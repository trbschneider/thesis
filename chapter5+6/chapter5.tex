%!TEX root = ../thesis.tex

\chapter[QSI of the healthy cord (I)]{Q-space imaging of the healthy cervical spinal cord (I)}
\label{chap:chap5 QSI in cord I}
In this chapter we investigate accuracy and sensitivity of spinal cord \gls{QSI} metrics in healthy controls and evaluate its potential for clinical application. Previous studies of \gls{QSI} on experimental MRI systems have shown that \gls{QSI} can provide accurate information about microscopic restriction in excised tissue \citep{Assaf:2000,Bar-Shir:2008,Ong:2008}. \Gls{QSI} requires an extensive sampling of different {\q}-values along a single axis. This restricts the number of diffusion gradient directions that can be sampled when scan time is limited. While \gls{QSI} application in the brain is limited by the need of high angular resolution of gradient directions to capture the variety of different fibre directions, this is less of a problem in the \gls{SC} due its relatively simple white matter structure. Although the conditions for true \gls{QSI}, such as the short gradient pulse, are impossible to achieve in clinical systems. Despite all its difficulties, previous proof-of-concept studies have shown the great potential in the assessment of {\gls{SC}} white matter and white pathologies in the human brain \citep{Assaf:2002,Yamada:2012} and in the spinal cord \citep{Farrell:2008}.

Following up on the encouraging results of these previous studies, we aim here to study the reproducibility of \gls{QSI} metrics in the cervical {\gls{SC}} on a standard 3T clinical MRI scanner. We also assess \gls{QSI} measures both in-plane (XY) and parallel to the main {\gls{SC}} axis (Z), not presented before. Previous work in \emph{in-vitro} rat spinal cord by \citet{Ong:2008,Ong:2010} suggest that \gls{QSI} parameters correlate with the axon diameter in different white matter regions. Our particular interest here is to explore whether clinical hardware constraints allow us to detect the structural differences between \gls{WM} and different ascending and descending \gls{WM} tracts of the cervical {\gls{SC}} with \gls{QSI}. We test whether \gls{QSI} can discriminate between \gls{WM} in the cervical in healthy subjects and compare conventional \gls{ADC} measures, both in plane and along the cord. Furthermore we also test whether any combination of \gls{QSI} derived FWHM and P0 metrics can better distinguish between WM \glspl{ROI} than the individual metrics alone.

The next two chapters will present two \gls{QSI} studies that both address the aims outlined above. This first chapter presents a study performed on 9 healthy controls that were scanned at the Wellcome Trust Centre for Neuroimaging as part of a pilot study to study the effect of brachial plexus avulsion (and was also used to study \gls{NMO}). Our preliminary findings in healthy controls were submitted for presentation at the Annual Meeting of the International Society of Magnetic Resonance in Medicine and were accepted for oral presentation. Following the encouraging results in this first experiment, we re-implemented an improved protocol on the Philips 3T MRI scanner, that was newly installed in our lab. in 2010. The results of a second \gls{QSI} pilot study using the new protocols comprise the next chapter (Chapter~\ref{chap:chap6 QSI in cord II}).

\section{Methods}
\label{sec:chapter 5 exp1 methods}
\subsection*{Study design}
Twenty right-handed male healthy subjects were recruited (mean age 35�11yrs) to be scanned on a 3T Tim Trio (Siemens Healthcare, Erlangen). Six subjects were recalled for a second scan on a different day to assess intra-subject reproducibility of \gls{QSI} derived parameters.

\subsection*{Data acquisition}
In each subject we perform cardiac-gated high b-value axial {\gls{DWI}} (matrix=96$\times$96, b-spline interpolated to 192$\times$192 in image space, FOV=$144\times144$ mm$^2$, slice thickness=5mm, 20 slices, TE=110ms, TR$\approx$4000ms). The \gls{QSI} set-up is based on parameters found in the most recent clinical \gls{QSI} study \citep{Farrell:2008}. However, our gradient system only allowed maximum \gls{gstr} of 23mT/m (\citet{Farrell:2008}: 60mT/m). To achieve similar {\q}-values it was necessary to increase the gradient duration \gls{smalldel} to 51ms. Reproduction of the protocol was further complicated by a limitation in the scanner software, which only permits a b-value to be specified in multiples of 50 mm/s$^2$ and means that {\q}-values can not be exactly linearly spaced. We acquire a total of 32 b-values between 0-3000s/mm$^2$ in three different {\gls{DWI}} directions: two directions perpendicular (XY) and one parallel (Z) to the main {\gls{SC}} axis. The full protocol is given in Table~\ref{tab:chap5exp1 protocol}.

After an initial quality check, we found that the prescription of the axial \gls{DWI} slices varied greatly between different subjects. Figure~\ref{fig:chapter5 positioning} shows two representive cases for correct and incorrect positioning observed in the dataset. \gls{QSI} is very sensitive to its alignment to the fibre direction as shown e.g. in \citep{Avram:2004} and the variation in slice positioning might overshadow the subtle differences between \gls{WM} we are interested in. We therefore measure the angulation between imaging plane and \gls{SC} longitudinal axis as seen on a T2w sagittal scan at C2/C3. We excluded 11 subjects and their subsequent data where the angle was less than 80$^\circ$ (ideally we assume the axial images perfectly perpendicular, i.e 90$^\circ$)

\begin{figure}
\centering
\subfloat[Correct]{\pgfimage[width=0.48\textwidth]{chapter5+6/figs/loc_good.png}}\hspace{0.015\textwidth}
\subfloat[Incorrect]{\pgfimage[width=0.48\textwidth]{chapter5+6/figs/loc_bad.png}}
\caption{Examples of correct and incorrect positioning of QSI scans on sagittal anatomical scans.}
\label{fig:chapter5 positioning}
\end{figure}

\subsection*{Data processing}
Similar to \citet{Farrell:2008} the two perpendicular diffusion directions were averaged to increase the signal-to-noise ratio. The measurements are linearly regridded to be equidistant in q-space and the  {\gls{dPDF}} is computed using inverse Fast Fourier Transformation. To increase the resolution of the  {\gls{dPDF}}, the signal was extrapolated in q-space to a maximum q=166mm$^{-1}$ by fitting a bi-exponential decay curve to the {\gls{DWI}} data as suggested in \citet{Cohen:2002, Farrell:2008}. Figure~\ref{fig:chapter5 exp1 processing pipeline} illustrates the processing pipeline. Maps of the full width at half maximum and zero displacement probability were derived for XY and Z as described in Section~\ref{sec:qspace}. For comparison we also computed the \gls{ADC} from the monoexponential part of the decay curve (b < 1100s/mm2) as in \citet{Farrell:2008} for both XY and Z directions using a constrained non-linear least squared fitting algorithm. Figure~\ref{fig:chapter5 exemplary maps} shows both \gls{ADC} maps and the four \gls{QSI} parameter maps in one randomly chosen subject.


\begin{figure}
  \pgfimage[width=\textwidth]{chapter5+6/figs/QSIprocessing.pdf}
  \caption{Cartoon of the individual steps in our QSI processing pipeline.}
  \label{fig:chapter5 exp1 processing pipeline}
\end{figure}


\begin{figure}
\centering
\subfloat[ADC$_{xy}$ $\times 10^{-9}m^2/s$]{
    \pgfimage[width=0.4\textwidth]{chapter5+6/figs/exp1_ADCX.png}
}
\subfloat[ADC$_{z}$ $\times 10^{-9}m^2/s$]{
    \pgfimage[width=0.4\textwidth]{chapter5+6/figs/exp1_ADCZ.png}
}\\
\subfloat[P0$_{xy}$]{
    \pgfimage[width=0.4\textwidth]{chapter5+6/figs/exp1_P0X.png}
}
\subfloat[FWHM$_{xy}$ $\times 10^{-6}m$]{
    \pgfimage[width=0.4\textwidth]{chapter5+6/figs/exp1_FWHMX.png}
}\\
\subfloat[P0$_{z}$]{
    \pgfimage[width=0.4\textwidth]{chapter5+6/figs/exp1_P0Z.png}
}
\subfloat[FWHM$_{z}$ $\times 10^{-6}m$]{
    \pgfimage[width=0.4\textwidth]{chapter5+6/figs/exp1_FWHMZ.png}
}
\caption{ADC maps and QSI parameter maps in one exemplary subject at the level of the C2--C3 disc.}
\label{fig:chapter5 exemplary maps}
\end{figure}

\subsection*{ROI analysis} We semi-automatically delineate the whole cervical {\gls{SCA}} between levels C1 and C3 on the b=0 images using the active surface segmentation by \citet{Horsfield:2010} available in Jim6. We perform a morphological erosion (2 iterations) of the obtained segmentation mask to exclude voxels with potential partial-volume average effect from surrounding \gls{CSF}. In addition, four regions of interest were manually placed in specific white matter tracts and one ROI was positioned in the gray matter on all slices between level C1 and C3. The four white matter regions comprised the left and right tracts (l\&r-LT) running in the lateral columns and the anterior (AT) and posterior tracts (PT) similar to \citet{Hesseltine:2006,Freund:2010} (see Figure~\ref{fig:chapter5 exp1 ROIs}).

  \begin{figure}
      \centering
      \pgfimage[height=5cm]{chapter5+6/figs/rois.png}
      \caption{Illustration of WM ROIs drawn on b0 image of the cord.}
      \label{fig:chapter5 exp1 ROIs}
  \end{figure}

  \begin{table}
      \centering
     \caption{QSI protocol displaying: Gradient strength (G), q-value (q) and b-value (b) for each of the 32 DWI volumes.}
		 \begin{adjustbox}{width={0.47\textwidth},keepaspectratio}
        \begin{tabular}{rrr}
        \addlinespace
            \multicolumn{3}{l}{}\\
        \toprule
            G $[mT/m]$ & q $[cm^{-1}]$ & b $[s/mm^2]$ \\
            \cmidrule(r){1-1}\cmidrule(lr){2-2}\cmidrule(l){3-3}
            0.0   & 0.0   & 0 \\
            3.0   & 56.2  & 50 \\
            4.2   & 79.4  & 100 \\
            5.1   & 97.3  & 150 \\
            5.9   & 112.3 & 200 \\
            6.6   & 125.6 & 250 \\
            8.4   & 158.9 & 400 \\
            9.8   & 186.3 & 550 \\
            11.5  & 217.5 & 750 \\
            12.9  & 244.8 & 950 \\
            14.5  & 275.2 & 1200 \\
            15.9  & 302.5 & 1450 \\
            17.5  & 332.3 & 1750 \\
            19.2  & 364.0 & 2100 \\
            20.7  & 393.2 & 2450 \\
            22.2  & 420.3 & 2800 \\
            \bottomrule
        \end{tabular}%
		 \end{adjustbox}
        \hspace{0.02\textwidth}%
		  \begin{adjustbox}{width={0.47\textwidth},keepaspectratio}
        \begin{tabular}{rrr}
        \addlinespace
            \multicolumn{3}{l}{\textit{... continued}}\\
        \toprule
            G $[mT/m]$ & q $[cm^{-1}]$ & b $[s/mm^2]$ \\
            \cmidrule(r){1-1}\cmidrule(lr){2-2}\cmidrule(l){3-3}
            0.0   & 0.0   & 0 \\
            3.0   & 56.2  & 50 \\
            4.2   & 79.4  & 100 \\
            5.1   & 97.3  & 150 \\
            5.9   & 112.3 & 200 \\
            7.8   & 148.6 & 350 \\
            9.4   & 177.6 & 500 \\
            10.7  & 202.5 & 650 \\
            12.2  & 231.6 & 850 \\
            13.9  & 263.4 & 1100 \\
            15.4  & 291.9 & 1350 \\
            16.7  & 317.7 & 1600 \\
            18.2  & 346.2 & 1900 \\
            19.9  & 376.8 & 2250 \\
            21.3  & 405.0 & 2600 \\
            22.9  & 435.1 & 3000 \\
            \bottomrule
        \end{tabular}%
     \label{tab:chap5exp1 protocol}
	 \end{adjustbox}
\end{table}

\subsection*{Statistical processing} We compare scan/re-scan reproducibility by computing the absolute difference and relative difference in ADC and \gls{QSI} parameters over the defined \glspl{ROI}.
%
%
Further, we investigate the correlation between individual \gls{ADC} and \gls{QSI} measurements in XY and Z directions. We pool all voxel-wise measurements over the segmented \gls{SC} area and compute Pearson's correlation coefficient over all voxels. We test for statistical significance of the correlations with a confidence interval of 95\%.


We then compare significant differences in individual metrics using a paired two-tailed t-test and further investigate statisitical significance in the group mean values of the \gls{ADC} parameters and \gls{QSI} metrics between tracts by performing the Hotellings-T$^2$ test (confidence interval=95\%). To investigate the relevance of measurements in the different \gls{DWI} directions, we compute the same significance test of XY-only \gls{QSI} parameters (P0$_{xy}$, FWHM$_{xy}$) and compare with Z-only (P0$_z$, FWHM$_z$) and the combination of both (P0$_{xy}$, FWHM$_{xy}$, P0$_z$, FWHM$_z$).

\section{Results}
\subsection*{Reproducibility}
\label{par:chapter5 exp1 reproducibility}
Tables~\ref{tab:chap5exp1 scan rescan} shows absolute and relative differences between scan and rescan of three healthy subjects in ADC$_{xy}$ and ADC$_z$ and \gls{QSI} metrics in XY and Z direction. We observe a general trend of measurements perpendicular to the long \gls{SC} fibres presenting higher variation between scan and rescan than parallel measurements in \gls{ADC} and both \gls{QSI} metrics in all subjects. In particular ADC$_{xy}$ shows very high intra-subject variation between 20--40\% on average in all white matter \glspl{ROI}, while only GM values show good reproducibility value of less than 11\% variation. In particular ADC$_z$ appears more reproducible in all three subjects with average relative variation between 5--16\%.


The perpendicular \gls{QSI} metrics P0${_{xy}}$ and FWHM$_{xy}$ present good reproducibility values of 6--12\% and are up to 4 times lower than ADC$_{xy}$ measurements in corresponding \glspl{ROI}. In both P0$_{z}$ and FWHM$_{z}$ we observe relative change between 4--13\% similar to values in ADC$_z$.%
\begin{table}
    \centering%
    \caption{Absolute and relative change (in percent) between scan and rescan of diffusivities and QSI parameters in 3 healthy volunteers}
	 \label{tab:chap5exp1 scan rescan}
    \footnotesize
\begin{adjustbox}{width={0.9\textwidth},keepaspectratio}
    		\begin{minipage}{\linewidth}
			\subfloat[Perpendicular (ADC$_{xy}$) and parallel diffusivity (ADC$_{z}$)]
			{
    		\begin{minipage}{\linewidth}
	        \begin{tabular}{rrrrrr}
            \addlinespace
			\multicolumn{6}{c}{\textbf{ADC$_{xy}$ $\times$ $10^{-9}m^2/s$}}\\
			\toprule
            subject & rLT   & lLT   & AT    & PT    & GM \\
            \midrule
            1     & 0.10 (30.4\%) & 0.00 (4.7\%) & 0.07 (27.6\%) & 0.06 (24.1\%) & 0.09 (12.0\%) \\
            2     & 0.06 (16.9\%) & 0.06 (34.4\%) & 0.12 (44.6\%) & 0.03 (11.0\%) & 0.05 (12.0\%) \\
            3     & 0.09 (25.5\%) & 0.12 (51.9\%) & 0.24 (57.2\%) & 0.20 (82.5\%) & 0.04 (8.6\%) \\
                  &       &       &       &       &  \\
            mean  & 0.08 (24.3\%) & 0.06 (30.4\%) & 0.14 (43.1\%) & 0.10 (39.2\%) & 0.06 (10.9\%) \\
            \bottomrule
            \end{tabular}%
			\\[0.5ex]
		\begin{tabular}{rrrrrr}
			\addlinespace
			\multicolumn{6}{c}{\textbf{ADC$_{z}$ $\times$ $10^{-9}m^2/s$}}\\
			\toprule
			subject & rLT   & lLT   & AT    & PT    & GM \\
		            \midrule
		            1     & 0.04 (3.3\%) & 0.07 (4.7\%) & 0.18 (12.2\%) & 0.03 (2.1\%) & 0.03 (2.4\%) \\
		            2     & 0.13 (9.0\%) & 0.17 (9.8\%) & 0.40 (23.2\%) & 0.03 (1.6\%) & 0.30 (16.9\%) \\
		            3     & 0.16 (12.5\%) & 0.10 (6.2\%) & 0.21 (12.9\%) & 0.16 (10.2\%) & 0.28 (16.6\%) \\
		                  &       &       &       &       &  \\
		            mean  & 0.11 (8.3\%) & 0.12 (6.9\%) & 0.26 (16.1\%) & 0.07 (4.7\%) & 0.20 (12.0\%) \\
		            \bottomrule
		\end{tabular}%
		\end{minipage}
		\label{tab:chap5exp1 scan rescan adc}
		}
		\end{minipage}
		\end{adjustbox}	
\begin{adjustbox}{width={0.9\textwidth},keepaspectratio}
\begin{minipage}{\linewidth}
	\subfloat[Perpendicular and parallel QSI parameters]
		    {
	    		\begin{minipage}{\linewidth}
				\begin{tabular}{rrrrrr}
		            \addlinespace
				\multicolumn{6}{c}{\textbf{P0$_{xy}$}}\\
				\toprule
				subject & rLT   & lLT   & AT    & PT    & GM \\
		            \midrule
		            1     & 0.01 (3.1\%) & 0.02 (6.8\%) & 0.01 (3.4\%) & 0.00 (1.7\%) & 0.00 (1.9\%) \\
		            2     & 0.00 (0.4\%) & 0.00 (0.3\%) & 0.01 (4.3\%) & 0.01 (3.3\%) & 0.00 (2.3\%) \\
		            3     & 0.01 (6.1\%) & 0.06 (28.2\%) & 0.04 (19.9\%) & 0.06 (26.7\%) & 0.03 (14.8\%) \\
		                  &       &       &       &       &  \\
		            mean  & 0.01 (3.2\%) & 0.03 (11.8\%) & 0.02 (9.2\%) & 0.02 (10.6\%) & 0.01 (6.3\%) \\
		            \bottomrule
		            \end{tabular}%
		        \\[0.5ex]
		        \begin{tabular}{rrrrrr}
		            \addlinespace
				\multicolumn{6}{c}{\textbf{FWHM$_{xy}$}}\\			
		            \toprule
		            subject & rLT   & lLT   & AT    & PT    & GM \\
		            \midrule
		            1     & 0.52 (2.5\%) & 0.67 (4.8\%) & 0.67 (3.5\%) & 0.29 (1.5\%) & 0.62 (2.4\%) \\
		            2     & 0.03 (0.1\%) & 0.29 (1.6\%) & 0.76 (4.1\%) & 0.36 (1.9\%) & 0.32 (1.5\%) \\
		            3     & 1.10 (5.2\%) & 5.69 (29.6\%) & 4.72 (20.6\%) & 5.10 (27.5\%) & 3.29 (15.5\%) \\
		                  &       &       &       &       &  \\
		            mean  & 0.55 (2.6\%) & 2.22 (12.0\%) & 2.05 (9.4\%) & 1.92 (10.3\%) & 1.41 (6.5\%) \\
		            \bottomrule
		        \end{tabular}%
			\\[0.5ex]
		        \begin{tabular}{rrrrrr}
		            \addlinespace
				\multicolumn{6}{c}{\textbf{P0$_{z}$}}\\
		            \toprule
		            subject & rLT   & lLT   & AT    & PT    & GM \\
		            \midrule
		            1     & 0.00 (4.5\%) & 0.00 (0.6\%) & 0.01 (6.9\%) & 0.00 (2.5\%) & 0.01 (5.6\%) \\
		            2     & 0.01 (9.2\%) & 0.01 (11.4\%) & 0.01 (11.0\%) & 0.00 (3.4\%) & 0.01 (10.6\%) \\
		            3     & 0.01 (6.0\%) & 0.00 (0.1\%) & 0.00 (1.2\%) & 0.01 (7.8\%) & 0.01 (14.9\%) \\
		                  &       &       &       &       &  \\
		            mean  & 0.01 (6.6\%) & 0.00 (4.1\%) & 0.01 (6.3\%) & 0.00 (4.6\%) & 0.01 (10.4\%) \\
		            \bottomrule
		        \end{tabular}%
		        \\[0.5ex]
		        \begin{tabular}{rrrrrr}
		            \addlinespace
				\multicolumn{6}{c}{\textbf{FWHM$_{z}$}}\\		
		            \toprule
		            subject & rLT   & lLT   & AT    & PT    & GM \\
		            \midrule
		            1     & 1.42 (3.9\%) & 1.67 (3.9\%) & 4.20 (10.4\%) & 1.07 (2.7\%) & 3.86 (10.7\%) \\
		            2     & 5.72 (15.4\%) & 9.04 (22.2\%) & 4.69 (11.6\%) & 4.47 (10.9\%) & 4.76 (12.5\%) \\
		            3     & 1.08 (3.0\%) & 0.13 (0.3\%) & 1.66 (4.1\%) & 4.89 (12.4\%) & 6.17 (16.1\%) \\
		                  &       &       &       &       &  \\
		            mean  & 2.74 (7.4\%) & 3.61 (8.8\%) & 3.52 (8.7\%) & 3.48 (8.6\%) & 4.93 (13.1\%) \\
		            \bottomrule
		        \end{tabular}%
				\label{tab:chap5exp1 scan rescan qsi}
	    		\end{minipage}
		    }
\end{minipage}
\end{adjustbox}
\end{table}

\subsection*{Differences between tract-specific ROI measurements}
Figure~\ref{fig:chapter5 exp1 vals} compares the average values and standard deviation over all 9 subjects between tract-specific for \gls{ADC} and \gls{QSI} metrics. As a general trend, we observe higher inter-subject variation in XY measurements compared to Z measurements among all 9 subjects which is in line with our results of intra-subject variation shown above.


Table~\ref{tab:chap5exp1 single ttest} present $p$-values for pairwise differences between different tract-\glspl{ROI} for ADC and \gls{QSI} metrics. The most notable differences are found between the GM \gls{ROI} and the white matter regions in ADC$_{xy}$ and both P0$_{xy}$/FWHM$_{xy}$ with high statistical significance (p<0.01 between WM tracts GM for all \gls{QSI}$_{xy}$ metrics), while ADC$_z$ and \gls{QSI}$_{z}$ metrics are less different between GM \gls{ROI} and WM \glspl{ROI}. In fact, significant differences are only found between rLT and GM in ADC$_z$. The \gls{QSI}$_z$ metrics only show significant differences between GM and rRT in P0$_z$ (p=0.01) and between GM and AT and PT (FWHM$_z$).


Between \gls{WM} \glspl{ROI} only the left LT but not the right LT is significantly different from both AT and PT in ADC$_{xy}$ perpendicular to long white matter fibres. Parallel to the long \gls{SC} axis we only find ADC$_{z}$ in the right LT significantly lower from AT and PT. Left and right LT show significant differences in both ADC$_{xy}$ and ADC$_{z}$ while we find no difference between AT or PT. In \gls{QSI} metrics we find the same tracts as with ADC to be significantly different in XY and Z direction. However, $p$-values are increased in \gls{QSI} compared to corresponding ADC, but remain below p<0.05.%
\begin{figure}[h!tp]
	\subfloat[]
	{
		\begin{minipage}{\linewidth}
		   \centering
		   \pgfplotsset{cutoff_vs_dti_barchart/.style={ybar,
                                            bar width=20pt,
                                            width=6cm,
                                            height=6cm,
                                            xtick={{1},{2},{3},{4},{5},{6}},
                                            xticklabels={rLT,lLT,AT,PT,GM,SCA}, fill=red},
                                            yticklabel style={/pgf/number format/.cd,
                                                              fixed,
                                                              fixed zerofill,
                                                              precision=2}}
\pgfplotsset{cutoff_vs_dti_barchart plot/.style={fill=olive!40!white,error bars/.cd, y dir=both, y explicit}}

\begin{tikzpicture}
\begin{axis}[cutoff_vs_dti_barchart, title=$ADC_{xy}$ $\times 10^{-9} m^2/s$, ymin=0]
    \addplot+[cutoff_vs_dti_barchart plot] table[y=ADCX, y error=ADCXerr] {chapter5+6/figs/exp1_qspacevals.dat};
\end{axis}
\end{tikzpicture}
\begin{tikzpicture}
\begin{axis}[cutoff_vs_dti_barchart, title=$ADC_{z}$ $\times 10^{-9} m^2/s$, ymin=0]
    \addplot+[cutoff_vs_dti_barchart plot] table[y=ADCZ, y error=ADCZerr] {chapter5+6/figs/exp1_qspacevals.dat};
\end{axis}
\end{tikzpicture}

		\end{minipage}%
	}\\
	\subfloat[]
	{
		\begin{minipage}{\linewidth}
	   	% Table generated by Excel2LaTeX from sheet 'Sheet2'

%!TEX root = ../chap4.tex
\pgfplotsset{cutoff_vs_dti_barchart/.style={ybar,
                                            bar width=20pt,
                                            width=6cm,
                                            height=6cm,
                                            xtick={{1},{2},{3},{4},{5},{6}},
                                            xticklabels={rLT,lLT,AT,PT,GM,SCA}},
                                            yticklabel style={/pgf/number format/.cd,
                                                              fixed,
                                                              fixed zerofill,
                                                              precision=2}}
\pgfplotsset{cutoff_vs_dti_barchart plot/.style={error bars/.cd, y dir=both, y explicit}}
\begin{tikzpicture}
\begin{axis}[cutoff_vs_dti_barchart, title=$P0_{xy}$, ymin=0]
    \addplot+[cutoff_vs_dti_barchart plot] table[y=P0X, y error=P0Xerr] {chapter5+6/figs/exp1_qspacevals.dat};
\end{axis}
\end{tikzpicture}
\begin{tikzpicture}
\begin{axis}[cutoff_vs_dti_barchart, title=$P0_{z}$, ymin=0]
    \addplot+[cutoff_vs_dti_barchart plot] table[y=P0Z, y error=P0Zerr] {chapter5+6/figs/exp1_qspacevals.dat};
\end{axis}
\end{tikzpicture}\\
\begin{tikzpicture}
\begin{axis}[cutoff_vs_dti_barchart, title=$FWHM_{xy}$ $\times 10^{-6} m$, ymin=0]
    \addplot+[cutoff_vs_dti_barchart plot] table[y=FWHMX, y error=FWHMXerr] {chapter5+6/figs/exp1_qspacevals.dat};
\end{axis}
\end{tikzpicture}
\begin{tikzpicture}
\begin{axis}[cutoff_vs_dti_barchart, title=$FWHM_{z}$ $\times 10^{-6} m$, ymin=0]
    \addplot+[cutoff_vs_dti_barchart plot] table[y=FWHMZ, y error=FWHMZerr] {chapter5+6/figs/exp1_qspacevals.dat};
\end{axis}
\end{tikzpicture}	
		\end{minipage}%
	}
   \caption{Mean and standard deviation of perpendicular and parallel ADC and QSI metrics for all ROIs over all 9 volunteers.}
   \label{fig:chapter5 exp1 vals}	
\end{figure}

\begin{table}
  \caption{Significance of pair-wise differences between SC tracts in diffusivities and QSI parameters (confidence interval: 95\%). Statistically significant differences are marked as follows: \textbf{bold} if p<0.05, \textbf{\emph{bold-italic}} if p<0.01.}	
  \label{tab:chap5exp1 single ttest}%
	\footnotesize
    \centering
    \subfloat[]
       	{
   				\begin{minipage}{\linewidth}%
              	\begin{tabular}{rrrrr}
				\addlinespace
				\multicolumn{5}{c}{\textbf{ADC$_{xy}$ $\times$ $10^{-9}m^2/s$}}\\
				\toprule
                      & lLT  & AT    & PT    & GM \\
                \midrule
                rLT  & \textbf{0.01}  & 0.60  & 0.84  & \textbf{\emph{<0.01}} \\
                lLT  &       & \textbf{\emph{<0.01}}  & \textbf{\emph{<0.01}}  & \textbf{\emph{<0.01}} \\
                AT    &       &       & 0.56  & \textbf{\emph{<0.01}} \\
                PT    &       &       &       & \textbf{\emph{<0.01}} \\
                \bottomrule
                \end{tabular}%
    			\hspace{0.5cm}
				%%
				%%
		        \begin{tabular}{rrrrr}
		        \addlinespace
				\multicolumn{5}{c}{\textbf{ADC$_{z}$ $\times$ $10^{-9}m^2/s$}}\\
				\toprule
		              & lLT  & AT    & PT    & GM \\
		        \midrule
		        rLT  & \textbf{0.01}  & \textbf{\emph{<0.01}}  & \textbf{\emph{<0.01}}  & \textbf{\emph{<0.01}} \\
		        lLT  &       & 0.85  & \textbf{\emph{<0.01}}  & 0.57 \\
		        AT    &       &       & 0.44  & 0.30 \\
		        PT    &       &       &       & 0.74 \\
		        \bottomrule
		        \end{tabular}%
				\end{minipage}%
				\label{tab:chap5exp1_adc single ttest}%
				
  	   }\\
  	   	\subfloat[]
  	  	{			
		\begin{minipage}{\linewidth}
  				    \begin{tabular}{rrrrr}
  				    \addlinespace
					\multicolumn{5}{c}{\textbf{P0$_{xy}$}}\\
					\toprule
  				          & lLT  & AT    & PT    & GM \\
  				    \midrule
  				    rLT  & \textbf{0.04}  & 0.27  & 0.48  & \textbf{\emph{<0.01}} \\
  				    lLT  &       & \textbf{0.05}  & 0.48  & \textbf{\emph{<0.01}} \\
  				    AT    &       &       & 0.97  & \textbf{\emph{<0.01}} \\
  				    PT    &       &       &       & \textbf{\emph{<0.01}} \\
  				    \bottomrule
  				    \end{tabular}%
  	  			    \hspace{0.5cm}
  				    \begin{tabular}{rrrrr}
  				    \addlinespace
					\multicolumn{5}{c}{\textbf{P0$_{z}$}}\\
					\toprule
  				          & lLT  & AT    & PT    & GM \\
  				    \midrule
  				    rLT  & \textbf{0.01}  & \textbf{\emph{<0.01}}  & \textbf{\emph{<0.01}}  & \textbf{0.01} \\
  				    lLT  &       & 0.94  & \textbf{\emph{<0.01}}  & 0.77 \\
  				    AT    &       &       & 0.40  & 0.69 \\
  				    PT    &       &       &       & 0.16 \\
  				    \bottomrule
  				    \end{tabular}%
  					\\[0.5ex]
  	 				\begin{tabular}{rrrrr}
  	 			    \addlinespace
					\multicolumn{5}{c}{\textbf{FWHM$_{xy}$}}\\		
					\toprule
  				        & lLT  & AT    & PT    & GM \\
  				    \midrule
  				    rLT  & \textbf{0.04}  & 0.56  & 0.37  & \textbf{\emph{<0.01}} \\
  				    lLT  &       & \textbf{0.02}  & 0.37  & \textbf{\emph{<0.01}} \\
  				    AT    &       &       & 0.72  & \textbf{0.01} \\
  				    PT    &       &       &       & \textbf{\emph{<0.01}} \\
  				    \bottomrule
  				    \end{tabular}%
  	  			    \hspace{0.5cm}
  				    \begin{tabular}{rrrrr}
  				    \addlinespace
					\multicolumn{5}{c}{\textbf{FWHM$_{z}$}}\\		
					\toprule
  				          & lLT  & AT    & PT    & GM \\
  				    \midrule
  				    rLT  & 0.21  & \textbf{0.02}  & \textbf{0.03}  & 0.99 \\
  				    lLT  &       & 0.20  & \textbf{0.03}  & 0.13 \\
  				    AT    &       &       & 1.00  & \textbf{0.01} \\
  				    PT    &       &       &       & \textbf{0.01} \\
  				    \bottomrule
  				    \end{tabular}%
  				\end{minipage}%
  	  	  	  }
\end{table}%

\subsection*{Multi-variate differences between tract-specific ROI measurements}
The single parameter comparisions above indicate that both \gls{ADC} and \gls{QSI} metrics can discriminate some WM tracts, but offer complementary information in perpendicular and parallel measurements. The multivariate Hotelling's-T$^2$ test allows us to test whether a combination of XY and Z metrics is better suited to characterize and discriminate WM measures in different \glspl{ROI}. In the following, we present results for the following combinations of parameters:
\begin{itemize}
    \item Both diffusivity parameters ADC$_{xy}$ and ADC$_z$ (Table~\ref{tab:chap5exp1_adc hotelling})
    \item Perpendicular only \gls{QSI} metrics P0$_{xy}$ and FWHM$_{xy}$ (Table~\ref{tab:chap5exp1_adc hotelling})
    \item Parallel only \gls{QSI} metrics P0$_z$ and FWHM$_z$ (Table~\ref{tab:chap5exp1_qsiz hotelling})
    \item Perpendicular and parallel \gls{QSI} metrics P0$_{xy}$, FWHM$_{xy}$, P0$_z$ and FWHM$_z$ (Table~\ref{tab:chap5exp1_qsiall hotelling})
\end{itemize}


Similar to the single t-test results shown above, GM and WM \glspl{ROI} can clearly be distinguished using either of the combinations of \gls{ADC} and \gls{QSI} parameters. However, GM/WM differences are more pronounced in XY than in Z direction. The combined ADC metrics show significant differences between the both lateral tracts and also l/r LT and the posterior WM \gls{ROI}. AT is only significantly different from the right but not the left LT.

For combination of \gls{QSI} parameters in Z only, as well as the combination of both XY and Z, the only two emerging differences are found between lLT/rLT and rLT/PT, both with p<0.05.%

\begin{table}
  \caption{Hotelling's-T$^2$ significance of pair-wise tract-specific differences for ADC and QSI parameters. (\textbf{bold} marks p<0.05, \textbf{\emph{bold-italic}} marks p<0.01).}
  \label{tab:chap5exp1  hotelling}%

  \centering
  \footnotesize
  \subfloat[Combined ADC$_{xy}$,ADC$_{z}$]
  {
	    \begin{tabular}{rrrrr}
	    \addlinespace
	    \toprule
	          & lLT  & AT    & PT    & GM \\
	    \midrule
	    rLT  & \textbf{\emph{<0.01}}  & \textbf{0.02}  & \textbf{0.01}  & \textbf{\emph{<0.01}} \\
	    lLT  &       & 0.10  & \textbf{0.01}  & \textbf{\emph{<0.01}} \\
	    AT    &       &       & 0.85  & \textbf{\emph{<0.01}} \\
	    PT    &       &       &       & \textbf{\emph{<0.01}} \\
	    \bottomrule
	    \end{tabular}%
	  \label{tab:chap5exp1_adc hotelling}%
  }\hspace{0.5cm}
  \subfloat[Combined perpendicular QSI parameters (P0$_{xy}$,FWHM$_{xy})$]{
      \begin{tabular}{rrrrr}
        \addlinespace
        \toprule
              & lLT  & AT    & PT    & GM \\
        \midrule
        rLT  & 0.13 & 0.79  & 0.71 &  \textbf{0.01} \\
        lLT  &       & 0.26  & 0.12 & \textbf{\emph{<0.01}} \\
        AT    &       &       & 0.76 & \textbf{0.02}  \\
        PT    &       &       &       & \textbf{\emph{<0.01}} \\
        \bottomrule
      \end{tabular}%
      \label{tab:chap5exp1_qsix hotelling}%
  }\\
  \subfloat[Combined parallel QSI parameters (P0$_{z}$,FWHM$_{z}$)]
  {
    \begin{tabular}{rrrrr}
    \addlinespace
    \toprule
          & lLT  & AT    & PT    & GM \\
    \midrule
    rLT  & \textbf{0.03}  & 0.08  & \textbf{0.02}  & \textbf{0.02} \\
    lLT  &       & 0.60  & 0.86  & 0.18 \\
    AT    &       &       & 0.49  & \textbf{0.01} \\
    PT    &       &       &       & \textbf{\emph{<0.01}} \\
    \bottomrule
    \end{tabular}%
    \label{tab:chap5exp1_qsiz hotelling}%
  }\hspace{0.5cm}
  \subfloat[Combined perpendicular and parallel QSI parameters (P0$_{xy}$,FWHM$_{xy}$,P0$_{z}$,FWHM$_{z}$)]
  {
    \begin{tabular}{rrrrr}
    \addlinespace
    \toprule
          & lLT  & AT    & PT    & GM \\
    \midrule
    rLT  & \textbf{0.04}  & 0.25  & \textbf{0.01}  & \textbf{\emph{<0.01}} \\
    lLT  &       & 0.62  & 0.50  & \textbf{\emph{<0.01}} \\
    AT    &       &       & 0.75  & \textbf{\emph{<0.01}} \\
    PT    &       &       &       & \textbf{\emph{<0.01}} \\
    \bottomrule
    \end{tabular}%
    \label{tab:chap5exp1_qsiall hotelling}%
  }
\end{table}%

\subsection*{Voxel-wise correlation of ADC and QSI metrics}
Table~\ref{tab:chapter5 exp1 correlations} shows the Pearson correlation coefficient $\mathit{r}$ and statistical significance of the correlation between ADC and \gls{QSI} parameters over all \gls{SC} voxels in all subjects. We observe significant correlations between \gls{ADC}$_{xy}$ and \gls{ADC}$_z$. Further we find significant correlations between
and \gls{QSI} parameters both within and across XY and Z direction. Interestingly, we find both FWHM$_{xy}$ and P0$_z$ are correlated with each other and also with both ADC parameters, while the other two \gls{QSI} parameters P0$_{xy}$ and FWHM$_z$ did neither correlate with each other nor any other metrics.%

\begin{table}
	\centering
    \caption{Pearson-correlation coefficient and significance between all ADC and QSI metrics. The $p$-values $<0.01$ are marked \textbf\emph{bold-italic}.}
	 \begin{adjustbox}{width=0.8\textwidth,keepaspectratio}
	 \begin{minipage}{\textwidth}
	 \begin{tabular}{rrrrrrrr}
	    \addlinespace
	    \toprule
	          &       & ADC$_{xy}$  & ADC$_{z}$  & P0$_{xy}$   & FWHM$_{xy}$   & P0$_{z}$   & FWHM$_{z}$ \\
	    \midrule
	    \multicolumn{1}{c}{\multirow{2}[0]{*}{ADC$_{xy}$}} & $\mathit{r}$   &       & 0.58  & 0.00  & -0.74 & 0.20  & 0.01 \\
	    \multicolumn{1}{c}{} & {p} & {} & {\textbf{\emph{<0.01}}} & {0.91} & {\textbf{\emph{<0.01}}} & {\textbf{\emph{<0.01}}} & {0.56} \\[2.0ex]
	    \multicolumn{1}{c}{\multirow{2}[0]{*}{ADC$_{z}$}} & $\mathit{r}$   & 0.58  &       & 0.00  & -0.29 & 0.71  & 0.00 \\
	    \multicolumn{1}{c}{} & {p} & {\textbf{\emph{<0.01}}} & {} & {0.87} & {\textbf{\emph{<0.01}}} & {\textbf{\emph{<0.01}}} & {0.82} \\[2.0ex]
	    \multicolumn{1}{c}{\multirow{2}[0]{*}{P0$_{xy}$}} & $\mathit{r}$   & 0.00  & 0.00  &       & 0.00  & 0.00  & 0.00 \\
	    \multicolumn{1}{c}{} & {p} \& {0.91} & {0.87} & {} & {0.82} & {0.99} & {1.00} \\[2.0ex]
	    \multicolumn{1}{c}{\multirow{2}[0]{*}{FWHM$_{xy}$}} & $\mathit{r}$   & -0.74 & -0.29 & 0.00  &       & -0.18 & -0.01 \\
	    \multicolumn{1}{c}{} & {p} & {\textbf{\emph{<0.01}}} & {\textbf{\emph{<0.01}}} & {0.82} & {} & {\textbf{\emph{<0.01}}} & {0.70} \\[2.0ex]
	    \multicolumn{1}{c}{\multirow{2}[0]{*}{P0$_{z}$}} & $\mathit{r}$   & 0.20  & 0.71  & 0.00  & -0.18 &       & 0.01 \\
	    \multicolumn{1}{c}{} & {p} & {\textbf{\emph{<0.01}}} & {\textbf{\emph{<0.01}}} & {0.99} & \textbf{\emph{<0.01}} & {} & {0.52} \\[2.0ex]
	    \multicolumn{1}{c}{\multirow{2}[0]{*}{FWHM$_{z}$}} & $\mathit{r}$   & 0.01  & 0.00  & 0.00  & -0.01 & 0.01  &  \\
	    \multicolumn{1}{c}{} & {p} & {0.56} & {0.82} & {1.00} & {0.70} & {0.52} & {} \\
    \bottomrule
    \end{tabular}%
	\end{minipage}
	\end{adjustbox}
  \label{tab:chapter5 exp1 correlations}
\end{table}%

\section{Discussion} \gls{QSI} metrics obtained without sequence development, using a standard {\gls{DWI}} protocol available on a 3T clinical scanner, show a good reproducibility that is superior to simple \gls{ADC} analysis. We observe tract-specific correlations between \gls{ADC} and \gls{QSI} parameters between several WM tracts. However some of the associations in \gls{QSI} metrics are weaker in XY compared to Z, particularly between lateral and posterior tracts. Together with the findings of weak correlation between \gls{QSI} and ADC metrics in both XY and Z, our results suggest that the Z direction provides additional information to perpendicular measurements. Our results also suggest that on a clinical scanner \gls{QSI} might not be able to reliably distinguish between individual \gls{WM} tracts.

\paragraph{} The results of this experiment need to be interpreted with caution due to the limitations in hardware and software in the experimental setup. In particular the low gradient strength used in this study might conceal differences between tracts. Simulations by \citet{Latt:2007} show that insufficient gradient strength might lead to overestimation of compartment size and suggest that gradients of at least 60 mT/m are required to be sensitive to the typical size of axons found in human WM. Beside the hardware limitations of the scanner, there are also several issues in the design of this study including:
\begin{enumerate}
	\item the high number of subjects that had to be excluded due to misalignment of the axial images
	\item the linear regridding that made necessary because of scanner software limitation
	\item the relatively high in-plane resolution of the acquired images
\end{enumerate}
To overcome these flaws in the study design, the data had to undergo a rather extensive pre-processing pipeline, which might weaken the confidence in our results. The installation of a new 3T scanner in our lab offered us the possibility to repeat this experiment with improved hardware and software capabilities. The results are described in the following chapter.
