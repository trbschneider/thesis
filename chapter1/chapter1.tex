%!TEX root = ../thesis.tex
Axon morphology plays an important role for normal signal conduction. Damage to axon or myelin integrity will compromise signal transmission and can cause severe neurological symptoms. Damage to the spinal cord axonal tissue results particularily dangerous as it results in a loss of function such as mobility or feeling. Frequent causes of damage are trauma (car accident, gunshot,  falls, etc.) or disease (polio, spina bifida, Friedreich's Ataxia, etc.). 

\Gls{MRI} represents a non-invasive clinical method to assess anatomical changes in the spinal cord after injury. The introduction of \gls{MRI} to the clinical practise has vastly improved the diagnosis and treatment monitoring of spinal cord pathologies such as \gls{SCI} and \gls{MS}. While routine \gls{MRI} scans are aiding the detection of macroscopic changes in the cord, they have a limited prognostic value because of their qualitative nature and because of their lack of specificity in terms of underlying microstructure changes. 

The sensitivity of {\gls{DWI}} to the diffusion of water molecules in the tissue in vivo has been exploited for more than 20 years to characterise the white matter tissue structure of the brain. Thanks to technological advances such as multi-channel coils for parallel imaging methods and 3T scanners, the past couple of years have made their application of \gls{DWI} in the {\gls{SC}} feasible \citep{CWK,Ellingson}. As a result, diffusion imaging techniques are emerging as useful clinical for methods for visualization and quantification of spinal cord damage as a result of trauma or during progressive degenerative diseases. 

Despite encouraging initial results, much works need still to be done bring \gls{DWI} in the cord to clinical practise. Specifically there is the need for in-vivo imaging biomarkers for human {\gls{SC}} examinations, which are sensitive to underlying tissue changes and which are capable of quantifying structural and functional pathologies.


%\begin{figure}[htbp]
%	\centering
%	\pgfimage[width=0.7\textwidth]{chapter1/figs/sc_injury}
%	\caption{Illustration of tissue damage after spinal cord injury in the acute, chronic and treatment stages of the disease.}
%\end{figure}



%{\Gls{SCI}} can have devastating effects on the life of people affected by it. Thanks to innovative treatment strategies, {\gls{SCI}} patients can now hope in the concrete possibility of new therapies leading to recovery of feeling and motor functions, with a dramatic repercussion on their future quality of life.


\section{Problem statement}
Despite some development work on \gls{DWI} for \gls{SC}, the following problems are mainly unresolved 
\begin{enumerate}
\item Current state-of-the-art \gls{DWI} analysis methods, such as \gls{DTI}, are unspecific to individual microstructural changes and therefore only have limited value in the evaluation of treatment and recovery in spinal cord pathology. Research on more advanced \gls{DWI} techniques is usually focussed on the only the brain in mind and is often not directly applicable to the \gls{SC} in the same manner.
\item \gls{DWI} acquisition itself is well established in the brain, but much less so in the \gls{SC}. The \gls{SC} is a more challenging structure to study because of several problems: the breathing motion, the artefacts arising from the surrounding bones, the pulsation of the {\gls{CSF}} and last but not least its limited size that requires high resolution.  
\end{enumerate} 
The key motivation of this work is to overcome the challenges above by optimising the whole process from the acquisition design to the analysis methods, based on known \gls{SC} tissue properties and to develop imaging biomarkers can provide insight in underlying mechanisms of tissue damage and functional recovery.

\section{Aims}
\begin{enumerate}
  \item Investigate existing \gls{DWI} methods and identify suitable metrics for \gls{SC} characterisation
  \item Optimise existing acquisition protocols and analysis methods to improve sensitivity to \gls{SC} pathologies
  \item Design new \gls{DWI} imaging protocols and white matter models and derive new imaging biomarkers specifically for a better quantification of \gls{SC} microstructure properties
\end{enumerate}

\section{Summary of contributions}
The work presented in this report is divided in four different experiment that each contributes towards the aims described above.
\begin{itemize}
  \item In Section~\ref{sec:chap3:experiment1} we describe and test an imaging protocol to visualise and quantify the presence of collateral sprouting fibres at different levels of the \gls{SC}. This experiment contributes towards project aim 1 and 2 as we investigate two different \gls{DWI} metrics from existing literature and focus on the optimisation of the acquisition protocol.
  \item In Section~\ref{sec:chap3:experiment2} we develop a novel post-processing method that corrects average \gls{DTI} metrics for partial volume effects. This experiment contributes towards project aim 2, as we aim to improve reliability and reduce inter-subject variability for \gls{DTI} acquisitions and measurements that are widely used in clinical studies.
  \item In Section~\ref{sec:chap3:experiment3} we present a reliability study of the \gls{QSI} method, applied to \gls{SC}. This experiment contributes towards project aim 1  as it investigates the use of \gls{QSI} metrics for the investigation of \gls{SC} microstructure. The experiment also contributes to project aim 3 as it looks for the first time into \gls{QSI} parallel to the major fibre direction as an additional imaging marker.
  \item In Section~\ref{sec:chap3:experiment4} we work towards project aim 3 by developing \gls{DWI} protocols that allow direct estimation of axon diameter and density of SC white matter tissue. Our method is an extension of the ``Active Imaging'' framework by \citet{Alexander:2010}. We modify the existing method to be able to exploit the characteristic a-priori know single major fibre orientation in structures like the \gls{SC}. 
\end{itemize}
%\section{Report structure}
%\section{Publications and presentations}
%\subsection*{Journal papers in preparation}
%\begin{itemize}
%  \item Ciccarelli, O., Thomas, D. L., De Vita, E., Wheeler-Kingshott, C. A. M., Schneider, T., Kachramanoglou, C., Toosy, A. T., \& Thompson, A. J. Spinal cord spectroscopy, tractography and q-space MRI in a case of NMO spectrum disorder.
%  \item Freund, P.$^*$, Schneider, T.$^*$, Nagy, Z., Wheeler-Kingshott, C. A. M., \& Thompson, A. J. Diffusion Tensor Imaging Detects Axonal Degeneration and its Extent is Associated with Disability in Chronic Spinal Cord Injury.  
%\end{itemize}
%\subsection*{Conference papers}
%\begin{itemize}
%  \item Schneider, T., Wheeler-Kingshott, C. A. M., \& Alexander, D. C. (2010). In-vivo estimates of axonal characteristics using optimized diffusion MRI protocols for single fibre orientation. \emph{13th International Conference on Medical Image Computing and Computer-Assisted Intervention (MICCAI2010)}  
%\end{itemize}
%\subsection*{Poster presentations}
%\begin{itemize}
%  \item Schneider, T., Alexander, D. C., \& Wheeler-Kingshott, C. A. M. (2010). Optimized diffusion MRI protocols for estimating axon diameter with known fibre orientation. \emph{18th Scientific Meeting of the International Society for Magnetic Resonance in Medicine}
%  \item Schneider, T., Alexander, D. C., \& Wheeler-Kingshott, C. A. M. (2009). Preliminary Investigation of Position Dependency of Radial Diffusivity in the Cervical Spinal Cord. \emph{17th Scientific Meeting of the International Society for Magnetic Resonance in Medicine}  
%\end{itemize}
%
%\subsection*{Invited talks}
%\begin{itemize}
%  \item ``Optimising Diffusion Pulse Sequences for Investigation of Spinal Cord Microstructure'', Departmental Seminar, Danish Research Centre for Magnetic Resonance, Copenhagen, August 2009
%  \item ``Spinal Cord Diffusion MRI'', CMIC Seminar, Centre for Medical Image Computing, UCL, London, January 2010
%  \item ``Imaging microstructure in the spinal cord with diffusion MRI'', Imaging \& Biophysics Unit Seminar Series, Institute of Child Health, UCL, London, November 2010   
%\end{itemize}


 
 
