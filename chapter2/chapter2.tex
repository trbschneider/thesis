%!TEX root = ../thesis.tex
\newcommand{\prot}{\ensuremath{\mathcal{P}}}
\label{chapter2}
\section{Anatomy of the spinal cord}
\glsresetall
This section will give a brief overview of the macroscopic and microscopic organisation of the \gls{SC} (see \citep{Carpenter:1991,Heimer:1995,Watson:2009} for more details).


The {\gls{SC}} is the part of the {\gls{CNS}} that connects the brain and peripheral nervous system. It controls the voluntary movement of limbs and trunk, receives sensory information from these regions and monitors and coordinates the internal organ function in the thorax, abdomen and pelvis. 


The \gls{SC} is divided into different segments, according to the surrounding vertebrae. The cervical cord is divided in 7 segments (C1--C7), followed by the thoracical, lumbar and sacral cord segments \citep{Heimer:1995,Watson:2009}.


The {\gls{SC}} is protected by the vertebral column and is located inside the vertebral canal. In cross-section, the cord can be divided in two regions: (i) the peripheral region containing neuronal white matter tracts, and (ii) the grey, butterfly-shaped central region made up of nerve cell bodies (see Figure~\ref{fig:chapter 2 spinal_cord_anatomy}). This \gls{GM} is centred around the central canal, containing \gls{CSF}.

\subsection{Organisation of the spinal cord}
The \gls{WM} of the {\gls{SC}} consists mostly of longitudinally oriented axons and glial cells. White matter axons are organized hierarchically grouped in bundles, tracts and pathways. Bundles of neighbouring white matter axons that share similar features are called fibre bundles. A tract is formed by fibre bundles with same origin, course, termination and function. Multiple tracts with the same function form a pathway.

\subsubsection{Ascending tracts}
\label{sec:chap2:ascendingtracts}
Figure \ref{fig:chapter 2 spinal_cord_anatomy} illustrates the location of the major ascending pathways in the {\gls{SC}}. These sensory tracts, arise either from cells of spinal ganglia in the white matter of the {\gls{SC}} or from intrinsic neurons within the gray matter that receive primary sensory input. The dorsal column holds the largest ascending tracts and are associated with tactile, pressure, and kinesthetic sense connecting with sensory areas of the cerebral cortex. Fibres of the spinothalamic tracts ascend in the lateral ventral part of the cord and convey signals related to pain and thermal sense. The anterior spinothalamic tract ascends more anteriorly in the {\gls{SC}}; conveying impulses related to light touch. At brain level the two spinothalamic tracts tend to merge and cannot be distinguished as separate entities. Anterior and posterior spinocerebellar tracts are involved in automatic muscle tone regulation. These tracts ascend peripherally in the dorsal and ventral margins of the cord.

\subsubsection{Descending tracts}
\label{sec:chap2:descendingtracts}
Tracts descending to the {\gls{SC}} as illustrated in Figure~\ref{fig:chapter 2 spinal_cord_anatomy} are concerned with modulation of ascending sensory signals and are associated with voluntary motor function such as muscle tone and reflexes. The largest and most important, the {\gls{CST}}, originates in broad regions of the cerebral cortex and descends in the lateral dorsal part of the {\gls{SC}} white matter. Smaller descending tracts like the rubrospinal tract, the vestibulospinal tract, and the reticulospinal tract originate in small and diffuse regions of the midbrain, pons, and medulla and descend ventrally and laterally.
\begin{figure}
 \centering
  \pgfimage[width=10cm]{chapter2/figs/spinalcordtracts.pdf}
  \caption{Illustration of the major ascending and descending fibre pathways of the {\protect\gls{SC}} (adapted from \url{http://en.wikipedia.org/wiki/Spinal_cord}).}
  \label{fig:chapter 2 spinal_cord_anatomy}
\end{figure}

\subsubsection{Microstructural organisation}
A neural cell, or Neuron, possesses a cell body and two typically two types of extending structures: axons and dendrites. The dendrites carry afferent signals to the cell, while axons relay efferent signals. the typical size of the axon lies in the range of 1--10$\mu m$ \citep{Waxman:1989,Beaulieu:2002}. Larger axons are usually surrounded by layers of myelin, which acts as an isolator for the electrical transmission of signals and allows for higher transmission speeds than unmyelinated axons. The structure of the axon is supported by longitudinal micro-filaments\citep{Beaulieu:2002}.

In the cord, the gray matter comprises the neuronal cell bodies and dendritic structures while the white matter mainly holds the axonal fibre bundles. The majority of white matter fibre bundles run parallel to the long axis of the cord. Peripheral nerves enter the spinal cord through the neuroanatomy, connecting to the gray matter\citep{Carpenter:1991}. 

\section{Principles of MRI}
\Gls{MRI} is a non-invasive imaging method widely used in medicine. \gls{MRI} is free of gamma-radiation (unlike e.g. X-ray methods), which makes it one of the major tools for application in neuroimaging. \Gls{MRI} can describe tissue in terms of many different properties such as relaxation, density, and diffusion. Specifically, in this thesis our main interest is in the sensitivity of MRI to the motion of water molecules to infer information about the microscopic tissue morphology. A full account of \gls{MRI} theory is beyond the scope of this chapter and can be found elsewhere \citep{McRobbie:2002,Bernstein:2004}. However, a brief overview about the principles of magnetic resonance and \gls{MRI} is given below.

\subsection{Magnetic resonance}
Nuclear magnetic resonance is a phenomenon that occurs when an element with a non-zero magnetic moment (possessing an odd number of protons or neutrons) interacts with an external magnetic field. Hydrogen ($^1$H) is such an element, and is most commonly used in MRI due to its abundance in the human body. When such an element is placed in a magnetic field, its nuclear spin will begin to precess with a frequency governed by the equation:
\begin{equation}
\omega =\gamma \cdot \mathbf{B}_0 
\end{equation}
where $\omega$ is the Larmor frequency, $\gamma$ is the nucleus specific gyromagnetic ratio, and $\mathbf{B}_0$ is the magnetic field strength. In equilibrium, the nuclear spins rotate around the axis of the magnetic field $\mathbf{B}_0$ and their magnetic moment is aligned with the $\mathbf{B}_0$ field. 

When a \gls{RF} pulse is applied perpendicular to the $\mathbf{B}_0$ field, with a frequency equal to the Larmor frequency (i.e. the resonance frequency) the magnetic proton spins tilt towards the transverse plane and precess around the axis of the $\mathbf{B}_0$ field. The precession induces a change of flux in the magnetic field of the receiver coil and produces a measurable signal. It is often present the nuclear spins in a reference frame rotating at the Larmor frequency $\omega$ about the $\mathbf{B}_0$ axis. In this rotating frame of reference, the bulk magnetisation is stationary and the \gls{RF} pulse results in a tip from its equilibrium position toward the transverse plane by the angle $\alpha$. A 90$^\circ$ pulse flips the magnetisation into the transverse plane and a 180$^\circ$ pulse inverts the bulk magnetisation.

Immediately after the excitation, the spins are completely in phase (coherent). After excitation follows a relaxation process, which is accompanied by a loss of phase coherence and a subsequent relaxation back to equilibrium state. The signal induces in the receiver coil after excitation is referred to as the \gls{FID} signal. Figure~\ref{fig:chapter2 spin FIDs} illustrates the excitation and relaxation process. 

\begin{figure}[H]
\centering
\pgfimage[width=0.9\textwidth]{chapter2/figs/spins.pdf}
\caption{Simplified illustration of spins during different steps of the FID signal formation after a 90$^\circ$ RF pulse is applied. Arrows represent the net magnetisation vectors of spin ensembles in the rotating frame of reference. Some figures were created using the SpinBench software \citep{Overall:2007}.}
\label{fig:chapter2 spin FIDs}
\end{figure}

Together with the density of nuclear spins $M_0$, the relevant time constants T1 and T2/T2$^*$ characterise the relaxation phenomena and are the principal source of contrast used in \gls{MRI}. In the case of a 90$^\circ$ excitation pulse, these time constants are defined as below:
\begin{description}
	\item[T1:] is the longitudinal relaxation time, which describes how long it takes for the net magnetisation to return to the longitudinal equilibrium. Formally, the T1 constant relates to the longitudinal component of the magnetisation $M_z$ at time $t$ after 90$^\circ$ excitation by the formula:
	\begin{equation}
		M_z(t) = M_0 \cdot (1-\exp(-t/T1)),
	\end{equation} 
	with $M_0$ (the proton density) being the total magnetisation, which is proportional to the total number of excited spins, 
	\item[T2:] is the transverse relaxation time, i.e. it describes the time it takes for the FID signal to decay due to randomly fluctuating internal magnetic fields caused by spin-spin interactions in the substance. This causes the spins to get out of phase and the transverse magnetization (and induced signal) is lost exponentially. Formally, the signal development of the transverse magnetisation $M_{xy}(t)$ at time $t$ after 90$^\circ$ excitation is described by:
	\begin{equation}
		M_{xy}(t) = M_0 \cdot \exp(-t/T2),
	\end{equation} 	
	In a non-ideal magnetic field, transverse magnetisation is also lost due to inhomogeneities in the $\mathbf{B}_0$ field, causing additional signal loss. In this case we distinguish between the T2 effect as the spin-spin interactions alone, and the T2$^*$ effect, as the signal loss due both spin-spin relaxation and $\mathbf{B}_0$ inhomogeneities.
\end{description}

The transverse magnetization decays more rapidly than it takes for the magnetisation to return to the longitudinal equilibrium. Both T1 and T2 are dependent on the magnetic field strength, but more importantly they are also specific to the macromolecular environment of the protons and therefore are specific for different types of tissue, e.g. for different tissue types within the live human brain (\gls{GM} T1/T2 = 2000/100 ms, \gls{WM} T1/T2 = 1100/70 ms at 3T magnetic field strength \citep{Stanisz:2005}).
\subsection{Spin-echo sequence}
\label{sec:chapter2 spin echo}
The {\gls{SE}} sequence is the central pulse sequence that is used in all experiments we present in this dissertation. Figure \ref{fig:chap2 SE sequence} illustrates the layout and signal development of the \gls{SE} experiment. The \gls{SE} sequence starts with a 90$^\circ$(P90) excitation pulse that flips magnetization in the transverse plane, followed by a 180$^\circ$RF pulse (P180) after time TE/2 and the signal readout after another TE/2, producing an echo at time TE. The P180 inversion pulse will reverse the demagnetization by field inhomogeneities so that the contrast is mainly driven by spin-spin relaxation constant T2 and the proton density $M_0$. When TR is sufficiently large for the transverse magnetisation to restore, the obtained signal is only dependent proton density $M_0$ and on T2. When TE is chosen appropriately (i.e. usually in the range of tens of milliseconds for in-vivo head or  cord images), the T2 signal decay is the main source of contrast the signal is called \gls{T2w}.

\begin{figure}[ht]
\centering
\pgfimage[width=0.9\textwidth]{chapter2/figs/spinecho.pdf}
\caption{Simplified illustration of spins during different steps of the FID signal formation after a 90$^\circ$ RF pulse is applied. Some figures were created using the SpinBench software \citep{Overall:2007}.}
\label{fig:chap2 SE sequence}
\end{figure}


\subsection{Gradients and Image formation}
A magnetic gradient field $\textbf{G}$ is a small spatially varying magnetic field, which is superimposed on the static magnetic field $\mathbf{B}_{0}$ and alters the spin frequency at a given position $x$ as follows:
\begin{equation}
\omega(x)= \omega_0 + \gamma \cdot \textbf{G}(x)
\end{equation}

Gradient fields are fundamental to many aspects of MR, e.g., to generate a signal response (the so-called gradient echo ) or to spatially encode the signal to allow the formation of an image, on which we concentrate here.  Specifically we discuss here the 2D slice encoded SE imaging sequence, which combines the principles of spin-echo formation as demonstrated in Section~\ref{sec:chapter2 spin echo} with spatial encoding gradients.

Figure~\ref{fig:chapter2 SE imaging sequence} illustrates such a simple imaging SE pulse imaging sequence. First, the slice selection gradient $\textbf{G}_{slice}$ is applied during the excitation RF pulse, which results in only the excitation of protons that precess with frequencies within the range of the excitation RF pulse. 

The two gradients $\textbf{G}_{read}$ and $\textbf{G}_{phase}$ are orthogonal to $\textbf{G}_{slice}$ and provide the spatial encoding within the excited slice. The phase encoding gradient $\textbf{G}_{phase}$ adds a phase shift to the spin frequency of the slice-selected magnetization, which encodes location in the  direction. During the spin echo at TE, the $\textbf{G}_{read}$ gradient is applied, making the resonant frequency of the nuclear magnetization vary with its location in the read-out direction. The signal is sampled $k_x$ times (typically between 128 and 512 $k_x$ samples are taken).
After waiting the \gls{TR} for the longitudinal magnetisation to restore, the whole sequence is repeated with a different phase-encoding gradient. After $k_y$ phase encoding steps, the object of interest is completely spatially encoded by means of frequency and phase. Typically also between 128 and 512 $k_y$ encoding steps are acquired. This 2-dimensional $k_x\times k_y$ frequency matrix is then reconstructed into an image using the Fourier Transform \citep{Ljunggren:1983,Twieg:1983}.
\begin{figure}[ht]
\centering
\pgfimage[width=0.95\textwidth]{chapter2/figs/imaging.pdf}
\caption{Spatial encoding by different gradient pulses during a 2D spin echo sequence.}
\label{fig:chapter2 SE imaging sequence}
\end{figure}

In the sequence described above, each line of k-space is acquired individually, which makes the scan time per image impractical for larger k-space matrices. In clinical reality, the image preparation is often combined with fast imaging techniques that allow to several lines of k-space at once\ref{XX,XX}. In clinical diffusion MRI, \gls{EPI} is predominantly used for acquisition, as it allows to acquire the full k-space matrix in one \lq shot \rq. 

\subsection{Small field of view imaging}
Imaging the spinal cord presents many challenges, mainly due to its small size and surrounding tissue like \gls{CSF} and bone. On the other hand, to image the cord below the the neck, conventional imaging methods require a rather large \gls{FOV} to avoid aliasing artefacts \citep{Bernstein:2004,McRobbie:2002}. Small \gls{FOV} methods allow to image a smaller volume and minimise the aliasing artefacts by avoiding to encode the surrounding tissue\citep{Feinberg:1985}. Such methods are well suited for the application to \gls{SC}, as they reduce the necessary encoding steps and allow for the high spatial resolution that is required for imaging of the cord.  


In this thesis we use a small \gls{FOV} modification of the \gls{SE} sequence called \gls{ZOOM} imaging, which was introduced by \citet{Wheeler-Kingshott:2002,Wheeler-Kingshott:2002a}. The central idea is to perform the encoding gradients for the excitation pulse P90 and refocussing pulse P180 at an angle $\alpha$ instead of parallel as in conventional SE. This way, only spins within the intersection of P90 and P180 are refocussed, effectively suppressing the unwanted signal outside the small \gls{FOV}. A disadvantage of the technique is that, dependent on the applied angle $\alpha$ between P90 and P180, spins within the adjacent region of the \gls{FOV} are also excited, which makes the continuous acquisition of adjacent slices within one \gls{TR} difficult.


\section{Principles of Diffusion MRI}
Diffusion MRI captures the average diffusion of water molecules, which probes the structure of the biological tissue at scales much smaller than the imaging resolution. The diffusion of water molecules is Brownian under normal unhindered conditions, but in fibrous structures, such as white matter, water molecules tend to diffuse preferably along the fibers. Due to this physical phenomenon, diffusion MRI is able to obtain information about the neural architecture \textit{in-vivo}. In the following section we will briefly review the principles of diffusion and its effect on the MRI signal.

\subsection{Brownian motion}
At a microscopic scale, water molecules freely move and collide with each other in a homogeneous medium according to Brownian motion \citep{Brown:1828}. At a macroscopic scale, this phenomenon yields a diffusion process. In the simplest case of pure molecular motion in the absence of any impeding barriers, the diffusion process can simply be characterised by the diffusion coefficient $d$ \citep{Fick:1855}. In an isotropic and homogeneous medium, the mean displacement after a given time $t$ is simply related to the diffusion coefficient $d$ by Einstein's formula, which in 3-d space, is: 
\begin{equation}
	 d = 6 \cdot \langle R^2 \rangle \cdot t
	\label{eq:chapter2 einsteins formula}
\end{equation}
where, $\langle\dots\rangle$ denotes the ensemble average and $R = r - r_0$ is the displacement between the original position $r_0$ of a particle and the position $r$ after the diffusion time $t$. 

\subsection[Types of diffusion]{Free, hindered and restricted diffusion in biological tissue}
In the simplest case, free diffusion (or unrestricted diffusion) is exactly described by the pure Brownian motion of water, i.e. molecules diffusing freely in all directions in the absence of any boundaries. In reality, free diffusion is rarely encountered in a biological tissue sample. Instead, the presence of restricting barriers, such as cell walls, membranes or axonal myelin sheaths impede the motion of the water molecules and alters their displacement pattern. In this case, the diffusion pattern is not only influenced by the diffusivity of the medium but more importantly informs about the characteristics of the surrounding environment on the scale of the mean displacement. 

The observed effects on the diffusion MR signal can be quite diverse, depending on the type and location of barriers within the sample. Figure~\ref{fig:chapter 2 types of diffusion} illustrates different diffusion environments and their effect on the \gls{RMSD} of molecules. It is helpful to further distinguish two different motion patterns in the presence of barriers as restricted and hindered diffusion.  Restricted diffusion is seen when the movement of water molecules is confined in closed spaces, such as impermeable cells wall. Those molecules experience restricted diffusion in that the molecules cannot displace farther than the confines of the cell. In hindered diffusion, the water movement of molecules is impeded however not confined within a limited space. Hindered diffusion best describes water motion in the space between densely packed cells or axons. The aim of diffusion MRI is to characterise the diffusion motion and thus infer some characteristics of the tissue non-invasively.


\begin{figure}[ht]
 \centering
  \subfloat[Illustration of molecular movement in different tissue environments]
  {
	\pgfimage[height=0.4\textwidth]{chapter2/figs/diff_types.pdf}
  }\hspace*{0.01\textwidth}
  \subfloat[Root-mean-squared displacement over prolonged diffusion time for different tissue environments.]
  {
	\pgfimage[height=0.4\textwidth]{chapter2/figs/rms_types.pdf}
  }
  \caption{Free, hindered and restricted,  diffusion patterns and root-mean-squared displacement over different diffusion times.}
  \label{fig:chapter 2 types of diffusion}
\end{figure}

\subsection{The Stejskal-Tanner PGSE experiment}
\begin{figure}[ht]
\centering
\pgfimage[width=0.6\textwidth]{chapter2/figs/PGSEdiagram.pdf}
\caption{Pulse sequence diagram of PGSE sequence. Image encoding gradients are omitted for clarity.}
\label{fig:chapter2 pgse_diagram}
\end{figure}

The MRI signal can be made sensitive to the movement of the water molecules within the tissue, providing contrast about their molecular displacement on a sub-voxel scale. By far the most commonly used method for diffusion MRI is the {\gls{PGSE}} sequence, introduced by \citet{Stejskal:1965}. The {\gls{PGSE}} sequence, as shown in Figure \ref{fig:chapter2 pgse_diagram}, is based on the standard SE sequence with an additional pair of identical diffusion weighting gradients, which make the sequence sensitive to the diffusion of water molecules. 


\begin{figure}[phtb]
\centering
\subfloat[Spin phase distribution in case of no molecular motion. The phase dispersion introduced by the first diffusion gradient is completely reversed by the second diffusion gradient.]
{
	\pgfimage[width=0.85\textwidth]{chapter2/figs/PGSEnomotion.pdf}
	\label{fig:chap2 diffusion illustration nodiffusion}
}\\
\subfloat[Spin phase distribution in case of diffusing molecules during the diffusion time $\Delta$. Because of motion, the individual molecules experience different phase offsets at the first and second diffusion gradients. As a result, there remains some phase incoherence after the second diffusion gradient, which culminates into an attenuation of the total spin echo response.]
{
	\pgfimage[width=0.85\textwidth]{chapter2/figs/PGSEmotion.pdf}
	\label{fig:chap2 diffusion illustration diffusion}
}
\caption{Cartoon of the principle of diffusion encoding in the PGSE experiment. The diagrams present the spin development over the course of the sequence in the case of: (a) no diffusion or (b) diffusing molecules.}
\label{fig:chap2 diffusion illustration}
\end{figure}


Figure~\ref{fig:chap2 diffusion illustration} illustrates the principle of diffusion encoding using the PGSE sequence. The first diffusion gradient adds a phase offset dependent on each molecule's position. If the molecule's position doesn't change, the second diffusion gradient will reverse the phase offset (illustrated in Figure~\ref{fig:chap2 diffusion illustration nodiffusion}). However, in the case of motion due to diffusion, the individual positions will differ between the first and second diffusion gradient, resulting in a reduced signal amplitude	(illustrated in Figure~\ref{fig:chap2 diffusion illustration diffusion}). The degree of signal loss is dependent on the rate of diffusion in the tissue but is also controlled by the parameters of the {\gls{PGSE}} sequence, namely:

\begin{itemize}
	\item the {\gls{gstr}} and {\gls{gdir}},
	\item the {\gls{smalldel}},
	\item the {\gls{bigdel}} between both gradient pulses.
\end{itemize}

In the literature the combination of those PGSE parameters is often summarised in terms of the diffusion weighting factor $b$-value \citep{LeBihan:1986}, which is defined as:
\begin{equation}
	b = \gamma^2|G|^2\delta^2(\Delta-\frac{\delta}{3}),
    \label{eq:bvalue}
\end{equation}
where $\gamma$ is the gyromagnetic ratio. The theoretical background of the b-value formula will be explained in Section~\ref{sec:chapter 2GPD}.


\paragraph{Eddy current distortions: } 
A common problem in diffusion imaging are image distortions caused by eddy currents that are induced in the gradient system by switching the strong diffusion encoding gradients. When the diffusion gradient pulses are switched on and off, the induced eddy currents set up magnetic field gradients that may persist after the primary gradients are switched off. Such residual gradient fields can combine with the imaging gradient pulses such that the actual gradients experienced in the imaged objects are not exactly the same as those that were programmed to encode the image. If this error is not taken into account, it produces geometric distortion in the final reconstructed images. 

A possible solution to eddy current artefacts is to modify the standard single spin-echo pulse sequence to nullify eddy current effects. Most common is the the so-called \gls{dPGSE} sequence\citep{Reese:2003} (shown in Figure~\ref{fig:chapter2 drpgse_diagram}), which relies on replacing the mono-polar gradients in the single PGSE sequence by two pairs of gradients with opposing polarity. Providing the duration of the eddy current fields is much longer than the duration of the gradients, the opposite eddy current fields tend to cancel out. By applying diffusion gradients with a shorter duration, the cancellation of long-term eddy currents can be achieved, enabling the minimization of the eddy currents on the image encoding.

The \gls{dPGSE} is arguably one of the most effective way of minimising eddy currents at the time of signal generation. However, the \gls{dPGSE} can not always be used as not all MRI scanner manufacturers supply it by default. Further concerns include potential signal-to-noise loss, and more severe $B_1$ inhomogeneity effects due to the presence of an extra 180$^\circ$ \gls{RF}-pulse. As an alternative, the diffusion weighted images can be corrected for eddy current distortions retrospectively, e.g. by employing affine image registration to align all \gls{DWI} data, using a non-diffusion weighted image as the target\citep{Rohde:2003}.


\begin{figure}[ht]
\centering
\pgfimage[width=0.6\textwidth]{chapter2/figs/DR_PGSE.pdf}
\caption{Diagram of the double refocussed PGSE sequence. Image encoding gradients are omitted for clarity.}
\label{fig:chapter2 drpgse_diagram}
\end{figure}

\section{Analysis of Diffusion MRI Data}
Like T1- and T2-weighted MRI, diffusion MRI can be used as qualitative imaging method, e.g., it is widely used in the early diagnosis of stroke \ref{XX}. Moreover, diffusion MRI has also proven to be a very powerful tool in the quantitative assessment of tissue properties through parameter maps. However, this type of analysis demands a systematic approach toward the acquisition of signal samples, especially in view of the inherently low SNR and large number of acquisition protocol parameters of diffusion MRI.


Most commonly, diffusion MRI is processed in terms of a model-based analysis, i.e. using a mathematical description of the diffusion signal that can be referred back to the tissue properties. We can break down the model-based analysis pipeline into its main building blocks: 
\paragraph{Acquisition:} The set of actual diffusion MR measurements. Any quantitative analysis of the diffusion MRI signal usually needs many samples of different PGSE parameters, e.g. many different gradient encoding directions and/or \gls{gstr},\gls{smalldel}, \gls{bigdel} combinations. We formally define such a combined set of $n$ singular PGSE acquisitions as a protocol (\prot):
\begin{equation}
	\mathcal{P} = \{(\vec{g}_1,|G|_1,\delta_1,\Delta_1),\cdots,(\vec{g}_n,|G|_n,\delta_n,\Delta_n)\},
\end{equation}
or alternatively using the shortcut term $b$ as defined in Equation \ref{eq:bvalue}:
\begin{equation*}		
	\mathcal{P} = \{(\vec{g}_1,b_1),\cdots,(\vec{g}_n,b_n)\}.
\end{equation*}
Several other terms are are found in the literature that to describe certain aspects of an acquisition protocol. We summarise the most commonly used terms in the following paragraph. 


A diffusion gradient scheme usually describes a set of diffusion gradient directions only without specifying PGSE pulse parameters or $b$-values. The term \gls{HARDI} describes a special case of gradient scheme with a high number diffusion directions (>60), which are uniformly sampled over the unit sphere\citep{Tuch:2002,Tournier:2011}. A \textit{shell} in the context of diffusion MRI refers to a protocol or subset of a protocol with several different gradient directions acquired at the same $b$-value. 


Different analysis methods have different requirements on the acquisition protocol. While it suffices for some methods to acquire few samples of the PGSE parameter space, other methods require one or more HARDI shells with different b-values and/or many different $(\vec{g},|G|,\delta,\Delta)$ combinations. 



\paragraph{Diffusion model:} The diffusion model is a mathematical approximation of the diffusion process. The diffusion model is usually controlled by a set of feature parameters $\Phi$, which can be (directly or indirectly) related back to the sample environment of the diffusion process. The diffusion model is usually associated closely with a mathematical formulation $S(\Phi;\prot_i)$ of the predicted diffusion MR signal for a given acquisition $\prot_i \in \prot$ and set of diffusion model parameters $\Phi$.
\paragraph{Fitting:} The fitting procedure links the observed signals from the acquisition to the diffusion model. The aim is to infer properties from the acquired data, which ideally provide insight in certain characteristics of interest of the underlying sample. In most cases, a forward-modelling approach is applied, i.e., the acquired signal is fitted via a signal model $S(\Phi;\prot_i)$ that has been determined \emph{a-priori} to find the particular $\Phi$ that explains the acquired data best.  
\paragraph{}
In the remainder of this section we will discuss some of the most common models and analysis methods, with particular focus on the techniques that were used in this dissertation. 
\subsection{Short gradient approximation and the q-space formalism}
If we assume the diffusion gradient pulse $\delta$ to be sufficiently short, multiple times smaller than the diffusion time $\Delta$, any motion of water molecules during the diffusion encoding gradient time can be neglected. In the so-called \gls{SGP} regime ($\delta \ll \Delta$), the diffusion echo attenuation $S$ for a specific PGSE acquisition can be expressed as the integral of the net phase shifts over all water over all molecule positions ($r$) weighted by the conditional probability $P(r|r')$ of the molecule's movement from position $r$ to $r'$\citep{Callaghan:1991}:
\begin{equation}
	S(|G|,\delta,\Delta)=\iint P(r)P(r|r',\Delta)\exp[-i\cdot \gamma \delta |G|\cdot (r'-r))] dr'dr.
	\label{eq:chapter2 signal in sgp}
\end{equation}
We can now describe ensemble molecule motion pattern over one voxel by the average \gls{dPDF} (often referred to as the average propagator\citep{Kaerger:1983}) as the average probability of all particles moving the distance $R$ independent of their starting position:
\begin{equation}
	\overline{P}(R,t)=\int P(r)P(r|r+R,t)dr.
	\label{eq:chapter2 dpdf}
\end{equation}

When Equation~\ref{eq:chapter2 dpdf} is substituted in the signal Equation~\ref{eq:chapter2 signal in sgp}, we obtain:
\begin{equation}
		S(|G|,\delta,\Delta)=\int \overline{P}(R,\Delta) \exp[-i\cdot \gamma \delta |G|\cdot R] dR,
\end{equation}
If we further introduce the $\textbf{q}$-value (or wavenumber) as
 \begin{equation}
\textbf{q}=\frac{\gamma \textbf{G}\delta}{2\pi},
\label{eq: chapter 2 q value definition}
\end{equation}
the signal equation can be written as:
\begin{equation}
		S(q,\Delta)=\int \overline{P}(R,\Delta) \exp[2\pi i \cdot q\cdot R] dR.
\label{eq:chapter 2 qspace formula}
\end{equation}
It is easy to see that Equation~\ref{eq:chapter 2 qspace formula} presents a simple Fourier relationship between the signal $S$ and the \gls{dPDF}. This relationship can be exploited in q-space analysis, where the diffusion signal is measured with many different q-values at a certain fixed diffusion time. The inverse Fourier transformation of the measured signal directly gives the \gls{dPDF} at a fixed diffusion time $\Delta$ without the need to impose any constraints on its shape.

\subsection{Q-space imaging}
\label{sec:qspace}
The combination of q-space analysis with MR imaging methods is called \gls{QSI}\citep{Callaghan:1991,Assaf:2000}. \Gls{QSI} provides the full displacement probability profile in each voxel of the imaged volume. However, the visualization and interpretation of the full displacement profile in each voxel is complicated and therefore impracticable for clinical application. Instead, it is more common to derive summary statistics from the \gls{dPDF} that describe specific features of the displacement profile. The most widely used parameters are: 
\begin{itemize}
\item zero displacement probability (P0)
\item full width of half maximum (FWHM)
\item kurtosis (K)
\end{itemize}
Figure~\ref{fig:chapter 2 QSI analysis} illustrates the QSI analysis performed steps and gives examples of P0, FWHM and K parameter maps in the spinal cord.
\begin{figure}[htbp]
 \centering
 \pgfimage[width=0.99\textwidth]{chapter2/figs/qsi_processing.pdf}
 \caption{QSI analysis pipeline and example parameter maps.}
 \label{fig:chapter 2 QSI analysis}
\end{figure}
\paragraph{}
The P0 and FWHM parameter describe the height and width of the displacement profile. Generally, high P0 and low FWHM can be interpreted as indicators of increased impedance of diffusive motion; low P0 and wide FWHM are related to more free (or less hindered) diffusion. The FWHM is of particular theoretical interest as it can be directly related to the size of the restricted compartment in simple geometries via the autocorrelation function when diffusion is completely restricted \citep{Cory:1990,Kuchel:1997}. Sometimes the \gls{RMSD} of Einstein's formula (see Equation~\ref{eq:chapter2 einsteins formula}) is reported instead of FWHM. A simple conversion factor between FWHM and RMSD was suggested by \citet{Cory:1990} as:
\begin{equation}
	\mbox{RMSD} = 1.443 \cdot \mbox{FWHM},
\end{equation}
although the equality is only true if the diffusion profile is truly Gaussian, but not for restricted diffusion.
\paragraph{}
The kurtosis parameter, here defined as the excess kurtosis \citep{Kenney:1957}, describes how much a distribution differs from the normal distribution. Kurtosis is defined as the standardised fourth central moment of a distribution minus 3 (to make the kurtosis of the normal distribution equal to zero). For a finite sample of n datapoints the kurtosis K is computed as:
\begin{equation}
	K=\frac{\tfrac{1}{n} \sum_{i=1}^n (x_i - \overline{x})^4}{\left(\tfrac{1}{n} \sum_{i=1}^n (x_i - \overline{x})^2\right)^2} - 3 
\end{equation}
with $\bar{x}$ being the sample mean. A high kurtosis distribution has a narrower peak and long, fat tail compared to a normal distribution. A low kurtosis distribution has a more rounded peak and a shorter, thinner tail. In the context of diffusion analysis, the kurtosis parameter can be used to quantify how much the \gls{dPDF} differs from a Gaussian displacement distribution \citep{Jensen:2010}. High K values can therefore be interpreted as an indicator of restricted diffusion in a sample.

\subsubsection{Limitations of QSI}
\Gls{QSI} parameters measured in nervous tissue are often interpreted as a direct indicator of axonal architecture, such as the \gls{MAD}.  Early studies have demonstrated that q-space analysis can indeed provide exact estimates of the geometry in simple samples, e.g. yeast cells \citep{Cory:1990} or blood cells \citep{Kuchel:1997}. However, experiments on real nervous tissue have shown that the interpretation of q-space parameters in axonal tissue is more complicated \citep{King:1994,Assaf:2000,Assaf:2000a,Bar-Shir:2008}. \citet{Assaf:2000} suggested that the displacement profile of nervous tissue can be expressed as a combination of at least two compartments exhibiting hindered and restricted diffusion. A recent study of QSI in the \emph{in-vivo} human brain by \citet{Nilsson:2009} confirmed that the FWHM perpendicular to white matter fibres did not change with diffusion time, while parallel FWHM increased linearly with the square root of diffusion time. This suggests the presence of restricted diffusion across \gls{WM} tracts and non-restricted diffusion along WM tracts respectively (see Figure~\ref{fig:chapter 2 types of diffusion}). It is sometimes assumed that hindered and restricted diffusion correspond to two different compartments: \gls{IC} and \gls{EC} water, although there is an ongoing debate over the interpretation of these results (see e.g. \citep{Kiselev:2007, Mulkern:2009}).


Since q-space analysis provides the average displacement probability over the whole voxel, the q-space measurement is affected by both IC and EC compartments as well as by the amount of exchange between the two. As a result, the \gls{dPDF} may be broader than the actual \gls{MAD} would suggest, due to the addition of displacements from hindered diffusion in the EC compartments. Other factors such as the distribution of sizes and variety of shapes further complicate the interpretation of q-space parameters to infer the real axon diameter distributions. 


\subsection{Apparent diffusion coefficient}
\label{subsec:adc}
In the absence of any diffusion impeding barriers, the \gls{dPDF} takes the form of a simple Gaussian probability distribution, which is only dependent on the diffusion time $t$ and the diffusion coefficient $d$:
\begin{equation}
P(\textbf{r}_{0},\textbf{r},\Delta) =  \frac{1}{\sqrt{(4\pi dt)^3}}\exp\bigg(-\frac{|\textbf{r}-\textbf{r}_{0}|^{2}}{4dt}\bigg).
\label{Gaussian PDF}
\end{equation}

This closed form solution for the \gls{dPDF} can be substituted in the general q-space formalism given in Equation~\ref{eq:chapter 2 qspace formula}, simplifying it to:
\begin{equation}
	S(s_0,d;\delta,\Delta,G) = s_0 \cdot \exp(-(2\pi\gamma\delta)^2\Delta \cdot d),
\end{equation}
with model parameters being the diffusion coefficient $d$ and the baseline signal $s_0$, i.e., the non-diffusion weighted T2w signal. It is often more convenient to rewrite above equation terms of the b-value as: 
\begin{equation}
	S(s_0,d;b) = s_0 \cdot \exp(-b \cdot d),
\end{equation}
with $b  \approx -(2\pi\gamma\delta)^2\Delta$ under the \gls{SGP} assumption of $\delta \ll \Delta$.
\paragraph{}
In true free diffusion, $d$ is simply the diffusion coefficient of the medium and the signal equation above is exact. However, in real biological tissue, virtually all molecules will have interacted with their environment within the timescale of a typical diffusion MR experiment. In this case the above expression is just an approximation of the underlying true \gls{dPDF} and $d$ above is not only related to the diffusivity of the medium but also informs about the diffusion impedance caused by molecules interacting with the environment. To highlight the difference to the classical definition of the diffusion coefficient, we refer to $d$ as the \gls{ADC}.

The model parameters $s_0$ and the \gls{ADC} are tissue dependent and can be estimated by acquiring a minimum of two diffusion weighted images with different $b$-value (usually $b=0$ and $b=800-1200mm/s^2$ for \emph{in-vivo} nervous tissue). Typically a simple log-transformation of Equation~\ref{eq:chapter 2 adc} is used to obtain a linear equation:
\begin{equation}
	log(S(s_0,ADC;\prot_i)) = log(s_{0}) - (b\cdot ADC),
    \label{eq:chapter 2 adc}
\end{equation}
for each measurement $\prot_i$ of the acquisition protocol \prot{}. The linear equation system can then be solved efficiently, e.g. using a least squares approach, to obtain maps of $s_0$ and $ADC$ values.

\subsection{Diffusion Tensor}
\label{subsec:dti}
In ordered tissue like white matter diffusion is directional, i.e., the \gls{ADC} will depend on the direction {\gls{gdir}} of the applied gradient. To reflect the directionality, Equation~\ref{eq:chapter 2 adc} can be extended from the scalar representation of the diffusion coefficient $d$ to reflect the complete 3-dimensional diffusion co-variance matrix \citep{Basser:1994}, obtaining the {\gls{DT}} formulation:
\begin{equation}
	S(\mat{D};b,\vec{g}) = S_{0}\exp(-b\vec{g}^T \mat{D}\vec{g}) \mbox{ with } \mat{D} = 
	\left[
	\begin{array}{ccc}
	d_{xx} & d_{xy} & d_{xz} \\
	{\color{gray} d_{xy}} & d_{yy} & d_{yz} \\
	{\color{gray} d_{xz}} & {\color{gray} d_{yz}} & d_{zz} 	
	\end{array} \right].	
    \label{eq:dti}
\end{equation}
As before, the parameters of the {\gls{DT}} model are the $s_0$ non-diffusion weighted signal baseline and the diffusivity $d$, now being a positive symmetric $3\times3$ co-variance matrix. The parameters can be estimated in a similar fashion to the ADC model using the log-transformation of the signal and a system of linear equations. In addition to the ADC model, the accurate estimation of all the directional {\gls{DT}} components requires a minimum of 6 different diffusion weighted measurements with non-coplanar gradient directions. However, we usually acquire more signals to overdetermine the solution, add noise control and increase directional resolution \citep{Jones:2004a}.


By an Eigen decomposition of the {\gls{DT}} we obtain the three eigenvectors $\vec{v}_1, \vec{v}_3, \vec{v}_3$ and their corresponding eigenvalues $\lambda_1\ge\lambda_2\ge\lambda_3$. The first eigenvector can be interpreted as the principal diffusion directions with $\lambda_1$ being the principal diffusivity. Usually $\lambda_1$ is also referred to as the {\gls{AD}} as it corresponds with the diffusivity parallel to white matter axons \citep{Basser:1996}. Other commonly used {\gls{DT}}metrics are:
\begin{itemize}
	\item The {\gls{MD}}, computed as:
	\begin{equation}
		MD = \frac{\mbox{Tr}(D)}{3} = \frac{\lambda_1 + \lambda_2 +\lambda_3}{3}.
	\end{equation}
	\item The {\gls{FA}} that represents the degree of diffusion anisotropy in each voxel.  {\gls{FA}} increases
	with directional dependence of particle displacements and is greatest when diffusion is highly directional.  {\gls{FA}} is computed by
	\begin{equation}
		FA = \sqrt{\frac{3}{2}}\frac{\sqrt{(\lambda_1-MD)^2+(\lambda_2-MD)^2+(\lambda_3-MD)^2}}{\sqrt{\lambda_1^2+\lambda_2^2+\lambda_3^2}}
	\end{equation}
	\item The {\gls{RD}} is the average diffusivity perpendicular to the major diffusion direction:
	\begin{equation}
		RD = \frac{\lambda_2 + \lambda_3}{2}.
	\end{equation}
\end{itemize}
It should be noted that the interpretation of \gls{AD} and \gls{RD} as parallel and perpendicular diffusivities only holds true in the case of a single fibre population within the image voxel, but breaks down in the case of more complex fibre configurations such as crossing or bending fibres. Furthermore, in case of WM  pathological processes the DT shape can undergo significant changes, which makes the notion of  "axial" and "radial" diffusivities misleading if not interpreted carefully \citep{Wheeler-Kingshott:inpress}. 

\subsection{Limitations of the SGP approximation}
\label{sec:chapter2 limits of SGP}
Unlike modern \gls{NMR} spectrometers and pre-clinical small bore scanners, most clinical MRI systems are only equipped with limited maximal gradient strength (usually 40-60 mT/m). On these systems the necessary high q-values, e.g., needed for q-space analysis cannot be achieved without prolonged diffusion gradient pulse durations. \citet{Mitra:1995} showed that the effective molecule displacement measured with a finite diffusion pulse $\delta$ is equivalent to the distance between the \gls{COM} of the molecule trajectories occurring while the diffusion gradients are applied. If the \gls{SGP} condition $ \delta \ll \Delta$ is fulfilled, the observed distance between the \glspl{COM} of the trajectories is approximately the same as the true displacement of the molecule. However, if $\delta$ is long, molecules movement will occur during the diffusion gradient pulses and only the displacement between the \gls{COM}s will be observed. As illustrated in Figure~\ref{fig:chapter2 com effect}, in the case of restricted diffusion, this increase in gradient pulse duration will cause the underestimation of the true displacement. 
When implementing QSI protocols on a clinical scanner, one has to be wary of the effect of the finite gradient pulse duration and its implications. Usually, clinical studies of QSI have to violate the \gls{SGP} condition to achieve sufficiently high q-values. As expected from the \gls{COM} effect, this causes an artifactual reduction of the \gls{RMSD}. This has been confirmed in simulation \citep{Linse:1995,Latt:2007a} and various experimental studies in phantoms \citep{Avram:2004,Latt:2007}, excised tissue \citep{Malmborg:2006,Bar-Shir:2008} and even in \emph{in-vivo} human scans \citep{Nilsson:2009}. As a consequence, the estimated displacement profile has to be interpreted with caution as it will not reflect the true displacement in the tissue. The \gls{SGP} violation is a fundamental problem in the above models and can only be avoided with an increase of the maximum gradient strength. 

\begin{figure}[htbp]
	\centering
	\pgfimage[width=0.8\textwidth]{chapter2/figs/com}
	\caption{Illustration of the centre-of-mass effect on the apparent molecules displacement for different gradient pulse durations.}
	\label{fig:chapter2 com effect}
\end{figure}

Some experimental clinical scanners are already equipped with gradient systems capable of generating up to 300mT/m \citep{Toga:2012}. However, those dedicated systems are usually designed for a specific research project and the general availability of those strong whole body gradients in the future is doubtful due to their high costs. Economic feasibility aside, the use of higher gradient strengths and shorter pulse width also increases the risk of \gls{PNS} and might cause more discomfort for the subjects. 

\subsection{Gaussian phase approximation}
\label{sec:chapter 2GPD}
As discussed above, the SGP approximation is often impossible to fulfil on typical clinical scanners. An alternative model of the diffusion process is given by the \gls{GPD} approximation. In contrast to the \gls{SGP}, the \gls{GPD} offers a description of the diffusion MR signal in the presence of finite $\delta$ under the assumption that the phases of the spins due to the magnetic field gradients are Gaussian distributed.

In the SGP approximation we use the probability density function of spin displacements, whereas the GPD approximation considers the distribution function of spin phases $P(\phi,\Delta)$ at the echo time TE  having phase $\phi$. The total signal in terms of $P(\phi,\Delta)$ is
\begin{equation}
S(\delta,\Delta,\textbf{G})  = \int P(\phi,\Delta)\cos\phi d\phi.
\end{equation}

For molecules undergoing free diffusion, characterised by a single diffusion coefficient $d$,  $P$ is Gaussian so that the signal can be expressed by the following formula (see \citet{Price:1998} for details):
\begin{equation}
S(\delta,\Delta,\textbf{G})  =  \exp\Big(- \gamma^{2} |\textbf{G}|^{2} \delta^{2} (\Delta - \delta/3) d\Big).
\label{freediff}
\end{equation}
This equation provides the theoretical underpinning of the definition of the popular $b$-value introduced in Equation~\ref{eq:bvalue}. Please note in the case of free diffusion the SGP approximation becomes a special case of the \gls{GPD} approximation:
\begin{align}
				   & S(d;\delta,\Delta,\textbf{G})   =  S_0\exp(-\gamma^{2} |\textbf{G}|^{2} \delta^{2} (\Delta - \delta/3) d) & \\ 
\Leftrightarrow & S(d;\delta,\Delta,\textbf{G})   =  S_0\exp(-\gamma^{2} |\textbf{G}|^{2} \delta^{2} \Delta d)	& \mbox{ if }\frac{\delta}{\Delta}\to 0 
\label{eq: chapter2 GPD vs SGP free diff}
\end{align}
\subsection{Models of restriction}
The above analytic models are all based on the assumption that the diffusion pattern can be described well with a diffusion process. However, many studies have shown that those models inadequately describe restricted diffusion, which is observed, e.g. in coherent white matter tracts. Over the years, various analytic solutions have presented for simple restricting geometries such spheres, parallel planes under either SGP or GPD approximation \citep{Balinov:1993, Linse:1995, Callaghan:1996}.

The cylinder geometry is particularly well suited to approximate diffusion within mylelinated axons, where diffusion is mainly restricted perpendicular and unrestricted parallel to the myelin barriers. We present here the analytic solutions for the diffusion MR signal in cylinders from PGSE with finite gradient pulses under the \gls{GPD} assumption. The following analytic solution for the diffusion signal from particles diffusing within the cylinder of radius $R$ was independently proposed by \citet{Stepisnik:1993} and \citet{Vangelderen:1994}:
{\scriptsize
\begin{equation}
\ln S = -2\gamma^{2}\textbf{G}^{2}\sum_{m=1}^{\infty}\frac{2da_{m}^{2}\delta-2+2e^{-da_{m}^{2}\delta}+2e^{-da_{m}^{2}\Delta}- e^{-da_{m}^{2}(\Delta-\delta)} -e^{-da_{m}^{2}(\Delta+\delta)}}{d^{2}a_{m}^{6}(R^{2}a_{m}^{2}- 1)}
\label{biganal}
\end{equation}
}

where  $a_{m}$ is the $m$th root of equation  $J'_{1}(a_{m}R)= 0$ and $J'_{1}$ is the derivative of the Bessel function of the first kind, order one.

\subsection{Compartment models}
\label{sec:multicompartment_modeling}
Using a-priori information about the microstructure of the investigated sample, the diffusion signal can be approximated by a combination of these simple geometric compartments. Each of the $n$ different compartments possesses the model parameters $\Phi_{i}$ from which the signal $S_i$ is computed. Each compartment is assigned a volume fraction $f_i$ with $0 \le f_i \le 1$ for all $1 \le i \le n$. For an acquisition protocol \prot{}, the signal model under the combined model parameter set $\Phi=\Phi_{1}\cup\dots\cup\Phi_{n}$ is then given by:
\begin{equation}
	S(\Phi;\prot)=\sum_{i=0}^{n}f_i\cdot S_i(\phi_i;\prot).
\end{equation}

\subsubsection{Bi-exponential model}
One of the simplest compartment models is the bi-exponential model, expressing diffusion as the summation of two separate mono-exponential decay curves (see Equation \ref{eq:chapter 2 adc}) with two different diffusion coefficients (usually named \gls{ADC}$_{slow}$ and \gls{ADC}$_{fast}$):
{\footnotesize
\begin{equation}
	S(f_{slow},f_{fast},ADC_{slow},ADC_{fast}; b) = f_{slow} \exp(-b\cdot ADC_{slow}) + f_{fast} \exp(-b\cdot ADC_{fast}).
\end{equation}
}
Experiments by \citet{Clark:2002} in \emph{in-vivo} brain data demonstrate good agreement between measurements and fitted signal curves over a range of $b$-values. However, the biophysical interpretation of the two compartments is still in debate and the relation between the compartments and the microstructural properties of white matter remains unclear. 
\subsection{Geometric multi-compartment models of nervous tissue}
\paragraph*{Stanisz' model}
\citet{Stanisz:1997} were the first to propose a model that reflects the underlying micro-anatomy of nervous tissue. They introduced a model of restricted diffusion in bovine optic nerve using a three-compartment model approach. In their model, prolate ellipsoids represented axons, spheres represented glial cells and Gaussian diffusion was assumed in a homogeneous extra-cellular medium surrounded by partially permeable membranes. Experimental data was in agreement with the signal predicted by their model and showed significant departure of the {\gls{DWI}} signal from the simple Gaussian model. However, the complexity of this models requires very high quality measurements, typically only achievable in \gls{NMR} spectroscopy rather than MRI.
\paragraph*{The CHARMED model} 
\citet{Assaf:2005} developed the \gls{CHARMED} model of cylindrical axons with gamma distributed radii to estimate axon diameter distributions in white matter tissue. The \gls{CHARMED} model assumes two compartments, representing diffusion in intra-axonal and extra-axonal space. The intra-axonal compartment is modeled by parallel cylinders, with the size of radii following a gamma-distribution. The extra-cellular compartment is modeled by a {\gls{DT}} with the principal diffusion direction $\vec{v}_1$ aligned with the long cylinder axis. \citet{Alexander:2008} validated the model in in-vitro optic and sciatic nerve samples and estimated parameters show good correlation with corresponding histology. In later work, \citet{Barazany:2009} extended the \gls{CHARMED} model by an isotropic diffusion compartment to account for partial volume effects and contributions from areas of {\gls{CSF}}. They apply their model to image axon size distributions in the corpus callosum of live rat brain. However, in both experiments, scan times are long and the high 7T magnetic field and maximum {\gls{gstr}} (400 mT/m) are impossible to achieve on human scanners, typically operating at 1.5-3T with maximum {\gls{gstr}} between 30-60 mT/m.

\paragraph*{Alexander's minimal model of white matter diffusion} 
\label{par:alexanders_model}
\citet{Alexander:2010} uses a model similar to \gls{CHARMED} to demonstrate measurements of axon diameter and density in excised monkey brain and live human brain on a standard clinical scanner with multi shell \gls{HARDI}. The \gls{MMWMD} introduces several modifications to the \gls{CHARMED} models \citep{Dyrby:2010}. The most distinguishing  difference to \gls{CHARMED} is that the distribution of cylinder radii is replaced by a fixed cylinder radius. The \gls{MMWMD} expresses diffusion in a white matter voxel as a combination of water particles trapped inside three different compartments: 
\begin{enumerate}
  \item Intra-axonal water experiencing diffusion restricted inside cylindrical axons with equal radius $R$ \citep{Stepisnik:1993,Vangelderen:1994}
  \item Extra-axonal water that is hindered due to the presence of adjacent axons. Diffusion is approximated by a diffusion tensor, with parallel diffusion coefficent $d_\parallel$ in the direction of the cylinders and symmetric diffusion $d_\perp$ in the perpendicular directions.
  \item Water that experiences unhindered diffusion, e.g., in the {\gls{CSF}}, modeled by an isotropic Gaussian distribution of displacements with diffusion coefficient $d_{I}$.
  \item Non-diffusing water, e.g., trapped in membranes (no parameters).
\end{enumerate}
He reduced the number of free model parameters by expressing $d_\perp$ using the tortuosity approximation proposed by \citet{Szafer:1995}.

\paragraph*{Model taxotomy}
The examples presented above only present a very small subset of possible compartment models that can be obtained by combining the different possible decriptions of diffusion. The selection of the best suited model is complicated; on the one hand complex models such as \gls{CHARMED} might better characterise the underlying tissue than e.g. the diffusion tensor. On the other hand, increasing model complexity can lead to overfitting and false parameter estimation. \citet{Panagiotaki:2012} approached this model selection problem systematically, comparing a large number of different compartment models. They propose a taxonomy of one-, two- and three-compartment models including the models described above. In this taxotomy the three compartments represent restricted, hindered and isotropic diffusion respectively. In detail the studied compartments were:
\begin{description}
	\item[Restricted diffusion:] is described by a Stick (cylinder with zero radius) , or a non-zero radius Cylinder either with a single radius or gamma-distributed radii
	\item[Hindered diffusion:] is either represented as a Tensor (full \gls{DT}), Zeppelin (cylindrically symmetric \gls{DT}) or Ball  (isotropic \gls{DT}).  
	\item[Isotropic diffusion:] is described by Dot (stationary molecules), Sphere (isotropically restricted) or Cylinders with isotropically distributed directions either as Astro-sticks (zero radius)  or Astro-Cylinders (non-zero radius).
\end{description}


A total of 47 different combinations of these compartments were tested using a very comprehensive dataset comprising many different combinations of, \gls{smalldel}, \gls{bigdel} and \gls{gstr} acquired in the \gls{CC} of fixed rat brain. They compared and ranked the models using the Bayesian Information Criterion, which rewards the goodness of fit between the data and predicted signal but also penalises a model's complexity.  They concluded that three- and two-compartment models including non-zero diameter Cylinder compartments explain the data well while DTI performs worse. A recent similar study of \emph{in-vivo} \gls{CC} \citep{Ferizi:2012} confirmed these findings although the hardware limitations of the clinical MR system give rise to preference of simpler models of restriction such as the Stick.

\subsection{Active Imaging}
\label{sec:protocol_optimisation}
More complex models usually require  {\gls{DWI}}  acquisitions with several different diffusion weightings at various diffusion times. For example \citet{Barazany:2009} perform approx. 900 different combinations of $0\le|\vec{g}|\le 300mT/m$, $0\le {\gls{smalldel}} \le 30ms$ and $0\le \Delta \le 30ms$ to estimate the axon diameter distribution of live rat brain. This extensive sampling of the \gls{PGSE} parameter space requires long acquisition times (between hours and days) and is infeasible for \emph{in-vivo} clinical scanning. 

The principle of the ``Active Imaging" protocol optimisation framework of \cite{Alexander:2008} is to find the protocol $\mathcal{P}$, that allows the most accurate estimation of the tissue model parameters under given hardware and time constraints. The \gls{FIM} provides a lower bound on the inverse covariance matrix of parameter estimates, i.e., the $\mathcal{P}$ that maximizes the \gls{FIM} will maximize the precision of those estimates. \citeauthor{Alexander:2008} uses the \gls{CRLB} as the optimality criterion \citep{OBrien:2003}, which is defined as the trace of the inverse \gls{FIM} of protocol $\mathcal{P}$ and tissue model parameters $\phi$:
\begin{equation}
	D(\phi,\mathcal{P})=\mbox{Tr}[(\mat{J}^T\Omega\mat{J})^{-1}], 
	\label{eq-optimality}
\end{equation}
where $\mat{J}$ is the $N\times \mbox{size}(\phi)$ Jacobian matrix with the $ij$st element defined as:

\begin{equation}
	\partial S(\vec{g}_i,\delta_i,\Delta_i) / \partial \phi_j.
\end{equation}
Intuitively, the \gls{CRLB} defines a lower bound on the variance of the fitted model parameters $\phi$ for a given protocol \prot. In the original approach $\Omega$ is the identity matrix, i.e. all measurements are assigned equal importance. 

\citet{Alexander:2008} then uses a stochastic optimization algorithm \citep{Zelinka:2010} that returns $\mathcal{P}'$ with minimal $D$ among all possible $\mathcal{P}$ with respect to the given scanner hardware limits. The optimisation framework was used in \citet{Alexander:2010} to estimate the parameters of the \gls{MMWMD}, described in section \ref{par:alexanders_model} using a standard clinical Philips 3T scanner with maximum {\gls{gstr}} of $60mT/m$ and a maximum scan time of one hour (total number of acquisitions $N=360$). To achieve estimates independent of fibre orientation, the $N$ acquisitions are divided in $M$ sets of different PGSE settings with gradient directions in each set being fixed and uniformly distributed over the sphere as in \cite{Cook:2007}. They performed \emph{in-vivo} scans of the corpus callosum and compared their axon diameter and density indices with high resolution scans of \emph{ex-vivo} monkey brain and previously published histology studies. They found that the trends in diameter and density agreed with both \emph{ex-vivo} scans and histology, although the axon diameter was over-estimated. This is mainly an effect of limited gradient strength as has been shown in \cite{Dyrby:2010}. 


It should be noted that this method by design produces protocol that minimise the variance in parameter estimates, but it does not account for any potential bias between the estimates and real tissue parameters. Therefore, this approach crucially depends on the careful selection of both the tissue model and a realistic set of model parameters a-priori to the optimisation process \citep{Alexander:2008, Alexander:2010}. 



\section{Diffusion MRI in healthy and diseased spinal cord}
Although diffusion \gls{MRI} in the spinal cord has been studied for over 10 years, it's clinical application is still relatively unexplored compared to the brain. This is mostly due to the technical challenges caused by it's small structure and problematic imaging conditions, including breathing and cardiac motion and susceptibility artefacts arising from  surrounding bony tissue. However, the latest developments in imaging and post-processing methods have enabled an increasing number of studies of healthy and diseased spinal cord within the last 5-10 years. To date acquisition protocols and analysis of spinal cord \gls{DWI} focussed on studying the cord under the assumption of Gaussian diffusion, using \gls{ADC} or \gls{DTI} derived parameters. In this section we will report the diffusion properties of  spinal cord tissue as they were observed in previous studies. Furthermore we summarise tissue changes that arise from tissue damage  and their effects on \gls{DWI}-derived parameters.

\subsection{Diffusion MRI in healthy spinal cord}
Water diffusivity in the \gls{WM} is highly anisotropic, i.e. diffusion occurs preferentially in a particular direction. In highly coherent structures such as the spinal cord WM, diffusion anisotropy is usually seen as caused by restricted diffusion by the axon membrane, myelin sheath, neurofilaments and microtubules, resulting reduced transverse diffusivity (\gls{RD}) compared to the longitudinal diffusivity (\gls{AD}) along the \gls{WM} tracts. A study by \citet{Schwartz:2005} showed a significant correlation between cellular morphological parameters and \gls{ADC} using combined histological analyses and high resolution \emph{ex-vivo} \gls{DTI}. On the other hand \gls{RD} has shown to be inversely correlated with both neurofilament and microtubule density as demonstrated in the rat optic nerve\citep{Kinoshita:1999}, implying hindered diffusion caused by neurofilaments and microtubules longitudinal to the axon orientation. 
%
The estimated apparent diffusion coefficients in the human spinal cord typically ranges from $1.0 \times 10^{-3} mm^2/s$ to $2.3 \times 10^{-3} mm^{2}/s$ along the \gls{WM} fibres and between  $0.1 \times 10^{-3} mm^2/s$ to $1.0 \times 10^{-3} mm^{2}/s$ across. The range in \gls{ADC} are highly dependent on the specific microstructure of the tissue under investigation, but also depend on pulse sequence parameters such as diffusion time and \gls{TE}. Despite differences in pulse sequences, a mean \gls{ADC} in the human cervical spinal cord of approximately  $1.0 \times 10^{-3} mm^2/s$ has been reported by several groups \citep{Wheeler-Kingshott:2002,Ellingson:2007}.

\subsection{Diffusion MRI in spinal cord injury}
Trauma to the spinal cord, and changes occurring during healing, result in alterations of tissue microstructure that are measurable via diffusion MRI.  The course of the disease is broadly staged in three distinct phases: acute, sub-acute and chronic. The remainder of this section will describe each phase in more detail.  

\paragraph{Acute phase: }
In the acute stages of SC trauma, the mechanical disruption of neural tissue structure results in immediate death of cells in the region of the injury. The cell death and disruption of the cell membranes results in axons that are spaced further apart. As a result water molecules can diffuse larger distances before barriers are encountered, which can be detected by increased \gls{ADC} in animal studies \citep{Ford:2005,Deo:2006}, with diffusivity as high as a double the diffusion measurements in healthy cord. In addition, edema also occurs in the first moments of traumatic \gls{SCI} primarily resulting from mechanical disruption of axon cell membranes and damage to local blood vessels \citep{Balentine:1978,Mautes:2000}.  \Gls{DTI} in acute spinal trauma often exhibits a decrease in \gls{AD}, resulting in an overall decrease in diffusion anisotropy in the lesion sites during the period of severe edema and hemorrhage \citep{Ford:2005}.  This decrease in the \gls{AD} has been largely attributed to metabolic dysfunction as opposed to specific changes in axon morphology \citep{Schwartz:2003}. 
%
%
\paragraph{Sub-acute phase: }
Following the initial response to spinal trauma there is infiltration of inflammatory cells from both the CNS and periphery.  It is unclear how the influx of reactive cell types influences diffusion measurements in the injured spinal cord.  Reactive cells, such as glia, produce collagenous scar tissue that is expected to have a relatively high impact on tissue diffusivity.  \Citet{Schwartz:2005a} demonstrated that  the principal DTI eigenvector orientation shows sensitivity to glial cell orientations, although only if they are in sufficient numbers to significantly affect the overall orientation of the particular voxel microstructure.  Furthermore, in the subacute stage of SCI, the presence of a large number numbers of astrocytes, microglia, and macrophages are also assumed to decrease the extracellular volume, which could decrease the overall apparent diffusion coefficient, counteracting the initial increase associated with edema.
%
%
% 
In addition to tissue changes directly at the site of injury,  there is also Wallerian degeneration, i.e. degeneration of axons distant from the site of injury \citep{Waxman:1989}, which causes changes in diffusivity even at locations away from the traumatic injury.  Axon degeneration first manifests as disintegration of the myelin sheath and cytoskeletal proteins including microtubules and neurofilaments, eventually followed by  complete anterograde degeneration. Experimental data suggests extensive retrograde degeneration following injury \citep{Cajal:1928,Kalil:1975,Pallini:1988}, resulting from both apoptosis and necrosis \citep{Crowe:1997}.


During the degeneration process, \gls{RD} is typically elevated \citep{Deo:2006}. The primary explanation for the increase in \gls{RD} lies in the tissue structural changes that occur during degeneration along with direct effects on the intra- and extracellular space.  Anterograde degeneration results in rapid degeneration of both the axonal membrane and myelin sheath, decreasing the number and extent of transverse diffusion boundaries.  This is expected to contribute to a higher diffusion coefficient perpendicular to the fibre bundles.  Retrograde degeneration also shows a similar, but slightly larger, increase in \gls{RD} in experimental animal models \citep{Deo:2006}, which is most likely due to axon swelling and the subsequent increase in intracellular space\citep{Balentine:1978}. 
%
%
\paragraph{Chronic phase: } The late phase of SCI, defined months to years after the initial injury, differences in tissue morphology in chronic injury likely impact DTI measurements.  Although most of the degenerative processes are stabilized by the chronic stage, there is evidence to suggest degeneration even long after the injury.  For example, progressive demyelination can occur even during chronic injury \citep{Blight:1986,Bunge:1993}.  Remyelination, if it occurs, results in significantly decreased myelin sheath thickness\citep{Blight:1986,Bunge:1993,Totoiu:2005,Harrison:2005} and thus alters the white matter structure in chronic injury.  Furthermore, a preferential loss of large diameter axons can also occur in chronic injury\citep{Blight:1986} resulting in a dominance of small, unmyelinated axons in damaged axonal tracts.  Finally, significant atrophy of the spinal cord also occurs in late stages of spinal cord injury\citep{Ellingson:2008a,Freund:2011,Lundell:2010} causing the remaining axons to be compressed and tightly packed. These structural changes are all expected to contribute to differences in water diffusivity in chronic injury.



Studies of experimental animal models have shown that a decrease in \gls{RD} and increase in \gls{L1} are indicators of the pathological processes in chronic \gls{SCI} \citet{Song:2002,Kim:2006,Budde:2008,Feng:2009,Kim:2009,Zhang:2009} and suggest of both axonal degeneration and progressive demyelination rostral to the trauma of the cord in the chronic stage of the disease. In humans, diffusion characteristics in chronic injury have not yet been so thoroughly explored. However preliminary evidence of gross morphological changes and atrophy have been illustrated using \gls{DTI} \citep{Ellingson:2008,Cohen-Adad:2011} (see also Appendix A). 

\subsection{Future directions}
Non-gaussian \gls{QSI} has been studies almost exclusively in post-mortem tissue and experimental animal models. Early work by \citet{Assaf:2001} applied \gls{QSI} in excised infant pig spinal cord during different stages of neurological development. They showed that \gls{QSI} parameters reflect more restricted diffusion with increasing age of the piglet, which is correlated with formation of myelin. XXXXXXXX

Specifically the reduction of \gls{RD}, which is dependent on axon diameter, may be linked to the preferential loss of larger diameter axons in the late phase of \gls{SCI}.  


\Gls{DWI} has the potential to  provide information about the location and severity of an injury that might prove useful in the diagnosis and prognosis of a spinal injury.  Further, DTI measures could be used as an indicator of neural degeneration and healing.  Because of the changes in tissue structure during inflammation and healing, DTI measures are likely to depend on the stage of injury, varying from the acute to chronic stages.  

 
\section{Summary}
We have discussed ways of inferring microstructual information from  {\gls{DWI}}, ranging from simple methods such as \gls{ADC} or \gls{DTI} to sophisticated multi-compartment modelling. \gls{ADC} and \gls{DTI} are easy to obtain but the simplistic underlying assumptions of Gaussian  {\gls{dPDF}} is often inaccurate. As a result, different microstructural changes in pathologies can have the same effect on those metrics and therefore cannot be told apart by \gls{DTI} alone. At least in theory, \gls{QSI} has the potential to overcome this limitation but requires both very strong diffusion gradients and long acquisition times. Furthermore, \gls{QSI} derived parameters  {\gls{dPDF}} measures only relate indirectly to white matter structure and must be carefully interpreted if the SGP is violated.


Using more advanced diffusion models, incorporating anatomical a-priori information about the different compartments of the investigated tissue can overcome the limitations of the simplistic \gls{DTI} model but at the same time allow more flexibility than \gls{QSI}. However, \emph{in-vivo} scans are limited in in maximum scan time and hardware capabilities. Under these conditions, finding the optimal set of acquisition parameters is not trivial. The optimisation framework of Alexander can be used to find the  {\gls{DWI}}  protocol that is best suited to estimate the model parameters of interest while it respects the limitations of the clinical setup.  

 	

%%% Local Variables:
%%% TeX-master: "../thesis"
%%% End:
