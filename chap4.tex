%!TEX root = ./report.tex
\section{Completed work}
In the first two studies we aimed to improve standard DTI when used to obtain imaging markers in the spinal cord. DTI is available on all major production MRI systems and therefore easy to integrate in clinical studies. Further, scan time can be kept relatively short compared to QSI or some tissue-model-based DWI techniques. In Experiment 1 we specifically design an imaging protocol to quantify the presence of collateral fibres. We carefully optimised the acquisition protocol and analysis to detect subtle changes due to sprouting fibre. We are able to demonstrate that DTI metrics are dependent on the image positioning along the long axis of a cervical spinal cord segment. We show evidence that the change in DTI parameters is likely caused by the presence of collateral fibres. By investigating the role of RD/$ADC_{\perp}$ in the assessment of collateral fibres we took a first step towards defining new imaging biomarkers sensitive to specific SC pathology.


In Experiment 2 we concentrated on improving the post-processing of spinal cord imaging markers. We devise a novel partial volume correction for whole cord averages of over MRI metrics. We evaluate our method in DTI performed in 9 healthy controls. We show that PVA correction helps to reduce bias in average whole cord DTI metrics and improves inter-subject variability. Since the achievable resolution in spinal cord DWI is low, PVA is a common problem to all spinal cord DWI techniques. Therefore, the PVA correction is expected to improve measures of other DWI derived imaging markers, e.g., those investigated in Experiments 3 and 4.


In Experiment 3, we turn towards the more experimental QSI. The setup and analysis of in-vivo QSI is more challenging, hence QSI measures have only been reported in a few case studies of in-vivo human spinal cord so far. For the first time, we perform QSI on a larger group of 9 healthy volunteers and examined inter- and intra-subject variability of QSI measures over the whole cord area and specific white matter tracts in cervical cord. We demonstrate that variability of QSI metrics is low both in individual subjects and among all controls, while conventional ADC measures show higher variability. Further, QSI metrics complement ADC measures when distinguishing features of different white matter tracts.


In Experiment 4, we construct an imaging protocol that is specifically designed to directly estimate axon diameter and density in structures with known single fibre orientation such as the spinal cord. We adapt the framework of Alexander\cite{alexander08} to this special case. In simulation we demonstrate improved efficacy of tissue microstructure estimates when applying our protocol compared to the more general protocol used in \cite{alexander10}. We chose to perform the first test of the in-vivo implementation of our method  in the corpus callosum rather. Similar to the spinal cord, fibres in the corpus callosum follow one main direction. However, the CC has the advantage of having a larger white matter area while suffering less from motion and aliasing artifacts than the spinal cord.

\section{Limitations} 
\begin{itemize}
	\item ex1: needs larger study, better imaging, more levels
	\item ex2: only controls, need patients
	\item ex3: needs patient, no small FOV, suboptimal acquisition (G...)
	\item ex4: spinal cord, compare wit histology, not very stable 
\end{itemize}

\section{Work required for completion}
\begin{itemize}
	\item ...
\end{itemize}