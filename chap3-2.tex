%!TEX root = ./report.tex
\section{Fuzzy partial volume correction of spinal cord DTI parameters}
\label{sec:chap3:experiment2}
Due to the small size of the cord and the limited spatial resolution, a large proportion of voxels are affected by partial volume averaging (PVA) from surrounding {\gls{CSF}}. Water molecules in {\gls{CSF}} are less hindered than in nervous tissue, resulting in increased diffusivity measures and decreased anisotropy in PVA voxels \citep{Alexander:2001,Pfefferbaum:2003}. This can lead to biased average measurements over specific regions of interest (ROIs) and over the whole cord volume and potentially conceal subtle disease effects. Existing correction methods as in \citet{Pasternak:2009} are often not applicable due to the low signal-to-noise ratio in {\gls{SC}} diffusion images. Therefore in common practice, {\gls{CSF}} affected voxels are excluded from analysis with a subjective and manual editing of the outlined ROIs. However, objectively deciding which voxels to exclude while retaining information can be problematic, particularly when the cord area is small and only few unaffected voxels exist, e.g., in patients with {\gls{SC}} atrophy. We introduce a robust partial volume correction method for average \gls{DTI} parameters that avoids the manual exclusion of PVA affected voxels. Instead, we introduce a contribution weighting factor for each affected voxel that depends of on its distance to the interface between {\gls{SC}} voxels and {\gls{CSF}}. We investigate the accuracy of our approach in healthy volunteers and demonstrate that our method significantly reduces PVA effects on mean \gls{DTI} indices. 
\subsection*{Methods} 
\paragraph{Data acquisition and DTI analysis}

\begin{figure}
  \centering
  \subfloat[]
  {
  		\pgfimage[width=0.35\textwidth]{pictures/chap3/sec2/weighting-illustration.pdf}
  }
  \subfloat[]
  {
  		\pgfimage[width=0.28\textwidth]{pictures/chap3/sec2/B0-isolines.png}
  }
  \subfloat[]
  {
  		\pgfimage[width=0.28\textwidth]{pictures/chap3/sec2/FA-isolines.png}
  }
  \captionbelow{(a) 1D illustration of computed weighting factors. (b\&c) Isolines of weighting factors overlayed on FA map (b) and B0 map (c) in one subject}
  \label{fig:experiment2_weightingillustration}
\end{figure}

We acquired diffusion-weighted images of 14 healthy volunteers (13 male, age=35±11). In each subject cardiac gated \gls{DTI} of the cervical cord was performed (acquisition matrix=96x96, sinc interpolated in image space to $192\times 192$, FOV=$144\times 144mm^2$, slice thickness=5mm, 20 slices, TE=88ms, TR≈4000ms) with a total of 100 b=1000s/mm$^2$ diffusion weighted volumes (20 unique diffusion directions repeated 5 times) and 5 non-diffusion weighted volumes. In each voxel the diffusion tensor was fitted to the data using camino \citep{Cook:2006} and maps of {\gls{FA}}, {\gls{MD}}, {\gls{AD}} and {\gls{RD}} were generated. 
\paragraph{PVA method} We semi-automatically delineate the cervical cord between levels C1/2 and C4/5 using the active surface segmentation \citep{Horsfield:2010} available in Jim6\footnote{\url{www.xinapse.com}}, performed on the computed  {\gls{FA}} maps similar to \citet{Wheeler-Kingshott:2002a}. A 2D distance transformation is applied to the binary segmentation masks, i.e., determining the distance d of each masked voxel to the border of the mask. Assuming that only voxels close to {\gls{CSF}} are affected by PVA, the fuzzy partial volume correction factor $w$ is then computed as: 
\begin{equation}
	x =\left\{
	\begin{array}{lll}
		d/\mbox{max}(d)&\mbox{ if } d\leq c\\
		1&\mbox{ otherwise } 
	\end{array}
	\right.,	
\end{equation}
where $c$ is a cutoff distance determined on the basis of the \gls{DTI} parameter values as illustrated in Figure~\ref{fig:experiment2_weightingillustration}. The weighted average using the weighting factors $w$ is computed for all \gls{DTI} parameters over the whole segmented {\gls{SC}} area. The linear weighting factor is dependent on the maximum cord diameter and ensures that for larger {\gls{SC}}, the border voxels are weighted less than in the case of a small cord. 


We determine the optimal cutoff voxel distance $c'$ in our dataset so that for $c\geq c'$ the average \gls{DTI} parameters over the cord area reach a stable plateau, i.e., assuming that {\gls{CSF}} contribution effects are minimized. In our data, \gls{DTI} parameters reach the desired plateau for $c\geq 2$ voxels (see Figure~\ref{fig:experiment2_cutoffgraph}) and are in agreement with previously reported values in the healthy cord \citep{Wheeler-Kingshott:2002a,Ellingson:2007}. Thus the cutoff value $c=2$ is chosen for further analysis. We perform a two-tailed paired t-test to compare significance of differences between uncorrected and corrected averages of all \gls{DTI} metrics among all subjects. 
\subsection*{Results and discussion} Table~\ref{tab:experiment2_reproducibility} shows lower standard deviation of diffusivity parameters among subjects when using our PVA correction method, suggesting lower inter-subject variability compared to the uncorrected measurements. Furthermore, the largest reduction of \gls{DTI} values is observed in the \gls{RD} (p<0.0001). We also find moderate decrease in the \gls{AD} and  {\gls{MD}} and increase in  {\gls{FA}} (all p<0.0001). These results can be explained by {\gls{CSF}} contribution to average measurements in uncorrected values and are in agreement with similar findings in simulations \citep{Alexander:2001} and in the brain \citep{Pfefferbaum:2003}.

\begin{figure}
  \centering
  \pgfimage[width=10cm]{pictures/chap3/sec2/cutoff-graph.pdf}
  \captionbelow{Weighted average and standard deviation among all subjects for \protect\gls{DTI} parameters computed with different cutoff values. Columns corresponding to the chosen cutoff value of 2 are colored red.}
  \label{fig:experiment2_cutoffgraph}
\end{figure}

\begin{table}
\begin{captionbeside}[]{Averaged relative change of mean and standard deviation between uncorrected and PVA-corrected \protect\gls{DTI} measurements over all subjects.}[o][\linewidth][1em]
\begin{tabular}{l >{\raggedleft\arraybackslash}p{3cm}>{\raggedleft\arraybackslash}p{3cm}}
\toprule
   & \centering Average change in  mean (\%) & \centering\arraybackslash Average change in std (\%)\\
\cmidrule(r){2-2}\cmidrule(l){3-3}
FA & +6.2$^*$  ± 0.7  & +2.1\\
MD & -12.2$^*$ ± 1.3  & -10.6\\
AD & -7.0$^*$  ± 1.0  & -6.6\\
RD & -21.0$^*$ ± 2.1  & -9.8\\
\bottomrule
\multicolumn{3}{l}{\footnotesize $^*$Significance p<0.001 (confidence interval 99\%)}
\end{tabular}
\end{captionbeside}
\label{tab:experiment2_reproducibility}
\end{table}


\subsection*{Conclusion} In this study we propose a novel fuzzy partial volume correction method that removes {\gls{CSF}} contribution effects in measurements of \gls{DTI} parameters over the whole {\gls{SC}} volume. We avoid fully excluding all potentially {\gls{CSF}} contaminated voxels, and introduce a weighting factor that is dependent on the size of the cord and therefore accounts for the variability in number of white matter voxels. This allows more reliable measurements, particularly in patients who might suffer from white matter atrophy. Our method can be easily extended to other analysis methods such as histogram analysis and other quantitative modalities such as magnetization transfer imaging.