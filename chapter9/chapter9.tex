%!TEX root = ../thesis.tex
%\newcommand{\SF}{{\ensuremath{\mathcal{SF}}}}
%\newcommand{\OI}{{\ensuremath{\mathcal{OI}}}}
%\newcommand{\SD}{{\ensuremath{\mathcal{SF}_{pulses}}}}
%\newcommand{\DO}{{\ensuremath{\mathcal{SF}_{dirs}}}}
%\newcommand{\FD}{{\SF}}
%\newcommand{\FDmod}{{\ensuremath{\FD_{mod}}}}
%\newcommand{\SFasym}{{\ensuremath{a\mathcal{SF}}}}
%
%\newcommand{\SFshort}{\SF$_{90}$}
%\newcommand{\SFlong}{\SF$_{360}$}
%\newcommand{\OIlong}{\OI$_{360}$}


\newsavebox{\poorBox}
\savebox{\poorBox}{\textcolor{red}{\rule{0.05in}{0.05in}}}
\newsavebox{\fairBox}
\savebox{\fairBox}{\textcolor{orange}{\rule{0.05in}{0.05in}}}
\newsavebox{\moderateBox}
\savebox{\moderateBox}{\textcolor{yellow}{\rule{0.05in}{0.05in}}}
\newsavebox{\substantialBox}
\savebox{\substantialBox}{\textcolor{lime}{\rule{0.05in}{0.05in}}}
\newsavebox{\perfectBox}
\savebox{\perfectBox}{\textcolor{green}{\rule{0.05in}{0.05in}}}


\chapter[Scan/rescan in the corpus callosum]{Scan/rescan reproducibility in the corpus callosum and spinal cord}
\section{Introduction}
In the previous chapter we have presented a implementation of our {\SFasym} protocols on a clinical system. We found that the 25 minute long \SFasym{} acquisition can produce comparable results to the \~3 times longer \OI{} method. Our initial work was focussed only on the optimisation of the acquisition protocol. However, clerly other non-diffusion related imaging parameters, such as accurate positioning or spatial resolution, also contribute significantly to the goodness of the parameter estimates coming from our protocol. In this chapter we address several shortcomings of the initial \SFasym{} setup. Our main aim is to improve SNR and spatial resolution of our dataset in order to maximise accuracy and  reproducibility of our microstructure maps. 


In detail, we make the following changes compared to the \SFasym{} experiment described in in Chapter~\ref{chapter8}:
\begin{enumerate}
\item We use of a small \gls{FOV} imaging sequence to increase both image resolution and SNR.
\item We develop a new method to better align the image volume with the dominant fibre direction of the CC using fast DTI tractography directly on the scanner console.
\item We extend the signal model used in the optimisation and fitting to allow the use of shortest available \gls{TE} for different combination of \gls{smalldel} and \gls{bigdel}. Furthermore we increase the nominal maximal gradient strength to 87mT/m using the the modified scanner software we developed for the QSI study in Chapter~\ref{chapter7}. The combined effect lower \gls{TE} and stronger gradient will increase SNR and aid the performance of the parameter fitting routine.
\end{enumerate}
We analyse our new imaging pipeline in 5 healthy volunteers, who were scanned at two different timepoint to assess both intra- and inter-subject reproducibility. Furthermore we propose the first implementation of our \SFasym{} protocol in the cervical spine and test it in one healthy volunteer. 


\section{Methods \& Experiments}
\subsection*{Protocol optimisation}
We use the same \SFasym{} optimisation as described in Chapter~\ref{chapter9}, but with one modification: we extend the tissue model to include an additional T2 decay factor that accounts for signal loss in the DWI with respect to the \gls{TE} on a per-acquisition basis. For simplicity, we assume here a mono-exponential T2 decay of the signal. Previously, the \gls{TE} was governed by the maximum TE in the whole protocol. This modification now allows the dynamic use of \glspl{TE} for different $(\gls{smalldel},\gls{bigdel})$ settings in the protocol optimisation. This has the advantage of effectively reducing the \gls{TE} in the low DWI acquisition, which improves greatly the SNR in those acquisitions.

We carry out the optimisation using as before, assuming T2=70ms, which is typical for white matter (WM) in the CC \citep{}. We also use the scanner software modification we described in Chapter~\ref{chapter7}, which allows to combine several orthogonal 62mT/m-gradients to increase the maximum gradient strength to 87mT/m. To improve directional resolution required for the fit of the diffusion direction, we add a single shell DTI acquisition (max b=800s/mm$^2$, 1 b=0, 16 uniformly distributed directions) to the optimised protocol. The full set of parameters is given in Table \ref{tab:chap9 protocol table}.

\begin{table}[htbp]
  \centering
  \caption{Optimised protocol parameters for the \SFasym{} method with variable TEs}
    \begin{tabular}{rrrrr}
    \toprule
    	  & $\delta$ $[ms]$   & $\Delta$ $[ms]$ & $G$ $[mT/m]$ & TE $[s]$\\
    \midrule
    {1b0} & {0} & {0} & {0} & {23} \\
    {1b0 + 5$\parallel$} & {8} & {22} & {78} & {46} \\
    {1b0 + 14$\perp$} & {13} & {20} & {87} & {54} \\
    {1b0 + 7$\perp$} & {22} & {56} & {48} & {96} \\
    {1b0 + 26$\perp$} & {23} & {29} & {87} & {73} \\
    {1b0 + 20$\perp$} & {27} & {50} & {63} & {93} \\
    {1b0 + 11$\perp$} & {35} & {42} & {81} & {93} \\
    {1b0 + 16 DTI} & \multicolumn{3}{c}{{b=800 s/mm$^2$}} & {47} \\
    \midrule
    \multicolumn{5}{r}{\textit{\textbf{total scan time 35min}}} \\
    \bottomrule
    \end{tabular}%
  \label{tab:chap9 protocol table}%
\end{table}%

\subsection{CC reproducibility study}
\subsubsection*{Data acquisition}
We recruited 5 healthy volunteers (3 female, 2 male, mean age=28 $\pm$ ) to be scanned on a Philips Achieva 3TX scanner. All subjects were recalled for a second scan on a different day to assess the intra-subject reproducibility of the experiment. In each scan session we acquire a the optimised \SF{} DWI protocol with following scan parameters: voxel size: 1x1x4mm$^3$, FOV=96x96mm$^2$, TR=6000ms, 2 averages, using an outer-volume suppressed ZOOM acquisition \citep{Wilm:2007} to avoid fold-over artifacts.  
\paragraph{}
The previous experiments have shown that our \SF{} methods benefit from accurate alignment of the gradient scheme with respect to the dominant fibre direction. While standard T2w localizers are adequate to align the scan volume to anatomical reference, it offers no information about the WM fibre orientation. To aid slice positioning, we acquire a fast DTI scan in addition to the conventional scout scans. The scout-DTI imaging parameters are as follows: voxelsize=2x2x4mm$^3$, 16 slices, FOV=232x232 mm$^2$, TE=78ms, TR=4200ms, 6 non-colinear diffusion weighted directions (b=1000) plus one non-diffusion weighted image. Total scan time of the scout DTI scan is 57 seconds. We use the PRIDE tools directly on the scan console to place a ROI in the mid-sagittal slice of the CC and perform FACT tractography on the DTI dataset (FA threshold=0.45, angle threshold=YY). The tracts are then overlayed on the colour-coded FA map, rasterised, and resliced to obtain a new 1x1x1 image volume. To plan the final \SF{} scan, we use the axial and coronal views of the tractography results to adjust the angulation of the axial slices with respect to the observed tracts.  Figure~\ref{fig:chap9 FOV positioning} shows an example of the final slice alignment based on both the structural localizer scan and tractography results. Since the whole scout-DTI processing is performed on the scanner console, the additional scan setup time by acquiring analysing the scout-DTI scan is kept to a minimum.

\begin{figure}[ht]
	\centering
	\begin{minipage}{0.35\textwidth}
	\subfloat[Sagittal localizer]
	{
		\pgfimage[width=\textwidth]{chapter9/figs/sag_pos.pdf}
	}\\
	\subfloat[Coronal FA map overlayed with tractography results (cyan lines)]
	{
		\pgfimage[width=\textwidth]{chapter9/figs/coronal_overlay.pdf}
	}	
	\end{minipage}\hspace{0.05\textwidth}
	\begin{minipage}{0.53\textwidth}
	\subfloat[Axial FA map with overlayed tractography results (cyan lines)]
	{
		\pgfimage[width=\textwidth]{chapter9/figs/axial_overlay.pdf}
	}	
	\end{minipage}
	
	\caption{Positioning of small FOV scans in white, overlayed on a sagittal scout image (a) and axial and coronal DTI tractography results (b\&c).}
	\label{fig:chap9 FOV positioning}	
\end{figure}
\subsubsection*{Post-processing}
We compensate for motion during the acquisition by aligning all scan volumes to the first b=0 image using the block-wise rigid registration algorithm \citep{Ourselin:2001} implemented in \citep{Modat:2010}. However, since the diffusion weighted images provide little contrast in non-coherently aligned WM tissue, we only register the interleaved b=0 images and apply the estimated transformation matrix to the subsequent intermediate b>0 images.   

To ensure anatomical correspondence between scan and rescan, we then register the rescan dataset to the scan dataset (using rigid registration) using the transformation estimates from registering  the first b=0 images of the two datasets. The transformation matrices for intra-scan motion and scan/rescan alignment are combined before applying them to the dataset to avoid unnecessary multiple interpolations. The data is then smoothed using the Unbiased-Non-Local-Means filter \citep{Tristan-Vega:2012} with a small filter radius of $3\times3\times4 mm^3$.

\subsubsection*{Data analysis}
We use the same fitting procedure as outlined in the previous chapters. However since we allowed for variable TE in each acquisition, we need to account for the resulting differences in T2 single decay within the data. We therefore estimate the voxel-wise mono-exponential decay curve using a linear regression model based on the non-diffusion-weighted acquisitions. The predicted MR signal $S$ from the tissue model is then adjusted based on its \glspl{TE} by:
\begin{equation}
	S' = S * exp(TE/T2).
\end{equation}
The adjusted signal S' is then used to compute the rician log-likelihood with the observed signal as before. As in the previous studies, we compute the posterior distributions of the model parameters using an MCMC method on a voxel-by-voxel basis. From the mean of the posterior distribution we compute the the axon diameter index $a$ and axonal density index $\rho$=f/$\pi/a^2$. In addition we also fitted the diffusion tensor to the 16-direction DTI data and derive the principal eigenvectors $v1$--$v3$ and scalar maps of FA, MD, AD, RD. All fitting is implemented using the Camino toolkit \citep{Cook:2006}.
\begin{figure}[ht]
	\centering
	\pgfimage[width=0.5\textwidth]{chapter9/figs/CC_ROIs}
	\caption{Example of CC subdivision scheme overlayed on the midsagittal slice of a b=0 image in one volunteer. The ROIs divide the CC in genu (G1--G3), midbody (B1--B3), isthmus region (I) and splenium (S1--S3).}
	\label{fig:chap9 CC ROIs}
\end{figure}
\subsubsection*{ROI analysis} 
In each subject we manually segmented the CC on the mid-sagittal slice. We then remove all voxels from the CC mask with FA<0.5 to exclude voxels with more than one single fibre orientation or significant CSF contamination. We further exclude voxel from the analysis where v1 deviated more than 10 degrees from the left-right fibre orientation we assumed in the protocol optimisation. The CC segmentation is then divided in 10 equidistant regions along the anterior-posterior baseline similar to \cite{Aboitiz:1992}. Figure \ref{fig:chap9 CC ROIs} shows an example of the CC subdivision in one subject. Mean $a$ and $\rho$ indices are then computed for each CC subdivision in each of the 10 datasets. Scan/rescan agreement is assessed visually as well as it is quantified by computing the \gls{ICC} \citep{Shrout:1979} over the whole CC and in each \gls{ROI}. To investigate the correlation between DTI metrics and $a$ and $\rho$, we pool all values in the CC ROI from all subjects separately for scan and rescan and report the robust correlation coefficient\citep{Huber:1996}. All statistical processing was performed using the software R\citep{RCoreTeam:2012} with packages 'ICC'\citep{Wolak:2011} and 'robust'\citep{Wang:2012}.

\subsection{SC experiment}
\subsubsection*{Data acquisition}
We performed the scans  our \SFasym{} in one healthy volunteers (41YO female). The \SF{} DWI protocol wasacquired with following scan parameters: voxel size: 1x1x5mm3, FOV=64x64mm$^2$, cardiac gated, TR=5RR, using the outer-volume suppressed ZOOM acquisition \citep{Wilm:2007} in the CC. We chose the 32-channel head coil to perform the scans as we found that it offers superior SNR in the cervical cord region than the dedicated 16-channel head-neck coil alternative.Scan were acquired between the discs C1/2 and C3/4, although severe motion artifacts made it necessary to exclude all slices except the 3 most caudal slices. 

\FloatBarrier
\section{Results}
\subsection*{Axon diameter and axon density indices in the CC}
Figure~\ref{fig:chap9 scan rescan maps per subject} shows side-by-side scan/rescan maps of $a$ and $\rho$ for all five subjects. Figure~\ref{fig:chap9 scan rescan scatterplots per subject} summarizes the mean $a$ and $\rho$ parameters measured in each ROI for all five subjects. In all subjects we can clearly see the variation along AP we expect from previous experiment and earlier studies \citep{Alexander:2010}. Furthermore, in comparison with those earlier results, our maps appear considerably less noisy and show improved contrast between different CC regions. Consistent with our previous results, we estimate values of $a$ in the range of 5--15$\mu m$. The largest $a$ estimates are found in the midbody of the CC. The smallest $a$ values are found in the splenium <8$\mu m$ and the anterior part of the genu (9--11$\mu m$). The axon density index $\rho$ is reciprocal to the $a$ trends, with $\rho$ being largest in the anterior genu and posterior splenium regions and smallest in the body and isthmus of the CC. The $a$ and $\rho$ pattern we observe here agrees very well with the microstructure that is seen in excised human CC tissue samples.


Unlike in the previous experiments, the high spatial resolution here provides a large number of voxels that are completely contained in the CC. As a consequence, the CC can be easily distinguished from surrounding tissue and the tissue parameter estimates are less influenced by CSF contamination, particularly in the thinning part of the CC (B3--S1). This is beneficial for subjects with smaller CC such as found in s2 \& s3, but becomes even more important in view of future applications in patients with neurological diseases such as MS or Alzheimers disease, who often suffer from severe CC atrophy. 
\begin{figure}[ht]
	\centering
	\subfloat[]
	{
		\pgfimage[width=0.49\textwidth]{chapter9/figs/diam_per_subj}
	}
	\subfloat[]
	{
		\pgfimage[width=0.49\textwidth]{chapter9/figs/dens_per_subj}
	}
 	\ref{leg:chap 9 maps per subj}
	\caption{Individual maps of $a$ and $\rho$ in the sagittal slice for each subject for the scan and rescan experiments.}
	\label{fig:chap9 scan rescan maps per subject}
\end{figure}

\subsection*{Inter- \& Intra-subject reproducibility}
Results in Figure~\ref{fig:chap9 scan rescan maps per subject} and  Figure~\ref{fig:chap9 scan rescan scatterplots per subject} suggest good ROI-wise consistency of the parameter maps between the five subjects. Figure~\ref{fig:chap9 scan rescan averages}, also shows that the average taken over all 5 subjects for the scan and rescan experiment agree very well with each other, both in the average trend as well as in the observed standard deviation. Moreover, both $a$ and $\rho$ show very little variation from the mean over all subjects. Inter-subject variation is lower in the mid-body and the proximal genu regions (G2--G3) than in the more distal anterior and posterior regions. The Bland-Altman plots shown in Figure \ref{fig:chap9 bland altman plot} show good scan/rescan variation independent of parameter estimate with the majority of ROI estimates within the confidence interval of 1.5 standard deviations. In a minority of ROIs we see outliers with scan/rescan large variation. This appears to be related to large axon density estimates, which appear more unstable than smaller $\rho$ values. Those outliers appear mostly in the most distal G1 and S3 regions. Such large variations might indicate cardiac pulsation artifacts as these regions are closest to the adjacent Arteria Cerebralis. In $a$ the scan/rescan variability appears more independent of the actual estimation values.


Table~\ref{tab:chapter9 ICC table} presents the \gls{ICC} for whole CC and individual ROIs. For both $a$ and $\rho$, we find the scan/rescan agreement being 'moderate' or better for both whole CC values and most ROIs. As noted before, the lowest ICC values are found in boundary regions (G1) or in the thin proximal part of the CC, which are most affected by imaging and analysis artefacts. Both Bland-Altman analysis as well as the ICC suggest that estimated values in those ROIs appear more prone to error and must be interpreted with caution.

\begin{figure}[ht]
	\centering
	\ref{leg:chap 9 diam_all}
	\subfloat[Axon diameter index]
	{
		%!TEX root = ../../thesis.tex
\begin{tikzpicture}[scale=0.7]
\begin{axis}[%
    		xlabel={ROI}, 
		ylabel={$a$ in $[\mu m]$},
		ymin=5,
		ymax=15,
		width=0.7\textwidth,
		height=0.6\textwidth,
		xtick={1,2,3,4,5,6,7,8,9,10},
		xticklabels={G1,G2,G3,B1,B2,B3,I,S1,S2,S3},		
		legend to name=leg:chap 9 diam_all,
		legend columns=-1,
		title=Scan
	]
	\pgfplotstableread{chapter9/figs/diam_all.dat}\tableall
	\addplot+[only marks,mark options={scale=1.5}] table[x=reg,y=s11] from \tableall;		
	\addplot+[only marks, mark options={scale=1.5}] table[x=reg,y=s21] from \tableall;		
	\addplot+[only marks, mark options={scale=1.5}] table[x=reg,y=s31] from \tableall;		
	\addplot+[only marks, mark options={scale=1.5}] table[x=reg,y=s41] from \tableall;		
	\addplot+[only marks, mark options={scale=1.5}] table[x=reg,y=s51] from \tableall;		
	\addplot[dashed, thick] table[x=reg,y=avg1] from \tableall;		
	\legend{s1,s2,s3,s4,s5,mean};
	\end{axis}
\end{tikzpicture}	
\begin{tikzpicture}[scale=0.7]
\begin{axis}[%
    		xlabel={ROI}, 
		%ylabel={$a$ in $[\mu m]$},
		ymin=5,
		ymax=15,
		width=0.7\textwidth,
		height=0.6\textwidth,
		xtick={1,2,3,4,5,6,7,8,9,10},
		xticklabels={G1,G2,G3,B1,B2,B3,I,S1,S2,S3},		
		title=Rescan
	]
	\pgfplotstableread{chapter9/figs/diam_all.dat}\tableall
	\addplot+[only marks,mark options={scale=1.5}] table[x=reg,y=s12] from \tableall;		
	\addplot+[only marks, mark options={scale=1.5}] table[x=reg,y=s22] from \tableall;		
	\addplot+[only marks, mark options={scale=1.5}] table[x=reg,y=s32] from \tableall;		
	\addplot+[only marks, mark options={scale=1.5}] table[x=reg,y=s42] from \tableall;		
	\addplot+[only marks, mark options={scale=1.5}] table[x=reg,y=s52] from \tableall;		
	\addplot[dashed, thick] table[x=reg,y=avg2] from \tableall;		
%	\legend{s1,s2,s3,s4,s5,mean};
	\end{axis}
\end{tikzpicture}


	}\\	
	\subfloat[Axon density index]
	{
		%!TEX root = ../../thesis.tex
\begin{tikzpicture}[scale=0.7]
\begin{axis}[%
    		xlabel={ROI}, 
		ylabel={$\rho$ in $[\mu m^{-2}]$},
		ymin=0,
		ymax=0.05,
		width=0.7\textwidth,
		height=0.6\textwidth,
		xtick={1,2,3,4,5,6,7,8,9,10},
		xticklabels={G1,G2,G3,B1,B2,B3,I,S1,S2,S3},		
%		legend to name=leg:chap 9 dens_all,
%		legend columns=-1,
		title=Scan,
		yticklabel style={%
		        /pgf/number format/.cd,
		        fixed,
		        fixed zerofill,
            		precision=2,
	        },
	]
	\pgfplotstableread{chapter9/figs/dens_all.dat}\tableall
	\addplot+[only marks,mark options={scale=1.5}] table[x=reg,y=s11] from \tableall;		
	\addplot+[only marks, mark options={scale=1.5}] table[x=reg,y=s21] from \tableall;		
	\addplot+[only marks, mark options={scale=1.5}] table[x=reg,y=s31] from \tableall;		
	\addplot+[only marks, mark options={scale=1.5}] table[x=reg,y=s41] from \tableall;		
	\addplot+[only marks, mark options={scale=1.5}] table[x=reg,y=s51] from \tableall;		
	\addplot[dashed, thick] table[x=reg,y=avg1] from \tableall;		
%	\legend{s1,s2,s3,s4,s5,mean};
	\end{axis}
\end{tikzpicture}	
\begin{tikzpicture}[scale=0.7]
\begin{axis}[%
    		xlabel={ROI}, 
		%ylabel={$a$ in $[\mu m]$},
		ymin=0,
		ymax=0.05,
		width=0.7\textwidth,
		height=0.6\textwidth,
		xtick={1,2,3,4,5,6,7,8,9,10},
		xticklabels={G1,G2,G3,B1,B2,B3,I,S1,S2,S3},		
		title=Rescan,
		yticklabel style={%
		        /pgf/number format/.cd,
		        fixed,
		        fixed zerofill,
            		precision=2,
	        },
	]
	\pgfplotstableread{chapter9/figs/dens_all.dat}\tableall
	\addplot+[only marks,mark options={scale=1.5}] table[x=reg,y=s12] from \tableall;		
	\addplot+[only marks, mark options={scale=1.5}] table[x=reg,y=s22] from \tableall;		
	\addplot+[only marks, mark options={scale=1.5}] table[x=reg,y=s32] from \tableall;		
	\addplot+[only marks, mark options={scale=1.5}] table[x=reg,y=s42] from \tableall;		
	\addplot+[only marks, mark options={scale=1.5}] table[x=reg,y=s52] from \tableall;		
	\addplot[dashed, thick] table[x=reg,y=avg2] from \tableall;		
%	\legend{s1,s2,s3,s4,s5,mean};
	\end{axis}
\end{tikzpicture}


	}\\
	\caption{Scatter plots of axon diameter ($a$) and axon density ($\rho$) indices in all 5 subjects in individual ROIs. The dashed line shows the average over all subjects.}
	\label{fig:chap9 scan rescan scatterplots per subject}
\end{figure}

\begin{figure}[ht]
	\centering
	\ref{leg:chap 9 dens_avg}\\	
	\subfloat[Axon diameter index]
	{
		%!TEX root = ../../thesis.tex
\begin{tikzpicture}[scale=0.7]
\begin{axis}[%
    		xlabel={ROI}, 
		ylabel={$a$ in $[\mu m]$},
		ymin=5,
		ymax=15,
		width=0.7\textwidth,
		height=0.6\textwidth,
		xtick={1,2,3,4,5,6,7,8,9,10},
		xticklabels={G1,G2,G3,B1,B2,B3,I,S1,S2,S3},		
		legend to name=leg:chap 9 dens_avg,
		legend columns=-1,
		yticklabel style={%
		        /pgf/number format/.cd,
		        fixed,
		        fixed zerofill,
            		precision=2,
	        },
	]
	\pgfplotstableread{chapter9/figs/diam_all.dat}\tableall
	\addplot+[blue, mark=*,error bars/.cd, y dir=both, y explicit] table[x=reg,y=avg1, y error = std1] from \tableall;		
	\addplot+[red, mark=*,error bars/.cd, y dir=both, y explicit] table[x=reg,y=avg2, y error = std2] from \tableall;		
	\legend{scan,rescan};
	\end{axis}
\end{tikzpicture}	

	}	
	\subfloat[Axon density index]
	{
		%!TEX root = ../../thesis.tex
\begin{tikzpicture}[scale=0.7]
\begin{axis}[%
    		xlabel={ROI}, 
		ylabel={$\rho$ in $[\mu m^{-2}]$},
		ymin=0,
		ymax=0.07,
		width=0.7\textwidth,
		height=0.6\textwidth,
		xtick={1,2,3,4,5,6,7,8,9,10},
		xticklabels={G1,G2,G3,B1,B2,B3,I,S1,S2,S3},		
%		legend to name=leg:chap 9 dens_all,
%		legend columns=-1,
		yticklabel style={%
		        /pgf/number format/.cd,
		        fixed,
		        fixed zerofill,
            		precision=2,
	        },
	]
	\pgfplotstableread{chapter9/figs/dens_all.dat}\tableall
	\addplot+[blue, mark=*,error bars/.cd, y dir=both, y explicit] table[x=reg,y=avg1, y error = std1] from \tableall;		
	\addplot+[red, mark=*,error bars/.cd, y dir=both, y explicit] table[x=reg,y=avg2, y error = std2] from \tableall;		
%	\legend{s1,s2,s3,s4,s5,mean};
	\end{axis}
\end{tikzpicture}	

	}
	\caption{Average and standard deviation $a$ and $\rho$ over the whole group of 5 subjects for different ROIs.}
	\label{fig:chap9 scan rescan averages}
\end{figure}

\begin{figure}[ht]
	\centering
	%!TEX root = ../../thesis.tex
%created by ICC.r
\definecolor{brewer10_1}{HTML}{9E0142}
\definecolor{brewer10_2}{HTML}{D53E4F}
\definecolor{brewer10_3}{HTML}{F46D43}
\definecolor{brewer10_4}{HTML}{FDAE61}
\definecolor{brewer10_5}{HTML}{FEE08B}
\definecolor{brewer10_6}{HTML}{E6F598}
\definecolor{brewer10_7}{HTML}{ABDDA4}
\definecolor{brewer10_8}{HTML}{66C2A5}
\definecolor{brewer10_9}{HTML}{3288BD}
\definecolor{brewer10_10}{HTML}{5E4FA2}
\pgfplotsset{blandaltman/.style={scatter/classes=
{
R1={color=brewer10_1, mark=*, mark options={scale=1.5}}, R2={color=brewer10_3, mark=o, mark options={scale=1.5}}, R3={color=brewer10_4, mark=asterisk, mark options={scale=1.5}}, R4={color=brewer10_5, mark=oplus*, mark options={scale=1.5}}, R5={color=brewer10_6, mark=triangle*, mark options={scale=1.5}}, R6={color=brewer10_7, mark=square*, mark options={scale=1.5}}, R7={color=brewer10_8, mark=diamond*, mark options={scale=1.5}}, R8={color=brewer10_9, mark=diamond*, mark options={scale=1.5}}, R9={color=brewer10_10, mark=star, mark options={scale=1.5}}, R10={color=brewer10_2, mark=+, mark options={scale=1.5}}
}
}}
\begin{tikzpicture}[scale=1]
\begin{axis}[legend columns=-1, blandaltman,width=0.8*\textwidth, height=0.6*\textwidth, xmin=8.877132,xmax=14.554980,ymin=-1.884580,ymax=1.884580, legend to name=leg:chap 9 bland altman,
	title=Axon diameter index,
	xlabel={scan/rescan average},
	ylabel={scan/rescan difference},]
\addplot+[only marks, scatter]  [scatter src=explicit symbolic] 
coordinates {
(11.42235, -0.35310)[R1]
(11.97285, 0.26170)[R2]
(12.73915, 0.32570)[R3]
(12.99505, 0.48330)[R4]
(12.93510, 0.44820)[R5]
(12.71710, -0.12500)[R6]
(12.22115, -0.26330)[R7]
(11.91405, 0.17950)[R8]
(11.47995, -0.13610)[R9]
(11.68225, 0.07950)[R10]
(10.70500, -0.94900)[R1]
(11.48135, -0.11810)[R2]
(11.86385, 0.47190)[R3]
(12.20675, 0.82150)[R4]
(12.05765, -0.25810)[R5]
(12.13630, 0.57940)[R6]
(11.92495, 0.66810)[R7]
(10.96030, 0.63480)[R8]
(10.63285, 0.54890)[R9]
(9.86348, -0.05920)[R10]
(11.39565, -0.30870)[R1]
(12.47305, -0.00370)[R2]
(12.81075, -0.17930)[R3]
(12.84975, -0.06810)[R4]
(12.55710, -0.27260)[R5]
(11.76920, -0.93260)[R6]
(11.76835, -1.58610)[R7]
(11.84845, -1.92110)[R8]
(11.49530, -0.41900)[R9]
(10.79545, -0.18510)[R10]
(11.66590, -1.71420)[R1]
(12.01435, -0.60150)[R2]
(12.35065, -0.41490)[R3]
(12.70670, 0.11360)[R4]
(12.63735, 0.25190)[R5]
(12.43770, 0.02420)[R6]
(12.25150, -0.59800)[R7]
(12.18665, -0.89310)[R8]
(11.86595, -0.38130)[R9]
(10.92505, -0.12490)[R10]
(12.15770, -0.32000)[R1]
(12.84275, 0.24230)[R2]
(12.96655, 0.76350)[R3]
(13.09315, 0.94730)[R4]
(13.23180, 0.03320)[R5]
(12.86005, 0.46430)[R6]
(12.69000, 0.87940)[R7]
(11.81980, 0.16620)[R8]
(11.45060, 0.29840)[R9]
(11.66275, -0.56250)[R10]
};
\draw[dashed] (axis cs:8.877132,-1.256387) -- (axis cs:14.554980,-1.256387);
\draw[dashed] (axis cs:8.877132,0.000000) -- (axis cs:14.554980,0.000000);
\draw[dashed] (axis cs:8.877132,1.256387) -- (axis cs:14.554980,1.256387);
\legend{G1,G2, G3, B1, B2, B3, I, S1, S2, S3};
\end{axis}
\end{tikzpicture}\\
\begin{tikzpicture}[scale=1]
\begin{axis}[scaled ticks=false, blandaltman,width=0.8*\textwidth, height=0.6*\textwidth, xmin=0.020790,xmax=0.050490,ymin=-0.012710,ymax=0.012710,%
    xticklabel style={/pgf/number format/.cd,fixed}, % Use fixed point notation
	yticklabel style={/pgf/number format/.cd,fixed, precision=3, zerofill}, % Use fixed point notation
	title=Axon density index,
	xlabel={scan/rescan average},
	ylabel={scan/rescan difference},]
\addplot+[only marks, scatter]  [scatter src=explicit symbolic] coordinates {
(0.03270, 0.00280)[R1]
(0.03210, -0.00040)[R2]
(0.02825, -0.00170)[R3]
(0.02685, -0.00130)[R4]
(0.02670, -0.00160)[R5]
(0.02705, 0.00010)[R6]
(0.02925, 0.00050)[R7]
(0.03055, -0.00090)[R8]
(0.03395, 0.00110)[R9]
(0.03400, -0.00140)[R10]
(0.04165, 0.00630)[R1]
(0.03485, 0.00150)[R2]
(0.03125, -0.00090)[R3]
(0.02815, -0.00090)[R4]
(0.02900, 0.00040)[R5]
(0.02825, -0.00130)[R6]
(0.02930, -0.00140)[R7]
(0.03845, -0.01090)[R8]
(0.04305, -0.01390)[R9]
(0.04590, -0.00340)[R10]
(0.03235, -0.00010)[R1]
(0.03035, 0.00330)[R2]
(0.02785, -0.00110)[R3]
(0.02600, -0.00360)[R4]
(0.02820, -0.00160)[R5]
(0.02785, 0.00190)[R6]
(0.02310, 0.00000)[R7]
(0.03205, 0.01310)[R8]
(0.03210, -0.00180)[R9]
(0.03395, -0.00150)[R10]
(0.03030, 0.00320)[R1]
(0.03150, 0.00300)[R2]
(0.02940, 0.00280)[R3]
(0.02820, 0.00060)[R4]
(0.02770, 0.00100)[R5]
(0.02745, 0.00130)[R6]
(0.02440, -0.00420)[R7]
(0.03000, 0.01020)[R8]
(0.03195, 0.00470)[R9]
(0.04200, 0.00860)[R10]
(0.03010, 0.00300)[R1]
(0.02670, 0.00100)[R2]
(0.02455, 0.00150)[R3]
(0.02480, -0.00060)[R4]
(0.02525, 0.00210)[R5]
(0.02535, 0.00230)[R6]
(0.02395, -0.00130)[R7]
(0.02840, 0.00240)[R8]
(0.03240, -0.00020)[R9]
(0.03290, 0.00420)[R10]
};
\draw[dashed] (axis cs:0.020790,-0.008473) -- (axis cs:0.050490,-0.008473);
\draw[dashed] (axis cs:0.020790,0.000000) -- (axis cs:0.050490,0.000000);
\draw[dashed] (axis cs:0.020790,0.008473) -- (axis cs:0.050490,0.008473);
%\legend{R1, R10, R2, R3, R4, R5, R6, R7, R8, R9};
\end{axis}
\end{tikzpicture}
\ref{leg:chap 9 bland altman}	
	\caption{Bland-Altman scan/rescan reproducibility analysis of $a$ and $\rho$ in all CC ROIs.}
	\label{fig:chap9 bland altman plot}	
\end{figure}	



\begin{table}[ht]
\caption{ICC values for whole CC and individual ROIs for $a$ and $\rho$ estimates.}
\label{tab:chapter9 ICC table}
\begin{adjustbox}{width={\textwidth},totalheight=\textheight,keepaspectratio}
\begin{tabular}{rrrrrrrrrrrr}
      \toprule
       & & \multicolumn{10}{c}{\textit{Individual ROIs}}                                             \\
       & \textit{whole CC} & G1    & G2    & G3    & B1    & B2    & B3    & I     & S1    & S2    & S3\\
       \cmidrule(rl){2-2} \cmidrule(l){3-12}
       \addlinespace
$a$    & 0.66~\usebox{\substantialBox} & 0.14~\usebox{\poorBox}  & 0.83~\usebox{\perfectBox} & 0.56~\usebox{\moderateBox}  & 0.14~\usebox{\poorBox}  & 0.81~\usebox{\perfectBox}  & 0.46~\usebox{\moderateBox}  & -0.25~\usebox{\poorBox} & -0.07~\usebox{\poorBox} & 0.70~\usebox{\substantialBox}  & 0.94~\usebox{\perfectBox}  \\
$\rho$ & 0.79~\usebox{\substantialBox} & 0.74~\usebox{\substantialBox}  & 0.77~\usebox{\substantialBox}  & 0.78~\usebox{\substantialBox}  & 0.44~\usebox{\moderateBox}  & 0.59~\usebox{\moderateBox}  & 0.34~\usebox{\fairBox}  & 0.79~\usebox{\substantialBox}  & -0.14~\usebox{\poorBox} & 0.34~\usebox{\fairBox}  & 0.73~\usebox{\substantialBox}  \\
\bottomrule
\end{tabular}
\end{adjustbox}
{\footnotesize Guidelines for agreement  \citep{Landis:1977}: \usebox{\poorBox}~$<0.2$: poor,  \usebox{\fairBox}~$0.2–0.4$:~fair,  \usebox{\moderateBox}~$0.4–0.6$:~moderate, \usebox{\substantialBox}~$0.6–0.8$:~substantial,  \usebox{\perfectBox}~$>0.8$:~almost perfect}
\end{table}

\bgroup
\tikzset{use png} % Will be exported to png for faster comp

\subsection*{Correlation with DTI metrics}
\begin{figure}[ht]
	\centering
				\subfloat[FA]{
					\begin{minipage}{0.5\textwidth}						
					\begin{adjustbox}{width={\textwidth},totalheight=\textheight,keepaspectratio}
						\strut
						% Created by tikzDevice version - on 2012-09-27 22:33:37
% !TEX encoding = UTF-8 Unicode
\begin{tikzpicture}[x=1pt,y=1pt]
\definecolor[named]{fillColor}{rgb}{1.00,1.00,1.00}
\path[use as bounding box,fill=fillColor,fill opacity=0.00] (0,0) rectangle (289.08,144.54);
\begin{scope}
\path[clip] (  0.00,  0.00) rectangle (289.08,144.54);
\definecolor[named]{fillColor}{rgb}{1.00,1.00,1.00}

\path[fill=fillColor] (  0.00,  0.00) rectangle (289.08,144.54);
\end{scope}
\begin{scope}
\path[clip] ( 39.69,119.86) rectangle (156.86,132.50);
\definecolor[named]{fillColor}{rgb}{0.80,0.80,0.80}

\path[fill=fillColor] ( 39.69,119.86) rectangle (156.86,132.50);
\definecolor[named]{drawColor}{rgb}{0.00,0.00,0.00}

\node[text=drawColor,anchor=base,inner sep=0pt, outer sep=0pt, scale=  0.96] at ( 98.27,122.87) {Scan (r=-0.652)};
\end{scope}
\begin{scope}
\path[clip] (159.87,119.86) rectangle (277.04,132.50);
\definecolor[named]{fillColor}{rgb}{0.80,0.80,0.80}

\path[fill=fillColor] (159.87,119.86) rectangle (277.03,132.50);
\definecolor[named]{drawColor}{rgb}{0.00,0.00,0.00}

\node[text=drawColor,anchor=base,inner sep=0pt, outer sep=0pt, scale=  0.96] at (218.45,122.87) {Rescan (r=-0.554)};
\end{scope}
\begin{scope}
\path[clip] (  0.00,  0.00) rectangle (289.08,144.54);
\definecolor[named]{drawColor}{rgb}{0.50,0.50,0.50}

\node[text=drawColor,anchor=base east,inner sep=0pt, outer sep=0pt, scale=  0.96] at ( 32.58, 36.28) {0.3};

\node[text=drawColor,anchor=base east,inner sep=0pt, outer sep=0pt, scale=  0.96] at ( 32.58, 48.92) {0.4};

\node[text=drawColor,anchor=base east,inner sep=0pt, outer sep=0pt, scale=  0.96] at ( 32.58, 61.57) {0.5};

\node[text=drawColor,anchor=base east,inner sep=0pt, outer sep=0pt, scale=  0.96] at ( 32.58, 74.21) {0.6};

\node[text=drawColor,anchor=base east,inner sep=0pt, outer sep=0pt, scale=  0.96] at ( 32.58, 86.86) {0.7};

\node[text=drawColor,anchor=base east,inner sep=0pt, outer sep=0pt, scale=  0.96] at ( 32.58, 99.50) {0.8};

\node[text=drawColor,anchor=base east,inner sep=0pt, outer sep=0pt, scale=  0.96] at ( 32.58,112.15) {0.9};
\end{scope}
\begin{scope}
\path[clip] (  0.00,  0.00) rectangle (289.08,144.54);
\definecolor[named]{drawColor}{rgb}{0.50,0.50,0.50}

\path[draw=drawColor,line width= 0.6pt,line join=round,line cap=round] ( 35.42, 39.58) -- ( 39.69, 39.58);

\path[draw=drawColor,line width= 0.6pt,line join=round,line cap=round] ( 35.42, 52.23) -- ( 39.69, 52.23);

\path[draw=drawColor,line width= 0.6pt,line join=round,line cap=round] ( 35.42, 64.87) -- ( 39.69, 64.87);

\path[draw=drawColor,line width= 0.6pt,line join=round,line cap=round] ( 35.42, 77.52) -- ( 39.69, 77.52);

\path[draw=drawColor,line width= 0.6pt,line join=round,line cap=round] ( 35.42, 90.16) -- ( 39.69, 90.16);

\path[draw=drawColor,line width= 0.6pt,line join=round,line cap=round] ( 35.42,102.81) -- ( 39.69,102.81);

\path[draw=drawColor,line width= 0.6pt,line join=round,line cap=round] ( 35.42,115.45) -- ( 39.69,115.45);
\end{scope}
\begin{scope}
\path[clip] ( 39.69, 34.04) rectangle (156.86,119.86);
\definecolor[named]{fillColor}{rgb}{0.90,0.90,0.90}

\path[fill=fillColor] ( 39.69, 34.04) rectangle (156.86,119.86);
\definecolor[named]{drawColor}{rgb}{0.95,0.95,0.95}

\path[draw=drawColor,line width= 0.3pt,line join=round,line cap=round] ( 39.69, 45.91) --
	(156.86, 45.91);

\path[draw=drawColor,line width= 0.3pt,line join=round,line cap=round] ( 39.69, 58.55) --
	(156.86, 58.55);

\path[draw=drawColor,line width= 0.3pt,line join=round,line cap=round] ( 39.69, 71.20) --
	(156.86, 71.20);

\path[draw=drawColor,line width= 0.3pt,line join=round,line cap=round] ( 39.69, 83.84) --
	(156.86, 83.84);

\path[draw=drawColor,line width= 0.3pt,line join=round,line cap=round] ( 39.69, 96.49) --
	(156.86, 96.49);

\path[draw=drawColor,line width= 0.3pt,line join=round,line cap=round] ( 39.69,109.13) --
	(156.86,109.13);

\path[draw=drawColor,line width= 0.3pt,line join=round,line cap=round] ( 58.23, 34.04) --
	( 58.23,119.86);

\path[draw=drawColor,line width= 0.3pt,line join=round,line cap=round] ( 78.30, 34.04) --
	( 78.30,119.86);

\path[draw=drawColor,line width= 0.3pt,line join=round,line cap=round] ( 98.36, 34.04) --
	( 98.36,119.86);

\path[draw=drawColor,line width= 0.3pt,line join=round,line cap=round] (118.43, 34.04) --
	(118.43,119.86);

\path[draw=drawColor,line width= 0.3pt,line join=round,line cap=round] (138.49, 34.04) --
	(138.49,119.86);
\definecolor[named]{drawColor}{rgb}{1.00,1.00,1.00}

\path[draw=drawColor,line width= 0.6pt,line join=round,line cap=round] ( 39.69, 39.58) --
	(156.86, 39.58);

\path[draw=drawColor,line width= 0.6pt,line join=round,line cap=round] ( 39.69, 52.23) --
	(156.86, 52.23);

\path[draw=drawColor,line width= 0.6pt,line join=round,line cap=round] ( 39.69, 64.87) --
	(156.86, 64.87);

\path[draw=drawColor,line width= 0.6pt,line join=round,line cap=round] ( 39.69, 77.52) --
	(156.86, 77.52);

\path[draw=drawColor,line width= 0.6pt,line join=round,line cap=round] ( 39.69, 90.16) --
	(156.86, 90.16);

\path[draw=drawColor,line width= 0.6pt,line join=round,line cap=round] ( 39.69,102.81) --
	(156.86,102.81);

\path[draw=drawColor,line width= 0.6pt,line join=round,line cap=round] ( 39.69,115.45) --
	(156.86,115.45);

\path[draw=drawColor,line width= 0.6pt,line join=round,line cap=round] ( 48.20, 34.04) --
	( 48.20,119.86);

\path[draw=drawColor,line width= 0.6pt,line join=round,line cap=round] ( 68.26, 34.04) --
	( 68.26,119.86);

\path[draw=drawColor,line width= 0.6pt,line join=round,line cap=round] ( 88.33, 34.04) --
	( 88.33,119.86);

\path[draw=drawColor,line width= 0.6pt,line join=round,line cap=round] (108.39, 34.04) --
	(108.39,119.86);

\path[draw=drawColor,line width= 0.6pt,line join=round,line cap=round] (128.46, 34.04) --
	(128.46,119.86);

\path[draw=drawColor,line width= 0.6pt,line join=round,line cap=round] (148.52, 34.04) --
	(148.52,119.86);
\definecolor[named]{fillColor}{rgb}{0.00,0.00,0.00}

\path[fill=fillColor,fill opacity=0.20] (141.50, 45.91) circle (  2.13);

\path[fill=fillColor,fill opacity=0.20] ( 99.36, 41.73) circle (  2.13);

\path[fill=fillColor,fill opacity=0.20] ( 94.35, 55.39) circle (  2.13);

\path[fill=fillColor,fill opacity=0.20] ( 89.33, 71.20) circle (  2.13);

\path[fill=fillColor,fill opacity=0.20] ( 91.34, 74.48) circle (  2.13);

\path[fill=fillColor,fill opacity=0.20] (102.37, 71.83) circle (  2.13);

\path[fill=fillColor,fill opacity=0.20] (109.40, 56.15) circle (  2.13);

\path[fill=fillColor,fill opacity=0.20] ( 86.32, 47.30) circle (  2.13);

\path[fill=fillColor,fill opacity=0.20] ( 88.33, 72.08) circle (  2.13);

\path[fill=fillColor,fill opacity=0.20] ( 83.31, 87.76) circle (  2.13);

\path[fill=fillColor,fill opacity=0.20] ( 77.29, 89.02) circle (  2.13);

\path[fill=fillColor,fill opacity=0.20] ( 83.31, 88.14) circle (  2.13);

\path[fill=fillColor,fill opacity=0.20] ( 74.28, 87.89) circle (  2.13);

\path[fill=fillColor,fill opacity=0.20] ( 82.31, 85.48) circle (  2.13);

\path[fill=fillColor,fill opacity=0.20] ( 87.33, 67.40) circle (  2.13);

\path[fill=fillColor,fill opacity=0.20] (105.38, 54.88) circle (  2.13);

\path[fill=fillColor,fill opacity=0.20] (127.45, 45.65) circle (  2.13);

\path[fill=fillColor,fill opacity=0.20] ( 82.31, 57.41) circle (  2.13);

\path[fill=fillColor,fill opacity=0.20] ( 83.31, 79.16) circle (  2.13);

\path[fill=fillColor,fill opacity=0.20] ( 82.31, 89.40) circle (  2.13);

\path[fill=fillColor,fill opacity=0.20] ( 83.31, 97.75) circle (  2.13);

\path[fill=fillColor,fill opacity=0.20] ( 80.30,101.04) circle (  2.13);

\path[fill=fillColor,fill opacity=0.20] ( 81.31, 95.22) circle (  2.13);

\path[fill=fillColor,fill opacity=0.20] ( 83.31, 87.25) circle (  2.13);

\path[fill=fillColor,fill opacity=0.20] ( 84.32, 86.87) circle (  2.13);

\path[fill=fillColor,fill opacity=0.20] ( 86.32, 89.28) circle (  2.13);

\path[fill=fillColor,fill opacity=0.20] (100.37, 73.09) circle (  2.13);

\path[fill=fillColor,fill opacity=0.20] ( 98.36, 53.87) circle (  2.13);

\path[fill=fillColor,fill opacity=0.20] (105.38, 41.48) circle (  2.13);

\path[fill=fillColor,fill opacity=0.20] ( 99.36, 38.44) circle (  2.13);

\path[fill=fillColor,fill opacity=0.20] ( 81.31, 55.89) circle (  2.13);

\path[fill=fillColor,fill opacity=0.20] ( 81.31, 79.79) circle (  2.13);

\path[fill=fillColor,fill opacity=0.20] ( 73.28, 98.51) circle (  2.13);

\path[fill=fillColor,fill opacity=0.20] ( 75.29, 94.08) circle (  2.13);

\path[fill=fillColor,fill opacity=0.20] ( 80.30, 97.62) circle (  2.13);

\path[fill=fillColor,fill opacity=0.20] ( 86.32,106.47) circle (  2.13);

\path[fill=fillColor,fill opacity=0.20] ( 83.31,100.15) circle (  2.13);

\path[fill=fillColor,fill opacity=0.20] ( 81.31, 91.17) circle (  2.13);

\path[fill=fillColor,fill opacity=0.20] ( 86.32, 89.66) circle (  2.13);

\path[fill=fillColor,fill opacity=0.20] ( 93.34, 86.75) circle (  2.13);

\path[fill=fillColor,fill opacity=0.20] ( 97.36, 73.22) circle (  2.13);

\path[fill=fillColor,fill opacity=0.20] ( 94.35, 60.83) circle (  2.13);

\path[fill=fillColor,fill opacity=0.20] ( 99.36, 48.05) circle (  2.13);

\path[fill=fillColor,fill opacity=0.20] ( 86.32,110.90) circle (  2.13);

\path[fill=fillColor,fill opacity=0.20] (101.37, 47.80) circle (  2.13);

\path[fill=fillColor,fill opacity=0.20] ( 94.35, 75.49) circle (  2.13);

\path[fill=fillColor,fill opacity=0.20] ( 81.31, 92.06) circle (  2.13);

\path[fill=fillColor,fill opacity=0.20] ( 76.29,105.59) circle (  2.13);

\path[fill=fillColor,fill opacity=0.20] ( 77.29,106.22) circle (  2.13);

\path[fill=fillColor,fill opacity=0.20] ( 77.29,102.30) circle (  2.13);

\path[fill=fillColor,fill opacity=0.20] ( 81.31, 99.14) circle (  2.13);

\path[fill=fillColor,fill opacity=0.20] ( 85.32, 97.24) circle (  2.13);

\path[fill=fillColor,fill opacity=0.20] ( 85.32, 97.88) circle (  2.13);

\path[fill=fillColor,fill opacity=0.20] ( 92.34, 94.21) circle (  2.13);

\path[fill=fillColor,fill opacity=0.20] (100.37, 84.22) circle (  2.13);

\path[fill=fillColor,fill opacity=0.20] ( 95.35, 71.70) circle (  2.13);

\path[fill=fillColor,fill opacity=0.20] ( 98.36, 57.79) circle (  2.13);

\path[fill=fillColor,fill opacity=0.20] (102.37, 38.32) circle (  2.13);

\path[fill=fillColor,fill opacity=0.20] ( 96.35, 76.25) circle (  2.13);

\path[fill=fillColor,fill opacity=0.20] ( 85.32,101.16) circle (  2.13);

\path[fill=fillColor,fill opacity=0.20] ( 84.32, 80.93) circle (  2.13);

\path[fill=fillColor,fill opacity=0.20] ( 92.34, 81.44) circle (  2.13);

\path[fill=fillColor,fill opacity=0.20] ( 92.34, 89.53) circle (  2.13);

\path[fill=fillColor,fill opacity=0.20] (102.37, 50.20) circle (  2.13);

\path[fill=fillColor,fill opacity=0.20] ( 96.35, 87.25) circle (  2.13);

\path[fill=fillColor,fill opacity=0.20] ( 80.30,102.55) circle (  2.13);

\path[fill=fillColor,fill opacity=0.20] ( 78.30,109.13) circle (  2.13);

\path[fill=fillColor,fill opacity=0.20] ( 79.30,114.69) circle (  2.13);

\path[fill=fillColor,fill opacity=0.20] ( 80.30, 98.00) circle (  2.13);

\path[fill=fillColor,fill opacity=0.20] ( 78.30, 90.79) circle (  2.13);

\path[fill=fillColor,fill opacity=0.20] ( 93.34, 93.32) circle (  2.13);

\path[fill=fillColor,fill opacity=0.20] (103.38, 91.93) circle (  2.13);

\path[fill=fillColor,fill opacity=0.20] (114.41, 84.35) circle (  2.13);

\path[fill=fillColor,fill opacity=0.20] ( 99.36, 69.42) circle (  2.13);

\path[fill=fillColor,fill opacity=0.20] ( 77.29, 96.61) circle (  2.13);

\path[fill=fillColor,fill opacity=0.20] ( 79.30, 91.81) circle (  2.13);

\path[fill=fillColor,fill opacity=0.20] ( 82.31, 96.86) circle (  2.13);

\path[fill=fillColor,fill opacity=0.20] ( 80.30, 97.62) circle (  2.13);

\path[fill=fillColor,fill opacity=0.20] ( 82.31, 96.23) circle (  2.13);

\path[fill=fillColor,fill opacity=0.20] ( 86.32,100.66) circle (  2.13);

\path[fill=fillColor,fill opacity=0.20] ( 97.36, 92.69) circle (  2.13);

\path[fill=fillColor,fill opacity=0.20] (101.37, 64.11) circle (  2.13);

\path[fill=fillColor,fill opacity=0.20] ( 95.35, 54.38) circle (  2.13);

\path[fill=fillColor,fill opacity=0.20] (100.37, 91.30) circle (  2.13);

\path[fill=fillColor,fill opacity=0.20] ( 91.34,110.14) circle (  2.13);

\path[fill=fillColor,fill opacity=0.20] ( 88.33,106.73) circle (  2.13);

\path[fill=fillColor,fill opacity=0.20] ( 87.33,103.69) circle (  2.13);

\path[fill=fillColor,fill opacity=0.20] ( 88.33,111.03) circle (  2.13);

\path[fill=fillColor,fill opacity=0.20] ( 86.32,104.45) circle (  2.13);

\path[fill=fillColor,fill opacity=0.20] ( 82.31, 86.12) circle (  2.13);

\path[fill=fillColor,fill opacity=0.20] (106.39, 83.46) circle (  2.13);

\path[fill=fillColor,fill opacity=0.20] ( 87.33, 76.38) circle (  2.13);

\path[fill=fillColor,fill opacity=0.20] ( 72.28,100.03) circle (  2.13);

\path[fill=fillColor,fill opacity=0.20] ( 69.27, 92.06) circle (  2.13);

\path[fill=fillColor,fill opacity=0.20] ( 70.27, 98.89) circle (  2.13);

\path[fill=fillColor,fill opacity=0.20] ( 67.36,110.39) circle (  2.13);

\path[fill=fillColor,fill opacity=0.20] ( 60.04,112.54) circle (  2.13);

\path[fill=fillColor,fill opacity=0.20] ( 74.28,106.85) circle (  2.13);

\path[fill=fillColor,fill opacity=0.20] ( 79.30,110.52) circle (  2.13);

\path[fill=fillColor,fill opacity=0.20] ( 82.31,104.32) circle (  2.13);

\path[fill=fillColor,fill opacity=0.20] (109.40, 73.98) circle (  2.13);

\path[fill=fillColor,fill opacity=0.20] (106.39, 56.15) circle (  2.13);

\path[fill=fillColor,fill opacity=0.20] (100.37, 88.52) circle (  2.13);

\path[fill=fillColor,fill opacity=0.20] ( 89.33,100.40) circle (  2.13);

\path[fill=fillColor,fill opacity=0.20] ( 93.34,102.81) circle (  2.13);

\path[fill=fillColor,fill opacity=0.20] ( 92.34, 96.23) circle (  2.13);

\path[fill=fillColor,fill opacity=0.20] ( 90.33, 96.23) circle (  2.13);

\path[fill=fillColor,fill opacity=0.20] ( 96.35, 98.26) circle (  2.13);

\path[fill=fillColor,fill opacity=0.20] ( 92.34, 86.12) circle (  2.13);

\path[fill=fillColor,fill opacity=0.20] ( 93.34, 76.63) circle (  2.13);

\path[fill=fillColor,fill opacity=0.20] (104.38, 75.62) circle (  2.13);

\path[fill=fillColor,fill opacity=0.20] ( 85.32, 60.70) circle (  2.13);

\path[fill=fillColor,fill opacity=0.20] ( 75.29,102.81) circle (  2.13);

\path[fill=fillColor,fill opacity=0.20] ( 69.27, 94.34) circle (  2.13);

\path[fill=fillColor,fill opacity=0.20] ( 66.06, 96.49) circle (  2.13);

\path[fill=fillColor,fill opacity=0.20] ( 64.55,113.94) circle (  2.13);

\path[fill=fillColor,fill opacity=0.20] ( 70.27,114.82) circle (  2.13);

\path[fill=fillColor,fill opacity=0.20] ( 68.26,107.87) circle (  2.13);

\path[fill=fillColor,fill opacity=0.20] ( 81.31,106.10) circle (  2.13);

\path[fill=fillColor,fill opacity=0.20] ( 84.32,108.88) circle (  2.13);

\path[fill=fillColor,fill opacity=0.20] ( 73.28,115.33) circle (  2.13);

\path[fill=fillColor,fill opacity=0.20] (100.37, 93.45) circle (  2.13);

\path[fill=fillColor,fill opacity=0.20] (115.42, 50.33) circle (  2.13);

\path[fill=fillColor,fill opacity=0.20] ( 97.36, 75.87) circle (  2.13);

\path[fill=fillColor,fill opacity=0.20] ( 93.34, 85.61) circle (  2.13);

\path[fill=fillColor,fill opacity=0.20] ( 89.33,103.19) circle (  2.13);

\path[fill=fillColor,fill opacity=0.20] ( 90.33,107.61) circle (  2.13);

\path[fill=fillColor,fill opacity=0.20] ( 92.34, 94.21) circle (  2.13);

\path[fill=fillColor,fill opacity=0.20] ( 87.33, 87.38) circle (  2.13);

\path[fill=fillColor,fill opacity=0.20] ( 93.34, 88.27) circle (  2.13);

\path[fill=fillColor,fill opacity=0.20] ( 90.33, 86.50) circle (  2.13);

\path[fill=fillColor,fill opacity=0.20] (102.37, 76.13) circle (  2.13);

\path[fill=fillColor,fill opacity=0.20] (133.47, 49.45) circle (  2.13);

\path[fill=fillColor,fill opacity=0.20] ( 77.29, 84.35) circle (  2.13);

\path[fill=fillColor,fill opacity=0.20] ( 69.27,102.18) circle (  2.13);

\path[fill=fillColor,fill opacity=0.20] ( 72.28,103.69) circle (  2.13);

\path[fill=fillColor,fill opacity=0.20] ( 81.31,111.03) circle (  2.13);

\path[fill=fillColor,fill opacity=0.20] ( 78.30,103.69) circle (  2.13);

\path[fill=fillColor,fill opacity=0.20] ( 77.29,100.78) circle (  2.13);

\path[fill=fillColor,fill opacity=0.20] ( 86.32,102.68) circle (  2.13);

\path[fill=fillColor,fill opacity=0.20] ( 92.34,101.04) circle (  2.13);

\path[fill=fillColor,fill opacity=0.20] ( 90.33,112.42) circle (  2.13);

\path[fill=fillColor,fill opacity=0.20] (109.40, 40.85) circle (  2.13);

\path[fill=fillColor,fill opacity=0.20] (101.37, 64.11) circle (  2.13);

\path[fill=fillColor,fill opacity=0.20] ( 92.34, 80.30) circle (  2.13);

\path[fill=fillColor,fill opacity=0.20] ( 84.32,104.70) circle (  2.13);

\path[fill=fillColor,fill opacity=0.20] ( 91.34,102.43) circle (  2.13);

\path[fill=fillColor,fill opacity=0.20] ( 93.34, 90.42) circle (  2.13);

\path[fill=fillColor,fill opacity=0.20] ( 95.35, 88.52) circle (  2.13);

\path[fill=fillColor,fill opacity=0.20] ( 96.35, 90.42) circle (  2.13);

\path[fill=fillColor,fill opacity=0.20] ( 97.36, 86.50) circle (  2.13);

\path[fill=fillColor,fill opacity=0.20] ( 66.66,104.45) circle (  2.13);

\path[fill=fillColor,fill opacity=0.20] ( 72.28,109.38) circle (  2.13);

\path[fill=fillColor,fill opacity=0.20] ( 76.29,102.05) circle (  2.13);

\path[fill=fillColor,fill opacity=0.20] ( 67.96,107.74) circle (  2.13);

\path[fill=fillColor,fill opacity=0.20] ( 79.30,105.72) circle (  2.13);

\path[fill=fillColor,fill opacity=0.20] ( 81.31,100.28) circle (  2.13);

\path[fill=fillColor,fill opacity=0.20] ( 78.30,100.53) circle (  2.13);

\path[fill=fillColor,fill opacity=0.20] ( 82.31, 98.38) circle (  2.13);

\path[fill=fillColor,fill opacity=0.20] ( 88.33, 95.60) circle (  2.13);

\path[fill=fillColor,fill opacity=0.20] ( 91.34,105.46) circle (  2.13);

\path[fill=fillColor,fill opacity=0.20] (100.37,103.06) circle (  2.13);

\path[fill=fillColor,fill opacity=0.20] (103.38, 54.76) circle (  2.13);

\path[fill=fillColor,fill opacity=0.20] ( 85.32, 79.03) circle (  2.13);

\path[fill=fillColor,fill opacity=0.20] ( 79.30, 96.36) circle (  2.13);

\path[fill=fillColor,fill opacity=0.20] ( 81.31,103.82) circle (  2.13);

\path[fill=fillColor,fill opacity=0.20] ( 91.34,104.96) circle (  2.13);

\path[fill=fillColor,fill opacity=0.20] ( 94.35,102.81) circle (  2.13);

\path[fill=fillColor,fill opacity=0.20] ( 94.35, 94.84) circle (  2.13);

\path[fill=fillColor,fill opacity=0.20] ( 98.36, 83.08) circle (  2.13);

\path[fill=fillColor,fill opacity=0.20] ( 95.35, 85.61) circle (  2.13);

\path[fill=fillColor,fill opacity=0.20] ( 87.33, 71.57) circle (  2.13);

\path[fill=fillColor,fill opacity=0.20] ( 69.27,111.03) circle (  2.13);

\path[fill=fillColor,fill opacity=0.20] ( 72.28, 93.70) circle (  2.13);

\path[fill=fillColor,fill opacity=0.20] ( 74.28, 95.22) circle (  2.13);

\path[fill=fillColor,fill opacity=0.20] ( 69.27,106.98) circle (  2.13);

\path[fill=fillColor,fill opacity=0.20] ( 77.29,105.21) circle (  2.13);

\path[fill=fillColor,fill opacity=0.20] ( 83.31,100.53) circle (  2.13);

\path[fill=fillColor,fill opacity=0.20] ( 76.29, 99.14) circle (  2.13);

\path[fill=fillColor,fill opacity=0.20] ( 91.34, 93.58) circle (  2.13);

\path[fill=fillColor,fill opacity=0.20] ( 88.33, 94.84) circle (  2.13);

\path[fill=fillColor,fill opacity=0.20] ( 93.34,103.31) circle (  2.13);

\path[fill=fillColor,fill opacity=0.20] (106.39, 82.20) circle (  2.13);

\path[fill=fillColor,fill opacity=0.20] ( 95.35, 74.48) circle (  2.13);

\path[fill=fillColor,fill opacity=0.20] ( 86.32, 84.22) circle (  2.13);

\path[fill=fillColor,fill opacity=0.20] ( 84.32, 89.91) circle (  2.13);

\path[fill=fillColor,fill opacity=0.20] ( 88.33, 97.62) circle (  2.13);

\path[fill=fillColor,fill opacity=0.20] ( 91.34,107.11) circle (  2.13);

\path[fill=fillColor,fill opacity=0.20] ( 89.33,102.43) circle (  2.13);

\path[fill=fillColor,fill opacity=0.20] ( 93.34, 81.94) circle (  2.13);

\path[fill=fillColor,fill opacity=0.20] ( 82.31, 77.39) circle (  2.13);

\path[fill=fillColor,fill opacity=0.20] ( 93.34, 81.18) circle (  2.13);

\path[fill=fillColor,fill opacity=0.20] ( 80.30, 97.62) circle (  2.13);

\path[fill=fillColor,fill opacity=0.20] ( 79.30, 98.63) circle (  2.13);

\path[fill=fillColor,fill opacity=0.20] ( 80.30, 89.53) circle (  2.13);

\path[fill=fillColor,fill opacity=0.20] ( 81.31, 95.73) circle (  2.13);

\path[fill=fillColor,fill opacity=0.20] ( 77.29,103.82) circle (  2.13);

\path[fill=fillColor,fill opacity=0.20] ( 79.30,100.28) circle (  2.13);

\path[fill=fillColor,fill opacity=0.20] ( 80.30, 99.77) circle (  2.13);

\path[fill=fillColor,fill opacity=0.20] ( 79.30,103.31) circle (  2.13);

\path[fill=fillColor,fill opacity=0.20] ( 82.31, 97.88) circle (  2.13);

\path[fill=fillColor,fill opacity=0.20] ( 93.34, 96.23) circle (  2.13);

\path[fill=fillColor,fill opacity=0.20] ( 94.35, 96.36) circle (  2.13);

\path[fill=fillColor,fill opacity=0.20] (110.40, 63.99) circle (  2.13);

\path[fill=fillColor,fill opacity=0.20] (100.37, 57.16) circle (  2.13);

\path[fill=fillColor,fill opacity=0.20] ( 95.35, 72.46) circle (  2.13);

\path[fill=fillColor,fill opacity=0.20] ( 95.35, 84.35) circle (  2.13);

\path[fill=fillColor,fill opacity=0.20] ( 87.33, 94.08) circle (  2.13);

\path[fill=fillColor,fill opacity=0.20] ( 91.34, 98.51) circle (  2.13);

\path[fill=fillColor,fill opacity=0.20] ( 90.33, 98.13) circle (  2.13);

\path[fill=fillColor,fill opacity=0.20] ( 87.33, 89.28) circle (  2.13);

\path[fill=fillColor,fill opacity=0.20] ( 96.35, 82.70) circle (  2.13);

\path[fill=fillColor,fill opacity=0.20] (101.37, 85.99) circle (  2.13);

\path[fill=fillColor,fill opacity=0.20] (114.41, 71.95) circle (  2.13);

\path[fill=fillColor,fill opacity=0.20] ( 80.30, 86.24) circle (  2.13);

\path[fill=fillColor,fill opacity=0.20] ( 69.27,100.40) circle (  2.13);

\path[fill=fillColor,fill opacity=0.20] ( 86.32, 92.31) circle (  2.13);

\path[fill=fillColor,fill opacity=0.20] ( 84.32, 93.58) circle (  2.13);

\path[fill=fillColor,fill opacity=0.20] ( 80.30,100.78) circle (  2.13);

\path[fill=fillColor,fill opacity=0.20] ( 80.30,103.82) circle (  2.13);

\path[fill=fillColor,fill opacity=0.20] ( 75.29, 99.77) circle (  2.13);

\path[fill=fillColor,fill opacity=0.20] ( 83.31,102.18) circle (  2.13);

\path[fill=fillColor,fill opacity=0.20] ( 77.29,108.62) circle (  2.13);

\path[fill=fillColor,fill opacity=0.20] ( 77.29,100.15) circle (  2.13);

\path[fill=fillColor,fill opacity=0.20] ( 89.33, 92.19) circle (  2.13);

\path[fill=fillColor,fill opacity=0.20] ( 95.35, 81.82) circle (  2.13);

\path[fill=fillColor,fill opacity=0.20] ( 98.36, 55.89) circle (  2.13);

\path[fill=fillColor,fill opacity=0.20] ( 86.32, 79.03) circle (  2.13);

\path[fill=fillColor,fill opacity=0.20] ( 87.33, 94.21) circle (  2.13);

\path[fill=fillColor,fill opacity=0.20] ( 89.33, 92.69) circle (  2.13);

\path[fill=fillColor,fill opacity=0.20] ( 87.33, 93.83) circle (  2.13);

\path[fill=fillColor,fill opacity=0.20] ( 88.33,100.66) circle (  2.13);

\path[fill=fillColor,fill opacity=0.20] ( 90.33,103.44) circle (  2.13);

\path[fill=fillColor,fill opacity=0.20] ( 97.36, 99.52) circle (  2.13);

\path[fill=fillColor,fill opacity=0.20] ( 97.36, 88.77) circle (  2.13);

\path[fill=fillColor,fill opacity=0.20] (105.38, 70.06) circle (  2.13);

\path[fill=fillColor,fill opacity=0.20] ( 84.32, 75.75) circle (  2.13);

\path[fill=fillColor,fill opacity=0.20] ( 76.29,104.07) circle (  2.13);

\path[fill=fillColor,fill opacity=0.20] ( 77.29, 97.50) circle (  2.13);

\path[fill=fillColor,fill opacity=0.20] ( 82.31,100.03) circle (  2.13);

\path[fill=fillColor,fill opacity=0.20] ( 75.29,102.18) circle (  2.13);

\path[fill=fillColor,fill opacity=0.20] ( 75.29,100.91) circle (  2.13);

\path[fill=fillColor,fill opacity=0.20] ( 75.29,101.42) circle (  2.13);

\path[fill=fillColor,fill opacity=0.20] ( 74.28,103.95) circle (  2.13);

\path[fill=fillColor,fill opacity=0.20] ( 78.30,106.98) circle (  2.13);

\path[fill=fillColor,fill opacity=0.20] ( 79.30,103.44) circle (  2.13);

\path[fill=fillColor,fill opacity=0.20] ( 84.32, 90.29) circle (  2.13);

\path[fill=fillColor,fill opacity=0.20] ( 81.31, 79.03) circle (  2.13);

\path[fill=fillColor,fill opacity=0.20] ( 90.33, 62.22) circle (  2.13);

\path[fill=fillColor,fill opacity=0.20] ( 83.31, 61.08) circle (  2.13);

\path[fill=fillColor,fill opacity=0.20] ( 77.29, 88.90) circle (  2.13);

\path[fill=fillColor,fill opacity=0.20] ( 90.33, 94.46) circle (  2.13);

\path[fill=fillColor,fill opacity=0.20] ( 86.32, 95.35) circle (  2.13);

\path[fill=fillColor,fill opacity=0.20] ( 89.33,102.55) circle (  2.13);

\path[fill=fillColor,fill opacity=0.20] ( 93.34,104.07) circle (  2.13);

\path[fill=fillColor,fill opacity=0.20] ( 87.33, 99.90) circle (  2.13);

\path[fill=fillColor,fill opacity=0.20] ( 92.34, 91.17) circle (  2.13);

\path[fill=fillColor,fill opacity=0.20] ( 96.35, 78.02) circle (  2.13);

\path[fill=fillColor,fill opacity=0.20] (111.40, 65.88) circle (  2.13);

\path[fill=fillColor,fill opacity=0.20] ( 82.31, 69.30) circle (  2.13);

\path[fill=fillColor,fill opacity=0.20] ( 78.30, 96.99) circle (  2.13);

\path[fill=fillColor,fill opacity=0.20] ( 76.29, 92.57) circle (  2.13);

\path[fill=fillColor,fill opacity=0.20] ( 80.30,100.15) circle (  2.13);

\path[fill=fillColor,fill opacity=0.20] ( 76.29,107.23) circle (  2.13);

\path[fill=fillColor,fill opacity=0.20] ( 75.29,103.69) circle (  2.13);

\path[fill=fillColor,fill opacity=0.20] ( 75.29, 96.36) circle (  2.13);

\path[fill=fillColor,fill opacity=0.20] ( 75.29, 93.70) circle (  2.13);

\path[fill=fillColor,fill opacity=0.20] ( 72.28, 98.63) circle (  2.13);

\path[fill=fillColor,fill opacity=0.20] ( 79.30,101.42) circle (  2.13);

\path[fill=fillColor,fill opacity=0.20] ( 78.30, 93.96) circle (  2.13);

\path[fill=fillColor,fill opacity=0.20] ( 80.30, 83.71) circle (  2.13);

\path[fill=fillColor,fill opacity=0.20] ( 83.31, 64.87) circle (  2.13);

\path[fill=fillColor,fill opacity=0.20] (100.37, 40.21) circle (  2.13);

\path[fill=fillColor,fill opacity=0.20] ( 94.35, 64.24) circle (  2.13);

\path[fill=fillColor,fill opacity=0.20] ( 88.33, 84.47) circle (  2.13);

\path[fill=fillColor,fill opacity=0.20] ( 90.33, 93.07) circle (  2.13);

\path[fill=fillColor,fill opacity=0.20] ( 93.34, 92.19) circle (  2.13);

\path[fill=fillColor,fill opacity=0.20] ( 86.32, 92.06) circle (  2.13);

\path[fill=fillColor,fill opacity=0.20] ( 90.33, 93.96) circle (  2.13);

\path[fill=fillColor,fill opacity=0.20] ( 89.33, 89.40) circle (  2.13);

\path[fill=fillColor,fill opacity=0.20] ( 95.35, 78.66) circle (  2.13);

\path[fill=fillColor,fill opacity=0.20] ( 96.35, 71.57) circle (  2.13);

\path[fill=fillColor,fill opacity=0.20] (105.38, 64.62) circle (  2.13);

\path[fill=fillColor,fill opacity=0.20] ( 89.33, 68.16) circle (  2.13);

\path[fill=fillColor,fill opacity=0.20] ( 80.30, 98.76) circle (  2.13);

\path[fill=fillColor,fill opacity=0.20] ( 77.29, 89.15) circle (  2.13);

\path[fill=fillColor,fill opacity=0.20] ( 80.30, 92.69) circle (  2.13);

\path[fill=fillColor,fill opacity=0.20] ( 78.30,100.28) circle (  2.13);

\path[fill=fillColor,fill opacity=0.20] ( 73.28, 97.75) circle (  2.13);

\path[fill=fillColor,fill opacity=0.20] ( 79.30, 96.74) circle (  2.13);

\path[fill=fillColor,fill opacity=0.20] ( 79.30, 96.86) circle (  2.13);

\path[fill=fillColor,fill opacity=0.20] ( 75.29, 94.97) circle (  2.13);

\path[fill=fillColor,fill opacity=0.20] ( 76.29, 92.06) circle (  2.13);

\path[fill=fillColor,fill opacity=0.20] ( 74.28, 86.24) circle (  2.13);

\path[fill=fillColor,fill opacity=0.20] ( 77.29, 85.74) circle (  2.13);

\path[fill=fillColor,fill opacity=0.20] ( 90.33, 82.95) circle (  2.13);

\path[fill=fillColor,fill opacity=0.20] (111.40, 38.07) circle (  2.13);

\path[fill=fillColor,fill opacity=0.20] ( 86.32, 57.92) circle (  2.13);

\path[fill=fillColor,fill opacity=0.20] ( 85.32, 80.68) circle (  2.13);

\path[fill=fillColor,fill opacity=0.20] ( 91.34, 85.86) circle (  2.13);

\path[fill=fillColor,fill opacity=0.20] ( 86.32, 93.96) circle (  2.13);

\path[fill=fillColor,fill opacity=0.20] ( 83.31,102.30) circle (  2.13);

\path[fill=fillColor,fill opacity=0.20] ( 90.33, 94.84) circle (  2.13);

\path[fill=fillColor,fill opacity=0.20] ( 91.34, 88.39) circle (  2.13);

\path[fill=fillColor,fill opacity=0.20] ( 91.34, 84.60) circle (  2.13);

\path[fill=fillColor,fill opacity=0.20] ( 93.34, 80.43) circle (  2.13);

\path[fill=fillColor,fill opacity=0.20] (103.38, 73.34) circle (  2.13);

\path[fill=fillColor,fill opacity=0.20] ( 95.35, 64.62) circle (  2.13);

\path[fill=fillColor,fill opacity=0.20] ( 84.32, 95.60) circle (  2.13);

\path[fill=fillColor,fill opacity=0.20] ( 79.30, 89.53) circle (  2.13);

\path[fill=fillColor,fill opacity=0.20] ( 74.28, 92.82) circle (  2.13);

\path[fill=fillColor,fill opacity=0.20] ( 77.29,103.69) circle (  2.13);

\path[fill=fillColor,fill opacity=0.20] ( 72.28,104.96) circle (  2.13);

\path[fill=fillColor,fill opacity=0.20] ( 73.28, 99.01) circle (  2.13);

\path[fill=fillColor,fill opacity=0.20] ( 76.29, 95.98) circle (  2.13);

\path[fill=fillColor,fill opacity=0.20] ( 75.29, 99.01) circle (  2.13);

\path[fill=fillColor,fill opacity=0.20] ( 71.27, 99.39) circle (  2.13);

\path[fill=fillColor,fill opacity=0.20] ( 72.28, 90.29) circle (  2.13);

\path[fill=fillColor,fill opacity=0.20] ( 45.69, 81.31) circle (  2.13);

\path[fill=fillColor,fill opacity=0.20] ( 78.30, 77.01) circle (  2.13);

\path[fill=fillColor,fill opacity=0.20] ( 98.36, 69.42) circle (  2.13);

\path[fill=fillColor,fill opacity=0.20] ( 85.32, 60.07) circle (  2.13);

\path[fill=fillColor,fill opacity=0.20] ( 81.31, 82.70) circle (  2.13);

\path[fill=fillColor,fill opacity=0.20] ( 82.31, 98.89) circle (  2.13);

\path[fill=fillColor,fill opacity=0.20] ( 76.29,105.97) circle (  2.13);

\path[fill=fillColor,fill opacity=0.20] ( 85.32, 97.88) circle (  2.13);

\path[fill=fillColor,fill opacity=0.20] ( 89.33, 94.08) circle (  2.13);

\path[fill=fillColor,fill opacity=0.20] ( 89.33, 88.90) circle (  2.13);

\path[fill=fillColor,fill opacity=0.20] ( 97.36, 82.58) circle (  2.13);

\path[fill=fillColor,fill opacity=0.20] (102.37, 80.55) circle (  2.13);

\path[fill=fillColor,fill opacity=0.20] (107.39, 70.82) circle (  2.13);

\path[fill=fillColor,fill opacity=0.20] ( 93.34, 65.25) circle (  2.13);

\path[fill=fillColor,fill opacity=0.20] ( 81.31, 89.53) circle (  2.13);

\path[fill=fillColor,fill opacity=0.20] ( 83.31, 86.75) circle (  2.13);

\path[fill=fillColor,fill opacity=0.20] ( 77.29, 92.31) circle (  2.13);

\path[fill=fillColor,fill opacity=0.20] ( 72.28,102.18) circle (  2.13);

\path[fill=fillColor,fill opacity=0.20] ( 69.27,108.37) circle (  2.13);

\path[fill=fillColor,fill opacity=0.20] ( 66.66,110.52) circle (  2.13);

\path[fill=fillColor,fill opacity=0.20] ( 72.28,108.62) circle (  2.13);

\path[fill=fillColor,fill opacity=0.20] ( 73.28,102.05) circle (  2.13);

\path[fill=fillColor,fill opacity=0.20] ( 71.27, 96.74) circle (  2.13);

\path[fill=fillColor,fill opacity=0.20] ( 73.28, 94.08) circle (  2.13);

\path[fill=fillColor,fill opacity=0.20] ( 67.96, 90.54) circle (  2.13);

\path[fill=fillColor,fill opacity=0.20] ( 75.29, 83.21) circle (  2.13);

\path[fill=fillColor,fill opacity=0.20] ( 95.35, 64.24) circle (  2.13);

\path[fill=fillColor,fill opacity=0.20] ( 74.28, 60.95) circle (  2.13);

\path[fill=fillColor,fill opacity=0.20] ( 79.30, 79.79) circle (  2.13);

\path[fill=fillColor,fill opacity=0.20] ( 78.30, 87.13) circle (  2.13);

\path[fill=fillColor,fill opacity=0.20] ( 80.30, 90.54) circle (  2.13);

\path[fill=fillColor,fill opacity=0.20] ( 89.33, 87.76) circle (  2.13);

\path[fill=fillColor,fill opacity=0.20] ( 87.33, 82.70) circle (  2.13);

\path[fill=fillColor,fill opacity=0.20] ( 91.34, 81.69) circle (  2.13);

\path[fill=fillColor,fill opacity=0.20] (102.37, 80.43) circle (  2.13);

\path[fill=fillColor,fill opacity=0.20] ( 99.36, 77.90) circle (  2.13);

\path[fill=fillColor,fill opacity=0.20] (104.38, 77.90) circle (  2.13);

\path[fill=fillColor,fill opacity=0.20] ( 99.36, 74.74) circle (  2.13);

\path[fill=fillColor,fill opacity=0.20] ( 80.30, 96.49) circle (  2.13);

\path[fill=fillColor,fill opacity=0.20] ( 66.26, 83.97) circle (  2.13);

\path[fill=fillColor,fill opacity=0.20] ( 75.29, 85.48) circle (  2.13);

\path[fill=fillColor,fill opacity=0.20] ( 71.27, 99.65) circle (  2.13);

\path[fill=fillColor,fill opacity=0.20] ( 69.27,100.66) circle (  2.13);

\path[fill=fillColor,fill opacity=0.20] ( 66.66, 98.26) circle (  2.13);

\path[fill=fillColor,fill opacity=0.20] ( 68.26,103.06) circle (  2.13);

\path[fill=fillColor,fill opacity=0.20] ( 72.28,108.12) circle (  2.13);

\path[fill=fillColor,fill opacity=0.20] ( 72.28,104.83) circle (  2.13);

\path[fill=fillColor,fill opacity=0.20] ( 73.28, 96.36) circle (  2.13);

\path[fill=fillColor,fill opacity=0.20] ( 76.29, 93.45) circle (  2.13);

\path[fill=fillColor,fill opacity=0.20] ( 75.29, 89.15) circle (  2.13);

\path[fill=fillColor,fill opacity=0.20] ( 96.35, 70.56) circle (  2.13);

\path[fill=fillColor,fill opacity=0.20] ( 75.29, 51.85) circle (  2.13);

\path[fill=fillColor,fill opacity=0.20] ( 76.29, 64.11) circle (  2.13);

\path[fill=fillColor,fill opacity=0.20] ( 86.32, 80.93) circle (  2.13);

\path[fill=fillColor,fill opacity=0.20] ( 89.33, 87.13) circle (  2.13);

\path[fill=fillColor,fill opacity=0.20] ( 85.32, 80.43) circle (  2.13);

\path[fill=fillColor,fill opacity=0.20] ( 89.33, 82.58) circle (  2.13);

\path[fill=fillColor,fill opacity=0.20] ( 96.35, 85.61) circle (  2.13);

\path[fill=fillColor,fill opacity=0.20] ( 99.36, 84.35) circle (  2.13);

\path[fill=fillColor,fill opacity=0.20] (105.38, 87.25) circle (  2.13);

\path[fill=fillColor,fill opacity=0.20] ( 96.35, 87.25) circle (  2.13);

\path[fill=fillColor,fill opacity=0.20] (105.38, 79.41) circle (  2.13);

\path[fill=fillColor,fill opacity=0.20] (112.41, 62.85) circle (  2.13);

\path[fill=fillColor,fill opacity=0.20] ( 93.34, 90.92) circle (  2.13);

\path[fill=fillColor,fill opacity=0.20] ( 85.32, 93.70) circle (  2.13);

\path[fill=fillColor,fill opacity=0.20] ( 72.28, 89.91) circle (  2.13);

\path[fill=fillColor,fill opacity=0.20] ( 77.29, 90.92) circle (  2.13);

\path[fill=fillColor,fill opacity=0.20] ( 74.28, 92.57) circle (  2.13);

\path[fill=fillColor,fill opacity=0.20] ( 73.28, 94.59) circle (  2.13);

\path[fill=fillColor,fill opacity=0.20] ( 70.27, 92.44) circle (  2.13);

\path[fill=fillColor,fill opacity=0.20] ( 67.66, 88.39) circle (  2.13);

\path[fill=fillColor,fill opacity=0.20] ( 67.66, 91.93) circle (  2.13);

\path[fill=fillColor,fill opacity=0.20] ( 69.27,100.40) circle (  2.13);

\path[fill=fillColor,fill opacity=0.20] ( 73.28,101.16) circle (  2.13);

\path[fill=fillColor,fill opacity=0.20] ( 76.29, 96.86) circle (  2.13);

\path[fill=fillColor,fill opacity=0.20] ( 90.33, 94.08) circle (  2.13);

\path[fill=fillColor,fill opacity=0.20] ( 95.35, 74.23) circle (  2.13);

\path[fill=fillColor,fill opacity=0.20] ( 93.34, 44.39) circle (  2.13);

\path[fill=fillColor,fill opacity=0.20] ( 88.33, 64.24) circle (  2.13);

\path[fill=fillColor,fill opacity=0.20] ( 86.32, 86.75) circle (  2.13);

\path[fill=fillColor,fill opacity=0.20] ( 83.31, 86.37) circle (  2.13);

\path[fill=fillColor,fill opacity=0.20] ( 86.32, 82.83) circle (  2.13);

\path[fill=fillColor,fill opacity=0.20] ( 93.34, 86.50) circle (  2.13);

\path[fill=fillColor,fill opacity=0.20] (101.37, 86.24) circle (  2.13);

\path[fill=fillColor,fill opacity=0.20] (105.38, 79.16) circle (  2.13);

\path[fill=fillColor,fill opacity=0.20] (104.38, 76.76) circle (  2.13);

\path[fill=fillColor,fill opacity=0.20] (103.38, 84.22) circle (  2.13);

\path[fill=fillColor,fill opacity=0.20] (107.39, 91.05) circle (  2.13);

\path[fill=fillColor,fill opacity=0.20] (108.39, 84.98) circle (  2.13);

\path[fill=fillColor,fill opacity=0.20] (115.42, 68.67) circle (  2.13);

\path[fill=fillColor,fill opacity=0.20] (106.39, 68.92) circle (  2.13);

\path[fill=fillColor,fill opacity=0.20] ( 97.36, 75.49) circle (  2.13);

\path[fill=fillColor,fill opacity=0.20] ( 86.32, 81.18) circle (  2.13);

\path[fill=fillColor,fill opacity=0.20] ( 81.31, 83.97) circle (  2.13);

\path[fill=fillColor,fill opacity=0.20] ( 57.13, 84.35) circle (  2.13);

\path[fill=fillColor,fill opacity=0.20] ( 76.29, 87.25) circle (  2.13);

\path[fill=fillColor,fill opacity=0.20] ( 82.31, 96.86) circle (  2.13);

\path[fill=fillColor,fill opacity=0.20] ( 73.28, 98.13) circle (  2.13);

\path[fill=fillColor,fill opacity=0.20] ( 71.27, 90.79) circle (  2.13);

\path[fill=fillColor,fill opacity=0.20] ( 71.27, 87.63) circle (  2.13);

\path[fill=fillColor,fill opacity=0.20] ( 69.27, 86.87) circle (  2.13);

\path[fill=fillColor,fill opacity=0.20] ( 66.46, 84.22) circle (  2.13);

\path[fill=fillColor,fill opacity=0.20] ( 74.28, 85.74) circle (  2.13);

\path[fill=fillColor,fill opacity=0.20] ( 78.30, 90.54) circle (  2.13);

\path[fill=fillColor,fill opacity=0.20] ( 92.34, 82.20) circle (  2.13);

\path[fill=fillColor,fill opacity=0.20] (102.37, 41.23) circle (  2.13);

\path[fill=fillColor,fill opacity=0.20] ( 90.33, 61.21) circle (  2.13);

\path[fill=fillColor,fill opacity=0.20] ( 83.31, 78.40) circle (  2.13);

\path[fill=fillColor,fill opacity=0.20] ( 85.32, 80.17) circle (  2.13);

\path[fill=fillColor,fill opacity=0.20] ( 91.34, 83.71) circle (  2.13);

\path[fill=fillColor,fill opacity=0.20] ( 90.33, 88.90) circle (  2.13);

\path[fill=fillColor,fill opacity=0.20] ( 93.34, 82.07) circle (  2.13);

\path[fill=fillColor,fill opacity=0.20] ( 95.35, 78.66) circle (  2.13);

\path[fill=fillColor,fill opacity=0.20] (100.37, 84.85) circle (  2.13);

\path[fill=fillColor,fill opacity=0.20] (101.37, 91.30) circle (  2.13);

\path[fill=fillColor,fill opacity=0.20] (100.37, 90.54) circle (  2.13);

\path[fill=fillColor,fill opacity=0.20] ( 99.36, 80.93) circle (  2.13);

\path[fill=fillColor,fill opacity=0.20] (105.38, 72.71) circle (  2.13);

\path[fill=fillColor,fill opacity=0.20] (109.40, 75.12) circle (  2.13);

\path[fill=fillColor,fill opacity=0.20] (111.40, 68.67) circle (  2.13);

\path[fill=fillColor,fill opacity=0.20] (107.39, 62.72) circle (  2.13);

\path[fill=fillColor,fill opacity=0.20] (106.39, 64.24) circle (  2.13);

\path[fill=fillColor,fill opacity=0.20] ( 78.30, 65.50) circle (  2.13);

\path[fill=fillColor,fill opacity=0.20] ( 87.33, 75.62) circle (  2.13);

\path[fill=fillColor,fill opacity=0.20] ( 83.31, 79.67) circle (  2.13);

\path[fill=fillColor,fill opacity=0.20] ( 73.28, 74.23) circle (  2.13);

\path[fill=fillColor,fill opacity=0.20] ( 77.29, 78.15) circle (  2.13);

\path[fill=fillColor,fill opacity=0.20] ( 76.29, 87.25) circle (  2.13);

\path[fill=fillColor,fill opacity=0.20] ( 84.32, 88.65) circle (  2.13);

\path[fill=fillColor,fill opacity=0.20] ( 76.29, 93.07) circle (  2.13);

\path[fill=fillColor,fill opacity=0.20] ( 69.27, 93.96) circle (  2.13);

\path[fill=fillColor,fill opacity=0.20] ( 61.24, 87.76) circle (  2.13);

\path[fill=fillColor,fill opacity=0.20] ( 73.28, 87.51) circle (  2.13);

\path[fill=fillColor,fill opacity=0.20] ( 77.29, 85.99) circle (  2.13);

\path[fill=fillColor,fill opacity=0.20] ( 75.29, 76.51) circle (  2.13);

\path[fill=fillColor,fill opacity=0.20] ( 86.32, 69.30) circle (  2.13);

\path[fill=fillColor,fill opacity=0.20] ( 89.33, 66.64) circle (  2.13);

\path[fill=fillColor,fill opacity=0.20] (100.37, 51.22) circle (  2.13);

\path[fill=fillColor,fill opacity=0.20] ( 89.33, 60.83) circle (  2.13);

\path[fill=fillColor,fill opacity=0.20] ( 87.33, 73.85) circle (  2.13);

\path[fill=fillColor,fill opacity=0.20] ( 82.31, 91.30) circle (  2.13);

\path[fill=fillColor,fill opacity=0.20] ( 82.31, 96.61) circle (  2.13);

\path[fill=fillColor,fill opacity=0.20] ( 90.33, 89.40) circle (  2.13);

\path[fill=fillColor,fill opacity=0.20] ( 97.36, 90.04) circle (  2.13);

\path[fill=fillColor,fill opacity=0.20] ( 96.35, 88.01) circle (  2.13);

\path[fill=fillColor,fill opacity=0.20] (101.37, 82.70) circle (  2.13);

\path[fill=fillColor,fill opacity=0.20] (101.37, 80.68) circle (  2.13);

\path[fill=fillColor,fill opacity=0.20] (103.38, 78.02) circle (  2.13);

\path[fill=fillColor,fill opacity=0.20] (107.39, 76.89) circle (  2.13);

\path[fill=fillColor,fill opacity=0.20] (106.39, 79.16) circle (  2.13);

\path[fill=fillColor,fill opacity=0.20] (103.38, 79.54) circle (  2.13);

\path[fill=fillColor,fill opacity=0.20] (102.37, 77.26) circle (  2.13);

\path[fill=fillColor,fill opacity=0.20] (102.37, 75.37) circle (  2.13);

\path[fill=fillColor,fill opacity=0.20] (104.38, 69.42) circle (  2.13);

\path[fill=fillColor,fill opacity=0.20] ( 98.36, 75.62) circle (  2.13);

\path[fill=fillColor,fill opacity=0.20] (108.39, 83.21) circle (  2.13);

\path[fill=fillColor,fill opacity=0.20] (109.40, 69.80) circle (  2.13);

\path[fill=fillColor,fill opacity=0.20] ( 99.36, 60.32) circle (  2.13);

\path[fill=fillColor,fill opacity=0.20] ( 98.36, 63.86) circle (  2.13);

\path[fill=fillColor,fill opacity=0.20] (101.37, 65.38) circle (  2.13);

\path[fill=fillColor,fill opacity=0.20] (102.37, 61.33) circle (  2.13);

\path[fill=fillColor,fill opacity=0.20] (103.38, 60.83) circle (  2.13);

\path[fill=fillColor,fill opacity=0.20] ( 96.35, 67.40) circle (  2.13);

\path[fill=fillColor,fill opacity=0.20] (100.37, 71.32) circle (  2.13);

\path[fill=fillColor,fill opacity=0.20] (101.37, 68.54) circle (  2.13);

\path[fill=fillColor,fill opacity=0.20] ( 99.36, 60.45) circle (  2.13);

\path[fill=fillColor,fill opacity=0.20] ( 98.36, 58.80) circle (  2.13);

\path[fill=fillColor,fill opacity=0.20] (100.37, 64.24) circle (  2.13);

\path[fill=fillColor,fill opacity=0.20] (100.37, 69.05) circle (  2.13);

\path[fill=fillColor,fill opacity=0.20] ( 94.35, 72.59) circle (  2.13);

\path[fill=fillColor,fill opacity=0.20] ( 88.33, 74.99) circle (  2.13);

\path[fill=fillColor,fill opacity=0.20] ( 81.31, 71.07) circle (  2.13);

\path[fill=fillColor,fill opacity=0.20] ( 79.30, 72.46) circle (  2.13);

\path[fill=fillColor,fill opacity=0.20] ( 85.32, 79.03) circle (  2.13);

\path[fill=fillColor,fill opacity=0.20] ( 81.31, 80.17) circle (  2.13);

\path[fill=fillColor,fill opacity=0.20] ( 76.29, 83.46) circle (  2.13);

\path[fill=fillColor,fill opacity=0.20] ( 79.30, 89.02) circle (  2.13);

\path[fill=fillColor,fill opacity=0.20] ( 80.30, 91.55) circle (  2.13);

\path[fill=fillColor,fill opacity=0.20] ( 81.31, 92.06) circle (  2.13);

\path[fill=fillColor,fill opacity=0.20] ( 78.30, 86.12) circle (  2.13);

\path[fill=fillColor,fill opacity=0.20] ( 82.31, 76.63) circle (  2.13);

\path[fill=fillColor,fill opacity=0.20] ( 90.33, 77.39) circle (  2.13);

\path[fill=fillColor,fill opacity=0.20] ( 76.29, 73.09) circle (  2.13);

\path[fill=fillColor,fill opacity=0.20] ( 88.33, 66.14) circle (  2.13);

\path[fill=fillColor,fill opacity=0.20] ( 95.35, 65.50) circle (  2.13);

\path[fill=fillColor,fill opacity=0.20] ( 89.33, 53.62) circle (  2.13);

\path[fill=fillColor,fill opacity=0.20] ( 91.34, 68.29) circle (  2.13);

\path[fill=fillColor,fill opacity=0.20] ( 88.33, 83.21) circle (  2.13);

\path[fill=fillColor,fill opacity=0.20] ( 87.33, 86.62) circle (  2.13);

\path[fill=fillColor,fill opacity=0.20] ( 89.33, 85.48) circle (  2.13);

\path[fill=fillColor,fill opacity=0.20] ( 92.34, 90.16) circle (  2.13);

\path[fill=fillColor,fill opacity=0.20] ( 97.36, 88.14) circle (  2.13);

\path[fill=fillColor,fill opacity=0.20] (102.37, 84.98) circle (  2.13);

\path[fill=fillColor,fill opacity=0.20] (102.37, 84.85) circle (  2.13);

\path[fill=fillColor,fill opacity=0.20] (101.37, 81.44) circle (  2.13);

\path[fill=fillColor,fill opacity=0.20] (103.38, 78.78) circle (  2.13);

\path[fill=fillColor,fill opacity=0.20] ( 97.36, 79.54) circle (  2.13);

\path[fill=fillColor,fill opacity=0.20] ( 88.33, 77.77) circle (  2.13);

\path[fill=fillColor,fill opacity=0.20] (100.37, 76.13) circle (  2.13);

\path[fill=fillColor,fill opacity=0.20] ( 99.36, 77.52) circle (  2.13);

\path[fill=fillColor,fill opacity=0.20] (102.37, 82.07) circle (  2.13);

\path[fill=fillColor,fill opacity=0.20] ( 99.36, 88.14) circle (  2.13);

\path[fill=fillColor,fill opacity=0.20] (102.37, 83.97) circle (  2.13);

\path[fill=fillColor,fill opacity=0.20] ( 98.36, 78.78) circle (  2.13);

\path[fill=fillColor,fill opacity=0.20] (103.38, 82.20) circle (  2.13);

\path[fill=fillColor,fill opacity=0.20] ( 96.35, 82.58) circle (  2.13);

\path[fill=fillColor,fill opacity=0.20] ( 93.34, 76.89) circle (  2.13);

\path[fill=fillColor,fill opacity=0.20] ( 91.34, 76.89) circle (  2.13);

\path[fill=fillColor,fill opacity=0.20] ( 89.33, 77.39) circle (  2.13);

\path[fill=fillColor,fill opacity=0.20] ( 92.34, 76.00) circle (  2.13);

\path[fill=fillColor,fill opacity=0.20] ( 92.34, 74.48) circle (  2.13);

\path[fill=fillColor,fill opacity=0.20] ( 90.33, 69.55) circle (  2.13);

\path[fill=fillColor,fill opacity=0.20] ( 90.33, 66.26) circle (  2.13);

\path[fill=fillColor,fill opacity=0.20] ( 86.32, 69.80) circle (  2.13);

\path[fill=fillColor,fill opacity=0.20] ( 86.32, 75.24) circle (  2.13);

\path[fill=fillColor,fill opacity=0.20] ( 82.31, 77.01) circle (  2.13);

\path[fill=fillColor,fill opacity=0.20] ( 81.31, 76.63) circle (  2.13);

\path[fill=fillColor,fill opacity=0.20] ( 87.33, 87.13) circle (  2.13);

\path[fill=fillColor,fill opacity=0.20] ( 83.31,101.54) circle (  2.13);

\path[fill=fillColor,fill opacity=0.20] ( 87.33, 92.82) circle (  2.13);

\path[fill=fillColor,fill opacity=0.20] ( 84.32, 86.50) circle (  2.13);

\path[fill=fillColor,fill opacity=0.20] ( 83.31, 90.04) circle (  2.13);

\path[fill=fillColor,fill opacity=0.20] ( 86.32, 83.71) circle (  2.13);

\path[fill=fillColor,fill opacity=0.20] ( 88.33, 78.91) circle (  2.13);

\path[fill=fillColor,fill opacity=0.20] ( 84.32, 75.24) circle (  2.13);

\path[fill=fillColor,fill opacity=0.20] ( 86.32, 63.99) circle (  2.13);

\path[fill=fillColor,fill opacity=0.20] ( 98.36, 60.07) circle (  2.13);

\path[fill=fillColor,fill opacity=0.20] ( 73.28, 59.81) circle (  2.13);

\path[fill=fillColor,fill opacity=0.20] (111.40, 41.86) circle (  2.13);

\path[fill=fillColor,fill opacity=0.20] ( 89.33, 51.47) circle (  2.13);

\path[fill=fillColor,fill opacity=0.20] ( 88.33, 62.60) circle (  2.13);

\path[fill=fillColor,fill opacity=0.20] ( 91.34, 70.56) circle (  2.13);

\path[fill=fillColor,fill opacity=0.20] ( 87.33, 79.54) circle (  2.13);

\path[fill=fillColor,fill opacity=0.20] ( 91.34, 90.67) circle (  2.13);

\path[fill=fillColor,fill opacity=0.20] ( 92.34, 92.94) circle (  2.13);

\path[fill=fillColor,fill opacity=0.20] ( 96.35, 86.62) circle (  2.13);

\path[fill=fillColor,fill opacity=0.20] ( 97.36, 80.55) circle (  2.13);

\path[fill=fillColor,fill opacity=0.20] ( 95.35, 80.68) circle (  2.13);

\path[fill=fillColor,fill opacity=0.20] ( 97.36, 79.79) circle (  2.13);

\path[fill=fillColor,fill opacity=0.20] ( 99.36, 76.51) circle (  2.13);

\path[fill=fillColor,fill opacity=0.20] ( 96.35, 77.14) circle (  2.13);

\path[fill=fillColor,fill opacity=0.20] ( 95.35, 80.30) circle (  2.13);

\path[fill=fillColor,fill opacity=0.20] ( 97.36, 80.17) circle (  2.13);

\path[fill=fillColor,fill opacity=0.20] ( 96.35, 78.53) circle (  2.13);

\path[fill=fillColor,fill opacity=0.20] ( 96.35, 79.54) circle (  2.13);

\path[fill=fillColor,fill opacity=0.20] (102.37, 80.93) circle (  2.13);

\path[fill=fillColor,fill opacity=0.20] ( 99.36, 82.95) circle (  2.13);

\path[fill=fillColor,fill opacity=0.20] ( 89.33, 82.58) circle (  2.13);

\path[fill=fillColor,fill opacity=0.20] ( 95.35, 81.82) circle (  2.13);

\path[fill=fillColor,fill opacity=0.20] (100.37, 81.31) circle (  2.13);

\path[fill=fillColor,fill opacity=0.20] (100.37, 77.52) circle (  2.13);

\path[fill=fillColor,fill opacity=0.20] ( 95.35, 75.62) circle (  2.13);

\path[fill=fillColor,fill opacity=0.20] ( 97.36, 76.63) circle (  2.13);

\path[fill=fillColor,fill opacity=0.20] ( 93.34, 77.39) circle (  2.13);

\path[fill=fillColor,fill opacity=0.20] ( 92.34, 78.78) circle (  2.13);

\path[fill=fillColor,fill opacity=0.20] ( 89.33, 79.54) circle (  2.13);

\path[fill=fillColor,fill opacity=0.20] ( 91.34, 82.20) circle (  2.13);

\path[fill=fillColor,fill opacity=0.20] ( 82.31, 84.35) circle (  2.13);

\path[fill=fillColor,fill opacity=0.20] ( 81.31, 79.67) circle (  2.13);

\path[fill=fillColor,fill opacity=0.20] ( 73.28, 75.49) circle (  2.13);

\path[fill=fillColor,fill opacity=0.20] ( 89.33, 77.90) circle (  2.13);

\path[fill=fillColor,fill opacity=0.20] ( 82.31, 77.26) circle (  2.13);

\path[fill=fillColor,fill opacity=0.20] ( 87.33, 71.20) circle (  2.13);

\path[fill=fillColor,fill opacity=0.20] ( 87.33, 64.62) circle (  2.13);

\path[fill=fillColor,fill opacity=0.20] (103.38, 46.28) circle (  2.13);

\path[fill=fillColor,fill opacity=0.20] ( 92.34, 54.12) circle (  2.13);

\path[fill=fillColor,fill opacity=0.20] ( 90.33, 69.30) circle (  2.13);

\path[fill=fillColor,fill opacity=0.20] ( 90.33, 79.92) circle (  2.13);

\path[fill=fillColor,fill opacity=0.20] ( 84.32, 76.25) circle (  2.13);

\path[fill=fillColor,fill opacity=0.20] ( 86.32, 73.60) circle (  2.13);

\path[fill=fillColor,fill opacity=0.20] ( 89.33, 78.15) circle (  2.13);

\path[fill=fillColor,fill opacity=0.20] ( 87.33, 80.93) circle (  2.13);

\path[fill=fillColor,fill opacity=0.20] ( 85.32, 80.30) circle (  2.13);

\path[fill=fillColor,fill opacity=0.20] ( 90.33, 82.70) circle (  2.13);

\path[fill=fillColor,fill opacity=0.20] ( 93.34, 86.12) circle (  2.13);

\path[fill=fillColor,fill opacity=0.20] ( 94.35, 90.16) circle (  2.13);

\path[fill=fillColor,fill opacity=0.20] ( 95.35, 88.01) circle (  2.13);

\path[fill=fillColor,fill opacity=0.20] (100.37, 82.07) circle (  2.13);

\path[fill=fillColor,fill opacity=0.20] ( 97.36, 81.44) circle (  2.13);

\path[fill=fillColor,fill opacity=0.20] ( 98.36, 86.24) circle (  2.13);

\path[fill=fillColor,fill opacity=0.20] ( 99.36, 90.42) circle (  2.13);

\path[fill=fillColor,fill opacity=0.20] ( 99.36, 92.31) circle (  2.13);

\path[fill=fillColor,fill opacity=0.20] ( 94.35, 87.76) circle (  2.13);

\path[fill=fillColor,fill opacity=0.20] ( 86.32, 81.18) circle (  2.13);

\path[fill=fillColor,fill opacity=0.20] ( 91.34, 80.17) circle (  2.13);

\path[fill=fillColor,fill opacity=0.20] ( 86.32, 80.93) circle (  2.13);

\path[fill=fillColor,fill opacity=0.20] ( 89.33, 79.16) circle (  2.13);

\path[fill=fillColor,fill opacity=0.20] ( 84.32, 79.79) circle (  2.13);

\path[fill=fillColor,fill opacity=0.20] ( 84.32, 76.00) circle (  2.13);

\path[fill=fillColor,fill opacity=0.20] ( 85.32, 67.40) circle (  2.13);

\path[fill=fillColor,fill opacity=0.20] ( 96.35, 62.34) circle (  2.13);

\path[fill=fillColor,fill opacity=0.20] ( 88.33, 56.78) circle (  2.13);

\path[fill=fillColor,fill opacity=0.20] ( 98.36, 44.39) circle (  2.13);

\path[fill=fillColor,fill opacity=0.20] ( 93.34, 42.49) circle (  2.13);

\path[fill=fillColor,fill opacity=0.20] ( 96.35, 47.93) circle (  2.13);

\path[fill=fillColor,fill opacity=0.20] ( 81.31, 53.11) circle (  2.13);

\path[fill=fillColor,fill opacity=0.20] ( 84.32, 59.56) circle (  2.13);

\path[fill=fillColor,fill opacity=0.20] ( 92.34, 63.36) circle (  2.13);

\path[fill=fillColor,fill opacity=0.20] ( 82.31, 71.32) circle (  2.13);

\path[fill=fillColor,fill opacity=0.20] ( 87.33, 84.09) circle (  2.13);

\path[fill=fillColor,fill opacity=0.20] ( 78.30, 80.68) circle (  2.13);

\path[fill=fillColor,fill opacity=0.20] ( 83.31, 73.34) circle (  2.13);

\path[fill=fillColor,fill opacity=0.20] ( 85.32, 83.46) circle (  2.13);

\path[fill=fillColor,fill opacity=0.20] ( 94.35, 91.93) circle (  2.13);

\path[fill=fillColor,fill opacity=0.20] ( 93.34, 78.91) circle (  2.13);

\path[fill=fillColor,fill opacity=0.20] ( 95.35, 72.08) circle (  2.13);

\path[fill=fillColor,fill opacity=0.20] ( 92.34, 75.24) circle (  2.13);

\path[fill=fillColor,fill opacity=0.20] ( 93.34, 76.89) circle (  2.13);

\path[fill=fillColor,fill opacity=0.20] ( 90.33, 76.25) circle (  2.13);

\path[fill=fillColor,fill opacity=0.20] ( 91.34, 71.45) circle (  2.13);

\path[fill=fillColor,fill opacity=0.20] ( 83.31, 66.64) circle (  2.13);

\path[fill=fillColor,fill opacity=0.20] ( 82.31, 63.61) circle (  2.13);

\path[fill=fillColor,fill opacity=0.20] ( 90.33, 56.91) circle (  2.13);

\path[fill=fillColor,fill opacity=0.20] ( 86.32, 52.61) circle (  2.13);

\path[fill=fillColor,fill opacity=0.20] ( 80.30, 52.48) circle (  2.13);

\path[fill=fillColor,fill opacity=0.20] (115.42, 38.70) circle (  2.13);

\path[fill=fillColor,fill opacity=0.20] (105.38, 43.38) circle (  2.13);

\path[fill=fillColor,fill opacity=0.20] (100.37, 52.48) circle (  2.13);

\path[fill=fillColor,fill opacity=0.20] ( 81.31, 52.35) circle (  2.13);

\path[fill=fillColor,fill opacity=0.20] ( 79.30, 50.08) circle (  2.13);

\path[fill=fillColor,fill opacity=0.20] ( 89.33, 52.73) circle (  2.13);

\path[fill=fillColor,fill opacity=0.20] ( 96.35, 52.99) circle (  2.13);

\path[fill=fillColor,fill opacity=0.20] ( 99.36, 51.85) circle (  2.13);

\path[fill=fillColor,fill opacity=0.20] ( 95.35, 49.95) circle (  2.13);

\path[fill=fillColor,fill opacity=0.20] ( 96.35, 47.55) circle (  2.13);

\path[fill=fillColor,fill opacity=0.20] ( 87.33, 46.03) circle (  2.13);

\path[fill=fillColor,fill opacity=0.20] ( 49.50, 47.30) circle (  2.13);

\path[fill=fillColor,fill opacity=0.20] ( 53.52, 63.61) circle (  2.13);

\path[fill=fillColor,fill opacity=0.20] ( 63.75, 77.14) circle (  2.13);

\path[fill=fillColor,fill opacity=0.20] ( 63.85, 80.43) circle (  2.13);

\path[fill=fillColor,fill opacity=0.20] ( 52.01, 69.68) circle (  2.13);

\path[fill=fillColor,fill opacity=0.20] ( 54.72, 58.93) circle (  2.13);

\path[fill=fillColor,fill opacity=0.20] ( 58.03, 79.16) circle (  2.13);

\path[fill=fillColor,fill opacity=0.20] ( 72.28, 97.75) circle (  2.13);

\path[fill=fillColor,fill opacity=0.20] ( 75.29,106.10) circle (  2.13);

\path[fill=fillColor,fill opacity=0.20] ( 75.29, 98.38) circle (  2.13);

\path[fill=fillColor,fill opacity=0.20] ( 65.96, 95.98) circle (  2.13);

\path[fill=fillColor,fill opacity=0.20] ( 70.27,100.03) circle (  2.13);

\path[fill=fillColor,fill opacity=0.20] ( 58.83, 88.01) circle (  2.13);

\path[fill=fillColor,fill opacity=0.20] ( 54.92, 63.73) circle (  2.13);

\path[fill=fillColor,fill opacity=0.20] ( 61.94,101.42) circle (  2.13);

\path[fill=fillColor,fill opacity=0.20] ( 65.86, 99.27) circle (  2.13);

\path[fill=fillColor,fill opacity=0.20] ( 71.27,108.24) circle (  2.13);

\path[fill=fillColor,fill opacity=0.20] ( 69.27,110.77) circle (  2.13);

\path[fill=fillColor,fill opacity=0.20] ( 70.27, 99.01) circle (  2.13);

\path[fill=fillColor,fill opacity=0.20] ( 68.26,100.03) circle (  2.13);

\path[fill=fillColor,fill opacity=0.20] ( 68.26,107.61) circle (  2.13);

\path[fill=fillColor,fill opacity=0.20] ( 72.28, 96.99) circle (  2.13);

\path[fill=fillColor,fill opacity=0.20] ( 83.31, 73.85) circle (  2.13);

\path[fill=fillColor,fill opacity=0.20] ( 55.22, 46.66) circle (  2.13);

\path[fill=fillColor,fill opacity=0.20] ( 60.84, 93.20) circle (  2.13);

\path[fill=fillColor,fill opacity=0.20] ( 74.28,105.72) circle (  2.13);

\path[fill=fillColor,fill opacity=0.20] ( 68.26, 91.93) circle (  2.13);

\path[fill=fillColor,fill opacity=0.20] ( 60.44,102.05) circle (  2.13);

\path[fill=fillColor,fill opacity=0.20] ( 62.95,104.45) circle (  2.13);

\path[fill=fillColor,fill opacity=0.20] ( 68.26, 99.27) circle (  2.13);

\path[fill=fillColor,fill opacity=0.20] ( 65.76, 99.39) circle (  2.13);

\path[fill=fillColor,fill opacity=0.20] ( 62.75,103.31) circle (  2.13);

\path[fill=fillColor,fill opacity=0.20] ( 67.76, 99.01) circle (  2.13);

\path[fill=fillColor,fill opacity=0.20] ( 85.32, 83.33) circle (  2.13);

\path[fill=fillColor,fill opacity=0.20] ( 78.30, 59.69) circle (  2.13);

\path[fill=fillColor,fill opacity=0.20] ( 67.96,106.98) circle (  2.13);

\path[fill=fillColor,fill opacity=0.20] ( 75.29,102.81) circle (  2.13);

\path[fill=fillColor,fill opacity=0.20] ( 64.85, 97.37) circle (  2.13);

\path[fill=fillColor,fill opacity=0.20] ( 53.42,104.07) circle (  2.13);

\path[fill=fillColor,fill opacity=0.20] ( 56.12,105.84) circle (  2.13);

\path[fill=fillColor,fill opacity=0.20] ( 73.28,111.41) circle (  2.13);

\path[fill=fillColor,fill opacity=0.20] ( 69.27,109.89) circle (  2.13);

\path[fill=fillColor,fill opacity=0.20] ( 58.93,101.16) circle (  2.13);

\path[fill=fillColor,fill opacity=0.20] ( 66.36,102.30) circle (  2.13);

\path[fill=fillColor,fill opacity=0.20] ( 90.33, 92.57) circle (  2.13);

\path[fill=fillColor,fill opacity=0.20] ( 91.34, 61.21) circle (  2.13);

\path[fill=fillColor,fill opacity=0.20] ( 75.29,102.30) circle (  2.13);

\path[fill=fillColor,fill opacity=0.20] ( 69.27,113.43) circle (  2.13);

\path[fill=fillColor,fill opacity=0.20] ( 65.86,110.39) circle (  2.13);

\path[fill=fillColor,fill opacity=0.20] ( 64.95,104.45) circle (  2.13);

\path[fill=fillColor,fill opacity=0.20] ( 64.85,104.83) circle (  2.13);

\path[fill=fillColor,fill opacity=0.20] ( 62.95,105.59) circle (  2.13);

\path[fill=fillColor,fill opacity=0.20] ( 68.26,101.80) circle (  2.13);

\path[fill=fillColor,fill opacity=0.20] ( 93.34, 55.01) circle (  2.13);

\path[fill=fillColor,fill opacity=0.20] ( 94.35, 54.50) circle (  2.13);

\path[fill=fillColor,fill opacity=0.20] ( 88.33, 48.05) circle (  2.13);

\path[fill=fillColor,fill opacity=0.20] ( 87.33, 52.48) circle (  2.13);

\path[fill=fillColor,fill opacity=0.20] ( 83.31, 49.19) circle (  2.13);

\path[fill=fillColor,fill opacity=0.20] ( 91.34, 62.98) circle (  2.13);

\path[fill=fillColor,fill opacity=0.20] ( 81.31, 91.68) circle (  2.13);

\path[fill=fillColor,fill opacity=0.20] ( 74.28,103.82) circle (  2.13);

\path[fill=fillColor,fill opacity=0.20] ( 70.27,110.02) circle (  2.13);

\path[fill=fillColor,fill opacity=0.20] ( 70.27,105.59) circle (  2.13);

\path[fill=fillColor,fill opacity=0.20] ( 66.56,105.21) circle (  2.13);

\path[fill=fillColor,fill opacity=0.20] ( 61.64,109.64) circle (  2.13);

\path[fill=fillColor,fill opacity=0.20] ( 61.74,110.14) circle (  2.13);

\path[fill=fillColor,fill opacity=0.20] ( 58.83,109.13) circle (  2.13);

\path[fill=fillColor,fill opacity=0.20] ( 64.65,101.04) circle (  2.13);

\path[fill=fillColor,fill opacity=0.20] ( 72.28, 48.94) circle (  2.13);

\path[fill=fillColor,fill opacity=0.20] ( 69.27, 72.71) circle (  2.13);

\path[fill=fillColor,fill opacity=0.20] ( 70.27, 84.73) circle (  2.13);

\path[fill=fillColor,fill opacity=0.20] ( 82.31, 91.68) circle (  2.13);

\path[fill=fillColor,fill opacity=0.20] ( 94.35, 86.87) circle (  2.13);

\path[fill=fillColor,fill opacity=0.20] ( 96.35, 74.61) circle (  2.13);

\path[fill=fillColor,fill opacity=0.20] ( 90.33, 55.39) circle (  2.13);

\path[fill=fillColor,fill opacity=0.20] ( 94.35, 40.21) circle (  2.13);

\path[fill=fillColor,fill opacity=0.20] ( 81.31, 79.92) circle (  2.13);

\path[fill=fillColor,fill opacity=0.20] ( 78.30, 95.98) circle (  2.13);

\path[fill=fillColor,fill opacity=0.20] ( 74.28, 96.49) circle (  2.13);

\path[fill=fillColor,fill opacity=0.20] ( 74.28,106.73) circle (  2.13);

\path[fill=fillColor,fill opacity=0.20] ( 65.45,109.13) circle (  2.13);

\path[fill=fillColor,fill opacity=0.20] ( 59.23,103.82) circle (  2.13);

\path[fill=fillColor,fill opacity=0.20] ( 58.73,107.99) circle (  2.13);

\path[fill=fillColor,fill opacity=0.20] ( 75.29,104.20) circle (  2.13);

\path[fill=fillColor,fill opacity=0.20] ( 72.28,103.95) circle (  2.13);

\path[fill=fillColor,fill opacity=0.20] ( 74.28, 93.70) circle (  2.13);

\path[fill=fillColor,fill opacity=0.20] ( 77.29, 90.92) circle (  2.13);

\path[fill=fillColor,fill opacity=0.20] ( 79.30, 89.15) circle (  2.13);

\path[fill=fillColor,fill opacity=0.20] ( 82.31, 83.08) circle (  2.13);

\path[fill=fillColor,fill opacity=0.20] ( 96.35, 77.01) circle (  2.13);

\path[fill=fillColor,fill opacity=0.20] ( 90.33, 66.90) circle (  2.13);

\path[fill=fillColor,fill opacity=0.20] ( 89.33, 55.39) circle (  2.13);

\path[fill=fillColor,fill opacity=0.20] ( 83.31, 40.21) circle (  2.13);

\path[fill=fillColor,fill opacity=0.20] ( 72.28,102.55) circle (  2.13);

\path[fill=fillColor,fill opacity=0.20] ( 73.28,113.30) circle (  2.13);

\path[fill=fillColor,fill opacity=0.20] ( 69.27,107.23) circle (  2.13);

\path[fill=fillColor,fill opacity=0.20] ( 67.96,103.69) circle (  2.13);

\path[fill=fillColor,fill opacity=0.20] ( 61.24,111.15) circle (  2.13);

\path[fill=fillColor,fill opacity=0.20] ( 61.44,105.21) circle (  2.13);

\path[fill=fillColor,fill opacity=0.20] ( 69.27,105.72) circle (  2.13);

\path[fill=fillColor,fill opacity=0.20] ( 88.33, 99.39) circle (  2.13);

\path[fill=fillColor,fill opacity=0.20] ( 70.27, 37.94) circle (  2.13);

\path[fill=fillColor,fill opacity=0.20] ( 72.28, 73.72) circle (  2.13);

\path[fill=fillColor,fill opacity=0.20] ( 72.28,112.67) circle (  2.13);

\path[fill=fillColor,fill opacity=0.20] ( 72.28,104.58) circle (  2.13);

\path[fill=fillColor,fill opacity=0.20] ( 72.28, 97.37) circle (  2.13);

\path[fill=fillColor,fill opacity=0.20] ( 77.29,102.43) circle (  2.13);

\path[fill=fillColor,fill opacity=0.20] ( 77.29, 97.62) circle (  2.13);

\path[fill=fillColor,fill opacity=0.20] ( 91.34, 91.55) circle (  2.13);

\path[fill=fillColor,fill opacity=0.20] ( 91.34, 94.34) circle (  2.13);

\path[fill=fillColor,fill opacity=0.20] ( 86.32, 91.55) circle (  2.13);

\path[fill=fillColor,fill opacity=0.20] ( 90.33, 77.77) circle (  2.13);

\path[fill=fillColor,fill opacity=0.20] ( 87.33, 68.79) circle (  2.13);

\path[fill=fillColor,fill opacity=0.20] ( 78.30, 97.50) circle (  2.13);

\path[fill=fillColor,fill opacity=0.20] ( 66.86, 98.63) circle (  2.13);

\path[fill=fillColor,fill opacity=0.20] ( 70.27,111.41) circle (  2.13);

\path[fill=fillColor,fill opacity=0.20] ( 70.27,106.98) circle (  2.13);

\path[fill=fillColor,fill opacity=0.20] ( 60.24, 98.63) circle (  2.13);

\path[fill=fillColor,fill opacity=0.20] ( 63.95,102.68) circle (  2.13);

\path[fill=fillColor,fill opacity=0.20] ( 63.75,105.46) circle (  2.13);

\path[fill=fillColor,fill opacity=0.20] ( 59.84,103.06) circle (  2.13);

\path[fill=fillColor,fill opacity=0.20] ( 62.55,101.29) circle (  2.13);

\path[fill=fillColor,fill opacity=0.20] ( 84.32,100.78) circle (  2.13);

\path[fill=fillColor,fill opacity=0.20] ( 50.91, 47.68) circle (  2.13);

\path[fill=fillColor,fill opacity=0.20] ( 79.30,102.18) circle (  2.13);

\path[fill=fillColor,fill opacity=0.20] ( 56.02,106.22) circle (  2.13);

\path[fill=fillColor,fill opacity=0.20] ( 77.29,109.13) circle (  2.13);

\path[fill=fillColor,fill opacity=0.20] ( 75.29,115.07) circle (  2.13);

\path[fill=fillColor,fill opacity=0.20] ( 82.31,102.43) circle (  2.13);

\path[fill=fillColor,fill opacity=0.20] ( 88.33, 96.49) circle (  2.13);

\path[fill=fillColor,fill opacity=0.20] ( 86.32, 97.75) circle (  2.13);

\path[fill=fillColor,fill opacity=0.20] ( 91.34, 90.79) circle (  2.13);

\path[fill=fillColor,fill opacity=0.20] ( 86.32, 57.41) circle (  2.13);

\path[fill=fillColor,fill opacity=0.20] ( 83.31, 91.17) circle (  2.13);

\path[fill=fillColor,fill opacity=0.20] ( 73.28, 86.50) circle (  2.13);

\path[fill=fillColor,fill opacity=0.20] ( 55.42, 92.06) circle (  2.13);

\path[fill=fillColor,fill opacity=0.20] ( 66.36,101.80) circle (  2.13);

\path[fill=fillColor,fill opacity=0.20] ( 68.26,102.43) circle (  2.13);

\path[fill=fillColor,fill opacity=0.20] ( 59.13,104.45) circle (  2.13);

\path[fill=fillColor,fill opacity=0.20] ( 49.10,101.92) circle (  2.13);

\path[fill=fillColor,fill opacity=0.20] ( 59.74,102.68) circle (  2.13);

\path[fill=fillColor,fill opacity=0.20] ( 68.26,102.18) circle (  2.13);

\path[fill=fillColor,fill opacity=0.20] ( 68.26, 51.97) circle (  2.13);

\path[fill=fillColor,fill opacity=0.20] ( 74.28, 98.63) circle (  2.13);

\path[fill=fillColor,fill opacity=0.20] ( 65.96,115.96) circle (  2.13);

\path[fill=fillColor,fill opacity=0.20] ( 84.32,105.34) circle (  2.13);

\path[fill=fillColor,fill opacity=0.20] ( 93.34, 98.51) circle (  2.13);

\path[fill=fillColor,fill opacity=0.20] ( 97.36, 94.21) circle (  2.13);

\path[fill=fillColor,fill opacity=0.20] (100.37, 81.94) circle (  2.13);

\path[fill=fillColor,fill opacity=0.20] ( 78.30, 54.63) circle (  2.13);

\path[fill=fillColor,fill opacity=0.20] ( 88.33, 84.60) circle (  2.13);

\path[fill=fillColor,fill opacity=0.20] ( 81.31, 83.71) circle (  2.13);

\path[fill=fillColor,fill opacity=0.20] ( 71.27, 88.77) circle (  2.13);

\path[fill=fillColor,fill opacity=0.20] ( 68.26, 97.62) circle (  2.13);

\path[fill=fillColor,fill opacity=0.20] ( 69.27,101.16) circle (  2.13);

\path[fill=fillColor,fill opacity=0.20] ( 64.05,107.36) circle (  2.13);

\path[fill=fillColor,fill opacity=0.20] ( 57.03,103.57) circle (  2.13);

\path[fill=fillColor,fill opacity=0.20] ( 60.54,108.12) circle (  2.13);

\path[fill=fillColor,fill opacity=0.20] ( 86.32,111.41) circle (  2.13);

\path[fill=fillColor,fill opacity=0.20] ( 89.33, 83.21) circle (  2.13);

\path[fill=fillColor,fill opacity=0.20] ( 67.66,105.97) circle (  2.13);

\path[fill=fillColor,fill opacity=0.20] ( 62.75,115.20) circle (  2.13);

\path[fill=fillColor,fill opacity=0.20] ( 85.32,108.75) circle (  2.13);

\path[fill=fillColor,fill opacity=0.20] ( 94.35,101.80) circle (  2.13);

\path[fill=fillColor,fill opacity=0.20] ( 98.36, 94.59) circle (  2.13);

\path[fill=fillColor,fill opacity=0.20] (101.37, 87.63) circle (  2.13);

\path[fill=fillColor,fill opacity=0.20] ( 92.34, 53.49) circle (  2.13);

\path[fill=fillColor,fill opacity=0.20] ( 82.31, 83.71) circle (  2.13);

\path[fill=fillColor,fill opacity=0.20] ( 80.30, 79.92) circle (  2.13);

\path[fill=fillColor,fill opacity=0.20] ( 77.29, 83.71) circle (  2.13);

\path[fill=fillColor,fill opacity=0.20] ( 74.28, 97.75) circle (  2.13);

\path[fill=fillColor,fill opacity=0.20] ( 72.28,101.80) circle (  2.13);

\path[fill=fillColor,fill opacity=0.20] ( 70.27,101.54) circle (  2.13);

\path[fill=fillColor,fill opacity=0.20] ( 64.95,106.98) circle (  2.13);

\path[fill=fillColor,fill opacity=0.20] ( 66.56,114.19) circle (  2.13);

\path[fill=fillColor,fill opacity=0.20] ( 66.16,100.91) circle (  2.13);

\path[fill=fillColor,fill opacity=0.20] ( 77.29, 97.75) circle (  2.13);

\path[fill=fillColor,fill opacity=0.20] (101.37, 67.02) circle (  2.13);

\path[fill=fillColor,fill opacity=0.20] ( 78.30,100.15) circle (  2.13);

\path[fill=fillColor,fill opacity=0.20] ( 61.14,108.37) circle (  2.13);

\path[fill=fillColor,fill opacity=0.20] ( 63.85,105.97) circle (  2.13);

\path[fill=fillColor,fill opacity=0.20] ( 79.30,106.85) circle (  2.13);

\path[fill=fillColor,fill opacity=0.20] ( 79.30,107.49) circle (  2.13);

\path[fill=fillColor,fill opacity=0.20] ( 87.33,100.15) circle (  2.13);

\path[fill=fillColor,fill opacity=0.20] ( 86.32, 95.09) circle (  2.13);

\path[fill=fillColor,fill opacity=0.20] ( 94.35, 94.46) circle (  2.13);

\path[fill=fillColor,fill opacity=0.20] (113.41, 70.56) circle (  2.13);

\path[fill=fillColor,fill opacity=0.20] ( 96.35, 55.01) circle (  2.13);

\path[fill=fillColor,fill opacity=0.20] ( 89.33, 80.30) circle (  2.13);

\path[fill=fillColor,fill opacity=0.20] ( 81.31, 83.71) circle (  2.13);

\path[fill=fillColor,fill opacity=0.20] ( 74.28, 89.91) circle (  2.13);

\path[fill=fillColor,fill opacity=0.20] ( 69.27, 92.44) circle (  2.13);

\path[fill=fillColor,fill opacity=0.20] ( 67.36, 98.38) circle (  2.13);

\path[fill=fillColor,fill opacity=0.20] ( 76.29,103.95) circle (  2.13);

\path[fill=fillColor,fill opacity=0.20] ( 72.28,103.69) circle (  2.13);

\path[fill=fillColor,fill opacity=0.20] ( 66.06,103.69) circle (  2.13);

\path[fill=fillColor,fill opacity=0.20] ( 70.27,102.43) circle (  2.13);

\path[fill=fillColor,fill opacity=0.20] ( 73.28, 98.76) circle (  2.13);

\path[fill=fillColor,fill opacity=0.20] ( 78.30, 95.73) circle (  2.13);

\path[fill=fillColor,fill opacity=0.20] ( 97.36, 85.10) circle (  2.13);

\path[fill=fillColor,fill opacity=0.20] ( 72.28,101.80) circle (  2.13);

\path[fill=fillColor,fill opacity=0.20] ( 70.27,105.21) circle (  2.13);

\path[fill=fillColor,fill opacity=0.20] ( 73.28,110.77) circle (  2.13);

\path[fill=fillColor,fill opacity=0.20] ( 73.28,105.84) circle (  2.13);

\path[fill=fillColor,fill opacity=0.20] ( 73.28, 97.37) circle (  2.13);

\path[fill=fillColor,fill opacity=0.20] ( 78.30, 96.99) circle (  2.13);

\path[fill=fillColor,fill opacity=0.20] ( 87.33, 90.67) circle (  2.13);

\path[fill=fillColor,fill opacity=0.20] ( 98.36, 74.74) circle (  2.13);

\path[fill=fillColor,fill opacity=0.20] (102.37, 54.50) circle (  2.13);

\path[fill=fillColor,fill opacity=0.20] ( 91.34, 82.07) circle (  2.13);

\path[fill=fillColor,fill opacity=0.20] ( 75.29, 79.29) circle (  2.13);

\path[fill=fillColor,fill opacity=0.20] ( 77.29, 88.65) circle (  2.13);

\path[fill=fillColor,fill opacity=0.20] ( 78.30,102.30) circle (  2.13);

\path[fill=fillColor,fill opacity=0.20] ( 68.26, 99.52) circle (  2.13);

\path[fill=fillColor,fill opacity=0.20] ( 67.26,105.34) circle (  2.13);

\path[fill=fillColor,fill opacity=0.20] ( 73.28,113.81) circle (  2.13);

\path[fill=fillColor,fill opacity=0.20] ( 69.27,110.65) circle (  2.13);

\path[fill=fillColor,fill opacity=0.20] ( 75.29,103.44) circle (  2.13);

\path[fill=fillColor,fill opacity=0.20] ( 84.32, 95.09) circle (  2.13);

\path[fill=fillColor,fill opacity=0.20] ( 91.34, 89.66) circle (  2.13);

\path[fill=fillColor,fill opacity=0.20] (113.41, 61.58) circle (  2.13);

\path[fill=fillColor,fill opacity=0.20] ( 83.31, 79.92) circle (  2.13);

\path[fill=fillColor,fill opacity=0.20] ( 67.66,102.55) circle (  2.13);

\path[fill=fillColor,fill opacity=0.20] ( 74.28,115.96) circle (  2.13);

\path[fill=fillColor,fill opacity=0.20] ( 72.28,105.21) circle (  2.13);

\path[fill=fillColor,fill opacity=0.20] ( 72.28, 99.77) circle (  2.13);

\path[fill=fillColor,fill opacity=0.20] ( 77.29,103.82) circle (  2.13);

\path[fill=fillColor,fill opacity=0.20] ( 82.31, 94.46) circle (  2.13);

\path[fill=fillColor,fill opacity=0.20] ( 89.33, 80.93) circle (  2.13);

\path[fill=fillColor,fill opacity=0.20] (104.38, 69.93) circle (  2.13);

\path[fill=fillColor,fill opacity=0.20] (117.42, 45.78) circle (  2.13);

\path[fill=fillColor,fill opacity=0.20] (110.40, 38.19) circle (  2.13);

\path[fill=fillColor,fill opacity=0.20] ( 94.35, 56.27) circle (  2.13);

\path[fill=fillColor,fill opacity=0.20] ( 89.33, 73.98) circle (  2.13);

\path[fill=fillColor,fill opacity=0.20] ( 82.31, 76.51) circle (  2.13);

\path[fill=fillColor,fill opacity=0.20] ( 75.29, 82.45) circle (  2.13);

\path[fill=fillColor,fill opacity=0.20] ( 71.27, 91.68) circle (  2.13);

\path[fill=fillColor,fill opacity=0.20] ( 70.27,104.45) circle (  2.13);

\path[fill=fillColor,fill opacity=0.20] ( 71.27,111.91) circle (  2.13);

\path[fill=fillColor,fill opacity=0.20] ( 70.27,105.34) circle (  2.13);

\path[fill=fillColor,fill opacity=0.20] ( 88.33,100.53) circle (  2.13);

\path[fill=fillColor,fill opacity=0.20] ( 89.33, 89.78) circle (  2.13);

\path[fill=fillColor,fill opacity=0.20] ( 83.31, 79.16) circle (  2.13);

\path[fill=fillColor,fill opacity=0.20] ( 91.34, 60.32) circle (  2.13);

\path[fill=fillColor,fill opacity=0.20] ( 67.66, 98.76) circle (  2.13);

\path[fill=fillColor,fill opacity=0.20] ( 67.66,100.66) circle (  2.13);

\path[fill=fillColor,fill opacity=0.20] ( 71.27, 98.38) circle (  2.13);

\path[fill=fillColor,fill opacity=0.20] ( 83.31,105.84) circle (  2.13);

\path[fill=fillColor,fill opacity=0.20] ( 83.31, 98.26) circle (  2.13);

\path[fill=fillColor,fill opacity=0.20] ( 84.32, 84.85) circle (  2.13);

\path[fill=fillColor,fill opacity=0.20] ( 91.34, 81.56) circle (  2.13);

\path[fill=fillColor,fill opacity=0.20] ( 91.34, 72.46) circle (  2.13);

\path[fill=fillColor,fill opacity=0.20] (101.37, 48.94) circle (  2.13);

\path[fill=fillColor,fill opacity=0.20] ( 78.30, 47.42) circle (  2.13);

\path[fill=fillColor,fill opacity=0.20] ( 92.34, 65.00) circle (  2.13);

\path[fill=fillColor,fill opacity=0.20] ( 77.29, 67.78) circle (  2.13);

\path[fill=fillColor,fill opacity=0.20] ( 75.29, 69.80) circle (  2.13);

\path[fill=fillColor,fill opacity=0.20] ( 73.28, 84.47) circle (  2.13);

\path[fill=fillColor,fill opacity=0.20] ( 72.28, 99.14) circle (  2.13);

\path[fill=fillColor,fill opacity=0.20] ( 67.46,107.49) circle (  2.13);

\path[fill=fillColor,fill opacity=0.20] ( 79.30, 88.77) circle (  2.13);

\path[fill=fillColor,fill opacity=0.20] ( 81.31,103.95) circle (  2.13);

\path[fill=fillColor,fill opacity=0.20] ( 55.12, 93.32) circle (  2.13);

\path[fill=fillColor,fill opacity=0.20] ( 75.29,101.29) circle (  2.13);

\path[fill=fillColor,fill opacity=0.20] ( 81.31, 97.62) circle (  2.13);

\path[fill=fillColor,fill opacity=0.20] ( 86.32, 93.45) circle (  2.13);

\path[fill=fillColor,fill opacity=0.20] ( 79.30, 89.78) circle (  2.13);

\path[fill=fillColor,fill opacity=0.20] ( 82.31, 85.48) circle (  2.13);

\path[fill=fillColor,fill opacity=0.20] ( 93.34, 79.41) circle (  2.13);

\path[fill=fillColor,fill opacity=0.20] ( 94.35, 54.12) circle (  2.13);

\path[fill=fillColor,fill opacity=0.20] (104.38, 40.59) circle (  2.13);

\path[fill=fillColor,fill opacity=0.20] ( 93.34, 45.91) circle (  2.13);

\path[fill=fillColor,fill opacity=0.20] ( 88.33, 57.92) circle (  2.13);

\path[fill=fillColor,fill opacity=0.20] ( 76.29, 68.29) circle (  2.13);

\path[fill=fillColor,fill opacity=0.20] ( 75.29, 66.39) circle (  2.13);

\path[fill=fillColor,fill opacity=0.20] ( 71.27, 70.44) circle (  2.13);

\path[fill=fillColor,fill opacity=0.20] ( 66.76, 85.86) circle (  2.13);

\path[fill=fillColor,fill opacity=0.20] ( 58.53, 97.62) circle (  2.13);

\path[fill=fillColor,fill opacity=0.20] ( 84.32,106.10) circle (  2.13);

\path[fill=fillColor,fill opacity=0.20] ( 84.32, 78.91) circle (  2.13);

\path[fill=fillColor,fill opacity=0.20] ( 77.29, 94.21) circle (  2.13);

\path[fill=fillColor,fill opacity=0.20] ( 74.28, 99.65) circle (  2.13);

\path[fill=fillColor,fill opacity=0.20] ( 77.29, 95.60) circle (  2.13);

\path[fill=fillColor,fill opacity=0.20] ( 80.30, 92.44) circle (  2.13);

\path[fill=fillColor,fill opacity=0.20] ( 84.32, 86.87) circle (  2.13);

\path[fill=fillColor,fill opacity=0.20] ( 79.30, 80.30) circle (  2.13);

\path[fill=fillColor,fill opacity=0.20] ( 86.32, 82.07) circle (  2.13);

\path[fill=fillColor,fill opacity=0.20] ( 90.33, 77.64) circle (  2.13);

\path[fill=fillColor,fill opacity=0.20] (100.37, 58.30) circle (  2.13);

\path[fill=fillColor,fill opacity=0.20] ( 86.32, 51.22) circle (  2.13);

\path[fill=fillColor,fill opacity=0.20] ( 89.33, 48.81) circle (  2.13);

\path[fill=fillColor,fill opacity=0.20] ( 91.34, 50.08) circle (  2.13);

\path[fill=fillColor,fill opacity=0.20] ( 81.31, 58.30) circle (  2.13);

\path[fill=fillColor,fill opacity=0.20] ( 78.30, 65.13) circle (  2.13);

\path[fill=fillColor,fill opacity=0.20] ( 75.29, 68.92) circle (  2.13);

\path[fill=fillColor,fill opacity=0.20] ( 69.27, 71.57) circle (  2.13);

\path[fill=fillColor,fill opacity=0.20] ( 74.28, 80.05) circle (  2.13);

\path[fill=fillColor,fill opacity=0.20] ( 84.32, 90.67) circle (  2.13);

\path[fill=fillColor,fill opacity=0.20] ( 78.30, 95.60) circle (  2.13);

\path[fill=fillColor,fill opacity=0.20] ( 72.28, 91.68) circle (  2.13);

\path[fill=fillColor,fill opacity=0.20] ( 79.30, 87.00) circle (  2.13);

\path[fill=fillColor,fill opacity=0.20] ( 83.31, 81.06) circle (  2.13);

\path[fill=fillColor,fill opacity=0.20] ( 85.32, 80.17) circle (  2.13);

\path[fill=fillColor,fill opacity=0.20] ( 58.93, 89.15) circle (  2.13);

\path[fill=fillColor,fill opacity=0.20] ( 91.34, 85.61) circle (  2.13);

\path[fill=fillColor,fill opacity=0.20] ( 96.35, 68.92) circle (  2.13);

\path[fill=fillColor,fill opacity=0.20] (104.38, 56.15) circle (  2.13);

\path[fill=fillColor,fill opacity=0.20] (106.39, 50.46) circle (  2.13);

\path[fill=fillColor,fill opacity=0.20] (121.43, 44.26) circle (  2.13);

\path[fill=fillColor,fill opacity=0.20] ( 93.34, 49.70) circle (  2.13);

\path[fill=fillColor,fill opacity=0.20] ( 89.33, 59.44) circle (  2.13);

\path[fill=fillColor,fill opacity=0.20] ( 85.32, 54.50) circle (  2.13);

\path[fill=fillColor,fill opacity=0.20] ( 84.32, 59.69) circle (  2.13);

\path[fill=fillColor,fill opacity=0.20] ( 77.29, 68.92) circle (  2.13);

\path[fill=fillColor,fill opacity=0.20] ( 71.27, 66.26) circle (  2.13);

\path[fill=fillColor,fill opacity=0.20] ( 65.15, 70.69) circle (  2.13);

\path[fill=fillColor,fill opacity=0.20] ( 82.31, 88.01) circle (  2.13);

\path[fill=fillColor,fill opacity=0.20] ( 90.33, 74.36) circle (  2.13);

\path[fill=fillColor,fill opacity=0.20] ( 82.31, 94.84) circle (  2.13);

\path[fill=fillColor,fill opacity=0.20] ( 83.31,101.54) circle (  2.13);

\path[fill=fillColor,fill opacity=0.20] ( 85.32, 90.29) circle (  2.13);

\path[fill=fillColor,fill opacity=0.20] ( 85.32, 87.89) circle (  2.13);

\path[fill=fillColor,fill opacity=0.20] ( 90.33, 96.86) circle (  2.13);

\path[fill=fillColor,fill opacity=0.20] ( 94.35, 91.55) circle (  2.13);

\path[fill=fillColor,fill opacity=0.20] ( 86.32, 80.05) circle (  2.13);

\path[fill=fillColor,fill opacity=0.20] ( 91.34, 76.00) circle (  2.13);

\path[fill=fillColor,fill opacity=0.20] ( 99.36, 70.94) circle (  2.13);

\path[fill=fillColor,fill opacity=0.20] ( 98.36, 67.53) circle (  2.13);

\path[fill=fillColor,fill opacity=0.20] (107.39, 58.80) circle (  2.13);

\path[fill=fillColor,fill opacity=0.20] (101.37, 46.66) circle (  2.13);

\path[fill=fillColor,fill opacity=0.20] ( 90.33, 38.82) circle (  2.13);

\path[fill=fillColor,fill opacity=0.20] ( 89.33, 38.07) circle (  2.13);

\path[fill=fillColor,fill opacity=0.20] ( 83.31, 38.95) circle (  2.13);

\path[fill=fillColor,fill opacity=0.20] ( 90.33, 49.32) circle (  2.13);

\path[fill=fillColor,fill opacity=0.20] ( 88.33, 66.01) circle (  2.13);

\path[fill=fillColor,fill opacity=0.20] ( 79.30, 64.87) circle (  2.13);

\path[fill=fillColor,fill opacity=0.20] ( 81.31, 64.24) circle (  2.13);

\path[fill=fillColor,fill opacity=0.20] ( 71.27, 72.33) circle (  2.13);

\path[fill=fillColor,fill opacity=0.20] ( 65.25, 75.87) circle (  2.13);

\path[fill=fillColor,fill opacity=0.20] ( 72.28, 85.23) circle (  2.13);

\path[fill=fillColor,fill opacity=0.20] ( 81.31, 99.77) circle (  2.13);

\path[fill=fillColor,fill opacity=0.20] ( 79.30, 92.94) circle (  2.13);

\path[fill=fillColor,fill opacity=0.20] ( 82.31, 87.51) circle (  2.13);

\path[fill=fillColor,fill opacity=0.20] ( 87.33, 86.12) circle (  2.13);

\path[fill=fillColor,fill opacity=0.20] ( 82.31, 81.69) circle (  2.13);

\path[fill=fillColor,fill opacity=0.20] ( 91.34, 81.69) circle (  2.13);

\path[fill=fillColor,fill opacity=0.20] ( 97.36, 83.33) circle (  2.13);

\path[fill=fillColor,fill opacity=0.20] ( 95.35, 79.41) circle (  2.13);

\path[fill=fillColor,fill opacity=0.20] ( 96.35, 77.52) circle (  2.13);

\path[fill=fillColor,fill opacity=0.20] (104.38, 78.66) circle (  2.13);

\path[fill=fillColor,fill opacity=0.20] ( 98.36, 72.59) circle (  2.13);

\path[fill=fillColor,fill opacity=0.20] ( 96.35, 60.32) circle (  2.13);

\path[fill=fillColor,fill opacity=0.20] ( 94.35, 50.20) circle (  2.13);

\path[fill=fillColor,fill opacity=0.20] ( 82.31, 43.88) circle (  2.13);

\path[fill=fillColor,fill opacity=0.20] ( 87.33, 39.46) circle (  2.13);

\path[fill=fillColor,fill opacity=0.20] ( 78.30, 41.23) circle (  2.13);

\path[fill=fillColor,fill opacity=0.20] ( 49.00, 45.91) circle (  2.13);

\path[fill=fillColor,fill opacity=0.20] (109.40, 44.26) circle (  2.13);

\path[fill=fillColor,fill opacity=0.20] (105.38, 40.21) circle (  2.13);

\path[fill=fillColor,fill opacity=0.20] (101.37, 38.07) circle (  2.13);

\path[fill=fillColor,fill opacity=0.20] ( 92.34, 43.63) circle (  2.13);

\path[fill=fillColor,fill opacity=0.20] ( 99.36, 48.18) circle (  2.13);

\path[fill=fillColor,fill opacity=0.20] ( 94.35, 45.27) circle (  2.13);

\path[fill=fillColor,fill opacity=0.20] ( 86.32, 45.78) circle (  2.13);

\path[fill=fillColor,fill opacity=0.20] ( 92.34, 57.54) circle (  2.13);

\path[fill=fillColor,fill opacity=0.20] ( 99.36, 79.54) circle (  2.13);

\path[fill=fillColor,fill opacity=0.20] (101.37, 76.63) circle (  2.13);

\path[fill=fillColor,fill opacity=0.20] ( 91.34, 71.32) circle (  2.13);

\path[fill=fillColor,fill opacity=0.20] ( 89.33, 76.76) circle (  2.13);

\path[fill=fillColor,fill opacity=0.20] ( 91.34, 72.21) circle (  2.13);

\path[fill=fillColor,fill opacity=0.20] ( 87.33, 68.29) circle (  2.13);

\path[fill=fillColor,fill opacity=0.20] ( 79.30, 81.69) circle (  2.13);

\path[fill=fillColor,fill opacity=0.20] ( 74.28, 98.26) circle (  2.13);

\path[fill=fillColor,fill opacity=0.20] ( 81.31,101.04) circle (  2.13);

\path[fill=fillColor,fill opacity=0.20] ( 81.31, 82.95) circle (  2.13);

\path[fill=fillColor,fill opacity=0.20] ( 82.31, 82.32) circle (  2.13);

\path[fill=fillColor,fill opacity=0.20] ( 81.31, 89.15) circle (  2.13);

\path[fill=fillColor,fill opacity=0.20] ( 83.31, 87.13) circle (  2.13);

\path[fill=fillColor,fill opacity=0.20] ( 93.34, 85.36) circle (  2.13);

\path[fill=fillColor,fill opacity=0.20] ( 92.34, 89.91) circle (  2.13);

\path[fill=fillColor,fill opacity=0.20] ( 95.35, 85.86) circle (  2.13);

\path[fill=fillColor,fill opacity=0.20] (101.37, 86.24) circle (  2.13);

\path[fill=fillColor,fill opacity=0.20] (109.40, 94.97) circle (  2.13);

\path[fill=fillColor,fill opacity=0.20] ( 96.35, 91.68) circle (  2.13);

\path[fill=fillColor,fill opacity=0.20] ( 85.32, 85.36) circle (  2.13);

\path[fill=fillColor,fill opacity=0.20] ( 90.33, 84.85) circle (  2.13);

\path[fill=fillColor,fill opacity=0.20] ( 87.33, 80.55) circle (  2.13);

\path[fill=fillColor,fill opacity=0.20] ( 78.30, 75.12) circle (  2.13);

\path[fill=fillColor,fill opacity=0.20] ( 89.33, 72.21) circle (  2.13);

\path[fill=fillColor,fill opacity=0.20] ( 96.35, 73.47) circle (  2.13);

\path[fill=fillColor,fill opacity=0.20] (108.39, 77.26) circle (  2.13);

\path[fill=fillColor,fill opacity=0.20] (101.37, 77.64) circle (  2.13);

\path[fill=fillColor,fill opacity=0.20] ( 83.31, 70.94) circle (  2.13);

\path[fill=fillColor,fill opacity=0.20] ( 90.33, 68.16) circle (  2.13);

\path[fill=fillColor,fill opacity=0.20] ( 96.35, 70.44) circle (  2.13);

\path[fill=fillColor,fill opacity=0.20] ( 96.35, 69.80) circle (  2.13);

\path[fill=fillColor,fill opacity=0.20] ( 91.34, 69.42) circle (  2.13);

\path[fill=fillColor,fill opacity=0.20] ( 97.36, 73.98) circle (  2.13);

\path[fill=fillColor,fill opacity=0.20] (100.37, 79.79) circle (  2.13);

\path[fill=fillColor,fill opacity=0.20] ( 90.33, 78.40) circle (  2.13);

\path[fill=fillColor,fill opacity=0.20] ( 89.33, 77.39) circle (  2.13);

\path[fill=fillColor,fill opacity=0.20] (102.37, 82.32) circle (  2.13);

\path[fill=fillColor,fill opacity=0.20] (101.37, 84.60) circle (  2.13);

\path[fill=fillColor,fill opacity=0.20] (101.37, 85.74) circle (  2.13);

\path[fill=fillColor,fill opacity=0.20] ( 91.34, 86.75) circle (  2.13);

\path[fill=fillColor,fill opacity=0.20] ( 87.33, 79.41) circle (  2.13);

\path[fill=fillColor,fill opacity=0.20] ( 82.31, 76.51) circle (  2.13);

\path[fill=fillColor,fill opacity=0.20] ( 85.32, 91.81) circle (  2.13);

\path[fill=fillColor,fill opacity=0.20] ( 82.31,111.66) circle (  2.13);

\path[fill=fillColor,fill opacity=0.20] ( 92.34, 76.00) circle (  2.13);

\path[fill=fillColor,fill opacity=0.20] ( 87.33, 88.27) circle (  2.13);

\path[fill=fillColor,fill opacity=0.20] ( 81.31, 88.90) circle (  2.13);

\path[fill=fillColor,fill opacity=0.20] ( 83.31, 89.15) circle (  2.13);

\path[fill=fillColor,fill opacity=0.20] ( 89.33, 87.38) circle (  2.13);

\path[fill=fillColor,fill opacity=0.20] ( 84.32, 86.12) circle (  2.13);

\path[fill=fillColor,fill opacity=0.20] ( 87.33, 93.70) circle (  2.13);

\path[fill=fillColor,fill opacity=0.20] ( 94.35, 97.12) circle (  2.13);

\path[fill=fillColor,fill opacity=0.20] ( 97.36, 87.89) circle (  2.13);

\path[fill=fillColor,fill opacity=0.20] ( 97.36, 84.98) circle (  2.13);

\path[fill=fillColor,fill opacity=0.20] (103.38, 85.74) circle (  2.13);

\path[fill=fillColor,fill opacity=0.20] ( 91.34, 85.86) circle (  2.13);

\path[fill=fillColor,fill opacity=0.20] ( 82.31, 86.87) circle (  2.13);

\path[fill=fillColor,fill opacity=0.20] ( 87.33, 78.66) circle (  2.13);

\path[fill=fillColor,fill opacity=0.20] ( 94.35, 71.83) circle (  2.13);

\path[fill=fillColor,fill opacity=0.20] ( 97.36, 78.15) circle (  2.13);

\path[fill=fillColor,fill opacity=0.20] ( 95.35, 84.60) circle (  2.13);

\path[fill=fillColor,fill opacity=0.20] ( 93.34, 79.92) circle (  2.13);

\path[fill=fillColor,fill opacity=0.20] ( 99.36, 82.45) circle (  2.13);

\path[fill=fillColor,fill opacity=0.20] ( 95.35, 89.40) circle (  2.13);

\path[fill=fillColor,fill opacity=0.20] ( 86.32, 84.47) circle (  2.13);

\path[fill=fillColor,fill opacity=0.20] ( 85.32, 79.92) circle (  2.13);

\path[fill=fillColor,fill opacity=0.20] ( 84.32, 81.94) circle (  2.13);

\path[fill=fillColor,fill opacity=0.20] ( 97.36, 79.29) circle (  2.13);

\path[fill=fillColor,fill opacity=0.20] ( 97.36, 75.12) circle (  2.13);

\path[fill=fillColor,fill opacity=0.20] ( 95.35, 75.24) circle (  2.13);

\path[fill=fillColor,fill opacity=0.20] ( 94.35, 77.64) circle (  2.13);

\path[fill=fillColor,fill opacity=0.20] ( 94.35, 78.78) circle (  2.13);

\path[fill=fillColor,fill opacity=0.20] ( 91.34, 86.24) circle (  2.13);

\path[fill=fillColor,fill opacity=0.20] ( 87.33, 92.31) circle (  2.13);

\path[fill=fillColor,fill opacity=0.20] ( 84.32, 86.37) circle (  2.13);

\path[fill=fillColor,fill opacity=0.20] ( 90.33, 88.65) circle (  2.13);

\path[fill=fillColor,fill opacity=0.20] ( 80.30, 84.22) circle (  2.13);

\path[fill=fillColor,fill opacity=0.20] ( 84.32, 87.89) circle (  2.13);

\path[fill=fillColor,fill opacity=0.20] ( 87.33, 95.35) circle (  2.13);

\path[fill=fillColor,fill opacity=0.20] ( 87.33, 97.88) circle (  2.13);

\path[fill=fillColor,fill opacity=0.20] ( 87.33, 92.19) circle (  2.13);

\path[fill=fillColor,fill opacity=0.20] ( 93.34, 85.99) circle (  2.13);

\path[fill=fillColor,fill opacity=0.20] ( 95.35, 83.84) circle (  2.13);

\path[fill=fillColor,fill opacity=0.20] ( 88.33, 85.74) circle (  2.13);

\path[fill=fillColor,fill opacity=0.20] ( 90.33, 90.67) circle (  2.13);

\path[fill=fillColor,fill opacity=0.20] ( 91.34, 85.10) circle (  2.13);

\path[fill=fillColor,fill opacity=0.20] ( 84.32, 76.51) circle (  2.13);

\path[fill=fillColor,fill opacity=0.20] ( 82.31, 78.40) circle (  2.13);

\path[fill=fillColor,fill opacity=0.20] ( 87.33, 80.93) circle (  2.13);

\path[fill=fillColor,fill opacity=0.20] ( 89.33, 81.31) circle (  2.13);

\path[fill=fillColor,fill opacity=0.20] ( 95.35, 86.75) circle (  2.13);

\path[fill=fillColor,fill opacity=0.20] ( 86.32, 91.68) circle (  2.13);

\path[fill=fillColor,fill opacity=0.20] ( 70.27, 85.74) circle (  2.13);

\path[fill=fillColor,fill opacity=0.20] ( 88.33, 80.05) circle (  2.13);

\path[fill=fillColor,fill opacity=0.20] ( 91.34, 83.46) circle (  2.13);

\path[fill=fillColor,fill opacity=0.20] ( 89.33, 87.51) circle (  2.13);

\path[fill=fillColor,fill opacity=0.20] ( 87.33, 85.48) circle (  2.13);

\path[fill=fillColor,fill opacity=0.20] ( 89.33, 85.36) circle (  2.13);

\path[fill=fillColor,fill opacity=0.20] ( 86.32, 86.87) circle (  2.13);

\path[fill=fillColor,fill opacity=0.20] ( 82.31, 90.54) circle (  2.13);

\path[fill=fillColor,fill opacity=0.20] ( 91.34, 94.21) circle (  2.13);

\path[fill=fillColor,fill opacity=0.20] ( 93.34, 85.48) circle (  2.13);

\path[fill=fillColor,fill opacity=0.20] (106.39, 87.63) circle (  2.13);

\path[fill=fillColor,fill opacity=0.20] ( 88.33, 86.50) circle (  2.13);

\path[fill=fillColor,fill opacity=0.20] ( 93.34, 85.99) circle (  2.13);

\path[fill=fillColor,fill opacity=0.20] ( 87.33, 91.68) circle (  2.13);

\path[fill=fillColor,fill opacity=0.20] ( 88.33, 93.32) circle (  2.13);

\path[fill=fillColor,fill opacity=0.20] ( 86.32, 96.74) circle (  2.13);

\path[fill=fillColor,fill opacity=0.20] ( 79.30,106.60) circle (  2.13);

\path[fill=fillColor,fill opacity=0.20] ( 84.32,106.35) circle (  2.13);

\path[fill=fillColor,fill opacity=0.20] ( 84.32, 97.24) circle (  2.13);

\path[fill=fillColor,fill opacity=0.20] ( 93.34, 91.68) circle (  2.13);

\path[fill=fillColor,fill opacity=0.20] ( 94.35, 91.81) circle (  2.13);

\path[fill=fillColor,fill opacity=0.20] ( 97.36, 94.08) circle (  2.13);

\path[fill=fillColor,fill opacity=0.20] ( 97.36, 92.06) circle (  2.13);

\path[fill=fillColor,fill opacity=0.20] ( 84.32, 93.58) circle (  2.13);

\path[fill=fillColor,fill opacity=0.20] ( 97.36, 93.83) circle (  2.13);

\path[fill=fillColor,fill opacity=0.20] ( 99.36, 91.43) circle (  2.13);

\path[fill=fillColor,fill opacity=0.20] ( 89.33, 95.60) circle (  2.13);

\path[fill=fillColor,fill opacity=0.20] ( 92.34,103.31) circle (  2.13);

\path[fill=fillColor,fill opacity=0.20] ( 99.36,102.68) circle (  2.13);

\path[fill=fillColor,fill opacity=0.20] (133.47, 39.96) circle (  2.13);

\path[fill=fillColor,fill opacity=0.20] (133.47, 40.97) circle (  2.13);

\path[fill=fillColor,fill opacity=0.20] (119.43, 52.23) circle (  2.13);

\path[fill=fillColor,fill opacity=0.20] (127.45, 57.29) circle (  2.13);

\path[fill=fillColor,fill opacity=0.20] (118.43, 59.94) circle (  2.13);

\path[fill=fillColor,fill opacity=0.20] ( 71.27, 58.42) circle (  2.13);

\path[fill=fillColor,fill opacity=0.20] (107.39, 58.80) circle (  2.13);

\path[fill=fillColor,fill opacity=0.20] (110.40, 59.44) circle (  2.13);

\path[fill=fillColor,fill opacity=0.20] ( 87.33, 52.61) circle (  2.13);

\path[fill=fillColor,fill opacity=0.20] (123.44, 48.05) circle (  2.13);

\path[fill=fillColor,fill opacity=0.20] (110.40, 46.41) circle (  2.13);

\path[fill=fillColor,fill opacity=0.20] (126.45, 41.23) circle (  2.13);

\path[fill=fillColor,fill opacity=0.20] (151.53, 45.78) circle (  2.13);

\path[fill=fillColor,fill opacity=0.20] (107.39, 57.41) circle (  2.13);

\path[fill=fillColor,fill opacity=0.20] ( 68.16, 65.38) circle (  2.13);

\path[fill=fillColor,fill opacity=0.20] ( 82.31, 67.53) circle (  2.13);

\path[fill=fillColor,fill opacity=0.20] ( 82.31, 71.70) circle (  2.13);

\path[fill=fillColor,fill opacity=0.20] ( 87.33, 79.92) circle (  2.13);

\path[fill=fillColor,fill opacity=0.20] ( 92.34, 82.83) circle (  2.13);

\path[fill=fillColor,fill opacity=0.20] ( 86.32, 80.30) circle (  2.13);

\path[fill=fillColor,fill opacity=0.20] ( 90.33, 77.52) circle (  2.13);

\path[fill=fillColor,fill opacity=0.20] ( 65.05, 56.65) circle (  2.13);

\path[fill=fillColor,fill opacity=0.20] ( 74.28, 69.68) circle (  2.13);

\path[fill=fillColor,fill opacity=0.20] ( 68.16, 81.18) circle (  2.13);

\path[fill=fillColor,fill opacity=0.20] ( 65.25, 85.61) circle (  2.13);

\path[fill=fillColor,fill opacity=0.20] ( 62.75, 88.90) circle (  2.13);

\path[fill=fillColor,fill opacity=0.20] ( 77.29, 92.06) circle (  2.13);

\path[fill=fillColor,fill opacity=0.20] ( 81.31, 95.09) circle (  2.13);

\path[fill=fillColor,fill opacity=0.20] ( 85.32, 93.20) circle (  2.13);

\path[fill=fillColor,fill opacity=0.20] ( 83.31, 76.38) circle (  2.13);

\path[fill=fillColor,fill opacity=0.20] ( 83.31, 69.42) circle (  2.13);

\path[fill=fillColor,fill opacity=0.20] ( 96.35, 46.54) circle (  2.13);

\path[fill=fillColor,fill opacity=0.20] ( 71.27, 68.92) circle (  2.13);

\path[fill=fillColor,fill opacity=0.20] ( 67.06, 82.07) circle (  2.13);

\path[fill=fillColor,fill opacity=0.20] ( 50.91, 93.83) circle (  2.13);

\path[fill=fillColor,fill opacity=0.20] ( 51.81, 97.37) circle (  2.13);

\path[fill=fillColor,fill opacity=0.20] ( 73.28, 98.00) circle (  2.13);

\path[fill=fillColor,fill opacity=0.20] ( 75.29, 91.93) circle (  2.13);

\path[fill=fillColor,fill opacity=0.20] ( 79.30, 83.97) circle (  2.13);

\path[fill=fillColor,fill opacity=0.20] ( 85.32, 87.63) circle (  2.13);

\path[fill=fillColor,fill opacity=0.20] ( 77.29, 90.16) circle (  2.13);

\path[fill=fillColor,fill opacity=0.20] ( 90.33, 71.07) circle (  2.13);

\path[fill=fillColor,fill opacity=0.20] ( 97.36,101.54) circle (  2.13);

\path[fill=fillColor,fill opacity=0.20] ( 83.31, 55.52) circle (  2.13);

\path[fill=fillColor,fill opacity=0.20] ( 52.11, 82.20) circle (  2.13);

\path[fill=fillColor,fill opacity=0.20] ( 64.45, 92.44) circle (  2.13);

\path[fill=fillColor,fill opacity=0.20] ( 80.30, 94.21) circle (  2.13);

\path[fill=fillColor,fill opacity=0.20] ( 80.30, 91.30) circle (  2.13);

\path[fill=fillColor,fill opacity=0.20] ( 80.30, 92.94) circle (  2.13);

\path[fill=fillColor,fill opacity=0.20] ( 80.30, 90.79) circle (  2.13);

\path[fill=fillColor,fill opacity=0.20] ( 84.32, 78.15) circle (  2.13);

\path[fill=fillColor,fill opacity=0.20] ( 87.33, 74.74) circle (  2.13);

\path[fill=fillColor,fill opacity=0.20] ( 91.34, 78.66) circle (  2.13);

\path[fill=fillColor,fill opacity=0.20] ( 91.34, 87.76) circle (  2.13);

\path[fill=fillColor,fill opacity=0.20] ( 82.31, 98.51) circle (  2.13);

\path[fill=fillColor,fill opacity=0.20] ( 71.27,100.53) circle (  2.13);

\path[fill=fillColor,fill opacity=0.20] ( 66.36,115.83) circle (  2.13);

\path[fill=fillColor,fill opacity=0.20] ( 70.27, 57.41) circle (  2.13);

\path[fill=fillColor,fill opacity=0.20] ( 75.29, 89.66) circle (  2.13);

\path[fill=fillColor,fill opacity=0.20] ( 63.65, 95.35) circle (  2.13);

\path[fill=fillColor,fill opacity=0.20] ( 77.29, 93.32) circle (  2.13);

\path[fill=fillColor,fill opacity=0.20] ( 80.30, 89.15) circle (  2.13);

\path[fill=fillColor,fill opacity=0.20] ( 86.32, 90.92) circle (  2.13);

\path[fill=fillColor,fill opacity=0.20] ( 82.31, 91.55) circle (  2.13);

\path[fill=fillColor,fill opacity=0.20] ( 85.32, 68.54) circle (  2.13);

\path[fill=fillColor,fill opacity=0.20] ( 89.33, 67.78) circle (  2.13);

\path[fill=fillColor,fill opacity=0.20] ( 75.29,101.42) circle (  2.13);

\path[fill=fillColor,fill opacity=0.20] ( 80.30,110.77) circle (  2.13);

\path[fill=fillColor,fill opacity=0.20] ( 79.30,102.05) circle (  2.13);

\path[fill=fillColor,fill opacity=0.20] ( 59.84,100.91) circle (  2.13);

\path[fill=fillColor,fill opacity=0.20] ( 58.83, 98.76) circle (  2.13);

\path[fill=fillColor,fill opacity=0.20] ( 53.12,102.93) circle (  2.13);

\path[fill=fillColor,fill opacity=0.20] ( 56.63,114.82) circle (  2.13);

\path[fill=fillColor,fill opacity=0.20] ( 79.30, 54.25) circle (  2.13);

\path[fill=fillColor,fill opacity=0.20] ( 76.29, 93.20) circle (  2.13);

\path[fill=fillColor,fill opacity=0.20] ( 79.30, 97.88) circle (  2.13);

\path[fill=fillColor,fill opacity=0.20] ( 76.29, 91.43) circle (  2.13);

\path[fill=fillColor,fill opacity=0.20] ( 83.31, 96.11) circle (  2.13);

\path[fill=fillColor,fill opacity=0.20] ( 76.29, 97.37) circle (  2.13);

\path[fill=fillColor,fill opacity=0.20] ( 83.31, 92.69) circle (  2.13);

\path[fill=fillColor,fill opacity=0.20] ( 91.34, 63.86) circle (  2.13);

\path[fill=fillColor,fill opacity=0.20] (110.40, 53.87) circle (  2.13);

\path[fill=fillColor,fill opacity=0.20] ( 87.33,103.19) circle (  2.13);

\path[fill=fillColor,fill opacity=0.20] ( 73.28, 97.88) circle (  2.13);

\path[fill=fillColor,fill opacity=0.20] ( 70.27, 99.01) circle (  2.13);

\path[fill=fillColor,fill opacity=0.20] ( 67.56,105.84) circle (  2.13);

\path[fill=fillColor,fill opacity=0.20] ( 59.94,101.29) circle (  2.13);

\path[fill=fillColor,fill opacity=0.20] ( 45.09, 96.11) circle (  2.13);

\path[fill=fillColor,fill opacity=0.20] ( 61.64,105.84) circle (  2.13);

\path[fill=fillColor,fill opacity=0.20] ( 69.27, 48.81) circle (  2.13);

\path[fill=fillColor,fill opacity=0.20] ( 59.54, 95.35) circle (  2.13);

\path[fill=fillColor,fill opacity=0.20] ( 85.32, 97.50) circle (  2.13);

\path[fill=fillColor,fill opacity=0.20] ( 86.32, 92.19) circle (  2.13);

\path[fill=fillColor,fill opacity=0.20] ( 87.33,103.69) circle (  2.13);

\path[fill=fillColor,fill opacity=0.20] ( 85.32,102.93) circle (  2.13);

\path[fill=fillColor,fill opacity=0.20] ( 91.34, 88.01) circle (  2.13);

\path[fill=fillColor,fill opacity=0.20] ( 67.36, 82.70) circle (  2.13);

\path[fill=fillColor,fill opacity=0.20] ( 99.36, 80.43) circle (  2.13);

\path[fill=fillColor,fill opacity=0.20] ( 79.30, 98.26) circle (  2.13);

\path[fill=fillColor,fill opacity=0.20] ( 70.27,100.66) circle (  2.13);

\path[fill=fillColor,fill opacity=0.20] ( 77.29,107.74) circle (  2.13);

\path[fill=fillColor,fill opacity=0.20] ( 76.29,111.91) circle (  2.13);

\path[fill=fillColor,fill opacity=0.20] ( 68.16,105.34) circle (  2.13);

\path[fill=fillColor,fill opacity=0.20] ( 65.25,102.43) circle (  2.13);

\path[fill=fillColor,fill opacity=0.20] ( 67.96, 88.52) circle (  2.13);

\path[fill=fillColor,fill opacity=0.20] ( 83.31, 93.58) circle (  2.13);

\path[fill=fillColor,fill opacity=0.20] ( 91.34, 92.57) circle (  2.13);

\path[fill=fillColor,fill opacity=0.20] ( 93.34,105.08) circle (  2.13);

\path[fill=fillColor,fill opacity=0.20] ( 96.35,104.45) circle (  2.13);

\path[fill=fillColor,fill opacity=0.20] ( 96.35, 89.53) circle (  2.13);

\path[fill=fillColor,fill opacity=0.20] ( 99.36, 83.71) circle (  2.13);

\path[fill=fillColor,fill opacity=0.20] (100.37, 90.29) circle (  2.13);

\path[fill=fillColor,fill opacity=0.20] (101.37, 89.78) circle (  2.13);

\path[fill=fillColor,fill opacity=0.20] ( 94.35, 73.22) circle (  2.13);

\path[fill=fillColor,fill opacity=0.20] ( 72.28,110.52) circle (  2.13);

\path[fill=fillColor,fill opacity=0.20] ( 75.29,102.55) circle (  2.13);

\path[fill=fillColor,fill opacity=0.20] ( 80.30,112.04) circle (  2.13);

\path[fill=fillColor,fill opacity=0.20] ( 74.28,109.38) circle (  2.13);

\path[fill=fillColor,fill opacity=0.20] ( 70.27,102.43) circle (  2.13);

\path[fill=fillColor,fill opacity=0.20] ( 59.34,111.41) circle (  2.13);

\path[fill=fillColor,fill opacity=0.20] ( 46.29,115.20) circle (  2.13);

\path[fill=fillColor,fill opacity=0.20] ( 51.91,114.19) circle (  2.13);

\path[fill=fillColor,fill opacity=0.20] ( 72.28, 64.24) circle (  2.13);

\path[fill=fillColor,fill opacity=0.20] ( 82.31, 83.59) circle (  2.13);

\path[fill=fillColor,fill opacity=0.20] ( 86.32, 87.63) circle (  2.13);

\path[fill=fillColor,fill opacity=0.20] ( 88.33, 93.58) circle (  2.13);

\path[fill=fillColor,fill opacity=0.20] ( 85.32, 94.71) circle (  2.13);

\path[fill=fillColor,fill opacity=0.20] ( 93.34, 91.43) circle (  2.13);

\path[fill=fillColor,fill opacity=0.20] ( 92.34, 90.92) circle (  2.13);

\path[fill=fillColor,fill opacity=0.20] ( 94.35, 94.97) circle (  2.13);

\path[fill=fillColor,fill opacity=0.20] ( 84.32, 88.65) circle (  2.13);

\path[fill=fillColor,fill opacity=0.20] (104.38, 75.62) circle (  2.13);

\path[fill=fillColor,fill opacity=0.20] (107.39, 55.64) circle (  2.13);

\path[fill=fillColor,fill opacity=0.20] ( 95.35, 85.74) circle (  2.13);

\path[fill=fillColor,fill opacity=0.20] ( 76.29,114.95) circle (  2.13);

\path[fill=fillColor,fill opacity=0.20] ( 78.30,100.28) circle (  2.13);

\path[fill=fillColor,fill opacity=0.20] ( 73.28,107.11) circle (  2.13);

\path[fill=fillColor,fill opacity=0.20] ( 73.28,113.30) circle (  2.13);

\path[fill=fillColor,fill opacity=0.20] ( 77.29,109.51) circle (  2.13);

\path[fill=fillColor,fill opacity=0.20] ( 71.27,112.04) circle (  2.13);

\path[fill=fillColor,fill opacity=0.20] ( 66.66,115.58) circle (  2.13);

\path[fill=fillColor,fill opacity=0.20] ( 56.63,113.81) circle (  2.13);

\path[fill=fillColor,fill opacity=0.20] ( 60.24,109.51) circle (  2.13);

\path[fill=fillColor,fill opacity=0.20] ( 72.28, 40.59) circle (  2.13);

\path[fill=fillColor,fill opacity=0.20] ( 73.28, 68.29) circle (  2.13);

\path[fill=fillColor,fill opacity=0.20] ( 82.31, 81.56) circle (  2.13);

\path[fill=fillColor,fill opacity=0.20] ( 81.31, 81.31) circle (  2.13);

\path[fill=fillColor,fill opacity=0.20] ( 91.34, 80.68) circle (  2.13);

\path[fill=fillColor,fill opacity=0.20] ( 95.35, 85.36) circle (  2.13);

\path[fill=fillColor,fill opacity=0.20] ( 91.34, 95.85) circle (  2.13);

\path[fill=fillColor,fill opacity=0.20] ( 95.35, 93.07) circle (  2.13);

\path[fill=fillColor,fill opacity=0.20] (100.37, 78.15) circle (  2.13);

\path[fill=fillColor,fill opacity=0.20] ( 99.36, 72.71) circle (  2.13);

\path[fill=fillColor,fill opacity=0.20] (106.39, 69.30) circle (  2.13);

\path[fill=fillColor,fill opacity=0.20] (107.39, 55.01) circle (  2.13);

\path[fill=fillColor,fill opacity=0.20] ( 84.32,100.28) circle (  2.13);

\path[fill=fillColor,fill opacity=0.20] ( 73.28,107.74) circle (  2.13);

\path[fill=fillColor,fill opacity=0.20] ( 73.28,110.14) circle (  2.13);

\path[fill=fillColor,fill opacity=0.20] ( 78.30,106.22) circle (  2.13);

\path[fill=fillColor,fill opacity=0.20] ( 77.29,110.77) circle (  2.13);

\path[fill=fillColor,fill opacity=0.20] ( 71.27,113.81) circle (  2.13);

\path[fill=fillColor,fill opacity=0.20] ( 70.27,106.22) circle (  2.13);

\path[fill=fillColor,fill opacity=0.20] ( 71.27,111.66) circle (  2.13);

\path[fill=fillColor,fill opacity=0.20] ( 64.05,106.60) circle (  2.13);

\path[fill=fillColor,fill opacity=0.20] ( 68.26,105.34) circle (  2.13);

\path[fill=fillColor,fill opacity=0.20] ( 71.27, 45.91) circle (  2.13);

\path[fill=fillColor,fill opacity=0.20] ( 80.30, 72.46) circle (  2.13);

\path[fill=fillColor,fill opacity=0.20] ( 88.33, 80.43) circle (  2.13);

\path[fill=fillColor,fill opacity=0.20] ( 94.35, 77.90) circle (  2.13);

\path[fill=fillColor,fill opacity=0.20] ( 99.36, 85.48) circle (  2.13);

\path[fill=fillColor,fill opacity=0.20] ( 97.36, 95.22) circle (  2.13);

\path[fill=fillColor,fill opacity=0.20] ( 99.36, 85.86) circle (  2.13);

\path[fill=fillColor,fill opacity=0.20] ( 97.36, 77.01) circle (  2.13);

\path[fill=fillColor,fill opacity=0.20] ( 91.34, 75.37) circle (  2.13);

\path[fill=fillColor,fill opacity=0.20] ( 93.34, 73.34) circle (  2.13);

\path[fill=fillColor,fill opacity=0.20] (101.37, 68.92) circle (  2.13);

\path[fill=fillColor,fill opacity=0.20] (103.38, 59.81) circle (  2.13);

\path[fill=fillColor,fill opacity=0.20] ( 73.28, 45.53) circle (  2.13);

\path[fill=fillColor,fill opacity=0.20] ( 88.33, 84.98) circle (  2.13);

\path[fill=fillColor,fill opacity=0.20] ( 70.27, 97.62) circle (  2.13);

\path[fill=fillColor,fill opacity=0.20] ( 76.29,105.72) circle (  2.13);

\path[fill=fillColor,fill opacity=0.20] ( 76.29,107.61) circle (  2.13);

\path[fill=fillColor,fill opacity=0.20] ( 76.29,111.91) circle (  2.13);

\path[fill=fillColor,fill opacity=0.20] ( 65.86,108.75) circle (  2.13);

\path[fill=fillColor,fill opacity=0.20] ( 60.64, 98.51) circle (  2.13);

\path[fill=fillColor,fill opacity=0.20] ( 72.28,101.42) circle (  2.13);

\path[fill=fillColor,fill opacity=0.20] ( 68.26,101.67) circle (  2.13);

\path[fill=fillColor,fill opacity=0.20] ( 64.25,100.03) circle (  2.13);

\path[fill=fillColor,fill opacity=0.20] ( 77.29,111.15) circle (  2.13);

\path[fill=fillColor,fill opacity=0.20] ( 58.43, 45.78) circle (  2.13);

\path[fill=fillColor,fill opacity=0.20] ( 90.33, 72.84) circle (  2.13);

\path[fill=fillColor,fill opacity=0.20] ( 92.34, 78.53) circle (  2.13);

\path[fill=fillColor,fill opacity=0.20] ( 90.33, 88.52) circle (  2.13);

\path[fill=fillColor,fill opacity=0.20] ( 90.33, 88.39) circle (  2.13);

\path[fill=fillColor,fill opacity=0.20] ( 97.36, 82.45) circle (  2.13);

\path[fill=fillColor,fill opacity=0.20] ( 92.34, 82.58) circle (  2.13);

\path[fill=fillColor,fill opacity=0.20] ( 76.29, 89.78) circle (  2.13);

\path[fill=fillColor,fill opacity=0.20] ( 91.34, 83.46) circle (  2.13);

\path[fill=fillColor,fill opacity=0.20] ( 98.36, 77.26) circle (  2.13);

\path[fill=fillColor,fill opacity=0.20] (103.38, 75.49) circle (  2.13);

\path[fill=fillColor,fill opacity=0.20] (102.37, 62.47) circle (  2.13);

\path[fill=fillColor,fill opacity=0.20] ( 87.33, 49.19) circle (  2.13);

\path[fill=fillColor,fill opacity=0.20] ( 82.31, 66.26) circle (  2.13);

\path[fill=fillColor,fill opacity=0.20] ( 85.32, 83.59) circle (  2.13);

\path[fill=fillColor,fill opacity=0.20] ( 78.30, 82.58) circle (  2.13);

\path[fill=fillColor,fill opacity=0.20] ( 76.29, 91.55) circle (  2.13);

\path[fill=fillColor,fill opacity=0.20] ( 72.28,107.61) circle (  2.13);

\path[fill=fillColor,fill opacity=0.20] ( 69.27,109.64) circle (  2.13);

\path[fill=fillColor,fill opacity=0.20] ( 66.66, 98.63) circle (  2.13);

\path[fill=fillColor,fill opacity=0.20] ( 69.27, 93.70) circle (  2.13);

\path[fill=fillColor,fill opacity=0.20] ( 71.27,103.06) circle (  2.13);

\path[fill=fillColor,fill opacity=0.20] ( 70.27,111.03) circle (  2.13);

\path[fill=fillColor,fill opacity=0.20] ( 68.16,106.85) circle (  2.13);

\path[fill=fillColor,fill opacity=0.20] ( 67.36,103.57) circle (  2.13);

\path[fill=fillColor,fill opacity=0.20] ( 71.27,100.53) circle (  2.13);

\path[fill=fillColor,fill opacity=0.20] ( 67.86, 95.22) circle (  2.13);

\path[fill=fillColor,fill opacity=0.20] ( 62.24, 40.85) circle (  2.13);

\path[fill=fillColor,fill opacity=0.20] ( 86.32, 56.40) circle (  2.13);

\path[fill=fillColor,fill opacity=0.20] ( 89.33, 65.88) circle (  2.13);

\path[fill=fillColor,fill opacity=0.20] ( 89.33, 72.08) circle (  2.13);

\path[fill=fillColor,fill opacity=0.20] ( 92.34, 79.67) circle (  2.13);

\path[fill=fillColor,fill opacity=0.20] ( 86.32, 86.12) circle (  2.13);

\path[fill=fillColor,fill opacity=0.20] ( 82.31, 90.42) circle (  2.13);

\path[fill=fillColor,fill opacity=0.20] ( 81.31, 87.63) circle (  2.13);

\path[fill=fillColor,fill opacity=0.20] ( 83.31, 83.97) circle (  2.13);

\path[fill=fillColor,fill opacity=0.20] ( 97.36, 85.48) circle (  2.13);

\path[fill=fillColor,fill opacity=0.20] ( 97.36, 80.68) circle (  2.13);

\path[fill=fillColor,fill opacity=0.20] ( 99.36, 71.57) circle (  2.13);

\path[fill=fillColor,fill opacity=0.20] (109.40, 65.13) circle (  2.13);

\path[fill=fillColor,fill opacity=0.20] (117.42, 53.49) circle (  2.13);

\path[fill=fillColor,fill opacity=0.20] ( 74.28, 45.02) circle (  2.13);

\path[fill=fillColor,fill opacity=0.20] ( 81.31, 53.11) circle (  2.13);

\path[fill=fillColor,fill opacity=0.20] ( 63.15, 59.56) circle (  2.13);

\path[fill=fillColor,fill opacity=0.20] ( 86.32, 58.68) circle (  2.13);

\path[fill=fillColor,fill opacity=0.20] ( 77.29, 70.18) circle (  2.13);

\path[fill=fillColor,fill opacity=0.20] ( 61.34, 86.62) circle (  2.13);

\path[fill=fillColor,fill opacity=0.20] ( 73.28, 91.93) circle (  2.13);

\path[fill=fillColor,fill opacity=0.20] ( 77.29, 85.48) circle (  2.13);

\path[fill=fillColor,fill opacity=0.20] ( 75.29, 90.04) circle (  2.13);

\path[fill=fillColor,fill opacity=0.20] ( 77.29,107.36) circle (  2.13);

\path[fill=fillColor,fill opacity=0.20] ( 73.28,109.26) circle (  2.13);

\path[fill=fillColor,fill opacity=0.20] ( 73.28, 98.51) circle (  2.13);

\path[fill=fillColor,fill opacity=0.20] ( 69.27, 99.77) circle (  2.13);

\path[fill=fillColor,fill opacity=0.20] ( 70.27,109.76) circle (  2.13);

\path[fill=fillColor,fill opacity=0.20] ( 71.27,102.05) circle (  2.13);

\path[fill=fillColor,fill opacity=0.20] ( 66.76, 97.50) circle (  2.13);

\path[fill=fillColor,fill opacity=0.20] ( 70.27,112.80) circle (  2.13);

\path[fill=fillColor,fill opacity=0.20] ( 77.29,108.24) circle (  2.13);

\path[fill=fillColor,fill opacity=0.20] ( 64.15, 85.36) circle (  2.13);

\path[fill=fillColor,fill opacity=0.20] ( 82.31, 40.09) circle (  2.13);

\path[fill=fillColor,fill opacity=0.20] ( 82.31, 47.68) circle (  2.13);

\path[fill=fillColor,fill opacity=0.20] ( 83.31, 59.18) circle (  2.13);

\path[fill=fillColor,fill opacity=0.20] ( 89.33, 61.84) circle (  2.13);

\path[fill=fillColor,fill opacity=0.20] ( 86.32, 64.62) circle (  2.13);

\path[fill=fillColor,fill opacity=0.20] ( 99.36, 73.09) circle (  2.13);

\path[fill=fillColor,fill opacity=0.20] ( 92.34, 83.84) circle (  2.13);

\path[fill=fillColor,fill opacity=0.20] ( 95.35, 88.27) circle (  2.13);

\path[fill=fillColor,fill opacity=0.20] (100.37, 86.62) circle (  2.13);

\path[fill=fillColor,fill opacity=0.20] (105.38, 88.01) circle (  2.13);

\path[fill=fillColor,fill opacity=0.20] ( 97.36, 81.18) circle (  2.13);

\path[fill=fillColor,fill opacity=0.20] ( 96.35, 70.94) circle (  2.13);

\path[fill=fillColor,fill opacity=0.20] ( 74.28, 66.90) circle (  2.13);

\path[fill=fillColor,fill opacity=0.20] ( 94.35, 60.19) circle (  2.13);

\path[fill=fillColor,fill opacity=0.20] ( 97.36, 56.40) circle (  2.13);

\path[fill=fillColor,fill opacity=0.20] ( 58.13, 61.21) circle (  2.13);

\path[fill=fillColor,fill opacity=0.20] ( 87.33, 65.76) circle (  2.13);

\path[fill=fillColor,fill opacity=0.20] ( 83.31, 80.05) circle (  2.13);

\path[fill=fillColor,fill opacity=0.20] ( 86.32, 81.18) circle (  2.13);

\path[fill=fillColor,fill opacity=0.20] ( 81.31, 73.47) circle (  2.13);

\path[fill=fillColor,fill opacity=0.20] ( 80.30, 80.81) circle (  2.13);

\path[fill=fillColor,fill opacity=0.20] ( 80.30, 87.89) circle (  2.13);

\path[fill=fillColor,fill opacity=0.20] ( 75.29, 97.24) circle (  2.13);

\path[fill=fillColor,fill opacity=0.20] ( 79.30,104.83) circle (  2.13);

\path[fill=fillColor,fill opacity=0.20] ( 77.29, 99.90) circle (  2.13);

\path[fill=fillColor,fill opacity=0.20] ( 88.33,103.31) circle (  2.13);

\path[fill=fillColor,fill opacity=0.20] ( 79.30,105.08) circle (  2.13);

\path[fill=fillColor,fill opacity=0.20] ( 78.30, 98.13) circle (  2.13);

\path[fill=fillColor,fill opacity=0.20] ( 73.28, 98.51) circle (  2.13);

\path[fill=fillColor,fill opacity=0.20] ( 72.28,109.38) circle (  2.13);

\path[fill=fillColor,fill opacity=0.20] ( 70.27,109.00) circle (  2.13);

\path[fill=fillColor,fill opacity=0.20] ( 69.27,102.93) circle (  2.13);

\path[fill=fillColor,fill opacity=0.20] ( 69.27,115.33) circle (  2.13);

\path[fill=fillColor,fill opacity=0.20] ( 67.06, 86.50) circle (  2.13);

\path[fill=fillColor,fill opacity=0.20] ( 87.33, 51.85) circle (  2.13);

\path[fill=fillColor,fill opacity=0.20] ( 96.35, 59.94) circle (  2.13);

\path[fill=fillColor,fill opacity=0.20] ( 97.36, 72.71) circle (  2.13);

\path[fill=fillColor,fill opacity=0.20] ( 97.36, 83.08) circle (  2.13);

\path[fill=fillColor,fill opacity=0.20] ( 96.35, 79.41) circle (  2.13);

\path[fill=fillColor,fill opacity=0.20] ( 92.34, 74.10) circle (  2.13);

\path[fill=fillColor,fill opacity=0.20] ( 87.33, 78.15) circle (  2.13);

\path[fill=fillColor,fill opacity=0.20] ( 92.34, 84.73) circle (  2.13);

\path[fill=fillColor,fill opacity=0.20] ( 99.36, 79.03) circle (  2.13);

\path[fill=fillColor,fill opacity=0.20] ( 78.30, 73.22) circle (  2.13);

\path[fill=fillColor,fill opacity=0.20] ( 95.35, 65.63) circle (  2.13);

\path[fill=fillColor,fill opacity=0.20] ( 97.36, 55.26) circle (  2.13);

\path[fill=fillColor,fill opacity=0.20] (117.42, 44.13) circle (  2.13);

\path[fill=fillColor,fill opacity=0.20] ( 97.36, 68.41) circle (  2.13);

\path[fill=fillColor,fill opacity=0.20] ( 89.33, 73.98) circle (  2.13);

\path[fill=fillColor,fill opacity=0.20] ( 91.34, 90.29) circle (  2.13);

\path[fill=fillColor,fill opacity=0.20] ( 81.31, 96.99) circle (  2.13);

\path[fill=fillColor,fill opacity=0.20] ( 80.30, 81.18) circle (  2.13);

\path[fill=fillColor,fill opacity=0.20] ( 77.29, 77.01) circle (  2.13);

\path[fill=fillColor,fill opacity=0.20] ( 85.32, 85.36) circle (  2.13);

\path[fill=fillColor,fill opacity=0.20] ( 83.31, 87.63) circle (  2.13);

\path[fill=fillColor,fill opacity=0.20] ( 66.26, 84.09) circle (  2.13);

\path[fill=fillColor,fill opacity=0.20] ( 79.30, 97.88) circle (  2.13);

\path[fill=fillColor,fill opacity=0.20] ( 83.31,109.76) circle (  2.13);

\path[fill=fillColor,fill opacity=0.20] ( 73.28,101.67) circle (  2.13);

\path[fill=fillColor,fill opacity=0.20] ( 82.31, 93.58) circle (  2.13);

\path[fill=fillColor,fill opacity=0.20] ( 84.32, 92.44) circle (  2.13);

\path[fill=fillColor,fill opacity=0.20] ( 78.30, 97.62) circle (  2.13);

\path[fill=fillColor,fill opacity=0.20] ( 75.29, 99.01) circle (  2.13);

\path[fill=fillColor,fill opacity=0.20] ( 75.29,102.81) circle (  2.13);

\path[fill=fillColor,fill opacity=0.20] ( 64.25,112.67) circle (  2.13);

\path[fill=fillColor,fill opacity=0.20] ( 72.28,110.27) circle (  2.13);

\path[fill=fillColor,fill opacity=0.20] ( 67.96,107.11) circle (  2.13);

\path[fill=fillColor,fill opacity=0.20] ( 65.35,113.43) circle (  2.13);

\path[fill=fillColor,fill opacity=0.20] ( 66.16, 85.23) circle (  2.13);

\path[fill=fillColor,fill opacity=0.20] ( 82.31, 38.82) circle (  2.13);

\path[fill=fillColor,fill opacity=0.20] ( 89.33, 44.13) circle (  2.13);

\path[fill=fillColor,fill opacity=0.20] ( 90.33, 54.12) circle (  2.13);

\path[fill=fillColor,fill opacity=0.20] ( 94.35, 58.04) circle (  2.13);

\path[fill=fillColor,fill opacity=0.20] ( 95.35, 60.95) circle (  2.13);

\path[fill=fillColor,fill opacity=0.20] ( 98.36, 69.30) circle (  2.13);

\path[fill=fillColor,fill opacity=0.20] ( 98.36, 75.87) circle (  2.13);

\path[fill=fillColor,fill opacity=0.20] ( 92.34, 72.71) circle (  2.13);

\path[fill=fillColor,fill opacity=0.20] ( 97.36, 77.14) circle (  2.13);

\path[fill=fillColor,fill opacity=0.20] ( 96.35, 74.36) circle (  2.13);

\path[fill=fillColor,fill opacity=0.20] ( 95.35, 67.65) circle (  2.13);

\path[fill=fillColor,fill opacity=0.20] ( 90.33, 67.15) circle (  2.13);

\path[fill=fillColor,fill opacity=0.20] ( 97.36, 67.78) circle (  2.13);

\path[fill=fillColor,fill opacity=0.20] ( 96.35, 58.55) circle (  2.13);

\path[fill=fillColor,fill opacity=0.20] (100.37, 47.93) circle (  2.13);

\path[fill=fillColor,fill opacity=0.20] ( 87.33, 51.60) circle (  2.13);

\path[fill=fillColor,fill opacity=0.20] ( 92.34, 65.25) circle (  2.13);

\path[fill=fillColor,fill opacity=0.20] ( 94.35, 70.31) circle (  2.13);

\path[fill=fillColor,fill opacity=0.20] ( 86.32, 72.71) circle (  2.13);

\path[fill=fillColor,fill opacity=0.20] ( 98.36, 90.04) circle (  2.13);

\path[fill=fillColor,fill opacity=0.20] ( 92.34, 95.35) circle (  2.13);

\path[fill=fillColor,fill opacity=0.20] ( 88.33, 81.82) circle (  2.13);

\path[fill=fillColor,fill opacity=0.20] ( 71.27, 79.16) circle (  2.13);

\path[fill=fillColor,fill opacity=0.20] ( 87.33, 84.60) circle (  2.13);

\path[fill=fillColor,fill opacity=0.20] ( 85.32, 93.07) circle (  2.13);

\path[fill=fillColor,fill opacity=0.20] ( 84.32, 99.14) circle (  2.13);

\path[fill=fillColor,fill opacity=0.20] ( 83.31, 96.23) circle (  2.13);

\path[fill=fillColor,fill opacity=0.20] ( 83.31, 94.71) circle (  2.13);

\path[fill=fillColor,fill opacity=0.20] ( 78.30, 93.96) circle (  2.13);

\path[fill=fillColor,fill opacity=0.20] ( 72.28, 87.38) circle (  2.13);

\path[fill=fillColor,fill opacity=0.20] ( 79.30, 89.91) circle (  2.13);

\path[fill=fillColor,fill opacity=0.20] ( 80.30,101.54) circle (  2.13);

\path[fill=fillColor,fill opacity=0.20] ( 74.28,102.68) circle (  2.13);

\path[fill=fillColor,fill opacity=0.20] ( 74.28,102.55) circle (  2.13);

\path[fill=fillColor,fill opacity=0.20] ( 75.29,105.34) circle (  2.13);

\path[fill=fillColor,fill opacity=0.20] ( 79.30,102.30) circle (  2.13);

\path[fill=fillColor,fill opacity=0.20] ( 69.27,102.68) circle (  2.13);

\path[fill=fillColor,fill opacity=0.20] ( 63.45,104.20) circle (  2.13);

\path[fill=fillColor,fill opacity=0.20] ( 68.16, 78.02) circle (  2.13);

\path[fill=fillColor,fill opacity=0.20] ( 94.35, 43.63) circle (  2.13);

\path[fill=fillColor,fill opacity=0.20] ( 96.35, 43.88) circle (  2.13);

\path[fill=fillColor,fill opacity=0.20] ( 96.35, 51.85) circle (  2.13);

\path[fill=fillColor,fill opacity=0.20] ( 94.35, 57.79) circle (  2.13);

\path[fill=fillColor,fill opacity=0.20] ( 98.36, 61.96) circle (  2.13);

\path[fill=fillColor,fill opacity=0.20] (100.37, 65.00) circle (  2.13);

\path[fill=fillColor,fill opacity=0.20] ( 97.36, 67.65) circle (  2.13);

\path[fill=fillColor,fill opacity=0.20] ( 97.36, 70.69) circle (  2.13);

\path[fill=fillColor,fill opacity=0.20] (103.38, 72.08) circle (  2.13);

\path[fill=fillColor,fill opacity=0.20] (100.37, 75.75) circle (  2.13);

\path[fill=fillColor,fill opacity=0.20] ( 97.36, 75.62) circle (  2.13);

\path[fill=fillColor,fill opacity=0.20] ( 96.35, 71.57) circle (  2.13);

\path[fill=fillColor,fill opacity=0.20] ( 96.35, 57.54) circle (  2.13);

\path[fill=fillColor,fill opacity=0.20] ( 95.35, 41.61) circle (  2.13);

\path[fill=fillColor,fill opacity=0.20] ( 91.34, 49.95) circle (  2.13);

\path[fill=fillColor,fill opacity=0.20] ( 86.32, 55.26) circle (  2.13);

\path[fill=fillColor,fill opacity=0.20] ( 83.31, 52.23) circle (  2.13);

\path[fill=fillColor,fill opacity=0.20] ( 90.33, 48.43) circle (  2.13);

\path[fill=fillColor,fill opacity=0.20] ( 89.33, 54.50) circle (  2.13);

\path[fill=fillColor,fill opacity=0.20] ( 97.36, 53.62) circle (  2.13);

\path[fill=fillColor,fill opacity=0.20] ( 94.35, 48.56) circle (  2.13);

\path[fill=fillColor,fill opacity=0.20] ( 92.34, 54.76) circle (  2.13);

\path[fill=fillColor,fill opacity=0.20] ( 92.34, 61.96) circle (  2.13);

\path[fill=fillColor,fill opacity=0.20] ( 89.33, 64.87) circle (  2.13);

\path[fill=fillColor,fill opacity=0.20] ( 88.33, 65.13) circle (  2.13);

\path[fill=fillColor,fill opacity=0.20] ( 92.34, 77.14) circle (  2.13);

\path[fill=fillColor,fill opacity=0.20] ( 88.33, 86.37) circle (  2.13);

\path[fill=fillColor,fill opacity=0.20] ( 90.33, 83.08) circle (  2.13);

\path[fill=fillColor,fill opacity=0.20] ( 88.33, 79.67) circle (  2.13);

\path[fill=fillColor,fill opacity=0.20] ( 81.31, 80.55) circle (  2.13);

\path[fill=fillColor,fill opacity=0.20] ( 80.30, 90.16) circle (  2.13);

\path[fill=fillColor,fill opacity=0.20] ( 83.31, 99.65) circle (  2.13);

\path[fill=fillColor,fill opacity=0.20] ( 87.33, 91.43) circle (  2.13);

\path[fill=fillColor,fill opacity=0.20] ( 82.31, 86.12) circle (  2.13);

\path[fill=fillColor,fill opacity=0.20] ( 77.29, 88.65) circle (  2.13);

\path[fill=fillColor,fill opacity=0.20] ( 73.28, 93.20) circle (  2.13);

\path[fill=fillColor,fill opacity=0.20] ( 78.30,102.43) circle (  2.13);

\path[fill=fillColor,fill opacity=0.20] ( 86.32,100.66) circle (  2.13);

\path[fill=fillColor,fill opacity=0.20] ( 79.30, 98.00) circle (  2.13);

\path[fill=fillColor,fill opacity=0.20] ( 80.30,104.32) circle (  2.13);

\path[fill=fillColor,fill opacity=0.20] ( 83.31, 99.90) circle (  2.13);

\path[fill=fillColor,fill opacity=0.20] ( 84.32, 97.62) circle (  2.13);

\path[fill=fillColor,fill opacity=0.20] ( 68.26,102.93) circle (  2.13);

\path[fill=fillColor,fill opacity=0.20] ( 68.26, 94.59) circle (  2.13);

\path[fill=fillColor,fill opacity=0.20] ( 72.28, 40.21) circle (  2.13);

\path[fill=fillColor,fill opacity=0.20] ( 94.35, 49.32) circle (  2.13);

\path[fill=fillColor,fill opacity=0.20] ( 94.35, 60.45) circle (  2.13);

\path[fill=fillColor,fill opacity=0.20] ( 99.36, 58.42) circle (  2.13);

\path[fill=fillColor,fill opacity=0.20] (105.38, 59.94) circle (  2.13);

\path[fill=fillColor,fill opacity=0.20] (107.39, 67.78) circle (  2.13);

\path[fill=fillColor,fill opacity=0.20] ( 98.36, 66.52) circle (  2.13);

\path[fill=fillColor,fill opacity=0.20] ( 90.33, 68.92) circle (  2.13);

\path[fill=fillColor,fill opacity=0.20] ( 88.33, 72.97) circle (  2.13);

\path[fill=fillColor,fill opacity=0.20] ( 90.33, 69.68) circle (  2.13);

\path[fill=fillColor,fill opacity=0.20] ( 95.35, 51.47) circle (  2.13);

\path[fill=fillColor,fill opacity=0.20] ( 91.34, 58.68) circle (  2.13);

\path[fill=fillColor,fill opacity=0.20] (105.38, 46.54) circle (  2.13);

\path[fill=fillColor,fill opacity=0.20] ( 65.25, 49.32) circle (  2.13);

\path[fill=fillColor,fill opacity=0.20] ( 92.34, 54.38) circle (  2.13);

\path[fill=fillColor,fill opacity=0.20] ( 81.31, 55.89) circle (  2.13);

\path[fill=fillColor,fill opacity=0.20] ( 88.33, 51.60) circle (  2.13);

\path[fill=fillColor,fill opacity=0.20] ( 95.35, 68.79) circle (  2.13);

\path[fill=fillColor,fill opacity=0.20] ( 85.32, 81.18) circle (  2.13);

\path[fill=fillColor,fill opacity=0.20] ( 82.31, 63.99) circle (  2.13);

\path[fill=fillColor,fill opacity=0.20] ( 91.34, 58.68) circle (  2.13);

\path[fill=fillColor,fill opacity=0.20] ( 91.34, 56.40) circle (  2.13);

\path[fill=fillColor,fill opacity=0.20] ( 83.31, 44.39) circle (  2.13);

\path[fill=fillColor,fill opacity=0.20] ( 56.73, 40.21) circle (  2.13);

\path[fill=fillColor,fill opacity=0.20] ( 89.33, 39.08) circle (  2.13);

\path[fill=fillColor,fill opacity=0.20] ( 80.30, 47.68) circle (  2.13);

\path[fill=fillColor,fill opacity=0.20] ( 82.31, 64.62) circle (  2.13);

\path[fill=fillColor,fill opacity=0.20] ( 82.31, 69.42) circle (  2.13);

\path[fill=fillColor,fill opacity=0.20] ( 82.31, 76.51) circle (  2.13);

\path[fill=fillColor,fill opacity=0.20] ( 78.30, 76.00) circle (  2.13);

\path[fill=fillColor,fill opacity=0.20] ( 79.30, 70.18) circle (  2.13);

\path[fill=fillColor,fill opacity=0.20] ( 76.29, 68.92) circle (  2.13);

\path[fill=fillColor,fill opacity=0.20] ( 86.32, 88.14) circle (  2.13);

\path[fill=fillColor,fill opacity=0.20] ( 81.31, 85.61) circle (  2.13);

\path[fill=fillColor,fill opacity=0.20] ( 78.30, 84.35) circle (  2.13);

\path[fill=fillColor,fill opacity=0.20] ( 81.31, 90.67) circle (  2.13);

\path[fill=fillColor,fill opacity=0.20] ( 80.30, 99.90) circle (  2.13);

\path[fill=fillColor,fill opacity=0.20] ( 90.33,104.96) circle (  2.13);

\path[fill=fillColor,fill opacity=0.20] ( 84.32, 93.07) circle (  2.13);

\path[fill=fillColor,fill opacity=0.20] ( 81.31, 87.38) circle (  2.13);

\path[fill=fillColor,fill opacity=0.20] ( 87.33, 95.35) circle (  2.13);

\path[fill=fillColor,fill opacity=0.20] ( 91.34, 93.45) circle (  2.13);

\path[fill=fillColor,fill opacity=0.20] ( 84.32, 92.69) circle (  2.13);

\path[fill=fillColor,fill opacity=0.20] ( 86.32, 95.47) circle (  2.13);

\path[fill=fillColor,fill opacity=0.20] ( 81.31, 84.47) circle (  2.13);

\path[fill=fillColor,fill opacity=0.20] ( 90.33, 40.72) circle (  2.13);

\path[fill=fillColor,fill opacity=0.20] ( 98.36, 40.59) circle (  2.13);

\path[fill=fillColor,fill opacity=0.20] ( 94.35, 43.76) circle (  2.13);

\path[fill=fillColor,fill opacity=0.20] ( 94.35, 43.12) circle (  2.13);

\path[fill=fillColor,fill opacity=0.20] (104.38, 52.86) circle (  2.13);

\path[fill=fillColor,fill opacity=0.20] (100.37, 68.16) circle (  2.13);

\path[fill=fillColor,fill opacity=0.20] (100.37, 67.78) circle (  2.13);

\path[fill=fillColor,fill opacity=0.20] (102.37, 67.40) circle (  2.13);

\path[fill=fillColor,fill opacity=0.20] ( 99.36, 64.37) circle (  2.13);

\path[fill=fillColor,fill opacity=0.20] ( 92.34, 61.21) circle (  2.13);

\path[fill=fillColor,fill opacity=0.20] ( 90.33, 71.95) circle (  2.13);

\path[fill=fillColor,fill opacity=0.20] ( 97.36, 86.87) circle (  2.13);

\path[fill=fillColor,fill opacity=0.20] ( 89.33, 67.91) circle (  2.13);

\path[fill=fillColor,fill opacity=0.20] ( 81.31, 65.50) circle (  2.13);

\path[fill=fillColor,fill opacity=0.20] ( 87.33, 78.53) circle (  2.13);

\path[fill=fillColor,fill opacity=0.20] ( 86.32, 77.52) circle (  2.13);

\path[fill=fillColor,fill opacity=0.20] ( 87.33, 74.61) circle (  2.13);

\path[fill=fillColor,fill opacity=0.20] ( 86.32, 72.08) circle (  2.13);

\path[fill=fillColor,fill opacity=0.20] ( 92.34, 68.03) circle (  2.13);

\path[fill=fillColor,fill opacity=0.20] ( 81.31, 73.47) circle (  2.13);

\path[fill=fillColor,fill opacity=0.20] ( 76.29, 76.13) circle (  2.13);

\path[fill=fillColor,fill opacity=0.20] ( 48.40, 71.70) circle (  2.13);

\path[fill=fillColor,fill opacity=0.20] ( 81.31, 73.60) circle (  2.13);

\path[fill=fillColor,fill opacity=0.20] ( 87.33, 78.91) circle (  2.13);

\path[fill=fillColor,fill opacity=0.20] ( 64.65, 70.31) circle (  2.13);

\path[fill=fillColor,fill opacity=0.20] ( 88.33, 57.16) circle (  2.13);

\path[fill=fillColor,fill opacity=0.20] ( 84.32, 54.12) circle (  2.13);

\path[fill=fillColor,fill opacity=0.20] ( 73.28, 63.73) circle (  2.13);

\path[fill=fillColor,fill opacity=0.20] ( 83.31, 69.30) circle (  2.13);

\path[fill=fillColor,fill opacity=0.20] ( 86.32, 50.33) circle (  2.13);

\path[fill=fillColor,fill opacity=0.20] ( 82.31, 43.50) circle (  2.13);

\path[fill=fillColor,fill opacity=0.20] ( 76.29, 40.47) circle (  2.13);

\path[fill=fillColor,fill opacity=0.20] ( 81.31, 50.84) circle (  2.13);

\path[fill=fillColor,fill opacity=0.20] ( 78.30, 53.37) circle (  2.13);

\path[fill=fillColor,fill opacity=0.20] ( 70.27, 56.78) circle (  2.13);

\path[fill=fillColor,fill opacity=0.20] ( 76.29, 56.02) circle (  2.13);

\path[fill=fillColor,fill opacity=0.20] ( 82.31, 65.25) circle (  2.13);

\path[fill=fillColor,fill opacity=0.20] ( 80.30, 73.22) circle (  2.13);

\path[fill=fillColor,fill opacity=0.20] ( 79.30, 77.39) circle (  2.13);

\path[fill=fillColor,fill opacity=0.20] ( 86.32, 83.21) circle (  2.13);

\path[fill=fillColor,fill opacity=0.20] ( 80.30, 90.04) circle (  2.13);

\path[fill=fillColor,fill opacity=0.20] ( 82.31, 88.77) circle (  2.13);

\path[fill=fillColor,fill opacity=0.20] ( 85.32, 89.53) circle (  2.13);

\path[fill=fillColor,fill opacity=0.20] ( 87.33, 84.60) circle (  2.13);

\path[fill=fillColor,fill opacity=0.20] ( 84.32, 81.44) circle (  2.13);

\path[fill=fillColor,fill opacity=0.20] ( 90.33, 83.71) circle (  2.13);

\path[fill=fillColor,fill opacity=0.20] ( 92.34, 80.43) circle (  2.13);

\path[fill=fillColor,fill opacity=0.20] ( 86.32, 81.18) circle (  2.13);

\path[fill=fillColor,fill opacity=0.20] ( 85.32, 83.59) circle (  2.13);

\path[fill=fillColor,fill opacity=0.20] ( 75.29, 70.18) circle (  2.13);

\path[fill=fillColor,fill opacity=0.20] ( 89.33, 37.94) circle (  2.13);

\path[fill=fillColor,fill opacity=0.20] ( 88.33, 39.46) circle (  2.13);

\path[fill=fillColor,fill opacity=0.20] (101.37, 42.87) circle (  2.13);

\path[fill=fillColor,fill opacity=0.20] ( 91.34, 50.08) circle (  2.13);

\path[fill=fillColor,fill opacity=0.20] ( 99.36, 56.27) circle (  2.13);

\path[fill=fillColor,fill opacity=0.20] ( 94.35, 62.85) circle (  2.13);

\path[fill=fillColor,fill opacity=0.20] (101.37, 71.57) circle (  2.13);

\path[fill=fillColor,fill opacity=0.20] ( 98.36, 72.21) circle (  2.13);

\path[fill=fillColor,fill opacity=0.20] ( 94.35, 74.36) circle (  2.13);

\path[fill=fillColor,fill opacity=0.20] ( 87.33, 80.81) circle (  2.13);

\path[fill=fillColor,fill opacity=0.20] ( 92.34, 85.23) circle (  2.13);

\path[fill=fillColor,fill opacity=0.20] ( 80.30, 82.45) circle (  2.13);

\path[fill=fillColor,fill opacity=0.20] ( 88.33, 76.89) circle (  2.13);

\path[fill=fillColor,fill opacity=0.20] ( 93.34, 79.92) circle (  2.13);

\path[fill=fillColor,fill opacity=0.20] ( 87.33, 81.31) circle (  2.13);

\path[fill=fillColor,fill opacity=0.20] ( 85.32, 77.39) circle (  2.13);

\path[fill=fillColor,fill opacity=0.20] ( 86.32, 71.45) circle (  2.13);

\path[fill=fillColor,fill opacity=0.20] ( 79.30, 64.87) circle (  2.13);

\path[fill=fillColor,fill opacity=0.20] ( 53.22, 63.48) circle (  2.13);

\path[fill=fillColor,fill opacity=0.20] ( 78.30, 67.53) circle (  2.13);

\path[fill=fillColor,fill opacity=0.20] ( 78.30, 59.56) circle (  2.13);

\path[fill=fillColor,fill opacity=0.20] ( 72.28, 47.42) circle (  2.13);

\path[fill=fillColor,fill opacity=0.20] ( 76.29, 44.77) circle (  2.13);

\path[fill=fillColor,fill opacity=0.20] ( 74.28, 61.84) circle (  2.13);

\path[fill=fillColor,fill opacity=0.20] ( 83.31, 73.98) circle (  2.13);

\path[fill=fillColor,fill opacity=0.20] ( 82.31, 70.82) circle (  2.13);

\path[fill=fillColor,fill opacity=0.20] ( 77.29, 69.30) circle (  2.13);

\path[fill=fillColor,fill opacity=0.20] ( 84.32, 77.90) circle (  2.13);

\path[fill=fillColor,fill opacity=0.20] ( 85.32, 80.81) circle (  2.13);

\path[fill=fillColor,fill opacity=0.20] ( 85.32, 81.82) circle (  2.13);

\path[fill=fillColor,fill opacity=0.20] ( 87.33, 81.69) circle (  2.13);

\path[fill=fillColor,fill opacity=0.20] ( 88.33, 71.07) circle (  2.13);

\path[fill=fillColor,fill opacity=0.20] ( 81.31, 71.32) circle (  2.13);

\path[fill=fillColor,fill opacity=0.20] ( 70.27, 68.29) circle (  2.13);

\path[fill=fillColor,fill opacity=0.20] ( 88.33, 40.47) circle (  2.13);

\path[fill=fillColor,fill opacity=0.20] ( 88.33, 43.12) circle (  2.13);

\path[fill=fillColor,fill opacity=0.20] ( 89.33, 45.15) circle (  2.13);

\path[fill=fillColor,fill opacity=0.20] ( 95.35, 46.79) circle (  2.13);

\path[fill=fillColor,fill opacity=0.20] ( 96.35, 53.62) circle (  2.13);

\path[fill=fillColor,fill opacity=0.20] ( 99.36, 63.48) circle (  2.13);

\path[fill=fillColor,fill opacity=0.20] (102.37, 75.62) circle (  2.13);

\path[fill=fillColor,fill opacity=0.20] ( 98.36, 80.81) circle (  2.13);

\path[fill=fillColor,fill opacity=0.20] ( 99.36, 70.06) circle (  2.13);

\path[fill=fillColor,fill opacity=0.20] ( 94.35, 65.76) circle (  2.13);

\path[fill=fillColor,fill opacity=0.20] ( 87.33, 76.63) circle (  2.13);

\path[fill=fillColor,fill opacity=0.20] ( 84.32, 81.69) circle (  2.13);

\path[fill=fillColor,fill opacity=0.20] ( 88.33, 76.00) circle (  2.13);

\path[fill=fillColor,fill opacity=0.20] ( 70.27, 71.57) circle (  2.13);

\path[fill=fillColor,fill opacity=0.20] ( 84.32, 64.62) circle (  2.13);

\path[fill=fillColor,fill opacity=0.20] ( 81.31, 57.41) circle (  2.13);

\path[fill=fillColor,fill opacity=0.20] ( 80.30, 51.85) circle (  2.13);

\path[fill=fillColor,fill opacity=0.20] ( 72.28, 49.45) circle (  2.13);

\path[fill=fillColor,fill opacity=0.20] ( 69.27, 46.79) circle (  2.13);

\path[fill=fillColor,fill opacity=0.20] ( 88.33, 47.42) circle (  2.13);

\path[fill=fillColor,fill opacity=0.20] ( 77.29, 46.28) circle (  2.13);

\path[fill=fillColor,fill opacity=0.20] ( 66.66, 51.34) circle (  2.13);

\path[fill=fillColor,fill opacity=0.20] ( 86.32, 73.85) circle (  2.13);

\path[fill=fillColor,fill opacity=0.20] ( 68.26, 66.39) circle (  2.13);

\path[fill=fillColor,fill opacity=0.20] ( 88.33, 42.49) circle (  2.13);

\path[fill=fillColor,fill opacity=0.20] ( 82.31, 51.72) circle (  2.13);

\path[fill=fillColor,fill opacity=0.20] ( 91.34, 59.56) circle (  2.13);

\path[fill=fillColor,fill opacity=0.20] ( 84.32, 54.63) circle (  2.13);

\path[fill=fillColor,fill opacity=0.20] ( 87.33, 49.57) circle (  2.13);

\path[fill=fillColor,fill opacity=0.20] ( 71.27, 57.92) circle (  2.13);

\path[fill=fillColor,fill opacity=0.20] ( 85.32, 69.55) circle (  2.13);

\path[fill=fillColor,fill opacity=0.20] ( 79.30, 59.56) circle (  2.13);

\path[fill=fillColor,fill opacity=0.20] ( 74.28, 52.48) circle (  2.13);

\path[fill=fillColor,fill opacity=0.20] ( 77.29, 50.58) circle (  2.13);

\path[fill=fillColor,fill opacity=0.20] ( 77.29, 46.16) circle (  2.13);

\path[fill=fillColor,fill opacity=0.20] ( 80.30, 44.13) circle (  2.13);

\path[fill=fillColor,fill opacity=0.20] ( 72.28, 40.85) circle (  2.13);

\path[fill=fillColor,fill opacity=0.20] ( 72.28, 38.44) circle (  2.13);

\path[fill=fillColor,fill opacity=0.20] ( 80.30, 43.25) circle (  2.13);

\path[fill=fillColor,fill opacity=0.20] ( 76.29, 67.78) circle (  2.13);

\path[fill=fillColor,fill opacity=0.20] ( 76.29, 77.39) circle (  2.13);

\path[fill=fillColor,fill opacity=0.20] ( 74.28, 78.91) circle (  2.13);

\path[fill=fillColor,fill opacity=0.20] ( 82.31, 84.35) circle (  2.13);

\path[fill=fillColor,fill opacity=0.20] ( 79.30, 71.83) circle (  2.13);

\path[fill=fillColor,fill opacity=0.20] ( 71.27, 95.35) circle (  2.13);

\path[fill=fillColor,fill opacity=0.20] ( 58.23,104.07) circle (  2.13);

\path[fill=fillColor,fill opacity=0.20] ( 75.29,102.30) circle (  2.13);

\path[fill=fillColor,fill opacity=0.20] ( 76.29, 93.45) circle (  2.13);

\path[fill=fillColor,fill opacity=0.20] ( 75.29, 81.94) circle (  2.13);

\path[fill=fillColor,fill opacity=0.20] (102.37, 38.95) circle (  2.13);

\path[fill=fillColor,fill opacity=0.20] ( 95.35, 41.10) circle (  2.13);

\path[fill=fillColor,fill opacity=0.20] ( 98.36, 44.64) circle (  2.13);

\path[fill=fillColor,fill opacity=0.20] (100.37, 43.38) circle (  2.13);

\path[fill=fillColor,fill opacity=0.20] ( 82.31, 59.44) circle (  2.13);

\path[fill=fillColor,fill opacity=0.20] ( 71.27, 98.63) circle (  2.13);

\path[fill=fillColor,fill opacity=0.20] ( 63.95, 97.12) circle (  2.13);

\path[fill=fillColor,fill opacity=0.20] ( 54.52, 96.99) circle (  2.13);

\path[fill=fillColor,fill opacity=0.20] ( 67.16,106.10) circle (  2.13);

\path[fill=fillColor,fill opacity=0.20] ( 74.28, 96.49) circle (  2.13);

\path[fill=fillColor,fill opacity=0.20] ( 57.13, 82.58) circle (  2.13);

\path[fill=fillColor,fill opacity=0.20] ( 89.33, 77.52) circle (  2.13);

\path[fill=fillColor,fill opacity=0.20] ( 89.33, 67.28) circle (  2.13);

\path[fill=fillColor,fill opacity=0.20] ( 94.35, 39.33) circle (  2.13);

\path[fill=fillColor,fill opacity=0.20] ( 89.33, 55.26) circle (  2.13);

\path[fill=fillColor,fill opacity=0.20] ( 95.35, 68.29) circle (  2.13);

\path[fill=fillColor,fill opacity=0.20] ( 93.34, 72.33) circle (  2.13);

\path[fill=fillColor,fill opacity=0.20] ( 91.34, 71.83) circle (  2.13);

\path[fill=fillColor,fill opacity=0.20] ( 87.33, 72.84) circle (  2.13);

\path[fill=fillColor,fill opacity=0.20] ( 85.32, 73.34) circle (  2.13);

\path[fill=fillColor,fill opacity=0.20] ( 86.32, 59.81) circle (  2.13);

\path[fill=fillColor,fill opacity=0.20] ( 95.35, 42.74) circle (  2.13);

\path[fill=fillColor,fill opacity=0.20] ( 74.28, 77.52) circle (  2.13);

\path[fill=fillColor,fill opacity=0.20] ( 67.46,106.47) circle (  2.13);

\path[fill=fillColor,fill opacity=0.20] ( 58.93, 87.25) circle (  2.13);

\path[fill=fillColor,fill opacity=0.20] ( 61.04, 91.68) circle (  2.13);

\path[fill=fillColor,fill opacity=0.20] ( 54.02, 97.88) circle (  2.13);

\path[fill=fillColor,fill opacity=0.20] ( 76.29, 90.54) circle (  2.13);

\path[fill=fillColor,fill opacity=0.20] ( 78.30, 87.25) circle (  2.13);

\path[fill=fillColor,fill opacity=0.20] ( 77.29, 89.15) circle (  2.13);

\path[fill=fillColor,fill opacity=0.20] ( 82.31, 90.29) circle (  2.13);

\path[fill=fillColor,fill opacity=0.20] ( 92.34, 81.18) circle (  2.13);

\path[fill=fillColor,fill opacity=0.20] ( 92.34, 49.83) circle (  2.13);

\path[fill=fillColor,fill opacity=0.20] ( 88.33, 90.67) circle (  2.13);

\path[fill=fillColor,fill opacity=0.20] ( 82.31,113.43) circle (  2.13);

\path[fill=fillColor,fill opacity=0.20] ( 84.32,110.02) circle (  2.13);

\path[fill=fillColor,fill opacity=0.20] ( 84.32,102.43) circle (  2.13);

\path[fill=fillColor,fill opacity=0.20] ( 89.33, 91.68) circle (  2.13);

\path[fill=fillColor,fill opacity=0.20] ( 88.33, 85.48) circle (  2.13);

\path[fill=fillColor,fill opacity=0.20] ( 87.33, 84.22) circle (  2.13);

\path[fill=fillColor,fill opacity=0.20] ( 58.93, 77.77) circle (  2.13);

\path[fill=fillColor,fill opacity=0.20] (100.37, 63.86) circle (  2.13);

\path[fill=fillColor,fill opacity=0.20] (115.42, 50.96) circle (  2.13);

\path[fill=fillColor,fill opacity=0.20] ( 76.29, 79.03) circle (  2.13);

\path[fill=fillColor,fill opacity=0.20] ( 68.26,100.03) circle (  2.13);

\path[fill=fillColor,fill opacity=0.20] ( 64.45, 85.23) circle (  2.13);

\path[fill=fillColor,fill opacity=0.20] ( 48.00, 91.30) circle (  2.13);

\path[fill=fillColor,fill opacity=0.20] ( 61.54, 85.48) circle (  2.13);

\path[fill=fillColor,fill opacity=0.20] ( 79.30, 95.47) circle (  2.13);

\path[fill=fillColor,fill opacity=0.20] ( 82.31,103.44) circle (  2.13);

\path[fill=fillColor,fill opacity=0.20] ( 84.32, 85.61) circle (  2.13);

\path[fill=fillColor,fill opacity=0.20] ( 86.32, 76.51) circle (  2.13);

\path[fill=fillColor,fill opacity=0.20] ( 89.33, 69.93) circle (  2.13);

\path[fill=fillColor,fill opacity=0.20] ( 86.32, 55.01) circle (  2.13);

\path[fill=fillColor,fill opacity=0.20] ( 81.31,100.78) circle (  2.13);

\path[fill=fillColor,fill opacity=0.20] ( 77.29,114.31) circle (  2.13);

\path[fill=fillColor,fill opacity=0.20] ( 71.27,109.89) circle (  2.13);

\path[fill=fillColor,fill opacity=0.20] ( 65.66,108.12) circle (  2.13);

\path[fill=fillColor,fill opacity=0.20] ( 71.27,106.10) circle (  2.13);

\path[fill=fillColor,fill opacity=0.20] ( 75.29,104.32) circle (  2.13);

\path[fill=fillColor,fill opacity=0.20] ( 81.31,102.68) circle (  2.13);

\path[fill=fillColor,fill opacity=0.20] ( 95.35, 88.27) circle (  2.13);

\path[fill=fillColor,fill opacity=0.20] (102.37, 72.71) circle (  2.13);

\path[fill=fillColor,fill opacity=0.20] ( 91.34, 70.31) circle (  2.13);

\path[fill=fillColor,fill opacity=0.20] (118.43, 49.45) circle (  2.13);

\path[fill=fillColor,fill opacity=0.20] ( 78.30, 73.98) circle (  2.13);

\path[fill=fillColor,fill opacity=0.20] ( 67.36,102.18) circle (  2.13);

\path[fill=fillColor,fill opacity=0.20] ( 60.84, 89.91) circle (  2.13);

\path[fill=fillColor,fill opacity=0.20] ( 47.20, 97.50) circle (  2.13);

\path[fill=fillColor,fill opacity=0.20] ( 61.64, 99.27) circle (  2.13);

\path[fill=fillColor,fill opacity=0.20] ( 51.91, 93.45) circle (  2.13);

\path[fill=fillColor,fill opacity=0.20] ( 70.27,104.58) circle (  2.13);

\path[fill=fillColor,fill opacity=0.20] ( 83.31,110.02) circle (  2.13);

\path[fill=fillColor,fill opacity=0.20] ( 83.31, 87.76) circle (  2.13);

\path[fill=fillColor,fill opacity=0.20] ( 83.31, 80.17) circle (  2.13);

\path[fill=fillColor,fill opacity=0.20] ( 91.34, 74.48) circle (  2.13);

\path[fill=fillColor,fill opacity=0.20] ( 80.30, 92.94) circle (  2.13);

\path[fill=fillColor,fill opacity=0.20] ( 65.45,114.95) circle (  2.13);

\path[fill=fillColor,fill opacity=0.20] ( 61.54,111.15) circle (  2.13);

\path[fill=fillColor,fill opacity=0.20] ( 52.51,108.37) circle (  2.13);

\path[fill=fillColor,fill opacity=0.20] ( 69.27,110.90) circle (  2.13);

\path[fill=fillColor,fill opacity=0.20] ( 81.31,114.19) circle (  2.13);

\path[fill=fillColor,fill opacity=0.20] ( 94.35, 96.49) circle (  2.13);

\path[fill=fillColor,fill opacity=0.20] (106.39, 74.86) circle (  2.13);

\path[fill=fillColor,fill opacity=0.20] ( 89.33, 80.81) circle (  2.13);

\path[fill=fillColor,fill opacity=0.20] ( 75.29,110.14) circle (  2.13);

\path[fill=fillColor,fill opacity=0.20] ( 69.27, 97.24) circle (  2.13);

\path[fill=fillColor,fill opacity=0.20] ( 66.56,101.67) circle (  2.13);

\path[fill=fillColor,fill opacity=0.20] ( 69.27,110.65) circle (  2.13);

\path[fill=fillColor,fill opacity=0.20] ( 70.27,106.85) circle (  2.13);

\path[fill=fillColor,fill opacity=0.20] ( 73.28,105.84) circle (  2.13);

\path[fill=fillColor,fill opacity=0.20] ( 80.30,101.92) circle (  2.13);

\path[fill=fillColor,fill opacity=0.20] ( 86.32, 95.60) circle (  2.13);

\path[fill=fillColor,fill opacity=0.20] ( 70.27, 94.21) circle (  2.13);

\path[fill=fillColor,fill opacity=0.20] ( 93.34, 60.32) circle (  2.13);

\path[fill=fillColor,fill opacity=0.20] ( 82.31,111.66) circle (  2.13);

\path[fill=fillColor,fill opacity=0.20] ( 48.50,108.12) circle (  2.13);

\path[fill=fillColor,fill opacity=0.20] ( 77.29, 99.52) circle (  2.13);

\path[fill=fillColor,fill opacity=0.20] ( 91.34, 93.70) circle (  2.13);

\path[fill=fillColor,fill opacity=0.20] ( 98.36, 92.94) circle (  2.13);

\path[fill=fillColor,fill opacity=0.20] (111.40, 80.17) circle (  2.13);

\path[fill=fillColor,fill opacity=0.20] ( 83.31, 85.10) circle (  2.13);

\path[fill=fillColor,fill opacity=0.20] ( 82.31, 98.89) circle (  2.13);

\path[fill=fillColor,fill opacity=0.20] ( 79.30, 93.70) circle (  2.13);

\path[fill=fillColor,fill opacity=0.20] ( 61.74, 97.50) circle (  2.13);

\path[fill=fillColor,fill opacity=0.20] ( 72.28,106.73) circle (  2.13);

\path[fill=fillColor,fill opacity=0.20] ( 73.28,104.07) circle (  2.13);

\path[fill=fillColor,fill opacity=0.20] ( 73.28, 98.38) circle (  2.13);

\path[fill=fillColor,fill opacity=0.20] ( 73.28,100.53) circle (  2.13);

\path[fill=fillColor,fill opacity=0.20] ( 83.31,101.16) circle (  2.13);

\path[fill=fillColor,fill opacity=0.20] ( 85.32, 94.59) circle (  2.13);

\path[fill=fillColor,fill opacity=0.20] ( 84.32, 76.13) circle (  2.13);

\path[fill=fillColor,fill opacity=0.20] ( 72.28,114.95) circle (  2.13);

\path[fill=fillColor,fill opacity=0.20] ( 65.66,110.65) circle (  2.13);

\path[fill=fillColor,fill opacity=0.20] ( 73.28,111.15) circle (  2.13);

\path[fill=fillColor,fill opacity=0.20] ( 75.29,101.42) circle (  2.13);

\path[fill=fillColor,fill opacity=0.20] ( 75.29, 90.79) circle (  2.13);

\path[fill=fillColor,fill opacity=0.20] ( 95.35, 80.68) circle (  2.13);

\path[fill=fillColor,fill opacity=0.20] (107.39, 76.25) circle (  2.13);

\path[fill=fillColor,fill opacity=0.20] ( 87.33, 62.72) circle (  2.13);

\path[fill=fillColor,fill opacity=0.20] ( 83.31, 87.51) circle (  2.13);

\path[fill=fillColor,fill opacity=0.20] ( 78.30, 93.83) circle (  2.13);

\path[fill=fillColor,fill opacity=0.20] ( 83.31, 94.97) circle (  2.13);

\path[fill=fillColor,fill opacity=0.20] ( 83.31, 97.50) circle (  2.13);

\path[fill=fillColor,fill opacity=0.20] ( 79.30, 98.00) circle (  2.13);

\path[fill=fillColor,fill opacity=0.20] ( 79.30, 97.37) circle (  2.13);

\path[fill=fillColor,fill opacity=0.20] ( 66.56,100.91) circle (  2.13);

\path[fill=fillColor,fill opacity=0.20] ( 73.28,104.32) circle (  2.13);

\path[fill=fillColor,fill opacity=0.20] ( 77.29, 98.13) circle (  2.13);

\path[fill=fillColor,fill opacity=0.20] ( 87.33, 88.14) circle (  2.13);

\path[fill=fillColor,fill opacity=0.20] ( 98.36, 71.95) circle (  2.13);

\path[fill=fillColor,fill opacity=0.20] (115.42, 50.33) circle (  2.13);

\path[fill=fillColor,fill opacity=0.20] ( 83.31, 81.06) circle (  2.13);

\path[fill=fillColor,fill opacity=0.20] ( 69.27,112.80) circle (  2.13);

\path[fill=fillColor,fill opacity=0.20] ( 67.46,104.07) circle (  2.13);

\path[fill=fillColor,fill opacity=0.20] ( 65.76,101.54) circle (  2.13);

\path[fill=fillColor,fill opacity=0.20] ( 75.29, 99.14) circle (  2.13);

\path[fill=fillColor,fill opacity=0.20] ( 87.33, 99.65) circle (  2.13);

\path[fill=fillColor,fill opacity=0.20] ( 88.33, 96.74) circle (  2.13);

\path[fill=fillColor,fill opacity=0.20] ( 90.33, 83.59) circle (  2.13);

\path[fill=fillColor,fill opacity=0.20] ( 88.33, 78.15) circle (  2.13);

\path[fill=fillColor,fill opacity=0.20] ( 84.32,100.78) circle (  2.13);

\path[fill=fillColor,fill opacity=0.20] ( 86.32,101.29) circle (  2.13);

\path[fill=fillColor,fill opacity=0.20] ( 76.29, 98.76) circle (  2.13);

\path[fill=fillColor,fill opacity=0.20] ( 75.29, 92.69) circle (  2.13);

\path[fill=fillColor,fill opacity=0.20] ( 72.28, 96.61) circle (  2.13);

\path[fill=fillColor,fill opacity=0.20] ( 82.31,106.47) circle (  2.13);

\path[fill=fillColor,fill opacity=0.20] ( 83.31,103.06) circle (  2.13);

\path[fill=fillColor,fill opacity=0.20] ( 92.34, 97.24) circle (  2.13);

\path[fill=fillColor,fill opacity=0.20] ( 92.34, 89.40) circle (  2.13);

\path[fill=fillColor,fill opacity=0.20] ( 84.32, 71.20) circle (  2.13);

\path[fill=fillColor,fill opacity=0.20] ( 77.29,107.87) circle (  2.13);

\path[fill=fillColor,fill opacity=0.20] ( 68.26,110.65) circle (  2.13);

\path[fill=fillColor,fill opacity=0.20] ( 68.26,106.85) circle (  2.13);

\path[fill=fillColor,fill opacity=0.20] ( 73.28, 95.47) circle (  2.13);

\path[fill=fillColor,fill opacity=0.20] ( 78.30, 90.92) circle (  2.13);

\path[fill=fillColor,fill opacity=0.20] ( 87.33, 96.11) circle (  2.13);

\path[fill=fillColor,fill opacity=0.20] ( 87.33, 91.17) circle (  2.13);

\path[fill=fillColor,fill opacity=0.20] ( 88.33, 79.03) circle (  2.13);

\path[fill=fillColor,fill opacity=0.20] (109.40, 60.83) circle (  2.13);

\path[fill=fillColor,fill opacity=0.20] ( 93.34, 54.25) circle (  2.13);

\path[fill=fillColor,fill opacity=0.20] ( 86.32, 81.44) circle (  2.13);

\path[fill=fillColor,fill opacity=0.20] ( 82.31, 99.39) circle (  2.13);

\path[fill=fillColor,fill opacity=0.20] ( 79.30, 94.97) circle (  2.13);

\path[fill=fillColor,fill opacity=0.20] ( 80.30, 92.94) circle (  2.13);

\path[fill=fillColor,fill opacity=0.20] ( 79.30, 98.51) circle (  2.13);

\path[fill=fillColor,fill opacity=0.20] ( 83.31, 95.22) circle (  2.13);

\path[fill=fillColor,fill opacity=0.20] ( 79.30, 95.22) circle (  2.13);

\path[fill=fillColor,fill opacity=0.20] ( 78.30, 97.88) circle (  2.13);

\path[fill=fillColor,fill opacity=0.20] ( 86.32, 97.37) circle (  2.13);

\path[fill=fillColor,fill opacity=0.20] ( 93.34, 99.27) circle (  2.13);

\path[fill=fillColor,fill opacity=0.20] ( 67.86, 88.27) circle (  2.13);

\path[fill=fillColor,fill opacity=0.20] ( 70.27,102.30) circle (  2.13);

\path[fill=fillColor,fill opacity=0.20] ( 66.86,111.15) circle (  2.13);

\path[fill=fillColor,fill opacity=0.20] ( 80.30,103.57) circle (  2.13);

\path[fill=fillColor,fill opacity=0.20] ( 85.32, 93.45) circle (  2.13);

\path[fill=fillColor,fill opacity=0.20] ( 89.33, 89.28) circle (  2.13);

\path[fill=fillColor,fill opacity=0.20] ( 89.33, 82.70) circle (  2.13);

\path[fill=fillColor,fill opacity=0.20] ( 91.34, 74.99) circle (  2.13);

\path[fill=fillColor,fill opacity=0.20] ( 97.36, 63.99) circle (  2.13);

\path[fill=fillColor,fill opacity=0.20] ( 82.31, 61.46) circle (  2.13);

\path[fill=fillColor,fill opacity=0.20] ( 84.32, 74.86) circle (  2.13);

\path[fill=fillColor,fill opacity=0.20] ( 77.29, 84.09) circle (  2.13);

\path[fill=fillColor,fill opacity=0.20] ( 71.27, 98.38) circle (  2.13);

\path[fill=fillColor,fill opacity=0.20] ( 77.29,110.65) circle (  2.13);

\path[fill=fillColor,fill opacity=0.20] ( 80.30,104.83) circle (  2.13);

\path[fill=fillColor,fill opacity=0.20] ( 72.28, 92.31) circle (  2.13);

\path[fill=fillColor,fill opacity=0.20] ( 77.29, 89.02) circle (  2.13);

\path[fill=fillColor,fill opacity=0.20] ( 81.31, 96.36) circle (  2.13);

\path[fill=fillColor,fill opacity=0.20] ( 89.33, 97.37) circle (  2.13);

\path[fill=fillColor,fill opacity=0.20] (104.38, 75.49) circle (  2.13);

\path[fill=fillColor,fill opacity=0.20] ( 82.31, 71.57) circle (  2.13);

\path[fill=fillColor,fill opacity=0.20] ( 67.76, 90.92) circle (  2.13);

\path[fill=fillColor,fill opacity=0.20] ( 57.93,107.87) circle (  2.13);

\path[fill=fillColor,fill opacity=0.20] ( 82.31,115.96) circle (  2.13);

\path[fill=fillColor,fill opacity=0.20] ( 81.31,102.18) circle (  2.13);

\path[fill=fillColor,fill opacity=0.20] ( 90.33, 92.31) circle (  2.13);

\path[fill=fillColor,fill opacity=0.20] ( 98.36, 87.13) circle (  2.13);

\path[fill=fillColor,fill opacity=0.20] ( 99.36, 76.89) circle (  2.13);

\path[fill=fillColor,fill opacity=0.20] (103.38, 66.39) circle (  2.13);

\path[fill=fillColor,fill opacity=0.20] (122.44, 54.12) circle (  2.13);

\path[fill=fillColor,fill opacity=0.20] ( 76.29, 76.51) circle (  2.13);

\path[fill=fillColor,fill opacity=0.20] ( 74.28, 82.70) circle (  2.13);

\path[fill=fillColor,fill opacity=0.20] ( 72.28, 89.91) circle (  2.13);

\path[fill=fillColor,fill opacity=0.20] ( 65.15,102.81) circle (  2.13);

\path[fill=fillColor,fill opacity=0.20] ( 74.28,108.50) circle (  2.13);

\path[fill=fillColor,fill opacity=0.20] ( 70.27,101.67) circle (  2.13);

\path[fill=fillColor,fill opacity=0.20] ( 77.29, 95.60) circle (  2.13);

\path[fill=fillColor,fill opacity=0.20] ( 84.32, 94.97) circle (  2.13);

\path[fill=fillColor,fill opacity=0.20] (112.41, 63.86) circle (  2.13);

\path[fill=fillColor,fill opacity=0.20] (131.47, 49.57) circle (  2.13);

\path[fill=fillColor,fill opacity=0.20] ( 90.33, 58.17) circle (  2.13);

\path[fill=fillColor,fill opacity=0.20] ( 77.29, 84.85) circle (  2.13);

\path[fill=fillColor,fill opacity=0.20] ( 81.31, 97.24) circle (  2.13);

\path[fill=fillColor,fill opacity=0.20] ( 81.31,104.20) circle (  2.13);

\path[fill=fillColor,fill opacity=0.20] ( 85.32,102.68) circle (  2.13);

\path[fill=fillColor,fill opacity=0.20] ( 90.33, 96.99) circle (  2.13);

\path[fill=fillColor,fill opacity=0.20] ( 92.34, 94.97) circle (  2.13);

\path[fill=fillColor,fill opacity=0.20] ( 85.32, 84.47) circle (  2.13);

\path[fill=fillColor,fill opacity=0.20] ( 95.35, 72.71) circle (  2.13);

\path[fill=fillColor,fill opacity=0.20] (108.39, 63.73) circle (  2.13);

\path[fill=fillColor,fill opacity=0.20] ( 79.30, 74.48) circle (  2.13);

\path[fill=fillColor,fill opacity=0.20] ( 73.28, 83.71) circle (  2.13);

\path[fill=fillColor,fill opacity=0.20] ( 72.28, 94.59) circle (  2.13);

\path[fill=fillColor,fill opacity=0.20] ( 71.27, 98.00) circle (  2.13);

\path[fill=fillColor,fill opacity=0.20] ( 63.45, 91.30) circle (  2.13);

\path[fill=fillColor,fill opacity=0.20] ( 65.66, 88.77) circle (  2.13);

\path[fill=fillColor,fill opacity=0.20] ( 78.30, 94.59) circle (  2.13);

\path[fill=fillColor,fill opacity=0.20] ( 83.31,103.31) circle (  2.13);

\path[fill=fillColor,fill opacity=0.20] ( 87.33,102.68) circle (  2.13);

\path[fill=fillColor,fill opacity=0.20] ( 81.31, 86.12) circle (  2.13);

\path[fill=fillColor,fill opacity=0.20] ( 83.31, 93.96) circle (  2.13);

\path[fill=fillColor,fill opacity=0.20] ( 88.33,101.29) circle (  2.13);

\path[fill=fillColor,fill opacity=0.20] ( 88.33, 98.26) circle (  2.13);

\path[fill=fillColor,fill opacity=0.20] ( 89.33, 96.61) circle (  2.13);

\path[fill=fillColor,fill opacity=0.20] ( 93.34, 96.74) circle (  2.13);

\path[fill=fillColor,fill opacity=0.20] ( 98.36, 76.00) circle (  2.13);

\path[fill=fillColor,fill opacity=0.20] (104.38, 57.79) circle (  2.13);

\path[fill=fillColor,fill opacity=0.20] ( 91.34, 47.55) circle (  2.13);

\path[fill=fillColor,fill opacity=0.20] ( 81.31, 66.77) circle (  2.13);

\path[fill=fillColor,fill opacity=0.20] ( 85.32, 68.79) circle (  2.13);

\path[fill=fillColor,fill opacity=0.20] ( 83.31, 79.03) circle (  2.13);

\path[fill=fillColor,fill opacity=0.20] ( 75.29, 87.13) circle (  2.13);

\path[fill=fillColor,fill opacity=0.20] ( 71.27, 92.06) circle (  2.13);

\path[fill=fillColor,fill opacity=0.20] ( 69.27, 93.32) circle (  2.13);

\path[fill=fillColor,fill opacity=0.20] ( 71.27, 84.85) circle (  2.13);

\path[fill=fillColor,fill opacity=0.20] ( 59.64, 84.98) circle (  2.13);

\path[fill=fillColor,fill opacity=0.20] ( 82.31, 99.39) circle (  2.13);

\path[fill=fillColor,fill opacity=0.20] ( 92.34,105.46) circle (  2.13);

\path[fill=fillColor,fill opacity=0.20] ( 85.32, 82.70) circle (  2.13);

\path[fill=fillColor,fill opacity=0.20] ( 95.35, 69.80) circle (  2.13);

\path[fill=fillColor,fill opacity=0.20] (105.38, 70.94) circle (  2.13);

\path[fill=fillColor,fill opacity=0.20] ( 75.29, 76.89) circle (  2.13);

\path[fill=fillColor,fill opacity=0.20] ( 82.31, 93.07) circle (  2.13);

\path[fill=fillColor,fill opacity=0.20] ( 87.33, 98.51) circle (  2.13);

\path[fill=fillColor,fill opacity=0.20] ( 92.34, 98.26) circle (  2.13);

\path[fill=fillColor,fill opacity=0.20] ( 91.34, 98.76) circle (  2.13);

\path[fill=fillColor,fill opacity=0.20] ( 97.36, 94.97) circle (  2.13);

\path[fill=fillColor,fill opacity=0.20] ( 85.32, 78.40) circle (  2.13);

\path[fill=fillColor,fill opacity=0.20] (101.37, 61.96) circle (  2.13);

\path[fill=fillColor,fill opacity=0.20] ( 67.16, 48.69) circle (  2.13);

\path[fill=fillColor,fill opacity=0.20] ( 89.33, 64.87) circle (  2.13);

\path[fill=fillColor,fill opacity=0.20] ( 89.33, 70.69) circle (  2.13);

\path[fill=fillColor,fill opacity=0.20] ( 84.32, 87.13) circle (  2.13);

\path[fill=fillColor,fill opacity=0.20] ( 83.31, 91.81) circle (  2.13);

\path[fill=fillColor,fill opacity=0.20] ( 80.30, 83.84) circle (  2.13);

\path[fill=fillColor,fill opacity=0.20] ( 77.29, 83.08) circle (  2.13);

\path[fill=fillColor,fill opacity=0.20] ( 72.28, 86.62) circle (  2.13);

\path[fill=fillColor,fill opacity=0.20] ( 81.31, 88.77) circle (  2.13);

\path[fill=fillColor,fill opacity=0.20] ( 86.32, 94.21) circle (  2.13);

\path[fill=fillColor,fill opacity=0.20] ( 88.33,101.16) circle (  2.13);

\path[fill=fillColor,fill opacity=0.20] ( 96.35, 90.04) circle (  2.13);

\path[fill=fillColor,fill opacity=0.20] (115.42, 71.83) circle (  2.13);

\path[fill=fillColor,fill opacity=0.20] ( 81.31, 69.93) circle (  2.13);

\path[fill=fillColor,fill opacity=0.20] ( 77.29, 85.99) circle (  2.13);

\path[fill=fillColor,fill opacity=0.20] ( 82.31, 84.22) circle (  2.13);

\path[fill=fillColor,fill opacity=0.20] ( 89.33, 92.44) circle (  2.13);

\path[fill=fillColor,fill opacity=0.20] ( 93.34, 96.49) circle (  2.13);

\path[fill=fillColor,fill opacity=0.20] (100.37, 85.48) circle (  2.13);

\path[fill=fillColor,fill opacity=0.20] (100.37, 78.15) circle (  2.13);

\path[fill=fillColor,fill opacity=0.20] ( 95.35, 80.43) circle (  2.13);

\path[fill=fillColor,fill opacity=0.20] ( 90.33, 69.17) circle (  2.13);

\path[fill=fillColor,fill opacity=0.20] (116.42, 54.63) circle (  2.13);

\path[fill=fillColor,fill opacity=0.20] ( 89.33, 65.38) circle (  2.13);

\path[fill=fillColor,fill opacity=0.20] ( 86.32, 69.93) circle (  2.13);

\path[fill=fillColor,fill opacity=0.20] ( 87.33, 75.37) circle (  2.13);

\path[fill=fillColor,fill opacity=0.20] ( 85.32, 93.58) circle (  2.13);

\path[fill=fillColor,fill opacity=0.20] ( 79.30,105.21) circle (  2.13);

\path[fill=fillColor,fill opacity=0.20] ( 74.28, 95.22) circle (  2.13);

\path[fill=fillColor,fill opacity=0.20] ( 82.31, 89.02) circle (  2.13);

\path[fill=fillColor,fill opacity=0.20] ( 77.29, 87.13) circle (  2.13);

\path[fill=fillColor,fill opacity=0.20] ( 80.30, 84.73) circle (  2.13);

\path[fill=fillColor,fill opacity=0.20] ( 87.33, 87.51) circle (  2.13);

\path[fill=fillColor,fill opacity=0.20] ( 98.36, 89.66) circle (  2.13);

\path[fill=fillColor,fill opacity=0.20] ( 92.34, 85.86) circle (  2.13);

\path[fill=fillColor,fill opacity=0.20] ( 96.35, 77.26) circle (  2.13);

\path[fill=fillColor,fill opacity=0.20] ( 63.65, 51.72) circle (  2.13);

\path[fill=fillColor,fill opacity=0.20] ( 82.31, 69.93) circle (  2.13);

\path[fill=fillColor,fill opacity=0.20] ( 80.30, 78.53) circle (  2.13);

\path[fill=fillColor,fill opacity=0.20] ( 88.33, 85.74) circle (  2.13);

\path[fill=fillColor,fill opacity=0.20] ( 94.35, 85.36) circle (  2.13);

\path[fill=fillColor,fill opacity=0.20] ( 88.33, 86.50) circle (  2.13);

\path[fill=fillColor,fill opacity=0.20] ( 95.35, 99.65) circle (  2.13);

\path[fill=fillColor,fill opacity=0.20] ( 97.36, 89.78) circle (  2.13);

\path[fill=fillColor,fill opacity=0.20] (103.38, 74.36) circle (  2.13);

\path[fill=fillColor,fill opacity=0.20] ( 96.35, 73.85) circle (  2.13);

\path[fill=fillColor,fill opacity=0.20] (104.38, 64.75) circle (  2.13);

\path[fill=fillColor,fill opacity=0.20] (118.43, 54.88) circle (  2.13);

\path[fill=fillColor,fill opacity=0.20] ( 86.32, 65.00) circle (  2.13);

\path[fill=fillColor,fill opacity=0.20] ( 83.31, 74.86) circle (  2.13);

\path[fill=fillColor,fill opacity=0.20] ( 81.31, 78.78) circle (  2.13);

\path[fill=fillColor,fill opacity=0.20] ( 79.30, 85.48) circle (  2.13);

\path[fill=fillColor,fill opacity=0.20] ( 81.31, 99.27) circle (  2.13);

\path[fill=fillColor,fill opacity=0.20] ( 77.29, 97.88) circle (  2.13);

\path[fill=fillColor,fill opacity=0.20] ( 79.30, 88.52) circle (  2.13);

\path[fill=fillColor,fill opacity=0.20] ( 75.29, 92.06) circle (  2.13);

\path[fill=fillColor,fill opacity=0.20] ( 82.31, 95.09) circle (  2.13);

\path[fill=fillColor,fill opacity=0.20] ( 84.32, 86.37) circle (  2.13);

\path[fill=fillColor,fill opacity=0.20] ( 88.33, 82.20) circle (  2.13);

\path[fill=fillColor,fill opacity=0.20] ( 95.35, 78.02) circle (  2.13);

\path[fill=fillColor,fill opacity=0.20] ( 97.36, 74.48) circle (  2.13);

\path[fill=fillColor,fill opacity=0.20] ( 98.36, 71.07) circle (  2.13);

\path[fill=fillColor,fill opacity=0.20] ( 66.36, 56.40) circle (  2.13);

\path[fill=fillColor,fill opacity=0.20] ( 80.30, 82.32) circle (  2.13);

\path[fill=fillColor,fill opacity=0.20] ( 77.29, 84.09) circle (  2.13);

\path[fill=fillColor,fill opacity=0.20] ( 84.32, 74.86) circle (  2.13);

\path[fill=fillColor,fill opacity=0.20] ( 78.30, 91.05) circle (  2.13);

\path[fill=fillColor,fill opacity=0.20] ( 85.32, 85.36) circle (  2.13);

\path[fill=fillColor,fill opacity=0.20] ( 91.34, 87.38) circle (  2.13);

\path[fill=fillColor,fill opacity=0.20] ( 92.34, 82.95) circle (  2.13);

\path[fill=fillColor,fill opacity=0.20] ( 92.34, 79.03) circle (  2.13);

\path[fill=fillColor,fill opacity=0.20] ( 99.36, 76.63) circle (  2.13);

\path[fill=fillColor,fill opacity=0.20] (103.38, 73.85) circle (  2.13);

\path[fill=fillColor,fill opacity=0.20] ( 99.36, 65.88) circle (  2.13);

\path[fill=fillColor,fill opacity=0.20] ( 83.31, 66.77) circle (  2.13);

\path[fill=fillColor,fill opacity=0.20] ( 82.31, 79.92) circle (  2.13);

\path[fill=fillColor,fill opacity=0.20] ( 77.29, 88.39) circle (  2.13);

\path[fill=fillColor,fill opacity=0.20] ( 75.29, 89.28) circle (  2.13);

\path[fill=fillColor,fill opacity=0.20] ( 77.29, 86.50) circle (  2.13);

\path[fill=fillColor,fill opacity=0.20] ( 76.29, 87.63) circle (  2.13);

\path[fill=fillColor,fill opacity=0.20] ( 77.29, 88.77) circle (  2.13);

\path[fill=fillColor,fill opacity=0.20] ( 82.31, 90.16) circle (  2.13);

\path[fill=fillColor,fill opacity=0.20] ( 83.31, 94.84) circle (  2.13);

\path[fill=fillColor,fill opacity=0.20] ( 88.33, 89.40) circle (  2.13);

\path[fill=fillColor,fill opacity=0.20] ( 97.36, 79.03) circle (  2.13);

\path[fill=fillColor,fill opacity=0.20] ( 92.34, 79.54) circle (  2.13);

\path[fill=fillColor,fill opacity=0.20] ( 94.35, 74.23) circle (  2.13);

\path[fill=fillColor,fill opacity=0.20] (102.37, 61.08) circle (  2.13);

\path[fill=fillColor,fill opacity=0.20] ( 69.27, 72.08) circle (  2.13);

\path[fill=fillColor,fill opacity=0.20] ( 77.29, 82.20) circle (  2.13);

\path[fill=fillColor,fill opacity=0.20] ( 88.33, 83.33) circle (  2.13);

\path[fill=fillColor,fill opacity=0.20] ( 83.31, 85.61) circle (  2.13);

\path[fill=fillColor,fill opacity=0.20] ( 83.31, 93.83) circle (  2.13);

\path[fill=fillColor,fill opacity=0.20] ( 81.31, 99.27) circle (  2.13);

\path[fill=fillColor,fill opacity=0.20] ( 87.33, 82.95) circle (  2.13);

\path[fill=fillColor,fill opacity=0.20] ( 84.32, 74.61) circle (  2.13);

\path[fill=fillColor,fill opacity=0.20] ( 98.36, 71.07) circle (  2.13);

\path[fill=fillColor,fill opacity=0.20] ( 94.35, 72.71) circle (  2.13);

\path[fill=fillColor,fill opacity=0.20] ( 96.35, 68.03) circle (  2.13);

\path[fill=fillColor,fill opacity=0.20] ( 88.33, 56.65) circle (  2.13);

\path[fill=fillColor,fill opacity=0.20] ( 84.32, 71.32) circle (  2.13);

\path[fill=fillColor,fill opacity=0.20] ( 85.32, 80.17) circle (  2.13);

\path[fill=fillColor,fill opacity=0.20] ( 65.96, 85.36) circle (  2.13);

\path[fill=fillColor,fill opacity=0.20] ( 76.29, 86.62) circle (  2.13);

\path[fill=fillColor,fill opacity=0.20] ( 74.28, 81.44) circle (  2.13);

\path[fill=fillColor,fill opacity=0.20] ( 74.28, 74.99) circle (  2.13);

\path[fill=fillColor,fill opacity=0.20] ( 79.30, 76.76) circle (  2.13);

\path[fill=fillColor,fill opacity=0.20] ( 75.29, 84.47) circle (  2.13);

\path[fill=fillColor,fill opacity=0.20] ( 88.33, 94.21) circle (  2.13);

\path[fill=fillColor,fill opacity=0.20] ( 88.33, 96.61) circle (  2.13);

\path[fill=fillColor,fill opacity=0.20] ( 97.36, 75.37) circle (  2.13);

\path[fill=fillColor,fill opacity=0.20] (102.37, 61.84) circle (  2.13);

\path[fill=fillColor,fill opacity=0.20] (109.40, 67.40) circle (  2.13);

\path[fill=fillColor,fill opacity=0.20] (126.45, 59.69) circle (  2.13);

\path[fill=fillColor,fill opacity=0.20] ( 71.27, 71.57) circle (  2.13);

\path[fill=fillColor,fill opacity=0.20] ( 85.32, 78.28) circle (  2.13);

\path[fill=fillColor,fill opacity=0.20] ( 84.32, 79.29) circle (  2.13);

\path[fill=fillColor,fill opacity=0.20] ( 83.31, 89.02) circle (  2.13);

\path[fill=fillColor,fill opacity=0.20] ( 81.31, 90.16) circle (  2.13);

\path[fill=fillColor,fill opacity=0.20] ( 87.33, 79.92) circle (  2.13);

\path[fill=fillColor,fill opacity=0.20] ( 86.32, 77.14) circle (  2.13);

\path[fill=fillColor,fill opacity=0.20] ( 93.34, 81.31) circle (  2.13);

\path[fill=fillColor,fill opacity=0.20] ( 93.34, 83.97) circle (  2.13);

\path[fill=fillColor,fill opacity=0.20] (101.37, 76.00) circle (  2.13);

\path[fill=fillColor,fill opacity=0.20] (104.38, 60.32) circle (  2.13);

\path[fill=fillColor,fill opacity=0.20] ( 98.36, 54.76) circle (  2.13);

\path[fill=fillColor,fill opacity=0.20] ( 99.36, 55.77) circle (  2.13);

\path[fill=fillColor,fill opacity=0.20] ( 93.34, 43.88) circle (  2.13);

\path[fill=fillColor,fill opacity=0.20] ( 79.30, 59.31) circle (  2.13);

\path[fill=fillColor,fill opacity=0.20] ( 80.30, 66.52) circle (  2.13);

\path[fill=fillColor,fill opacity=0.20] ( 81.31, 67.91) circle (  2.13);

\path[fill=fillColor,fill opacity=0.20] ( 82.31, 72.33) circle (  2.13);

\path[fill=fillColor,fill opacity=0.20] ( 78.30, 76.13) circle (  2.13);

\path[fill=fillColor,fill opacity=0.20] ( 82.31, 78.40) circle (  2.13);

\path[fill=fillColor,fill opacity=0.20] ( 76.29, 76.51) circle (  2.13);

\path[fill=fillColor,fill opacity=0.20] ( 80.30, 69.93) circle (  2.13);

\path[fill=fillColor,fill opacity=0.20] ( 88.33, 73.85) circle (  2.13);

\path[fill=fillColor,fill opacity=0.20] (102.37, 78.15) circle (  2.13);

\path[fill=fillColor,fill opacity=0.20] ( 97.36, 69.42) circle (  2.13);

\path[fill=fillColor,fill opacity=0.20] (101.37, 60.45) circle (  2.13);

\path[fill=fillColor,fill opacity=0.20] (101.37, 51.09) circle (  2.13);

\path[fill=fillColor,fill opacity=0.20] (139.49, 47.17) circle (  2.13);

\path[fill=fillColor,fill opacity=0.20] ( 88.33, 78.15) circle (  2.13);

\path[fill=fillColor,fill opacity=0.20] ( 74.28, 88.01) circle (  2.13);

\path[fill=fillColor,fill opacity=0.20] ( 79.30, 80.17) circle (  2.13);

\path[fill=fillColor,fill opacity=0.20] ( 87.33, 78.02) circle (  2.13);

\path[fill=fillColor,fill opacity=0.20] ( 79.30, 88.52) circle (  2.13);

\path[fill=fillColor,fill opacity=0.20] ( 75.29,104.58) circle (  2.13);

\path[fill=fillColor,fill opacity=0.20] ( 87.33,101.80) circle (  2.13);

\path[fill=fillColor,fill opacity=0.20] ( 95.35, 87.25) circle (  2.13);

\path[fill=fillColor,fill opacity=0.20] ( 88.33, 76.76) circle (  2.13);

\path[fill=fillColor,fill opacity=0.20] ( 94.35, 70.06) circle (  2.13);

\path[fill=fillColor,fill opacity=0.20] ( 91.34, 62.72) circle (  2.13);

\path[fill=fillColor,fill opacity=0.20] (100.37, 56.91) circle (  2.13);

\path[fill=fillColor,fill opacity=0.20] (106.39, 53.62) circle (  2.13);

\path[fill=fillColor,fill opacity=0.20] ( 88.33, 47.55) circle (  2.13);

\path[fill=fillColor,fill opacity=0.20] ( 78.30, 58.42) circle (  2.13);

\path[fill=fillColor,fill opacity=0.20] ( 68.06, 67.53) circle (  2.13);

\path[fill=fillColor,fill opacity=0.20] ( 73.28, 69.55) circle (  2.13);

\path[fill=fillColor,fill opacity=0.20] ( 73.28, 76.13) circle (  2.13);

\path[fill=fillColor,fill opacity=0.20] ( 84.32, 77.52) circle (  2.13);

\path[fill=fillColor,fill opacity=0.20] ( 85.32, 72.97) circle (  2.13);

\path[fill=fillColor,fill opacity=0.20] ( 82.31, 77.77) circle (  2.13);

\path[fill=fillColor,fill opacity=0.20] ( 94.35, 79.16) circle (  2.13);

\path[fill=fillColor,fill opacity=0.20] (102.37, 68.92) circle (  2.13);

\path[fill=fillColor,fill opacity=0.20] (112.41, 66.90) circle (  2.13);

\path[fill=fillColor,fill opacity=0.20] (105.38, 59.18) circle (  2.13);

\path[fill=fillColor,fill opacity=0.20] (120.43, 43.63) circle (  2.13);

\path[fill=fillColor,fill opacity=0.20] ( 46.39, 86.12) circle (  2.13);

\path[fill=fillColor,fill opacity=0.20] ( 73.28, 85.99) circle (  2.13);

\path[fill=fillColor,fill opacity=0.20] ( 73.28, 93.96) circle (  2.13);

\path[fill=fillColor,fill opacity=0.20] ( 72.28, 98.89) circle (  2.13);

\path[fill=fillColor,fill opacity=0.20] ( 82.31, 95.22) circle (  2.13);

\path[fill=fillColor,fill opacity=0.20] ( 85.32, 94.46) circle (  2.13);

\path[fill=fillColor,fill opacity=0.20] ( 88.33, 93.45) circle (  2.13);

\path[fill=fillColor,fill opacity=0.20] ( 96.35, 84.60) circle (  2.13);

\path[fill=fillColor,fill opacity=0.20] ( 91.34, 69.80) circle (  2.13);

\path[fill=fillColor,fill opacity=0.20] (102.37, 62.09) circle (  2.13);

\path[fill=fillColor,fill opacity=0.20] (108.39, 66.14) circle (  2.13);

\path[fill=fillColor,fill opacity=0.20] (110.40, 67.65) circle (  2.13);

\path[fill=fillColor,fill opacity=0.20] (108.39, 57.16) circle (  2.13);

\path[fill=fillColor,fill opacity=0.20] ( 71.27, 51.97) circle (  2.13);

\path[fill=fillColor,fill opacity=0.20] ( 84.32, 52.10) circle (  2.13);

\path[fill=fillColor,fill opacity=0.20] ( 88.33, 55.26) circle (  2.13);

\path[fill=fillColor,fill opacity=0.20] ( 77.29, 61.84) circle (  2.13);

\path[fill=fillColor,fill opacity=0.20] ( 81.31, 68.92) circle (  2.13);

\path[fill=fillColor,fill opacity=0.20] ( 84.32, 77.01) circle (  2.13);

\path[fill=fillColor,fill opacity=0.20] ( 67.46, 94.59) circle (  2.13);

\path[fill=fillColor,fill opacity=0.20] ( 89.33, 94.21) circle (  2.13);

\path[fill=fillColor,fill opacity=0.20] ( 66.06, 68.03) circle (  2.13);

\path[fill=fillColor,fill opacity=0.20] ( 99.36, 54.76) circle (  2.13);

\path[fill=fillColor,fill opacity=0.20] (111.40, 52.35) circle (  2.13);

\path[fill=fillColor,fill opacity=0.20] ( 78.30,105.72) circle (  2.13);

\path[fill=fillColor,fill opacity=0.20] ( 75.29, 90.67) circle (  2.13);

\path[fill=fillColor,fill opacity=0.20] ( 77.29, 82.45) circle (  2.13);

\path[fill=fillColor,fill opacity=0.20] ( 81.31, 85.23) circle (  2.13);

\path[fill=fillColor,fill opacity=0.20] ( 90.33, 89.66) circle (  2.13);

\path[fill=fillColor,fill opacity=0.20] ( 89.33, 91.81) circle (  2.13);

\path[fill=fillColor,fill opacity=0.20] ( 86.32, 85.99) circle (  2.13);

\path[fill=fillColor,fill opacity=0.20] ( 96.35, 76.25) circle (  2.13);

\path[fill=fillColor,fill opacity=0.20] ( 96.35, 71.07) circle (  2.13);

\path[fill=fillColor,fill opacity=0.20] (103.38, 69.42) circle (  2.13);

\path[fill=fillColor,fill opacity=0.20] (109.40, 66.77) circle (  2.13);

\path[fill=fillColor,fill opacity=0.20] ( 96.35, 59.81) circle (  2.13);

\path[fill=fillColor,fill opacity=0.20] ( 88.33, 55.89) circle (  2.13);

\path[fill=fillColor,fill opacity=0.20] ( 85.32, 55.26) circle (  2.13);

\path[fill=fillColor,fill opacity=0.20] ( 89.33, 62.98) circle (  2.13);

\path[fill=fillColor,fill opacity=0.20] ( 91.34, 63.99) circle (  2.13);

\path[fill=fillColor,fill opacity=0.20] ( 99.36, 64.87) circle (  2.13);

\path[fill=fillColor,fill opacity=0.20] ( 95.35, 73.22) circle (  2.13);

\path[fill=fillColor,fill opacity=0.20] ( 94.35, 73.72) circle (  2.13);

\path[fill=fillColor,fill opacity=0.20] ( 97.36, 72.84) circle (  2.13);

\path[fill=fillColor,fill opacity=0.20] ( 83.31, 63.48) circle (  2.13);

\path[fill=fillColor,fill opacity=0.20] ( 98.36, 54.63) circle (  2.13);

\path[fill=fillColor,fill opacity=0.20] (116.42, 50.96) circle (  2.13);

\path[fill=fillColor,fill opacity=0.20] ( 55.92, 56.65) circle (  2.13);

\path[fill=fillColor,fill opacity=0.20] ( 90.33, 83.33) circle (  2.13);

\path[fill=fillColor,fill opacity=0.20] ( 77.29, 86.24) circle (  2.13);

\path[fill=fillColor,fill opacity=0.20] ( 78.30, 85.74) circle (  2.13);

\path[fill=fillColor,fill opacity=0.20] ( 88.33, 82.83) circle (  2.13);

\path[fill=fillColor,fill opacity=0.20] ( 88.33, 85.99) circle (  2.13);

\path[fill=fillColor,fill opacity=0.20] ( 90.33, 89.78) circle (  2.13);

\path[fill=fillColor,fill opacity=0.20] ( 97.36, 82.07) circle (  2.13);

\path[fill=fillColor,fill opacity=0.20] ( 87.33, 75.37) circle (  2.13);

\path[fill=fillColor,fill opacity=0.20] ( 88.33, 70.69) circle (  2.13);

\path[fill=fillColor,fill opacity=0.20] ( 98.36, 65.88) circle (  2.13);

\path[fill=fillColor,fill opacity=0.20] (101.37, 64.11) circle (  2.13);

\path[fill=fillColor,fill opacity=0.20] (107.39, 71.57) circle (  2.13);

\path[fill=fillColor,fill opacity=0.20] (108.39, 79.79) circle (  2.13);

\path[fill=fillColor,fill opacity=0.20] (107.39, 67.53) circle (  2.13);

\path[fill=fillColor,fill opacity=0.20] (107.39, 54.76) circle (  2.13);

\path[fill=fillColor,fill opacity=0.20] ( 98.36, 57.41) circle (  2.13);

\path[fill=fillColor,fill opacity=0.20] ( 89.33, 63.86) circle (  2.13);

\path[fill=fillColor,fill opacity=0.20] ( 87.33, 61.96) circle (  2.13);

\path[fill=fillColor,fill opacity=0.20] ( 86.32, 61.84) circle (  2.13);

\path[fill=fillColor,fill opacity=0.20] ( 88.33, 76.38) circle (  2.13);

\path[fill=fillColor,fill opacity=0.20] ( 80.30, 75.49) circle (  2.13);

\path[fill=fillColor,fill opacity=0.20] ( 83.31, 68.79) circle (  2.13);

\path[fill=fillColor,fill opacity=0.20] ( 92.34, 67.40) circle (  2.13);

\path[fill=fillColor,fill opacity=0.20] (102.37, 66.14) circle (  2.13);

\path[fill=fillColor,fill opacity=0.20] ( 74.28, 69.05) circle (  2.13);

\path[fill=fillColor,fill opacity=0.20] ( 75.29, 73.98) circle (  2.13);

\path[fill=fillColor,fill opacity=0.20] ( 73.28, 81.31) circle (  2.13);

\path[fill=fillColor,fill opacity=0.20] ( 83.31, 77.52) circle (  2.13);

\path[fill=fillColor,fill opacity=0.20] ( 85.32, 80.68) circle (  2.13);

\path[fill=fillColor,fill opacity=0.20] ( 87.33, 90.92) circle (  2.13);

\path[fill=fillColor,fill opacity=0.20] ( 89.33, 87.38) circle (  2.13);

\path[fill=fillColor,fill opacity=0.20] ( 86.32, 77.14) circle (  2.13);

\path[fill=fillColor,fill opacity=0.20] ( 83.31, 72.33) circle (  2.13);

\path[fill=fillColor,fill opacity=0.20] ( 89.33, 69.68) circle (  2.13);

\path[fill=fillColor,fill opacity=0.20] ( 97.36, 72.21) circle (  2.13);

\path[fill=fillColor,fill opacity=0.20] (100.37, 70.94) circle (  2.13);

\path[fill=fillColor,fill opacity=0.20] ( 98.36, 68.92) circle (  2.13);

\path[fill=fillColor,fill opacity=0.20] ( 98.36, 70.82) circle (  2.13);

\path[fill=fillColor,fill opacity=0.20] ( 87.33, 73.85) circle (  2.13);

\path[fill=fillColor,fill opacity=0.20] (104.38, 72.33) circle (  2.13);

\path[fill=fillColor,fill opacity=0.20] (104.38, 72.46) circle (  2.13);

\path[fill=fillColor,fill opacity=0.20] ( 97.36, 70.18) circle (  2.13);

\path[fill=fillColor,fill opacity=0.20] ( 95.35, 62.60) circle (  2.13);

\path[fill=fillColor,fill opacity=0.20] (103.38, 58.17) circle (  2.13);

\path[fill=fillColor,fill opacity=0.20] (100.37, 56.53) circle (  2.13);

\path[fill=fillColor,fill opacity=0.20] (104.38, 58.17) circle (  2.13);

\path[fill=fillColor,fill opacity=0.20] (102.37, 66.52) circle (  2.13);

\path[fill=fillColor,fill opacity=0.20] (102.37, 70.56) circle (  2.13);

\path[fill=fillColor,fill opacity=0.20] ( 97.36, 65.13) circle (  2.13);

\path[fill=fillColor,fill opacity=0.20] ( 97.36, 64.49) circle (  2.13);

\path[fill=fillColor,fill opacity=0.20] ( 93.34, 67.65) circle (  2.13);

\path[fill=fillColor,fill opacity=0.20] ( 95.35, 68.67) circle (  2.13);

\path[fill=fillColor,fill opacity=0.20] ( 98.36, 69.05) circle (  2.13);

\path[fill=fillColor,fill opacity=0.20] ( 93.34, 73.85) circle (  2.13);

\path[fill=fillColor,fill opacity=0.20] ( 91.34, 78.78) circle (  2.13);

\path[fill=fillColor,fill opacity=0.20] ( 79.30, 88.90) circle (  2.13);

\path[fill=fillColor,fill opacity=0.20] ( 79.30, 92.44) circle (  2.13);

\path[fill=fillColor,fill opacity=0.20] ( 78.30, 76.51) circle (  2.13);

\path[fill=fillColor,fill opacity=0.20] ( 86.32, 66.64) circle (  2.13);

\path[fill=fillColor,fill opacity=0.20] ( 86.32, 75.12) circle (  2.13);

\path[fill=fillColor,fill opacity=0.20] ( 86.32, 90.67) circle (  2.13);

\path[fill=fillColor,fill opacity=0.20] ( 84.32, 81.31) circle (  2.13);

\path[fill=fillColor,fill opacity=0.20] ( 92.34, 70.56) circle (  2.13);

\path[fill=fillColor,fill opacity=0.20] ( 95.35, 60.32) circle (  2.13);

\path[fill=fillColor,fill opacity=0.20] ( 57.33, 49.95) circle (  2.13);

\path[fill=fillColor,fill opacity=0.20] ( 87.33, 64.62) circle (  2.13);

\path[fill=fillColor,fill opacity=0.20] ( 77.29, 64.62) circle (  2.13);

\path[fill=fillColor,fill opacity=0.20] ( 93.34, 67.91) circle (  2.13);

\path[fill=fillColor,fill opacity=0.20] ( 92.34, 76.63) circle (  2.13);

\path[fill=fillColor,fill opacity=0.20] ( 84.32, 78.91) circle (  2.13);

\path[fill=fillColor,fill opacity=0.20] ( 76.29, 77.26) circle (  2.13);

\path[fill=fillColor,fill opacity=0.20] ( 87.33, 81.31) circle (  2.13);

\path[fill=fillColor,fill opacity=0.20] ( 90.33, 78.28) circle (  2.13);

\path[fill=fillColor,fill opacity=0.20] ( 88.33, 66.90) circle (  2.13);

\path[fill=fillColor,fill opacity=0.20] ( 94.35, 62.22) circle (  2.13);

\path[fill=fillColor,fill opacity=0.20] ( 95.35, 67.53) circle (  2.13);

\path[fill=fillColor,fill opacity=0.20] ( 95.35, 70.31) circle (  2.13);

\path[fill=fillColor,fill opacity=0.20] ( 91.34, 76.76) circle (  2.13);

\path[fill=fillColor,fill opacity=0.20] ( 93.34, 67.28) circle (  2.13);

\path[fill=fillColor,fill opacity=0.20] ( 97.36, 69.55) circle (  2.13);

\path[fill=fillColor,fill opacity=0.20] (102.37, 78.28) circle (  2.13);

\path[fill=fillColor,fill opacity=0.20] ( 98.36, 82.07) circle (  2.13);

\path[fill=fillColor,fill opacity=0.20] ( 97.36, 73.72) circle (  2.13);

\path[fill=fillColor,fill opacity=0.20] ( 95.35, 67.40) circle (  2.13);

\path[fill=fillColor,fill opacity=0.20] ( 98.36, 72.59) circle (  2.13);

\path[fill=fillColor,fill opacity=0.20] ( 93.34, 78.02) circle (  2.13);

\path[fill=fillColor,fill opacity=0.20] ( 88.33, 73.47) circle (  2.13);

\path[fill=fillColor,fill opacity=0.20] ( 97.36, 76.13) circle (  2.13);

\path[fill=fillColor,fill opacity=0.20] ( 95.35, 87.25) circle (  2.13);

\path[fill=fillColor,fill opacity=0.20] ( 95.35, 86.62) circle (  2.13);

\path[fill=fillColor,fill opacity=0.20] ( 88.33, 77.64) circle (  2.13);

\path[fill=fillColor,fill opacity=0.20] ( 83.31, 77.90) circle (  2.13);

\path[fill=fillColor,fill opacity=0.20] ( 84.32, 82.32) circle (  2.13);

\path[fill=fillColor,fill opacity=0.20] ( 88.33, 75.87) circle (  2.13);

\path[fill=fillColor,fill opacity=0.20] ( 96.35, 68.29) circle (  2.13);

\path[fill=fillColor,fill opacity=0.20] ( 92.34, 73.34) circle (  2.13);

\path[fill=fillColor,fill opacity=0.20] ( 80.30, 74.48) circle (  2.13);

\path[fill=fillColor,fill opacity=0.20] ( 66.66, 59.69) circle (  2.13);

\path[fill=fillColor,fill opacity=0.20] (104.38, 50.71) circle (  2.13);

\path[fill=fillColor,fill opacity=0.20] ( 91.34, 66.39) circle (  2.13);

\path[fill=fillColor,fill opacity=0.20] ( 85.32, 77.77) circle (  2.13);

\path[fill=fillColor,fill opacity=0.20] ( 96.35, 88.90) circle (  2.13);

\path[fill=fillColor,fill opacity=0.20] ( 92.34, 83.33) circle (  2.13);

\path[fill=fillColor,fill opacity=0.20] ( 88.33, 68.41) circle (  2.13);

\path[fill=fillColor,fill opacity=0.20] ( 90.33, 67.02) circle (  2.13);

\path[fill=fillColor,fill opacity=0.20] ( 88.33, 72.08) circle (  2.13);

\path[fill=fillColor,fill opacity=0.20] ( 89.33, 68.29) circle (  2.13);

\path[fill=fillColor,fill opacity=0.20] ( 87.33, 71.45) circle (  2.13);

\path[fill=fillColor,fill opacity=0.20] ( 83.31, 71.70) circle (  2.13);

\path[fill=fillColor,fill opacity=0.20] ( 89.33, 64.75) circle (  2.13);

\path[fill=fillColor,fill opacity=0.20] ( 90.33, 71.57) circle (  2.13);

\path[fill=fillColor,fill opacity=0.20] ( 93.34, 83.33) circle (  2.13);

\path[fill=fillColor,fill opacity=0.20] ( 90.33, 77.39) circle (  2.13);

\path[fill=fillColor,fill opacity=0.20] ( 88.33, 71.95) circle (  2.13);

\path[fill=fillColor,fill opacity=0.20] ( 88.33, 74.99) circle (  2.13);

\path[fill=fillColor,fill opacity=0.20] ( 76.29, 69.05) circle (  2.13);

\path[fill=fillColor,fill opacity=0.20] ( 92.34, 73.98) circle (  2.13);

\path[fill=fillColor,fill opacity=0.20] ( 92.34, 91.68) circle (  2.13);

\path[fill=fillColor,fill opacity=0.20] ( 92.34, 91.43) circle (  2.13);

\path[fill=fillColor,fill opacity=0.20] ( 87.33, 78.66) circle (  2.13);

\path[fill=fillColor,fill opacity=0.20] ( 86.32, 73.34) circle (  2.13);

\path[fill=fillColor,fill opacity=0.20] ( 90.33, 72.21) circle (  2.13);

\path[fill=fillColor,fill opacity=0.20] ( 94.35, 68.16) circle (  2.13);

\path[fill=fillColor,fill opacity=0.20] ( 98.36, 56.78) circle (  2.13);

\path[fill=fillColor,fill opacity=0.20] (102.37, 47.68) circle (  2.13);

\path[fill=fillColor,fill opacity=0.20] (109.40, 39.71) circle (  2.13);

\path[fill=fillColor,fill opacity=0.20] ( 97.36, 52.10) circle (  2.13);

\path[fill=fillColor,fill opacity=0.20] ( 96.35, 65.38) circle (  2.13);

\path[fill=fillColor,fill opacity=0.20] ( 98.36, 78.66) circle (  2.13);

\path[fill=fillColor,fill opacity=0.20] ( 87.33, 75.12) circle (  2.13);

\path[fill=fillColor,fill opacity=0.20] ( 89.33, 74.86) circle (  2.13);

\path[fill=fillColor,fill opacity=0.20] (105.38, 77.14) circle (  2.13);

\path[fill=fillColor,fill opacity=0.20] ( 95.35, 70.56) circle (  2.13);

\path[fill=fillColor,fill opacity=0.20] ( 87.33, 76.13) circle (  2.13);

\path[fill=fillColor,fill opacity=0.20] ( 91.34, 86.24) circle (  2.13);

\path[fill=fillColor,fill opacity=0.20] ( 95.35, 80.17) circle (  2.13);

\path[fill=fillColor,fill opacity=0.20] ( 71.27, 77.90) circle (  2.13);

\path[fill=fillColor,fill opacity=0.20] ( 83.31, 88.27) circle (  2.13);

\path[fill=fillColor,fill opacity=0.20] ( 88.33, 85.86) circle (  2.13);

\path[fill=fillColor,fill opacity=0.20] ( 90.33, 73.60) circle (  2.13);

\path[fill=fillColor,fill opacity=0.20] ( 85.32, 70.44) circle (  2.13);

\path[fill=fillColor,fill opacity=0.20] ( 89.33, 74.48) circle (  2.13);

\path[fill=fillColor,fill opacity=0.20] ( 80.30, 77.39) circle (  2.13);

\path[fill=fillColor,fill opacity=0.20] ( 78.30, 76.00) circle (  2.13);

\path[fill=fillColor,fill opacity=0.20] ( 84.32, 75.75) circle (  2.13);

\path[fill=fillColor,fill opacity=0.20] ( 90.33, 82.70) circle (  2.13);

\path[fill=fillColor,fill opacity=0.20] ( 94.35, 90.04) circle (  2.13);

\path[fill=fillColor,fill opacity=0.20] ( 97.36, 88.77) circle (  2.13);

\path[fill=fillColor,fill opacity=0.20] ( 95.35, 80.30) circle (  2.13);

\path[fill=fillColor,fill opacity=0.20] ( 95.35, 70.69) circle (  2.13);

\path[fill=fillColor,fill opacity=0.20] ( 93.34, 62.22) circle (  2.13);

\path[fill=fillColor,fill opacity=0.20] ( 98.36, 49.70) circle (  2.13);

\path[fill=fillColor,fill opacity=0.20] (105.38, 51.97) circle (  2.13);

\path[fill=fillColor,fill opacity=0.20] (108.39, 58.04) circle (  2.13);

\path[fill=fillColor,fill opacity=0.20] ( 96.35, 59.06) circle (  2.13);

\path[fill=fillColor,fill opacity=0.20] (103.38, 65.13) circle (  2.13);

\path[fill=fillColor,fill opacity=0.20] (112.41, 73.47) circle (  2.13);

\path[fill=fillColor,fill opacity=0.20] (109.40, 75.37) circle (  2.13);

\path[fill=fillColor,fill opacity=0.20] ( 51.61, 79.54) circle (  2.13);

\path[fill=fillColor,fill opacity=0.20] ( 90.33, 99.27) circle (  2.13);

\path[fill=fillColor,fill opacity=0.20] ( 95.35, 93.32) circle (  2.13);

\path[fill=fillColor,fill opacity=0.20] ( 97.36, 76.63) circle (  2.13);

\path[fill=fillColor,fill opacity=0.20] ( 97.36, 80.05) circle (  2.13);

\path[fill=fillColor,fill opacity=0.20] ( 97.36, 87.25) circle (  2.13);

\path[fill=fillColor,fill opacity=0.20] ( 90.33, 80.17) circle (  2.13);

\path[fill=fillColor,fill opacity=0.20] ( 95.35, 80.30) circle (  2.13);

\path[fill=fillColor,fill opacity=0.20] ( 94.35, 96.11) circle (  2.13);

\path[fill=fillColor,fill opacity=0.20] ( 99.36, 97.24) circle (  2.13);

\path[fill=fillColor,fill opacity=0.20] ( 99.36, 75.37) circle (  2.13);

\path[fill=fillColor,fill opacity=0.20] ( 51.31, 63.10) circle (  2.13);

\path[fill=fillColor,fill opacity=0.20] ( 89.33, 57.79) circle (  2.13);

\path[fill=fillColor,fill opacity=0.20] (123.44, 38.95) circle (  2.13);

\path[fill=fillColor,fill opacity=0.20] ( 98.36, 54.00) circle (  2.13);

\path[fill=fillColor,fill opacity=0.20] (101.37, 71.83) circle (  2.13);

\path[fill=fillColor,fill opacity=0.20] (100.37, 63.48) circle (  2.13);

\path[fill=fillColor,fill opacity=0.20] (102.37, 65.13) circle (  2.13);

\path[fill=fillColor,fill opacity=0.20] ( 99.36, 65.38) circle (  2.13);

\path[fill=fillColor,fill opacity=0.20] ( 89.33, 58.68) circle (  2.13);

\path[fill=fillColor,fill opacity=0.20] ( 74.28, 59.69) circle (  2.13);

\path[fill=fillColor,fill opacity=0.20] (109.40, 63.61) circle (  2.13);

\path[fill=fillColor,fill opacity=0.20] (121.43, 57.92) circle (  2.13);

\path[fill=fillColor,fill opacity=0.20] (117.42, 48.81) circle (  2.13);

\path[fill=fillColor,fill opacity=0.20] (109.40, 41.73) circle (  2.13);

\path[fill=fillColor,fill opacity=0.20] (120.43, 40.72) circle (  2.13);

\path[fill=fillColor,fill opacity=0.20] ( 69.27, 60.83) circle (  2.13);

\path[fill=fillColor,fill opacity=0.20] ( 70.27, 65.50) circle (  2.13);

\path[fill=fillColor,fill opacity=0.20] ( 84.32, 66.14) circle (  2.13);

\path[fill=fillColor,fill opacity=0.20] ( 85.32, 61.33) circle (  2.13);

\path[fill=fillColor,fill opacity=0.20] ( 79.30, 47.80) circle (  2.13);

\path[fill=fillColor,fill opacity=0.20] ( 71.27, 62.85) circle (  2.13);

\path[fill=fillColor,fill opacity=0.20] ( 73.28, 72.08) circle (  2.13);

\path[fill=fillColor,fill opacity=0.20] ( 78.30, 81.31) circle (  2.13);

\path[fill=fillColor,fill opacity=0.20] ( 79.30, 87.00) circle (  2.13);

\path[fill=fillColor,fill opacity=0.20] ( 78.30, 81.82) circle (  2.13);

\path[fill=fillColor,fill opacity=0.20] ( 89.33, 77.77) circle (  2.13);

\path[fill=fillColor,fill opacity=0.20] ( 90.33, 71.70) circle (  2.13);

\path[fill=fillColor,fill opacity=0.20] ( 84.32, 56.27) circle (  2.13);

\path[fill=fillColor,fill opacity=0.20] (106.39, 45.53) circle (  2.13);

\path[fill=fillColor,fill opacity=0.20] (110.40, 40.97) circle (  2.13);

\path[fill=fillColor,fill opacity=0.20] ( 83.31, 49.19) circle (  2.13);

\path[fill=fillColor,fill opacity=0.20] ( 84.32, 72.21) circle (  2.13);

\path[fill=fillColor,fill opacity=0.20] ( 80.30, 83.97) circle (  2.13);

\path[fill=fillColor,fill opacity=0.20] ( 77.29, 88.14) circle (  2.13);

\path[fill=fillColor,fill opacity=0.20] ( 81.31, 95.73) circle (  2.13);

\path[fill=fillColor,fill opacity=0.20] ( 83.31, 97.88) circle (  2.13);

\path[fill=fillColor,fill opacity=0.20] ( 86.32, 95.35) circle (  2.13);

\path[fill=fillColor,fill opacity=0.20] ( 92.34, 89.40) circle (  2.13);

\path[fill=fillColor,fill opacity=0.20] ( 90.33, 82.20) circle (  2.13);

\path[fill=fillColor,fill opacity=0.20] ( 96.35, 76.13) circle (  2.13);

\path[fill=fillColor,fill opacity=0.20] ( 90.33, 68.41) circle (  2.13);

\path[fill=fillColor,fill opacity=0.20] ( 88.33, 49.07) circle (  2.13);

\path[fill=fillColor,fill opacity=0.20] ( 95.35, 46.16) circle (  2.13);

\path[fill=fillColor,fill opacity=0.20] ( 85.32, 74.61) circle (  2.13);

\path[fill=fillColor,fill opacity=0.20] ( 78.30, 92.69) circle (  2.13);

\path[fill=fillColor,fill opacity=0.20] ( 75.29, 99.01) circle (  2.13);

\path[fill=fillColor,fill opacity=0.20] ( 73.28, 98.26) circle (  2.13);

\path[fill=fillColor,fill opacity=0.20] ( 81.31,100.15) circle (  2.13);

\path[fill=fillColor,fill opacity=0.20] ( 83.31, 98.38) circle (  2.13);

\path[fill=fillColor,fill opacity=0.20] ( 88.33, 98.00) circle (  2.13);

\path[fill=fillColor,fill opacity=0.20] ( 94.35, 95.09) circle (  2.13);

\path[fill=fillColor,fill opacity=0.20] ( 95.35, 86.62) circle (  2.13);

\path[fill=fillColor,fill opacity=0.20] ( 97.36, 74.36) circle (  2.13);

\path[fill=fillColor,fill opacity=0.20] ( 89.33, 57.79) circle (  2.13);

\path[fill=fillColor,fill opacity=0.20] (105.38, 46.41) circle (  2.13);

\path[fill=fillColor,fill opacity=0.20] (100.37, 39.96) circle (  2.13);

\path[fill=fillColor,fill opacity=0.20] ( 79.30, 86.87) circle (  2.13);

\path[fill=fillColor,fill opacity=0.20] ( 82.31, 92.06) circle (  2.13);

\path[fill=fillColor,fill opacity=0.20] ( 85.32, 99.65) circle (  2.13);

\path[fill=fillColor,fill opacity=0.20] ( 82.31, 97.37) circle (  2.13);

\path[fill=fillColor,fill opacity=0.20] ( 84.32, 94.34) circle (  2.13);

\path[fill=fillColor,fill opacity=0.20] ( 89.33, 94.34) circle (  2.13);

\path[fill=fillColor,fill opacity=0.20] ( 96.35, 92.44) circle (  2.13);

\path[fill=fillColor,fill opacity=0.20] ( 95.35, 90.42) circle (  2.13);

\path[fill=fillColor,fill opacity=0.20] ( 99.36, 90.67) circle (  2.13);

\path[fill=fillColor,fill opacity=0.20] (106.39, 80.43) circle (  2.13);

\path[fill=fillColor,fill opacity=0.20] (102.37, 61.84) circle (  2.13);

\path[fill=fillColor,fill opacity=0.20] ( 94.35, 50.84) circle (  2.13);

\path[fill=fillColor,fill opacity=0.20] ( 92.34, 71.95) circle (  2.13);

\path[fill=fillColor,fill opacity=0.20] ( 81.31, 93.96) circle (  2.13);

\path[fill=fillColor,fill opacity=0.20] ( 86.32, 93.07) circle (  2.13);

\path[fill=fillColor,fill opacity=0.20] ( 85.32,102.18) circle (  2.13);

\path[fill=fillColor,fill opacity=0.20] ( 92.34, 97.88) circle (  2.13);

\path[fill=fillColor,fill opacity=0.20] ( 94.35, 86.75) circle (  2.13);

\path[fill=fillColor,fill opacity=0.20] ( 92.34, 86.50) circle (  2.13);

\path[fill=fillColor,fill opacity=0.20] ( 96.35, 87.63) circle (  2.13);

\path[fill=fillColor,fill opacity=0.20] ( 98.36, 83.33) circle (  2.13);

\path[fill=fillColor,fill opacity=0.20] (101.37, 85.61) circle (  2.13);

\path[fill=fillColor,fill opacity=0.20] (121.43, 80.81) circle (  2.13);

\path[fill=fillColor,fill opacity=0.20] ( 82.31, 54.88) circle (  2.13);

\path[fill=fillColor,fill opacity=0.20] ( 85.32, 81.56) circle (  2.13);

\path[fill=fillColor,fill opacity=0.20] ( 82.31,100.91) circle (  2.13);

\path[fill=fillColor,fill opacity=0.20] ( 83.31,100.03) circle (  2.13);

\path[fill=fillColor,fill opacity=0.20] ( 75.29,103.19) circle (  2.13);

\path[fill=fillColor,fill opacity=0.20] ( 93.34, 97.75) circle (  2.13);

\path[fill=fillColor,fill opacity=0.20] ( 98.36, 88.14) circle (  2.13);

\path[fill=fillColor,fill opacity=0.20] ( 99.36, 82.45) circle (  2.13);

\path[fill=fillColor,fill opacity=0.20] (104.38, 84.47) circle (  2.13);

\path[fill=fillColor,fill opacity=0.20] (104.38, 77.52) circle (  2.13);

\path[fill=fillColor,fill opacity=0.20] (119.43, 65.38) circle (  2.13);

\path[fill=fillColor,fill opacity=0.20] ( 92.34, 81.94) circle (  2.13);

\path[fill=fillColor,fill opacity=0.20] ( 88.33, 83.21) circle (  2.13);

\path[fill=fillColor,fill opacity=0.20] ( 89.33, 79.54) circle (  2.13);

\path[fill=fillColor,fill opacity=0.20] ( 96.35, 71.07) circle (  2.13);

\path[fill=fillColor,fill opacity=0.20] ( 92.34, 75.12) circle (  2.13);

\path[fill=fillColor,fill opacity=0.20] ( 86.32, 92.31) circle (  2.13);

\path[fill=fillColor,fill opacity=0.20] ( 83.31, 94.34) circle (  2.13);

\path[fill=fillColor,fill opacity=0.20] ( 85.32, 97.50) circle (  2.13);

\path[fill=fillColor,fill opacity=0.20] ( 93.34, 95.09) circle (  2.13);

\path[fill=fillColor,fill opacity=0.20] ( 96.35, 91.55) circle (  2.13);

\path[fill=fillColor,fill opacity=0.20] (101.37, 84.85) circle (  2.13);

\path[fill=fillColor,fill opacity=0.20] (110.40, 77.64) circle (  2.13);

\path[fill=fillColor,fill opacity=0.20] (115.42, 67.28) circle (  2.13);

\path[fill=fillColor,fill opacity=0.20] (135.48, 52.10) circle (  2.13);

\path[fill=fillColor,fill opacity=0.20] ( 90.33, 77.52) circle (  2.13);

\path[fill=fillColor,fill opacity=0.20] ( 99.36, 85.74) circle (  2.13);

\path[fill=fillColor,fill opacity=0.20] ( 89.33, 75.24) circle (  2.13);

\path[fill=fillColor,fill opacity=0.20] ( 83.31, 74.86) circle (  2.13);

\path[fill=fillColor,fill opacity=0.20] ( 91.34, 75.24) circle (  2.13);

\path[fill=fillColor,fill opacity=0.20] ( 97.36, 75.62) circle (  2.13);

\path[fill=fillColor,fill opacity=0.20] ( 98.36, 81.82) circle (  2.13);

\path[fill=fillColor,fill opacity=0.20] ( 91.34, 48.81) circle (  2.13);

\path[fill=fillColor,fill opacity=0.20] (102.37, 69.05) circle (  2.13);

\path[fill=fillColor,fill opacity=0.20] ( 75.29, 87.89) circle (  2.13);

\path[fill=fillColor,fill opacity=0.20] ( 83.31, 92.19) circle (  2.13);

\path[fill=fillColor,fill opacity=0.20] ( 91.34, 99.52) circle (  2.13);

\path[fill=fillColor,fill opacity=0.20] ( 93.34, 95.73) circle (  2.13);

\path[fill=fillColor,fill opacity=0.20] ( 88.33, 91.55) circle (  2.13);

\path[fill=fillColor,fill opacity=0.20] (104.38, 87.13) circle (  2.13);

\path[fill=fillColor,fill opacity=0.20] (111.40, 58.04) circle (  2.13);

\path[fill=fillColor,fill opacity=0.20] ( 87.33, 87.13) circle (  2.13);

\path[fill=fillColor,fill opacity=0.20] ( 89.33, 85.61) circle (  2.13);

\path[fill=fillColor,fill opacity=0.20] ( 87.33, 81.94) circle (  2.13);

\path[fill=fillColor,fill opacity=0.20] ( 89.33, 84.60) circle (  2.13);

\path[fill=fillColor,fill opacity=0.20] ( 84.32, 82.20) circle (  2.13);

\path[fill=fillColor,fill opacity=0.20] ( 89.33, 76.00) circle (  2.13);

\path[fill=fillColor,fill opacity=0.20] (101.37, 73.72) circle (  2.13);

\path[fill=fillColor,fill opacity=0.20] ( 94.35, 60.95) circle (  2.13);

\path[fill=fillColor,fill opacity=0.20] (102.37, 67.15) circle (  2.13);

\path[fill=fillColor,fill opacity=0.20] ( 75.29,101.16) circle (  2.13);

\path[fill=fillColor,fill opacity=0.20] ( 69.27, 98.76) circle (  2.13);

\path[fill=fillColor,fill opacity=0.20] ( 92.34,104.70) circle (  2.13);

\path[fill=fillColor,fill opacity=0.20] ( 87.33, 97.50) circle (  2.13);

\path[fill=fillColor,fill opacity=0.20] ( 88.33, 87.89) circle (  2.13);

\path[fill=fillColor,fill opacity=0.20] (104.38, 88.65) circle (  2.13);

\path[fill=fillColor,fill opacity=0.20] ( 99.36, 74.36) circle (  2.13);

\path[fill=fillColor,fill opacity=0.20] (106.39, 56.91) circle (  2.13);

\path[fill=fillColor,fill opacity=0.20] ( 84.32, 78.28) circle (  2.13);

\path[fill=fillColor,fill opacity=0.20] ( 78.30, 94.08) circle (  2.13);

\path[fill=fillColor,fill opacity=0.20] ( 81.31, 92.69) circle (  2.13);

\path[fill=fillColor,fill opacity=0.20] ( 88.33, 95.60) circle (  2.13);

\path[fill=fillColor,fill opacity=0.20] ( 97.36, 95.47) circle (  2.13);

\path[fill=fillColor,fill opacity=0.20] ( 92.34, 97.24) circle (  2.13);

\path[fill=fillColor,fill opacity=0.20] ( 91.34, 88.52) circle (  2.13);

\path[fill=fillColor,fill opacity=0.20] ( 97.36, 76.38) circle (  2.13);

\path[fill=fillColor,fill opacity=0.20] ( 92.34, 68.03) circle (  2.13);

\path[fill=fillColor,fill opacity=0.20] ( 99.36, 51.97) circle (  2.13);

\path[fill=fillColor,fill opacity=0.20] ( 82.31,101.29) circle (  2.13);

\path[fill=fillColor,fill opacity=0.20] ( 83.31,100.03) circle (  2.13);

\path[fill=fillColor,fill opacity=0.20] ( 92.34, 96.61) circle (  2.13);

\path[fill=fillColor,fill opacity=0.20] ( 94.35, 92.82) circle (  2.13);

\path[fill=fillColor,fill opacity=0.20] ( 97.36, 86.50) circle (  2.13);

\path[fill=fillColor,fill opacity=0.20] (104.38, 88.39) circle (  2.13);

\path[fill=fillColor,fill opacity=0.20] (103.38, 84.22) circle (  2.13);

\path[fill=fillColor,fill opacity=0.20] (104.38, 68.79) circle (  2.13);

\path[fill=fillColor,fill opacity=0.20] ( 83.31, 92.44) circle (  2.13);

\path[fill=fillColor,fill opacity=0.20] ( 75.29,101.16) circle (  2.13);

\path[fill=fillColor,fill opacity=0.20] ( 82.31, 90.42) circle (  2.13);

\path[fill=fillColor,fill opacity=0.20] ( 86.32, 94.21) circle (  2.13);

\path[fill=fillColor,fill opacity=0.20] ( 86.32, 98.13) circle (  2.13);

\path[fill=fillColor,fill opacity=0.20] ( 92.34,105.46) circle (  2.13);

\path[fill=fillColor,fill opacity=0.20] ( 89.33, 97.12) circle (  2.13);

\path[fill=fillColor,fill opacity=0.20] ( 96.35, 83.84) circle (  2.13);

\path[fill=fillColor,fill opacity=0.20] (101.37, 79.16) circle (  2.13);

\path[fill=fillColor,fill opacity=0.20] ( 83.31, 50.96) circle (  2.13);

\path[fill=fillColor,fill opacity=0.20] (115.42, 65.76) circle (  2.13);

\path[fill=fillColor,fill opacity=0.20] ( 94.35, 89.91) circle (  2.13);

\path[fill=fillColor,fill opacity=0.20] ( 85.32, 87.38) circle (  2.13);

\path[fill=fillColor,fill opacity=0.20] ( 86.32, 86.62) circle (  2.13);

\path[fill=fillColor,fill opacity=0.20] ( 93.34, 89.53) circle (  2.13);

\path[fill=fillColor,fill opacity=0.20] ( 97.36, 89.02) circle (  2.13);

\path[fill=fillColor,fill opacity=0.20] (106.39, 85.10) circle (  2.13);

\path[fill=fillColor,fill opacity=0.20] (108.39, 73.34) circle (  2.13);

\path[fill=fillColor,fill opacity=0.20] ( 77.29, 94.34) circle (  2.13);

\path[fill=fillColor,fill opacity=0.20] ( 78.30, 93.58) circle (  2.13);

\path[fill=fillColor,fill opacity=0.20] ( 82.31, 85.48) circle (  2.13);

\path[fill=fillColor,fill opacity=0.20] ( 83.31, 94.34) circle (  2.13);

\path[fill=fillColor,fill opacity=0.20] ( 82.31,101.67) circle (  2.13);

\path[fill=fillColor,fill opacity=0.20] ( 88.33,104.58) circle (  2.13);

\path[fill=fillColor,fill opacity=0.20] ( 83.31, 94.08) circle (  2.13);

\path[fill=fillColor,fill opacity=0.20] ( 91.34, 83.21) circle (  2.13);

\path[fill=fillColor,fill opacity=0.20] ( 99.36, 84.98) circle (  2.13);

\path[fill=fillColor,fill opacity=0.20] (101.37, 65.13) circle (  2.13);

\path[fill=fillColor,fill opacity=0.20] ( 95.35, 77.26) circle (  2.13);

\path[fill=fillColor,fill opacity=0.20] ( 84.32, 91.81) circle (  2.13);

\path[fill=fillColor,fill opacity=0.20] ( 84.32, 91.68) circle (  2.13);

\path[fill=fillColor,fill opacity=0.20] ( 93.34, 92.44) circle (  2.13);

\path[fill=fillColor,fill opacity=0.20] ( 95.35, 85.61) circle (  2.13);

\path[fill=fillColor,fill opacity=0.20] ( 97.36, 72.97) circle (  2.13);

\path[fill=fillColor,fill opacity=0.20] (106.39, 63.61) circle (  2.13);

\path[fill=fillColor,fill opacity=0.20] (110.40, 58.30) circle (  2.13);

\path[fill=fillColor,fill opacity=0.20] ( 82.31, 94.08) circle (  2.13);

\path[fill=fillColor,fill opacity=0.20] ( 86.32, 91.05) circle (  2.13);

\path[fill=fillColor,fill opacity=0.20] ( 84.32, 91.81) circle (  2.13);

\path[fill=fillColor,fill opacity=0.20] ( 87.33, 99.77) circle (  2.13);

\path[fill=fillColor,fill opacity=0.20] ( 87.33, 97.50) circle (  2.13);

\path[fill=fillColor,fill opacity=0.20] ( 90.33, 94.97) circle (  2.13);

\path[fill=fillColor,fill opacity=0.20] ( 83.31, 91.17) circle (  2.13);

\path[fill=fillColor,fill opacity=0.20] ( 83.31, 84.85) circle (  2.13);

\path[fill=fillColor,fill opacity=0.20] ( 96.35, 84.98) circle (  2.13);

\path[fill=fillColor,fill opacity=0.20] ( 90.33, 81.31) circle (  2.13);

\path[fill=fillColor,fill opacity=0.20] (147.52, 52.23) circle (  2.13);

\path[fill=fillColor,fill opacity=0.20] (108.39, 80.17) circle (  2.13);

\path[fill=fillColor,fill opacity=0.20] ( 96.35, 97.88) circle (  2.13);

\path[fill=fillColor,fill opacity=0.20] ( 95.35, 85.99) circle (  2.13);

\path[fill=fillColor,fill opacity=0.20] ( 94.35, 76.89) circle (  2.13);

\path[fill=fillColor,fill opacity=0.20] ( 89.33, 70.94) circle (  2.13);

\path[fill=fillColor,fill opacity=0.20] ( 96.35, 64.24) circle (  2.13);

\path[fill=fillColor,fill opacity=0.20] (104.38, 60.95) circle (  2.13);

\path[fill=fillColor,fill opacity=0.20] (111.40, 55.14) circle (  2.13);

\path[fill=fillColor,fill opacity=0.20] ( 91.34, 78.40) circle (  2.13);

\path[fill=fillColor,fill opacity=0.20] ( 91.34, 97.62) circle (  2.13);

\path[fill=fillColor,fill opacity=0.20] ( 91.34,104.45) circle (  2.13);

\path[fill=fillColor,fill opacity=0.20] ( 87.33,103.06) circle (  2.13);

\path[fill=fillColor,fill opacity=0.20] ( 86.32, 95.98) circle (  2.13);

\path[fill=fillColor,fill opacity=0.20] ( 77.29, 90.79) circle (  2.13);

\path[fill=fillColor,fill opacity=0.20] ( 79.30, 90.29) circle (  2.13);

\path[fill=fillColor,fill opacity=0.20] ( 81.31, 92.19) circle (  2.13);

\path[fill=fillColor,fill opacity=0.20] ( 79.30, 87.63) circle (  2.13);

\path[fill=fillColor,fill opacity=0.20] ( 91.34, 81.18) circle (  2.13);

\path[fill=fillColor,fill opacity=0.20] ( 95.35, 73.34) circle (  2.13);

\path[fill=fillColor,fill opacity=0.20] ( 86.32, 48.05) circle (  2.13);

\path[fill=fillColor,fill opacity=0.20] (143.51, 80.81) circle (  2.13);

\path[fill=fillColor,fill opacity=0.20] (109.40, 81.56) circle (  2.13);

\path[fill=fillColor,fill opacity=0.20] ( 94.35, 73.98) circle (  2.13);

\path[fill=fillColor,fill opacity=0.20] ( 86.32, 77.64) circle (  2.13);

\path[fill=fillColor,fill opacity=0.20] ( 95.35, 75.87) circle (  2.13);

\path[fill=fillColor,fill opacity=0.20] (100.37, 63.73) circle (  2.13);

\path[fill=fillColor,fill opacity=0.20] (101.37, 56.27) circle (  2.13);

\path[fill=fillColor,fill opacity=0.20] (101.37, 48.56) circle (  2.13);

\path[fill=fillColor,fill opacity=0.20] ( 89.33, 83.33) circle (  2.13);

\path[fill=fillColor,fill opacity=0.20] ( 87.33, 88.01) circle (  2.13);

\path[fill=fillColor,fill opacity=0.20] ( 83.31,101.67) circle (  2.13);

\path[fill=fillColor,fill opacity=0.20] ( 84.32,111.91) circle (  2.13);

\path[fill=fillColor,fill opacity=0.20] ( 83.31,101.80) circle (  2.13);

\path[fill=fillColor,fill opacity=0.20] ( 80.30, 93.07) circle (  2.13);

\path[fill=fillColor,fill opacity=0.20] ( 75.29, 93.20) circle (  2.13);

\path[fill=fillColor,fill opacity=0.20] ( 76.29, 96.49) circle (  2.13);

\path[fill=fillColor,fill opacity=0.20] ( 80.30, 94.97) circle (  2.13);

\path[fill=fillColor,fill opacity=0.20] ( 78.30, 86.62) circle (  2.13);

\path[fill=fillColor,fill opacity=0.20] ( 84.32, 79.41) circle (  2.13);

\path[fill=fillColor,fill opacity=0.20] ( 97.36, 67.53) circle (  2.13);

\path[fill=fillColor,fill opacity=0.20] (129.46, 44.64) circle (  2.13);

\path[fill=fillColor,fill opacity=0.20] (136.48, 73.98) circle (  2.13);

\path[fill=fillColor,fill opacity=0.20] (100.37, 78.53) circle (  2.13);

\path[fill=fillColor,fill opacity=0.20] ( 88.33, 78.15) circle (  2.13);

\path[fill=fillColor,fill opacity=0.20] ( 95.35, 77.52) circle (  2.13);

\path[fill=fillColor,fill opacity=0.20] ( 97.36, 69.17) circle (  2.13);

\path[fill=fillColor,fill opacity=0.20] ( 97.36, 59.44) circle (  2.13);

\path[fill=fillColor,fill opacity=0.20] (105.38, 59.81) circle (  2.13);

\path[fill=fillColor,fill opacity=0.20] ( 84.32, 74.99) circle (  2.13);

\path[fill=fillColor,fill opacity=0.20] ( 80.30, 87.13) circle (  2.13);

\path[fill=fillColor,fill opacity=0.20] ( 79.30, 90.42) circle (  2.13);

\path[fill=fillColor,fill opacity=0.20] ( 75.29, 97.75) circle (  2.13);

\path[fill=fillColor,fill opacity=0.20] ( 76.29, 93.58) circle (  2.13);

\path[fill=fillColor,fill opacity=0.20] ( 71.27, 91.93) circle (  2.13);

\path[fill=fillColor,fill opacity=0.20] ( 75.29, 96.61) circle (  2.13);

\path[fill=fillColor,fill opacity=0.20] ( 77.29,100.40) circle (  2.13);

\path[fill=fillColor,fill opacity=0.20] ( 75.29,103.69) circle (  2.13);

\path[fill=fillColor,fill opacity=0.20] ( 77.29,100.91) circle (  2.13);

\path[fill=fillColor,fill opacity=0.20] ( 73.28, 87.89) circle (  2.13);

\path[fill=fillColor,fill opacity=0.20] ( 84.32, 84.09) circle (  2.13);

\path[fill=fillColor,fill opacity=0.20] ( 94.35, 74.86) circle (  2.13);

\path[fill=fillColor,fill opacity=0.20] (130.46, 44.26) circle (  2.13);

\path[fill=fillColor,fill opacity=0.20] ( 82.31, 68.16) circle (  2.13);

\path[fill=fillColor,fill opacity=0.20] (100.37, 73.85) circle (  2.13);

\path[fill=fillColor,fill opacity=0.20] ( 93.34, 72.08) circle (  2.13);

\path[fill=fillColor,fill opacity=0.20] ( 98.36, 72.97) circle (  2.13);

\path[fill=fillColor,fill opacity=0.20] ( 99.36, 70.06) circle (  2.13);

\path[fill=fillColor,fill opacity=0.20] (101.37, 68.67) circle (  2.13);

\path[fill=fillColor,fill opacity=0.20] (104.38, 64.11) circle (  2.13);

\path[fill=fillColor,fill opacity=0.20] ( 87.33, 74.48) circle (  2.13);

\path[fill=fillColor,fill opacity=0.20] ( 86.32, 81.82) circle (  2.13);

\path[fill=fillColor,fill opacity=0.20] ( 82.31, 87.89) circle (  2.13);

\path[fill=fillColor,fill opacity=0.20] ( 80.30, 99.27) circle (  2.13);

\path[fill=fillColor,fill opacity=0.20] ( 78.30, 93.32) circle (  2.13);

\path[fill=fillColor,fill opacity=0.20] ( 75.29, 82.20) circle (  2.13);

\path[fill=fillColor,fill opacity=0.20] ( 69.27, 86.24) circle (  2.13);

\path[fill=fillColor,fill opacity=0.20] ( 74.28, 99.01) circle (  2.13);

\path[fill=fillColor,fill opacity=0.20] ( 83.31,106.22) circle (  2.13);

\path[fill=fillColor,fill opacity=0.20] ( 78.30,109.51) circle (  2.13);

\path[fill=fillColor,fill opacity=0.20] ( 77.29, 99.65) circle (  2.13);

\path[fill=fillColor,fill opacity=0.20] ( 74.28, 86.62) circle (  2.13);

\path[fill=fillColor,fill opacity=0.20] ( 89.33, 85.36) circle (  2.13);

\path[fill=fillColor,fill opacity=0.20] (102.37, 71.95) circle (  2.13);

\path[fill=fillColor,fill opacity=0.20] (150.53, 41.73) circle (  2.13);

\path[fill=fillColor,fill opacity=0.20] (108.39, 61.84) circle (  2.13);

\path[fill=fillColor,fill opacity=0.20] (107.39, 70.56) circle (  2.13);

\path[fill=fillColor,fill opacity=0.20] ( 89.33, 69.42) circle (  2.13);

\path[fill=fillColor,fill opacity=0.20] ( 94.35, 74.23) circle (  2.13);

\path[fill=fillColor,fill opacity=0.20] ( 93.34, 73.72) circle (  2.13);

\path[fill=fillColor,fill opacity=0.20] ( 93.34, 66.64) circle (  2.13);

\path[fill=fillColor,fill opacity=0.20] (100.37, 60.83) circle (  2.13);

\path[fill=fillColor,fill opacity=0.20] (108.39, 52.86) circle (  2.13);

\path[fill=fillColor,fill opacity=0.20] ( 96.35, 70.82) circle (  2.13);

\path[fill=fillColor,fill opacity=0.20] ( 91.34, 83.33) circle (  2.13);

\path[fill=fillColor,fill opacity=0.20] ( 80.30, 88.90) circle (  2.13);

\path[fill=fillColor,fill opacity=0.20] ( 82.31, 96.61) circle (  2.13);

\path[fill=fillColor,fill opacity=0.20] ( 78.30,100.15) circle (  2.13);

\path[fill=fillColor,fill opacity=0.20] ( 79.30, 93.83) circle (  2.13);

\path[fill=fillColor,fill opacity=0.20] ( 75.29, 88.27) circle (  2.13);

\path[fill=fillColor,fill opacity=0.20] ( 71.27, 90.16) circle (  2.13);

\path[fill=fillColor,fill opacity=0.20] ( 74.28, 98.89) circle (  2.13);

\path[fill=fillColor,fill opacity=0.20] ( 81.31,109.51) circle (  2.13);

\path[fill=fillColor,fill opacity=0.20] ( 78.30,103.19) circle (  2.13);

\path[fill=fillColor,fill opacity=0.20] ( 71.27, 88.90) circle (  2.13);

\path[fill=fillColor,fill opacity=0.20] ( 74.28, 83.59) circle (  2.13);

\path[fill=fillColor,fill opacity=0.20] ( 88.33, 77.64) circle (  2.13);

\path[fill=fillColor,fill opacity=0.20] (101.37, 56.78) circle (  2.13);

\path[fill=fillColor,fill opacity=0.20] ( 69.27, 66.14) circle (  2.13);

\path[fill=fillColor,fill opacity=0.20] ( 90.33, 75.87) circle (  2.13);

\path[fill=fillColor,fill opacity=0.20] ( 92.34, 78.53) circle (  2.13);

\path[fill=fillColor,fill opacity=0.20] ( 89.33, 70.69) circle (  2.13);

\path[fill=fillColor,fill opacity=0.20] (101.37, 67.78) circle (  2.13);

\path[fill=fillColor,fill opacity=0.20] (104.38, 65.76) circle (  2.13);

\path[fill=fillColor,fill opacity=0.20] (107.39, 48.05) circle (  2.13);

\path[fill=fillColor,fill opacity=0.20] ( 97.36, 68.16) circle (  2.13);

\path[fill=fillColor,fill opacity=0.20] ( 95.35, 73.85) circle (  2.13);

\path[fill=fillColor,fill opacity=0.20] ( 78.30, 85.36) circle (  2.13);

\path[fill=fillColor,fill opacity=0.20] ( 79.30, 92.69) circle (  2.13);

\path[fill=fillColor,fill opacity=0.20] ( 79.30, 93.83) circle (  2.13);

\path[fill=fillColor,fill opacity=0.20] ( 76.29, 97.75) circle (  2.13);

\path[fill=fillColor,fill opacity=0.20] ( 79.30, 99.39) circle (  2.13);

\path[fill=fillColor,fill opacity=0.20] ( 75.29, 98.51) circle (  2.13);

\path[fill=fillColor,fill opacity=0.20] ( 76.29, 97.24) circle (  2.13);

\path[fill=fillColor,fill opacity=0.20] ( 78.30,101.67) circle (  2.13);

\path[fill=fillColor,fill opacity=0.20] ( 76.29,103.44) circle (  2.13);

\path[fill=fillColor,fill opacity=0.20] ( 58.23, 92.06) circle (  2.13);

\path[fill=fillColor,fill opacity=0.20] ( 70.27, 86.12) circle (  2.13);

\path[fill=fillColor,fill opacity=0.20] ( 82.31, 82.70) circle (  2.13);

\path[fill=fillColor,fill opacity=0.20] (106.39, 64.24) circle (  2.13);

\path[fill=fillColor,fill opacity=0.20] (104.38, 57.79) circle (  2.13);

\path[fill=fillColor,fill opacity=0.20] (108.39, 80.81) circle (  2.13);

\path[fill=fillColor,fill opacity=0.20] ( 99.36, 80.93) circle (  2.13);

\path[fill=fillColor,fill opacity=0.20] ( 91.34, 73.09) circle (  2.13);

\path[fill=fillColor,fill opacity=0.20] ( 97.36, 79.16) circle (  2.13);

\path[fill=fillColor,fill opacity=0.20] (100.37, 74.23) circle (  2.13);

\path[fill=fillColor,fill opacity=0.20] (102.37, 60.57) circle (  2.13);

\path[fill=fillColor,fill opacity=0.20] (101.37, 52.73) circle (  2.13);

\path[fill=fillColor,fill opacity=0.20] ( 96.35, 70.82) circle (  2.13);

\path[fill=fillColor,fill opacity=0.20] ( 90.33, 79.03) circle (  2.13);

\path[fill=fillColor,fill opacity=0.20] ( 84.32, 78.66) circle (  2.13);

\path[fill=fillColor,fill opacity=0.20] ( 77.29, 85.99) circle (  2.13);

\path[fill=fillColor,fill opacity=0.20] ( 76.29, 91.93) circle (  2.13);

\path[fill=fillColor,fill opacity=0.20] ( 76.29, 91.68) circle (  2.13);

\path[fill=fillColor,fill opacity=0.20] ( 75.29,100.91) circle (  2.13);

\path[fill=fillColor,fill opacity=0.20] ( 75.29,106.35) circle (  2.13);

\path[fill=fillColor,fill opacity=0.20] ( 75.29, 97.12) circle (  2.13);

\path[fill=fillColor,fill opacity=0.20] ( 51.61, 96.11) circle (  2.13);

\path[fill=fillColor,fill opacity=0.20] ( 77.29,101.67) circle (  2.13);

\path[fill=fillColor,fill opacity=0.20] ( 76.29, 97.75) circle (  2.13);

\path[fill=fillColor,fill opacity=0.20] ( 81.31, 91.68) circle (  2.13);

\path[fill=fillColor,fill opacity=0.20] ( 81.31, 88.27) circle (  2.13);

\path[fill=fillColor,fill opacity=0.20] ( 76.29, 70.82) circle (  2.13);

\path[fill=fillColor,fill opacity=0.20] (129.46, 56.65) circle (  2.13);

\path[fill=fillColor,fill opacity=0.20] (107.39, 72.59) circle (  2.13);

\path[fill=fillColor,fill opacity=0.20] ( 97.36, 79.92) circle (  2.13);

\path[fill=fillColor,fill opacity=0.20] ( 93.34, 84.85) circle (  2.13);

\path[fill=fillColor,fill opacity=0.20] ( 90.33, 85.48) circle (  2.13);

\path[fill=fillColor,fill opacity=0.20] ( 97.36, 81.82) circle (  2.13);

\path[fill=fillColor,fill opacity=0.20] (100.37, 72.59) circle (  2.13);

\path[fill=fillColor,fill opacity=0.20] (103.38, 61.84) circle (  2.13);

\path[fill=fillColor,fill opacity=0.20] ( 87.33, 81.82) circle (  2.13);

\path[fill=fillColor,fill opacity=0.20] ( 82.31, 91.17) circle (  2.13);

\path[fill=fillColor,fill opacity=0.20] ( 79.30, 85.86) circle (  2.13);

\path[fill=fillColor,fill opacity=0.20] ( 79.30, 87.63) circle (  2.13);

\path[fill=fillColor,fill opacity=0.20] ( 74.28, 94.21) circle (  2.13);

\path[fill=fillColor,fill opacity=0.20] ( 73.28, 95.98) circle (  2.13);

\path[fill=fillColor,fill opacity=0.20] ( 66.96,104.83) circle (  2.13);

\path[fill=fillColor,fill opacity=0.20] ( 71.27,109.13) circle (  2.13);

\path[fill=fillColor,fill opacity=0.20] ( 72.28, 95.98) circle (  2.13);

\path[fill=fillColor,fill opacity=0.20] ( 71.27, 94.08) circle (  2.13);

\path[fill=fillColor,fill opacity=0.20] ( 77.29, 97.24) circle (  2.13);

\path[fill=fillColor,fill opacity=0.20] ( 81.31, 87.63) circle (  2.13);

\path[fill=fillColor,fill opacity=0.20] ( 91.34, 80.55) circle (  2.13);

\path[fill=fillColor,fill opacity=0.20] (128.46, 69.42) circle (  2.13);

\path[fill=fillColor,fill opacity=0.20] (140.50, 48.94) circle (  2.13);

\path[fill=fillColor,fill opacity=0.20] ( 85.32, 75.62) circle (  2.13);

\path[fill=fillColor,fill opacity=0.20] ( 99.36, 82.70) circle (  2.13);

\path[fill=fillColor,fill opacity=0.20] ( 94.35, 83.71) circle (  2.13);

\path[fill=fillColor,fill opacity=0.20] (100.37, 86.75) circle (  2.13);

\path[fill=fillColor,fill opacity=0.20] (103.38, 79.54) circle (  2.13);

\path[fill=fillColor,fill opacity=0.20] (112.41, 71.20) circle (  2.13);

\path[fill=fillColor,fill opacity=0.20] (113.41, 65.00) circle (  2.13);

\path[fill=fillColor,fill opacity=0.20] (100.37, 66.90) circle (  2.13);

\path[fill=fillColor,fill opacity=0.20] ( 94.35, 69.42) circle (  2.13);

\path[fill=fillColor,fill opacity=0.20] ( 85.32, 74.10) circle (  2.13);

\path[fill=fillColor,fill opacity=0.20] ( 81.31, 85.61) circle (  2.13);

\path[fill=fillColor,fill opacity=0.20] ( 81.31, 90.54) circle (  2.13);

\path[fill=fillColor,fill opacity=0.20] ( 80.30, 89.78) circle (  2.13);

\path[fill=fillColor,fill opacity=0.20] ( 78.30, 93.45) circle (  2.13);

\path[fill=fillColor,fill opacity=0.20] ( 76.29, 95.98) circle (  2.13);

\path[fill=fillColor,fill opacity=0.20] ( 74.28, 94.97) circle (  2.13);

\path[fill=fillColor,fill opacity=0.20] ( 64.95, 98.26) circle (  2.13);

\path[fill=fillColor,fill opacity=0.20] ( 45.29,101.54) circle (  2.13);

\path[fill=fillColor,fill opacity=0.20] ( 75.29, 98.76) circle (  2.13);

\path[fill=fillColor,fill opacity=0.20] ( 79.30, 94.21) circle (  2.13);

\path[fill=fillColor,fill opacity=0.20] ( 86.32, 86.50) circle (  2.13);

\path[fill=fillColor,fill opacity=0.20] ( 95.35, 65.25) circle (  2.13);

\path[fill=fillColor,fill opacity=0.20] ( 95.35, 72.21) circle (  2.13);

\path[fill=fillColor,fill opacity=0.20] ( 95.35, 76.00) circle (  2.13);

\path[fill=fillColor,fill opacity=0.20] (101.37, 72.59) circle (  2.13);

\path[fill=fillColor,fill opacity=0.20] (104.38, 72.71) circle (  2.13);

\path[fill=fillColor,fill opacity=0.20] (103.38, 66.77) circle (  2.13);

\path[fill=fillColor,fill opacity=0.20] (106.39, 65.00) circle (  2.13);

\path[fill=fillColor,fill opacity=0.20] (107.39, 62.85) circle (  2.13);

\path[fill=fillColor,fill opacity=0.20] (103.38, 56.40) circle (  2.13);

\path[fill=fillColor,fill opacity=0.20] (101.37, 59.06) circle (  2.13);

\path[fill=fillColor,fill opacity=0.20] ( 92.34, 68.67) circle (  2.13);

\path[fill=fillColor,fill opacity=0.20] ( 88.33, 79.29) circle (  2.13);

\path[fill=fillColor,fill opacity=0.20] ( 82.31, 79.29) circle (  2.13);

\path[fill=fillColor,fill opacity=0.20] ( 76.29, 83.71) circle (  2.13);

\path[fill=fillColor,fill opacity=0.20] ( 77.29, 93.32) circle (  2.13);

\path[fill=fillColor,fill opacity=0.20] ( 81.31, 94.34) circle (  2.13);

\path[fill=fillColor,fill opacity=0.20] ( 75.29, 92.57) circle (  2.13);

\path[fill=fillColor,fill opacity=0.20] ( 74.28, 93.83) circle (  2.13);

\path[fill=fillColor,fill opacity=0.20] ( 73.28, 91.93) circle (  2.13);

\path[fill=fillColor,fill opacity=0.20] ( 78.30, 92.94) circle (  2.13);

\path[fill=fillColor,fill opacity=0.20] ( 80.30, 92.44) circle (  2.13);

\path[fill=fillColor,fill opacity=0.20] ( 79.30, 88.01) circle (  2.13);

\path[fill=fillColor,fill opacity=0.20] ( 79.30, 85.86) circle (  2.13);

\path[fill=fillColor,fill opacity=0.20] ( 91.34, 80.05) circle (  2.13);

\path[fill=fillColor,fill opacity=0.20] (139.49, 58.42) circle (  2.13);

\path[fill=fillColor,fill opacity=0.20] (131.47, 50.71) circle (  2.13);

\path[fill=fillColor,fill opacity=0.20] (113.41, 68.16) circle (  2.13);

\path[fill=fillColor,fill opacity=0.20] ( 88.33, 66.39) circle (  2.13);

\path[fill=fillColor,fill opacity=0.20] ( 92.34, 66.52) circle (  2.13);

\path[fill=fillColor,fill opacity=0.20] ( 95.35, 65.00) circle (  2.13);

\path[fill=fillColor,fill opacity=0.20] ( 89.33, 66.90) circle (  2.13);

\path[fill=fillColor,fill opacity=0.20] ( 96.35, 69.05) circle (  2.13);

\path[fill=fillColor,fill opacity=0.20] (102.37, 67.02) circle (  2.13);

\path[fill=fillColor,fill opacity=0.20] (102.37, 52.73) circle (  2.13);

\path[fill=fillColor,fill opacity=0.20] (101.37, 43.88) circle (  2.13);

\path[fill=fillColor,fill opacity=0.20] ( 99.36, 66.39) circle (  2.13);

\path[fill=fillColor,fill opacity=0.20] ( 97.36, 72.33) circle (  2.13);

\path[fill=fillColor,fill opacity=0.20] ( 89.33, 72.21) circle (  2.13);

\path[fill=fillColor,fill opacity=0.20] ( 85.32, 79.67) circle (  2.13);

\path[fill=fillColor,fill opacity=0.20] ( 77.29, 81.18) circle (  2.13);

\path[fill=fillColor,fill opacity=0.20] ( 72.28, 78.78) circle (  2.13);

\path[fill=fillColor,fill opacity=0.20] ( 72.28, 85.74) circle (  2.13);

\path[fill=fillColor,fill opacity=0.20] ( 69.27, 99.14) circle (  2.13);

\path[fill=fillColor,fill opacity=0.20] ( 79.30,101.54) circle (  2.13);

\path[fill=fillColor,fill opacity=0.20] ( 77.29, 97.50) circle (  2.13);

\path[fill=fillColor,fill opacity=0.20] ( 82.31, 93.20) circle (  2.13);

\path[fill=fillColor,fill opacity=0.20] ( 83.31, 89.02) circle (  2.13);

\path[fill=fillColor,fill opacity=0.20] ( 88.33, 81.82) circle (  2.13);

\path[fill=fillColor,fill opacity=0.20] ( 95.35, 71.57) circle (  2.13);

\path[fill=fillColor,fill opacity=0.20] (116.42, 65.50) circle (  2.13);

\path[fill=fillColor,fill opacity=0.20] (113.41, 63.99) circle (  2.13);

\path[fill=fillColor,fill opacity=0.20] (102.37, 64.11) circle (  2.13);

\path[fill=fillColor,fill opacity=0.20] (100.37, 70.06) circle (  2.13);

\path[fill=fillColor,fill opacity=0.20] ( 88.33, 71.95) circle (  2.13);

\path[fill=fillColor,fill opacity=0.20] ( 92.34, 70.18) circle (  2.13);

\path[fill=fillColor,fill opacity=0.20] ( 94.35, 69.55) circle (  2.13);

\path[fill=fillColor,fill opacity=0.20] ( 94.35, 71.70) circle (  2.13);

\path[fill=fillColor,fill opacity=0.20] ( 93.34, 76.13) circle (  2.13);

\path[fill=fillColor,fill opacity=0.20] (105.38, 71.95) circle (  2.13);

\path[fill=fillColor,fill opacity=0.20] (103.38, 62.34) circle (  2.13);

\path[fill=fillColor,fill opacity=0.20] (104.38, 61.08) circle (  2.13);

\path[fill=fillColor,fill opacity=0.20] (108.39, 64.24) circle (  2.13);

\path[fill=fillColor,fill opacity=0.20] ( 96.35, 50.20) circle (  2.13);

\path[fill=fillColor,fill opacity=0.20] ( 99.36, 65.13) circle (  2.13);

\path[fill=fillColor,fill opacity=0.20] ( 93.34, 67.91) circle (  2.13);

\path[fill=fillColor,fill opacity=0.20] ( 91.34, 64.75) circle (  2.13);

\path[fill=fillColor,fill opacity=0.20] ( 84.32, 67.78) circle (  2.13);

\path[fill=fillColor,fill opacity=0.20] ( 76.29, 73.85) circle (  2.13);

\path[fill=fillColor,fill opacity=0.20] ( 79.30, 76.76) circle (  2.13);

\path[fill=fillColor,fill opacity=0.20] ( 76.29, 78.28) circle (  2.13);

\path[fill=fillColor,fill opacity=0.20] ( 71.27, 84.35) circle (  2.13);

\path[fill=fillColor,fill opacity=0.20] ( 75.29, 87.89) circle (  2.13);

\path[fill=fillColor,fill opacity=0.20] ( 81.31, 87.13) circle (  2.13);

\path[fill=fillColor,fill opacity=0.20] ( 87.33, 86.75) circle (  2.13);

\path[fill=fillColor,fill opacity=0.20] ( 95.35, 82.58) circle (  2.13);

\path[fill=fillColor,fill opacity=0.20] (101.37, 74.48) circle (  2.13);

\path[fill=fillColor,fill opacity=0.20] (110.40, 68.67) circle (  2.13);

\path[fill=fillColor,fill opacity=0.20] ( 99.36, 56.91) circle (  2.13);

\path[fill=fillColor,fill opacity=0.20] (103.38, 69.30) circle (  2.13);

\path[fill=fillColor,fill opacity=0.20] ( 92.34, 70.06) circle (  2.13);

\path[fill=fillColor,fill opacity=0.20] ( 98.36, 67.53) circle (  2.13);

\path[fill=fillColor,fill opacity=0.20] ( 88.33, 70.06) circle (  2.13);

\path[fill=fillColor,fill opacity=0.20] ( 96.35, 76.00) circle (  2.13);

\path[fill=fillColor,fill opacity=0.20] ( 99.36, 73.98) circle (  2.13);

\path[fill=fillColor,fill opacity=0.20] (104.38, 69.80) circle (  2.13);

\path[fill=fillColor,fill opacity=0.20] ( 93.34, 69.93) circle (  2.13);

\path[fill=fillColor,fill opacity=0.20] ( 94.35, 66.77) circle (  2.13);

\path[fill=fillColor,fill opacity=0.20] ( 99.36, 69.17) circle (  2.13);

\path[fill=fillColor,fill opacity=0.20] ( 98.36, 63.61) circle (  2.13);

\path[fill=fillColor,fill opacity=0.20] ( 93.34, 58.68) circle (  2.13);

\path[fill=fillColor,fill opacity=0.20] ( 88.33, 62.72) circle (  2.13);

\path[fill=fillColor,fill opacity=0.20] ( 79.30, 64.87) circle (  2.13);

\path[fill=fillColor,fill opacity=0.20] ( 74.28, 66.39) circle (  2.13);

\path[fill=fillColor,fill opacity=0.20] ( 79.30, 73.34) circle (  2.13);

\path[fill=fillColor,fill opacity=0.20] ( 80.30, 80.43) circle (  2.13);

\path[fill=fillColor,fill opacity=0.20] ( 77.29, 81.82) circle (  2.13);

\path[fill=fillColor,fill opacity=0.20] ( 84.32, 83.71) circle (  2.13);

\path[fill=fillColor,fill opacity=0.20] ( 91.34, 81.31) circle (  2.13);

\path[fill=fillColor,fill opacity=0.20] ( 58.43, 69.80) circle (  2.13);

\path[fill=fillColor,fill opacity=0.20] (110.40, 58.93) circle (  2.13);

\path[fill=fillColor,fill opacity=0.20] (105.38, 54.38) circle (  2.13);

\path[fill=fillColor,fill opacity=0.20] ( 87.33, 58.42) circle (  2.13);

\path[fill=fillColor,fill opacity=0.20] (104.38, 66.64) circle (  2.13);

\path[fill=fillColor,fill opacity=0.20] (100.37, 70.69) circle (  2.13);

\path[fill=fillColor,fill opacity=0.20] ( 97.36, 71.83) circle (  2.13);

\path[fill=fillColor,fill opacity=0.20] (102.37, 68.54) circle (  2.13);

\path[fill=fillColor,fill opacity=0.20] (103.38, 68.16) circle (  2.13);

\path[fill=fillColor,fill opacity=0.20] (104.38, 76.38) circle (  2.13);

\path[fill=fillColor,fill opacity=0.20] ( 96.35, 72.21) circle (  2.13);

\path[fill=fillColor,fill opacity=0.20] ( 94.35, 64.11) circle (  2.13);

\path[fill=fillColor,fill opacity=0.20] (101.37, 66.01) circle (  2.13);

\path[fill=fillColor,fill opacity=0.20] (103.38, 66.01) circle (  2.13);

\path[fill=fillColor,fill opacity=0.20] (102.37, 60.32) circle (  2.13);

\path[fill=fillColor,fill opacity=0.20] ( 89.33, 53.87) circle (  2.13);

\path[fill=fillColor,fill opacity=0.20] ( 96.35, 53.24) circle (  2.13);

\path[fill=fillColor,fill opacity=0.20] ( 96.35, 51.97) circle (  2.13);

\path[fill=fillColor,fill opacity=0.20] (108.39, 52.73) circle (  2.13);

\path[fill=fillColor,fill opacity=0.20] (103.38, 51.34) circle (  2.13);

\path[fill=fillColor,fill opacity=0.20] (111.40, 54.88) circle (  2.13);

\path[fill=fillColor,fill opacity=0.20] (111.40, 56.27) circle (  2.13);

\path[fill=fillColor,fill opacity=0.20] (104.38, 58.04) circle (  2.13);

\path[fill=fillColor,fill opacity=0.20] (101.37, 59.44) circle (  2.13);

\path[fill=fillColor,fill opacity=0.20] ( 90.33, 62.22) circle (  2.13);

\path[fill=fillColor,fill opacity=0.20] ( 83.31, 61.84) circle (  2.13);

\path[fill=fillColor,fill opacity=0.20] ( 84.32, 70.06) circle (  2.13);

\path[fill=fillColor,fill opacity=0.20] ( 91.34, 71.20) circle (  2.13);

\path[fill=fillColor,fill opacity=0.20] ( 89.33, 78.53) circle (  2.13);

\path[fill=fillColor,fill opacity=0.20] ( 87.33, 89.02) circle (  2.13);

\path[fill=fillColor,fill opacity=0.20] ( 56.73, 81.56) circle (  2.13);

\path[fill=fillColor,fill opacity=0.20] (101.37, 70.56) circle (  2.13);

\path[fill=fillColor,fill opacity=0.20] (111.40, 60.95) circle (  2.13);

\path[fill=fillColor,fill opacity=0.20] (114.41, 47.68) circle (  2.13);

\path[fill=fillColor,fill opacity=0.20] (108.39, 52.61) circle (  2.13);

\path[fill=fillColor,fill opacity=0.20] (112.41, 67.40) circle (  2.13);

\path[fill=fillColor,fill opacity=0.20] ( 99.36, 66.77) circle (  2.13);

\path[fill=fillColor,fill opacity=0.20] (102.37, 67.91) circle (  2.13);

\path[fill=fillColor,fill opacity=0.20] ( 98.36, 76.89) circle (  2.13);

\path[fill=fillColor,fill opacity=0.20] ( 97.36, 71.95) circle (  2.13);

\path[fill=fillColor,fill opacity=0.20] ( 95.35, 63.48) circle (  2.13);

\path[fill=fillColor,fill opacity=0.20] ( 93.34, 64.24) circle (  2.13);

\path[fill=fillColor,fill opacity=0.20] (101.37, 70.69) circle (  2.13);

\path[fill=fillColor,fill opacity=0.20] (100.37, 67.65) circle (  2.13);

\path[fill=fillColor,fill opacity=0.20] (107.39, 64.37) circle (  2.13);

\path[fill=fillColor,fill opacity=0.20] (104.38, 68.16) circle (  2.13);

\path[fill=fillColor,fill opacity=0.20] ( 98.36, 71.70) circle (  2.13);

\path[fill=fillColor,fill opacity=0.20] ( 99.36, 66.26) circle (  2.13);

\path[fill=fillColor,fill opacity=0.20] (103.38, 63.86) circle (  2.13);

\path[fill=fillColor,fill opacity=0.20] ( 96.35, 53.87) circle (  2.13);

\path[fill=fillColor,fill opacity=0.20] ( 79.30, 49.19) circle (  2.13);

\path[fill=fillColor,fill opacity=0.20] ( 86.32, 58.68) circle (  2.13);

\path[fill=fillColor,fill opacity=0.20] ( 89.33, 65.76) circle (  2.13);

\path[fill=fillColor,fill opacity=0.20] ( 98.36, 56.27) circle (  2.13);

\path[fill=fillColor,fill opacity=0.20] (104.38, 56.78) circle (  2.13);

\path[fill=fillColor,fill opacity=0.20] (106.39, 58.68) circle (  2.13);

\path[fill=fillColor,fill opacity=0.20] (109.40, 57.03) circle (  2.13);

\path[fill=fillColor,fill opacity=0.20] (101.37, 61.46) circle (  2.13);

\path[fill=fillColor,fill opacity=0.20] ( 98.36, 63.23) circle (  2.13);

\path[fill=fillColor,fill opacity=0.20] ( 79.30, 60.19) circle (  2.13);

\path[fill=fillColor,fill opacity=0.20] ( 78.30, 67.02) circle (  2.13);

\path[fill=fillColor,fill opacity=0.20] ( 89.33, 78.78) circle (  2.13);

\path[fill=fillColor,fill opacity=0.20] ( 84.32, 77.77) circle (  2.13);

\path[fill=fillColor,fill opacity=0.20] ( 83.31, 67.65) circle (  2.13);

\path[fill=fillColor,fill opacity=0.20] (129.46, 56.78) circle (  2.13);

\path[fill=fillColor,fill opacity=0.20] (109.40, 45.91) circle (  2.13);

\path[fill=fillColor,fill opacity=0.20] (125.45, 58.42) circle (  2.13);

\path[fill=fillColor,fill opacity=0.20] (101.37, 67.02) circle (  2.13);

\path[fill=fillColor,fill opacity=0.20] (105.38, 62.60) circle (  2.13);

\path[fill=fillColor,fill opacity=0.20] (104.38, 61.21) circle (  2.13);

\path[fill=fillColor,fill opacity=0.20] (111.40, 68.41) circle (  2.13);

\path[fill=fillColor,fill opacity=0.20] (114.41, 78.28) circle (  2.13);

\path[fill=fillColor,fill opacity=0.20] (107.39, 72.33) circle (  2.13);

\path[fill=fillColor,fill opacity=0.20] ( 95.35, 63.86) circle (  2.13);

\path[fill=fillColor,fill opacity=0.20] (104.38, 73.98) circle (  2.13);

\path[fill=fillColor,fill opacity=0.20] ( 99.36, 79.54) circle (  2.13);

\path[fill=fillColor,fill opacity=0.20] ( 89.33, 64.24) circle (  2.13);

\path[fill=fillColor,fill opacity=0.20] (102.37, 60.70) circle (  2.13);

\path[fill=fillColor,fill opacity=0.20] (111.40, 59.31) circle (  2.13);

\path[fill=fillColor,fill opacity=0.20] (102.37, 55.52) circle (  2.13);

\path[fill=fillColor,fill opacity=0.20] (103.38, 62.72) circle (  2.13);

\path[fill=fillColor,fill opacity=0.20] (106.39, 65.88) circle (  2.13);

\path[fill=fillColor,fill opacity=0.20] (103.38, 56.78) circle (  2.13);

\path[fill=fillColor,fill opacity=0.20] (100.37, 61.46) circle (  2.13);

\path[fill=fillColor,fill opacity=0.20] ( 92.34, 56.53) circle (  2.13);

\path[fill=fillColor,fill opacity=0.20] ( 89.33, 52.10) circle (  2.13);

\path[fill=fillColor,fill opacity=0.20] ( 89.33, 64.62) circle (  2.13);

\path[fill=fillColor,fill opacity=0.20] ( 83.31, 72.97) circle (  2.13);

\path[fill=fillColor,fill opacity=0.20] ( 77.29, 80.43) circle (  2.13);

\path[fill=fillColor,fill opacity=0.20] ( 92.34, 66.77) circle (  2.13);

\path[fill=fillColor,fill opacity=0.20] (110.40, 62.34) circle (  2.13);

\path[fill=fillColor,fill opacity=0.20] ( 93.34, 66.26) circle (  2.13);

\path[fill=fillColor,fill opacity=0.20] (109.40, 43.12) circle (  2.13);

\path[fill=fillColor,fill opacity=0.20] (115.42, 48.94) circle (  2.13);

\path[fill=fillColor,fill opacity=0.20] (111.40, 58.04) circle (  2.13);

\path[fill=fillColor,fill opacity=0.20] (111.40, 70.56) circle (  2.13);

\path[fill=fillColor,fill opacity=0.20] (113.41, 83.97) circle (  2.13);

\path[fill=fillColor,fill opacity=0.20] (104.38, 81.44) circle (  2.13);

\path[fill=fillColor,fill opacity=0.20] (105.38, 67.02) circle (  2.13);

\path[fill=fillColor,fill opacity=0.20] (103.38, 67.02) circle (  2.13);

\path[fill=fillColor,fill opacity=0.20] (105.38, 69.68) circle (  2.13);

\path[fill=fillColor,fill opacity=0.20] (101.37, 60.70) circle (  2.13);

\path[fill=fillColor,fill opacity=0.20] ( 99.36, 60.19) circle (  2.13);

\path[fill=fillColor,fill opacity=0.20] ( 95.35, 60.70) circle (  2.13);

\path[fill=fillColor,fill opacity=0.20] ( 90.33, 66.64) circle (  2.13);

\path[fill=fillColor,fill opacity=0.20] ( 74.28, 58.42) circle (  2.13);

\path[fill=fillColor,fill opacity=0.20] ( 83.31, 64.11) circle (  2.13);

\path[fill=fillColor,fill opacity=0.20] ( 80.30, 73.09) circle (  2.13);

\path[fill=fillColor,fill opacity=0.20] ( 94.35, 74.86) circle (  2.13);

\path[fill=fillColor,fill opacity=0.20] ( 86.32, 66.77) circle (  2.13);

\path[fill=fillColor,fill opacity=0.20] ( 94.35, 57.54) circle (  2.13);

\path[fill=fillColor,fill opacity=0.20] (110.40, 54.63) circle (  2.13);

\path[fill=fillColor,fill opacity=0.20] (115.42, 58.80) circle (  2.13);

\path[fill=fillColor,fill opacity=0.20] (134.48, 49.70) circle (  2.13);

\path[fill=fillColor,fill opacity=0.20] (123.44, 57.54) circle (  2.13);

\path[fill=fillColor,fill opacity=0.20] (114.41, 61.96) circle (  2.13);

\path[fill=fillColor,fill opacity=0.20] (102.37, 68.41) circle (  2.13);

\path[fill=fillColor,fill opacity=0.20] (103.38, 64.37) circle (  2.13);

\path[fill=fillColor,fill opacity=0.20] ( 89.33, 72.46) circle (  2.13);

\path[fill=fillColor,fill opacity=0.20] ( 87.33, 89.02) circle (  2.13);

\path[fill=fillColor,fill opacity=0.20] ( 96.35, 61.08) circle (  2.13);

\path[fill=fillColor,fill opacity=0.20] ( 88.33, 79.03) circle (  2.13);

\path[fill=fillColor,fill opacity=0.20] ( 96.35,101.42) circle (  2.13);

\path[fill=fillColor,fill opacity=0.20] (103.38, 90.54) circle (  2.13);

\path[fill=fillColor,fill opacity=0.20] ( 97.36, 70.82) circle (  2.13);

\path[fill=fillColor,fill opacity=0.20] (101.37, 59.69) circle (  2.13);

\path[fill=fillColor,fill opacity=0.20] ( 83.31, 61.96) circle (  2.13);

\path[fill=fillColor,fill opacity=0.20] ( 98.36, 66.39) circle (  2.13);

\path[fill=fillColor,fill opacity=0.20] (119.43, 54.63) circle (  2.13);

\path[fill=fillColor,fill opacity=0.20] (133.47, 45.40) circle (  2.13);

\path[fill=fillColor,fill opacity=0.20] (134.48, 39.46) circle (  2.13);

\path[fill=fillColor,fill opacity=0.20] (128.46, 46.92) circle (  2.13);

\path[fill=fillColor,fill opacity=0.20] ( 99.36, 54.88) circle (  2.13);

\path[fill=fillColor,fill opacity=0.20] (101.37, 69.93) circle (  2.13);

\path[fill=fillColor,fill opacity=0.20] ( 92.34, 82.32) circle (  2.13);

\path[fill=fillColor,fill opacity=0.20] ( 89.33, 82.70) circle (  2.13);

\path[fill=fillColor,fill opacity=0.20] ( 85.32, 78.53) circle (  2.13);

\path[fill=fillColor,fill opacity=0.20] ( 90.33, 65.63) circle (  2.13);

\path[fill=fillColor,fill opacity=0.20] (113.41, 68.16) circle (  2.13);

\path[fill=fillColor,fill opacity=0.20] (130.46, 57.66) circle (  2.13);

\path[fill=fillColor,fill opacity=0.20] (127.45, 49.19) circle (  2.13);

\path[fill=fillColor,fill opacity=0.20] (124.44, 42.36) circle (  2.13);

\path[fill=fillColor,fill opacity=0.20] (118.43, 50.20) circle (  2.13);

\path[fill=fillColor,fill opacity=0.20] ( 88.33, 50.84) circle (  2.13);

\path[fill=fillColor,fill opacity=0.20] (112.41, 46.92) circle (  2.13);

\path[fill=fillColor,fill opacity=0.20] (105.38, 47.55) circle (  2.13);

\path[fill=fillColor,fill opacity=0.20] ( 94.35, 52.23) circle (  2.13);

\path[fill=fillColor,fill opacity=0.20] (147.52, 43.12) circle (  2.13);
\end{scope}
\begin{scope}
\path[clip] (159.87, 34.04) rectangle (277.04,119.86);
\definecolor[named]{fillColor}{rgb}{0.90,0.90,0.90}

\path[fill=fillColor] (159.87, 34.04) rectangle (277.03,119.86);
\definecolor[named]{drawColor}{rgb}{0.95,0.95,0.95}

\path[draw=drawColor,line width= 0.3pt,line join=round,line cap=round] (159.87, 45.91) --
	(277.04, 45.91);

\path[draw=drawColor,line width= 0.3pt,line join=round,line cap=round] (159.87, 58.55) --
	(277.04, 58.55);

\path[draw=drawColor,line width= 0.3pt,line join=round,line cap=round] (159.87, 71.20) --
	(277.04, 71.20);

\path[draw=drawColor,line width= 0.3pt,line join=round,line cap=round] (159.87, 83.84) --
	(277.04, 83.84);

\path[draw=drawColor,line width= 0.3pt,line join=round,line cap=round] (159.87, 96.49) --
	(277.04, 96.49);

\path[draw=drawColor,line width= 0.3pt,line join=round,line cap=round] (159.87,109.13) --
	(277.04,109.13);

\path[draw=drawColor,line width= 0.3pt,line join=round,line cap=round] (178.41, 34.04) --
	(178.41,119.86);

\path[draw=drawColor,line width= 0.3pt,line join=round,line cap=round] (198.47, 34.04) --
	(198.47,119.86);

\path[draw=drawColor,line width= 0.3pt,line join=round,line cap=round] (218.54, 34.04) --
	(218.54,119.86);

\path[draw=drawColor,line width= 0.3pt,line join=round,line cap=round] (238.60, 34.04) --
	(238.60,119.86);

\path[draw=drawColor,line width= 0.3pt,line join=round,line cap=round] (258.67, 34.04) --
	(258.67,119.86);
\definecolor[named]{drawColor}{rgb}{1.00,1.00,1.00}

\path[draw=drawColor,line width= 0.6pt,line join=round,line cap=round] (159.87, 39.58) --
	(277.04, 39.58);

\path[draw=drawColor,line width= 0.6pt,line join=round,line cap=round] (159.87, 52.23) --
	(277.04, 52.23);

\path[draw=drawColor,line width= 0.6pt,line join=round,line cap=round] (159.87, 64.87) --
	(277.04, 64.87);

\path[draw=drawColor,line width= 0.6pt,line join=round,line cap=round] (159.87, 77.52) --
	(277.04, 77.52);

\path[draw=drawColor,line width= 0.6pt,line join=round,line cap=round] (159.87, 90.16) --
	(277.04, 90.16);

\path[draw=drawColor,line width= 0.6pt,line join=round,line cap=round] (159.87,102.81) --
	(277.04,102.81);

\path[draw=drawColor,line width= 0.6pt,line join=round,line cap=round] (159.87,115.45) --
	(277.04,115.45);

\path[draw=drawColor,line width= 0.6pt,line join=round,line cap=round] (168.38, 34.04) --
	(168.38,119.86);

\path[draw=drawColor,line width= 0.6pt,line join=round,line cap=round] (188.44, 34.04) --
	(188.44,119.86);

\path[draw=drawColor,line width= 0.6pt,line join=round,line cap=round] (208.51, 34.04) --
	(208.51,119.86);

\path[draw=drawColor,line width= 0.6pt,line join=round,line cap=round] (228.57, 34.04) --
	(228.57,119.86);

\path[draw=drawColor,line width= 0.6pt,line join=round,line cap=round] (248.64, 34.04) --
	(248.64,119.86);

\path[draw=drawColor,line width= 0.6pt,line join=round,line cap=round] (268.70, 34.04) --
	(268.70,119.86);
\definecolor[named]{fillColor}{rgb}{0.00,0.00,0.00}

\path[fill=fillColor,fill opacity=0.20] (218.54, 47.17) circle (  2.13);

\path[fill=fillColor,fill opacity=0.20] (213.52, 49.70) circle (  2.13);

\path[fill=fillColor,fill opacity=0.20] (221.55, 52.23) circle (  2.13);

\path[fill=fillColor,fill opacity=0.20] (213.52, 55.64) circle (  2.13);

\path[fill=fillColor,fill opacity=0.20] (223.55, 53.75) circle (  2.13);

\path[fill=fillColor,fill opacity=0.20] (235.59, 41.73) circle (  2.13);

\path[fill=fillColor,fill opacity=0.20] (216.53, 50.96) circle (  2.13);

\path[fill=fillColor,fill opacity=0.20] (206.50, 62.72) circle (  2.13);

\path[fill=fillColor,fill opacity=0.20] (201.48, 64.75) circle (  2.13);

\path[fill=fillColor,fill opacity=0.20] (201.48, 66.14) circle (  2.13);

\path[fill=fillColor,fill opacity=0.20] (207.50, 71.20) circle (  2.13);

\path[fill=fillColor,fill opacity=0.20] (204.49, 72.46) circle (  2.13);

\path[fill=fillColor,fill opacity=0.20] (205.50, 61.71) circle (  2.13);

\path[fill=fillColor,fill opacity=0.20] (216.53, 49.07) circle (  2.13);

\path[fill=fillColor,fill opacity=0.20] (228.57, 40.72) circle (  2.13);

\path[fill=fillColor,fill opacity=0.20] (210.51, 52.48) circle (  2.13);

\path[fill=fillColor,fill opacity=0.20] (201.48, 64.11) circle (  2.13);

\path[fill=fillColor,fill opacity=0.20] (196.47, 73.09) circle (  2.13);

\path[fill=fillColor,fill opacity=0.20] (196.47, 78.28) circle (  2.13);

\path[fill=fillColor,fill opacity=0.20] (203.49, 83.33) circle (  2.13);

\path[fill=fillColor,fill opacity=0.20] (207.50, 82.70) circle (  2.13);

\path[fill=fillColor,fill opacity=0.20] (210.51, 73.98) circle (  2.13);

\path[fill=fillColor,fill opacity=0.20] (209.51, 63.86) circle (  2.13);

\path[fill=fillColor,fill opacity=0.20] (205.50, 54.76) circle (  2.13);

\path[fill=fillColor,fill opacity=0.20] (217.53, 44.89) circle (  2.13);

\path[fill=fillColor,fill opacity=0.20] (207.50, 54.76) circle (  2.13);

\path[fill=fillColor,fill opacity=0.20] (198.47, 70.82) circle (  2.13);

\path[fill=fillColor,fill opacity=0.20] (191.45, 77.39) circle (  2.13);

\path[fill=fillColor,fill opacity=0.20] (191.45, 77.26) circle (  2.13);

\path[fill=fillColor,fill opacity=0.20] (196.47, 81.31) circle (  2.13);

\path[fill=fillColor,fill opacity=0.20] (200.48, 85.99) circle (  2.13);

\path[fill=fillColor,fill opacity=0.20] (201.48, 81.82) circle (  2.13);

\path[fill=fillColor,fill opacity=0.20] (206.50, 73.34) circle (  2.13);

\path[fill=fillColor,fill opacity=0.20] (205.50, 67.91) circle (  2.13);

\path[fill=fillColor,fill opacity=0.20] (198.47, 66.77) circle (  2.13);

\path[fill=fillColor,fill opacity=0.20] (200.48, 63.10) circle (  2.13);

\path[fill=fillColor,fill opacity=0.20] (210.51, 55.89) circle (  2.13);

\path[fill=fillColor,fill opacity=0.20] (217.53, 40.47) circle (  2.13);

\path[fill=fillColor,fill opacity=0.20] (206.50, 66.77) circle (  2.13);

\path[fill=fillColor,fill opacity=0.20] (194.46, 91.30) circle (  2.13);

\path[fill=fillColor,fill opacity=0.20] (191.45, 92.44) circle (  2.13);

\path[fill=fillColor,fill opacity=0.20] (194.46, 81.06) circle (  2.13);

\path[fill=fillColor,fill opacity=0.20] (196.47, 78.91) circle (  2.13);

\path[fill=fillColor,fill opacity=0.20] (198.47, 81.44) circle (  2.13);

\path[fill=fillColor,fill opacity=0.20] (201.48, 81.31) circle (  2.13);

\path[fill=fillColor,fill opacity=0.20] (202.49, 76.13) circle (  2.13);

\path[fill=fillColor,fill opacity=0.20] (208.51, 68.16) circle (  2.13);

\path[fill=fillColor,fill opacity=0.20] (201.48, 69.80) circle (  2.13);

\path[fill=fillColor,fill opacity=0.20] (197.47, 68.41) circle (  2.13);

\path[fill=fillColor,fill opacity=0.20] (203.49, 67.15) circle (  2.13);

\path[fill=fillColor,fill opacity=0.20] (207.50, 68.79) circle (  2.13);

\path[fill=fillColor,fill opacity=0.20] (212.52, 69.93) circle (  2.13);

\path[fill=fillColor,fill opacity=0.20] (216.53, 61.08) circle (  2.13);

\path[fill=fillColor,fill opacity=0.20] (214.53, 42.62) circle (  2.13);

\path[fill=fillColor,fill opacity=0.20] (209.51, 69.93) circle (  2.13);

\path[fill=fillColor,fill opacity=0.20] (200.48, 89.78) circle (  2.13);

\path[fill=fillColor,fill opacity=0.20] (197.47, 91.93) circle (  2.13);

\path[fill=fillColor,fill opacity=0.20] (197.47, 84.98) circle (  2.13);

\path[fill=fillColor,fill opacity=0.20] (199.48, 81.44) circle (  2.13);

\path[fill=fillColor,fill opacity=0.20] (201.48, 82.07) circle (  2.13);

\path[fill=fillColor,fill opacity=0.20] (204.49, 83.84) circle (  2.13);

\path[fill=fillColor,fill opacity=0.20] (207.50, 78.15) circle (  2.13);

\path[fill=fillColor,fill opacity=0.20] (220.54, 64.24) circle (  2.13);

\path[fill=fillColor,fill opacity=0.20] (197.47, 64.11) circle (  2.13);

\path[fill=fillColor,fill opacity=0.20] (193.46, 71.70) circle (  2.13);

\path[fill=fillColor,fill opacity=0.20] (195.46, 70.18) circle (  2.13);

\path[fill=fillColor,fill opacity=0.20] (196.47, 74.36) circle (  2.13);

\path[fill=fillColor,fill opacity=0.20] (196.47, 79.03) circle (  2.13);

\path[fill=fillColor,fill opacity=0.20] (204.49, 80.05) circle (  2.13);

\path[fill=fillColor,fill opacity=0.20] (204.49, 79.29) circle (  2.13);

\path[fill=fillColor,fill opacity=0.20] (208.51, 71.95) circle (  2.13);

\path[fill=fillColor,fill opacity=0.20] (213.52, 40.09) circle (  2.13);

\path[fill=fillColor,fill opacity=0.20] (211.52, 64.75) circle (  2.13);

\path[fill=fillColor,fill opacity=0.20] (201.48, 78.28) circle (  2.13);

\path[fill=fillColor,fill opacity=0.20] (199.48, 83.71) circle (  2.13);

\path[fill=fillColor,fill opacity=0.20] (198.47, 85.48) circle (  2.13);

\path[fill=fillColor,fill opacity=0.20] (199.48, 85.48) circle (  2.13);

\path[fill=fillColor,fill opacity=0.20] (204.49, 84.35) circle (  2.13);

\path[fill=fillColor,fill opacity=0.20] (207.50, 83.71) circle (  2.13);

\path[fill=fillColor,fill opacity=0.20] (212.52, 79.16) circle (  2.13);

\path[fill=fillColor,fill opacity=0.20] (200.48, 77.77) circle (  2.13);

\path[fill=fillColor,fill opacity=0.20] (193.46, 79.92) circle (  2.13);

\path[fill=fillColor,fill opacity=0.20] (196.47, 81.44) circle (  2.13);

\path[fill=fillColor,fill opacity=0.20] (198.47, 89.15) circle (  2.13);

\path[fill=fillColor,fill opacity=0.20] (200.48, 90.42) circle (  2.13);

\path[fill=fillColor,fill opacity=0.20] (201.48, 85.36) circle (  2.13);

\path[fill=fillColor,fill opacity=0.20] (201.48, 83.46) circle (  2.13);

\path[fill=fillColor,fill opacity=0.20] (211.52, 80.05) circle (  2.13);

\path[fill=fillColor,fill opacity=0.20] (209.51, 57.29) circle (  2.13);

\path[fill=fillColor,fill opacity=0.20] (199.48, 72.59) circle (  2.13);

\path[fill=fillColor,fill opacity=0.20] (199.48, 79.79) circle (  2.13);

\path[fill=fillColor,fill opacity=0.20] (202.49, 82.58) circle (  2.13);

\path[fill=fillColor,fill opacity=0.20] (200.48, 84.98) circle (  2.13);

\path[fill=fillColor,fill opacity=0.20] (203.49, 85.99) circle (  2.13);

\path[fill=fillColor,fill opacity=0.20] (206.50, 84.35) circle (  2.13);

\path[fill=fillColor,fill opacity=0.20] (212.52, 80.68) circle (  2.13);

\path[fill=fillColor,fill opacity=0.20] (202.49, 72.46) circle (  2.13);

\path[fill=fillColor,fill opacity=0.20] (196.47, 93.58) circle (  2.13);

\path[fill=fillColor,fill opacity=0.20] (198.47, 94.34) circle (  2.13);

\path[fill=fillColor,fill opacity=0.20] (198.47,100.40) circle (  2.13);

\path[fill=fillColor,fill opacity=0.20] (199.48,101.16) circle (  2.13);

\path[fill=fillColor,fill opacity=0.20] (203.49, 90.92) circle (  2.13);

\path[fill=fillColor,fill opacity=0.20] (205.50, 89.28) circle (  2.13);

\path[fill=fillColor,fill opacity=0.20] (211.52, 90.16) circle (  2.13);

\path[fill=fillColor,fill opacity=0.20] (215.53, 81.69) circle (  2.13);

\path[fill=fillColor,fill opacity=0.20] (220.54, 70.94) circle (  2.13);

\path[fill=fillColor,fill opacity=0.20] (208.51, 48.81) circle (  2.13);

\path[fill=fillColor,fill opacity=0.20] (201.48, 68.54) circle (  2.13);

\path[fill=fillColor,fill opacity=0.20] (201.48, 78.78) circle (  2.13);

\path[fill=fillColor,fill opacity=0.20] (204.49, 81.44) circle (  2.13);

\path[fill=fillColor,fill opacity=0.20] (202.49, 86.24) circle (  2.13);

\path[fill=fillColor,fill opacity=0.20] (203.49, 93.70) circle (  2.13);

\path[fill=fillColor,fill opacity=0.20] (207.50, 91.81) circle (  2.13);

\path[fill=fillColor,fill opacity=0.20] (210.51, 83.84) circle (  2.13);

\path[fill=fillColor,fill opacity=0.20] (217.53, 73.98) circle (  2.13);

\path[fill=fillColor,fill opacity=0.20] (198.47, 99.52) circle (  2.13);

\path[fill=fillColor,fill opacity=0.20] (195.46,108.75) circle (  2.13);

\path[fill=fillColor,fill opacity=0.20] (201.48,102.05) circle (  2.13);

\path[fill=fillColor,fill opacity=0.20] (202.49,108.37) circle (  2.13);

\path[fill=fillColor,fill opacity=0.20] (203.49,102.68) circle (  2.13);

\path[fill=fillColor,fill opacity=0.20] (206.50, 92.06) circle (  2.13);

\path[fill=fillColor,fill opacity=0.20] (208.51, 98.13) circle (  2.13);

\path[fill=fillColor,fill opacity=0.20] (214.53, 97.75) circle (  2.13);

\path[fill=fillColor,fill opacity=0.20] (219.54, 80.43) circle (  2.13);

\path[fill=fillColor,fill opacity=0.20] (223.55, 68.29) circle (  2.13);

\path[fill=fillColor,fill opacity=0.20] (209.51, 40.85) circle (  2.13);

\path[fill=fillColor,fill opacity=0.20] (202.49, 64.11) circle (  2.13);

\path[fill=fillColor,fill opacity=0.20] (199.48, 79.03) circle (  2.13);

\path[fill=fillColor,fill opacity=0.20] (198.47, 86.24) circle (  2.13);

\path[fill=fillColor,fill opacity=0.20] (200.48, 94.59) circle (  2.13);

\path[fill=fillColor,fill opacity=0.20] (203.49,101.16) circle (  2.13);

\path[fill=fillColor,fill opacity=0.20] (209.51, 96.23) circle (  2.13);

\path[fill=fillColor,fill opacity=0.20] (213.52, 88.52) circle (  2.13);

\path[fill=fillColor,fill opacity=0.20] (215.53, 84.85) circle (  2.13);

\path[fill=fillColor,fill opacity=0.20] (226.56, 66.77) circle (  2.13);

\path[fill=fillColor,fill opacity=0.20] (205.50, 67.28) circle (  2.13);

\path[fill=fillColor,fill opacity=0.20] (190.45,113.18) circle (  2.13);

\path[fill=fillColor,fill opacity=0.20] (199.48,109.26) circle (  2.13);

\path[fill=fillColor,fill opacity=0.20] (205.50,103.44) circle (  2.13);

\path[fill=fillColor,fill opacity=0.20] (203.49,105.08) circle (  2.13);

\path[fill=fillColor,fill opacity=0.20] (202.49, 99.65) circle (  2.13);

\path[fill=fillColor,fill opacity=0.20] (204.49, 99.39) circle (  2.13);

\path[fill=fillColor,fill opacity=0.20] (207.50,107.23) circle (  2.13);

\path[fill=fillColor,fill opacity=0.20] (210.51,100.66) circle (  2.13);

\path[fill=fillColor,fill opacity=0.20] (218.54, 83.84) circle (  2.13);

\path[fill=fillColor,fill opacity=0.20] (215.53, 71.07) circle (  2.13);

\path[fill=fillColor,fill opacity=0.20] (239.61, 47.04) circle (  2.13);

\path[fill=fillColor,fill opacity=0.20] (202.49, 58.42) circle (  2.13);

\path[fill=fillColor,fill opacity=0.20] (197.47, 78.66) circle (  2.13);

\path[fill=fillColor,fill opacity=0.20] (195.46, 95.35) circle (  2.13);

\path[fill=fillColor,fill opacity=0.20] (198.47,103.95) circle (  2.13);

\path[fill=fillColor,fill opacity=0.20] (204.49, 98.76) circle (  2.13);

\path[fill=fillColor,fill opacity=0.20] (208.51, 94.84) circle (  2.13);

\path[fill=fillColor,fill opacity=0.20] (215.53, 93.45) circle (  2.13);

\path[fill=fillColor,fill opacity=0.20] (213.52, 89.66) circle (  2.13);

\path[fill=fillColor,fill opacity=0.20] (220.54, 79.29) circle (  2.13);

\path[fill=fillColor,fill opacity=0.20] (205.50, 90.42) circle (  2.13);

\path[fill=fillColor,fill opacity=0.20] (199.48,104.58) circle (  2.13);

\path[fill=fillColor,fill opacity=0.20] (200.48,104.20) circle (  2.13);

\path[fill=fillColor,fill opacity=0.20] (204.49,108.75) circle (  2.13);

\path[fill=fillColor,fill opacity=0.20] (202.49,104.83) circle (  2.13);

\path[fill=fillColor,fill opacity=0.20] (204.49, 98.00) circle (  2.13);

\path[fill=fillColor,fill opacity=0.20] (209.51,100.91) circle (  2.13);

\path[fill=fillColor,fill opacity=0.20] (211.52,102.18) circle (  2.13);

\path[fill=fillColor,fill opacity=0.20] (212.52, 94.71) circle (  2.13);

\path[fill=fillColor,fill opacity=0.20] (211.52, 84.98) circle (  2.13);

\path[fill=fillColor,fill opacity=0.20] (221.55, 69.55) circle (  2.13);

\path[fill=fillColor,fill opacity=0.20] (240.61, 41.35) circle (  2.13);

\path[fill=fillColor,fill opacity=0.20] (211.52, 47.55) circle (  2.13);

\path[fill=fillColor,fill opacity=0.20] (205.50, 71.32) circle (  2.13);

\path[fill=fillColor,fill opacity=0.20] (198.47, 91.81) circle (  2.13);

\path[fill=fillColor,fill opacity=0.20] (202.49, 97.62) circle (  2.13);

\path[fill=fillColor,fill opacity=0.20] (205.50, 94.08) circle (  2.13);

\path[fill=fillColor,fill opacity=0.20] (203.49, 96.61) circle (  2.13);

\path[fill=fillColor,fill opacity=0.20] (209.51, 96.11) circle (  2.13);

\path[fill=fillColor,fill opacity=0.20] (214.53, 87.76) circle (  2.13);

\path[fill=fillColor,fill opacity=0.20] (220.54, 80.05) circle (  2.13);

\path[fill=fillColor,fill opacity=0.20] (211.52, 72.21) circle (  2.13);

\path[fill=fillColor,fill opacity=0.20] (205.50,102.05) circle (  2.13);

\path[fill=fillColor,fill opacity=0.20] (204.49, 99.01) circle (  2.13);

\path[fill=fillColor,fill opacity=0.20] (206.50,104.07) circle (  2.13);

\path[fill=fillColor,fill opacity=0.20] (199.48,109.64) circle (  2.13);

\path[fill=fillColor,fill opacity=0.20] (203.49,102.68) circle (  2.13);

\path[fill=fillColor,fill opacity=0.20] (202.49, 97.24) circle (  2.13);

\path[fill=fillColor,fill opacity=0.20] (203.49, 94.21) circle (  2.13);

\path[fill=fillColor,fill opacity=0.20] (210.51, 89.66) circle (  2.13);

\path[fill=fillColor,fill opacity=0.20] (212.52, 87.13) circle (  2.13);

\path[fill=fillColor,fill opacity=0.20] (214.53, 79.67) circle (  2.13);

\path[fill=fillColor,fill opacity=0.20] (223.55, 56.02) circle (  2.13);

\path[fill=fillColor,fill opacity=0.20] (210.51, 53.75) circle (  2.13);

\path[fill=fillColor,fill opacity=0.20] (203.49, 77.01) circle (  2.13);

\path[fill=fillColor,fill opacity=0.20] (204.49, 84.22) circle (  2.13);

\path[fill=fillColor,fill opacity=0.20] (206.50, 90.16) circle (  2.13);

\path[fill=fillColor,fill opacity=0.20] (204.49, 96.11) circle (  2.13);

\path[fill=fillColor,fill opacity=0.20] (204.49, 92.69) circle (  2.13);

\path[fill=fillColor,fill opacity=0.20] (210.51, 85.74) circle (  2.13);

\path[fill=fillColor,fill opacity=0.20] (212.52, 80.30) circle (  2.13);

\path[fill=fillColor,fill opacity=0.20] (219.54, 73.22) circle (  2.13);

\path[fill=fillColor,fill opacity=0.20] (202.49, 88.77) circle (  2.13);

\path[fill=fillColor,fill opacity=0.20] (208.51, 97.12) circle (  2.13);

\path[fill=fillColor,fill opacity=0.20] (208.51, 96.74) circle (  2.13);

\path[fill=fillColor,fill opacity=0.20] (205.50,102.43) circle (  2.13);

\path[fill=fillColor,fill opacity=0.20] (200.48,102.55) circle (  2.13);

\path[fill=fillColor,fill opacity=0.20] (199.48, 95.47) circle (  2.13);

\path[fill=fillColor,fill opacity=0.20] (210.51, 92.19) circle (  2.13);

\path[fill=fillColor,fill opacity=0.20] (203.49, 87.51) circle (  2.13);

\path[fill=fillColor,fill opacity=0.20] (208.51, 82.95) circle (  2.13);

\path[fill=fillColor,fill opacity=0.20] (211.52, 83.33) circle (  2.13);

\path[fill=fillColor,fill opacity=0.20] (217.53, 72.71) circle (  2.13);

\path[fill=fillColor,fill opacity=0.20] (210.51, 57.16) circle (  2.13);

\path[fill=fillColor,fill opacity=0.20] (207.50, 70.44) circle (  2.13);

\path[fill=fillColor,fill opacity=0.20] (207.50, 79.54) circle (  2.13);

\path[fill=fillColor,fill opacity=0.20] (207.50, 84.73) circle (  2.13);

\path[fill=fillColor,fill opacity=0.20] (206.50, 85.36) circle (  2.13);

\path[fill=fillColor,fill opacity=0.20] (203.49, 87.63) circle (  2.13);

\path[fill=fillColor,fill opacity=0.20] (206.50, 85.23) circle (  2.13);

\path[fill=fillColor,fill opacity=0.20] (210.51, 79.54) circle (  2.13);

\path[fill=fillColor,fill opacity=0.20] (220.54, 76.13) circle (  2.13);

\path[fill=fillColor,fill opacity=0.20] (230.58, 64.11) circle (  2.13);

\path[fill=fillColor,fill opacity=0.20] (199.48, 84.98) circle (  2.13);

\path[fill=fillColor,fill opacity=0.20] (200.48, 92.57) circle (  2.13);

\path[fill=fillColor,fill opacity=0.20] (206.50, 92.31) circle (  2.13);

\path[fill=fillColor,fill opacity=0.20] (205.50, 90.04) circle (  2.13);

\path[fill=fillColor,fill opacity=0.20] (197.47, 95.85) circle (  2.13);

\path[fill=fillColor,fill opacity=0.20] (200.48, 99.90) circle (  2.13);

\path[fill=fillColor,fill opacity=0.20] (195.46, 92.69) circle (  2.13);

\path[fill=fillColor,fill opacity=0.20] (209.51, 87.25) circle (  2.13);

\path[fill=fillColor,fill opacity=0.20] (206.50, 86.24) circle (  2.13);

\path[fill=fillColor,fill opacity=0.20] (212.52, 82.95) circle (  2.13);

\path[fill=fillColor,fill opacity=0.20] (214.53, 76.89) circle (  2.13);

\path[fill=fillColor,fill opacity=0.20] (221.55, 60.57) circle (  2.13);

\path[fill=fillColor,fill opacity=0.20] (212.52, 50.71) circle (  2.13);

\path[fill=fillColor,fill opacity=0.20] (208.51, 63.23) circle (  2.13);

\path[fill=fillColor,fill opacity=0.20] (207.50, 75.24) circle (  2.13);

\path[fill=fillColor,fill opacity=0.20] (205.50, 83.08) circle (  2.13);

\path[fill=fillColor,fill opacity=0.20] (206.50, 89.91) circle (  2.13);

\path[fill=fillColor,fill opacity=0.20] (207.50, 88.01) circle (  2.13);

\path[fill=fillColor,fill opacity=0.20] (212.52, 82.07) circle (  2.13);

\path[fill=fillColor,fill opacity=0.20] (216.53, 81.69) circle (  2.13);

\path[fill=fillColor,fill opacity=0.20] (221.55, 79.79) circle (  2.13);

\path[fill=fillColor,fill opacity=0.20] (201.48, 84.22) circle (  2.13);

\path[fill=fillColor,fill opacity=0.20] (195.46, 88.14) circle (  2.13);

\path[fill=fillColor,fill opacity=0.20] (201.48, 90.16) circle (  2.13);

\path[fill=fillColor,fill opacity=0.20] (197.47, 90.16) circle (  2.13);

\path[fill=fillColor,fill opacity=0.20] (196.47, 86.62) circle (  2.13);

\path[fill=fillColor,fill opacity=0.20] (196.47, 90.79) circle (  2.13);

\path[fill=fillColor,fill opacity=0.20] (201.48, 97.50) circle (  2.13);

\path[fill=fillColor,fill opacity=0.20] (206.50, 93.58) circle (  2.13);

\path[fill=fillColor,fill opacity=0.20] (206.50, 88.65) circle (  2.13);

\path[fill=fillColor,fill opacity=0.20] (208.51, 89.40) circle (  2.13);

\path[fill=fillColor,fill opacity=0.20] (216.53, 81.18) circle (  2.13);

\path[fill=fillColor,fill opacity=0.20] (223.55, 62.85) circle (  2.13);

\path[fill=fillColor,fill opacity=0.20] (212.52, 46.16) circle (  2.13);

\path[fill=fillColor,fill opacity=0.20] (203.49, 68.41) circle (  2.13);

\path[fill=fillColor,fill opacity=0.20] (205.50, 84.09) circle (  2.13);

\path[fill=fillColor,fill opacity=0.20] (210.51, 90.16) circle (  2.13);

\path[fill=fillColor,fill opacity=0.20] (207.50, 89.91) circle (  2.13);

\path[fill=fillColor,fill opacity=0.20] (207.50, 87.13) circle (  2.13);

\path[fill=fillColor,fill opacity=0.20] (208.51, 87.13) circle (  2.13);

\path[fill=fillColor,fill opacity=0.20] (205.50, 86.12) circle (  2.13);

\path[fill=fillColor,fill opacity=0.20] (214.53, 78.53) circle (  2.13);

\path[fill=fillColor,fill opacity=0.20] (197.47, 84.47) circle (  2.13);

\path[fill=fillColor,fill opacity=0.20] (199.48, 87.76) circle (  2.13);

\path[fill=fillColor,fill opacity=0.20] (196.47, 91.05) circle (  2.13);

\path[fill=fillColor,fill opacity=0.20] (195.46, 92.82) circle (  2.13);

\path[fill=fillColor,fill opacity=0.20] (198.47, 92.82) circle (  2.13);

\path[fill=fillColor,fill opacity=0.20] (196.47, 88.90) circle (  2.13);

\path[fill=fillColor,fill opacity=0.20] (202.49, 86.12) circle (  2.13);

\path[fill=fillColor,fill opacity=0.20] (202.49, 87.25) circle (  2.13);

\path[fill=fillColor,fill opacity=0.20] (205.50, 88.65) circle (  2.13);

\path[fill=fillColor,fill opacity=0.20] (206.50, 87.63) circle (  2.13);

\path[fill=fillColor,fill opacity=0.20] (217.53, 72.71) circle (  2.13);

\path[fill=fillColor,fill opacity=0.20] (226.56, 47.80) circle (  2.13);

\path[fill=fillColor,fill opacity=0.20] (205.50, 73.60) circle (  2.13);

\path[fill=fillColor,fill opacity=0.20] (208.51, 84.73) circle (  2.13);

\path[fill=fillColor,fill opacity=0.20] (209.51, 93.20) circle (  2.13);

\path[fill=fillColor,fill opacity=0.20] (212.52, 95.73) circle (  2.13);

\path[fill=fillColor,fill opacity=0.20] (212.52, 91.81) circle (  2.13);

\path[fill=fillColor,fill opacity=0.20] (210.51, 90.79) circle (  2.13);

\path[fill=fillColor,fill opacity=0.20] (205.50, 84.47) circle (  2.13);

\path[fill=fillColor,fill opacity=0.20] (211.52, 71.57) circle (  2.13);

\path[fill=fillColor,fill opacity=0.20] (222.55, 61.58) circle (  2.13);

\path[fill=fillColor,fill opacity=0.20] (210.51, 79.79) circle (  2.13);

\path[fill=fillColor,fill opacity=0.20] (198.47, 80.43) circle (  2.13);

\path[fill=fillColor,fill opacity=0.20] (198.47, 85.48) circle (  2.13);

\path[fill=fillColor,fill opacity=0.20] (201.48, 93.70) circle (  2.13);

\path[fill=fillColor,fill opacity=0.20] (197.47, 96.36) circle (  2.13);

\path[fill=fillColor,fill opacity=0.20] (199.48, 99.65) circle (  2.13);

\path[fill=fillColor,fill opacity=0.20] (200.48, 99.52) circle (  2.13);

\path[fill=fillColor,fill opacity=0.20] (200.48, 86.62) circle (  2.13);

\path[fill=fillColor,fill opacity=0.20] (199.48, 77.14) circle (  2.13);

\path[fill=fillColor,fill opacity=0.20] (203.49, 78.66) circle (  2.13);

\path[fill=fillColor,fill opacity=0.20] (212.52, 82.83) circle (  2.13);

\path[fill=fillColor,fill opacity=0.20] (217.53, 75.75) circle (  2.13);

\path[fill=fillColor,fill opacity=0.20] (211.52, 56.65) circle (  2.13);

\path[fill=fillColor,fill opacity=0.20] (214.53, 53.11) circle (  2.13);

\path[fill=fillColor,fill opacity=0.20] (206.50, 73.85) circle (  2.13);

\path[fill=fillColor,fill opacity=0.20] (205.50, 85.48) circle (  2.13);

\path[fill=fillColor,fill opacity=0.20] (206.50, 87.38) circle (  2.13);

\path[fill=fillColor,fill opacity=0.20] (210.51, 85.74) circle (  2.13);

\path[fill=fillColor,fill opacity=0.20] (214.53, 88.65) circle (  2.13);

\path[fill=fillColor,fill opacity=0.20] (215.53, 90.92) circle (  2.13);

\path[fill=fillColor,fill opacity=0.20] (211.52, 79.16) circle (  2.13);

\path[fill=fillColor,fill opacity=0.20] (218.54, 66.90) circle (  2.13);

\path[fill=fillColor,fill opacity=0.20] (217.53, 58.30) circle (  2.13);

\path[fill=fillColor,fill opacity=0.20] (207.50, 89.78) circle (  2.13);

\path[fill=fillColor,fill opacity=0.20] (203.49, 82.95) circle (  2.13);

\path[fill=fillColor,fill opacity=0.20] (194.46, 82.58) circle (  2.13);

\path[fill=fillColor,fill opacity=0.20] (196.47, 92.69) circle (  2.13);

\path[fill=fillColor,fill opacity=0.20] (196.47, 95.85) circle (  2.13);

\path[fill=fillColor,fill opacity=0.20] (198.47, 94.84) circle (  2.13);

\path[fill=fillColor,fill opacity=0.20] (199.48, 98.00) circle (  2.13);

\path[fill=fillColor,fill opacity=0.20] (200.48, 94.84) circle (  2.13);

\path[fill=fillColor,fill opacity=0.20] (206.50, 83.33) circle (  2.13);

\path[fill=fillColor,fill opacity=0.20] (207.50, 77.77) circle (  2.13);

\path[fill=fillColor,fill opacity=0.20] (216.53, 80.05) circle (  2.13);

\path[fill=fillColor,fill opacity=0.20] (212.52, 76.25) circle (  2.13);

\path[fill=fillColor,fill opacity=0.20] (229.57, 55.77) circle (  2.13);

\path[fill=fillColor,fill opacity=0.20] (209.51, 53.49) circle (  2.13);

\path[fill=fillColor,fill opacity=0.20] (200.48, 62.72) circle (  2.13);

\path[fill=fillColor,fill opacity=0.20] (202.49, 64.62) circle (  2.13);

\path[fill=fillColor,fill opacity=0.20] (201.48, 75.49) circle (  2.13);

\path[fill=fillColor,fill opacity=0.20] (209.51, 83.46) circle (  2.13);

\path[fill=fillColor,fill opacity=0.20] (213.52, 81.06) circle (  2.13);

\path[fill=fillColor,fill opacity=0.20] (218.54, 81.94) circle (  2.13);

\path[fill=fillColor,fill opacity=0.20] (215.53, 78.91) circle (  2.13);

\path[fill=fillColor,fill opacity=0.20] (217.53, 73.98) circle (  2.13);

\path[fill=fillColor,fill opacity=0.20] (218.54, 77.14) circle (  2.13);

\path[fill=fillColor,fill opacity=0.20] (226.56, 74.74) circle (  2.13);

\path[fill=fillColor,fill opacity=0.20] (213.52, 70.31) circle (  2.13);

\path[fill=fillColor,fill opacity=0.20] (204.49, 76.00) circle (  2.13);

\path[fill=fillColor,fill opacity=0.20] (201.48, 76.00) circle (  2.13);

\path[fill=fillColor,fill opacity=0.20] (197.47, 80.81) circle (  2.13);

\path[fill=fillColor,fill opacity=0.20] (197.47, 89.91) circle (  2.13);

\path[fill=fillColor,fill opacity=0.20] (198.47, 95.35) circle (  2.13);

\path[fill=fillColor,fill opacity=0.20] (197.47, 91.30) circle (  2.13);

\path[fill=fillColor,fill opacity=0.20] (196.47, 88.01) circle (  2.13);

\path[fill=fillColor,fill opacity=0.20] (198.47, 91.17) circle (  2.13);

\path[fill=fillColor,fill opacity=0.20] (206.50, 91.05) circle (  2.13);

\path[fill=fillColor,fill opacity=0.20] (210.51, 84.22) circle (  2.13);

\path[fill=fillColor,fill opacity=0.20] (214.53, 81.69) circle (  2.13);

\path[fill=fillColor,fill opacity=0.20] (219.54, 79.16) circle (  2.13);

\path[fill=fillColor,fill opacity=0.20] (234.59, 58.93) circle (  2.13);

\path[fill=fillColor,fill opacity=0.20] (205.50, 48.18) circle (  2.13);

\path[fill=fillColor,fill opacity=0.20] (205.50, 62.85) circle (  2.13);

\path[fill=fillColor,fill opacity=0.20] (203.49, 77.52) circle (  2.13);

\path[fill=fillColor,fill opacity=0.20] (206.50, 75.87) circle (  2.13);

\path[fill=fillColor,fill opacity=0.20] (217.53, 80.93) circle (  2.13);

\path[fill=fillColor,fill opacity=0.20] (218.54, 76.63) circle (  2.13);

\path[fill=fillColor,fill opacity=0.20] (215.53, 73.34) circle (  2.13);

\path[fill=fillColor,fill opacity=0.20] (218.54, 72.84) circle (  2.13);

\path[fill=fillColor,fill opacity=0.20] (219.54, 66.90) circle (  2.13);

\path[fill=fillColor,fill opacity=0.20] (223.55, 61.84) circle (  2.13);

\path[fill=fillColor,fill opacity=0.20] (228.57, 60.70) circle (  2.13);

\path[fill=fillColor,fill opacity=0.20] (211.52, 63.86) circle (  2.13);

\path[fill=fillColor,fill opacity=0.20] (208.51, 69.42) circle (  2.13);

\path[fill=fillColor,fill opacity=0.20] (203.49, 71.70) circle (  2.13);

\path[fill=fillColor,fill opacity=0.20] (200.48, 69.93) circle (  2.13);

\path[fill=fillColor,fill opacity=0.20] (199.48, 71.83) circle (  2.13);

\path[fill=fillColor,fill opacity=0.20] (200.48, 82.95) circle (  2.13);

\path[fill=fillColor,fill opacity=0.20] (202.49, 93.07) circle (  2.13);

\path[fill=fillColor,fill opacity=0.20] (203.49, 91.68) circle (  2.13);

\path[fill=fillColor,fill opacity=0.20] (199.48, 87.38) circle (  2.13);

\path[fill=fillColor,fill opacity=0.20] (196.47, 86.37) circle (  2.13);

\path[fill=fillColor,fill opacity=0.20] (198.47, 87.51) circle (  2.13);

\path[fill=fillColor,fill opacity=0.20] (207.50, 87.76) circle (  2.13);

\path[fill=fillColor,fill opacity=0.20] (211.52, 81.31) circle (  2.13);

\path[fill=fillColor,fill opacity=0.20] (224.56, 68.41) circle (  2.13);

\path[fill=fillColor,fill opacity=0.20] (220.54, 51.72) circle (  2.13);

\path[fill=fillColor,fill opacity=0.20] (206.50, 73.98) circle (  2.13);

\path[fill=fillColor,fill opacity=0.20] (202.49, 77.26) circle (  2.13);

\path[fill=fillColor,fill opacity=0.20] (210.51, 76.38) circle (  2.13);

\path[fill=fillColor,fill opacity=0.20] (215.53, 75.24) circle (  2.13);

\path[fill=fillColor,fill opacity=0.20] (213.52, 72.71) circle (  2.13);

\path[fill=fillColor,fill opacity=0.20] (219.54, 69.80) circle (  2.13);

\path[fill=fillColor,fill opacity=0.20] (221.55, 67.53) circle (  2.13);

\path[fill=fillColor,fill opacity=0.20] (219.54, 66.14) circle (  2.13);

\path[fill=fillColor,fill opacity=0.20] (218.54, 66.52) circle (  2.13);

\path[fill=fillColor,fill opacity=0.20] (223.55, 66.39) circle (  2.13);

\path[fill=fillColor,fill opacity=0.20] (238.60, 58.93) circle (  2.13);

\path[fill=fillColor,fill opacity=0.20] (213.52, 57.03) circle (  2.13);

\path[fill=fillColor,fill opacity=0.20] (182.02, 64.49) circle (  2.13);

\path[fill=fillColor,fill opacity=0.20] (201.48, 67.40) circle (  2.13);

\path[fill=fillColor,fill opacity=0.20] (202.49, 66.39) circle (  2.13);

\path[fill=fillColor,fill opacity=0.20] (201.48, 71.45) circle (  2.13);

\path[fill=fillColor,fill opacity=0.20] (203.49, 77.77) circle (  2.13);

\path[fill=fillColor,fill opacity=0.20] (201.48, 80.68) circle (  2.13);

\path[fill=fillColor,fill opacity=0.20] (200.48, 86.62) circle (  2.13);

\path[fill=fillColor,fill opacity=0.20] (201.48, 92.82) circle (  2.13);

\path[fill=fillColor,fill opacity=0.20] (203.49, 91.17) circle (  2.13);

\path[fill=fillColor,fill opacity=0.20] (205.50, 86.37) circle (  2.13);

\path[fill=fillColor,fill opacity=0.20] (208.51, 80.30) circle (  2.13);

\path[fill=fillColor,fill opacity=0.20] (207.50, 72.59) circle (  2.13);

\path[fill=fillColor,fill opacity=0.20] (209.51, 66.26) circle (  2.13);

\path[fill=fillColor,fill opacity=0.20] (226.56, 56.15) circle (  2.13);

\path[fill=fillColor,fill opacity=0.20] (210.51, 50.33) circle (  2.13);

\path[fill=fillColor,fill opacity=0.20] (201.48, 60.83) circle (  2.13);

\path[fill=fillColor,fill opacity=0.20] (211.52, 64.62) circle (  2.13);

\path[fill=fillColor,fill opacity=0.20] (211.52, 71.32) circle (  2.13);

\path[fill=fillColor,fill opacity=0.20] (215.53, 74.74) circle (  2.13);

\path[fill=fillColor,fill opacity=0.20] (219.54, 73.09) circle (  2.13);

\path[fill=fillColor,fill opacity=0.20] (215.53, 71.07) circle (  2.13);

\path[fill=fillColor,fill opacity=0.20] (218.54, 65.38) circle (  2.13);

\path[fill=fillColor,fill opacity=0.20] (219.54, 61.84) circle (  2.13);

\path[fill=fillColor,fill opacity=0.20] (216.53, 63.10) circle (  2.13);

\path[fill=fillColor,fill opacity=0.20] (218.54, 65.38) circle (  2.13);

\path[fill=fillColor,fill opacity=0.20] (216.53, 63.23) circle (  2.13);

\path[fill=fillColor,fill opacity=0.20] (219.54, 62.85) circle (  2.13);

\path[fill=fillColor,fill opacity=0.20] (216.53, 67.65) circle (  2.13);

\path[fill=fillColor,fill opacity=0.20] (225.56, 69.42) circle (  2.13);

\path[fill=fillColor,fill opacity=0.20] (223.55, 63.99) circle (  2.13);

\path[fill=fillColor,fill opacity=0.20] (229.57, 60.83) circle (  2.13);

\path[fill=fillColor,fill opacity=0.20] (227.57, 61.46) circle (  2.13);

\path[fill=fillColor,fill opacity=0.20] (210.51, 59.06) circle (  2.13);

\path[fill=fillColor,fill opacity=0.20] (224.56, 54.38) circle (  2.13);

\path[fill=fillColor,fill opacity=0.20] (224.56, 53.62) circle (  2.13);

\path[fill=fillColor,fill opacity=0.20] (219.54, 54.25) circle (  2.13);

\path[fill=fillColor,fill opacity=0.20] (217.53, 52.23) circle (  2.13);

\path[fill=fillColor,fill opacity=0.20] (221.55, 52.73) circle (  2.13);

\path[fill=fillColor,fill opacity=0.20] (216.53, 58.55) circle (  2.13);

\path[fill=fillColor,fill opacity=0.20] (207.50, 60.45) circle (  2.13);

\path[fill=fillColor,fill opacity=0.20] (213.52, 56.91) circle (  2.13);

\path[fill=fillColor,fill opacity=0.20] (213.52, 57.79) circle (  2.13);

\path[fill=fillColor,fill opacity=0.20] (214.53, 60.45) circle (  2.13);

\path[fill=fillColor,fill opacity=0.20] (211.52, 59.81) circle (  2.13);

\path[fill=fillColor,fill opacity=0.20] (206.50, 60.95) circle (  2.13);

\path[fill=fillColor,fill opacity=0.20] (207.50, 62.09) circle (  2.13);

\path[fill=fillColor,fill opacity=0.20] (207.50, 62.22) circle (  2.13);

\path[fill=fillColor,fill opacity=0.20] (201.48, 67.15) circle (  2.13);

\path[fill=fillColor,fill opacity=0.20] (200.48, 73.60) circle (  2.13);

\path[fill=fillColor,fill opacity=0.20] (200.48, 77.52) circle (  2.13);

\path[fill=fillColor,fill opacity=0.20] (203.49, 83.08) circle (  2.13);

\path[fill=fillColor,fill opacity=0.20] (203.49, 87.25) circle (  2.13);

\path[fill=fillColor,fill opacity=0.20] (202.49, 88.52) circle (  2.13);

\path[fill=fillColor,fill opacity=0.20] (199.48, 89.02) circle (  2.13);

\path[fill=fillColor,fill opacity=0.20] (210.51, 86.12) circle (  2.13);

\path[fill=fillColor,fill opacity=0.20] (209.51, 81.06) circle (  2.13);

\path[fill=fillColor,fill opacity=0.20] (212.52, 74.36) circle (  2.13);

\path[fill=fillColor,fill opacity=0.20] (220.54, 61.33) circle (  2.13);

\path[fill=fillColor,fill opacity=0.20] (220.54, 48.43) circle (  2.13);

\path[fill=fillColor,fill opacity=0.20] (213.52, 50.08) circle (  2.13);

\path[fill=fillColor,fill opacity=0.20] (206.50, 61.08) circle (  2.13);

\path[fill=fillColor,fill opacity=0.20] (214.53, 70.69) circle (  2.13);

\path[fill=fillColor,fill opacity=0.20] (213.52, 74.61) circle (  2.13);

\path[fill=fillColor,fill opacity=0.20] (211.52, 73.60) circle (  2.13);

\path[fill=fillColor,fill opacity=0.20] (210.51, 68.41) circle (  2.13);

\path[fill=fillColor,fill opacity=0.20] (212.52, 64.37) circle (  2.13);

\path[fill=fillColor,fill opacity=0.20] (211.52, 65.50) circle (  2.13);

\path[fill=fillColor,fill opacity=0.20] (212.52, 66.64) circle (  2.13);

\path[fill=fillColor,fill opacity=0.20] (213.52, 64.87) circle (  2.13);

\path[fill=fillColor,fill opacity=0.20] (213.52, 64.24) circle (  2.13);

\path[fill=fillColor,fill opacity=0.20] (212.52, 69.93) circle (  2.13);

\path[fill=fillColor,fill opacity=0.20] (205.50, 77.14) circle (  2.13);

\path[fill=fillColor,fill opacity=0.20] (215.53, 76.38) circle (  2.13);

\path[fill=fillColor,fill opacity=0.20] (217.53, 74.99) circle (  2.13);

\path[fill=fillColor,fill opacity=0.20] (214.53, 77.01) circle (  2.13);

\path[fill=fillColor,fill opacity=0.20] (213.52, 75.12) circle (  2.13);

\path[fill=fillColor,fill opacity=0.20] (214.53, 68.79) circle (  2.13);

\path[fill=fillColor,fill opacity=0.20] (213.52, 66.26) circle (  2.13);

\path[fill=fillColor,fill opacity=0.20] (208.51, 70.82) circle (  2.13);

\path[fill=fillColor,fill opacity=0.20] (215.53, 77.39) circle (  2.13);

\path[fill=fillColor,fill opacity=0.20] (217.53, 74.61) circle (  2.13);

\path[fill=fillColor,fill opacity=0.20] (213.52, 69.17) circle (  2.13);

\path[fill=fillColor,fill opacity=0.20] (211.52, 70.69) circle (  2.13);

\path[fill=fillColor,fill opacity=0.20] (209.51, 72.46) circle (  2.13);

\path[fill=fillColor,fill opacity=0.20] (213.52, 71.57) circle (  2.13);

\path[fill=fillColor,fill opacity=0.20] (209.51, 74.74) circle (  2.13);

\path[fill=fillColor,fill opacity=0.20] (208.51, 73.85) circle (  2.13);

\path[fill=fillColor,fill opacity=0.20] (208.51, 69.05) circle (  2.13);

\path[fill=fillColor,fill opacity=0.20] (207.50, 72.46) circle (  2.13);

\path[fill=fillColor,fill opacity=0.20] (205.50, 76.76) circle (  2.13);

\path[fill=fillColor,fill opacity=0.20] (206.50, 72.97) circle (  2.13);

\path[fill=fillColor,fill opacity=0.20] (205.50, 73.09) circle (  2.13);

\path[fill=fillColor,fill opacity=0.20] (204.49, 80.05) circle (  2.13);

\path[fill=fillColor,fill opacity=0.20] (205.50, 83.97) circle (  2.13);

\path[fill=fillColor,fill opacity=0.20] (205.50, 79.54) circle (  2.13);

\path[fill=fillColor,fill opacity=0.20] (206.50, 72.08) circle (  2.13);

\path[fill=fillColor,fill opacity=0.20] (210.51, 67.91) circle (  2.13);

\path[fill=fillColor,fill opacity=0.20] (211.52, 64.87) circle (  2.13);

\path[fill=fillColor,fill opacity=0.20] (220.54, 59.06) circle (  2.13);

\path[fill=fillColor,fill opacity=0.20] (221.55, 53.11) circle (  2.13);

\path[fill=fillColor,fill opacity=0.20] (224.56, 48.56) circle (  2.13);

\path[fill=fillColor,fill opacity=0.20] (213.52, 64.87) circle (  2.13);

\path[fill=fillColor,fill opacity=0.20] (205.50, 68.67) circle (  2.13);

\path[fill=fillColor,fill opacity=0.20] (210.51, 69.30) circle (  2.13);

\path[fill=fillColor,fill opacity=0.20] (205.50, 69.55) circle (  2.13);

\path[fill=fillColor,fill opacity=0.20] (211.52, 71.45) circle (  2.13);

\path[fill=fillColor,fill opacity=0.20] (213.52, 76.25) circle (  2.13);

\path[fill=fillColor,fill opacity=0.20] (215.53, 75.24) circle (  2.13);

\path[fill=fillColor,fill opacity=0.20] (210.51, 71.20) circle (  2.13);

\path[fill=fillColor,fill opacity=0.20] (216.53, 70.18) circle (  2.13);

\path[fill=fillColor,fill opacity=0.20] (206.50, 69.05) circle (  2.13);

\path[fill=fillColor,fill opacity=0.20] (213.52, 69.42) circle (  2.13);

\path[fill=fillColor,fill opacity=0.20] (211.52, 72.71) circle (  2.13);

\path[fill=fillColor,fill opacity=0.20] (216.53, 77.01) circle (  2.13);

\path[fill=fillColor,fill opacity=0.20] (213.52, 77.77) circle (  2.13);

\path[fill=fillColor,fill opacity=0.20] (212.52, 72.84) circle (  2.13);

\path[fill=fillColor,fill opacity=0.20] (211.52, 69.30) circle (  2.13);

\path[fill=fillColor,fill opacity=0.20] (213.52, 72.33) circle (  2.13);

\path[fill=fillColor,fill opacity=0.20] (215.53, 75.87) circle (  2.13);

\path[fill=fillColor,fill opacity=0.20] (206.50, 77.77) circle (  2.13);

\path[fill=fillColor,fill opacity=0.20] (213.52, 77.26) circle (  2.13);

\path[fill=fillColor,fill opacity=0.20] (213.52, 73.34) circle (  2.13);

\path[fill=fillColor,fill opacity=0.20] (213.52, 72.84) circle (  2.13);

\path[fill=fillColor,fill opacity=0.20] (214.53, 77.26) circle (  2.13);

\path[fill=fillColor,fill opacity=0.20] (211.52, 81.94) circle (  2.13);

\path[fill=fillColor,fill opacity=0.20] (209.51, 83.46) circle (  2.13);

\path[fill=fillColor,fill opacity=0.20] (209.51, 79.92) circle (  2.13);

\path[fill=fillColor,fill opacity=0.20] (209.51, 74.61) circle (  2.13);

\path[fill=fillColor,fill opacity=0.20] (209.51, 72.97) circle (  2.13);

\path[fill=fillColor,fill opacity=0.20] (212.52, 71.45) circle (  2.13);

\path[fill=fillColor,fill opacity=0.20] (215.53, 65.88) circle (  2.13);

\path[fill=fillColor,fill opacity=0.20] (216.53, 61.84) circle (  2.13);

\path[fill=fillColor,fill opacity=0.20] (214.53, 60.32) circle (  2.13);

\path[fill=fillColor,fill opacity=0.20] (213.52, 55.89) circle (  2.13);

\path[fill=fillColor,fill opacity=0.20] (216.53, 49.07) circle (  2.13);

\path[fill=fillColor,fill opacity=0.20] (223.55, 44.39) circle (  2.13);

\path[fill=fillColor,fill opacity=0.20] (230.58, 42.36) circle (  2.13);

\path[fill=fillColor,fill opacity=0.20] (215.53, 53.24) circle (  2.13);

\path[fill=fillColor,fill opacity=0.20] (213.52, 57.79) circle (  2.13);

\path[fill=fillColor,fill opacity=0.20] (211.52, 58.68) circle (  2.13);

\path[fill=fillColor,fill opacity=0.20] (208.51, 63.86) circle (  2.13);

\path[fill=fillColor,fill opacity=0.20] (214.53, 76.89) circle (  2.13);

\path[fill=fillColor,fill opacity=0.20] (214.53, 81.94) circle (  2.13);

\path[fill=fillColor,fill opacity=0.20] (216.53, 78.28) circle (  2.13);

\path[fill=fillColor,fill opacity=0.20] (211.52, 74.36) circle (  2.13);

\path[fill=fillColor,fill opacity=0.20] (213.52, 66.26) circle (  2.13);

\path[fill=fillColor,fill opacity=0.20] (211.52, 65.13) circle (  2.13);

\path[fill=fillColor,fill opacity=0.20] (216.53, 72.21) circle (  2.13);

\path[fill=fillColor,fill opacity=0.20] (216.53, 78.40) circle (  2.13);

\path[fill=fillColor,fill opacity=0.20] (215.53, 77.39) circle (  2.13);

\path[fill=fillColor,fill opacity=0.20] (214.53, 72.84) circle (  2.13);

\path[fill=fillColor,fill opacity=0.20] (208.51, 70.94) circle (  2.13);

\path[fill=fillColor,fill opacity=0.20] (217.53, 75.87) circle (  2.13);

\path[fill=fillColor,fill opacity=0.20] (217.53, 77.90) circle (  2.13);

\path[fill=fillColor,fill opacity=0.20] (213.52, 73.60) circle (  2.13);

\path[fill=fillColor,fill opacity=0.20] (213.52, 70.18) circle (  2.13);

\path[fill=fillColor,fill opacity=0.20] (210.51, 68.29) circle (  2.13);

\path[fill=fillColor,fill opacity=0.20] (213.52, 68.67) circle (  2.13);

\path[fill=fillColor,fill opacity=0.20] (214.53, 71.45) circle (  2.13);

\path[fill=fillColor,fill opacity=0.20] (210.51, 72.21) circle (  2.13);

\path[fill=fillColor,fill opacity=0.20] (214.53, 67.15) circle (  2.13);

\path[fill=fillColor,fill opacity=0.20] (215.53, 61.46) circle (  2.13);

\path[fill=fillColor,fill opacity=0.20] (218.54, 56.15) circle (  2.13);

\path[fill=fillColor,fill opacity=0.20] (222.55, 43.63) circle (  2.13);

\path[fill=fillColor,fill opacity=0.20] (217.53, 52.99) circle (  2.13);

\path[fill=fillColor,fill opacity=0.20] (212.52, 58.30) circle (  2.13);

\path[fill=fillColor,fill opacity=0.20] (215.53, 61.08) circle (  2.13);

\path[fill=fillColor,fill opacity=0.20] (209.51, 61.21) circle (  2.13);

\path[fill=fillColor,fill opacity=0.20] (212.52, 58.30) circle (  2.13);

\path[fill=fillColor,fill opacity=0.20] (211.52, 60.07) circle (  2.13);

\path[fill=fillColor,fill opacity=0.20] (214.53, 65.38) circle (  2.13);

\path[fill=fillColor,fill opacity=0.20] (216.53, 67.53) circle (  2.13);

\path[fill=fillColor,fill opacity=0.20] (218.54, 67.28) circle (  2.13);

\path[fill=fillColor,fill opacity=0.20] (217.53, 64.62) circle (  2.13);

\path[fill=fillColor,fill opacity=0.20] (220.54, 61.96) circle (  2.13);

\path[fill=fillColor,fill opacity=0.20] (218.54, 63.61) circle (  2.13);

\path[fill=fillColor,fill opacity=0.20] (217.53, 63.86) circle (  2.13);

\path[fill=fillColor,fill opacity=0.20] (215.53, 58.55) circle (  2.13);

\path[fill=fillColor,fill opacity=0.20] (221.55, 52.99) circle (  2.13);

\path[fill=fillColor,fill opacity=0.20] (213.52, 49.57) circle (  2.13);

\path[fill=fillColor,fill opacity=0.20] (221.55, 46.03) circle (  2.13);

\path[fill=fillColor,fill opacity=0.20] (212.52, 41.61) circle (  2.13);

\path[fill=fillColor,fill opacity=0.20] (221.55, 42.24) circle (  2.13);

\path[fill=fillColor,fill opacity=0.20] (222.55, 45.27) circle (  2.13);

\path[fill=fillColor,fill opacity=0.20] (221.55, 40.72) circle (  2.13);

\path[fill=fillColor,fill opacity=0.20] (224.56, 40.09) circle (  2.13);

\path[fill=fillColor,fill opacity=0.20] (178.11, 89.78) circle (  2.13);

\path[fill=fillColor,fill opacity=0.20] (185.23, 79.79) circle (  2.13);

\path[fill=fillColor,fill opacity=0.20] (189.44, 94.34) circle (  2.13);

\path[fill=fillColor,fill opacity=0.20] (188.44, 85.86) circle (  2.13);

\path[fill=fillColor,fill opacity=0.20] (187.84, 92.19) circle (  2.13);

\path[fill=fillColor,fill opacity=0.20] (188.44, 93.83) circle (  2.13);

\path[fill=fillColor,fill opacity=0.20] (189.44, 93.45) circle (  2.13);

\path[fill=fillColor,fill opacity=0.20] (187.34,101.67) circle (  2.13);

\path[fill=fillColor,fill opacity=0.20] (194.46, 94.97) circle (  2.13);

\path[fill=fillColor,fill opacity=0.20] (184.53,110.52) circle (  2.13);

\path[fill=fillColor,fill opacity=0.20] (189.44,107.87) circle (  2.13);

\path[fill=fillColor,fill opacity=0.20] (187.94,103.69) circle (  2.13);

\path[fill=fillColor,fill opacity=0.20] (183.83,110.65) circle (  2.13);

\path[fill=fillColor,fill opacity=0.20] (184.03,106.85) circle (  2.13);

\path[fill=fillColor,fill opacity=0.20] (181.32,102.30) circle (  2.13);

\path[fill=fillColor,fill opacity=0.20] (180.62,101.80) circle (  2.13);

\path[fill=fillColor,fill opacity=0.20] (190.45, 92.31) circle (  2.13);

\path[fill=fillColor,fill opacity=0.20] (204.49, 76.38) circle (  2.13);

\path[fill=fillColor,fill opacity=0.20] (175.00, 97.50) circle (  2.13);

\path[fill=fillColor,fill opacity=0.20] (179.61,103.95) circle (  2.13);

\path[fill=fillColor,fill opacity=0.20] (183.32,115.33) circle (  2.13);

\path[fill=fillColor,fill opacity=0.20] (182.22,111.28) circle (  2.13);

\path[fill=fillColor,fill opacity=0.20] (175.20,110.02) circle (  2.13);

\path[fill=fillColor,fill opacity=0.20] (183.43, 94.71) circle (  2.13);

\path[fill=fillColor,fill opacity=0.20] (205.50, 83.46) circle (  2.13);

\path[fill=fillColor,fill opacity=0.20] (185.73,106.22) circle (  2.13);

\path[fill=fillColor,fill opacity=0.20] (172.49,107.36) circle (  2.13);

\path[fill=fillColor,fill opacity=0.20] (171.19,111.41) circle (  2.13);

\path[fill=fillColor,fill opacity=0.20] (167.37,110.27) circle (  2.13);

\path[fill=fillColor,fill opacity=0.20] (169.48,107.11) circle (  2.13);

\path[fill=fillColor,fill opacity=0.20] (167.37,102.30) circle (  2.13);

\path[fill=fillColor,fill opacity=0.20] (184.83, 99.90) circle (  2.13);

\path[fill=fillColor,fill opacity=0.20] (195.46,101.67) circle (  2.13);

\path[fill=fillColor,fill opacity=0.20] (186.84,112.29) circle (  2.13);

\path[fill=fillColor,fill opacity=0.20] (175.60,106.98) circle (  2.13);

\path[fill=fillColor,fill opacity=0.20] (180.42,106.47) circle (  2.13);

\path[fill=fillColor,fill opacity=0.20] (182.32,105.97) circle (  2.13);

\path[fill=fillColor,fill opacity=0.20] (178.01,106.47) circle (  2.13);

\path[fill=fillColor,fill opacity=0.20] (177.21,105.46) circle (  2.13);

\path[fill=fillColor,fill opacity=0.20] (178.81,104.32) circle (  2.13);

\path[fill=fillColor,fill opacity=0.20] (195.46,101.92) circle (  2.13);

\path[fill=fillColor,fill opacity=0.20] (210.51, 84.73) circle (  2.13);

\path[fill=fillColor,fill opacity=0.20] (192.45, 46.03) circle (  2.13);

\path[fill=fillColor,fill opacity=0.20] (193.46, 58.80) circle (  2.13);

\path[fill=fillColor,fill opacity=0.20] (201.48, 59.56) circle (  2.13);

\path[fill=fillColor,fill opacity=0.20] (210.51, 65.38) circle (  2.13);

\path[fill=fillColor,fill opacity=0.20] (214.53, 66.52) circle (  2.13);

\path[fill=fillColor,fill opacity=0.20] (212.52, 58.55) circle (  2.13);

\path[fill=fillColor,fill opacity=0.20] (213.52, 50.96) circle (  2.13);

\path[fill=fillColor,fill opacity=0.20] (233.59, 50.71) circle (  2.13);

\path[fill=fillColor,fill opacity=0.20] (198.47, 92.69) circle (  2.13);

\path[fill=fillColor,fill opacity=0.20] (187.94,100.91) circle (  2.13);

\path[fill=fillColor,fill opacity=0.20] (186.13,100.78) circle (  2.13);

\path[fill=fillColor,fill opacity=0.20] (186.33,103.31) circle (  2.13);

\path[fill=fillColor,fill opacity=0.20] (185.43,105.59) circle (  2.13);

\path[fill=fillColor,fill opacity=0.20] (181.52,103.69) circle (  2.13);

\path[fill=fillColor,fill opacity=0.20] (179.21,106.85) circle (  2.13);

\path[fill=fillColor,fill opacity=0.20] (183.93,109.64) circle (  2.13);

\path[fill=fillColor,fill opacity=0.20] (205.50,100.40) circle (  2.13);

\path[fill=fillColor,fill opacity=0.20] (247.63, 82.20) circle (  2.13);

\path[fill=fillColor,fill opacity=0.20] (196.47, 57.54) circle (  2.13);

\path[fill=fillColor,fill opacity=0.20] (193.46, 75.87) circle (  2.13);

\path[fill=fillColor,fill opacity=0.20] (191.45, 83.21) circle (  2.13);

\path[fill=fillColor,fill opacity=0.20] (200.48, 79.67) circle (  2.13);

\path[fill=fillColor,fill opacity=0.20] (208.51, 79.16) circle (  2.13);

\path[fill=fillColor,fill opacity=0.20] (202.49, 81.94) circle (  2.13);

\path[fill=fillColor,fill opacity=0.20] (203.49, 72.97) circle (  2.13);

\path[fill=fillColor,fill opacity=0.20] (224.56, 57.92) circle (  2.13);

\path[fill=fillColor,fill opacity=0.20] (201.48, 92.82) circle (  2.13);

\path[fill=fillColor,fill opacity=0.20] (190.45, 93.96) circle (  2.13);

\path[fill=fillColor,fill opacity=0.20] (177.21, 98.26) circle (  2.13);

\path[fill=fillColor,fill opacity=0.20] (181.32,105.84) circle (  2.13);

\path[fill=fillColor,fill opacity=0.20] (183.53,108.62) circle (  2.13);

\path[fill=fillColor,fill opacity=0.20] (182.92,110.02) circle (  2.13);

\path[fill=fillColor,fill opacity=0.20] (183.12,107.61) circle (  2.13);

\path[fill=fillColor,fill opacity=0.20] (195.46,102.05) circle (  2.13);

\path[fill=fillColor,fill opacity=0.20] (207.50, 88.52) circle (  2.13);

\path[fill=fillColor,fill opacity=0.20] (199.48, 67.65) circle (  2.13);

\path[fill=fillColor,fill opacity=0.20] (195.46, 89.15) circle (  2.13);

\path[fill=fillColor,fill opacity=0.20] (195.46, 85.23) circle (  2.13);

\path[fill=fillColor,fill opacity=0.20] (198.47, 87.00) circle (  2.13);

\path[fill=fillColor,fill opacity=0.20] (198.47, 85.99) circle (  2.13);

\path[fill=fillColor,fill opacity=0.20] (204.49, 84.47) circle (  2.13);

\path[fill=fillColor,fill opacity=0.20] (206.50, 89.15) circle (  2.13);

\path[fill=fillColor,fill opacity=0.20] (204.49, 83.59) circle (  2.13);

\path[fill=fillColor,fill opacity=0.20] (209.51, 70.06) circle (  2.13);

\path[fill=fillColor,fill opacity=0.20] (192.45, 92.44) circle (  2.13);

\path[fill=fillColor,fill opacity=0.20] (194.46,110.65) circle (  2.13);

\path[fill=fillColor,fill opacity=0.20] (184.63,101.29) circle (  2.13);

\path[fill=fillColor,fill opacity=0.20] (167.57, 96.74) circle (  2.13);

\path[fill=fillColor,fill opacity=0.20] (165.37,106.60) circle (  2.13);

\path[fill=fillColor,fill opacity=0.20] (179.31,112.67) circle (  2.13);

\path[fill=fillColor,fill opacity=0.20] (172.79,109.00) circle (  2.13);

\path[fill=fillColor,fill opacity=0.20] (181.32,108.24) circle (  2.13);

\path[fill=fillColor,fill opacity=0.20] (185.83,109.26) circle (  2.13);

\path[fill=fillColor,fill opacity=0.20] (208.51, 98.63) circle (  2.13);

\path[fill=fillColor,fill opacity=0.20] (206.50, 51.85) circle (  2.13);

\path[fill=fillColor,fill opacity=0.20] (198.47, 83.71) circle (  2.13);

\path[fill=fillColor,fill opacity=0.20] (187.14, 96.36) circle (  2.13);

\path[fill=fillColor,fill opacity=0.20] (193.46, 99.01) circle (  2.13);

\path[fill=fillColor,fill opacity=0.20] (200.48, 96.74) circle (  2.13);

\path[fill=fillColor,fill opacity=0.20] (203.49, 89.53) circle (  2.13);

\path[fill=fillColor,fill opacity=0.20] (206.50, 88.52) circle (  2.13);

\path[fill=fillColor,fill opacity=0.20] (208.51, 92.44) circle (  2.13);

\path[fill=fillColor,fill opacity=0.20] (211.52, 87.25) circle (  2.13);

\path[fill=fillColor,fill opacity=0.20] (219.54, 77.77) circle (  2.13);

\path[fill=fillColor,fill opacity=0.20] (179.91, 86.62) circle (  2.13);

\path[fill=fillColor,fill opacity=0.20] (190.45, 91.68) circle (  2.13);

\path[fill=fillColor,fill opacity=0.20] (189.44, 98.63) circle (  2.13);

\path[fill=fillColor,fill opacity=0.20] (184.53, 94.71) circle (  2.13);

\path[fill=fillColor,fill opacity=0.20] (181.62,102.43) circle (  2.13);

\path[fill=fillColor,fill opacity=0.20] (180.62,110.52) circle (  2.13);

\path[fill=fillColor,fill opacity=0.20] (177.00,102.18) circle (  2.13);

\path[fill=fillColor,fill opacity=0.20] (195.46,107.99) circle (  2.13);

\path[fill=fillColor,fill opacity=0.20] (228.57, 98.00) circle (  2.13);

\path[fill=fillColor,fill opacity=0.20] (213.52, 56.78) circle (  2.13);

\path[fill=fillColor,fill opacity=0.20] (197.47, 85.74) circle (  2.13);

\path[fill=fillColor,fill opacity=0.20] (191.45, 94.46) circle (  2.13);

\path[fill=fillColor,fill opacity=0.20] (198.47,104.96) circle (  2.13);

\path[fill=fillColor,fill opacity=0.20] (202.49,107.49) circle (  2.13);

\path[fill=fillColor,fill opacity=0.20] (208.51, 97.62) circle (  2.13);

\path[fill=fillColor,fill opacity=0.20] (213.52, 96.61) circle (  2.13);

\path[fill=fillColor,fill opacity=0.20] (215.53, 98.38) circle (  2.13);

\path[fill=fillColor,fill opacity=0.20] (211.52, 90.67) circle (  2.13);

\path[fill=fillColor,fill opacity=0.20] (243.62, 80.93) circle (  2.13);

\path[fill=fillColor,fill opacity=0.20] (198.47, 77.77) circle (  2.13);

\path[fill=fillColor,fill opacity=0.20] (194.46, 90.29) circle (  2.13);

\path[fill=fillColor,fill opacity=0.20] (191.45, 86.62) circle (  2.13);

\path[fill=fillColor,fill opacity=0.20] (186.94, 85.99) circle (  2.13);

\path[fill=fillColor,fill opacity=0.20] (189.44, 94.34) circle (  2.13);

\path[fill=fillColor,fill opacity=0.20] (188.34,103.95) circle (  2.13);

\path[fill=fillColor,fill opacity=0.20] (181.32,107.11) circle (  2.13);

\path[fill=fillColor,fill opacity=0.20] (183.32,105.08) circle (  2.13);

\path[fill=fillColor,fill opacity=0.20] (184.33, 97.88) circle (  2.13);

\path[fill=fillColor,fill opacity=0.20] (180.42, 95.09) circle (  2.13);

\path[fill=fillColor,fill opacity=0.20] (209.51, 94.71) circle (  2.13);

\path[fill=fillColor,fill opacity=0.20] (271.71, 80.81) circle (  2.13);

\path[fill=fillColor,fill opacity=0.20] (220.54, 51.72) circle (  2.13);

\path[fill=fillColor,fill opacity=0.20] (203.49, 84.60) circle (  2.13);

\path[fill=fillColor,fill opacity=0.20] (192.45, 93.20) circle (  2.13);

\path[fill=fillColor,fill opacity=0.20] (196.47, 98.76) circle (  2.13);

\path[fill=fillColor,fill opacity=0.20] (203.49,107.11) circle (  2.13);

\path[fill=fillColor,fill opacity=0.20] (201.48,103.95) circle (  2.13);

\path[fill=fillColor,fill opacity=0.20] (207.50,102.68) circle (  2.13);

\path[fill=fillColor,fill opacity=0.20] (213.52,105.08) circle (  2.13);

\path[fill=fillColor,fill opacity=0.20] (227.57, 93.07) circle (  2.13);

\path[fill=fillColor,fill opacity=0.20] (200.48, 88.01) circle (  2.13);

\path[fill=fillColor,fill opacity=0.20] (191.45, 91.68) circle (  2.13);

\path[fill=fillColor,fill opacity=0.20] (191.45, 96.49) circle (  2.13);

\path[fill=fillColor,fill opacity=0.20] (188.34, 87.63) circle (  2.13);

\path[fill=fillColor,fill opacity=0.20] (187.14, 89.91) circle (  2.13);

\path[fill=fillColor,fill opacity=0.20] (189.44,104.83) circle (  2.13);

\path[fill=fillColor,fill opacity=0.20] (187.44,106.98) circle (  2.13);

\path[fill=fillColor,fill opacity=0.20] (188.24,101.67) circle (  2.13);

\path[fill=fillColor,fill opacity=0.20] (190.45, 98.51) circle (  2.13);

\path[fill=fillColor,fill opacity=0.20] (213.52, 91.43) circle (  2.13);

\path[fill=fillColor,fill opacity=0.20] (260.67, 78.91) circle (  2.13);

\path[fill=fillColor,fill opacity=0.20] (215.53, 77.01) circle (  2.13);

\path[fill=fillColor,fill opacity=0.20] (190.45, 94.08) circle (  2.13);

\path[fill=fillColor,fill opacity=0.20] (188.44, 96.99) circle (  2.13);

\path[fill=fillColor,fill opacity=0.20] (192.45,104.83) circle (  2.13);

\path[fill=fillColor,fill opacity=0.20] (192.45,106.22) circle (  2.13);

\path[fill=fillColor,fill opacity=0.20] (190.45, 98.63) circle (  2.13);

\path[fill=fillColor,fill opacity=0.20] (200.48, 97.75) circle (  2.13);

\path[fill=fillColor,fill opacity=0.20] (211.52, 94.97) circle (  2.13);

\path[fill=fillColor,fill opacity=0.20] (196.47, 84.47) circle (  2.13);

\path[fill=fillColor,fill opacity=0.20] (181.32, 89.53) circle (  2.13);

\path[fill=fillColor,fill opacity=0.20] (182.62,100.78) circle (  2.13);

\path[fill=fillColor,fill opacity=0.20] (189.44, 98.63) circle (  2.13);

\path[fill=fillColor,fill opacity=0.20] (189.44, 90.42) circle (  2.13);

\path[fill=fillColor,fill opacity=0.20] (187.74, 88.27) circle (  2.13);

\path[fill=fillColor,fill opacity=0.20] (191.45, 94.71) circle (  2.13);

\path[fill=fillColor,fill opacity=0.20] (195.46, 98.76) circle (  2.13);

\path[fill=fillColor,fill opacity=0.20] (203.49,102.05) circle (  2.13);

\path[fill=fillColor,fill opacity=0.20] (224.56,103.57) circle (  2.13);

\path[fill=fillColor,fill opacity=0.20] (202.49, 87.63) circle (  2.13);

\path[fill=fillColor,fill opacity=0.20] (194.46, 96.99) circle (  2.13);

\path[fill=fillColor,fill opacity=0.20] (194.46,105.08) circle (  2.13);

\path[fill=fillColor,fill opacity=0.20] (185.83,107.61) circle (  2.13);

\path[fill=fillColor,fill opacity=0.20] (186.64, 96.11) circle (  2.13);

\path[fill=fillColor,fill opacity=0.20] (195.46, 87.76) circle (  2.13);

\path[fill=fillColor,fill opacity=0.20] (201.48, 92.06) circle (  2.13);

\path[fill=fillColor,fill opacity=0.20] (209.51, 76.63) circle (  2.13);

\path[fill=fillColor,fill opacity=0.20] (199.48, 84.60) circle (  2.13);

\path[fill=fillColor,fill opacity=0.20] (183.43, 87.00) circle (  2.13);

\path[fill=fillColor,fill opacity=0.20] (185.23, 92.06) circle (  2.13);

\path[fill=fillColor,fill opacity=0.20] (190.45, 96.11) circle (  2.13);

\path[fill=fillColor,fill opacity=0.20] (191.45, 99.52) circle (  2.13);

\path[fill=fillColor,fill opacity=0.20] (193.46, 94.21) circle (  2.13);

\path[fill=fillColor,fill opacity=0.20] (206.50, 92.44) circle (  2.13);

\path[fill=fillColor,fill opacity=0.20] (213.52, 93.96) circle (  2.13);

\path[fill=fillColor,fill opacity=0.20] (221.55, 94.71) circle (  2.13);

\path[fill=fillColor,fill opacity=0.20] (210.51, 66.39) circle (  2.13);

\path[fill=fillColor,fill opacity=0.20] (199.48, 87.00) circle (  2.13);

\path[fill=fillColor,fill opacity=0.20] (199.48,100.66) circle (  2.13);

\path[fill=fillColor,fill opacity=0.20] (193.46,101.42) circle (  2.13);

\path[fill=fillColor,fill opacity=0.20] (193.46, 96.86) circle (  2.13);

\path[fill=fillColor,fill opacity=0.20] (198.47, 91.93) circle (  2.13);

\path[fill=fillColor,fill opacity=0.20] (199.48, 89.78) circle (  2.13);

\path[fill=fillColor,fill opacity=0.20] (204.49, 82.07) circle (  2.13);

\path[fill=fillColor,fill opacity=0.20] (196.47, 85.23) circle (  2.13);

\path[fill=fillColor,fill opacity=0.20] (192.45, 84.35) circle (  2.13);

\path[fill=fillColor,fill opacity=0.20] (180.21, 82.58) circle (  2.13);

\path[fill=fillColor,fill opacity=0.20] (181.22, 87.13) circle (  2.13);

\path[fill=fillColor,fill opacity=0.20] (189.44, 94.46) circle (  2.13);

\path[fill=fillColor,fill opacity=0.20] (201.48, 99.90) circle (  2.13);

\path[fill=fillColor,fill opacity=0.20] (212.52, 99.77) circle (  2.13);

\path[fill=fillColor,fill opacity=0.20] (213.52, 89.15) circle (  2.13);

\path[fill=fillColor,fill opacity=0.20] (220.54, 85.99) circle (  2.13);

\path[fill=fillColor,fill opacity=0.20] (223.55, 86.12) circle (  2.13);

\path[fill=fillColor,fill opacity=0.20] (211.52, 68.92) circle (  2.13);

\path[fill=fillColor,fill opacity=0.20] (202.49, 93.45) circle (  2.13);

\path[fill=fillColor,fill opacity=0.20] (197.47, 96.49) circle (  2.13);

\path[fill=fillColor,fill opacity=0.20] (194.46, 98.13) circle (  2.13);

\path[fill=fillColor,fill opacity=0.20] (196.47,101.16) circle (  2.13);

\path[fill=fillColor,fill opacity=0.20] (196.47, 95.98) circle (  2.13);

\path[fill=fillColor,fill opacity=0.20] (199.48, 87.00) circle (  2.13);

\path[fill=fillColor,fill opacity=0.20] (201.48, 81.56) circle (  2.13);

\path[fill=fillColor,fill opacity=0.20] (191.45, 67.15) circle (  2.13);

\path[fill=fillColor,fill opacity=0.20] (184.13, 71.83) circle (  2.13);

\path[fill=fillColor,fill opacity=0.20] (186.23, 81.18) circle (  2.13);

\path[fill=fillColor,fill opacity=0.20] (174.40, 89.53) circle (  2.13);

\path[fill=fillColor,fill opacity=0.20] (186.33, 99.27) circle (  2.13);

\path[fill=fillColor,fill opacity=0.20] (178.01, 98.89) circle (  2.13);

\path[fill=fillColor,fill opacity=0.20] (230.58, 83.84) circle (  2.13);

\path[fill=fillColor,fill opacity=0.20] (219.54, 51.85) circle (  2.13);

\path[fill=fillColor,fill opacity=0.20] (207.50, 87.89) circle (  2.13);

\path[fill=fillColor,fill opacity=0.20] (197.47, 92.57) circle (  2.13);

\path[fill=fillColor,fill opacity=0.20] (194.46, 93.58) circle (  2.13);

\path[fill=fillColor,fill opacity=0.20] (196.47, 99.27) circle (  2.13);

\path[fill=fillColor,fill opacity=0.20] (198.47,101.16) circle (  2.13);

\path[fill=fillColor,fill opacity=0.20] (197.47, 95.73) circle (  2.13);

\path[fill=fillColor,fill opacity=0.20] (198.47, 86.75) circle (  2.13);

\path[fill=fillColor,fill opacity=0.20] (203.49, 83.46) circle (  2.13);

\path[fill=fillColor,fill opacity=0.20] (189.44, 58.68) circle (  2.13);

\path[fill=fillColor,fill opacity=0.20] (190.45, 62.34) circle (  2.13);

\path[fill=fillColor,fill opacity=0.20] (195.46, 74.48) circle (  2.13);

\path[fill=fillColor,fill opacity=0.20] (189.44, 87.89) circle (  2.13);

\path[fill=fillColor,fill opacity=0.20] (213.52, 95.22) circle (  2.13);

\path[fill=fillColor,fill opacity=0.20] (232.58, 70.69) circle (  2.13);

\path[fill=fillColor,fill opacity=0.20] (218.54, 68.54) circle (  2.13);

\path[fill=fillColor,fill opacity=0.20] (203.49, 76.63) circle (  2.13);

\path[fill=fillColor,fill opacity=0.20] (198.47, 84.09) circle (  2.13);

\path[fill=fillColor,fill opacity=0.20] (197.47, 93.20) circle (  2.13);

\path[fill=fillColor,fill opacity=0.20] (198.47, 95.73) circle (  2.13);

\path[fill=fillColor,fill opacity=0.20] (203.49, 95.22) circle (  2.13);

\path[fill=fillColor,fill opacity=0.20] (204.49, 90.79) circle (  2.13);

\path[fill=fillColor,fill opacity=0.20] (205.50, 95.35) circle (  2.13);

\path[fill=fillColor,fill opacity=0.20] (207.50, 98.13) circle (  2.13);

\path[fill=fillColor,fill opacity=0.20] (187.84, 56.40) circle (  2.13);

\path[fill=fillColor,fill opacity=0.20] (190.45, 58.80) circle (  2.13);

\path[fill=fillColor,fill opacity=0.20] (196.47, 58.68) circle (  2.13);

\path[fill=fillColor,fill opacity=0.20] (204.49, 70.94) circle (  2.13);

\path[fill=fillColor,fill opacity=0.20] (214.53, 95.73) circle (  2.13);

\path[fill=fillColor,fill opacity=0.20] (190.45, 86.37) circle (  2.13);

\path[fill=fillColor,fill opacity=0.20] (226.56, 62.72) circle (  2.13);

\path[fill=fillColor,fill opacity=0.20] (209.51, 83.33) circle (  2.13);

\path[fill=fillColor,fill opacity=0.20] (196.47, 90.42) circle (  2.13);

\path[fill=fillColor,fill opacity=0.20] (204.49, 92.94) circle (  2.13);

\path[fill=fillColor,fill opacity=0.20] (203.49, 99.39) circle (  2.13);

\path[fill=fillColor,fill opacity=0.20] (193.46, 98.26) circle (  2.13);

\path[fill=fillColor,fill opacity=0.20] (199.48, 93.58) circle (  2.13);

\path[fill=fillColor,fill opacity=0.20] (199.48, 94.34) circle (  2.13);

\path[fill=fillColor,fill opacity=0.20] (202.49, 81.31) circle (  2.13);

\path[fill=fillColor,fill opacity=0.20] (197.47, 72.33) circle (  2.13);

\path[fill=fillColor,fill opacity=0.20] (187.94, 65.63) circle (  2.13);

\path[fill=fillColor,fill opacity=0.20] (194.46, 63.36) circle (  2.13);

\path[fill=fillColor,fill opacity=0.20] (203.49, 81.18) circle (  2.13);

\path[fill=fillColor,fill opacity=0.20] (242.62, 59.44) circle (  2.13);

\path[fill=fillColor,fill opacity=0.20] (208.51, 79.03) circle (  2.13);

\path[fill=fillColor,fill opacity=0.20] (207.50, 95.98) circle (  2.13);

\path[fill=fillColor,fill opacity=0.20] (197.47,105.21) circle (  2.13);

\path[fill=fillColor,fill opacity=0.20] (196.47, 94.46) circle (  2.13);

\path[fill=fillColor,fill opacity=0.20] (196.47, 88.14) circle (  2.13);

\path[fill=fillColor,fill opacity=0.20] (199.48, 90.67) circle (  2.13);

\path[fill=fillColor,fill opacity=0.20] (206.50, 93.07) circle (  2.13);

\path[fill=fillColor,fill opacity=0.20] (201.48, 85.10) circle (  2.13);

\path[fill=fillColor,fill opacity=0.20] (204.49, 74.61) circle (  2.13);

\path[fill=fillColor,fill opacity=0.20] (202.49, 67.28) circle (  2.13);

\path[fill=fillColor,fill opacity=0.20] (201.48, 65.50) circle (  2.13);

\path[fill=fillColor,fill opacity=0.20] (198.47, 72.71) circle (  2.13);

\path[fill=fillColor,fill opacity=0.20] (196.47, 79.79) circle (  2.13);

\path[fill=fillColor,fill opacity=0.20] (196.47, 75.75) circle (  2.13);

\path[fill=fillColor,fill opacity=0.20] (229.57, 51.22) circle (  2.13);

\path[fill=fillColor,fill opacity=0.20] (205.50, 72.84) circle (  2.13);

\path[fill=fillColor,fill opacity=0.20] (197.47, 84.98) circle (  2.13);

\path[fill=fillColor,fill opacity=0.20] (191.45, 87.25) circle (  2.13);

\path[fill=fillColor,fill opacity=0.20] (194.46, 89.78) circle (  2.13);

\path[fill=fillColor,fill opacity=0.20] (197.47, 90.92) circle (  2.13);

\path[fill=fillColor,fill opacity=0.20] (205.50, 94.71) circle (  2.13);

\path[fill=fillColor,fill opacity=0.20] (202.49, 86.50) circle (  2.13);

\path[fill=fillColor,fill opacity=0.20] (202.49, 79.79) circle (  2.13);

\path[fill=fillColor,fill opacity=0.20] (209.51, 84.47) circle (  2.13);

\path[fill=fillColor,fill opacity=0.20] (215.53, 89.15) circle (  2.13);

\path[fill=fillColor,fill opacity=0.20] (218.54, 81.44) circle (  2.13);

\path[fill=fillColor,fill opacity=0.20] (214.53, 61.96) circle (  2.13);

\path[fill=fillColor,fill opacity=0.20] (210.51, 72.33) circle (  2.13);

\path[fill=fillColor,fill opacity=0.20] (206.50, 78.91) circle (  2.13);

\path[fill=fillColor,fill opacity=0.20] (203.49, 81.82) circle (  2.13);

\path[fill=fillColor,fill opacity=0.20] (203.49, 76.89) circle (  2.13);

\path[fill=fillColor,fill opacity=0.20] (197.47, 76.25) circle (  2.13);

\path[fill=fillColor,fill opacity=0.20] (195.46, 82.58) circle (  2.13);

\path[fill=fillColor,fill opacity=0.20] (198.47, 87.89) circle (  2.13);

\path[fill=fillColor,fill opacity=0.20] (215.53, 88.65) circle (  2.13);

\path[fill=fillColor,fill opacity=0.20] (214.53, 62.60) circle (  2.13);

\path[fill=fillColor,fill opacity=0.20] (230.58, 53.11) circle (  2.13);

\path[fill=fillColor,fill opacity=0.20] (210.51, 71.45) circle (  2.13);

\path[fill=fillColor,fill opacity=0.20] (197.47, 85.36) circle (  2.13);

\path[fill=fillColor,fill opacity=0.20] (198.47, 90.54) circle (  2.13);

\path[fill=fillColor,fill opacity=0.20] (200.48, 96.49) circle (  2.13);

\path[fill=fillColor,fill opacity=0.20] (202.49, 90.29) circle (  2.13);

\path[fill=fillColor,fill opacity=0.20] (203.49, 80.30) circle (  2.13);

\path[fill=fillColor,fill opacity=0.20] (205.50, 82.20) circle (  2.13);

\path[fill=fillColor,fill opacity=0.20] (206.50, 88.77) circle (  2.13);

\path[fill=fillColor,fill opacity=0.20] (209.51, 90.16) circle (  2.13);

\path[fill=fillColor,fill opacity=0.20] (217.53, 89.02) circle (  2.13);

\path[fill=fillColor,fill opacity=0.20] (203.49, 81.31) circle (  2.13);

\path[fill=fillColor,fill opacity=0.20] (216.53, 67.40) circle (  2.13);

\path[fill=fillColor,fill opacity=0.20] (212.52, 71.07) circle (  2.13);

\path[fill=fillColor,fill opacity=0.20] (209.51, 81.94) circle (  2.13);

\path[fill=fillColor,fill opacity=0.20] (214.53, 92.57) circle (  2.13);

\path[fill=fillColor,fill opacity=0.20] (213.52, 93.96) circle (  2.13);

\path[fill=fillColor,fill opacity=0.20] (212.52, 88.90) circle (  2.13);

\path[fill=fillColor,fill opacity=0.20] (200.48, 89.02) circle (  2.13);

\path[fill=fillColor,fill opacity=0.20] (196.47, 93.83) circle (  2.13);

\path[fill=fillColor,fill opacity=0.20] (200.48, 87.89) circle (  2.13);

\path[fill=fillColor,fill opacity=0.20] (204.49, 86.75) circle (  2.13);

\path[fill=fillColor,fill opacity=0.20] (203.49, 96.11) circle (  2.13);

\path[fill=fillColor,fill opacity=0.20] (222.55, 94.84) circle (  2.13);

\path[fill=fillColor,fill opacity=0.20] (199.48, 78.66) circle (  2.13);

\path[fill=fillColor,fill opacity=0.20] (231.58, 54.38) circle (  2.13);

\path[fill=fillColor,fill opacity=0.20] (203.49, 71.95) circle (  2.13);

\path[fill=fillColor,fill opacity=0.20] (202.49, 87.76) circle (  2.13);

\path[fill=fillColor,fill opacity=0.20] (202.49, 94.71) circle (  2.13);

\path[fill=fillColor,fill opacity=0.20] (200.48, 92.19) circle (  2.13);

\path[fill=fillColor,fill opacity=0.20] (204.49, 88.65) circle (  2.13);

\path[fill=fillColor,fill opacity=0.20] (200.48, 87.00) circle (  2.13);

\path[fill=fillColor,fill opacity=0.20] (198.47, 88.90) circle (  2.13);

\path[fill=fillColor,fill opacity=0.20] (206.50, 90.67) circle (  2.13);

\path[fill=fillColor,fill opacity=0.20] (211.52, 85.48) circle (  2.13);

\path[fill=fillColor,fill opacity=0.20] (212.52, 83.33) circle (  2.13);

\path[fill=fillColor,fill opacity=0.20] (212.52, 88.27) circle (  2.13);

\path[fill=fillColor,fill opacity=0.20] (210.51, 92.19) circle (  2.13);

\path[fill=fillColor,fill opacity=0.20] (214.53, 93.07) circle (  2.13);

\path[fill=fillColor,fill opacity=0.20] (210.51, 93.58) circle (  2.13);

\path[fill=fillColor,fill opacity=0.20] (213.52, 89.40) circle (  2.13);

\path[fill=fillColor,fill opacity=0.20] (211.52, 91.68) circle (  2.13);

\path[fill=fillColor,fill opacity=0.20] (209.51, 96.86) circle (  2.13);

\path[fill=fillColor,fill opacity=0.20] (208.51, 92.06) circle (  2.13);

\path[fill=fillColor,fill opacity=0.20] (214.53, 87.00) circle (  2.13);

\path[fill=fillColor,fill opacity=0.20] (216.53, 82.45) circle (  2.13);

\path[fill=fillColor,fill opacity=0.20] (211.52, 80.05) circle (  2.13);

\path[fill=fillColor,fill opacity=0.20] (211.52, 85.36) circle (  2.13);

\path[fill=fillColor,fill opacity=0.20] (213.52, 89.66) circle (  2.13);

\path[fill=fillColor,fill opacity=0.20] (212.52, 86.87) circle (  2.13);

\path[fill=fillColor,fill opacity=0.20] (209.51, 85.61) circle (  2.13);

\path[fill=fillColor,fill opacity=0.20] (205.50, 84.85) circle (  2.13);

\path[fill=fillColor,fill opacity=0.20] (204.49, 89.02) circle (  2.13);

\path[fill=fillColor,fill opacity=0.20] (202.49, 90.29) circle (  2.13);

\path[fill=fillColor,fill opacity=0.20] (200.48, 85.23) circle (  2.13);

\path[fill=fillColor,fill opacity=0.20] (200.48, 88.14) circle (  2.13);

\path[fill=fillColor,fill opacity=0.20] (197.47, 96.99) circle (  2.13);

\path[fill=fillColor,fill opacity=0.20] (204.49, 97.62) circle (  2.13);

\path[fill=fillColor,fill opacity=0.20] (213.52, 97.24) circle (  2.13);

\path[fill=fillColor,fill opacity=0.20] (206.50, 59.44) circle (  2.13);

\path[fill=fillColor,fill opacity=0.20] (222.55, 73.47) circle (  2.13);

\path[fill=fillColor,fill opacity=0.20] (211.52, 94.46) circle (  2.13);

\path[fill=fillColor,fill opacity=0.20] (200.48, 95.35) circle (  2.13);

\path[fill=fillColor,fill opacity=0.20] (201.48, 82.32) circle (  2.13);

\path[fill=fillColor,fill opacity=0.20] (199.48, 81.18) circle (  2.13);

\path[fill=fillColor,fill opacity=0.20] (201.48, 88.52) circle (  2.13);

\path[fill=fillColor,fill opacity=0.20] (202.49, 87.51) circle (  2.13);

\path[fill=fillColor,fill opacity=0.20] (206.50, 81.56) circle (  2.13);

\path[fill=fillColor,fill opacity=0.20] (201.48, 79.79) circle (  2.13);

\path[fill=fillColor,fill opacity=0.20] (199.48, 83.46) circle (  2.13);

\path[fill=fillColor,fill opacity=0.20] (201.48, 90.92) circle (  2.13);

\path[fill=fillColor,fill opacity=0.20] (206.50, 94.08) circle (  2.13);

\path[fill=fillColor,fill opacity=0.20] (205.50, 91.05) circle (  2.13);

\path[fill=fillColor,fill opacity=0.20] (205.50, 90.92) circle (  2.13);

\path[fill=fillColor,fill opacity=0.20] (207.50, 93.07) circle (  2.13);

\path[fill=fillColor,fill opacity=0.20] (205.50, 93.32) circle (  2.13);

\path[fill=fillColor,fill opacity=0.20] (208.51, 91.93) circle (  2.13);

\path[fill=fillColor,fill opacity=0.20] (210.51, 89.02) circle (  2.13);

\path[fill=fillColor,fill opacity=0.20] (203.49, 87.51) circle (  2.13);

\path[fill=fillColor,fill opacity=0.20] (206.50, 86.24) circle (  2.13);

\path[fill=fillColor,fill opacity=0.20] (203.49, 85.99) circle (  2.13);

\path[fill=fillColor,fill opacity=0.20] (201.48, 85.86) circle (  2.13);

\path[fill=fillColor,fill opacity=0.20] (201.48, 82.70) circle (  2.13);

\path[fill=fillColor,fill opacity=0.20] (195.46, 81.94) circle (  2.13);

\path[fill=fillColor,fill opacity=0.20] (198.47, 90.16) circle (  2.13);

\path[fill=fillColor,fill opacity=0.20] (201.48, 94.46) circle (  2.13);

\path[fill=fillColor,fill opacity=0.20] (191.45, 89.53) circle (  2.13);

\path[fill=fillColor,fill opacity=0.20] (215.53, 93.20) circle (  2.13);

\path[fill=fillColor,fill opacity=0.20] (223.55, 92.06) circle (  2.13);

\path[fill=fillColor,fill opacity=0.20] (190.45, 76.00) circle (  2.13);

\path[fill=fillColor,fill opacity=0.20] (264.69, 56.15) circle (  2.13);

\path[fill=fillColor,fill opacity=0.20] (228.57, 65.50) circle (  2.13);

\path[fill=fillColor,fill opacity=0.20] (186.64, 64.75) circle (  2.13);

\path[fill=fillColor,fill opacity=0.20] (202.49, 65.25) circle (  2.13);

\path[fill=fillColor,fill opacity=0.20] (214.53, 75.75) circle (  2.13);

\path[fill=fillColor,fill opacity=0.20] (207.50, 87.13) circle (  2.13);

\path[fill=fillColor,fill opacity=0.20] (206.50, 83.08) circle (  2.13);

\path[fill=fillColor,fill opacity=0.20] (193.46, 79.03) circle (  2.13);

\path[fill=fillColor,fill opacity=0.20] (197.47, 84.60) circle (  2.13);

\path[fill=fillColor,fill opacity=0.20] (199.48, 93.58) circle (  2.13);

\path[fill=fillColor,fill opacity=0.20] (204.49, 96.86) circle (  2.13);

\path[fill=fillColor,fill opacity=0.20] (200.48, 94.46) circle (  2.13);

\path[fill=fillColor,fill opacity=0.20] (203.49, 89.78) circle (  2.13);

\path[fill=fillColor,fill opacity=0.20] (202.49, 87.13) circle (  2.13);

\path[fill=fillColor,fill opacity=0.20] (207.50, 87.00) circle (  2.13);

\path[fill=fillColor,fill opacity=0.20] (211.52, 87.00) circle (  2.13);

\path[fill=fillColor,fill opacity=0.20] (212.52, 86.50) circle (  2.13);

\path[fill=fillColor,fill opacity=0.20] (200.48, 90.79) circle (  2.13);

\path[fill=fillColor,fill opacity=0.20] (206.50, 96.36) circle (  2.13);

\path[fill=fillColor,fill opacity=0.20] (215.53, 95.73) circle (  2.13);

\path[fill=fillColor,fill opacity=0.20] (191.45, 91.55) circle (  2.13);

\path[fill=fillColor,fill opacity=0.20] (208.51, 93.70) circle (  2.13);

\path[fill=fillColor,fill opacity=0.20] (209.51, 98.76) circle (  2.13);

\path[fill=fillColor,fill opacity=0.20] (195.46, 99.01) circle (  2.13);

\path[fill=fillColor,fill opacity=0.20] (228.57, 96.86) circle (  2.13);

\path[fill=fillColor,fill opacity=0.20] (236.60, 88.27) circle (  2.13);

\path[fill=fillColor,fill opacity=0.20] (257.66, 60.07) circle (  2.13);

\path[fill=fillColor,fill opacity=0.20] (251.64, 68.29) circle (  2.13);

\path[fill=fillColor,fill opacity=0.20] (234.59, 77.90) circle (  2.13);

\path[fill=fillColor,fill opacity=0.20] (240.61, 86.37) circle (  2.13);

\path[fill=fillColor,fill opacity=0.20] (238.60, 88.14) circle (  2.13);

\path[fill=fillColor,fill opacity=0.20] (216.53, 89.53) circle (  2.13);

\path[fill=fillColor,fill opacity=0.20] (226.56, 90.29) circle (  2.13);

\path[fill=fillColor,fill opacity=0.20] (229.57, 85.23) circle (  2.13);

\path[fill=fillColor,fill opacity=0.20] (239.61, 79.54) circle (  2.13);

\path[fill=fillColor,fill opacity=0.20] (223.55, 74.99) circle (  2.13);

\path[fill=fillColor,fill opacity=0.20] (220.54, 71.83) circle (  2.13);

\path[fill=fillColor,fill opacity=0.20] (222.55, 67.91) circle (  2.13);

\path[fill=fillColor,fill opacity=0.20] (226.56, 65.63) circle (  2.13);

\path[fill=fillColor,fill opacity=0.20] (203.49, 81.31) circle (  2.13);

\path[fill=fillColor,fill opacity=0.20] (218.54, 73.22) circle (  2.13);

\path[fill=fillColor,fill opacity=0.20] (250.64, 67.65) circle (  2.13);

\path[fill=fillColor,fill opacity=0.20] (209.51, 49.83) circle (  2.13);

\path[fill=fillColor,fill opacity=0.20] (197.47, 45.40) circle (  2.13);

\path[fill=fillColor,fill opacity=0.20] (187.84, 53.49) circle (  2.13);

\path[fill=fillColor,fill opacity=0.20] (191.45, 55.77) circle (  2.13);

\path[fill=fillColor,fill opacity=0.20] (197.47, 55.89) circle (  2.13);

\path[fill=fillColor,fill opacity=0.20] (205.50, 59.94) circle (  2.13);

\path[fill=fillColor,fill opacity=0.20] (208.51, 65.38) circle (  2.13);

\path[fill=fillColor,fill opacity=0.20] (208.51, 64.75) circle (  2.13);

\path[fill=fillColor,fill opacity=0.20] (209.51, 58.04) circle (  2.13);

\path[fill=fillColor,fill opacity=0.20] (214.53, 49.95) circle (  2.13);

\path[fill=fillColor,fill opacity=0.20] (217.53, 43.63) circle (  2.13);

\path[fill=fillColor,fill opacity=0.20] (210.51, 38.44) circle (  2.13);

\path[fill=fillColor,fill opacity=0.20] (192.45, 50.46) circle (  2.13);

\path[fill=fillColor,fill opacity=0.20] (200.48, 69.30) circle (  2.13);

\path[fill=fillColor,fill opacity=0.20] (173.29, 71.57) circle (  2.13);

\path[fill=fillColor,fill opacity=0.20] (197.47, 67.91) circle (  2.13);

\path[fill=fillColor,fill opacity=0.20] (199.48, 68.54) circle (  2.13);

\path[fill=fillColor,fill opacity=0.20] (204.49, 72.46) circle (  2.13);

\path[fill=fillColor,fill opacity=0.20] (209.51, 74.99) circle (  2.13);

\path[fill=fillColor,fill opacity=0.20] (209.51, 75.24) circle (  2.13);

\path[fill=fillColor,fill opacity=0.20] (209.51, 70.69) circle (  2.13);

\path[fill=fillColor,fill opacity=0.20] (217.53, 57.66) circle (  2.13);

\path[fill=fillColor,fill opacity=0.20] (219.54, 47.68) circle (  2.13);

\path[fill=fillColor,fill opacity=0.20] (218.54, 42.87) circle (  2.13);

\path[fill=fillColor,fill opacity=0.20] (196.47, 43.25) circle (  2.13);

\path[fill=fillColor,fill opacity=0.20] (195.46, 64.62) circle (  2.13);

\path[fill=fillColor,fill opacity=0.20] (199.48, 83.08) circle (  2.13);

\path[fill=fillColor,fill opacity=0.20] (200.48, 86.75) circle (  2.13);

\path[fill=fillColor,fill opacity=0.20] (201.48, 83.84) circle (  2.13);

\path[fill=fillColor,fill opacity=0.20] (203.49, 83.84) circle (  2.13);

\path[fill=fillColor,fill opacity=0.20] (205.50, 83.84) circle (  2.13);

\path[fill=fillColor,fill opacity=0.20] (209.51, 79.67) circle (  2.13);

\path[fill=fillColor,fill opacity=0.20] (211.52, 76.00) circle (  2.13);

\path[fill=fillColor,fill opacity=0.20] (215.53, 71.07) circle (  2.13);

\path[fill=fillColor,fill opacity=0.20] (224.56, 59.81) circle (  2.13);

\path[fill=fillColor,fill opacity=0.20] (168.38, 70.82) circle (  2.13);

\path[fill=fillColor,fill opacity=0.20] (195.46, 50.46) circle (  2.13);

\path[fill=fillColor,fill opacity=0.20] (200.48, 71.57) circle (  2.13);

\path[fill=fillColor,fill opacity=0.20] (202.49, 86.12) circle (  2.13);

\path[fill=fillColor,fill opacity=0.20] (203.49, 92.57) circle (  2.13);

\path[fill=fillColor,fill opacity=0.20] (205.50, 92.82) circle (  2.13);

\path[fill=fillColor,fill opacity=0.20] (206.50, 91.30) circle (  2.13);

\path[fill=fillColor,fill opacity=0.20] (209.51, 88.65) circle (  2.13);

\path[fill=fillColor,fill opacity=0.20] (211.52, 80.55) circle (  2.13);

\path[fill=fillColor,fill opacity=0.20] (215.53, 72.46) circle (  2.13);

\path[fill=fillColor,fill opacity=0.20] (223.55, 66.77) circle (  2.13);

\path[fill=fillColor,fill opacity=0.20] (238.60, 57.03) circle (  2.13);

\path[fill=fillColor,fill opacity=0.20] (207.50, 80.17) circle (  2.13);

\path[fill=fillColor,fill opacity=0.20] (204.49, 83.08) circle (  2.13);

\path[fill=fillColor,fill opacity=0.20] (188.44, 89.15) circle (  2.13);

\path[fill=fillColor,fill opacity=0.20] (210.51, 37.94) circle (  2.13);

\path[fill=fillColor,fill opacity=0.20] (198.47, 58.04) circle (  2.13);

\path[fill=fillColor,fill opacity=0.20] (201.48, 78.53) circle (  2.13);

\path[fill=fillColor,fill opacity=0.20] (205.50, 87.76) circle (  2.13);

\path[fill=fillColor,fill opacity=0.20] (205.50, 92.82) circle (  2.13);

\path[fill=fillColor,fill opacity=0.20] (208.51, 93.45) circle (  2.13);

\path[fill=fillColor,fill opacity=0.20] (210.51, 88.65) circle (  2.13);

\path[fill=fillColor,fill opacity=0.20] (212.52, 84.98) circle (  2.13);

\path[fill=fillColor,fill opacity=0.20] (213.52, 78.53) circle (  2.13);

\path[fill=fillColor,fill opacity=0.20] (222.55, 69.05) circle (  2.13);

\path[fill=fillColor,fill opacity=0.20] (209.51, 81.18) circle (  2.13);

\path[fill=fillColor,fill opacity=0.20] (211.52, 87.63) circle (  2.13);

\path[fill=fillColor,fill opacity=0.20] (201.48, 88.14) circle (  2.13);

\path[fill=fillColor,fill opacity=0.20] (193.46, 95.98) circle (  2.13);

\path[fill=fillColor,fill opacity=0.20] (189.44, 99.39) circle (  2.13);

\path[fill=fillColor,fill opacity=0.20] (189.44, 98.63) circle (  2.13);

\path[fill=fillColor,fill opacity=0.20] (196.47, 93.96) circle (  2.13);

\path[fill=fillColor,fill opacity=0.20] (204.49, 87.13) circle (  2.13);

\path[fill=fillColor,fill opacity=0.20] (207.50, 40.85) circle (  2.13);

\path[fill=fillColor,fill opacity=0.20] (196.47, 65.00) circle (  2.13);

\path[fill=fillColor,fill opacity=0.20] (203.49, 84.47) circle (  2.13);

\path[fill=fillColor,fill opacity=0.20] (208.51, 88.90) circle (  2.13);

\path[fill=fillColor,fill opacity=0.20] (208.51, 93.07) circle (  2.13);

\path[fill=fillColor,fill opacity=0.20] (214.53, 93.58) circle (  2.13);

\path[fill=fillColor,fill opacity=0.20] (214.53, 86.50) circle (  2.13);

\path[fill=fillColor,fill opacity=0.20] (212.52, 80.55) circle (  2.13);

\path[fill=fillColor,fill opacity=0.20] (214.53, 74.61) circle (  2.13);

\path[fill=fillColor,fill opacity=0.20] (206.50, 85.23) circle (  2.13);

\path[fill=fillColor,fill opacity=0.20] (199.48, 85.61) circle (  2.13);

\path[fill=fillColor,fill opacity=0.20] (198.47, 90.04) circle (  2.13);

\path[fill=fillColor,fill opacity=0.20] (201.48, 98.76) circle (  2.13);

\path[fill=fillColor,fill opacity=0.20] (189.44,105.08) circle (  2.13);

\path[fill=fillColor,fill opacity=0.20] (189.44,106.98) circle (  2.13);

\path[fill=fillColor,fill opacity=0.20] (194.46,101.67) circle (  2.13);

\path[fill=fillColor,fill opacity=0.20] (198.47, 95.60) circle (  2.13);

\path[fill=fillColor,fill opacity=0.20] (198.47, 92.19) circle (  2.13);

\path[fill=fillColor,fill opacity=0.20] (165.87, 87.89) circle (  2.13);

\path[fill=fillColor,fill opacity=0.20] (210.51, 39.84) circle (  2.13);

\path[fill=fillColor,fill opacity=0.20] (198.47, 61.33) circle (  2.13);

\path[fill=fillColor,fill opacity=0.20] (206.50, 84.73) circle (  2.13);

\path[fill=fillColor,fill opacity=0.20] (209.51, 89.40) circle (  2.13);

\path[fill=fillColor,fill opacity=0.20] (206.50, 89.53) circle (  2.13);

\path[fill=fillColor,fill opacity=0.20] (210.51, 91.43) circle (  2.13);

\path[fill=fillColor,fill opacity=0.20] (215.53, 89.15) circle (  2.13);

\path[fill=fillColor,fill opacity=0.20] (212.52, 81.06) circle (  2.13);

\path[fill=fillColor,fill opacity=0.20] (213.52, 70.31) circle (  2.13);

\path[fill=fillColor,fill opacity=0.20] (206.50, 66.52) circle (  2.13);

\path[fill=fillColor,fill opacity=0.20] (200.48, 99.14) circle (  2.13);

\path[fill=fillColor,fill opacity=0.20] (202.49, 92.94) circle (  2.13);

\path[fill=fillColor,fill opacity=0.20] (199.48, 94.84) circle (  2.13);

\path[fill=fillColor,fill opacity=0.20] (191.45,104.70) circle (  2.13);

\path[fill=fillColor,fill opacity=0.20] (189.44,109.51) circle (  2.13);

\path[fill=fillColor,fill opacity=0.20] (189.44,107.74) circle (  2.13);

\path[fill=fillColor,fill opacity=0.20] (192.45,103.95) circle (  2.13);

\path[fill=fillColor,fill opacity=0.20] (195.46,105.72) circle (  2.13);

\path[fill=fillColor,fill opacity=0.20] (198.47,107.11) circle (  2.13);

\path[fill=fillColor,fill opacity=0.20] (199.48, 94.08) circle (  2.13);

\path[fill=fillColor,fill opacity=0.20] (193.46, 79.41) circle (  2.13);

\path[fill=fillColor,fill opacity=0.20] (203.49, 57.41) circle (  2.13);

\path[fill=fillColor,fill opacity=0.20] (203.49, 84.73) circle (  2.13);

\path[fill=fillColor,fill opacity=0.20] (206.50, 90.29) circle (  2.13);

\path[fill=fillColor,fill opacity=0.20] (207.50, 84.85) circle (  2.13);

\path[fill=fillColor,fill opacity=0.20] (204.49, 87.63) circle (  2.13);

\path[fill=fillColor,fill opacity=0.20] (209.51, 93.70) circle (  2.13);

\path[fill=fillColor,fill opacity=0.20] (215.53, 86.87) circle (  2.13);

\path[fill=fillColor,fill opacity=0.20] (217.53, 72.08) circle (  2.13);

\path[fill=fillColor,fill opacity=0.20] (218.54, 63.48) circle (  2.13);

\path[fill=fillColor,fill opacity=0.20] (223.55, 57.54) circle (  2.13);

\path[fill=fillColor,fill opacity=0.20] (202.49, 81.06) circle (  2.13);

\path[fill=fillColor,fill opacity=0.20] (202.49,114.19) circle (  2.13);

\path[fill=fillColor,fill opacity=0.20] (199.48,101.67) circle (  2.13);

\path[fill=fillColor,fill opacity=0.20] (199.48,105.46) circle (  2.13);

\path[fill=fillColor,fill opacity=0.20] (195.46,114.19) circle (  2.13);

\path[fill=fillColor,fill opacity=0.20] (195.46,109.89) circle (  2.13);

\path[fill=fillColor,fill opacity=0.20] (192.45,105.21) circle (  2.13);

\path[fill=fillColor,fill opacity=0.20] (189.44,105.46) circle (  2.13);

\path[fill=fillColor,fill opacity=0.20] (190.45,109.89) circle (  2.13);

\path[fill=fillColor,fill opacity=0.20] (193.46,115.58) circle (  2.13);

\path[fill=fillColor,fill opacity=0.20] (188.44,103.06) circle (  2.13);

\path[fill=fillColor,fill opacity=0.20] (201.48, 86.12) circle (  2.13);

\path[fill=fillColor,fill opacity=0.20] (191.45, 84.73) circle (  2.13);

\path[fill=fillColor,fill opacity=0.20] (202.49, 58.30) circle (  2.13);

\path[fill=fillColor,fill opacity=0.20] (198.47, 86.24) circle (  2.13);

\path[fill=fillColor,fill opacity=0.20] (204.49, 91.30) circle (  2.13);

\path[fill=fillColor,fill opacity=0.20] (208.51, 86.75) circle (  2.13);

\path[fill=fillColor,fill opacity=0.20] (213.52, 91.05) circle (  2.13);

\path[fill=fillColor,fill opacity=0.20] (218.54, 97.12) circle (  2.13);

\path[fill=fillColor,fill opacity=0.20] (217.53, 90.67) circle (  2.13);

\path[fill=fillColor,fill opacity=0.20] (216.53, 79.79) circle (  2.13);

\path[fill=fillColor,fill opacity=0.20] (219.54, 73.34) circle (  2.13);

\path[fill=fillColor,fill opacity=0.20] (221.55, 68.92) circle (  2.13);

\path[fill=fillColor,fill opacity=0.20] (199.48, 82.95) circle (  2.13);

\path[fill=fillColor,fill opacity=0.20] (194.46,106.10) circle (  2.13);

\path[fill=fillColor,fill opacity=0.20] (202.49,100.15) circle (  2.13);

\path[fill=fillColor,fill opacity=0.20] (200.48,110.39) circle (  2.13);

\path[fill=fillColor,fill opacity=0.20] (198.47,112.92) circle (  2.13);

\path[fill=fillColor,fill opacity=0.20] (196.47,106.35) circle (  2.13);

\path[fill=fillColor,fill opacity=0.20] (194.46,106.10) circle (  2.13);

\path[fill=fillColor,fill opacity=0.20] (191.45,109.76) circle (  2.13);

\path[fill=fillColor,fill opacity=0.20] (191.45,110.52) circle (  2.13);

\path[fill=fillColor,fill opacity=0.20] (190.45,114.19) circle (  2.13);

\path[fill=fillColor,fill opacity=0.20] (190.45,111.41) circle (  2.13);

\path[fill=fillColor,fill opacity=0.20] (202.49, 98.38) circle (  2.13);

\path[fill=fillColor,fill opacity=0.20] (210.51, 89.15) circle (  2.13);

\path[fill=fillColor,fill opacity=0.20] (205.50, 51.97) circle (  2.13);

\path[fill=fillColor,fill opacity=0.20] (198.47, 79.79) circle (  2.13);

\path[fill=fillColor,fill opacity=0.20] (205.50, 89.15) circle (  2.13);

\path[fill=fillColor,fill opacity=0.20] (211.52, 90.04) circle (  2.13);

\path[fill=fillColor,fill opacity=0.20] (213.52, 96.11) circle (  2.13);

\path[fill=fillColor,fill opacity=0.20] (216.53, 96.99) circle (  2.13);

\path[fill=fillColor,fill opacity=0.20] (213.52, 88.90) circle (  2.13);

\path[fill=fillColor,fill opacity=0.20] (214.53, 82.58) circle (  2.13);

\path[fill=fillColor,fill opacity=0.20] (214.53, 80.30) circle (  2.13);

\path[fill=fillColor,fill opacity=0.20] (217.53, 80.68) circle (  2.13);

\path[fill=fillColor,fill opacity=0.20] (223.55, 74.86) circle (  2.13);

\path[fill=fillColor,fill opacity=0.20] (200.48, 76.76) circle (  2.13);

\path[fill=fillColor,fill opacity=0.20] (193.46, 95.85) circle (  2.13);

\path[fill=fillColor,fill opacity=0.20] (198.47, 96.11) circle (  2.13);

\path[fill=fillColor,fill opacity=0.20] (199.48,103.69) circle (  2.13);

\path[fill=fillColor,fill opacity=0.20] (200.48,101.54) circle (  2.13);

\path[fill=fillColor,fill opacity=0.20] (195.46,103.95) circle (  2.13);

\path[fill=fillColor,fill opacity=0.20] (194.46,111.15) circle (  2.13);

\path[fill=fillColor,fill opacity=0.20] (193.46,109.26) circle (  2.13);

\path[fill=fillColor,fill opacity=0.20] (192.45,108.12) circle (  2.13);

\path[fill=fillColor,fill opacity=0.20] (190.45,114.44) circle (  2.13);

\path[fill=fillColor,fill opacity=0.20] (198.47,109.76) circle (  2.13);

\path[fill=fillColor,fill opacity=0.20] (208.51, 96.23) circle (  2.13);

\path[fill=fillColor,fill opacity=0.20] (225.56, 38.44) circle (  2.13);

\path[fill=fillColor,fill opacity=0.20] (207.50, 64.24) circle (  2.13);

\path[fill=fillColor,fill opacity=0.20] (207.50, 82.07) circle (  2.13);

\path[fill=fillColor,fill opacity=0.20] (210.51, 86.24) circle (  2.13);

\path[fill=fillColor,fill opacity=0.20] (211.52, 91.30) circle (  2.13);

\path[fill=fillColor,fill opacity=0.20] (210.51, 94.71) circle (  2.13);

\path[fill=fillColor,fill opacity=0.20] (212.52, 89.78) circle (  2.13);

\path[fill=fillColor,fill opacity=0.20] (212.52, 81.44) circle (  2.13);

\path[fill=fillColor,fill opacity=0.20] (215.53, 78.53) circle (  2.13);

\path[fill=fillColor,fill opacity=0.20] (215.53, 85.48) circle (  2.13);

\path[fill=fillColor,fill opacity=0.20] (216.53, 89.53) circle (  2.13);

\path[fill=fillColor,fill opacity=0.20] (224.56, 75.12) circle (  2.13);

\path[fill=fillColor,fill opacity=0.20] (228.57, 56.40) circle (  2.13);

\path[fill=fillColor,fill opacity=0.20] (202.49, 91.43) circle (  2.13);

\path[fill=fillColor,fill opacity=0.20] (197.47, 94.71) circle (  2.13);

\path[fill=fillColor,fill opacity=0.20] (197.47, 99.27) circle (  2.13);

\path[fill=fillColor,fill opacity=0.20] (194.46, 95.22) circle (  2.13);

\path[fill=fillColor,fill opacity=0.20] (190.45,102.05) circle (  2.13);

\path[fill=fillColor,fill opacity=0.20] (190.45,112.92) circle (  2.13);

\path[fill=fillColor,fill opacity=0.20] (190.45,104.07) circle (  2.13);

\path[fill=fillColor,fill opacity=0.20] (191.45,102.81) circle (  2.13);

\path[fill=fillColor,fill opacity=0.20] (192.45,114.95) circle (  2.13);

\path[fill=fillColor,fill opacity=0.20] (203.49,110.14) circle (  2.13);

\path[fill=fillColor,fill opacity=0.20] (205.50,103.31) circle (  2.13);

\path[fill=fillColor,fill opacity=0.20] (211.52, 79.79) circle (  2.13);

\path[fill=fillColor,fill opacity=0.20] (228.57, 48.18) circle (  2.13);

\path[fill=fillColor,fill opacity=0.20] (209.51, 70.44) circle (  2.13);

\path[fill=fillColor,fill opacity=0.20] (209.51, 82.58) circle (  2.13);

\path[fill=fillColor,fill opacity=0.20] (207.50, 87.63) circle (  2.13);

\path[fill=fillColor,fill opacity=0.20] (209.51, 91.55) circle (  2.13);

\path[fill=fillColor,fill opacity=0.20] (209.51, 92.44) circle (  2.13);

\path[fill=fillColor,fill opacity=0.20] (211.52, 86.24) circle (  2.13);

\path[fill=fillColor,fill opacity=0.20] (211.52, 81.82) circle (  2.13);

\path[fill=fillColor,fill opacity=0.20] (210.51, 86.24) circle (  2.13);

\path[fill=fillColor,fill opacity=0.20] (213.52, 89.15) circle (  2.13);

\path[fill=fillColor,fill opacity=0.20] (222.55, 79.03) circle (  2.13);

\path[fill=fillColor,fill opacity=0.20] (219.54, 68.03) circle (  2.13);

\path[fill=fillColor,fill opacity=0.20] (223.55, 59.18) circle (  2.13);

\path[fill=fillColor,fill opacity=0.20] (209.51, 77.39) circle (  2.13);

\path[fill=fillColor,fill opacity=0.20] (201.48, 90.67) circle (  2.13);

\path[fill=fillColor,fill opacity=0.20] (200.48, 94.08) circle (  2.13);

\path[fill=fillColor,fill opacity=0.20] (194.46, 99.52) circle (  2.13);

\path[fill=fillColor,fill opacity=0.20] (191.45, 98.38) circle (  2.13);

\path[fill=fillColor,fill opacity=0.20] (191.45,100.53) circle (  2.13);

\path[fill=fillColor,fill opacity=0.20] (191.45,106.73) circle (  2.13);

\path[fill=fillColor,fill opacity=0.20] (189.44,102.30) circle (  2.13);

\path[fill=fillColor,fill opacity=0.20] (194.46,103.19) circle (  2.13);

\path[fill=fillColor,fill opacity=0.20] (196.47,111.28) circle (  2.13);

\path[fill=fillColor,fill opacity=0.20] (194.46,107.49) circle (  2.13);

\path[fill=fillColor,fill opacity=0.20] (198.47,105.34) circle (  2.13);

\path[fill=fillColor,fill opacity=0.20] (207.50,106.98) circle (  2.13);

\path[fill=fillColor,fill opacity=0.20] (184.73, 79.16) circle (  2.13);

\path[fill=fillColor,fill opacity=0.20] (218.54, 54.25) circle (  2.13);

\path[fill=fillColor,fill opacity=0.20] (200.48, 79.54) circle (  2.13);

\path[fill=fillColor,fill opacity=0.20] (210.51, 90.16) circle (  2.13);

\path[fill=fillColor,fill opacity=0.20] (211.52, 89.53) circle (  2.13);

\path[fill=fillColor,fill opacity=0.20] (208.51, 90.92) circle (  2.13);

\path[fill=fillColor,fill opacity=0.20] (213.52, 92.06) circle (  2.13);

\path[fill=fillColor,fill opacity=0.20] (193.46, 89.66) circle (  2.13);

\path[fill=fillColor,fill opacity=0.20] (210.51, 88.27) circle (  2.13);

\path[fill=fillColor,fill opacity=0.20] (216.53, 85.10) circle (  2.13);

\path[fill=fillColor,fill opacity=0.20] (218.54, 78.91) circle (  2.13);

\path[fill=fillColor,fill opacity=0.20] (217.53, 72.84) circle (  2.13);

\path[fill=fillColor,fill opacity=0.20] (219.54, 66.52) circle (  2.13);

\path[fill=fillColor,fill opacity=0.20] (221.55, 63.61) circle (  2.13);

\path[fill=fillColor,fill opacity=0.20] (224.56, 58.30) circle (  2.13);

\path[fill=fillColor,fill opacity=0.20] (205.50, 68.16) circle (  2.13);

\path[fill=fillColor,fill opacity=0.20] (200.48, 83.33) circle (  2.13);

\path[fill=fillColor,fill opacity=0.20] (200.48, 92.06) circle (  2.13);

\path[fill=fillColor,fill opacity=0.20] (200.48, 93.20) circle (  2.13);

\path[fill=fillColor,fill opacity=0.20] (200.48, 99.27) circle (  2.13);

\path[fill=fillColor,fill opacity=0.20] (196.47,106.10) circle (  2.13);

\path[fill=fillColor,fill opacity=0.20] (189.44,106.47) circle (  2.13);

\path[fill=fillColor,fill opacity=0.20] (192.45,103.06) circle (  2.13);

\path[fill=fillColor,fill opacity=0.20] (191.45,103.06) circle (  2.13);

\path[fill=fillColor,fill opacity=0.20] (193.46,109.26) circle (  2.13);

\path[fill=fillColor,fill opacity=0.20] (198.47,106.22) circle (  2.13);

\path[fill=fillColor,fill opacity=0.20] (193.46, 98.26) circle (  2.13);

\path[fill=fillColor,fill opacity=0.20] (192.45,103.95) circle (  2.13);

\path[fill=fillColor,fill opacity=0.20] (209.51,104.32) circle (  2.13);

\path[fill=fillColor,fill opacity=0.20] (239.61, 56.27) circle (  2.13);

\path[fill=fillColor,fill opacity=0.20] (224.56, 75.12) circle (  2.13);

\path[fill=fillColor,fill opacity=0.20] (215.53, 83.84) circle (  2.13);

\path[fill=fillColor,fill opacity=0.20] (206.50, 88.65) circle (  2.13);

\path[fill=fillColor,fill opacity=0.20] (211.52, 90.92) circle (  2.13);

\path[fill=fillColor,fill opacity=0.20] (214.53, 92.31) circle (  2.13);

\path[fill=fillColor,fill opacity=0.20] (205.50, 91.93) circle (  2.13);

\path[fill=fillColor,fill opacity=0.20] (214.53, 84.47) circle (  2.13);

\path[fill=fillColor,fill opacity=0.20] (214.53, 79.16) circle (  2.13);

\path[fill=fillColor,fill opacity=0.20] (217.53, 77.01) circle (  2.13);

\path[fill=fillColor,fill opacity=0.20] (220.54, 70.31) circle (  2.13);

\path[fill=fillColor,fill opacity=0.20] (215.53, 68.54) circle (  2.13);

\path[fill=fillColor,fill opacity=0.20] (216.53, 69.30) circle (  2.13);

\path[fill=fillColor,fill opacity=0.20] (218.54, 60.32) circle (  2.13);

\path[fill=fillColor,fill opacity=0.20] (215.53, 54.76) circle (  2.13);

\path[fill=fillColor,fill opacity=0.20] (210.51, 63.73) circle (  2.13);

\path[fill=fillColor,fill opacity=0.20] (209.51, 66.77) circle (  2.13);

\path[fill=fillColor,fill opacity=0.20] (207.50, 70.82) circle (  2.13);

\path[fill=fillColor,fill opacity=0.20] (203.49, 72.46) circle (  2.13);

\path[fill=fillColor,fill opacity=0.20] (199.48, 78.40) circle (  2.13);

\path[fill=fillColor,fill opacity=0.20] (196.47, 86.87) circle (  2.13);

\path[fill=fillColor,fill opacity=0.20] (196.47, 92.31) circle (  2.13);

\path[fill=fillColor,fill opacity=0.20] (198.47, 94.21) circle (  2.13);

\path[fill=fillColor,fill opacity=0.20] (202.49, 98.13) circle (  2.13);

\path[fill=fillColor,fill opacity=0.20] (196.47,105.72) circle (  2.13);

\path[fill=fillColor,fill opacity=0.20] (196.47,110.77) circle (  2.13);

\path[fill=fillColor,fill opacity=0.20] (189.44,106.10) circle (  2.13);

\path[fill=fillColor,fill opacity=0.20] (193.46,103.69) circle (  2.13);

\path[fill=fillColor,fill opacity=0.20] (194.46,106.73) circle (  2.13);

\path[fill=fillColor,fill opacity=0.20] (194.46,101.80) circle (  2.13);

\path[fill=fillColor,fill opacity=0.20] (192.45, 97.24) circle (  2.13);

\path[fill=fillColor,fill opacity=0.20] (202.49,103.06) circle (  2.13);

\path[fill=fillColor,fill opacity=0.20] (208.51, 96.74) circle (  2.13);

\path[fill=fillColor,fill opacity=0.20] (168.08, 75.12) circle (  2.13);

\path[fill=fillColor,fill opacity=0.20] (233.59, 43.88) circle (  2.13);

\path[fill=fillColor,fill opacity=0.20] (224.56, 60.83) circle (  2.13);

\path[fill=fillColor,fill opacity=0.20] (216.53, 76.38) circle (  2.13);

\path[fill=fillColor,fill opacity=0.20] (218.54, 83.33) circle (  2.13);

\path[fill=fillColor,fill opacity=0.20] (209.51, 89.91) circle (  2.13);

\path[fill=fillColor,fill opacity=0.20] (209.51, 93.20) circle (  2.13);

\path[fill=fillColor,fill opacity=0.20] (210.51, 83.97) circle (  2.13);

\path[fill=fillColor,fill opacity=0.20] (210.51, 76.25) circle (  2.13);

\path[fill=fillColor,fill opacity=0.20] (217.53, 77.26) circle (  2.13);

\path[fill=fillColor,fill opacity=0.20] (218.54, 74.74) circle (  2.13);

\path[fill=fillColor,fill opacity=0.20] (212.52, 71.32) circle (  2.13);

\path[fill=fillColor,fill opacity=0.20] (213.52, 69.68) circle (  2.13);

\path[fill=fillColor,fill opacity=0.20] (213.52, 66.64) circle (  2.13);

\path[fill=fillColor,fill opacity=0.20] (217.53, 69.17) circle (  2.13);

\path[fill=fillColor,fill opacity=0.20] (225.56, 71.83) circle (  2.13);

\path[fill=fillColor,fill opacity=0.20] (226.56, 61.33) circle (  2.13);

\path[fill=fillColor,fill opacity=0.20] (215.53, 53.24) circle (  2.13);

\path[fill=fillColor,fill opacity=0.20] (216.53, 57.92) circle (  2.13);

\path[fill=fillColor,fill opacity=0.20] (215.53, 61.84) circle (  2.13);

\path[fill=fillColor,fill opacity=0.20] (197.47, 67.65) circle (  2.13);

\path[fill=fillColor,fill opacity=0.20] (200.48, 69.68) circle (  2.13);

\path[fill=fillColor,fill opacity=0.20] (203.49, 72.08) circle (  2.13);

\path[fill=fillColor,fill opacity=0.20] (203.49, 78.28) circle (  2.13);

\path[fill=fillColor,fill opacity=0.20] (199.48, 78.78) circle (  2.13);

\path[fill=fillColor,fill opacity=0.20] (199.48, 81.82) circle (  2.13);

\path[fill=fillColor,fill opacity=0.20] (198.47, 89.66) circle (  2.13);

\path[fill=fillColor,fill opacity=0.20] (199.48, 95.22) circle (  2.13);

\path[fill=fillColor,fill opacity=0.20] (198.47, 95.35) circle (  2.13);

\path[fill=fillColor,fill opacity=0.20] (198.47, 96.61) circle (  2.13);

\path[fill=fillColor,fill opacity=0.20] (199.48, 97.37) circle (  2.13);

\path[fill=fillColor,fill opacity=0.20] (196.47,100.53) circle (  2.13);

\path[fill=fillColor,fill opacity=0.20] (194.46,105.21) circle (  2.13);

\path[fill=fillColor,fill opacity=0.20] (192.45,103.44) circle (  2.13);

\path[fill=fillColor,fill opacity=0.20] (196.47, 97.50) circle (  2.13);

\path[fill=fillColor,fill opacity=0.20] (196.47, 98.38) circle (  2.13);

\path[fill=fillColor,fill opacity=0.20] (193.46,102.93) circle (  2.13);

\path[fill=fillColor,fill opacity=0.20] (195.46,103.06) circle (  2.13);

\path[fill=fillColor,fill opacity=0.20] (213.52, 93.07) circle (  2.13);

\path[fill=fillColor,fill opacity=0.20] (208.51, 74.61) circle (  2.13);

\path[fill=fillColor,fill opacity=0.20] (182.72, 45.91) circle (  2.13);

\path[fill=fillColor,fill opacity=0.20] (229.57, 58.30) circle (  2.13);

\path[fill=fillColor,fill opacity=0.20] (223.55, 70.18) circle (  2.13);

\path[fill=fillColor,fill opacity=0.20] (207.50, 78.53) circle (  2.13);

\path[fill=fillColor,fill opacity=0.20] (209.51, 80.93) circle (  2.13);

\path[fill=fillColor,fill opacity=0.20] (209.51, 76.89) circle (  2.13);

\path[fill=fillColor,fill opacity=0.20] (210.51, 77.26) circle (  2.13);

\path[fill=fillColor,fill opacity=0.20] (209.51, 79.67) circle (  2.13);

\path[fill=fillColor,fill opacity=0.20] (212.52, 78.66) circle (  2.13);

\path[fill=fillColor,fill opacity=0.20] (210.51, 76.38) circle (  2.13);

\path[fill=fillColor,fill opacity=0.20] (214.53, 78.28) circle (  2.13);

\path[fill=fillColor,fill opacity=0.20] (218.54, 82.45) circle (  2.13);

\path[fill=fillColor,fill opacity=0.20] (220.54, 76.89) circle (  2.13);

\path[fill=fillColor,fill opacity=0.20] (220.54, 69.05) circle (  2.13);

\path[fill=fillColor,fill opacity=0.20] (218.54, 69.55) circle (  2.13);

\path[fill=fillColor,fill opacity=0.20] (219.54, 64.49) circle (  2.13);

\path[fill=fillColor,fill opacity=0.20] (212.52, 58.93) circle (  2.13);

\path[fill=fillColor,fill opacity=0.20] (212.52, 62.09) circle (  2.13);

\path[fill=fillColor,fill opacity=0.20] (211.52, 70.69) circle (  2.13);

\path[fill=fillColor,fill opacity=0.20] (200.48, 81.31) circle (  2.13);

\path[fill=fillColor,fill opacity=0.20] (200.48, 83.46) circle (  2.13);

\path[fill=fillColor,fill opacity=0.20] (206.50, 83.21) circle (  2.13);

\path[fill=fillColor,fill opacity=0.20] (206.50, 88.90) circle (  2.13);

\path[fill=fillColor,fill opacity=0.20] (196.47, 89.53) circle (  2.13);

\path[fill=fillColor,fill opacity=0.20] (193.46, 90.16) circle (  2.13);

\path[fill=fillColor,fill opacity=0.20] (200.48, 98.89) circle (  2.13);

\path[fill=fillColor,fill opacity=0.20] (198.47,102.30) circle (  2.13);

\path[fill=fillColor,fill opacity=0.20] (197.47, 96.11) circle (  2.13);

\path[fill=fillColor,fill opacity=0.20] (199.48, 94.59) circle (  2.13);

\path[fill=fillColor,fill opacity=0.20] (198.47, 92.94) circle (  2.13);

\path[fill=fillColor,fill opacity=0.20] (199.48, 92.31) circle (  2.13);

\path[fill=fillColor,fill opacity=0.20] (199.48, 99.65) circle (  2.13);

\path[fill=fillColor,fill opacity=0.20] (199.48,100.91) circle (  2.13);

\path[fill=fillColor,fill opacity=0.20] (202.49, 91.93) circle (  2.13);

\path[fill=fillColor,fill opacity=0.20] (201.48, 93.83) circle (  2.13);

\path[fill=fillColor,fill opacity=0.20] (194.46,104.07) circle (  2.13);

\path[fill=fillColor,fill opacity=0.20] (197.47,106.10) circle (  2.13);

\path[fill=fillColor,fill opacity=0.20] (216.53, 97.50) circle (  2.13);

\path[fill=fillColor,fill opacity=0.20] (223.55, 72.59) circle (  2.13);

\path[fill=fillColor,fill opacity=0.20] (243.62, 39.46) circle (  2.13);

\path[fill=fillColor,fill opacity=0.20] (232.58, 49.95) circle (  2.13);

\path[fill=fillColor,fill opacity=0.20] (219.54, 65.38) circle (  2.13);

\path[fill=fillColor,fill opacity=0.20] (219.54, 71.95) circle (  2.13);

\path[fill=fillColor,fill opacity=0.20] (217.53, 72.84) circle (  2.13);

\path[fill=fillColor,fill opacity=0.20] (208.51, 81.56) circle (  2.13);

\path[fill=fillColor,fill opacity=0.20] (212.52, 81.69) circle (  2.13);

\path[fill=fillColor,fill opacity=0.20] (214.53, 77.39) circle (  2.13);

\path[fill=fillColor,fill opacity=0.20] (206.50, 85.61) circle (  2.13);

\path[fill=fillColor,fill opacity=0.20] (216.53, 88.77) circle (  2.13);

\path[fill=fillColor,fill opacity=0.20] (214.53, 75.37) circle (  2.13);

\path[fill=fillColor,fill opacity=0.20] (214.53, 72.97) circle (  2.13);

\path[fill=fillColor,fill opacity=0.20] (214.53, 87.13) circle (  2.13);

\path[fill=fillColor,fill opacity=0.20] (214.53, 84.73) circle (  2.13);

\path[fill=fillColor,fill opacity=0.20] (218.54, 63.23) circle (  2.13);

\path[fill=fillColor,fill opacity=0.20] (210.51, 44.77) circle (  2.13);

\path[fill=fillColor,fill opacity=0.20] (207.50, 61.33) circle (  2.13);

\path[fill=fillColor,fill opacity=0.20] (209.51, 81.94) circle (  2.13);

\path[fill=fillColor,fill opacity=0.20] (215.53, 77.90) circle (  2.13);

\path[fill=fillColor,fill opacity=0.20] (215.53, 71.32) circle (  2.13);

\path[fill=fillColor,fill opacity=0.20] (207.50, 72.97) circle (  2.13);

\path[fill=fillColor,fill opacity=0.20] (212.52, 81.44) circle (  2.13);

\path[fill=fillColor,fill opacity=0.20] (210.51, 95.35) circle (  2.13);

\path[fill=fillColor,fill opacity=0.20] (208.51,101.92) circle (  2.13);

\path[fill=fillColor,fill opacity=0.20] (208.51, 95.85) circle (  2.13);

\path[fill=fillColor,fill opacity=0.20] (206.50, 92.31) circle (  2.13);

\path[fill=fillColor,fill opacity=0.20] (200.48, 94.71) circle (  2.13);

\path[fill=fillColor,fill opacity=0.20] (200.48, 99.90) circle (  2.13);

\path[fill=fillColor,fill opacity=0.20] (202.49,105.72) circle (  2.13);

\path[fill=fillColor,fill opacity=0.20] (193.46,104.96) circle (  2.13);

\path[fill=fillColor,fill opacity=0.20] (201.48, 98.26) circle (  2.13);

\path[fill=fillColor,fill opacity=0.20] (200.48, 95.98) circle (  2.13);

\path[fill=fillColor,fill opacity=0.20] (202.49, 93.83) circle (  2.13);

\path[fill=fillColor,fill opacity=0.20] (203.49, 92.44) circle (  2.13);

\path[fill=fillColor,fill opacity=0.20] (203.49, 96.99) circle (  2.13);

\path[fill=fillColor,fill opacity=0.20] (200.48, 98.51) circle (  2.13);

\path[fill=fillColor,fill opacity=0.20] (199.48, 91.43) circle (  2.13);

\path[fill=fillColor,fill opacity=0.20] (197.47, 90.04) circle (  2.13);

\path[fill=fillColor,fill opacity=0.20] (202.49, 99.27) circle (  2.13);

\path[fill=fillColor,fill opacity=0.20] (205.50,106.85) circle (  2.13);

\path[fill=fillColor,fill opacity=0.20] (219.54, 91.43) circle (  2.13);

\path[fill=fillColor,fill opacity=0.20] (243.62, 39.46) circle (  2.13);

\path[fill=fillColor,fill opacity=0.20] (232.58, 44.01) circle (  2.13);

\path[fill=fillColor,fill opacity=0.20] (233.59, 51.85) circle (  2.13);

\path[fill=fillColor,fill opacity=0.20] (213.52, 64.75) circle (  2.13);

\path[fill=fillColor,fill opacity=0.20] (219.54, 66.14) circle (  2.13);

\path[fill=fillColor,fill opacity=0.20] (217.53, 67.78) circle (  2.13);

\path[fill=fillColor,fill opacity=0.20] (214.53, 81.44) circle (  2.13);

\path[fill=fillColor,fill opacity=0.20] (209.51, 85.23) circle (  2.13);

\path[fill=fillColor,fill opacity=0.20] (212.52, 75.24) circle (  2.13);

\path[fill=fillColor,fill opacity=0.20] (215.53, 78.91) circle (  2.13);

\path[fill=fillColor,fill opacity=0.20] (215.53, 92.06) circle (  2.13);

\path[fill=fillColor,fill opacity=0.20] (208.51, 85.10) circle (  2.13);

\path[fill=fillColor,fill opacity=0.20] (213.52, 69.42) circle (  2.13);

\path[fill=fillColor,fill opacity=0.20] (217.53, 65.88) circle (  2.13);

\path[fill=fillColor,fill opacity=0.20] (212.52, 69.93) circle (  2.13);

\path[fill=fillColor,fill opacity=0.20] (216.53, 70.31) circle (  2.13);

\path[fill=fillColor,fill opacity=0.20] (219.54, 61.96) circle (  2.13);

\path[fill=fillColor,fill opacity=0.20] (211.52, 51.85) circle (  2.13);

\path[fill=fillColor,fill opacity=0.20] (213.52, 61.21) circle (  2.13);

\path[fill=fillColor,fill opacity=0.20] (216.53, 59.18) circle (  2.13);

\path[fill=fillColor,fill opacity=0.20] (224.56, 53.87) circle (  2.13);

\path[fill=fillColor,fill opacity=0.20] (220.54, 55.77) circle (  2.13);

\path[fill=fillColor,fill opacity=0.20] (216.53, 63.36) circle (  2.13);

\path[fill=fillColor,fill opacity=0.20] (215.53, 75.49) circle (  2.13);

\path[fill=fillColor,fill opacity=0.20] (208.51, 94.34) circle (  2.13);

\path[fill=fillColor,fill opacity=0.20] (209.51,104.96) circle (  2.13);

\path[fill=fillColor,fill opacity=0.20] (212.52,101.92) circle (  2.13);

\path[fill=fillColor,fill opacity=0.20] (210.51,101.04) circle (  2.13);

\path[fill=fillColor,fill opacity=0.20] (205.50,100.03) circle (  2.13);

\path[fill=fillColor,fill opacity=0.20] (205.50, 95.47) circle (  2.13);

\path[fill=fillColor,fill opacity=0.20] (199.48, 95.47) circle (  2.13);

\path[fill=fillColor,fill opacity=0.20] (204.49, 99.27) circle (  2.13);

\path[fill=fillColor,fill opacity=0.20] (201.48,100.91) circle (  2.13);

\path[fill=fillColor,fill opacity=0.20] (200.48, 99.39) circle (  2.13);

\path[fill=fillColor,fill opacity=0.20] (200.48, 99.39) circle (  2.13);

\path[fill=fillColor,fill opacity=0.20] (205.50, 99.27) circle (  2.13);

\path[fill=fillColor,fill opacity=0.20] (204.49, 95.35) circle (  2.13);

\path[fill=fillColor,fill opacity=0.20] (205.50, 93.07) circle (  2.13);

\path[fill=fillColor,fill opacity=0.20] (206.50, 95.35) circle (  2.13);

\path[fill=fillColor,fill opacity=0.20] (204.49, 97.24) circle (  2.13);

\path[fill=fillColor,fill opacity=0.20] (196.47, 93.45) circle (  2.13);

\path[fill=fillColor,fill opacity=0.20] (193.46, 91.05) circle (  2.13);

\path[fill=fillColor,fill opacity=0.20] (208.51, 96.86) circle (  2.13);

\path[fill=fillColor,fill opacity=0.20] (217.53, 91.17) circle (  2.13);

\path[fill=fillColor,fill opacity=0.20] (217.53, 62.60) circle (  2.13);

\path[fill=fillColor,fill opacity=0.20] (259.67, 39.71) circle (  2.13);

\path[fill=fillColor,fill opacity=0.20] (229.57, 43.50) circle (  2.13);

\path[fill=fillColor,fill opacity=0.20] (226.56, 49.57) circle (  2.13);

\path[fill=fillColor,fill opacity=0.20] (220.54, 60.19) circle (  2.13);

\path[fill=fillColor,fill opacity=0.20] (209.51, 67.40) circle (  2.13);

\path[fill=fillColor,fill opacity=0.20] (210.51, 73.22) circle (  2.13);

\path[fill=fillColor,fill opacity=0.20] (217.53, 81.31) circle (  2.13);

\path[fill=fillColor,fill opacity=0.20] (216.53, 86.37) circle (  2.13);

\path[fill=fillColor,fill opacity=0.20] (213.52, 87.00) circle (  2.13);

\path[fill=fillColor,fill opacity=0.20] (214.53, 81.56) circle (  2.13);

\path[fill=fillColor,fill opacity=0.20] (216.53, 76.51) circle (  2.13);

\path[fill=fillColor,fill opacity=0.20] (205.50, 79.16) circle (  2.13);

\path[fill=fillColor,fill opacity=0.20] (222.55, 78.91) circle (  2.13);

\path[fill=fillColor,fill opacity=0.20] (224.56, 72.97) circle (  2.13);

\path[fill=fillColor,fill opacity=0.20] (221.55, 74.48) circle (  2.13);

\path[fill=fillColor,fill opacity=0.20] (216.53, 79.16) circle (  2.13);

\path[fill=fillColor,fill opacity=0.20] (216.53, 72.46) circle (  2.13);

\path[fill=fillColor,fill opacity=0.20] (211.52, 56.40) circle (  2.13);

\path[fill=fillColor,fill opacity=0.20] (216.53, 62.60) circle (  2.13);

\path[fill=fillColor,fill opacity=0.20] (221.55, 67.65) circle (  2.13);

\path[fill=fillColor,fill opacity=0.20] (208.51, 63.73) circle (  2.13);

\path[fill=fillColor,fill opacity=0.20] (218.54, 63.48) circle (  2.13);

\path[fill=fillColor,fill opacity=0.20] (213.52, 71.70) circle (  2.13);

\path[fill=fillColor,fill opacity=0.20] (218.54, 73.09) circle (  2.13);

\path[fill=fillColor,fill opacity=0.20] (211.52, 64.87) circle (  2.13);

\path[fill=fillColor,fill opacity=0.20] (211.52, 62.22) circle (  2.13);

\path[fill=fillColor,fill opacity=0.20] (209.51, 61.08) circle (  2.13);

\path[fill=fillColor,fill opacity=0.20] (217.53, 55.14) circle (  2.13);

\path[fill=fillColor,fill opacity=0.20] (226.56, 52.61) circle (  2.13);

\path[fill=fillColor,fill opacity=0.20] (233.59, 47.80) circle (  2.13);

\path[fill=fillColor,fill opacity=0.20] (238.60, 38.70) circle (  2.13);

\path[fill=fillColor,fill opacity=0.20] (217.53, 60.45) circle (  2.13);

\path[fill=fillColor,fill opacity=0.20] (212.52, 66.01) circle (  2.13);

\path[fill=fillColor,fill opacity=0.20] (211.52, 69.55) circle (  2.13);

\path[fill=fillColor,fill opacity=0.20] (207.50, 78.66) circle (  2.13);

\path[fill=fillColor,fill opacity=0.20] (195.46, 88.65) circle (  2.13);

\path[fill=fillColor,fill opacity=0.20] (206.50, 89.66) circle (  2.13);

\path[fill=fillColor,fill opacity=0.20] (200.48, 90.54) circle (  2.13);

\path[fill=fillColor,fill opacity=0.20] (201.48, 91.93) circle (  2.13);

\path[fill=fillColor,fill opacity=0.20] (203.49, 91.68) circle (  2.13);

\path[fill=fillColor,fill opacity=0.20] (204.49, 93.83) circle (  2.13);

\path[fill=fillColor,fill opacity=0.20] (204.49, 95.98) circle (  2.13);

\path[fill=fillColor,fill opacity=0.20] (202.49, 93.83) circle (  2.13);

\path[fill=fillColor,fill opacity=0.20] (202.49, 90.79) circle (  2.13);

\path[fill=fillColor,fill opacity=0.20] (197.47, 89.28) circle (  2.13);

\path[fill=fillColor,fill opacity=0.20] (204.49, 90.79) circle (  2.13);

\path[fill=fillColor,fill opacity=0.20] (200.48, 91.81) circle (  2.13);

\path[fill=fillColor,fill opacity=0.20] (208.51, 89.40) circle (  2.13);

\path[fill=fillColor,fill opacity=0.20] (221.55, 79.92) circle (  2.13);

\path[fill=fillColor,fill opacity=0.20] (230.58, 60.95) circle (  2.13);

\path[fill=fillColor,fill opacity=0.20] (222.55, 38.82) circle (  2.13);

\path[fill=fillColor,fill opacity=0.20] (217.53, 45.27) circle (  2.13);

\path[fill=fillColor,fill opacity=0.20] (221.55, 54.00) circle (  2.13);

\path[fill=fillColor,fill opacity=0.20] (222.55, 60.57) circle (  2.13);

\path[fill=fillColor,fill opacity=0.20] (216.53, 66.26) circle (  2.13);

\path[fill=fillColor,fill opacity=0.20] (209.51, 79.92) circle (  2.13);

\path[fill=fillColor,fill opacity=0.20] (218.54, 90.79) circle (  2.13);

\path[fill=fillColor,fill opacity=0.20] (217.53, 87.51) circle (  2.13);

\path[fill=fillColor,fill opacity=0.20] (218.54, 84.85) circle (  2.13);

\path[fill=fillColor,fill opacity=0.20] (218.54, 78.78) circle (  2.13);

\path[fill=fillColor,fill opacity=0.20] (212.52, 72.97) circle (  2.13);

\path[fill=fillColor,fill opacity=0.20] (214.53, 77.01) circle (  2.13);

\path[fill=fillColor,fill opacity=0.20] (224.56, 86.62) circle (  2.13);

\path[fill=fillColor,fill opacity=0.20] (214.53, 86.50) circle (  2.13);

\path[fill=fillColor,fill opacity=0.20] (218.54, 80.81) circle (  2.13);

\path[fill=fillColor,fill opacity=0.20] (222.55, 74.23) circle (  2.13);

\path[fill=fillColor,fill opacity=0.20] (224.56, 70.18) circle (  2.13);

\path[fill=fillColor,fill opacity=0.20] (224.56, 71.83) circle (  2.13);

\path[fill=fillColor,fill opacity=0.20] (221.55, 75.49) circle (  2.13);

\path[fill=fillColor,fill opacity=0.20] (220.54, 76.25) circle (  2.13);

\path[fill=fillColor,fill opacity=0.20] (218.54, 72.97) circle (  2.13);

\path[fill=fillColor,fill opacity=0.20] (218.54, 67.53) circle (  2.13);

\path[fill=fillColor,fill opacity=0.20] (220.54, 65.00) circle (  2.13);

\path[fill=fillColor,fill opacity=0.20] (220.54, 69.68) circle (  2.13);

\path[fill=fillColor,fill opacity=0.20] (217.53, 75.87) circle (  2.13);

\path[fill=fillColor,fill opacity=0.20] (210.51, 79.29) circle (  2.13);

\path[fill=fillColor,fill opacity=0.20] (214.53, 82.07) circle (  2.13);

\path[fill=fillColor,fill opacity=0.20] (210.51, 83.59) circle (  2.13);

\path[fill=fillColor,fill opacity=0.20] (209.51, 82.83) circle (  2.13);

\path[fill=fillColor,fill opacity=0.20] (215.53, 85.23) circle (  2.13);

\path[fill=fillColor,fill opacity=0.20] (212.52, 84.47) circle (  2.13);

\path[fill=fillColor,fill opacity=0.20] (210.51, 74.23) circle (  2.13);

\path[fill=fillColor,fill opacity=0.20] (211.52, 69.17) circle (  2.13);

\path[fill=fillColor,fill opacity=0.20] (211.52, 73.09) circle (  2.13);

\path[fill=fillColor,fill opacity=0.20] (215.53, 72.46) circle (  2.13);

\path[fill=fillColor,fill opacity=0.20] (214.53, 62.60) circle (  2.13);

\path[fill=fillColor,fill opacity=0.20] (206.50, 55.01) circle (  2.13);

\path[fill=fillColor,fill opacity=0.20] (218.54, 46.92) circle (  2.13);

\path[fill=fillColor,fill opacity=0.20] (219.54, 38.82) circle (  2.13);

\path[fill=fillColor,fill opacity=0.20] (219.54, 42.62) circle (  2.13);

\path[fill=fillColor,fill opacity=0.20] (217.53, 48.56) circle (  2.13);

\path[fill=fillColor,fill opacity=0.20] (206.50, 55.39) circle (  2.13);

\path[fill=fillColor,fill opacity=0.20] (200.48, 65.63) circle (  2.13);

\path[fill=fillColor,fill opacity=0.20] (202.49, 72.71) circle (  2.13);

\path[fill=fillColor,fill opacity=0.20] (207.50, 74.99) circle (  2.13);

\path[fill=fillColor,fill opacity=0.20] (206.50, 77.77) circle (  2.13);

\path[fill=fillColor,fill opacity=0.20] (201.48, 79.67) circle (  2.13);

\path[fill=fillColor,fill opacity=0.20] (203.49, 80.43) circle (  2.13);

\path[fill=fillColor,fill opacity=0.20] (207.50, 79.92) circle (  2.13);

\path[fill=fillColor,fill opacity=0.20] (207.50, 74.10) circle (  2.13);

\path[fill=fillColor,fill opacity=0.20] (208.51, 71.20) circle (  2.13);

\path[fill=fillColor,fill opacity=0.20] (212.52, 72.46) circle (  2.13);

\path[fill=fillColor,fill opacity=0.20] (215.53, 63.73) circle (  2.13);

\path[fill=fillColor,fill opacity=0.20] (224.56, 48.05) circle (  2.13);

\path[fill=fillColor,fill opacity=0.20] (230.58, 65.13) circle (  2.13);

\path[fill=fillColor,fill opacity=0.20] (222.55, 72.97) circle (  2.13);

\path[fill=fillColor,fill opacity=0.20] (216.53, 72.97) circle (  2.13);

\path[fill=fillColor,fill opacity=0.20] (213.52, 72.84) circle (  2.13);

\path[fill=fillColor,fill opacity=0.20] (213.52, 74.74) circle (  2.13);

\path[fill=fillColor,fill opacity=0.20] (221.55, 78.78) circle (  2.13);

\path[fill=fillColor,fill opacity=0.20] (214.53, 82.58) circle (  2.13);

\path[fill=fillColor,fill opacity=0.20] (219.54, 85.48) circle (  2.13);

\path[fill=fillColor,fill opacity=0.20] (217.53, 85.48) circle (  2.13);

\path[fill=fillColor,fill opacity=0.20] (222.55, 80.30) circle (  2.13);

\path[fill=fillColor,fill opacity=0.20] (220.54, 77.52) circle (  2.13);

\path[fill=fillColor,fill opacity=0.20] (216.53, 80.68) circle (  2.13);

\path[fill=fillColor,fill opacity=0.20] (216.53, 83.59) circle (  2.13);

\path[fill=fillColor,fill opacity=0.20] (214.53, 86.37) circle (  2.13);

\path[fill=fillColor,fill opacity=0.20] (219.54, 89.78) circle (  2.13);

\path[fill=fillColor,fill opacity=0.20] (219.54, 88.27) circle (  2.13);

\path[fill=fillColor,fill opacity=0.20] (214.53, 87.00) circle (  2.13);

\path[fill=fillColor,fill opacity=0.20] (216.53, 87.25) circle (  2.13);

\path[fill=fillColor,fill opacity=0.20] (213.52, 83.21) circle (  2.13);

\path[fill=fillColor,fill opacity=0.20] (212.52, 79.54) circle (  2.13);

\path[fill=fillColor,fill opacity=0.20] (197.47, 79.92) circle (  2.13);

\path[fill=fillColor,fill opacity=0.20] (208.51, 84.73) circle (  2.13);

\path[fill=fillColor,fill opacity=0.20] (212.52, 89.91) circle (  2.13);

\path[fill=fillColor,fill opacity=0.20] (208.51, 87.63) circle (  2.13);

\path[fill=fillColor,fill opacity=0.20] (207.50, 82.45) circle (  2.13);

\path[fill=fillColor,fill opacity=0.20] (207.50, 84.85) circle (  2.13);

\path[fill=fillColor,fill opacity=0.20] (207.50, 87.25) circle (  2.13);

\path[fill=fillColor,fill opacity=0.20] (210.51, 76.38) circle (  2.13);

\path[fill=fillColor,fill opacity=0.20] (210.51, 62.98) circle (  2.13);

\path[fill=fillColor,fill opacity=0.20] (216.53, 54.12) circle (  2.13);

\path[fill=fillColor,fill opacity=0.20] (226.56, 46.79) circle (  2.13);

\path[fill=fillColor,fill opacity=0.20] (212.52, 43.76) circle (  2.13);

\path[fill=fillColor,fill opacity=0.20] (214.53, 47.42) circle (  2.13);

\path[fill=fillColor,fill opacity=0.20] (212.52, 53.11) circle (  2.13);

\path[fill=fillColor,fill opacity=0.20] (206.50, 56.40) circle (  2.13);

\path[fill=fillColor,fill opacity=0.20] (205.50, 57.41) circle (  2.13);

\path[fill=fillColor,fill opacity=0.20] (207.50, 56.65) circle (  2.13);

\path[fill=fillColor,fill opacity=0.20] (210.51, 51.97) circle (  2.13);

\path[fill=fillColor,fill opacity=0.20] (220.54, 46.79) circle (  2.13);

\path[fill=fillColor,fill opacity=0.20] (217.53, 59.56) circle (  2.13);

\path[fill=fillColor,fill opacity=0.20] (214.53, 67.02) circle (  2.13);

\path[fill=fillColor,fill opacity=0.20] (209.51, 72.97) circle (  2.13);

\path[fill=fillColor,fill opacity=0.20] (212.52, 74.36) circle (  2.13);

\path[fill=fillColor,fill opacity=0.20] (215.53, 73.72) circle (  2.13);

\path[fill=fillColor,fill opacity=0.20] (219.54, 77.64) circle (  2.13);

\path[fill=fillColor,fill opacity=0.20] (217.53, 81.06) circle (  2.13);

\path[fill=fillColor,fill opacity=0.20] (214.53, 81.44) circle (  2.13);

\path[fill=fillColor,fill opacity=0.20] (210.51, 83.46) circle (  2.13);

\path[fill=fillColor,fill opacity=0.20] (207.50, 83.46) circle (  2.13);

\path[fill=fillColor,fill opacity=0.20] (214.53, 81.56) circle (  2.13);

\path[fill=fillColor,fill opacity=0.20] (217.53, 83.71) circle (  2.13);

\path[fill=fillColor,fill opacity=0.20] (214.53, 86.50) circle (  2.13);

\path[fill=fillColor,fill opacity=0.20] (212.52, 87.63) circle (  2.13);

\path[fill=fillColor,fill opacity=0.20] (209.51, 88.77) circle (  2.13);

\path[fill=fillColor,fill opacity=0.20] (211.52, 88.01) circle (  2.13);

\path[fill=fillColor,fill opacity=0.20] (204.49, 84.09) circle (  2.13);

\path[fill=fillColor,fill opacity=0.20] (201.48, 79.29) circle (  2.13);

\path[fill=fillColor,fill opacity=0.20] (204.49, 80.17) circle (  2.13);

\path[fill=fillColor,fill opacity=0.20] (205.50, 87.25) circle (  2.13);

\path[fill=fillColor,fill opacity=0.20] (204.49, 88.52) circle (  2.13);

\path[fill=fillColor,fill opacity=0.20] (204.49, 80.05) circle (  2.13);

\path[fill=fillColor,fill opacity=0.20] (206.50, 74.74) circle (  2.13);

\path[fill=fillColor,fill opacity=0.20] (215.53, 71.32) circle (  2.13);

\path[fill=fillColor,fill opacity=0.20] (217.53, 59.81) circle (  2.13);

\path[fill=fillColor,fill opacity=0.20] (220.54, 46.16) circle (  2.13);

\path[fill=fillColor,fill opacity=0.20] (232.58, 38.82) circle (  2.13);

\path[fill=fillColor,fill opacity=0.20] (207.50, 47.68) circle (  2.13);

\path[fill=fillColor,fill opacity=0.20] (216.53, 57.54) circle (  2.13);

\path[fill=fillColor,fill opacity=0.20] (218.54, 60.07) circle (  2.13);

\path[fill=fillColor,fill opacity=0.20] (219.54, 65.00) circle (  2.13);

\path[fill=fillColor,fill opacity=0.20] (215.53, 75.87) circle (  2.13);

\path[fill=fillColor,fill opacity=0.20] (213.52, 80.17) circle (  2.13);

\path[fill=fillColor,fill opacity=0.20] (210.51, 77.14) circle (  2.13);

\path[fill=fillColor,fill opacity=0.20] (211.52, 77.64) circle (  2.13);

\path[fill=fillColor,fill opacity=0.20] (207.50, 80.17) circle (  2.13);

\path[fill=fillColor,fill opacity=0.20] (209.51, 81.06) circle (  2.13);

\path[fill=fillColor,fill opacity=0.20] (209.51, 82.20) circle (  2.13);

\path[fill=fillColor,fill opacity=0.20] (204.49, 86.12) circle (  2.13);

\path[fill=fillColor,fill opacity=0.20] (205.50, 88.90) circle (  2.13);

\path[fill=fillColor,fill opacity=0.20] (206.50, 88.90) circle (  2.13);

\path[fill=fillColor,fill opacity=0.20] (199.48, 89.78) circle (  2.13);

\path[fill=fillColor,fill opacity=0.20] (198.47, 87.13) circle (  2.13);

\path[fill=fillColor,fill opacity=0.20] (208.51, 81.94) circle (  2.13);

\path[fill=fillColor,fill opacity=0.20] (209.51, 81.82) circle (  2.13);

\path[fill=fillColor,fill opacity=0.20] (208.51, 82.07) circle (  2.13);

\path[fill=fillColor,fill opacity=0.20] (208.51, 72.71) circle (  2.13);

\path[fill=fillColor,fill opacity=0.20] (207.50, 59.69) circle (  2.13);

\path[fill=fillColor,fill opacity=0.20] (222.55, 47.42) circle (  2.13);

\path[fill=fillColor,fill opacity=0.20] (220.54, 52.86) circle (  2.13);

\path[fill=fillColor,fill opacity=0.20] (212.52, 58.55) circle (  2.13);

\path[fill=fillColor,fill opacity=0.20] (196.47, 59.44) circle (  2.13);

\path[fill=fillColor,fill opacity=0.20] (218.54, 64.87) circle (  2.13);

\path[fill=fillColor,fill opacity=0.20] (211.52, 79.16) circle (  2.13);

\path[fill=fillColor,fill opacity=0.20] (205.50, 85.48) circle (  2.13);

\path[fill=fillColor,fill opacity=0.20] (207.50, 80.55) circle (  2.13);

\path[fill=fillColor,fill opacity=0.20] (205.50, 82.45) circle (  2.13);

\path[fill=fillColor,fill opacity=0.20] (205.50, 84.98) circle (  2.13);

\path[fill=fillColor,fill opacity=0.20] (207.50, 79.79) circle (  2.13);

\path[fill=fillColor,fill opacity=0.20] (197.47, 77.14) circle (  2.13);

\path[fill=fillColor,fill opacity=0.20] (211.52, 79.41) circle (  2.13);

\path[fill=fillColor,fill opacity=0.20] (215.53, 76.38) circle (  2.13);

\path[fill=fillColor,fill opacity=0.20] (208.51, 68.29) circle (  2.13);

\path[fill=fillColor,fill opacity=0.20] (220.54, 60.19) circle (  2.13);

\path[fill=fillColor,fill opacity=0.20] (230.58, 51.85) circle (  2.13);

\path[fill=fillColor,fill opacity=0.20] (211.52, 56.65) circle (  2.13);

\path[fill=fillColor,fill opacity=0.20] (210.51, 66.64) circle (  2.13);

\path[fill=fillColor,fill opacity=0.20] (215.53, 66.77) circle (  2.13);

\path[fill=fillColor,fill opacity=0.20] (213.52, 63.23) circle (  2.13);

\path[fill=fillColor,fill opacity=0.20] (211.52, 61.08) circle (  2.13);

\path[fill=fillColor,fill opacity=0.20] (218.54, 57.16) circle (  2.13);

\path[fill=fillColor,fill opacity=0.20] (220.54, 52.23) circle (  2.13);

\path[fill=fillColor,fill opacity=0.20] (224.56, 51.34) circle (  2.13);

\path[fill=fillColor,fill opacity=0.20] (237.60, 52.23) circle (  2.13);

\path[fill=fillColor,fill opacity=0.20] (187.54, 69.17) circle (  2.13);

\path[fill=fillColor,fill opacity=0.20] (186.33, 65.88) circle (  2.13);

\path[fill=fillColor,fill opacity=0.20] (204.49, 63.36) circle (  2.13);

\path[fill=fillColor,fill opacity=0.20] (203.49, 93.70) circle (  2.13);

\path[fill=fillColor,fill opacity=0.20] (201.48, 98.89) circle (  2.13);

\path[fill=fillColor,fill opacity=0.20] (200.48, 93.83) circle (  2.13);

\path[fill=fillColor,fill opacity=0.20] (201.48, 86.37) circle (  2.13);

\path[fill=fillColor,fill opacity=0.20] (203.49, 85.74) circle (  2.13);

\path[fill=fillColor,fill opacity=0.20] (242.62, 46.92) circle (  2.13);

\path[fill=fillColor,fill opacity=0.20] (245.63, 61.71) circle (  2.13);

\path[fill=fillColor,fill opacity=0.20] (241.61, 70.44) circle (  2.13);

\path[fill=fillColor,fill opacity=0.20] (237.60, 62.98) circle (  2.13);

\path[fill=fillColor,fill opacity=0.20] (238.60, 65.38) circle (  2.13);

\path[fill=fillColor,fill opacity=0.20] (241.61, 69.05) circle (  2.13);

\path[fill=fillColor,fill opacity=0.20] (208.51,100.28) circle (  2.13);

\path[fill=fillColor,fill opacity=0.20] (203.49,105.34) circle (  2.13);

\path[fill=fillColor,fill opacity=0.20] (200.48, 97.88) circle (  2.13);

\path[fill=fillColor,fill opacity=0.20] (196.47, 97.24) circle (  2.13);

\path[fill=fillColor,fill opacity=0.20] (196.47, 96.11) circle (  2.13);

\path[fill=fillColor,fill opacity=0.20] (199.48, 91.05) circle (  2.13);

\path[fill=fillColor,fill opacity=0.20] (202.49, 84.35) circle (  2.13);

\path[fill=fillColor,fill opacity=0.20] (205.50, 76.13) circle (  2.13);

\path[fill=fillColor,fill opacity=0.20] (240.61, 40.21) circle (  2.13);

\path[fill=fillColor,fill opacity=0.20] (235.59, 52.73) circle (  2.13);

\path[fill=fillColor,fill opacity=0.20] (239.61, 59.18) circle (  2.13);

\path[fill=fillColor,fill opacity=0.20] (230.58, 72.97) circle (  2.13);

\path[fill=fillColor,fill opacity=0.20] (220.54, 84.73) circle (  2.13);

\path[fill=fillColor,fill opacity=0.20] (217.53, 80.05) circle (  2.13);

\path[fill=fillColor,fill opacity=0.20] (221.55, 80.30) circle (  2.13);

\path[fill=fillColor,fill opacity=0.20] (224.56, 81.44) circle (  2.13);

\path[fill=fillColor,fill opacity=0.20] (228.57, 77.01) circle (  2.13);

\path[fill=fillColor,fill opacity=0.20] (232.58, 66.77) circle (  2.13);

\path[fill=fillColor,fill opacity=0.20] (208.51, 80.81) circle (  2.13);

\path[fill=fillColor,fill opacity=0.20] (201.48,106.47) circle (  2.13);

\path[fill=fillColor,fill opacity=0.20] (198.47, 93.20) circle (  2.13);

\path[fill=fillColor,fill opacity=0.20] (194.46, 98.63) circle (  2.13);

\path[fill=fillColor,fill opacity=0.20] (191.45,103.57) circle (  2.13);

\path[fill=fillColor,fill opacity=0.20] (193.46,106.47) circle (  2.13);

\path[fill=fillColor,fill opacity=0.20] (195.46,102.43) circle (  2.13);

\path[fill=fillColor,fill opacity=0.20] (200.48, 96.36) circle (  2.13);

\path[fill=fillColor,fill opacity=0.20] (206.50, 93.58) circle (  2.13);

\path[fill=fillColor,fill opacity=0.20] (210.51, 80.43) circle (  2.13);

\path[fill=fillColor,fill opacity=0.20] (202.49, 46.28) circle (  2.13);

\path[fill=fillColor,fill opacity=0.20] (228.57, 61.46) circle (  2.13);

\path[fill=fillColor,fill opacity=0.20] (218.54, 73.47) circle (  2.13);

\path[fill=fillColor,fill opacity=0.20] (223.55, 90.04) circle (  2.13);

\path[fill=fillColor,fill opacity=0.20] (218.54,100.03) circle (  2.13);

\path[fill=fillColor,fill opacity=0.20] (214.53, 92.19) circle (  2.13);

\path[fill=fillColor,fill opacity=0.20] (212.52, 83.84) circle (  2.13);

\path[fill=fillColor,fill opacity=0.20] (213.52, 88.77) circle (  2.13);

\path[fill=fillColor,fill opacity=0.20] (218.54, 89.91) circle (  2.13);

\path[fill=fillColor,fill opacity=0.20] (214.53, 77.26) circle (  2.13);

\path[fill=fillColor,fill opacity=0.20] (205.50, 99.90) circle (  2.13);

\path[fill=fillColor,fill opacity=0.20] (197.47, 96.49) circle (  2.13);

\path[fill=fillColor,fill opacity=0.20] (195.46, 87.63) circle (  2.13);

\path[fill=fillColor,fill opacity=0.20] (191.45, 98.26) circle (  2.13);

\path[fill=fillColor,fill opacity=0.20] (192.45,109.89) circle (  2.13);

\path[fill=fillColor,fill opacity=0.20] (194.46,115.20) circle (  2.13);

\path[fill=fillColor,fill opacity=0.20] (199.48,115.33) circle (  2.13);

\path[fill=fillColor,fill opacity=0.20] (203.49,114.95) circle (  2.13);

\path[fill=fillColor,fill opacity=0.20] (207.50, 96.99) circle (  2.13);

\path[fill=fillColor,fill opacity=0.20] (241.61, 44.64) circle (  2.13);

\path[fill=fillColor,fill opacity=0.20] (233.59, 78.91) circle (  2.13);

\path[fill=fillColor,fill opacity=0.20] (227.57, 78.28) circle (  2.13);

\path[fill=fillColor,fill opacity=0.20] (223.55, 79.41) circle (  2.13);

\path[fill=fillColor,fill opacity=0.20] (220.54, 94.46) circle (  2.13);

\path[fill=fillColor,fill opacity=0.20] (214.53,100.53) circle (  2.13);

\path[fill=fillColor,fill opacity=0.20] (210.51, 97.24) circle (  2.13);

\path[fill=fillColor,fill opacity=0.20] (211.52, 91.81) circle (  2.13);

\path[fill=fillColor,fill opacity=0.20] (213.52, 89.28) circle (  2.13);

\path[fill=fillColor,fill opacity=0.20] (218.54, 92.94) circle (  2.13);

\path[fill=fillColor,fill opacity=0.20] (228.57, 87.51) circle (  2.13);

\path[fill=fillColor,fill opacity=0.20] (234.59, 74.10) circle (  2.13);

\path[fill=fillColor,fill opacity=0.20] (204.49, 87.00) circle (  2.13);

\path[fill=fillColor,fill opacity=0.20] (197.47,101.04) circle (  2.13);

\path[fill=fillColor,fill opacity=0.20] (195.46, 92.69) circle (  2.13);

\path[fill=fillColor,fill opacity=0.20] (192.45, 95.09) circle (  2.13);

\path[fill=fillColor,fill opacity=0.20] (192.45,110.77) circle (  2.13);

\path[fill=fillColor,fill opacity=0.20] (193.46,114.44) circle (  2.13);

\path[fill=fillColor,fill opacity=0.20] (202.49,109.89) circle (  2.13);

\path[fill=fillColor,fill opacity=0.20] (207.50, 84.22) circle (  2.13);

\path[fill=fillColor,fill opacity=0.20] (256.66, 42.36) circle (  2.13);

\path[fill=fillColor,fill opacity=0.20] (223.55, 67.91) circle (  2.13);

\path[fill=fillColor,fill opacity=0.20] (215.53, 91.81) circle (  2.13);

\path[fill=fillColor,fill opacity=0.20] (219.54, 92.06) circle (  2.13);

\path[fill=fillColor,fill opacity=0.20] (217.53, 92.06) circle (  2.13);

\path[fill=fillColor,fill opacity=0.20] (213.52, 97.62) circle (  2.13);

\path[fill=fillColor,fill opacity=0.20] (209.51, 98.38) circle (  2.13);

\path[fill=fillColor,fill opacity=0.20] (208.51,101.04) circle (  2.13);

\path[fill=fillColor,fill opacity=0.20] (208.51, 98.26) circle (  2.13);

\path[fill=fillColor,fill opacity=0.20] (206.50, 84.98) circle (  2.13);

\path[fill=fillColor,fill opacity=0.20] (214.53, 84.35) circle (  2.13);

\path[fill=fillColor,fill opacity=0.20] (233.59, 77.39) circle (  2.13);

\path[fill=fillColor,fill opacity=0.20] (205.50, 92.44) circle (  2.13);

\path[fill=fillColor,fill opacity=0.20] (199.48,109.13) circle (  2.13);

\path[fill=fillColor,fill opacity=0.20] (197.47,101.29) circle (  2.13);

\path[fill=fillColor,fill opacity=0.20] (190.45,101.67) circle (  2.13);

\path[fill=fillColor,fill opacity=0.20] (190.45,112.04) circle (  2.13);

\path[fill=fillColor,fill opacity=0.20] (193.46,105.72) circle (  2.13);

\path[fill=fillColor,fill opacity=0.20] (194.46, 99.14) circle (  2.13);

\path[fill=fillColor,fill opacity=0.20] (196.47,105.84) circle (  2.13);

\path[fill=fillColor,fill opacity=0.20] (201.48,108.75) circle (  2.13);

\path[fill=fillColor,fill opacity=0.20] (206.50, 92.57) circle (  2.13);

\path[fill=fillColor,fill opacity=0.20] (210.51, 59.06) circle (  2.13);

\path[fill=fillColor,fill opacity=0.20] (240.61, 57.66) circle (  2.13);

\path[fill=fillColor,fill opacity=0.20] (220.54, 89.78) circle (  2.13);

\path[fill=fillColor,fill opacity=0.20] (215.53, 97.50) circle (  2.13);

\path[fill=fillColor,fill opacity=0.20] (208.51,112.92) circle (  2.13);

\path[fill=fillColor,fill opacity=0.20] (206.50,112.92) circle (  2.13);

\path[fill=fillColor,fill opacity=0.20] (202.49,102.18) circle (  2.13);

\path[fill=fillColor,fill opacity=0.20] (198.47, 98.76) circle (  2.13);

\path[fill=fillColor,fill opacity=0.20] (202.49,101.42) circle (  2.13);

\path[fill=fillColor,fill opacity=0.20] (203.49, 96.49) circle (  2.13);

\path[fill=fillColor,fill opacity=0.20] (202.49, 85.23) circle (  2.13);

\path[fill=fillColor,fill opacity=0.20] (210.51, 73.60) circle (  2.13);

\path[fill=fillColor,fill opacity=0.20] (205.50,110.90) circle (  2.13);

\path[fill=fillColor,fill opacity=0.20] (200.48,106.60) circle (  2.13);

\path[fill=fillColor,fill opacity=0.20] (199.48,104.20) circle (  2.13);

\path[fill=fillColor,fill opacity=0.20] (191.45,106.85) circle (  2.13);

\path[fill=fillColor,fill opacity=0.20] (191.45,109.89) circle (  2.13);

\path[fill=fillColor,fill opacity=0.20] (195.46,114.31) circle (  2.13);

\path[fill=fillColor,fill opacity=0.20] (195.46,103.44) circle (  2.13);

\path[fill=fillColor,fill opacity=0.20] (195.46, 93.07) circle (  2.13);

\path[fill=fillColor,fill opacity=0.20] (197.47,102.05) circle (  2.13);

\path[fill=fillColor,fill opacity=0.20] (202.49,109.38) circle (  2.13);

\path[fill=fillColor,fill opacity=0.20] (208.51, 94.84) circle (  2.13);

\path[fill=fillColor,fill opacity=0.20] (211.52, 61.21) circle (  2.13);

\path[fill=fillColor,fill opacity=0.20] (222.55, 57.03) circle (  2.13);

\path[fill=fillColor,fill opacity=0.20] (214.53, 93.58) circle (  2.13);

\path[fill=fillColor,fill opacity=0.20] (211.52,108.24) circle (  2.13);

\path[fill=fillColor,fill opacity=0.20] (194.46, 99.77) circle (  2.13);

\path[fill=fillColor,fill opacity=0.20] (196.47, 99.77) circle (  2.13);

\path[fill=fillColor,fill opacity=0.20] (198.47,105.97) circle (  2.13);

\path[fill=fillColor,fill opacity=0.20] (199.48, 93.83) circle (  2.13);

\path[fill=fillColor,fill opacity=0.20] (202.49, 80.81) circle (  2.13);

\path[fill=fillColor,fill opacity=0.20] (214.53, 60.70) circle (  2.13);

\path[fill=fillColor,fill opacity=0.20] (214.53, 96.61) circle (  2.13);

\path[fill=fillColor,fill opacity=0.20] (204.49,102.30) circle (  2.13);

\path[fill=fillColor,fill opacity=0.20] (200.48, 96.74) circle (  2.13);

\path[fill=fillColor,fill opacity=0.20] (199.48, 99.27) circle (  2.13);

\path[fill=fillColor,fill opacity=0.20] (197.47, 99.39) circle (  2.13);

\path[fill=fillColor,fill opacity=0.20] (197.47,107.61) circle (  2.13);

\path[fill=fillColor,fill opacity=0.20] (195.46,113.56) circle (  2.13);

\path[fill=fillColor,fill opacity=0.20] (193.46,106.60) circle (  2.13);

\path[fill=fillColor,fill opacity=0.20] (197.47,101.92) circle (  2.13);

\path[fill=fillColor,fill opacity=0.20] (200.48,107.74) circle (  2.13);

\path[fill=fillColor,fill opacity=0.20] (203.49,104.70) circle (  2.13);

\path[fill=fillColor,fill opacity=0.20] (210.51, 89.66) circle (  2.13);

\path[fill=fillColor,fill opacity=0.20] (246.63, 41.48) circle (  2.13);

\path[fill=fillColor,fill opacity=0.20] (218.54, 56.78) circle (  2.13);

\path[fill=fillColor,fill opacity=0.20] (210.51, 85.74) circle (  2.13);

\path[fill=fillColor,fill opacity=0.20] (210.51,104.45) circle (  2.13);

\path[fill=fillColor,fill opacity=0.20] (205.50,109.76) circle (  2.13);

\path[fill=fillColor,fill opacity=0.20] (203.49,102.81) circle (  2.13);

\path[fill=fillColor,fill opacity=0.20] (200.48, 92.57) circle (  2.13);

\path[fill=fillColor,fill opacity=0.20] (199.48, 91.81) circle (  2.13);

\path[fill=fillColor,fill opacity=0.20] (200.48,101.04) circle (  2.13);

\path[fill=fillColor,fill opacity=0.20] (205.50, 92.44) circle (  2.13);

\path[fill=fillColor,fill opacity=0.20] (212.52, 72.33) circle (  2.13);

\path[fill=fillColor,fill opacity=0.20] (222.55, 70.69) circle (  2.13);

\path[fill=fillColor,fill opacity=0.20] (210.51, 87.76) circle (  2.13);

\path[fill=fillColor,fill opacity=0.20] (202.49, 90.54) circle (  2.13);

\path[fill=fillColor,fill opacity=0.20] (199.48, 96.36) circle (  2.13);

\path[fill=fillColor,fill opacity=0.20] (199.48, 96.36) circle (  2.13);

\path[fill=fillColor,fill opacity=0.20] (200.48, 98.38) circle (  2.13);

\path[fill=fillColor,fill opacity=0.20] (199.48,106.10) circle (  2.13);

\path[fill=fillColor,fill opacity=0.20] (196.47,111.91) circle (  2.13);

\path[fill=fillColor,fill opacity=0.20] (195.46,109.76) circle (  2.13);

\path[fill=fillColor,fill opacity=0.20] (200.48,106.98) circle (  2.13);

\path[fill=fillColor,fill opacity=0.20] (201.48, 98.00) circle (  2.13);

\path[fill=fillColor,fill opacity=0.20] (207.50, 90.04) circle (  2.13);

\path[fill=fillColor,fill opacity=0.20] (214.53, 77.14) circle (  2.13);

\path[fill=fillColor,fill opacity=0.20] (240.61, 45.02) circle (  2.13);

\path[fill=fillColor,fill opacity=0.20] (218.54, 61.08) circle (  2.13);

\path[fill=fillColor,fill opacity=0.20] (212.52, 79.41) circle (  2.13);

\path[fill=fillColor,fill opacity=0.20] (206.50, 97.50) circle (  2.13);

\path[fill=fillColor,fill opacity=0.20] (203.49,105.34) circle (  2.13);

\path[fill=fillColor,fill opacity=0.20] (201.48, 99.65) circle (  2.13);

\path[fill=fillColor,fill opacity=0.20] (203.49, 93.58) circle (  2.13);

\path[fill=fillColor,fill opacity=0.20] (204.49, 89.78) circle (  2.13);

\path[fill=fillColor,fill opacity=0.20] (206.50, 88.65) circle (  2.13);

\path[fill=fillColor,fill opacity=0.20] (212.52, 87.25) circle (  2.13);

\path[fill=fillColor,fill opacity=0.20] (214.53, 73.34) circle (  2.13);

\path[fill=fillColor,fill opacity=0.20] (208.51, 79.92) circle (  2.13);

\path[fill=fillColor,fill opacity=0.20] (202.49, 98.76) circle (  2.13);

\path[fill=fillColor,fill opacity=0.20] (199.48,105.84) circle (  2.13);

\path[fill=fillColor,fill opacity=0.20] (198.47,100.66) circle (  2.13);

\path[fill=fillColor,fill opacity=0.20] (194.46,104.45) circle (  2.13);

\path[fill=fillColor,fill opacity=0.20] (199.48,108.37) circle (  2.13);

\path[fill=fillColor,fill opacity=0.20] (195.46,106.47) circle (  2.13);

\path[fill=fillColor,fill opacity=0.20] (196.47,105.72) circle (  2.13);

\path[fill=fillColor,fill opacity=0.20] (200.48, 97.75) circle (  2.13);

\path[fill=fillColor,fill opacity=0.20] (204.49, 83.71) circle (  2.13);

\path[fill=fillColor,fill opacity=0.20] (209.51, 78.40) circle (  2.13);

\path[fill=fillColor,fill opacity=0.20] (216.53, 59.81) circle (  2.13);

\path[fill=fillColor,fill opacity=0.20] (215.53, 59.69) circle (  2.13);

\path[fill=fillColor,fill opacity=0.20] (206.50, 79.79) circle (  2.13);

\path[fill=fillColor,fill opacity=0.20] (201.48, 90.92) circle (  2.13);

\path[fill=fillColor,fill opacity=0.20] (199.48, 97.50) circle (  2.13);

\path[fill=fillColor,fill opacity=0.20] (200.48, 96.36) circle (  2.13);

\path[fill=fillColor,fill opacity=0.20] (203.49, 96.61) circle (  2.13);

\path[fill=fillColor,fill opacity=0.20] (205.50, 96.23) circle (  2.13);

\path[fill=fillColor,fill opacity=0.20] (207.50, 85.10) circle (  2.13);

\path[fill=fillColor,fill opacity=0.20] (215.53, 78.28) circle (  2.13);

\path[fill=fillColor,fill opacity=0.20] (198.47, 70.94) circle (  2.13);

\path[fill=fillColor,fill opacity=0.20] (212.52, 78.02) circle (  2.13);

\path[fill=fillColor,fill opacity=0.20] (204.49, 87.00) circle (  2.13);

\path[fill=fillColor,fill opacity=0.20] (200.48,106.35) circle (  2.13);

\path[fill=fillColor,fill opacity=0.20] (197.47,115.45) circle (  2.13);

\path[fill=fillColor,fill opacity=0.20] (194.46,105.46) circle (  2.13);

\path[fill=fillColor,fill opacity=0.20] (194.46,108.37) circle (  2.13);

\path[fill=fillColor,fill opacity=0.20] (195.46,110.65) circle (  2.13);

\path[fill=fillColor,fill opacity=0.20] (195.46,100.28) circle (  2.13);

\path[fill=fillColor,fill opacity=0.20] (198.47, 95.22) circle (  2.13);

\path[fill=fillColor,fill opacity=0.20] (200.48, 92.57) circle (  2.13);

\path[fill=fillColor,fill opacity=0.20] (205.50, 85.48) circle (  2.13);

\path[fill=fillColor,fill opacity=0.20] (213.52, 72.46) circle (  2.13);

\path[fill=fillColor,fill opacity=0.20] (217.53, 43.38) circle (  2.13);

\path[fill=fillColor,fill opacity=0.20] (223.55, 51.22) circle (  2.13);

\path[fill=fillColor,fill opacity=0.20] (208.51, 76.76) circle (  2.13);

\path[fill=fillColor,fill opacity=0.20] (201.48, 83.59) circle (  2.13);

\path[fill=fillColor,fill opacity=0.20] (200.48, 90.79) circle (  2.13);

\path[fill=fillColor,fill opacity=0.20] (204.49, 97.50) circle (  2.13);

\path[fill=fillColor,fill opacity=0.20] (208.51, 97.88) circle (  2.13);

\path[fill=fillColor,fill opacity=0.20] (206.50, 95.22) circle (  2.13);

\path[fill=fillColor,fill opacity=0.20] (208.51, 87.63) circle (  2.13);

\path[fill=fillColor,fill opacity=0.20] (215.53, 75.62) circle (  2.13);

\path[fill=fillColor,fill opacity=0.20] (217.53, 78.28) circle (  2.13);

\path[fill=fillColor,fill opacity=0.20] (210.51, 86.37) circle (  2.13);

\path[fill=fillColor,fill opacity=0.20] (201.48, 88.14) circle (  2.13);

\path[fill=fillColor,fill opacity=0.20] (200.48, 90.04) circle (  2.13);

\path[fill=fillColor,fill opacity=0.20] (199.48, 98.76) circle (  2.13);

\path[fill=fillColor,fill opacity=0.20] (193.46,104.45) circle (  2.13);

\path[fill=fillColor,fill opacity=0.20] (191.45,105.21) circle (  2.13);

\path[fill=fillColor,fill opacity=0.20] (192.45,103.31) circle (  2.13);

\path[fill=fillColor,fill opacity=0.20] (194.46, 95.98) circle (  2.13);

\path[fill=fillColor,fill opacity=0.20] (196.47, 89.02) circle (  2.13);

\path[fill=fillColor,fill opacity=0.20] (197.47, 91.93) circle (  2.13);

\path[fill=fillColor,fill opacity=0.20] (209.51, 91.17) circle (  2.13);

\path[fill=fillColor,fill opacity=0.20] (218.54, 65.13) circle (  2.13);

\path[fill=fillColor,fill opacity=0.20] (234.59, 45.91) circle (  2.13);

\path[fill=fillColor,fill opacity=0.20] (212.52, 67.02) circle (  2.13);

\path[fill=fillColor,fill opacity=0.20] (205.50, 76.13) circle (  2.13);

\path[fill=fillColor,fill opacity=0.20] (205.50, 89.78) circle (  2.13);

\path[fill=fillColor,fill opacity=0.20] (206.50,103.31) circle (  2.13);

\path[fill=fillColor,fill opacity=0.20] (203.49, 96.74) circle (  2.13);

\path[fill=fillColor,fill opacity=0.20] (209.51, 90.16) circle (  2.13);

\path[fill=fillColor,fill opacity=0.20] (212.52, 91.81) circle (  2.13);

\path[fill=fillColor,fill opacity=0.20] (213.52, 88.39) circle (  2.13);

\path[fill=fillColor,fill opacity=0.20] (219.54, 72.46) circle (  2.13);

\path[fill=fillColor,fill opacity=0.20] (221.55, 74.36) circle (  2.13);

\path[fill=fillColor,fill opacity=0.20] (213.52, 92.31) circle (  2.13);

\path[fill=fillColor,fill opacity=0.20] (203.49, 93.20) circle (  2.13);

\path[fill=fillColor,fill opacity=0.20] (202.49, 87.51) circle (  2.13);

\path[fill=fillColor,fill opacity=0.20] (201.48, 84.47) circle (  2.13);

\path[fill=fillColor,fill opacity=0.20] (200.48, 90.04) circle (  2.13);

\path[fill=fillColor,fill opacity=0.20] (194.46, 99.90) circle (  2.13);

\path[fill=fillColor,fill opacity=0.20] (190.45, 99.39) circle (  2.13);

\path[fill=fillColor,fill opacity=0.20] (189.44, 92.06) circle (  2.13);

\path[fill=fillColor,fill opacity=0.20] (189.44, 90.54) circle (  2.13);

\path[fill=fillColor,fill opacity=0.20] (191.45, 91.17) circle (  2.13);

\path[fill=fillColor,fill opacity=0.20] (195.46, 88.39) circle (  2.13);

\path[fill=fillColor,fill opacity=0.20] (209.51, 78.15) circle (  2.13);

\path[fill=fillColor,fill opacity=0.20] (223.55, 50.20) circle (  2.13);

\path[fill=fillColor,fill opacity=0.20] (212.52, 67.65) circle (  2.13);

\path[fill=fillColor,fill opacity=0.20] (208.51, 88.01) circle (  2.13);

\path[fill=fillColor,fill opacity=0.20] (206.50, 99.27) circle (  2.13);

\path[fill=fillColor,fill opacity=0.20] (207.50, 94.21) circle (  2.13);

\path[fill=fillColor,fill opacity=0.20] (214.53, 93.32) circle (  2.13);

\path[fill=fillColor,fill opacity=0.20] (214.53,100.40) circle (  2.13);

\path[fill=fillColor,fill opacity=0.20] (211.52,100.40) circle (  2.13);

\path[fill=fillColor,fill opacity=0.20] (215.53, 91.17) circle (  2.13);

\path[fill=fillColor,fill opacity=0.20] (228.57, 71.57) circle (  2.13);

\path[fill=fillColor,fill opacity=0.20] (217.53, 67.53) circle (  2.13);

\path[fill=fillColor,fill opacity=0.20] (210.51, 84.73) circle (  2.13);

\path[fill=fillColor,fill opacity=0.20] (208.51, 93.07) circle (  2.13);

\path[fill=fillColor,fill opacity=0.20] (207.50, 90.42) circle (  2.13);

\path[fill=fillColor,fill opacity=0.20] (203.49, 93.70) circle (  2.13);

\path[fill=fillColor,fill opacity=0.20] (199.48, 97.88) circle (  2.13);

\path[fill=fillColor,fill opacity=0.20] (198.47, 95.09) circle (  2.13);

\path[fill=fillColor,fill opacity=0.20] (194.46, 97.62) circle (  2.13);

\path[fill=fillColor,fill opacity=0.20] (191.45, 98.26) circle (  2.13);

\path[fill=fillColor,fill opacity=0.20] (193.46, 91.68) circle (  2.13);

\path[fill=fillColor,fill opacity=0.20] (190.45, 90.67) circle (  2.13);

\path[fill=fillColor,fill opacity=0.20] (194.46, 92.31) circle (  2.13);

\path[fill=fillColor,fill opacity=0.20] (207.50, 80.43) circle (  2.13);

\path[fill=fillColor,fill opacity=0.20] (219.54, 56.02) circle (  2.13);

\path[fill=fillColor,fill opacity=0.20] (209.51, 74.23) circle (  2.13);

\path[fill=fillColor,fill opacity=0.20] (208.51, 86.50) circle (  2.13);

\path[fill=fillColor,fill opacity=0.20] (209.51, 95.22) circle (  2.13);

\path[fill=fillColor,fill opacity=0.20] (210.51,100.03) circle (  2.13);

\path[fill=fillColor,fill opacity=0.20] (212.52,100.53) circle (  2.13);

\path[fill=fillColor,fill opacity=0.20] (209.51, 97.24) circle (  2.13);

\path[fill=fillColor,fill opacity=0.20] (211.52, 93.70) circle (  2.13);

\path[fill=fillColor,fill opacity=0.20] (219.54, 90.67) circle (  2.13);

\path[fill=fillColor,fill opacity=0.20] (230.58, 75.37) circle (  2.13);

\path[fill=fillColor,fill opacity=0.20] (209.51, 59.18) circle (  2.13);

\path[fill=fillColor,fill opacity=0.20] (212.52, 83.33) circle (  2.13);

\path[fill=fillColor,fill opacity=0.20] (208.51, 88.14) circle (  2.13);

\path[fill=fillColor,fill opacity=0.20] (208.51, 82.58) circle (  2.13);

\path[fill=fillColor,fill opacity=0.20] (203.49, 89.02) circle (  2.13);

\path[fill=fillColor,fill opacity=0.20] (200.48,100.91) circle (  2.13);

\path[fill=fillColor,fill opacity=0.20] (195.46,109.38) circle (  2.13);

\path[fill=fillColor,fill opacity=0.20] (194.46,101.67) circle (  2.13);

\path[fill=fillColor,fill opacity=0.20] (189.44, 96.49) circle (  2.13);

\path[fill=fillColor,fill opacity=0.20] (192.45, 99.65) circle (  2.13);

\path[fill=fillColor,fill opacity=0.20] (199.48,101.92) circle (  2.13);

\path[fill=fillColor,fill opacity=0.20] (197.47, 96.23) circle (  2.13);

\path[fill=fillColor,fill opacity=0.20] (204.49, 82.20) circle (  2.13);

\path[fill=fillColor,fill opacity=0.20] (215.53, 70.56) circle (  2.13);

\path[fill=fillColor,fill opacity=0.20] (213.52, 59.18) circle (  2.13);

\path[fill=fillColor,fill opacity=0.20] (209.51, 78.53) circle (  2.13);

\path[fill=fillColor,fill opacity=0.20] (210.51,100.53) circle (  2.13);

\path[fill=fillColor,fill opacity=0.20] (210.51,106.22) circle (  2.13);

\path[fill=fillColor,fill opacity=0.20] (210.51, 98.38) circle (  2.13);

\path[fill=fillColor,fill opacity=0.20] (208.51, 92.94) circle (  2.13);

\path[fill=fillColor,fill opacity=0.20] (210.51, 93.07) circle (  2.13);

\path[fill=fillColor,fill opacity=0.20] (218.54, 95.35) circle (  2.13);

\path[fill=fillColor,fill opacity=0.20] (223.55, 93.83) circle (  2.13);

\path[fill=fillColor,fill opacity=0.20] (233.59, 82.07) circle (  2.13);

\path[fill=fillColor,fill opacity=0.20] (220.54, 57.92) circle (  2.13);

\path[fill=fillColor,fill opacity=0.20] (209.51, 68.41) circle (  2.13);

\path[fill=fillColor,fill opacity=0.20] (206.50, 81.06) circle (  2.13);

\path[fill=fillColor,fill opacity=0.20] (210.51, 86.87) circle (  2.13);

\path[fill=fillColor,fill opacity=0.20] (210.51, 83.84) circle (  2.13);

\path[fill=fillColor,fill opacity=0.20] (200.48, 89.53) circle (  2.13);

\path[fill=fillColor,fill opacity=0.20] (197.47,100.15) circle (  2.13);

\path[fill=fillColor,fill opacity=0.20] (195.46, 99.90) circle (  2.13);

\path[fill=fillColor,fill opacity=0.20] (193.46, 92.82) circle (  2.13);

\path[fill=fillColor,fill opacity=0.20] (189.44, 92.82) circle (  2.13);

\path[fill=fillColor,fill opacity=0.20] (166.37, 94.21) circle (  2.13);

\path[fill=fillColor,fill opacity=0.20] (202.49, 98.89) circle (  2.13);

\path[fill=fillColor,fill opacity=0.20] (206.50, 94.84) circle (  2.13);

\path[fill=fillColor,fill opacity=0.20] (212.52, 72.84) circle (  2.13);

\path[fill=fillColor,fill opacity=0.20] (223.55, 47.68) circle (  2.13);

\path[fill=fillColor,fill opacity=0.20] (215.53, 66.64) circle (  2.13);

\path[fill=fillColor,fill opacity=0.20] (209.51, 91.81) circle (  2.13);

\path[fill=fillColor,fill opacity=0.20] (210.51,100.28) circle (  2.13);

\path[fill=fillColor,fill opacity=0.20] (211.52, 92.94) circle (  2.13);

\path[fill=fillColor,fill opacity=0.20] (210.51, 91.43) circle (  2.13);

\path[fill=fillColor,fill opacity=0.20] (215.53, 94.59) circle (  2.13);

\path[fill=fillColor,fill opacity=0.20] (211.52, 95.35) circle (  2.13);

\path[fill=fillColor,fill opacity=0.20] (218.54, 88.90) circle (  2.13);

\path[fill=fillColor,fill opacity=0.20] (225.56, 85.61) circle (  2.13);

\path[fill=fillColor,fill opacity=0.20] (234.59, 83.46) circle (  2.13);

\path[fill=fillColor,fill opacity=0.20] (222.55, 60.57) circle (  2.13);

\path[fill=fillColor,fill opacity=0.20] (214.53, 81.44) circle (  2.13);

\path[fill=fillColor,fill opacity=0.20] (210.51, 76.00) circle (  2.13);

\path[fill=fillColor,fill opacity=0.20] (204.49, 77.90) circle (  2.13);

\path[fill=fillColor,fill opacity=0.20] (200.48, 91.68) circle (  2.13);

\path[fill=fillColor,fill opacity=0.20] (204.49, 95.60) circle (  2.13);

\path[fill=fillColor,fill opacity=0.20] (201.48, 89.66) circle (  2.13);

\path[fill=fillColor,fill opacity=0.20] (199.48, 91.93) circle (  2.13);

\path[fill=fillColor,fill opacity=0.20] (198.47, 91.93) circle (  2.13);

\path[fill=fillColor,fill opacity=0.20] (196.47, 88.65) circle (  2.13);

\path[fill=fillColor,fill opacity=0.20] (198.47, 86.24) circle (  2.13);

\path[fill=fillColor,fill opacity=0.20] (201.48, 81.69) circle (  2.13);

\path[fill=fillColor,fill opacity=0.20] (207.50, 78.53) circle (  2.13);

\path[fill=fillColor,fill opacity=0.20] (215.53, 79.16) circle (  2.13);

\path[fill=fillColor,fill opacity=0.20] (216.53, 43.50) circle (  2.13);

\path[fill=fillColor,fill opacity=0.20] (207.50, 69.55) circle (  2.13);

\path[fill=fillColor,fill opacity=0.20] (208.51, 87.00) circle (  2.13);

\path[fill=fillColor,fill opacity=0.20] (211.52, 86.37) circle (  2.13);

\path[fill=fillColor,fill opacity=0.20] (216.53, 88.14) circle (  2.13);

\path[fill=fillColor,fill opacity=0.20] (214.53, 95.35) circle (  2.13);

\path[fill=fillColor,fill opacity=0.20] (211.52, 93.20) circle (  2.13);

\path[fill=fillColor,fill opacity=0.20] (212.52, 85.86) circle (  2.13);

\path[fill=fillColor,fill opacity=0.20] (220.54, 82.58) circle (  2.13);

\path[fill=fillColor,fill opacity=0.20] (223.55, 87.13) circle (  2.13);

\path[fill=fillColor,fill opacity=0.20] (222.55, 86.75) circle (  2.13);

\path[fill=fillColor,fill opacity=0.20] (227.57, 64.49) circle (  2.13);

\path[fill=fillColor,fill opacity=0.20] (223.55, 62.60) circle (  2.13);

\path[fill=fillColor,fill opacity=0.20] (219.54, 84.98) circle (  2.13);

\path[fill=fillColor,fill opacity=0.20] (212.52,101.29) circle (  2.13);

\path[fill=fillColor,fill opacity=0.20] (209.51, 86.62) circle (  2.13);

\path[fill=fillColor,fill opacity=0.20] (207.50, 81.56) circle (  2.13);

\path[fill=fillColor,fill opacity=0.20] (195.46, 90.79) circle (  2.13);

\path[fill=fillColor,fill opacity=0.20] (200.48, 96.49) circle (  2.13);

\path[fill=fillColor,fill opacity=0.20] (200.48, 86.87) circle (  2.13);

\path[fill=fillColor,fill opacity=0.20] (203.49, 85.10) circle (  2.13);

\path[fill=fillColor,fill opacity=0.20] (201.48, 86.62) circle (  2.13);

\path[fill=fillColor,fill opacity=0.20] (194.46, 85.61) circle (  2.13);

\path[fill=fillColor,fill opacity=0.20] (206.50, 86.62) circle (  2.13);

\path[fill=fillColor,fill opacity=0.20] (210.51, 75.62) circle (  2.13);

\path[fill=fillColor,fill opacity=0.20] (213.52, 62.22) circle (  2.13);

\path[fill=fillColor,fill opacity=0.20] (222.55, 54.50) circle (  2.13);

\path[fill=fillColor,fill opacity=0.20] (207.50, 40.97) circle (  2.13);

\path[fill=fillColor,fill opacity=0.20] (209.51, 60.95) circle (  2.13);

\path[fill=fillColor,fill opacity=0.20] (212.52, 83.97) circle (  2.13);

\path[fill=fillColor,fill opacity=0.20] (211.52, 92.19) circle (  2.13);

\path[fill=fillColor,fill opacity=0.20] (212.52, 95.09) circle (  2.13);

\path[fill=fillColor,fill opacity=0.20] (214.53, 96.49) circle (  2.13);

\path[fill=fillColor,fill opacity=0.20] (217.53, 90.67) circle (  2.13);

\path[fill=fillColor,fill opacity=0.20] (215.53, 84.35) circle (  2.13);

\path[fill=fillColor,fill opacity=0.20] (218.54, 80.05) circle (  2.13);

\path[fill=fillColor,fill opacity=0.20] (217.53, 84.60) circle (  2.13);

\path[fill=fillColor,fill opacity=0.20] (224.56, 94.59) circle (  2.13);

\path[fill=fillColor,fill opacity=0.20] (231.58, 83.21) circle (  2.13);

\path[fill=fillColor,fill opacity=0.20] (220.54, 70.94) circle (  2.13);

\path[fill=fillColor,fill opacity=0.20] (211.52, 84.47) circle (  2.13);

\path[fill=fillColor,fill opacity=0.20] (214.53, 92.06) circle (  2.13);

\path[fill=fillColor,fill opacity=0.20] (215.53, 95.35) circle (  2.13);

\path[fill=fillColor,fill opacity=0.20] (209.51, 90.54) circle (  2.13);

\path[fill=fillColor,fill opacity=0.20] (205.50, 86.87) circle (  2.13);

\path[fill=fillColor,fill opacity=0.20] (203.49, 89.40) circle (  2.13);

\path[fill=fillColor,fill opacity=0.20] (200.48, 89.40) circle (  2.13);

\path[fill=fillColor,fill opacity=0.20] (194.46, 86.12) circle (  2.13);

\path[fill=fillColor,fill opacity=0.20] (207.50, 82.45) circle (  2.13);

\path[fill=fillColor,fill opacity=0.20] (206.50, 73.85) circle (  2.13);

\path[fill=fillColor,fill opacity=0.20] (205.50, 66.52) circle (  2.13);

\path[fill=fillColor,fill opacity=0.20] (213.52, 68.29) circle (  2.13);

\path[fill=fillColor,fill opacity=0.20] (219.54, 59.18) circle (  2.13);

\path[fill=fillColor,fill opacity=0.20] (212.52, 49.07) circle (  2.13);

\path[fill=fillColor,fill opacity=0.20] (210.51, 69.42) circle (  2.13);

\path[fill=fillColor,fill opacity=0.20] (208.51, 84.60) circle (  2.13);

\path[fill=fillColor,fill opacity=0.20] (211.52, 94.08) circle (  2.13);

\path[fill=fillColor,fill opacity=0.20] (205.50, 95.98) circle (  2.13);

\path[fill=fillColor,fill opacity=0.20] (214.53, 88.77) circle (  2.13);

\path[fill=fillColor,fill opacity=0.20] (209.51, 76.00) circle (  2.13);

\path[fill=fillColor,fill opacity=0.20] (208.51, 77.64) circle (  2.13);

\path[fill=fillColor,fill opacity=0.20] (216.53,100.53) circle (  2.13);

\path[fill=fillColor,fill opacity=0.20] (216.53,114.69) circle (  2.13);

\path[fill=fillColor,fill opacity=0.20] (222.55, 86.12) circle (  2.13);

\path[fill=fillColor,fill opacity=0.20] (227.57, 74.36) circle (  2.13);

\path[fill=fillColor,fill opacity=0.20] (218.54, 81.56) circle (  2.13);

\path[fill=fillColor,fill opacity=0.20] (211.52, 80.30) circle (  2.13);

\path[fill=fillColor,fill opacity=0.20] (207.50, 80.68) circle (  2.13);

\path[fill=fillColor,fill opacity=0.20] (209.51, 85.23) circle (  2.13);

\path[fill=fillColor,fill opacity=0.20] (210.51, 85.99) circle (  2.13);

\path[fill=fillColor,fill opacity=0.20] (212.52, 83.97) circle (  2.13);

\path[fill=fillColor,fill opacity=0.20] (210.51, 82.83) circle (  2.13);

\path[fill=fillColor,fill opacity=0.20] (210.51, 82.07) circle (  2.13);

\path[fill=fillColor,fill opacity=0.20] (215.53, 73.47) circle (  2.13);

\path[fill=fillColor,fill opacity=0.20] (215.53, 61.08) circle (  2.13);

\path[fill=fillColor,fill opacity=0.20] (211.52, 59.56) circle (  2.13);

\path[fill=fillColor,fill opacity=0.20] (205.50, 88.90) circle (  2.13);

\path[fill=fillColor,fill opacity=0.20] (210.51,102.68) circle (  2.13);

\path[fill=fillColor,fill opacity=0.20] (209.51, 94.59) circle (  2.13);

\path[fill=fillColor,fill opacity=0.20] (208.51, 77.77) circle (  2.13);

\path[fill=fillColor,fill opacity=0.20] (205.50, 80.30) circle (  2.13);

\path[fill=fillColor,fill opacity=0.20] (209.51,101.80) circle (  2.13);

\path[fill=fillColor,fill opacity=0.20] (209.51,103.95) circle (  2.13);

\path[fill=fillColor,fill opacity=0.20] (210.51, 84.98) circle (  2.13);

\path[fill=fillColor,fill opacity=0.20] (211.52, 80.55) circle (  2.13);

\path[fill=fillColor,fill opacity=0.20] (222.55, 76.13) circle (  2.13);

\path[fill=fillColor,fill opacity=0.20] (220.54, 70.69) circle (  2.13);

\path[fill=fillColor,fill opacity=0.20] (218.54, 79.41) circle (  2.13);

\path[fill=fillColor,fill opacity=0.20] (213.52, 83.08) circle (  2.13);

\path[fill=fillColor,fill opacity=0.20] (207.50, 83.84) circle (  2.13);

\path[fill=fillColor,fill opacity=0.20] (205.50, 80.43) circle (  2.13);

\path[fill=fillColor,fill opacity=0.20] (204.49, 77.14) circle (  2.13);

\path[fill=fillColor,fill opacity=0.20] (209.51, 71.57) circle (  2.13);

\path[fill=fillColor,fill opacity=0.20] (214.53, 66.52) circle (  2.13);

\path[fill=fillColor,fill opacity=0.20] (217.53, 61.08) circle (  2.13);

\path[fill=fillColor,fill opacity=0.20] (223.55, 57.79) circle (  2.13);

\path[fill=fillColor,fill opacity=0.20] (225.56, 51.60) circle (  2.13);

\path[fill=fillColor,fill opacity=0.20] (206.50, 54.38) circle (  2.13);

\path[fill=fillColor,fill opacity=0.20] (209.51, 80.81) circle (  2.13);

\path[fill=fillColor,fill opacity=0.20] (205.50, 88.52) circle (  2.13);

\path[fill=fillColor,fill opacity=0.20] (205.50, 79.41) circle (  2.13);

\path[fill=fillColor,fill opacity=0.20] (206.50, 86.75) circle (  2.13);

\path[fill=fillColor,fill opacity=0.20] (207.50,106.10) circle (  2.13);

\path[fill=fillColor,fill opacity=0.20] (208.51, 97.75) circle (  2.13);

\path[fill=fillColor,fill opacity=0.20] (205.50, 82.20) circle (  2.13);

\path[fill=fillColor,fill opacity=0.20] (215.53, 88.01) circle (  2.13);

\path[fill=fillColor,fill opacity=0.20] (215.53, 98.00) circle (  2.13);

\path[fill=fillColor,fill opacity=0.20] (213.52, 57.29) circle (  2.13);

\path[fill=fillColor,fill opacity=0.20] (212.52, 73.22) circle (  2.13);

\path[fill=fillColor,fill opacity=0.20] (212.52, 81.69) circle (  2.13);

\path[fill=fillColor,fill opacity=0.20] (212.52, 86.50) circle (  2.13);

\path[fill=fillColor,fill opacity=0.20] (206.50, 89.78) circle (  2.13);

\path[fill=fillColor,fill opacity=0.20] (206.50, 78.40) circle (  2.13);

\path[fill=fillColor,fill opacity=0.20] (214.53, 63.99) circle (  2.13);

\path[fill=fillColor,fill opacity=0.20] (209.51, 52.61) circle (  2.13);

\path[fill=fillColor,fill opacity=0.20] (214.53, 44.77) circle (  2.13);

\path[fill=fillColor,fill opacity=0.20] (209.51, 61.33) circle (  2.13);

\path[fill=fillColor,fill opacity=0.20] (208.51, 73.60) circle (  2.13);

\path[fill=fillColor,fill opacity=0.20] (209.51, 90.67) circle (  2.13);

\path[fill=fillColor,fill opacity=0.20] (208.51,104.83) circle (  2.13);

\path[fill=fillColor,fill opacity=0.20] (205.50, 99.27) circle (  2.13);

\path[fill=fillColor,fill opacity=0.20] (205.50, 91.55) circle (  2.13);

\path[fill=fillColor,fill opacity=0.20] (209.51, 90.29) circle (  2.13);

\path[fill=fillColor,fill opacity=0.20] (213.52, 86.87) circle (  2.13);

\path[fill=fillColor,fill opacity=0.20] (210.51, 88.27) circle (  2.13);

\path[fill=fillColor,fill opacity=0.20] (218.54,105.08) circle (  2.13);

\path[fill=fillColor,fill opacity=0.20] (221.55,110.65) circle (  2.13);

\path[fill=fillColor,fill opacity=0.20] (223.55, 85.61) circle (  2.13);

\path[fill=fillColor,fill opacity=0.20] (221.55, 69.30) circle (  2.13);

\path[fill=fillColor,fill opacity=0.20] (218.54, 56.91) circle (  2.13);

\path[fill=fillColor,fill opacity=0.20] (212.52, 73.98) circle (  2.13);

\path[fill=fillColor,fill opacity=0.20] (212.52, 90.29) circle (  2.13);

\path[fill=fillColor,fill opacity=0.20] (211.52, 78.78) circle (  2.13);

\path[fill=fillColor,fill opacity=0.20] (213.52, 73.60) circle (  2.13);

\path[fill=fillColor,fill opacity=0.20] (214.53, 68.03) circle (  2.13);

\path[fill=fillColor,fill opacity=0.20] (213.52, 55.77) circle (  2.13);

\path[fill=fillColor,fill opacity=0.20] (215.53, 45.65) circle (  2.13);

\path[fill=fillColor,fill opacity=0.20] (213.52, 51.34) circle (  2.13);

\path[fill=fillColor,fill opacity=0.20] (209.51, 75.62) circle (  2.13);

\path[fill=fillColor,fill opacity=0.20] (209.51, 90.29) circle (  2.13);

\path[fill=fillColor,fill opacity=0.20] (206.50, 96.36) circle (  2.13);

\path[fill=fillColor,fill opacity=0.20] (210.51,101.16) circle (  2.13);

\path[fill=fillColor,fill opacity=0.20] (207.50, 96.11) circle (  2.13);

\path[fill=fillColor,fill opacity=0.20] (209.51, 82.83) circle (  2.13);

\path[fill=fillColor,fill opacity=0.20] (211.52, 78.15) circle (  2.13);

\path[fill=fillColor,fill opacity=0.20] (212.52, 87.00) circle (  2.13);

\path[fill=fillColor,fill opacity=0.20] (214.53, 94.08) circle (  2.13);

\path[fill=fillColor,fill opacity=0.20] (221.55, 86.12) circle (  2.13);

\path[fill=fillColor,fill opacity=0.20] (218.54, 85.86) circle (  2.13);

\path[fill=fillColor,fill opacity=0.20] (221.55, 89.28) circle (  2.13);

\path[fill=fillColor,fill opacity=0.20] (223.55, 78.40) circle (  2.13);

\path[fill=fillColor,fill opacity=0.20] (218.54, 62.09) circle (  2.13);

\path[fill=fillColor,fill opacity=0.20] (216.53, 48.81) circle (  2.13);

\path[fill=fillColor,fill opacity=0.20] (213.52, 60.95) circle (  2.13);

\path[fill=fillColor,fill opacity=0.20] (212.52, 66.90) circle (  2.13);

\path[fill=fillColor,fill opacity=0.20] (214.53, 74.23) circle (  2.13);

\path[fill=fillColor,fill opacity=0.20] (216.53, 77.39) circle (  2.13);

\path[fill=fillColor,fill opacity=0.20] (213.52, 76.89) circle (  2.13);

\path[fill=fillColor,fill opacity=0.20] (190.45, 82.07) circle (  2.13);

\path[fill=fillColor,fill opacity=0.20] (224.56, 64.62) circle (  2.13);

\path[fill=fillColor,fill opacity=0.20] (217.53, 46.16) circle (  2.13);

\path[fill=fillColor,fill opacity=0.20] (224.56, 40.09) circle (  2.13);

\path[fill=fillColor,fill opacity=0.20] (214.53, 48.05) circle (  2.13);

\path[fill=fillColor,fill opacity=0.20] (205.50, 66.01) circle (  2.13);

\path[fill=fillColor,fill opacity=0.20] (208.51, 82.70) circle (  2.13);

\path[fill=fillColor,fill opacity=0.20] (209.51, 96.49) circle (  2.13);

\path[fill=fillColor,fill opacity=0.20] (210.51, 93.45) circle (  2.13);

\path[fill=fillColor,fill opacity=0.20] (209.51, 81.82) circle (  2.13);

\path[fill=fillColor,fill opacity=0.20] (207.50, 76.25) circle (  2.13);

\path[fill=fillColor,fill opacity=0.20] (213.52, 77.52) circle (  2.13);

\path[fill=fillColor,fill opacity=0.20] (213.52, 74.36) circle (  2.13);

\path[fill=fillColor,fill opacity=0.20] (212.52, 75.87) circle (  2.13);

\path[fill=fillColor,fill opacity=0.20] (214.53, 90.04) circle (  2.13);

\path[fill=fillColor,fill opacity=0.20] (220.54, 97.75) circle (  2.13);

\path[fill=fillColor,fill opacity=0.20] (219.54, 86.24) circle (  2.13);

\path[fill=fillColor,fill opacity=0.20] (215.53, 80.17) circle (  2.13);

\path[fill=fillColor,fill opacity=0.20] (218.54, 85.74) circle (  2.13);

\path[fill=fillColor,fill opacity=0.20] (221.55, 84.35) circle (  2.13);

\path[fill=fillColor,fill opacity=0.20] (223.55, 71.45) circle (  2.13);

\path[fill=fillColor,fill opacity=0.20] (222.55, 61.08) circle (  2.13);

\path[fill=fillColor,fill opacity=0.20] (218.54, 59.18) circle (  2.13);

\path[fill=fillColor,fill opacity=0.20] (216.53, 58.93) circle (  2.13);

\path[fill=fillColor,fill opacity=0.20] (214.53, 63.10) circle (  2.13);

\path[fill=fillColor,fill opacity=0.20] (216.53, 60.70) circle (  2.13);

\path[fill=fillColor,fill opacity=0.20] (221.55, 60.45) circle (  2.13);

\path[fill=fillColor,fill opacity=0.20] (215.53, 70.44) circle (  2.13);

\path[fill=fillColor,fill opacity=0.20] (216.53, 74.48) circle (  2.13);

\path[fill=fillColor,fill opacity=0.20] (214.53, 78.91) circle (  2.13);

\path[fill=fillColor,fill opacity=0.20] (212.52, 73.85) circle (  2.13);

\path[fill=fillColor,fill opacity=0.20] (211.52, 74.36) circle (  2.13);

\path[fill=fillColor,fill opacity=0.20] (215.53, 84.60) circle (  2.13);

\path[fill=fillColor,fill opacity=0.20] (213.52, 83.84) circle (  2.13);

\path[fill=fillColor,fill opacity=0.20] (214.53, 88.27) circle (  2.13);

\path[fill=fillColor,fill opacity=0.20] (219.54, 86.75) circle (  2.13);

\path[fill=fillColor,fill opacity=0.20] (221.55, 61.58) circle (  2.13);

\path[fill=fillColor,fill opacity=0.20] (208.51, 57.03) circle (  2.13);

\path[fill=fillColor,fill opacity=0.20] (206.50, 66.77) circle (  2.13);

\path[fill=fillColor,fill opacity=0.20] (209.51, 67.78) circle (  2.13);

\path[fill=fillColor,fill opacity=0.20] (213.52, 71.83) circle (  2.13);

\path[fill=fillColor,fill opacity=0.20] (215.53, 78.02) circle (  2.13);

\path[fill=fillColor,fill opacity=0.20] (214.53, 79.16) circle (  2.13);

\path[fill=fillColor,fill opacity=0.20] (210.51, 76.89) circle (  2.13);

\path[fill=fillColor,fill opacity=0.20] (209.51, 82.45) circle (  2.13);

\path[fill=fillColor,fill opacity=0.20] (213.52, 91.93) circle (  2.13);

\path[fill=fillColor,fill opacity=0.20] (216.53, 94.08) circle (  2.13);

\path[fill=fillColor,fill opacity=0.20] (216.53, 85.74) circle (  2.13);

\path[fill=fillColor,fill opacity=0.20] (214.53, 82.32) circle (  2.13);

\path[fill=fillColor,fill opacity=0.20] (215.53, 89.66) circle (  2.13);

\path[fill=fillColor,fill opacity=0.20] (220.54, 91.55) circle (  2.13);

\path[fill=fillColor,fill opacity=0.20] (217.53, 82.45) circle (  2.13);

\path[fill=fillColor,fill opacity=0.20] (215.53, 77.90) circle (  2.13);

\path[fill=fillColor,fill opacity=0.20] (213.52, 78.78) circle (  2.13);

\path[fill=fillColor,fill opacity=0.20] (216.53, 88.27) circle (  2.13);

\path[fill=fillColor,fill opacity=0.20] (216.53, 89.53) circle (  2.13);

\path[fill=fillColor,fill opacity=0.20] (214.53, 81.56) circle (  2.13);

\path[fill=fillColor,fill opacity=0.20] (210.51, 82.95) circle (  2.13);

\path[fill=fillColor,fill opacity=0.20] (214.53, 84.73) circle (  2.13);

\path[fill=fillColor,fill opacity=0.20] (215.53, 82.32) circle (  2.13);

\path[fill=fillColor,fill opacity=0.20] (214.53, 88.01) circle (  2.13);

\path[fill=fillColor,fill opacity=0.20] (214.53, 91.55) circle (  2.13);

\path[fill=fillColor,fill opacity=0.20] (213.52, 83.71) circle (  2.13);

\path[fill=fillColor,fill opacity=0.20] (208.51, 86.75) circle (  2.13);

\path[fill=fillColor,fill opacity=0.20] (216.53, 94.46) circle (  2.13);

\path[fill=fillColor,fill opacity=0.20] (211.52, 92.44) circle (  2.13);

\path[fill=fillColor,fill opacity=0.20] (213.52, 96.99) circle (  2.13);

\path[fill=fillColor,fill opacity=0.20] (200.48, 98.63) circle (  2.13);

\path[fill=fillColor,fill opacity=0.20] (212.52, 86.50) circle (  2.13);

\path[fill=fillColor,fill opacity=0.20] (214.53, 78.53) circle (  2.13);

\path[fill=fillColor,fill opacity=0.20] (217.53, 71.20) circle (  2.13);

\path[fill=fillColor,fill opacity=0.20] (217.53, 62.22) circle (  2.13);

\path[fill=fillColor,fill opacity=0.20] (220.54, 47.17) circle (  2.13);

\path[fill=fillColor,fill opacity=0.20] (215.53, 61.21) circle (  2.13);

\path[fill=fillColor,fill opacity=0.20] (213.52, 65.13) circle (  2.13);

\path[fill=fillColor,fill opacity=0.20] (210.51, 73.72) circle (  2.13);

\path[fill=fillColor,fill opacity=0.20] (208.51, 76.76) circle (  2.13);

\path[fill=fillColor,fill opacity=0.20] (209.51, 79.41) circle (  2.13);

\path[fill=fillColor,fill opacity=0.20] (212.52, 83.08) circle (  2.13);

\path[fill=fillColor,fill opacity=0.20] (214.53, 80.55) circle (  2.13);

\path[fill=fillColor,fill opacity=0.20] (215.53, 81.18) circle (  2.13);

\path[fill=fillColor,fill opacity=0.20] (213.52, 87.89) circle (  2.13);

\path[fill=fillColor,fill opacity=0.20] (211.52, 87.51) circle (  2.13);

\path[fill=fillColor,fill opacity=0.20] (215.53, 82.58) circle (  2.13);

\path[fill=fillColor,fill opacity=0.20] (211.52, 81.69) circle (  2.13);

\path[fill=fillColor,fill opacity=0.20] (207.50, 79.67) circle (  2.13);

\path[fill=fillColor,fill opacity=0.20] (211.52, 87.76) circle (  2.13);

\path[fill=fillColor,fill opacity=0.20] (215.53, 97.62) circle (  2.13);

\path[fill=fillColor,fill opacity=0.20] (214.53, 92.69) circle (  2.13);

\path[fill=fillColor,fill opacity=0.20] (212.52, 88.52) circle (  2.13);

\path[fill=fillColor,fill opacity=0.20] (212.52, 93.58) circle (  2.13);

\path[fill=fillColor,fill opacity=0.20] (214.53, 90.67) circle (  2.13);

\path[fill=fillColor,fill opacity=0.20] (213.52, 88.90) circle (  2.13);

\path[fill=fillColor,fill opacity=0.20] (212.52, 94.21) circle (  2.13);

\path[fill=fillColor,fill opacity=0.20] (210.51, 93.58) circle (  2.13);

\path[fill=fillColor,fill opacity=0.20] (209.51, 93.96) circle (  2.13);

\path[fill=fillColor,fill opacity=0.20] (208.51, 98.63) circle (  2.13);

\path[fill=fillColor,fill opacity=0.20] (209.51, 91.81) circle (  2.13);

\path[fill=fillColor,fill opacity=0.20] (210.51, 87.89) circle (  2.13);

\path[fill=fillColor,fill opacity=0.20] (215.53, 79.92) circle (  2.13);

\path[fill=fillColor,fill opacity=0.20] (217.53, 63.36) circle (  2.13);

\path[fill=fillColor,fill opacity=0.20] (207.50, 44.64) circle (  2.13);

\path[fill=fillColor,fill opacity=0.20] (208.51, 46.79) circle (  2.13);

\path[fill=fillColor,fill opacity=0.20] (209.51, 51.97) circle (  2.13);

\path[fill=fillColor,fill opacity=0.20] (213.52, 66.01) circle (  2.13);

\path[fill=fillColor,fill opacity=0.20] (217.53, 65.13) circle (  2.13);

\path[fill=fillColor,fill opacity=0.20] (214.53, 63.36) circle (  2.13);

\path[fill=fillColor,fill opacity=0.20] (211.52, 71.45) circle (  2.13);

\path[fill=fillColor,fill opacity=0.20] (212.52, 81.69) circle (  2.13);

\path[fill=fillColor,fill opacity=0.20] (213.52, 84.85) circle (  2.13);

\path[fill=fillColor,fill opacity=0.20] (209.51, 87.63) circle (  2.13);

\path[fill=fillColor,fill opacity=0.20] (209.51, 88.65) circle (  2.13);

\path[fill=fillColor,fill opacity=0.20] (209.51, 88.27) circle (  2.13);

\path[fill=fillColor,fill opacity=0.20] (210.51, 96.74) circle (  2.13);

\path[fill=fillColor,fill opacity=0.20] (212.52,104.20) circle (  2.13);

\path[fill=fillColor,fill opacity=0.20] (211.52,103.95) circle (  2.13);

\path[fill=fillColor,fill opacity=0.20] (211.52,100.66) circle (  2.13);

\path[fill=fillColor,fill opacity=0.20] (211.52, 96.99) circle (  2.13);

\path[fill=fillColor,fill opacity=0.20] (209.51, 89.66) circle (  2.13);

\path[fill=fillColor,fill opacity=0.20] (209.51, 82.83) circle (  2.13);

\path[fill=fillColor,fill opacity=0.20] (209.51, 85.74) circle (  2.13);

\path[fill=fillColor,fill opacity=0.20] (208.51, 86.12) circle (  2.13);

\path[fill=fillColor,fill opacity=0.20] (207.50, 79.54) circle (  2.13);

\path[fill=fillColor,fill opacity=0.20] (209.51, 71.57) circle (  2.13);

\path[fill=fillColor,fill opacity=0.20] (212.52, 60.57) circle (  2.13);

\path[fill=fillColor,fill opacity=0.20] (220.54, 44.89) circle (  2.13);

\path[fill=fillColor,fill opacity=0.20] (215.53, 38.82) circle (  2.13);

\path[fill=fillColor,fill opacity=0.20] (212.52, 42.87) circle (  2.13);

\path[fill=fillColor,fill opacity=0.20] (215.53, 51.72) circle (  2.13);

\path[fill=fillColor,fill opacity=0.20] (215.53, 58.55) circle (  2.13);

\path[fill=fillColor,fill opacity=0.20] (212.52, 66.90) circle (  2.13);

\path[fill=fillColor,fill opacity=0.20] (211.52, 82.45) circle (  2.13);

\path[fill=fillColor,fill opacity=0.20] (209.51, 88.01) circle (  2.13);

\path[fill=fillColor,fill opacity=0.20] (209.51, 85.36) circle (  2.13);

\path[fill=fillColor,fill opacity=0.20] (211.52, 88.14) circle (  2.13);

\path[fill=fillColor,fill opacity=0.20] (212.52, 90.54) circle (  2.13);

\path[fill=fillColor,fill opacity=0.20] (214.53, 87.51) circle (  2.13);

\path[fill=fillColor,fill opacity=0.20] (212.52, 83.46) circle (  2.13);

\path[fill=fillColor,fill opacity=0.20] (209.51, 70.06) circle (  2.13);

\path[fill=fillColor,fill opacity=0.20] (210.51, 62.47) circle (  2.13);

\path[fill=fillColor,fill opacity=0.20] (215.53, 61.58) circle (  2.13);

\path[fill=fillColor,fill opacity=0.20] (215.53, 51.22) circle (  2.13);

\path[fill=fillColor,fill opacity=0.20] (217.53, 43.25) circle (  2.13);

\path[fill=fillColor,fill opacity=0.20] (216.53, 48.56) circle (  2.13);

\path[fill=fillColor,fill opacity=0.20] (211.52, 55.39) circle (  2.13);

\path[fill=fillColor,fill opacity=0.20] (210.51, 54.25) circle (  2.13);

\path[fill=fillColor,fill opacity=0.20] (203.49, 50.46) circle (  2.13);

\path[fill=fillColor,fill opacity=0.20] (213.52, 50.71) circle (  2.13);

\path[fill=fillColor,fill opacity=0.20] (215.53, 52.48) circle (  2.13);

\path[fill=fillColor,fill opacity=0.20] (202.49, 49.19) circle (  2.13);

\path[fill=fillColor,fill opacity=0.20] (200.48, 40.59) circle (  2.13);

\path[fill=fillColor,fill opacity=0.20] (208.51, 52.35) circle (  2.13);

\path[fill=fillColor,fill opacity=0.20] (212.52, 52.48) circle (  2.13);

\path[fill=fillColor,fill opacity=0.20] (212.52, 52.23) circle (  2.13);

\path[fill=fillColor,fill opacity=0.20] (214.53, 45.53) circle (  2.13);

\path[fill=fillColor,fill opacity=0.20] (210.51, 51.85) circle (  2.13);

\path[fill=fillColor,fill opacity=0.20] (205.50, 57.92) circle (  2.13);

\path[fill=fillColor,fill opacity=0.20] (204.49, 62.85) circle (  2.13);

\path[fill=fillColor,fill opacity=0.20] (209.51, 66.90) circle (  2.13);

\path[fill=fillColor,fill opacity=0.20] (211.52, 69.55) circle (  2.13);

\path[fill=fillColor,fill opacity=0.20] (214.53, 67.78) circle (  2.13);

\path[fill=fillColor,fill opacity=0.20] (221.55, 60.95) circle (  2.13);

\path[fill=fillColor,fill opacity=0.20] (228.57, 56.65) circle (  2.13);

\path[fill=fillColor,fill opacity=0.20] (206.50, 45.53) circle (  2.13);

\path[fill=fillColor,fill opacity=0.20] (207.50, 57.66) circle (  2.13);

\path[fill=fillColor,fill opacity=0.20] (198.47, 65.76) circle (  2.13);

\path[fill=fillColor,fill opacity=0.20] (203.49, 71.70) circle (  2.13);

\path[fill=fillColor,fill opacity=0.20] (204.49, 78.66) circle (  2.13);

\path[fill=fillColor,fill opacity=0.20] (208.51, 81.69) circle (  2.13);

\path[fill=fillColor,fill opacity=0.20] (211.52, 79.29) circle (  2.13);

\path[fill=fillColor,fill opacity=0.20] (214.53, 75.12) circle (  2.13);

\path[fill=fillColor,fill opacity=0.20] (220.54, 68.41) circle (  2.13);

\path[fill=fillColor,fill opacity=0.20] (239.61, 61.33) circle (  2.13);

\path[fill=fillColor,fill opacity=0.20] (247.63, 51.85) circle (  2.13);

\path[fill=fillColor,fill opacity=0.20] (212.52, 40.34) circle (  2.13);

\path[fill=fillColor,fill opacity=0.20] (207.50, 58.42) circle (  2.13);

\path[fill=fillColor,fill opacity=0.20] (201.48, 76.51) circle (  2.13);

\path[fill=fillColor,fill opacity=0.20] (204.49, 83.71) circle (  2.13);

\path[fill=fillColor,fill opacity=0.20] (204.49, 88.27) circle (  2.13);

\path[fill=fillColor,fill opacity=0.20] (206.50, 91.68) circle (  2.13);

\path[fill=fillColor,fill opacity=0.20] (212.52, 90.29) circle (  2.13);

\path[fill=fillColor,fill opacity=0.20] (211.52, 85.86) circle (  2.13);

\path[fill=fillColor,fill opacity=0.20] (217.53, 80.30) circle (  2.13);

\path[fill=fillColor,fill opacity=0.20] (223.55, 70.69) circle (  2.13);

\path[fill=fillColor,fill opacity=0.20] (244.62, 59.31) circle (  2.13);

\path[fill=fillColor,fill opacity=0.20] (211.52, 50.71) circle (  2.13);

\path[fill=fillColor,fill opacity=0.20] (209.51, 72.84) circle (  2.13);

\path[fill=fillColor,fill opacity=0.20] (209.51, 92.69) circle (  2.13);

\path[fill=fillColor,fill opacity=0.20] (208.51,101.04) circle (  2.13);

\path[fill=fillColor,fill opacity=0.20] (208.51,100.15) circle (  2.13);

\path[fill=fillColor,fill opacity=0.20] (210.51, 96.11) circle (  2.13);

\path[fill=fillColor,fill opacity=0.20] (214.53, 91.17) circle (  2.13);

\path[fill=fillColor,fill opacity=0.20] (221.55, 89.66) circle (  2.13);

\path[fill=fillColor,fill opacity=0.20] (220.54, 86.12) circle (  2.13);

\path[fill=fillColor,fill opacity=0.20] (235.59, 73.47) circle (  2.13);

\path[fill=fillColor,fill opacity=0.20] (213.52, 65.25) circle (  2.13);

\path[fill=fillColor,fill opacity=0.20] (212.52, 84.09) circle (  2.13);

\path[fill=fillColor,fill opacity=0.20] (212.52, 97.62) circle (  2.13);

\path[fill=fillColor,fill opacity=0.20] (211.52,107.61) circle (  2.13);

\path[fill=fillColor,fill opacity=0.20] (209.51,106.60) circle (  2.13);

\path[fill=fillColor,fill opacity=0.20] (211.52, 97.75) circle (  2.13);

\path[fill=fillColor,fill opacity=0.20] (215.53, 92.69) circle (  2.13);

\path[fill=fillColor,fill opacity=0.20] (225.56, 90.29) circle (  2.13);

\path[fill=fillColor,fill opacity=0.20] (231.58, 84.22) circle (  2.13);

\path[fill=fillColor,fill opacity=0.20] (244.62, 70.31) circle (  2.13);

\path[fill=fillColor,fill opacity=0.20] (210.51, 40.47) circle (  2.13);

\path[fill=fillColor,fill opacity=0.20] (209.51, 41.99) circle (  2.13);

\path[fill=fillColor,fill opacity=0.20] (213.52, 74.23) circle (  2.13);

\path[fill=fillColor,fill opacity=0.20] (213.52, 90.42) circle (  2.13);

\path[fill=fillColor,fill opacity=0.20] (210.51, 98.63) circle (  2.13);

\path[fill=fillColor,fill opacity=0.20] (208.51,108.88) circle (  2.13);

\path[fill=fillColor,fill opacity=0.20] (213.52,107.23) circle (  2.13);

\path[fill=fillColor,fill opacity=0.20] (215.53, 98.38) circle (  2.13);

\path[fill=fillColor,fill opacity=0.20] (216.53, 93.83) circle (  2.13);

\path[fill=fillColor,fill opacity=0.20] (226.56, 86.75) circle (  2.13);

\path[fill=fillColor,fill opacity=0.20] (228.57, 70.06) circle (  2.13);

\path[fill=fillColor,fill opacity=0.20] (208.51, 52.10) circle (  2.13);

\path[fill=fillColor,fill opacity=0.20] (207.50, 55.64) circle (  2.13);

\path[fill=fillColor,fill opacity=0.20] (207.50, 53.11) circle (  2.13);

\path[fill=fillColor,fill opacity=0.20] (210.51, 48.56) circle (  2.13);

\path[fill=fillColor,fill opacity=0.20] (210.51, 73.09) circle (  2.13);

\path[fill=fillColor,fill opacity=0.20] (208.51, 90.04) circle (  2.13);

\path[fill=fillColor,fill opacity=0.20] (210.51, 99.65) circle (  2.13);

\path[fill=fillColor,fill opacity=0.20] (209.51,106.60) circle (  2.13);

\path[fill=fillColor,fill opacity=0.20] (209.51,103.06) circle (  2.13);

\path[fill=fillColor,fill opacity=0.20] (218.54, 95.47) circle (  2.13);

\path[fill=fillColor,fill opacity=0.20] (219.54, 92.44) circle (  2.13);

\path[fill=fillColor,fill opacity=0.20] (225.56, 81.82) circle (  2.13);

\path[fill=fillColor,fill opacity=0.20] (204.49, 65.63) circle (  2.13);

\path[fill=fillColor,fill opacity=0.20] (209.51, 65.88) circle (  2.13);

\path[fill=fillColor,fill opacity=0.20] (212.52, 63.99) circle (  2.13);

\path[fill=fillColor,fill opacity=0.20] (206.50, 57.92) circle (  2.13);

\path[fill=fillColor,fill opacity=0.20] (212.52, 71.45) circle (  2.13);

\path[fill=fillColor,fill opacity=0.20] (204.49, 88.39) circle (  2.13);

\path[fill=fillColor,fill opacity=0.20] (206.50, 98.13) circle (  2.13);

\path[fill=fillColor,fill opacity=0.20] (208.51,100.28) circle (  2.13);

\path[fill=fillColor,fill opacity=0.20] (214.53, 98.13) circle (  2.13);

\path[fill=fillColor,fill opacity=0.20] (220.54, 95.85) circle (  2.13);

\path[fill=fillColor,fill opacity=0.20] (220.54, 92.31) circle (  2.13);

\path[fill=fillColor,fill opacity=0.20] (221.55, 80.55) circle (  2.13);

\path[fill=fillColor,fill opacity=0.20] (214.53, 66.64) circle (  2.13);

\path[fill=fillColor,fill opacity=0.20] (205.50, 71.32) circle (  2.13);

\path[fill=fillColor,fill opacity=0.20] (209.51, 75.87) circle (  2.13);

\path[fill=fillColor,fill opacity=0.20] (215.53, 73.60) circle (  2.13);

\path[fill=fillColor,fill opacity=0.20] (212.52, 70.31) circle (  2.13);

\path[fill=fillColor,fill opacity=0.20] (209.51, 65.38) circle (  2.13);

\path[fill=fillColor,fill opacity=0.20] (210.51, 51.34) circle (  2.13);

\path[fill=fillColor,fill opacity=0.20] (223.55, 66.52) circle (  2.13);

\path[fill=fillColor,fill opacity=0.20] (204.49, 85.61) circle (  2.13);

\path[fill=fillColor,fill opacity=0.20] (209.51, 94.21) circle (  2.13);

\path[fill=fillColor,fill opacity=0.20] (208.51, 95.73) circle (  2.13);

\path[fill=fillColor,fill opacity=0.20] (212.52, 96.11) circle (  2.13);

\path[fill=fillColor,fill opacity=0.20] (215.53,101.16) circle (  2.13);

\path[fill=fillColor,fill opacity=0.20] (219.54, 99.14) circle (  2.13);

\path[fill=fillColor,fill opacity=0.20] (219.54, 82.83) circle (  2.13);

\path[fill=fillColor,fill opacity=0.20] (220.54, 62.60) circle (  2.13);

\path[fill=fillColor,fill opacity=0.20] (217.53, 70.94) circle (  2.13);

\path[fill=fillColor,fill opacity=0.20] (218.54, 78.91) circle (  2.13);

\path[fill=fillColor,fill opacity=0.20] (216.53, 80.81) circle (  2.13);

\path[fill=fillColor,fill opacity=0.20] (219.54, 77.14) circle (  2.13);

\path[fill=fillColor,fill opacity=0.20] (208.51, 72.21) circle (  2.13);

\path[fill=fillColor,fill opacity=0.20] (216.53, 65.88) circle (  2.13);

\path[fill=fillColor,fill opacity=0.20] (215.53, 54.76) circle (  2.13);

\path[fill=fillColor,fill opacity=0.20] (229.57, 41.23) circle (  2.13);

\path[fill=fillColor,fill opacity=0.20] (213.52, 50.84) circle (  2.13);

\path[fill=fillColor,fill opacity=0.20] (212.52, 77.14) circle (  2.13);

\path[fill=fillColor,fill opacity=0.20] (209.51, 91.05) circle (  2.13);

\path[fill=fillColor,fill opacity=0.20] (212.52, 94.84) circle (  2.13);

\path[fill=fillColor,fill opacity=0.20] (214.53, 93.32) circle (  2.13);

\path[fill=fillColor,fill opacity=0.20] (215.53, 98.76) circle (  2.13);

\path[fill=fillColor,fill opacity=0.20] (215.53,101.04) circle (  2.13);

\path[fill=fillColor,fill opacity=0.20] (214.53, 84.47) circle (  2.13);

\path[fill=fillColor,fill opacity=0.20] (223.55, 61.84) circle (  2.13);

\path[fill=fillColor,fill opacity=0.20] (213.52, 61.84) circle (  2.13);

\path[fill=fillColor,fill opacity=0.20] (217.53, 69.80) circle (  2.13);

\path[fill=fillColor,fill opacity=0.20] (217.53, 78.02) circle (  2.13);

\path[fill=fillColor,fill opacity=0.20] (222.55, 83.97) circle (  2.13);

\path[fill=fillColor,fill opacity=0.20] (221.55, 84.22) circle (  2.13);

\path[fill=fillColor,fill opacity=0.20] (223.55, 76.38) circle (  2.13);

\path[fill=fillColor,fill opacity=0.20] (223.55, 69.80) circle (  2.13);

\path[fill=fillColor,fill opacity=0.20] (215.53, 62.22) circle (  2.13);

\path[fill=fillColor,fill opacity=0.20] (232.58, 46.28) circle (  2.13);

\path[fill=fillColor,fill opacity=0.20] (234.59, 66.90) circle (  2.13);

\path[fill=fillColor,fill opacity=0.20] (213.52, 85.99) circle (  2.13);

\path[fill=fillColor,fill opacity=0.20] (210.51, 92.94) circle (  2.13);

\path[fill=fillColor,fill opacity=0.20] (203.49, 91.17) circle (  2.13);

\path[fill=fillColor,fill opacity=0.20] (217.53, 90.29) circle (  2.13);

\path[fill=fillColor,fill opacity=0.20] (218.54, 95.09) circle (  2.13);

\path[fill=fillColor,fill opacity=0.20] (215.53, 86.75) circle (  2.13);

\path[fill=fillColor,fill opacity=0.20] (218.54, 70.69) circle (  2.13);

\path[fill=fillColor,fill opacity=0.20] (210.51, 65.76) circle (  2.13);

\path[fill=fillColor,fill opacity=0.20] (214.53, 73.09) circle (  2.13);

\path[fill=fillColor,fill opacity=0.20] (214.53, 81.31) circle (  2.13);

\path[fill=fillColor,fill opacity=0.20] (220.54, 87.76) circle (  2.13);

\path[fill=fillColor,fill opacity=0.20] (216.53, 89.40) circle (  2.13);

\path[fill=fillColor,fill opacity=0.20] (217.53, 80.81) circle (  2.13);

\path[fill=fillColor,fill opacity=0.20] (222.55, 73.22) circle (  2.13);

\path[fill=fillColor,fill opacity=0.20] (218.54, 65.63) circle (  2.13);

\path[fill=fillColor,fill opacity=0.20] (224.56, 49.83) circle (  2.13);

\path[fill=fillColor,fill opacity=0.20] (269.70, 57.41) circle (  2.13);

\path[fill=fillColor,fill opacity=0.20] (239.61, 75.62) circle (  2.13);

\path[fill=fillColor,fill opacity=0.20] (218.54, 85.10) circle (  2.13);

\path[fill=fillColor,fill opacity=0.20] (199.48, 92.31) circle (  2.13);

\path[fill=fillColor,fill opacity=0.20] (216.53, 90.79) circle (  2.13);

\path[fill=fillColor,fill opacity=0.20] (218.54, 91.17) circle (  2.13);

\path[fill=fillColor,fill opacity=0.20] (213.52, 93.07) circle (  2.13);

\path[fill=fillColor,fill opacity=0.20] (217.53, 83.21) circle (  2.13);

\path[fill=fillColor,fill opacity=0.20] (222.55, 60.57) circle (  2.13);

\path[fill=fillColor,fill opacity=0.20] (216.53, 74.10) circle (  2.13);

\path[fill=fillColor,fill opacity=0.20] (210.51, 83.46) circle (  2.13);

\path[fill=fillColor,fill opacity=0.20] (209.51, 87.76) circle (  2.13);

\path[fill=fillColor,fill opacity=0.20] (217.53, 88.90) circle (  2.13);

\path[fill=fillColor,fill opacity=0.20] (217.53, 81.31) circle (  2.13);

\path[fill=fillColor,fill opacity=0.20] (212.52, 71.32) circle (  2.13);

\path[fill=fillColor,fill opacity=0.20] (218.54, 62.34) circle (  2.13);

\path[fill=fillColor,fill opacity=0.20] (231.58, 50.96) circle (  2.13);

\path[fill=fillColor,fill opacity=0.20] (235.59, 76.89) circle (  2.13);

\path[fill=fillColor,fill opacity=0.20] (217.53, 96.11) circle (  2.13);

\path[fill=fillColor,fill opacity=0.20] (214.53, 95.98) circle (  2.13);

\path[fill=fillColor,fill opacity=0.20] (211.52, 92.82) circle (  2.13);

\path[fill=fillColor,fill opacity=0.20] (216.53, 94.46) circle (  2.13);

\path[fill=fillColor,fill opacity=0.20] (215.53, 89.28) circle (  2.13);

\path[fill=fillColor,fill opacity=0.20] (215.53, 73.22) circle (  2.13);

\path[fill=fillColor,fill opacity=0.20] (209.51, 63.99) circle (  2.13);

\path[fill=fillColor,fill opacity=0.20] (207.50, 74.61) circle (  2.13);

\path[fill=fillColor,fill opacity=0.20] (206.50, 85.86) circle (  2.13);

\path[fill=fillColor,fill opacity=0.20] (207.50, 86.24) circle (  2.13);

\path[fill=fillColor,fill opacity=0.20] (213.52, 87.51) circle (  2.13);

\path[fill=fillColor,fill opacity=0.20] (210.51, 89.28) circle (  2.13);

\path[fill=fillColor,fill opacity=0.20] (210.51, 81.69) circle (  2.13);

\path[fill=fillColor,fill opacity=0.20] (213.52, 71.83) circle (  2.13);

\path[fill=fillColor,fill opacity=0.20] (216.53, 63.23) circle (  2.13);

\path[fill=fillColor,fill opacity=0.20] (226.56, 52.10) circle (  2.13);

\path[fill=fillColor,fill opacity=0.20] (264.69, 41.35) circle (  2.13);

\path[fill=fillColor,fill opacity=0.20] (232.58, 85.48) circle (  2.13);

\path[fill=fillColor,fill opacity=0.20] (212.52, 91.68) circle (  2.13);

\path[fill=fillColor,fill opacity=0.20] (205.50, 90.67) circle (  2.13);

\path[fill=fillColor,fill opacity=0.20] (210.51, 92.06) circle (  2.13);

\path[fill=fillColor,fill opacity=0.20] (214.53, 90.16) circle (  2.13);

\path[fill=fillColor,fill opacity=0.20] (213.52, 85.23) circle (  2.13);

\path[fill=fillColor,fill opacity=0.20] (214.53, 74.36) circle (  2.13);

\path[fill=fillColor,fill opacity=0.20] (200.48, 61.21) circle (  2.13);

\path[fill=fillColor,fill opacity=0.20] (202.49, 71.95) circle (  2.13);

\path[fill=fillColor,fill opacity=0.20] (202.49, 80.30) circle (  2.13);

\path[fill=fillColor,fill opacity=0.20] (204.49, 83.33) circle (  2.13);

\path[fill=fillColor,fill opacity=0.20] (204.49, 88.27) circle (  2.13);

\path[fill=fillColor,fill opacity=0.20] (208.51, 90.92) circle (  2.13);

\path[fill=fillColor,fill opacity=0.20] (211.52, 85.61) circle (  2.13);

\path[fill=fillColor,fill opacity=0.20] (217.53, 78.66) circle (  2.13);

\path[fill=fillColor,fill opacity=0.20] (217.53, 72.21) circle (  2.13);

\path[fill=fillColor,fill opacity=0.20] (225.56, 55.89) circle (  2.13);

\path[fill=fillColor,fill opacity=0.20] (244.62, 59.06) circle (  2.13);

\path[fill=fillColor,fill opacity=0.20] (218.54, 74.99) circle (  2.13);

\path[fill=fillColor,fill opacity=0.20] (212.52, 84.98) circle (  2.13);

\path[fill=fillColor,fill opacity=0.20] (210.51, 90.67) circle (  2.13);

\path[fill=fillColor,fill opacity=0.20] (211.52, 93.45) circle (  2.13);

\path[fill=fillColor,fill opacity=0.20] (211.52, 91.17) circle (  2.13);

\path[fill=fillColor,fill opacity=0.20] (212.52, 82.32) circle (  2.13);

\path[fill=fillColor,fill opacity=0.20] (215.53, 75.24) circle (  2.13);

\path[fill=fillColor,fill opacity=0.20] (221.55, 57.29) circle (  2.13);

\path[fill=fillColor,fill opacity=0.20] (201.48, 62.09) circle (  2.13);

\path[fill=fillColor,fill opacity=0.20] (200.48, 69.05) circle (  2.13);

\path[fill=fillColor,fill opacity=0.20] (202.49, 79.03) circle (  2.13);

\path[fill=fillColor,fill opacity=0.20] (202.49, 83.97) circle (  2.13);

\path[fill=fillColor,fill opacity=0.20] (203.49, 85.99) circle (  2.13);

\path[fill=fillColor,fill opacity=0.20] (201.48, 88.27) circle (  2.13);

\path[fill=fillColor,fill opacity=0.20] (202.49, 88.77) circle (  2.13);

\path[fill=fillColor,fill opacity=0.20] (204.49, 86.75) circle (  2.13);

\path[fill=fillColor,fill opacity=0.20] (211.52, 81.69) circle (  2.13);

\path[fill=fillColor,fill opacity=0.20] (219.54, 74.23) circle (  2.13);

\path[fill=fillColor,fill opacity=0.20] (241.61, 58.68) circle (  2.13);

\path[fill=fillColor,fill opacity=0.20] (266.69, 54.00) circle (  2.13);

\path[fill=fillColor,fill opacity=0.20] (251.64, 71.45) circle (  2.13);

\path[fill=fillColor,fill opacity=0.20] (222.55, 84.47) circle (  2.13);

\path[fill=fillColor,fill opacity=0.20] (212.52, 89.91) circle (  2.13);

\path[fill=fillColor,fill opacity=0.20] (212.52, 88.14) circle (  2.13);

\path[fill=fillColor,fill opacity=0.20] (212.52, 83.08) circle (  2.13);

\path[fill=fillColor,fill opacity=0.20] (213.52, 81.06) circle (  2.13);

\path[fill=fillColor,fill opacity=0.20] (218.54, 76.25) circle (  2.13);

\path[fill=fillColor,fill opacity=0.20] (200.48, 64.87) circle (  2.13);

\path[fill=fillColor,fill opacity=0.20] (197.47, 74.61) circle (  2.13);

\path[fill=fillColor,fill opacity=0.20] (199.48, 82.32) circle (  2.13);

\path[fill=fillColor,fill opacity=0.20] (203.49, 92.44) circle (  2.13);

\path[fill=fillColor,fill opacity=0.20] (203.49, 91.68) circle (  2.13);

\path[fill=fillColor,fill opacity=0.20] (203.49, 87.00) circle (  2.13);

\path[fill=fillColor,fill opacity=0.20] (207.50, 88.77) circle (  2.13);

\path[fill=fillColor,fill opacity=0.20] (206.50, 88.90) circle (  2.13);

\path[fill=fillColor,fill opacity=0.20] (207.50, 82.58) circle (  2.13);

\path[fill=fillColor,fill opacity=0.20] (209.51, 73.09) circle (  2.13);

\path[fill=fillColor,fill opacity=0.20] (215.53, 62.98) circle (  2.13);

\path[fill=fillColor,fill opacity=0.20] (251.64, 53.37) circle (  2.13);

\path[fill=fillColor,fill opacity=0.20] (266.69, 47.93) circle (  2.13);

\path[fill=fillColor,fill opacity=0.20] (239.61, 64.87) circle (  2.13);

\path[fill=fillColor,fill opacity=0.20] (226.56, 78.78) circle (  2.13);

\path[fill=fillColor,fill opacity=0.20] (216.53, 83.46) circle (  2.13);

\path[fill=fillColor,fill opacity=0.20] (212.52, 85.10) circle (  2.13);

\path[fill=fillColor,fill opacity=0.20] (213.52, 82.58) circle (  2.13);

\path[fill=fillColor,fill opacity=0.20] (212.52, 81.82) circle (  2.13);

\path[fill=fillColor,fill opacity=0.20] (212.52, 76.00) circle (  2.13);

\path[fill=fillColor,fill opacity=0.20] (196.47, 67.40) circle (  2.13);

\path[fill=fillColor,fill opacity=0.20] (194.46, 76.63) circle (  2.13);

\path[fill=fillColor,fill opacity=0.20] (189.44, 82.07) circle (  2.13);

\path[fill=fillColor,fill opacity=0.20] (198.47, 87.38) circle (  2.13);

\path[fill=fillColor,fill opacity=0.20] (202.49, 94.21) circle (  2.13);

\path[fill=fillColor,fill opacity=0.20] (200.48, 89.78) circle (  2.13);

\path[fill=fillColor,fill opacity=0.20] (204.49, 83.84) circle (  2.13);

\path[fill=fillColor,fill opacity=0.20] (210.51, 86.87) circle (  2.13);

\path[fill=fillColor,fill opacity=0.20] (208.51, 87.13) circle (  2.13);

\path[fill=fillColor,fill opacity=0.20] (206.50, 78.78) circle (  2.13);

\path[fill=fillColor,fill opacity=0.20] (212.52, 67.28) circle (  2.13);

\path[fill=fillColor,fill opacity=0.20] (247.63, 52.48) circle (  2.13);

\path[fill=fillColor,fill opacity=0.20] (227.57, 74.74) circle (  2.13);

\path[fill=fillColor,fill opacity=0.20] (207.50, 82.58) circle (  2.13);

\path[fill=fillColor,fill opacity=0.20] (208.51, 83.46) circle (  2.13);

\path[fill=fillColor,fill opacity=0.20] (211.52, 80.55) circle (  2.13);

\path[fill=fillColor,fill opacity=0.20] (217.53, 80.68) circle (  2.13);

\path[fill=fillColor,fill opacity=0.20] (216.53, 74.36) circle (  2.13);

\path[fill=fillColor,fill opacity=0.20] (195.46, 68.16) circle (  2.13);

\path[fill=fillColor,fill opacity=0.20] (192.45, 78.53) circle (  2.13);

\path[fill=fillColor,fill opacity=0.20] (192.45, 81.31) circle (  2.13);

\path[fill=fillColor,fill opacity=0.20] (187.94, 83.08) circle (  2.13);

\path[fill=fillColor,fill opacity=0.20] (198.47, 88.14) circle (  2.13);

\path[fill=fillColor,fill opacity=0.20] (201.48, 90.29) circle (  2.13);

\path[fill=fillColor,fill opacity=0.20] (199.48, 86.75) circle (  2.13);

\path[fill=fillColor,fill opacity=0.20] (205.50, 85.74) circle (  2.13);

\path[fill=fillColor,fill opacity=0.20] (211.52, 85.36) circle (  2.13);

\path[fill=fillColor,fill opacity=0.20] (213.52, 79.41) circle (  2.13);

\path[fill=fillColor,fill opacity=0.20] (215.53, 72.59) circle (  2.13);

\path[fill=fillColor,fill opacity=0.20] (230.58, 63.86) circle (  2.13);

\path[fill=fillColor,fill opacity=0.20] (246.63, 47.04) circle (  2.13);

\path[fill=fillColor,fill opacity=0.20] (235.59, 74.10) circle (  2.13);

\path[fill=fillColor,fill opacity=0.20] (215.53, 82.95) circle (  2.13);

\path[fill=fillColor,fill opacity=0.20] (217.53, 81.56) circle (  2.13);

\path[fill=fillColor,fill opacity=0.20] (210.51, 81.18) circle (  2.13);

\path[fill=fillColor,fill opacity=0.20] (210.51, 84.73) circle (  2.13);

\path[fill=fillColor,fill opacity=0.20] (211.52, 75.49) circle (  2.13);

\path[fill=fillColor,fill opacity=0.20] (193.46, 60.45) circle (  2.13);

\path[fill=fillColor,fill opacity=0.20] (197.47, 74.48) circle (  2.13);

\path[fill=fillColor,fill opacity=0.20] (195.46, 85.61) circle (  2.13);

\path[fill=fillColor,fill opacity=0.20] (191.45, 88.14) circle (  2.13);

\path[fill=fillColor,fill opacity=0.20] (192.45, 87.76) circle (  2.13);

\path[fill=fillColor,fill opacity=0.20] (196.47, 89.02) circle (  2.13);

\path[fill=fillColor,fill opacity=0.20] (201.48, 93.45) circle (  2.13);

\path[fill=fillColor,fill opacity=0.20] (199.48, 90.29) circle (  2.13);

\path[fill=fillColor,fill opacity=0.20] (201.48, 85.61) circle (  2.13);

\path[fill=fillColor,fill opacity=0.20] (207.50, 89.78) circle (  2.13);

\path[fill=fillColor,fill opacity=0.20] (220.54, 87.25) circle (  2.13);

\path[fill=fillColor,fill opacity=0.20] (222.55, 72.46) circle (  2.13);

\path[fill=fillColor,fill opacity=0.20] (255.66, 62.22) circle (  2.13);

\path[fill=fillColor,fill opacity=0.20] (240.61, 53.49) circle (  2.13);

\path[fill=fillColor,fill opacity=0.20] (240.61, 74.99) circle (  2.13);

\path[fill=fillColor,fill opacity=0.20] (222.55, 80.17) circle (  2.13);

\path[fill=fillColor,fill opacity=0.20] (212.52, 82.95) circle (  2.13);

\path[fill=fillColor,fill opacity=0.20] (211.52, 86.12) circle (  2.13);

\path[fill=fillColor,fill opacity=0.20] (207.50, 86.37) circle (  2.13);

\path[fill=fillColor,fill opacity=0.20] (204.49, 82.07) circle (  2.13);

\path[fill=fillColor,fill opacity=0.20] (211.52, 67.28) circle (  2.13);

\path[fill=fillColor,fill opacity=0.20] (193.46, 65.63) circle (  2.13);

\path[fill=fillColor,fill opacity=0.20] (191.45, 75.24) circle (  2.13);

\path[fill=fillColor,fill opacity=0.20] (191.45, 81.18) circle (  2.13);

\path[fill=fillColor,fill opacity=0.20] (195.46, 93.58) circle (  2.13);

\path[fill=fillColor,fill opacity=0.20] (195.46, 98.13) circle (  2.13);

\path[fill=fillColor,fill opacity=0.20] (192.45, 93.07) circle (  2.13);

\path[fill=fillColor,fill opacity=0.20] (194.46, 94.71) circle (  2.13);

\path[fill=fillColor,fill opacity=0.20] (187.64, 94.97) circle (  2.13);

\path[fill=fillColor,fill opacity=0.20] (199.48, 94.46) circle (  2.13);

\path[fill=fillColor,fill opacity=0.20] (200.48, 89.78) circle (  2.13);

\path[fill=fillColor,fill opacity=0.20] (209.51, 81.69) circle (  2.13);

\path[fill=fillColor,fill opacity=0.20] (218.54, 83.71) circle (  2.13);

\path[fill=fillColor,fill opacity=0.20] (236.60, 82.32) circle (  2.13);

\path[fill=fillColor,fill opacity=0.20] (258.67, 63.23) circle (  2.13);

\path[fill=fillColor,fill opacity=0.20] (238.60, 47.68) circle (  2.13);

\path[fill=fillColor,fill opacity=0.20] (251.64, 73.47) circle (  2.13);

\path[fill=fillColor,fill opacity=0.20] (224.56, 86.12) circle (  2.13);

\path[fill=fillColor,fill opacity=0.20] (212.52, 90.29) circle (  2.13);

\path[fill=fillColor,fill opacity=0.20] (210.51, 89.28) circle (  2.13);

\path[fill=fillColor,fill opacity=0.20] (208.51, 89.91) circle (  2.13);

\path[fill=fillColor,fill opacity=0.20] (207.50, 85.10) circle (  2.13);

\path[fill=fillColor,fill opacity=0.20] (212.52, 72.33) circle (  2.13);

\path[fill=fillColor,fill opacity=0.20] (189.44, 78.28) circle (  2.13);

\path[fill=fillColor,fill opacity=0.20] (187.54, 87.00) circle (  2.13);

\path[fill=fillColor,fill opacity=0.20] (184.33, 89.66) circle (  2.13);

\path[fill=fillColor,fill opacity=0.20] (195.46, 90.79) circle (  2.13);

\path[fill=fillColor,fill opacity=0.20] (194.46, 94.71) circle (  2.13);

\path[fill=fillColor,fill opacity=0.20] (196.47, 94.46) circle (  2.13);

\path[fill=fillColor,fill opacity=0.20] (195.46, 92.57) circle (  2.13);

\path[fill=fillColor,fill opacity=0.20] (194.46, 90.04) circle (  2.13);

\path[fill=fillColor,fill opacity=0.20] (200.48, 86.12) circle (  2.13);

\path[fill=fillColor,fill opacity=0.20] (207.50, 87.76) circle (  2.13);

\path[fill=fillColor,fill opacity=0.20] (219.54, 82.45) circle (  2.13);

\path[fill=fillColor,fill opacity=0.20] (227.57, 71.70) circle (  2.13);

\path[fill=fillColor,fill opacity=0.20] (242.62, 66.64) circle (  2.13);

\path[fill=fillColor,fill opacity=0.20] (248.64, 60.19) circle (  2.13);

\path[fill=fillColor,fill opacity=0.20] (233.59, 44.77) circle (  2.13);

\path[fill=fillColor,fill opacity=0.20] (255.66, 78.66) circle (  2.13);

\path[fill=fillColor,fill opacity=0.20] (216.53, 87.51) circle (  2.13);

\path[fill=fillColor,fill opacity=0.20] (209.51, 84.98) circle (  2.13);

\path[fill=fillColor,fill opacity=0.20] (207.50, 80.81) circle (  2.13);

\path[fill=fillColor,fill opacity=0.20] (209.51, 81.31) circle (  2.13);

\path[fill=fillColor,fill opacity=0.20] (211.52, 82.70) circle (  2.13);

\path[fill=fillColor,fill opacity=0.20] (207.50, 79.41) circle (  2.13);

\path[fill=fillColor,fill opacity=0.20] (210.51, 66.64) circle (  2.13);

\path[fill=fillColor,fill opacity=0.20] (190.45, 75.87) circle (  2.13);

\path[fill=fillColor,fill opacity=0.20] (191.45, 90.04) circle (  2.13);

\path[fill=fillColor,fill opacity=0.20] (189.44, 94.59) circle (  2.13);

\path[fill=fillColor,fill opacity=0.20] (177.51, 93.32) circle (  2.13);

\path[fill=fillColor,fill opacity=0.20] (194.46, 94.46) circle (  2.13);

\path[fill=fillColor,fill opacity=0.20] (195.46, 94.84) circle (  2.13);

\path[fill=fillColor,fill opacity=0.20] (191.45, 92.69) circle (  2.13);

\path[fill=fillColor,fill opacity=0.20] (192.45, 90.92) circle (  2.13);

\path[fill=fillColor,fill opacity=0.20] (195.46, 84.60) circle (  2.13);

\path[fill=fillColor,fill opacity=0.20] (204.49, 78.02) circle (  2.13);

\path[fill=fillColor,fill opacity=0.20] (220.54, 75.37) circle (  2.13);

\path[fill=fillColor,fill opacity=0.20] (232.58, 71.45) circle (  2.13);

\path[fill=fillColor,fill opacity=0.20] (258.67, 60.83) circle (  2.13);

\path[fill=fillColor,fill opacity=0.20] (243.62, 51.97) circle (  2.13);

\path[fill=fillColor,fill opacity=0.20] (234.59, 47.93) circle (  2.13);

\path[fill=fillColor,fill opacity=0.20] (222.55, 74.74) circle (  2.13);

\path[fill=fillColor,fill opacity=0.20] (216.53, 78.28) circle (  2.13);

\path[fill=fillColor,fill opacity=0.20] (211.52, 82.83) circle (  2.13);

\path[fill=fillColor,fill opacity=0.20] (209.51, 84.98) circle (  2.13);

\path[fill=fillColor,fill opacity=0.20] (209.51, 87.00) circle (  2.13);

\path[fill=fillColor,fill opacity=0.20] (207.50, 86.62) circle (  2.13);

\path[fill=fillColor,fill opacity=0.20] (212.52, 77.14) circle (  2.13);

\path[fill=fillColor,fill opacity=0.20] (216.53, 64.49) circle (  2.13);

\path[fill=fillColor,fill opacity=0.20] (194.46, 61.58) circle (  2.13);

\path[fill=fillColor,fill opacity=0.20] (190.45, 75.37) circle (  2.13);

\path[fill=fillColor,fill opacity=0.20] (188.04, 84.09) circle (  2.13);

\path[fill=fillColor,fill opacity=0.20] (191.45, 88.52) circle (  2.13);

\path[fill=fillColor,fill opacity=0.20] (195.46, 94.08) circle (  2.13);

\path[fill=fillColor,fill opacity=0.20] (195.46, 96.49) circle (  2.13);

\path[fill=fillColor,fill opacity=0.20] (192.45, 93.83) circle (  2.13);

\path[fill=fillColor,fill opacity=0.20] (196.47, 94.08) circle (  2.13);

\path[fill=fillColor,fill opacity=0.20] (203.49, 96.99) circle (  2.13);

\path[fill=fillColor,fill opacity=0.20] (208.51, 92.82) circle (  2.13);

\path[fill=fillColor,fill opacity=0.20] (209.51, 83.08) circle (  2.13);

\path[fill=fillColor,fill opacity=0.20] (213.52, 72.71) circle (  2.13);

\path[fill=fillColor,fill opacity=0.20] (209.51, 65.13) circle (  2.13);

\path[fill=fillColor,fill opacity=0.20] (233.59, 59.69) circle (  2.13);

\path[fill=fillColor,fill opacity=0.20] (247.63, 51.60) circle (  2.13);

\path[fill=fillColor,fill opacity=0.20] (217.53, 78.78) circle (  2.13);

\path[fill=fillColor,fill opacity=0.20] (223.55, 89.91) circle (  2.13);

\path[fill=fillColor,fill opacity=0.20] (212.52, 89.66) circle (  2.13);

\path[fill=fillColor,fill opacity=0.20] (208.51, 83.97) circle (  2.13);

\path[fill=fillColor,fill opacity=0.20] (206.50, 82.32) circle (  2.13);

\path[fill=fillColor,fill opacity=0.20] (209.51, 82.07) circle (  2.13);

\path[fill=fillColor,fill opacity=0.20] (214.53, 79.54) circle (  2.13);

\path[fill=fillColor,fill opacity=0.20] (213.52, 74.86) circle (  2.13);

\path[fill=fillColor,fill opacity=0.20] (210.51, 67.53) circle (  2.13);

\path[fill=fillColor,fill opacity=0.20] (183.22, 66.52) circle (  2.13);

\path[fill=fillColor,fill opacity=0.20] (193.46, 79.54) circle (  2.13);

\path[fill=fillColor,fill opacity=0.20] (190.45, 89.15) circle (  2.13);

\path[fill=fillColor,fill opacity=0.20] (187.44, 89.66) circle (  2.13);

\path[fill=fillColor,fill opacity=0.20] (193.46, 86.62) circle (  2.13);

\path[fill=fillColor,fill opacity=0.20] (201.48, 86.75) circle (  2.13);

\path[fill=fillColor,fill opacity=0.20] (207.50, 90.29) circle (  2.13);

\path[fill=fillColor,fill opacity=0.20] (196.47, 89.91) circle (  2.13);

\path[fill=fillColor,fill opacity=0.20] (211.52, 89.40) circle (  2.13);

\path[fill=fillColor,fill opacity=0.20] (226.56, 87.25) circle (  2.13);

\path[fill=fillColor,fill opacity=0.20] (243.62, 76.89) circle (  2.13);

\path[fill=fillColor,fill opacity=0.20] (249.64, 53.62) circle (  2.13);

\path[fill=fillColor,fill opacity=0.20] (230.58, 47.30) circle (  2.13);

\path[fill=fillColor,fill opacity=0.20] (259.67, 43.25) circle (  2.13);

\path[fill=fillColor,fill opacity=0.20] (269.70, 61.21) circle (  2.13);

\path[fill=fillColor,fill opacity=0.20] (238.60, 77.64) circle (  2.13);

\path[fill=fillColor,fill opacity=0.20] (224.56, 86.24) circle (  2.13);

\path[fill=fillColor,fill opacity=0.20] (211.52, 81.94) circle (  2.13);

\path[fill=fillColor,fill opacity=0.20] (207.50, 78.66) circle (  2.13);

\path[fill=fillColor,fill opacity=0.20] (214.53, 85.74) circle (  2.13);

\path[fill=fillColor,fill opacity=0.20] (214.53, 86.24) circle (  2.13);

\path[fill=fillColor,fill opacity=0.20] (214.53, 79.67) circle (  2.13);

\path[fill=fillColor,fill opacity=0.20] (212.52, 76.00) circle (  2.13);

\path[fill=fillColor,fill opacity=0.20] (210.51, 73.22) circle (  2.13);

\path[fill=fillColor,fill opacity=0.20] (207.50, 66.39) circle (  2.13);

\path[fill=fillColor,fill opacity=0.20] (197.47, 70.44) circle (  2.13);

\path[fill=fillColor,fill opacity=0.20] (196.47, 78.28) circle (  2.13);

\path[fill=fillColor,fill opacity=0.20] (197.47, 78.78) circle (  2.13);

\path[fill=fillColor,fill opacity=0.20] (198.47, 80.68) circle (  2.13);

\path[fill=fillColor,fill opacity=0.20] (198.47, 85.10) circle (  2.13);

\path[fill=fillColor,fill opacity=0.20] (199.48, 86.12) circle (  2.13);

\path[fill=fillColor,fill opacity=0.20] (214.53, 82.07) circle (  2.13);

\path[fill=fillColor,fill opacity=0.20] (213.52, 78.78) circle (  2.13);

\path[fill=fillColor,fill opacity=0.20] (236.60, 78.53) circle (  2.13);

\path[fill=fillColor,fill opacity=0.20] (245.63, 74.99) circle (  2.13);

\path[fill=fillColor,fill opacity=0.20] (237.60, 66.90) circle (  2.13);

\path[fill=fillColor,fill opacity=0.20] (237.60, 59.18) circle (  2.13);

\path[fill=fillColor,fill opacity=0.20] (219.54, 50.84) circle (  2.13);

\path[fill=fillColor,fill opacity=0.20] (265.69, 51.34) circle (  2.13);

\path[fill=fillColor,fill opacity=0.20] (233.59, 78.02) circle (  2.13);

\path[fill=fillColor,fill opacity=0.20] (221.55, 83.46) circle (  2.13);

\path[fill=fillColor,fill opacity=0.20] (212.52, 93.32) circle (  2.13);

\path[fill=fillColor,fill opacity=0.20] (210.51, 91.93) circle (  2.13);

\path[fill=fillColor,fill opacity=0.20] (213.52, 82.07) circle (  2.13);

\path[fill=fillColor,fill opacity=0.20] (212.52, 78.53) circle (  2.13);

\path[fill=fillColor,fill opacity=0.20] (211.52, 79.29) circle (  2.13);

\path[fill=fillColor,fill opacity=0.20] (217.53, 77.77) circle (  2.13);

\path[fill=fillColor,fill opacity=0.20] (218.54, 75.75) circle (  2.13);

\path[fill=fillColor,fill opacity=0.20] (215.53, 69.55) circle (  2.13);

\path[fill=fillColor,fill opacity=0.20] (194.46, 71.20) circle (  2.13);

\path[fill=fillColor,fill opacity=0.20] (196.47, 76.38) circle (  2.13);

\path[fill=fillColor,fill opacity=0.20] (199.48, 82.70) circle (  2.13);

\path[fill=fillColor,fill opacity=0.20] (204.49, 85.99) circle (  2.13);

\path[fill=fillColor,fill opacity=0.20] (206.50, 86.87) circle (  2.13);

\path[fill=fillColor,fill opacity=0.20] (208.51, 81.56) circle (  2.13);

\path[fill=fillColor,fill opacity=0.20] (219.54, 75.24) circle (  2.13);

\path[fill=fillColor,fill opacity=0.20] (224.56, 74.48) circle (  2.13);

\path[fill=fillColor,fill opacity=0.20] (227.57, 73.72) circle (  2.13);

\path[fill=fillColor,fill opacity=0.20] (235.59, 67.91) circle (  2.13);

\path[fill=fillColor,fill opacity=0.20] (242.62, 64.11) circle (  2.13);

\path[fill=fillColor,fill opacity=0.20] (231.58, 61.58) circle (  2.13);

\path[fill=fillColor,fill opacity=0.20] (243.62, 53.24) circle (  2.13);

\path[fill=fillColor,fill opacity=0.20] (264.69, 45.91) circle (  2.13);

\path[fill=fillColor,fill opacity=0.20] (257.66, 57.66) circle (  2.13);

\path[fill=fillColor,fill opacity=0.20] (255.66, 69.68) circle (  2.13);

\path[fill=fillColor,fill opacity=0.20] (209.51, 80.05) circle (  2.13);

\path[fill=fillColor,fill opacity=0.20] (224.56, 82.70) circle (  2.13);

\path[fill=fillColor,fill opacity=0.20] (214.53, 81.31) circle (  2.13);

\path[fill=fillColor,fill opacity=0.20] (210.51, 83.97) circle (  2.13);

\path[fill=fillColor,fill opacity=0.20] (213.52, 83.97) circle (  2.13);

\path[fill=fillColor,fill opacity=0.20] (216.53, 79.67) circle (  2.13);

\path[fill=fillColor,fill opacity=0.20] (218.54, 80.55) circle (  2.13);

\path[fill=fillColor,fill opacity=0.20] (218.54, 84.35) circle (  2.13);

\path[fill=fillColor,fill opacity=0.20] (219.54, 79.79) circle (  2.13);

\path[fill=fillColor,fill opacity=0.20] (223.55, 73.60) circle (  2.13);

\path[fill=fillColor,fill opacity=0.20] (227.57, 72.46) circle (  2.13);

\path[fill=fillColor,fill opacity=0.20] (216.53, 70.06) circle (  2.13);

\path[fill=fillColor,fill opacity=0.20] (208.51, 66.26) circle (  2.13);

\path[fill=fillColor,fill opacity=0.20] (223.55, 66.64) circle (  2.13);

\path[fill=fillColor,fill opacity=0.20] (213.52, 68.16) circle (  2.13);

\path[fill=fillColor,fill opacity=0.20] (216.53, 67.40) circle (  2.13);

\path[fill=fillColor,fill opacity=0.20] (216.53, 65.76) circle (  2.13);

\path[fill=fillColor,fill opacity=0.20] (217.53, 64.75) circle (  2.13);

\path[fill=fillColor,fill opacity=0.20] (221.55, 65.63) circle (  2.13);

\path[fill=fillColor,fill opacity=0.20] (204.49, 71.45) circle (  2.13);

\path[fill=fillColor,fill opacity=0.20] (203.49, 74.10) circle (  2.13);

\path[fill=fillColor,fill opacity=0.20] (198.47, 76.00) circle (  2.13);

\path[fill=fillColor,fill opacity=0.20] (201.48, 77.39) circle (  2.13);

\path[fill=fillColor,fill opacity=0.20] (201.48, 81.56) circle (  2.13);

\path[fill=fillColor,fill opacity=0.20] (198.47, 87.51) circle (  2.13);

\path[fill=fillColor,fill opacity=0.20] (199.48, 85.86) circle (  2.13);

\path[fill=fillColor,fill opacity=0.20] (194.46, 80.17) circle (  2.13);

\path[fill=fillColor,fill opacity=0.20] (213.52, 79.92) circle (  2.13);

\path[fill=fillColor,fill opacity=0.20] (217.53, 80.17) circle (  2.13);

\path[fill=fillColor,fill opacity=0.20] (231.58, 74.10) circle (  2.13);

\path[fill=fillColor,fill opacity=0.20] (233.59, 68.67) circle (  2.13);

\path[fill=fillColor,fill opacity=0.20] (223.55, 62.98) circle (  2.13);

\path[fill=fillColor,fill opacity=0.20] (271.71, 58.68) circle (  2.13);

\path[fill=fillColor,fill opacity=0.20] (247.63, 53.87) circle (  2.13);

\path[fill=fillColor,fill opacity=0.20] (233.59, 49.70) circle (  2.13);

\path[fill=fillColor,fill opacity=0.20] (231.58, 47.04) circle (  2.13);

\path[fill=fillColor,fill opacity=0.20] (244.62, 44.01) circle (  2.13);

\path[fill=fillColor,fill opacity=0.20] (251.64, 52.86) circle (  2.13);

\path[fill=fillColor,fill opacity=0.20] (237.60, 67.65) circle (  2.13);

\path[fill=fillColor,fill opacity=0.20] (217.53, 75.12) circle (  2.13);

\path[fill=fillColor,fill opacity=0.20] (218.54, 80.68) circle (  2.13);

\path[fill=fillColor,fill opacity=0.20] (218.54, 80.93) circle (  2.13);

\path[fill=fillColor,fill opacity=0.20] (219.54, 85.36) circle (  2.13);

\path[fill=fillColor,fill opacity=0.20] (217.53, 89.78) circle (  2.13);

\path[fill=fillColor,fill opacity=0.20] (210.51, 85.48) circle (  2.13);

\path[fill=fillColor,fill opacity=0.20] (218.54, 83.08) circle (  2.13);

\path[fill=fillColor,fill opacity=0.20] (220.54, 83.08) circle (  2.13);

\path[fill=fillColor,fill opacity=0.20] (218.54, 82.70) circle (  2.13);

\path[fill=fillColor,fill opacity=0.20] (215.53, 80.30) circle (  2.13);

\path[fill=fillColor,fill opacity=0.20] (224.56, 79.03) circle (  2.13);

\path[fill=fillColor,fill opacity=0.20] (219.54, 79.67) circle (  2.13);

\path[fill=fillColor,fill opacity=0.20] (213.52, 79.54) circle (  2.13);

\path[fill=fillColor,fill opacity=0.20] (215.53, 79.92) circle (  2.13);

\path[fill=fillColor,fill opacity=0.20] (209.51, 80.55) circle (  2.13);

\path[fill=fillColor,fill opacity=0.20] (210.51, 81.82) circle (  2.13);

\path[fill=fillColor,fill opacity=0.20] (213.52, 83.21) circle (  2.13);

\path[fill=fillColor,fill opacity=0.20] (214.53, 83.71) circle (  2.13);

\path[fill=fillColor,fill opacity=0.20] (204.49, 80.43) circle (  2.13);

\path[fill=fillColor,fill opacity=0.20] (204.49, 78.28) circle (  2.13);

\path[fill=fillColor,fill opacity=0.20] (202.49, 77.64) circle (  2.13);

\path[fill=fillColor,fill opacity=0.20] (201.48, 81.06) circle (  2.13);

\path[fill=fillColor,fill opacity=0.20] (201.48, 87.00) circle (  2.13);

\path[fill=fillColor,fill opacity=0.20] (209.51, 89.28) circle (  2.13);

\path[fill=fillColor,fill opacity=0.20] (215.53, 88.27) circle (  2.13);

\path[fill=fillColor,fill opacity=0.20] (218.54, 80.81) circle (  2.13);

\path[fill=fillColor,fill opacity=0.20] (237.60, 71.20) circle (  2.13);

\path[fill=fillColor,fill opacity=0.20] (244.62, 66.14) circle (  2.13);

\path[fill=fillColor,fill opacity=0.20] (247.63, 60.45) circle (  2.13);

\path[fill=fillColor,fill opacity=0.20] (262.68, 50.84) circle (  2.13);

\path[fill=fillColor,fill opacity=0.20] (249.64, 43.00) circle (  2.13);

\path[fill=fillColor,fill opacity=0.20] (243.62, 43.25) circle (  2.13);

\path[fill=fillColor,fill opacity=0.20] (246.63, 48.05) circle (  2.13);

\path[fill=fillColor,fill opacity=0.20] (246.63, 55.64) circle (  2.13);

\path[fill=fillColor,fill opacity=0.20] (263.68, 66.01) circle (  2.13);

\path[fill=fillColor,fill opacity=0.20] (233.59, 77.26) circle (  2.13);

\path[fill=fillColor,fill opacity=0.20] (235.59, 86.62) circle (  2.13);

\path[fill=fillColor,fill opacity=0.20] (221.55, 88.14) circle (  2.13);

\path[fill=fillColor,fill opacity=0.20] (218.54, 84.09) circle (  2.13);

\path[fill=fillColor,fill opacity=0.20] (215.53, 84.22) circle (  2.13);

\path[fill=fillColor,fill opacity=0.20] (213.52, 87.63) circle (  2.13);

\path[fill=fillColor,fill opacity=0.20] (209.51, 87.76) circle (  2.13);

\path[fill=fillColor,fill opacity=0.20] (215.53, 88.14) circle (  2.13);

\path[fill=fillColor,fill opacity=0.20] (214.53, 87.13) circle (  2.13);

\path[fill=fillColor,fill opacity=0.20] (208.51, 83.46) circle (  2.13);

\path[fill=fillColor,fill opacity=0.20] (208.51, 80.17) circle (  2.13);

\path[fill=fillColor,fill opacity=0.20] (209.51, 80.43) circle (  2.13);

\path[fill=fillColor,fill opacity=0.20] (207.50, 83.71) circle (  2.13);

\path[fill=fillColor,fill opacity=0.20] (208.51, 84.35) circle (  2.13);

\path[fill=fillColor,fill opacity=0.20] (216.53, 84.73) circle (  2.13);

\path[fill=fillColor,fill opacity=0.20] (212.52, 83.59) circle (  2.13);

\path[fill=fillColor,fill opacity=0.20] (212.52, 78.53) circle (  2.13);

\path[fill=fillColor,fill opacity=0.20] (220.54, 74.48) circle (  2.13);

\path[fill=fillColor,fill opacity=0.20] (216.53, 73.85) circle (  2.13);

\path[fill=fillColor,fill opacity=0.20] (219.54, 76.25) circle (  2.13);

\path[fill=fillColor,fill opacity=0.20] (230.58, 81.18) circle (  2.13);

\path[fill=fillColor,fill opacity=0.20] (245.63, 80.43) circle (  2.13);

\path[fill=fillColor,fill opacity=0.20] (246.63, 70.44) circle (  2.13);

\path[fill=fillColor,fill opacity=0.20] (241.61, 53.11) circle (  2.13);

\path[fill=fillColor,fill opacity=0.20] (243.62, 47.04) circle (  2.13);

\path[fill=fillColor,fill opacity=0.20] (243.62, 56.27) circle (  2.13);

\path[fill=fillColor,fill opacity=0.20] (258.67, 61.46) circle (  2.13);

\path[fill=fillColor,fill opacity=0.20] (224.56, 71.07) circle (  2.13);

\path[fill=fillColor,fill opacity=0.20] (227.57, 79.03) circle (  2.13);

\path[fill=fillColor,fill opacity=0.20] (220.54, 81.18) circle (  2.13);

\path[fill=fillColor,fill opacity=0.20] (217.53, 81.18) circle (  2.13);

\path[fill=fillColor,fill opacity=0.20] (215.53, 84.85) circle (  2.13);

\path[fill=fillColor,fill opacity=0.20] (212.52, 91.43) circle (  2.13);

\path[fill=fillColor,fill opacity=0.20] (207.50, 88.27) circle (  2.13);

\path[fill=fillColor,fill opacity=0.20] (208.51, 80.17) circle (  2.13);

\path[fill=fillColor,fill opacity=0.20] (217.53, 78.91) circle (  2.13);

\path[fill=fillColor,fill opacity=0.20] (221.55, 80.93) circle (  2.13);

\path[fill=fillColor,fill opacity=0.20] (225.56, 80.30) circle (  2.13);

\path[fill=fillColor,fill opacity=0.20] (241.61, 77.01) circle (  2.13);

\path[fill=fillColor,fill opacity=0.20] (232.58, 76.38) circle (  2.13);

\path[fill=fillColor,fill opacity=0.20] (261.68, 75.62) circle (  2.13);

\path[fill=fillColor,fill opacity=0.20] (268.70, 75.24) circle (  2.13);

\path[fill=fillColor,fill opacity=0.20] (243.62, 70.56) circle (  2.13);

\path[fill=fillColor,fill opacity=0.20] (236.60, 62.60) circle (  2.13);

\path[fill=fillColor,fill opacity=0.20] (238.60, 60.70) circle (  2.13);

\path[fill=fillColor,fill opacity=0.20] (236.60, 58.55) circle (  2.13);

\path[fill=fillColor,fill opacity=0.20] (248.64, 49.83) circle (  2.13);

\path[fill=fillColor,fill opacity=0.20] (244.62, 40.97) circle (  2.13);

\path[fill=fillColor,fill opacity=0.20] (253.65, 48.56) circle (  2.13);

\path[fill=fillColor,fill opacity=0.20] (252.65, 58.68) circle (  2.13);

\path[fill=fillColor,fill opacity=0.20] (259.67, 61.08) circle (  2.13);

\path[fill=fillColor,fill opacity=0.20] (236.60, 62.34) circle (  2.13);

\path[fill=fillColor,fill opacity=0.20] (208.51, 66.90) circle (  2.13);

\path[fill=fillColor,fill opacity=0.20] (234.59, 73.47) circle (  2.13);

\path[fill=fillColor,fill opacity=0.20] (231.58, 78.02) circle (  2.13);

\path[fill=fillColor,fill opacity=0.20] (219.54, 75.49) circle (  2.13);

\path[fill=fillColor,fill opacity=0.20] (265.69, 71.20) circle (  2.13);

\path[fill=fillColor,fill opacity=0.20] (252.65, 66.64) circle (  2.13);

\path[fill=fillColor,fill opacity=0.20] (270.71, 60.32) circle (  2.13);

\path[fill=fillColor,fill opacity=0.20] (244.62, 54.76) circle (  2.13);

\path[fill=fillColor,fill opacity=0.20] (244.62, 53.87) circle (  2.13);

\path[fill=fillColor,fill opacity=0.20] (243.62, 55.77) circle (  2.13);

\path[fill=fillColor,fill opacity=0.20] (230.58, 54.88) circle (  2.13);

\path[fill=fillColor,fill opacity=0.20] (209.51, 51.34) circle (  2.13);

\path[fill=fillColor,fill opacity=0.20] (228.57, 41.48) circle (  2.13);

\path[fill=fillColor,fill opacity=0.20] (236.60, 44.13) circle (  2.13);

\path[fill=fillColor,fill opacity=0.20] (238.60, 47.68) circle (  2.13);

\path[fill=fillColor,fill opacity=0.20] (235.59, 49.57) circle (  2.13);

\path[fill=fillColor,fill opacity=0.20] (232.58, 54.38) circle (  2.13);

\path[fill=fillColor,fill opacity=0.20] (253.65, 56.15) circle (  2.13);

\path[fill=fillColor,fill opacity=0.20] (249.64, 50.71) circle (  2.13);

\path[fill=fillColor,fill opacity=0.20] (242.62, 44.89) circle (  2.13);
\end{scope}
\begin{scope}
\path[clip] (  0.00,  0.00) rectangle (289.08,144.54);
\definecolor[named]{drawColor}{rgb}{0.50,0.50,0.50}

\node[text=drawColor,anchor=base,inner sep=0pt, outer sep=0pt, scale=  0.96] at ( 48.20, 20.31) {8};

\node[text=drawColor,anchor=base,inner sep=0pt, outer sep=0pt, scale=  0.96] at ( 68.26, 20.31) {10};

\node[text=drawColor,anchor=base,inner sep=0pt, outer sep=0pt, scale=  0.96] at ( 88.33, 20.31) {12};

\node[text=drawColor,anchor=base,inner sep=0pt, outer sep=0pt, scale=  0.96] at (108.39, 20.31) {14};

\node[text=drawColor,anchor=base,inner sep=0pt, outer sep=0pt, scale=  0.96] at (128.46, 20.31) {16};

\node[text=drawColor,anchor=base,inner sep=0pt, outer sep=0pt, scale=  0.96] at (148.52, 20.31) {18};
\end{scope}
\begin{scope}
\path[clip] (  0.00,  0.00) rectangle (289.08,144.54);
\definecolor[named]{drawColor}{rgb}{0.50,0.50,0.50}

\path[draw=drawColor,line width= 0.6pt,line join=round,line cap=round] ( 48.20, 29.77) -- ( 48.20, 34.04);

\path[draw=drawColor,line width= 0.6pt,line join=round,line cap=round] ( 68.26, 29.77) -- ( 68.26, 34.04);

\path[draw=drawColor,line width= 0.6pt,line join=round,line cap=round] ( 88.33, 29.77) -- ( 88.33, 34.04);

\path[draw=drawColor,line width= 0.6pt,line join=round,line cap=round] (108.39, 29.77) -- (108.39, 34.04);

\path[draw=drawColor,line width= 0.6pt,line join=round,line cap=round] (128.46, 29.77) -- (128.46, 34.04);

\path[draw=drawColor,line width= 0.6pt,line join=round,line cap=round] (148.52, 29.77) -- (148.52, 34.04);
\end{scope}
\begin{scope}
\path[clip] (  0.00,  0.00) rectangle (289.08,144.54);
\definecolor[named]{drawColor}{rgb}{0.50,0.50,0.50}

\node[text=drawColor,anchor=base,inner sep=0pt, outer sep=0pt, scale=  0.96] at (168.38, 20.31) {8};

\node[text=drawColor,anchor=base,inner sep=0pt, outer sep=0pt, scale=  0.96] at (188.44, 20.31) {10};

\node[text=drawColor,anchor=base,inner sep=0pt, outer sep=0pt, scale=  0.96] at (208.51, 20.31) {12};

\node[text=drawColor,anchor=base,inner sep=0pt, outer sep=0pt, scale=  0.96] at (228.57, 20.31) {14};

\node[text=drawColor,anchor=base,inner sep=0pt, outer sep=0pt, scale=  0.96] at (248.64, 20.31) {16};

\node[text=drawColor,anchor=base,inner sep=0pt, outer sep=0pt, scale=  0.96] at (268.70, 20.31) {18};
\end{scope}
\begin{scope}
\path[clip] (  0.00,  0.00) rectangle (289.08,144.54);
\definecolor[named]{drawColor}{rgb}{0.50,0.50,0.50}

\path[draw=drawColor,line width= 0.6pt,line join=round,line cap=round] (168.38, 29.77) -- (168.38, 34.04);

\path[draw=drawColor,line width= 0.6pt,line join=round,line cap=round] (188.44, 29.77) -- (188.44, 34.04);

\path[draw=drawColor,line width= 0.6pt,line join=round,line cap=round] (208.51, 29.77) -- (208.51, 34.04);

\path[draw=drawColor,line width= 0.6pt,line join=round,line cap=round] (228.57, 29.77) -- (228.57, 34.04);

\path[draw=drawColor,line width= 0.6pt,line join=round,line cap=round] (248.64, 29.77) -- (248.64, 34.04);

\path[draw=drawColor,line width= 0.6pt,line join=round,line cap=round] (268.70, 29.77) -- (268.70, 34.04);
\end{scope}
\begin{scope}
\path[clip] (  0.00,  0.00) rectangle (289.08,144.54);
\definecolor[named]{drawColor}{rgb}{0.00,0.00,0.00}

\node[text=drawColor,anchor=base,inner sep=0pt, outer sep=0pt, scale=  1.20] at (158.36,  9.03) {$a$ $[\mu m]$};
\end{scope}
\begin{scope}
\path[clip] (  0.00,  0.00) rectangle (289.08,144.54);
\definecolor[named]{drawColor}{rgb}{0.00,0.00,0.00}

\node[text=drawColor,rotate= 90.00,anchor=base,inner sep=0pt, outer sep=0pt, scale=  1.20] at ( 17.30, 76.95) {FA};
\end{scope}
\end{tikzpicture}

					\end{adjustbox}\\
					\begin{adjustbox}{width={\textwidth},totalheight=\textheight,keepaspectratio}
						\strut
						% Created by tikzDevice version 0.6.2-92-0ad2792 on 2012-09-27 18:24:49
% !TEX encoding = UTF-8 Unicode
\begin{tikzpicture}[x=1pt,y=1pt]
\definecolor[named]{fillColor}{rgb}{1.00,1.00,1.00}
\path[use as bounding box,fill=fillColor,fill opacity=0.00] (0,0) rectangle (289.08,144.54);
\begin{scope}
\path[clip] (  0.00,  0.00) rectangle (289.08,144.54);
\definecolor[named]{drawColor}{rgb}{1.00,1.00,1.00}
\definecolor[named]{fillColor}{rgb}{1.00,1.00,1.00}

\path[draw=drawColor,line width= 0.6pt,line join=round,line cap=round,fill=fillColor] ( -0.00,  0.00) rectangle (289.08,144.54);
\end{scope}
\begin{scope}
\path[clip] ( 39.69,119.86) rectangle (156.86,132.50);
\definecolor[named]{fillColor}{rgb}{0.80,0.80,0.80}

\path[fill=fillColor] ( 39.69,119.86) rectangle (156.86,132.50);
\definecolor[named]{drawColor}{rgb}{0.00,0.00,0.00}

\node[text=drawColor,anchor=base,inner sep=0pt, outer sep=0pt, scale=  0.96] at ( 98.27,122.87) {Scan (r=0.763)};
\end{scope}
\begin{scope}
\path[clip] (159.87,119.86) rectangle (277.03,132.50);
\definecolor[named]{fillColor}{rgb}{0.80,0.80,0.80}

\path[fill=fillColor] (159.87,119.86) rectangle (277.03,132.50);
\definecolor[named]{drawColor}{rgb}{0.00,0.00,0.00}

\node[text=drawColor,anchor=base,inner sep=0pt, outer sep=0pt, scale=  0.96] at (218.45,122.87) {Rescan (r=0.642)};
\end{scope}
\begin{scope}
\path[clip] ( 39.69, 34.04) rectangle (156.86,119.86);
\definecolor[named]{fillColor}{rgb}{0.90,0.90,0.90}

\path[fill=fillColor] ( 39.69, 34.04) rectangle (156.86,119.86);
\definecolor[named]{drawColor}{rgb}{0.95,0.95,0.95}

\path[draw=drawColor,line width= 0.3pt,line join=round] ( 39.69, 39.58) --
	(156.86, 39.58);

\path[draw=drawColor,line width= 0.3pt,line join=round] ( 39.69, 64.87) --
	(156.86, 64.87);

\path[draw=drawColor,line width= 0.3pt,line join=round] ( 39.69, 90.16) --
	(156.86, 90.16);

\path[draw=drawColor,line width= 0.3pt,line join=round] ( 39.69,115.45) --
	(156.86,115.45);

\path[draw=drawColor,line width= 0.3pt,line join=round] ( 49.61, 34.04) --
	( 49.61,119.86);

\path[draw=drawColor,line width= 0.3pt,line join=round] ( 71.46, 34.04) --
	( 71.46,119.86);

\path[draw=drawColor,line width= 0.3pt,line join=round] ( 93.31, 34.04) --
	( 93.31,119.86);

\path[draw=drawColor,line width= 0.3pt,line join=round] (115.16, 34.04) --
	(115.16,119.86);

\path[draw=drawColor,line width= 0.3pt,line join=round] (137.01, 34.04) --
	(137.01,119.86);
\definecolor[named]{drawColor}{rgb}{1.00,1.00,1.00}

\path[draw=drawColor,line width= 0.6pt,line join=round] ( 39.69, 52.23) --
	(156.86, 52.23);

\path[draw=drawColor,line width= 0.6pt,line join=round] ( 39.69, 77.52) --
	(156.86, 77.52);

\path[draw=drawColor,line width= 0.6pt,line join=round] ( 39.69,102.81) --
	(156.86,102.81);

\path[draw=drawColor,line width= 0.6pt,line join=round] ( 60.53, 34.04) --
	( 60.53,119.86);

\path[draw=drawColor,line width= 0.6pt,line join=round] ( 82.38, 34.04) --
	( 82.38,119.86);

\path[draw=drawColor,line width= 0.6pt,line join=round] (104.23, 34.04) --
	(104.23,119.86);

\path[draw=drawColor,line width= 0.6pt,line join=round] (126.08, 34.04) --
	(126.08,119.86);

\path[draw=drawColor,line width= 0.6pt,line join=round] (147.93, 34.04) --
	(147.93,119.86);
\definecolor[named]{fillColor}{rgb}{0.00,0.00,0.00}

\path[fill=fillColor,fill opacity=0.20] ( 52.23, 45.91) circle (  2.13);

\path[fill=fillColor,fill opacity=0.20] ( 64.25, 41.73) circle (  2.13);

\path[fill=fillColor,fill opacity=0.20] ( 76.26, 55.39) circle (  2.13);

\path[fill=fillColor,fill opacity=0.20] ( 86.97, 71.20) circle (  2.13);

\path[fill=fillColor,fill opacity=0.20] ( 81.07, 74.48) circle (  2.13);

\path[fill=fillColor,fill opacity=0.20] ( 75.17, 71.83) circle (  2.13);

\path[fill=fillColor,fill opacity=0.20] ( 71.02, 56.15) circle (  2.13);

\path[fill=fillColor,fill opacity=0.20] ( 75.17, 47.30) circle (  2.13);

\path[fill=fillColor,fill opacity=0.20] ( 85.88, 72.08) circle (  2.13);

\path[fill=fillColor,fill opacity=0.20] ( 97.68, 87.76) circle (  2.13);

\path[fill=fillColor,fill opacity=0.20] (108.60, 89.02) circle (  2.13);

\path[fill=fillColor,fill opacity=0.20] ( 96.36, 88.14) circle (  2.13);

\path[fill=fillColor,fill opacity=0.20] (103.79, 87.89) circle (  2.13);

\path[fill=fillColor,fill opacity=0.20] (100.30, 85.48) circle (  2.13);

\path[fill=fillColor,fill opacity=0.20] ( 91.78, 67.40) circle (  2.13);

\path[fill=fillColor,fill opacity=0.20] ( 69.49, 54.88) circle (  2.13);

\path[fill=fillColor,fill opacity=0.20] ( 58.13, 45.65) circle (  2.13);

\path[fill=fillColor,fill opacity=0.20] ( 85.22, 57.41) circle (  2.13);

\path[fill=fillColor,fill opacity=0.20] ( 90.68, 79.16) circle (  2.13);

\path[fill=fillColor,fill opacity=0.20] ( 94.84, 89.40) circle (  2.13);

\path[fill=fillColor,fill opacity=0.20] ( 98.33, 97.75) circle (  2.13);

\path[fill=fillColor,fill opacity=0.20] (102.92,101.04) circle (  2.13);

\path[fill=fillColor,fill opacity=0.20] (100.08, 95.22) circle (  2.13);

\path[fill=fillColor,fill opacity=0.20] ( 95.27, 87.25) circle (  2.13);

\path[fill=fillColor,fill opacity=0.20] ( 93.52, 86.87) circle (  2.13);

\path[fill=fillColor,fill opacity=0.20] ( 88.06, 89.28) circle (  2.13);

\path[fill=fillColor,fill opacity=0.20] ( 76.04, 73.09) circle (  2.13);

\path[fill=fillColor,fill opacity=0.20] ( 74.95, 53.87) circle (  2.13);

\path[fill=fillColor,fill opacity=0.20] ( 71.46, 41.48) circle (  2.13);

\path[fill=fillColor,fill opacity=0.20] ( 58.56, 38.44) circle (  2.13);

\path[fill=fillColor,fill opacity=0.20] ( 81.51, 55.89) circle (  2.13);

\path[fill=fillColor,fill opacity=0.20] ( 98.11, 79.79) circle (  2.13);

\path[fill=fillColor,fill opacity=0.20] (111.66, 98.51) circle (  2.13);

\path[fill=fillColor,fill opacity=0.20] (106.42, 94.08) circle (  2.13);

\path[fill=fillColor,fill opacity=0.20] ( 99.21, 97.62) circle (  2.13);

\path[fill=fillColor,fill opacity=0.20] ( 92.87,106.47) circle (  2.13);

\path[fill=fillColor,fill opacity=0.20] ( 97.68,100.15) circle (  2.13);

\path[fill=fillColor,fill opacity=0.20] ( 99.86, 91.17) circle (  2.13);

\path[fill=fillColor,fill opacity=0.20] ( 88.94, 89.66) circle (  2.13);

\path[fill=fillColor,fill opacity=0.20] ( 79.76, 86.75) circle (  2.13);

\path[fill=fillColor,fill opacity=0.20] ( 77.14, 73.22) circle (  2.13);

\path[fill=fillColor,fill opacity=0.20] ( 77.79, 60.83) circle (  2.13);

\path[fill=fillColor,fill opacity=0.20] ( 74.30, 48.05) circle (  2.13);

\path[fill=fillColor,fill opacity=0.20] ( 90.25,110.90) circle (  2.13);

\path[fill=fillColor,fill opacity=0.20] ( 60.97, 47.80) circle (  2.13);

\path[fill=fillColor,fill opacity=0.20] ( 76.48, 75.49) circle (  2.13);

\path[fill=fillColor,fill opacity=0.20] ( 98.77, 92.06) circle (  2.13);

\path[fill=fillColor,fill opacity=0.20] (107.95,105.59) circle (  2.13);

\path[fill=fillColor,fill opacity=0.20] (105.10,106.22) circle (  2.13);

\path[fill=fillColor,fill opacity=0.20] (103.58,102.30) circle (  2.13);

\path[fill=fillColor,fill opacity=0.20] ( 95.27, 99.14) circle (  2.13);

\path[fill=fillColor,fill opacity=0.20] ( 83.25, 97.24) circle (  2.13);

\path[fill=fillColor,fill opacity=0.20] ( 89.37, 97.88) circle (  2.13);

\path[fill=fillColor,fill opacity=0.20] ( 85.22, 94.21) circle (  2.13);

\path[fill=fillColor,fill opacity=0.20] ( 76.92, 84.22) circle (  2.13);

\path[fill=fillColor,fill opacity=0.20] ( 74.30, 71.70) circle (  2.13);

\path[fill=fillColor,fill opacity=0.20] ( 65.56, 57.79) circle (  2.13);

\path[fill=fillColor,fill opacity=0.20] ( 54.19, 38.32) circle (  2.13);

\path[fill=fillColor,fill opacity=0.20] ( 70.80, 76.25) circle (  2.13);

\path[fill=fillColor,fill opacity=0.20] ( 90.68,101.16) circle (  2.13);

\path[fill=fillColor,fill opacity=0.20] ( 81.51, 80.93) circle (  2.13);

\path[fill=fillColor,fill opacity=0.20] ( 81.29, 81.44) circle (  2.13);

\path[fill=fillColor,fill opacity=0.20] ( 83.47, 89.53) circle (  2.13);

\path[fill=fillColor,fill opacity=0.20] ( 64.25, 50.20) circle (  2.13);

\path[fill=fillColor,fill opacity=0.20] ( 78.23, 87.25) circle (  2.13);

\path[fill=fillColor,fill opacity=0.20] ( 95.93,102.55) circle (  2.13);

\path[fill=fillColor,fill opacity=0.20] (102.48,109.13) circle (  2.13);

\path[fill=fillColor,fill opacity=0.20] (102.05,114.69) circle (  2.13);

\path[fill=fillColor,fill opacity=0.20] ( 99.21, 98.00) circle (  2.13);

\path[fill=fillColor,fill opacity=0.20] ( 87.84, 90.79) circle (  2.13);

\path[fill=fillColor,fill opacity=0.20] ( 73.86, 93.32) circle (  2.13);

\path[fill=fillColor,fill opacity=0.20] ( 71.67, 91.93) circle (  2.13);

\path[fill=fillColor,fill opacity=0.20] ( 64.25, 84.35) circle (  2.13);

\path[fill=fillColor,fill opacity=0.20] ( 66.65, 69.42) circle (  2.13);

\path[fill=fillColor,fill opacity=0.20] (105.32, 96.61) circle (  2.13);

\path[fill=fillColor,fill opacity=0.20] ( 97.02, 91.81) circle (  2.13);

\path[fill=fillColor,fill opacity=0.20] ( 94.84, 96.86) circle (  2.13);

\path[fill=fillColor,fill opacity=0.20] ( 99.86, 97.62) circle (  2.13);

\path[fill=fillColor,fill opacity=0.20] ( 99.42, 96.23) circle (  2.13);

\path[fill=fillColor,fill opacity=0.20] ( 95.71,100.66) circle (  2.13);

\path[fill=fillColor,fill opacity=0.20] ( 81.29, 92.69) circle (  2.13);

\path[fill=fillColor,fill opacity=0.20] ( 55.94, 64.11) circle (  2.13);

\path[fill=fillColor,fill opacity=0.20] ( 70.36, 54.38) circle (  2.13);

\path[fill=fillColor,fill opacity=0.20] ( 77.57, 91.30) circle (  2.13);

\path[fill=fillColor,fill opacity=0.20] ( 88.06,110.14) circle (  2.13);

\path[fill=fillColor,fill opacity=0.20] ( 90.47,106.73) circle (  2.13);

\path[fill=fillColor,fill opacity=0.20] ( 87.62,103.69) circle (  2.13);

\path[fill=fillColor,fill opacity=0.20] ( 89.59,111.03) circle (  2.13);

\path[fill=fillColor,fill opacity=0.20] ( 91.99,104.45) circle (  2.13);

\path[fill=fillColor,fill opacity=0.20] ( 85.44, 86.12) circle (  2.13);

\path[fill=fillColor,fill opacity=0.20] ( 64.46, 83.46) circle (  2.13);

\path[fill=fillColor,fill opacity=0.20] ( 77.57, 76.38) circle (  2.13);

\path[fill=fillColor,fill opacity=0.20] (111.88,100.03) circle (  2.13);

\path[fill=fillColor,fill opacity=0.20] (110.35, 92.06) circle (  2.13);

\path[fill=fillColor,fill opacity=0.20] (116.03, 98.89) circle (  2.13);

\path[fill=fillColor,fill opacity=0.20] (126.95,110.39) circle (  2.13);

\path[fill=fillColor,fill opacity=0.20] (146.84,112.54) circle (  2.13);

\path[fill=fillColor,fill opacity=0.20] (113.41,106.85) circle (  2.13);

\path[fill=fillColor,fill opacity=0.20] (105.54,110.52) circle (  2.13);

\path[fill=fillColor,fill opacity=0.20] (101.39,104.32) circle (  2.13);

\path[fill=fillColor,fill opacity=0.20] ( 65.34, 73.98) circle (  2.13);

\path[fill=fillColor,fill opacity=0.20] ( 67.96, 56.15) circle (  2.13);

\path[fill=fillColor,fill opacity=0.20] ( 78.01, 88.52) circle (  2.13);

\path[fill=fillColor,fill opacity=0.20] ( 88.94,100.40) circle (  2.13);

\path[fill=fillColor,fill opacity=0.20] ( 81.94,102.81) circle (  2.13);

\path[fill=fillColor,fill opacity=0.20] ( 80.41, 96.23) circle (  2.13);

\path[fill=fillColor,fill opacity=0.20] ( 83.47, 96.23) circle (  2.13);

\path[fill=fillColor,fill opacity=0.20] ( 80.20, 98.26) circle (  2.13);

\path[fill=fillColor,fill opacity=0.20] ( 78.67, 86.12) circle (  2.13);

\path[fill=fillColor,fill opacity=0.20] ( 77.79, 76.63) circle (  2.13);

\path[fill=fillColor,fill opacity=0.20] ( 67.96, 75.62) circle (  2.13);

\path[fill=fillColor,fill opacity=0.20] ( 59.88, 60.70) circle (  2.13);

\path[fill=fillColor,fill opacity=0.20] (111.66,102.81) circle (  2.13);

\path[fill=fillColor,fill opacity=0.20] (115.59, 94.34) circle (  2.13);

\path[fill=fillColor,fill opacity=0.20] (119.09, 96.49) circle (  2.13);

\path[fill=fillColor,fill opacity=0.20] (133.73,113.94) circle (  2.13);

\path[fill=fillColor,fill opacity=0.20] (120.40,114.82) circle (  2.13);

\path[fill=fillColor,fill opacity=0.20] (122.58,107.87) circle (  2.13);

\path[fill=fillColor,fill opacity=0.20] (100.95,106.10) circle (  2.13);

\path[fill=fillColor,fill opacity=0.20] ( 96.15,108.88) circle (  2.13);

\path[fill=fillColor,fill opacity=0.20] (115.59,115.33) circle (  2.13);

\path[fill=fillColor,fill opacity=0.20] ( 79.98, 93.45) circle (  2.13);

\path[fill=fillColor,fill opacity=0.20] ( 60.31, 50.33) circle (  2.13);

\path[fill=fillColor,fill opacity=0.20] ( 81.07, 75.87) circle (  2.13);

\path[fill=fillColor,fill opacity=0.20] ( 81.94, 85.61) circle (  2.13);

\path[fill=fillColor,fill opacity=0.20] ( 83.91,103.19) circle (  2.13);

\path[fill=fillColor,fill opacity=0.20] ( 86.97,107.61) circle (  2.13);

\path[fill=fillColor,fill opacity=0.20] ( 83.04, 94.21) circle (  2.13);

\path[fill=fillColor,fill opacity=0.20] ( 86.10, 87.38) circle (  2.13);

\path[fill=fillColor,fill opacity=0.20] ( 80.63, 88.27) circle (  2.13);

\path[fill=fillColor,fill opacity=0.20] ( 82.82, 86.50) circle (  2.13);

\path[fill=fillColor,fill opacity=0.20] ( 71.24, 76.13) circle (  2.13);

\path[fill=fillColor,fill opacity=0.20] ( 89.15, 84.35) circle (  2.13);

\path[fill=fillColor,fill opacity=0.20] (117.78,102.18) circle (  2.13);

\path[fill=fillColor,fill opacity=0.20] (109.26,103.69) circle (  2.13);

\path[fill=fillColor,fill opacity=0.20] (101.17,111.03) circle (  2.13);

\path[fill=fillColor,fill opacity=0.20] (100.08,103.69) circle (  2.13);

\path[fill=fillColor,fill opacity=0.20] (102.48,100.78) circle (  2.13);

\path[fill=fillColor,fill opacity=0.20] ( 94.40,102.68) circle (  2.13);

\path[fill=fillColor,fill opacity=0.20] ( 82.60,101.04) circle (  2.13);

\path[fill=fillColor,fill opacity=0.20] ( 84.78,112.42) circle (  2.13);

\path[fill=fillColor,fill opacity=0.20] ( 56.38, 40.85) circle (  2.13);

\path[fill=fillColor,fill opacity=0.20] ( 75.39, 64.11) circle (  2.13);

\path[fill=fillColor,fill opacity=0.20] ( 81.73, 80.30) circle (  2.13);

\path[fill=fillColor,fill opacity=0.20] ( 91.78,104.70) circle (  2.13);

\path[fill=fillColor,fill opacity=0.20] ( 86.53,102.43) circle (  2.13);

\path[fill=fillColor,fill opacity=0.20] ( 83.47, 90.42) circle (  2.13);

\path[fill=fillColor,fill opacity=0.20] ( 81.94, 88.52) circle (  2.13);

\path[fill=fillColor,fill opacity=0.20] ( 80.41, 90.42) circle (  2.13);

\path[fill=fillColor,fill opacity=0.20] ( 78.23, 86.50) circle (  2.13);

\path[fill=fillColor,fill opacity=0.20] (129.14,104.45) circle (  2.13);

\path[fill=fillColor,fill opacity=0.20] (114.72,109.38) circle (  2.13);

\path[fill=fillColor,fill opacity=0.20] (102.70,102.05) circle (  2.13);

\path[fill=fillColor,fill opacity=0.20] (119.96,107.74) circle (  2.13);

\path[fill=fillColor,fill opacity=0.20] (100.95,105.72) circle (  2.13);

\path[fill=fillColor,fill opacity=0.20] ( 93.31,100.28) circle (  2.13);

\path[fill=fillColor,fill opacity=0.20] (100.73,100.53) circle (  2.13);

\path[fill=fillColor,fill opacity=0.20] ( 94.18, 98.38) circle (  2.13);

\path[fill=fillColor,fill opacity=0.20] ( 86.31, 95.60) circle (  2.13);

\path[fill=fillColor,fill opacity=0.20] ( 88.28,105.46) circle (  2.13);

\path[fill=fillColor,fill opacity=0.20] ( 79.32,103.06) circle (  2.13);

\path[fill=fillColor,fill opacity=0.20] ( 72.33, 54.76) circle (  2.13);

\path[fill=fillColor,fill opacity=0.20] ( 90.90, 79.03) circle (  2.13);

\path[fill=fillColor,fill opacity=0.20] ( 99.64, 96.36) circle (  2.13);

\path[fill=fillColor,fill opacity=0.20] ( 97.46,103.82) circle (  2.13);

\path[fill=fillColor,fill opacity=0.20] ( 88.06,104.96) circle (  2.13);

\path[fill=fillColor,fill opacity=0.20] ( 84.78,102.81) circle (  2.13);

\path[fill=fillColor,fill opacity=0.20] ( 83.04, 94.84) circle (  2.13);

\path[fill=fillColor,fill opacity=0.20] ( 76.48, 83.08) circle (  2.13);

\path[fill=fillColor,fill opacity=0.20] ( 79.98, 85.61) circle (  2.13);

\path[fill=fillColor,fill opacity=0.20] ( 66.21, 71.57) circle (  2.13);

\path[fill=fillColor,fill opacity=0.20] (117.34,111.03) circle (  2.13);

\path[fill=fillColor,fill opacity=0.20] (103.36, 93.70) circle (  2.13);

\path[fill=fillColor,fill opacity=0.20] (109.69, 95.22) circle (  2.13);

\path[fill=fillColor,fill opacity=0.20] (122.15,106.98) circle (  2.13);

\path[fill=fillColor,fill opacity=0.20] (106.20,105.21) circle (  2.13);

\path[fill=fillColor,fill opacity=0.20] ( 96.58,100.53) circle (  2.13);

\path[fill=fillColor,fill opacity=0.20] (106.63, 99.14) circle (  2.13);

\path[fill=fillColor,fill opacity=0.20] ( 83.25, 93.58) circle (  2.13);

\path[fill=fillColor,fill opacity=0.20] ( 88.06, 94.84) circle (  2.13);

\path[fill=fillColor,fill opacity=0.20] ( 87.62,103.31) circle (  2.13);

\path[fill=fillColor,fill opacity=0.20] ( 74.30, 82.20) circle (  2.13);

\path[fill=fillColor,fill opacity=0.20] ( 82.60, 74.48) circle (  2.13);

\path[fill=fillColor,fill opacity=0.20] ( 90.68, 84.22) circle (  2.13);

\path[fill=fillColor,fill opacity=0.20] ( 92.43, 89.91) circle (  2.13);

\path[fill=fillColor,fill opacity=0.20] ( 90.90, 97.62) circle (  2.13);

\path[fill=fillColor,fill opacity=0.20] ( 88.28,107.11) circle (  2.13);

\path[fill=fillColor,fill opacity=0.20] ( 90.68,102.43) circle (  2.13);

\path[fill=fillColor,fill opacity=0.20] ( 81.29, 81.94) circle (  2.13);

\path[fill=fillColor,fill opacity=0.20] ( 89.81, 77.39) circle (  2.13);

\path[fill=fillColor,fill opacity=0.20] ( 82.60, 81.18) circle (  2.13);

\path[fill=fillColor,fill opacity=0.20] ( 95.93, 97.62) circle (  2.13);

\path[fill=fillColor,fill opacity=0.20] ( 89.81, 98.63) circle (  2.13);

\path[fill=fillColor,fill opacity=0.20] ( 88.28, 89.53) circle (  2.13);

\path[fill=fillColor,fill opacity=0.20] ( 97.89, 95.73) circle (  2.13);

\path[fill=fillColor,fill opacity=0.20] (107.73,103.82) circle (  2.13);

\path[fill=fillColor,fill opacity=0.20] (100.52,100.28) circle (  2.13);

\path[fill=fillColor,fill opacity=0.20] ( 98.33, 99.77) circle (  2.13);

\path[fill=fillColor,fill opacity=0.20] (101.17,103.31) circle (  2.13);

\path[fill=fillColor,fill opacity=0.20] ( 92.87, 97.88) circle (  2.13);

\path[fill=fillColor,fill opacity=0.20] ( 85.88, 96.23) circle (  2.13);

\path[fill=fillColor,fill opacity=0.20] ( 85.66, 96.36) circle (  2.13);

\path[fill=fillColor,fill opacity=0.20] ( 66.43, 63.99) circle (  2.13);

\path[fill=fillColor,fill opacity=0.20] ( 72.11, 57.16) circle (  2.13);

\path[fill=fillColor,fill opacity=0.20] ( 80.63, 72.46) circle (  2.13);

\path[fill=fillColor,fill opacity=0.20] ( 82.38, 84.35) circle (  2.13);

\path[fill=fillColor,fill opacity=0.20] ( 91.99, 94.08) circle (  2.13);

\path[fill=fillColor,fill opacity=0.20] ( 86.75, 98.51) circle (  2.13);

\path[fill=fillColor,fill opacity=0.20] ( 87.62, 98.13) circle (  2.13);

\path[fill=fillColor,fill opacity=0.20] ( 88.50, 89.28) circle (  2.13);

\path[fill=fillColor,fill opacity=0.20] ( 79.32, 82.70) circle (  2.13);

\path[fill=fillColor,fill opacity=0.20] ( 75.83, 85.99) circle (  2.13);

\path[fill=fillColor,fill opacity=0.20] ( 64.90, 71.95) circle (  2.13);

\path[fill=fillColor,fill opacity=0.20] ( 86.31, 86.24) circle (  2.13);

\path[fill=fillColor,fill opacity=0.20] (110.13,100.40) circle (  2.13);

\path[fill=fillColor,fill opacity=0.20] ( 83.25, 92.31) circle (  2.13);

\path[fill=fillColor,fill opacity=0.20] ( 87.62, 93.58) circle (  2.13);

\path[fill=fillColor,fill opacity=0.20] ( 95.05,100.78) circle (  2.13);

\path[fill=fillColor,fill opacity=0.20] ( 98.33,103.82) circle (  2.13);

\path[fill=fillColor,fill opacity=0.20] (101.17, 99.77) circle (  2.13);

\path[fill=fillColor,fill opacity=0.20] ( 91.56,102.18) circle (  2.13);

\path[fill=fillColor,fill opacity=0.20] (106.63,108.62) circle (  2.13);

\path[fill=fillColor,fill opacity=0.20] (102.92,100.15) circle (  2.13);

\path[fill=fillColor,fill opacity=0.20] ( 90.47, 92.19) circle (  2.13);

\path[fill=fillColor,fill opacity=0.20] ( 83.25, 81.82) circle (  2.13);

\path[fill=fillColor,fill opacity=0.20] ( 71.46, 55.89) circle (  2.13);

\path[fill=fillColor,fill opacity=0.20] ( 89.59, 79.03) circle (  2.13);

\path[fill=fillColor,fill opacity=0.20] ( 90.68, 94.21) circle (  2.13);

\path[fill=fillColor,fill opacity=0.20] ( 85.00, 92.69) circle (  2.13);

\path[fill=fillColor,fill opacity=0.20] ( 89.59, 93.83) circle (  2.13);

\path[fill=fillColor,fill opacity=0.20] ( 90.68,100.66) circle (  2.13);

\path[fill=fillColor,fill opacity=0.20] ( 86.75,103.44) circle (  2.13);

\path[fill=fillColor,fill opacity=0.20] ( 77.79, 99.52) circle (  2.13);

\path[fill=fillColor,fill opacity=0.20] ( 76.26, 88.77) circle (  2.13);

\path[fill=fillColor,fill opacity=0.20] ( 67.52, 70.06) circle (  2.13);

\path[fill=fillColor,fill opacity=0.20] ( 72.99, 75.75) circle (  2.13);

\path[fill=fillColor,fill opacity=0.20] ( 97.24,104.07) circle (  2.13);

\path[fill=fillColor,fill opacity=0.20] ( 96.15, 97.50) circle (  2.13);

\path[fill=fillColor,fill opacity=0.20] ( 96.80,100.03) circle (  2.13);

\path[fill=fillColor,fill opacity=0.20] (105.32,102.18) circle (  2.13);

\path[fill=fillColor,fill opacity=0.20] (102.26,100.91) circle (  2.13);

\path[fill=fillColor,fill opacity=0.20] ( 99.64,101.42) circle (  2.13);

\path[fill=fillColor,fill opacity=0.20] (100.95,103.95) circle (  2.13);

\path[fill=fillColor,fill opacity=0.20] ( 98.99,106.98) circle (  2.13);

\path[fill=fillColor,fill opacity=0.20] (103.36,103.44) circle (  2.13);

\path[fill=fillColor,fill opacity=0.20] ( 95.93, 90.29) circle (  2.13);

\path[fill=fillColor,fill opacity=0.20] ( 94.84, 79.03) circle (  2.13);

\path[fill=fillColor,fill opacity=0.20] ( 78.67, 62.22) circle (  2.13);

\path[fill=fillColor,fill opacity=0.20] ( 88.06, 61.08) circle (  2.13);

\path[fill=fillColor,fill opacity=0.20] ( 99.42, 88.90) circle (  2.13);

\path[fill=fillColor,fill opacity=0.20] ( 84.35, 94.46) circle (  2.13);

\path[fill=fillColor,fill opacity=0.20] ( 89.37, 95.35) circle (  2.13);

\path[fill=fillColor,fill opacity=0.20] ( 89.81,102.55) circle (  2.13);

\path[fill=fillColor,fill opacity=0.20] ( 85.22,104.07) circle (  2.13);

\path[fill=fillColor,fill opacity=0.20] ( 86.97, 99.90) circle (  2.13);

\path[fill=fillColor,fill opacity=0.20] ( 78.45, 91.17) circle (  2.13);

\path[fill=fillColor,fill opacity=0.20] ( 75.17, 78.02) circle (  2.13);

\path[fill=fillColor,fill opacity=0.20] ( 60.31, 65.88) circle (  2.13);

\path[fill=fillColor,fill opacity=0.20] ( 75.17, 69.30) circle (  2.13);

\path[fill=fillColor,fill opacity=0.20] ( 97.46, 96.99) circle (  2.13);

\path[fill=fillColor,fill opacity=0.20] ( 94.18, 92.57) circle (  2.13);

\path[fill=fillColor,fill opacity=0.20] ( 97.46,100.15) circle (  2.13);

\path[fill=fillColor,fill opacity=0.20] (107.29,107.23) circle (  2.13);

\path[fill=fillColor,fill opacity=0.20] (108.38,103.69) circle (  2.13);

\path[fill=fillColor,fill opacity=0.20] (102.70, 96.36) circle (  2.13);

\path[fill=fillColor,fill opacity=0.20] ( 98.33, 93.70) circle (  2.13);

\path[fill=fillColor,fill opacity=0.20] (107.29, 98.63) circle (  2.13);

\path[fill=fillColor,fill opacity=0.20] ( 98.33,101.42) circle (  2.13);

\path[fill=fillColor,fill opacity=0.20] (100.52, 93.96) circle (  2.13);

\path[fill=fillColor,fill opacity=0.20] (100.52, 83.71) circle (  2.13);

\path[fill=fillColor,fill opacity=0.20] ( 86.10, 64.87) circle (  2.13);

\path[fill=fillColor,fill opacity=0.20] ( 57.69, 40.21) circle (  2.13);

\path[fill=fillColor,fill opacity=0.20] ( 75.61, 64.24) circle (  2.13);

\path[fill=fillColor,fill opacity=0.20] ( 88.28, 84.47) circle (  2.13);

\path[fill=fillColor,fill opacity=0.20] ( 86.53, 93.07) circle (  2.13);

\path[fill=fillColor,fill opacity=0.20] ( 81.73, 92.19) circle (  2.13);

\path[fill=fillColor,fill opacity=0.20] ( 89.37, 92.06) circle (  2.13);

\path[fill=fillColor,fill opacity=0.20] ( 86.97, 93.96) circle (  2.13);

\path[fill=fillColor,fill opacity=0.20] ( 86.10, 89.40) circle (  2.13);

\path[fill=fillColor,fill opacity=0.20] ( 77.36, 78.66) circle (  2.13);

\path[fill=fillColor,fill opacity=0.20] ( 72.77, 71.57) circle (  2.13);

\path[fill=fillColor,fill opacity=0.20] ( 65.77, 64.62) circle (  2.13);

\path[fill=fillColor,fill opacity=0.20] ( 68.40, 68.16) circle (  2.13);

\path[fill=fillColor,fill opacity=0.20] ( 97.68, 98.76) circle (  2.13);

\path[fill=fillColor,fill opacity=0.20] ( 94.18, 89.15) circle (  2.13);

\path[fill=fillColor,fill opacity=0.20] ( 95.05, 92.69) circle (  2.13);

\path[fill=fillColor,fill opacity=0.20] (100.95,100.28) circle (  2.13);

\path[fill=fillColor,fill opacity=0.20] (102.05, 97.75) circle (  2.13);

\path[fill=fillColor,fill opacity=0.20] ( 99.21, 96.74) circle (  2.13);

\path[fill=fillColor,fill opacity=0.20] (100.08, 96.86) circle (  2.13);

\path[fill=fillColor,fill opacity=0.20] (103.58, 94.97) circle (  2.13);

\path[fill=fillColor,fill opacity=0.20] (100.08, 92.06) circle (  2.13);

\path[fill=fillColor,fill opacity=0.20] (104.23, 86.24) circle (  2.13);

\path[fill=fillColor,fill opacity=0.20] (102.26, 85.74) circle (  2.13);

\path[fill=fillColor,fill opacity=0.20] ( 88.06, 82.95) circle (  2.13);

\path[fill=fillColor,fill opacity=0.20] ( 55.94, 38.07) circle (  2.13);

\path[fill=fillColor,fill opacity=0.20] ( 85.00, 57.92) circle (  2.13);

\path[fill=fillColor,fill opacity=0.20] ( 93.74, 80.68) circle (  2.13);

\path[fill=fillColor,fill opacity=0.20] ( 85.22, 85.86) circle (  2.13);

\path[fill=fillColor,fill opacity=0.20] ( 91.34, 93.96) circle (  2.13);

\path[fill=fillColor,fill opacity=0.20] ( 98.11,102.30) circle (  2.13);

\path[fill=fillColor,fill opacity=0.20] ( 86.75, 94.84) circle (  2.13);

\path[fill=fillColor,fill opacity=0.20] ( 84.78, 88.39) circle (  2.13);

\path[fill=fillColor,fill opacity=0.20] ( 83.25, 84.60) circle (  2.13);

\path[fill=fillColor,fill opacity=0.20] ( 83.47, 80.43) circle (  2.13);

\path[fill=fillColor,fill opacity=0.20] ( 68.83, 73.34) circle (  2.13);

\path[fill=fillColor,fill opacity=0.20] ( 59.22, 64.62) circle (  2.13);

\path[fill=fillColor,fill opacity=0.20] ( 92.21, 95.60) circle (  2.13);

\path[fill=fillColor,fill opacity=0.20] ( 93.09, 89.53) circle (  2.13);

\path[fill=fillColor,fill opacity=0.20] (103.14, 92.82) circle (  2.13);

\path[fill=fillColor,fill opacity=0.20] (103.36,103.69) circle (  2.13);

\path[fill=fillColor,fill opacity=0.20] (105.76,104.96) circle (  2.13);

\path[fill=fillColor,fill opacity=0.20] (102.48, 99.01) circle (  2.13);

\path[fill=fillColor,fill opacity=0.20] (100.73, 95.98) circle (  2.13);

\path[fill=fillColor,fill opacity=0.20] (103.79, 99.01) circle (  2.13);

\path[fill=fillColor,fill opacity=0.20] (110.35, 99.39) circle (  2.13);

\path[fill=fillColor,fill opacity=0.20] (104.89, 90.29) circle (  2.13);

\path[fill=fillColor,fill opacity=0.20] ( 98.11, 77.01) circle (  2.13);

\path[fill=fillColor,fill opacity=0.20] ( 76.70, 69.42) circle (  2.13);

\path[fill=fillColor,fill opacity=0.20] ( 91.12, 60.07) circle (  2.13);

\path[fill=fillColor,fill opacity=0.20] (101.39, 82.70) circle (  2.13);

\path[fill=fillColor,fill opacity=0.20] ( 98.33, 98.89) circle (  2.13);

\path[fill=fillColor,fill opacity=0.20] (102.48,105.97) circle (  2.13);

\path[fill=fillColor,fill opacity=0.20] ( 91.78, 97.88) circle (  2.13);

\path[fill=fillColor,fill opacity=0.20] ( 83.69, 94.08) circle (  2.13);

\path[fill=fillColor,fill opacity=0.20] ( 83.91, 88.90) circle (  2.13);

\path[fill=fillColor,fill opacity=0.20] ( 76.26, 82.58) circle (  2.13);

\path[fill=fillColor,fill opacity=0.20] ( 72.77, 80.55) circle (  2.13);

\path[fill=fillColor,fill opacity=0.20] ( 66.87, 70.82) circle (  2.13);

\path[fill=fillColor,fill opacity=0.20] ( 62.72, 65.25) circle (  2.13);

\path[fill=fillColor,fill opacity=0.20] ( 91.34, 89.53) circle (  2.13);

\path[fill=fillColor,fill opacity=0.20] ( 86.31, 86.75) circle (  2.13);

\path[fill=fillColor,fill opacity=0.20] ( 98.11, 92.31) circle (  2.13);

\path[fill=fillColor,fill opacity=0.20] (113.41,102.18) circle (  2.13);

\path[fill=fillColor,fill opacity=0.20] (116.69,108.37) circle (  2.13);

\path[fill=fillColor,fill opacity=0.20] (117.34,110.52) circle (  2.13);

\path[fill=fillColor,fill opacity=0.20] (109.04,108.62) circle (  2.13);

\path[fill=fillColor,fill opacity=0.20] (106.63,102.05) circle (  2.13);

\path[fill=fillColor,fill opacity=0.20] (101.61, 96.74) circle (  2.13);

\path[fill=fillColor,fill opacity=0.20] (100.08, 94.08) circle (  2.13);

\path[fill=fillColor,fill opacity=0.20] (111.88, 90.54) circle (  2.13);

\path[fill=fillColor,fill opacity=0.20] (104.67, 83.21) circle (  2.13);

\path[fill=fillColor,fill opacity=0.20] ( 74.73, 64.24) circle (  2.13);

\path[fill=fillColor,fill opacity=0.20] ( 98.77, 60.95) circle (  2.13);

\path[fill=fillColor,fill opacity=0.20] (102.92, 79.79) circle (  2.13);

\path[fill=fillColor,fill opacity=0.20] ( 97.02, 87.13) circle (  2.13);

\path[fill=fillColor,fill opacity=0.20] ( 97.68, 90.54) circle (  2.13);

\path[fill=fillColor,fill opacity=0.20] ( 84.78, 87.76) circle (  2.13);

\path[fill=fillColor,fill opacity=0.20] ( 82.82, 82.70) circle (  2.13);

\path[fill=fillColor,fill opacity=0.20] ( 81.51, 81.69) circle (  2.13);

\path[fill=fillColor,fill opacity=0.20] ( 72.33, 80.43) circle (  2.13);

\path[fill=fillColor,fill opacity=0.20] ( 72.33, 77.90) circle (  2.13);

\path[fill=fillColor,fill opacity=0.20] ( 68.18, 77.90) circle (  2.13);

\path[fill=fillColor,fill opacity=0.20] ( 70.36, 74.74) circle (  2.13);

\path[fill=fillColor,fill opacity=0.20] ( 97.24, 96.49) circle (  2.13);

\path[fill=fillColor,fill opacity=0.20] (102.70, 83.97) circle (  2.13);

\path[fill=fillColor,fill opacity=0.20] ( 95.71, 85.48) circle (  2.13);

\path[fill=fillColor,fill opacity=0.20] (112.10, 99.65) circle (  2.13);

\path[fill=fillColor,fill opacity=0.20] (112.10,100.66) circle (  2.13);

\path[fill=fillColor,fill opacity=0.20] (116.90, 98.26) circle (  2.13);

\path[fill=fillColor,fill opacity=0.20] (114.72,103.06) circle (  2.13);

\path[fill=fillColor,fill opacity=0.20] (110.79,108.12) circle (  2.13);

\path[fill=fillColor,fill opacity=0.20] (109.47,104.83) circle (  2.13);

\path[fill=fillColor,fill opacity=0.20] (100.95, 96.36) circle (  2.13);

\path[fill=fillColor,fill opacity=0.20] ( 98.33, 93.45) circle (  2.13);

\path[fill=fillColor,fill opacity=0.20] ( 97.68, 89.15) circle (  2.13);

\path[fill=fillColor,fill opacity=0.20] ( 70.14, 70.56) circle (  2.13);

\path[fill=fillColor,fill opacity=0.20] ( 91.34, 51.85) circle (  2.13);

\path[fill=fillColor,fill opacity=0.20] (100.30, 64.11) circle (  2.13);

\path[fill=fillColor,fill opacity=0.20] ( 91.34, 80.93) circle (  2.13);

\path[fill=fillColor,fill opacity=0.20] ( 87.41, 87.13) circle (  2.13);

\path[fill=fillColor,fill opacity=0.20] ( 84.78, 80.43) circle (  2.13);

\path[fill=fillColor,fill opacity=0.20] ( 82.60, 82.58) circle (  2.13);

\path[fill=fillColor,fill opacity=0.20] ( 80.41, 85.61) circle (  2.13);

\path[fill=fillColor,fill opacity=0.20] ( 72.77, 84.35) circle (  2.13);

\path[fill=fillColor,fill opacity=0.20] ( 70.80, 87.25) circle (  2.13);

\path[fill=fillColor,fill opacity=0.20] ( 81.73, 87.25) circle (  2.13);

\path[fill=fillColor,fill opacity=0.20] ( 70.58, 79.41) circle (  2.13);

\path[fill=fillColor,fill opacity=0.20] ( 58.35, 62.85) circle (  2.13);

\path[fill=fillColor,fill opacity=0.20] ( 78.01, 90.92) circle (  2.13);

\path[fill=fillColor,fill opacity=0.20] ( 87.19, 93.70) circle (  2.13);

\path[fill=fillColor,fill opacity=0.20] ( 98.33, 89.91) circle (  2.13);

\path[fill=fillColor,fill opacity=0.20] ( 92.65, 90.92) circle (  2.13);

\path[fill=fillColor,fill opacity=0.20] (102.92, 92.57) circle (  2.13);

\path[fill=fillColor,fill opacity=0.20] (105.98, 94.59) circle (  2.13);

\path[fill=fillColor,fill opacity=0.20] (106.42, 92.44) circle (  2.13);

\path[fill=fillColor,fill opacity=0.20] (110.79, 88.39) circle (  2.13);

\path[fill=fillColor,fill opacity=0.20] (111.00, 91.93) circle (  2.13);

\path[fill=fillColor,fill opacity=0.20] (111.66,100.40) circle (  2.13);

\path[fill=fillColor,fill opacity=0.20] (106.85,101.16) circle (  2.13);

\path[fill=fillColor,fill opacity=0.20] (100.08, 96.86) circle (  2.13);

\path[fill=fillColor,fill opacity=0.20] ( 85.88, 94.08) circle (  2.13);

\path[fill=fillColor,fill opacity=0.20] ( 71.24, 74.23) circle (  2.13);

\path[fill=fillColor,fill opacity=0.20] ( 68.62, 44.39) circle (  2.13);

\path[fill=fillColor,fill opacity=0.20] ( 85.44, 64.24) circle (  2.13);

\path[fill=fillColor,fill opacity=0.20] ( 94.40, 86.75) circle (  2.13);

\path[fill=fillColor,fill opacity=0.20] ( 92.65, 86.37) circle (  2.13);

\path[fill=fillColor,fill opacity=0.20] ( 85.88, 82.83) circle (  2.13);

\path[fill=fillColor,fill opacity=0.20] ( 84.13, 86.50) circle (  2.13);

\path[fill=fillColor,fill opacity=0.20] ( 76.70, 86.24) circle (  2.13);

\path[fill=fillColor,fill opacity=0.20] ( 70.36, 79.16) circle (  2.13);

\path[fill=fillColor,fill opacity=0.20] ( 69.27, 76.76) circle (  2.13);

\path[fill=fillColor,fill opacity=0.20] ( 73.86, 84.22) circle (  2.13);

\path[fill=fillColor,fill opacity=0.20] ( 71.89, 91.05) circle (  2.13);

\path[fill=fillColor,fill opacity=0.20] ( 72.33, 84.98) circle (  2.13);

\path[fill=fillColor,fill opacity=0.20] ( 64.90, 68.67) circle (  2.13);

\path[fill=fillColor,fill opacity=0.20] ( 67.52, 68.92) circle (  2.13);

\path[fill=fillColor,fill opacity=0.20] ( 76.04, 75.49) circle (  2.13);

\path[fill=fillColor,fill opacity=0.20] ( 85.66, 81.18) circle (  2.13);

\path[fill=fillColor,fill opacity=0.20] ( 86.75, 83.97) circle (  2.13);

\path[fill=fillColor,fill opacity=0.20] (116.90, 84.35) circle (  2.13);

\path[fill=fillColor,fill opacity=0.20] ( 93.09, 87.25) circle (  2.13);

\path[fill=fillColor,fill opacity=0.20] ( 93.96, 96.86) circle (  2.13);

\path[fill=fillColor,fill opacity=0.20] (103.58, 98.13) circle (  2.13);

\path[fill=fillColor,fill opacity=0.20] (105.10, 90.79) circle (  2.13);

\path[fill=fillColor,fill opacity=0.20] (105.54, 87.63) circle (  2.13);

\path[fill=fillColor,fill opacity=0.20] (108.38, 86.87) circle (  2.13);

\path[fill=fillColor,fill opacity=0.20] (109.47, 84.22) circle (  2.13);

\path[fill=fillColor,fill opacity=0.20] ( 99.86, 85.74) circle (  2.13);

\path[fill=fillColor,fill opacity=0.20] ( 94.62, 90.54) circle (  2.13);

\path[fill=fillColor,fill opacity=0.20] ( 80.85, 82.20) circle (  2.13);

\path[fill=fillColor,fill opacity=0.20] ( 67.96, 41.23) circle (  2.13);

\path[fill=fillColor,fill opacity=0.20] ( 78.67, 61.21) circle (  2.13);

\path[fill=fillColor,fill opacity=0.20] ( 91.34, 78.40) circle (  2.13);

\path[fill=fillColor,fill opacity=0.20] ( 93.74, 80.17) circle (  2.13);

\path[fill=fillColor,fill opacity=0.20] ( 87.84, 83.71) circle (  2.13);

\path[fill=fillColor,fill opacity=0.20] ( 89.15, 88.90) circle (  2.13);

\path[fill=fillColor,fill opacity=0.20] ( 80.41, 82.07) circle (  2.13);

\path[fill=fillColor,fill opacity=0.20] ( 77.57, 78.66) circle (  2.13);

\path[fill=fillColor,fill opacity=0.20] ( 75.61, 84.85) circle (  2.13);

\path[fill=fillColor,fill opacity=0.20] ( 75.17, 91.30) circle (  2.13);

\path[fill=fillColor,fill opacity=0.20] ( 75.39, 90.54) circle (  2.13);

\path[fill=fillColor,fill opacity=0.20] ( 77.57, 80.93) circle (  2.13);

\path[fill=fillColor,fill opacity=0.20] ( 70.80, 72.71) circle (  2.13);

\path[fill=fillColor,fill opacity=0.20] ( 70.58, 75.12) circle (  2.13);

\path[fill=fillColor,fill opacity=0.20] ( 65.99, 68.67) circle (  2.13);

\path[fill=fillColor,fill opacity=0.20] ( 66.65, 62.72) circle (  2.13);

\path[fill=fillColor,fill opacity=0.20] ( 66.21, 64.24) circle (  2.13);

\path[fill=fillColor,fill opacity=0.20] ( 90.68, 65.50) circle (  2.13);

\path[fill=fillColor,fill opacity=0.20] ( 85.66, 75.62) circle (  2.13);

\path[fill=fillColor,fill opacity=0.20] ( 86.10, 79.67) circle (  2.13);

\path[fill=fillColor,fill opacity=0.20] ( 95.71, 74.23) circle (  2.13);

\path[fill=fillColor,fill opacity=0.20] ( 89.59, 78.15) circle (  2.13);

\path[fill=fillColor,fill opacity=0.20] ( 97.68, 87.25) circle (  2.13);

\path[fill=fillColor,fill opacity=0.20] ( 90.68, 88.65) circle (  2.13);

\path[fill=fillColor,fill opacity=0.20] (101.83, 93.07) circle (  2.13);

\path[fill=fillColor,fill opacity=0.20] (107.95, 93.96) circle (  2.13);

\path[fill=fillColor,fill opacity=0.20] (119.31, 87.76) circle (  2.13);

\path[fill=fillColor,fill opacity=0.20] (105.98, 87.51) circle (  2.13);

\path[fill=fillColor,fill opacity=0.20] ( 96.80, 85.99) circle (  2.13);

\path[fill=fillColor,fill opacity=0.20] ( 92.21, 76.51) circle (  2.13);

\path[fill=fillColor,fill opacity=0.20] ( 78.67, 69.30) circle (  2.13);

\path[fill=fillColor,fill opacity=0.20] ( 75.61, 66.64) circle (  2.13);

\path[fill=fillColor,fill opacity=0.20] ( 67.30, 51.22) circle (  2.13);

\path[fill=fillColor,fill opacity=0.20] ( 80.63, 60.83) circle (  2.13);

\path[fill=fillColor,fill opacity=0.20] ( 91.34, 73.85) circle (  2.13);

\path[fill=fillColor,fill opacity=0.20] ( 97.89, 91.30) circle (  2.13);

\path[fill=fillColor,fill opacity=0.20] ( 98.11, 96.61) circle (  2.13);

\path[fill=fillColor,fill opacity=0.20] ( 87.84, 89.40) circle (  2.13);

\path[fill=fillColor,fill opacity=0.20] ( 79.10, 90.04) circle (  2.13);

\path[fill=fillColor,fill opacity=0.20] ( 78.01, 88.01) circle (  2.13);

\path[fill=fillColor,fill opacity=0.20] ( 75.83, 82.70) circle (  2.13);

\path[fill=fillColor,fill opacity=0.20] ( 76.70, 80.68) circle (  2.13);

\path[fill=fillColor,fill opacity=0.20] ( 73.20, 78.02) circle (  2.13);

\path[fill=fillColor,fill opacity=0.20] ( 71.46, 76.89) circle (  2.13);

\path[fill=fillColor,fill opacity=0.20] ( 71.67, 79.16) circle (  2.13);

\path[fill=fillColor,fill opacity=0.20] ( 72.33, 79.54) circle (  2.13);

\path[fill=fillColor,fill opacity=0.20] ( 74.95, 77.26) circle (  2.13);

\path[fill=fillColor,fill opacity=0.20] ( 71.46, 75.37) circle (  2.13);

\path[fill=fillColor,fill opacity=0.20] ( 65.99, 69.42) circle (  2.13);

\path[fill=fillColor,fill opacity=0.20] ( 76.92, 75.62) circle (  2.13);

\path[fill=fillColor,fill opacity=0.20] ( 71.24, 83.21) circle (  2.13);

\path[fill=fillColor,fill opacity=0.20] ( 67.30, 69.80) circle (  2.13);

\path[fill=fillColor,fill opacity=0.20] ( 68.62, 60.32) circle (  2.13);

\path[fill=fillColor,fill opacity=0.20] ( 74.30, 63.86) circle (  2.13);

\path[fill=fillColor,fill opacity=0.20] ( 72.11, 65.38) circle (  2.13);

\path[fill=fillColor,fill opacity=0.20] ( 71.46, 61.33) circle (  2.13);

\path[fill=fillColor,fill opacity=0.20] ( 69.71, 60.83) circle (  2.13);

\path[fill=fillColor,fill opacity=0.20] ( 77.57, 67.40) circle (  2.13);

\path[fill=fillColor,fill opacity=0.20] ( 74.08, 71.32) circle (  2.13);

\path[fill=fillColor,fill opacity=0.20] ( 76.26, 68.54) circle (  2.13);

\path[fill=fillColor,fill opacity=0.20] ( 78.23, 60.45) circle (  2.13);

\path[fill=fillColor,fill opacity=0.20] ( 74.95, 58.80) circle (  2.13);

\path[fill=fillColor,fill opacity=0.20] ( 76.70, 64.24) circle (  2.13);

\path[fill=fillColor,fill opacity=0.20] ( 76.26, 69.05) circle (  2.13);

\path[fill=fillColor,fill opacity=0.20] ( 82.82, 72.59) circle (  2.13);

\path[fill=fillColor,fill opacity=0.20] ( 80.20, 74.99) circle (  2.13);

\path[fill=fillColor,fill opacity=0.20] ( 88.94, 71.07) circle (  2.13);

\path[fill=fillColor,fill opacity=0.20] ( 89.81, 72.46) circle (  2.13);

\path[fill=fillColor,fill opacity=0.20] ( 84.13, 79.03) circle (  2.13);

\path[fill=fillColor,fill opacity=0.20] ( 88.72, 80.17) circle (  2.13);

\path[fill=fillColor,fill opacity=0.20] ( 99.42, 83.46) circle (  2.13);

\path[fill=fillColor,fill opacity=0.20] (101.39, 89.02) circle (  2.13);

\path[fill=fillColor,fill opacity=0.20] (100.30, 91.55) circle (  2.13);

\path[fill=fillColor,fill opacity=0.20] ( 97.68, 92.06) circle (  2.13);

\path[fill=fillColor,fill opacity=0.20] ( 97.02, 86.12) circle (  2.13);

\path[fill=fillColor,fill opacity=0.20] ( 88.72, 76.63) circle (  2.13);

\path[fill=fillColor,fill opacity=0.20] ( 80.20, 77.39) circle (  2.13);

\path[fill=fillColor,fill opacity=0.20] ( 89.15, 73.09) circle (  2.13);

\path[fill=fillColor,fill opacity=0.20] ( 74.73, 66.14) circle (  2.13);

\path[fill=fillColor,fill opacity=0.20] ( 73.20, 65.50) circle (  2.13);

\path[fill=fillColor,fill opacity=0.20] ( 74.30, 53.62) circle (  2.13);

\path[fill=fillColor,fill opacity=0.20] ( 82.82, 68.29) circle (  2.13);

\path[fill=fillColor,fill opacity=0.20] ( 90.90, 83.21) circle (  2.13);

\path[fill=fillColor,fill opacity=0.20] ( 91.78, 86.62) circle (  2.13);

\path[fill=fillColor,fill opacity=0.20] ( 87.19, 85.48) circle (  2.13);

\path[fill=fillColor,fill opacity=0.20] ( 82.82, 90.16) circle (  2.13);

\path[fill=fillColor,fill opacity=0.20] ( 79.76, 88.14) circle (  2.13);

\path[fill=fillColor,fill opacity=0.20] ( 74.95, 84.98) circle (  2.13);

\path[fill=fillColor,fill opacity=0.20] ( 74.51, 84.85) circle (  2.13);

\path[fill=fillColor,fill opacity=0.20] ( 74.73, 81.44) circle (  2.13);

\path[fill=fillColor,fill opacity=0.20] ( 71.02, 78.78) circle (  2.13);

\path[fill=fillColor,fill opacity=0.20] ( 75.61, 79.54) circle (  2.13);

\path[fill=fillColor,fill opacity=0.20] ( 82.82, 77.77) circle (  2.13);

\path[fill=fillColor,fill opacity=0.20] ( 73.20, 76.13) circle (  2.13);

\path[fill=fillColor,fill opacity=0.20] ( 76.48, 77.52) circle (  2.13);

\path[fill=fillColor,fill opacity=0.20] ( 74.08, 82.07) circle (  2.13);

\path[fill=fillColor,fill opacity=0.20] ( 76.26, 88.14) circle (  2.13);

\path[fill=fillColor,fill opacity=0.20] ( 75.83, 83.97) circle (  2.13);

\path[fill=fillColor,fill opacity=0.20] ( 79.32, 78.78) circle (  2.13);

\path[fill=fillColor,fill opacity=0.20] ( 72.77, 82.20) circle (  2.13);

\path[fill=fillColor,fill opacity=0.20] ( 79.32, 82.58) circle (  2.13);

\path[fill=fillColor,fill opacity=0.20] ( 77.79, 76.89) circle (  2.13);

\path[fill=fillColor,fill opacity=0.20] ( 79.76, 76.89) circle (  2.13);

\path[fill=fillColor,fill opacity=0.20] ( 85.22, 77.39) circle (  2.13);

\path[fill=fillColor,fill opacity=0.20] ( 82.60, 76.00) circle (  2.13);

\path[fill=fillColor,fill opacity=0.20] ( 83.69, 74.48) circle (  2.13);

\path[fill=fillColor,fill opacity=0.20] ( 82.38, 69.55) circle (  2.13);

\path[fill=fillColor,fill opacity=0.20] ( 79.54, 66.26) circle (  2.13);

\path[fill=fillColor,fill opacity=0.20] ( 86.97, 69.80) circle (  2.13);

\path[fill=fillColor,fill opacity=0.20] ( 89.15, 75.24) circle (  2.13);

\path[fill=fillColor,fill opacity=0.20] ( 92.21, 77.01) circle (  2.13);

\path[fill=fillColor,fill opacity=0.20] ( 95.71, 76.63) circle (  2.13);

\path[fill=fillColor,fill opacity=0.20] ( 90.68, 87.13) circle (  2.13);

\path[fill=fillColor,fill opacity=0.20] ( 96.15,101.54) circle (  2.13);

\path[fill=fillColor,fill opacity=0.20] ( 90.68, 92.82) circle (  2.13);

\path[fill=fillColor,fill opacity=0.20] ( 90.68, 86.50) circle (  2.13);

\path[fill=fillColor,fill opacity=0.20] ( 95.27, 90.04) circle (  2.13);

\path[fill=fillColor,fill opacity=0.20] ( 87.19, 83.71) circle (  2.13);

\path[fill=fillColor,fill opacity=0.20] ( 82.16, 78.91) circle (  2.13);

\path[fill=fillColor,fill opacity=0.20] ( 81.51, 75.24) circle (  2.13);

\path[fill=fillColor,fill opacity=0.20] ( 75.61, 63.99) circle (  2.13);

\path[fill=fillColor,fill opacity=0.20] ( 67.09, 60.07) circle (  2.13);

\path[fill=fillColor,fill opacity=0.20] ( 83.25, 59.81) circle (  2.13);

\path[fill=fillColor,fill opacity=0.20] ( 61.19, 41.86) circle (  2.13);

\path[fill=fillColor,fill opacity=0.20] ( 76.92, 51.47) circle (  2.13);

\path[fill=fillColor,fill opacity=0.20] ( 79.32, 62.60) circle (  2.13);

\path[fill=fillColor,fill opacity=0.20] ( 85.00, 70.56) circle (  2.13);

\path[fill=fillColor,fill opacity=0.20] ( 90.68, 79.54) circle (  2.13);

\path[fill=fillColor,fill opacity=0.20] ( 86.75, 90.67) circle (  2.13);

\path[fill=fillColor,fill opacity=0.20] ( 85.44, 92.94) circle (  2.13);

\path[fill=fillColor,fill opacity=0.20] ( 80.41, 86.62) circle (  2.13);

\path[fill=fillColor,fill opacity=0.20] ( 76.48, 80.55) circle (  2.13);

\path[fill=fillColor,fill opacity=0.20] ( 78.45, 80.68) circle (  2.13);

\path[fill=fillColor,fill opacity=0.20] ( 76.04, 79.79) circle (  2.13);

\path[fill=fillColor,fill opacity=0.20] ( 72.55, 76.51) circle (  2.13);

\path[fill=fillColor,fill opacity=0.20] ( 75.17, 77.14) circle (  2.13);

\path[fill=fillColor,fill opacity=0.20] ( 77.36, 80.30) circle (  2.13);

\path[fill=fillColor,fill opacity=0.20] ( 73.86, 80.17) circle (  2.13);

\path[fill=fillColor,fill opacity=0.20] ( 76.92, 78.53) circle (  2.13);

\path[fill=fillColor,fill opacity=0.20] ( 78.23, 79.54) circle (  2.13);

\path[fill=fillColor,fill opacity=0.20] ( 71.67, 80.93) circle (  2.13);

\path[fill=fillColor,fill opacity=0.20] ( 74.51, 82.95) circle (  2.13);

\path[fill=fillColor,fill opacity=0.20] ( 78.01, 82.58) circle (  2.13);

\path[fill=fillColor,fill opacity=0.20] ( 77.79, 81.82) circle (  2.13);

\path[fill=fillColor,fill opacity=0.20] ( 78.01, 81.31) circle (  2.13);

\path[fill=fillColor,fill opacity=0.20] ( 74.73, 77.52) circle (  2.13);

\path[fill=fillColor,fill opacity=0.20] ( 74.73, 75.62) circle (  2.13);

\path[fill=fillColor,fill opacity=0.20] ( 77.36, 76.63) circle (  2.13);

\path[fill=fillColor,fill opacity=0.20] ( 80.85, 77.39) circle (  2.13);

\path[fill=fillColor,fill opacity=0.20] ( 83.25, 78.78) circle (  2.13);

\path[fill=fillColor,fill opacity=0.20] ( 88.94, 79.54) circle (  2.13);

\path[fill=fillColor,fill opacity=0.20] ( 88.06, 82.20) circle (  2.13);

\path[fill=fillColor,fill opacity=0.20] ( 98.77, 84.35) circle (  2.13);

\path[fill=fillColor,fill opacity=0.20] ( 96.36, 79.67) circle (  2.13);

\path[fill=fillColor,fill opacity=0.20] ( 98.33, 75.49) circle (  2.13);

\path[fill=fillColor,fill opacity=0.20] ( 81.07, 77.90) circle (  2.13);

\path[fill=fillColor,fill opacity=0.20] ( 85.00, 77.26) circle (  2.13);

\path[fill=fillColor,fill opacity=0.20] ( 76.70, 71.20) circle (  2.13);

\path[fill=fillColor,fill opacity=0.20] ( 70.58, 64.62) circle (  2.13);

\path[fill=fillColor,fill opacity=0.20] ( 54.63, 46.28) circle (  2.13);

\path[fill=fillColor,fill opacity=0.20] ( 75.17, 54.12) circle (  2.13);

\path[fill=fillColor,fill opacity=0.20] ( 81.73, 69.30) circle (  2.13);

\path[fill=fillColor,fill opacity=0.20] ( 84.57, 79.92) circle (  2.13);

\path[fill=fillColor,fill opacity=0.20] ( 90.47, 76.25) circle (  2.13);

\path[fill=fillColor,fill opacity=0.20] ( 89.37, 73.60) circle (  2.13);

\path[fill=fillColor,fill opacity=0.20] ( 87.19, 78.15) circle (  2.13);

\path[fill=fillColor,fill opacity=0.20] ( 89.37, 80.93) circle (  2.13);

\path[fill=fillColor,fill opacity=0.20] ( 88.94, 80.30) circle (  2.13);

\path[fill=fillColor,fill opacity=0.20] ( 81.73, 82.70) circle (  2.13);

\path[fill=fillColor,fill opacity=0.20] ( 79.76, 86.12) circle (  2.13);

\path[fill=fillColor,fill opacity=0.20] ( 82.60, 90.16) circle (  2.13);

\path[fill=fillColor,fill opacity=0.20] ( 82.38, 88.01) circle (  2.13);

\path[fill=fillColor,fill opacity=0.20] ( 77.79, 82.07) circle (  2.13);

\path[fill=fillColor,fill opacity=0.20] ( 79.32, 81.44) circle (  2.13);

\path[fill=fillColor,fill opacity=0.20] ( 78.67, 86.24) circle (  2.13);

\path[fill=fillColor,fill opacity=0.20] ( 78.23, 90.42) circle (  2.13);

\path[fill=fillColor,fill opacity=0.20] ( 79.32, 92.31) circle (  2.13);

\path[fill=fillColor,fill opacity=0.20] ( 83.47, 87.76) circle (  2.13);

\path[fill=fillColor,fill opacity=0.20] ( 89.37, 81.18) circle (  2.13);

\path[fill=fillColor,fill opacity=0.20] ( 83.04, 80.17) circle (  2.13);

\path[fill=fillColor,fill opacity=0.20] ( 87.62, 80.93) circle (  2.13);

\path[fill=fillColor,fill opacity=0.20] ( 88.50, 79.16) circle (  2.13);

\path[fill=fillColor,fill opacity=0.20] ( 92.87, 79.79) circle (  2.13);

\path[fill=fillColor,fill opacity=0.20] ( 89.81, 76.00) circle (  2.13);

\path[fill=fillColor,fill opacity=0.20] ( 83.69, 67.40) circle (  2.13);

\path[fill=fillColor,fill opacity=0.20] ( 71.67, 62.34) circle (  2.13);

\path[fill=fillColor,fill opacity=0.20] ( 67.52, 56.78) circle (  2.13);

\path[fill=fillColor,fill opacity=0.20] ( 52.01, 44.39) circle (  2.13);

\path[fill=fillColor,fill opacity=0.20] ( 53.54, 42.49) circle (  2.13);

\path[fill=fillColor,fill opacity=0.20] ( 67.52, 47.93) circle (  2.13);

\path[fill=fillColor,fill opacity=0.20] ( 81.07, 53.11) circle (  2.13);

\path[fill=fillColor,fill opacity=0.20] ( 88.28, 59.56) circle (  2.13);

\path[fill=fillColor,fill opacity=0.20] ( 79.32, 63.36) circle (  2.13);

\path[fill=fillColor,fill opacity=0.20] ( 93.96, 71.32) circle (  2.13);

\path[fill=fillColor,fill opacity=0.20] ( 92.65, 84.09) circle (  2.13);

\path[fill=fillColor,fill opacity=0.20] ( 97.89, 80.68) circle (  2.13);

\path[fill=fillColor,fill opacity=0.20] ( 90.68, 73.34) circle (  2.13);

\path[fill=fillColor,fill opacity=0.20] ( 93.96, 83.46) circle (  2.13);

\path[fill=fillColor,fill opacity=0.20] ( 85.22, 91.93) circle (  2.13);

\path[fill=fillColor,fill opacity=0.20] ( 85.44, 78.91) circle (  2.13);

\path[fill=fillColor,fill opacity=0.20] ( 82.82, 72.08) circle (  2.13);

\path[fill=fillColor,fill opacity=0.20] ( 85.22, 75.24) circle (  2.13);

\path[fill=fillColor,fill opacity=0.20] ( 85.88, 76.89) circle (  2.13);

\path[fill=fillColor,fill opacity=0.20] ( 87.62, 76.25) circle (  2.13);

\path[fill=fillColor,fill opacity=0.20] ( 83.91, 71.45) circle (  2.13);

\path[fill=fillColor,fill opacity=0.20] ( 89.37, 66.64) circle (  2.13);

\path[fill=fillColor,fill opacity=0.20] ( 86.97, 63.61) circle (  2.13);

\path[fill=fillColor,fill opacity=0.20] ( 75.17, 56.91) circle (  2.13);

\path[fill=fillColor,fill opacity=0.20] ( 74.30, 52.61) circle (  2.13);

\path[fill=fillColor,fill opacity=0.20] ( 76.26, 52.48) circle (  2.13);

\path[fill=fillColor,fill opacity=0.20] ( 52.01, 38.70) circle (  2.13);

\path[fill=fillColor,fill opacity=0.20] ( 57.69, 43.38) circle (  2.13);

\path[fill=fillColor,fill opacity=0.20] ( 66.21, 52.48) circle (  2.13);

\path[fill=fillColor,fill opacity=0.20] ( 80.85, 52.35) circle (  2.13);

\path[fill=fillColor,fill opacity=0.20] ( 79.76, 50.08) circle (  2.13);

\path[fill=fillColor,fill opacity=0.20] ( 77.14, 52.73) circle (  2.13);

\path[fill=fillColor,fill opacity=0.20] ( 72.55, 52.99) circle (  2.13);

\path[fill=fillColor,fill opacity=0.20] ( 69.27, 51.85) circle (  2.13);

\path[fill=fillColor,fill opacity=0.20] ( 71.67, 49.95) circle (  2.13);

\path[fill=fillColor,fill opacity=0.20] ( 71.24, 47.55) circle (  2.13);

\path[fill=fillColor,fill opacity=0.20] ( 77.36, 46.03) circle (  2.13);

\path[fill=fillColor,fill opacity=0.20] ( 75.17, 47.30) circle (  2.13);

\path[fill=fillColor,fill opacity=0.20] ( 79.10, 63.61) circle (  2.13);

\path[fill=fillColor,fill opacity=0.20] ( 74.95, 77.14) circle (  2.13);

\path[fill=fillColor,fill opacity=0.20] ( 73.20, 80.43) circle (  2.13);

\path[fill=fillColor,fill opacity=0.20] ( 91.99, 69.68) circle (  2.13);

\path[fill=fillColor,fill opacity=0.20] (124.77, 61.58) circle (  2.13);

\path[fill=fillColor,fill opacity=0.20] ( 76.26, 39.33) circle (  2.13);

\path[fill=fillColor,fill opacity=0.20] (103.58, 58.93) circle (  2.13);

\path[fill=fillColor,fill opacity=0.20] (111.66, 79.16) circle (  2.13);

\path[fill=fillColor,fill opacity=0.20] ( 97.68, 97.75) circle (  2.13);

\path[fill=fillColor,fill opacity=0.20] ( 96.58,106.10) circle (  2.13);

\path[fill=fillColor,fill opacity=0.20] ( 89.15, 98.38) circle (  2.13);

\path[fill=fillColor,fill opacity=0.20] ( 94.18, 95.98) circle (  2.13);

\path[fill=fillColor,fill opacity=0.20] ( 88.72,100.03) circle (  2.13);

\path[fill=fillColor,fill opacity=0.20] ( 86.75, 88.01) circle (  2.13);

\path[fill=fillColor,fill opacity=0.20] (102.48, 63.73) circle (  2.13);

\path[fill=fillColor,fill opacity=0.20] (130.89,101.42) circle (  2.13);

\path[fill=fillColor,fill opacity=0.20] (112.10, 99.27) circle (  2.13);

\path[fill=fillColor,fill opacity=0.20] (109.91,108.24) circle (  2.13);

\path[fill=fillColor,fill opacity=0.20] (118.21,110.77) circle (  2.13);

\path[fill=fillColor,fill opacity=0.20] (108.82, 99.01) circle (  2.13);

\path[fill=fillColor,fill opacity=0.20] (108.38,100.03) circle (  2.13);

\path[fill=fillColor,fill opacity=0.20] (109.26,107.61) circle (  2.13);

\path[fill=fillColor,fill opacity=0.20] ( 90.25, 96.99) circle (  2.13);

\path[fill=fillColor,fill opacity=0.20] ( 62.28, 73.85) circle (  2.13);

\path[fill=fillColor,fill opacity=0.20] ( 61.19, 46.66) circle (  2.13);

\path[fill=fillColor,fill opacity=0.20] (128.48, 93.20) circle (  2.13);

\path[fill=fillColor,fill opacity=0.20] (106.20,105.72) circle (  2.13);

\path[fill=fillColor,fill opacity=0.20] (101.83, 91.93) circle (  2.13);

\path[fill=fillColor,fill opacity=0.20] (124.99,102.05) circle (  2.13);

\path[fill=fillColor,fill opacity=0.20] (122.80,104.45) circle (  2.13);

\path[fill=fillColor,fill opacity=0.20] (114.28, 99.27) circle (  2.13);

\path[fill=fillColor,fill opacity=0.20] (119.96, 99.39) circle (  2.13);

\path[fill=fillColor,fill opacity=0.20] (130.67,103.31) circle (  2.13);

\path[fill=fillColor,fill opacity=0.20] (114.94, 99.01) circle (  2.13);

\path[fill=fillColor,fill opacity=0.20] ( 75.39, 83.33) circle (  2.13);

\path[fill=fillColor,fill opacity=0.20] ( 65.12, 59.69) circle (  2.13);

\path[fill=fillColor,fill opacity=0.20] (124.99,106.98) circle (  2.13);

\path[fill=fillColor,fill opacity=0.20] (106.20,102.81) circle (  2.13);

\path[fill=fillColor,fill opacity=0.20] (116.90, 97.37) circle (  2.13);

\path[fill=fillColor,fill opacity=0.20] (146.40,104.07) circle (  2.13);

\path[fill=fillColor,fill opacity=0.20] (136.13,105.84) circle (  2.13);

\path[fill=fillColor,fill opacity=0.20] (113.63,111.41) circle (  2.13);

\path[fill=fillColor,fill opacity=0.20] (122.15,109.89) circle (  2.13);

\path[fill=fillColor,fill opacity=0.20] (139.41,101.16) circle (  2.13);

\path[fill=fillColor,fill opacity=0.20] (125.43,102.30) circle (  2.13);

\path[fill=fillColor,fill opacity=0.20] ( 85.00, 92.57) circle (  2.13);

\path[fill=fillColor,fill opacity=0.20] ( 59.22, 61.21) circle (  2.13);

\path[fill=fillColor,fill opacity=0.20] (112.53,102.30) circle (  2.13);

\path[fill=fillColor,fill opacity=0.20] (120.40,113.43) circle (  2.13);

\path[fill=fillColor,fill opacity=0.20] (123.90,110.39) circle (  2.13);

\path[fill=fillColor,fill opacity=0.20] (125.21,104.45) circle (  2.13);

\path[fill=fillColor,fill opacity=0.20] (126.08,104.83) circle (  2.13);

\path[fill=fillColor,fill opacity=0.20] (135.04,105.59) circle (  2.13);

\path[fill=fillColor,fill opacity=0.20] (119.31,101.80) circle (  2.13);

\path[fill=fillColor,fill opacity=0.20] ( 71.02, 55.01) circle (  2.13);

\path[fill=fillColor,fill opacity=0.20] ( 79.98, 54.50) circle (  2.13);

\path[fill=fillColor,fill opacity=0.20] ( 92.21, 48.05) circle (  2.13);

\path[fill=fillColor,fill opacity=0.20] ( 93.52, 52.48) circle (  2.13);

\path[fill=fillColor,fill opacity=0.20] ( 94.18, 49.19) circle (  2.13);

\path[fill=fillColor,fill opacity=0.20] ( 63.37, 62.98) circle (  2.13);

\path[fill=fillColor,fill opacity=0.20] ( 93.09, 91.68) circle (  2.13);

\path[fill=fillColor,fill opacity=0.20] (104.89,103.82) circle (  2.13);

\path[fill=fillColor,fill opacity=0.20] (114.06,110.02) circle (  2.13);

\path[fill=fillColor,fill opacity=0.20] (118.87,105.59) circle (  2.13);

\path[fill=fillColor,fill opacity=0.20] (126.95,105.21) circle (  2.13);

\path[fill=fillColor,fill opacity=0.20] (134.38,109.64) circle (  2.13);

\path[fill=fillColor,fill opacity=0.20] (138.32,110.14) circle (  2.13);

\path[fill=fillColor,fill opacity=0.20] (144.22,109.13) circle (  2.13);

\path[fill=fillColor,fill opacity=0.20] (122.37,101.04) circle (  2.13);

\path[fill=fillColor,fill opacity=0.20] ( 84.57, 48.94) circle (  2.13);

\path[fill=fillColor,fill opacity=0.20] (104.23, 72.71) circle (  2.13);

\path[fill=fillColor,fill opacity=0.20] (111.44, 84.73) circle (  2.13);

\path[fill=fillColor,fill opacity=0.20] ( 97.68, 91.68) circle (  2.13);

\path[fill=fillColor,fill opacity=0.20] ( 84.35, 86.87) circle (  2.13);

\path[fill=fillColor,fill opacity=0.20] ( 80.85, 74.61) circle (  2.13);

\path[fill=fillColor,fill opacity=0.20] ( 78.45, 55.39) circle (  2.13);

\path[fill=fillColor,fill opacity=0.20] ( 72.11, 40.21) circle (  2.13);

\path[fill=fillColor,fill opacity=0.20] ( 87.41, 79.92) circle (  2.13);

\path[fill=fillColor,fill opacity=0.20] ( 92.21, 95.98) circle (  2.13);

\path[fill=fillColor,fill opacity=0.20] ( 97.02, 96.49) circle (  2.13);

\path[fill=fillColor,fill opacity=0.20] (106.42,106.73) circle (  2.13);

\path[fill=fillColor,fill opacity=0.20] (126.08,109.13) circle (  2.13);

\path[fill=fillColor,fill opacity=0.20] (135.04,103.82) circle (  2.13);

\path[fill=fillColor,fill opacity=0.20] (142.91,107.99) circle (  2.13);

\path[fill=fillColor,fill opacity=0.20] (104.67,104.20) circle (  2.13);

\path[fill=fillColor,fill opacity=0.20] (112.53,103.95) circle (  2.13);

\path[fill=fillColor,fill opacity=0.20] (111.44, 93.70) circle (  2.13);

\path[fill=fillColor,fill opacity=0.20] (106.42, 90.92) circle (  2.13);

\path[fill=fillColor,fill opacity=0.20] ( 94.62, 89.15) circle (  2.13);

\path[fill=fillColor,fill opacity=0.20] ( 89.15, 83.08) circle (  2.13);

\path[fill=fillColor,fill opacity=0.20] ( 78.01, 77.01) circle (  2.13);

\path[fill=fillColor,fill opacity=0.20] ( 82.82, 66.90) circle (  2.13);

\path[fill=fillColor,fill opacity=0.20] ( 80.85, 55.39) circle (  2.13);

\path[fill=fillColor,fill opacity=0.20] ( 45.24, 40.21) circle (  2.13);

\path[fill=fillColor,fill opacity=0.20] (109.26,102.55) circle (  2.13);

\path[fill=fillColor,fill opacity=0.20] (106.42,113.30) circle (  2.13);

\path[fill=fillColor,fill opacity=0.20] (107.95,107.23) circle (  2.13);

\path[fill=fillColor,fill opacity=0.20] (114.28,103.69) circle (  2.13);

\path[fill=fillColor,fill opacity=0.20] (135.69,111.15) circle (  2.13);

\path[fill=fillColor,fill opacity=0.20] (135.04,105.21) circle (  2.13);

\path[fill=fillColor,fill opacity=0.20] (119.96,105.72) circle (  2.13);

\path[fill=fillColor,fill opacity=0.20] ( 88.94, 99.39) circle (  2.13);

\path[fill=fillColor,fill opacity=0.20] ( 50.70, 37.94) circle (  2.13);

\path[fill=fillColor,fill opacity=0.20] ( 85.00, 73.72) circle (  2.13);

\path[fill=fillColor,fill opacity=0.20] (109.47,112.67) circle (  2.13);

\path[fill=fillColor,fill opacity=0.20] (107.29,104.58) circle (  2.13);

\path[fill=fillColor,fill opacity=0.20] (105.10, 97.37) circle (  2.13);

\path[fill=fillColor,fill opacity=0.20] (104.23,102.43) circle (  2.13);

\path[fill=fillColor,fill opacity=0.20] (101.17, 97.62) circle (  2.13);

\path[fill=fillColor,fill opacity=0.20] ( 87.19, 91.55) circle (  2.13);

\path[fill=fillColor,fill opacity=0.20] ( 86.10, 94.34) circle (  2.13);

\path[fill=fillColor,fill opacity=0.20] ( 92.87, 91.55) circle (  2.13);

\path[fill=fillColor,fill opacity=0.20] ( 89.81, 77.77) circle (  2.13);

\path[fill=fillColor,fill opacity=0.20] ( 73.42, 68.79) circle (  2.13);

\path[fill=fillColor,fill opacity=0.20] ( 88.28, 97.50) circle (  2.13);

\path[fill=fillColor,fill opacity=0.20] (105.32, 98.63) circle (  2.13);

\path[fill=fillColor,fill opacity=0.20] (114.28,111.41) circle (  2.13);

\path[fill=fillColor,fill opacity=0.20] (113.84,106.98) circle (  2.13);

\path[fill=fillColor,fill opacity=0.20] (126.08, 98.63) circle (  2.13);

\path[fill=fillColor,fill opacity=0.20] (119.31,102.68) circle (  2.13);

\path[fill=fillColor,fill opacity=0.20] (119.96,105.46) circle (  2.13);

\path[fill=fillColor,fill opacity=0.20] (137.01,103.06) circle (  2.13);

\path[fill=fillColor,fill opacity=0.20] (131.32,101.29) circle (  2.13);

\path[fill=fillColor,fill opacity=0.20] ( 94.40,100.78) circle (  2.13);

\path[fill=fillColor,fill opacity=0.20] ( 84.13, 47.68) circle (  2.13);

\path[fill=fillColor,fill opacity=0.20] ( 95.49,102.18) circle (  2.13);

\path[fill=fillColor,fill opacity=0.20] (138.97,106.22) circle (  2.13);

\path[fill=fillColor,fill opacity=0.20] (104.45,109.13) circle (  2.13);

\path[fill=fillColor,fill opacity=0.20] (110.35,115.07) circle (  2.13);

\path[fill=fillColor,fill opacity=0.20] ( 98.77,102.43) circle (  2.13);

\path[fill=fillColor,fill opacity=0.20] ( 91.34, 96.49) circle (  2.13);

\path[fill=fillColor,fill opacity=0.20] ( 93.52, 97.75) circle (  2.13);

\path[fill=fillColor,fill opacity=0.20] ( 87.62, 90.79) circle (  2.13);

\path[fill=fillColor,fill opacity=0.20] ( 69.27, 57.41) circle (  2.13);

\path[fill=fillColor,fill opacity=0.20] ( 89.15, 91.17) circle (  2.13);

\path[fill=fillColor,fill opacity=0.20] ( 95.05, 86.50) circle (  2.13);

\path[fill=fillColor,fill opacity=0.20] (127.83, 92.06) circle (  2.13);

\path[fill=fillColor,fill opacity=0.20] (118.21,101.80) circle (  2.13);

\path[fill=fillColor,fill opacity=0.20] (118.43,102.43) circle (  2.13);

\path[fill=fillColor,fill opacity=0.20] (137.44,104.45) circle (  2.13);

\path[fill=fillColor,fill opacity=0.20] (138.54,102.68) circle (  2.13);

\path[fill=fillColor,fill opacity=0.20] (119.31,102.18) circle (  2.13);

\path[fill=fillColor,fill opacity=0.20] ( 72.77, 51.97) circle (  2.13);

\path[fill=fillColor,fill opacity=0.20] ( 98.55, 98.63) circle (  2.13);

\path[fill=fillColor,fill opacity=0.20] (124.77,115.96) circle (  2.13);

\path[fill=fillColor,fill opacity=0.20] ( 96.58,105.34) circle (  2.13);

\path[fill=fillColor,fill opacity=0.20] ( 83.25, 98.51) circle (  2.13);

\path[fill=fillColor,fill opacity=0.20] ( 80.41, 94.21) circle (  2.13);

\path[fill=fillColor,fill opacity=0.20] ( 76.70, 81.94) circle (  2.13);

\path[fill=fillColor,fill opacity=0.20] ( 74.73, 54.63) circle (  2.13);

\path[fill=fillColor,fill opacity=0.20] ( 81.07, 84.60) circle (  2.13);

\path[fill=fillColor,fill opacity=0.20] ( 89.37, 83.71) circle (  2.13);

\path[fill=fillColor,fill opacity=0.20] (104.67, 88.77) circle (  2.13);

\path[fill=fillColor,fill opacity=0.20] (111.00, 97.62) circle (  2.13);

\path[fill=fillColor,fill opacity=0.20] (108.38,101.16) circle (  2.13);

\path[fill=fillColor,fill opacity=0.20] (123.24,107.36) circle (  2.13);

\path[fill=fillColor,fill opacity=0.20] (147.28,103.57) circle (  2.13);

\path[fill=fillColor,fill opacity=0.20] (138.97,108.12) circle (  2.13);

\path[fill=fillColor,fill opacity=0.20] ( 93.74,111.41) circle (  2.13);

\path[fill=fillColor,fill opacity=0.20] ( 82.60, 83.21) circle (  2.13);

\path[fill=fillColor,fill opacity=0.20] (122.80,105.97) circle (  2.13);

\path[fill=fillColor,fill opacity=0.20] (131.76,115.20) circle (  2.13);

\path[fill=fillColor,fill opacity=0.20] ( 93.96,108.75) circle (  2.13);

\path[fill=fillColor,fill opacity=0.20] ( 80.63,101.80) circle (  2.13);

\path[fill=fillColor,fill opacity=0.20] ( 75.17, 94.59) circle (  2.13);

\path[fill=fillColor,fill opacity=0.20] ( 72.99, 87.63) circle (  2.13);

\path[fill=fillColor,fill opacity=0.20] ( 62.93, 53.49) circle (  2.13);

\path[fill=fillColor,fill opacity=0.20] ( 88.50, 83.71) circle (  2.13);

\path[fill=fillColor,fill opacity=0.20] ( 89.15, 79.92) circle (  2.13);

\path[fill=fillColor,fill opacity=0.20] ( 95.05, 83.71) circle (  2.13);

\path[fill=fillColor,fill opacity=0.20] (108.60, 97.75) circle (  2.13);

\path[fill=fillColor,fill opacity=0.20] (100.08,101.80) circle (  2.13);

\path[fill=fillColor,fill opacity=0.20] (103.58,101.54) circle (  2.13);

\path[fill=fillColor,fill opacity=0.20] (121.27,106.98) circle (  2.13);

\path[fill=fillColor,fill opacity=0.20] (123.90,114.19) circle (  2.13);

\path[fill=fillColor,fill opacity=0.20] (125.64,100.91) circle (  2.13);

\path[fill=fillColor,fill opacity=0.20] ( 99.42, 97.75) circle (  2.13);

\path[fill=fillColor,fill opacity=0.20] ( 74.30, 67.02) circle (  2.13);

\path[fill=fillColor,fill opacity=0.20] (107.73,100.15) circle (  2.13);

\path[fill=fillColor,fill opacity=0.20] (135.91,108.37) circle (  2.13);

\path[fill=fillColor,fill opacity=0.20] (123.02,105.97) circle (  2.13);

\path[fill=fillColor,fill opacity=0.20] (102.92,106.85) circle (  2.13);

\path[fill=fillColor,fill opacity=0.20] (100.30,107.49) circle (  2.13);

\path[fill=fillColor,fill opacity=0.20] ( 88.50,100.15) circle (  2.13);

\path[fill=fillColor,fill opacity=0.20] ( 89.37, 95.09) circle (  2.13);

\path[fill=fillColor,fill opacity=0.20] ( 80.41, 94.46) circle (  2.13);

\path[fill=fillColor,fill opacity=0.20] ( 61.40, 70.56) circle (  2.13);

\path[fill=fillColor,fill opacity=0.20] ( 63.15, 55.01) circle (  2.13);

\path[fill=fillColor,fill opacity=0.20] ( 78.88, 80.30) circle (  2.13);

\path[fill=fillColor,fill opacity=0.20] ( 88.72, 83.71) circle (  2.13);

\path[fill=fillColor,fill opacity=0.20] ( 99.64, 89.91) circle (  2.13);

\path[fill=fillColor,fill opacity=0.20] (112.75, 92.44) circle (  2.13);

\path[fill=fillColor,fill opacity=0.20] (118.65, 98.38) circle (  2.13);

\path[fill=fillColor,fill opacity=0.20] ( 99.42,103.95) circle (  2.13);

\path[fill=fillColor,fill opacity=0.20] (102.92,103.69) circle (  2.13);

\path[fill=fillColor,fill opacity=0.20] (120.84,103.69) circle (  2.13);

\path[fill=fillColor,fill opacity=0.20] (114.06,102.43) circle (  2.13);

\path[fill=fillColor,fill opacity=0.20] (107.73, 98.76) circle (  2.13);

\path[fill=fillColor,fill opacity=0.20] ( 97.24, 95.73) circle (  2.13);

\path[fill=fillColor,fill opacity=0.20] ( 81.73, 85.10) circle (  2.13);

\path[fill=fillColor,fill opacity=0.20] (108.82,101.80) circle (  2.13);

\path[fill=fillColor,fill opacity=0.20] (116.90,105.21) circle (  2.13);

\path[fill=fillColor,fill opacity=0.20] (114.50,110.77) circle (  2.13);

\path[fill=fillColor,fill opacity=0.20] (109.26,105.84) circle (  2.13);

\path[fill=fillColor,fill opacity=0.20] (108.82, 97.37) circle (  2.13);

\path[fill=fillColor,fill opacity=0.20] (102.92, 96.99) circle (  2.13);

\path[fill=fillColor,fill opacity=0.20] ( 88.50, 90.67) circle (  2.13);

\path[fill=fillColor,fill opacity=0.20] ( 73.42, 74.74) circle (  2.13);

\path[fill=fillColor,fill opacity=0.20] ( 63.37, 54.50) circle (  2.13);

\path[fill=fillColor,fill opacity=0.20] ( 82.38, 82.07) circle (  2.13);

\path[fill=fillColor,fill opacity=0.20] ( 94.40, 79.29) circle (  2.13);

\path[fill=fillColor,fill opacity=0.20] ( 95.27, 88.65) circle (  2.13);

\path[fill=fillColor,fill opacity=0.20] (102.70,102.30) circle (  2.13);

\path[fill=fillColor,fill opacity=0.20] (116.47, 99.52) circle (  2.13);

\path[fill=fillColor,fill opacity=0.20] (116.90,105.34) circle (  2.13);

\path[fill=fillColor,fill opacity=0.20] (109.04,113.81) circle (  2.13);

\path[fill=fillColor,fill opacity=0.20] (112.97,110.65) circle (  2.13);

\path[fill=fillColor,fill opacity=0.20] (105.54,103.44) circle (  2.13);

\path[fill=fillColor,fill opacity=0.20] ( 94.62, 95.09) circle (  2.13);

\path[fill=fillColor,fill opacity=0.20] ( 81.29, 89.66) circle (  2.13);

\path[fill=fillColor,fill opacity=0.20] ( 66.21, 61.58) circle (  2.13);

\path[fill=fillColor,fill opacity=0.20] ( 91.34, 79.92) circle (  2.13);

\path[fill=fillColor,fill opacity=0.20] (113.41,102.55) circle (  2.13);

\path[fill=fillColor,fill opacity=0.20] (111.22,115.96) circle (  2.13);

\path[fill=fillColor,fill opacity=0.20] (111.66,105.21) circle (  2.13);

\path[fill=fillColor,fill opacity=0.20] (108.82, 99.77) circle (  2.13);

\path[fill=fillColor,fill opacity=0.20] (104.01,103.82) circle (  2.13);

\path[fill=fillColor,fill opacity=0.20] ( 94.84, 94.46) circle (  2.13);

\path[fill=fillColor,fill opacity=0.20] ( 80.85, 80.93) circle (  2.13);

\path[fill=fillColor,fill opacity=0.20] ( 69.05, 69.93) circle (  2.13);

\path[fill=fillColor,fill opacity=0.20] ( 50.48, 45.78) circle (  2.13);

\path[fill=fillColor,fill opacity=0.20] ( 52.01, 38.19) circle (  2.13);

\path[fill=fillColor,fill opacity=0.20] ( 70.80, 56.27) circle (  2.13);

\path[fill=fillColor,fill opacity=0.20] ( 78.23, 73.98) circle (  2.13);

\path[fill=fillColor,fill opacity=0.20] ( 87.62, 76.51) circle (  2.13);

\path[fill=fillColor,fill opacity=0.20] ( 99.42, 82.45) circle (  2.13);

\path[fill=fillColor,fill opacity=0.20] (108.38, 91.68) circle (  2.13);

\path[fill=fillColor,fill opacity=0.20] (114.28,104.45) circle (  2.13);

\path[fill=fillColor,fill opacity=0.20] (110.13,111.91) circle (  2.13);

\path[fill=fillColor,fill opacity=0.20] (104.45,105.34) circle (  2.13);

\path[fill=fillColor,fill opacity=0.20] ( 77.57,100.53) circle (  2.13);

\path[fill=fillColor,fill opacity=0.20] ( 77.14, 89.78) circle (  2.13);

\path[fill=fillColor,fill opacity=0.20] ( 83.47, 79.16) circle (  2.13);

\path[fill=fillColor,fill opacity=0.20] ( 80.85, 60.32) circle (  2.13);

\path[fill=fillColor,fill opacity=0.20] (124.77, 98.76) circle (  2.13);

\path[fill=fillColor,fill opacity=0.20] (112.10,100.66) circle (  2.13);

\path[fill=fillColor,fill opacity=0.20] (107.51, 98.38) circle (  2.13);

\path[fill=fillColor,fill opacity=0.20] ( 97.02,105.84) circle (  2.13);

\path[fill=fillColor,fill opacity=0.20] ( 96.36, 98.26) circle (  2.13);

\path[fill=fillColor,fill opacity=0.20] ( 88.72, 84.85) circle (  2.13);

\path[fill=fillColor,fill opacity=0.20] ( 83.47, 81.56) circle (  2.13);

\path[fill=fillColor,fill opacity=0.20] ( 79.98, 72.46) circle (  2.13);

\path[fill=fillColor,fill opacity=0.20] ( 57.91, 48.94) circle (  2.13);

\path[fill=fillColor,fill opacity=0.20] ( 77.36, 47.42) circle (  2.13);

\path[fill=fillColor,fill opacity=0.20] ( 74.73, 65.00) circle (  2.13);

\path[fill=fillColor,fill opacity=0.20] ( 93.96, 67.78) circle (  2.13);

\path[fill=fillColor,fill opacity=0.20] ( 95.93, 69.80) circle (  2.13);

\path[fill=fillColor,fill opacity=0.20] (103.14, 84.47) circle (  2.13);

\path[fill=fillColor,fill opacity=0.20] (110.79, 99.14) circle (  2.13);

\path[fill=fillColor,fill opacity=0.20] (117.78,107.49) circle (  2.13);

\path[fill=fillColor,fill opacity=0.20] (102.05, 88.77) circle (  2.13);

\path[fill=fillColor,fill opacity=0.20] (102.26,103.95) circle (  2.13);

\path[fill=fillColor,fill opacity=0.20] (134.82, 93.32) circle (  2.13);

\path[fill=fillColor,fill opacity=0.20] (103.79,101.29) circle (  2.13);

\path[fill=fillColor,fill opacity=0.20] ( 96.58, 97.62) circle (  2.13);

\path[fill=fillColor,fill opacity=0.20] ( 92.65, 93.45) circle (  2.13);

\path[fill=fillColor,fill opacity=0.20] ( 95.05, 89.78) circle (  2.13);

\path[fill=fillColor,fill opacity=0.20] ( 84.35, 85.48) circle (  2.13);

\path[fill=fillColor,fill opacity=0.20] ( 75.39, 79.41) circle (  2.13);

\path[fill=fillColor,fill opacity=0.20] ( 63.37, 54.12) circle (  2.13);

\path[fill=fillColor,fill opacity=0.20] ( 53.98, 40.59) circle (  2.13);

\path[fill=fillColor,fill opacity=0.20] ( 64.90, 45.91) circle (  2.13);

\path[fill=fillColor,fill opacity=0.20] ( 78.88, 57.92) circle (  2.13);

\path[fill=fillColor,fill opacity=0.20] ( 95.05, 68.29) circle (  2.13);

\path[fill=fillColor,fill opacity=0.20] ( 95.93, 66.39) circle (  2.13);

\path[fill=fillColor,fill opacity=0.20] (106.42, 70.44) circle (  2.13);

\path[fill=fillColor,fill opacity=0.20] (115.37, 85.86) circle (  2.13);

\path[fill=fillColor,fill opacity=0.20] (127.83, 97.62) circle (  2.13);

\path[fill=fillColor,fill opacity=0.20] ( 83.25,106.10) circle (  2.13);

\path[fill=fillColor,fill opacity=0.20] ( 87.84, 78.91) circle (  2.13);

\path[fill=fillColor,fill opacity=0.20] (102.05, 94.21) circle (  2.13);

\path[fill=fillColor,fill opacity=0.20] (107.51, 99.65) circle (  2.13);

\path[fill=fillColor,fill opacity=0.20] (100.73, 95.60) circle (  2.13);

\path[fill=fillColor,fill opacity=0.20] ( 97.46, 92.44) circle (  2.13);

\path[fill=fillColor,fill opacity=0.20] ( 88.28, 86.87) circle (  2.13);

\path[fill=fillColor,fill opacity=0.20] ( 87.41, 80.30) circle (  2.13);

\path[fill=fillColor,fill opacity=0.20] ( 82.82, 82.07) circle (  2.13);

\path[fill=fillColor,fill opacity=0.20] ( 79.98, 77.64) circle (  2.13);

\path[fill=fillColor,fill opacity=0.20] ( 64.68, 58.30) circle (  2.13);

\path[fill=fillColor,fill opacity=0.20] ( 64.90, 51.22) circle (  2.13);

\path[fill=fillColor,fill opacity=0.20] ( 65.12, 48.81) circle (  2.13);

\path[fill=fillColor,fill opacity=0.20] ( 67.52, 50.08) circle (  2.13);

\path[fill=fillColor,fill opacity=0.20] ( 80.20, 58.30) circle (  2.13);

\path[fill=fillColor,fill opacity=0.20] ( 89.81, 65.13) circle (  2.13);

\path[fill=fillColor,fill opacity=0.20] ( 95.27, 68.92) circle (  2.13);

\path[fill=fillColor,fill opacity=0.20] (106.63, 71.57) circle (  2.13);

\path[fill=fillColor,fill opacity=0.20] ( 97.24, 80.05) circle (  2.13);

\path[fill=fillColor,fill opacity=0.20] ( 91.34, 90.67) circle (  2.13);

\path[fill=fillColor,fill opacity=0.20] ( 96.36, 95.60) circle (  2.13);

\path[fill=fillColor,fill opacity=0.20] (101.17, 91.68) circle (  2.13);

\path[fill=fillColor,fill opacity=0.20] ( 93.96, 87.00) circle (  2.13);

\path[fill=fillColor,fill opacity=0.20] ( 89.15, 81.06) circle (  2.13);

\path[fill=fillColor,fill opacity=0.20] ( 87.62, 80.17) circle (  2.13);

\path[fill=fillColor,fill opacity=0.20] (124.11, 89.15) circle (  2.13);

\path[fill=fillColor,fill opacity=0.20] ( 83.91, 85.61) circle (  2.13);

\path[fill=fillColor,fill opacity=0.20] ( 78.67, 68.92) circle (  2.13);

\path[fill=fillColor,fill opacity=0.20] ( 63.15, 56.15) circle (  2.13);

\path[fill=fillColor,fill opacity=0.20] ( 55.72, 50.46) circle (  2.13);

\path[fill=fillColor,fill opacity=0.20] ( 46.98, 44.26) circle (  2.13);

\path[fill=fillColor,fill opacity=0.20] ( 64.46, 49.70) circle (  2.13);

\path[fill=fillColor,fill opacity=0.20] ( 78.01, 59.44) circle (  2.13);

\path[fill=fillColor,fill opacity=0.20] ( 78.67, 54.50) circle (  2.13);

\path[fill=fillColor,fill opacity=0.20] ( 85.44, 59.69) circle (  2.13);

\path[fill=fillColor,fill opacity=0.20] ( 95.93, 68.92) circle (  2.13);

\path[fill=fillColor,fill opacity=0.20] (103.58, 66.26) circle (  2.13);

\path[fill=fillColor,fill opacity=0.20] (115.37, 70.69) circle (  2.13);

\path[fill=fillColor,fill opacity=0.20] ( 81.94, 88.01) circle (  2.13);

\path[fill=fillColor,fill opacity=0.20] ( 79.10, 74.36) circle (  2.13);

\path[fill=fillColor,fill opacity=0.20] ( 91.78, 94.84) circle (  2.13);

\path[fill=fillColor,fill opacity=0.20] ( 93.96,101.54) circle (  2.13);

\path[fill=fillColor,fill opacity=0.20] ( 92.65, 90.29) circle (  2.13);

\path[fill=fillColor,fill opacity=0.20] ( 92.43, 87.89) circle (  2.13);

\path[fill=fillColor,fill opacity=0.20] ( 88.72, 96.86) circle (  2.13);

\path[fill=fillColor,fill opacity=0.20] ( 83.47, 91.55) circle (  2.13);

\path[fill=fillColor,fill opacity=0.20] ( 86.31, 80.05) circle (  2.13);

\path[fill=fillColor,fill opacity=0.20] ( 81.51, 76.00) circle (  2.13);

\path[fill=fillColor,fill opacity=0.20] ( 71.89, 70.94) circle (  2.13);

\path[fill=fillColor,fill opacity=0.20] ( 70.58, 67.53) circle (  2.13);

\path[fill=fillColor,fill opacity=0.20] ( 63.37, 58.80) circle (  2.13);

\path[fill=fillColor,fill opacity=0.20] ( 55.94, 46.66) circle (  2.13);

\path[fill=fillColor,fill opacity=0.20] ( 49.17, 38.82) circle (  2.13);

\path[fill=fillColor,fill opacity=0.20] ( 65.77, 38.07) circle (  2.13);

\path[fill=fillColor,fill opacity=0.20] ( 68.40, 38.95) circle (  2.13);

\path[fill=fillColor,fill opacity=0.20] ( 74.51, 49.32) circle (  2.13);

\path[fill=fillColor,fill opacity=0.20] ( 84.78, 66.01) circle (  2.13);

\path[fill=fillColor,fill opacity=0.20] ( 97.89, 64.87) circle (  2.13);

\path[fill=fillColor,fill opacity=0.20] ( 99.86, 64.24) circle (  2.13);

\path[fill=fillColor,fill opacity=0.20] (108.60, 72.33) circle (  2.13);

\path[fill=fillColor,fill opacity=0.20] (118.21, 75.87) circle (  2.13);

\path[fill=fillColor,fill opacity=0.20] (102.92, 85.23) circle (  2.13);

\path[fill=fillColor,fill opacity=0.20] ( 97.46, 99.77) circle (  2.13);

\path[fill=fillColor,fill opacity=0.20] ( 98.33, 92.94) circle (  2.13);

\path[fill=fillColor,fill opacity=0.20] ( 90.68, 87.51) circle (  2.13);

\path[fill=fillColor,fill opacity=0.20] ( 87.62, 86.12) circle (  2.13);

\path[fill=fillColor,fill opacity=0.20] ( 87.62, 81.69) circle (  2.13);

\path[fill=fillColor,fill opacity=0.20] ( 78.23, 81.69) circle (  2.13);

\path[fill=fillColor,fill opacity=0.20] ( 75.83, 83.33) circle (  2.13);

\path[fill=fillColor,fill opacity=0.20] ( 78.45, 79.41) circle (  2.13);

\path[fill=fillColor,fill opacity=0.20] ( 79.10, 77.52) circle (  2.13);

\path[fill=fillColor,fill opacity=0.20] ( 70.58, 78.66) circle (  2.13);

\path[fill=fillColor,fill opacity=0.20] ( 73.86, 72.59) circle (  2.13);

\path[fill=fillColor,fill opacity=0.20] ( 70.80, 60.32) circle (  2.13);

\path[fill=fillColor,fill opacity=0.20] ( 63.37, 50.20) circle (  2.13);

\path[fill=fillColor,fill opacity=0.20] ( 61.84, 43.88) circle (  2.13);

\path[fill=fillColor,fill opacity=0.20] ( 56.60, 39.46) circle (  2.13);

\path[fill=fillColor,fill opacity=0.20] ( 64.25, 41.23) circle (  2.13);

\path[fill=fillColor,fill opacity=0.20] ( 97.02, 45.91) circle (  2.13);

\path[fill=fillColor,fill opacity=0.20] ( 52.01, 44.26) circle (  2.13);

\path[fill=fillColor,fill opacity=0.20] ( 47.64, 40.21) circle (  2.13);

\path[fill=fillColor,fill opacity=0.20] ( 55.51, 38.07) circle (  2.13);

\path[fill=fillColor,fill opacity=0.20] ( 61.62, 43.63) circle (  2.13);

\path[fill=fillColor,fill opacity=0.20] ( 62.72, 48.18) circle (  2.13);

\path[fill=fillColor,fill opacity=0.20] ( 65.77, 45.27) circle (  2.13);

\path[fill=fillColor,fill opacity=0.20] ( 69.27, 45.78) circle (  2.13);

\path[fill=fillColor,fill opacity=0.20] ( 75.17, 57.54) circle (  2.13);

\path[fill=fillColor,fill opacity=0.20] ( 76.70, 79.54) circle (  2.13);

\path[fill=fillColor,fill opacity=0.20] ( 74.95, 76.63) circle (  2.13);

\path[fill=fillColor,fill opacity=0.20] ( 82.38, 71.32) circle (  2.13);

\path[fill=fillColor,fill opacity=0.20] ( 84.78, 76.76) circle (  2.13);

\path[fill=fillColor,fill opacity=0.20] ( 83.47, 72.21) circle (  2.13);

\path[fill=fillColor,fill opacity=0.20] ( 90.68, 68.29) circle (  2.13);

\path[fill=fillColor,fill opacity=0.20] (104.23, 81.69) circle (  2.13);

\path[fill=fillColor,fill opacity=0.20] (108.60, 98.26) circle (  2.13);

\path[fill=fillColor,fill opacity=0.20] ( 89.59,101.04) circle (  2.13);

\path[fill=fillColor,fill opacity=0.20] ( 90.68, 82.95) circle (  2.13);

\path[fill=fillColor,fill opacity=0.20] ( 94.84, 82.32) circle (  2.13);

\path[fill=fillColor,fill opacity=0.20] ( 98.55, 89.15) circle (  2.13);

\path[fill=fillColor,fill opacity=0.20] ( 93.52, 87.13) circle (  2.13);

\path[fill=fillColor,fill opacity=0.20] ( 78.67, 85.36) circle (  2.13);

\path[fill=fillColor,fill opacity=0.20] ( 81.07, 89.91) circle (  2.13);

\path[fill=fillColor,fill opacity=0.20] ( 77.14, 85.86) circle (  2.13);

\path[fill=fillColor,fill opacity=0.20] ( 76.26, 86.24) circle (  2.13);

\path[fill=fillColor,fill opacity=0.20] ( 71.02, 94.97) circle (  2.13);

\path[fill=fillColor,fill opacity=0.20] ( 81.51, 91.68) circle (  2.13);

\path[fill=fillColor,fill opacity=0.20] ( 90.25, 85.36) circle (  2.13);

\path[fill=fillColor,fill opacity=0.20] ( 85.88, 84.85) circle (  2.13);

\path[fill=fillColor,fill opacity=0.20] ( 83.69, 80.55) circle (  2.13);

\path[fill=fillColor,fill opacity=0.20] ( 85.88, 75.12) circle (  2.13);

\path[fill=fillColor,fill opacity=0.20] ( 77.57, 72.21) circle (  2.13);

\path[fill=fillColor,fill opacity=0.20] ( 76.70, 73.47) circle (  2.13);

\path[fill=fillColor,fill opacity=0.20] ( 70.14, 77.26) circle (  2.13);

\path[fill=fillColor,fill opacity=0.20] ( 70.80, 77.64) circle (  2.13);

\path[fill=fillColor,fill opacity=0.20] ( 77.57, 70.94) circle (  2.13);

\path[fill=fillColor,fill opacity=0.20] ( 74.95, 68.16) circle (  2.13);

\path[fill=fillColor,fill opacity=0.20] ( 75.83, 70.44) circle (  2.13);

\path[fill=fillColor,fill opacity=0.20] ( 76.26, 69.80) circle (  2.13);

\path[fill=fillColor,fill opacity=0.20] ( 81.29, 69.42) circle (  2.13);

\path[fill=fillColor,fill opacity=0.20] ( 75.83, 73.98) circle (  2.13);

\path[fill=fillColor,fill opacity=0.20] ( 72.77, 79.79) circle (  2.13);

\path[fill=fillColor,fill opacity=0.20] ( 79.10, 78.40) circle (  2.13);

\path[fill=fillColor,fill opacity=0.20] ( 80.85, 77.39) circle (  2.13);

\path[fill=fillColor,fill opacity=0.20] ( 75.83, 82.32) circle (  2.13);

\path[fill=fillColor,fill opacity=0.20] ( 75.61, 84.60) circle (  2.13);

\path[fill=fillColor,fill opacity=0.20] ( 77.36, 85.74) circle (  2.13);

\path[fill=fillColor,fill opacity=0.20] ( 88.06, 86.75) circle (  2.13);

\path[fill=fillColor,fill opacity=0.20] ( 91.99, 79.41) circle (  2.13);

\path[fill=fillColor,fill opacity=0.20] ( 92.21, 76.51) circle (  2.13);

\path[fill=fillColor,fill opacity=0.20] ( 92.21, 91.81) circle (  2.13);

\path[fill=fillColor,fill opacity=0.20] ( 94.62,111.66) circle (  2.13);

\path[fill=fillColor,fill opacity=0.20] ( 85.66, 76.00) circle (  2.13);

\path[fill=fillColor,fill opacity=0.20] ( 93.09, 88.27) circle (  2.13);

\path[fill=fillColor,fill opacity=0.20] (102.48, 88.90) circle (  2.13);

\path[fill=fillColor,fill opacity=0.20] ( 98.77, 89.15) circle (  2.13);

\path[fill=fillColor,fill opacity=0.20] ( 86.75, 87.38) circle (  2.13);

\path[fill=fillColor,fill opacity=0.20] ( 87.62, 86.12) circle (  2.13);

\path[fill=fillColor,fill opacity=0.20] ( 88.28, 93.70) circle (  2.13);

\path[fill=fillColor,fill opacity=0.20] ( 82.16, 97.12) circle (  2.13);

\path[fill=fillColor,fill opacity=0.20] ( 77.14, 87.89) circle (  2.13);

\path[fill=fillColor,fill opacity=0.20] ( 77.57, 84.98) circle (  2.13);

\path[fill=fillColor,fill opacity=0.20] ( 74.30, 85.74) circle (  2.13);

\path[fill=fillColor,fill opacity=0.20] ( 84.13, 85.86) circle (  2.13);

\path[fill=fillColor,fill opacity=0.20] ( 88.94, 86.87) circle (  2.13);

\path[fill=fillColor,fill opacity=0.20] ( 78.88, 78.66) circle (  2.13);

\path[fill=fillColor,fill opacity=0.20] ( 75.17, 71.83) circle (  2.13);

\path[fill=fillColor,fill opacity=0.20] ( 77.36, 78.15) circle (  2.13);

\path[fill=fillColor,fill opacity=0.20] ( 79.54, 84.60) circle (  2.13);

\path[fill=fillColor,fill opacity=0.20] ( 79.10, 79.92) circle (  2.13);

\path[fill=fillColor,fill opacity=0.20] ( 79.54, 82.45) circle (  2.13);

\path[fill=fillColor,fill opacity=0.20] ( 83.25, 89.40) circle (  2.13);

\path[fill=fillColor,fill opacity=0.20] ( 83.69, 84.47) circle (  2.13);

\path[fill=fillColor,fill opacity=0.20] ( 82.60, 79.92) circle (  2.13);

\path[fill=fillColor,fill opacity=0.20] ( 87.19, 81.94) circle (  2.13);

\path[fill=fillColor,fill opacity=0.20] ( 76.04, 79.29) circle (  2.13);

\path[fill=fillColor,fill opacity=0.20] ( 74.51, 75.12) circle (  2.13);

\path[fill=fillColor,fill opacity=0.20] ( 76.92, 75.24) circle (  2.13);

\path[fill=fillColor,fill opacity=0.20] ( 80.41, 77.64) circle (  2.13);

\path[fill=fillColor,fill opacity=0.20] ( 78.67, 78.78) circle (  2.13);

\path[fill=fillColor,fill opacity=0.20] ( 87.19, 86.24) circle (  2.13);

\path[fill=fillColor,fill opacity=0.20] ( 94.18, 92.31) circle (  2.13);

\path[fill=fillColor,fill opacity=0.20] ( 91.99, 86.37) circle (  2.13);

\path[fill=fillColor,fill opacity=0.20] ( 81.51, 88.65) circle (  2.13);

\path[fill=fillColor,fill opacity=0.20] ( 92.65, 84.22) circle (  2.13);

\path[fill=fillColor,fill opacity=0.20] ( 91.56, 87.89) circle (  2.13);

\path[fill=fillColor,fill opacity=0.20] ( 91.99, 95.35) circle (  2.13);

\path[fill=fillColor,fill opacity=0.20] ( 91.56, 97.88) circle (  2.13);

\path[fill=fillColor,fill opacity=0.20] ( 90.90, 92.19) circle (  2.13);

\path[fill=fillColor,fill opacity=0.20] ( 83.25, 85.99) circle (  2.13);

\path[fill=fillColor,fill opacity=0.20] ( 82.60, 83.84) circle (  2.13);

\path[fill=fillColor,fill opacity=0.20] ( 88.94, 85.74) circle (  2.13);

\path[fill=fillColor,fill opacity=0.20] ( 88.28, 90.67) circle (  2.13);

\path[fill=fillColor,fill opacity=0.20] ( 84.57, 85.10) circle (  2.13);

\path[fill=fillColor,fill opacity=0.20] ( 84.35, 76.51) circle (  2.13);

\path[fill=fillColor,fill opacity=0.20] ( 90.03, 78.40) circle (  2.13);

\path[fill=fillColor,fill opacity=0.20] ( 90.03, 80.93) circle (  2.13);

\path[fill=fillColor,fill opacity=0.20] ( 88.72, 81.31) circle (  2.13);

\path[fill=fillColor,fill opacity=0.20] ( 81.73, 86.75) circle (  2.13);

\path[fill=fillColor,fill opacity=0.20] ( 90.47, 91.68) circle (  2.13);

\path[fill=fillColor,fill opacity=0.20] (103.58, 85.74) circle (  2.13);

\path[fill=fillColor,fill opacity=0.20] ( 81.73, 80.05) circle (  2.13);

\path[fill=fillColor,fill opacity=0.20] ( 80.20, 83.46) circle (  2.13);

\path[fill=fillColor,fill opacity=0.20] ( 88.72, 87.51) circle (  2.13);

\path[fill=fillColor,fill opacity=0.20] ( 87.84, 85.48) circle (  2.13);

\path[fill=fillColor,fill opacity=0.20] ( 85.00, 85.36) circle (  2.13);

\path[fill=fillColor,fill opacity=0.20] ( 91.34, 86.87) circle (  2.13);

\path[fill=fillColor,fill opacity=0.20] ( 91.34, 90.54) circle (  2.13);

\path[fill=fillColor,fill opacity=0.20] ( 83.69, 94.21) circle (  2.13);

\path[fill=fillColor,fill opacity=0.20] ( 85.44, 85.48) circle (  2.13);

\path[fill=fillColor,fill opacity=0.20] ( 74.30, 87.63) circle (  2.13);

\path[fill=fillColor,fill opacity=0.20] ( 81.73, 86.50) circle (  2.13);

\path[fill=fillColor,fill opacity=0.20] ( 83.04, 85.99) circle (  2.13);

\path[fill=fillColor,fill opacity=0.20] ( 92.65, 91.68) circle (  2.13);

\path[fill=fillColor,fill opacity=0.20] ( 91.78, 93.32) circle (  2.13);

\path[fill=fillColor,fill opacity=0.20] ( 94.40, 96.74) circle (  2.13);

\path[fill=fillColor,fill opacity=0.20] (105.98,106.60) circle (  2.13);

\path[fill=fillColor,fill opacity=0.20] ( 95.93,106.35) circle (  2.13);

\path[fill=fillColor,fill opacity=0.20] ( 94.40, 97.24) circle (  2.13);

\path[fill=fillColor,fill opacity=0.20] ( 86.75, 91.68) circle (  2.13);

\path[fill=fillColor,fill opacity=0.20] ( 84.78, 91.81) circle (  2.13);

\path[fill=fillColor,fill opacity=0.20] ( 78.23, 94.08) circle (  2.13);

\path[fill=fillColor,fill opacity=0.20] ( 74.95, 92.06) circle (  2.13);

\path[fill=fillColor,fill opacity=0.20] ( 83.04, 93.58) circle (  2.13);

\path[fill=fillColor,fill opacity=0.20] ( 78.67, 93.83) circle (  2.13);

\path[fill=fillColor,fill opacity=0.20] ( 79.32, 91.43) circle (  2.13);

\path[fill=fillColor,fill opacity=0.20] ( 86.10, 95.60) circle (  2.13);

\path[fill=fillColor,fill opacity=0.20] ( 79.98,103.31) circle (  2.13);

\path[fill=fillColor,fill opacity=0.20] ( 78.23,102.68) circle (  2.13);

\path[fill=fillColor,fill opacity=0.20] ( 49.61, 39.96) circle (  2.13);

\path[fill=fillColor,fill opacity=0.20] ( 47.64, 40.97) circle (  2.13);

\path[fill=fillColor,fill opacity=0.20] ( 45.02, 38.19) circle (  2.13);

\path[fill=fillColor,fill opacity=0.20] ( 51.79, 52.23) circle (  2.13);

\path[fill=fillColor,fill opacity=0.20] ( 52.23, 57.29) circle (  2.13);

\path[fill=fillColor,fill opacity=0.20] ( 54.19, 59.94) circle (  2.13);

\path[fill=fillColor,fill opacity=0.20] ( 86.97, 58.42) circle (  2.13);

\path[fill=fillColor,fill opacity=0.20] ( 59.66, 58.80) circle (  2.13);

\path[fill=fillColor,fill opacity=0.20] ( 58.35, 59.44) circle (  2.13);

\path[fill=fillColor,fill opacity=0.20] ( 72.55, 52.61) circle (  2.13);

\path[fill=fillColor,fill opacity=0.20] ( 55.29, 48.05) circle (  2.13);

\path[fill=fillColor,fill opacity=0.20] ( 55.51, 46.41) circle (  2.13);

\path[fill=fillColor,fill opacity=0.20] ( 51.79, 41.23) circle (  2.13);

\path[fill=fillColor,fill opacity=0.20] ( 45.67, 45.78) circle (  2.13);

\path[fill=fillColor,fill opacity=0.20] ( 54.63, 57.41) circle (  2.13);

\path[fill=fillColor,fill opacity=0.20] ( 77.57, 65.38) circle (  2.13);

\path[fill=fillColor,fill opacity=0.20] ( 64.46, 67.53) circle (  2.13);

\path[fill=fillColor,fill opacity=0.20] ( 71.02, 71.70) circle (  2.13);

\path[fill=fillColor,fill opacity=0.20] ( 73.42, 79.92) circle (  2.13);

\path[fill=fillColor,fill opacity=0.20] ( 68.62, 82.83) circle (  2.13);

\path[fill=fillColor,fill opacity=0.20] ( 73.20, 80.30) circle (  2.13);

\path[fill=fillColor,fill opacity=0.20] ( 74.08, 77.52) circle (  2.13);

\path[fill=fillColor,fill opacity=0.20] (140.72, 68.29) circle (  2.13);

\path[fill=fillColor,fill opacity=0.20] ( 89.37, 56.65) circle (  2.13);

\path[fill=fillColor,fill opacity=0.20] ( 75.83, 69.68) circle (  2.13);

\path[fill=fillColor,fill opacity=0.20] ( 84.78, 81.18) circle (  2.13);

\path[fill=fillColor,fill opacity=0.20] ( 95.71, 85.61) circle (  2.13);

\path[fill=fillColor,fill opacity=0.20] (103.36, 88.90) circle (  2.13);

\path[fill=fillColor,fill opacity=0.20] ( 88.06, 92.06) circle (  2.13);

\path[fill=fillColor,fill opacity=0.20] ( 84.57, 95.09) circle (  2.13);

\path[fill=fillColor,fill opacity=0.20] ( 83.04, 93.20) circle (  2.13);

\path[fill=fillColor,fill opacity=0.20] ( 81.73, 76.38) circle (  2.13);

\path[fill=fillColor,fill opacity=0.20] ( 78.01, 69.42) circle (  2.13);

\path[fill=fillColor,fill opacity=0.20] ( 57.25, 46.54) circle (  2.13);

\path[fill=fillColor,fill opacity=0.20] ( 79.76, 68.92) circle (  2.13);

\path[fill=fillColor,fill opacity=0.20] ( 93.31, 82.07) circle (  2.13);

\path[fill=fillColor,fill opacity=0.20] (126.52, 93.83) circle (  2.13);

\path[fill=fillColor,fill opacity=0.20] (134.82, 97.37) circle (  2.13);

\path[fill=fillColor,fill opacity=0.20] (103.14, 98.00) circle (  2.13);

\path[fill=fillColor,fill opacity=0.20] ( 93.09, 91.93) circle (  2.13);

\path[fill=fillColor,fill opacity=0.20] ( 85.00, 83.97) circle (  2.13);

\path[fill=fillColor,fill opacity=0.20] ( 86.31, 87.63) circle (  2.13);

\path[fill=fillColor,fill opacity=0.20] (102.05, 90.16) circle (  2.13);

\path[fill=fillColor,fill opacity=0.20] ( 83.91, 71.07) circle (  2.13);

\path[fill=fillColor,fill opacity=0.20] ( 74.73,101.54) circle (  2.13);

\path[fill=fillColor,fill opacity=0.20] ( 67.09, 55.52) circle (  2.13);

\path[fill=fillColor,fill opacity=0.20] (124.99, 82.20) circle (  2.13);

\path[fill=fillColor,fill opacity=0.20] (112.53, 92.44) circle (  2.13);

\path[fill=fillColor,fill opacity=0.20] ( 93.31, 94.21) circle (  2.13);

\path[fill=fillColor,fill opacity=0.20] ( 91.78, 91.30) circle (  2.13);

\path[fill=fillColor,fill opacity=0.20] ( 93.09, 92.94) circle (  2.13);

\path[fill=fillColor,fill opacity=0.20] ( 94.40, 90.79) circle (  2.13);

\path[fill=fillColor,fill opacity=0.20] ( 83.91, 78.15) circle (  2.13);

\path[fill=fillColor,fill opacity=0.20] ( 81.51, 74.74) circle (  2.13);

\path[fill=fillColor,fill opacity=0.20] ( 85.44, 78.66) circle (  2.13);

\path[fill=fillColor,fill opacity=0.20] ( 79.98, 87.76) circle (  2.13);

\path[fill=fillColor,fill opacity=0.20] ( 91.34, 98.51) circle (  2.13);

\path[fill=fillColor,fill opacity=0.20] (102.92,100.53) circle (  2.13);

\path[fill=fillColor,fill opacity=0.20] (105.76,115.83) circle (  2.13);

\path[fill=fillColor,fill opacity=0.20] ( 86.53, 57.41) circle (  2.13);

\path[fill=fillColor,fill opacity=0.20] (103.36, 89.66) circle (  2.13);

\path[fill=fillColor,fill opacity=0.20] (119.31, 95.35) circle (  2.13);

\path[fill=fillColor,fill opacity=0.20] ( 99.64, 93.32) circle (  2.13);

\path[fill=fillColor,fill opacity=0.20] ( 93.31, 89.15) circle (  2.13);

\path[fill=fillColor,fill opacity=0.20] ( 84.78, 90.92) circle (  2.13);

\path[fill=fillColor,fill opacity=0.20] ( 95.05, 91.55) circle (  2.13);

\path[fill=fillColor,fill opacity=0.20] ( 78.45, 68.54) circle (  2.13);

\path[fill=fillColor,fill opacity=0.20] ( 79.98, 67.78) circle (  2.13);

\path[fill=fillColor,fill opacity=0.20] ( 91.34,101.42) circle (  2.13);

\path[fill=fillColor,fill opacity=0.20] (101.39,110.77) circle (  2.13);

\path[fill=fillColor,fill opacity=0.20] ( 98.33,102.05) circle (  2.13);

\path[fill=fillColor,fill opacity=0.20] (123.24,100.91) circle (  2.13);

\path[fill=fillColor,fill opacity=0.20] (123.24, 98.76) circle (  2.13);

\path[fill=fillColor,fill opacity=0.20] (128.27,102.93) circle (  2.13);

\path[fill=fillColor,fill opacity=0.20] (138.32,114.82) circle (  2.13);

\path[fill=fillColor,fill opacity=0.20] ( 88.50, 54.25) circle (  2.13);

\path[fill=fillColor,fill opacity=0.20] (107.07, 93.20) circle (  2.13);

\path[fill=fillColor,fill opacity=0.20] ( 95.71, 97.88) circle (  2.13);

\path[fill=fillColor,fill opacity=0.20] (100.52, 91.43) circle (  2.13);

\path[fill=fillColor,fill opacity=0.20] ( 95.27, 96.11) circle (  2.13);

\path[fill=fillColor,fill opacity=0.20] ( 96.80, 97.37) circle (  2.13);

\path[fill=fillColor,fill opacity=0.20] ( 92.43, 92.69) circle (  2.13);

\path[fill=fillColor,fill opacity=0.20] ( 78.23, 63.86) circle (  2.13);

\path[fill=fillColor,fill opacity=0.20] ( 63.81, 53.87) circle (  2.13);

\path[fill=fillColor,fill opacity=0.20] ( 86.97,103.19) circle (  2.13);

\path[fill=fillColor,fill opacity=0.20] ( 94.62, 97.88) circle (  2.13);

\path[fill=fillColor,fill opacity=0.20] (102.92, 99.01) circle (  2.13);

\path[fill=fillColor,fill opacity=0.20] (120.18,105.84) circle (  2.13);

\path[fill=fillColor,fill opacity=0.20] (131.54,101.29) circle (  2.13);

\path[fill=fillColor,fill opacity=0.20] (121.27,105.84) circle (  2.13);

\path[fill=fillColor,fill opacity=0.20] ( 98.55, 48.81) circle (  2.13);

\path[fill=fillColor,fill opacity=0.20] (131.76, 95.35) circle (  2.13);

\path[fill=fillColor,fill opacity=0.20] ( 85.88, 97.50) circle (  2.13);

\path[fill=fillColor,fill opacity=0.20] ( 87.84, 92.19) circle (  2.13);

\path[fill=fillColor,fill opacity=0.20] ( 92.43,103.69) circle (  2.13);

\path[fill=fillColor,fill opacity=0.20] ( 91.34,102.93) circle (  2.13);

\path[fill=fillColor,fill opacity=0.20] ( 84.13, 88.01) circle (  2.13);

\path[fill=fillColor,fill opacity=0.20] (112.10, 82.70) circle (  2.13);

\path[fill=fillColor,fill opacity=0.20] ( 77.14, 80.43) circle (  2.13);

\path[fill=fillColor,fill opacity=0.20] ( 97.46, 98.26) circle (  2.13);

\path[fill=fillColor,fill opacity=0.20] ( 99.86,100.66) circle (  2.13);

\path[fill=fillColor,fill opacity=0.20] ( 96.36,107.74) circle (  2.13);

\path[fill=fillColor,fill opacity=0.20] (105.98,111.91) circle (  2.13);

\path[fill=fillColor,fill opacity=0.20] (123.02,105.34) circle (  2.13);

\path[fill=fillColor,fill opacity=0.20] (125.43,102.43) circle (  2.13);

\path[fill=fillColor,fill opacity=0.20] (139.63, 42.87) circle (  2.13);

\path[fill=fillColor,fill opacity=0.20] (112.75, 88.52) circle (  2.13);

\path[fill=fillColor,fill opacity=0.20] ( 91.99, 93.58) circle (  2.13);

\path[fill=fillColor,fill opacity=0.20] ( 84.35, 92.57) circle (  2.13);

\path[fill=fillColor,fill opacity=0.20] ( 85.44,105.08) circle (  2.13);

\path[fill=fillColor,fill opacity=0.20] ( 82.16,104.45) circle (  2.13);

\path[fill=fillColor,fill opacity=0.20] ( 80.85, 89.53) circle (  2.13);

\path[fill=fillColor,fill opacity=0.20] ( 78.01, 83.71) circle (  2.13);

\path[fill=fillColor,fill opacity=0.20] ( 74.73, 90.29) circle (  2.13);

\path[fill=fillColor,fill opacity=0.20] ( 73.64, 89.78) circle (  2.13);

\path[fill=fillColor,fill opacity=0.20] ( 69.93, 73.22) circle (  2.13);

\path[fill=fillColor,fill opacity=0.20] (116.03,110.52) circle (  2.13);

\path[fill=fillColor,fill opacity=0.20] (102.70,102.55) circle (  2.13);

\path[fill=fillColor,fill opacity=0.20] (100.73,112.04) circle (  2.13);

\path[fill=fillColor,fill opacity=0.20] (109.47,109.38) circle (  2.13);

\path[fill=fillColor,fill opacity=0.20] (118.21,102.43) circle (  2.13);

\path[fill=fillColor,fill opacity=0.20] (139.63,111.41) circle (  2.13);

\path[fill=fillColor,fill opacity=0.20] (145.53,114.19) circle (  2.13);

\path[fill=fillColor,fill opacity=0.20] (102.26, 64.24) circle (  2.13);

\path[fill=fillColor,fill opacity=0.20] ( 97.02, 83.59) circle (  2.13);

\path[fill=fillColor,fill opacity=0.20] ( 91.12, 87.63) circle (  2.13);

\path[fill=fillColor,fill opacity=0.20] ( 86.53, 93.58) circle (  2.13);

\path[fill=fillColor,fill opacity=0.20] ( 92.87, 94.71) circle (  2.13);

\path[fill=fillColor,fill opacity=0.20] ( 84.57, 91.43) circle (  2.13);

\path[fill=fillColor,fill opacity=0.20] ( 85.00, 90.92) circle (  2.13);

\path[fill=fillColor,fill opacity=0.20] ( 79.98, 94.97) circle (  2.13);

\path[fill=fillColor,fill opacity=0.20] ( 88.06, 88.65) circle (  2.13);

\path[fill=fillColor,fill opacity=0.20] ( 69.71, 75.62) circle (  2.13);

\path[fill=fillColor,fill opacity=0.20] ( 55.07, 55.64) circle (  2.13);

\path[fill=fillColor,fill opacity=0.20] ( 75.61, 85.74) circle (  2.13);

\path[fill=fillColor,fill opacity=0.20] (110.79,114.95) circle (  2.13);

\path[fill=fillColor,fill opacity=0.20] (101.83,100.28) circle (  2.13);

\path[fill=fillColor,fill opacity=0.20] (110.57,107.11) circle (  2.13);

\path[fill=fillColor,fill opacity=0.20] (111.22,113.30) circle (  2.13);

\path[fill=fillColor,fill opacity=0.20] (105.32,109.51) circle (  2.13);

\path[fill=fillColor,fill opacity=0.20] (115.16,112.04) circle (  2.13);

\path[fill=fillColor,fill opacity=0.20] (125.64,115.58) circle (  2.13);

\path[fill=fillColor,fill opacity=0.20] (141.38,113.81) circle (  2.13);

\path[fill=fillColor,fill opacity=0.20] (129.58,109.51) circle (  2.13);

\path[fill=fillColor,fill opacity=0.20] ( 98.33, 40.59) circle (  2.13);

\path[fill=fillColor,fill opacity=0.20] (102.48, 68.29) circle (  2.13);

\path[fill=fillColor,fill opacity=0.20] ( 96.58, 81.56) circle (  2.13);

\path[fill=fillColor,fill opacity=0.20] ( 92.21, 81.31) circle (  2.13);

\path[fill=fillColor,fill opacity=0.20] ( 79.76, 80.68) circle (  2.13);

\path[fill=fillColor,fill opacity=0.20] ( 81.07, 85.36) circle (  2.13);

\path[fill=fillColor,fill opacity=0.20] ( 86.97, 95.85) circle (  2.13);

\path[fill=fillColor,fill opacity=0.20] ( 83.25, 93.07) circle (  2.13);

\path[fill=fillColor,fill opacity=0.20] ( 77.36, 78.15) circle (  2.13);

\path[fill=fillColor,fill opacity=0.20] ( 71.24, 72.71) circle (  2.13);

\path[fill=fillColor,fill opacity=0.20] ( 61.40, 69.30) circle (  2.13);

\path[fill=fillColor,fill opacity=0.20] ( 53.76, 55.01) circle (  2.13);

\path[fill=fillColor,fill opacity=0.20] ( 97.68,100.28) circle (  2.13);

\path[fill=fillColor,fill opacity=0.20] (109.04,107.74) circle (  2.13);

\path[fill=fillColor,fill opacity=0.20] (107.95,110.14) circle (  2.13);

\path[fill=fillColor,fill opacity=0.20] (102.26,106.22) circle (  2.13);

\path[fill=fillColor,fill opacity=0.20] (106.20,110.77) circle (  2.13);

\path[fill=fillColor,fill opacity=0.20] (116.69,113.81) circle (  2.13);

\path[fill=fillColor,fill opacity=0.20] (110.57,106.22) circle (  2.13);

\path[fill=fillColor,fill opacity=0.20] (117.78,111.66) circle (  2.13);

\path[fill=fillColor,fill opacity=0.20] (125.43,106.60) circle (  2.13);

\path[fill=fillColor,fill opacity=0.20] (109.04,105.34) circle (  2.13);

\path[fill=fillColor,fill opacity=0.20] (101.83, 45.91) circle (  2.13);

\path[fill=fillColor,fill opacity=0.20] ( 96.58, 72.46) circle (  2.13);

\path[fill=fillColor,fill opacity=0.20] ( 88.06, 80.43) circle (  2.13);

\path[fill=fillColor,fill opacity=0.20] ( 77.79, 77.90) circle (  2.13);

\path[fill=fillColor,fill opacity=0.20] ( 77.57, 85.48) circle (  2.13);

\path[fill=fillColor,fill opacity=0.20] ( 81.73, 95.22) circle (  2.13);

\path[fill=fillColor,fill opacity=0.20] ( 76.26, 85.86) circle (  2.13);

\path[fill=fillColor,fill opacity=0.20] ( 74.95, 77.01) circle (  2.13);

\path[fill=fillColor,fill opacity=0.20] ( 82.60, 75.37) circle (  2.13);

\path[fill=fillColor,fill opacity=0.20] ( 74.30, 73.34) circle (  2.13);

\path[fill=fillColor,fill opacity=0.20] ( 64.46, 68.92) circle (  2.13);

\path[fill=fillColor,fill opacity=0.20] ( 56.82, 59.81) circle (  2.13);

\path[fill=fillColor,fill opacity=0.20] ( 64.90, 45.53) circle (  2.13);

\path[fill=fillColor,fill opacity=0.20] ( 91.12, 84.98) circle (  2.13);

\path[fill=fillColor,fill opacity=0.20] (114.28, 97.62) circle (  2.13);

\path[fill=fillColor,fill opacity=0.20] (104.23,105.72) circle (  2.13);

\path[fill=fillColor,fill opacity=0.20] (103.58,107.61) circle (  2.13);

\path[fill=fillColor,fill opacity=0.20] (103.58,111.91) circle (  2.13);

\path[fill=fillColor,fill opacity=0.20] (127.39,108.75) circle (  2.13);

\path[fill=fillColor,fill opacity=0.20] (129.80, 98.51) circle (  2.13);

\path[fill=fillColor,fill opacity=0.20] (109.47,101.42) circle (  2.13);

\path[fill=fillColor,fill opacity=0.20] (120.40,101.67) circle (  2.13);

\path[fill=fillColor,fill opacity=0.20] (116.69,100.03) circle (  2.13);

\path[fill=fillColor,fill opacity=0.20] (100.08,111.15) circle (  2.13);

\path[fill=fillColor,fill opacity=0.20] (116.90, 45.78) circle (  2.13);

\path[fill=fillColor,fill opacity=0.20] ( 87.19, 72.84) circle (  2.13);

\path[fill=fillColor,fill opacity=0.20] ( 85.00, 78.53) circle (  2.13);

\path[fill=fillColor,fill opacity=0.20] ( 83.69, 88.52) circle (  2.13);

\path[fill=fillColor,fill opacity=0.20] ( 84.57, 88.39) circle (  2.13);

\path[fill=fillColor,fill opacity=0.20] ( 72.11, 82.45) circle (  2.13);

\path[fill=fillColor,fill opacity=0.20] ( 75.83, 82.58) circle (  2.13);

\path[fill=fillColor,fill opacity=0.20] ( 96.80, 89.78) circle (  2.13);

\path[fill=fillColor,fill opacity=0.20] ( 78.45, 83.46) circle (  2.13);

\path[fill=fillColor,fill opacity=0.20] ( 72.99, 77.26) circle (  2.13);

\path[fill=fillColor,fill opacity=0.20] ( 63.81, 75.49) circle (  2.13);

\path[fill=fillColor,fill opacity=0.20] ( 61.84, 62.47) circle (  2.13);

\path[fill=fillColor,fill opacity=0.20] ( 47.64, 49.19) circle (  2.13);

\path[fill=fillColor,fill opacity=0.20] ( 70.36, 66.26) circle (  2.13);

\path[fill=fillColor,fill opacity=0.20] ( 83.04, 83.59) circle (  2.13);

\path[fill=fillColor,fill opacity=0.20] ( 93.31, 82.58) circle (  2.13);

\path[fill=fillColor,fill opacity=0.20] (104.23, 91.55) circle (  2.13);

\path[fill=fillColor,fill opacity=0.20] (113.41,107.61) circle (  2.13);

\path[fill=fillColor,fill opacity=0.20] (116.25,109.64) circle (  2.13);

\path[fill=fillColor,fill opacity=0.20] (115.37, 98.63) circle (  2.13);

\path[fill=fillColor,fill opacity=0.20] (109.47, 93.70) circle (  2.13);

\path[fill=fillColor,fill opacity=0.20] (116.03,103.06) circle (  2.13);

\path[fill=fillColor,fill opacity=0.20] (118.43,111.03) circle (  2.13);

\path[fill=fillColor,fill opacity=0.20] (119.74,106.85) circle (  2.13);

\path[fill=fillColor,fill opacity=0.20] (122.80,103.57) circle (  2.13);

\path[fill=fillColor,fill opacity=0.20] (116.25,100.53) circle (  2.13);

\path[fill=fillColor,fill opacity=0.20] (114.50, 95.22) circle (  2.13);

\path[fill=fillColor,fill opacity=0.20] (107.73, 40.85) circle (  2.13);

\path[fill=fillColor,fill opacity=0.20] ( 90.03, 56.40) circle (  2.13);

\path[fill=fillColor,fill opacity=0.20] ( 81.51, 65.88) circle (  2.13);

\path[fill=fillColor,fill opacity=0.20] ( 84.13, 72.08) circle (  2.13);

\path[fill=fillColor,fill opacity=0.20] ( 82.60, 79.67) circle (  2.13);

\path[fill=fillColor,fill opacity=0.20] ( 86.75, 86.12) circle (  2.13);

\path[fill=fillColor,fill opacity=0.20] ( 88.72, 90.42) circle (  2.13);

\path[fill=fillColor,fill opacity=0.20] ( 90.25, 87.63) circle (  2.13);

\path[fill=fillColor,fill opacity=0.20] ( 87.62, 83.97) circle (  2.13);

\path[fill=fillColor,fill opacity=0.20] ( 77.57, 85.48) circle (  2.13);

\path[fill=fillColor,fill opacity=0.20] ( 73.64, 80.68) circle (  2.13);

\path[fill=fillColor,fill opacity=0.20] ( 67.52, 71.57) circle (  2.13);

\path[fill=fillColor,fill opacity=0.20] ( 60.75, 65.13) circle (  2.13);

\path[fill=fillColor,fill opacity=0.20] ( 54.41, 53.49) circle (  2.13);

\path[fill=fillColor,fill opacity=0.20] ( 53.76, 45.02) circle (  2.13);

\path[fill=fillColor,fill opacity=0.20] ( 56.82, 53.11) circle (  2.13);

\path[fill=fillColor,fill opacity=0.20] ( 64.90, 59.56) circle (  2.13);

\path[fill=fillColor,fill opacity=0.20] ( 56.16, 58.68) circle (  2.13);

\path[fill=fillColor,fill opacity=0.20] ( 72.55, 70.18) circle (  2.13);

\path[fill=fillColor,fill opacity=0.20] (103.58, 86.62) circle (  2.13);

\path[fill=fillColor,fill opacity=0.20] ( 93.09, 91.93) circle (  2.13);

\path[fill=fillColor,fill opacity=0.20] ( 91.78, 85.48) circle (  2.13);

\path[fill=fillColor,fill opacity=0.20] (104.89, 90.04) circle (  2.13);

\path[fill=fillColor,fill opacity=0.20] (108.82,107.36) circle (  2.13);

\path[fill=fillColor,fill opacity=0.20] (114.72,109.26) circle (  2.13);

\path[fill=fillColor,fill opacity=0.20] (109.47, 98.51) circle (  2.13);

\path[fill=fillColor,fill opacity=0.20] (116.03, 99.77) circle (  2.13);

\path[fill=fillColor,fill opacity=0.20] (118.65,109.76) circle (  2.13);

\path[fill=fillColor,fill opacity=0.20] (112.75,102.05) circle (  2.13);

\path[fill=fillColor,fill opacity=0.20] (115.16, 97.50) circle (  2.13);

\path[fill=fillColor,fill opacity=0.20] (113.63,112.80) circle (  2.13);

\path[fill=fillColor,fill opacity=0.20] (106.42,108.24) circle (  2.13);

\path[fill=fillColor,fill opacity=0.20] (115.16, 85.36) circle (  2.13);

\path[fill=fillColor,fill opacity=0.20] ( 84.35, 40.09) circle (  2.13);

\path[fill=fillColor,fill opacity=0.20] ( 91.34, 47.68) circle (  2.13);

\path[fill=fillColor,fill opacity=0.20] ( 97.24, 59.18) circle (  2.13);

\path[fill=fillColor,fill opacity=0.20] ( 85.88, 61.84) circle (  2.13);

\path[fill=fillColor,fill opacity=0.20] ( 85.44, 64.62) circle (  2.13);

\path[fill=fillColor,fill opacity=0.20] ( 78.23, 73.09) circle (  2.13);

\path[fill=fillColor,fill opacity=0.20] ( 83.25, 83.84) circle (  2.13);

\path[fill=fillColor,fill opacity=0.20] ( 78.88, 88.27) circle (  2.13);

\path[fill=fillColor,fill opacity=0.20] ( 72.33, 86.62) circle (  2.13);

\path[fill=fillColor,fill opacity=0.20] ( 69.27, 88.01) circle (  2.13);

\path[fill=fillColor,fill opacity=0.20] ( 71.24, 81.18) circle (  2.13);

\path[fill=fillColor,fill opacity=0.20] ( 68.40, 70.94) circle (  2.13);

\path[fill=fillColor,fill opacity=0.20] ( 75.17, 66.90) circle (  2.13);

\path[fill=fillColor,fill opacity=0.20] ( 60.31, 60.19) circle (  2.13);

\path[fill=fillColor,fill opacity=0.20] ( 55.51, 56.40) circle (  2.13);

\path[fill=fillColor,fill opacity=0.20] ( 83.91, 61.21) circle (  2.13);

\path[fill=fillColor,fill opacity=0.20] ( 66.21, 65.76) circle (  2.13);

\path[fill=fillColor,fill opacity=0.20] ( 74.51, 80.05) circle (  2.13);

\path[fill=fillColor,fill opacity=0.20] ( 74.73, 81.18) circle (  2.13);

\path[fill=fillColor,fill opacity=0.20] ( 74.08, 73.47) circle (  2.13);

\path[fill=fillColor,fill opacity=0.20] ( 78.23, 80.81) circle (  2.13);

\path[fill=fillColor,fill opacity=0.20] ( 83.91, 87.89) circle (  2.13);

\path[fill=fillColor,fill opacity=0.20] ( 95.71, 97.24) circle (  2.13);

\path[fill=fillColor,fill opacity=0.20] ( 92.21,104.83) circle (  2.13);

\path[fill=fillColor,fill opacity=0.20] ( 99.42, 99.90) circle (  2.13);

\path[fill=fillColor,fill opacity=0.20] ( 91.12,103.31) circle (  2.13);

\path[fill=fillColor,fill opacity=0.20] (104.23,105.08) circle (  2.13);

\path[fill=fillColor,fill opacity=0.20] (104.67, 98.13) circle (  2.13);

\path[fill=fillColor,fill opacity=0.20] (109.04, 98.51) circle (  2.13);

\path[fill=fillColor,fill opacity=0.20] (110.57,109.38) circle (  2.13);

\path[fill=fillColor,fill opacity=0.20] (119.09,109.00) circle (  2.13);

\path[fill=fillColor,fill opacity=0.20] (115.81,102.93) circle (  2.13);

\path[fill=fillColor,fill opacity=0.20] (115.37,115.33) circle (  2.13);

\path[fill=fillColor,fill opacity=0.20] (105.32, 86.50) circle (  2.13);

\path[fill=fillColor,fill opacity=0.20] ( 85.88, 51.85) circle (  2.13);

\path[fill=fillColor,fill opacity=0.20] ( 79.54, 59.94) circle (  2.13);

\path[fill=fillColor,fill opacity=0.20] ( 76.70, 72.71) circle (  2.13);

\path[fill=fillColor,fill opacity=0.20] ( 79.32, 83.08) circle (  2.13);

\path[fill=fillColor,fill opacity=0.20] ( 79.76, 79.41) circle (  2.13);

\path[fill=fillColor,fill opacity=0.20] ( 78.67, 74.10) circle (  2.13);

\path[fill=fillColor,fill opacity=0.20] ( 81.51, 78.15) circle (  2.13);

\path[fill=fillColor,fill opacity=0.20] ( 77.79, 84.73) circle (  2.13);

\path[fill=fillColor,fill opacity=0.20] ( 65.34, 79.03) circle (  2.13);

\path[fill=fillColor,fill opacity=0.20] ( 77.14, 73.22) circle (  2.13);

\path[fill=fillColor,fill opacity=0.20] ( 64.90, 65.63) circle (  2.13);

\path[fill=fillColor,fill opacity=0.20] ( 57.25, 55.26) circle (  2.13);

\path[fill=fillColor,fill opacity=0.20] ( 52.01, 44.13) circle (  2.13);

\path[fill=fillColor,fill opacity=0.20] ( 66.43, 68.41) circle (  2.13);

\path[fill=fillColor,fill opacity=0.20] ( 76.04, 73.98) circle (  2.13);

\path[fill=fillColor,fill opacity=0.20] ( 78.67, 90.29) circle (  2.13);

\path[fill=fillColor,fill opacity=0.20] ( 89.59, 96.99) circle (  2.13);

\path[fill=fillColor,fill opacity=0.20] ( 94.84, 81.18) circle (  2.13);

\path[fill=fillColor,fill opacity=0.20] ( 94.40, 77.01) circle (  2.13);

\path[fill=fillColor,fill opacity=0.20] ( 84.57, 85.36) circle (  2.13);

\path[fill=fillColor,fill opacity=0.20] ( 86.10, 87.63) circle (  2.13);

\path[fill=fillColor,fill opacity=0.20] (106.42, 84.09) circle (  2.13);

\path[fill=fillColor,fill opacity=0.20] ( 98.77, 97.88) circle (  2.13);

\path[fill=fillColor,fill opacity=0.20] ( 90.47,109.76) circle (  2.13);

\path[fill=fillColor,fill opacity=0.20] ( 95.49,101.67) circle (  2.13);

\path[fill=fillColor,fill opacity=0.20] ( 92.21, 93.58) circle (  2.13);

\path[fill=fillColor,fill opacity=0.20] ( 96.58, 92.44) circle (  2.13);

\path[fill=fillColor,fill opacity=0.20] (103.58, 97.62) circle (  2.13);

\path[fill=fillColor,fill opacity=0.20] (100.95, 99.01) circle (  2.13);

\path[fill=fillColor,fill opacity=0.20] (100.52,102.81) circle (  2.13);

\path[fill=fillColor,fill opacity=0.20] (129.36,112.67) circle (  2.13);

\path[fill=fillColor,fill opacity=0.20] (114.50,110.27) circle (  2.13);

\path[fill=fillColor,fill opacity=0.20] (121.06,107.11) circle (  2.13);

\path[fill=fillColor,fill opacity=0.20] (128.70,113.43) circle (  2.13);

\path[fill=fillColor,fill opacity=0.20] (104.89, 85.23) circle (  2.13);

\path[fill=fillColor,fill opacity=0.20] ( 86.10, 38.82) circle (  2.13);

\path[fill=fillColor,fill opacity=0.20] ( 77.79, 44.13) circle (  2.13);

\path[fill=fillColor,fill opacity=0.20] ( 79.10, 54.12) circle (  2.13);

\path[fill=fillColor,fill opacity=0.20] ( 82.16, 58.04) circle (  2.13);

\path[fill=fillColor,fill opacity=0.20] ( 77.36, 60.95) circle (  2.13);

\path[fill=fillColor,fill opacity=0.20] ( 76.70, 69.30) circle (  2.13);

\path[fill=fillColor,fill opacity=0.20] ( 77.79, 75.87) circle (  2.13);

\path[fill=fillColor,fill opacity=0.20] ( 79.32, 72.71) circle (  2.13);

\path[fill=fillColor,fill opacity=0.20] ( 75.61, 77.14) circle (  2.13);

\path[fill=fillColor,fill opacity=0.20] ( 74.51, 74.36) circle (  2.13);

\path[fill=fillColor,fill opacity=0.20] ( 70.80, 67.65) circle (  2.13);

\path[fill=fillColor,fill opacity=0.20] ( 68.62, 67.15) circle (  2.13);

\path[fill=fillColor,fill opacity=0.20] ( 62.50, 67.78) circle (  2.13);

\path[fill=fillColor,fill opacity=0.20] ( 59.66, 58.55) circle (  2.13);

\path[fill=fillColor,fill opacity=0.20] ( 52.66, 47.93) circle (  2.13);

\path[fill=fillColor,fill opacity=0.20] ( 69.71, 51.60) circle (  2.13);

\path[fill=fillColor,fill opacity=0.20] ( 73.86, 65.25) circle (  2.13);

\path[fill=fillColor,fill opacity=0.20] ( 76.04, 70.31) circle (  2.13);

\path[fill=fillColor,fill opacity=0.20] ( 85.66, 72.71) circle (  2.13);

\path[fill=fillColor,fill opacity=0.20] ( 77.36, 90.04) circle (  2.13);

\path[fill=fillColor,fill opacity=0.20] ( 85.88, 95.35) circle (  2.13);

\path[fill=fillColor,fill opacity=0.20] ( 87.84, 81.82) circle (  2.13);

\path[fill=fillColor,fill opacity=0.20] (101.17, 79.16) circle (  2.13);

\path[fill=fillColor,fill opacity=0.20] ( 85.66, 84.60) circle (  2.13);

\path[fill=fillColor,fill opacity=0.20] ( 90.90, 93.07) circle (  2.13);

\path[fill=fillColor,fill opacity=0.20] ( 90.68, 99.14) circle (  2.13);

\path[fill=fillColor,fill opacity=0.20] ( 91.99, 96.23) circle (  2.13);

\path[fill=fillColor,fill opacity=0.20] ( 89.81, 94.71) circle (  2.13);

\path[fill=fillColor,fill opacity=0.20] ( 86.97, 93.96) circle (  2.13);

\path[fill=fillColor,fill opacity=0.20] ( 98.99, 87.38) circle (  2.13);

\path[fill=fillColor,fill opacity=0.20] ( 99.86, 89.91) circle (  2.13);

\path[fill=fillColor,fill opacity=0.20] ( 99.64,101.54) circle (  2.13);

\path[fill=fillColor,fill opacity=0.20] (100.08,102.68) circle (  2.13);

\path[fill=fillColor,fill opacity=0.20] (101.39,102.55) circle (  2.13);

\path[fill=fillColor,fill opacity=0.20] (108.16,105.34) circle (  2.13);

\path[fill=fillColor,fill opacity=0.20] ( 97.02,102.30) circle (  2.13);

\path[fill=fillColor,fill opacity=0.20] (117.34,102.68) circle (  2.13);

\path[fill=fillColor,fill opacity=0.20] (136.79,104.20) circle (  2.13);

\path[fill=fillColor,fill opacity=0.20] (103.36, 78.02) circle (  2.13);

\path[fill=fillColor,fill opacity=0.20] ( 77.57, 43.63) circle (  2.13);

\path[fill=fillColor,fill opacity=0.20] ( 73.42, 43.88) circle (  2.13);

\path[fill=fillColor,fill opacity=0.20] ( 80.41, 51.85) circle (  2.13);

\path[fill=fillColor,fill opacity=0.20] ( 79.98, 57.79) circle (  2.13);

\path[fill=fillColor,fill opacity=0.20] ( 78.88, 61.96) circle (  2.13);

\path[fill=fillColor,fill opacity=0.20] ( 76.04, 65.00) circle (  2.13);

\path[fill=fillColor,fill opacity=0.20] ( 76.92, 67.65) circle (  2.13);

\path[fill=fillColor,fill opacity=0.20] ( 71.89, 70.69) circle (  2.13);

\path[fill=fillColor,fill opacity=0.20] ( 68.62, 72.08) circle (  2.13);

\path[fill=fillColor,fill opacity=0.20] ( 71.46, 75.75) circle (  2.13);

\path[fill=fillColor,fill opacity=0.20] ( 68.18, 75.62) circle (  2.13);

\path[fill=fillColor,fill opacity=0.20] ( 65.56, 71.57) circle (  2.13);

\path[fill=fillColor,fill opacity=0.20] ( 59.44, 57.54) circle (  2.13);

\path[fill=fillColor,fill opacity=0.20] ( 53.54, 41.61) circle (  2.13);

\path[fill=fillColor,fill opacity=0.20] ( 58.35, 49.95) circle (  2.13);

\path[fill=fillColor,fill opacity=0.20] ( 64.03, 55.26) circle (  2.13);

\path[fill=fillColor,fill opacity=0.20] ( 72.11, 52.23) circle (  2.13);

\path[fill=fillColor,fill opacity=0.20] ( 64.68, 48.43) circle (  2.13);

\path[fill=fillColor,fill opacity=0.20] ( 70.14, 54.50) circle (  2.13);

\path[fill=fillColor,fill opacity=0.20] ( 69.49, 53.62) circle (  2.13);

\path[fill=fillColor,fill opacity=0.20] ( 70.58, 48.56) circle (  2.13);

\path[fill=fillColor,fill opacity=0.20] ( 71.46, 54.76) circle (  2.13);

\path[fill=fillColor,fill opacity=0.20] ( 76.04, 61.96) circle (  2.13);

\path[fill=fillColor,fill opacity=0.20] ( 80.20, 64.87) circle (  2.13);

\path[fill=fillColor,fill opacity=0.20] ( 79.76, 65.13) circle (  2.13);

\path[fill=fillColor,fill opacity=0.20] ( 81.94, 77.14) circle (  2.13);

\path[fill=fillColor,fill opacity=0.20] ( 86.53, 86.37) circle (  2.13);

\path[fill=fillColor,fill opacity=0.20] ( 83.25, 83.08) circle (  2.13);

\path[fill=fillColor,fill opacity=0.20] ( 84.13, 79.67) circle (  2.13);

\path[fill=fillColor,fill opacity=0.20] ( 91.34, 80.55) circle (  2.13);

\path[fill=fillColor,fill opacity=0.20] ( 98.55, 90.16) circle (  2.13);

\path[fill=fillColor,fill opacity=0.20] ( 93.09, 99.65) circle (  2.13);

\path[fill=fillColor,fill opacity=0.20] ( 84.57, 91.43) circle (  2.13);

\path[fill=fillColor,fill opacity=0.20] ( 87.62, 86.12) circle (  2.13);

\path[fill=fillColor,fill opacity=0.20] ( 95.71, 88.65) circle (  2.13);

\path[fill=fillColor,fill opacity=0.20] (100.95, 93.20) circle (  2.13);

\path[fill=fillColor,fill opacity=0.20] (100.73,102.43) circle (  2.13);

\path[fill=fillColor,fill opacity=0.20] ( 92.87,100.66) circle (  2.13);

\path[fill=fillColor,fill opacity=0.20] ( 99.64, 98.00) circle (  2.13);

\path[fill=fillColor,fill opacity=0.20] ( 98.99,104.32) circle (  2.13);

\path[fill=fillColor,fill opacity=0.20] ( 95.93, 99.90) circle (  2.13);

\path[fill=fillColor,fill opacity=0.20] ( 91.78, 97.62) circle (  2.13);

\path[fill=fillColor,fill opacity=0.20] (115.37,102.93) circle (  2.13);

\path[fill=fillColor,fill opacity=0.20] (123.24, 94.59) circle (  2.13);

\path[fill=fillColor,fill opacity=0.20] ( 92.65, 40.21) circle (  2.13);

\path[fill=fillColor,fill opacity=0.20] ( 75.17, 49.32) circle (  2.13);

\path[fill=fillColor,fill opacity=0.20] ( 79.10, 60.45) circle (  2.13);

\path[fill=fillColor,fill opacity=0.20] ( 77.36, 58.42) circle (  2.13);

\path[fill=fillColor,fill opacity=0.20] ( 72.99, 59.94) circle (  2.13);

\path[fill=fillColor,fill opacity=0.20] ( 69.71, 67.78) circle (  2.13);

\path[fill=fillColor,fill opacity=0.20] ( 76.48, 66.52) circle (  2.13);

\path[fill=fillColor,fill opacity=0.20] ( 79.32, 68.92) circle (  2.13);

\path[fill=fillColor,fill opacity=0.20] ( 76.26, 72.97) circle (  2.13);

\path[fill=fillColor,fill opacity=0.20] ( 71.24, 69.68) circle (  2.13);

\path[fill=fillColor,fill opacity=0.20] ( 55.29, 51.47) circle (  2.13);

\path[fill=fillColor,fill opacity=0.20] ( 56.82, 58.68) circle (  2.13);

\path[fill=fillColor,fill opacity=0.20] ( 47.20, 46.54) circle (  2.13);

\path[fill=fillColor,fill opacity=0.20] ( 69.27, 49.32) circle (  2.13);

\path[fill=fillColor,fill opacity=0.20] ( 54.85, 54.38) circle (  2.13);

\path[fill=fillColor,fill opacity=0.20] ( 63.81, 55.89) circle (  2.13);

\path[fill=fillColor,fill opacity=0.20] ( 65.77, 51.60) circle (  2.13);

\path[fill=fillColor,fill opacity=0.20] ( 62.72, 68.79) circle (  2.13);

\path[fill=fillColor,fill opacity=0.20] ( 74.08, 81.18) circle (  2.13);

\path[fill=fillColor,fill opacity=0.20] ( 81.29, 63.99) circle (  2.13);

\path[fill=fillColor,fill opacity=0.20] ( 75.17, 58.68) circle (  2.13);

\path[fill=fillColor,fill opacity=0.20] ( 80.85, 56.40) circle (  2.13);

\path[fill=fillColor,fill opacity=0.20] ( 78.23, 44.39) circle (  2.13);

\path[fill=fillColor,fill opacity=0.20] (109.69, 40.21) circle (  2.13);

\path[fill=fillColor,fill opacity=0.20] ( 72.77, 39.08) circle (  2.13);

\path[fill=fillColor,fill opacity=0.20] ( 82.82, 47.68) circle (  2.13);

\path[fill=fillColor,fill opacity=0.20] ( 83.91, 64.62) circle (  2.13);

\path[fill=fillColor,fill opacity=0.20] ( 93.96, 69.42) circle (  2.13);

\path[fill=fillColor,fill opacity=0.20] ( 92.43, 76.51) circle (  2.13);

\path[fill=fillColor,fill opacity=0.20] ( 93.52, 76.00) circle (  2.13);

\path[fill=fillColor,fill opacity=0.20] ( 97.02, 70.18) circle (  2.13);

\path[fill=fillColor,fill opacity=0.20] ( 95.49, 68.92) circle (  2.13);

\path[fill=fillColor,fill opacity=0.20] ( 89.81, 88.14) circle (  2.13);

\path[fill=fillColor,fill opacity=0.20] ( 91.34, 85.61) circle (  2.13);

\path[fill=fillColor,fill opacity=0.20] ( 94.62, 84.35) circle (  2.13);

\path[fill=fillColor,fill opacity=0.20] ( 93.31, 90.67) circle (  2.13);

\path[fill=fillColor,fill opacity=0.20] ( 95.05, 99.90) circle (  2.13);

\path[fill=fillColor,fill opacity=0.20] ( 86.10,104.96) circle (  2.13);

\path[fill=fillColor,fill opacity=0.20] ( 93.52, 93.07) circle (  2.13);

\path[fill=fillColor,fill opacity=0.20] ( 93.31, 87.38) circle (  2.13);

\path[fill=fillColor,fill opacity=0.20] ( 90.03, 95.35) circle (  2.13);

\path[fill=fillColor,fill opacity=0.20] ( 85.00, 93.45) circle (  2.13);

\path[fill=fillColor,fill opacity=0.20] ( 92.87, 92.69) circle (  2.13);

\path[fill=fillColor,fill opacity=0.20] ( 93.09, 95.47) circle (  2.13);

\path[fill=fillColor,fill opacity=0.20] ( 99.64, 84.47) circle (  2.13);

\path[fill=fillColor,fill opacity=0.20] ( 79.32, 40.72) circle (  2.13);

\path[fill=fillColor,fill opacity=0.20] ( 77.14, 40.59) circle (  2.13);

\path[fill=fillColor,fill opacity=0.20] ( 81.94, 43.76) circle (  2.13);

\path[fill=fillColor,fill opacity=0.20] ( 77.36, 43.12) circle (  2.13);

\path[fill=fillColor,fill opacity=0.20] ( 71.67, 52.86) circle (  2.13);

\path[fill=fillColor,fill opacity=0.20] ( 76.04, 68.16) circle (  2.13);

\path[fill=fillColor,fill opacity=0.20] ( 74.08, 67.78) circle (  2.13);

\path[fill=fillColor,fill opacity=0.20] ( 73.42, 67.40) circle (  2.13);

\path[fill=fillColor,fill opacity=0.20] ( 70.58, 64.37) circle (  2.13);

\path[fill=fillColor,fill opacity=0.20] ( 69.93, 61.21) circle (  2.13);

\path[fill=fillColor,fill opacity=0.20] ( 75.39, 71.95) circle (  2.13);

\path[fill=fillColor,fill opacity=0.20] ( 69.49, 86.87) circle (  2.13);

\path[fill=fillColor,fill opacity=0.20] ( 68.62, 67.91) circle (  2.13);

\path[fill=fillColor,fill opacity=0.20] ( 68.83, 65.50) circle (  2.13);

\path[fill=fillColor,fill opacity=0.20] ( 63.81, 78.53) circle (  2.13);

\path[fill=fillColor,fill opacity=0.20] ( 64.25, 77.52) circle (  2.13);

\path[fill=fillColor,fill opacity=0.20] ( 64.03, 74.61) circle (  2.13);

\path[fill=fillColor,fill opacity=0.20] ( 71.24, 72.08) circle (  2.13);

\path[fill=fillColor,fill opacity=0.20] ( 64.25, 68.03) circle (  2.13);

\path[fill=fillColor,fill opacity=0.20] ( 68.18, 73.47) circle (  2.13);

\path[fill=fillColor,fill opacity=0.20] ( 75.61, 76.13) circle (  2.13);

\path[fill=fillColor,fill opacity=0.20] (115.81, 71.70) circle (  2.13);

\path[fill=fillColor,fill opacity=0.20] ( 78.01, 73.60) circle (  2.13);

\path[fill=fillColor,fill opacity=0.20] ( 71.67, 78.91) circle (  2.13);

\path[fill=fillColor,fill opacity=0.20] (101.83, 70.31) circle (  2.13);

\path[fill=fillColor,fill opacity=0.20] ( 82.16, 57.16) circle (  2.13);

\path[fill=fillColor,fill opacity=0.20] ( 75.61, 54.12) circle (  2.13);

\path[fill=fillColor,fill opacity=0.20] ( 82.16, 63.73) circle (  2.13);

\path[fill=fillColor,fill opacity=0.20] ( 77.36, 69.30) circle (  2.13);

\path[fill=fillColor,fill opacity=0.20] ( 81.73, 50.33) circle (  2.13);

\path[fill=fillColor,fill opacity=0.20] ( 80.63, 43.50) circle (  2.13);

\path[fill=fillColor,fill opacity=0.20] ( 87.19, 40.47) circle (  2.13);

\path[fill=fillColor,fill opacity=0.20] ( 85.22, 50.84) circle (  2.13);

\path[fill=fillColor,fill opacity=0.20] ( 92.87, 53.37) circle (  2.13);

\path[fill=fillColor,fill opacity=0.20] (104.45, 56.78) circle (  2.13);

\path[fill=fillColor,fill opacity=0.20] ( 94.62, 56.02) circle (  2.13);

\path[fill=fillColor,fill opacity=0.20] ( 84.57, 65.25) circle (  2.13);

\path[fill=fillColor,fill opacity=0.20] ( 89.15, 73.22) circle (  2.13);

\path[fill=fillColor,fill opacity=0.20] ( 93.52, 77.39) circle (  2.13);

\path[fill=fillColor,fill opacity=0.20] ( 90.03, 83.21) circle (  2.13);

\path[fill=fillColor,fill opacity=0.20] ( 92.65, 90.04) circle (  2.13);

\path[fill=fillColor,fill opacity=0.20] ( 94.40, 88.77) circle (  2.13);

\path[fill=fillColor,fill opacity=0.20] ( 92.21, 89.53) circle (  2.13);

\path[fill=fillColor,fill opacity=0.20] ( 87.41, 84.60) circle (  2.13);

\path[fill=fillColor,fill opacity=0.20] ( 89.37, 81.44) circle (  2.13);

\path[fill=fillColor,fill opacity=0.20] ( 85.22, 83.71) circle (  2.13);

\path[fill=fillColor,fill opacity=0.20] ( 78.88, 80.43) circle (  2.13);

\path[fill=fillColor,fill opacity=0.20] ( 90.68, 81.18) circle (  2.13);

\path[fill=fillColor,fill opacity=0.20] ( 94.40, 83.59) circle (  2.13);

\path[fill=fillColor,fill opacity=0.20] ( 97.24, 70.18) circle (  2.13);

\path[fill=fillColor,fill opacity=0.20] ( 80.20, 37.94) circle (  2.13);

\path[fill=fillColor,fill opacity=0.20] ( 77.36, 39.46) circle (  2.13);

\path[fill=fillColor,fill opacity=0.20] ( 71.46, 42.87) circle (  2.13);

\path[fill=fillColor,fill opacity=0.20] ( 81.73, 50.08) circle (  2.13);

\path[fill=fillColor,fill opacity=0.20] ( 77.79, 56.27) circle (  2.13);

\path[fill=fillColor,fill opacity=0.20] ( 80.85, 62.85) circle (  2.13);

\path[fill=fillColor,fill opacity=0.20] ( 74.73, 71.57) circle (  2.13);

\path[fill=fillColor,fill opacity=0.20] ( 75.39, 72.21) circle (  2.13);

\path[fill=fillColor,fill opacity=0.20] ( 76.04, 74.36) circle (  2.13);

\path[fill=fillColor,fill opacity=0.20] ( 84.78, 80.81) circle (  2.13);

\path[fill=fillColor,fill opacity=0.20] ( 78.45, 85.23) circle (  2.13);

\path[fill=fillColor,fill opacity=0.20] ( 81.73, 82.45) circle (  2.13);

\path[fill=fillColor,fill opacity=0.20] ( 76.04, 76.89) circle (  2.13);

\path[fill=fillColor,fill opacity=0.20] ( 74.30, 79.92) circle (  2.13);

\path[fill=fillColor,fill opacity=0.20] ( 77.79, 81.31) circle (  2.13);

\path[fill=fillColor,fill opacity=0.20] ( 79.76, 77.39) circle (  2.13);

\path[fill=fillColor,fill opacity=0.20] ( 81.29, 71.45) circle (  2.13);

\path[fill=fillColor,fill opacity=0.20] ( 83.91, 64.87) circle (  2.13);

\path[fill=fillColor,fill opacity=0.20] (123.02, 63.48) circle (  2.13);

\path[fill=fillColor,fill opacity=0.20] ( 91.78, 67.53) circle (  2.13);

\path[fill=fillColor,fill opacity=0.20] ( 88.06, 59.56) circle (  2.13);

\path[fill=fillColor,fill opacity=0.20] ( 86.10, 47.42) circle (  2.13);

\path[fill=fillColor,fill opacity=0.20] ( 94.18, 44.77) circle (  2.13);

\path[fill=fillColor,fill opacity=0.20] ( 98.99, 61.84) circle (  2.13);

\path[fill=fillColor,fill opacity=0.20] ( 93.96, 73.98) circle (  2.13);

\path[fill=fillColor,fill opacity=0.20] ( 90.03, 70.82) circle (  2.13);

\path[fill=fillColor,fill opacity=0.20] ( 96.80, 69.30) circle (  2.13);

\path[fill=fillColor,fill opacity=0.20] ( 94.18, 77.90) circle (  2.13);

\path[fill=fillColor,fill opacity=0.20] ( 87.84, 80.81) circle (  2.13);

\path[fill=fillColor,fill opacity=0.20] ( 91.99, 81.82) circle (  2.13);

\path[fill=fillColor,fill opacity=0.20] ( 86.75, 81.69) circle (  2.13);

\path[fill=fillColor,fill opacity=0.20] ( 86.75, 71.07) circle (  2.13);

\path[fill=fillColor,fill opacity=0.20] ( 93.52, 71.32) circle (  2.13);

\path[fill=fillColor,fill opacity=0.20] (102.48, 68.29) circle (  2.13);

\path[fill=fillColor,fill opacity=0.20] ( 84.13, 40.47) circle (  2.13);

\path[fill=fillColor,fill opacity=0.20] ( 84.35, 43.12) circle (  2.13);

\path[fill=fillColor,fill opacity=0.20] ( 84.35, 45.15) circle (  2.13);

\path[fill=fillColor,fill opacity=0.20] ( 73.42, 46.79) circle (  2.13);

\path[fill=fillColor,fill opacity=0.20] ( 75.61, 53.62) circle (  2.13);

\path[fill=fillColor,fill opacity=0.20] ( 79.32, 63.48) circle (  2.13);

\path[fill=fillColor,fill opacity=0.20] ( 76.26, 75.62) circle (  2.13);

\path[fill=fillColor,fill opacity=0.20] ( 79.98, 80.81) circle (  2.13);

\path[fill=fillColor,fill opacity=0.20] ( 75.39, 70.06) circle (  2.13);

\path[fill=fillColor,fill opacity=0.20] ( 76.70, 65.76) circle (  2.13);

\path[fill=fillColor,fill opacity=0.20] ( 83.91, 76.63) circle (  2.13);

\path[fill=fillColor,fill opacity=0.20] ( 82.82, 81.69) circle (  2.13);

\path[fill=fillColor,fill opacity=0.20] ( 82.82, 76.00) circle (  2.13);

\path[fill=fillColor,fill opacity=0.20] (100.30, 71.57) circle (  2.13);

\path[fill=fillColor,fill opacity=0.20] ( 81.29, 64.62) circle (  2.13);

\path[fill=fillColor,fill opacity=0.20] ( 84.35, 57.41) circle (  2.13);

\path[fill=fillColor,fill opacity=0.20] ( 85.00, 51.85) circle (  2.13);

\path[fill=fillColor,fill opacity=0.20] ( 92.65, 49.45) circle (  2.13);

\path[fill=fillColor,fill opacity=0.20] (100.52, 46.79) circle (  2.13);

\path[fill=fillColor,fill opacity=0.20] ( 84.57, 47.42) circle (  2.13);

\path[fill=fillColor,fill opacity=0.20] ( 94.18, 46.28) circle (  2.13);

\path[fill=fillColor,fill opacity=0.20] (105.54, 51.34) circle (  2.13);

\path[fill=fillColor,fill opacity=0.20] ( 88.50, 73.85) circle (  2.13);

\path[fill=fillColor,fill opacity=0.20] (108.16, 66.39) circle (  2.13);

\path[fill=fillColor,fill opacity=0.20] ( 82.38, 42.49) circle (  2.13);

\path[fill=fillColor,fill opacity=0.20] ( 89.59, 51.72) circle (  2.13);

\path[fill=fillColor,fill opacity=0.20] ( 80.41, 59.56) circle (  2.13);

\path[fill=fillColor,fill opacity=0.20] ( 86.75, 54.63) circle (  2.13);

\path[fill=fillColor,fill opacity=0.20] ( 82.82, 49.57) circle (  2.13);

\path[fill=fillColor,fill opacity=0.20] ( 97.02, 57.92) circle (  2.13);

\path[fill=fillColor,fill opacity=0.20] ( 87.41, 69.55) circle (  2.13);

\path[fill=fillColor,fill opacity=0.20] ( 90.68, 59.56) circle (  2.13);

\path[fill=fillColor,fill opacity=0.20] ( 94.18, 52.48) circle (  2.13);

\path[fill=fillColor,fill opacity=0.20] ( 89.59, 50.58) circle (  2.13);

\path[fill=fillColor,fill opacity=0.20] ( 88.72, 46.16) circle (  2.13);

\path[fill=fillColor,fill opacity=0.20] ( 89.37, 44.13) circle (  2.13);

\path[fill=fillColor,fill opacity=0.20] ( 97.46, 40.85) circle (  2.13);

\path[fill=fillColor,fill opacity=0.20] ( 95.93, 38.44) circle (  2.13);

\path[fill=fillColor,fill opacity=0.20] ( 88.94, 43.25) circle (  2.13);

\path[fill=fillColor,fill opacity=0.20] (100.08, 67.78) circle (  2.13);

\path[fill=fillColor,fill opacity=0.20] (101.83, 77.39) circle (  2.13);

\path[fill=fillColor,fill opacity=0.20] (100.95, 78.91) circle (  2.13);

\path[fill=fillColor,fill opacity=0.20] ( 90.47, 84.35) circle (  2.13);

\path[fill=fillColor,fill opacity=0.20] ( 93.52, 71.83) circle (  2.13);

\path[fill=fillColor,fill opacity=0.20] (108.60, 95.35) circle (  2.13);

\path[fill=fillColor,fill opacity=0.20] (130.23,104.07) circle (  2.13);

\path[fill=fillColor,fill opacity=0.20] (102.92,102.30) circle (  2.13);

\path[fill=fillColor,fill opacity=0.20] ( 93.52, 93.45) circle (  2.13);

\path[fill=fillColor,fill opacity=0.20] ( 89.15, 81.94) circle (  2.13);

\path[fill=fillColor,fill opacity=0.20] ( 75.83, 38.95) circle (  2.13);

\path[fill=fillColor,fill opacity=0.20] ( 82.82, 41.10) circle (  2.13);

\path[fill=fillColor,fill opacity=0.20] ( 78.67, 44.64) circle (  2.13);

\path[fill=fillColor,fill opacity=0.20] ( 76.70, 43.38) circle (  2.13);

\path[fill=fillColor,fill opacity=0.20] ( 92.87, 59.44) circle (  2.13);

\path[fill=fillColor,fill opacity=0.20] (100.95, 98.63) circle (  2.13);

\path[fill=fillColor,fill opacity=0.20] (110.13, 97.12) circle (  2.13);

\path[fill=fillColor,fill opacity=0.20] (136.57, 96.99) circle (  2.13);

\path[fill=fillColor,fill opacity=0.20] (110.57,106.10) circle (  2.13);

\path[fill=fillColor,fill opacity=0.20] ( 97.02, 96.49) circle (  2.13);

\path[fill=fillColor,fill opacity=0.20] (119.53, 82.58) circle (  2.13);

\path[fill=fillColor,fill opacity=0.20] ( 85.88, 77.52) circle (  2.13);

\path[fill=fillColor,fill opacity=0.20] ( 72.77, 67.28) circle (  2.13);

\path[fill=fillColor,fill opacity=0.20] ( 86.75, 39.33) circle (  2.13);

\path[fill=fillColor,fill opacity=0.20] ( 92.21, 55.26) circle (  2.13);

\path[fill=fillColor,fill opacity=0.20] ( 84.57, 68.29) circle (  2.13);

\path[fill=fillColor,fill opacity=0.20] ( 85.66, 72.33) circle (  2.13);

\path[fill=fillColor,fill opacity=0.20] ( 87.19, 71.83) circle (  2.13);

\path[fill=fillColor,fill opacity=0.20] ( 90.25, 72.84) circle (  2.13);

\path[fill=fillColor,fill opacity=0.20] ( 95.27, 73.34) circle (  2.13);

\path[fill=fillColor,fill opacity=0.20] ( 94.84, 59.81) circle (  2.13);

\path[fill=fillColor,fill opacity=0.20] ( 81.29, 42.74) circle (  2.13);

\path[fill=fillColor,fill opacity=0.20] (105.10, 77.52) circle (  2.13);

\path[fill=fillColor,fill opacity=0.20] (107.73,106.47) circle (  2.13);

\path[fill=fillColor,fill opacity=0.20] (122.58, 87.25) circle (  2.13);

\path[fill=fillColor,fill opacity=0.20] (123.90, 91.68) circle (  2.13);

\path[fill=fillColor,fill opacity=0.20] (136.35, 97.88) circle (  2.13);

\path[fill=fillColor,fill opacity=0.20] (100.08, 90.54) circle (  2.13);

\path[fill=fillColor,fill opacity=0.20] ( 98.33, 87.25) circle (  2.13);

\path[fill=fillColor,fill opacity=0.20] ( 98.55, 89.15) circle (  2.13);

\path[fill=fillColor,fill opacity=0.20] ( 98.11, 90.29) circle (  2.13);

\path[fill=fillColor,fill opacity=0.20] ( 77.36, 81.18) circle (  2.13);

\path[fill=fillColor,fill opacity=0.20] ( 88.50, 49.83) circle (  2.13);

\path[fill=fillColor,fill opacity=0.20] ( 92.87, 90.67) circle (  2.13);

\path[fill=fillColor,fill opacity=0.20] (100.95,113.43) circle (  2.13);

\path[fill=fillColor,fill opacity=0.20] ( 96.36,110.02) circle (  2.13);

\path[fill=fillColor,fill opacity=0.20] ( 96.36,102.43) circle (  2.13);

\path[fill=fillColor,fill opacity=0.20] ( 87.84, 91.68) circle (  2.13);

\path[fill=fillColor,fill opacity=0.20] ( 90.90, 85.48) circle (  2.13);

\path[fill=fillColor,fill opacity=0.20] ( 91.34, 84.22) circle (  2.13);

\path[fill=fillColor,fill opacity=0.20] (126.52, 77.77) circle (  2.13);

\path[fill=fillColor,fill opacity=0.20] ( 74.51, 63.86) circle (  2.13);

\path[fill=fillColor,fill opacity=0.20] ( 60.97, 50.96) circle (  2.13);

\path[fill=fillColor,fill opacity=0.20] (106.85, 79.03) circle (  2.13);

\path[fill=fillColor,fill opacity=0.20] (113.84,100.03) circle (  2.13);

\path[fill=fillColor,fill opacity=0.20] (115.37, 85.23) circle (  2.13);

\path[fill=fillColor,fill opacity=0.20] (151.43, 91.30) circle (  2.13);

\path[fill=fillColor,fill opacity=0.20] (123.02, 85.48) circle (  2.13);

\path[fill=fillColor,fill opacity=0.20] (101.17, 95.47) circle (  2.13);

\path[fill=fillColor,fill opacity=0.20] ( 91.56,103.44) circle (  2.13);

\path[fill=fillColor,fill opacity=0.20] ( 89.37, 85.61) circle (  2.13);

\path[fill=fillColor,fill opacity=0.20] ( 85.44, 76.51) circle (  2.13);

\path[fill=fillColor,fill opacity=0.20] ( 76.48, 69.93) circle (  2.13);

\path[fill=fillColor,fill opacity=0.20] ( 94.84, 55.01) circle (  2.13);

\path[fill=fillColor,fill opacity=0.20] (102.05,100.78) circle (  2.13);

\path[fill=fillColor,fill opacity=0.20] (106.85,114.31) circle (  2.13);

\path[fill=fillColor,fill opacity=0.20] (113.41,109.89) circle (  2.13);

\path[fill=fillColor,fill opacity=0.20] (123.24,108.12) circle (  2.13);

\path[fill=fillColor,fill opacity=0.20] (114.50,106.10) circle (  2.13);

\path[fill=fillColor,fill opacity=0.20] (106.63,104.32) circle (  2.13);

\path[fill=fillColor,fill opacity=0.20] ( 97.46,102.68) circle (  2.13);

\path[fill=fillColor,fill opacity=0.20] ( 81.51, 88.27) circle (  2.13);

\path[fill=fillColor,fill opacity=0.20] ( 73.64, 72.71) circle (  2.13);

\path[fill=fillColor,fill opacity=0.20] ( 76.48, 70.31) circle (  2.13);

\path[fill=fillColor,fill opacity=0.20] ( 52.01, 49.45) circle (  2.13);

\path[fill=fillColor,fill opacity=0.20] (102.70, 73.98) circle (  2.13);

\path[fill=fillColor,fill opacity=0.20] (112.53,102.18) circle (  2.13);

\path[fill=fillColor,fill opacity=0.20] (124.55, 89.91) circle (  2.13);

\path[fill=fillColor,fill opacity=0.20] (124.11, 99.27) circle (  2.13);

\path[fill=fillColor,fill opacity=0.20] (145.53, 93.45) circle (  2.13);

\path[fill=fillColor,fill opacity=0.20] (117.78,104.58) circle (  2.13);

\path[fill=fillColor,fill opacity=0.20] ( 94.18,110.02) circle (  2.13);

\path[fill=fillColor,fill opacity=0.20] ( 88.50, 87.76) circle (  2.13);

\path[fill=fillColor,fill opacity=0.20] ( 92.21, 80.17) circle (  2.13);

\path[fill=fillColor,fill opacity=0.20] ( 80.20, 74.48) circle (  2.13);

\path[fill=fillColor,fill opacity=0.20] (103.14, 92.94) circle (  2.13);

\path[fill=fillColor,fill opacity=0.20] (125.43,114.95) circle (  2.13);

\path[fill=fillColor,fill opacity=0.20] (131.11,111.15) circle (  2.13);

\path[fill=fillColor,fill opacity=0.20] (148.80,108.37) circle (  2.13);

\path[fill=fillColor,fill opacity=0.20] (109.47,110.90) circle (  2.13);

\path[fill=fillColor,fill opacity=0.20] ( 95.93,114.19) circle (  2.13);

\path[fill=fillColor,fill opacity=0.20] ( 83.69, 96.49) circle (  2.13);

\path[fill=fillColor,fill opacity=0.20] ( 66.21, 74.86) circle (  2.13);

\path[fill=fillColor,fill opacity=0.20] ( 83.69, 80.81) circle (  2.13);

\path[fill=fillColor,fill opacity=0.20] ( 96.15,110.14) circle (  2.13);

\path[fill=fillColor,fill opacity=0.20] (110.35, 97.24) circle (  2.13);

\path[fill=fillColor,fill opacity=0.20] (124.11,101.67) circle (  2.13);

\path[fill=fillColor,fill opacity=0.20] (114.28,110.65) circle (  2.13);

\path[fill=fillColor,fill opacity=0.20] (111.22,106.85) circle (  2.13);

\path[fill=fillColor,fill opacity=0.20] (112.32,105.84) circle (  2.13);

\path[fill=fillColor,fill opacity=0.20] ( 99.64,101.92) circle (  2.13);

\path[fill=fillColor,fill opacity=0.20] ( 89.59, 95.60) circle (  2.13);

\path[fill=fillColor,fill opacity=0.20] (114.28, 94.21) circle (  2.13);

\path[fill=fillColor,fill opacity=0.20] ( 87.62, 60.32) circle (  2.13);

\path[fill=fillColor,fill opacity=0.20] ( 99.42,111.66) circle (  2.13);

\path[fill=fillColor,fill opacity=0.20] ( 95.49, 99.52) circle (  2.13);

\path[fill=fillColor,fill opacity=0.20] ( 83.91, 93.70) circle (  2.13);

\path[fill=fillColor,fill opacity=0.20] ( 77.79, 92.94) circle (  2.13);

\path[fill=fillColor,fill opacity=0.20] ( 63.15, 80.17) circle (  2.13);

\path[fill=fillColor,fill opacity=0.20] ( 91.78, 85.10) circle (  2.13);

\path[fill=fillColor,fill opacity=0.20] ( 90.03, 98.89) circle (  2.13);

\path[fill=fillColor,fill opacity=0.20] ( 94.62, 93.70) circle (  2.13);

\path[fill=fillColor,fill opacity=0.20] (121.49, 97.50) circle (  2.13);

\path[fill=fillColor,fill opacity=0.20] (106.20,106.73) circle (  2.13);

\path[fill=fillColor,fill opacity=0.20] (107.29,104.07) circle (  2.13);

\path[fill=fillColor,fill opacity=0.20] (111.22, 98.38) circle (  2.13);

\path[fill=fillColor,fill opacity=0.20] (110.13,100.53) circle (  2.13);

\path[fill=fillColor,fill opacity=0.20] ( 95.27,101.16) circle (  2.13);

\path[fill=fillColor,fill opacity=0.20] ( 91.34, 94.59) circle (  2.13);

\path[fill=fillColor,fill opacity=0.20] ( 98.77, 76.13) circle (  2.13);

\path[fill=fillColor,fill opacity=0.20] (112.97,114.95) circle (  2.13);

\path[fill=fillColor,fill opacity=0.20] (120.84,110.65) circle (  2.13);

\path[fill=fillColor,fill opacity=0.20] (105.76,111.15) circle (  2.13);

\path[fill=fillColor,fill opacity=0.20] (105.32,101.42) circle (  2.13);

\path[fill=fillColor,fill opacity=0.20] (101.61, 90.79) circle (  2.13);

\path[fill=fillColor,fill opacity=0.20] ( 78.01, 80.68) circle (  2.13);

\path[fill=fillColor,fill opacity=0.20] ( 68.18, 76.25) circle (  2.13);

\path[fill=fillColor,fill opacity=0.20] ( 76.26, 62.72) circle (  2.13);

\path[fill=fillColor,fill opacity=0.20] ( 90.25, 87.51) circle (  2.13);

\path[fill=fillColor,fill opacity=0.20] ( 99.21, 93.83) circle (  2.13);

\path[fill=fillColor,fill opacity=0.20] ( 92.65, 94.97) circle (  2.13);

\path[fill=fillColor,fill opacity=0.20] ( 88.06, 97.50) circle (  2.13);

\path[fill=fillColor,fill opacity=0.20] ( 94.84, 98.00) circle (  2.13);

\path[fill=fillColor,fill opacity=0.20] ( 99.64, 97.37) circle (  2.13);

\path[fill=fillColor,fill opacity=0.20] (117.56,100.91) circle (  2.13);

\path[fill=fillColor,fill opacity=0.20] (109.47,104.32) circle (  2.13);

\path[fill=fillColor,fill opacity=0.20] ( 98.77, 98.13) circle (  2.13);

\path[fill=fillColor,fill opacity=0.20] ( 82.38, 88.14) circle (  2.13);

\path[fill=fillColor,fill opacity=0.20] ( 65.77, 71.95) circle (  2.13);

\path[fill=fillColor,fill opacity=0.20] ( 59.88, 50.33) circle (  2.13);

\path[fill=fillColor,fill opacity=0.20] ( 97.02, 81.06) circle (  2.13);

\path[fill=fillColor,fill opacity=0.20] (118.65,112.80) circle (  2.13);

\path[fill=fillColor,fill opacity=0.20] (119.96,104.07) circle (  2.13);

\path[fill=fillColor,fill opacity=0.20] (114.94,101.54) circle (  2.13);

\path[fill=fillColor,fill opacity=0.20] (100.52, 99.14) circle (  2.13);

\path[fill=fillColor,fill opacity=0.20] ( 87.19, 99.65) circle (  2.13);

\path[fill=fillColor,fill opacity=0.20] ( 86.97, 96.74) circle (  2.13);

\path[fill=fillColor,fill opacity=0.20] ( 80.85, 83.59) circle (  2.13);

\path[fill=fillColor,fill opacity=0.20] ( 82.38, 78.15) circle (  2.13);

\path[fill=fillColor,fill opacity=0.20] ( 88.50,100.78) circle (  2.13);

\path[fill=fillColor,fill opacity=0.20] ( 89.37,101.29) circle (  2.13);

\path[fill=fillColor,fill opacity=0.20] (102.26, 98.76) circle (  2.13);

\path[fill=fillColor,fill opacity=0.20] ( 96.58, 92.69) circle (  2.13);

\path[fill=fillColor,fill opacity=0.20] (102.05, 96.61) circle (  2.13);

\path[fill=fillColor,fill opacity=0.20] ( 93.96,106.47) circle (  2.13);

\path[fill=fillColor,fill opacity=0.20] ( 91.99,103.06) circle (  2.13);

\path[fill=fillColor,fill opacity=0.20] ( 84.78, 97.24) circle (  2.13);

\path[fill=fillColor,fill opacity=0.20] ( 73.86, 89.40) circle (  2.13);

\path[fill=fillColor,fill opacity=0.20] ( 94.84, 71.20) circle (  2.13);

\path[fill=fillColor,fill opacity=0.20] (104.89,107.87) circle (  2.13);

\path[fill=fillColor,fill opacity=0.20] (119.96,110.65) circle (  2.13);

\path[fill=fillColor,fill opacity=0.20] (121.49,106.85) circle (  2.13);

\path[fill=fillColor,fill opacity=0.20] (107.95, 95.47) circle (  2.13);

\path[fill=fillColor,fill opacity=0.20] ( 96.15, 90.92) circle (  2.13);

\path[fill=fillColor,fill opacity=0.20] ( 83.25, 96.11) circle (  2.13);

\path[fill=fillColor,fill opacity=0.20] ( 82.38, 91.17) circle (  2.13);

\path[fill=fillColor,fill opacity=0.20] ( 79.76, 79.03) circle (  2.13);

\path[fill=fillColor,fill opacity=0.20] ( 62.50, 60.83) circle (  2.13);

\path[fill=fillColor,fill opacity=0.20] ( 77.79, 54.25) circle (  2.13);

\path[fill=fillColor,fill opacity=0.20] ( 82.82, 81.44) circle (  2.13);

\path[fill=fillColor,fill opacity=0.20] ( 85.66, 99.39) circle (  2.13);

\path[fill=fillColor,fill opacity=0.20] ( 94.18, 94.97) circle (  2.13);

\path[fill=fillColor,fill opacity=0.20] ( 98.11, 92.94) circle (  2.13);

\path[fill=fillColor,fill opacity=0.20] ( 98.77, 98.51) circle (  2.13);

\path[fill=fillColor,fill opacity=0.20] ( 92.87, 95.22) circle (  2.13);

\path[fill=fillColor,fill opacity=0.20] ( 93.09, 95.22) circle (  2.13);

\path[fill=fillColor,fill opacity=0.20] ( 93.31, 97.88) circle (  2.13);

\path[fill=fillColor,fill opacity=0.20] ( 88.94, 97.37) circle (  2.13);

\path[fill=fillColor,fill opacity=0.20] ( 83.69, 99.27) circle (  2.13);

\path[fill=fillColor,fill opacity=0.20] (120.62, 88.27) circle (  2.13);

\path[fill=fillColor,fill opacity=0.20] (114.50,102.30) circle (  2.13);

\path[fill=fillColor,fill opacity=0.20] (124.77,111.15) circle (  2.13);

\path[fill=fillColor,fill opacity=0.20] (101.61,103.57) circle (  2.13);

\path[fill=fillColor,fill opacity=0.20] ( 92.87, 93.45) circle (  2.13);

\path[fill=fillColor,fill opacity=0.20] ( 83.47, 89.28) circle (  2.13);

\path[fill=fillColor,fill opacity=0.20] ( 77.14, 82.70) circle (  2.13);

\path[fill=fillColor,fill opacity=0.20] ( 78.67, 74.99) circle (  2.13);

\path[fill=fillColor,fill opacity=0.20] ( 72.11, 63.99) circle (  2.13);

\path[fill=fillColor,fill opacity=0.20] ( 84.13, 61.46) circle (  2.13);

\path[fill=fillColor,fill opacity=0.20] ( 83.25, 74.86) circle (  2.13);

\path[fill=fillColor,fill opacity=0.20] ( 92.21, 84.09) circle (  2.13);

\path[fill=fillColor,fill opacity=0.20] (108.16, 98.38) circle (  2.13);

\path[fill=fillColor,fill opacity=0.20] (101.61,110.65) circle (  2.13);

\path[fill=fillColor,fill opacity=0.20] ( 94.40,104.83) circle (  2.13);

\path[fill=fillColor,fill opacity=0.20] (101.83, 92.31) circle (  2.13);

\path[fill=fillColor,fill opacity=0.20] ( 95.93, 89.02) circle (  2.13);

\path[fill=fillColor,fill opacity=0.20] ( 95.27, 96.36) circle (  2.13);

\path[fill=fillColor,fill opacity=0.20] ( 88.06, 97.37) circle (  2.13);

\path[fill=fillColor,fill opacity=0.20] ( 68.40, 75.49) circle (  2.13);

\path[fill=fillColor,fill opacity=0.20] ( 99.42, 71.57) circle (  2.13);

\path[fill=fillColor,fill opacity=0.20] (120.84, 90.92) circle (  2.13);

\path[fill=fillColor,fill opacity=0.20] (138.97,107.87) circle (  2.13);

\path[fill=fillColor,fill opacity=0.20] ( 99.21,115.96) circle (  2.13);

\path[fill=fillColor,fill opacity=0.20] ( 99.86,102.18) circle (  2.13);

\path[fill=fillColor,fill opacity=0.20] ( 88.94, 92.31) circle (  2.13);

\path[fill=fillColor,fill opacity=0.20] ( 75.17, 87.13) circle (  2.13);

\path[fill=fillColor,fill opacity=0.20] ( 74.95, 76.89) circle (  2.13);

\path[fill=fillColor,fill opacity=0.20] ( 73.42, 66.39) circle (  2.13);

\path[fill=fillColor,fill opacity=0.20] ( 57.03, 54.12) circle (  2.13);

\path[fill=fillColor,fill opacity=0.20] ( 94.18, 76.51) circle (  2.13);

\path[fill=fillColor,fill opacity=0.20] (105.32, 82.70) circle (  2.13);

\path[fill=fillColor,fill opacity=0.20] (108.60, 89.91) circle (  2.13);

\path[fill=fillColor,fill opacity=0.20] (118.00,102.81) circle (  2.13);

\path[fill=fillColor,fill opacity=0.20] (101.17,108.50) circle (  2.13);

\path[fill=fillColor,fill opacity=0.20] (109.26,101.67) circle (  2.13);

\path[fill=fillColor,fill opacity=0.20] ( 96.80, 95.60) circle (  2.13);

\path[fill=fillColor,fill opacity=0.20] ( 94.18, 94.97) circle (  2.13);

\path[fill=fillColor,fill opacity=0.20] ( 64.25, 63.86) circle (  2.13);

\path[fill=fillColor,fill opacity=0.20] ( 57.47, 49.57) circle (  2.13);

\path[fill=fillColor,fill opacity=0.20] ( 83.69, 58.17) circle (  2.13);

\path[fill=fillColor,fill opacity=0.20] (103.14, 84.85) circle (  2.13);

\path[fill=fillColor,fill opacity=0.20] ( 99.86, 97.24) circle (  2.13);

\path[fill=fillColor,fill opacity=0.20] ( 99.21,104.20) circle (  2.13);

\path[fill=fillColor,fill opacity=0.20] ( 95.49,102.68) circle (  2.13);

\path[fill=fillColor,fill opacity=0.20] ( 90.25, 96.99) circle (  2.13);

\path[fill=fillColor,fill opacity=0.20] ( 86.75, 94.97) circle (  2.13);

\path[fill=fillColor,fill opacity=0.20] ( 86.75, 84.47) circle (  2.13);

\path[fill=fillColor,fill opacity=0.20] ( 74.08, 72.71) circle (  2.13);

\path[fill=fillColor,fill opacity=0.20] ( 65.12, 63.73) circle (  2.13);

\path[fill=fillColor,fill opacity=0.20] ( 88.94, 74.48) circle (  2.13);

\path[fill=fillColor,fill opacity=0.20] (102.26, 83.71) circle (  2.13);

\path[fill=fillColor,fill opacity=0.20] (111.44, 94.59) circle (  2.13);

\path[fill=fillColor,fill opacity=0.20] (108.60, 98.00) circle (  2.13);

\path[fill=fillColor,fill opacity=0.20] (115.59, 91.30) circle (  2.13);

\path[fill=fillColor,fill opacity=0.20] (113.41, 88.77) circle (  2.13);

\path[fill=fillColor,fill opacity=0.20] ( 98.77, 94.59) circle (  2.13);

\path[fill=fillColor,fill opacity=0.20] ( 89.15,103.31) circle (  2.13);

\path[fill=fillColor,fill opacity=0.20] ( 88.28,102.68) circle (  2.13);

\path[fill=fillColor,fill opacity=0.20] ( 99.86, 86.12) circle (  2.13);

\path[fill=fillColor,fill opacity=0.20] ( 95.93, 93.96) circle (  2.13);

\path[fill=fillColor,fill opacity=0.20] ( 88.28,101.29) circle (  2.13);

\path[fill=fillColor,fill opacity=0.20] ( 88.94, 98.26) circle (  2.13);

\path[fill=fillColor,fill opacity=0.20] ( 87.62, 96.61) circle (  2.13);

\path[fill=fillColor,fill opacity=0.20] ( 78.45, 96.74) circle (  2.13);

\path[fill=fillColor,fill opacity=0.20] ( 71.02, 76.00) circle (  2.13);

\path[fill=fillColor,fill opacity=0.20] ( 68.62, 57.79) circle (  2.13);

\path[fill=fillColor,fill opacity=0.20] ( 74.30, 47.55) circle (  2.13);

\path[fill=fillColor,fill opacity=0.20] ( 85.22, 66.77) circle (  2.13);

\path[fill=fillColor,fill opacity=0.20] ( 86.31, 68.79) circle (  2.13);

\path[fill=fillColor,fill opacity=0.20] ( 90.47, 79.03) circle (  2.13);

\path[fill=fillColor,fill opacity=0.20] ( 97.46, 87.13) circle (  2.13);

\path[fill=fillColor,fill opacity=0.20] (105.32, 92.06) circle (  2.13);

\path[fill=fillColor,fill opacity=0.20] (107.73, 93.32) circle (  2.13);

\path[fill=fillColor,fill opacity=0.20] (104.67, 84.85) circle (  2.13);

\path[fill=fillColor,fill opacity=0.20] (121.27, 84.98) circle (  2.13);

\path[fill=fillColor,fill opacity=0.20] ( 88.06, 99.39) circle (  2.13);

\path[fill=fillColor,fill opacity=0.20] ( 77.79,105.46) circle (  2.13);

\path[fill=fillColor,fill opacity=0.20] ( 83.25, 82.70) circle (  2.13);

\path[fill=fillColor,fill opacity=0.20] ( 74.30, 69.80) circle (  2.13);

\path[fill=fillColor,fill opacity=0.20] ( 64.03, 70.94) circle (  2.13);

\path[fill=fillColor,fill opacity=0.20] (106.85, 76.89) circle (  2.13);

\path[fill=fillColor,fill opacity=0.20] ( 97.02, 93.07) circle (  2.13);

\path[fill=fillColor,fill opacity=0.20] ( 81.29, 98.51) circle (  2.13);

\path[fill=fillColor,fill opacity=0.20] ( 79.76, 98.26) circle (  2.13);

\path[fill=fillColor,fill opacity=0.20] ( 80.63, 98.76) circle (  2.13);

\path[fill=fillColor,fill opacity=0.20] ( 76.04, 94.97) circle (  2.13);

\path[fill=fillColor,fill opacity=0.20] ( 85.22, 78.40) circle (  2.13);

\path[fill=fillColor,fill opacity=0.20] ( 70.36, 61.96) circle (  2.13);

\path[fill=fillColor,fill opacity=0.20] ( 99.21, 48.69) circle (  2.13);

\path[fill=fillColor,fill opacity=0.20] ( 81.07, 64.87) circle (  2.13);

\path[fill=fillColor,fill opacity=0.20] ( 85.88, 70.69) circle (  2.13);

\path[fill=fillColor,fill opacity=0.20] ( 90.47, 87.13) circle (  2.13);

\path[fill=fillColor,fill opacity=0.20] ( 91.34, 91.81) circle (  2.13);

\path[fill=fillColor,fill opacity=0.20] ( 92.65, 83.84) circle (  2.13);

\path[fill=fillColor,fill opacity=0.20] ( 91.78, 83.08) circle (  2.13);

\path[fill=fillColor,fill opacity=0.20] ( 98.33, 86.62) circle (  2.13);

\path[fill=fillColor,fill opacity=0.20] ( 92.65, 88.77) circle (  2.13);

\path[fill=fillColor,fill opacity=0.20] ( 83.91, 94.21) circle (  2.13);

\path[fill=fillColor,fill opacity=0.20] ( 83.25,101.16) circle (  2.13);

\path[fill=fillColor,fill opacity=0.20] ( 74.08, 90.04) circle (  2.13);

\path[fill=fillColor,fill opacity=0.20] ( 53.76, 71.83) circle (  2.13);

\path[fill=fillColor,fill opacity=0.20] ( 90.47, 69.93) circle (  2.13);

\path[fill=fillColor,fill opacity=0.20] (102.92, 85.99) circle (  2.13);

\path[fill=fillColor,fill opacity=0.20] ( 92.21, 84.22) circle (  2.13);

\path[fill=fillColor,fill opacity=0.20] ( 78.23, 92.44) circle (  2.13);

\path[fill=fillColor,fill opacity=0.20] ( 77.14, 96.49) circle (  2.13);

\path[fill=fillColor,fill opacity=0.20] ( 77.36, 85.48) circle (  2.13);

\path[fill=fillColor,fill opacity=0.20] ( 78.01, 78.15) circle (  2.13);

\path[fill=fillColor,fill opacity=0.20] ( 71.89, 80.43) circle (  2.13);

\path[fill=fillColor,fill opacity=0.20] ( 76.04, 69.17) circle (  2.13);

\path[fill=fillColor,fill opacity=0.20] ( 60.75, 54.63) circle (  2.13);

\path[fill=fillColor,fill opacity=0.20] ( 84.13, 65.38) circle (  2.13);

\path[fill=fillColor,fill opacity=0.20] ( 89.37, 69.93) circle (  2.13);

\path[fill=fillColor,fill opacity=0.20] ( 90.68, 75.37) circle (  2.13);

\path[fill=fillColor,fill opacity=0.20] ( 87.41, 93.58) circle (  2.13);

\path[fill=fillColor,fill opacity=0.20] ( 91.12,105.21) circle (  2.13);

\path[fill=fillColor,fill opacity=0.20] (104.45, 95.22) circle (  2.13);

\path[fill=fillColor,fill opacity=0.20] ( 92.43, 89.02) circle (  2.13);

\path[fill=fillColor,fill opacity=0.20] ( 91.78, 87.13) circle (  2.13);

\path[fill=fillColor,fill opacity=0.20] ( 88.50, 84.73) circle (  2.13);

\path[fill=fillColor,fill opacity=0.20] ( 86.97, 87.51) circle (  2.13);

\path[fill=fillColor,fill opacity=0.20] ( 75.39, 89.66) circle (  2.13);

\path[fill=fillColor,fill opacity=0.20] ( 83.47, 85.86) circle (  2.13);

\path[fill=fillColor,fill opacity=0.20] ( 82.82, 77.26) circle (  2.13);

\path[fill=fillColor,fill opacity=0.20] ( 84.57, 51.72) circle (  2.13);

\path[fill=fillColor,fill opacity=0.20] ( 89.37, 69.93) circle (  2.13);

\path[fill=fillColor,fill opacity=0.20] ( 99.64, 78.53) circle (  2.13);

\path[fill=fillColor,fill opacity=0.20] ( 82.82, 85.74) circle (  2.13);

\path[fill=fillColor,fill opacity=0.20] ( 80.20, 85.36) circle (  2.13);

\path[fill=fillColor,fill opacity=0.20] ( 89.37, 86.50) circle (  2.13);

\path[fill=fillColor,fill opacity=0.20] ( 80.41, 99.65) circle (  2.13);

\path[fill=fillColor,fill opacity=0.20] ( 74.30, 89.78) circle (  2.13);

\path[fill=fillColor,fill opacity=0.20] ( 73.20, 74.36) circle (  2.13);

\path[fill=fillColor,fill opacity=0.20] ( 75.39, 73.85) circle (  2.13);

\path[fill=fillColor,fill opacity=0.20] ( 67.52, 64.75) circle (  2.13);

\path[fill=fillColor,fill opacity=0.20] ( 60.09, 54.88) circle (  2.13);

\path[fill=fillColor,fill opacity=0.20] ( 85.66, 65.00) circle (  2.13);

\path[fill=fillColor,fill opacity=0.20] ( 89.81, 74.86) circle (  2.13);

\path[fill=fillColor,fill opacity=0.20] ( 91.12, 78.78) circle (  2.13);

\path[fill=fillColor,fill opacity=0.20] ( 92.43, 85.48) circle (  2.13);

\path[fill=fillColor,fill opacity=0.20] ( 89.15, 99.27) circle (  2.13);

\path[fill=fillColor,fill opacity=0.20] ( 94.18, 97.88) circle (  2.13);

\path[fill=fillColor,fill opacity=0.20] ( 96.80, 88.52) circle (  2.13);

\path[fill=fillColor,fill opacity=0.20] (102.26, 92.06) circle (  2.13);

\path[fill=fillColor,fill opacity=0.20] ( 86.10, 95.09) circle (  2.13);

\path[fill=fillColor,fill opacity=0.20] ( 91.12, 86.37) circle (  2.13);

\path[fill=fillColor,fill opacity=0.20] ( 85.22, 82.20) circle (  2.13);

\path[fill=fillColor,fill opacity=0.20] ( 81.94, 78.02) circle (  2.13);

\path[fill=fillColor,fill opacity=0.20] ( 81.94, 74.48) circle (  2.13);

\path[fill=fillColor,fill opacity=0.20] ( 75.83, 71.07) circle (  2.13);

\path[fill=fillColor,fill opacity=0.20] ( 79.98, 56.40) circle (  2.13);

\path[fill=fillColor,fill opacity=0.20] ( 91.78, 82.32) circle (  2.13);

\path[fill=fillColor,fill opacity=0.20] (103.79, 84.09) circle (  2.13);

\path[fill=fillColor,fill opacity=0.20] ( 91.34, 74.86) circle (  2.13);

\path[fill=fillColor,fill opacity=0.20] ( 93.09, 91.05) circle (  2.13);

\path[fill=fillColor,fill opacity=0.20] ( 87.19, 85.36) circle (  2.13);

\path[fill=fillColor,fill opacity=0.20] ( 82.16, 87.38) circle (  2.13);

\path[fill=fillColor,fill opacity=0.20] ( 79.10, 82.95) circle (  2.13);

\path[fill=fillColor,fill opacity=0.20] ( 78.23, 79.03) circle (  2.13);

\path[fill=fillColor,fill opacity=0.20] ( 75.39, 76.63) circle (  2.13);

\path[fill=fillColor,fill opacity=0.20] ( 71.02, 73.85) circle (  2.13);

\path[fill=fillColor,fill opacity=0.20] ( 72.11, 65.88) circle (  2.13);

\path[fill=fillColor,fill opacity=0.20] ( 86.97, 66.77) circle (  2.13);

\path[fill=fillColor,fill opacity=0.20] ( 90.47, 79.92) circle (  2.13);

\path[fill=fillColor,fill opacity=0.20] ( 94.18, 88.39) circle (  2.13);

\path[fill=fillColor,fill opacity=0.20] ( 93.74, 89.28) circle (  2.13);

\path[fill=fillColor,fill opacity=0.20] ( 94.62, 86.50) circle (  2.13);

\path[fill=fillColor,fill opacity=0.20] ( 97.24, 87.63) circle (  2.13);

\path[fill=fillColor,fill opacity=0.20] ( 97.68, 88.77) circle (  2.13);

\path[fill=fillColor,fill opacity=0.20] ( 93.31, 90.16) circle (  2.13);

\path[fill=fillColor,fill opacity=0.20] ( 91.56, 94.84) circle (  2.13);

\path[fill=fillColor,fill opacity=0.20] ( 82.38, 89.40) circle (  2.13);

\path[fill=fillColor,fill opacity=0.20] ( 77.79, 79.03) circle (  2.13);

\path[fill=fillColor,fill opacity=0.20] ( 79.10, 79.54) circle (  2.13);

\path[fill=fillColor,fill opacity=0.20] ( 75.17, 74.23) circle (  2.13);

\path[fill=fillColor,fill opacity=0.20] ( 70.36, 61.08) circle (  2.13);

\path[fill=fillColor,fill opacity=0.20] ( 90.47, 72.08) circle (  2.13);

\path[fill=fillColor,fill opacity=0.20] ( 97.46, 82.20) circle (  2.13);

\path[fill=fillColor,fill opacity=0.20] ( 82.38, 83.33) circle (  2.13);

\path[fill=fillColor,fill opacity=0.20] ( 90.03, 85.61) circle (  2.13);

\path[fill=fillColor,fill opacity=0.20] ( 89.59, 93.83) circle (  2.13);

\path[fill=fillColor,fill opacity=0.20] ( 90.03, 99.27) circle (  2.13);

\path[fill=fillColor,fill opacity=0.20] ( 86.31, 82.95) circle (  2.13);

\path[fill=fillColor,fill opacity=0.20] ( 91.99, 74.61) circle (  2.13);

\path[fill=fillColor,fill opacity=0.20] ( 79.76, 71.07) circle (  2.13);

\path[fill=fillColor,fill opacity=0.20] ( 76.92, 72.71) circle (  2.13);

\path[fill=fillColor,fill opacity=0.20] ( 73.64, 68.03) circle (  2.13);

\path[fill=fillColor,fill opacity=0.20] ( 77.57, 56.65) circle (  2.13);

\path[fill=fillColor,fill opacity=0.20] ( 82.60, 71.32) circle (  2.13);

\path[fill=fillColor,fill opacity=0.20] ( 83.69, 80.17) circle (  2.13);

\path[fill=fillColor,fill opacity=0.20] (104.23, 85.36) circle (  2.13);

\path[fill=fillColor,fill opacity=0.20] ( 99.86, 86.62) circle (  2.13);

\path[fill=fillColor,fill opacity=0.20] (100.73, 81.44) circle (  2.13);

\path[fill=fillColor,fill opacity=0.20] (101.61, 74.99) circle (  2.13);

\path[fill=fillColor,fill opacity=0.20] ( 97.24, 76.76) circle (  2.13);

\path[fill=fillColor,fill opacity=0.20] (102.05, 84.47) circle (  2.13);

\path[fill=fillColor,fill opacity=0.20] ( 86.75, 94.21) circle (  2.13);

\path[fill=fillColor,fill opacity=0.20] ( 83.69, 96.61) circle (  2.13);

\path[fill=fillColor,fill opacity=0.20] ( 71.89, 75.37) circle (  2.13);

\path[fill=fillColor,fill opacity=0.20] ( 69.93, 61.84) circle (  2.13);

\path[fill=fillColor,fill opacity=0.20] ( 66.21, 67.40) circle (  2.13);

\path[fill=fillColor,fill opacity=0.20] ( 55.72, 59.69) circle (  2.13);

\path[fill=fillColor,fill opacity=0.20] ( 64.03, 71.57) circle (  2.13);

\path[fill=fillColor,fill opacity=0.20] ( 81.29, 78.28) circle (  2.13);

\path[fill=fillColor,fill opacity=0.20] ( 95.49, 79.29) circle (  2.13);

\path[fill=fillColor,fill opacity=0.20] ( 95.49, 89.02) circle (  2.13);

\path[fill=fillColor,fill opacity=0.20] ( 94.40, 90.16) circle (  2.13);

\path[fill=fillColor,fill opacity=0.20] ( 87.84, 79.92) circle (  2.13);

\path[fill=fillColor,fill opacity=0.20] ( 92.21, 77.14) circle (  2.13);

\path[fill=fillColor,fill opacity=0.20] ( 85.66, 81.31) circle (  2.13);

\path[fill=fillColor,fill opacity=0.20] ( 82.60, 83.97) circle (  2.13);

\path[fill=fillColor,fill opacity=0.20] ( 76.04, 76.00) circle (  2.13);

\path[fill=fillColor,fill opacity=0.20] ( 74.51, 60.32) circle (  2.13);

\path[fill=fillColor,fill opacity=0.20] ( 71.02, 54.76) circle (  2.13);

\path[fill=fillColor,fill opacity=0.20] ( 69.27, 55.77) circle (  2.13);

\path[fill=fillColor,fill opacity=0.20] ( 69.27, 43.88) circle (  2.13);

\path[fill=fillColor,fill opacity=0.20] ( 83.25, 59.31) circle (  2.13);

\path[fill=fillColor,fill opacity=0.20] ( 85.00, 66.52) circle (  2.13);

\path[fill=fillColor,fill opacity=0.20] ( 89.59, 67.91) circle (  2.13);

\path[fill=fillColor,fill opacity=0.20] ( 92.87, 72.33) circle (  2.13);

\path[fill=fillColor,fill opacity=0.20] (100.30, 76.13) circle (  2.13);

\path[fill=fillColor,fill opacity=0.20] ( 97.02, 78.40) circle (  2.13);

\path[fill=fillColor,fill opacity=0.20] ( 97.24, 76.51) circle (  2.13);

\path[fill=fillColor,fill opacity=0.20] ( 90.68, 69.93) circle (  2.13);

\path[fill=fillColor,fill opacity=0.20] ( 83.47, 73.85) circle (  2.13);

\path[fill=fillColor,fill opacity=0.20] ( 69.05, 78.15) circle (  2.13);

\path[fill=fillColor,fill opacity=0.20] ( 68.40, 69.42) circle (  2.13);

\path[fill=fillColor,fill opacity=0.20] ( 65.77, 60.45) circle (  2.13);

\path[fill=fillColor,fill opacity=0.20] ( 67.09, 51.09) circle (  2.13);

\path[fill=fillColor,fill opacity=0.20] ( 53.76, 47.17) circle (  2.13);

\path[fill=fillColor,fill opacity=0.20] ( 73.20, 78.15) circle (  2.13);

\path[fill=fillColor,fill opacity=0.20] (106.20, 88.01) circle (  2.13);

\path[fill=fillColor,fill opacity=0.20] ( 99.42, 80.17) circle (  2.13);

\path[fill=fillColor,fill opacity=0.20] ( 87.62, 78.02) circle (  2.13);

\path[fill=fillColor,fill opacity=0.20] ( 95.49, 88.52) circle (  2.13);

\path[fill=fillColor,fill opacity=0.20] ( 99.64,104.58) circle (  2.13);

\path[fill=fillColor,fill opacity=0.20] ( 90.47,101.80) circle (  2.13);

\path[fill=fillColor,fill opacity=0.20] ( 81.73, 87.25) circle (  2.13);

\path[fill=fillColor,fill opacity=0.20] ( 87.84, 76.76) circle (  2.13);

\path[fill=fillColor,fill opacity=0.20] ( 76.48, 70.06) circle (  2.13);

\path[fill=fillColor,fill opacity=0.20] ( 75.39, 62.72) circle (  2.13);

\path[fill=fillColor,fill opacity=0.20] ( 72.77, 56.91) circle (  2.13);

\path[fill=fillColor,fill opacity=0.20] ( 70.80, 53.62) circle (  2.13);

\path[fill=fillColor,fill opacity=0.20] ( 78.23, 47.55) circle (  2.13);

\path[fill=fillColor,fill opacity=0.20] ( 88.72, 58.42) circle (  2.13);

\path[fill=fillColor,fill opacity=0.20] (110.35, 67.53) circle (  2.13);

\path[fill=fillColor,fill opacity=0.20] (114.72, 69.55) circle (  2.13);

\path[fill=fillColor,fill opacity=0.20] (111.88, 76.13) circle (  2.13);

\path[fill=fillColor,fill opacity=0.20] ( 89.15, 77.52) circle (  2.13);

\path[fill=fillColor,fill opacity=0.20] ( 81.94, 72.97) circle (  2.13);

\path[fill=fillColor,fill opacity=0.20] ( 78.67, 77.77) circle (  2.13);

\path[fill=fillColor,fill opacity=0.20] ( 68.18, 79.16) circle (  2.13);

\path[fill=fillColor,fill opacity=0.20] ( 68.62, 68.92) circle (  2.13);

\path[fill=fillColor,fill opacity=0.20] ( 61.62, 66.90) circle (  2.13);

\path[fill=fillColor,fill opacity=0.20] ( 59.22, 59.18) circle (  2.13);

\path[fill=fillColor,fill opacity=0.20] ( 53.10, 43.63) circle (  2.13);

\path[fill=fillColor,fill opacity=0.20] (142.25, 73.85) circle (  2.13);

\path[fill=fillColor,fill opacity=0.20] (145.09, 86.12) circle (  2.13);

\path[fill=fillColor,fill opacity=0.20] (107.29, 85.99) circle (  2.13);

\path[fill=fillColor,fill opacity=0.20] (100.08, 93.96) circle (  2.13);

\path[fill=fillColor,fill opacity=0.20] (102.05, 98.89) circle (  2.13);

\path[fill=fillColor,fill opacity=0.20] ( 92.87, 95.22) circle (  2.13);

\path[fill=fillColor,fill opacity=0.20] ( 85.00, 94.46) circle (  2.13);

\path[fill=fillColor,fill opacity=0.20] ( 83.25, 93.45) circle (  2.13);

\path[fill=fillColor,fill opacity=0.20] ( 79.54, 84.60) circle (  2.13);

\path[fill=fillColor,fill opacity=0.20] ( 80.20, 69.80) circle (  2.13);

\path[fill=fillColor,fill opacity=0.20] ( 72.11, 62.09) circle (  2.13);

\path[fill=fillColor,fill opacity=0.20] ( 69.49, 66.14) circle (  2.13);

\path[fill=fillColor,fill opacity=0.20] ( 67.74, 67.65) circle (  2.13);

\path[fill=fillColor,fill opacity=0.20] ( 65.99, 57.16) circle (  2.13);

\path[fill=fillColor,fill opacity=0.20] ( 91.56, 51.97) circle (  2.13);

\path[fill=fillColor,fill opacity=0.20] ( 84.13, 52.10) circle (  2.13);

\path[fill=fillColor,fill opacity=0.20] ( 86.75, 55.26) circle (  2.13);

\path[fill=fillColor,fill opacity=0.20] ( 97.02, 61.84) circle (  2.13);

\path[fill=fillColor,fill opacity=0.20] ( 95.71, 68.92) circle (  2.13);

\path[fill=fillColor,fill opacity=0.20] ( 87.62, 77.01) circle (  2.13);

\path[fill=fillColor,fill opacity=0.20] ( 88.50, 94.59) circle (  2.13);

\path[fill=fillColor,fill opacity=0.20] ( 59.66, 94.21) circle (  2.13);

\path[fill=fillColor,fill opacity=0.20] ( 62.50, 68.03) circle (  2.13);

\path[fill=fillColor,fill opacity=0.20] ( 47.42, 54.76) circle (  2.13);

\path[fill=fillColor,fill opacity=0.20] ( 54.41, 52.35) circle (  2.13);

\path[fill=fillColor,fill opacity=0.20] ( 89.37,105.72) circle (  2.13);

\path[fill=fillColor,fill opacity=0.20] (103.79, 90.67) circle (  2.13);

\path[fill=fillColor,fill opacity=0.20] (101.61, 82.45) circle (  2.13);

\path[fill=fillColor,fill opacity=0.20] ( 93.74, 85.23) circle (  2.13);

\path[fill=fillColor,fill opacity=0.20] ( 83.69, 89.66) circle (  2.13);

\path[fill=fillColor,fill opacity=0.20] ( 85.66, 91.81) circle (  2.13);

\path[fill=fillColor,fill opacity=0.20] ( 86.10, 85.99) circle (  2.13);

\path[fill=fillColor,fill opacity=0.20] ( 76.48, 76.25) circle (  2.13);

\path[fill=fillColor,fill opacity=0.20] ( 76.26, 71.07) circle (  2.13);

\path[fill=fillColor,fill opacity=0.20] ( 70.36, 69.42) circle (  2.13);

\path[fill=fillColor,fill opacity=0.20] ( 70.80, 66.77) circle (  2.13);

\path[fill=fillColor,fill opacity=0.20] ( 79.98, 59.81) circle (  2.13);

\path[fill=fillColor,fill opacity=0.20] ( 82.16, 55.89) circle (  2.13);

\path[fill=fillColor,fill opacity=0.20] ( 86.97, 55.26) circle (  2.13);

\path[fill=fillColor,fill opacity=0.20] ( 83.25, 62.98) circle (  2.13);

\path[fill=fillColor,fill opacity=0.20] ( 83.47, 63.99) circle (  2.13);

\path[fill=fillColor,fill opacity=0.20] ( 74.51, 64.87) circle (  2.13);

\path[fill=fillColor,fill opacity=0.20] ( 71.89, 73.22) circle (  2.13);

\path[fill=fillColor,fill opacity=0.20] ( 67.74, 73.72) circle (  2.13);

\path[fill=fillColor,fill opacity=0.20] ( 58.13, 72.84) circle (  2.13);

\path[fill=fillColor,fill opacity=0.20] ( 51.79, 63.48) circle (  2.13);

\path[fill=fillColor,fill opacity=0.20] ( 66.65, 56.65) circle (  2.13);

\path[fill=fillColor,fill opacity=0.20] ( 70.58, 83.33) circle (  2.13);

\path[fill=fillColor,fill opacity=0.20] ( 97.24, 86.24) circle (  2.13);

\path[fill=fillColor,fill opacity=0.20] (100.30, 85.74) circle (  2.13);

\path[fill=fillColor,fill opacity=0.20] ( 91.34, 82.83) circle (  2.13);

\path[fill=fillColor,fill opacity=0.20] ( 91.56, 85.99) circle (  2.13);

\path[fill=fillColor,fill opacity=0.20] ( 87.84, 89.78) circle (  2.13);

\path[fill=fillColor,fill opacity=0.20] ( 79.54, 82.07) circle (  2.13);

\path[fill=fillColor,fill opacity=0.20] ( 84.13, 75.37) circle (  2.13);

\path[fill=fillColor,fill opacity=0.20] ( 83.69, 70.69) circle (  2.13);

\path[fill=fillColor,fill opacity=0.20] ( 78.45, 65.88) circle (  2.13);

\path[fill=fillColor,fill opacity=0.20] ( 77.36, 64.11) circle (  2.13);

\path[fill=fillColor,fill opacity=0.20] ( 67.74, 71.57) circle (  2.13);

\path[fill=fillColor,fill opacity=0.20] ( 68.83, 79.79) circle (  2.13);

\path[fill=fillColor,fill opacity=0.20] ( 69.71, 67.53) circle (  2.13);

\path[fill=fillColor,fill opacity=0.20] ( 70.14, 54.76) circle (  2.13);

\path[fill=fillColor,fill opacity=0.20] ( 75.39, 57.41) circle (  2.13);

\path[fill=fillColor,fill opacity=0.20] ( 81.07, 63.86) circle (  2.13);

\path[fill=fillColor,fill opacity=0.20] ( 77.57, 61.96) circle (  2.13);

\path[fill=fillColor,fill opacity=0.20] ( 87.41, 61.84) circle (  2.13);

\path[fill=fillColor,fill opacity=0.20] ( 89.37, 76.38) circle (  2.13);

\path[fill=fillColor,fill opacity=0.20] ( 94.40, 75.49) circle (  2.13);

\path[fill=fillColor,fill opacity=0.20] ( 93.31, 68.79) circle (  2.13);

\path[fill=fillColor,fill opacity=0.20] ( 82.82, 67.40) circle (  2.13);

\path[fill=fillColor,fill opacity=0.20] ( 64.90, 66.14) circle (  2.13);

\path[fill=fillColor,fill opacity=0.20] ( 63.81, 69.05) circle (  2.13);

\path[fill=fillColor,fill opacity=0.20] ( 70.14, 73.98) circle (  2.13);

\path[fill=fillColor,fill opacity=0.20] ( 94.18, 81.31) circle (  2.13);

\path[fill=fillColor,fill opacity=0.20] ( 91.12, 77.52) circle (  2.13);

\path[fill=fillColor,fill opacity=0.20] ( 89.37, 80.68) circle (  2.13);

\path[fill=fillColor,fill opacity=0.20] ( 91.34, 90.92) circle (  2.13);

\path[fill=fillColor,fill opacity=0.20] ( 90.25, 87.38) circle (  2.13);

\path[fill=fillColor,fill opacity=0.20] ( 94.62, 77.14) circle (  2.13);

\path[fill=fillColor,fill opacity=0.20] ( 95.93, 72.33) circle (  2.13);

\path[fill=fillColor,fill opacity=0.20] ( 88.72, 69.68) circle (  2.13);

\path[fill=fillColor,fill opacity=0.20] ( 81.51, 72.21) circle (  2.13);

\path[fill=fillColor,fill opacity=0.20] ( 75.83, 70.94) circle (  2.13);

\path[fill=fillColor,fill opacity=0.20] ( 74.73, 68.92) circle (  2.13);

\path[fill=fillColor,fill opacity=0.20] ( 74.51, 70.82) circle (  2.13);

\path[fill=fillColor,fill opacity=0.20] ( 81.29, 73.85) circle (  2.13);

\path[fill=fillColor,fill opacity=0.20] ( 70.80, 72.33) circle (  2.13);

\path[fill=fillColor,fill opacity=0.20] ( 71.02, 72.46) circle (  2.13);

\path[fill=fillColor,fill opacity=0.20] ( 71.24, 70.18) circle (  2.13);

\path[fill=fillColor,fill opacity=0.20] ( 72.55, 62.60) circle (  2.13);

\path[fill=fillColor,fill opacity=0.20] ( 75.17, 58.17) circle (  2.13);

\path[fill=fillColor,fill opacity=0.20] ( 78.88, 56.53) circle (  2.13);

\path[fill=fillColor,fill opacity=0.20] ( 72.77, 58.17) circle (  2.13);

\path[fill=fillColor,fill opacity=0.20] ( 72.99, 66.52) circle (  2.13);

\path[fill=fillColor,fill opacity=0.20] ( 77.14, 70.56) circle (  2.13);

\path[fill=fillColor,fill opacity=0.20] ( 79.98, 65.13) circle (  2.13);

\path[fill=fillColor,fill opacity=0.20] ( 77.36, 64.49) circle (  2.13);

\path[fill=fillColor,fill opacity=0.20] ( 79.10, 67.65) circle (  2.13);

\path[fill=fillColor,fill opacity=0.20] ( 80.20, 68.67) circle (  2.13);

\path[fill=fillColor,fill opacity=0.20] ( 78.01, 69.05) circle (  2.13);

\path[fill=fillColor,fill opacity=0.20] ( 78.23, 73.85) circle (  2.13);

\path[fill=fillColor,fill opacity=0.20] ( 83.25, 78.78) circle (  2.13);

\path[fill=fillColor,fill opacity=0.20] ( 92.65, 88.90) circle (  2.13);

\path[fill=fillColor,fill opacity=0.20] ( 90.25, 92.44) circle (  2.13);

\path[fill=fillColor,fill opacity=0.20] ( 93.52, 76.51) circle (  2.13);

\path[fill=fillColor,fill opacity=0.20] ( 91.34, 66.64) circle (  2.13);

\path[fill=fillColor,fill opacity=0.20] ( 93.52, 75.12) circle (  2.13);

\path[fill=fillColor,fill opacity=0.20] ( 90.68, 90.67) circle (  2.13);

\path[fill=fillColor,fill opacity=0.20] ( 82.82, 81.31) circle (  2.13);

\path[fill=fillColor,fill opacity=0.20] ( 61.19, 70.56) circle (  2.13);

\path[fill=fillColor,fill opacity=0.20] ( 46.55, 60.32) circle (  2.13);

\path[fill=fillColor,fill opacity=0.20] ( 67.09, 49.95) circle (  2.13);

\path[fill=fillColor,fill opacity=0.20] ( 61.19, 64.62) circle (  2.13);

\path[fill=fillColor,fill opacity=0.20] ( 72.99, 64.62) circle (  2.13);

\path[fill=fillColor,fill opacity=0.20] ( 71.24, 67.91) circle (  2.13);

\path[fill=fillColor,fill opacity=0.20] ( 82.82, 76.63) circle (  2.13);

\path[fill=fillColor,fill opacity=0.20] ( 98.11, 78.91) circle (  2.13);

\path[fill=fillColor,fill opacity=0.20] (109.69, 77.26) circle (  2.13);

\path[fill=fillColor,fill opacity=0.20] ( 93.74, 81.31) circle (  2.13);

\path[fill=fillColor,fill opacity=0.20] ( 87.62, 78.28) circle (  2.13);

\path[fill=fillColor,fill opacity=0.20] ( 90.25, 66.90) circle (  2.13);

\path[fill=fillColor,fill opacity=0.20] ( 85.66, 62.22) circle (  2.13);

\path[fill=fillColor,fill opacity=0.20] ( 80.85, 67.53) circle (  2.13);

\path[fill=fillColor,fill opacity=0.20] ( 78.67, 70.31) circle (  2.13);

\path[fill=fillColor,fill opacity=0.20] ( 76.26, 76.76) circle (  2.13);

\path[fill=fillColor,fill opacity=0.20] ( 76.48, 67.28) circle (  2.13);

\path[fill=fillColor,fill opacity=0.20] ( 79.10, 69.55) circle (  2.13);

\path[fill=fillColor,fill opacity=0.20] ( 73.42, 78.28) circle (  2.13);

\path[fill=fillColor,fill opacity=0.20] ( 69.93, 82.07) circle (  2.13);

\path[fill=fillColor,fill opacity=0.20] ( 71.89, 73.72) circle (  2.13);

\path[fill=fillColor,fill opacity=0.20] ( 77.79, 67.40) circle (  2.13);

\path[fill=fillColor,fill opacity=0.20] ( 74.73, 72.59) circle (  2.13);

\path[fill=fillColor,fill opacity=0.20] ( 75.17, 78.02) circle (  2.13);

\path[fill=fillColor,fill opacity=0.20] ( 80.41, 73.47) circle (  2.13);

\path[fill=fillColor,fill opacity=0.20] ( 78.45, 76.13) circle (  2.13);

\path[fill=fillColor,fill opacity=0.20] ( 80.63, 87.25) circle (  2.13);

\path[fill=fillColor,fill opacity=0.20] ( 80.41, 86.62) circle (  2.13);

\path[fill=fillColor,fill opacity=0.20] ( 89.81, 77.64) circle (  2.13);

\path[fill=fillColor,fill opacity=0.20] ( 95.71, 77.90) circle (  2.13);

\path[fill=fillColor,fill opacity=0.20] ( 90.90, 82.32) circle (  2.13);

\path[fill=fillColor,fill opacity=0.20] ( 78.01, 75.87) circle (  2.13);

\path[fill=fillColor,fill opacity=0.20] ( 68.40, 68.29) circle (  2.13);

\path[fill=fillColor,fill opacity=0.20] ( 66.43, 73.34) circle (  2.13);

\path[fill=fillColor,fill opacity=0.20] ( 67.96, 74.48) circle (  2.13);

\path[fill=fillColor,fill opacity=0.20] ( 69.71, 59.69) circle (  2.13);

\path[fill=fillColor,fill opacity=0.20] ( 78.88, 50.71) circle (  2.13);

\path[fill=fillColor,fill opacity=0.20] ( 55.94, 50.71) circle (  2.13);

\path[fill=fillColor,fill opacity=0.20] ( 74.51, 66.39) circle (  2.13);

\path[fill=fillColor,fill opacity=0.20] ( 90.47, 77.77) circle (  2.13);

\path[fill=fillColor,fill opacity=0.20] ( 79.76, 88.90) circle (  2.13);

\path[fill=fillColor,fill opacity=0.20] ( 82.38, 83.33) circle (  2.13);

\path[fill=fillColor,fill opacity=0.20] ( 88.28, 68.41) circle (  2.13);

\path[fill=fillColor,fill opacity=0.20] ( 88.72, 67.02) circle (  2.13);

\path[fill=fillColor,fill opacity=0.20] ( 92.87, 72.08) circle (  2.13);

\path[fill=fillColor,fill opacity=0.20] ( 91.56, 68.29) circle (  2.13);

\path[fill=fillColor,fill opacity=0.20] ( 94.40, 71.45) circle (  2.13);

\path[fill=fillColor,fill opacity=0.20] ( 96.80, 71.70) circle (  2.13);

\path[fill=fillColor,fill opacity=0.20] ( 90.90, 64.75) circle (  2.13);

\path[fill=fillColor,fill opacity=0.20] ( 88.06, 71.57) circle (  2.13);

\path[fill=fillColor,fill opacity=0.20] ( 79.32, 83.33) circle (  2.13);

\path[fill=fillColor,fill opacity=0.20] ( 79.32, 77.39) circle (  2.13);

\path[fill=fillColor,fill opacity=0.20] ( 83.69, 71.95) circle (  2.13);

\path[fill=fillColor,fill opacity=0.20] ( 85.44, 74.99) circle (  2.13);

\path[fill=fillColor,fill opacity=0.20] (103.14, 69.05) circle (  2.13);

\path[fill=fillColor,fill opacity=0.20] ( 86.75, 73.98) circle (  2.13);

\path[fill=fillColor,fill opacity=0.20] ( 86.31, 91.68) circle (  2.13);

\path[fill=fillColor,fill opacity=0.20] ( 85.00, 91.43) circle (  2.13);

\path[fill=fillColor,fill opacity=0.20] ( 90.03, 78.66) circle (  2.13);

\path[fill=fillColor,fill opacity=0.20] ( 91.99, 73.34) circle (  2.13);

\path[fill=fillColor,fill opacity=0.20] ( 78.45, 72.21) circle (  2.13);

\path[fill=fillColor,fill opacity=0.20] ( 60.53, 68.16) circle (  2.13);

\path[fill=fillColor,fill opacity=0.20] ( 52.01, 56.78) circle (  2.13);

\path[fill=fillColor,fill opacity=0.20] ( 48.51, 47.68) circle (  2.13);

\path[fill=fillColor,fill opacity=0.20] ( 45.02, 39.71) circle (  2.13);

\path[fill=fillColor,fill opacity=0.20] ( 55.29, 52.10) circle (  2.13);

\path[fill=fillColor,fill opacity=0.20] ( 59.44, 65.38) circle (  2.13);

\path[fill=fillColor,fill opacity=0.20] ( 61.19, 78.66) circle (  2.13);

\path[fill=fillColor,fill opacity=0.20] ( 69.49, 75.12) circle (  2.13);

\path[fill=fillColor,fill opacity=0.20] ( 74.73, 74.86) circle (  2.13);

\path[fill=fillColor,fill opacity=0.20] ( 71.02, 77.14) circle (  2.13);

\path[fill=fillColor,fill opacity=0.20] ( 80.41, 70.56) circle (  2.13);

\path[fill=fillColor,fill opacity=0.20] ( 91.99, 76.13) circle (  2.13);

\path[fill=fillColor,fill opacity=0.20] ( 84.35, 86.24) circle (  2.13);

\path[fill=fillColor,fill opacity=0.20] ( 83.69, 80.17) circle (  2.13);

\path[fill=fillColor,fill opacity=0.20] (118.87, 77.90) circle (  2.13);

\path[fill=fillColor,fill opacity=0.20] ( 99.21, 88.27) circle (  2.13);

\path[fill=fillColor,fill opacity=0.20] ( 91.34, 85.86) circle (  2.13);

\path[fill=fillColor,fill opacity=0.20] ( 90.25, 73.60) circle (  2.13);

\path[fill=fillColor,fill opacity=0.20] ( 95.93, 70.44) circle (  2.13);

\path[fill=fillColor,fill opacity=0.20] ( 88.50, 74.48) circle (  2.13);

\path[fill=fillColor,fill opacity=0.20] (101.17, 77.39) circle (  2.13);

\path[fill=fillColor,fill opacity=0.20] (105.98, 76.00) circle (  2.13);

\path[fill=fillColor,fill opacity=0.20] ( 96.36, 75.75) circle (  2.13);

\path[fill=fillColor,fill opacity=0.20] ( 85.22, 82.70) circle (  2.13);

\path[fill=fillColor,fill opacity=0.20] ( 80.20, 90.04) circle (  2.13);

\path[fill=fillColor,fill opacity=0.20] ( 75.17, 88.77) circle (  2.13);

\path[fill=fillColor,fill opacity=0.20] ( 73.64, 80.30) circle (  2.13);

\path[fill=fillColor,fill opacity=0.20] ( 69.05, 70.69) circle (  2.13);

\path[fill=fillColor,fill opacity=0.20] ( 62.50, 62.22) circle (  2.13);

\path[fill=fillColor,fill opacity=0.20] ( 52.23, 49.70) circle (  2.13);

\path[fill=fillColor,fill opacity=0.20] ( 50.48, 51.97) circle (  2.13);

\path[fill=fillColor,fill opacity=0.20] ( 50.92, 58.04) circle (  2.13);

\path[fill=fillColor,fill opacity=0.20] ( 57.25, 59.06) circle (  2.13);

\path[fill=fillColor,fill opacity=0.20] ( 54.85, 65.13) circle (  2.13);

\path[fill=fillColor,fill opacity=0.20] ( 49.17, 73.47) circle (  2.13);

\path[fill=fillColor,fill opacity=0.20] ( 51.79, 75.37) circle (  2.13);

\path[fill=fillColor,fill opacity=0.20] (122.15, 79.54) circle (  2.13);

\path[fill=fillColor,fill opacity=0.20] ( 85.00, 99.27) circle (  2.13);

\path[fill=fillColor,fill opacity=0.20] ( 78.67, 93.32) circle (  2.13);

\path[fill=fillColor,fill opacity=0.20] ( 78.45, 76.63) circle (  2.13);

\path[fill=fillColor,fill opacity=0.20] ( 74.95, 80.05) circle (  2.13);

\path[fill=fillColor,fill opacity=0.20] ( 71.67, 87.25) circle (  2.13);

\path[fill=fillColor,fill opacity=0.20] ( 78.01, 80.17) circle (  2.13);

\path[fill=fillColor,fill opacity=0.20] ( 74.30, 80.30) circle (  2.13);

\path[fill=fillColor,fill opacity=0.20] ( 70.14, 96.11) circle (  2.13);

\path[fill=fillColor,fill opacity=0.20] ( 63.15, 97.24) circle (  2.13);

\path[fill=fillColor,fill opacity=0.20] ( 65.12, 75.37) circle (  2.13);

\path[fill=fillColor,fill opacity=0.20] (105.10, 63.10) circle (  2.13);

\path[fill=fillColor,fill opacity=0.20] ( 66.43, 57.79) circle (  2.13);

\path[fill=fillColor,fill opacity=0.20] ( 58.78, 54.00) circle (  2.13);

\path[fill=fillColor,fill opacity=0.20] ( 59.22, 71.83) circle (  2.13);

\path[fill=fillColor,fill opacity=0.20] ( 56.82, 63.48) circle (  2.13);

\path[fill=fillColor,fill opacity=0.20] ( 54.85, 65.13) circle (  2.13);

\path[fill=fillColor,fill opacity=0.20] ( 54.85, 65.38) circle (  2.13);

\path[fill=fillColor,fill opacity=0.20] ( 56.38, 58.68) circle (  2.13);

\path[fill=fillColor,fill opacity=0.20] ( 59.22, 59.69) circle (  2.13);

\path[fill=fillColor,fill opacity=0.20] ( 46.98, 63.61) circle (  2.13);

\path[fill=fillColor,fill opacity=0.20] ( 48.08, 57.92) circle (  2.13);

\path[fill=fillColor,fill opacity=0.20] ( 50.70, 48.81) circle (  2.13);

\path[fill=fillColor,fill opacity=0.20] ( 54.41, 41.73) circle (  2.13);

\path[fill=fillColor,fill opacity=0.20] ( 53.98, 40.72) circle (  2.13);

\path[fill=fillColor,fill opacity=0.20] ( 85.66, 60.83) circle (  2.13);

\path[fill=fillColor,fill opacity=0.20] (133.07, 63.48) circle (  2.13);

\path[fill=fillColor,fill opacity=0.20] ( 95.93, 65.50) circle (  2.13);

\path[fill=fillColor,fill opacity=0.20] ( 81.51, 66.14) circle (  2.13);

\path[fill=fillColor,fill opacity=0.20] ( 78.23, 61.33) circle (  2.13);

\path[fill=fillColor,fill opacity=0.20] ( 73.42, 47.80) circle (  2.13);

\path[fill=fillColor,fill opacity=0.20] ( 93.52, 62.85) circle (  2.13);

\path[fill=fillColor,fill opacity=0.20] (101.17, 72.08) circle (  2.13);

\path[fill=fillColor,fill opacity=0.20] ( 97.24, 81.31) circle (  2.13);

\path[fill=fillColor,fill opacity=0.20] ( 98.11, 87.00) circle (  2.13);

\path[fill=fillColor,fill opacity=0.20] ( 94.84, 81.82) circle (  2.13);

\path[fill=fillColor,fill opacity=0.20] ( 81.73, 77.77) circle (  2.13);

\path[fill=fillColor,fill opacity=0.20] ( 79.76, 71.70) circle (  2.13);

\path[fill=fillColor,fill opacity=0.20] ( 85.66, 56.27) circle (  2.13);

\path[fill=fillColor,fill opacity=0.20] ( 70.14, 45.53) circle (  2.13);

\path[fill=fillColor,fill opacity=0.20] ( 69.49, 40.97) circle (  2.13);

\path[fill=fillColor,fill opacity=0.20] ( 72.77, 49.19) circle (  2.13);

\path[fill=fillColor,fill opacity=0.20] ( 86.31, 72.21) circle (  2.13);

\path[fill=fillColor,fill opacity=0.20] ( 92.87, 83.97) circle (  2.13);

\path[fill=fillColor,fill opacity=0.20] ( 99.64, 88.14) circle (  2.13);

\path[fill=fillColor,fill opacity=0.20] ( 95.93, 95.73) circle (  2.13);

\path[fill=fillColor,fill opacity=0.20] ( 92.21, 97.88) circle (  2.13);

\path[fill=fillColor,fill opacity=0.20] ( 89.15, 95.35) circle (  2.13);

\path[fill=fillColor,fill opacity=0.20] ( 80.20, 89.40) circle (  2.13);

\path[fill=fillColor,fill opacity=0.20] ( 81.29, 82.20) circle (  2.13);

\path[fill=fillColor,fill opacity=0.20] ( 77.14, 76.13) circle (  2.13);

\path[fill=fillColor,fill opacity=0.20] ( 79.32, 68.41) circle (  2.13);

\path[fill=fillColor,fill opacity=0.20] ( 78.23, 49.07) circle (  2.13);

\path[fill=fillColor,fill opacity=0.20] ( 60.31, 46.16) circle (  2.13);

\path[fill=fillColor,fill opacity=0.20] ( 84.35, 74.61) circle (  2.13);

\path[fill=fillColor,fill opacity=0.20] ( 95.05, 92.69) circle (  2.13);

\path[fill=fillColor,fill opacity=0.20] ( 96.58, 99.01) circle (  2.13);

\path[fill=fillColor,fill opacity=0.20] ( 96.15, 98.26) circle (  2.13);

\path[fill=fillColor,fill opacity=0.20] ( 91.12,100.15) circle (  2.13);

\path[fill=fillColor,fill opacity=0.20] ( 87.62, 98.38) circle (  2.13);

\path[fill=fillColor,fill opacity=0.20] ( 85.00, 98.00) circle (  2.13);

\path[fill=fillColor,fill opacity=0.20] ( 82.38, 95.09) circle (  2.13);

\path[fill=fillColor,fill opacity=0.20] ( 77.57, 86.62) circle (  2.13);

\path[fill=fillColor,fill opacity=0.20] ( 73.64, 74.36) circle (  2.13);

\path[fill=fillColor,fill opacity=0.20] ( 74.51, 57.79) circle (  2.13);

\path[fill=fillColor,fill opacity=0.20] ( 62.93, 46.41) circle (  2.13);

\path[fill=fillColor,fill opacity=0.20] ( 62.28, 39.96) circle (  2.13);

\path[fill=fillColor,fill opacity=0.20] ( 95.93, 86.87) circle (  2.13);

\path[fill=fillColor,fill opacity=0.20] ( 94.84, 92.06) circle (  2.13);

\path[fill=fillColor,fill opacity=0.20] ( 90.90, 99.65) circle (  2.13);

\path[fill=fillColor,fill opacity=0.20] ( 87.19, 97.37) circle (  2.13);

\path[fill=fillColor,fill opacity=0.20] ( 85.44, 94.34) circle (  2.13);

\path[fill=fillColor,fill opacity=0.20] ( 81.73, 94.34) circle (  2.13);

\path[fill=fillColor,fill opacity=0.20] ( 76.26, 92.44) circle (  2.13);

\path[fill=fillColor,fill opacity=0.20] ( 78.67, 90.42) circle (  2.13);

\path[fill=fillColor,fill opacity=0.20] ( 76.70, 90.67) circle (  2.13);

\path[fill=fillColor,fill opacity=0.20] ( 64.90, 80.43) circle (  2.13);

\path[fill=fillColor,fill opacity=0.20] ( 60.53, 61.84) circle (  2.13);

\path[fill=fillColor,fill opacity=0.20] ( 61.84, 50.84) circle (  2.13);

\path[fill=fillColor,fill opacity=0.20] ( 74.08, 71.95) circle (  2.13);

\path[fill=fillColor,fill opacity=0.20] ( 96.36, 93.96) circle (  2.13);

\path[fill=fillColor,fill opacity=0.20] ( 92.43, 93.07) circle (  2.13);

\path[fill=fillColor,fill opacity=0.20] ( 95.49,102.18) circle (  2.13);

\path[fill=fillColor,fill opacity=0.20] ( 83.69, 97.88) circle (  2.13);

\path[fill=fillColor,fill opacity=0.20] ( 76.70, 86.75) circle (  2.13);

\path[fill=fillColor,fill opacity=0.20] ( 79.32, 86.50) circle (  2.13);

\path[fill=fillColor,fill opacity=0.20] ( 76.92, 87.63) circle (  2.13);

\path[fill=fillColor,fill opacity=0.20] ( 73.42, 83.33) circle (  2.13);

\path[fill=fillColor,fill opacity=0.20] ( 67.74, 85.61) circle (  2.13);

\path[fill=fillColor,fill opacity=0.20] ( 53.98, 80.81) circle (  2.13);

\path[fill=fillColor,fill opacity=0.20] ( 68.62, 54.88) circle (  2.13);

\path[fill=fillColor,fill opacity=0.20] ( 80.20, 81.56) circle (  2.13);

\path[fill=fillColor,fill opacity=0.20] ( 91.78,100.91) circle (  2.13);

\path[fill=fillColor,fill opacity=0.20] ( 91.12,100.03) circle (  2.13);

\path[fill=fillColor,fill opacity=0.20] (104.01,103.19) circle (  2.13);

\path[fill=fillColor,fill opacity=0.20] ( 84.13, 97.75) circle (  2.13);

\path[fill=fillColor,fill opacity=0.20] ( 77.14, 88.14) circle (  2.13);

\path[fill=fillColor,fill opacity=0.20] ( 74.08, 82.45) circle (  2.13);

\path[fill=fillColor,fill opacity=0.20] ( 71.89, 84.47) circle (  2.13);

\path[fill=fillColor,fill opacity=0.20] ( 67.96, 77.52) circle (  2.13);

\path[fill=fillColor,fill opacity=0.20] ( 56.16, 65.38) circle (  2.13);

\path[fill=fillColor,fill opacity=0.20] ( 72.77, 81.94) circle (  2.13);

\path[fill=fillColor,fill opacity=0.20] ( 79.10, 83.21) circle (  2.13);

\path[fill=fillColor,fill opacity=0.20] ( 77.57, 79.54) circle (  2.13);

\path[fill=fillColor,fill opacity=0.20] ( 72.33, 71.07) circle (  2.13);

\path[fill=fillColor,fill opacity=0.20] ( 76.04, 75.12) circle (  2.13);

\path[fill=fillColor,fill opacity=0.20] ( 93.31, 92.31) circle (  2.13);

\path[fill=fillColor,fill opacity=0.20] ( 89.59, 94.34) circle (  2.13);

\path[fill=fillColor,fill opacity=0.20] ( 85.44, 97.50) circle (  2.13);

\path[fill=fillColor,fill opacity=0.20] ( 78.45, 95.09) circle (  2.13);

\path[fill=fillColor,fill opacity=0.20] ( 78.45, 91.55) circle (  2.13);

\path[fill=fillColor,fill opacity=0.20] ( 73.86, 84.85) circle (  2.13);

\path[fill=fillColor,fill opacity=0.20] ( 68.62, 77.64) circle (  2.13);

\path[fill=fillColor,fill opacity=0.20] ( 61.62, 67.28) circle (  2.13);

\path[fill=fillColor,fill opacity=0.20] ( 50.04, 52.10) circle (  2.13);

\path[fill=fillColor,fill opacity=0.20] ( 76.92, 77.52) circle (  2.13);

\path[fill=fillColor,fill opacity=0.20] ( 74.30, 85.74) circle (  2.13);

\path[fill=fillColor,fill opacity=0.20] ( 79.98, 75.24) circle (  2.13);

\path[fill=fillColor,fill opacity=0.20] ( 88.06, 74.86) circle (  2.13);

\path[fill=fillColor,fill opacity=0.20] ( 79.76, 75.24) circle (  2.13);

\path[fill=fillColor,fill opacity=0.20] ( 78.67, 75.62) circle (  2.13);

\path[fill=fillColor,fill opacity=0.20] ( 77.14, 81.82) circle (  2.13);

\path[fill=fillColor,fill opacity=0.20] ( 66.87, 48.81) circle (  2.13);

\path[fill=fillColor,fill opacity=0.20] ( 64.68, 69.05) circle (  2.13);

\path[fill=fillColor,fill opacity=0.20] (112.10, 87.89) circle (  2.13);

\path[fill=fillColor,fill opacity=0.20] ( 94.84, 92.19) circle (  2.13);

\path[fill=fillColor,fill opacity=0.20] ( 84.13, 99.52) circle (  2.13);

\path[fill=fillColor,fill opacity=0.20] ( 82.60, 95.73) circle (  2.13);

\path[fill=fillColor,fill opacity=0.20] ( 83.25, 91.55) circle (  2.13);

\path[fill=fillColor,fill opacity=0.20] ( 72.77, 87.13) circle (  2.13);

\path[fill=fillColor,fill opacity=0.20] ( 66.21, 58.04) circle (  2.13);

\path[fill=fillColor,fill opacity=0.20] ( 86.97, 87.13) circle (  2.13);

\path[fill=fillColor,fill opacity=0.20] ( 84.13, 85.61) circle (  2.13);

\path[fill=fillColor,fill opacity=0.20] ( 88.72, 81.94) circle (  2.13);

\path[fill=fillColor,fill opacity=0.20] ( 86.97, 84.60) circle (  2.13);

\path[fill=fillColor,fill opacity=0.20] ( 86.10, 82.20) circle (  2.13);

\path[fill=fillColor,fill opacity=0.20] ( 81.94, 76.00) circle (  2.13);

\path[fill=fillColor,fill opacity=0.20] ( 75.83, 73.72) circle (  2.13);

\path[fill=fillColor,fill opacity=0.20] ( 78.88, 60.95) circle (  2.13);

\path[fill=fillColor,fill opacity=0.20] ( 59.88, 67.15) circle (  2.13);

\path[fill=fillColor,fill opacity=0.20] (110.57,101.16) circle (  2.13);

\path[fill=fillColor,fill opacity=0.20] (114.28, 98.76) circle (  2.13);

\path[fill=fillColor,fill opacity=0.20] ( 88.06,104.70) circle (  2.13);

\path[fill=fillColor,fill opacity=0.20] ( 91.99, 97.50) circle (  2.13);

\path[fill=fillColor,fill opacity=0.20] ( 89.15, 87.89) circle (  2.13);

\path[fill=fillColor,fill opacity=0.20] ( 72.33, 88.65) circle (  2.13);

\path[fill=fillColor,fill opacity=0.20] ( 75.83, 74.36) circle (  2.13);

\path[fill=fillColor,fill opacity=0.20] ( 68.40, 56.91) circle (  2.13);

\path[fill=fillColor,fill opacity=0.20] ( 86.53, 78.28) circle (  2.13);

\path[fill=fillColor,fill opacity=0.20] ( 94.84, 94.08) circle (  2.13);

\path[fill=fillColor,fill opacity=0.20] ( 94.84, 92.69) circle (  2.13);

\path[fill=fillColor,fill opacity=0.20] ( 90.90, 95.60) circle (  2.13);

\path[fill=fillColor,fill opacity=0.20] ( 81.51, 95.47) circle (  2.13);

\path[fill=fillColor,fill opacity=0.20] ( 85.44, 97.24) circle (  2.13);

\path[fill=fillColor,fill opacity=0.20] ( 86.53, 88.52) circle (  2.13);

\path[fill=fillColor,fill opacity=0.20] ( 77.36, 76.38) circle (  2.13);

\path[fill=fillColor,fill opacity=0.20] ( 84.78, 68.03) circle (  2.13);

\path[fill=fillColor,fill opacity=0.20] ( 59.66, 51.97) circle (  2.13);

\path[fill=fillColor,fill opacity=0.20] ( 97.68,101.29) circle (  2.13);

\path[fill=fillColor,fill opacity=0.20] ( 98.77,100.03) circle (  2.13);

\path[fill=fillColor,fill opacity=0.20] ( 84.78, 96.61) circle (  2.13);

\path[fill=fillColor,fill opacity=0.20] ( 85.66, 92.82) circle (  2.13);

\path[fill=fillColor,fill opacity=0.20] ( 82.16, 86.50) circle (  2.13);

\path[fill=fillColor,fill opacity=0.20] ( 74.95, 88.39) circle (  2.13);

\path[fill=fillColor,fill opacity=0.20] ( 74.51, 84.22) circle (  2.13);

\path[fill=fillColor,fill opacity=0.20] ( 68.62, 68.79) circle (  2.13);

\path[fill=fillColor,fill opacity=0.20] ( 94.62, 92.44) circle (  2.13);

\path[fill=fillColor,fill opacity=0.20] (102.26,101.16) circle (  2.13);

\path[fill=fillColor,fill opacity=0.20] ( 94.62, 90.42) circle (  2.13);

\path[fill=fillColor,fill opacity=0.20] ( 92.21, 94.21) circle (  2.13);

\path[fill=fillColor,fill opacity=0.20] ( 92.43, 98.13) circle (  2.13);

\path[fill=fillColor,fill opacity=0.20] ( 87.41,105.46) circle (  2.13);

\path[fill=fillColor,fill opacity=0.20] ( 91.12, 97.12) circle (  2.13);

\path[fill=fillColor,fill opacity=0.20] ( 81.94, 83.84) circle (  2.13);

\path[fill=fillColor,fill opacity=0.20] ( 78.67, 79.16) circle (  2.13);

\path[fill=fillColor,fill opacity=0.20] ( 91.56, 50.96) circle (  2.13);

\path[fill=fillColor,fill opacity=0.20] ( 65.56, 65.76) circle (  2.13);

\path[fill=fillColor,fill opacity=0.20] ( 82.60, 89.91) circle (  2.13);

\path[fill=fillColor,fill opacity=0.20] ( 89.81, 87.38) circle (  2.13);

\path[fill=fillColor,fill opacity=0.20] ( 85.88, 86.62) circle (  2.13);

\path[fill=fillColor,fill opacity=0.20] ( 82.60, 89.53) circle (  2.13);

\path[fill=fillColor,fill opacity=0.20] ( 81.29, 89.02) circle (  2.13);

\path[fill=fillColor,fill opacity=0.20] ( 71.89, 85.10) circle (  2.13);

\path[fill=fillColor,fill opacity=0.20] ( 66.65, 73.34) circle (  2.13);

\path[fill=fillColor,fill opacity=0.20] (104.67, 94.34) circle (  2.13);

\path[fill=fillColor,fill opacity=0.20] ( 98.11, 93.58) circle (  2.13);

\path[fill=fillColor,fill opacity=0.20] ( 91.99, 85.48) circle (  2.13);

\path[fill=fillColor,fill opacity=0.20] ( 96.58, 94.34) circle (  2.13);

\path[fill=fillColor,fill opacity=0.20] ( 99.21,101.67) circle (  2.13);

\path[fill=fillColor,fill opacity=0.20] ( 92.87,104.58) circle (  2.13);

\path[fill=fillColor,fill opacity=0.20] ( 99.21, 94.08) circle (  2.13);

\path[fill=fillColor,fill opacity=0.20] ( 88.06, 83.21) circle (  2.13);

\path[fill=fillColor,fill opacity=0.20] ( 80.41, 84.98) circle (  2.13);

\path[fill=fillColor,fill opacity=0.20] ( 78.23, 65.13) circle (  2.13);

\path[fill=fillColor,fill opacity=0.20] ( 76.04, 77.26) circle (  2.13);

\path[fill=fillColor,fill opacity=0.20] ( 87.84, 91.81) circle (  2.13);

\path[fill=fillColor,fill opacity=0.20] ( 90.47, 91.68) circle (  2.13);

\path[fill=fillColor,fill opacity=0.20] ( 77.79, 92.44) circle (  2.13);

\path[fill=fillColor,fill opacity=0.20] ( 79.32, 85.61) circle (  2.13);

\path[fill=fillColor,fill opacity=0.20] ( 75.17, 72.97) circle (  2.13);

\path[fill=fillColor,fill opacity=0.20] ( 65.34, 63.61) circle (  2.13);

\path[fill=fillColor,fill opacity=0.20] ( 67.74, 58.30) circle (  2.13);

\path[fill=fillColor,fill opacity=0.20] ( 96.36, 94.08) circle (  2.13);

\path[fill=fillColor,fill opacity=0.20] ( 87.41, 91.05) circle (  2.13);

\path[fill=fillColor,fill opacity=0.20] ( 89.59, 91.81) circle (  2.13);

\path[fill=fillColor,fill opacity=0.20] ( 92.43, 99.77) circle (  2.13);

\path[fill=fillColor,fill opacity=0.20] ( 93.52, 97.50) circle (  2.13);

\path[fill=fillColor,fill opacity=0.20] ( 87.41, 94.97) circle (  2.13);

\path[fill=fillColor,fill opacity=0.20] ( 93.31, 91.17) circle (  2.13);

\path[fill=fillColor,fill opacity=0.20] ( 92.65, 84.85) circle (  2.13);

\path[fill=fillColor,fill opacity=0.20] ( 80.41, 84.98) circle (  2.13);

\path[fill=fillColor,fill opacity=0.20] ( 89.37, 81.31) circle (  2.13);

\path[fill=fillColor,fill opacity=0.20] ( 46.55, 52.23) circle (  2.13);

\path[fill=fillColor,fill opacity=0.20] ( 66.43, 80.17) circle (  2.13);

\path[fill=fillColor,fill opacity=0.20] ( 83.04, 97.88) circle (  2.13);

\path[fill=fillColor,fill opacity=0.20] ( 82.16, 85.99) circle (  2.13);

\path[fill=fillColor,fill opacity=0.20] ( 76.70, 76.89) circle (  2.13);

\path[fill=fillColor,fill opacity=0.20] ( 78.67, 70.94) circle (  2.13);

\path[fill=fillColor,fill opacity=0.20] ( 72.77, 64.24) circle (  2.13);

\path[fill=fillColor,fill opacity=0.20] ( 69.71, 60.95) circle (  2.13);

\path[fill=fillColor,fill opacity=0.20] ( 66.87, 55.14) circle (  2.13);

\path[fill=fillColor,fill opacity=0.20] ( 78.88, 78.40) circle (  2.13);

\path[fill=fillColor,fill opacity=0.20] ( 84.78, 97.62) circle (  2.13);

\path[fill=fillColor,fill opacity=0.20] ( 83.91,104.45) circle (  2.13);

\path[fill=fillColor,fill opacity=0.20] ( 91.12,103.06) circle (  2.13);

\path[fill=fillColor,fill opacity=0.20] ( 93.31, 95.98) circle (  2.13);

\path[fill=fillColor,fill opacity=0.20] (100.08, 90.79) circle (  2.13);

\path[fill=fillColor,fill opacity=0.20] ( 98.55, 90.29) circle (  2.13);

\path[fill=fillColor,fill opacity=0.20] ( 95.49, 92.19) circle (  2.13);

\path[fill=fillColor,fill opacity=0.20] ( 91.12, 87.63) circle (  2.13);

\path[fill=fillColor,fill opacity=0.20] ( 84.13, 81.18) circle (  2.13);

\path[fill=fillColor,fill opacity=0.20] ( 82.60, 73.34) circle (  2.13);

\path[fill=fillColor,fill opacity=0.20] ( 74.51, 48.05) circle (  2.13);

\path[fill=fillColor,fill opacity=0.20] ( 52.45, 80.81) circle (  2.13);

\path[fill=fillColor,fill opacity=0.20] ( 71.89, 81.56) circle (  2.13);

\path[fill=fillColor,fill opacity=0.20] ( 80.63, 73.98) circle (  2.13);

\path[fill=fillColor,fill opacity=0.20] ( 86.10, 77.64) circle (  2.13);

\path[fill=fillColor,fill opacity=0.20] ( 78.01, 75.87) circle (  2.13);

\path[fill=fillColor,fill opacity=0.20] ( 72.33, 63.73) circle (  2.13);

\path[fill=fillColor,fill opacity=0.20] ( 68.62, 56.27) circle (  2.13);

\path[fill=fillColor,fill opacity=0.20] ( 70.80, 48.56) circle (  2.13);

\path[fill=fillColor,fill opacity=0.20] ( 76.26, 83.33) circle (  2.13);

\path[fill=fillColor,fill opacity=0.20] ( 81.07, 88.01) circle (  2.13);

\path[fill=fillColor,fill opacity=0.20] ( 90.90,101.67) circle (  2.13);

\path[fill=fillColor,fill opacity=0.20] ( 95.05,111.91) circle (  2.13);

\path[fill=fillColor,fill opacity=0.20] ( 94.18,101.80) circle (  2.13);

\path[fill=fillColor,fill opacity=0.20] ( 97.89, 93.07) circle (  2.13);

\path[fill=fillColor,fill opacity=0.20] (104.45, 93.20) circle (  2.13);

\path[fill=fillColor,fill opacity=0.20] (107.07, 96.49) circle (  2.13);

\path[fill=fillColor,fill opacity=0.20] ( 99.42, 94.97) circle (  2.13);

\path[fill=fillColor,fill opacity=0.20] ( 93.96, 86.62) circle (  2.13);

\path[fill=fillColor,fill opacity=0.20] ( 91.12, 79.41) circle (  2.13);

\path[fill=fillColor,fill opacity=0.20] ( 81.07, 67.53) circle (  2.13);

\path[fill=fillColor,fill opacity=0.20] ( 52.88, 44.64) circle (  2.13);

\path[fill=fillColor,fill opacity=0.20] ( 54.19, 73.98) circle (  2.13);

\path[fill=fillColor,fill opacity=0.20] ( 77.36, 78.53) circle (  2.13);

\path[fill=fillColor,fill opacity=0.20] ( 90.90, 78.15) circle (  2.13);

\path[fill=fillColor,fill opacity=0.20] ( 79.98, 77.52) circle (  2.13);

\path[fill=fillColor,fill opacity=0.20] ( 75.61, 69.17) circle (  2.13);

\path[fill=fillColor,fill opacity=0.20] ( 73.42, 59.44) circle (  2.13);

\path[fill=fillColor,fill opacity=0.20] ( 69.49, 59.81) circle (  2.13);

\path[fill=fillColor,fill opacity=0.20] ( 82.82, 74.99) circle (  2.13);

\path[fill=fillColor,fill opacity=0.20] ( 89.15, 87.13) circle (  2.13);

\path[fill=fillColor,fill opacity=0.20] ( 86.75, 90.42) circle (  2.13);

\path[fill=fillColor,fill opacity=0.20] ( 97.02, 97.75) circle (  2.13);

\path[fill=fillColor,fill opacity=0.20] (101.83, 93.58) circle (  2.13);

\path[fill=fillColor,fill opacity=0.20] (109.91, 91.93) circle (  2.13);

\path[fill=fillColor,fill opacity=0.20] (105.10, 96.61) circle (  2.13);

\path[fill=fillColor,fill opacity=0.20] (104.45,100.40) circle (  2.13);

\path[fill=fillColor,fill opacity=0.20] (100.52,103.69) circle (  2.13);

\path[fill=fillColor,fill opacity=0.20] (100.08,100.91) circle (  2.13);

\path[fill=fillColor,fill opacity=0.20] (101.39, 87.89) circle (  2.13);

\path[fill=fillColor,fill opacity=0.20] ( 94.18, 84.09) circle (  2.13);

\path[fill=fillColor,fill opacity=0.20] ( 83.04, 74.86) circle (  2.13);

\path[fill=fillColor,fill opacity=0.20] ( 53.32, 44.26) circle (  2.13);

\path[fill=fillColor,fill opacity=0.20] ( 80.85, 68.16) circle (  2.13);

\path[fill=fillColor,fill opacity=0.20] ( 76.26, 73.85) circle (  2.13);

\path[fill=fillColor,fill opacity=0.20] ( 85.66, 72.08) circle (  2.13);

\path[fill=fillColor,fill opacity=0.20] ( 79.54, 72.97) circle (  2.13);

\path[fill=fillColor,fill opacity=0.20] ( 76.70, 70.06) circle (  2.13);

\path[fill=fillColor,fill opacity=0.20] ( 74.51, 68.67) circle (  2.13);

\path[fill=fillColor,fill opacity=0.20] ( 72.99, 64.11) circle (  2.13);

\path[fill=fillColor,fill opacity=0.20] ( 80.85, 74.48) circle (  2.13);

\path[fill=fillColor,fill opacity=0.20] ( 88.50, 81.82) circle (  2.13);

\path[fill=fillColor,fill opacity=0.20] ( 90.68, 87.89) circle (  2.13);

\path[fill=fillColor,fill opacity=0.20] ( 92.65, 99.27) circle (  2.13);

\path[fill=fillColor,fill opacity=0.20] ( 97.46, 93.32) circle (  2.13);

\path[fill=fillColor,fill opacity=0.20] ( 96.15, 82.20) circle (  2.13);

\path[fill=fillColor,fill opacity=0.20] (106.85, 86.24) circle (  2.13);

\path[fill=fillColor,fill opacity=0.20] (105.76, 99.01) circle (  2.13);

\path[fill=fillColor,fill opacity=0.20] ( 97.02,106.22) circle (  2.13);

\path[fill=fillColor,fill opacity=0.20] ( 99.42,109.51) circle (  2.13);

\path[fill=fillColor,fill opacity=0.20] ( 99.86, 99.65) circle (  2.13);

\path[fill=fillColor,fill opacity=0.20] (102.48, 86.62) circle (  2.13);

\path[fill=fillColor,fill opacity=0.20] ( 89.37, 85.36) circle (  2.13);

\path[fill=fillColor,fill opacity=0.20] ( 74.30, 71.95) circle (  2.13);

\path[fill=fillColor,fill opacity=0.20] ( 48.08, 41.73) circle (  2.13);

\path[fill=fillColor,fill opacity=0.20] ( 62.72, 61.84) circle (  2.13);

\path[fill=fillColor,fill opacity=0.20] ( 71.89, 70.56) circle (  2.13);

\path[fill=fillColor,fill opacity=0.20] ( 87.19, 69.42) circle (  2.13);

\path[fill=fillColor,fill opacity=0.20] ( 81.07, 74.23) circle (  2.13);

\path[fill=fillColor,fill opacity=0.20] ( 78.01, 73.72) circle (  2.13);

\path[fill=fillColor,fill opacity=0.20] ( 82.60, 66.64) circle (  2.13);

\path[fill=fillColor,fill opacity=0.20] ( 73.20, 60.83) circle (  2.13);

\path[fill=fillColor,fill opacity=0.20] ( 64.90, 52.86) circle (  2.13);

\path[fill=fillColor,fill opacity=0.20] ( 73.20, 70.82) circle (  2.13);

\path[fill=fillColor,fill opacity=0.20] ( 86.97, 83.33) circle (  2.13);

\path[fill=fillColor,fill opacity=0.20] ( 97.24, 88.90) circle (  2.13);

\path[fill=fillColor,fill opacity=0.20] ( 91.34, 96.61) circle (  2.13);

\path[fill=fillColor,fill opacity=0.20] (101.17,100.15) circle (  2.13);

\path[fill=fillColor,fill opacity=0.20] (100.73, 93.83) circle (  2.13);

\path[fill=fillColor,fill opacity=0.20] (102.05, 88.27) circle (  2.13);

\path[fill=fillColor,fill opacity=0.20] (104.45, 90.16) circle (  2.13);

\path[fill=fillColor,fill opacity=0.20] (105.32, 98.89) circle (  2.13);

\path[fill=fillColor,fill opacity=0.20] (100.08,109.51) circle (  2.13);

\path[fill=fillColor,fill opacity=0.20] (103.79,103.19) circle (  2.13);

\path[fill=fillColor,fill opacity=0.20] (105.76, 88.90) circle (  2.13);

\path[fill=fillColor,fill opacity=0.20] (102.48, 83.59) circle (  2.13);

\path[fill=fillColor,fill opacity=0.20] ( 86.53, 77.64) circle (  2.13);

\path[fill=fillColor,fill opacity=0.20] ( 69.49, 56.78) circle (  2.13);

\path[fill=fillColor,fill opacity=0.20] (104.23, 66.14) circle (  2.13);

\path[fill=fillColor,fill opacity=0.20] ( 86.75, 75.87) circle (  2.13);

\path[fill=fillColor,fill opacity=0.20] ( 84.13, 78.53) circle (  2.13);

\path[fill=fillColor,fill opacity=0.20] ( 81.29, 70.69) circle (  2.13);

\path[fill=fillColor,fill opacity=0.20] ( 72.11, 67.78) circle (  2.13);

\path[fill=fillColor,fill opacity=0.20] ( 73.20, 65.76) circle (  2.13);

\path[fill=fillColor,fill opacity=0.20] ( 63.37, 48.05) circle (  2.13);

\path[fill=fillColor,fill opacity=0.20] ( 72.77, 68.16) circle (  2.13);

\path[fill=fillColor,fill opacity=0.20] ( 75.83, 73.85) circle (  2.13);

\path[fill=fillColor,fill opacity=0.20] ( 95.71, 85.36) circle (  2.13);

\path[fill=fillColor,fill opacity=0.20] ( 98.77, 92.69) circle (  2.13);

\path[fill=fillColor,fill opacity=0.20] ( 90.68, 93.83) circle (  2.13);

\path[fill=fillColor,fill opacity=0.20] ( 96.80, 97.75) circle (  2.13);

\path[fill=fillColor,fill opacity=0.20] ( 99.86, 99.39) circle (  2.13);

\path[fill=fillColor,fill opacity=0.20] (107.29, 98.51) circle (  2.13);

\path[fill=fillColor,fill opacity=0.20] (101.83, 97.24) circle (  2.13);

\path[fill=fillColor,fill opacity=0.20] ( 98.77,101.67) circle (  2.13);

\path[fill=fillColor,fill opacity=0.20] (105.32,103.44) circle (  2.13);

\path[fill=fillColor,fill opacity=0.20] (130.45, 92.06) circle (  2.13);

\path[fill=fillColor,fill opacity=0.20] (100.30, 86.12) circle (  2.13);

\path[fill=fillColor,fill opacity=0.20] ( 88.50, 82.70) circle (  2.13);

\path[fill=fillColor,fill opacity=0.20] ( 68.62, 64.24) circle (  2.13);

\path[fill=fillColor,fill opacity=0.20] ( 68.62, 57.79) circle (  2.13);

\path[fill=fillColor,fill opacity=0.20] ( 72.33, 80.81) circle (  2.13);

\path[fill=fillColor,fill opacity=0.20] ( 78.88, 80.93) circle (  2.13);

\path[fill=fillColor,fill opacity=0.20] ( 76.26, 73.09) circle (  2.13);

\path[fill=fillColor,fill opacity=0.20] ( 74.08, 79.16) circle (  2.13);

\path[fill=fillColor,fill opacity=0.20] ( 73.42, 74.23) circle (  2.13);

\path[fill=fillColor,fill opacity=0.20] ( 69.49, 60.57) circle (  2.13);

\path[fill=fillColor,fill opacity=0.20] ( 69.27, 52.73) circle (  2.13);

\path[fill=fillColor,fill opacity=0.20] ( 77.14, 70.82) circle (  2.13);

\path[fill=fillColor,fill opacity=0.20] ( 80.20, 79.03) circle (  2.13);

\path[fill=fillColor,fill opacity=0.20] ( 87.19, 78.66) circle (  2.13);

\path[fill=fillColor,fill opacity=0.20] ( 96.58, 85.99) circle (  2.13);

\path[fill=fillColor,fill opacity=0.20] (101.17, 91.93) circle (  2.13);

\path[fill=fillColor,fill opacity=0.20] ( 98.99, 91.68) circle (  2.13);

\path[fill=fillColor,fill opacity=0.20] ( 97.24,100.91) circle (  2.13);

\path[fill=fillColor,fill opacity=0.20] ( 99.86,106.35) circle (  2.13);

\path[fill=fillColor,fill opacity=0.20] (101.83, 97.12) circle (  2.13);

\path[fill=fillColor,fill opacity=0.20] (138.75, 96.11) circle (  2.13);

\path[fill=fillColor,fill opacity=0.20] (101.17,101.67) circle (  2.13);

\path[fill=fillColor,fill opacity=0.20] ( 93.96, 97.75) circle (  2.13);

\path[fill=fillColor,fill opacity=0.20] ( 83.69, 91.68) circle (  2.13);

\path[fill=fillColor,fill opacity=0.20] ( 83.47, 88.27) circle (  2.13);

\path[fill=fillColor,fill opacity=0.20] ( 88.28, 70.82) circle (  2.13);

\path[fill=fillColor,fill opacity=0.20] ( 55.51, 56.65) circle (  2.13);

\path[fill=fillColor,fill opacity=0.20] ( 68.83, 72.59) circle (  2.13);

\path[fill=fillColor,fill opacity=0.20] ( 77.36, 79.92) circle (  2.13);

\path[fill=fillColor,fill opacity=0.20] ( 77.79, 84.85) circle (  2.13);

\path[fill=fillColor,fill opacity=0.20] ( 81.94, 85.48) circle (  2.13);

\path[fill=fillColor,fill opacity=0.20] ( 78.88, 81.82) circle (  2.13);

\path[fill=fillColor,fill opacity=0.20] ( 75.61, 72.59) circle (  2.13);

\path[fill=fillColor,fill opacity=0.20] ( 71.02, 61.84) circle (  2.13);

\path[fill=fillColor,fill opacity=0.20] ( 85.22, 81.82) circle (  2.13);

\path[fill=fillColor,fill opacity=0.20] ( 91.78, 91.17) circle (  2.13);

\path[fill=fillColor,fill opacity=0.20] ( 97.68, 85.86) circle (  2.13);

\path[fill=fillColor,fill opacity=0.20] ( 98.11, 87.63) circle (  2.13);

\path[fill=fillColor,fill opacity=0.20] (102.92, 94.21) circle (  2.13);

\path[fill=fillColor,fill opacity=0.20] (104.23, 95.98) circle (  2.13);

\path[fill=fillColor,fill opacity=0.20] (111.66,104.83) circle (  2.13);

\path[fill=fillColor,fill opacity=0.20] (105.32,109.13) circle (  2.13);

\path[fill=fillColor,fill opacity=0.20] ( 98.77, 95.98) circle (  2.13);

\path[fill=fillColor,fill opacity=0.20] ( 99.86, 94.08) circle (  2.13);

\path[fill=fillColor,fill opacity=0.20] ( 94.40, 97.24) circle (  2.13);

\path[fill=fillColor,fill opacity=0.20] ( 81.07, 87.63) circle (  2.13);

\path[fill=fillColor,fill opacity=0.20] ( 73.42, 80.55) circle (  2.13);

\path[fill=fillColor,fill opacity=0.20] ( 57.91, 69.42) circle (  2.13);

\path[fill=fillColor,fill opacity=0.20] ( 51.57, 48.94) circle (  2.13);

\path[fill=fillColor,fill opacity=0.20] ( 93.09, 75.62) circle (  2.13);

\path[fill=fillColor,fill opacity=0.20] ( 80.20, 82.70) circle (  2.13);

\path[fill=fillColor,fill opacity=0.20] ( 81.73, 83.71) circle (  2.13);

\path[fill=fillColor,fill opacity=0.20] ( 75.83, 86.75) circle (  2.13);

\path[fill=fillColor,fill opacity=0.20] ( 74.51, 79.54) circle (  2.13);

\path[fill=fillColor,fill opacity=0.20] ( 69.05, 71.20) circle (  2.13);

\path[fill=fillColor,fill opacity=0.20] ( 68.18, 65.00) circle (  2.13);

\path[fill=fillColor,fill opacity=0.20] ( 74.73, 66.90) circle (  2.13);

\path[fill=fillColor,fill opacity=0.20] ( 74.51, 69.42) circle (  2.13);

\path[fill=fillColor,fill opacity=0.20] ( 83.47, 74.10) circle (  2.13);

\path[fill=fillColor,fill opacity=0.20] ( 93.96, 85.61) circle (  2.13);

\path[fill=fillColor,fill opacity=0.20] ( 97.24, 90.54) circle (  2.13);

\path[fill=fillColor,fill opacity=0.20] ( 95.49, 89.78) circle (  2.13);

\path[fill=fillColor,fill opacity=0.20] (100.08, 93.45) circle (  2.13);

\path[fill=fillColor,fill opacity=0.20] ( 98.33, 95.98) circle (  2.13);

\path[fill=fillColor,fill opacity=0.20] ( 99.21, 94.97) circle (  2.13);

\path[fill=fillColor,fill opacity=0.20] (111.22, 98.26) circle (  2.13);

\path[fill=fillColor,fill opacity=0.20] ( 92.87, 98.76) circle (  2.13);

\path[fill=fillColor,fill opacity=0.20] ( 86.10, 94.21) circle (  2.13);

\path[fill=fillColor,fill opacity=0.20] ( 77.14, 86.50) circle (  2.13);

\path[fill=fillColor,fill opacity=0.20] ( 70.36, 65.25) circle (  2.13);

\path[fill=fillColor,fill opacity=0.20] ( 46.98, 50.71) circle (  2.13);

\path[fill=fillColor,fill opacity=0.20] ( 82.82, 72.21) circle (  2.13);

\path[fill=fillColor,fill opacity=0.20] ( 82.60, 76.00) circle (  2.13);

\path[fill=fillColor,fill opacity=0.20] ( 76.70, 72.59) circle (  2.13);

\path[fill=fillColor,fill opacity=0.20] ( 74.73, 72.71) circle (  2.13);

\path[fill=fillColor,fill opacity=0.20] ( 71.89, 66.77) circle (  2.13);

\path[fill=fillColor,fill opacity=0.20] ( 72.11, 65.00) circle (  2.13);

\path[fill=fillColor,fill opacity=0.20] ( 71.67, 62.85) circle (  2.13);

\path[fill=fillColor,fill opacity=0.20] ( 69.49, 56.40) circle (  2.13);

\path[fill=fillColor,fill opacity=0.20] ( 70.14, 59.06) circle (  2.13);

\path[fill=fillColor,fill opacity=0.20] ( 79.32, 68.67) circle (  2.13);

\path[fill=fillColor,fill opacity=0.20] ( 85.22, 79.29) circle (  2.13);

\path[fill=fillColor,fill opacity=0.20] ( 88.94, 79.29) circle (  2.13);

\path[fill=fillColor,fill opacity=0.20] ( 96.80, 83.71) circle (  2.13);

\path[fill=fillColor,fill opacity=0.20] ( 98.55, 93.32) circle (  2.13);

\path[fill=fillColor,fill opacity=0.20] ( 96.80, 94.34) circle (  2.13);

\path[fill=fillColor,fill opacity=0.20] (100.52, 92.57) circle (  2.13);

\path[fill=fillColor,fill opacity=0.20] (102.48, 93.83) circle (  2.13);

\path[fill=fillColor,fill opacity=0.20] (104.67, 91.93) circle (  2.13);

\path[fill=fillColor,fill opacity=0.20] ( 95.49, 92.94) circle (  2.13);

\path[fill=fillColor,fill opacity=0.20] ( 88.94, 92.44) circle (  2.13);

\path[fill=fillColor,fill opacity=0.20] ( 89.59, 88.01) circle (  2.13);

\path[fill=fillColor,fill opacity=0.20] ( 87.62, 85.86) circle (  2.13);

\path[fill=fillColor,fill opacity=0.20] ( 74.95, 80.05) circle (  2.13);

\path[fill=fillColor,fill opacity=0.20] ( 52.01, 58.42) circle (  2.13);

\path[fill=fillColor,fill opacity=0.20] ( 52.01, 50.71) circle (  2.13);

\path[fill=fillColor,fill opacity=0.20] ( 69.05, 68.16) circle (  2.13);

\path[fill=fillColor,fill opacity=0.20] ( 90.25, 66.39) circle (  2.13);

\path[fill=fillColor,fill opacity=0.20] ( 78.23, 66.52) circle (  2.13);

\path[fill=fillColor,fill opacity=0.20] ( 74.51, 65.00) circle (  2.13);

\path[fill=fillColor,fill opacity=0.20] ( 82.16, 66.90) circle (  2.13);

\path[fill=fillColor,fill opacity=0.20] ( 79.98, 69.05) circle (  2.13);

\path[fill=fillColor,fill opacity=0.20] ( 70.14, 67.02) circle (  2.13);

\path[fill=fillColor,fill opacity=0.20] ( 62.50, 52.73) circle (  2.13);

\path[fill=fillColor,fill opacity=0.20] ( 58.56, 43.88) circle (  2.13);

\path[fill=fillColor,fill opacity=0.20] ( 71.89, 66.39) circle (  2.13);

\path[fill=fillColor,fill opacity=0.20] ( 77.14, 72.33) circle (  2.13);

\path[fill=fillColor,fill opacity=0.20] ( 78.88, 72.21) circle (  2.13);

\path[fill=fillColor,fill opacity=0.20] ( 89.59, 79.67) circle (  2.13);

\path[fill=fillColor,fill opacity=0.20] ( 98.99, 81.18) circle (  2.13);

\path[fill=fillColor,fill opacity=0.20] (101.61, 78.78) circle (  2.13);

\path[fill=fillColor,fill opacity=0.20] (105.10, 85.74) circle (  2.13);

\path[fill=fillColor,fill opacity=0.20] (115.16, 99.14) circle (  2.13);

\path[fill=fillColor,fill opacity=0.20] ( 93.09,101.54) circle (  2.13);

\path[fill=fillColor,fill opacity=0.20] ( 91.12, 97.50) circle (  2.13);

\path[fill=fillColor,fill opacity=0.20] ( 83.91, 93.20) circle (  2.13);

\path[fill=fillColor,fill opacity=0.20] ( 83.04, 89.02) circle (  2.13);

\path[fill=fillColor,fill opacity=0.20] ( 78.67, 81.82) circle (  2.13);

\path[fill=fillColor,fill opacity=0.20] ( 70.36, 71.57) circle (  2.13);

\path[fill=fillColor,fill opacity=0.20] ( 61.62, 65.50) circle (  2.13);

\path[fill=fillColor,fill opacity=0.20] ( 59.88, 63.99) circle (  2.13);

\path[fill=fillColor,fill opacity=0.20] ( 45.67, 50.58) circle (  2.13);

\path[fill=fillColor,fill opacity=0.20] ( 75.17, 64.11) circle (  2.13);

\path[fill=fillColor,fill opacity=0.20] ( 75.83, 70.06) circle (  2.13);

\path[fill=fillColor,fill opacity=0.20] ( 77.57, 71.95) circle (  2.13);

\path[fill=fillColor,fill opacity=0.20] ( 77.57, 70.18) circle (  2.13);

\path[fill=fillColor,fill opacity=0.20] ( 76.26, 69.55) circle (  2.13);

\path[fill=fillColor,fill opacity=0.20] ( 80.41, 71.70) circle (  2.13);

\path[fill=fillColor,fill opacity=0.20] ( 80.63, 76.13) circle (  2.13);

\path[fill=fillColor,fill opacity=0.20] ( 70.58, 71.95) circle (  2.13);

\path[fill=fillColor,fill opacity=0.20] ( 66.43, 62.34) circle (  2.13);

\path[fill=fillColor,fill opacity=0.20] ( 68.40, 61.08) circle (  2.13);

\path[fill=fillColor,fill opacity=0.20] ( 66.87, 64.24) circle (  2.13);

\path[fill=fillColor,fill opacity=0.20] ( 71.89, 50.20) circle (  2.13);

\path[fill=fillColor,fill opacity=0.20] ( 72.11, 65.13) circle (  2.13);

\path[fill=fillColor,fill opacity=0.20] ( 74.95, 67.91) circle (  2.13);

\path[fill=fillColor,fill opacity=0.20] ( 78.01, 64.75) circle (  2.13);

\path[fill=fillColor,fill opacity=0.20] ( 87.41, 67.78) circle (  2.13);

\path[fill=fillColor,fill opacity=0.20] ( 97.24, 73.85) circle (  2.13);

\path[fill=fillColor,fill opacity=0.20] ( 93.52, 76.76) circle (  2.13);

\path[fill=fillColor,fill opacity=0.20] ( 91.78, 78.28) circle (  2.13);

\path[fill=fillColor,fill opacity=0.20] ( 92.21, 84.35) circle (  2.13);

\path[fill=fillColor,fill opacity=0.20] ( 92.87, 87.89) circle (  2.13);

\path[fill=fillColor,fill opacity=0.20] ( 87.84, 87.13) circle (  2.13);

\path[fill=fillColor,fill opacity=0.20] ( 78.45, 86.75) circle (  2.13);

\path[fill=fillColor,fill opacity=0.20] ( 68.18, 82.58) circle (  2.13);

\path[fill=fillColor,fill opacity=0.20] ( 65.12, 74.48) circle (  2.13);

\path[fill=fillColor,fill opacity=0.20] ( 59.22, 68.67) circle (  2.13);

\path[fill=fillColor,fill opacity=0.20] ( 59.66, 56.91) circle (  2.13);

\path[fill=fillColor,fill opacity=0.20] ( 74.08, 69.30) circle (  2.13);

\path[fill=fillColor,fill opacity=0.20] ( 76.26, 70.06) circle (  2.13);

\path[fill=fillColor,fill opacity=0.20] ( 70.36, 67.53) circle (  2.13);

\path[fill=fillColor,fill opacity=0.20] ( 79.98, 70.06) circle (  2.13);

\path[fill=fillColor,fill opacity=0.20] ( 83.04, 76.00) circle (  2.13);

\path[fill=fillColor,fill opacity=0.20] ( 79.54, 73.98) circle (  2.13);

\path[fill=fillColor,fill opacity=0.20] ( 75.39, 69.80) circle (  2.13);

\path[fill=fillColor,fill opacity=0.20] ( 75.39, 69.93) circle (  2.13);

\path[fill=fillColor,fill opacity=0.20] ( 77.79, 66.77) circle (  2.13);

\path[fill=fillColor,fill opacity=0.20] ( 74.73, 69.17) circle (  2.13);

\path[fill=fillColor,fill opacity=0.20] ( 66.43, 63.61) circle (  2.13);

\path[fill=fillColor,fill opacity=0.20] ( 77.36, 58.68) circle (  2.13);

\path[fill=fillColor,fill opacity=0.20] ( 79.76, 62.72) circle (  2.13);

\path[fill=fillColor,fill opacity=0.20] ( 88.94, 64.87) circle (  2.13);

\path[fill=fillColor,fill opacity=0.20] ( 98.11, 66.39) circle (  2.13);

\path[fill=fillColor,fill opacity=0.20] ( 89.59, 73.34) circle (  2.13);

\path[fill=fillColor,fill opacity=0.20] ( 89.81, 80.43) circle (  2.13);

\path[fill=fillColor,fill opacity=0.20] ( 91.12, 81.82) circle (  2.13);

\path[fill=fillColor,fill opacity=0.20] ( 75.83, 83.71) circle (  2.13);

\path[fill=fillColor,fill opacity=0.20] ( 67.96, 81.31) circle (  2.13);

\path[fill=fillColor,fill opacity=0.20] (103.36, 69.80) circle (  2.13);

\path[fill=fillColor,fill opacity=0.20] ( 60.31, 58.93) circle (  2.13);

\path[fill=fillColor,fill opacity=0.20] ( 58.13, 54.38) circle (  2.13);

\path[fill=fillColor,fill opacity=0.20] ( 83.04, 58.42) circle (  2.13);

\path[fill=fillColor,fill opacity=0.20] ( 69.71, 66.64) circle (  2.13);

\path[fill=fillColor,fill opacity=0.20] ( 71.02, 70.69) circle (  2.13);

\path[fill=fillColor,fill opacity=0.20] ( 78.01, 71.83) circle (  2.13);

\path[fill=fillColor,fill opacity=0.20] ( 75.39, 68.54) circle (  2.13);

\path[fill=fillColor,fill opacity=0.20] ( 74.95, 68.16) circle (  2.13);

\path[fill=fillColor,fill opacity=0.20] ( 73.20, 76.38) circle (  2.13);

\path[fill=fillColor,fill opacity=0.20] ( 75.61, 72.21) circle (  2.13);

\path[fill=fillColor,fill opacity=0.20] ( 76.92, 64.11) circle (  2.13);

\path[fill=fillColor,fill opacity=0.20] ( 77.36, 66.01) circle (  2.13);

\path[fill=fillColor,fill opacity=0.20] ( 72.77, 66.01) circle (  2.13);

\path[fill=fillColor,fill opacity=0.20] ( 68.62, 60.32) circle (  2.13);

\path[fill=fillColor,fill opacity=0.20] ( 78.88, 53.87) circle (  2.13);

\path[fill=fillColor,fill opacity=0.20] ( 72.77, 53.24) circle (  2.13);

\path[fill=fillColor,fill opacity=0.20] ( 71.89, 51.97) circle (  2.13);

\path[fill=fillColor,fill opacity=0.20] ( 65.56, 52.73) circle (  2.13);

\path[fill=fillColor,fill opacity=0.20] ( 68.40, 51.34) circle (  2.13);

\path[fill=fillColor,fill opacity=0.20] ( 67.30, 54.88) circle (  2.13);

\path[fill=fillColor,fill opacity=0.20] ( 62.06, 56.27) circle (  2.13);

\path[fill=fillColor,fill opacity=0.20] ( 63.59, 58.04) circle (  2.13);

\path[fill=fillColor,fill opacity=0.20] ( 68.83, 59.44) circle (  2.13);

\path[fill=fillColor,fill opacity=0.20] ( 78.67, 62.22) circle (  2.13);

\path[fill=fillColor,fill opacity=0.20] ( 80.85, 61.84) circle (  2.13);

\path[fill=fillColor,fill opacity=0.20] ( 78.45, 70.06) circle (  2.13);

\path[fill=fillColor,fill opacity=0.20] ( 74.51, 71.20) circle (  2.13);

\path[fill=fillColor,fill opacity=0.20] ( 72.77, 78.53) circle (  2.13);

\path[fill=fillColor,fill opacity=0.20] ( 70.14, 89.02) circle (  2.13);

\path[fill=fillColor,fill opacity=0.20] (106.85, 81.56) circle (  2.13);

\path[fill=fillColor,fill opacity=0.20] ( 64.68, 70.56) circle (  2.13);

\path[fill=fillColor,fill opacity=0.20] ( 57.03, 60.95) circle (  2.13);

\path[fill=fillColor,fill opacity=0.20] ( 51.35, 47.68) circle (  2.13);

\path[fill=fillColor,fill opacity=0.20] ( 66.65, 52.61) circle (  2.13);

\path[fill=fillColor,fill opacity=0.20] ( 61.84, 67.40) circle (  2.13);

\path[fill=fillColor,fill opacity=0.20] ( 67.52, 66.77) circle (  2.13);

\path[fill=fillColor,fill opacity=0.20] ( 69.49, 67.91) circle (  2.13);

\path[fill=fillColor,fill opacity=0.20] ( 71.46, 76.89) circle (  2.13);

\path[fill=fillColor,fill opacity=0.20] ( 74.08, 71.95) circle (  2.13);

\path[fill=fillColor,fill opacity=0.20] ( 77.57, 63.48) circle (  2.13);

\path[fill=fillColor,fill opacity=0.20] ( 78.01, 64.24) circle (  2.13);

\path[fill=fillColor,fill opacity=0.20] ( 76.70, 70.69) circle (  2.13);

\path[fill=fillColor,fill opacity=0.20] ( 79.32, 67.65) circle (  2.13);

\path[fill=fillColor,fill opacity=0.20] ( 71.46, 64.37) circle (  2.13);

\path[fill=fillColor,fill opacity=0.20] ( 72.99, 68.16) circle (  2.13);

\path[fill=fillColor,fill opacity=0.20] ( 81.07, 71.70) circle (  2.13);

\path[fill=fillColor,fill opacity=0.20] ( 78.01, 66.26) circle (  2.13);

\path[fill=fillColor,fill opacity=0.20] ( 75.61, 63.86) circle (  2.13);

\path[fill=fillColor,fill opacity=0.20] ( 76.92, 53.87) circle (  2.13);

\path[fill=fillColor,fill opacity=0.20] ( 88.28, 49.19) circle (  2.13);

\path[fill=fillColor,fill opacity=0.20] ( 86.75, 58.68) circle (  2.13);

\path[fill=fillColor,fill opacity=0.20] ( 87.19, 65.76) circle (  2.13);

\path[fill=fillColor,fill opacity=0.20] ( 80.41, 56.27) circle (  2.13);

\path[fill=fillColor,fill opacity=0.20] ( 74.08, 56.78) circle (  2.13);

\path[fill=fillColor,fill opacity=0.20] ( 71.89, 58.68) circle (  2.13);

\path[fill=fillColor,fill opacity=0.20] ( 71.24, 57.03) circle (  2.13);

\path[fill=fillColor,fill opacity=0.20] ( 70.14, 61.46) circle (  2.13);

\path[fill=fillColor,fill opacity=0.20] ( 72.55, 63.23) circle (  2.13);

\path[fill=fillColor,fill opacity=0.20] ( 91.56, 60.19) circle (  2.13);

\path[fill=fillColor,fill opacity=0.20] ( 85.00, 67.02) circle (  2.13);

\path[fill=fillColor,fill opacity=0.20] ( 66.43, 78.78) circle (  2.13);

\path[fill=fillColor,fill opacity=0.20] ( 69.05, 77.77) circle (  2.13);

\path[fill=fillColor,fill opacity=0.20] ( 69.05, 67.65) circle (  2.13);

\path[fill=fillColor,fill opacity=0.20] ( 49.17, 56.78) circle (  2.13);

\path[fill=fillColor,fill opacity=0.20] ( 55.72, 45.91) circle (  2.13);

\path[fill=fillColor,fill opacity=0.20] ( 54.63, 58.42) circle (  2.13);

\path[fill=fillColor,fill opacity=0.20] ( 67.52, 67.02) circle (  2.13);

\path[fill=fillColor,fill opacity=0.20] ( 65.56, 62.60) circle (  2.13);

\path[fill=fillColor,fill opacity=0.20] ( 69.27, 61.21) circle (  2.13);

\path[fill=fillColor,fill opacity=0.20] ( 68.83, 68.41) circle (  2.13);

\path[fill=fillColor,fill opacity=0.20] ( 67.09, 78.28) circle (  2.13);

\path[fill=fillColor,fill opacity=0.20] ( 73.20, 72.33) circle (  2.13);

\path[fill=fillColor,fill opacity=0.20] ( 83.04, 63.86) circle (  2.13);

\path[fill=fillColor,fill opacity=0.20] ( 75.17, 73.98) circle (  2.13);

\path[fill=fillColor,fill opacity=0.20] ( 79.54, 79.54) circle (  2.13);

\path[fill=fillColor,fill opacity=0.20] ( 87.41, 64.24) circle (  2.13);

\path[fill=fillColor,fill opacity=0.20] ( 74.95, 60.70) circle (  2.13);

\path[fill=fillColor,fill opacity=0.20] ( 70.58, 59.31) circle (  2.13);

\path[fill=fillColor,fill opacity=0.20] ( 76.48, 55.52) circle (  2.13);

\path[fill=fillColor,fill opacity=0.20] ( 76.48, 62.72) circle (  2.13);

\path[fill=fillColor,fill opacity=0.20] ( 74.30, 65.88) circle (  2.13);

\path[fill=fillColor,fill opacity=0.20] ( 72.99, 56.78) circle (  2.13);

\path[fill=fillColor,fill opacity=0.20] ( 75.61, 61.46) circle (  2.13);

\path[fill=fillColor,fill opacity=0.20] ( 78.01, 56.53) circle (  2.13);

\path[fill=fillColor,fill opacity=0.20] ( 78.67, 52.10) circle (  2.13);

\path[fill=fillColor,fill opacity=0.20] ( 76.04, 64.62) circle (  2.13);

\path[fill=fillColor,fill opacity=0.20] ( 78.23, 72.97) circle (  2.13);

\path[fill=fillColor,fill opacity=0.20] ( 80.85, 80.43) circle (  2.13);

\path[fill=fillColor,fill opacity=0.20] ( 66.43, 66.77) circle (  2.13);

\path[fill=fillColor,fill opacity=0.20] ( 55.51, 62.34) circle (  2.13);

\path[fill=fillColor,fill opacity=0.20] ( 62.93, 66.26) circle (  2.13);

\path[fill=fillColor,fill opacity=0.20] ( 59.22, 43.12) circle (  2.13);

\path[fill=fillColor,fill opacity=0.20] ( 60.75, 48.94) circle (  2.13);

\path[fill=fillColor,fill opacity=0.20] ( 62.50, 58.04) circle (  2.13);

\path[fill=fillColor,fill opacity=0.20] ( 64.90, 70.56) circle (  2.13);

\path[fill=fillColor,fill opacity=0.20] ( 62.93, 83.97) circle (  2.13);

\path[fill=fillColor,fill opacity=0.20] ( 67.74, 81.44) circle (  2.13);

\path[fill=fillColor,fill opacity=0.20] ( 70.80, 67.02) circle (  2.13);

\path[fill=fillColor,fill opacity=0.20] ( 73.20, 67.02) circle (  2.13);

\path[fill=fillColor,fill opacity=0.20] ( 71.46, 69.68) circle (  2.13);

\path[fill=fillColor,fill opacity=0.20] ( 75.17, 60.70) circle (  2.13);

\path[fill=fillColor,fill opacity=0.20] ( 75.39, 60.19) circle (  2.13);

\path[fill=fillColor,fill opacity=0.20] ( 80.41, 60.70) circle (  2.13);

\path[fill=fillColor,fill opacity=0.20] ( 77.36, 66.64) circle (  2.13);

\path[fill=fillColor,fill opacity=0.20] ( 85.88, 58.42) circle (  2.13);

\path[fill=fillColor,fill opacity=0.20] ( 84.57, 64.11) circle (  2.13);

\path[fill=fillColor,fill opacity=0.20] ( 88.50, 73.09) circle (  2.13);

\path[fill=fillColor,fill opacity=0.20] ( 81.51, 74.86) circle (  2.13);

\path[fill=fillColor,fill opacity=0.20] ( 88.72, 66.77) circle (  2.13);

\path[fill=fillColor,fill opacity=0.20] ( 74.51, 57.54) circle (  2.13);

\path[fill=fillColor,fill opacity=0.20] ( 58.13, 54.63) circle (  2.13);

\path[fill=fillColor,fill opacity=0.20] ( 53.10, 58.80) circle (  2.13);

\path[fill=fillColor,fill opacity=0.20] ( 53.76, 49.70) circle (  2.13);

\path[fill=fillColor,fill opacity=0.20] ( 59.00, 57.54) circle (  2.13);

\path[fill=fillColor,fill opacity=0.20] ( 61.62, 61.96) circle (  2.13);

\path[fill=fillColor,fill opacity=0.20] ( 69.05, 68.41) circle (  2.13);

\path[fill=fillColor,fill opacity=0.20] ( 67.74, 64.37) circle (  2.13);

\path[fill=fillColor,fill opacity=0.20] ( 82.38, 72.46) circle (  2.13);

\path[fill=fillColor,fill opacity=0.20] ( 80.85, 89.02) circle (  2.13);

\path[fill=fillColor,fill opacity=0.20] ( 76.70, 61.08) circle (  2.13);

\path[fill=fillColor,fill opacity=0.20] ( 81.29, 79.03) circle (  2.13);

\path[fill=fillColor,fill opacity=0.20] ( 71.89,101.42) circle (  2.13);

\path[fill=fillColor,fill opacity=0.20] ( 68.18, 90.54) circle (  2.13);

\path[fill=fillColor,fill opacity=0.20] ( 70.36, 70.82) circle (  2.13);

\path[fill=fillColor,fill opacity=0.20] ( 64.68, 59.69) circle (  2.13);

\path[fill=fillColor,fill opacity=0.20] ( 79.54, 61.96) circle (  2.13);

\path[fill=fillColor,fill opacity=0.20] ( 67.74, 66.39) circle (  2.13);

\path[fill=fillColor,fill opacity=0.20] ( 60.31, 54.63) circle (  2.13);

\path[fill=fillColor,fill opacity=0.20] ( 55.29, 45.40) circle (  2.13);

\path[fill=fillColor,fill opacity=0.20] ( 53.98, 39.46) circle (  2.13);

\path[fill=fillColor,fill opacity=0.20] ( 55.29, 46.92) circle (  2.13);

\path[fill=fillColor,fill opacity=0.20] ( 69.27, 54.88) circle (  2.13);

\path[fill=fillColor,fill opacity=0.20] ( 68.18, 69.93) circle (  2.13);

\path[fill=fillColor,fill opacity=0.20] ( 76.70, 82.32) circle (  2.13);

\path[fill=fillColor,fill opacity=0.20] ( 79.32, 82.70) circle (  2.13);

\path[fill=fillColor,fill opacity=0.20] ( 86.97, 78.53) circle (  2.13);

\path[fill=fillColor,fill opacity=0.20] ( 83.91, 65.63) circle (  2.13);

\path[fill=fillColor,fill opacity=0.20] ( 58.56, 68.16) circle (  2.13);

\path[fill=fillColor,fill opacity=0.20] ( 49.61, 57.66) circle (  2.13);

\path[fill=fillColor,fill opacity=0.20] ( 46.77, 49.19) circle (  2.13);

\path[fill=fillColor,fill opacity=0.20] ( 58.35, 42.36) circle (  2.13);

\path[fill=fillColor,fill opacity=0.20] ( 62.50, 50.20) circle (  2.13);

\path[fill=fillColor,fill opacity=0.20] ( 78.01, 50.84) circle (  2.13);

\path[fill=fillColor,fill opacity=0.20] ( 65.99, 46.92) circle (  2.13);

\path[fill=fillColor,fill opacity=0.20] ( 69.49, 47.55) circle (  2.13);

\path[fill=fillColor,fill opacity=0.20] ( 78.45, 52.23) circle (  2.13);

\path[fill=fillColor,fill opacity=0.20] ( 48.95, 43.12) circle (  2.13);
\end{scope}
\begin{scope}
\path[clip] (159.87, 34.04) rectangle (277.03,119.86);
\definecolor[named]{fillColor}{rgb}{0.90,0.90,0.90}

\path[fill=fillColor] (159.87, 34.04) rectangle (277.03,119.86);
\definecolor[named]{drawColor}{rgb}{0.95,0.95,0.95}

\path[draw=drawColor,line width= 0.3pt,line join=round] (159.87, 39.58) --
	(277.03, 39.58);

\path[draw=drawColor,line width= 0.3pt,line join=round] (159.87, 64.87) --
	(277.03, 64.87);

\path[draw=drawColor,line width= 0.3pt,line join=round] (159.87, 90.16) --
	(277.03, 90.16);

\path[draw=drawColor,line width= 0.3pt,line join=round] (159.87,115.45) --
	(277.03,115.45);

\path[draw=drawColor,line width= 0.3pt,line join=round] (169.78, 34.04) --
	(169.78,119.86);

\path[draw=drawColor,line width= 0.3pt,line join=round] (191.63, 34.04) --
	(191.63,119.86);

\path[draw=drawColor,line width= 0.3pt,line join=round] (213.48, 34.04) --
	(213.48,119.86);

\path[draw=drawColor,line width= 0.3pt,line join=round] (235.33, 34.04) --
	(235.33,119.86);

\path[draw=drawColor,line width= 0.3pt,line join=round] (257.18, 34.04) --
	(257.18,119.86);
\definecolor[named]{drawColor}{rgb}{1.00,1.00,1.00}

\path[draw=drawColor,line width= 0.6pt,line join=round] (159.87, 52.23) --
	(277.03, 52.23);

\path[draw=drawColor,line width= 0.6pt,line join=round] (159.87, 77.52) --
	(277.03, 77.52);

\path[draw=drawColor,line width= 0.6pt,line join=round] (159.87,102.81) --
	(277.03,102.81);

\path[draw=drawColor,line width= 0.6pt,line join=round] (180.71, 34.04) --
	(180.71,119.86);

\path[draw=drawColor,line width= 0.6pt,line join=round] (202.56, 34.04) --
	(202.56,119.86);

\path[draw=drawColor,line width= 0.6pt,line join=round] (224.41, 34.04) --
	(224.41,119.86);

\path[draw=drawColor,line width= 0.6pt,line join=round] (246.26, 34.04) --
	(246.26,119.86);

\path[draw=drawColor,line width= 0.6pt,line join=round] (268.11, 34.04) --
	(268.11,119.86);
\definecolor[named]{fillColor}{rgb}{0.00,0.00,0.00}

\path[fill=fillColor,fill opacity=0.20] (185.95, 47.17) circle (  2.13);

\path[fill=fillColor,fill opacity=0.20] (193.16, 49.70) circle (  2.13);

\path[fill=fillColor,fill opacity=0.20] (191.41, 52.23) circle (  2.13);

\path[fill=fillColor,fill opacity=0.20] (196.88, 55.64) circle (  2.13);

\path[fill=fillColor,fill opacity=0.20] (186.61, 53.75) circle (  2.13);

\path[fill=fillColor,fill opacity=0.20] (177.65, 41.73) circle (  2.13);

\path[fill=fillColor,fill opacity=0.20] (187.48, 50.96) circle (  2.13);

\path[fill=fillColor,fill opacity=0.20] (202.56, 62.72) circle (  2.13);

\path[fill=fillColor,fill opacity=0.20] (213.05, 64.75) circle (  2.13);

\path[fill=fillColor,fill opacity=0.20] (214.14, 66.14) circle (  2.13);

\path[fill=fillColor,fill opacity=0.20] (207.58, 71.20) circle (  2.13);

\path[fill=fillColor,fill opacity=0.20] (209.99, 72.46) circle (  2.13);

\path[fill=fillColor,fill opacity=0.20] (210.42, 61.71) circle (  2.13);

\path[fill=fillColor,fill opacity=0.20] (200.15, 49.07) circle (  2.13);

\path[fill=fillColor,fill opacity=0.20] (186.83, 40.72) circle (  2.13);

\path[fill=fillColor,fill opacity=0.20] (194.69, 52.48) circle (  2.13);

\path[fill=fillColor,fill opacity=0.20] (211.30, 64.11) circle (  2.13);

\path[fill=fillColor,fill opacity=0.20] (222.00, 73.09) circle (  2.13);

\path[fill=fillColor,fill opacity=0.20] (219.38, 78.28) circle (  2.13);

\path[fill=fillColor,fill opacity=0.20] (211.52, 83.33) circle (  2.13);

\path[fill=fillColor,fill opacity=0.20] (206.93, 82.70) circle (  2.13);

\path[fill=fillColor,fill opacity=0.20] (202.12, 73.98) circle (  2.13);

\path[fill=fillColor,fill opacity=0.20] (202.56, 63.86) circle (  2.13);

\path[fill=fillColor,fill opacity=0.20] (201.47, 54.76) circle (  2.13);

\path[fill=fillColor,fill opacity=0.20] (194.91, 44.89) circle (  2.13);

\path[fill=fillColor,fill opacity=0.20] (189.67, 54.76) circle (  2.13);

\path[fill=fillColor,fill opacity=0.20] (206.27, 70.82) circle (  2.13);

\path[fill=fillColor,fill opacity=0.20] (220.69, 77.39) circle (  2.13);

\path[fill=fillColor,fill opacity=0.20] (221.13, 77.26) circle (  2.13);

\path[fill=fillColor,fill opacity=0.20] (216.54, 81.31) circle (  2.13);

\path[fill=fillColor,fill opacity=0.20] (216.54, 85.99) circle (  2.13);

\path[fill=fillColor,fill opacity=0.20] (213.70, 81.82) circle (  2.13);

\path[fill=fillColor,fill opacity=0.20] (205.62, 73.34) circle (  2.13);

\path[fill=fillColor,fill opacity=0.20] (203.00, 67.91) circle (  2.13);

\path[fill=fillColor,fill opacity=0.20] (214.79, 66.77) circle (  2.13);

\path[fill=fillColor,fill opacity=0.20] (214.79, 63.10) circle (  2.13);

\path[fill=fillColor,fill opacity=0.20] (200.81, 55.89) circle (  2.13);

\path[fill=fillColor,fill opacity=0.20] (182.24, 40.47) circle (  2.13);

\path[fill=fillColor,fill opacity=0.20] (195.57, 66.77) circle (  2.13);

\path[fill=fillColor,fill opacity=0.20] (215.67, 91.30) circle (  2.13);

\path[fill=fillColor,fill opacity=0.20] (222.22, 92.44) circle (  2.13);

\path[fill=fillColor,fill opacity=0.20] (215.67, 81.06) circle (  2.13);

\path[fill=fillColor,fill opacity=0.20] (218.07, 78.91) circle (  2.13);

\path[fill=fillColor,fill opacity=0.20] (219.60, 81.44) circle (  2.13);

\path[fill=fillColor,fill opacity=0.20] (209.33, 81.31) circle (  2.13);

\path[fill=fillColor,fill opacity=0.20] (204.96, 76.13) circle (  2.13);

\path[fill=fillColor,fill opacity=0.20] (199.50, 68.16) circle (  2.13);

\path[fill=fillColor,fill opacity=0.20] (215.01, 69.80) circle (  2.13);

\path[fill=fillColor,fill opacity=0.20] (215.23, 68.41) circle (  2.13);

\path[fill=fillColor,fill opacity=0.20] (208.46, 67.15) circle (  2.13);

\path[fill=fillColor,fill opacity=0.20] (204.96, 68.79) circle (  2.13);

\path[fill=fillColor,fill opacity=0.20] (201.90, 69.93) circle (  2.13);

\path[fill=fillColor,fill opacity=0.20] (195.57, 61.08) circle (  2.13);

\path[fill=fillColor,fill opacity=0.20] (184.64, 42.62) circle (  2.13);

\path[fill=fillColor,fill opacity=0.20] (197.97, 69.93) circle (  2.13);

\path[fill=fillColor,fill opacity=0.20] (215.89, 89.78) circle (  2.13);

\path[fill=fillColor,fill opacity=0.20] (218.73, 91.93) circle (  2.13);

\path[fill=fillColor,fill opacity=0.20] (213.92, 84.98) circle (  2.13);

\path[fill=fillColor,fill opacity=0.20] (211.74, 81.44) circle (  2.13);

\path[fill=fillColor,fill opacity=0.20] (210.86, 82.07) circle (  2.13);

\path[fill=fillColor,fill opacity=0.20] (206.71, 83.84) circle (  2.13);

\path[fill=fillColor,fill opacity=0.20] (203.65, 78.15) circle (  2.13);

\path[fill=fillColor,fill opacity=0.20] (190.32, 64.24) circle (  2.13);

\path[fill=fillColor,fill opacity=0.20] (209.77, 64.11) circle (  2.13);

\path[fill=fillColor,fill opacity=0.20] (221.13, 71.70) circle (  2.13);

\path[fill=fillColor,fill opacity=0.20] (216.54, 70.18) circle (  2.13);

\path[fill=fillColor,fill opacity=0.20] (215.45, 74.36) circle (  2.13);

\path[fill=fillColor,fill opacity=0.20] (219.60, 79.03) circle (  2.13);

\path[fill=fillColor,fill opacity=0.20] (213.05, 80.05) circle (  2.13);

\path[fill=fillColor,fill opacity=0.20] (210.64, 79.29) circle (  2.13);

\path[fill=fillColor,fill opacity=0.20] (208.68, 71.95) circle (  2.13);

\path[fill=fillColor,fill opacity=0.20] (185.52, 40.09) circle (  2.13);

\path[fill=fillColor,fill opacity=0.20] (199.50, 64.75) circle (  2.13);

\path[fill=fillColor,fill opacity=0.20] (215.45, 78.28) circle (  2.13);

\path[fill=fillColor,fill opacity=0.20] (215.89, 83.71) circle (  2.13);

\path[fill=fillColor,fill opacity=0.20] (213.48, 85.48) circle (  2.13);

\path[fill=fillColor,fill opacity=0.20] (211.74, 85.48) circle (  2.13);

\path[fill=fillColor,fill opacity=0.20] (206.71, 84.35) circle (  2.13);

\path[fill=fillColor,fill opacity=0.20] (206.93, 83.71) circle (  2.13);

\path[fill=fillColor,fill opacity=0.20] (201.68, 79.16) circle (  2.13);

\path[fill=fillColor,fill opacity=0.20] (214.79, 77.77) circle (  2.13);

\path[fill=fillColor,fill opacity=0.20] (220.26, 79.92) circle (  2.13);

\path[fill=fillColor,fill opacity=0.20] (218.95, 81.44) circle (  2.13);

\path[fill=fillColor,fill opacity=0.20] (222.00, 89.15) circle (  2.13);

\path[fill=fillColor,fill opacity=0.20] (218.95, 90.42) circle (  2.13);

\path[fill=fillColor,fill opacity=0.20] (217.42, 85.36) circle (  2.13);

\path[fill=fillColor,fill opacity=0.20] (218.51, 83.46) circle (  2.13);

\path[fill=fillColor,fill opacity=0.20] (206.93, 80.05) circle (  2.13);

\path[fill=fillColor,fill opacity=0.20] (199.72, 57.29) circle (  2.13);

\path[fill=fillColor,fill opacity=0.20] (213.26, 72.59) circle (  2.13);

\path[fill=fillColor,fill opacity=0.20] (213.70, 79.79) circle (  2.13);

\path[fill=fillColor,fill opacity=0.20] (211.74, 82.58) circle (  2.13);

\path[fill=fillColor,fill opacity=0.20] (215.23, 84.98) circle (  2.13);

\path[fill=fillColor,fill opacity=0.20] (210.64, 85.99) circle (  2.13);

\path[fill=fillColor,fill opacity=0.20] (208.46, 84.35) circle (  2.13);

\path[fill=fillColor,fill opacity=0.20] (201.25, 80.68) circle (  2.13);

\path[fill=fillColor,fill opacity=0.20] (206.05, 72.46) circle (  2.13);

\path[fill=fillColor,fill opacity=0.20] (227.03, 93.58) circle (  2.13);

\path[fill=fillColor,fill opacity=0.20] (219.38, 94.34) circle (  2.13);

\path[fill=fillColor,fill opacity=0.20] (224.19,100.40) circle (  2.13);

\path[fill=fillColor,fill opacity=0.20] (223.10,101.16) circle (  2.13);

\path[fill=fillColor,fill opacity=0.20] (213.05, 90.92) circle (  2.13);

\path[fill=fillColor,fill opacity=0.20] (212.61, 89.28) circle (  2.13);

\path[fill=fillColor,fill opacity=0.20] (206.71, 90.16) circle (  2.13);

\path[fill=fillColor,fill opacity=0.20] (203.21, 81.69) circle (  2.13);

\path[fill=fillColor,fill opacity=0.20] (196.88, 70.94) circle (  2.13);

\path[fill=fillColor,fill opacity=0.20] (199.94, 48.81) circle (  2.13);

\path[fill=fillColor,fill opacity=0.20] (209.33, 68.54) circle (  2.13);

\path[fill=fillColor,fill opacity=0.20] (212.39, 78.78) circle (  2.13);

\path[fill=fillColor,fill opacity=0.20] (211.30, 81.44) circle (  2.13);

\path[fill=fillColor,fill opacity=0.20] (215.23, 86.24) circle (  2.13);

\path[fill=fillColor,fill opacity=0.20] (214.14, 93.70) circle (  2.13);

\path[fill=fillColor,fill opacity=0.20] (206.49, 91.81) circle (  2.13);

\path[fill=fillColor,fill opacity=0.20] (201.68, 83.84) circle (  2.13);

\path[fill=fillColor,fill opacity=0.20] (189.89, 73.98) circle (  2.13);

\path[fill=fillColor,fill opacity=0.20] (224.85, 99.52) circle (  2.13);

\path[fill=fillColor,fill opacity=0.20] (232.27,108.75) circle (  2.13);

\path[fill=fillColor,fill opacity=0.20] (218.51,102.05) circle (  2.13);

\path[fill=fillColor,fill opacity=0.20] (219.82,108.37) circle (  2.13);

\path[fill=fillColor,fill opacity=0.20] (215.01,102.68) circle (  2.13);

\path[fill=fillColor,fill opacity=0.20] (207.37, 92.06) circle (  2.13);

\path[fill=fillColor,fill opacity=0.20] (211.30, 98.13) circle (  2.13);

\path[fill=fillColor,fill opacity=0.20] (204.52, 97.75) circle (  2.13);

\path[fill=fillColor,fill opacity=0.20] (199.94, 80.43) circle (  2.13);

\path[fill=fillColor,fill opacity=0.20] (194.04, 68.29) circle (  2.13);

\path[fill=fillColor,fill opacity=0.20] (195.78, 40.85) circle (  2.13);

\path[fill=fillColor,fill opacity=0.20] (210.42, 64.11) circle (  2.13);

\path[fill=fillColor,fill opacity=0.20] (217.85, 79.03) circle (  2.13);

\path[fill=fillColor,fill opacity=0.20] (215.89, 86.24) circle (  2.13);

\path[fill=fillColor,fill opacity=0.20] (216.98, 94.59) circle (  2.13);

\path[fill=fillColor,fill opacity=0.20] (215.23,101.16) circle (  2.13);

\path[fill=fillColor,fill opacity=0.20] (205.62, 96.23) circle (  2.13);

\path[fill=fillColor,fill opacity=0.20] (199.06, 88.52) circle (  2.13);

\path[fill=fillColor,fill opacity=0.20] (195.57, 84.85) circle (  2.13);

\path[fill=fillColor,fill opacity=0.20] (176.99, 66.77) circle (  2.13);

\path[fill=fillColor,fill opacity=0.20] (185.73, 67.28) circle (  2.13);

\path[fill=fillColor,fill opacity=0.20] (240.58,113.18) circle (  2.13);

\path[fill=fillColor,fill opacity=0.20] (222.00,109.26) circle (  2.13);

\path[fill=fillColor,fill opacity=0.20] (214.36,103.44) circle (  2.13);

\path[fill=fillColor,fill opacity=0.20] (217.20,105.08) circle (  2.13);

\path[fill=fillColor,fill opacity=0.20] (216.76, 99.65) circle (  2.13);

\path[fill=fillColor,fill opacity=0.20] (215.01, 99.39) circle (  2.13);

\path[fill=fillColor,fill opacity=0.20] (214.58,107.23) circle (  2.13);

\path[fill=fillColor,fill opacity=0.20] (209.33,100.66) circle (  2.13);

\path[fill=fillColor,fill opacity=0.20] (201.03, 83.84) circle (  2.13);

\path[fill=fillColor,fill opacity=0.20] (204.09, 71.07) circle (  2.13);

\path[fill=fillColor,fill opacity=0.20] (183.11, 47.04) circle (  2.13);

\path[fill=fillColor,fill opacity=0.20] (210.86, 58.42) circle (  2.13);

\path[fill=fillColor,fill opacity=0.20] (221.35, 78.66) circle (  2.13);

\path[fill=fillColor,fill opacity=0.20] (221.79, 95.35) circle (  2.13);

\path[fill=fillColor,fill opacity=0.20] (216.76,103.95) circle (  2.13);

\path[fill=fillColor,fill opacity=0.20] (213.26, 98.76) circle (  2.13);

\path[fill=fillColor,fill opacity=0.20] (208.89, 94.84) circle (  2.13);

\path[fill=fillColor,fill opacity=0.20] (200.59, 93.45) circle (  2.13);

\path[fill=fillColor,fill opacity=0.20] (201.03, 89.66) circle (  2.13);

\path[fill=fillColor,fill opacity=0.20] (190.76, 79.29) circle (  2.13);

\path[fill=fillColor,fill opacity=0.20] (204.96, 90.42) circle (  2.13);

\path[fill=fillColor,fill opacity=0.20] (221.79,104.58) circle (  2.13);

\path[fill=fillColor,fill opacity=0.20] (213.48,104.20) circle (  2.13);

\path[fill=fillColor,fill opacity=0.20] (214.79,108.75) circle (  2.13);

\path[fill=fillColor,fill opacity=0.20] (218.07,104.83) circle (  2.13);

\path[fill=fillColor,fill opacity=0.20] (214.36, 98.00) circle (  2.13);

\path[fill=fillColor,fill opacity=0.20] (211.30,100.91) circle (  2.13);

\path[fill=fillColor,fill opacity=0.20] (209.77,102.18) circle (  2.13);

\path[fill=fillColor,fill opacity=0.20] (207.15, 94.71) circle (  2.13);

\path[fill=fillColor,fill opacity=0.20] (208.68, 84.98) circle (  2.13);

\path[fill=fillColor,fill opacity=0.20] (198.19, 69.55) circle (  2.13);

\path[fill=fillColor,fill opacity=0.20] (180.71, 41.35) circle (  2.13);

\path[fill=fillColor,fill opacity=0.20] (198.63, 47.55) circle (  2.13);

\path[fill=fillColor,fill opacity=0.20] (208.89, 71.32) circle (  2.13);

\path[fill=fillColor,fill opacity=0.20] (218.95, 91.81) circle (  2.13);

\path[fill=fillColor,fill opacity=0.20] (214.14, 97.62) circle (  2.13);

\path[fill=fillColor,fill opacity=0.20] (210.64, 94.08) circle (  2.13);

\path[fill=fillColor,fill opacity=0.20] (215.89, 96.61) circle (  2.13);

\path[fill=fillColor,fill opacity=0.20] (209.33, 96.11) circle (  2.13);

\path[fill=fillColor,fill opacity=0.20] (204.09, 87.76) circle (  2.13);

\path[fill=fillColor,fill opacity=0.20] (197.31, 80.05) circle (  2.13);

\path[fill=fillColor,fill opacity=0.20] (186.83, 72.21) circle (  2.13);

\path[fill=fillColor,fill opacity=0.20] (211.52,102.05) circle (  2.13);

\path[fill=fillColor,fill opacity=0.20] (208.89, 99.01) circle (  2.13);

\path[fill=fillColor,fill opacity=0.20] (209.33,104.07) circle (  2.13);

\path[fill=fillColor,fill opacity=0.20] (221.35,109.64) circle (  2.13);

\path[fill=fillColor,fill opacity=0.20] (214.79,102.68) circle (  2.13);

\path[fill=fillColor,fill opacity=0.20] (218.07, 97.24) circle (  2.13);

\path[fill=fillColor,fill opacity=0.20] (218.73, 94.21) circle (  2.13);

\path[fill=fillColor,fill opacity=0.20] (210.21, 89.66) circle (  2.13);

\path[fill=fillColor,fill opacity=0.20] (206.71, 87.13) circle (  2.13);

\path[fill=fillColor,fill opacity=0.20] (205.84, 79.67) circle (  2.13);

\path[fill=fillColor,fill opacity=0.20] (194.91, 56.02) circle (  2.13);

\path[fill=fillColor,fill opacity=0.20] (198.41, 53.75) circle (  2.13);

\path[fill=fillColor,fill opacity=0.20] (213.05, 77.01) circle (  2.13);

\path[fill=fillColor,fill opacity=0.20] (212.83, 84.22) circle (  2.13);

\path[fill=fillColor,fill opacity=0.20] (209.33, 90.16) circle (  2.13);

\path[fill=fillColor,fill opacity=0.20] (214.58, 96.11) circle (  2.13);

\path[fill=fillColor,fill opacity=0.20] (216.76, 92.69) circle (  2.13);

\path[fill=fillColor,fill opacity=0.20] (206.93, 85.74) circle (  2.13);

\path[fill=fillColor,fill opacity=0.20] (201.25, 80.30) circle (  2.13);

\path[fill=fillColor,fill opacity=0.20] (187.48, 73.22) circle (  2.13);

\path[fill=fillColor,fill opacity=0.20] (208.46, 88.77) circle (  2.13);

\path[fill=fillColor,fill opacity=0.20] (206.49, 97.12) circle (  2.13);

\path[fill=fillColor,fill opacity=0.20] (209.11, 96.74) circle (  2.13);

\path[fill=fillColor,fill opacity=0.20] (213.48,102.43) circle (  2.13);

\path[fill=fillColor,fill opacity=0.20] (217.42,102.55) circle (  2.13);

\path[fill=fillColor,fill opacity=0.20] (215.01, 95.47) circle (  2.13);

\path[fill=fillColor,fill opacity=0.20] (207.15, 92.19) circle (  2.13);

\path[fill=fillColor,fill opacity=0.20] (215.45, 87.51) circle (  2.13);

\path[fill=fillColor,fill opacity=0.20] (210.86, 82.95) circle (  2.13);

\path[fill=fillColor,fill opacity=0.20] (208.24, 83.33) circle (  2.13);

\path[fill=fillColor,fill opacity=0.20] (199.94, 72.71) circle (  2.13);

\path[fill=fillColor,fill opacity=0.20] (203.00, 57.16) circle (  2.13);

\path[fill=fillColor,fill opacity=0.20] (209.33, 70.44) circle (  2.13);

\path[fill=fillColor,fill opacity=0.20] (208.02, 79.54) circle (  2.13);

\path[fill=fillColor,fill opacity=0.20] (208.68, 84.73) circle (  2.13);

\path[fill=fillColor,fill opacity=0.20] (211.95, 85.36) circle (  2.13);

\path[fill=fillColor,fill opacity=0.20] (209.11, 87.63) circle (  2.13);

\path[fill=fillColor,fill opacity=0.20] (204.74, 85.23) circle (  2.13);

\path[fill=fillColor,fill opacity=0.20] (197.97, 79.54) circle (  2.13);

\path[fill=fillColor,fill opacity=0.20] (189.01, 76.13) circle (  2.13);

\path[fill=fillColor,fill opacity=0.20] (176.12, 64.11) circle (  2.13);

\path[fill=fillColor,fill opacity=0.20] (210.64, 84.98) circle (  2.13);

\path[fill=fillColor,fill opacity=0.20] (212.83, 92.57) circle (  2.13);

\path[fill=fillColor,fill opacity=0.20] (208.89, 92.31) circle (  2.13);

\path[fill=fillColor,fill opacity=0.20] (213.48, 90.04) circle (  2.13);

\path[fill=fillColor,fill opacity=0.20] (222.22, 95.85) circle (  2.13);

\path[fill=fillColor,fill opacity=0.20] (217.85, 99.90) circle (  2.13);

\path[fill=fillColor,fill opacity=0.20] (219.82, 92.69) circle (  2.13);

\path[fill=fillColor,fill opacity=0.20] (205.40, 87.25) circle (  2.13);

\path[fill=fillColor,fill opacity=0.20] (208.02, 86.24) circle (  2.13);

\path[fill=fillColor,fill opacity=0.20] (205.84, 82.95) circle (  2.13);

\path[fill=fillColor,fill opacity=0.20] (204.74, 76.89) circle (  2.13);

\path[fill=fillColor,fill opacity=0.20] (196.88, 60.57) circle (  2.13);

\path[fill=fillColor,fill opacity=0.20] (200.15, 50.71) circle (  2.13);

\path[fill=fillColor,fill opacity=0.20] (208.24, 63.23) circle (  2.13);

\path[fill=fillColor,fill opacity=0.20] (209.55, 75.24) circle (  2.13);

\path[fill=fillColor,fill opacity=0.20] (210.42, 83.08) circle (  2.13);

\path[fill=fillColor,fill opacity=0.20] (207.58, 89.91) circle (  2.13);

\path[fill=fillColor,fill opacity=0.20] (207.58, 88.01) circle (  2.13);

\path[fill=fillColor,fill opacity=0.20] (202.56, 82.07) circle (  2.13);

\path[fill=fillColor,fill opacity=0.20] (200.59, 81.69) circle (  2.13);

\path[fill=fillColor,fill opacity=0.20] (190.98, 79.79) circle (  2.13);

\path[fill=fillColor,fill opacity=0.20] (208.89, 84.22) circle (  2.13);

\path[fill=fillColor,fill opacity=0.20] (217.63, 88.14) circle (  2.13);

\path[fill=fillColor,fill opacity=0.20] (210.86, 90.16) circle (  2.13);

\path[fill=fillColor,fill opacity=0.20] (216.32, 90.16) circle (  2.13);

\path[fill=fillColor,fill opacity=0.20] (217.63, 86.62) circle (  2.13);

\path[fill=fillColor,fill opacity=0.20] (222.00, 90.79) circle (  2.13);

\path[fill=fillColor,fill opacity=0.20] (215.45, 97.50) circle (  2.13);

\path[fill=fillColor,fill opacity=0.20] (210.42, 93.58) circle (  2.13);

\path[fill=fillColor,fill opacity=0.20] (211.08, 88.65) circle (  2.13);

\path[fill=fillColor,fill opacity=0.20] (210.42, 89.40) circle (  2.13);

\path[fill=fillColor,fill opacity=0.20] (202.78, 81.18) circle (  2.13);

\path[fill=fillColor,fill opacity=0.20] (196.22, 62.85) circle (  2.13);

\path[fill=fillColor,fill opacity=0.20] (201.03, 46.16) circle (  2.13);

\path[fill=fillColor,fill opacity=0.20] (217.42, 68.41) circle (  2.13);

\path[fill=fillColor,fill opacity=0.20] (211.52, 84.09) circle (  2.13);

\path[fill=fillColor,fill opacity=0.20] (204.09, 90.16) circle (  2.13);

\path[fill=fillColor,fill opacity=0.20] (211.52, 89.91) circle (  2.13);

\path[fill=fillColor,fill opacity=0.20] (209.77, 87.13) circle (  2.13);

\path[fill=fillColor,fill opacity=0.20] (206.93, 87.13) circle (  2.13);

\path[fill=fillColor,fill opacity=0.20] (204.96, 86.12) circle (  2.13);

\path[fill=fillColor,fill opacity=0.20] (192.94, 78.53) circle (  2.13);

\path[fill=fillColor,fill opacity=0.20] (214.14, 84.47) circle (  2.13);

\path[fill=fillColor,fill opacity=0.20] (211.74, 87.76) circle (  2.13);

\path[fill=fillColor,fill opacity=0.20] (219.38, 91.05) circle (  2.13);

\path[fill=fillColor,fill opacity=0.20] (221.79, 92.82) circle (  2.13);

\path[fill=fillColor,fill opacity=0.20] (218.07, 92.82) circle (  2.13);

\path[fill=fillColor,fill opacity=0.20] (218.29, 88.90) circle (  2.13);

\path[fill=fillColor,fill opacity=0.20] (215.23, 86.12) circle (  2.13);

\path[fill=fillColor,fill opacity=0.20] (218.51, 87.25) circle (  2.13);

\path[fill=fillColor,fill opacity=0.20] (213.92, 88.65) circle (  2.13);

\path[fill=fillColor,fill opacity=0.20] (213.26, 87.63) circle (  2.13);

\path[fill=fillColor,fill opacity=0.20] (202.12, 72.71) circle (  2.13);

\path[fill=fillColor,fill opacity=0.20] (190.54, 47.80) circle (  2.13);

\path[fill=fillColor,fill opacity=0.20] (215.45, 73.60) circle (  2.13);

\path[fill=fillColor,fill opacity=0.20] (209.99, 84.73) circle (  2.13);

\path[fill=fillColor,fill opacity=0.20] (208.02, 93.20) circle (  2.13);

\path[fill=fillColor,fill opacity=0.20] (206.05, 95.73) circle (  2.13);

\path[fill=fillColor,fill opacity=0.20] (203.65, 91.81) circle (  2.13);

\path[fill=fillColor,fill opacity=0.20] (203.87, 90.79) circle (  2.13);

\path[fill=fillColor,fill opacity=0.20] (207.15, 84.47) circle (  2.13);

\path[fill=fillColor,fill opacity=0.20] (200.15, 71.57) circle (  2.13);

\path[fill=fillColor,fill opacity=0.20] (182.24, 61.58) circle (  2.13);

\path[fill=fillColor,fill opacity=0.20] (202.56, 79.79) circle (  2.13);

\path[fill=fillColor,fill opacity=0.20] (212.17, 80.43) circle (  2.13);

\path[fill=fillColor,fill opacity=0.20] (215.01, 85.48) circle (  2.13);

\path[fill=fillColor,fill opacity=0.20] (215.89, 93.70) circle (  2.13);

\path[fill=fillColor,fill opacity=0.20] (222.22, 96.36) circle (  2.13);

\path[fill=fillColor,fill opacity=0.20] (219.60, 99.65) circle (  2.13);

\path[fill=fillColor,fill opacity=0.20] (220.48, 99.52) circle (  2.13);

\path[fill=fillColor,fill opacity=0.20] (216.98, 86.62) circle (  2.13);

\path[fill=fillColor,fill opacity=0.20] (215.89, 77.14) circle (  2.13);

\path[fill=fillColor,fill opacity=0.20] (212.61, 78.66) circle (  2.13);

\path[fill=fillColor,fill opacity=0.20] (205.62, 82.83) circle (  2.13);

\path[fill=fillColor,fill opacity=0.20] (201.90, 75.75) circle (  2.13);

\path[fill=fillColor,fill opacity=0.20] (202.78, 56.65) circle (  2.13);

\path[fill=fillColor,fill opacity=0.20] (201.25, 53.11) circle (  2.13);

\path[fill=fillColor,fill opacity=0.20] (212.39, 73.85) circle (  2.13);

\path[fill=fillColor,fill opacity=0.20] (213.48, 85.48) circle (  2.13);

\path[fill=fillColor,fill opacity=0.20] (214.36, 87.38) circle (  2.13);

\path[fill=fillColor,fill opacity=0.20] (207.58, 85.74) circle (  2.13);

\path[fill=fillColor,fill opacity=0.20] (203.87, 88.65) circle (  2.13);

\path[fill=fillColor,fill opacity=0.20] (200.59, 90.92) circle (  2.13);

\path[fill=fillColor,fill opacity=0.20] (199.94, 79.16) circle (  2.13);

\path[fill=fillColor,fill opacity=0.20] (193.60, 66.90) circle (  2.13);

\path[fill=fillColor,fill opacity=0.20] (172.84, 58.30) circle (  2.13);

\path[fill=fillColor,fill opacity=0.20] (205.62, 89.78) circle (  2.13);

\path[fill=fillColor,fill opacity=0.20] (206.49, 82.95) circle (  2.13);

\path[fill=fillColor,fill opacity=0.20] (216.98, 82.58) circle (  2.13);

\path[fill=fillColor,fill opacity=0.20] (221.79, 92.69) circle (  2.13);

\path[fill=fillColor,fill opacity=0.20] (225.50, 95.85) circle (  2.13);

\path[fill=fillColor,fill opacity=0.20] (223.32, 94.84) circle (  2.13);

\path[fill=fillColor,fill opacity=0.20] (222.66, 98.00) circle (  2.13);

\path[fill=fillColor,fill opacity=0.20] (219.16, 94.84) circle (  2.13);

\path[fill=fillColor,fill opacity=0.20] (209.77, 83.33) circle (  2.13);

\path[fill=fillColor,fill opacity=0.20] (205.18, 77.77) circle (  2.13);

\path[fill=fillColor,fill opacity=0.20] (202.12, 80.05) circle (  2.13);

\path[fill=fillColor,fill opacity=0.20] (204.74, 76.25) circle (  2.13);

\path[fill=fillColor,fill opacity=0.20] (187.92, 55.77) circle (  2.13);

\path[fill=fillColor,fill opacity=0.20] (206.71, 53.49) circle (  2.13);

\path[fill=fillColor,fill opacity=0.20] (220.91, 62.72) circle (  2.13);

\path[fill=fillColor,fill opacity=0.20] (218.51, 64.62) circle (  2.13);

\path[fill=fillColor,fill opacity=0.20] (218.51, 75.49) circle (  2.13);

\path[fill=fillColor,fill opacity=0.20] (208.24, 83.46) circle (  2.13);

\path[fill=fillColor,fill opacity=0.20] (201.90, 81.06) circle (  2.13);

\path[fill=fillColor,fill opacity=0.20] (197.75, 81.94) circle (  2.13);

\path[fill=fillColor,fill opacity=0.20] (198.19, 78.91) circle (  2.13);

\path[fill=fillColor,fill opacity=0.20] (198.41, 73.98) circle (  2.13);

\path[fill=fillColor,fill opacity=0.20] (193.16, 77.14) circle (  2.13);

\path[fill=fillColor,fill opacity=0.20] (180.05, 74.74) circle (  2.13);

\path[fill=fillColor,fill opacity=0.20] (193.60, 70.31) circle (  2.13);

\path[fill=fillColor,fill opacity=0.20] (206.93, 76.00) circle (  2.13);

\path[fill=fillColor,fill opacity=0.20] (208.46, 76.00) circle (  2.13);

\path[fill=fillColor,fill opacity=0.20] (210.86, 80.81) circle (  2.13);

\path[fill=fillColor,fill opacity=0.20] (217.20, 89.91) circle (  2.13);

\path[fill=fillColor,fill opacity=0.20] (218.07, 95.35) circle (  2.13);

\path[fill=fillColor,fill opacity=0.20] (220.48, 91.30) circle (  2.13);

\path[fill=fillColor,fill opacity=0.20] (221.35, 88.01) circle (  2.13);

\path[fill=fillColor,fill opacity=0.20] (222.00, 91.17) circle (  2.13);

\path[fill=fillColor,fill opacity=0.20] (209.55, 91.05) circle (  2.13);

\path[fill=fillColor,fill opacity=0.20] (205.84, 84.22) circle (  2.13);

\path[fill=fillColor,fill opacity=0.20] (201.90, 81.69) circle (  2.13);

\path[fill=fillColor,fill opacity=0.20] (200.15, 79.16) circle (  2.13);

\path[fill=fillColor,fill opacity=0.20] (185.30, 58.93) circle (  2.13);

\path[fill=fillColor,fill opacity=0.20] (207.37, 48.18) circle (  2.13);

\path[fill=fillColor,fill opacity=0.20] (212.83, 62.85) circle (  2.13);

\path[fill=fillColor,fill opacity=0.20] (211.08, 77.52) circle (  2.13);

\path[fill=fillColor,fill opacity=0.20] (203.65, 75.87) circle (  2.13);

\path[fill=fillColor,fill opacity=0.20] (197.75, 80.93) circle (  2.13);

\path[fill=fillColor,fill opacity=0.20] (199.06, 76.63) circle (  2.13);

\path[fill=fillColor,fill opacity=0.20] (200.81, 73.34) circle (  2.13);

\path[fill=fillColor,fill opacity=0.20] (196.88, 72.84) circle (  2.13);

\path[fill=fillColor,fill opacity=0.20] (190.98, 66.90) circle (  2.13);

\path[fill=fillColor,fill opacity=0.20] (183.99, 61.84) circle (  2.13);

\path[fill=fillColor,fill opacity=0.20] (178.30, 60.70) circle (  2.13);

\path[fill=fillColor,fill opacity=0.20] (193.60, 63.86) circle (  2.13);

\path[fill=fillColor,fill opacity=0.20] (203.87, 69.42) circle (  2.13);

\path[fill=fillColor,fill opacity=0.20] (207.58, 71.70) circle (  2.13);

\path[fill=fillColor,fill opacity=0.20] (206.71, 69.93) circle (  2.13);

\path[fill=fillColor,fill opacity=0.20] (207.37, 71.83) circle (  2.13);

\path[fill=fillColor,fill opacity=0.20] (211.74, 82.95) circle (  2.13);

\path[fill=fillColor,fill opacity=0.20] (213.48, 93.07) circle (  2.13);

\path[fill=fillColor,fill opacity=0.20] (213.05, 91.68) circle (  2.13);

\path[fill=fillColor,fill opacity=0.20] (215.89, 87.38) circle (  2.13);

\path[fill=fillColor,fill opacity=0.20] (219.16, 86.37) circle (  2.13);

\path[fill=fillColor,fill opacity=0.20] (222.22, 87.51) circle (  2.13);

\path[fill=fillColor,fill opacity=0.20] (210.86, 87.76) circle (  2.13);

\path[fill=fillColor,fill opacity=0.20] (206.49, 81.31) circle (  2.13);

\path[fill=fillColor,fill opacity=0.20] (194.04, 68.41) circle (  2.13);

\path[fill=fillColor,fill opacity=0.20] (190.54, 51.72) circle (  2.13);

\path[fill=fillColor,fill opacity=0.20] (208.68, 73.98) circle (  2.13);

\path[fill=fillColor,fill opacity=0.20] (211.74, 77.26) circle (  2.13);

\path[fill=fillColor,fill opacity=0.20] (204.52, 76.38) circle (  2.13);

\path[fill=fillColor,fill opacity=0.20] (198.19, 75.24) circle (  2.13);

\path[fill=fillColor,fill opacity=0.20] (201.03, 72.71) circle (  2.13);

\path[fill=fillColor,fill opacity=0.20] (194.47, 69.80) circle (  2.13);

\path[fill=fillColor,fill opacity=0.20] (191.85, 67.53) circle (  2.13);

\path[fill=fillColor,fill opacity=0.20] (194.47, 66.14) circle (  2.13);

\path[fill=fillColor,fill opacity=0.20] (196.66, 66.52) circle (  2.13);

\path[fill=fillColor,fill opacity=0.20] (189.67, 66.39) circle (  2.13);

\path[fill=fillColor,fill opacity=0.20] (171.97, 58.93) circle (  2.13);

\path[fill=fillColor,fill opacity=0.20] (193.16, 57.03) circle (  2.13);

\path[fill=fillColor,fill opacity=0.20] (224.63, 64.49) circle (  2.13);

\path[fill=fillColor,fill opacity=0.20] (207.37, 67.40) circle (  2.13);

\path[fill=fillColor,fill opacity=0.20] (206.05, 66.39) circle (  2.13);

\path[fill=fillColor,fill opacity=0.20] (205.18, 71.45) circle (  2.13);

\path[fill=fillColor,fill opacity=0.20] (207.15, 77.77) circle (  2.13);

\path[fill=fillColor,fill opacity=0.20] (211.74, 80.68) circle (  2.13);

\path[fill=fillColor,fill opacity=0.20] (213.05, 86.62) circle (  2.13);

\path[fill=fillColor,fill opacity=0.20] (216.98, 92.82) circle (  2.13);

\path[fill=fillColor,fill opacity=0.20] (217.42, 91.17) circle (  2.13);

\path[fill=fillColor,fill opacity=0.20] (211.30, 86.37) circle (  2.13);

\path[fill=fillColor,fill opacity=0.20] (209.55, 80.30) circle (  2.13);

\path[fill=fillColor,fill opacity=0.20] (209.11, 72.59) circle (  2.13);

\path[fill=fillColor,fill opacity=0.20] (204.31, 66.26) circle (  2.13);

\path[fill=fillColor,fill opacity=0.20] (189.89, 56.15) circle (  2.13);

\path[fill=fillColor,fill opacity=0.20] (206.27, 50.33) circle (  2.13);

\path[fill=fillColor,fill opacity=0.20] (217.42, 60.83) circle (  2.13);

\path[fill=fillColor,fill opacity=0.20] (203.65, 64.62) circle (  2.13);

\path[fill=fillColor,fill opacity=0.20] (203.65, 71.32) circle (  2.13);

\path[fill=fillColor,fill opacity=0.20] (199.50, 74.74) circle (  2.13);

\path[fill=fillColor,fill opacity=0.20] (196.44, 73.09) circle (  2.13);

\path[fill=fillColor,fill opacity=0.20] (197.97, 71.07) circle (  2.13);

\path[fill=fillColor,fill opacity=0.20] (194.47, 65.38) circle (  2.13);

\path[fill=fillColor,fill opacity=0.20] (194.04, 61.84) circle (  2.13);

\path[fill=fillColor,fill opacity=0.20] (198.19, 63.10) circle (  2.13);

\path[fill=fillColor,fill opacity=0.20] (190.98, 65.38) circle (  2.13);

\path[fill=fillColor,fill opacity=0.20] (190.98, 63.23) circle (  2.13);

\path[fill=fillColor,fill opacity=0.20] (184.86, 62.85) circle (  2.13);

\path[fill=fillColor,fill opacity=0.20] (180.49, 67.65) circle (  2.13);

\path[fill=fillColor,fill opacity=0.20] (179.83, 69.42) circle (  2.13);

\path[fill=fillColor,fill opacity=0.20] (181.15, 63.99) circle (  2.13);

\path[fill=fillColor,fill opacity=0.20] (176.12, 60.83) circle (  2.13);

\path[fill=fillColor,fill opacity=0.20] (176.34, 61.46) circle (  2.13);

\path[fill=fillColor,fill opacity=0.20] (185.52, 59.06) circle (  2.13);

\path[fill=fillColor,fill opacity=0.20] (179.40, 54.38) circle (  2.13);

\path[fill=fillColor,fill opacity=0.20] (180.05, 53.62) circle (  2.13);

\path[fill=fillColor,fill opacity=0.20] (180.27, 54.25) circle (  2.13);

\path[fill=fillColor,fill opacity=0.20] (180.93, 52.23) circle (  2.13);

\path[fill=fillColor,fill opacity=0.20] (180.05, 52.73) circle (  2.13);

\path[fill=fillColor,fill opacity=0.20] (186.61, 58.55) circle (  2.13);

\path[fill=fillColor,fill opacity=0.20] (196.00, 60.45) circle (  2.13);

\path[fill=fillColor,fill opacity=0.20] (187.70, 56.91) circle (  2.13);

\path[fill=fillColor,fill opacity=0.20] (189.23, 57.79) circle (  2.13);

\path[fill=fillColor,fill opacity=0.20] (191.85, 60.45) circle (  2.13);

\path[fill=fillColor,fill opacity=0.20] (194.47, 59.81) circle (  2.13);

\path[fill=fillColor,fill opacity=0.20] (200.15, 60.95) circle (  2.13);

\path[fill=fillColor,fill opacity=0.20] (198.19, 62.09) circle (  2.13);

\path[fill=fillColor,fill opacity=0.20] (197.53, 62.22) circle (  2.13);

\path[fill=fillColor,fill opacity=0.20] (206.93, 67.15) circle (  2.13);

\path[fill=fillColor,fill opacity=0.20] (208.46, 73.60) circle (  2.13);

\path[fill=fillColor,fill opacity=0.20] (211.74, 77.52) circle (  2.13);

\path[fill=fillColor,fill opacity=0.20] (215.67, 83.08) circle (  2.13);

\path[fill=fillColor,fill opacity=0.20] (215.45, 87.25) circle (  2.13);

\path[fill=fillColor,fill opacity=0.20] (213.70, 88.52) circle (  2.13);

\path[fill=fillColor,fill opacity=0.20] (217.20, 89.02) circle (  2.13);

\path[fill=fillColor,fill opacity=0.20] (206.93, 86.12) circle (  2.13);

\path[fill=fillColor,fill opacity=0.20] (208.46, 81.06) circle (  2.13);

\path[fill=fillColor,fill opacity=0.20] (204.09, 74.36) circle (  2.13);

\path[fill=fillColor,fill opacity=0.20] (196.00, 61.33) circle (  2.13);

\path[fill=fillColor,fill opacity=0.20] (190.76, 48.43) circle (  2.13);

\path[fill=fillColor,fill opacity=0.20] (202.12, 50.08) circle (  2.13);

\path[fill=fillColor,fill opacity=0.20] (212.39, 61.08) circle (  2.13);

\path[fill=fillColor,fill opacity=0.20] (205.18, 70.69) circle (  2.13);

\path[fill=fillColor,fill opacity=0.20] (204.74, 74.61) circle (  2.13);

\path[fill=fillColor,fill opacity=0.20] (203.65, 73.60) circle (  2.13);

\path[fill=fillColor,fill opacity=0.20] (200.59, 68.41) circle (  2.13);

\path[fill=fillColor,fill opacity=0.20] (195.78, 64.37) circle (  2.13);

\path[fill=fillColor,fill opacity=0.20] (199.50, 65.50) circle (  2.13);

\path[fill=fillColor,fill opacity=0.20] (199.50, 66.64) circle (  2.13);

\path[fill=fillColor,fill opacity=0.20] (197.31, 64.87) circle (  2.13);

\path[fill=fillColor,fill opacity=0.20] (196.66, 64.24) circle (  2.13);

\path[fill=fillColor,fill opacity=0.20] (196.66, 69.93) circle (  2.13);

\path[fill=fillColor,fill opacity=0.20] (205.40, 77.14) circle (  2.13);

\path[fill=fillColor,fill opacity=0.20] (198.84, 76.38) circle (  2.13);

\path[fill=fillColor,fill opacity=0.20] (194.26, 74.99) circle (  2.13);

\path[fill=fillColor,fill opacity=0.20] (198.41, 77.01) circle (  2.13);

\path[fill=fillColor,fill opacity=0.20] (200.59, 75.12) circle (  2.13);

\path[fill=fillColor,fill opacity=0.20] (201.25, 68.79) circle (  2.13);

\path[fill=fillColor,fill opacity=0.20] (198.84, 66.26) circle (  2.13);

\path[fill=fillColor,fill opacity=0.20] (200.81, 70.82) circle (  2.13);

\path[fill=fillColor,fill opacity=0.20] (199.72, 77.39) circle (  2.13);

\path[fill=fillColor,fill opacity=0.20] (196.44, 74.61) circle (  2.13);

\path[fill=fillColor,fill opacity=0.20] (197.97, 69.17) circle (  2.13);

\path[fill=fillColor,fill opacity=0.20] (199.28, 70.69) circle (  2.13);

\path[fill=fillColor,fill opacity=0.20] (201.90, 72.46) circle (  2.13);

\path[fill=fillColor,fill opacity=0.20] (200.37, 71.57) circle (  2.13);

\path[fill=fillColor,fill opacity=0.20] (202.78, 74.74) circle (  2.13);

\path[fill=fillColor,fill opacity=0.20] (203.43, 73.85) circle (  2.13);

\path[fill=fillColor,fill opacity=0.20] (206.71, 69.05) circle (  2.13);

\path[fill=fillColor,fill opacity=0.20] (206.93, 72.46) circle (  2.13);

\path[fill=fillColor,fill opacity=0.20] (209.11, 76.76) circle (  2.13);

\path[fill=fillColor,fill opacity=0.20] (208.02, 72.97) circle (  2.13);

\path[fill=fillColor,fill opacity=0.20] (209.99, 73.09) circle (  2.13);

\path[fill=fillColor,fill opacity=0.20] (212.83, 80.05) circle (  2.13);

\path[fill=fillColor,fill opacity=0.20] (214.14, 83.97) circle (  2.13);

\path[fill=fillColor,fill opacity=0.20] (215.67, 79.54) circle (  2.13);

\path[fill=fillColor,fill opacity=0.20] (209.33, 72.08) circle (  2.13);

\path[fill=fillColor,fill opacity=0.20] (204.09, 67.91) circle (  2.13);

\path[fill=fillColor,fill opacity=0.20] (201.47, 64.87) circle (  2.13);

\path[fill=fillColor,fill opacity=0.20] (189.67, 59.06) circle (  2.13);

\path[fill=fillColor,fill opacity=0.20] (187.70, 53.11) circle (  2.13);

\path[fill=fillColor,fill opacity=0.20] (182.46, 48.56) circle (  2.13);

\path[fill=fillColor,fill opacity=0.20] (205.40, 64.87) circle (  2.13);

\path[fill=fillColor,fill opacity=0.20] (212.83, 68.67) circle (  2.13);

\path[fill=fillColor,fill opacity=0.20] (204.31, 69.30) circle (  2.13);

\path[fill=fillColor,fill opacity=0.20] (205.62, 69.55) circle (  2.13);

\path[fill=fillColor,fill opacity=0.20] (204.31, 71.45) circle (  2.13);

\path[fill=fillColor,fill opacity=0.20] (203.43, 76.25) circle (  2.13);

\path[fill=fillColor,fill opacity=0.20] (199.06, 75.24) circle (  2.13);

\path[fill=fillColor,fill opacity=0.20] (198.84, 71.20) circle (  2.13);

\path[fill=fillColor,fill opacity=0.20] (197.31, 70.18) circle (  2.13);

\path[fill=fillColor,fill opacity=0.20] (204.96, 69.05) circle (  2.13);

\path[fill=fillColor,fill opacity=0.20] (199.06, 69.42) circle (  2.13);

\path[fill=fillColor,fill opacity=0.20] (201.25, 72.71) circle (  2.13);

\path[fill=fillColor,fill opacity=0.20] (197.97, 77.01) circle (  2.13);

\path[fill=fillColor,fill opacity=0.20] (201.25, 77.77) circle (  2.13);

\path[fill=fillColor,fill opacity=0.20] (203.00, 72.84) circle (  2.13);

\path[fill=fillColor,fill opacity=0.20] (200.59, 69.30) circle (  2.13);

\path[fill=fillColor,fill opacity=0.20] (199.72, 72.33) circle (  2.13);

\path[fill=fillColor,fill opacity=0.20] (199.50, 75.87) circle (  2.13);

\path[fill=fillColor,fill opacity=0.20] (206.49, 77.77) circle (  2.13);

\path[fill=fillColor,fill opacity=0.20] (199.50, 77.26) circle (  2.13);

\path[fill=fillColor,fill opacity=0.20] (201.03, 73.34) circle (  2.13);

\path[fill=fillColor,fill opacity=0.20] (201.47, 72.84) circle (  2.13);

\path[fill=fillColor,fill opacity=0.20] (203.87, 77.26) circle (  2.13);

\path[fill=fillColor,fill opacity=0.20] (208.02, 81.94) circle (  2.13);

\path[fill=fillColor,fill opacity=0.20] (208.02, 83.46) circle (  2.13);

\path[fill=fillColor,fill opacity=0.20] (208.68, 79.92) circle (  2.13);

\path[fill=fillColor,fill opacity=0.20] (208.89, 74.61) circle (  2.13);

\path[fill=fillColor,fill opacity=0.20] (208.46, 72.97) circle (  2.13);

\path[fill=fillColor,fill opacity=0.20] (205.18, 71.45) circle (  2.13);

\path[fill=fillColor,fill opacity=0.20] (200.81, 65.88) circle (  2.13);

\path[fill=fillColor,fill opacity=0.20] (199.28, 61.84) circle (  2.13);

\path[fill=fillColor,fill opacity=0.20] (199.72, 60.32) circle (  2.13);

\path[fill=fillColor,fill opacity=0.20] (195.78, 55.89) circle (  2.13);

\path[fill=fillColor,fill opacity=0.20] (187.70, 49.07) circle (  2.13);

\path[fill=fillColor,fill opacity=0.20] (179.83, 44.39) circle (  2.13);

\path[fill=fillColor,fill opacity=0.20] (178.52, 42.36) circle (  2.13);

\path[fill=fillColor,fill opacity=0.20] (198.19, 53.24) circle (  2.13);

\path[fill=fillColor,fill opacity=0.20] (201.47, 57.79) circle (  2.13);

\path[fill=fillColor,fill opacity=0.20] (202.78, 58.68) circle (  2.13);

\path[fill=fillColor,fill opacity=0.20] (206.27, 63.86) circle (  2.13);

\path[fill=fillColor,fill opacity=0.20] (200.37, 76.89) circle (  2.13);

\path[fill=fillColor,fill opacity=0.20] (201.68, 81.94) circle (  2.13);

\path[fill=fillColor,fill opacity=0.20] (201.90, 78.28) circle (  2.13);

\path[fill=fillColor,fill opacity=0.20] (204.96, 74.36) circle (  2.13);

\path[fill=fillColor,fill opacity=0.20] (203.00, 66.26) circle (  2.13);

\path[fill=fillColor,fill opacity=0.20] (201.68, 65.13) circle (  2.13);

\path[fill=fillColor,fill opacity=0.20] (202.34, 72.21) circle (  2.13);

\path[fill=fillColor,fill opacity=0.20] (202.12, 78.40) circle (  2.13);

\path[fill=fillColor,fill opacity=0.20] (202.34, 77.39) circle (  2.13);

\path[fill=fillColor,fill opacity=0.20] (200.81, 72.84) circle (  2.13);

\path[fill=fillColor,fill opacity=0.20] (204.74, 70.94) circle (  2.13);

\path[fill=fillColor,fill opacity=0.20] (198.41, 75.87) circle (  2.13);

\path[fill=fillColor,fill opacity=0.20] (199.72, 77.90) circle (  2.13);

\path[fill=fillColor,fill opacity=0.20] (204.74, 73.60) circle (  2.13);

\path[fill=fillColor,fill opacity=0.20] (202.34, 70.18) circle (  2.13);

\path[fill=fillColor,fill opacity=0.20] (204.74, 68.29) circle (  2.13);

\path[fill=fillColor,fill opacity=0.20] (203.65, 68.67) circle (  2.13);

\path[fill=fillColor,fill opacity=0.20] (204.31, 71.45) circle (  2.13);

\path[fill=fillColor,fill opacity=0.20] (208.46, 72.21) circle (  2.13);

\path[fill=fillColor,fill opacity=0.20] (201.90, 67.15) circle (  2.13);

\path[fill=fillColor,fill opacity=0.20] (198.84, 61.46) circle (  2.13);

\path[fill=fillColor,fill opacity=0.20] (193.60, 56.15) circle (  2.13);

\path[fill=fillColor,fill opacity=0.20] (187.92, 43.63) circle (  2.13);

\path[fill=fillColor,fill opacity=0.20] (194.04, 52.99) circle (  2.13);

\path[fill=fillColor,fill opacity=0.20] (200.37, 58.30) circle (  2.13);

\path[fill=fillColor,fill opacity=0.20] (202.12, 61.08) circle (  2.13);

\path[fill=fillColor,fill opacity=0.20] (209.11, 61.21) circle (  2.13);

\path[fill=fillColor,fill opacity=0.20] (205.18, 58.30) circle (  2.13);

\path[fill=fillColor,fill opacity=0.20] (204.74, 60.07) circle (  2.13);

\path[fill=fillColor,fill opacity=0.20] (204.52, 65.38) circle (  2.13);

\path[fill=fillColor,fill opacity=0.20] (203.43, 67.53) circle (  2.13);

\path[fill=fillColor,fill opacity=0.20] (200.37, 67.28) circle (  2.13);

\path[fill=fillColor,fill opacity=0.20] (200.59, 64.62) circle (  2.13);

\path[fill=fillColor,fill opacity=0.20] (197.10, 61.96) circle (  2.13);

\path[fill=fillColor,fill opacity=0.20] (198.84, 63.61) circle (  2.13);

\path[fill=fillColor,fill opacity=0.20] (199.50, 63.86) circle (  2.13);

\path[fill=fillColor,fill opacity=0.20] (199.94, 58.55) circle (  2.13);

\path[fill=fillColor,fill opacity=0.20] (197.31, 52.99) circle (  2.13);

\path[fill=fillColor,fill opacity=0.20] (198.84, 49.57) circle (  2.13);

\path[fill=fillColor,fill opacity=0.20] (191.20, 46.03) circle (  2.13);

\path[fill=fillColor,fill opacity=0.20] (191.63, 41.61) circle (  2.13);

\path[fill=fillColor,fill opacity=0.20] (191.41, 42.24) circle (  2.13);

\path[fill=fillColor,fill opacity=0.20] (192.94, 45.27) circle (  2.13);

\path[fill=fillColor,fill opacity=0.20] (191.20, 40.72) circle (  2.13);

\path[fill=fillColor,fill opacity=0.20] (190.98, 40.09) circle (  2.13);

\path[fill=fillColor,fill opacity=0.20] (197.53, 89.78) circle (  2.13);

\path[fill=fillColor,fill opacity=0.20] (250.19, 79.79) circle (  2.13);

\path[fill=fillColor,fill opacity=0.20] (237.52, 94.34) circle (  2.13);

\path[fill=fillColor,fill opacity=0.20] (240.36, 85.86) circle (  2.13);

\path[fill=fillColor,fill opacity=0.20] (240.14, 92.19) circle (  2.13);

\path[fill=fillColor,fill opacity=0.20] (238.17, 93.83) circle (  2.13);

\path[fill=fillColor,fill opacity=0.20] (239.70, 93.45) circle (  2.13);

\path[fill=fillColor,fill opacity=0.20] (239.92,101.67) circle (  2.13);

\path[fill=fillColor,fill opacity=0.20] (214.14, 94.97) circle (  2.13);

\path[fill=fillColor,fill opacity=0.20] (235.55,110.52) circle (  2.13);

\path[fill=fillColor,fill opacity=0.20] (234.24,107.87) circle (  2.13);

\path[fill=fillColor,fill opacity=0.20] (243.20,103.69) circle (  2.13);

\path[fill=fillColor,fill opacity=0.20] (247.79,110.65) circle (  2.13);

\path[fill=fillColor,fill opacity=0.20] (250.85,106.85) circle (  2.13);

\path[fill=fillColor,fill opacity=0.20] (258.28,102.30) circle (  2.13);

\path[fill=fillColor,fill opacity=0.20] (260.24,101.80) circle (  2.13);

\path[fill=fillColor,fill opacity=0.20] (230.96, 92.31) circle (  2.13);

\path[fill=fillColor,fill opacity=0.20] (196.44, 76.38) circle (  2.13);

\path[fill=fillColor,fill opacity=0.20] (266.14, 97.50) circle (  2.13);

\path[fill=fillColor,fill opacity=0.20] (253.03,103.95) circle (  2.13);

\path[fill=fillColor,fill opacity=0.20] (251.72,115.33) circle (  2.13);

\path[fill=fillColor,fill opacity=0.20] (256.96,111.28) circle (  2.13);

\path[fill=fillColor,fill opacity=0.20] (252.81, 94.71) circle (  2.13);

\path[fill=fillColor,fill opacity=0.20] (200.81, 83.46) circle (  2.13);

\path[fill=fillColor,fill opacity=0.20] (242.33,106.22) circle (  2.13);

\path[fill=fillColor,fill opacity=0.20] (269.86,107.36) circle (  2.13);

\path[fill=fillColor,fill opacity=0.20] (243.85, 99.90) circle (  2.13);

\path[fill=fillColor,fill opacity=0.20] (217.63,101.67) circle (  2.13);

\path[fill=fillColor,fill opacity=0.20] (231.40,112.29) circle (  2.13);

\path[fill=fillColor,fill opacity=0.20] (261.12,106.98) circle (  2.13);

\path[fill=fillColor,fill opacity=0.20] (253.25,106.47) circle (  2.13);

\path[fill=fillColor,fill opacity=0.20] (247.57,105.97) circle (  2.13);

\path[fill=fillColor,fill opacity=0.20] (256.09,106.47) circle (  2.13);

\path[fill=fillColor,fill opacity=0.20] (264.83,105.46) circle (  2.13);

\path[fill=fillColor,fill opacity=0.20] (259.15,104.32) circle (  2.13);

\path[fill=fillColor,fill opacity=0.20] (222.88,101.92) circle (  2.13);

\path[fill=fillColor,fill opacity=0.20] (194.47, 84.73) circle (  2.13);

\path[fill=fillColor,fill opacity=0.20] (232.06, 46.03) circle (  2.13);

\path[fill=fillColor,fill opacity=0.20] (229.87, 58.80) circle (  2.13);

\path[fill=fillColor,fill opacity=0.20] (217.85, 59.56) circle (  2.13);

\path[fill=fillColor,fill opacity=0.20] (207.80, 65.38) circle (  2.13);

\path[fill=fillColor,fill opacity=0.20] (203.65, 66.52) circle (  2.13);

\path[fill=fillColor,fill opacity=0.20] (203.43, 58.55) circle (  2.13);

\path[fill=fillColor,fill opacity=0.20] (196.44, 50.96) circle (  2.13);

\path[fill=fillColor,fill opacity=0.20] (172.19, 50.71) circle (  2.13);

\path[fill=fillColor,fill opacity=0.20] (216.54, 92.69) circle (  2.13);

\path[fill=fillColor,fill opacity=0.20] (232.06,100.91) circle (  2.13);

\path[fill=fillColor,fill opacity=0.20] (235.77,100.78) circle (  2.13);

\path[fill=fillColor,fill opacity=0.20] (240.36,103.31) circle (  2.13);

\path[fill=fillColor,fill opacity=0.20] (244.07,105.59) circle (  2.13);

\path[fill=fillColor,fill opacity=0.20] (251.94,103.69) circle (  2.13);

\path[fill=fillColor,fill opacity=0.20] (256.31,106.85) circle (  2.13);

\path[fill=fillColor,fill opacity=0.20] (246.48,109.64) circle (  2.13);

\path[fill=fillColor,fill opacity=0.20] (207.58,100.40) circle (  2.13);

\path[fill=fillColor,fill opacity=0.20] (175.03, 82.20) circle (  2.13);

\path[fill=fillColor,fill opacity=0.20] (225.06, 57.54) circle (  2.13);

\path[fill=fillColor,fill opacity=0.20] (228.12, 75.87) circle (  2.13);

\path[fill=fillColor,fill opacity=0.20] (228.78, 83.21) circle (  2.13);

\path[fill=fillColor,fill opacity=0.20] (214.58, 79.67) circle (  2.13);

\path[fill=fillColor,fill opacity=0.20] (207.58, 79.16) circle (  2.13);

\path[fill=fillColor,fill opacity=0.20] (215.89, 81.94) circle (  2.13);

\path[fill=fillColor,fill opacity=0.20] (208.89, 72.97) circle (  2.13);

\path[fill=fillColor,fill opacity=0.20] (180.05, 57.92) circle (  2.13);

\path[fill=fillColor,fill opacity=0.20] (217.85, 92.82) circle (  2.13);

\path[fill=fillColor,fill opacity=0.20] (233.80, 93.96) circle (  2.13);

\path[fill=fillColor,fill opacity=0.20] (260.46, 98.26) circle (  2.13);

\path[fill=fillColor,fill opacity=0.20] (252.16,105.84) circle (  2.13);

\path[fill=fillColor,fill opacity=0.20] (247.79,108.62) circle (  2.13);

\path[fill=fillColor,fill opacity=0.20] (253.69,110.02) circle (  2.13);

\path[fill=fillColor,fill opacity=0.20] (255.65,107.61) circle (  2.13);

\path[fill=fillColor,fill opacity=0.20] (228.12,102.05) circle (  2.13);

\path[fill=fillColor,fill opacity=0.20] (198.84, 88.52) circle (  2.13);

\path[fill=fillColor,fill opacity=0.20] (220.69, 67.65) circle (  2.13);

\path[fill=fillColor,fill opacity=0.20] (226.59, 89.15) circle (  2.13);

\path[fill=fillColor,fill opacity=0.20] (229.00, 85.23) circle (  2.13);

\path[fill=fillColor,fill opacity=0.20] (222.88, 87.00) circle (  2.13);

\path[fill=fillColor,fill opacity=0.20] (218.29, 85.99) circle (  2.13);

\path[fill=fillColor,fill opacity=0.20] (216.11, 84.47) circle (  2.13);

\path[fill=fillColor,fill opacity=0.20] (213.48, 89.15) circle (  2.13);

\path[fill=fillColor,fill opacity=0.20] (210.86, 83.59) circle (  2.13);

\path[fill=fillColor,fill opacity=0.20] (201.25, 70.06) circle (  2.13);

\path[fill=fillColor,fill opacity=0.20] (217.63, 92.44) circle (  2.13);

\path[fill=fillColor,fill opacity=0.20] (226.16,110.65) circle (  2.13);

\path[fill=fillColor,fill opacity=0.20] (242.33,101.29) circle (  2.13);

\path[fill=fillColor,fill opacity=0.20] (258.06,112.67) circle (  2.13);

\path[fill=fillColor,fill opacity=0.20] (271.39,109.00) circle (  2.13);

\path[fill=fillColor,fill opacity=0.20] (258.28,108.24) circle (  2.13);

\path[fill=fillColor,fill opacity=0.20] (251.28,109.26) circle (  2.13);

\path[fill=fillColor,fill opacity=0.20] (209.77, 98.63) circle (  2.13);

\path[fill=fillColor,fill opacity=0.20] (212.17, 51.85) circle (  2.13);

\path[fill=fillColor,fill opacity=0.20] (221.57, 83.71) circle (  2.13);

\path[fill=fillColor,fill opacity=0.20] (242.33, 96.36) circle (  2.13);

\path[fill=fillColor,fill opacity=0.20] (233.80, 99.01) circle (  2.13);

\path[fill=fillColor,fill opacity=0.20] (222.88, 96.74) circle (  2.13);

\path[fill=fillColor,fill opacity=0.20] (216.98, 89.53) circle (  2.13);

\path[fill=fillColor,fill opacity=0.20] (213.26, 88.52) circle (  2.13);

\path[fill=fillColor,fill opacity=0.20] (211.95, 92.44) circle (  2.13);

\path[fill=fillColor,fill opacity=0.20] (207.15, 87.25) circle (  2.13);

\path[fill=fillColor,fill opacity=0.20] (186.39, 77.77) circle (  2.13);

\path[fill=fillColor,fill opacity=0.20] (235.33, 86.62) circle (  2.13);

\path[fill=fillColor,fill opacity=0.20] (228.12, 91.68) circle (  2.13);

\path[fill=fillColor,fill opacity=0.20] (233.59, 98.63) circle (  2.13);

\path[fill=fillColor,fill opacity=0.20] (242.33, 94.71) circle (  2.13);

\path[fill=fillColor,fill opacity=0.20] (252.16,102.43) circle (  2.13);

\path[fill=fillColor,fill opacity=0.20] (249.75,110.52) circle (  2.13);

\path[fill=fillColor,fill opacity=0.20] (260.24,102.18) circle (  2.13);

\path[fill=fillColor,fill opacity=0.20] (221.79,107.99) circle (  2.13);

\path[fill=fillColor,fill opacity=0.20] (185.08, 98.00) circle (  2.13);

\path[fill=fillColor,fill opacity=0.20] (204.74, 56.78) circle (  2.13);

\path[fill=fillColor,fill opacity=0.20] (227.90, 85.74) circle (  2.13);

\path[fill=fillColor,fill opacity=0.20] (235.11, 94.46) circle (  2.13);

\path[fill=fillColor,fill opacity=0.20] (224.19,104.96) circle (  2.13);

\path[fill=fillColor,fill opacity=0.20] (219.38,107.49) circle (  2.13);

\path[fill=fillColor,fill opacity=0.20] (209.77, 97.62) circle (  2.13);

\path[fill=fillColor,fill opacity=0.20] (204.09, 96.61) circle (  2.13);

\path[fill=fillColor,fill opacity=0.20] (204.09, 98.38) circle (  2.13);

\path[fill=fillColor,fill opacity=0.20] (201.25, 90.67) circle (  2.13);

\path[fill=fillColor,fill opacity=0.20] (168.47, 80.93) circle (  2.13);

\path[fill=fillColor,fill opacity=0.20] (214.36, 77.77) circle (  2.13);

\path[fill=fillColor,fill opacity=0.20] (219.38, 90.29) circle (  2.13);

\path[fill=fillColor,fill opacity=0.20] (223.32, 86.62) circle (  2.13);

\path[fill=fillColor,fill opacity=0.20] (230.09, 85.99) circle (  2.13);

\path[fill=fillColor,fill opacity=0.20] (233.59, 94.34) circle (  2.13);

\path[fill=fillColor,fill opacity=0.20] (240.80,103.95) circle (  2.13);

\path[fill=fillColor,fill opacity=0.20] (249.10,107.11) circle (  2.13);

\path[fill=fillColor,fill opacity=0.20] (240.36,105.08) circle (  2.13);

\path[fill=fillColor,fill opacity=0.20] (242.54, 97.88) circle (  2.13);

\path[fill=fillColor,fill opacity=0.20] (246.48, 95.09) circle (  2.13);

\path[fill=fillColor,fill opacity=0.20] (196.00, 94.71) circle (  2.13);

\path[fill=fillColor,fill opacity=0.20] (197.10, 51.72) circle (  2.13);

\path[fill=fillColor,fill opacity=0.20] (218.51, 84.60) circle (  2.13);

\path[fill=fillColor,fill opacity=0.20] (235.11, 93.20) circle (  2.13);

\path[fill=fillColor,fill opacity=0.20] (227.69, 98.76) circle (  2.13);

\path[fill=fillColor,fill opacity=0.20] (218.95,107.11) circle (  2.13);

\path[fill=fillColor,fill opacity=0.20] (218.95,103.95) circle (  2.13);

\path[fill=fillColor,fill opacity=0.20] (206.93,102.68) circle (  2.13);

\path[fill=fillColor,fill opacity=0.20] (201.68,105.08) circle (  2.13);

\path[fill=fillColor,fill opacity=0.20] (191.41, 93.07) circle (  2.13);

\path[fill=fillColor,fill opacity=0.20] (217.42, 88.01) circle (  2.13);

\path[fill=fillColor,fill opacity=0.20] (230.09, 91.68) circle (  2.13);

\path[fill=fillColor,fill opacity=0.20] (225.28, 96.49) circle (  2.13);

\path[fill=fillColor,fill opacity=0.20] (223.97, 87.63) circle (  2.13);

\path[fill=fillColor,fill opacity=0.20] (235.11, 89.91) circle (  2.13);

\path[fill=fillColor,fill opacity=0.20] (237.74,104.83) circle (  2.13);

\path[fill=fillColor,fill opacity=0.20] (235.55,106.98) circle (  2.13);

\path[fill=fillColor,fill opacity=0.20] (233.37,101.67) circle (  2.13);

\path[fill=fillColor,fill opacity=0.20] (234.90, 98.51) circle (  2.13);

\path[fill=fillColor,fill opacity=0.20] (197.53, 91.43) circle (  2.13);

\path[fill=fillColor,fill opacity=0.20] (169.78, 78.91) circle (  2.13);

\path[fill=fillColor,fill opacity=0.20] (202.12, 77.01) circle (  2.13);

\path[fill=fillColor,fill opacity=0.20] (237.96, 94.08) circle (  2.13);

\path[fill=fillColor,fill opacity=0.20] (241.89, 96.99) circle (  2.13);

\path[fill=fillColor,fill opacity=0.20] (234.24,104.83) circle (  2.13);

\path[fill=fillColor,fill opacity=0.20] (232.71,106.22) circle (  2.13);

\path[fill=fillColor,fill opacity=0.20] (231.40, 98.63) circle (  2.13);

\path[fill=fillColor,fill opacity=0.20] (214.36, 97.75) circle (  2.13);

\path[fill=fillColor,fill opacity=0.20] (203.65, 94.97) circle (  2.13);

\path[fill=fillColor,fill opacity=0.20] (216.76, 84.47) circle (  2.13);

\path[fill=fillColor,fill opacity=0.20] (245.60, 89.53) circle (  2.13);

\path[fill=fillColor,fill opacity=0.20] (245.17,100.78) circle (  2.13);

\path[fill=fillColor,fill opacity=0.20] (233.15, 98.63) circle (  2.13);

\path[fill=fillColor,fill opacity=0.20] (230.31, 90.42) circle (  2.13);

\path[fill=fillColor,fill opacity=0.20] (234.46, 88.27) circle (  2.13);

\path[fill=fillColor,fill opacity=0.20] (234.90, 94.71) circle (  2.13);

\path[fill=fillColor,fill opacity=0.20] (227.90, 98.76) circle (  2.13);

\path[fill=fillColor,fill opacity=0.20] (216.98,102.05) circle (  2.13);

\path[fill=fillColor,fill opacity=0.20] (191.85,103.57) circle (  2.13);

\path[fill=fillColor,fill opacity=0.20] (167.38, 85.23) circle (  2.13);

\path[fill=fillColor,fill opacity=0.20] (212.61, 87.63) circle (  2.13);

\path[fill=fillColor,fill opacity=0.20] (227.03, 96.99) circle (  2.13);

\path[fill=fillColor,fill opacity=0.20] (231.62,105.08) circle (  2.13);

\path[fill=fillColor,fill opacity=0.20] (244.73,107.61) circle (  2.13);

\path[fill=fillColor,fill opacity=0.20] (240.58, 96.11) circle (  2.13);

\path[fill=fillColor,fill opacity=0.20] (223.10, 87.76) circle (  2.13);

\path[fill=fillColor,fill opacity=0.20] (216.76, 92.06) circle (  2.13);

\path[fill=fillColor,fill opacity=0.20] (201.47, 76.63) circle (  2.13);

\path[fill=fillColor,fill opacity=0.20] (212.83, 84.60) circle (  2.13);

\path[fill=fillColor,fill opacity=0.20] (245.38, 87.00) circle (  2.13);

\path[fill=fillColor,fill opacity=0.20] (248.01, 92.06) circle (  2.13);

\path[fill=fillColor,fill opacity=0.20] (237.52, 96.11) circle (  2.13);

\path[fill=fillColor,fill opacity=0.20] (231.62, 99.52) circle (  2.13);

\path[fill=fillColor,fill opacity=0.20] (221.57, 94.21) circle (  2.13);

\path[fill=fillColor,fill opacity=0.20] (211.08, 92.44) circle (  2.13);

\path[fill=fillColor,fill opacity=0.20] (204.52, 93.96) circle (  2.13);

\path[fill=fillColor,fill opacity=0.20] (189.23, 94.71) circle (  2.13);

\path[fill=fillColor,fill opacity=0.20] (198.84, 66.39) circle (  2.13);

\path[fill=fillColor,fill opacity=0.20] (213.05, 87.00) circle (  2.13);

\path[fill=fillColor,fill opacity=0.20] (224.19,100.66) circle (  2.13);

\path[fill=fillColor,fill opacity=0.20] (231.62,101.42) circle (  2.13);

\path[fill=fillColor,fill opacity=0.20] (226.81, 96.86) circle (  2.13);

\path[fill=fillColor,fill opacity=0.20] (221.57, 91.93) circle (  2.13);

\path[fill=fillColor,fill opacity=0.20] (216.98, 89.78) circle (  2.13);

\path[fill=fillColor,fill opacity=0.20] (206.27, 82.07) circle (  2.13);

\path[fill=fillColor,fill opacity=0.20] (220.04, 85.23) circle (  2.13);

\path[fill=fillColor,fill opacity=0.20] (232.93, 84.35) circle (  2.13);

\path[fill=fillColor,fill opacity=0.20] (265.49, 82.58) circle (  2.13);

\path[fill=fillColor,fill opacity=0.20] (260.46, 87.13) circle (  2.13);

\path[fill=fillColor,fill opacity=0.20] (228.78, 94.46) circle (  2.13);

\path[fill=fillColor,fill opacity=0.20] (203.00, 99.90) circle (  2.13);

\path[fill=fillColor,fill opacity=0.20] (185.73, 99.77) circle (  2.13);

\path[fill=fillColor,fill opacity=0.20] (183.77, 89.15) circle (  2.13);

\path[fill=fillColor,fill opacity=0.20] (185.08, 85.99) circle (  2.13);

\path[fill=fillColor,fill opacity=0.20] (181.80, 86.12) circle (  2.13);

\path[fill=fillColor,fill opacity=0.20] (206.05, 68.92) circle (  2.13);

\path[fill=fillColor,fill opacity=0.20] (220.04, 93.45) circle (  2.13);

\path[fill=fillColor,fill opacity=0.20] (223.10, 96.49) circle (  2.13);

\path[fill=fillColor,fill opacity=0.20] (224.85, 98.13) circle (  2.13);

\path[fill=fillColor,fill opacity=0.20] (223.97,101.16) circle (  2.13);

\path[fill=fillColor,fill opacity=0.20] (220.04, 95.98) circle (  2.13);

\path[fill=fillColor,fill opacity=0.20] (211.95, 87.00) circle (  2.13);

\path[fill=fillColor,fill opacity=0.20] (210.64, 81.56) circle (  2.13);

\path[fill=fillColor,fill opacity=0.20] (230.96, 67.15) circle (  2.13);

\path[fill=fillColor,fill opacity=0.20] (254.56, 71.83) circle (  2.13);

\path[fill=fillColor,fill opacity=0.20] (249.10, 81.18) circle (  2.13);

\path[fill=fillColor,fill opacity=0.20] (251.50, 89.53) circle (  2.13);

\path[fill=fillColor,fill opacity=0.20] (203.43, 99.27) circle (  2.13);

\path[fill=fillColor,fill opacity=0.20] (201.25, 98.89) circle (  2.13);

\path[fill=fillColor,fill opacity=0.20] (168.04, 83.84) circle (  2.13);

\path[fill=fillColor,fill opacity=0.20] (192.51, 51.85) circle (  2.13);

\path[fill=fillColor,fill opacity=0.20] (210.21, 87.89) circle (  2.13);

\path[fill=fillColor,fill opacity=0.20] (217.42, 92.57) circle (  2.13);

\path[fill=fillColor,fill opacity=0.20] (228.56, 93.58) circle (  2.13);

\path[fill=fillColor,fill opacity=0.20] (223.75, 99.27) circle (  2.13);

\path[fill=fillColor,fill opacity=0.20] (218.29,101.16) circle (  2.13);

\path[fill=fillColor,fill opacity=0.20] (218.07, 95.73) circle (  2.13);

\path[fill=fillColor,fill opacity=0.20] (215.01, 86.75) circle (  2.13);

\path[fill=fillColor,fill opacity=0.20] (210.64, 83.46) circle (  2.13);

\path[fill=fillColor,fill opacity=0.20] (232.71, 58.68) circle (  2.13);

\path[fill=fillColor,fill opacity=0.20] (239.70, 62.34) circle (  2.13);

\path[fill=fillColor,fill opacity=0.20] (224.63, 74.48) circle (  2.13);

\path[fill=fillColor,fill opacity=0.20] (211.95, 87.89) circle (  2.13);

\path[fill=fillColor,fill opacity=0.20] (177.65, 95.22) circle (  2.13);

\path[fill=fillColor,fill opacity=0.20] (183.55, 68.54) circle (  2.13);

\path[fill=fillColor,fill opacity=0.20] (199.72, 76.63) circle (  2.13);

\path[fill=fillColor,fill opacity=0.20] (217.63, 84.09) circle (  2.13);

\path[fill=fillColor,fill opacity=0.20] (224.85, 93.20) circle (  2.13);

\path[fill=fillColor,fill opacity=0.20] (222.88, 95.73) circle (  2.13);

\path[fill=fillColor,fill opacity=0.20] (213.92, 95.22) circle (  2.13);

\path[fill=fillColor,fill opacity=0.20] (211.52, 90.79) circle (  2.13);

\path[fill=fillColor,fill opacity=0.20] (210.21, 95.35) circle (  2.13);

\path[fill=fillColor,fill opacity=0.20] (208.02, 98.13) circle (  2.13);

\path[fill=fillColor,fill opacity=0.20] (231.62, 56.40) circle (  2.13);

\path[fill=fillColor,fill opacity=0.20] (233.59, 58.80) circle (  2.13);

\path[fill=fillColor,fill opacity=0.20] (204.31, 58.68) circle (  2.13);

\path[fill=fillColor,fill opacity=0.20] (185.95, 70.94) circle (  2.13);

\path[fill=fillColor,fill opacity=0.20] (170.66, 95.73) circle (  2.13);

\path[fill=fillColor,fill opacity=0.20] (170.66, 86.37) circle (  2.13);

\path[fill=fillColor,fill opacity=0.20] (182.89, 62.72) circle (  2.13);

\path[fill=fillColor,fill opacity=0.20] (202.56, 83.33) circle (  2.13);

\path[fill=fillColor,fill opacity=0.20] (224.85, 90.42) circle (  2.13);

\path[fill=fillColor,fill opacity=0.20] (216.11, 92.94) circle (  2.13);

\path[fill=fillColor,fill opacity=0.20] (216.98, 99.39) circle (  2.13);

\path[fill=fillColor,fill opacity=0.20] (224.63, 98.26) circle (  2.13);

\path[fill=fillColor,fill opacity=0.20] (216.98, 93.58) circle (  2.13);

\path[fill=fillColor,fill opacity=0.20] (217.85, 94.34) circle (  2.13);

\path[fill=fillColor,fill opacity=0.20] (218.07, 81.31) circle (  2.13);

\path[fill=fillColor,fill opacity=0.20] (217.63, 72.33) circle (  2.13);

\path[fill=fillColor,fill opacity=0.20] (245.82, 65.63) circle (  2.13);

\path[fill=fillColor,fill opacity=0.20] (229.22, 63.36) circle (  2.13);

\path[fill=fillColor,fill opacity=0.20] (182.89, 81.18) circle (  2.13);

\path[fill=fillColor,fill opacity=0.20] (208.89, 82.70) circle (  2.13);

\path[fill=fillColor,fill opacity=0.20] (174.37, 59.44) circle (  2.13);

\path[fill=fillColor,fill opacity=0.20] (194.26, 79.03) circle (  2.13);

\path[fill=fillColor,fill opacity=0.20] (209.99, 95.98) circle (  2.13);

\path[fill=fillColor,fill opacity=0.20] (224.85,105.21) circle (  2.13);

\path[fill=fillColor,fill opacity=0.20] (225.50, 94.46) circle (  2.13);

\path[fill=fillColor,fill opacity=0.20] (223.10, 88.14) circle (  2.13);

\path[fill=fillColor,fill opacity=0.20] (218.07, 90.67) circle (  2.13);

\path[fill=fillColor,fill opacity=0.20] (209.33, 93.07) circle (  2.13);

\path[fill=fillColor,fill opacity=0.20] (213.70, 85.10) circle (  2.13);

\path[fill=fillColor,fill opacity=0.20] (209.55, 74.61) circle (  2.13);

\path[fill=fillColor,fill opacity=0.20] (211.52, 67.28) circle (  2.13);

\path[fill=fillColor,fill opacity=0.20] (216.98, 65.50) circle (  2.13);

\path[fill=fillColor,fill opacity=0.20] (223.53, 72.71) circle (  2.13);

\path[fill=fillColor,fill opacity=0.20] (230.53, 79.79) circle (  2.13);

\path[fill=fillColor,fill opacity=0.20] (215.01, 75.75) circle (  2.13);

\path[fill=fillColor,fill opacity=0.20] (211.52, 83.97) circle (  2.13);

\path[fill=fillColor,fill opacity=0.20] (177.65, 51.22) circle (  2.13);

\path[fill=fillColor,fill opacity=0.20] (188.57, 72.84) circle (  2.13);

\path[fill=fillColor,fill opacity=0.20] (208.89, 84.98) circle (  2.13);

\path[fill=fillColor,fill opacity=0.20] (229.43, 87.25) circle (  2.13);

\path[fill=fillColor,fill opacity=0.20] (229.00, 89.78) circle (  2.13);

\path[fill=fillColor,fill opacity=0.20] (222.00, 90.92) circle (  2.13);

\path[fill=fillColor,fill opacity=0.20] (212.39, 94.71) circle (  2.13);

\path[fill=fillColor,fill opacity=0.20] (209.33, 86.50) circle (  2.13);

\path[fill=fillColor,fill opacity=0.20] (207.58, 79.79) circle (  2.13);

\path[fill=fillColor,fill opacity=0.20] (199.94, 84.47) circle (  2.13);

\path[fill=fillColor,fill opacity=0.20] (197.97, 89.15) circle (  2.13);

\path[fill=fillColor,fill opacity=0.20] (198.63, 81.44) circle (  2.13);

\path[fill=fillColor,fill opacity=0.20] (203.00, 61.96) circle (  2.13);

\path[fill=fillColor,fill opacity=0.20] (206.93, 72.33) circle (  2.13);

\path[fill=fillColor,fill opacity=0.20] (208.24, 78.91) circle (  2.13);

\path[fill=fillColor,fill opacity=0.20] (211.74, 81.82) circle (  2.13);

\path[fill=fillColor,fill opacity=0.20] (216.32, 76.89) circle (  2.13);

\path[fill=fillColor,fill opacity=0.20] (229.00, 76.25) circle (  2.13);

\path[fill=fillColor,fill opacity=0.20] (223.53, 82.58) circle (  2.13);

\path[fill=fillColor,fill opacity=0.20] (197.31, 87.89) circle (  2.13);

\path[fill=fillColor,fill opacity=0.20] (166.51, 88.65) circle (  2.13);

\path[fill=fillColor,fill opacity=0.20] (174.15, 53.11) circle (  2.13);

\path[fill=fillColor,fill opacity=0.20] (189.67, 71.45) circle (  2.13);

\path[fill=fillColor,fill opacity=0.20] (213.26, 85.36) circle (  2.13);

\path[fill=fillColor,fill opacity=0.20] (221.35, 90.54) circle (  2.13);

\path[fill=fillColor,fill opacity=0.20] (221.13, 96.49) circle (  2.13);

\path[fill=fillColor,fill opacity=0.20] (217.63, 90.29) circle (  2.13);

\path[fill=fillColor,fill opacity=0.20] (212.17, 80.30) circle (  2.13);

\path[fill=fillColor,fill opacity=0.20] (206.27, 82.20) circle (  2.13);

\path[fill=fillColor,fill opacity=0.20] (205.84, 88.77) circle (  2.13);

\path[fill=fillColor,fill opacity=0.20] (201.25, 90.16) circle (  2.13);

\path[fill=fillColor,fill opacity=0.20] (195.57, 89.02) circle (  2.13);

\path[fill=fillColor,fill opacity=0.20] (210.64, 81.31) circle (  2.13);

\path[fill=fillColor,fill opacity=0.20] (199.72, 67.40) circle (  2.13);

\path[fill=fillColor,fill opacity=0.20] (199.94, 71.07) circle (  2.13);

\path[fill=fillColor,fill opacity=0.20] (201.68, 81.94) circle (  2.13);

\path[fill=fillColor,fill opacity=0.20] (199.06, 92.57) circle (  2.13);

\path[fill=fillColor,fill opacity=0.20] (200.37, 93.96) circle (  2.13);

\path[fill=fillColor,fill opacity=0.20] (203.43, 88.90) circle (  2.13);

\path[fill=fillColor,fill opacity=0.20] (220.69, 89.02) circle (  2.13);

\path[fill=fillColor,fill opacity=0.20] (224.85, 93.83) circle (  2.13);

\path[fill=fillColor,fill opacity=0.20] (221.35, 87.89) circle (  2.13);

\path[fill=fillColor,fill opacity=0.20] (213.70, 86.75) circle (  2.13);

\path[fill=fillColor,fill opacity=0.20] (197.31, 96.11) circle (  2.13);

\path[fill=fillColor,fill opacity=0.20] (171.09, 94.84) circle (  2.13);

\path[fill=fillColor,fill opacity=0.20] (167.38, 78.66) circle (  2.13);

\path[fill=fillColor,fill opacity=0.20] (175.68, 54.38) circle (  2.13);

\path[fill=fillColor,fill opacity=0.20] (202.56, 71.95) circle (  2.13);

\path[fill=fillColor,fill opacity=0.20] (211.52, 87.76) circle (  2.13);

\path[fill=fillColor,fill opacity=0.20] (215.89, 94.71) circle (  2.13);

\path[fill=fillColor,fill opacity=0.20] (221.57, 92.19) circle (  2.13);

\path[fill=fillColor,fill opacity=0.20] (217.42, 88.65) circle (  2.13);

\path[fill=fillColor,fill opacity=0.20] (219.60, 87.00) circle (  2.13);

\path[fill=fillColor,fill opacity=0.20] (217.85, 88.90) circle (  2.13);

\path[fill=fillColor,fill opacity=0.20] (208.68, 90.67) circle (  2.13);

\path[fill=fillColor,fill opacity=0.20] (206.05, 85.48) circle (  2.13);

\path[fill=fillColor,fill opacity=0.20] (205.40, 83.33) circle (  2.13);

\path[fill=fillColor,fill opacity=0.20] (201.47, 88.27) circle (  2.13);

\path[fill=fillColor,fill opacity=0.20] (201.68, 92.19) circle (  2.13);

\path[fill=fillColor,fill opacity=0.20] (201.25, 93.07) circle (  2.13);

\path[fill=fillColor,fill opacity=0.20] (203.00, 93.58) circle (  2.13);

\path[fill=fillColor,fill opacity=0.20] (198.63, 89.40) circle (  2.13);

\path[fill=fillColor,fill opacity=0.20] (202.56, 91.68) circle (  2.13);

\path[fill=fillColor,fill opacity=0.20] (203.65, 96.86) circle (  2.13);

\path[fill=fillColor,fill opacity=0.20] (203.00, 92.06) circle (  2.13);

\path[fill=fillColor,fill opacity=0.20] (196.22, 87.00) circle (  2.13);

\path[fill=fillColor,fill opacity=0.20] (199.50, 82.45) circle (  2.13);

\path[fill=fillColor,fill opacity=0.20] (201.25, 80.05) circle (  2.13);

\path[fill=fillColor,fill opacity=0.20] (198.41, 85.36) circle (  2.13);

\path[fill=fillColor,fill opacity=0.20] (200.81, 89.66) circle (  2.13);

\path[fill=fillColor,fill opacity=0.20] (203.21, 86.87) circle (  2.13);

\path[fill=fillColor,fill opacity=0.20] (203.87, 85.61) circle (  2.13);

\path[fill=fillColor,fill opacity=0.20] (208.68, 84.85) circle (  2.13);

\path[fill=fillColor,fill opacity=0.20] (214.36, 89.02) circle (  2.13);

\path[fill=fillColor,fill opacity=0.20] (216.98, 90.29) circle (  2.13);

\path[fill=fillColor,fill opacity=0.20] (219.38, 85.23) circle (  2.13);

\path[fill=fillColor,fill opacity=0.20] (217.63, 88.14) circle (  2.13);

\path[fill=fillColor,fill opacity=0.20] (210.42, 96.99) circle (  2.13);

\path[fill=fillColor,fill opacity=0.20] (192.73, 97.62) circle (  2.13);

\path[fill=fillColor,fill opacity=0.20] (175.46, 97.24) circle (  2.13);

\path[fill=fillColor,fill opacity=0.20] (215.89, 86.50) circle (  2.13);

\path[fill=fillColor,fill opacity=0.20] (190.76, 59.44) circle (  2.13);

\path[fill=fillColor,fill opacity=0.20] (186.39, 73.47) circle (  2.13);

\path[fill=fillColor,fill opacity=0.20] (196.00, 94.46) circle (  2.13);

\path[fill=fillColor,fill opacity=0.20] (211.52, 95.35) circle (  2.13);

\path[fill=fillColor,fill opacity=0.20] (215.45, 82.32) circle (  2.13);

\path[fill=fillColor,fill opacity=0.20] (217.20, 81.18) circle (  2.13);

\path[fill=fillColor,fill opacity=0.20] (219.38, 88.52) circle (  2.13);

\path[fill=fillColor,fill opacity=0.20] (218.95, 87.51) circle (  2.13);

\path[fill=fillColor,fill opacity=0.20] (215.01, 81.56) circle (  2.13);

\path[fill=fillColor,fill opacity=0.20] (220.04, 79.79) circle (  2.13);

\path[fill=fillColor,fill opacity=0.20] (218.95, 83.46) circle (  2.13);

\path[fill=fillColor,fill opacity=0.20] (215.45, 90.92) circle (  2.13);

\path[fill=fillColor,fill opacity=0.20] (211.08, 94.08) circle (  2.13);

\path[fill=fillColor,fill opacity=0.20] (214.14, 91.05) circle (  2.13);

\path[fill=fillColor,fill opacity=0.20] (213.05, 90.92) circle (  2.13);

\path[fill=fillColor,fill opacity=0.20] (211.95, 93.07) circle (  2.13);

\path[fill=fillColor,fill opacity=0.20] (212.39, 93.32) circle (  2.13);

\path[fill=fillColor,fill opacity=0.20] (208.24, 91.93) circle (  2.13);

\path[fill=fillColor,fill opacity=0.20] (208.68, 89.02) circle (  2.13);

\path[fill=fillColor,fill opacity=0.20] (216.32, 87.51) circle (  2.13);

\path[fill=fillColor,fill opacity=0.20] (213.70, 86.24) circle (  2.13);

\path[fill=fillColor,fill opacity=0.20] (217.42, 85.99) circle (  2.13);

\path[fill=fillColor,fill opacity=0.20] (218.29, 85.86) circle (  2.13);

\path[fill=fillColor,fill opacity=0.20] (221.57, 82.70) circle (  2.13);

\path[fill=fillColor,fill opacity=0.20] (231.40, 81.94) circle (  2.13);

\path[fill=fillColor,fill opacity=0.20] (225.06, 90.16) circle (  2.13);

\path[fill=fillColor,fill opacity=0.20] (212.61, 94.46) circle (  2.13);

\path[fill=fillColor,fill opacity=0.20] (211.74, 89.53) circle (  2.13);

\path[fill=fillColor,fill opacity=0.20] (179.83, 93.20) circle (  2.13);

\path[fill=fillColor,fill opacity=0.20] (169.56, 92.06) circle (  2.13);

\path[fill=fillColor,fill opacity=0.20] (179.40, 76.00) circle (  2.13);

\path[fill=fillColor,fill opacity=0.20] (166.07, 56.15) circle (  2.13);

\path[fill=fillColor,fill opacity=0.20] (175.68, 65.50) circle (  2.13);

\path[fill=fillColor,fill opacity=0.20] (208.46, 64.75) circle (  2.13);

\path[fill=fillColor,fill opacity=0.20] (196.88, 65.25) circle (  2.13);

\path[fill=fillColor,fill opacity=0.20] (192.29, 75.75) circle (  2.13);

\path[fill=fillColor,fill opacity=0.20] (198.84, 87.13) circle (  2.13);

\path[fill=fillColor,fill opacity=0.20] (205.84, 83.08) circle (  2.13);

\path[fill=fillColor,fill opacity=0.20] (224.85, 79.03) circle (  2.13);

\path[fill=fillColor,fill opacity=0.20] (217.85, 84.60) circle (  2.13);

\path[fill=fillColor,fill opacity=0.20] (216.54, 93.58) circle (  2.13);

\path[fill=fillColor,fill opacity=0.20] (215.23, 96.86) circle (  2.13);

\path[fill=fillColor,fill opacity=0.20] (221.57, 94.46) circle (  2.13);

\path[fill=fillColor,fill opacity=0.20] (213.48, 89.78) circle (  2.13);

\path[fill=fillColor,fill opacity=0.20] (214.14, 87.13) circle (  2.13);

\path[fill=fillColor,fill opacity=0.20] (206.27, 87.00) circle (  2.13);

\path[fill=fillColor,fill opacity=0.20] (203.43, 87.00) circle (  2.13);

\path[fill=fillColor,fill opacity=0.20] (192.73, 86.50) circle (  2.13);

\path[fill=fillColor,fill opacity=0.20] (198.41, 90.79) circle (  2.13);

\path[fill=fillColor,fill opacity=0.20] (202.34, 96.36) circle (  2.13);

\path[fill=fillColor,fill opacity=0.20] (197.31, 95.73) circle (  2.13);

\path[fill=fillColor,fill opacity=0.20] (212.83, 91.55) circle (  2.13);

\path[fill=fillColor,fill opacity=0.20] (193.38, 93.70) circle (  2.13);

\path[fill=fillColor,fill opacity=0.20] (192.07, 98.76) circle (  2.13);

\path[fill=fillColor,fill opacity=0.20] (192.94, 99.01) circle (  2.13);

\path[fill=fillColor,fill opacity=0.20] (174.81, 96.86) circle (  2.13);

\path[fill=fillColor,fill opacity=0.20] (168.47, 88.27) circle (  2.13);

\path[fill=fillColor,fill opacity=0.20] (165.19, 60.07) circle (  2.13);

\path[fill=fillColor,fill opacity=0.20] (166.94, 68.29) circle (  2.13);

\path[fill=fillColor,fill opacity=0.20] (172.41, 77.90) circle (  2.13);

\path[fill=fillColor,fill opacity=0.20] (169.35, 86.37) circle (  2.13);

\path[fill=fillColor,fill opacity=0.20] (169.35, 88.14) circle (  2.13);

\path[fill=fillColor,fill opacity=0.20] (177.21, 89.53) circle (  2.13);

\path[fill=fillColor,fill opacity=0.20] (172.41, 90.29) circle (  2.13);

\path[fill=fillColor,fill opacity=0.20] (173.50, 85.23) circle (  2.13);

\path[fill=fillColor,fill opacity=0.20] (169.78, 79.54) circle (  2.13);

\path[fill=fillColor,fill opacity=0.20] (175.25, 74.99) circle (  2.13);

\path[fill=fillColor,fill opacity=0.20] (177.65, 71.83) circle (  2.13);

\path[fill=fillColor,fill opacity=0.20] (170.00, 67.91) circle (  2.13);

\path[fill=fillColor,fill opacity=0.20] (165.19, 65.63) circle (  2.13);

\path[fill=fillColor,fill opacity=0.20] (233.59, 73.22) circle (  2.13);

\path[fill=fillColor,fill opacity=0.20] (183.77, 81.31) circle (  2.13);

\path[fill=fillColor,fill opacity=0.20] (172.62, 73.22) circle (  2.13);

\path[fill=fillColor,fill opacity=0.20] (204.09, 49.83) circle (  2.13);

\path[fill=fillColor,fill opacity=0.20] (215.01, 45.40) circle (  2.13);

\path[fill=fillColor,fill opacity=0.20] (230.74, 53.49) circle (  2.13);

\path[fill=fillColor,fill opacity=0.20] (224.85, 55.77) circle (  2.13);

\path[fill=fillColor,fill opacity=0.20] (215.89, 55.89) circle (  2.13);

\path[fill=fillColor,fill opacity=0.20] (206.05, 59.94) circle (  2.13);

\path[fill=fillColor,fill opacity=0.20] (202.56, 65.38) circle (  2.13);

\path[fill=fillColor,fill opacity=0.20] (201.90, 64.75) circle (  2.13);

\path[fill=fillColor,fill opacity=0.20] (202.78, 58.04) circle (  2.13);

\path[fill=fillColor,fill opacity=0.20] (196.66, 49.95) circle (  2.13);

\path[fill=fillColor,fill opacity=0.20] (191.85, 43.63) circle (  2.13);

\path[fill=fillColor,fill opacity=0.20] (191.63, 38.44) circle (  2.13);

\path[fill=fillColor,fill opacity=0.20] (223.10, 50.46) circle (  2.13);

\path[fill=fillColor,fill opacity=0.20] (212.39, 69.30) circle (  2.13);

\path[fill=fillColor,fill opacity=0.20] (258.71, 71.57) circle (  2.13);

\path[fill=fillColor,fill opacity=0.20] (216.98, 67.91) circle (  2.13);

\path[fill=fillColor,fill opacity=0.20] (211.74, 68.54) circle (  2.13);

\path[fill=fillColor,fill opacity=0.20] (204.52, 72.46) circle (  2.13);

\path[fill=fillColor,fill opacity=0.20] (199.06, 74.99) circle (  2.13);

\path[fill=fillColor,fill opacity=0.20] (198.41, 75.24) circle (  2.13);

\path[fill=fillColor,fill opacity=0.20] (196.22, 70.69) circle (  2.13);

\path[fill=fillColor,fill opacity=0.20] (189.23, 57.66) circle (  2.13);

\path[fill=fillColor,fill opacity=0.20] (180.71, 47.68) circle (  2.13);

\path[fill=fillColor,fill opacity=0.20] (174.37, 42.87) circle (  2.13);

\path[fill=fillColor,fill opacity=0.20] (214.58, 43.25) circle (  2.13);

\path[fill=fillColor,fill opacity=0.20] (218.29, 64.62) circle (  2.13);

\path[fill=fillColor,fill opacity=0.20] (209.33, 83.08) circle (  2.13);

\path[fill=fillColor,fill opacity=0.20] (211.08, 86.75) circle (  2.13);

\path[fill=fillColor,fill opacity=0.20] (211.08, 83.84) circle (  2.13);

\path[fill=fillColor,fill opacity=0.20] (207.58, 83.84) circle (  2.13);

\path[fill=fillColor,fill opacity=0.20] (204.52, 83.84) circle (  2.13);

\path[fill=fillColor,fill opacity=0.20] (198.19, 79.67) circle (  2.13);

\path[fill=fillColor,fill opacity=0.20] (197.75, 76.00) circle (  2.13);

\path[fill=fillColor,fill opacity=0.20] (192.73, 71.07) circle (  2.13);

\path[fill=fillColor,fill opacity=0.20] (180.05, 59.81) circle (  2.13);

\path[fill=fillColor,fill opacity=0.20] (213.70, 70.82) circle (  2.13);

\path[fill=fillColor,fill opacity=0.20] (252.16, 73.60) circle (  2.13);

\path[fill=fillColor,fill opacity=0.20] (214.79, 50.46) circle (  2.13);

\path[fill=fillColor,fill opacity=0.20] (213.26, 71.57) circle (  2.13);

\path[fill=fillColor,fill opacity=0.20] (208.68, 86.12) circle (  2.13);

\path[fill=fillColor,fill opacity=0.20] (205.84, 92.57) circle (  2.13);

\path[fill=fillColor,fill opacity=0.20] (205.18, 92.82) circle (  2.13);

\path[fill=fillColor,fill opacity=0.20] (203.65, 91.30) circle (  2.13);

\path[fill=fillColor,fill opacity=0.20] (201.47, 88.65) circle (  2.13);

\path[fill=fillColor,fill opacity=0.20] (197.53, 80.55) circle (  2.13);

\path[fill=fillColor,fill opacity=0.20] (191.20, 72.46) circle (  2.13);

\path[fill=fillColor,fill opacity=0.20] (183.11, 66.77) circle (  2.13);

\path[fill=fillColor,fill opacity=0.20] (168.25, 57.03) circle (  2.13);

\path[fill=fillColor,fill opacity=0.20] (199.06, 80.17) circle (  2.13);

\path[fill=fillColor,fill opacity=0.20] (203.65, 83.08) circle (  2.13);

\path[fill=fillColor,fill opacity=0.20] (212.39, 89.15) circle (  2.13);

\path[fill=fillColor,fill opacity=0.20] (193.60, 37.94) circle (  2.13);

\path[fill=fillColor,fill opacity=0.20] (215.23, 58.04) circle (  2.13);

\path[fill=fillColor,fill opacity=0.20] (215.67, 78.53) circle (  2.13);

\path[fill=fillColor,fill opacity=0.20] (206.71, 87.76) circle (  2.13);

\path[fill=fillColor,fill opacity=0.20] (204.96, 92.82) circle (  2.13);

\path[fill=fillColor,fill opacity=0.20] (202.34, 93.45) circle (  2.13);

\path[fill=fillColor,fill opacity=0.20] (200.15, 88.65) circle (  2.13);

\path[fill=fillColor,fill opacity=0.20] (199.94, 84.98) circle (  2.13);

\path[fill=fillColor,fill opacity=0.20] (197.10, 78.53) circle (  2.13);

\path[fill=fillColor,fill opacity=0.20] (187.26, 69.05) circle (  2.13);

\path[fill=fillColor,fill opacity=0.20] (205.62, 81.18) circle (  2.13);

\path[fill=fillColor,fill opacity=0.20] (205.40, 87.63) circle (  2.13);

\path[fill=fillColor,fill opacity=0.20] (214.36, 88.14) circle (  2.13);

\path[fill=fillColor,fill opacity=0.20] (224.19, 95.98) circle (  2.13);

\path[fill=fillColor,fill opacity=0.20] (225.06, 99.39) circle (  2.13);

\path[fill=fillColor,fill opacity=0.20] (219.60, 98.63) circle (  2.13);

\path[fill=fillColor,fill opacity=0.20] (213.05, 93.96) circle (  2.13);

\path[fill=fillColor,fill opacity=0.20] (207.37, 87.13) circle (  2.13);

\path[fill=fillColor,fill opacity=0.20] (193.82, 40.85) circle (  2.13);

\path[fill=fillColor,fill opacity=0.20] (218.07, 65.00) circle (  2.13);

\path[fill=fillColor,fill opacity=0.20] (212.83, 84.47) circle (  2.13);

\path[fill=fillColor,fill opacity=0.20] (206.27, 88.90) circle (  2.13);

\path[fill=fillColor,fill opacity=0.20] (205.18, 93.07) circle (  2.13);

\path[fill=fillColor,fill opacity=0.20] (200.15, 93.58) circle (  2.13);

\path[fill=fillColor,fill opacity=0.20] (197.10, 86.50) circle (  2.13);

\path[fill=fillColor,fill opacity=0.20] (197.97, 80.55) circle (  2.13);

\path[fill=fillColor,fill opacity=0.20] (194.91, 74.61) circle (  2.13);

\path[fill=fillColor,fill opacity=0.20] (208.89, 85.23) circle (  2.13);

\path[fill=fillColor,fill opacity=0.20] (218.29, 85.61) circle (  2.13);

\path[fill=fillColor,fill opacity=0.20] (222.22, 90.04) circle (  2.13);

\path[fill=fillColor,fill opacity=0.20] (219.38, 98.76) circle (  2.13);

\path[fill=fillColor,fill opacity=0.20] (234.46,105.08) circle (  2.13);

\path[fill=fillColor,fill opacity=0.20] (229.87,106.98) circle (  2.13);

\path[fill=fillColor,fill opacity=0.20] (219.60,101.67) circle (  2.13);

\path[fill=fillColor,fill opacity=0.20] (215.89, 95.60) circle (  2.13);

\path[fill=fillColor,fill opacity=0.20] (210.64, 92.19) circle (  2.13);

\path[fill=fillColor,fill opacity=0.20] (227.90, 87.89) circle (  2.13);

\path[fill=fillColor,fill opacity=0.20] (191.41, 39.84) circle (  2.13);

\path[fill=fillColor,fill opacity=0.20] (217.85, 61.33) circle (  2.13);

\path[fill=fillColor,fill opacity=0.20] (207.80, 84.73) circle (  2.13);

\path[fill=fillColor,fill opacity=0.20] (202.12, 89.40) circle (  2.13);

\path[fill=fillColor,fill opacity=0.20] (203.00, 89.53) circle (  2.13);

\path[fill=fillColor,fill opacity=0.20] (200.81, 91.43) circle (  2.13);

\path[fill=fillColor,fill opacity=0.20] (198.63, 89.15) circle (  2.13);

\path[fill=fillColor,fill opacity=0.20] (197.75, 81.06) circle (  2.13);

\path[fill=fillColor,fill opacity=0.20] (195.13, 70.31) circle (  2.13);

\path[fill=fillColor,fill opacity=0.20] (206.71, 66.52) circle (  2.13);

\path[fill=fillColor,fill opacity=0.20] (211.30, 99.14) circle (  2.13);

\path[fill=fillColor,fill opacity=0.20] (211.95, 92.94) circle (  2.13);

\path[fill=fillColor,fill opacity=0.20] (219.60, 94.84) circle (  2.13);

\path[fill=fillColor,fill opacity=0.20] (234.68,104.70) circle (  2.13);

\path[fill=fillColor,fill opacity=0.20] (238.61,109.51) circle (  2.13);

\path[fill=fillColor,fill opacity=0.20] (237.52,107.74) circle (  2.13);

\path[fill=fillColor,fill opacity=0.20] (230.09,103.95) circle (  2.13);

\path[fill=fillColor,fill opacity=0.20] (224.41,105.72) circle (  2.13);

\path[fill=fillColor,fill opacity=0.20] (213.05,107.11) circle (  2.13);

\path[fill=fillColor,fill opacity=0.20] (201.68, 94.08) circle (  2.13);

\path[fill=fillColor,fill opacity=0.20] (196.22, 79.41) circle (  2.13);

\path[fill=fillColor,fill opacity=0.20] (212.83, 57.41) circle (  2.13);

\path[fill=fillColor,fill opacity=0.20] (215.23, 84.73) circle (  2.13);

\path[fill=fillColor,fill opacity=0.20] (203.21, 90.29) circle (  2.13);

\path[fill=fillColor,fill opacity=0.20] (203.43, 84.85) circle (  2.13);

\path[fill=fillColor,fill opacity=0.20] (210.42, 87.63) circle (  2.13);

\path[fill=fillColor,fill opacity=0.20] (203.65, 93.70) circle (  2.13);

\path[fill=fillColor,fill opacity=0.20] (196.00, 86.87) circle (  2.13);

\path[fill=fillColor,fill opacity=0.20] (194.04, 72.08) circle (  2.13);

\path[fill=fillColor,fill opacity=0.20] (194.91, 63.48) circle (  2.13);

\path[fill=fillColor,fill opacity=0.20] (188.57, 57.54) circle (  2.13);

\path[fill=fillColor,fill opacity=0.20] (219.16, 81.06) circle (  2.13);

\path[fill=fillColor,fill opacity=0.20] (215.45,114.19) circle (  2.13);

\path[fill=fillColor,fill opacity=0.20] (217.42,101.67) circle (  2.13);

\path[fill=fillColor,fill opacity=0.20] (221.35,105.46) circle (  2.13);

\path[fill=fillColor,fill opacity=0.20] (227.03,114.19) circle (  2.13);

\path[fill=fillColor,fill opacity=0.20] (229.22,109.89) circle (  2.13);

\path[fill=fillColor,fill opacity=0.20] (234.02,105.21) circle (  2.13);

\path[fill=fillColor,fill opacity=0.20] (239.27,105.46) circle (  2.13);

\path[fill=fillColor,fill opacity=0.20] (237.96,109.89) circle (  2.13);

\path[fill=fillColor,fill opacity=0.20] (226.16,115.58) circle (  2.13);

\path[fill=fillColor,fill opacity=0.20] (218.07,103.06) circle (  2.13);

\path[fill=fillColor,fill opacity=0.20] (196.00, 86.12) circle (  2.13);

\path[fill=fillColor,fill opacity=0.20] (196.00, 84.73) circle (  2.13);

\path[fill=fillColor,fill opacity=0.20] (208.46, 58.30) circle (  2.13);

\path[fill=fillColor,fill opacity=0.20] (222.44, 86.24) circle (  2.13);

\path[fill=fillColor,fill opacity=0.20] (211.74, 91.30) circle (  2.13);

\path[fill=fillColor,fill opacity=0.20] (207.15, 86.75) circle (  2.13);

\path[fill=fillColor,fill opacity=0.20] (205.40, 91.05) circle (  2.13);

\path[fill=fillColor,fill opacity=0.20] (198.84, 97.12) circle (  2.13);

\path[fill=fillColor,fill opacity=0.20] (192.94, 90.67) circle (  2.13);

\path[fill=fillColor,fill opacity=0.20] (194.91, 79.79) circle (  2.13);

\path[fill=fillColor,fill opacity=0.20] (196.44, 73.34) circle (  2.13);

\path[fill=fillColor,fill opacity=0.20] (193.82, 68.92) circle (  2.13);

\path[fill=fillColor,fill opacity=0.20] (225.28, 82.95) circle (  2.13);

\path[fill=fillColor,fill opacity=0.20] (228.56,106.10) circle (  2.13);

\path[fill=fillColor,fill opacity=0.20] (218.07,100.15) circle (  2.13);

\path[fill=fillColor,fill opacity=0.20] (222.88,110.39) circle (  2.13);

\path[fill=fillColor,fill opacity=0.20] (225.72,112.92) circle (  2.13);

\path[fill=fillColor,fill opacity=0.20] (226.16,106.35) circle (  2.13);

\path[fill=fillColor,fill opacity=0.20] (230.09,106.10) circle (  2.13);

\path[fill=fillColor,fill opacity=0.20] (235.99,109.76) circle (  2.13);

\path[fill=fillColor,fill opacity=0.20] (238.39,110.52) circle (  2.13);

\path[fill=fillColor,fill opacity=0.20] (234.68,114.19) circle (  2.13);

\path[fill=fillColor,fill opacity=0.20] (224.41,111.41) circle (  2.13);

\path[fill=fillColor,fill opacity=0.20] (204.74, 98.38) circle (  2.13);

\path[fill=fillColor,fill opacity=0.20] (196.00, 89.15) circle (  2.13);

\path[fill=fillColor,fill opacity=0.20] (196.00, 51.97) circle (  2.13);

\path[fill=fillColor,fill opacity=0.20] (219.16, 79.79) circle (  2.13);

\path[fill=fillColor,fill opacity=0.20] (214.14, 89.15) circle (  2.13);

\path[fill=fillColor,fill opacity=0.20] (205.84, 90.04) circle (  2.13);

\path[fill=fillColor,fill opacity=0.20] (204.31, 96.11) circle (  2.13);

\path[fill=fillColor,fill opacity=0.20] (200.15, 96.99) circle (  2.13);

\path[fill=fillColor,fill opacity=0.20] (199.28, 88.90) circle (  2.13);

\path[fill=fillColor,fill opacity=0.20] (197.53, 82.58) circle (  2.13);

\path[fill=fillColor,fill opacity=0.20] (200.15, 80.30) circle (  2.13);

\path[fill=fillColor,fill opacity=0.20] (198.84, 80.68) circle (  2.13);

\path[fill=fillColor,fill opacity=0.20] (194.04, 74.86) circle (  2.13);

\path[fill=fillColor,fill opacity=0.20] (222.88, 76.76) circle (  2.13);

\path[fill=fillColor,fill opacity=0.20] (231.62, 95.85) circle (  2.13);

\path[fill=fillColor,fill opacity=0.20] (226.16, 96.11) circle (  2.13);

\path[fill=fillColor,fill opacity=0.20] (224.41,103.69) circle (  2.13);

\path[fill=fillColor,fill opacity=0.20] (220.04,101.54) circle (  2.13);

\path[fill=fillColor,fill opacity=0.20] (226.16,103.95) circle (  2.13);

\path[fill=fillColor,fill opacity=0.20] (232.06,111.15) circle (  2.13);

\path[fill=fillColor,fill opacity=0.20] (234.02,109.26) circle (  2.13);

\path[fill=fillColor,fill opacity=0.20] (234.02,108.12) circle (  2.13);

\path[fill=fillColor,fill opacity=0.20] (238.17,114.44) circle (  2.13);

\path[fill=fillColor,fill opacity=0.20] (215.23,109.76) circle (  2.13);

\path[fill=fillColor,fill opacity=0.20] (196.00, 96.23) circle (  2.13);

\path[fill=fillColor,fill opacity=0.20] (177.87, 38.44) circle (  2.13);

\path[fill=fillColor,fill opacity=0.20] (203.65, 64.24) circle (  2.13);

\path[fill=fillColor,fill opacity=0.20] (209.99, 82.07) circle (  2.13);

\path[fill=fillColor,fill opacity=0.20] (206.71, 86.24) circle (  2.13);

\path[fill=fillColor,fill opacity=0.20] (205.40, 91.30) circle (  2.13);

\path[fill=fillColor,fill opacity=0.20] (207.15, 94.71) circle (  2.13);

\path[fill=fillColor,fill opacity=0.20] (204.52, 89.78) circle (  2.13);

\path[fill=fillColor,fill opacity=0.20] (203.43, 81.44) circle (  2.13);

\path[fill=fillColor,fill opacity=0.20] (199.28, 78.53) circle (  2.13);

\path[fill=fillColor,fill opacity=0.20] (201.90, 85.48) circle (  2.13);

\path[fill=fillColor,fill opacity=0.20] (198.84, 89.53) circle (  2.13);

\path[fill=fillColor,fill opacity=0.20] (192.73, 75.12) circle (  2.13);

\path[fill=fillColor,fill opacity=0.20] (186.83, 56.40) circle (  2.13);

\path[fill=fillColor,fill opacity=0.20] (219.16, 91.43) circle (  2.13);

\path[fill=fillColor,fill opacity=0.20] (227.03, 94.71) circle (  2.13);

\path[fill=fillColor,fill opacity=0.20] (225.50, 99.27) circle (  2.13);

\path[fill=fillColor,fill opacity=0.20] (223.75, 95.22) circle (  2.13);

\path[fill=fillColor,fill opacity=0.20] (232.71,102.05) circle (  2.13);

\path[fill=fillColor,fill opacity=0.20] (236.21,112.92) circle (  2.13);

\path[fill=fillColor,fill opacity=0.20] (237.08,104.07) circle (  2.13);

\path[fill=fillColor,fill opacity=0.20] (235.33,102.81) circle (  2.13);

\path[fill=fillColor,fill opacity=0.20] (236.21,114.95) circle (  2.13);

\path[fill=fillColor,fill opacity=0.20] (215.23,110.14) circle (  2.13);

\path[fill=fillColor,fill opacity=0.20] (199.50,103.31) circle (  2.13);

\path[fill=fillColor,fill opacity=0.20] (181.58, 79.79) circle (  2.13);

\path[fill=fillColor,fill opacity=0.20] (179.18, 48.18) circle (  2.13);

\path[fill=fillColor,fill opacity=0.20] (203.43, 70.44) circle (  2.13);

\path[fill=fillColor,fill opacity=0.20] (208.89, 82.58) circle (  2.13);

\path[fill=fillColor,fill opacity=0.20] (209.77, 87.63) circle (  2.13);

\path[fill=fillColor,fill opacity=0.20] (204.52, 91.55) circle (  2.13);

\path[fill=fillColor,fill opacity=0.20] (206.05, 92.44) circle (  2.13);

\path[fill=fillColor,fill opacity=0.20] (205.18, 86.24) circle (  2.13);

\path[fill=fillColor,fill opacity=0.20] (203.87, 81.82) circle (  2.13);

\path[fill=fillColor,fill opacity=0.20] (206.71, 86.24) circle (  2.13);

\path[fill=fillColor,fill opacity=0.20] (202.34, 89.15) circle (  2.13);

\path[fill=fillColor,fill opacity=0.20] (194.47, 79.03) circle (  2.13);

\path[fill=fillColor,fill opacity=0.20] (194.69, 68.03) circle (  2.13);

\path[fill=fillColor,fill opacity=0.20] (190.54, 59.18) circle (  2.13);

\path[fill=fillColor,fill opacity=0.20] (204.52, 77.39) circle (  2.13);

\path[fill=fillColor,fill opacity=0.20] (217.20, 90.67) circle (  2.13);

\path[fill=fillColor,fill opacity=0.20] (222.22, 94.08) circle (  2.13);

\path[fill=fillColor,fill opacity=0.20] (228.12, 99.52) circle (  2.13);

\path[fill=fillColor,fill opacity=0.20] (229.00, 98.38) circle (  2.13);

\path[fill=fillColor,fill opacity=0.20] (233.15,100.53) circle (  2.13);

\path[fill=fillColor,fill opacity=0.20] (231.40,106.73) circle (  2.13);

\path[fill=fillColor,fill opacity=0.20] (234.02,102.30) circle (  2.13);

\path[fill=fillColor,fill opacity=0.20] (232.49,103.19) circle (  2.13);

\path[fill=fillColor,fill opacity=0.20] (230.74,111.28) circle (  2.13);

\path[fill=fillColor,fill opacity=0.20] (232.27,107.49) circle (  2.13);

\path[fill=fillColor,fill opacity=0.20] (222.88,105.34) circle (  2.13);

\path[fill=fillColor,fill opacity=0.20] (199.28,106.98) circle (  2.13);

\path[fill=fillColor,fill opacity=0.20] (197.31, 79.16) circle (  2.13);

\path[fill=fillColor,fill opacity=0.20] (183.99, 54.25) circle (  2.13);

\path[fill=fillColor,fill opacity=0.20] (203.21, 79.54) circle (  2.13);

\path[fill=fillColor,fill opacity=0.20] (204.74, 90.16) circle (  2.13);

\path[fill=fillColor,fill opacity=0.20] (205.84, 89.53) circle (  2.13);

\path[fill=fillColor,fill opacity=0.20] (208.68, 90.92) circle (  2.13);

\path[fill=fillColor,fill opacity=0.20] (205.18, 92.06) circle (  2.13);

\path[fill=fillColor,fill opacity=0.20] (224.41, 89.66) circle (  2.13);

\path[fill=fillColor,fill opacity=0.20] (203.87, 88.27) circle (  2.13);

\path[fill=fillColor,fill opacity=0.20] (200.37, 85.10) circle (  2.13);

\path[fill=fillColor,fill opacity=0.20] (198.63, 78.91) circle (  2.13);

\path[fill=fillColor,fill opacity=0.20] (198.63, 72.84) circle (  2.13);

\path[fill=fillColor,fill opacity=0.20] (196.44, 66.52) circle (  2.13);

\path[fill=fillColor,fill opacity=0.20] (194.26, 63.61) circle (  2.13);

\path[fill=fillColor,fill opacity=0.20] (191.41, 58.30) circle (  2.13);

\path[fill=fillColor,fill opacity=0.20] (196.88, 68.16) circle (  2.13);

\path[fill=fillColor,fill opacity=0.20] (211.95, 83.33) circle (  2.13);

\path[fill=fillColor,fill opacity=0.20] (217.63, 92.06) circle (  2.13);

\path[fill=fillColor,fill opacity=0.20] (219.82, 93.20) circle (  2.13);

\path[fill=fillColor,fill opacity=0.20] (220.48, 99.27) circle (  2.13);

\path[fill=fillColor,fill opacity=0.20] (226.59,106.10) circle (  2.13);

\path[fill=fillColor,fill opacity=0.20] (239.92,106.47) circle (  2.13);

\path[fill=fillColor,fill opacity=0.20] (228.12,103.06) circle (  2.13);

\path[fill=fillColor,fill opacity=0.20] (231.84,103.06) circle (  2.13);

\path[fill=fillColor,fill opacity=0.20] (233.59,109.26) circle (  2.13);

\path[fill=fillColor,fill opacity=0.20] (224.19,106.22) circle (  2.13);

\path[fill=fillColor,fill opacity=0.20] (232.49, 98.26) circle (  2.13);

\path[fill=fillColor,fill opacity=0.20] (236.64,103.95) circle (  2.13);

\path[fill=fillColor,fill opacity=0.20] (200.15,104.32) circle (  2.13);

\path[fill=fillColor,fill opacity=0.20] (218.73, 77.26) circle (  2.13);

\path[fill=fillColor,fill opacity=0.20] (173.72, 56.27) circle (  2.13);

\path[fill=fillColor,fill opacity=0.20] (186.61, 75.12) circle (  2.13);

\path[fill=fillColor,fill opacity=0.20] (199.28, 83.84) circle (  2.13);

\path[fill=fillColor,fill opacity=0.20] (205.62, 88.65) circle (  2.13);

\path[fill=fillColor,fill opacity=0.20] (205.62, 90.92) circle (  2.13);

\path[fill=fillColor,fill opacity=0.20] (203.00, 92.31) circle (  2.13);

\path[fill=fillColor,fill opacity=0.20] (208.89, 91.93) circle (  2.13);

\path[fill=fillColor,fill opacity=0.20] (199.50, 84.47) circle (  2.13);

\path[fill=fillColor,fill opacity=0.20] (198.84, 79.16) circle (  2.13);

\path[fill=fillColor,fill opacity=0.20] (197.75, 77.01) circle (  2.13);

\path[fill=fillColor,fill opacity=0.20] (195.35, 70.31) circle (  2.13);

\path[fill=fillColor,fill opacity=0.20] (195.13, 68.54) circle (  2.13);

\path[fill=fillColor,fill opacity=0.20] (196.00, 69.30) circle (  2.13);

\path[fill=fillColor,fill opacity=0.20] (193.60, 60.32) circle (  2.13);

\path[fill=fillColor,fill opacity=0.20] (189.01, 54.76) circle (  2.13);

\path[fill=fillColor,fill opacity=0.20] (193.16, 63.73) circle (  2.13);

\path[fill=fillColor,fill opacity=0.20] (192.51, 66.77) circle (  2.13);

\path[fill=fillColor,fill opacity=0.20] (197.53, 70.82) circle (  2.13);

\path[fill=fillColor,fill opacity=0.20] (205.84, 72.46) circle (  2.13);

\path[fill=fillColor,fill opacity=0.20] (214.79, 78.40) circle (  2.13);

\path[fill=fillColor,fill opacity=0.20] (217.42, 86.87) circle (  2.13);

\path[fill=fillColor,fill opacity=0.20] (216.11, 92.31) circle (  2.13);

\path[fill=fillColor,fill opacity=0.20] (214.58, 94.21) circle (  2.13);

\path[fill=fillColor,fill opacity=0.20] (217.42, 98.13) circle (  2.13);

\path[fill=fillColor,fill opacity=0.20] (228.78,105.72) circle (  2.13);

\path[fill=fillColor,fill opacity=0.20] (226.37,110.77) circle (  2.13);

\path[fill=fillColor,fill opacity=0.20] (232.93,106.10) circle (  2.13);

\path[fill=fillColor,fill opacity=0.20] (230.96,103.69) circle (  2.13);

\path[fill=fillColor,fill opacity=0.20] (229.65,106.73) circle (  2.13);

\path[fill=fillColor,fill opacity=0.20] (223.53,101.80) circle (  2.13);

\path[fill=fillColor,fill opacity=0.20] (233.15, 97.24) circle (  2.13);

\path[fill=fillColor,fill opacity=0.20] (221.13,103.06) circle (  2.13);

\path[fill=fillColor,fill opacity=0.20] (199.50, 96.74) circle (  2.13);

\path[fill=fillColor,fill opacity=0.20] (204.31, 75.12) circle (  2.13);

\path[fill=fillColor,fill opacity=0.20] (173.50, 43.88) circle (  2.13);

\path[fill=fillColor,fill opacity=0.20] (177.43, 60.83) circle (  2.13);

\path[fill=fillColor,fill opacity=0.20] (183.55, 76.38) circle (  2.13);

\path[fill=fillColor,fill opacity=0.20] (190.54, 83.33) circle (  2.13);

\path[fill=fillColor,fill opacity=0.20] (204.31, 89.91) circle (  2.13);

\path[fill=fillColor,fill opacity=0.20] (202.56, 93.20) circle (  2.13);

\path[fill=fillColor,fill opacity=0.20] (202.78, 83.97) circle (  2.13);

\path[fill=fillColor,fill opacity=0.20] (200.81, 76.25) circle (  2.13);

\path[fill=fillColor,fill opacity=0.20] (198.84, 77.26) circle (  2.13);

\path[fill=fillColor,fill opacity=0.20] (195.35, 74.74) circle (  2.13);

\path[fill=fillColor,fill opacity=0.20] (197.97, 71.32) circle (  2.13);

\path[fill=fillColor,fill opacity=0.20] (202.12, 69.68) circle (  2.13);

\path[fill=fillColor,fill opacity=0.20] (198.84, 66.64) circle (  2.13);

\path[fill=fillColor,fill opacity=0.20] (196.88, 69.17) circle (  2.13);

\path[fill=fillColor,fill opacity=0.20] (192.94, 71.83) circle (  2.13);

\path[fill=fillColor,fill opacity=0.20] (190.98, 61.33) circle (  2.13);

\path[fill=fillColor,fill opacity=0.20] (187.26, 53.24) circle (  2.13);

\path[fill=fillColor,fill opacity=0.20] (192.73, 57.92) circle (  2.13);

\path[fill=fillColor,fill opacity=0.20] (197.97, 61.84) circle (  2.13);

\path[fill=fillColor,fill opacity=0.20] (213.26, 67.65) circle (  2.13);

\path[fill=fillColor,fill opacity=0.20] (209.33, 69.68) circle (  2.13);

\path[fill=fillColor,fill opacity=0.20] (206.71, 72.08) circle (  2.13);

\path[fill=fillColor,fill opacity=0.20] (204.31, 78.28) circle (  2.13);

\path[fill=fillColor,fill opacity=0.20] (213.70, 78.78) circle (  2.13);

\path[fill=fillColor,fill opacity=0.20] (218.51, 81.82) circle (  2.13);

\path[fill=fillColor,fill opacity=0.20] (214.79, 89.66) circle (  2.13);

\path[fill=fillColor,fill opacity=0.20] (210.64, 95.22) circle (  2.13);

\path[fill=fillColor,fill opacity=0.20] (213.70, 95.35) circle (  2.13);

\path[fill=fillColor,fill opacity=0.20] (216.98, 96.61) circle (  2.13);

\path[fill=fillColor,fill opacity=0.20] (220.69, 97.37) circle (  2.13);

\path[fill=fillColor,fill opacity=0.20] (223.53,100.53) circle (  2.13);

\path[fill=fillColor,fill opacity=0.20] (226.16,105.21) circle (  2.13);

\path[fill=fillColor,fill opacity=0.20] (230.31,103.44) circle (  2.13);

\path[fill=fillColor,fill opacity=0.20] (224.85, 97.50) circle (  2.13);

\path[fill=fillColor,fill opacity=0.20] (223.10, 98.38) circle (  2.13);

\path[fill=fillColor,fill opacity=0.20] (232.27,102.93) circle (  2.13);

\path[fill=fillColor,fill opacity=0.20] (230.53,103.06) circle (  2.13);

\path[fill=fillColor,fill opacity=0.20] (194.47, 93.07) circle (  2.13);

\path[fill=fillColor,fill opacity=0.20] (171.53, 74.61) circle (  2.13);

\path[fill=fillColor,fill opacity=0.20] (196.88, 45.91) circle (  2.13);

\path[fill=fillColor,fill opacity=0.20] (174.37, 58.30) circle (  2.13);

\path[fill=fillColor,fill opacity=0.20] (184.42, 70.18) circle (  2.13);

\path[fill=fillColor,fill opacity=0.20] (198.84, 78.53) circle (  2.13);

\path[fill=fillColor,fill opacity=0.20] (205.40, 80.93) circle (  2.13);

\path[fill=fillColor,fill opacity=0.20] (205.40, 76.89) circle (  2.13);

\path[fill=fillColor,fill opacity=0.20] (205.62, 77.26) circle (  2.13);

\path[fill=fillColor,fill opacity=0.20] (204.96, 79.67) circle (  2.13);

\path[fill=fillColor,fill opacity=0.20] (204.96, 78.66) circle (  2.13);

\path[fill=fillColor,fill opacity=0.20] (205.62, 76.38) circle (  2.13);

\path[fill=fillColor,fill opacity=0.20] (201.03, 78.28) circle (  2.13);

\path[fill=fillColor,fill opacity=0.20] (196.44, 82.45) circle (  2.13);

\path[fill=fillColor,fill opacity=0.20] (194.91, 76.89) circle (  2.13);

\path[fill=fillColor,fill opacity=0.20] (194.91, 69.05) circle (  2.13);

\path[fill=fillColor,fill opacity=0.20] (198.84, 69.55) circle (  2.13);

\path[fill=fillColor,fill opacity=0.20] (193.60, 64.49) circle (  2.13);

\path[fill=fillColor,fill opacity=0.20] (194.26, 58.93) circle (  2.13);

\path[fill=fillColor,fill opacity=0.20] (196.44, 62.09) circle (  2.13);

\path[fill=fillColor,fill opacity=0.20] (201.03, 70.69) circle (  2.13);

\path[fill=fillColor,fill opacity=0.20] (209.55, 81.31) circle (  2.13);

\path[fill=fillColor,fill opacity=0.20] (210.86, 83.46) circle (  2.13);

\path[fill=fillColor,fill opacity=0.20] (206.27, 83.21) circle (  2.13);

\path[fill=fillColor,fill opacity=0.20] (205.40, 88.90) circle (  2.13);

\path[fill=fillColor,fill opacity=0.20] (216.98, 89.53) circle (  2.13);

\path[fill=fillColor,fill opacity=0.20] (226.81, 90.16) circle (  2.13);

\path[fill=fillColor,fill opacity=0.20] (216.54, 98.89) circle (  2.13);

\path[fill=fillColor,fill opacity=0.20] (218.51,102.30) circle (  2.13);

\path[fill=fillColor,fill opacity=0.20] (221.35, 96.11) circle (  2.13);

\path[fill=fillColor,fill opacity=0.20] (217.42, 94.59) circle (  2.13);

\path[fill=fillColor,fill opacity=0.20] (216.98, 92.94) circle (  2.13);

\path[fill=fillColor,fill opacity=0.20] (218.51, 92.31) circle (  2.13);

\path[fill=fillColor,fill opacity=0.20] (221.35, 99.65) circle (  2.13);

\path[fill=fillColor,fill opacity=0.20] (222.44,100.91) circle (  2.13);

\path[fill=fillColor,fill opacity=0.20] (217.63, 91.93) circle (  2.13);

\path[fill=fillColor,fill opacity=0.20] (220.69, 93.83) circle (  2.13);

\path[fill=fillColor,fill opacity=0.20] (230.74,104.07) circle (  2.13);

\path[fill=fillColor,fill opacity=0.20] (222.44,106.10) circle (  2.13);

\path[fill=fillColor,fill opacity=0.20] (190.76, 97.50) circle (  2.13);

\path[fill=fillColor,fill opacity=0.20] (168.04, 72.59) circle (  2.13);

\path[fill=fillColor,fill opacity=0.20] (172.62, 39.46) circle (  2.13);

\path[fill=fillColor,fill opacity=0.20] (176.99, 49.95) circle (  2.13);

\path[fill=fillColor,fill opacity=0.20] (189.01, 65.38) circle (  2.13);

\path[fill=fillColor,fill opacity=0.20] (194.69, 71.95) circle (  2.13);

\path[fill=fillColor,fill opacity=0.20] (197.53, 72.84) circle (  2.13);

\path[fill=fillColor,fill opacity=0.20] (204.96, 81.56) circle (  2.13);

\path[fill=fillColor,fill opacity=0.20] (204.31, 81.69) circle (  2.13);

\path[fill=fillColor,fill opacity=0.20] (203.43, 77.39) circle (  2.13);

\path[fill=fillColor,fill opacity=0.20] (211.52, 85.61) circle (  2.13);

\path[fill=fillColor,fill opacity=0.20] (201.68, 88.77) circle (  2.13);

\path[fill=fillColor,fill opacity=0.20] (199.06, 75.37) circle (  2.13);

\path[fill=fillColor,fill opacity=0.20] (197.75, 72.97) circle (  2.13);

\path[fill=fillColor,fill opacity=0.20] (198.19, 87.13) circle (  2.13);

\path[fill=fillColor,fill opacity=0.20] (196.66, 84.73) circle (  2.13);

\path[fill=fillColor,fill opacity=0.20] (192.94, 63.23) circle (  2.13);

\path[fill=fillColor,fill opacity=0.20] (200.15, 44.77) circle (  2.13);

\path[fill=fillColor,fill opacity=0.20] (208.68, 61.33) circle (  2.13);

\path[fill=fillColor,fill opacity=0.20] (205.84, 81.94) circle (  2.13);

\path[fill=fillColor,fill opacity=0.20] (193.82, 77.90) circle (  2.13);

\path[fill=fillColor,fill opacity=0.20] (193.38, 71.32) circle (  2.13);

\path[fill=fillColor,fill opacity=0.20] (201.03, 72.97) circle (  2.13);

\path[fill=fillColor,fill opacity=0.20] (201.25, 81.44) circle (  2.13);

\path[fill=fillColor,fill opacity=0.20] (203.21, 95.35) circle (  2.13);

\path[fill=fillColor,fill opacity=0.20] (204.09,101.92) circle (  2.13);

\path[fill=fillColor,fill opacity=0.20] (206.71, 95.85) circle (  2.13);

\path[fill=fillColor,fill opacity=0.20] (207.80, 92.31) circle (  2.13);

\path[fill=fillColor,fill opacity=0.20] (214.58, 94.71) circle (  2.13);

\path[fill=fillColor,fill opacity=0.20] (217.20, 99.90) circle (  2.13);

\path[fill=fillColor,fill opacity=0.20] (217.42,105.72) circle (  2.13);

\path[fill=fillColor,fill opacity=0.20] (231.18,104.96) circle (  2.13);

\path[fill=fillColor,fill opacity=0.20] (217.42, 98.26) circle (  2.13);

\path[fill=fillColor,fill opacity=0.20] (214.36, 95.98) circle (  2.13);

\path[fill=fillColor,fill opacity=0.20] (212.39, 93.83) circle (  2.13);

\path[fill=fillColor,fill opacity=0.20] (215.01, 92.44) circle (  2.13);

\path[fill=fillColor,fill opacity=0.20] (217.42, 96.99) circle (  2.13);

\path[fill=fillColor,fill opacity=0.20] (217.42, 98.51) circle (  2.13);

\path[fill=fillColor,fill opacity=0.20] (218.51, 91.43) circle (  2.13);

\path[fill=fillColor,fill opacity=0.20] (225.50, 90.04) circle (  2.13);

\path[fill=fillColor,fill opacity=0.20] (221.57, 99.27) circle (  2.13);

\path[fill=fillColor,fill opacity=0.20] (213.05,106.85) circle (  2.13);

\path[fill=fillColor,fill opacity=0.20] (185.08, 91.43) circle (  2.13);

\path[fill=fillColor,fill opacity=0.20] (175.25, 39.46) circle (  2.13);

\path[fill=fillColor,fill opacity=0.20] (179.62, 44.01) circle (  2.13);

\path[fill=fillColor,fill opacity=0.20] (181.58, 51.85) circle (  2.13);

\path[fill=fillColor,fill opacity=0.20] (193.16, 64.75) circle (  2.13);

\path[fill=fillColor,fill opacity=0.20] (194.26, 66.14) circle (  2.13);

\path[fill=fillColor,fill opacity=0.20] (199.06, 67.78) circle (  2.13);

\path[fill=fillColor,fill opacity=0.20] (204.31, 81.44) circle (  2.13);

\path[fill=fillColor,fill opacity=0.20] (209.33, 85.23) circle (  2.13);

\path[fill=fillColor,fill opacity=0.20] (201.68, 75.24) circle (  2.13);

\path[fill=fillColor,fill opacity=0.20] (200.15, 78.91) circle (  2.13);

\path[fill=fillColor,fill opacity=0.20] (200.37, 92.06) circle (  2.13);

\path[fill=fillColor,fill opacity=0.20] (204.96, 85.10) circle (  2.13);

\path[fill=fillColor,fill opacity=0.20] (193.82, 69.42) circle (  2.13);

\path[fill=fillColor,fill opacity=0.20] (192.29, 65.88) circle (  2.13);

\path[fill=fillColor,fill opacity=0.20] (199.72, 69.93) circle (  2.13);

\path[fill=fillColor,fill opacity=0.20] (199.50, 70.31) circle (  2.13);

\path[fill=fillColor,fill opacity=0.20] (198.19, 61.96) circle (  2.13);

\path[fill=fillColor,fill opacity=0.20] (205.40, 51.85) circle (  2.13);

\path[fill=fillColor,fill opacity=0.20] (204.09, 61.21) circle (  2.13);

\path[fill=fillColor,fill opacity=0.20] (196.00, 59.18) circle (  2.13);

\path[fill=fillColor,fill opacity=0.20] (189.23, 53.87) circle (  2.13);

\path[fill=fillColor,fill opacity=0.20] (195.35, 55.77) circle (  2.13);

\path[fill=fillColor,fill opacity=0.20] (202.12, 63.36) circle (  2.13);

\path[fill=fillColor,fill opacity=0.20] (202.78, 75.49) circle (  2.13);

\path[fill=fillColor,fill opacity=0.20] (206.05, 94.34) circle (  2.13);

\path[fill=fillColor,fill opacity=0.20] (209.99,104.96) circle (  2.13);

\path[fill=fillColor,fill opacity=0.20] (207.58,101.92) circle (  2.13);

\path[fill=fillColor,fill opacity=0.20] (207.58,101.04) circle (  2.13);

\path[fill=fillColor,fill opacity=0.20] (207.15,100.03) circle (  2.13);

\path[fill=fillColor,fill opacity=0.20] (210.21, 95.47) circle (  2.13);

\path[fill=fillColor,fill opacity=0.20] (217.20, 95.47) circle (  2.13);

\path[fill=fillColor,fill opacity=0.20] (211.30, 99.27) circle (  2.13);

\path[fill=fillColor,fill opacity=0.20] (216.32,100.91) circle (  2.13);

\path[fill=fillColor,fill opacity=0.20] (218.29, 99.39) circle (  2.13);

\path[fill=fillColor,fill opacity=0.20] (215.45, 99.39) circle (  2.13);

\path[fill=fillColor,fill opacity=0.20] (209.11, 99.27) circle (  2.13);

\path[fill=fillColor,fill opacity=0.20] (211.08, 95.35) circle (  2.13);

\path[fill=fillColor,fill opacity=0.20] (210.64, 93.07) circle (  2.13);

\path[fill=fillColor,fill opacity=0.20] (209.33, 95.35) circle (  2.13);

\path[fill=fillColor,fill opacity=0.20] (210.64, 97.24) circle (  2.13);

\path[fill=fillColor,fill opacity=0.20] (217.42, 93.45) circle (  2.13);

\path[fill=fillColor,fill opacity=0.20] (226.16, 91.05) circle (  2.13);

\path[fill=fillColor,fill opacity=0.20] (212.17, 96.86) circle (  2.13);

\path[fill=fillColor,fill opacity=0.20] (200.37, 91.17) circle (  2.13);

\path[fill=fillColor,fill opacity=0.20] (179.40, 62.60) circle (  2.13);

\path[fill=fillColor,fill opacity=0.20] (172.41, 39.71) circle (  2.13);

\path[fill=fillColor,fill opacity=0.20] (181.80, 43.50) circle (  2.13);

\path[fill=fillColor,fill opacity=0.20] (184.42, 49.57) circle (  2.13);

\path[fill=fillColor,fill opacity=0.20] (191.85, 60.19) circle (  2.13);

\path[fill=fillColor,fill opacity=0.20] (200.37, 67.40) circle (  2.13);

\path[fill=fillColor,fill opacity=0.20] (202.12, 73.22) circle (  2.13);

\path[fill=fillColor,fill opacity=0.20] (199.94, 81.31) circle (  2.13);

\path[fill=fillColor,fill opacity=0.20] (201.25, 86.37) circle (  2.13);

\path[fill=fillColor,fill opacity=0.20] (206.27, 87.00) circle (  2.13);

\path[fill=fillColor,fill opacity=0.20] (201.47, 81.56) circle (  2.13);

\path[fill=fillColor,fill opacity=0.20] (197.75, 76.51) circle (  2.13);

\path[fill=fillColor,fill opacity=0.20] (206.93, 79.16) circle (  2.13);

\path[fill=fillColor,fill opacity=0.20] (194.69, 78.91) circle (  2.13);

\path[fill=fillColor,fill opacity=0.20] (192.94, 72.97) circle (  2.13);

\path[fill=fillColor,fill opacity=0.20] (196.66, 74.48) circle (  2.13);

\path[fill=fillColor,fill opacity=0.20] (200.37, 79.16) circle (  2.13);

\path[fill=fillColor,fill opacity=0.20] (197.10, 72.46) circle (  2.13);

\path[fill=fillColor,fill opacity=0.20] (190.10, 56.40) circle (  2.13);

\path[fill=fillColor,fill opacity=0.20] (190.76, 62.60) circle (  2.13);

\path[fill=fillColor,fill opacity=0.20] (189.01, 67.65) circle (  2.13);

\path[fill=fillColor,fill opacity=0.20] (198.84, 63.73) circle (  2.13);

\path[fill=fillColor,fill opacity=0.20] (194.26, 63.48) circle (  2.13);

\path[fill=fillColor,fill opacity=0.20] (200.81, 71.70) circle (  2.13);

\path[fill=fillColor,fill opacity=0.20] (196.22, 73.09) circle (  2.13);

\path[fill=fillColor,fill opacity=0.20] (204.09, 64.87) circle (  2.13);

\path[fill=fillColor,fill opacity=0.20] (204.52, 62.22) circle (  2.13);

\path[fill=fillColor,fill opacity=0.20] (204.52, 61.08) circle (  2.13);

\path[fill=fillColor,fill opacity=0.20] (198.19, 55.14) circle (  2.13);

\path[fill=fillColor,fill opacity=0.20] (188.57, 52.61) circle (  2.13);

\path[fill=fillColor,fill opacity=0.20] (176.34, 47.80) circle (  2.13);

\path[fill=fillColor,fill opacity=0.20] (166.29, 38.70) circle (  2.13);

\path[fill=fillColor,fill opacity=0.20] (200.59, 60.45) circle (  2.13);

\path[fill=fillColor,fill opacity=0.20] (208.68, 66.01) circle (  2.13);

\path[fill=fillColor,fill opacity=0.20] (208.68, 69.55) circle (  2.13);

\path[fill=fillColor,fill opacity=0.20] (210.64, 78.66) circle (  2.13);

\path[fill=fillColor,fill opacity=0.20] (225.50, 88.65) circle (  2.13);

\path[fill=fillColor,fill opacity=0.20] (210.64, 89.66) circle (  2.13);

\path[fill=fillColor,fill opacity=0.20] (215.01, 90.54) circle (  2.13);

\path[fill=fillColor,fill opacity=0.20] (213.70, 91.93) circle (  2.13);

\path[fill=fillColor,fill opacity=0.20] (212.17, 91.68) circle (  2.13);

\path[fill=fillColor,fill opacity=0.20] (212.61, 93.83) circle (  2.13);

\path[fill=fillColor,fill opacity=0.20] (213.05, 95.98) circle (  2.13);

\path[fill=fillColor,fill opacity=0.20] (211.95, 93.83) circle (  2.13);

\path[fill=fillColor,fill opacity=0.20] (211.95, 90.79) circle (  2.13);

\path[fill=fillColor,fill opacity=0.20] (216.32, 89.28) circle (  2.13);

\path[fill=fillColor,fill opacity=0.20] (209.11, 90.79) circle (  2.13);

\path[fill=fillColor,fill opacity=0.20] (216.54, 91.81) circle (  2.13);

\path[fill=fillColor,fill opacity=0.20] (211.52, 89.40) circle (  2.13);

\path[fill=fillColor,fill opacity=0.20] (198.19, 79.92) circle (  2.13);

\path[fill=fillColor,fill opacity=0.20] (182.24, 60.95) circle (  2.13);

\path[fill=fillColor,fill opacity=0.20] (182.24, 38.82) circle (  2.13);

\path[fill=fillColor,fill opacity=0.20] (184.86, 45.27) circle (  2.13);

\path[fill=fillColor,fill opacity=0.20] (183.11, 54.00) circle (  2.13);

\path[fill=fillColor,fill opacity=0.20] (187.04, 60.57) circle (  2.13);

\path[fill=fillColor,fill opacity=0.20] (194.26, 66.26) circle (  2.13);

\path[fill=fillColor,fill opacity=0.20] (206.27, 79.92) circle (  2.13);

\path[fill=fillColor,fill opacity=0.20] (200.15, 90.79) circle (  2.13);

\path[fill=fillColor,fill opacity=0.20] (201.25, 87.51) circle (  2.13);

\path[fill=fillColor,fill opacity=0.20] (200.81, 84.85) circle (  2.13);

\path[fill=fillColor,fill opacity=0.20] (200.15, 78.78) circle (  2.13);

\path[fill=fillColor,fill opacity=0.20] (199.28, 72.97) circle (  2.13);

\path[fill=fillColor,fill opacity=0.20] (199.06, 77.01) circle (  2.13);

\path[fill=fillColor,fill opacity=0.20] (192.29, 86.62) circle (  2.13);

\path[fill=fillColor,fill opacity=0.20] (201.25, 86.50) circle (  2.13);

\path[fill=fillColor,fill opacity=0.20] (197.53, 80.81) circle (  2.13);

\path[fill=fillColor,fill opacity=0.20] (193.82, 74.23) circle (  2.13);

\path[fill=fillColor,fill opacity=0.20] (191.85, 70.18) circle (  2.13);

\path[fill=fillColor,fill opacity=0.20] (191.41, 71.83) circle (  2.13);

\path[fill=fillColor,fill opacity=0.20] (193.38, 75.49) circle (  2.13);

\path[fill=fillColor,fill opacity=0.20] (198.84, 76.25) circle (  2.13);

\path[fill=fillColor,fill opacity=0.20] (196.66, 72.97) circle (  2.13);

\path[fill=fillColor,fill opacity=0.20] (193.82, 67.53) circle (  2.13);

\path[fill=fillColor,fill opacity=0.20] (191.41, 65.00) circle (  2.13);

\path[fill=fillColor,fill opacity=0.20] (192.94, 69.68) circle (  2.13);

\path[fill=fillColor,fill opacity=0.20] (197.10, 75.87) circle (  2.13);

\path[fill=fillColor,fill opacity=0.20] (200.59, 79.29) circle (  2.13);

\path[fill=fillColor,fill opacity=0.20] (201.25, 82.07) circle (  2.13);

\path[fill=fillColor,fill opacity=0.20] (204.31, 83.59) circle (  2.13);

\path[fill=fillColor,fill opacity=0.20] (205.40, 82.83) circle (  2.13);

\path[fill=fillColor,fill opacity=0.20] (203.00, 85.23) circle (  2.13);

\path[fill=fillColor,fill opacity=0.20] (204.09, 84.47) circle (  2.13);

\path[fill=fillColor,fill opacity=0.20] (206.05, 74.23) circle (  2.13);

\path[fill=fillColor,fill opacity=0.20] (205.62, 69.17) circle (  2.13);

\path[fill=fillColor,fill opacity=0.20] (204.74, 73.09) circle (  2.13);

\path[fill=fillColor,fill opacity=0.20] (201.47, 72.46) circle (  2.13);

\path[fill=fillColor,fill opacity=0.20] (200.59, 62.60) circle (  2.13);

\path[fill=fillColor,fill opacity=0.20] (203.00, 55.01) circle (  2.13);

\path[fill=fillColor,fill opacity=0.20] (184.86, 46.92) circle (  2.13);

\path[fill=fillColor,fill opacity=0.20] (175.25, 38.82) circle (  2.13);

\path[fill=fillColor,fill opacity=0.20] (190.32, 42.62) circle (  2.13);

\path[fill=fillColor,fill opacity=0.20] (196.00, 48.56) circle (  2.13);

\path[fill=fillColor,fill opacity=0.20] (207.58, 55.39) circle (  2.13);

\path[fill=fillColor,fill opacity=0.20] (216.32, 65.63) circle (  2.13);

\path[fill=fillColor,fill opacity=0.20] (212.17, 72.71) circle (  2.13);

\path[fill=fillColor,fill opacity=0.20] (207.37, 74.99) circle (  2.13);

\path[fill=fillColor,fill opacity=0.20] (209.11, 77.77) circle (  2.13);

\path[fill=fillColor,fill opacity=0.20] (216.98, 79.67) circle (  2.13);

\path[fill=fillColor,fill opacity=0.20] (214.14, 80.43) circle (  2.13);

\path[fill=fillColor,fill opacity=0.20] (207.58, 79.92) circle (  2.13);

\path[fill=fillColor,fill opacity=0.20] (209.77, 74.10) circle (  2.13);

\path[fill=fillColor,fill opacity=0.20] (209.55, 71.20) circle (  2.13);

\path[fill=fillColor,fill opacity=0.20] (203.87, 72.46) circle (  2.13);

\path[fill=fillColor,fill opacity=0.20] (198.84, 63.73) circle (  2.13);

\path[fill=fillColor,fill opacity=0.20] (181.36, 48.05) circle (  2.13);

\path[fill=fillColor,fill opacity=0.20] (183.33, 65.13) circle (  2.13);

\path[fill=fillColor,fill opacity=0.20] (192.51, 72.97) circle (  2.13);

\path[fill=fillColor,fill opacity=0.20] (202.34, 72.97) circle (  2.13);

\path[fill=fillColor,fill opacity=0.20] (206.71, 72.84) circle (  2.13);

\path[fill=fillColor,fill opacity=0.20] (204.96, 74.74) circle (  2.13);

\path[fill=fillColor,fill opacity=0.20] (196.44, 78.78) circle (  2.13);

\path[fill=fillColor,fill opacity=0.20] (201.25, 82.58) circle (  2.13);

\path[fill=fillColor,fill opacity=0.20] (198.84, 85.48) circle (  2.13);

\path[fill=fillColor,fill opacity=0.20] (200.59, 85.48) circle (  2.13);

\path[fill=fillColor,fill opacity=0.20] (196.66, 80.30) circle (  2.13);

\path[fill=fillColor,fill opacity=0.20] (196.44, 77.52) circle (  2.13);

\path[fill=fillColor,fill opacity=0.20] (198.84, 80.68) circle (  2.13);

\path[fill=fillColor,fill opacity=0.20] (197.97, 83.59) circle (  2.13);

\path[fill=fillColor,fill opacity=0.20] (201.90, 86.37) circle (  2.13);

\path[fill=fillColor,fill opacity=0.20] (199.94, 89.78) circle (  2.13);

\path[fill=fillColor,fill opacity=0.20] (198.41, 88.27) circle (  2.13);

\path[fill=fillColor,fill opacity=0.20] (203.43, 87.00) circle (  2.13);

\path[fill=fillColor,fill opacity=0.20] (200.37, 87.25) circle (  2.13);

\path[fill=fillColor,fill opacity=0.20] (199.94, 83.21) circle (  2.13);

\path[fill=fillColor,fill opacity=0.20] (201.03, 79.54) circle (  2.13);

\path[fill=fillColor,fill opacity=0.20] (217.20, 79.92) circle (  2.13);

\path[fill=fillColor,fill opacity=0.20] (209.33, 84.73) circle (  2.13);

\path[fill=fillColor,fill opacity=0.20] (206.05, 89.91) circle (  2.13);

\path[fill=fillColor,fill opacity=0.20] (207.58, 87.63) circle (  2.13);

\path[fill=fillColor,fill opacity=0.20] (211.08, 82.45) circle (  2.13);

\path[fill=fillColor,fill opacity=0.20] (213.05, 84.85) circle (  2.13);

\path[fill=fillColor,fill opacity=0.20] (209.11, 87.25) circle (  2.13);

\path[fill=fillColor,fill opacity=0.20] (202.56, 76.38) circle (  2.13);

\path[fill=fillColor,fill opacity=0.20] (196.88, 62.98) circle (  2.13);

\path[fill=fillColor,fill opacity=0.20] (187.92, 54.12) circle (  2.13);

\path[fill=fillColor,fill opacity=0.20] (178.52, 46.79) circle (  2.13);

\path[fill=fillColor,fill opacity=0.20] (199.28, 43.76) circle (  2.13);

\path[fill=fillColor,fill opacity=0.20] (195.57, 47.42) circle (  2.13);

\path[fill=fillColor,fill opacity=0.20] (198.41, 53.11) circle (  2.13);

\path[fill=fillColor,fill opacity=0.20] (205.40, 56.40) circle (  2.13);

\path[fill=fillColor,fill opacity=0.20] (206.05, 57.41) circle (  2.13);

\path[fill=fillColor,fill opacity=0.20] (200.37, 56.65) circle (  2.13);

\path[fill=fillColor,fill opacity=0.20] (199.94, 51.97) circle (  2.13);

\path[fill=fillColor,fill opacity=0.20] (185.95, 46.79) circle (  2.13);

\path[fill=fillColor,fill opacity=0.20] (193.82, 59.56) circle (  2.13);

\path[fill=fillColor,fill opacity=0.20] (199.28, 67.02) circle (  2.13);

\path[fill=fillColor,fill opacity=0.20] (209.11, 72.97) circle (  2.13);

\path[fill=fillColor,fill opacity=0.20] (206.05, 74.36) circle (  2.13);

\path[fill=fillColor,fill opacity=0.20] (202.34, 73.72) circle (  2.13);

\path[fill=fillColor,fill opacity=0.20] (199.28, 77.64) circle (  2.13);

\path[fill=fillColor,fill opacity=0.20] (201.68, 81.06) circle (  2.13);

\path[fill=fillColor,fill opacity=0.20] (203.43, 81.44) circle (  2.13);

\path[fill=fillColor,fill opacity=0.20] (202.34, 83.46) circle (  2.13);

\path[fill=fillColor,fill opacity=0.20] (205.62, 83.46) circle (  2.13);

\path[fill=fillColor,fill opacity=0.20] (202.12, 81.56) circle (  2.13);

\path[fill=fillColor,fill opacity=0.20] (201.03, 83.71) circle (  2.13);

\path[fill=fillColor,fill opacity=0.20] (200.59, 86.50) circle (  2.13);

\path[fill=fillColor,fill opacity=0.20] (204.74, 87.63) circle (  2.13);

\path[fill=fillColor,fill opacity=0.20] (205.40, 88.77) circle (  2.13);

\path[fill=fillColor,fill opacity=0.20] (203.65, 88.01) circle (  2.13);

\path[fill=fillColor,fill opacity=0.20] (210.42, 84.09) circle (  2.13);

\path[fill=fillColor,fill opacity=0.20] (212.17, 79.29) circle (  2.13);

\path[fill=fillColor,fill opacity=0.20] (211.95, 80.17) circle (  2.13);

\path[fill=fillColor,fill opacity=0.20] (213.48, 87.25) circle (  2.13);

\path[fill=fillColor,fill opacity=0.20] (212.17, 88.52) circle (  2.13);

\path[fill=fillColor,fill opacity=0.20] (211.08, 80.05) circle (  2.13);

\path[fill=fillColor,fill opacity=0.20] (204.52, 74.74) circle (  2.13);

\path[fill=fillColor,fill opacity=0.20] (191.41, 71.32) circle (  2.13);

\path[fill=fillColor,fill opacity=0.20] (181.36, 59.81) circle (  2.13);

\path[fill=fillColor,fill opacity=0.20] (175.68, 46.16) circle (  2.13);

\path[fill=fillColor,fill opacity=0.20] (180.93, 38.82) circle (  2.13);

\path[fill=fillColor,fill opacity=0.20] (203.00, 47.68) circle (  2.13);

\path[fill=fillColor,fill opacity=0.20] (194.69, 57.54) circle (  2.13);

\path[fill=fillColor,fill opacity=0.20] (194.47, 60.07) circle (  2.13);

\path[fill=fillColor,fill opacity=0.20] (196.66, 65.00) circle (  2.13);

\path[fill=fillColor,fill opacity=0.20] (200.59, 75.87) circle (  2.13);

\path[fill=fillColor,fill opacity=0.20] (203.43, 80.17) circle (  2.13);

\path[fill=fillColor,fill opacity=0.20] (205.84, 77.14) circle (  2.13);

\path[fill=fillColor,fill opacity=0.20] (206.71, 77.64) circle (  2.13);

\path[fill=fillColor,fill opacity=0.20] (209.11, 80.17) circle (  2.13);

\path[fill=fillColor,fill opacity=0.20] (208.68, 81.06) circle (  2.13);

\path[fill=fillColor,fill opacity=0.20] (207.80, 82.20) circle (  2.13);

\path[fill=fillColor,fill opacity=0.20] (212.17, 86.12) circle (  2.13);

\path[fill=fillColor,fill opacity=0.20] (213.48, 88.90) circle (  2.13);

\path[fill=fillColor,fill opacity=0.20] (211.30, 88.90) circle (  2.13);

\path[fill=fillColor,fill opacity=0.20] (218.29, 89.78) circle (  2.13);

\path[fill=fillColor,fill opacity=0.20] (219.16, 87.13) circle (  2.13);

\path[fill=fillColor,fill opacity=0.20] (208.89, 81.94) circle (  2.13);

\path[fill=fillColor,fill opacity=0.20] (206.27, 81.82) circle (  2.13);

\path[fill=fillColor,fill opacity=0.20] (202.34, 82.07) circle (  2.13);

\path[fill=fillColor,fill opacity=0.20] (197.75, 72.71) circle (  2.13);

\path[fill=fillColor,fill opacity=0.20] (188.57, 59.69) circle (  2.13);

\path[fill=fillColor,fill opacity=0.20] (184.86, 47.42) circle (  2.13);

\path[fill=fillColor,fill opacity=0.20] (188.79, 52.86) circle (  2.13);

\path[fill=fillColor,fill opacity=0.20] (197.10, 58.55) circle (  2.13);

\path[fill=fillColor,fill opacity=0.20] (219.60, 59.44) circle (  2.13);

\path[fill=fillColor,fill opacity=0.20] (199.06, 64.87) circle (  2.13);

\path[fill=fillColor,fill opacity=0.20] (206.93, 79.16) circle (  2.13);

\path[fill=fillColor,fill opacity=0.20] (212.17, 85.48) circle (  2.13);

\path[fill=fillColor,fill opacity=0.20] (211.95, 80.55) circle (  2.13);

\path[fill=fillColor,fill opacity=0.20] (215.23, 82.45) circle (  2.13);

\path[fill=fillColor,fill opacity=0.20] (213.05, 84.98) circle (  2.13);

\path[fill=fillColor,fill opacity=0.20] (208.89, 79.79) circle (  2.13);

\path[fill=fillColor,fill opacity=0.20] (214.58, 77.14) circle (  2.13);

\path[fill=fillColor,fill opacity=0.20] (199.94, 79.41) circle (  2.13);

\path[fill=fillColor,fill opacity=0.20] (194.47, 76.38) circle (  2.13);

\path[fill=fillColor,fill opacity=0.20] (195.13, 68.29) circle (  2.13);

\path[fill=fillColor,fill opacity=0.20] (183.99, 60.19) circle (  2.13);

\path[fill=fillColor,fill opacity=0.20] (179.62, 51.85) circle (  2.13);

\path[fill=fillColor,fill opacity=0.20] (196.66, 56.65) circle (  2.13);

\path[fill=fillColor,fill opacity=0.20] (200.59, 66.64) circle (  2.13);

\path[fill=fillColor,fill opacity=0.20] (195.57, 66.77) circle (  2.13);

\path[fill=fillColor,fill opacity=0.20] (196.22, 63.23) circle (  2.13);

\path[fill=fillColor,fill opacity=0.20] (196.44, 61.08) circle (  2.13);

\path[fill=fillColor,fill opacity=0.20] (189.89, 57.16) circle (  2.13);

\path[fill=fillColor,fill opacity=0.20] (185.30, 52.23) circle (  2.13);

\path[fill=fillColor,fill opacity=0.20] (182.46, 51.34) circle (  2.13);

\path[fill=fillColor,fill opacity=0.20] (177.65, 52.23) circle (  2.13);

\path[fill=fillColor,fill opacity=0.20] (197.31, 69.17) circle (  2.13);

\path[fill=fillColor,fill opacity=0.20] (200.37, 65.88) circle (  2.13);

\path[fill=fillColor,fill opacity=0.20] (185.08, 63.36) circle (  2.13);

\path[fill=fillColor,fill opacity=0.20] (204.74, 93.70) circle (  2.13);

\path[fill=fillColor,fill opacity=0.20] (209.99, 98.89) circle (  2.13);

\path[fill=fillColor,fill opacity=0.20] (210.21, 93.83) circle (  2.13);

\path[fill=fillColor,fill opacity=0.20] (209.55, 86.37) circle (  2.13);

\path[fill=fillColor,fill opacity=0.20] (205.62, 85.74) circle (  2.13);

\path[fill=fillColor,fill opacity=0.20] (180.05, 46.92) circle (  2.13);

\path[fill=fillColor,fill opacity=0.20] (177.21, 61.71) circle (  2.13);

\path[fill=fillColor,fill opacity=0.20] (182.24, 70.44) circle (  2.13);

\path[fill=fillColor,fill opacity=0.20] (183.99, 62.98) circle (  2.13);

\path[fill=fillColor,fill opacity=0.20] (183.11, 65.38) circle (  2.13);

\path[fill=fillColor,fill opacity=0.20] (180.49, 69.05) circle (  2.13);

\path[fill=fillColor,fill opacity=0.20] (201.90,100.28) circle (  2.13);

\path[fill=fillColor,fill opacity=0.20] (215.67,105.34) circle (  2.13);

\path[fill=fillColor,fill opacity=0.20] (217.20, 97.88) circle (  2.13);

\path[fill=fillColor,fill opacity=0.20] (223.32, 97.24) circle (  2.13);

\path[fill=fillColor,fill opacity=0.20] (225.28, 96.11) circle (  2.13);

\path[fill=fillColor,fill opacity=0.20] (220.04, 91.05) circle (  2.13);

\path[fill=fillColor,fill opacity=0.20] (212.61, 84.35) circle (  2.13);

\path[fill=fillColor,fill opacity=0.20] (202.34, 76.13) circle (  2.13);

\path[fill=fillColor,fill opacity=0.20] (180.49, 40.21) circle (  2.13);

\path[fill=fillColor,fill opacity=0.20] (181.15, 52.73) circle (  2.13);

\path[fill=fillColor,fill opacity=0.20] (182.46, 59.18) circle (  2.13);

\path[fill=fillColor,fill opacity=0.20] (190.54, 72.97) circle (  2.13);

\path[fill=fillColor,fill opacity=0.20] (199.06, 84.73) circle (  2.13);

\path[fill=fillColor,fill opacity=0.20] (201.90, 80.05) circle (  2.13);

\path[fill=fillColor,fill opacity=0.20] (198.63, 80.30) circle (  2.13);

\path[fill=fillColor,fill opacity=0.20] (194.47, 81.44) circle (  2.13);

\path[fill=fillColor,fill opacity=0.20] (190.10, 77.01) circle (  2.13);

\path[fill=fillColor,fill opacity=0.20] (185.52, 66.77) circle (  2.13);

\path[fill=fillColor,fill opacity=0.20] (185.73, 80.81) circle (  2.13);

\path[fill=fillColor,fill opacity=0.20] (219.38,106.47) circle (  2.13);

\path[fill=fillColor,fill opacity=0.20] (220.91, 93.20) circle (  2.13);

\path[fill=fillColor,fill opacity=0.20] (224.19, 98.63) circle (  2.13);

\path[fill=fillColor,fill opacity=0.20] (232.93,103.57) circle (  2.13);

\path[fill=fillColor,fill opacity=0.20] (234.02,106.47) circle (  2.13);

\path[fill=fillColor,fill opacity=0.20] (229.00,102.43) circle (  2.13);

\path[fill=fillColor,fill opacity=0.20] (217.42, 96.36) circle (  2.13);

\path[fill=fillColor,fill opacity=0.20] (206.93, 93.58) circle (  2.13);

\path[fill=fillColor,fill opacity=0.20] (197.75, 80.43) circle (  2.13);

\path[fill=fillColor,fill opacity=0.20] (206.71, 46.28) circle (  2.13);

\path[fill=fillColor,fill opacity=0.20] (190.10, 61.46) circle (  2.13);

\path[fill=fillColor,fill opacity=0.20] (197.53, 73.47) circle (  2.13);

\path[fill=fillColor,fill opacity=0.20] (196.66, 90.04) circle (  2.13);

\path[fill=fillColor,fill opacity=0.20] (201.68,100.03) circle (  2.13);

\path[fill=fillColor,fill opacity=0.20] (204.96, 92.19) circle (  2.13);

\path[fill=fillColor,fill opacity=0.20] (206.71, 83.84) circle (  2.13);

\path[fill=fillColor,fill opacity=0.20] (205.40, 88.77) circle (  2.13);

\path[fill=fillColor,fill opacity=0.20] (197.97, 89.91) circle (  2.13);

\path[fill=fillColor,fill opacity=0.20] (195.57, 77.26) circle (  2.13);

\path[fill=fillColor,fill opacity=0.20] (198.84, 99.90) circle (  2.13);

\path[fill=fillColor,fill opacity=0.20] (227.25, 96.49) circle (  2.13);

\path[fill=fillColor,fill opacity=0.20] (225.72, 87.63) circle (  2.13);

\path[fill=fillColor,fill opacity=0.20] (233.59, 98.26) circle (  2.13);

\path[fill=fillColor,fill opacity=0.20] (234.68,109.89) circle (  2.13);

\path[fill=fillColor,fill opacity=0.20] (232.27,115.20) circle (  2.13);

\path[fill=fillColor,fill opacity=0.20] (222.44,115.33) circle (  2.13);

\path[fill=fillColor,fill opacity=0.20] (217.20,114.95) circle (  2.13);

\path[fill=fillColor,fill opacity=0.20] (206.27, 96.99) circle (  2.13);

\path[fill=fillColor,fill opacity=0.20] (179.18, 44.64) circle (  2.13);

\path[fill=fillColor,fill opacity=0.20] (188.14, 78.91) circle (  2.13);

\path[fill=fillColor,fill opacity=0.20] (192.29, 78.28) circle (  2.13);

\path[fill=fillColor,fill opacity=0.20] (196.44, 79.41) circle (  2.13);

\path[fill=fillColor,fill opacity=0.20] (199.28, 94.46) circle (  2.13);

\path[fill=fillColor,fill opacity=0.20] (204.74,100.53) circle (  2.13);

\path[fill=fillColor,fill opacity=0.20] (208.02, 97.24) circle (  2.13);

\path[fill=fillColor,fill opacity=0.20] (206.27, 91.81) circle (  2.13);

\path[fill=fillColor,fill opacity=0.20] (205.18, 89.28) circle (  2.13);

\path[fill=fillColor,fill opacity=0.20] (194.69, 92.94) circle (  2.13);

\path[fill=fillColor,fill opacity=0.20] (185.73, 87.51) circle (  2.13);

\path[fill=fillColor,fill opacity=0.20] (177.21, 74.10) circle (  2.13);

\path[fill=fillColor,fill opacity=0.20] (197.97, 87.00) circle (  2.13);

\path[fill=fillColor,fill opacity=0.20] (225.06,101.04) circle (  2.13);

\path[fill=fillColor,fill opacity=0.20] (225.50, 92.69) circle (  2.13);

\path[fill=fillColor,fill opacity=0.20] (234.24, 95.09) circle (  2.13);

\path[fill=fillColor,fill opacity=0.20] (235.33,110.77) circle (  2.13);

\path[fill=fillColor,fill opacity=0.20] (233.80,114.44) circle (  2.13);

\path[fill=fillColor,fill opacity=0.20] (216.11,109.89) circle (  2.13);

\path[fill=fillColor,fill opacity=0.20] (205.62, 84.22) circle (  2.13);

\path[fill=fillColor,fill opacity=0.20] (175.25, 42.36) circle (  2.13);

\path[fill=fillColor,fill opacity=0.20] (192.94, 67.91) circle (  2.13);

\path[fill=fillColor,fill opacity=0.20] (201.68, 91.81) circle (  2.13);

\path[fill=fillColor,fill opacity=0.20] (199.94, 92.06) circle (  2.13);

\path[fill=fillColor,fill opacity=0.20] (201.25, 92.06) circle (  2.13);

\path[fill=fillColor,fill opacity=0.20] (204.31, 97.62) circle (  2.13);

\path[fill=fillColor,fill opacity=0.20] (207.80, 98.38) circle (  2.13);

\path[fill=fillColor,fill opacity=0.20] (209.33,101.04) circle (  2.13);

\path[fill=fillColor,fill opacity=0.20] (206.93, 98.26) circle (  2.13);

\path[fill=fillColor,fill opacity=0.20] (204.31, 84.98) circle (  2.13);

\path[fill=fillColor,fill opacity=0.20] (196.22, 84.35) circle (  2.13);

\path[fill=fillColor,fill opacity=0.20] (173.28, 77.39) circle (  2.13);

\path[fill=fillColor,fill opacity=0.20] (197.10, 92.44) circle (  2.13);

\path[fill=fillColor,fill opacity=0.20] (220.91,109.13) circle (  2.13);

\path[fill=fillColor,fill opacity=0.20] (223.53,101.29) circle (  2.13);

\path[fill=fillColor,fill opacity=0.20] (234.46,101.67) circle (  2.13);

\path[fill=fillColor,fill opacity=0.20] (238.17,112.04) circle (  2.13);

\path[fill=fillColor,fill opacity=0.20] (230.31,105.72) circle (  2.13);

\path[fill=fillColor,fill opacity=0.20] (229.22, 99.14) circle (  2.13);

\path[fill=fillColor,fill opacity=0.20] (226.37,105.84) circle (  2.13);

\path[fill=fillColor,fill opacity=0.20] (216.54,108.75) circle (  2.13);

\path[fill=fillColor,fill opacity=0.20] (208.02, 92.57) circle (  2.13);

\path[fill=fillColor,fill opacity=0.20] (194.26, 59.06) circle (  2.13);

\path[fill=fillColor,fill opacity=0.20] (183.11, 57.66) circle (  2.13);

\path[fill=fillColor,fill opacity=0.20] (198.63, 89.78) circle (  2.13);

\path[fill=fillColor,fill opacity=0.20] (202.56, 97.50) circle (  2.13);

\path[fill=fillColor,fill opacity=0.20] (208.46,112.92) circle (  2.13);

\path[fill=fillColor,fill opacity=0.20] (210.42,112.92) circle (  2.13);

\path[fill=fillColor,fill opacity=0.20] (213.05,102.18) circle (  2.13);

\path[fill=fillColor,fill opacity=0.20] (215.45, 98.76) circle (  2.13);

\path[fill=fillColor,fill opacity=0.20] (212.17,101.42) circle (  2.13);

\path[fill=fillColor,fill opacity=0.20] (208.89, 96.49) circle (  2.13);

\path[fill=fillColor,fill opacity=0.20] (203.43, 85.23) circle (  2.13);

\path[fill=fillColor,fill opacity=0.20] (189.67, 73.60) circle (  2.13);

\path[fill=fillColor,fill opacity=0.20] (202.34,110.90) circle (  2.13);

\path[fill=fillColor,fill opacity=0.20] (219.38,106.60) circle (  2.13);

\path[fill=fillColor,fill opacity=0.20] (220.91,104.20) circle (  2.13);

\path[fill=fillColor,fill opacity=0.20] (234.68,106.85) circle (  2.13);

\path[fill=fillColor,fill opacity=0.20] (237.30,109.89) circle (  2.13);

\path[fill=fillColor,fill opacity=0.20] (231.18,114.31) circle (  2.13);

\path[fill=fillColor,fill opacity=0.20] (227.90,103.44) circle (  2.13);

\path[fill=fillColor,fill opacity=0.20] (225.28, 93.07) circle (  2.13);

\path[fill=fillColor,fill opacity=0.20] (222.44,102.05) circle (  2.13);

\path[fill=fillColor,fill opacity=0.20] (216.32,109.38) circle (  2.13);

\path[fill=fillColor,fill opacity=0.20] (204.52, 94.84) circle (  2.13);

\path[fill=fillColor,fill opacity=0.20] (191.63, 61.21) circle (  2.13);

\path[fill=fillColor,fill opacity=0.20] (194.04, 57.03) circle (  2.13);

\path[fill=fillColor,fill opacity=0.20] (204.31, 93.58) circle (  2.13);

\path[fill=fillColor,fill opacity=0.20] (208.46,108.24) circle (  2.13);

\path[fill=fillColor,fill opacity=0.20] (220.04, 99.77) circle (  2.13);

\path[fill=fillColor,fill opacity=0.20] (217.42, 99.77) circle (  2.13);

\path[fill=fillColor,fill opacity=0.20] (214.14,105.97) circle (  2.13);

\path[fill=fillColor,fill opacity=0.20] (209.55, 93.83) circle (  2.13);

\path[fill=fillColor,fill opacity=0.20] (196.22, 80.81) circle (  2.13);

\path[fill=fillColor,fill opacity=0.20] (174.81, 60.70) circle (  2.13);

\path[fill=fillColor,fill opacity=0.20] (187.92, 96.61) circle (  2.13);

\path[fill=fillColor,fill opacity=0.20] (213.05,102.30) circle (  2.13);

\path[fill=fillColor,fill opacity=0.20] (217.85, 96.74) circle (  2.13);

\path[fill=fillColor,fill opacity=0.20] (216.98, 99.27) circle (  2.13);

\path[fill=fillColor,fill opacity=0.20] (222.88, 99.39) circle (  2.13);

\path[fill=fillColor,fill opacity=0.20] (227.25,107.61) circle (  2.13);

\path[fill=fillColor,fill opacity=0.20] (231.18,113.56) circle (  2.13);

\path[fill=fillColor,fill opacity=0.20] (232.27,106.60) circle (  2.13);

\path[fill=fillColor,fill opacity=0.20] (224.63,101.92) circle (  2.13);

\path[fill=fillColor,fill opacity=0.20] (220.04,107.74) circle (  2.13);

\path[fill=fillColor,fill opacity=0.20] (216.76,104.70) circle (  2.13);

\path[fill=fillColor,fill opacity=0.20] (204.74, 89.66) circle (  2.13);

\path[fill=fillColor,fill opacity=0.20] (177.43, 41.48) circle (  2.13);

\path[fill=fillColor,fill opacity=0.20] (196.66, 56.78) circle (  2.13);

\path[fill=fillColor,fill opacity=0.20] (206.49, 85.74) circle (  2.13);

\path[fill=fillColor,fill opacity=0.20] (209.77,104.45) circle (  2.13);

\path[fill=fillColor,fill opacity=0.20] (215.23,109.76) circle (  2.13);

\path[fill=fillColor,fill opacity=0.20] (215.45,102.81) circle (  2.13);

\path[fill=fillColor,fill opacity=0.20] (215.23, 92.57) circle (  2.13);

\path[fill=fillColor,fill opacity=0.20] (215.01, 91.81) circle (  2.13);

\path[fill=fillColor,fill opacity=0.20] (211.95,101.04) circle (  2.13);

\path[fill=fillColor,fill opacity=0.20] (202.78, 92.44) circle (  2.13);

\path[fill=fillColor,fill opacity=0.20] (184.20, 72.33) circle (  2.13);

\path[fill=fillColor,fill opacity=0.20] (178.96, 70.69) circle (  2.13);

\path[fill=fillColor,fill opacity=0.20] (203.00, 87.76) circle (  2.13);

\path[fill=fillColor,fill opacity=0.20] (213.70, 90.54) circle (  2.13);

\path[fill=fillColor,fill opacity=0.20] (216.32, 96.36) circle (  2.13);

\path[fill=fillColor,fill opacity=0.20] (213.92, 96.36) circle (  2.13);

\path[fill=fillColor,fill opacity=0.20] (218.95, 98.38) circle (  2.13);

\path[fill=fillColor,fill opacity=0.20] (223.53,106.10) circle (  2.13);

\path[fill=fillColor,fill opacity=0.20] (228.78,111.91) circle (  2.13);

\path[fill=fillColor,fill opacity=0.20] (227.47,109.76) circle (  2.13);

\path[fill=fillColor,fill opacity=0.20] (219.60,106.98) circle (  2.13);

\path[fill=fillColor,fill opacity=0.20] (219.38, 98.00) circle (  2.13);

\path[fill=fillColor,fill opacity=0.20] (211.74, 90.04) circle (  2.13);

\path[fill=fillColor,fill opacity=0.20] (198.84, 77.14) circle (  2.13);

\path[fill=fillColor,fill opacity=0.20] (179.62, 45.02) circle (  2.13);

\path[fill=fillColor,fill opacity=0.20] (194.26, 61.08) circle (  2.13);

\path[fill=fillColor,fill opacity=0.20] (204.52, 79.41) circle (  2.13);

\path[fill=fillColor,fill opacity=0.20] (212.61, 97.50) circle (  2.13);

\path[fill=fillColor,fill opacity=0.20] (217.85,105.34) circle (  2.13);

\path[fill=fillColor,fill opacity=0.20] (218.73, 99.65) circle (  2.13);

\path[fill=fillColor,fill opacity=0.20] (215.23, 93.58) circle (  2.13);

\path[fill=fillColor,fill opacity=0.20] (211.95, 89.78) circle (  2.13);

\path[fill=fillColor,fill opacity=0.20] (207.15, 88.65) circle (  2.13);

\path[fill=fillColor,fill opacity=0.20] (194.91, 87.25) circle (  2.13);

\path[fill=fillColor,fill opacity=0.20] (191.41, 73.34) circle (  2.13);

\path[fill=fillColor,fill opacity=0.20] (202.12, 79.92) circle (  2.13);

\path[fill=fillColor,fill opacity=0.20] (209.11, 98.76) circle (  2.13);

\path[fill=fillColor,fill opacity=0.20] (213.92,105.84) circle (  2.13);

\path[fill=fillColor,fill opacity=0.20] (218.07,100.66) circle (  2.13);

\path[fill=fillColor,fill opacity=0.20] (226.16,104.45) circle (  2.13);

\path[fill=fillColor,fill opacity=0.20] (220.26,108.37) circle (  2.13);

\path[fill=fillColor,fill opacity=0.20] (224.41,106.47) circle (  2.13);

\path[fill=fillColor,fill opacity=0.20] (223.10,105.72) circle (  2.13);

\path[fill=fillColor,fill opacity=0.20] (217.85, 97.75) circle (  2.13);

\path[fill=fillColor,fill opacity=0.20] (213.26, 83.71) circle (  2.13);

\path[fill=fillColor,fill opacity=0.20] (207.37, 78.40) circle (  2.13);

\path[fill=fillColor,fill opacity=0.20] (193.60, 59.81) circle (  2.13);

\path[fill=fillColor,fill opacity=0.20] (194.26, 59.69) circle (  2.13);

\path[fill=fillColor,fill opacity=0.20] (210.86, 79.79) circle (  2.13);

\path[fill=fillColor,fill opacity=0.20] (215.67, 90.92) circle (  2.13);

\path[fill=fillColor,fill opacity=0.20] (219.16, 97.50) circle (  2.13);

\path[fill=fillColor,fill opacity=0.20] (217.85, 96.36) circle (  2.13);

\path[fill=fillColor,fill opacity=0.20] (212.39, 96.61) circle (  2.13);

\path[fill=fillColor,fill opacity=0.20] (208.46, 96.23) circle (  2.13);

\path[fill=fillColor,fill opacity=0.20] (204.52, 85.10) circle (  2.13);

\path[fill=fillColor,fill opacity=0.20] (191.63, 78.28) circle (  2.13);

\path[fill=fillColor,fill opacity=0.20] (194.47, 70.94) circle (  2.13);

\path[fill=fillColor,fill opacity=0.20] (196.00, 78.02) circle (  2.13);

\path[fill=fillColor,fill opacity=0.20] (203.65, 87.00) circle (  2.13);

\path[fill=fillColor,fill opacity=0.20] (208.46,106.35) circle (  2.13);

\path[fill=fillColor,fill opacity=0.20] (215.67,115.45) circle (  2.13);

\path[fill=fillColor,fill opacity=0.20] (223.10,105.46) circle (  2.13);

\path[fill=fillColor,fill opacity=0.20] (226.37,108.37) circle (  2.13);

\path[fill=fillColor,fill opacity=0.20] (223.10,110.65) circle (  2.13);

\path[fill=fillColor,fill opacity=0.20] (220.48,100.28) circle (  2.13);

\path[fill=fillColor,fill opacity=0.20] (218.73, 95.22) circle (  2.13);

\path[fill=fillColor,fill opacity=0.20] (216.54, 92.57) circle (  2.13);

\path[fill=fillColor,fill opacity=0.20] (209.11, 85.48) circle (  2.13);

\path[fill=fillColor,fill opacity=0.20] (198.84, 72.46) circle (  2.13);

\path[fill=fillColor,fill opacity=0.20] (184.86, 43.38) circle (  2.13);

\path[fill=fillColor,fill opacity=0.20] (191.20, 51.22) circle (  2.13);

\path[fill=fillColor,fill opacity=0.20] (208.46, 76.76) circle (  2.13);

\path[fill=fillColor,fill opacity=0.20] (214.58, 83.59) circle (  2.13);

\path[fill=fillColor,fill opacity=0.20] (214.58, 90.79) circle (  2.13);

\path[fill=fillColor,fill opacity=0.20] (209.55, 97.50) circle (  2.13);

\path[fill=fillColor,fill opacity=0.20] (204.31, 97.88) circle (  2.13);

\path[fill=fillColor,fill opacity=0.20] (205.62, 95.22) circle (  2.13);

\path[fill=fillColor,fill opacity=0.20] (203.87, 87.63) circle (  2.13);

\path[fill=fillColor,fill opacity=0.20] (194.26, 75.62) circle (  2.13);

\path[fill=fillColor,fill opacity=0.20] (187.92, 78.28) circle (  2.13);

\path[fill=fillColor,fill opacity=0.20] (199.06, 86.37) circle (  2.13);

\path[fill=fillColor,fill opacity=0.20] (209.11, 88.14) circle (  2.13);

\path[fill=fillColor,fill opacity=0.20] (211.74, 90.04) circle (  2.13);

\path[fill=fillColor,fill opacity=0.20] (213.26, 98.76) circle (  2.13);

\path[fill=fillColor,fill opacity=0.20] (223.53,104.45) circle (  2.13);

\path[fill=fillColor,fill opacity=0.20] (228.12,105.21) circle (  2.13);

\path[fill=fillColor,fill opacity=0.20] (225.28,103.31) circle (  2.13);

\path[fill=fillColor,fill opacity=0.20] (222.22, 95.98) circle (  2.13);

\path[fill=fillColor,fill opacity=0.20] (220.69, 89.02) circle (  2.13);

\path[fill=fillColor,fill opacity=0.20] (222.22, 91.93) circle (  2.13);

\path[fill=fillColor,fill opacity=0.20] (206.49, 91.17) circle (  2.13);

\path[fill=fillColor,fill opacity=0.20] (190.32, 65.13) circle (  2.13);

\path[fill=fillColor,fill opacity=0.20] (184.42, 45.91) circle (  2.13);

\path[fill=fillColor,fill opacity=0.20] (201.03, 67.02) circle (  2.13);

\path[fill=fillColor,fill opacity=0.20] (210.86, 76.13) circle (  2.13);

\path[fill=fillColor,fill opacity=0.20] (210.42, 89.78) circle (  2.13);

\path[fill=fillColor,fill opacity=0.20] (209.11,103.31) circle (  2.13);

\path[fill=fillColor,fill opacity=0.20] (209.11, 96.74) circle (  2.13);

\path[fill=fillColor,fill opacity=0.20] (203.43, 90.16) circle (  2.13);

\path[fill=fillColor,fill opacity=0.20] (201.90, 91.81) circle (  2.13);

\path[fill=fillColor,fill opacity=0.20] (199.06, 88.39) circle (  2.13);

\path[fill=fillColor,fill opacity=0.20] (183.55, 72.46) circle (  2.13);

\path[fill=fillColor,fill opacity=0.20] (185.08, 74.36) circle (  2.13);

\path[fill=fillColor,fill opacity=0.20] (197.31, 92.31) circle (  2.13);

\path[fill=fillColor,fill opacity=0.20] (210.64, 93.20) circle (  2.13);

\path[fill=fillColor,fill opacity=0.20] (214.58, 87.51) circle (  2.13);

\path[fill=fillColor,fill opacity=0.20] (213.92, 84.47) circle (  2.13);

\path[fill=fillColor,fill opacity=0.20] (213.70, 90.04) circle (  2.13);

\path[fill=fillColor,fill opacity=0.20] (223.97, 99.90) circle (  2.13);

\path[fill=fillColor,fill opacity=0.20] (231.84, 99.39) circle (  2.13);

\path[fill=fillColor,fill opacity=0.20] (231.40, 92.06) circle (  2.13);

\path[fill=fillColor,fill opacity=0.20] (231.84, 90.54) circle (  2.13);

\path[fill=fillColor,fill opacity=0.20] (229.22, 91.17) circle (  2.13);

\path[fill=fillColor,fill opacity=0.20] (220.69, 88.39) circle (  2.13);

\path[fill=fillColor,fill opacity=0.20] (201.68, 78.15) circle (  2.13);

\path[fill=fillColor,fill opacity=0.20] (191.63, 50.20) circle (  2.13);

\path[fill=fillColor,fill opacity=0.20] (204.09, 67.65) circle (  2.13);

\path[fill=fillColor,fill opacity=0.20] (208.89, 88.01) circle (  2.13);

\path[fill=fillColor,fill opacity=0.20] (209.99, 99.27) circle (  2.13);

\path[fill=fillColor,fill opacity=0.20] (207.58, 94.21) circle (  2.13);

\path[fill=fillColor,fill opacity=0.20] (201.90, 93.32) circle (  2.13);

\path[fill=fillColor,fill opacity=0.20] (201.90,100.40) circle (  2.13);

\path[fill=fillColor,fill opacity=0.20] (204.96,100.40) circle (  2.13);

\path[fill=fillColor,fill opacity=0.20] (198.41, 91.17) circle (  2.13);

\path[fill=fillColor,fill opacity=0.20] (177.87, 71.57) circle (  2.13);

\path[fill=fillColor,fill opacity=0.20] (185.52, 67.53) circle (  2.13);

\path[fill=fillColor,fill opacity=0.20] (197.10, 84.73) circle (  2.13);

\path[fill=fillColor,fill opacity=0.20] (203.21, 93.07) circle (  2.13);

\path[fill=fillColor,fill opacity=0.20] (209.99, 90.42) circle (  2.13);

\path[fill=fillColor,fill opacity=0.20] (215.45, 93.70) circle (  2.13);

\path[fill=fillColor,fill opacity=0.20] (217.42, 97.88) circle (  2.13);

\path[fill=fillColor,fill opacity=0.20] (218.51, 95.09) circle (  2.13);

\path[fill=fillColor,fill opacity=0.20] (224.63, 97.62) circle (  2.13);

\path[fill=fillColor,fill opacity=0.20] (230.53, 98.26) circle (  2.13);

\path[fill=fillColor,fill opacity=0.20] (227.90, 91.68) circle (  2.13);

\path[fill=fillColor,fill opacity=0.20] (232.49, 90.67) circle (  2.13);

\path[fill=fillColor,fill opacity=0.20] (222.88, 92.31) circle (  2.13);

\path[fill=fillColor,fill opacity=0.20] (203.87, 80.43) circle (  2.13);

\path[fill=fillColor,fill opacity=0.20] (198.63, 56.02) circle (  2.13);

\path[fill=fillColor,fill opacity=0.20] (207.58, 74.23) circle (  2.13);

\path[fill=fillColor,fill opacity=0.20] (208.68, 86.50) circle (  2.13);

\path[fill=fillColor,fill opacity=0.20] (207.37, 95.22) circle (  2.13);

\path[fill=fillColor,fill opacity=0.20] (207.37,100.03) circle (  2.13);

\path[fill=fillColor,fill opacity=0.20] (203.65,100.53) circle (  2.13);

\path[fill=fillColor,fill opacity=0.20] (205.40, 97.24) circle (  2.13);

\path[fill=fillColor,fill opacity=0.20] (202.12, 93.70) circle (  2.13);

\path[fill=fillColor,fill opacity=0.20] (190.10, 90.67) circle (  2.13);

\path[fill=fillColor,fill opacity=0.20] (176.56, 75.37) circle (  2.13);

\path[fill=fillColor,fill opacity=0.20] (186.61, 59.18) circle (  2.13);

\path[fill=fillColor,fill opacity=0.20] (193.16, 83.33) circle (  2.13);

\path[fill=fillColor,fill opacity=0.20] (201.90, 88.14) circle (  2.13);

\path[fill=fillColor,fill opacity=0.20] (202.56, 82.58) circle (  2.13);

\path[fill=fillColor,fill opacity=0.20] (208.68, 89.02) circle (  2.13);

\path[fill=fillColor,fill opacity=0.20] (215.89,100.91) circle (  2.13);

\path[fill=fillColor,fill opacity=0.20] (225.72,109.38) circle (  2.13);

\path[fill=fillColor,fill opacity=0.20] (223.75,101.67) circle (  2.13);

\path[fill=fillColor,fill opacity=0.20] (227.90, 96.49) circle (  2.13);

\path[fill=fillColor,fill opacity=0.20] (222.66, 99.65) circle (  2.13);

\path[fill=fillColor,fill opacity=0.20] (219.16,101.92) circle (  2.13);

\path[fill=fillColor,fill opacity=0.20] (224.41, 96.23) circle (  2.13);

\path[fill=fillColor,fill opacity=0.20] (210.64, 82.20) circle (  2.13);

\path[fill=fillColor,fill opacity=0.20] (195.35, 70.56) circle (  2.13);

\path[fill=fillColor,fill opacity=0.20] (203.21, 59.18) circle (  2.13);

\path[fill=fillColor,fill opacity=0.20] (208.02, 78.53) circle (  2.13);

\path[fill=fillColor,fill opacity=0.20] (208.68,100.53) circle (  2.13);

\path[fill=fillColor,fill opacity=0.20] (209.11,106.22) circle (  2.13);

\path[fill=fillColor,fill opacity=0.20] (207.80, 98.38) circle (  2.13);

\path[fill=fillColor,fill opacity=0.20] (206.27, 92.94) circle (  2.13);

\path[fill=fillColor,fill opacity=0.20] (202.34, 93.07) circle (  2.13);

\path[fill=fillColor,fill opacity=0.20] (195.78, 95.35) circle (  2.13);

\path[fill=fillColor,fill opacity=0.20] (191.85, 93.83) circle (  2.13);

\path[fill=fillColor,fill opacity=0.20] (180.49, 82.07) circle (  2.13);

\path[fill=fillColor,fill opacity=0.20] (182.02, 57.92) circle (  2.13);

\path[fill=fillColor,fill opacity=0.20] (194.91, 68.41) circle (  2.13);

\path[fill=fillColor,fill opacity=0.20] (203.00, 81.06) circle (  2.13);

\path[fill=fillColor,fill opacity=0.20] (201.47, 86.87) circle (  2.13);

\path[fill=fillColor,fill opacity=0.20] (200.15, 83.84) circle (  2.13);

\path[fill=fillColor,fill opacity=0.20] (210.21, 89.53) circle (  2.13);

\path[fill=fillColor,fill opacity=0.20] (218.29,100.15) circle (  2.13);

\path[fill=fillColor,fill opacity=0.20] (225.06, 99.90) circle (  2.13);

\path[fill=fillColor,fill opacity=0.20] (225.28, 92.82) circle (  2.13);

\path[fill=fillColor,fill opacity=0.20] (225.06, 92.82) circle (  2.13);

\path[fill=fillColor,fill opacity=0.20] (270.51, 94.21) circle (  2.13);

\path[fill=fillColor,fill opacity=0.20] (212.61, 98.89) circle (  2.13);

\path[fill=fillColor,fill opacity=0.20] (209.33, 94.84) circle (  2.13);

\path[fill=fillColor,fill opacity=0.20] (200.37, 72.84) circle (  2.13);

\path[fill=fillColor,fill opacity=0.20] (190.32, 47.68) circle (  2.13);

\path[fill=fillColor,fill opacity=0.20] (201.03, 66.64) circle (  2.13);

\path[fill=fillColor,fill opacity=0.20] (209.99, 91.81) circle (  2.13);

\path[fill=fillColor,fill opacity=0.20] (208.02,100.28) circle (  2.13);

\path[fill=fillColor,fill opacity=0.20] (206.27, 92.94) circle (  2.13);

\path[fill=fillColor,fill opacity=0.20] (206.27, 91.43) circle (  2.13);

\path[fill=fillColor,fill opacity=0.20] (201.47, 94.59) circle (  2.13);

\path[fill=fillColor,fill opacity=0.20] (204.52, 95.35) circle (  2.13);

\path[fill=fillColor,fill opacity=0.20] (199.28, 88.90) circle (  2.13);

\path[fill=fillColor,fill opacity=0.20] (193.38, 85.61) circle (  2.13);

\path[fill=fillColor,fill opacity=0.20] (181.36, 83.46) circle (  2.13);

\path[fill=fillColor,fill opacity=0.20] (181.36, 60.57) circle (  2.13);

\path[fill=fillColor,fill opacity=0.20] (195.57, 81.44) circle (  2.13);

\path[fill=fillColor,fill opacity=0.20] (204.31, 76.00) circle (  2.13);

\path[fill=fillColor,fill opacity=0.20] (208.68, 77.90) circle (  2.13);

\path[fill=fillColor,fill opacity=0.20] (214.14, 91.68) circle (  2.13);

\path[fill=fillColor,fill opacity=0.20] (208.46, 95.60) circle (  2.13);

\path[fill=fillColor,fill opacity=0.20] (213.70, 89.66) circle (  2.13);

\path[fill=fillColor,fill opacity=0.20] (216.76, 91.93) circle (  2.13);

\path[fill=fillColor,fill opacity=0.20] (218.29, 91.93) circle (  2.13);

\path[fill=fillColor,fill opacity=0.20] (219.82, 88.65) circle (  2.13);

\path[fill=fillColor,fill opacity=0.20] (214.79, 86.24) circle (  2.13);

\path[fill=fillColor,fill opacity=0.20] (211.30, 81.69) circle (  2.13);

\path[fill=fillColor,fill opacity=0.20] (208.89, 78.53) circle (  2.13);

\path[fill=fillColor,fill opacity=0.20] (198.63, 79.16) circle (  2.13);

\path[fill=fillColor,fill opacity=0.20] (194.69, 43.50) circle (  2.13);

\path[fill=fillColor,fill opacity=0.20] (208.02, 69.55) circle (  2.13);

\path[fill=fillColor,fill opacity=0.20] (209.55, 87.00) circle (  2.13);

\path[fill=fillColor,fill opacity=0.20] (206.71, 86.37) circle (  2.13);

\path[fill=fillColor,fill opacity=0.20] (200.81, 88.14) circle (  2.13);

\path[fill=fillColor,fill opacity=0.20] (202.78, 95.35) circle (  2.13);

\path[fill=fillColor,fill opacity=0.20] (205.18, 93.20) circle (  2.13);

\path[fill=fillColor,fill opacity=0.20] (201.25, 85.86) circle (  2.13);

\path[fill=fillColor,fill opacity=0.20] (193.60, 82.58) circle (  2.13);

\path[fill=fillColor,fill opacity=0.20] (191.20, 87.13) circle (  2.13);

\path[fill=fillColor,fill opacity=0.20] (188.57, 86.75) circle (  2.13);

\path[fill=fillColor,fill opacity=0.20] (177.65, 64.49) circle (  2.13);

\path[fill=fillColor,fill opacity=0.20] (180.27, 62.60) circle (  2.13);

\path[fill=fillColor,fill opacity=0.20] (191.41, 84.98) circle (  2.13);

\path[fill=fillColor,fill opacity=0.20] (201.90,101.29) circle (  2.13);

\path[fill=fillColor,fill opacity=0.20] (204.52, 86.62) circle (  2.13);

\path[fill=fillColor,fill opacity=0.20] (205.84, 81.56) circle (  2.13);

\path[fill=fillColor,fill opacity=0.20] (217.63, 90.79) circle (  2.13);

\path[fill=fillColor,fill opacity=0.20] (215.45, 96.49) circle (  2.13);

\path[fill=fillColor,fill opacity=0.20] (216.32, 86.87) circle (  2.13);

\path[fill=fillColor,fill opacity=0.20] (212.17, 85.10) circle (  2.13);

\path[fill=fillColor,fill opacity=0.20] (211.52, 86.62) circle (  2.13);

\path[fill=fillColor,fill opacity=0.20] (220.91, 85.61) circle (  2.13);

\path[fill=fillColor,fill opacity=0.20] (210.86, 86.62) circle (  2.13);

\path[fill=fillColor,fill opacity=0.20] (202.56, 75.62) circle (  2.13);

\path[fill=fillColor,fill opacity=0.20] (200.37, 62.22) circle (  2.13);

\path[fill=fillColor,fill opacity=0.20] (186.17, 54.50) circle (  2.13);

\path[fill=fillColor,fill opacity=0.20] (198.63, 40.97) circle (  2.13);

\path[fill=fillColor,fill opacity=0.20] (204.09, 60.95) circle (  2.13);

\path[fill=fillColor,fill opacity=0.20] (204.09, 83.97) circle (  2.13);

\path[fill=fillColor,fill opacity=0.20] (205.84, 92.19) circle (  2.13);

\path[fill=fillColor,fill opacity=0.20] (204.96, 95.09) circle (  2.13);

\path[fill=fillColor,fill opacity=0.20] (202.56, 96.49) circle (  2.13);

\path[fill=fillColor,fill opacity=0.20] (196.88, 90.67) circle (  2.13);

\path[fill=fillColor,fill opacity=0.20] (198.41, 84.35) circle (  2.13);

\path[fill=fillColor,fill opacity=0.20] (195.35, 80.05) circle (  2.13);

\path[fill=fillColor,fill opacity=0.20] (197.53, 84.60) circle (  2.13);

\path[fill=fillColor,fill opacity=0.20] (191.85, 94.59) circle (  2.13);

\path[fill=fillColor,fill opacity=0.20] (180.93, 83.21) circle (  2.13);

\path[fill=fillColor,fill opacity=0.20] (183.55, 70.94) circle (  2.13);

\path[fill=fillColor,fill opacity=0.20] (197.75, 84.47) circle (  2.13);

\path[fill=fillColor,fill opacity=0.20] (199.28, 92.06) circle (  2.13);

\path[fill=fillColor,fill opacity=0.20] (195.78, 95.35) circle (  2.13);

\path[fill=fillColor,fill opacity=0.20] (203.87, 90.54) circle (  2.13);

\path[fill=fillColor,fill opacity=0.20] (209.33, 86.87) circle (  2.13);

\path[fill=fillColor,fill opacity=0.20] (212.83, 89.40) circle (  2.13);

\path[fill=fillColor,fill opacity=0.20] (216.76, 89.40) circle (  2.13);

\path[fill=fillColor,fill opacity=0.20] (224.19, 86.12) circle (  2.13);

\path[fill=fillColor,fill opacity=0.20] (207.58, 82.45) circle (  2.13);

\path[fill=fillColor,fill opacity=0.20] (207.80, 73.85) circle (  2.13);

\path[fill=fillColor,fill opacity=0.20] (208.02, 66.52) circle (  2.13);

\path[fill=fillColor,fill opacity=0.20] (201.03, 68.29) circle (  2.13);

\path[fill=fillColor,fill opacity=0.20] (192.29, 59.18) circle (  2.13);

\path[fill=fillColor,fill opacity=0.20] (191.20, 49.07) circle (  2.13);

\path[fill=fillColor,fill opacity=0.20] (199.94, 69.42) circle (  2.13);

\path[fill=fillColor,fill opacity=0.20] (208.89, 84.60) circle (  2.13);

\path[fill=fillColor,fill opacity=0.20] (205.18, 94.08) circle (  2.13);

\path[fill=fillColor,fill opacity=0.20] (211.52, 95.98) circle (  2.13);

\path[fill=fillColor,fill opacity=0.20] (203.00, 88.77) circle (  2.13);

\path[fill=fillColor,fill opacity=0.20] (203.87, 76.00) circle (  2.13);

\path[fill=fillColor,fill opacity=0.20] (204.52, 77.64) circle (  2.13);

\path[fill=fillColor,fill opacity=0.20] (201.90,100.53) circle (  2.13);

\path[fill=fillColor,fill opacity=0.20] (202.34,114.69) circle (  2.13);

\path[fill=fillColor,fill opacity=0.20] (187.70, 86.12) circle (  2.13);

\path[fill=fillColor,fill opacity=0.20] (182.46, 74.36) circle (  2.13);

\path[fill=fillColor,fill opacity=0.20] (192.94, 81.56) circle (  2.13);

\path[fill=fillColor,fill opacity=0.20] (201.25, 80.30) circle (  2.13);

\path[fill=fillColor,fill opacity=0.20] (203.21, 80.68) circle (  2.13);

\path[fill=fillColor,fill opacity=0.20] (203.21, 85.23) circle (  2.13);

\path[fill=fillColor,fill opacity=0.20] (205.40, 85.99) circle (  2.13);

\path[fill=fillColor,fill opacity=0.20] (204.74, 83.97) circle (  2.13);

\path[fill=fillColor,fill opacity=0.20] (205.40, 82.83) circle (  2.13);

\path[fill=fillColor,fill opacity=0.20] (204.96, 82.07) circle (  2.13);

\path[fill=fillColor,fill opacity=0.20] (198.41, 73.47) circle (  2.13);

\path[fill=fillColor,fill opacity=0.20] (197.53, 61.08) circle (  2.13);

\path[fill=fillColor,fill opacity=0.20] (197.31, 59.56) circle (  2.13);

\path[fill=fillColor,fill opacity=0.20] (211.74, 88.90) circle (  2.13);

\path[fill=fillColor,fill opacity=0.20] (208.24,102.68) circle (  2.13);

\path[fill=fillColor,fill opacity=0.20] (207.37, 94.59) circle (  2.13);

\path[fill=fillColor,fill opacity=0.20] (201.47, 77.77) circle (  2.13);

\path[fill=fillColor,fill opacity=0.20] (204.31, 80.30) circle (  2.13);

\path[fill=fillColor,fill opacity=0.20] (208.46,101.80) circle (  2.13);

\path[fill=fillColor,fill opacity=0.20] (209.99,103.95) circle (  2.13);

\path[fill=fillColor,fill opacity=0.20] (203.00, 84.98) circle (  2.13);

\path[fill=fillColor,fill opacity=0.20] (192.94, 80.55) circle (  2.13);

\path[fill=fillColor,fill opacity=0.20] (182.02, 76.13) circle (  2.13);

\path[fill=fillColor,fill opacity=0.20] (185.08, 70.69) circle (  2.13);

\path[fill=fillColor,fill opacity=0.20] (192.51, 79.41) circle (  2.13);

\path[fill=fillColor,fill opacity=0.20] (199.94, 83.08) circle (  2.13);

\path[fill=fillColor,fill opacity=0.20] (208.46, 83.84) circle (  2.13);

\path[fill=fillColor,fill opacity=0.20] (209.99, 80.43) circle (  2.13);

\path[fill=fillColor,fill opacity=0.20] (207.80, 77.14) circle (  2.13);

\path[fill=fillColor,fill opacity=0.20] (201.47, 71.57) circle (  2.13);

\path[fill=fillColor,fill opacity=0.20] (197.75, 66.52) circle (  2.13);

\path[fill=fillColor,fill opacity=0.20] (194.26, 61.08) circle (  2.13);

\path[fill=fillColor,fill opacity=0.20] (190.10, 57.79) circle (  2.13);

\path[fill=fillColor,fill opacity=0.20] (185.95, 51.60) circle (  2.13);

\path[fill=fillColor,fill opacity=0.20] (197.75, 54.38) circle (  2.13);

\path[fill=fillColor,fill opacity=0.20] (207.37, 80.81) circle (  2.13);

\path[fill=fillColor,fill opacity=0.20] (211.95, 88.52) circle (  2.13);

\path[fill=fillColor,fill opacity=0.20] (208.46, 79.41) circle (  2.13);

\path[fill=fillColor,fill opacity=0.20] (206.93, 86.75) circle (  2.13);

\path[fill=fillColor,fill opacity=0.20] (209.33,106.10) circle (  2.13);

\path[fill=fillColor,fill opacity=0.20] (208.02, 97.75) circle (  2.13);

\path[fill=fillColor,fill opacity=0.20] (205.18, 82.20) circle (  2.13);

\path[fill=fillColor,fill opacity=0.20] (195.35, 88.01) circle (  2.13);

\path[fill=fillColor,fill opacity=0.20] (197.31, 98.00) circle (  2.13);

\path[fill=fillColor,fill opacity=0.20] (186.83, 57.29) circle (  2.13);

\path[fill=fillColor,fill opacity=0.20] (194.04, 73.22) circle (  2.13);

\path[fill=fillColor,fill opacity=0.20] (200.59, 81.69) circle (  2.13);

\path[fill=fillColor,fill opacity=0.20] (201.03, 86.50) circle (  2.13);

\path[fill=fillColor,fill opacity=0.20] (208.24, 89.78) circle (  2.13);

\path[fill=fillColor,fill opacity=0.20] (204.09, 78.40) circle (  2.13);

\path[fill=fillColor,fill opacity=0.20] (192.73, 63.99) circle (  2.13);

\path[fill=fillColor,fill opacity=0.20] (188.79, 52.61) circle (  2.13);

\path[fill=fillColor,fill opacity=0.20] (185.08, 44.77) circle (  2.13);

\path[fill=fillColor,fill opacity=0.20] (199.72, 61.33) circle (  2.13);

\path[fill=fillColor,fill opacity=0.20] (209.11, 73.60) circle (  2.13);

\path[fill=fillColor,fill opacity=0.20] (208.46, 90.67) circle (  2.13);

\path[fill=fillColor,fill opacity=0.20] (207.58,104.83) circle (  2.13);

\path[fill=fillColor,fill opacity=0.20] (207.80, 99.27) circle (  2.13);

\path[fill=fillColor,fill opacity=0.20] (208.68, 91.55) circle (  2.13);

\path[fill=fillColor,fill opacity=0.20] (203.00, 90.29) circle (  2.13);

\path[fill=fillColor,fill opacity=0.20] (201.25, 86.87) circle (  2.13);

\path[fill=fillColor,fill opacity=0.20] (202.78, 88.27) circle (  2.13);

\path[fill=fillColor,fill opacity=0.20] (196.88,105.08) circle (  2.13);

\path[fill=fillColor,fill opacity=0.20] (192.94,110.65) circle (  2.13);

\path[fill=fillColor,fill opacity=0.20] (187.04, 85.61) circle (  2.13);

\path[fill=fillColor,fill opacity=0.20] (181.36, 69.30) circle (  2.13);

\path[fill=fillColor,fill opacity=0.20] (183.11, 56.91) circle (  2.13);

\path[fill=fillColor,fill opacity=0.20] (191.20, 73.98) circle (  2.13);

\path[fill=fillColor,fill opacity=0.20] (194.26, 90.29) circle (  2.13);

\path[fill=fillColor,fill opacity=0.20] (200.37, 78.78) circle (  2.13);

\path[fill=fillColor,fill opacity=0.20] (196.44, 73.60) circle (  2.13);

\path[fill=fillColor,fill opacity=0.20] (189.45, 68.03) circle (  2.13);

\path[fill=fillColor,fill opacity=0.20] (183.33, 55.77) circle (  2.13);

\path[fill=fillColor,fill opacity=0.20] (175.90, 45.65) circle (  2.13);

\path[fill=fillColor,fill opacity=0.20] (192.07, 51.34) circle (  2.13);

\path[fill=fillColor,fill opacity=0.20] (203.21, 75.62) circle (  2.13);

\path[fill=fillColor,fill opacity=0.20] (206.49, 90.29) circle (  2.13);

\path[fill=fillColor,fill opacity=0.20] (209.77, 96.36) circle (  2.13);

\path[fill=fillColor,fill opacity=0.20] (207.15,101.16) circle (  2.13);

\path[fill=fillColor,fill opacity=0.20] (206.71, 96.11) circle (  2.13);

\path[fill=fillColor,fill opacity=0.20] (203.00, 82.83) circle (  2.13);

\path[fill=fillColor,fill opacity=0.20] (200.37, 78.15) circle (  2.13);

\path[fill=fillColor,fill opacity=0.20] (199.94, 87.00) circle (  2.13);

\path[fill=fillColor,fill opacity=0.20] (198.19, 94.08) circle (  2.13);

\path[fill=fillColor,fill opacity=0.20] (192.94, 86.12) circle (  2.13);

\path[fill=fillColor,fill opacity=0.20] (193.38, 85.86) circle (  2.13);

\path[fill=fillColor,fill opacity=0.20] (190.10, 89.28) circle (  2.13);

\path[fill=fillColor,fill opacity=0.20] (186.83, 78.40) circle (  2.13);

\path[fill=fillColor,fill opacity=0.20] (182.02, 62.09) circle (  2.13);

\path[fill=fillColor,fill opacity=0.20] (172.41, 48.81) circle (  2.13);

\path[fill=fillColor,fill opacity=0.20] (186.39, 60.95) circle (  2.13);

\path[fill=fillColor,fill opacity=0.20] (192.07, 66.90) circle (  2.13);

\path[fill=fillColor,fill opacity=0.20] (194.91, 74.23) circle (  2.13);

\path[fill=fillColor,fill opacity=0.20] (194.26, 77.39) circle (  2.13);

\path[fill=fillColor,fill opacity=0.20] (198.63, 76.89) circle (  2.13);

\path[fill=fillColor,fill opacity=0.20] (220.48, 82.07) circle (  2.13);

\path[fill=fillColor,fill opacity=0.20] (182.24, 64.62) circle (  2.13);

\path[fill=fillColor,fill opacity=0.20] (175.68, 46.16) circle (  2.13);

\path[fill=fillColor,fill opacity=0.20] (167.60, 40.09) circle (  2.13);

\path[fill=fillColor,fill opacity=0.20] (184.86, 48.05) circle (  2.13);

\path[fill=fillColor,fill opacity=0.20] (203.21, 66.01) circle (  2.13);

\path[fill=fillColor,fill opacity=0.20] (209.99, 82.70) circle (  2.13);

\path[fill=fillColor,fill opacity=0.20] (206.05, 96.49) circle (  2.13);

\path[fill=fillColor,fill opacity=0.20] (205.18, 93.45) circle (  2.13);

\path[fill=fillColor,fill opacity=0.20] (206.27, 81.82) circle (  2.13);

\path[fill=fillColor,fill opacity=0.20] (205.40, 76.25) circle (  2.13);

\path[fill=fillColor,fill opacity=0.20] (199.94, 77.52) circle (  2.13);

\path[fill=fillColor,fill opacity=0.20] (199.94, 74.36) circle (  2.13);

\path[fill=fillColor,fill opacity=0.20] (197.97, 75.87) circle (  2.13);

\path[fill=fillColor,fill opacity=0.20] (197.31, 90.04) circle (  2.13);

\path[fill=fillColor,fill opacity=0.20] (196.00, 97.75) circle (  2.13);

\path[fill=fillColor,fill opacity=0.20] (196.22, 86.24) circle (  2.13);

\path[fill=fillColor,fill opacity=0.20] (193.38, 80.17) circle (  2.13);

\path[fill=fillColor,fill opacity=0.20] (190.10, 85.74) circle (  2.13);

\path[fill=fillColor,fill opacity=0.20] (187.70, 84.35) circle (  2.13);

\path[fill=fillColor,fill opacity=0.20] (183.33, 71.45) circle (  2.13);

\path[fill=fillColor,fill opacity=0.20] (178.74, 61.08) circle (  2.13);

\path[fill=fillColor,fill opacity=0.20] (178.09, 59.18) circle (  2.13);

\path[fill=fillColor,fill opacity=0.20] (178.74, 58.93) circle (  2.13);

\path[fill=fillColor,fill opacity=0.20] (181.36, 63.10) circle (  2.13);

\path[fill=fillColor,fill opacity=0.20] (180.93, 60.70) circle (  2.13);

\path[fill=fillColor,fill opacity=0.20] (179.18, 60.45) circle (  2.13);

\path[fill=fillColor,fill opacity=0.20] (188.79, 70.44) circle (  2.13);

\path[fill=fillColor,fill opacity=0.20] (189.45, 74.48) circle (  2.13);

\path[fill=fillColor,fill opacity=0.20] (192.73, 78.91) circle (  2.13);

\path[fill=fillColor,fill opacity=0.20] (197.31, 73.85) circle (  2.13);

\path[fill=fillColor,fill opacity=0.20] (199.94, 74.36) circle (  2.13);

\path[fill=fillColor,fill opacity=0.20] (198.63, 84.60) circle (  2.13);

\path[fill=fillColor,fill opacity=0.20] (199.72, 83.84) circle (  2.13);

\path[fill=fillColor,fill opacity=0.20] (198.84, 88.27) circle (  2.13);

\path[fill=fillColor,fill opacity=0.20] (192.07, 86.75) circle (  2.13);

\path[fill=fillColor,fill opacity=0.20] (182.02, 61.58) circle (  2.13);

\path[fill=fillColor,fill opacity=0.20] (195.57, 57.03) circle (  2.13);

\path[fill=fillColor,fill opacity=0.20] (202.12, 66.77) circle (  2.13);

\path[fill=fillColor,fill opacity=0.20] (202.56, 67.78) circle (  2.13);

\path[fill=fillColor,fill opacity=0.20] (201.25, 71.83) circle (  2.13);

\path[fill=fillColor,fill opacity=0.20] (201.68, 78.02) circle (  2.13);

\path[fill=fillColor,fill opacity=0.20] (204.09, 79.16) circle (  2.13);

\path[fill=fillColor,fill opacity=0.20] (208.02, 76.89) circle (  2.13);

\path[fill=fillColor,fill opacity=0.20] (206.71, 82.45) circle (  2.13);

\path[fill=fillColor,fill opacity=0.20] (200.59, 91.93) circle (  2.13);

\path[fill=fillColor,fill opacity=0.20] (199.50, 94.08) circle (  2.13);

\path[fill=fillColor,fill opacity=0.20] (198.19, 85.74) circle (  2.13);

\path[fill=fillColor,fill opacity=0.20] (195.78, 82.32) circle (  2.13);

\path[fill=fillColor,fill opacity=0.20] (195.35, 89.66) circle (  2.13);

\path[fill=fillColor,fill opacity=0.20] (194.91, 91.55) circle (  2.13);

\path[fill=fillColor,fill opacity=0.20] (194.47, 82.45) circle (  2.13);

\path[fill=fillColor,fill opacity=0.20] (195.35, 77.90) circle (  2.13);

\path[fill=fillColor,fill opacity=0.20] (195.78, 78.78) circle (  2.13);

\path[fill=fillColor,fill opacity=0.20] (194.91, 88.27) circle (  2.13);

\path[fill=fillColor,fill opacity=0.20] (196.22, 89.53) circle (  2.13);

\path[fill=fillColor,fill opacity=0.20] (195.13, 81.56) circle (  2.13);

\path[fill=fillColor,fill opacity=0.20] (197.31, 82.95) circle (  2.13);

\path[fill=fillColor,fill opacity=0.20] (195.13, 84.73) circle (  2.13);

\path[fill=fillColor,fill opacity=0.20] (195.35, 82.32) circle (  2.13);

\path[fill=fillColor,fill opacity=0.20] (195.78, 88.01) circle (  2.13);

\path[fill=fillColor,fill opacity=0.20] (195.78, 91.55) circle (  2.13);

\path[fill=fillColor,fill opacity=0.20] (195.35, 83.71) circle (  2.13);

\path[fill=fillColor,fill opacity=0.20] (200.81, 86.75) circle (  2.13);

\path[fill=fillColor,fill opacity=0.20] (196.88, 94.46) circle (  2.13);

\path[fill=fillColor,fill opacity=0.20] (200.59, 92.44) circle (  2.13);

\path[fill=fillColor,fill opacity=0.20] (200.59, 96.99) circle (  2.13);

\path[fill=fillColor,fill opacity=0.20] (214.14, 98.63) circle (  2.13);

\path[fill=fillColor,fill opacity=0.20] (201.68, 86.50) circle (  2.13);

\path[fill=fillColor,fill opacity=0.20] (197.97, 78.53) circle (  2.13);

\path[fill=fillColor,fill opacity=0.20] (194.04, 71.20) circle (  2.13);

\path[fill=fillColor,fill opacity=0.20] (186.17, 62.22) circle (  2.13);

\path[fill=fillColor,fill opacity=0.20] (175.46, 47.17) circle (  2.13);

\path[fill=fillColor,fill opacity=0.20] (193.60, 61.21) circle (  2.13);

\path[fill=fillColor,fill opacity=0.20] (198.41, 65.13) circle (  2.13);

\path[fill=fillColor,fill opacity=0.20] (206.49, 73.72) circle (  2.13);

\path[fill=fillColor,fill opacity=0.20] (208.24, 76.76) circle (  2.13);

\path[fill=fillColor,fill opacity=0.20] (208.24, 79.41) circle (  2.13);

\path[fill=fillColor,fill opacity=0.20] (205.62, 83.08) circle (  2.13);

\path[fill=fillColor,fill opacity=0.20] (203.00, 80.55) circle (  2.13);

\path[fill=fillColor,fill opacity=0.20] (198.41, 81.18) circle (  2.13);

\path[fill=fillColor,fill opacity=0.20] (201.47, 87.89) circle (  2.13);

\path[fill=fillColor,fill opacity=0.20] (203.43, 87.51) circle (  2.13);

\path[fill=fillColor,fill opacity=0.20] (199.50, 82.58) circle (  2.13);

\path[fill=fillColor,fill opacity=0.20] (201.25, 81.69) circle (  2.13);

\path[fill=fillColor,fill opacity=0.20] (203.21, 79.67) circle (  2.13);

\path[fill=fillColor,fill opacity=0.20] (198.41, 87.76) circle (  2.13);

\path[fill=fillColor,fill opacity=0.20] (196.22, 97.62) circle (  2.13);

\path[fill=fillColor,fill opacity=0.20] (198.19, 92.69) circle (  2.13);

\path[fill=fillColor,fill opacity=0.20] (199.28, 88.52) circle (  2.13);

\path[fill=fillColor,fill opacity=0.20] (201.68, 93.58) circle (  2.13);

\path[fill=fillColor,fill opacity=0.20] (201.47, 90.67) circle (  2.13);

\path[fill=fillColor,fill opacity=0.20] (199.06, 88.90) circle (  2.13);

\path[fill=fillColor,fill opacity=0.20] (200.37, 94.21) circle (  2.13);

\path[fill=fillColor,fill opacity=0.20] (201.68, 93.58) circle (  2.13);

\path[fill=fillColor,fill opacity=0.20] (203.43, 93.96) circle (  2.13);

\path[fill=fillColor,fill opacity=0.20] (204.31, 98.63) circle (  2.13);

\path[fill=fillColor,fill opacity=0.20] (206.71, 91.81) circle (  2.13);

\path[fill=fillColor,fill opacity=0.20] (204.74, 87.89) circle (  2.13);

\path[fill=fillColor,fill opacity=0.20] (195.57, 79.92) circle (  2.13);

\path[fill=fillColor,fill opacity=0.20] (184.20, 63.36) circle (  2.13);

\path[fill=fillColor,fill opacity=0.20] (190.76, 44.64) circle (  2.13);

\path[fill=fillColor,fill opacity=0.20] (195.13, 46.79) circle (  2.13);

\path[fill=fillColor,fill opacity=0.20] (198.41, 51.97) circle (  2.13);

\path[fill=fillColor,fill opacity=0.20] (199.72, 66.01) circle (  2.13);

\path[fill=fillColor,fill opacity=0.20] (197.53, 65.13) circle (  2.13);

\path[fill=fillColor,fill opacity=0.20] (197.75, 63.36) circle (  2.13);

\path[fill=fillColor,fill opacity=0.20] (201.90, 71.45) circle (  2.13);

\path[fill=fillColor,fill opacity=0.20] (200.59, 81.69) circle (  2.13);

\path[fill=fillColor,fill opacity=0.20] (202.34, 84.85) circle (  2.13);

\path[fill=fillColor,fill opacity=0.20] (206.71, 87.63) circle (  2.13);

\path[fill=fillColor,fill opacity=0.20] (207.15, 88.65) circle (  2.13);

\path[fill=fillColor,fill opacity=0.20] (206.05, 88.27) circle (  2.13);

\path[fill=fillColor,fill opacity=0.20] (204.52, 96.74) circle (  2.13);

\path[fill=fillColor,fill opacity=0.20] (203.65,104.20) circle (  2.13);

\path[fill=fillColor,fill opacity=0.20] (205.40,103.95) circle (  2.13);

\path[fill=fillColor,fill opacity=0.20] (207.37,100.66) circle (  2.13);

\path[fill=fillColor,fill opacity=0.20] (206.93, 96.99) circle (  2.13);

\path[fill=fillColor,fill opacity=0.20] (205.18, 89.66) circle (  2.13);

\path[fill=fillColor,fill opacity=0.20] (203.00, 82.83) circle (  2.13);

\path[fill=fillColor,fill opacity=0.20] (205.18, 85.74) circle (  2.13);

\path[fill=fillColor,fill opacity=0.20] (206.49, 86.12) circle (  2.13);

\path[fill=fillColor,fill opacity=0.20] (205.84, 79.54) circle (  2.13);

\path[fill=fillColor,fill opacity=0.20] (201.90, 71.57) circle (  2.13);

\path[fill=fillColor,fill opacity=0.20] (194.04, 60.57) circle (  2.13);

\path[fill=fillColor,fill opacity=0.20] (184.86, 44.89) circle (  2.13);

\path[fill=fillColor,fill opacity=0.20] (186.17, 38.82) circle (  2.13);

\path[fill=fillColor,fill opacity=0.20] (188.79, 42.87) circle (  2.13);

\path[fill=fillColor,fill opacity=0.20] (187.26, 51.72) circle (  2.13);

\path[fill=fillColor,fill opacity=0.20] (187.92, 58.55) circle (  2.13);

\path[fill=fillColor,fill opacity=0.20] (196.44, 66.90) circle (  2.13);

\path[fill=fillColor,fill opacity=0.20] (203.87, 82.45) circle (  2.13);

\path[fill=fillColor,fill opacity=0.20] (205.18, 88.01) circle (  2.13);

\path[fill=fillColor,fill opacity=0.20] (204.96, 85.36) circle (  2.13);

\path[fill=fillColor,fill opacity=0.20] (203.43, 88.14) circle (  2.13);

\path[fill=fillColor,fill opacity=0.20] (202.56, 90.54) circle (  2.13);

\path[fill=fillColor,fill opacity=0.20] (200.59, 87.51) circle (  2.13);

\path[fill=fillColor,fill opacity=0.20] (201.90, 83.46) circle (  2.13);

\path[fill=fillColor,fill opacity=0.20] (201.25, 70.06) circle (  2.13);

\path[fill=fillColor,fill opacity=0.20] (198.41, 62.47) circle (  2.13);

\path[fill=fillColor,fill opacity=0.20] (194.91, 61.58) circle (  2.13);

\path[fill=fillColor,fill opacity=0.20] (189.01, 51.22) circle (  2.13);

\path[fill=fillColor,fill opacity=0.20] (182.89, 43.25) circle (  2.13);

\path[fill=fillColor,fill opacity=0.20] (186.61, 48.56) circle (  2.13);

\path[fill=fillColor,fill opacity=0.20] (191.41, 55.39) circle (  2.13);

\path[fill=fillColor,fill opacity=0.20] (191.41, 54.25) circle (  2.13);

\path[fill=fillColor,fill opacity=0.20] (196.66, 50.46) circle (  2.13);

\path[fill=fillColor,fill opacity=0.20] (187.70, 50.71) circle (  2.13);

\path[fill=fillColor,fill opacity=0.20] (183.77, 52.48) circle (  2.13);

\path[fill=fillColor,fill opacity=0.20] (190.98, 49.19) circle (  2.13);

\path[fill=fillColor,fill opacity=0.20] (189.67, 40.59) circle (  2.13);

\path[fill=fillColor,fill opacity=0.20] (204.31, 52.35) circle (  2.13);

\path[fill=fillColor,fill opacity=0.20] (203.21, 52.48) circle (  2.13);

\path[fill=fillColor,fill opacity=0.20] (203.43, 52.23) circle (  2.13);

\path[fill=fillColor,fill opacity=0.20] (198.84, 45.53) circle (  2.13);

\path[fill=fillColor,fill opacity=0.20] (202.34, 51.85) circle (  2.13);

\path[fill=fillColor,fill opacity=0.20] (207.37, 57.92) circle (  2.13);

\path[fill=fillColor,fill opacity=0.20] (207.58, 62.85) circle (  2.13);

\path[fill=fillColor,fill opacity=0.20] (206.05, 66.90) circle (  2.13);

\path[fill=fillColor,fill opacity=0.20] (205.18, 69.55) circle (  2.13);

\path[fill=fillColor,fill opacity=0.20] (200.15, 67.78) circle (  2.13);

\path[fill=fillColor,fill opacity=0.20] (192.73, 60.95) circle (  2.13);

\path[fill=fillColor,fill opacity=0.20] (181.58, 56.65) circle (  2.13);

\path[fill=fillColor,fill opacity=0.20] (207.37, 45.53) circle (  2.13);

\path[fill=fillColor,fill opacity=0.20] (206.49, 57.66) circle (  2.13);

\path[fill=fillColor,fill opacity=0.20] (213.92, 65.76) circle (  2.13);

\path[fill=fillColor,fill opacity=0.20] (210.86, 71.70) circle (  2.13);

\path[fill=fillColor,fill opacity=0.20] (210.86, 78.66) circle (  2.13);

\path[fill=fillColor,fill opacity=0.20] (206.71, 81.69) circle (  2.13);

\path[fill=fillColor,fill opacity=0.20] (203.87, 79.29) circle (  2.13);

\path[fill=fillColor,fill opacity=0.20] (201.03, 75.12) circle (  2.13);

\path[fill=fillColor,fill opacity=0.20] (190.10, 68.41) circle (  2.13);

\path[fill=fillColor,fill opacity=0.20] (171.97, 61.33) circle (  2.13);

\path[fill=fillColor,fill opacity=0.20] (200.59, 40.34) circle (  2.13);

\path[fill=fillColor,fill opacity=0.20] (207.58, 58.42) circle (  2.13);

\path[fill=fillColor,fill opacity=0.20] (212.39, 76.51) circle (  2.13);

\path[fill=fillColor,fill opacity=0.20] (215.01, 83.71) circle (  2.13);

\path[fill=fillColor,fill opacity=0.20] (211.74, 88.27) circle (  2.13);

\path[fill=fillColor,fill opacity=0.20] (209.11, 91.68) circle (  2.13);

\path[fill=fillColor,fill opacity=0.20] (204.31, 90.29) circle (  2.13);

\path[fill=fillColor,fill opacity=0.20] (200.37, 85.86) circle (  2.13);

\path[fill=fillColor,fill opacity=0.20] (193.82, 80.30) circle (  2.13);

\path[fill=fillColor,fill opacity=0.20] (181.58, 70.69) circle (  2.13);

\path[fill=fillColor,fill opacity=0.20] (166.94, 59.31) circle (  2.13);

\path[fill=fillColor,fill opacity=0.20] (205.18, 50.71) circle (  2.13);

\path[fill=fillColor,fill opacity=0.20] (208.24, 72.84) circle (  2.13);

\path[fill=fillColor,fill opacity=0.20] (208.68, 92.69) circle (  2.13);

\path[fill=fillColor,fill opacity=0.20] (207.37,101.04) circle (  2.13);

\path[fill=fillColor,fill opacity=0.20] (207.80,100.15) circle (  2.13);

\path[fill=fillColor,fill opacity=0.20] (204.96, 96.11) circle (  2.13);

\path[fill=fillColor,fill opacity=0.20] (199.06, 91.17) circle (  2.13);

\path[fill=fillColor,fill opacity=0.20] (192.29, 89.66) circle (  2.13);

\path[fill=fillColor,fill opacity=0.20] (183.77, 86.12) circle (  2.13);

\path[fill=fillColor,fill opacity=0.20] (168.69, 73.47) circle (  2.13);

\path[fill=fillColor,fill opacity=0.20] (204.31, 65.25) circle (  2.13);

\path[fill=fillColor,fill opacity=0.20] (205.40, 84.09) circle (  2.13);

\path[fill=fillColor,fill opacity=0.20] (206.71, 97.62) circle (  2.13);

\path[fill=fillColor,fill opacity=0.20] (205.18,107.61) circle (  2.13);

\path[fill=fillColor,fill opacity=0.20] (205.40,106.60) circle (  2.13);

\path[fill=fillColor,fill opacity=0.20] (201.03, 97.75) circle (  2.13);

\path[fill=fillColor,fill opacity=0.20] (198.84, 92.69) circle (  2.13);

\path[fill=fillColor,fill opacity=0.20] (187.48, 90.29) circle (  2.13);

\path[fill=fillColor,fill opacity=0.20] (172.41, 84.22) circle (  2.13);

\path[fill=fillColor,fill opacity=0.20] (207.58, 40.47) circle (  2.13);

\path[fill=fillColor,fill opacity=0.20] (208.89, 41.99) circle (  2.13);

\path[fill=fillColor,fill opacity=0.20] (203.87, 74.23) circle (  2.13);

\path[fill=fillColor,fill opacity=0.20] (205.40, 90.42) circle (  2.13);

\path[fill=fillColor,fill opacity=0.20] (207.37, 98.63) circle (  2.13);

\path[fill=fillColor,fill opacity=0.20] (209.77,108.88) circle (  2.13);

\path[fill=fillColor,fill opacity=0.20] (201.03,107.23) circle (  2.13);

\path[fill=fillColor,fill opacity=0.20] (200.81, 98.38) circle (  2.13);

\path[fill=fillColor,fill opacity=0.20] (199.06, 93.83) circle (  2.13);

\path[fill=fillColor,fill opacity=0.20] (186.17, 86.75) circle (  2.13);

\path[fill=fillColor,fill opacity=0.20] (170.00, 70.06) circle (  2.13);

\path[fill=fillColor,fill opacity=0.20] (209.55, 52.10) circle (  2.13);

\path[fill=fillColor,fill opacity=0.20] (210.86, 55.64) circle (  2.13);

\path[fill=fillColor,fill opacity=0.20] (210.64, 53.11) circle (  2.13);

\path[fill=fillColor,fill opacity=0.20] (207.37, 48.56) circle (  2.13);

\path[fill=fillColor,fill opacity=0.20] (205.18, 73.09) circle (  2.13);

\path[fill=fillColor,fill opacity=0.20] (209.11, 90.04) circle (  2.13);

\path[fill=fillColor,fill opacity=0.20] (207.37, 99.65) circle (  2.13);

\path[fill=fillColor,fill opacity=0.20] (208.02,106.60) circle (  2.13);

\path[fill=fillColor,fill opacity=0.20] (204.09,103.06) circle (  2.13);

\path[fill=fillColor,fill opacity=0.20] (200.15, 95.47) circle (  2.13);

\path[fill=fillColor,fill opacity=0.20] (199.72, 92.44) circle (  2.13);

\path[fill=fillColor,fill opacity=0.20] (188.57, 81.82) circle (  2.13);

\path[fill=fillColor,fill opacity=0.20] (212.17, 65.63) circle (  2.13);

\path[fill=fillColor,fill opacity=0.20] (207.37, 65.88) circle (  2.13);

\path[fill=fillColor,fill opacity=0.20] (206.49, 63.99) circle (  2.13);

\path[fill=fillColor,fill opacity=0.20] (208.89, 57.92) circle (  2.13);

\path[fill=fillColor,fill opacity=0.20] (196.44, 71.45) circle (  2.13);

\path[fill=fillColor,fill opacity=0.20] (211.74, 88.39) circle (  2.13);

\path[fill=fillColor,fill opacity=0.20] (208.02, 98.13) circle (  2.13);

\path[fill=fillColor,fill opacity=0.20] (208.46,100.28) circle (  2.13);

\path[fill=fillColor,fill opacity=0.20] (202.78, 98.13) circle (  2.13);

\path[fill=fillColor,fill opacity=0.20] (196.00, 95.85) circle (  2.13);

\path[fill=fillColor,fill opacity=0.20] (197.10, 92.31) circle (  2.13);

\path[fill=fillColor,fill opacity=0.20] (192.29, 80.55) circle (  2.13);

\path[fill=fillColor,fill opacity=0.20] (196.88, 66.64) circle (  2.13);

\path[fill=fillColor,fill opacity=0.20] (208.46, 71.32) circle (  2.13);

\path[fill=fillColor,fill opacity=0.20] (206.05, 75.87) circle (  2.13);

\path[fill=fillColor,fill opacity=0.20] (202.34, 73.60) circle (  2.13);

\path[fill=fillColor,fill opacity=0.20] (201.03, 70.31) circle (  2.13);

\path[fill=fillColor,fill opacity=0.20] (209.11, 65.38) circle (  2.13);

\path[fill=fillColor,fill opacity=0.20] (204.31, 51.34) circle (  2.13);

\path[fill=fillColor,fill opacity=0.20] (182.67, 66.52) circle (  2.13);

\path[fill=fillColor,fill opacity=0.20] (206.93, 85.61) circle (  2.13);

\path[fill=fillColor,fill opacity=0.20] (206.71, 94.21) circle (  2.13);

\path[fill=fillColor,fill opacity=0.20] (205.40, 95.73) circle (  2.13);

\path[fill=fillColor,fill opacity=0.20] (203.00, 96.11) circle (  2.13);

\path[fill=fillColor,fill opacity=0.20] (197.31,101.16) circle (  2.13);

\path[fill=fillColor,fill opacity=0.20] (197.53, 99.14) circle (  2.13);

\path[fill=fillColor,fill opacity=0.20] (197.53, 82.83) circle (  2.13);

\path[fill=fillColor,fill opacity=0.20] (188.14, 62.60) circle (  2.13);

\path[fill=fillColor,fill opacity=0.20] (192.73, 70.94) circle (  2.13);

\path[fill=fillColor,fill opacity=0.20] (195.35, 78.91) circle (  2.13);

\path[fill=fillColor,fill opacity=0.20] (199.28, 80.81) circle (  2.13);

\path[fill=fillColor,fill opacity=0.20] (197.53, 77.14) circle (  2.13);

\path[fill=fillColor,fill opacity=0.20] (204.31, 72.21) circle (  2.13);

\path[fill=fillColor,fill opacity=0.20] (197.10, 65.88) circle (  2.13);

\path[fill=fillColor,fill opacity=0.20] (200.37, 54.76) circle (  2.13);

\path[fill=fillColor,fill opacity=0.20] (186.83, 41.23) circle (  2.13);

\path[fill=fillColor,fill opacity=0.20] (185.73, 50.84) circle (  2.13);

\path[fill=fillColor,fill opacity=0.20] (199.94, 77.14) circle (  2.13);

\path[fill=fillColor,fill opacity=0.20] (208.02, 91.05) circle (  2.13);

\path[fill=fillColor,fill opacity=0.20] (202.78, 94.84) circle (  2.13);

\path[fill=fillColor,fill opacity=0.20] (200.15, 93.32) circle (  2.13);

\path[fill=fillColor,fill opacity=0.20] (200.59, 98.76) circle (  2.13);

\path[fill=fillColor,fill opacity=0.20] (199.28,101.04) circle (  2.13);

\path[fill=fillColor,fill opacity=0.20] (201.90, 84.47) circle (  2.13);

\path[fill=fillColor,fill opacity=0.20] (191.41, 61.84) circle (  2.13);

\path[fill=fillColor,fill opacity=0.20] (192.29, 61.84) circle (  2.13);

\path[fill=fillColor,fill opacity=0.20] (196.88, 69.80) circle (  2.13);

\path[fill=fillColor,fill opacity=0.20] (201.03, 78.02) circle (  2.13);

\path[fill=fillColor,fill opacity=0.20] (194.26, 83.97) circle (  2.13);

\path[fill=fillColor,fill opacity=0.20] (194.69, 84.22) circle (  2.13);

\path[fill=fillColor,fill opacity=0.20] (194.69, 76.38) circle (  2.13);

\path[fill=fillColor,fill opacity=0.20] (194.91, 69.80) circle (  2.13);

\path[fill=fillColor,fill opacity=0.20] (201.47, 62.22) circle (  2.13);

\path[fill=fillColor,fill opacity=0.20] (186.17, 46.28) circle (  2.13);

\path[fill=fillColor,fill opacity=0.20] (181.36, 66.90) circle (  2.13);

\path[fill=fillColor,fill opacity=0.20] (199.06, 85.99) circle (  2.13);

\path[fill=fillColor,fill opacity=0.20] (204.31, 92.94) circle (  2.13);

\path[fill=fillColor,fill opacity=0.20] (216.76, 91.17) circle (  2.13);

\path[fill=fillColor,fill opacity=0.20] (201.68, 90.29) circle (  2.13);

\path[fill=fillColor,fill opacity=0.20] (199.28, 95.09) circle (  2.13);

\path[fill=fillColor,fill opacity=0.20] (203.21, 86.75) circle (  2.13);

\path[fill=fillColor,fill opacity=0.20] (198.63, 70.69) circle (  2.13);

\path[fill=fillColor,fill opacity=0.20] (200.15, 65.76) circle (  2.13);

\path[fill=fillColor,fill opacity=0.20] (200.37, 73.09) circle (  2.13);

\path[fill=fillColor,fill opacity=0.20] (201.47, 81.31) circle (  2.13);

\path[fill=fillColor,fill opacity=0.20] (197.31, 87.76) circle (  2.13);

\path[fill=fillColor,fill opacity=0.20] (200.59, 89.40) circle (  2.13);

\path[fill=fillColor,fill opacity=0.20] (199.28, 80.81) circle (  2.13);

\path[fill=fillColor,fill opacity=0.20] (194.47, 73.22) circle (  2.13);

\path[fill=fillColor,fill opacity=0.20] (199.72, 65.63) circle (  2.13);

\path[fill=fillColor,fill opacity=0.20] (191.20, 49.83) circle (  2.13);

\path[fill=fillColor,fill opacity=0.20] (167.82, 57.41) circle (  2.13);

\path[fill=fillColor,fill opacity=0.20] (175.46, 75.62) circle (  2.13);

\path[fill=fillColor,fill opacity=0.20] (194.47, 85.10) circle (  2.13);

\path[fill=fillColor,fill opacity=0.20] (214.79, 92.31) circle (  2.13);

\path[fill=fillColor,fill opacity=0.20] (202.34, 90.79) circle (  2.13);

\path[fill=fillColor,fill opacity=0.20] (200.59, 91.17) circle (  2.13);

\path[fill=fillColor,fill opacity=0.20] (204.09, 93.07) circle (  2.13);

\path[fill=fillColor,fill opacity=0.20] (200.59, 83.21) circle (  2.13);

\path[fill=fillColor,fill opacity=0.20] (195.13, 60.57) circle (  2.13);

\path[fill=fillColor,fill opacity=0.20] (196.88, 74.10) circle (  2.13);

\path[fill=fillColor,fill opacity=0.20] (201.03, 83.46) circle (  2.13);

\path[fill=fillColor,fill opacity=0.20] (206.27, 87.76) circle (  2.13);

\path[fill=fillColor,fill opacity=0.20] (198.63, 88.90) circle (  2.13);

\path[fill=fillColor,fill opacity=0.20] (197.10, 81.31) circle (  2.13);

\path[fill=fillColor,fill opacity=0.20] (204.52, 71.32) circle (  2.13);

\path[fill=fillColor,fill opacity=0.20] (198.63, 62.34) circle (  2.13);

\path[fill=fillColor,fill opacity=0.20] (185.73, 50.96) circle (  2.13);

\path[fill=fillColor,fill opacity=0.20] (175.68, 76.89) circle (  2.13);

\path[fill=fillColor,fill opacity=0.20] (193.16, 96.11) circle (  2.13);

\path[fill=fillColor,fill opacity=0.20] (202.56, 95.98) circle (  2.13);

\path[fill=fillColor,fill opacity=0.20] (206.71, 92.82) circle (  2.13);

\path[fill=fillColor,fill opacity=0.20] (201.03, 94.46) circle (  2.13);

\path[fill=fillColor,fill opacity=0.20] (200.37, 89.28) circle (  2.13);

\path[fill=fillColor,fill opacity=0.20] (199.94, 73.22) circle (  2.13);

\path[fill=fillColor,fill opacity=0.20] (197.97, 63.99) circle (  2.13);

\path[fill=fillColor,fill opacity=0.20] (205.18, 74.61) circle (  2.13);

\path[fill=fillColor,fill opacity=0.20] (210.21, 85.86) circle (  2.13);

\path[fill=fillColor,fill opacity=0.20] (209.55, 86.24) circle (  2.13);

\path[fill=fillColor,fill opacity=0.20] (200.59, 87.51) circle (  2.13);

\path[fill=fillColor,fill opacity=0.20] (199.72, 89.28) circle (  2.13);

\path[fill=fillColor,fill opacity=0.20] (204.31, 81.69) circle (  2.13);

\path[fill=fillColor,fill opacity=0.20] (201.90, 71.83) circle (  2.13);

\path[fill=fillColor,fill opacity=0.20] (199.06, 63.23) circle (  2.13);

\path[fill=fillColor,fill opacity=0.20] (191.63, 52.10) circle (  2.13);

\path[fill=fillColor,fill opacity=0.20] (171.97, 41.35) circle (  2.13);

\path[fill=fillColor,fill opacity=0.20] (167.38, 62.09) circle (  2.13);

\path[fill=fillColor,fill opacity=0.20] (177.43, 85.48) circle (  2.13);

\path[fill=fillColor,fill opacity=0.20] (192.73, 91.68) circle (  2.13);

\path[fill=fillColor,fill opacity=0.20] (205.62, 90.67) circle (  2.13);

\path[fill=fillColor,fill opacity=0.20] (203.43, 92.06) circle (  2.13);

\path[fill=fillColor,fill opacity=0.20] (199.94, 90.16) circle (  2.13);

\path[fill=fillColor,fill opacity=0.20] (200.15, 85.23) circle (  2.13);

\path[fill=fillColor,fill opacity=0.20] (196.22, 74.36) circle (  2.13);

\path[fill=fillColor,fill opacity=0.20] (211.95, 61.21) circle (  2.13);

\path[fill=fillColor,fill opacity=0.20] (211.08, 71.95) circle (  2.13);

\path[fill=fillColor,fill opacity=0.20] (210.64, 80.30) circle (  2.13);

\path[fill=fillColor,fill opacity=0.20] (211.74, 83.33) circle (  2.13);

\path[fill=fillColor,fill opacity=0.20] (207.80, 88.27) circle (  2.13);

\path[fill=fillColor,fill opacity=0.20] (207.58, 90.92) circle (  2.13);

\path[fill=fillColor,fill opacity=0.20] (202.12, 85.61) circle (  2.13);

\path[fill=fillColor,fill opacity=0.20] (199.06, 78.66) circle (  2.13);

\path[fill=fillColor,fill opacity=0.20] (199.50, 72.21) circle (  2.13);

\path[fill=fillColor,fill opacity=0.20] (189.23, 55.89) circle (  2.13);

\path[fill=fillColor,fill opacity=0.20] (178.96, 59.06) circle (  2.13);

\path[fill=fillColor,fill opacity=0.20] (183.77, 74.99) circle (  2.13);

\path[fill=fillColor,fill opacity=0.20] (186.83, 84.98) circle (  2.13);

\path[fill=fillColor,fill opacity=0.20] (200.59, 90.67) circle (  2.13);

\path[fill=fillColor,fill opacity=0.20] (204.74, 93.45) circle (  2.13);

\path[fill=fillColor,fill opacity=0.20] (202.56, 91.17) circle (  2.13);

\path[fill=fillColor,fill opacity=0.20] (198.19, 82.32) circle (  2.13);

\path[fill=fillColor,fill opacity=0.20] (194.91, 75.24) circle (  2.13);

\path[fill=fillColor,fill opacity=0.20] (190.54, 57.29) circle (  2.13);

\path[fill=fillColor,fill opacity=0.20] (210.21, 62.09) circle (  2.13);

\path[fill=fillColor,fill opacity=0.20] (217.20, 69.05) circle (  2.13);

\path[fill=fillColor,fill opacity=0.20] (213.48, 79.03) circle (  2.13);

\path[fill=fillColor,fill opacity=0.20] (209.99, 83.97) circle (  2.13);

\path[fill=fillColor,fill opacity=0.20] (209.33, 85.99) circle (  2.13);

\path[fill=fillColor,fill opacity=0.20] (213.70, 88.27) circle (  2.13);

\path[fill=fillColor,fill opacity=0.20] (212.61, 88.77) circle (  2.13);

\path[fill=fillColor,fill opacity=0.20] (208.68, 86.75) circle (  2.13);

\path[fill=fillColor,fill opacity=0.20] (203.00, 81.69) circle (  2.13);

\path[fill=fillColor,fill opacity=0.20] (192.51, 74.23) circle (  2.13);

\path[fill=fillColor,fill opacity=0.20] (181.36, 58.68) circle (  2.13);

\path[fill=fillColor,fill opacity=0.20] (170.88, 54.00) circle (  2.13);

\path[fill=fillColor,fill opacity=0.20] (169.56, 71.45) circle (  2.13);

\path[fill=fillColor,fill opacity=0.20] (180.27, 84.47) circle (  2.13);

\path[fill=fillColor,fill opacity=0.20] (198.19, 89.91) circle (  2.13);

\path[fill=fillColor,fill opacity=0.20] (203.21, 88.14) circle (  2.13);

\path[fill=fillColor,fill opacity=0.20] (201.25, 83.08) circle (  2.13);

\path[fill=fillColor,fill opacity=0.20] (201.03, 81.06) circle (  2.13);

\path[fill=fillColor,fill opacity=0.20] (194.04, 76.25) circle (  2.13);

\path[fill=fillColor,fill opacity=0.20] (210.21, 64.87) circle (  2.13);

\path[fill=fillColor,fill opacity=0.20] (220.04, 74.61) circle (  2.13);

\path[fill=fillColor,fill opacity=0.20] (219.38, 82.32) circle (  2.13);

\path[fill=fillColor,fill opacity=0.20] (212.83, 92.44) circle (  2.13);

\path[fill=fillColor,fill opacity=0.20] (211.08, 91.68) circle (  2.13);

\path[fill=fillColor,fill opacity=0.20] (206.93, 87.00) circle (  2.13);

\path[fill=fillColor,fill opacity=0.20] (205.84, 88.77) circle (  2.13);

\path[fill=fillColor,fill opacity=0.20] (206.71, 88.90) circle (  2.13);

\path[fill=fillColor,fill opacity=0.20] (204.31, 82.58) circle (  2.13);

\path[fill=fillColor,fill opacity=0.20] (200.15, 73.09) circle (  2.13);

\path[fill=fillColor,fill opacity=0.20] (188.57, 62.98) circle (  2.13);

\path[fill=fillColor,fill opacity=0.20] (177.87, 53.37) circle (  2.13);

\path[fill=fillColor,fill opacity=0.20] (170.88, 47.93) circle (  2.13);

\path[fill=fillColor,fill opacity=0.20] (175.90, 64.87) circle (  2.13);

\path[fill=fillColor,fill opacity=0.20] (181.15, 78.78) circle (  2.13);

\path[fill=fillColor,fill opacity=0.20] (194.69, 83.46) circle (  2.13);

\path[fill=fillColor,fill opacity=0.20] (203.87, 85.10) circle (  2.13);

\path[fill=fillColor,fill opacity=0.20] (203.87, 82.58) circle (  2.13);

\path[fill=fillColor,fill opacity=0.20] (202.34, 81.82) circle (  2.13);

\path[fill=fillColor,fill opacity=0.20] (198.41, 76.00) circle (  2.13);

\path[fill=fillColor,fill opacity=0.20] (214.14, 67.40) circle (  2.13);

\path[fill=fillColor,fill opacity=0.20] (218.29, 76.63) circle (  2.13);

\path[fill=fillColor,fill opacity=0.20] (227.69, 82.07) circle (  2.13);

\path[fill=fillColor,fill opacity=0.20] (218.07, 87.38) circle (  2.13);

\path[fill=fillColor,fill opacity=0.20] (213.05, 94.21) circle (  2.13);

\path[fill=fillColor,fill opacity=0.20] (215.23, 89.78) circle (  2.13);

\path[fill=fillColor,fill opacity=0.20] (208.68, 83.84) circle (  2.13);

\path[fill=fillColor,fill opacity=0.20] (202.78, 86.87) circle (  2.13);

\path[fill=fillColor,fill opacity=0.20] (202.78, 87.13) circle (  2.13);

\path[fill=fillColor,fill opacity=0.20] (201.90, 78.78) circle (  2.13);

\path[fill=fillColor,fill opacity=0.20] (196.22, 67.28) circle (  2.13);

\path[fill=fillColor,fill opacity=0.20] (177.65, 52.48) circle (  2.13);

\path[fill=fillColor,fill opacity=0.20] (166.07, 42.62) circle (  2.13);

\path[fill=fillColor,fill opacity=0.20] (182.02, 74.74) circle (  2.13);

\path[fill=fillColor,fill opacity=0.20] (201.90, 82.58) circle (  2.13);

\path[fill=fillColor,fill opacity=0.20] (208.24, 83.46) circle (  2.13);

\path[fill=fillColor,fill opacity=0.20] (206.05, 80.55) circle (  2.13);

\path[fill=fillColor,fill opacity=0.20] (201.03, 80.68) circle (  2.13);

\path[fill=fillColor,fill opacity=0.20] (200.15, 74.36) circle (  2.13);

\path[fill=fillColor,fill opacity=0.20] (217.42, 68.16) circle (  2.13);

\path[fill=fillColor,fill opacity=0.20] (222.44, 78.53) circle (  2.13);

\path[fill=fillColor,fill opacity=0.20] (220.69, 81.31) circle (  2.13);

\path[fill=fillColor,fill opacity=0.20] (223.10, 83.08) circle (  2.13);

\path[fill=fillColor,fill opacity=0.20] (215.45, 88.14) circle (  2.13);

\path[fill=fillColor,fill opacity=0.20] (213.92, 90.29) circle (  2.13);

\path[fill=fillColor,fill opacity=0.20] (214.14, 86.75) circle (  2.13);

\path[fill=fillColor,fill opacity=0.20] (212.17, 85.74) circle (  2.13);

\path[fill=fillColor,fill opacity=0.20] (207.37, 85.36) circle (  2.13);

\path[fill=fillColor,fill opacity=0.20] (196.88, 79.41) circle (  2.13);

\path[fill=fillColor,fill opacity=0.20] (190.98, 72.59) circle (  2.13);

\path[fill=fillColor,fill opacity=0.20] (184.20, 63.86) circle (  2.13);

\path[fill=fillColor,fill opacity=0.20] (180.71, 47.04) circle (  2.13);

\path[fill=fillColor,fill opacity=0.20] (165.41, 57.16) circle (  2.13);

\path[fill=fillColor,fill opacity=0.20] (178.74, 74.10) circle (  2.13);

\path[fill=fillColor,fill opacity=0.20] (193.82, 82.95) circle (  2.13);

\path[fill=fillColor,fill opacity=0.20] (198.63, 81.56) circle (  2.13);

\path[fill=fillColor,fill opacity=0.20] (208.46, 81.18) circle (  2.13);

\path[fill=fillColor,fill opacity=0.20] (207.37, 84.73) circle (  2.13);

\path[fill=fillColor,fill opacity=0.20] (204.52, 75.49) circle (  2.13);

\path[fill=fillColor,fill opacity=0.20] (219.82, 60.45) circle (  2.13);

\path[fill=fillColor,fill opacity=0.20] (214.36, 74.48) circle (  2.13);

\path[fill=fillColor,fill opacity=0.20] (219.82, 85.61) circle (  2.13);

\path[fill=fillColor,fill opacity=0.20] (223.97, 88.14) circle (  2.13);

\path[fill=fillColor,fill opacity=0.20] (219.82, 87.76) circle (  2.13);

\path[fill=fillColor,fill opacity=0.20] (212.39, 89.02) circle (  2.13);

\path[fill=fillColor,fill opacity=0.20] (212.61, 93.45) circle (  2.13);

\path[fill=fillColor,fill opacity=0.20] (217.20, 90.29) circle (  2.13);

\path[fill=fillColor,fill opacity=0.20] (208.89, 85.61) circle (  2.13);

\path[fill=fillColor,fill opacity=0.20] (203.43, 89.78) circle (  2.13);

\path[fill=fillColor,fill opacity=0.20] (192.94, 87.25) circle (  2.13);

\path[fill=fillColor,fill opacity=0.20] (185.08, 72.46) circle (  2.13);

\path[fill=fillColor,fill opacity=0.20] (175.68, 62.22) circle (  2.13);

\path[fill=fillColor,fill opacity=0.20] (183.99, 53.49) circle (  2.13);

\path[fill=fillColor,fill opacity=0.20] (166.94, 57.66) circle (  2.13);

\path[fill=fillColor,fill opacity=0.20] (175.46, 74.99) circle (  2.13);

\path[fill=fillColor,fill opacity=0.20] (187.48, 80.17) circle (  2.13);

\path[fill=fillColor,fill opacity=0.20] (201.90, 82.95) circle (  2.13);

\path[fill=fillColor,fill opacity=0.20] (206.71, 86.12) circle (  2.13);

\path[fill=fillColor,fill opacity=0.20] (204.09, 86.37) circle (  2.13);

\path[fill=fillColor,fill opacity=0.20] (204.52, 82.07) circle (  2.13);

\path[fill=fillColor,fill opacity=0.20] (198.41, 67.28) circle (  2.13);

\path[fill=fillColor,fill opacity=0.20] (216.76, 65.63) circle (  2.13);

\path[fill=fillColor,fill opacity=0.20] (221.57, 75.24) circle (  2.13);

\path[fill=fillColor,fill opacity=0.20] (225.72, 81.18) circle (  2.13);

\path[fill=fillColor,fill opacity=0.20] (219.60, 93.58) circle (  2.13);

\path[fill=fillColor,fill opacity=0.20] (220.48, 98.13) circle (  2.13);

\path[fill=fillColor,fill opacity=0.20] (227.03, 93.07) circle (  2.13);

\path[fill=fillColor,fill opacity=0.20] (219.38, 94.71) circle (  2.13);

\path[fill=fillColor,fill opacity=0.20] (232.06, 94.97) circle (  2.13);

\path[fill=fillColor,fill opacity=0.20] (219.38, 94.46) circle (  2.13);

\path[fill=fillColor,fill opacity=0.20] (208.46, 89.78) circle (  2.13);

\path[fill=fillColor,fill opacity=0.20] (198.41, 81.69) circle (  2.13);

\path[fill=fillColor,fill opacity=0.20] (192.29, 83.71) circle (  2.13);

\path[fill=fillColor,fill opacity=0.20] (178.52, 82.32) circle (  2.13);

\path[fill=fillColor,fill opacity=0.20] (174.15, 63.23) circle (  2.13);

\path[fill=fillColor,fill opacity=0.20] (185.30, 47.68) circle (  2.13);

\path[fill=fillColor,fill opacity=0.20] (169.56, 73.47) circle (  2.13);

\path[fill=fillColor,fill opacity=0.20] (183.99, 86.12) circle (  2.13);

\path[fill=fillColor,fill opacity=0.20] (197.75, 90.29) circle (  2.13);

\path[fill=fillColor,fill opacity=0.20] (207.15, 89.28) circle (  2.13);

\path[fill=fillColor,fill opacity=0.20] (205.18, 89.91) circle (  2.13);

\path[fill=fillColor,fill opacity=0.20] (203.87, 85.10) circle (  2.13);

\path[fill=fillColor,fill opacity=0.20] (197.97, 72.33) circle (  2.13);

\path[fill=fillColor,fill opacity=0.20] (225.72, 78.28) circle (  2.13);

\path[fill=fillColor,fill opacity=0.20] (228.56, 87.00) circle (  2.13);

\path[fill=fillColor,fill opacity=0.20] (234.02, 89.66) circle (  2.13);

\path[fill=fillColor,fill opacity=0.20] (222.88, 90.79) circle (  2.13);

\path[fill=fillColor,fill opacity=0.20] (223.97, 94.71) circle (  2.13);

\path[fill=fillColor,fill opacity=0.20] (216.76, 94.46) circle (  2.13);

\path[fill=fillColor,fill opacity=0.20] (221.35, 92.57) circle (  2.13);

\path[fill=fillColor,fill opacity=0.20] (219.38, 90.04) circle (  2.13);

\path[fill=fillColor,fill opacity=0.20] (209.77, 86.12) circle (  2.13);

\path[fill=fillColor,fill opacity=0.20] (202.12, 87.76) circle (  2.13);

\path[fill=fillColor,fill opacity=0.20] (187.48, 82.45) circle (  2.13);

\path[fill=fillColor,fill opacity=0.20] (183.11, 71.70) circle (  2.13);

\path[fill=fillColor,fill opacity=0.20] (182.67, 66.64) circle (  2.13);

\path[fill=fillColor,fill opacity=0.20] (179.83, 60.19) circle (  2.13);

\path[fill=fillColor,fill opacity=0.20] (188.79, 44.77) circle (  2.13);

\path[fill=fillColor,fill opacity=0.20] (166.94, 78.66) circle (  2.13);

\path[fill=fillColor,fill opacity=0.20] (185.30, 87.51) circle (  2.13);

\path[fill=fillColor,fill opacity=0.20] (197.31, 84.98) circle (  2.13);

\path[fill=fillColor,fill opacity=0.20] (205.40, 80.81) circle (  2.13);

\path[fill=fillColor,fill opacity=0.20] (206.27, 81.31) circle (  2.13);

\path[fill=fillColor,fill opacity=0.20] (200.37, 82.70) circle (  2.13);

\path[fill=fillColor,fill opacity=0.20] (200.59, 79.41) circle (  2.13);

\path[fill=fillColor,fill opacity=0.20] (198.41, 66.64) circle (  2.13);

\path[fill=fillColor,fill opacity=0.20] (223.53, 75.87) circle (  2.13);

\path[fill=fillColor,fill opacity=0.20] (225.50, 90.04) circle (  2.13);

\path[fill=fillColor,fill opacity=0.20] (230.74, 94.59) circle (  2.13);

\path[fill=fillColor,fill opacity=0.20] (247.35, 93.32) circle (  2.13);

\path[fill=fillColor,fill opacity=0.20] (219.38, 94.46) circle (  2.13);

\path[fill=fillColor,fill opacity=0.20] (219.60, 94.84) circle (  2.13);

\path[fill=fillColor,fill opacity=0.20] (221.13, 92.69) circle (  2.13);

\path[fill=fillColor,fill opacity=0.20] (214.36, 90.92) circle (  2.13);

\path[fill=fillColor,fill opacity=0.20] (206.49, 84.60) circle (  2.13);

\path[fill=fillColor,fill opacity=0.20] (196.66, 78.02) circle (  2.13);

\path[fill=fillColor,fill opacity=0.20] (182.24, 75.37) circle (  2.13);

\path[fill=fillColor,fill opacity=0.20] (182.02, 71.45) circle (  2.13);

\path[fill=fillColor,fill opacity=0.20] (174.81, 60.83) circle (  2.13);

\path[fill=fillColor,fill opacity=0.20] (182.67, 51.97) circle (  2.13);

\path[fill=fillColor,fill opacity=0.20] (188.14, 47.93) circle (  2.13);

\path[fill=fillColor,fill opacity=0.20] (180.27, 74.74) circle (  2.13);

\path[fill=fillColor,fill opacity=0.20] (189.89, 78.28) circle (  2.13);

\path[fill=fillColor,fill opacity=0.20] (198.41, 82.83) circle (  2.13);

\path[fill=fillColor,fill opacity=0.20] (204.96, 84.98) circle (  2.13);

\path[fill=fillColor,fill opacity=0.20] (201.25, 87.00) circle (  2.13);

\path[fill=fillColor,fill opacity=0.20] (200.81, 86.62) circle (  2.13);

\path[fill=fillColor,fill opacity=0.20] (201.03, 77.14) circle (  2.13);

\path[fill=fillColor,fill opacity=0.20] (196.88, 64.49) circle (  2.13);

\path[fill=fillColor,fill opacity=0.20] (211.30, 61.58) circle (  2.13);

\path[fill=fillColor,fill opacity=0.20] (222.44, 75.37) circle (  2.13);

\path[fill=fillColor,fill opacity=0.20] (230.09, 84.09) circle (  2.13);

\path[fill=fillColor,fill opacity=0.20] (225.28, 88.52) circle (  2.13);

\path[fill=fillColor,fill opacity=0.20] (220.26, 94.08) circle (  2.13);

\path[fill=fillColor,fill opacity=0.20] (219.82, 96.49) circle (  2.13);

\path[fill=fillColor,fill opacity=0.20] (217.63, 93.83) circle (  2.13);

\path[fill=fillColor,fill opacity=0.20] (208.89, 94.08) circle (  2.13);

\path[fill=fillColor,fill opacity=0.20] (200.37, 96.99) circle (  2.13);

\path[fill=fillColor,fill opacity=0.20] (191.41, 92.82) circle (  2.13);

\path[fill=fillColor,fill opacity=0.20] (190.32, 83.08) circle (  2.13);

\path[fill=fillColor,fill opacity=0.20] (183.99, 72.71) circle (  2.13);

\path[fill=fillColor,fill opacity=0.20] (191.41, 65.13) circle (  2.13);

\path[fill=fillColor,fill opacity=0.20] (184.64, 59.69) circle (  2.13);

\path[fill=fillColor,fill opacity=0.20] (179.62, 51.60) circle (  2.13);

\path[fill=fillColor,fill opacity=0.20] (168.04, 45.02) circle (  2.13);

\path[fill=fillColor,fill opacity=0.20] (183.77, 78.78) circle (  2.13);

\path[fill=fillColor,fill opacity=0.20] (183.55, 89.91) circle (  2.13);

\path[fill=fillColor,fill opacity=0.20] (198.41, 89.66) circle (  2.13);

\path[fill=fillColor,fill opacity=0.20] (201.25, 83.97) circle (  2.13);

\path[fill=fillColor,fill opacity=0.20] (203.87, 82.32) circle (  2.13);

\path[fill=fillColor,fill opacity=0.20] (205.84, 82.07) circle (  2.13);

\path[fill=fillColor,fill opacity=0.20] (200.81, 79.54) circle (  2.13);

\path[fill=fillColor,fill opacity=0.20] (197.10, 74.86) circle (  2.13);

\path[fill=fillColor,fill opacity=0.20] (199.06, 67.53) circle (  2.13);

\path[fill=fillColor,fill opacity=0.20] (233.80, 66.52) circle (  2.13);

\path[fill=fillColor,fill opacity=0.20] (217.42, 79.54) circle (  2.13);

\path[fill=fillColor,fill opacity=0.20] (224.41, 89.15) circle (  2.13);

\path[fill=fillColor,fill opacity=0.20] (227.25, 89.66) circle (  2.13);

\path[fill=fillColor,fill opacity=0.20] (212.61, 86.62) circle (  2.13);

\path[fill=fillColor,fill opacity=0.20] (205.84, 86.75) circle (  2.13);

\path[fill=fillColor,fill opacity=0.20] (200.59, 90.29) circle (  2.13);

\path[fill=fillColor,fill opacity=0.20] (208.46, 89.91) circle (  2.13);

\path[fill=fillColor,fill opacity=0.20] (189.23, 89.40) circle (  2.13);

\path[fill=fillColor,fill opacity=0.20] (185.73, 87.25) circle (  2.13);

\path[fill=fillColor,fill opacity=0.20] (182.02, 76.89) circle (  2.13);

\path[fill=fillColor,fill opacity=0.20] (165.19, 62.22) circle (  2.13);

\path[fill=fillColor,fill opacity=0.20] (178.96, 53.62) circle (  2.13);

\path[fill=fillColor,fill opacity=0.20] (190.98, 47.30) circle (  2.13);

\path[fill=fillColor,fill opacity=0.20] (173.72, 43.25) circle (  2.13);

\path[fill=fillColor,fill opacity=0.20] (170.00, 61.21) circle (  2.13);

\path[fill=fillColor,fill opacity=0.20] (175.68, 77.64) circle (  2.13);

\path[fill=fillColor,fill opacity=0.20] (180.05, 86.24) circle (  2.13);

\path[fill=fillColor,fill opacity=0.20] (193.60, 81.94) circle (  2.13);

\path[fill=fillColor,fill opacity=0.20] (200.59, 78.66) circle (  2.13);

\path[fill=fillColor,fill opacity=0.20] (198.63, 85.74) circle (  2.13);

\path[fill=fillColor,fill opacity=0.20] (202.34, 86.24) circle (  2.13);

\path[fill=fillColor,fill opacity=0.20] (198.63, 79.67) circle (  2.13);

\path[fill=fillColor,fill opacity=0.20] (198.84, 76.00) circle (  2.13);

\path[fill=fillColor,fill opacity=0.20] (200.37, 73.22) circle (  2.13);

\path[fill=fillColor,fill opacity=0.20] (200.59, 66.39) circle (  2.13);

\path[fill=fillColor,fill opacity=0.20] (212.39, 70.44) circle (  2.13);

\path[fill=fillColor,fill opacity=0.20] (209.77, 78.28) circle (  2.13);

\path[fill=fillColor,fill opacity=0.20] (209.11, 78.78) circle (  2.13);

\path[fill=fillColor,fill opacity=0.20] (205.84, 80.68) circle (  2.13);

\path[fill=fillColor,fill opacity=0.20] (201.90, 85.10) circle (  2.13);

\path[fill=fillColor,fill opacity=0.20] (198.84, 86.12) circle (  2.13);

\path[fill=fillColor,fill opacity=0.20] (186.17, 82.07) circle (  2.13);

\path[fill=fillColor,fill opacity=0.20] (192.07, 78.78) circle (  2.13);

\path[fill=fillColor,fill opacity=0.20] (183.55, 78.53) circle (  2.13);

\path[fill=fillColor,fill opacity=0.20] (180.27, 74.99) circle (  2.13);

\path[fill=fillColor,fill opacity=0.20] (185.73, 66.90) circle (  2.13);

\path[fill=fillColor,fill opacity=0.20] (186.39, 59.18) circle (  2.13);

\path[fill=fillColor,fill opacity=0.20] (201.03, 50.84) circle (  2.13);

\path[fill=fillColor,fill opacity=0.20] (171.31, 51.34) circle (  2.13);

\path[fill=fillColor,fill opacity=0.20] (176.34, 78.02) circle (  2.13);

\path[fill=fillColor,fill opacity=0.20] (184.86, 83.46) circle (  2.13);

\path[fill=fillColor,fill opacity=0.20] (194.04, 93.32) circle (  2.13);

\path[fill=fillColor,fill opacity=0.20] (197.75, 91.93) circle (  2.13);

\path[fill=fillColor,fill opacity=0.20] (195.78, 82.07) circle (  2.13);

\path[fill=fillColor,fill opacity=0.20] (201.03, 78.53) circle (  2.13);

\path[fill=fillColor,fill opacity=0.20] (203.87, 79.29) circle (  2.13);

\path[fill=fillColor,fill opacity=0.20] (199.06, 77.77) circle (  2.13);

\path[fill=fillColor,fill opacity=0.20] (193.82, 75.75) circle (  2.13);

\path[fill=fillColor,fill opacity=0.20] (196.44, 69.55) circle (  2.13);

\path[fill=fillColor,fill opacity=0.20] (215.89, 71.20) circle (  2.13);

\path[fill=fillColor,fill opacity=0.20] (209.33, 76.38) circle (  2.13);

\path[fill=fillColor,fill opacity=0.20] (202.12, 82.70) circle (  2.13);

\path[fill=fillColor,fill opacity=0.20] (192.51, 85.99) circle (  2.13);

\path[fill=fillColor,fill opacity=0.20] (185.95, 86.87) circle (  2.13);

\path[fill=fillColor,fill opacity=0.20] (185.30, 81.56) circle (  2.13);

\path[fill=fillColor,fill opacity=0.20] (179.18, 75.24) circle (  2.13);

\path[fill=fillColor,fill opacity=0.20] (179.40, 74.48) circle (  2.13);

\path[fill=fillColor,fill opacity=0.20] (180.27, 73.72) circle (  2.13);

\path[fill=fillColor,fill opacity=0.20] (185.08, 67.91) circle (  2.13);

\path[fill=fillColor,fill opacity=0.20] (183.11, 64.11) circle (  2.13);

\path[fill=fillColor,fill opacity=0.20] (190.32, 61.58) circle (  2.13);

\path[fill=fillColor,fill opacity=0.20] (182.24, 53.24) circle (  2.13);

\path[fill=fillColor,fill opacity=0.20] (172.19, 45.91) circle (  2.13);

\path[fill=fillColor,fill opacity=0.20] (174.59, 57.66) circle (  2.13);

\path[fill=fillColor,fill opacity=0.20] (175.46, 69.68) circle (  2.13);

\path[fill=fillColor,fill opacity=0.20] (192.07, 80.05) circle (  2.13);

\path[fill=fillColor,fill opacity=0.20] (182.67, 82.70) circle (  2.13);

\path[fill=fillColor,fill opacity=0.20] (187.70, 81.31) circle (  2.13);

\path[fill=fillColor,fill opacity=0.20] (196.22, 83.97) circle (  2.13);

\path[fill=fillColor,fill opacity=0.20] (200.37, 83.97) circle (  2.13);

\path[fill=fillColor,fill opacity=0.20] (199.28, 79.67) circle (  2.13);

\path[fill=fillColor,fill opacity=0.20] (194.91, 80.55) circle (  2.13);

\path[fill=fillColor,fill opacity=0.20] (193.60, 84.35) circle (  2.13);

\path[fill=fillColor,fill opacity=0.20] (195.35, 79.79) circle (  2.13);

\path[fill=fillColor,fill opacity=0.20] (192.94, 73.60) circle (  2.13);

\path[fill=fillColor,fill opacity=0.20] (190.54, 72.46) circle (  2.13);

\path[fill=fillColor,fill opacity=0.20] (196.88, 70.06) circle (  2.13);

\path[fill=fillColor,fill opacity=0.20] (204.09, 66.26) circle (  2.13);

\path[fill=fillColor,fill opacity=0.20] (190.54, 66.64) circle (  2.13);

\path[fill=fillColor,fill opacity=0.20] (199.06, 68.16) circle (  2.13);

\path[fill=fillColor,fill opacity=0.20] (198.41, 67.40) circle (  2.13);

\path[fill=fillColor,fill opacity=0.20] (199.50, 65.76) circle (  2.13);

\path[fill=fillColor,fill opacity=0.20] (198.41, 64.75) circle (  2.13);

\path[fill=fillColor,fill opacity=0.20] (195.78, 65.63) circle (  2.13);

\path[fill=fillColor,fill opacity=0.20] (206.93, 71.45) circle (  2.13);

\path[fill=fillColor,fill opacity=0.20] (209.33, 74.10) circle (  2.13);

\path[fill=fillColor,fill opacity=0.20] (215.67, 76.00) circle (  2.13);

\path[fill=fillColor,fill opacity=0.20] (212.61, 77.39) circle (  2.13);

\path[fill=fillColor,fill opacity=0.20] (209.55, 81.56) circle (  2.13);

\path[fill=fillColor,fill opacity=0.20] (209.55, 87.51) circle (  2.13);

\path[fill=fillColor,fill opacity=0.20] (201.90, 85.86) circle (  2.13);

\path[fill=fillColor,fill opacity=0.20] (199.50, 80.17) circle (  2.13);

\path[fill=fillColor,fill opacity=0.20] (185.08, 79.92) circle (  2.13);

\path[fill=fillColor,fill opacity=0.20] (182.02, 80.17) circle (  2.13);

\path[fill=fillColor,fill opacity=0.20] (175.03, 74.10) circle (  2.13);

\path[fill=fillColor,fill opacity=0.20] (173.28, 68.67) circle (  2.13);

\path[fill=fillColor,fill opacity=0.20] (182.89, 62.98) circle (  2.13);

\path[fill=fillColor,fill opacity=0.20] (168.91, 58.68) circle (  2.13);

\path[fill=fillColor,fill opacity=0.20] (179.83, 53.87) circle (  2.13);

\path[fill=fillColor,fill opacity=0.20] (188.57, 49.70) circle (  2.13);

\path[fill=fillColor,fill opacity=0.20] (190.10, 47.04) circle (  2.13);

\path[fill=fillColor,fill opacity=0.20] (181.58, 44.01) circle (  2.13);

\path[fill=fillColor,fill opacity=0.20] (177.87, 52.86) circle (  2.13);

\path[fill=fillColor,fill opacity=0.20] (169.13, 61.33) circle (  2.13);

\path[fill=fillColor,fill opacity=0.20] (175.25, 67.65) circle (  2.13);

\path[fill=fillColor,fill opacity=0.20] (185.52, 75.12) circle (  2.13);

\path[fill=fillColor,fill opacity=0.20] (188.79, 80.68) circle (  2.13);

\path[fill=fillColor,fill opacity=0.20] (189.01, 80.93) circle (  2.13);

\path[fill=fillColor,fill opacity=0.20] (188.14, 85.36) circle (  2.13);

\path[fill=fillColor,fill opacity=0.20] (190.54, 89.78) circle (  2.13);

\path[fill=fillColor,fill opacity=0.20] (199.94, 85.48) circle (  2.13);

\path[fill=fillColor,fill opacity=0.20] (196.44, 83.08) circle (  2.13);

\path[fill=fillColor,fill opacity=0.20] (196.66, 83.08) circle (  2.13);

\path[fill=fillColor,fill opacity=0.20] (198.41, 82.70) circle (  2.13);

\path[fill=fillColor,fill opacity=0.20] (197.97, 80.30) circle (  2.13);

\path[fill=fillColor,fill opacity=0.20] (192.07, 79.03) circle (  2.13);

\path[fill=fillColor,fill opacity=0.20] (196.00, 79.67) circle (  2.13);

\path[fill=fillColor,fill opacity=0.20] (201.47, 79.54) circle (  2.13);

\path[fill=fillColor,fill opacity=0.20] (202.34, 79.92) circle (  2.13);

\path[fill=fillColor,fill opacity=0.20] (208.24, 80.55) circle (  2.13);

\path[fill=fillColor,fill opacity=0.20] (202.12, 81.82) circle (  2.13);

\path[fill=fillColor,fill opacity=0.20] (202.34, 83.21) circle (  2.13);

\path[fill=fillColor,fill opacity=0.20] (199.94, 83.71) circle (  2.13);

\path[fill=fillColor,fill opacity=0.20] (206.27, 80.43) circle (  2.13);

\path[fill=fillColor,fill opacity=0.20] (206.27, 78.28) circle (  2.13);

\path[fill=fillColor,fill opacity=0.20] (206.93, 77.64) circle (  2.13);

\path[fill=fillColor,fill opacity=0.20] (205.18, 81.06) circle (  2.13);

\path[fill=fillColor,fill opacity=0.20] (201.68, 87.00) circle (  2.13);

\path[fill=fillColor,fill opacity=0.20] (192.29, 89.28) circle (  2.13);

\path[fill=fillColor,fill opacity=0.20] (182.02, 88.27) circle (  2.13);

\path[fill=fillColor,fill opacity=0.20] (175.68, 80.81) circle (  2.13);

\path[fill=fillColor,fill opacity=0.20] (168.04, 71.20) circle (  2.13);

\path[fill=fillColor,fill opacity=0.20] (175.46, 66.14) circle (  2.13);

\path[fill=fillColor,fill opacity=0.20] (178.74, 60.45) circle (  2.13);

\path[fill=fillColor,fill opacity=0.20] (172.84, 50.84) circle (  2.13);

\path[fill=fillColor,fill opacity=0.20] (179.40, 43.00) circle (  2.13);

\path[fill=fillColor,fill opacity=0.20] (182.46, 43.25) circle (  2.13);

\path[fill=fillColor,fill opacity=0.20] (180.05, 48.05) circle (  2.13);

\path[fill=fillColor,fill opacity=0.20] (179.83, 55.64) circle (  2.13);

\path[fill=fillColor,fill opacity=0.20] (172.19, 66.01) circle (  2.13);

\path[fill=fillColor,fill opacity=0.20] (178.09, 77.26) circle (  2.13);

\path[fill=fillColor,fill opacity=0.20] (175.46, 86.62) circle (  2.13);

\path[fill=fillColor,fill opacity=0.20] (182.02, 88.14) circle (  2.13);

\path[fill=fillColor,fill opacity=0.20] (189.01, 84.09) circle (  2.13);

\path[fill=fillColor,fill opacity=0.20] (192.94, 84.22) circle (  2.13);

\path[fill=fillColor,fill opacity=0.20] (194.91, 87.63) circle (  2.13);

\path[fill=fillColor,fill opacity=0.20] (199.94, 87.76) circle (  2.13);

\path[fill=fillColor,fill opacity=0.20] (197.31, 88.14) circle (  2.13);

\path[fill=fillColor,fill opacity=0.20] (200.15, 87.13) circle (  2.13);

\path[fill=fillColor,fill opacity=0.20] (204.74, 83.46) circle (  2.13);

\path[fill=fillColor,fill opacity=0.20] (208.02, 80.17) circle (  2.13);

\path[fill=fillColor,fill opacity=0.20] (203.43, 80.43) circle (  2.13);

\path[fill=fillColor,fill opacity=0.20] (203.65, 83.71) circle (  2.13);

\path[fill=fillColor,fill opacity=0.20] (196.66, 84.35) circle (  2.13);

\path[fill=fillColor,fill opacity=0.20] (188.14, 84.73) circle (  2.13);

\path[fill=fillColor,fill opacity=0.20] (190.54, 83.59) circle (  2.13);

\path[fill=fillColor,fill opacity=0.20] (189.45, 78.53) circle (  2.13);

\path[fill=fillColor,fill opacity=0.20] (183.55, 74.48) circle (  2.13);

\path[fill=fillColor,fill opacity=0.20] (185.30, 73.85) circle (  2.13);

\path[fill=fillColor,fill opacity=0.20] (183.99, 76.25) circle (  2.13);

\path[fill=fillColor,fill opacity=0.20] (178.96, 81.18) circle (  2.13);

\path[fill=fillColor,fill opacity=0.20] (174.59, 80.43) circle (  2.13);

\path[fill=fillColor,fill opacity=0.20] (173.50, 70.44) circle (  2.13);

\path[fill=fillColor,fill opacity=0.20] (183.55, 53.11) circle (  2.13);

\path[fill=fillColor,fill opacity=0.20] (182.46, 47.04) circle (  2.13);

\path[fill=fillColor,fill opacity=0.20] (182.24, 56.27) circle (  2.13);

\path[fill=fillColor,fill opacity=0.20] (174.15, 61.46) circle (  2.13);

\path[fill=fillColor,fill opacity=0.20] (167.82, 65.25) circle (  2.13);

\path[fill=fillColor,fill opacity=0.20] (180.27, 71.07) circle (  2.13);

\path[fill=fillColor,fill opacity=0.20] (179.40, 79.03) circle (  2.13);

\path[fill=fillColor,fill opacity=0.20] (184.20, 81.18) circle (  2.13);

\path[fill=fillColor,fill opacity=0.20] (188.14, 81.18) circle (  2.13);

\path[fill=fillColor,fill opacity=0.20] (193.38, 84.85) circle (  2.13);

\path[fill=fillColor,fill opacity=0.20] (195.35, 91.43) circle (  2.13);

\path[fill=fillColor,fill opacity=0.20] (199.50, 88.27) circle (  2.13);

\path[fill=fillColor,fill opacity=0.20] (196.88, 80.17) circle (  2.13);

\path[fill=fillColor,fill opacity=0.20] (188.57, 78.91) circle (  2.13);

\path[fill=fillColor,fill opacity=0.20] (183.55, 80.93) circle (  2.13);

\path[fill=fillColor,fill opacity=0.20] (178.09, 80.30) circle (  2.13);

\path[fill=fillColor,fill opacity=0.20] (174.15, 77.01) circle (  2.13);

\path[fill=fillColor,fill opacity=0.20] (176.78, 76.38) circle (  2.13);

\path[fill=fillColor,fill opacity=0.20] (171.09, 75.62) circle (  2.13);

\path[fill=fillColor,fill opacity=0.20] (170.66, 75.24) circle (  2.13);

\path[fill=fillColor,fill opacity=0.20] (181.36, 70.56) circle (  2.13);

\path[fill=fillColor,fill opacity=0.20] (185.73, 62.60) circle (  2.13);

\path[fill=fillColor,fill opacity=0.20] (185.08, 60.70) circle (  2.13);

\path[fill=fillColor,fill opacity=0.20] (186.61, 58.55) circle (  2.13);

\path[fill=fillColor,fill opacity=0.20] (179.40, 49.83) circle (  2.13);

\path[fill=fillColor,fill opacity=0.20] (181.80, 40.97) circle (  2.13);

\path[fill=fillColor,fill opacity=0.20] (176.56, 48.56) circle (  2.13);

\path[fill=fillColor,fill opacity=0.20] (177.21, 58.68) circle (  2.13);

\path[fill=fillColor,fill opacity=0.20] (171.97, 61.08) circle (  2.13);

\path[fill=fillColor,fill opacity=0.20] (180.49, 62.34) circle (  2.13);

\path[fill=fillColor,fill opacity=0.20] (193.38, 66.90) circle (  2.13);

\path[fill=fillColor,fill opacity=0.20] (181.36, 73.47) circle (  2.13);

\path[fill=fillColor,fill opacity=0.20] (178.96, 78.02) circle (  2.13);

\path[fill=fillColor,fill opacity=0.20] (180.05, 75.49) circle (  2.13);

\path[fill=fillColor,fill opacity=0.20] (170.66, 71.20) circle (  2.13);

\path[fill=fillColor,fill opacity=0.20] (177.43, 66.64) circle (  2.13);

\path[fill=fillColor,fill opacity=0.20] (169.56, 60.32) circle (  2.13);

\path[fill=fillColor,fill opacity=0.20] (181.36, 54.76) circle (  2.13);

\path[fill=fillColor,fill opacity=0.20] (181.80, 53.87) circle (  2.13);

\path[fill=fillColor,fill opacity=0.20] (182.24, 55.77) circle (  2.13);

\path[fill=fillColor,fill opacity=0.20] (190.76, 54.88) circle (  2.13);

\path[fill=fillColor,fill opacity=0.20] (210.64, 51.34) circle (  2.13);

\path[fill=fillColor,fill opacity=0.20] (192.07, 41.48) circle (  2.13);

\path[fill=fillColor,fill opacity=0.20] (186.83, 44.13) circle (  2.13);

\path[fill=fillColor,fill opacity=0.20] (185.30, 47.68) circle (  2.13);

\path[fill=fillColor,fill opacity=0.20] (187.26, 49.57) circle (  2.13);

\path[fill=fillColor,fill opacity=0.20] (189.67, 54.38) circle (  2.13);

\path[fill=fillColor,fill opacity=0.20] (176.56, 56.15) circle (  2.13);

\path[fill=fillColor,fill opacity=0.20] (178.74, 50.71) circle (  2.13);

\path[fill=fillColor,fill opacity=0.20] (182.89, 44.89) circle (  2.13);
\end{scope}
\begin{scope}
\path[clip] (  0.00,  0.00) rectangle (289.08,144.54);
\definecolor[named]{drawColor}{rgb}{0.50,0.50,0.50}

\node[text=drawColor,anchor=base east,inner sep=0pt, outer sep=0pt, scale=  0.96] at ( 32.58, 48.92) {0.4};

\node[text=drawColor,anchor=base east,inner sep=0pt, outer sep=0pt, scale=  0.96] at ( 32.58, 74.21) {0.6};

\node[text=drawColor,anchor=base east,inner sep=0pt, outer sep=0pt, scale=  0.96] at ( 32.58, 99.50) {0.8};
\end{scope}
\begin{scope}
\path[clip] (  0.00,  0.00) rectangle (289.08,144.54);
\definecolor[named]{drawColor}{rgb}{0.50,0.50,0.50}

\path[draw=drawColor,line width= 0.6pt,line join=round] ( 35.42, 52.23) --
	( 39.69, 52.23);

\path[draw=drawColor,line width= 0.6pt,line join=round] ( 35.42, 77.52) --
	( 39.69, 77.52);

\path[draw=drawColor,line width= 0.6pt,line join=round] ( 35.42,102.81) --
	( 39.69,102.81);
\end{scope}
\begin{scope}
\path[clip] (  0.00,  0.00) rectangle (289.08,144.54);
\definecolor[named]{drawColor}{rgb}{0.50,0.50,0.50}

\path[draw=drawColor,line width= 0.6pt,line join=round] ( 60.53, 29.77) --
	( 60.53, 34.04);

\path[draw=drawColor,line width= 0.6pt,line join=round] ( 82.38, 29.77) --
	( 82.38, 34.04);

\path[draw=drawColor,line width= 0.6pt,line join=round] (104.23, 29.77) --
	(104.23, 34.04);

\path[draw=drawColor,line width= 0.6pt,line join=round] (126.08, 29.77) --
	(126.08, 34.04);

\path[draw=drawColor,line width= 0.6pt,line join=round] (147.93, 29.77) --
	(147.93, 34.04);
\end{scope}
\begin{scope}
\path[clip] (  0.00,  0.00) rectangle (289.08,144.54);
\definecolor[named]{drawColor}{rgb}{0.50,0.50,0.50}

\node[text=drawColor,anchor=base,inner sep=0pt, outer sep=0pt, scale=  0.96] at ( 60.53, 20.31) {0.02};

\node[text=drawColor,anchor=base,inner sep=0pt, outer sep=0pt, scale=  0.96] at ( 82.38, 20.31) {0.03};

\node[text=drawColor,anchor=base,inner sep=0pt, outer sep=0pt, scale=  0.96] at (104.23, 20.31) {0.04};

\node[text=drawColor,anchor=base,inner sep=0pt, outer sep=0pt, scale=  0.96] at (126.08, 20.31) {0.05};

\node[text=drawColor,anchor=base,inner sep=0pt, outer sep=0pt, scale=  0.96] at (147.93, 20.31) {0.06};
\end{scope}
\begin{scope}
\path[clip] (  0.00,  0.00) rectangle (289.08,144.54);
\definecolor[named]{drawColor}{rgb}{0.50,0.50,0.50}

\path[draw=drawColor,line width= 0.6pt,line join=round] (180.71, 29.77) --
	(180.71, 34.04);

\path[draw=drawColor,line width= 0.6pt,line join=round] (202.56, 29.77) --
	(202.56, 34.04);

\path[draw=drawColor,line width= 0.6pt,line join=round] (224.41, 29.77) --
	(224.41, 34.04);

\path[draw=drawColor,line width= 0.6pt,line join=round] (246.26, 29.77) --
	(246.26, 34.04);

\path[draw=drawColor,line width= 0.6pt,line join=round] (268.11, 29.77) --
	(268.11, 34.04);
\end{scope}
\begin{scope}
\path[clip] (  0.00,  0.00) rectangle (289.08,144.54);
\definecolor[named]{drawColor}{rgb}{0.50,0.50,0.50}

\node[text=drawColor,anchor=base,inner sep=0pt, outer sep=0pt, scale=  0.96] at (180.71, 20.31) {0.02};

\node[text=drawColor,anchor=base,inner sep=0pt, outer sep=0pt, scale=  0.96] at (202.56, 20.31) {0.03};

\node[text=drawColor,anchor=base,inner sep=0pt, outer sep=0pt, scale=  0.96] at (224.41, 20.31) {0.04};

\node[text=drawColor,anchor=base,inner sep=0pt, outer sep=0pt, scale=  0.96] at (246.26, 20.31) {0.05};

\node[text=drawColor,anchor=base,inner sep=0pt, outer sep=0pt, scale=  0.96] at (268.11, 20.31) {0.06};
\end{scope}
\begin{scope}
\path[clip] (  0.00,  0.00) rectangle (289.08,144.54);
\definecolor[named]{drawColor}{rgb}{0.00,0.00,0.00}

\node[text=drawColor,anchor=base,inner sep=0pt, outer sep=0pt, scale=  1.20] at (158.36,  9.03) {$\rho$ $[\mu m^{-2}]$};
\end{scope}
\begin{scope}
\path[clip] (  0.00,  0.00) rectangle (289.08,144.54);
\definecolor[named]{drawColor}{rgb}{0.00,0.00,0.00}

\node[text=drawColor,rotate= 90.00,anchor=base,inner sep=0pt, outer sep=0pt, scale=  1.20] at ( 17.30, 76.95) {FA};
\end{scope}
\end{tikzpicture}

					\end{adjustbox}
					\end{minipage}
					}
					\subfloat[MD]{
						\begin{minipage}{0.5\textwidth}						
						\begin{adjustbox}{width={\textwidth},totalheight=\textheight,keepaspectratio}
							\strut
							% Created by tikzDevice version 0.6.2-92-0ad2792 on 2012-09-27 18:25:06
% !TEX encoding = UTF-8 Unicode
\begin{tikzpicture}[x=1pt,y=1pt]
\definecolor[named]{fillColor}{rgb}{1.00,1.00,1.00}
\path[use as bounding box,fill=fillColor,fill opacity=0.00] (0,0) rectangle (289.08,144.54);
\begin{scope}
\path[clip] (  0.00,  0.00) rectangle (289.08,144.54);
\definecolor[named]{drawColor}{rgb}{1.00,1.00,1.00}
\definecolor[named]{fillColor}{rgb}{1.00,1.00,1.00}

\path[draw=drawColor,line width= 0.6pt,line join=round,line cap=round,fill=fillColor] ( -0.00,  0.00) rectangle (289.08,144.54);
\end{scope}
\begin{scope}
\path[clip] ( 39.69,119.86) rectangle (156.86,132.50);
\definecolor[named]{fillColor}{rgb}{0.80,0.80,0.80}

\path[fill=fillColor] ( 39.69,119.86) rectangle (156.86,132.50);
\definecolor[named]{drawColor}{rgb}{0.00,0.00,0.00}

\node[text=drawColor,anchor=base,inner sep=0pt, outer sep=0pt, scale=  0.96] at ( 98.27,122.87) {Scan (r=0.055)};
\end{scope}
\begin{scope}
\path[clip] (159.87,119.86) rectangle (277.03,132.50);
\definecolor[named]{fillColor}{rgb}{0.80,0.80,0.80}

\path[fill=fillColor] (159.87,119.86) rectangle (277.03,132.50);
\definecolor[named]{drawColor}{rgb}{0.00,0.00,0.00}

\node[text=drawColor,anchor=base,inner sep=0pt, outer sep=0pt, scale=  0.96] at (218.45,122.87) {Rescan (r=0.088)};
\end{scope}
\begin{scope}
\path[clip] ( 39.69, 34.04) rectangle (156.86,119.86);
\definecolor[named]{fillColor}{rgb}{0.90,0.90,0.90}

\path[fill=fillColor] ( 39.69, 34.04) rectangle (156.86,119.86);
\definecolor[named]{drawColor}{rgb}{0.95,0.95,0.95}

\path[draw=drawColor,line width= 0.3pt,line join=round] ( 39.69, 50.55) --
	(156.86, 50.55);

\path[draw=drawColor,line width= 0.3pt,line join=round] ( 39.69, 71.32) --
	(156.86, 71.32);

\path[draw=drawColor,line width= 0.3pt,line join=round] ( 39.69, 92.08) --
	(156.86, 92.08);

\path[draw=drawColor,line width= 0.3pt,line join=round] ( 39.69,112.84) --
	(156.86,112.84);

\path[draw=drawColor,line width= 0.3pt,line join=round] ( 55.72, 34.04) --
	( 55.72,119.86);

\path[draw=drawColor,line width= 0.3pt,line join=round] ( 80.80, 34.04) --
	( 80.80,119.86);

\path[draw=drawColor,line width= 0.3pt,line join=round] (105.88, 34.04) --
	(105.88,119.86);

\path[draw=drawColor,line width= 0.3pt,line join=round] (130.97, 34.04) --
	(130.97,119.86);

\path[draw=drawColor,line width= 0.3pt,line join=round] (156.05, 34.04) --
	(156.05,119.86);
\definecolor[named]{drawColor}{rgb}{1.00,1.00,1.00}

\path[draw=drawColor,line width= 0.6pt,line join=round] ( 39.69, 40.17) --
	(156.86, 40.17);

\path[draw=drawColor,line width= 0.6pt,line join=round] ( 39.69, 60.93) --
	(156.86, 60.93);

\path[draw=drawColor,line width= 0.6pt,line join=round] ( 39.69, 81.70) --
	(156.86, 81.70);

\path[draw=drawColor,line width= 0.6pt,line join=round] ( 39.69,102.46) --
	(156.86,102.46);

\path[draw=drawColor,line width= 0.6pt,line join=round] ( 43.18, 34.04) --
	( 43.18,119.86);

\path[draw=drawColor,line width= 0.6pt,line join=round] ( 68.26, 34.04) --
	( 68.26,119.86);

\path[draw=drawColor,line width= 0.6pt,line join=round] ( 93.34, 34.04) --
	( 93.34,119.86);

\path[draw=drawColor,line width= 0.6pt,line join=round] (118.43, 34.04) --
	(118.43,119.86);

\path[draw=drawColor,line width= 0.6pt,line join=round] (143.51, 34.04) --
	(143.51,119.86);
\definecolor[named]{fillColor}{rgb}{0.00,0.00,0.00}

\path[fill=fillColor,fill opacity=0.20] (141.50, 72.46) circle (  2.13);

\path[fill=fillColor,fill opacity=0.20] ( 99.36, 80.76) circle (  2.13);

\path[fill=fillColor,fill opacity=0.20] ( 94.35, 68.82) circle (  2.13);

\path[fill=fillColor,fill opacity=0.20] ( 89.33, 56.47) circle (  2.13);

\path[fill=fillColor,fill opacity=0.20] ( 91.34, 53.67) circle (  2.13);

\path[fill=fillColor,fill opacity=0.20] (102.37, 53.46) circle (  2.13);

\path[fill=fillColor,fill opacity=0.20] (109.40, 69.65) circle (  2.13);

\path[fill=fillColor,fill opacity=0.20] (107.39, 93.12) circle (  2.13);

\path[fill=fillColor,fill opacity=0.20] ( 86.32, 83.77) circle (  2.13);

\path[fill=fillColor,fill opacity=0.20] ( 88.33, 58.65) circle (  2.13);

\path[fill=fillColor,fill opacity=0.20] ( 83.31, 51.49) circle (  2.13);

\path[fill=fillColor,fill opacity=0.20] ( 77.29, 57.71) circle (  2.13);

\path[fill=fillColor,fill opacity=0.20] ( 83.31, 57.92) circle (  2.13);

\path[fill=fillColor,fill opacity=0.20] ( 74.28, 50.45) circle (  2.13);

\path[fill=fillColor,fill opacity=0.20] ( 82.31, 44.94) circle (  2.13);

\path[fill=fillColor,fill opacity=0.20] ( 87.33, 56.68) circle (  2.13);

\path[fill=fillColor,fill opacity=0.20] (105.38, 59.79) circle (  2.13);

\path[fill=fillColor,fill opacity=0.20] (127.45, 57.09) circle (  2.13);

\path[fill=fillColor,fill opacity=0.20] ( 81.31,101.42) circle (  2.13);

\path[fill=fillColor,fill opacity=0.20] ( 82.31, 73.18) circle (  2.13);

\path[fill=fillColor,fill opacity=0.20] ( 83.31, 60.10) circle (  2.13);

\path[fill=fillColor,fill opacity=0.20] ( 82.31, 60.52) circle (  2.13);

\path[fill=fillColor,fill opacity=0.20] ( 83.31, 59.48) circle (  2.13);

\path[fill=fillColor,fill opacity=0.20] ( 80.30, 59.27) circle (  2.13);

\path[fill=fillColor,fill opacity=0.20] ( 81.31, 64.46) circle (  2.13);

\path[fill=fillColor,fill opacity=0.20] ( 83.31, 65.50) circle (  2.13);

\path[fill=fillColor,fill opacity=0.20] ( 84.32, 55.33) circle (  2.13);

\path[fill=fillColor,fill opacity=0.20] ( 86.32, 42.35) circle (  2.13);

\path[fill=fillColor,fill opacity=0.20] (100.37, 47.12) circle (  2.13);

\path[fill=fillColor,fill opacity=0.20] ( 98.36, 56.26) circle (  2.13);

\path[fill=fillColor,fill opacity=0.20] (105.38, 59.79) circle (  2.13);

\path[fill=fillColor,fill opacity=0.20] ( 99.36, 85.85) circle (  2.13);

\path[fill=fillColor,fill opacity=0.20] ( 81.31, 76.30) circle (  2.13);

\path[fill=fillColor,fill opacity=0.20] ( 81.31, 64.05) circle (  2.13);

\path[fill=fillColor,fill opacity=0.20] ( 73.28, 55.85) circle (  2.13);

\path[fill=fillColor,fill opacity=0.20] ( 75.29, 72.66) circle (  2.13);

\path[fill=fillColor,fill opacity=0.20] ( 80.30, 72.04) circle (  2.13);

\path[fill=fillColor,fill opacity=0.20] ( 86.32, 59.48) circle (  2.13);

\path[fill=fillColor,fill opacity=0.20] ( 83.31, 63.53) circle (  2.13);

\path[fill=fillColor,fill opacity=0.20] ( 81.31, 67.37) circle (  2.13);

\path[fill=fillColor,fill opacity=0.20] ( 86.32, 60.00) circle (  2.13);

\path[fill=fillColor,fill opacity=0.20] ( 93.34, 54.50) circle (  2.13);

\path[fill=fillColor,fill opacity=0.20] ( 97.36, 58.44) circle (  2.13);

\path[fill=fillColor,fill opacity=0.20] ( 94.35, 59.58) circle (  2.13);

\path[fill=fillColor,fill opacity=0.20] ( 99.36, 62.08) circle (  2.13);

\path[fill=fillColor,fill opacity=0.20] (101.37, 80.35) circle (  2.13);

\path[fill=fillColor,fill opacity=0.20] ( 94.35, 63.84) circle (  2.13);

\path[fill=fillColor,fill opacity=0.20] ( 81.31, 64.67) circle (  2.13);

\path[fill=fillColor,fill opacity=0.20] ( 76.29, 58.23) circle (  2.13);

\path[fill=fillColor,fill opacity=0.20] ( 77.29, 61.76) circle (  2.13);

\path[fill=fillColor,fill opacity=0.20] ( 77.29, 67.89) circle (  2.13);

\path[fill=fillColor,fill opacity=0.20] ( 81.31, 70.69) circle (  2.13);

\path[fill=fillColor,fill opacity=0.20] ( 85.32, 67.89) circle (  2.13);

\path[fill=fillColor,fill opacity=0.20] ( 85.32, 59.17) circle (  2.13);

\path[fill=fillColor,fill opacity=0.20] ( 92.34, 56.26) circle (  2.13);

\path[fill=fillColor,fill opacity=0.20] (100.37, 63.42) circle (  2.13);

\path[fill=fillColor,fill opacity=0.20] ( 95.35, 71.94) circle (  2.13);

\path[fill=fillColor,fill opacity=0.20] ( 98.36, 81.70) circle (  2.13);

\path[fill=fillColor,fill opacity=0.20] (102.37,106.61) circle (  2.13);

\path[fill=fillColor,fill opacity=0.20] ( 96.35, 75.68) circle (  2.13);

\path[fill=fillColor,fill opacity=0.20] ( 85.32, 45.05) circle (  2.13);

\path[fill=fillColor,fill opacity=0.20] ( 84.32, 76.71) circle (  2.13);

\path[fill=fillColor,fill opacity=0.20] ( 92.34, 76.71) circle (  2.13);

\path[fill=fillColor,fill opacity=0.20] ( 92.34, 64.88) circle (  2.13);

\path[fill=fillColor,fill opacity=0.20] (102.37, 85.85) circle (  2.13);

\path[fill=fillColor,fill opacity=0.20] ( 96.35, 58.34) circle (  2.13);

\path[fill=fillColor,fill opacity=0.20] ( 80.30, 60.00) circle (  2.13);

\path[fill=fillColor,fill opacity=0.20] ( 78.30, 58.03) circle (  2.13);

\path[fill=fillColor,fill opacity=0.20] ( 82.31, 46.50) circle (  2.13);

\path[fill=fillColor,fill opacity=0.20] ( 79.30, 48.89) circle (  2.13);

\path[fill=fillColor,fill opacity=0.20] ( 80.30, 69.97) circle (  2.13);

\path[fill=fillColor,fill opacity=0.20] ( 78.30, 76.19) circle (  2.13);

\path[fill=fillColor,fill opacity=0.20] ( 93.34, 63.22) circle (  2.13);

\path[fill=fillColor,fill opacity=0.20] (103.38, 57.51) circle (  2.13);

\path[fill=fillColor,fill opacity=0.20] (114.41, 63.01) circle (  2.13);

\path[fill=fillColor,fill opacity=0.20] ( 99.36, 86.89) circle (  2.13);

\path[fill=fillColor,fill opacity=0.20] ( 77.29, 52.63) circle (  2.13);

\path[fill=fillColor,fill opacity=0.20] ( 79.30, 64.15) circle (  2.13);

\path[fill=fillColor,fill opacity=0.20] ( 82.31, 60.41) circle (  2.13);

\path[fill=fillColor,fill opacity=0.20] ( 80.30, 61.14) circle (  2.13);

\path[fill=fillColor,fill opacity=0.20] ( 82.31, 62.18) circle (  2.13);

\path[fill=fillColor,fill opacity=0.20] ( 86.32, 54.81) circle (  2.13);

\path[fill=fillColor,fill opacity=0.20] ( 97.36, 65.19) circle (  2.13);

\path[fill=fillColor,fill opacity=0.20] (101.37,114.92) circle (  2.13);

\path[fill=fillColor,fill opacity=0.20] ( 95.35, 79.93) circle (  2.13);

\path[fill=fillColor,fill opacity=0.20] (100.37, 56.68) circle (  2.13);

\path[fill=fillColor,fill opacity=0.20] ( 91.34, 51.28) circle (  2.13);

\path[fill=fillColor,fill opacity=0.20] ( 88.33, 61.04) circle (  2.13);

\path[fill=fillColor,fill opacity=0.20] ( 87.33, 63.74) circle (  2.13);

\path[fill=fillColor,fill opacity=0.20] ( 88.33, 51.17) circle (  2.13);

\path[fill=fillColor,fill opacity=0.20] ( 86.32, 57.30) circle (  2.13);

\path[fill=fillColor,fill opacity=0.20] ( 82.31, 79.21) circle (  2.13);

\path[fill=fillColor,fill opacity=0.20] (106.39, 64.77) circle (  2.13);

\path[fill=fillColor,fill opacity=0.20] ( 87.33, 79.41) circle (  2.13);

\path[fill=fillColor,fill opacity=0.20] ( 72.28, 50.76) circle (  2.13);

\path[fill=fillColor,fill opacity=0.20] ( 69.27, 67.47) circle (  2.13);

\path[fill=fillColor,fill opacity=0.20] ( 70.27, 60.93) circle (  2.13);

\path[fill=fillColor,fill opacity=0.20] ( 67.36, 47.33) circle (  2.13);

\path[fill=fillColor,fill opacity=0.20] ( 60.04, 44.74) circle (  2.13);

\path[fill=fillColor,fill opacity=0.20] ( 74.28, 50.24) circle (  2.13);

\path[fill=fillColor,fill opacity=0.20] ( 79.30, 44.22) circle (  2.13);

\path[fill=fillColor,fill opacity=0.20] ( 82.31, 49.31) circle (  2.13);

\path[fill=fillColor,fill opacity=0.20] (109.40, 90.00) circle (  2.13);

\path[fill=fillColor,fill opacity=0.20] (106.39, 71.94) circle (  2.13);

\path[fill=fillColor,fill opacity=0.20] (100.37, 55.53) circle (  2.13);

\path[fill=fillColor,fill opacity=0.20] ( 89.33, 60.93) circle (  2.13);

\path[fill=fillColor,fill opacity=0.20] ( 93.34, 63.63) circle (  2.13);

\path[fill=fillColor,fill opacity=0.20] ( 92.34, 72.15) circle (  2.13);

\path[fill=fillColor,fill opacity=0.20] ( 90.33, 69.45) circle (  2.13);

\path[fill=fillColor,fill opacity=0.20] ( 96.35, 62.18) circle (  2.13);

\path[fill=fillColor,fill opacity=0.20] ( 92.34, 73.81) circle (  2.13);

\path[fill=fillColor,fill opacity=0.20] ( 93.34, 78.58) circle (  2.13);

\path[fill=fillColor,fill opacity=0.20] (104.38, 70.38) circle (  2.13);

\path[fill=fillColor,fill opacity=0.20] ( 75.29, 49.82) circle (  2.13);

\path[fill=fillColor,fill opacity=0.20] ( 69.27, 65.29) circle (  2.13);

\path[fill=fillColor,fill opacity=0.20] ( 66.06, 65.81) circle (  2.13);

\path[fill=fillColor,fill opacity=0.20] ( 64.55, 44.43) circle (  2.13);

\path[fill=fillColor,fill opacity=0.20] ( 70.27, 43.49) circle (  2.13);

\path[fill=fillColor,fill opacity=0.20] ( 68.26, 52.00) circle (  2.13);

\path[fill=fillColor,fill opacity=0.20] ( 81.31, 51.80) circle (  2.13);

\path[fill=fillColor,fill opacity=0.20] ( 84.32, 45.78) circle (  2.13);

\path[fill=fillColor,fill opacity=0.20] (100.37, 56.05) circle (  2.13);

\path[fill=fillColor,fill opacity=0.20] (115.42, 74.12) circle (  2.13);

\path[fill=fillColor,fill opacity=0.20] ( 97.36, 62.39) circle (  2.13);

\path[fill=fillColor,fill opacity=0.20] ( 93.34, 74.64) circle (  2.13);

\path[fill=fillColor,fill opacity=0.20] ( 89.33, 58.34) circle (  2.13);

\path[fill=fillColor,fill opacity=0.20] ( 90.33, 53.04) circle (  2.13);

\path[fill=fillColor,fill opacity=0.20] ( 92.34, 70.28) circle (  2.13);

\path[fill=fillColor,fill opacity=0.20] ( 87.33, 75.68) circle (  2.13);

\path[fill=fillColor,fill opacity=0.20] ( 93.34, 67.47) circle (  2.13);

\path[fill=fillColor,fill opacity=0.20] ( 90.33, 60.52) circle (  2.13);

\path[fill=fillColor,fill opacity=0.20] (102.37, 65.40) circle (  2.13);

\path[fill=fillColor,fill opacity=0.20] (133.47,109.73) circle (  2.13);

\path[fill=fillColor,fill opacity=0.20] ( 77.29, 78.06) circle (  2.13);

\path[fill=fillColor,fill opacity=0.20] ( 69.27, 58.75) circle (  2.13);

\path[fill=fillColor,fill opacity=0.20] ( 72.28, 58.55) circle (  2.13);

\path[fill=fillColor,fill opacity=0.20] ( 81.31, 48.68) circle (  2.13);

\path[fill=fillColor,fill opacity=0.20] ( 78.30, 58.23) circle (  2.13);

\path[fill=fillColor,fill opacity=0.20] ( 77.29, 61.04) circle (  2.13);

\path[fill=fillColor,fill opacity=0.20] ( 86.32, 55.22) circle (  2.13);

\path[fill=fillColor,fill opacity=0.20] ( 92.34, 54.91) circle (  2.13);

\path[fill=fillColor,fill opacity=0.20] (109.40, 84.81) circle (  2.13);

\path[fill=fillColor,fill opacity=0.20] (101.37, 68.93) circle (  2.13);

\path[fill=fillColor,fill opacity=0.20] ( 92.34, 71.52) circle (  2.13);

\path[fill=fillColor,fill opacity=0.20] ( 84.32, 52.11) circle (  2.13);

\path[fill=fillColor,fill opacity=0.20] ( 86.32, 38.82) circle (  2.13);

\path[fill=fillColor,fill opacity=0.20] ( 91.34, 56.99) circle (  2.13);

\path[fill=fillColor,fill opacity=0.20] ( 93.34, 70.59) circle (  2.13);

\path[fill=fillColor,fill opacity=0.20] ( 95.35, 67.16) circle (  2.13);

\path[fill=fillColor,fill opacity=0.20] ( 96.35, 56.57) circle (  2.13);

\path[fill=fillColor,fill opacity=0.20] ( 97.36, 53.15) circle (  2.13);

\path[fill=fillColor,fill opacity=0.20] ( 66.66, 53.15) circle (  2.13);

\path[fill=fillColor,fill opacity=0.20] ( 72.28, 48.99) circle (  2.13);

\path[fill=fillColor,fill opacity=0.20] ( 76.29, 61.35) circle (  2.13);

\path[fill=fillColor,fill opacity=0.20] ( 67.96, 53.98) circle (  2.13);

\path[fill=fillColor,fill opacity=0.20] ( 79.30, 56.05) circle (  2.13);

\path[fill=fillColor,fill opacity=0.20] ( 81.31, 62.39) circle (  2.13);

\path[fill=fillColor,fill opacity=0.20] ( 78.30, 60.41) circle (  2.13);

\path[fill=fillColor,fill opacity=0.20] ( 82.31, 60.10) circle (  2.13);

\path[fill=fillColor,fill opacity=0.20] ( 88.33, 60.41) circle (  2.13);

\path[fill=fillColor,fill opacity=0.20] ( 91.34, 42.56) circle (  2.13);

\path[fill=fillColor,fill opacity=0.20] (103.38, 70.28) circle (  2.13);

\path[fill=fillColor,fill opacity=0.20] ( 85.32, 60.52) circle (  2.13);

\path[fill=fillColor,fill opacity=0.20] ( 79.30, 56.05) circle (  2.13);

\path[fill=fillColor,fill opacity=0.20] ( 81.31, 53.77) circle (  2.13);

\path[fill=fillColor,fill opacity=0.20] ( 91.34, 53.15) circle (  2.13);

\path[fill=fillColor,fill opacity=0.20] ( 94.35, 53.15) circle (  2.13);

\path[fill=fillColor,fill opacity=0.20] ( 94.35, 60.10) circle (  2.13);

\path[fill=fillColor,fill opacity=0.20] ( 98.36, 71.11) circle (  2.13);

\path[fill=fillColor,fill opacity=0.20] ( 95.35, 58.96) circle (  2.13);

\path[fill=fillColor,fill opacity=0.20] ( 87.33,108.69) circle (  2.13);

\path[fill=fillColor,fill opacity=0.20] ( 69.27, 46.29) circle (  2.13);

\path[fill=fillColor,fill opacity=0.20] ( 72.28, 72.25) circle (  2.13);

\path[fill=fillColor,fill opacity=0.20] ( 74.28, 71.21) circle (  2.13);

\path[fill=fillColor,fill opacity=0.20] ( 69.27, 54.18) circle (  2.13);

\path[fill=fillColor,fill opacity=0.20] ( 77.29, 55.53) circle (  2.13);

\path[fill=fillColor,fill opacity=0.20] ( 83.31, 61.04) circle (  2.13);

\path[fill=fillColor,fill opacity=0.20] ( 76.29, 61.14) circle (  2.13);

\path[fill=fillColor,fill opacity=0.20] ( 91.34, 65.71) circle (  2.13);

\path[fill=fillColor,fill opacity=0.20] ( 88.33, 58.34) circle (  2.13);

\path[fill=fillColor,fill opacity=0.20] ( 93.34, 39.86) circle (  2.13);

\path[fill=fillColor,fill opacity=0.20] (106.39, 52.94) circle (  2.13);

\path[fill=fillColor,fill opacity=0.20] ( 95.35, 51.69) circle (  2.13);

\path[fill=fillColor,fill opacity=0.20] ( 86.32, 60.83) circle (  2.13);

\path[fill=fillColor,fill opacity=0.20] ( 84.32, 66.33) circle (  2.13);

\path[fill=fillColor,fill opacity=0.20] ( 88.33, 59.17) circle (  2.13);

\path[fill=fillColor,fill opacity=0.20] ( 91.34, 47.02) circle (  2.13);

\path[fill=fillColor,fill opacity=0.20] ( 89.33, 50.65) circle (  2.13);

\path[fill=fillColor,fill opacity=0.20] ( 93.34, 75.47) circle (  2.13);

\path[fill=fillColor,fill opacity=0.20] ( 82.31, 74.33) circle (  2.13);

\path[fill=fillColor,fill opacity=0.20] ( 93.34, 60.31) circle (  2.13);

\path[fill=fillColor,fill opacity=0.20] ( 80.30, 62.08) circle (  2.13);

\path[fill=fillColor,fill opacity=0.20] ( 79.30, 62.18) circle (  2.13);

\path[fill=fillColor,fill opacity=0.20] ( 80.30, 77.86) circle (  2.13);

\path[fill=fillColor,fill opacity=0.20] ( 81.31, 68.93) circle (  2.13);

\path[fill=fillColor,fill opacity=0.20] ( 77.29, 57.20) circle (  2.13);

\path[fill=fillColor,fill opacity=0.20] ( 79.30, 60.73) circle (  2.13);

\path[fill=fillColor,fill opacity=0.20] ( 80.30, 59.89) circle (  2.13);

\path[fill=fillColor,fill opacity=0.20] ( 79.30, 53.67) circle (  2.13);

\path[fill=fillColor,fill opacity=0.20] ( 82.31, 56.68) circle (  2.13);

\path[fill=fillColor,fill opacity=0.20] ( 93.34, 51.59) circle (  2.13);

\path[fill=fillColor,fill opacity=0.20] ( 94.35, 40.17) circle (  2.13);

\path[fill=fillColor,fill opacity=0.20] (110.40, 64.98) circle (  2.13);

\path[fill=fillColor,fill opacity=0.20] (100.37, 60.00) circle (  2.13);

\path[fill=fillColor,fill opacity=0.20] ( 95.35, 63.32) circle (  2.13);

\path[fill=fillColor,fill opacity=0.20] ( 95.35, 64.15) circle (  2.13);

\path[fill=fillColor,fill opacity=0.20] ( 87.33, 57.92) circle (  2.13);

\path[fill=fillColor,fill opacity=0.20] ( 91.34, 55.53) circle (  2.13);

\path[fill=fillColor,fill opacity=0.20] ( 90.33, 55.22) circle (  2.13);

\path[fill=fillColor,fill opacity=0.20] ( 87.33, 64.15) circle (  2.13);

\path[fill=fillColor,fill opacity=0.20] ( 96.35, 67.58) circle (  2.13);

\path[fill=fillColor,fill opacity=0.20] (101.37, 56.05) circle (  2.13);

\path[fill=fillColor,fill opacity=0.20] (114.41, 69.24) circle (  2.13);

\path[fill=fillColor,fill opacity=0.20] ( 80.30, 78.69) circle (  2.13);

\path[fill=fillColor,fill opacity=0.20] ( 69.27, 59.06) circle (  2.13);

\path[fill=fillColor,fill opacity=0.20] ( 86.32, 71.83) circle (  2.13);

\path[fill=fillColor,fill opacity=0.20] ( 84.32, 70.28) circle (  2.13);

\path[fill=fillColor,fill opacity=0.20] ( 80.30, 60.31) circle (  2.13);

\path[fill=fillColor,fill opacity=0.20] ( 80.30, 55.85) circle (  2.13);

\path[fill=fillColor,fill opacity=0.20] ( 75.29, 60.52) circle (  2.13);

\path[fill=fillColor,fill opacity=0.20] ( 83.31, 55.43) circle (  2.13);

\path[fill=fillColor,fill opacity=0.20] ( 77.29, 44.94) circle (  2.13);

\path[fill=fillColor,fill opacity=0.20] ( 77.29, 50.45) circle (  2.13);

\path[fill=fillColor,fill opacity=0.20] ( 89.33, 50.14) circle (  2.13);

\path[fill=fillColor,fill opacity=0.20] ( 95.35, 47.33) circle (  2.13);

\path[fill=fillColor,fill opacity=0.20] ( 98.36, 73.50) circle (  2.13);

\path[fill=fillColor,fill opacity=0.20] ( 86.32, 58.55) circle (  2.13);

\path[fill=fillColor,fill opacity=0.20] ( 87.33, 49.82) circle (  2.13);

\path[fill=fillColor,fill opacity=0.20] ( 89.33, 58.03) circle (  2.13);

\path[fill=fillColor,fill opacity=0.20] ( 87.33, 59.89) circle (  2.13);

\path[fill=fillColor,fill opacity=0.20] ( 88.33, 50.34) circle (  2.13);

\path[fill=fillColor,fill opacity=0.20] ( 90.33, 43.49) circle (  2.13);

\path[fill=fillColor,fill opacity=0.20] ( 97.36, 43.08) circle (  2.13);

\path[fill=fillColor,fill opacity=0.20] ( 97.36, 49.62) circle (  2.13);

\path[fill=fillColor,fill opacity=0.20] (105.38, 68.93) circle (  2.13);

\path[fill=fillColor,fill opacity=0.20] ( 84.32, 96.23) circle (  2.13);

\path[fill=fillColor,fill opacity=0.20] ( 76.29, 53.46) circle (  2.13);

\path[fill=fillColor,fill opacity=0.20] ( 77.29, 63.74) circle (  2.13);

\path[fill=fillColor,fill opacity=0.20] ( 82.31, 60.00) circle (  2.13);

\path[fill=fillColor,fill opacity=0.20] ( 75.29, 57.09) circle (  2.13);

\path[fill=fillColor,fill opacity=0.20] ( 75.29, 59.06) circle (  2.13);

\path[fill=fillColor,fill opacity=0.20] ( 75.29, 58.34) circle (  2.13);

\path[fill=fillColor,fill opacity=0.20] ( 74.28, 53.56) circle (  2.13);

\path[fill=fillColor,fill opacity=0.20] ( 78.30, 47.64) circle (  2.13);

\path[fill=fillColor,fill opacity=0.20] ( 79.30, 48.79) circle (  2.13);

\path[fill=fillColor,fill opacity=0.20] ( 84.32, 58.23) circle (  2.13);

\path[fill=fillColor,fill opacity=0.20] ( 81.31, 58.96) circle (  2.13);

\path[fill=fillColor,fill opacity=0.20] ( 90.33, 62.08) circle (  2.13);

\path[fill=fillColor,fill opacity=0.20] ( 83.31, 68.30) circle (  2.13);

\path[fill=fillColor,fill opacity=0.20] ( 77.29, 44.94) circle (  2.13);

\path[fill=fillColor,fill opacity=0.20] ( 90.33, 47.64) circle (  2.13);

\path[fill=fillColor,fill opacity=0.20] ( 86.32, 53.35) circle (  2.13);

\path[fill=fillColor,fill opacity=0.20] ( 89.33, 47.54) circle (  2.13);

\path[fill=fillColor,fill opacity=0.20] ( 93.34, 44.43) circle (  2.13);

\path[fill=fillColor,fill opacity=0.20] ( 87.33, 45.05) circle (  2.13);

\path[fill=fillColor,fill opacity=0.20] ( 92.34, 50.86) circle (  2.13);

\path[fill=fillColor,fill opacity=0.20] ( 96.35, 64.46) circle (  2.13);

\path[fill=fillColor,fill opacity=0.20] (111.40, 79.10) circle (  2.13);

\path[fill=fillColor,fill opacity=0.20] ( 82.31,107.65) circle (  2.13);

\path[fill=fillColor,fill opacity=0.20] ( 78.30, 60.83) circle (  2.13);

\path[fill=fillColor,fill opacity=0.20] ( 76.29, 69.24) circle (  2.13);

\path[fill=fillColor,fill opacity=0.20] ( 80.30, 58.96) circle (  2.13);

\path[fill=fillColor,fill opacity=0.20] ( 76.29, 49.51) circle (  2.13);

\path[fill=fillColor,fill opacity=0.20] ( 75.29, 54.81) circle (  2.13);

\path[fill=fillColor,fill opacity=0.20] ( 75.29, 64.36) circle (  2.13);

\path[fill=fillColor,fill opacity=0.20] ( 75.29, 67.68) circle (  2.13);

\path[fill=fillColor,fill opacity=0.20] ( 72.28, 58.65) circle (  2.13);

\path[fill=fillColor,fill opacity=0.20] ( 79.30, 51.69) circle (  2.13);

\path[fill=fillColor,fill opacity=0.20] ( 78.30, 55.22) circle (  2.13);

\path[fill=fillColor,fill opacity=0.20] ( 80.30, 58.23) circle (  2.13);

\path[fill=fillColor,fill opacity=0.20] ( 83.31, 67.16) circle (  2.13);

\path[fill=fillColor,fill opacity=0.20] (100.37, 88.96) circle (  2.13);

\path[fill=fillColor,fill opacity=0.20] (112.41, 95.19) circle (  2.13);

\path[fill=fillColor,fill opacity=0.20] ( 94.35, 60.10) circle (  2.13);

\path[fill=fillColor,fill opacity=0.20] ( 88.33, 48.16) circle (  2.13);

\path[fill=fillColor,fill opacity=0.20] ( 90.33, 48.79) circle (  2.13);

\path[fill=fillColor,fill opacity=0.20] ( 93.34, 56.57) circle (  2.13);

\path[fill=fillColor,fill opacity=0.20] ( 86.32, 58.55) circle (  2.13);

\path[fill=fillColor,fill opacity=0.20] ( 90.33, 54.18) circle (  2.13);

\path[fill=fillColor,fill opacity=0.20] ( 89.33, 56.99) circle (  2.13);

\path[fill=fillColor,fill opacity=0.20] ( 95.35, 68.62) circle (  2.13);

\path[fill=fillColor,fill opacity=0.20] ( 96.35, 75.68) circle (  2.13);

\path[fill=fillColor,fill opacity=0.20] (105.38, 80.87) circle (  2.13);

\path[fill=fillColor,fill opacity=0.20] ( 89.33,111.81) circle (  2.13);

\path[fill=fillColor,fill opacity=0.20] ( 80.30, 58.13) circle (  2.13);

\path[fill=fillColor,fill opacity=0.20] ( 77.29, 72.87) circle (  2.13);

\path[fill=fillColor,fill opacity=0.20] ( 80.30, 69.55) circle (  2.13);

\path[fill=fillColor,fill opacity=0.20] ( 78.30, 59.17) circle (  2.13);

\path[fill=fillColor,fill opacity=0.20] ( 73.28, 63.11) circle (  2.13);

\path[fill=fillColor,fill opacity=0.20] ( 79.30, 64.98) circle (  2.13);

\path[fill=fillColor,fill opacity=0.20] ( 79.30, 63.74) circle (  2.13);

\path[fill=fillColor,fill opacity=0.20] ( 75.29, 64.77) circle (  2.13);

\path[fill=fillColor,fill opacity=0.20] ( 76.29, 66.02) circle (  2.13);

\path[fill=fillColor,fill opacity=0.20] ( 74.28, 68.62) circle (  2.13);

\path[fill=fillColor,fill opacity=0.20] ( 77.29, 59.17) circle (  2.13);

\path[fill=fillColor,fill opacity=0.20] ( 90.33, 47.75) circle (  2.13);

\path[fill=fillColor,fill opacity=0.20] (111.40, 94.16) circle (  2.13);

\path[fill=fillColor,fill opacity=0.20] ( 86.32, 71.32) circle (  2.13);

\path[fill=fillColor,fill opacity=0.20] ( 85.32, 53.04) circle (  2.13);

\path[fill=fillColor,fill opacity=0.20] ( 91.34, 56.88) circle (  2.13);

\path[fill=fillColor,fill opacity=0.20] ( 86.32, 53.56) circle (  2.13);

\path[fill=fillColor,fill opacity=0.20] ( 83.31, 45.57) circle (  2.13);

\path[fill=fillColor,fill opacity=0.20] ( 90.33, 53.56) circle (  2.13);

\path[fill=fillColor,fill opacity=0.20] ( 91.34, 59.58) circle (  2.13);

\path[fill=fillColor,fill opacity=0.20] ( 91.34, 61.87) circle (  2.13);

\path[fill=fillColor,fill opacity=0.20] ( 93.34, 62.91) circle (  2.13);

\path[fill=fillColor,fill opacity=0.20] (103.38, 66.95) circle (  2.13);

\path[fill=fillColor,fill opacity=0.20] ( 84.32, 64.88) circle (  2.13);

\path[fill=fillColor,fill opacity=0.20] ( 79.30, 74.85) circle (  2.13);

\path[fill=fillColor,fill opacity=0.20] ( 74.28, 69.65) circle (  2.13);

\path[fill=fillColor,fill opacity=0.20] ( 77.29, 55.74) circle (  2.13);

\path[fill=fillColor,fill opacity=0.20] ( 72.28, 55.64) circle (  2.13);

\path[fill=fillColor,fill opacity=0.20] ( 73.28, 63.22) circle (  2.13);

\path[fill=fillColor,fill opacity=0.20] ( 76.29, 67.27) circle (  2.13);

\path[fill=fillColor,fill opacity=0.20] ( 75.29, 60.31) circle (  2.13);

\path[fill=fillColor,fill opacity=0.20] ( 71.27, 57.61) circle (  2.13);

\path[fill=fillColor,fill opacity=0.20] ( 72.28, 66.33) circle (  2.13);

\path[fill=fillColor,fill opacity=0.20] ( 45.69, 69.86) circle (  2.13);

\path[fill=fillColor,fill opacity=0.20] ( 78.30, 60.41) circle (  2.13);

\path[fill=fillColor,fill opacity=0.20] ( 98.36, 50.45) circle (  2.13);

\path[fill=fillColor,fill opacity=0.20] (112.41,101.42) circle (  2.13);

\path[fill=fillColor,fill opacity=0.20] ( 85.32, 66.12) circle (  2.13);

\path[fill=fillColor,fill opacity=0.20] ( 81.31, 49.93) circle (  2.13);

\path[fill=fillColor,fill opacity=0.20] ( 82.31, 41.41) circle (  2.13);

\path[fill=fillColor,fill opacity=0.20] ( 76.29, 39.55) circle (  2.13);

\path[fill=fillColor,fill opacity=0.20] ( 85.32, 49.82) circle (  2.13);

\path[fill=fillColor,fill opacity=0.20] ( 89.33, 53.87) circle (  2.13);

\path[fill=fillColor,fill opacity=0.20] ( 89.33, 59.48) circle (  2.13);

\path[fill=fillColor,fill opacity=0.20] ( 97.36, 64.88) circle (  2.13);

\path[fill=fillColor,fill opacity=0.20] (102.37, 63.74) circle (  2.13);

\path[fill=fillColor,fill opacity=0.20] (107.39, 75.99) circle (  2.13);

\path[fill=fillColor,fill opacity=0.20] ( 93.34,108.69) circle (  2.13);

\path[fill=fillColor,fill opacity=0.20] ( 81.31, 71.94) circle (  2.13);

\path[fill=fillColor,fill opacity=0.20] ( 83.31, 80.24) circle (  2.13);

\path[fill=fillColor,fill opacity=0.20] ( 77.29, 71.52) circle (  2.13);

\path[fill=fillColor,fill opacity=0.20] ( 72.28, 57.71) circle (  2.13);

\path[fill=fillColor,fill opacity=0.20] ( 69.27, 51.38) circle (  2.13);

\path[fill=fillColor,fill opacity=0.20] ( 66.66, 49.20) circle (  2.13);

\path[fill=fillColor,fill opacity=0.20] ( 72.28, 51.38) circle (  2.13);

\path[fill=fillColor,fill opacity=0.20] ( 73.28, 57.92) circle (  2.13);

\path[fill=fillColor,fill opacity=0.20] ( 71.27, 61.97) circle (  2.13);

\path[fill=fillColor,fill opacity=0.20] ( 73.28, 61.87) circle (  2.13);

\path[fill=fillColor,fill opacity=0.20] ( 67.96, 60.62) circle (  2.13);

\path[fill=fillColor,fill opacity=0.20] ( 75.29, 56.57) circle (  2.13);

\path[fill=fillColor,fill opacity=0.20] ( 95.35, 63.32) circle (  2.13);

\path[fill=fillColor,fill opacity=0.20] (110.40, 95.19) circle (  2.13);

\path[fill=fillColor,fill opacity=0.20] ( 74.28, 62.59) circle (  2.13);

\path[fill=fillColor,fill opacity=0.20] ( 79.30, 51.80) circle (  2.13);

\path[fill=fillColor,fill opacity=0.20] ( 78.30, 52.63) circle (  2.13);

\path[fill=fillColor,fill opacity=0.20] ( 80.30, 53.35) circle (  2.13);

\path[fill=fillColor,fill opacity=0.20] ( 89.33, 59.58) circle (  2.13);

\path[fill=fillColor,fill opacity=0.20] ( 87.33, 68.51) circle (  2.13);

\path[fill=fillColor,fill opacity=0.20] ( 91.34, 68.10) circle (  2.13);

\path[fill=fillColor,fill opacity=0.20] (102.37, 67.58) circle (  2.13);

\path[fill=fillColor,fill opacity=0.20] ( 99.36, 69.03) circle (  2.13);

\path[fill=fillColor,fill opacity=0.20] (104.38, 65.92) circle (  2.13);

\path[fill=fillColor,fill opacity=0.20] ( 99.36, 80.14) circle (  2.13);

\path[fill=fillColor,fill opacity=0.20] ( 80.30, 54.81) circle (  2.13);

\path[fill=fillColor,fill opacity=0.20] ( 66.26, 80.14) circle (  2.13);

\path[fill=fillColor,fill opacity=0.20] ( 75.29, 81.39) circle (  2.13);

\path[fill=fillColor,fill opacity=0.20] ( 71.27, 59.89) circle (  2.13);

\path[fill=fillColor,fill opacity=0.20] ( 69.27, 59.06) circle (  2.13);

\path[fill=fillColor,fill opacity=0.20] ( 66.66, 64.15) circle (  2.13);

\path[fill=fillColor,fill opacity=0.20] ( 68.26, 57.92) circle (  2.13);

\path[fill=fillColor,fill opacity=0.20] ( 72.28, 49.72) circle (  2.13);

\path[fill=fillColor,fill opacity=0.20] ( 72.28, 51.38) circle (  2.13);

\path[fill=fillColor,fill opacity=0.20] ( 73.28, 58.75) circle (  2.13);

\path[fill=fillColor,fill opacity=0.20] ( 76.29, 56.26) circle (  2.13);

\path[fill=fillColor,fill opacity=0.20] ( 75.29, 52.00) circle (  2.13);

\path[fill=fillColor,fill opacity=0.20] ( 96.35, 60.10) circle (  2.13);

\path[fill=fillColor,fill opacity=0.20] ( 75.29, 74.64) circle (  2.13);

\path[fill=fillColor,fill opacity=0.20] ( 76.29, 68.41) circle (  2.13);

\path[fill=fillColor,fill opacity=0.20] ( 86.32, 53.87) circle (  2.13);

\path[fill=fillColor,fill opacity=0.20] ( 89.33, 52.52) circle (  2.13);

\path[fill=fillColor,fill opacity=0.20] ( 85.32, 66.95) circle (  2.13);

\path[fill=fillColor,fill opacity=0.20] ( 89.33, 66.33) circle (  2.13);

\path[fill=fillColor,fill opacity=0.20] ( 96.35, 61.56) circle (  2.13);

\path[fill=fillColor,fill opacity=0.20] ( 99.36, 62.80) circle (  2.13);

\path[fill=fillColor,fill opacity=0.20] (105.38, 57.09) circle (  2.13);

\path[fill=fillColor,fill opacity=0.20] ( 96.35, 54.08) circle (  2.13);

\path[fill=fillColor,fill opacity=0.20] (105.38, 60.73) circle (  2.13);

\path[fill=fillColor,fill opacity=0.20] (112.41, 82.74) circle (  2.13);

\path[fill=fillColor,fill opacity=0.20] ( 93.34, 52.52) circle (  2.13);

\path[fill=fillColor,fill opacity=0.20] ( 85.32, 53.77) circle (  2.13);

\path[fill=fillColor,fill opacity=0.20] ( 72.28, 63.94) circle (  2.13);

\path[fill=fillColor,fill opacity=0.20] ( 77.29, 67.58) circle (  2.13);

\path[fill=fillColor,fill opacity=0.20] ( 74.28, 68.10) circle (  2.13);

\path[fill=fillColor,fill opacity=0.20] ( 73.28, 65.29) circle (  2.13);

\path[fill=fillColor,fill opacity=0.20] ( 70.27, 68.20) circle (  2.13);

\path[fill=fillColor,fill opacity=0.20] ( 67.66, 74.95) circle (  2.13);

\path[fill=fillColor,fill opacity=0.20] ( 67.66, 68.72) circle (  2.13);

\path[fill=fillColor,fill opacity=0.20] ( 69.27, 54.70) circle (  2.13);

\path[fill=fillColor,fill opacity=0.20] ( 73.28, 50.86) circle (  2.13);

\path[fill=fillColor,fill opacity=0.20] ( 76.29, 50.97) circle (  2.13);

\path[fill=fillColor,fill opacity=0.20] ( 90.33, 46.40) circle (  2.13);

\path[fill=fillColor,fill opacity=0.20] ( 95.35, 57.20) circle (  2.13);

\path[fill=fillColor,fill opacity=0.20] ( 93.34, 82.74) circle (  2.13);

\path[fill=fillColor,fill opacity=0.20] ( 88.33, 59.48) circle (  2.13);

\path[fill=fillColor,fill opacity=0.20] ( 86.32, 41.93) circle (  2.13);

\path[fill=fillColor,fill opacity=0.20] ( 83.31, 49.41) circle (  2.13);

\path[fill=fillColor,fill opacity=0.20] ( 86.32, 60.52) circle (  2.13);

\path[fill=fillColor,fill opacity=0.20] ( 93.34, 59.06) circle (  2.13);

\path[fill=fillColor,fill opacity=0.20] (101.37, 61.66) circle (  2.13);

\path[fill=fillColor,fill opacity=0.20] (105.38, 72.35) circle (  2.13);

\path[fill=fillColor,fill opacity=0.20] (104.38, 74.22) circle (  2.13);

\path[fill=fillColor,fill opacity=0.20] (103.38, 60.73) circle (  2.13);

\path[fill=fillColor,fill opacity=0.20] (107.39, 48.79) circle (  2.13);

\path[fill=fillColor,fill opacity=0.20] (108.39, 52.21) circle (  2.13);

\path[fill=fillColor,fill opacity=0.20] (115.42, 70.38) circle (  2.13);

\path[fill=fillColor,fill opacity=0.20] (106.39, 65.40) circle (  2.13);

\path[fill=fillColor,fill opacity=0.20] ( 97.36, 63.74) circle (  2.13);

\path[fill=fillColor,fill opacity=0.20] ( 86.32, 63.42) circle (  2.13);

\path[fill=fillColor,fill opacity=0.20] ( 81.31, 64.46) circle (  2.13);

\path[fill=fillColor,fill opacity=0.20] ( 57.13, 68.62) circle (  2.13);

\path[fill=fillColor,fill opacity=0.20] ( 76.29, 68.72) circle (  2.13);

\path[fill=fillColor,fill opacity=0.20] ( 82.31, 58.03) circle (  2.13);

\path[fill=fillColor,fill opacity=0.20] ( 73.28, 58.34) circle (  2.13);

\path[fill=fillColor,fill opacity=0.20] ( 71.27, 67.68) circle (  2.13);

\path[fill=fillColor,fill opacity=0.20] ( 71.27, 70.80) circle (  2.13);

\path[fill=fillColor,fill opacity=0.20] ( 69.27, 71.83) circle (  2.13);

\path[fill=fillColor,fill opacity=0.20] ( 66.46, 74.43) circle (  2.13);

\path[fill=fillColor,fill opacity=0.20] ( 74.28, 67.79) circle (  2.13);

\path[fill=fillColor,fill opacity=0.20] ( 78.30, 55.43) circle (  2.13);

\path[fill=fillColor,fill opacity=0.20] ( 92.34, 58.13) circle (  2.13);

\path[fill=fillColor,fill opacity=0.20] (102.37, 80.97) circle (  2.13);

\path[fill=fillColor,fill opacity=0.20] ( 90.33, 58.75) circle (  2.13);

\path[fill=fillColor,fill opacity=0.20] ( 83.31, 47.23) circle (  2.13);

\path[fill=fillColor,fill opacity=0.20] ( 85.32, 54.50) circle (  2.13);

\path[fill=fillColor,fill opacity=0.20] ( 91.34, 57.61) circle (  2.13);

\path[fill=fillColor,fill opacity=0.20] ( 90.33, 56.05) circle (  2.13);

\path[fill=fillColor,fill opacity=0.20] ( 93.34, 68.51) circle (  2.13);

\path[fill=fillColor,fill opacity=0.20] ( 95.35, 74.53) circle (  2.13);

\path[fill=fillColor,fill opacity=0.20] (100.37, 65.71) circle (  2.13);

\path[fill=fillColor,fill opacity=0.20] (101.37, 55.12) circle (  2.13);

\path[fill=fillColor,fill opacity=0.20] (100.37, 52.63) circle (  2.13);

\path[fill=fillColor,fill opacity=0.20] ( 99.36, 61.76) circle (  2.13);

\path[fill=fillColor,fill opacity=0.20] (105.38, 69.24) circle (  2.13);

\path[fill=fillColor,fill opacity=0.20] (109.40, 60.93) circle (  2.13);

\path[fill=fillColor,fill opacity=0.20] (111.40, 68.10) circle (  2.13);

\path[fill=fillColor,fill opacity=0.20] (107.39, 66.95) circle (  2.13);

\path[fill=fillColor,fill opacity=0.20] (106.39, 69.86) circle (  2.13);

\path[fill=fillColor,fill opacity=0.20] ( 78.30, 73.81) circle (  2.13);

\path[fill=fillColor,fill opacity=0.20] ( 87.33, 62.39) circle (  2.13);

\path[fill=fillColor,fill opacity=0.20] ( 83.31, 62.91) circle (  2.13);

\path[fill=fillColor,fill opacity=0.20] ( 73.28, 78.79) circle (  2.13);

\path[fill=fillColor,fill opacity=0.20] ( 77.29, 76.30) circle (  2.13);

\path[fill=fillColor,fill opacity=0.20] ( 76.29, 64.15) circle (  2.13);

\path[fill=fillColor,fill opacity=0.20] ( 84.32, 64.67) circle (  2.13);

\path[fill=fillColor,fill opacity=0.20] ( 76.29, 60.62) circle (  2.13);

\path[fill=fillColor,fill opacity=0.20] ( 69.27, 60.31) circle (  2.13);

\path[fill=fillColor,fill opacity=0.20] ( 61.24, 67.16) circle (  2.13);

\path[fill=fillColor,fill opacity=0.20] ( 73.28, 63.84) circle (  2.13);

\path[fill=fillColor,fill opacity=0.20] ( 77.29, 63.84) circle (  2.13);

\path[fill=fillColor,fill opacity=0.20] ( 75.29, 73.81) circle (  2.13);

\path[fill=fillColor,fill opacity=0.20] ( 86.32, 78.48) circle (  2.13);

\path[fill=fillColor,fill opacity=0.20] ( 89.33, 72.46) circle (  2.13);

\path[fill=fillColor,fill opacity=0.20] (100.37, 72.77) circle (  2.13);

\path[fill=fillColor,fill opacity=0.20] ( 89.33, 69.45) circle (  2.13);

\path[fill=fillColor,fill opacity=0.20] ( 87.33, 61.14) circle (  2.13);

\path[fill=fillColor,fill opacity=0.20] ( 82.31, 46.19) circle (  2.13);

\path[fill=fillColor,fill opacity=0.20] ( 82.31, 45.15) circle (  2.13);

\path[fill=fillColor,fill opacity=0.20] ( 90.33, 56.68) circle (  2.13);

\path[fill=fillColor,fill opacity=0.20] ( 97.36, 59.06) circle (  2.13);

\path[fill=fillColor,fill opacity=0.20] ( 96.35, 62.08) circle (  2.13);

\path[fill=fillColor,fill opacity=0.20] (101.37, 67.89) circle (  2.13);

\path[fill=fillColor,fill opacity=0.20] (101.37, 67.37) circle (  2.13);

\path[fill=fillColor,fill opacity=0.20] (103.38, 68.93) circle (  2.13);

\path[fill=fillColor,fill opacity=0.20] (107.39, 67.89) circle (  2.13);

\path[fill=fillColor,fill opacity=0.20] (106.39, 61.56) circle (  2.13);

\path[fill=fillColor,fill opacity=0.20] (103.38, 58.96) circle (  2.13);

\path[fill=fillColor,fill opacity=0.20] (102.37, 59.79) circle (  2.13);

\path[fill=fillColor,fill opacity=0.20] (102.37, 62.59) circle (  2.13);

\path[fill=fillColor,fill opacity=0.20] (104.38, 70.38) circle (  2.13);

\path[fill=fillColor,fill opacity=0.20] ( 98.36, 58.44) circle (  2.13);

\path[fill=fillColor,fill opacity=0.20] (108.39, 45.67) circle (  2.13);

\path[fill=fillColor,fill opacity=0.20] (109.40, 60.83) circle (  2.13);

\path[fill=fillColor,fill opacity=0.20] ( 99.36, 76.19) circle (  2.13);

\path[fill=fillColor,fill opacity=0.20] ( 98.36, 69.13) circle (  2.13);

\path[fill=fillColor,fill opacity=0.20] (101.37, 64.15) circle (  2.13);

\path[fill=fillColor,fill opacity=0.20] (102.37, 69.45) circle (  2.13);

\path[fill=fillColor,fill opacity=0.20] (103.38, 68.62) circle (  2.13);

\path[fill=fillColor,fill opacity=0.20] ( 96.35, 58.75) circle (  2.13);

\path[fill=fillColor,fill opacity=0.20] (100.37, 53.35) circle (  2.13);

\path[fill=fillColor,fill opacity=0.20] (101.37, 55.74) circle (  2.13);

\path[fill=fillColor,fill opacity=0.20] ( 99.36, 66.12) circle (  2.13);

\path[fill=fillColor,fill opacity=0.20] ( 98.36, 70.80) circle (  2.13);

\path[fill=fillColor,fill opacity=0.20] (100.37, 64.26) circle (  2.13);

\path[fill=fillColor,fill opacity=0.20] (100.37, 60.21) circle (  2.13);

\path[fill=fillColor,fill opacity=0.20] ( 94.35, 57.40) circle (  2.13);

\path[fill=fillColor,fill opacity=0.20] ( 88.33, 57.20) circle (  2.13);

\path[fill=fillColor,fill opacity=0.20] ( 81.31, 68.10) circle (  2.13);

\path[fill=fillColor,fill opacity=0.20] ( 79.30, 71.00) circle (  2.13);

\path[fill=fillColor,fill opacity=0.20] ( 85.32, 64.36) circle (  2.13);

\path[fill=fillColor,fill opacity=0.20] ( 81.31, 67.37) circle (  2.13);

\path[fill=fillColor,fill opacity=0.20] ( 76.29, 66.02) circle (  2.13);

\path[fill=fillColor,fill opacity=0.20] ( 79.30, 59.89) circle (  2.13);

\path[fill=fillColor,fill opacity=0.20] ( 80.30, 56.36) circle (  2.13);

\path[fill=fillColor,fill opacity=0.20] ( 81.31, 55.85) circle (  2.13);

\path[fill=fillColor,fill opacity=0.20] ( 78.30, 64.26) circle (  2.13);

\path[fill=fillColor,fill opacity=0.20] ( 82.31, 77.23) circle (  2.13);

\path[fill=fillColor,fill opacity=0.20] ( 90.33, 71.73) circle (  2.13);

\path[fill=fillColor,fill opacity=0.20] ( 76.29, 72.56) circle (  2.13);

\path[fill=fillColor,fill opacity=0.20] ( 88.33, 78.38) circle (  2.13);

\path[fill=fillColor,fill opacity=0.20] ( 95.35, 70.38) circle (  2.13);

\path[fill=fillColor,fill opacity=0.20] ( 89.33, 75.47) circle (  2.13);

\path[fill=fillColor,fill opacity=0.20] ( 91.34, 60.21) circle (  2.13);

\path[fill=fillColor,fill opacity=0.20] ( 88.33, 48.37) circle (  2.13);

\path[fill=fillColor,fill opacity=0.20] ( 87.33, 51.17) circle (  2.13);

\path[fill=fillColor,fill opacity=0.20] ( 89.33, 57.09) circle (  2.13);

\path[fill=fillColor,fill opacity=0.20] ( 92.34, 53.98) circle (  2.13);

\path[fill=fillColor,fill opacity=0.20] ( 97.36, 56.99) circle (  2.13);

\path[fill=fillColor,fill opacity=0.20] (102.37, 60.73) circle (  2.13);

\path[fill=fillColor,fill opacity=0.20] (102.37, 60.73) circle (  2.13);

\path[fill=fillColor,fill opacity=0.20] (101.37, 65.50) circle (  2.13);

\path[fill=fillColor,fill opacity=0.20] (103.38, 68.20) circle (  2.13);

\path[fill=fillColor,fill opacity=0.20] ( 97.36, 65.19) circle (  2.13);

\path[fill=fillColor,fill opacity=0.20] ( 88.33, 66.85) circle (  2.13);

\path[fill=fillColor,fill opacity=0.20] (100.37, 68.41) circle (  2.13);

\path[fill=fillColor,fill opacity=0.20] ( 99.36, 64.88) circle (  2.13);

\path[fill=fillColor,fill opacity=0.20] (102.37, 56.78) circle (  2.13);

\path[fill=fillColor,fill opacity=0.20] ( 99.36, 46.61) circle (  2.13);

\path[fill=fillColor,fill opacity=0.20] (102.37, 49.93) circle (  2.13);

\path[fill=fillColor,fill opacity=0.20] ( 98.36, 56.99) circle (  2.13);

\path[fill=fillColor,fill opacity=0.20] (103.38, 52.32) circle (  2.13);

\path[fill=fillColor,fill opacity=0.20] ( 96.35, 52.42) circle (  2.13);

\path[fill=fillColor,fill opacity=0.20] ( 93.34, 59.17) circle (  2.13);

\path[fill=fillColor,fill opacity=0.20] ( 91.34, 60.41) circle (  2.13);

\path[fill=fillColor,fill opacity=0.20] ( 89.33, 61.14) circle (  2.13);

\path[fill=fillColor,fill opacity=0.20] ( 92.34, 63.01) circle (  2.13);

\path[fill=fillColor,fill opacity=0.20] ( 92.34, 62.39) circle (  2.13);

\path[fill=fillColor,fill opacity=0.20] ( 90.33, 66.85) circle (  2.13);

\path[fill=fillColor,fill opacity=0.20] ( 90.33, 71.21) circle (  2.13);

\path[fill=fillColor,fill opacity=0.20] ( 86.32, 67.06) circle (  2.13);

\path[fill=fillColor,fill opacity=0.20] ( 86.32, 59.69) circle (  2.13);

\path[fill=fillColor,fill opacity=0.20] ( 82.31, 57.92) circle (  2.13);

\path[fill=fillColor,fill opacity=0.20] ( 81.31, 61.14) circle (  2.13);

\path[fill=fillColor,fill opacity=0.20] ( 87.33, 50.76) circle (  2.13);

\path[fill=fillColor,fill opacity=0.20] ( 87.33, 48.58) circle (  2.13);

\path[fill=fillColor,fill opacity=0.20] ( 84.32, 58.96) circle (  2.13);

\path[fill=fillColor,fill opacity=0.20] ( 83.31, 55.12) circle (  2.13);

\path[fill=fillColor,fill opacity=0.20] ( 86.32, 62.28) circle (  2.13);

\path[fill=fillColor,fill opacity=0.20] ( 88.33, 66.44) circle (  2.13);

\path[fill=fillColor,fill opacity=0.20] ( 84.32, 68.20) circle (  2.13);

\path[fill=fillColor,fill opacity=0.20] ( 86.32, 83.77) circle (  2.13);

\path[fill=fillColor,fill opacity=0.20] ( 98.36, 86.89) circle (  2.13);

\path[fill=fillColor,fill opacity=0.20] ( 73.28, 79.31) circle (  2.13);

\path[fill=fillColor,fill opacity=0.20] (111.40, 86.89) circle (  2.13);

\path[fill=fillColor,fill opacity=0.20] ( 89.33, 75.57) circle (  2.13);

\path[fill=fillColor,fill opacity=0.20] ( 88.33, 66.23) circle (  2.13);

\path[fill=fillColor,fill opacity=0.20] ( 91.34, 63.63) circle (  2.13);

\path[fill=fillColor,fill opacity=0.20] ( 87.33, 56.78) circle (  2.13);

\path[fill=fillColor,fill opacity=0.20] ( 91.34, 46.71) circle (  2.13);

\path[fill=fillColor,fill opacity=0.20] ( 92.34, 45.88) circle (  2.13);

\path[fill=fillColor,fill opacity=0.20] ( 96.35, 55.22) circle (  2.13);

\path[fill=fillColor,fill opacity=0.20] ( 97.36, 65.19) circle (  2.13);

\path[fill=fillColor,fill opacity=0.20] ( 95.35, 66.12) circle (  2.13);

\path[fill=fillColor,fill opacity=0.20] ( 97.36, 67.58) circle (  2.13);

\path[fill=fillColor,fill opacity=0.20] ( 99.36, 72.77) circle (  2.13);

\path[fill=fillColor,fill opacity=0.20] ( 96.35, 71.63) circle (  2.13);

\path[fill=fillColor,fill opacity=0.20] ( 95.35, 65.71) circle (  2.13);

\path[fill=fillColor,fill opacity=0.20] ( 97.36, 64.98) circle (  2.13);

\path[fill=fillColor,fill opacity=0.20] ( 96.35, 65.09) circle (  2.13);

\path[fill=fillColor,fill opacity=0.20] ( 96.35, 62.91) circle (  2.13);

\path[fill=fillColor,fill opacity=0.20] (102.37, 60.93) circle (  2.13);

\path[fill=fillColor,fill opacity=0.20] ( 99.36, 59.58) circle (  2.13);

\path[fill=fillColor,fill opacity=0.20] ( 89.33, 60.62) circle (  2.13);

\path[fill=fillColor,fill opacity=0.20] ( 95.35, 60.73) circle (  2.13);

\path[fill=fillColor,fill opacity=0.20] (100.37, 62.28) circle (  2.13);

\path[fill=fillColor,fill opacity=0.20] (100.37, 69.76) circle (  2.13);

\path[fill=fillColor,fill opacity=0.20] ( 95.35, 72.25) circle (  2.13);

\path[fill=fillColor,fill opacity=0.20] ( 97.36, 68.10) circle (  2.13);

\path[fill=fillColor,fill opacity=0.20] ( 93.34, 63.84) circle (  2.13);

\path[fill=fillColor,fill opacity=0.20] ( 92.34, 59.58) circle (  2.13);

\path[fill=fillColor,fill opacity=0.20] ( 89.33, 57.82) circle (  2.13);

\path[fill=fillColor,fill opacity=0.20] ( 91.34, 54.70) circle (  2.13);

\path[fill=fillColor,fill opacity=0.20] ( 82.31, 50.76) circle (  2.13);

\path[fill=fillColor,fill opacity=0.20] ( 81.31, 58.23) circle (  2.13);

\path[fill=fillColor,fill opacity=0.20] ( 73.28, 65.92) circle (  2.13);

\path[fill=fillColor,fill opacity=0.20] ( 89.33, 62.49) circle (  2.13);

\path[fill=fillColor,fill opacity=0.20] ( 82.31, 63.11) circle (  2.13);

\path[fill=fillColor,fill opacity=0.20] ( 87.33, 72.56) circle (  2.13);

\path[fill=fillColor,fill opacity=0.20] ( 87.33, 81.70) circle (  2.13);

\path[fill=fillColor,fill opacity=0.20] (103.38,104.54) circle (  2.13);

\path[fill=fillColor,fill opacity=0.20] ( 92.34, 80.24) circle (  2.13);

\path[fill=fillColor,fill opacity=0.20] ( 90.33, 59.69) circle (  2.13);

\path[fill=fillColor,fill opacity=0.20] ( 90.33, 49.31) circle (  2.13);

\path[fill=fillColor,fill opacity=0.20] ( 84.32, 58.03) circle (  2.13);

\path[fill=fillColor,fill opacity=0.20] ( 86.32, 66.12) circle (  2.13);

\path[fill=fillColor,fill opacity=0.20] ( 89.33, 62.18) circle (  2.13);

\path[fill=fillColor,fill opacity=0.20] ( 87.33, 60.62) circle (  2.13);

\path[fill=fillColor,fill opacity=0.20] ( 85.32, 63.22) circle (  2.13);

\path[fill=fillColor,fill opacity=0.20] ( 90.33, 60.41) circle (  2.13);

\path[fill=fillColor,fill opacity=0.20] ( 93.34, 55.53) circle (  2.13);

\path[fill=fillColor,fill opacity=0.20] ( 94.35, 50.55) circle (  2.13);

\path[fill=fillColor,fill opacity=0.20] ( 95.35, 52.00) circle (  2.13);

\path[fill=fillColor,fill opacity=0.20] (100.37, 58.44) circle (  2.13);

\path[fill=fillColor,fill opacity=0.20] ( 97.36, 58.96) circle (  2.13);

\path[fill=fillColor,fill opacity=0.20] ( 98.36, 52.52) circle (  2.13);

\path[fill=fillColor,fill opacity=0.20] ( 99.36, 47.75) circle (  2.13);

\path[fill=fillColor,fill opacity=0.20] ( 99.36, 44.94) circle (  2.13);

\path[fill=fillColor,fill opacity=0.20] ( 94.35, 50.14) circle (  2.13);

\path[fill=fillColor,fill opacity=0.20] ( 86.32, 58.86) circle (  2.13);

\path[fill=fillColor,fill opacity=0.20] ( 91.34, 59.38) circle (  2.13);

\path[fill=fillColor,fill opacity=0.20] ( 86.32, 57.51) circle (  2.13);

\path[fill=fillColor,fill opacity=0.20] ( 89.33, 57.51) circle (  2.13);

\path[fill=fillColor,fill opacity=0.20] ( 84.32, 54.29) circle (  2.13);

\path[fill=fillColor,fill opacity=0.20] ( 84.32, 57.82) circle (  2.13);

\path[fill=fillColor,fill opacity=0.20] ( 85.32, 70.48) circle (  2.13);

\path[fill=fillColor,fill opacity=0.20] ( 96.35, 77.65) circle (  2.13);

\path[fill=fillColor,fill opacity=0.20] ( 88.33, 85.85) circle (  2.13);

\path[fill=fillColor,fill opacity=0.20] ( 98.36,109.73) circle (  2.13);

\path[fill=fillColor,fill opacity=0.20] ( 93.34,113.88) circle (  2.13);

\path[fill=fillColor,fill opacity=0.20] ( 96.35, 80.24) circle (  2.13);

\path[fill=fillColor,fill opacity=0.20] ( 81.31, 75.36) circle (  2.13);

\path[fill=fillColor,fill opacity=0.20] ( 84.32, 70.69) circle (  2.13);

\path[fill=fillColor,fill opacity=0.20] ( 92.34, 69.55) circle (  2.13);

\path[fill=fillColor,fill opacity=0.20] ( 82.31, 60.93) circle (  2.13);

\path[fill=fillColor,fill opacity=0.20] ( 87.33, 46.92) circle (  2.13);

\path[fill=fillColor,fill opacity=0.20] ( 78.30, 52.63) circle (  2.13);

\path[fill=fillColor,fill opacity=0.20] ( 83.31, 62.59) circle (  2.13);

\path[fill=fillColor,fill opacity=0.20] ( 85.32, 48.37) circle (  2.13);

\path[fill=fillColor,fill opacity=0.20] ( 93.34, 51.28) circle (  2.13);

\path[fill=fillColor,fill opacity=0.20] ( 95.35, 57.61) circle (  2.13);

\path[fill=fillColor,fill opacity=0.20] ( 92.34, 52.94) circle (  2.13);

\path[fill=fillColor,fill opacity=0.20] ( 93.34, 49.93) circle (  2.13);

\path[fill=fillColor,fill opacity=0.20] ( 90.33, 49.62) circle (  2.13);

\path[fill=fillColor,fill opacity=0.20] ( 91.34, 54.70) circle (  2.13);

\path[fill=fillColor,fill opacity=0.20] ( 83.31, 61.76) circle (  2.13);

\path[fill=fillColor,fill opacity=0.20] ( 82.31, 65.71) circle (  2.13);

\path[fill=fillColor,fill opacity=0.20] ( 90.33, 76.71) circle (  2.13);

\path[fill=fillColor,fill opacity=0.20] ( 86.32, 84.81) circle (  2.13);

\path[fill=fillColor,fill opacity=0.20] ( 80.30, 81.70) circle (  2.13);

\path[fill=fillColor,fill opacity=0.20] (115.42,101.42) circle (  2.13);

\path[fill=fillColor,fill opacity=0.20] (105.38, 93.12) circle (  2.13);

\path[fill=fillColor,fill opacity=0.20] (100.37, 74.95) circle (  2.13);

\path[fill=fillColor,fill opacity=0.20] ( 81.31, 78.48) circle (  2.13);

\path[fill=fillColor,fill opacity=0.20] ( 79.30, 84.81) circle (  2.13);

\path[fill=fillColor,fill opacity=0.20] ( 89.33, 77.86) circle (  2.13);

\path[fill=fillColor,fill opacity=0.20] ( 96.35, 74.85) circle (  2.13);

\path[fill=fillColor,fill opacity=0.20] ( 99.36, 74.12) circle (  2.13);

\path[fill=fillColor,fill opacity=0.20] ( 95.35, 74.53) circle (  2.13);

\path[fill=fillColor,fill opacity=0.20] ( 96.35, 77.23) circle (  2.13);

\path[fill=fillColor,fill opacity=0.20] ( 87.33, 77.96) circle (  2.13);

\path[fill=fillColor,fill opacity=0.20] ( 63.75,102.46) circle (  2.13);

\path[fill=fillColor,fill opacity=0.20] ( 63.85, 95.19) circle (  2.13);

\path[fill=fillColor,fill opacity=0.20] ( 52.01,113.88) circle (  2.13);

\path[fill=fillColor,fill opacity=0.20] ( 58.03, 95.19) circle (  2.13);

\path[fill=fillColor,fill opacity=0.20] ( 72.28, 68.72) circle (  2.13);

\path[fill=fillColor,fill opacity=0.20] ( 75.29, 59.38) circle (  2.13);

\path[fill=fillColor,fill opacity=0.20] ( 75.29, 71.11) circle (  2.13);

\path[fill=fillColor,fill opacity=0.20] ( 65.96, 71.11) circle (  2.13);

\path[fill=fillColor,fill opacity=0.20] ( 70.27, 61.35) circle (  2.13);

\path[fill=fillColor,fill opacity=0.20] ( 58.83, 77.13) circle (  2.13);

\path[fill=fillColor,fill opacity=0.20] ( 61.94, 58.96) circle (  2.13);

\path[fill=fillColor,fill opacity=0.20] ( 65.86, 65.71) circle (  2.13);

\path[fill=fillColor,fill opacity=0.20] ( 71.27, 55.95) circle (  2.13);

\path[fill=fillColor,fill opacity=0.20] ( 69.27, 54.29) circle (  2.13);

\path[fill=fillColor,fill opacity=0.20] ( 70.27, 71.73) circle (  2.13);

\path[fill=fillColor,fill opacity=0.20] ( 68.26, 66.64) circle (  2.13);

\path[fill=fillColor,fill opacity=0.20] ( 68.26, 52.84) circle (  2.13);

\path[fill=fillColor,fill opacity=0.20] ( 72.28, 65.09) circle (  2.13);

\path[fill=fillColor,fill opacity=0.20] ( 83.31, 99.35) circle (  2.13);

\path[fill=fillColor,fill opacity=0.20] ( 60.84, 70.90) circle (  2.13);

\path[fill=fillColor,fill opacity=0.20] ( 74.28, 55.53) circle (  2.13);

\path[fill=fillColor,fill opacity=0.20] ( 68.26, 79.10) circle (  2.13);

\path[fill=fillColor,fill opacity=0.20] ( 60.44, 65.71) circle (  2.13);

\path[fill=fillColor,fill opacity=0.20] ( 62.95, 63.74) circle (  2.13);

\path[fill=fillColor,fill opacity=0.20] ( 68.26, 72.25) circle (  2.13);

\path[fill=fillColor,fill opacity=0.20] ( 65.76, 68.62) circle (  2.13);

\path[fill=fillColor,fill opacity=0.20] ( 62.75, 58.34) circle (  2.13);

\path[fill=fillColor,fill opacity=0.20] ( 67.76, 61.04) circle (  2.13);

\path[fill=fillColor,fill opacity=0.20] ( 85.32, 79.41) circle (  2.13);

\path[fill=fillColor,fill opacity=0.20] ( 67.96, 52.52) circle (  2.13);

\path[fill=fillColor,fill opacity=0.20] ( 75.29, 61.14) circle (  2.13);

\path[fill=fillColor,fill opacity=0.20] ( 64.85, 71.52) circle (  2.13);

\path[fill=fillColor,fill opacity=0.20] ( 53.42, 63.74) circle (  2.13);

\path[fill=fillColor,fill opacity=0.20] ( 56.12, 63.42) circle (  2.13);

\path[fill=fillColor,fill opacity=0.20] ( 73.28, 55.43) circle (  2.13);

\path[fill=fillColor,fill opacity=0.20] ( 69.27, 53.67) circle (  2.13);

\path[fill=fillColor,fill opacity=0.20] ( 58.93, 61.04) circle (  2.13);

\path[fill=fillColor,fill opacity=0.20] ( 66.36, 55.43) circle (  2.13);

\path[fill=fillColor,fill opacity=0.20] ( 90.33, 62.18) circle (  2.13);

\path[fill=fillColor,fill opacity=0.20] (142.50, 88.96) circle (  2.13);

\path[fill=fillColor,fill opacity=0.20] ( 75.29, 57.20) circle (  2.13);

\path[fill=fillColor,fill opacity=0.20] ( 69.27, 46.19) circle (  2.13);

\path[fill=fillColor,fill opacity=0.20] ( 65.86, 53.46) circle (  2.13);

\path[fill=fillColor,fill opacity=0.20] ( 64.95, 65.29) circle (  2.13);

\path[fill=fillColor,fill opacity=0.20] ( 64.85, 67.16) circle (  2.13);

\path[fill=fillColor,fill opacity=0.20] ( 68.06, 49.20) circle (  2.13);

\path[fill=fillColor,fill opacity=0.20] ( 65.76, 42.87) circle (  2.13);

\path[fill=fillColor,fill opacity=0.20] ( 62.95, 54.91) circle (  2.13);

\path[fill=fillColor,fill opacity=0.20] ( 68.26, 55.12) circle (  2.13);

\path[fill=fillColor,fill opacity=0.20] ( 89.33,104.54) circle (  2.13);

\path[fill=fillColor,fill opacity=0.20] ( 93.34, 61.24) circle (  2.13);

\path[fill=fillColor,fill opacity=0.20] ( 94.35, 60.31) circle (  2.13);

\path[fill=fillColor,fill opacity=0.20] ( 88.33, 67.79) circle (  2.13);

\path[fill=fillColor,fill opacity=0.20] ( 87.33, 50.86) circle (  2.13);

\path[fill=fillColor,fill opacity=0.20] ( 83.31, 43.70) circle (  2.13);

\path[fill=fillColor,fill opacity=0.20] (104.38, 58.44) circle (  2.13);

\path[fill=fillColor,fill opacity=0.20] ( 99.36, 53.56) circle (  2.13);

\path[fill=fillColor,fill opacity=0.20] (103.38, 52.21) circle (  2.13);

\path[fill=fillColor,fill opacity=0.20] ( 81.31, 70.17) circle (  2.13);

\path[fill=fillColor,fill opacity=0.20] ( 74.28, 58.03) circle (  2.13);

\path[fill=fillColor,fill opacity=0.20] ( 70.27, 53.46) circle (  2.13);

\path[fill=fillColor,fill opacity=0.20] ( 70.27, 63.22) circle (  2.13);

\path[fill=fillColor,fill opacity=0.20] ( 66.56, 67.27) circle (  2.13);

\path[fill=fillColor,fill opacity=0.20] ( 61.64, 60.31) circle (  2.13);

\path[fill=fillColor,fill opacity=0.20] ( 61.74, 54.39) circle (  2.13);

\path[fill=fillColor,fill opacity=0.20] ( 58.83, 49.82) circle (  2.13);

\path[fill=fillColor,fill opacity=0.20] ( 64.65, 54.60) circle (  2.13);

\path[fill=fillColor,fill opacity=0.20] ( 72.28, 97.27) circle (  2.13);

\path[fill=fillColor,fill opacity=0.20] ( 69.27, 60.21) circle (  2.13);

\path[fill=fillColor,fill opacity=0.20] ( 70.27, 45.88) circle (  2.13);

\path[fill=fillColor,fill opacity=0.20] ( 96.35, 40.38) circle (  2.13);

\path[fill=fillColor,fill opacity=0.20] ( 90.33, 52.52) circle (  2.13);

\path[fill=fillColor,fill opacity=0.20] ( 94.35, 69.03) circle (  2.13);

\path[fill=fillColor,fill opacity=0.20] ( 96.35, 69.45) circle (  2.13);

\path[fill=fillColor,fill opacity=0.20] ( 81.31, 81.70) circle (  2.13);

\path[fill=fillColor,fill opacity=0.20] ( 78.30, 64.15) circle (  2.13);

\path[fill=fillColor,fill opacity=0.20] ( 74.28, 69.45) circle (  2.13);

\path[fill=fillColor,fill opacity=0.20] ( 74.28, 58.86) circle (  2.13);

\path[fill=fillColor,fill opacity=0.20] ( 74.28, 47.02) circle (  2.13);

\path[fill=fillColor,fill opacity=0.20] ( 70.27, 48.16) circle (  2.13);

\path[fill=fillColor,fill opacity=0.20] ( 65.45, 59.06) circle (  2.13);

\path[fill=fillColor,fill opacity=0.20] ( 59.23, 61.45) circle (  2.13);

\path[fill=fillColor,fill opacity=0.20] ( 58.73, 48.79) circle (  2.13);

\path[fill=fillColor,fill opacity=0.20] ( 75.29, 46.81) circle (  2.13);

\path[fill=fillColor,fill opacity=0.20] ( 72.28, 41.52) circle (  2.13);

\path[fill=fillColor,fill opacity=0.20] ( 74.28, 57.92) circle (  2.13);

\path[fill=fillColor,fill opacity=0.20] ( 77.29, 60.41) circle (  2.13);

\path[fill=fillColor,fill opacity=0.20] ( 79.30, 57.40) circle (  2.13);

\path[fill=fillColor,fill opacity=0.20] ( 82.31, 58.86) circle (  2.13);

\path[fill=fillColor,fill opacity=0.20] ( 96.35, 57.30) circle (  2.13);

\path[fill=fillColor,fill opacity=0.20] ( 90.33, 59.58) circle (  2.13);

\path[fill=fillColor,fill opacity=0.20] ( 89.33, 65.40) circle (  2.13);

\path[fill=fillColor,fill opacity=0.20] ( 72.28, 51.38) circle (  2.13);

\path[fill=fillColor,fill opacity=0.20] ( 73.28, 43.49) circle (  2.13);

\path[fill=fillColor,fill opacity=0.20] ( 69.27, 55.53) circle (  2.13);

\path[fill=fillColor,fill opacity=0.20] ( 67.96, 64.67) circle (  2.13);

\path[fill=fillColor,fill opacity=0.20] ( 71.27, 44.94) circle (  2.13);

\path[fill=fillColor,fill opacity=0.20] ( 67.06, 37.99) circle (  2.13);

\path[fill=fillColor,fill opacity=0.20] ( 61.24, 53.98) circle (  2.13);

\path[fill=fillColor,fill opacity=0.20] ( 61.44, 56.47) circle (  2.13);

\path[fill=fillColor,fill opacity=0.20] ( 69.27, 48.16) circle (  2.13);

\path[fill=fillColor,fill opacity=0.20] ( 88.33, 46.50) circle (  2.13);

\path[fill=fillColor,fill opacity=0.20] ( 72.28, 85.85) circle (  2.13);

\path[fill=fillColor,fill opacity=0.20] ( 72.28, 44.32) circle (  2.13);

\path[fill=fillColor,fill opacity=0.20] ( 72.28, 60.41) circle (  2.13);

\path[fill=fillColor,fill opacity=0.20] ( 72.28, 73.81) circle (  2.13);

\path[fill=fillColor,fill opacity=0.20] ( 77.29, 64.05) circle (  2.13);

\path[fill=fillColor,fill opacity=0.20] ( 77.29, 65.81) circle (  2.13);

\path[fill=fillColor,fill opacity=0.20] ( 91.34, 67.58) circle (  2.13);

\path[fill=fillColor,fill opacity=0.20] ( 91.34, 53.87) circle (  2.13);

\path[fill=fillColor,fill opacity=0.20] ( 86.32, 47.75) circle (  2.13);

\path[fill=fillColor,fill opacity=0.20] ( 90.33, 53.67) circle (  2.13);

\path[fill=fillColor,fill opacity=0.20] ( 87.33, 87.93) circle (  2.13);

\path[fill=fillColor,fill opacity=0.20] ( 78.30, 53.35) circle (  2.13);

\path[fill=fillColor,fill opacity=0.20] ( 66.86, 58.75) circle (  2.13);

\path[fill=fillColor,fill opacity=0.20] ( 70.27, 47.02) circle (  2.13);

\path[fill=fillColor,fill opacity=0.20] ( 70.27, 56.36) circle (  2.13);

\path[fill=fillColor,fill opacity=0.20] ( 60.24, 72.56) circle (  2.13);

\path[fill=fillColor,fill opacity=0.20] ( 63.95, 68.20) circle (  2.13);

\path[fill=fillColor,fill opacity=0.20] ( 63.75, 64.05) circle (  2.13);

\path[fill=fillColor,fill opacity=0.20] ( 59.84, 63.01) circle (  2.13);

\path[fill=fillColor,fill opacity=0.20] ( 62.55, 58.13) circle (  2.13);

\path[fill=fillColor,fill opacity=0.20] ( 84.32, 50.65) circle (  2.13);

\path[fill=fillColor,fill opacity=0.20] ( 79.30, 54.39) circle (  2.13);

\path[fill=fillColor,fill opacity=0.20] ( 70.27, 42.87) circle (  2.13);

\path[fill=fillColor,fill opacity=0.20] ( 56.02, 67.99) circle (  2.13);

\path[fill=fillColor,fill opacity=0.20] ( 77.29, 63.74) circle (  2.13);

\path[fill=fillColor,fill opacity=0.20] ( 75.29, 54.29) circle (  2.13);

\path[fill=fillColor,fill opacity=0.20] ( 82.31, 68.51) circle (  2.13);

\path[fill=fillColor,fill opacity=0.20] ( 88.33, 69.03) circle (  2.13);

\path[fill=fillColor,fill opacity=0.20] ( 86.32, 57.09) circle (  2.13);

\path[fill=fillColor,fill opacity=0.20] ( 91.34, 54.81) circle (  2.13);

\path[fill=fillColor,fill opacity=0.20] ( 86.32,106.61) circle (  2.13);

\path[fill=fillColor,fill opacity=0.20] ( 83.31, 55.43) circle (  2.13);

\path[fill=fillColor,fill opacity=0.20] ( 73.28, 70.38) circle (  2.13);

\path[fill=fillColor,fill opacity=0.20] ( 55.42, 68.62) circle (  2.13);

\path[fill=fillColor,fill opacity=0.20] ( 66.36, 59.38) circle (  2.13);

\path[fill=fillColor,fill opacity=0.20] ( 68.26, 62.91) circle (  2.13);

\path[fill=fillColor,fill opacity=0.20] ( 59.13, 63.22) circle (  2.13);

\path[fill=fillColor,fill opacity=0.20] ( 49.10, 68.62) circle (  2.13);

\path[fill=fillColor,fill opacity=0.20] ( 59.74, 61.04) circle (  2.13);

\path[fill=fillColor,fill opacity=0.20] ( 68.26, 53.77) circle (  2.13);

\path[fill=fillColor,fill opacity=0.20] ( 68.26,114.92) circle (  2.13);

\path[fill=fillColor,fill opacity=0.20] ( 74.28, 57.92) circle (  2.13);

\path[fill=fillColor,fill opacity=0.20] ( 68.26, 48.99) circle (  2.13);

\path[fill=fillColor,fill opacity=0.20] ( 65.96, 53.56) circle (  2.13);

\path[fill=fillColor,fill opacity=0.20] ( 79.30, 39.65) circle (  2.13);

\path[fill=fillColor,fill opacity=0.20] ( 81.31, 42.56) circle (  2.13);

\path[fill=fillColor,fill opacity=0.20] ( 84.32, 64.46) circle (  2.13);

\path[fill=fillColor,fill opacity=0.20] ( 93.34, 66.12) circle (  2.13);

\path[fill=fillColor,fill opacity=0.20] ( 97.36, 60.93) circle (  2.13);

\path[fill=fillColor,fill opacity=0.20] (100.37, 64.67) circle (  2.13);

\path[fill=fillColor,fill opacity=0.20] ( 78.30,108.69) circle (  2.13);

\path[fill=fillColor,fill opacity=0.20] ( 88.33, 61.87) circle (  2.13);

\path[fill=fillColor,fill opacity=0.20] ( 81.31, 70.48) circle (  2.13);

\path[fill=fillColor,fill opacity=0.20] ( 71.27, 70.59) circle (  2.13);

\path[fill=fillColor,fill opacity=0.20] ( 68.26, 62.70) circle (  2.13);

\path[fill=fillColor,fill opacity=0.20] ( 69.27, 61.66) circle (  2.13);

\path[fill=fillColor,fill opacity=0.20] ( 64.05, 56.26) circle (  2.13);

\path[fill=fillColor,fill opacity=0.20] ( 62.34, 41.93) circle (  2.13);

\path[fill=fillColor,fill opacity=0.20] ( 46.39, 45.05) circle (  2.13);

\path[fill=fillColor,fill opacity=0.20] ( 57.03, 61.35) circle (  2.13);

\path[fill=fillColor,fill opacity=0.20] ( 60.54, 49.72) circle (  2.13);

\path[fill=fillColor,fill opacity=0.20] ( 89.33, 73.18) circle (  2.13);

\path[fill=fillColor,fill opacity=0.20] ( 67.66, 59.27) circle (  2.13);

\path[fill=fillColor,fill opacity=0.20] ( 62.75, 53.25) circle (  2.13);

\path[fill=fillColor,fill opacity=0.20] ( 76.29, 45.57) circle (  2.13);

\path[fill=fillColor,fill opacity=0.20] ( 80.30, 50.45) circle (  2.13);

\path[fill=fillColor,fill opacity=0.20] ( 85.32, 59.38) circle (  2.13);

\path[fill=fillColor,fill opacity=0.20] ( 94.35, 61.45) circle (  2.13);

\path[fill=fillColor,fill opacity=0.20] ( 98.36, 60.21) circle (  2.13);

\path[fill=fillColor,fill opacity=0.20] (101.37, 56.47) circle (  2.13);

\path[fill=fillColor,fill opacity=0.20] ( 92.34,108.69) circle (  2.13);

\path[fill=fillColor,fill opacity=0.20] ( 82.31, 63.94) circle (  2.13);

\path[fill=fillColor,fill opacity=0.20] ( 80.30, 78.48) circle (  2.13);

\path[fill=fillColor,fill opacity=0.20] ( 77.29, 78.06) circle (  2.13);

\path[fill=fillColor,fill opacity=0.20] ( 74.28, 63.22) circle (  2.13);

\path[fill=fillColor,fill opacity=0.20] ( 72.28, 61.76) circle (  2.13);

\path[fill=fillColor,fill opacity=0.20] ( 70.27, 64.15) circle (  2.13);

\path[fill=fillColor,fill opacity=0.20] ( 64.95, 57.09) circle (  2.13);

\path[fill=fillColor,fill opacity=0.20] ( 66.56, 45.46) circle (  2.13);

\path[fill=fillColor,fill opacity=0.20] ( 66.16, 59.69) circle (  2.13);

\path[fill=fillColor,fill opacity=0.20] ( 77.29, 55.64) circle (  2.13);

\path[fill=fillColor,fill opacity=0.20] (101.37, 85.85) circle (  2.13);

\path[fill=fillColor,fill opacity=0.20] ( 78.30, 58.86) circle (  2.13);

\path[fill=fillColor,fill opacity=0.20] ( 61.14, 59.79) circle (  2.13);

\path[fill=fillColor,fill opacity=0.20] ( 63.85, 68.62) circle (  2.13);

\path[fill=fillColor,fill opacity=0.20] ( 79.30, 66.54) circle (  2.13);

\path[fill=fillColor,fill opacity=0.20] ( 79.30, 61.76) circle (  2.13);

\path[fill=fillColor,fill opacity=0.20] ( 87.33, 64.88) circle (  2.13);

\path[fill=fillColor,fill opacity=0.20] ( 86.32, 61.76) circle (  2.13);

\path[fill=fillColor,fill opacity=0.20] ( 94.35, 51.28) circle (  2.13);

\path[fill=fillColor,fill opacity=0.20] (113.41, 73.60) circle (  2.13);

\path[fill=fillColor,fill opacity=0.20] ( 96.35, 99.35) circle (  2.13);

\path[fill=fillColor,fill opacity=0.20] ( 89.33, 66.95) circle (  2.13);

\path[fill=fillColor,fill opacity=0.20] ( 81.31, 71.73) circle (  2.13);

\path[fill=fillColor,fill opacity=0.20] ( 74.28, 68.72) circle (  2.13);

\path[fill=fillColor,fill opacity=0.20] ( 69.27, 70.48) circle (  2.13);

\path[fill=fillColor,fill opacity=0.20] ( 67.36, 66.12) circle (  2.13);

\path[fill=fillColor,fill opacity=0.20] ( 76.29, 61.87) circle (  2.13);

\path[fill=fillColor,fill opacity=0.20] ( 72.28, 62.49) circle (  2.13);

\path[fill=fillColor,fill opacity=0.20] ( 66.06, 60.93) circle (  2.13);

\path[fill=fillColor,fill opacity=0.20] ( 70.27, 60.21) circle (  2.13);

\path[fill=fillColor,fill opacity=0.20] ( 73.28, 61.76) circle (  2.13);

\path[fill=fillColor,fill opacity=0.20] ( 78.30, 58.86) circle (  2.13);

\path[fill=fillColor,fill opacity=0.20] ( 97.36, 67.06) circle (  2.13);

\path[fill=fillColor,fill opacity=0.20] ( 72.28, 61.35) circle (  2.13);

\path[fill=fillColor,fill opacity=0.20] ( 70.27, 65.61) circle (  2.13);

\path[fill=fillColor,fill opacity=0.20] ( 73.28, 59.79) circle (  2.13);

\path[fill=fillColor,fill opacity=0.20] ( 73.28, 64.67) circle (  2.13);

\path[fill=fillColor,fill opacity=0.20] ( 73.28, 71.00) circle (  2.13);

\path[fill=fillColor,fill opacity=0.20] ( 78.30, 62.70) circle (  2.13);

\path[fill=fillColor,fill opacity=0.20] ( 87.33, 61.66) circle (  2.13);

\path[fill=fillColor,fill opacity=0.20] ( 98.36, 74.43) circle (  2.13);

\path[fill=fillColor,fill opacity=0.20] (102.37, 94.16) circle (  2.13);

\path[fill=fillColor,fill opacity=0.20] ( 91.34, 60.93) circle (  2.13);

\path[fill=fillColor,fill opacity=0.20] ( 75.29, 75.47) circle (  2.13);

\path[fill=fillColor,fill opacity=0.20] ( 77.29, 67.27) circle (  2.13);

\path[fill=fillColor,fill opacity=0.20] ( 78.30, 53.15) circle (  2.13);

\path[fill=fillColor,fill opacity=0.20] ( 68.26, 59.89) circle (  2.13);

\path[fill=fillColor,fill opacity=0.20] ( 67.26, 55.33) circle (  2.13);

\path[fill=fillColor,fill opacity=0.20] ( 73.28, 45.57) circle (  2.13);

\path[fill=fillColor,fill opacity=0.20] ( 69.27, 47.64) circle (  2.13);

\path[fill=fillColor,fill opacity=0.20] ( 75.29, 54.50) circle (  2.13);

\path[fill=fillColor,fill opacity=0.20] ( 84.32, 61.14) circle (  2.13);

\path[fill=fillColor,fill opacity=0.20] ( 91.34, 61.76) circle (  2.13);

\path[fill=fillColor,fill opacity=0.20] (113.41, 84.81) circle (  2.13);

\path[fill=fillColor,fill opacity=0.20] ( 83.31, 78.89) circle (  2.13);

\path[fill=fillColor,fill opacity=0.20] ( 67.66, 61.04) circle (  2.13);

\path[fill=fillColor,fill opacity=0.20] ( 74.28, 48.06) circle (  2.13);

\path[fill=fillColor,fill opacity=0.20] ( 72.28, 63.84) circle (  2.13);

\path[fill=fillColor,fill opacity=0.20] ( 72.28, 67.79) circle (  2.13);

\path[fill=fillColor,fill opacity=0.20] ( 77.29, 55.33) circle (  2.13);

\path[fill=fillColor,fill opacity=0.20] ( 82.31, 61.24) circle (  2.13);

\path[fill=fillColor,fill opacity=0.20] ( 89.33, 71.94) circle (  2.13);

\path[fill=fillColor,fill opacity=0.20] (104.38, 77.03) circle (  2.13);

\path[fill=fillColor,fill opacity=0.20] (117.42,113.88) circle (  2.13);

\path[fill=fillColor,fill opacity=0.20] ( 94.35, 86.89) circle (  2.13);

\path[fill=fillColor,fill opacity=0.20] ( 89.33, 68.93) circle (  2.13);

\path[fill=fillColor,fill opacity=0.20] ( 82.31, 75.47) circle (  2.13);

\path[fill=fillColor,fill opacity=0.20] ( 75.29, 73.60) circle (  2.13);

\path[fill=fillColor,fill opacity=0.20] ( 71.27, 63.63) circle (  2.13);

\path[fill=fillColor,fill opacity=0.20] ( 70.27, 48.58) circle (  2.13);

\path[fill=fillColor,fill opacity=0.20] ( 71.27, 39.55) circle (  2.13);

\path[fill=fillColor,fill opacity=0.20] ( 70.27, 47.23) circle (  2.13);

\path[fill=fillColor,fill opacity=0.20] ( 88.33, 52.63) circle (  2.13);

\path[fill=fillColor,fill opacity=0.20] ( 89.33, 62.70) circle (  2.13);

\path[fill=fillColor,fill opacity=0.20] ( 83.31, 72.77) circle (  2.13);

\path[fill=fillColor,fill opacity=0.20] ( 91.34, 92.08) circle (  2.13);

\path[fill=fillColor,fill opacity=0.20] ( 67.66, 52.11) circle (  2.13);

\path[fill=fillColor,fill opacity=0.20] ( 74.28, 39.55) circle (  2.13);

\path[fill=fillColor,fill opacity=0.20] ( 67.66, 64.98) circle (  2.13);

\path[fill=fillColor,fill opacity=0.20] ( 71.27, 69.34) circle (  2.13);

\path[fill=fillColor,fill opacity=0.20] ( 83.31, 54.60) circle (  2.13);

\path[fill=fillColor,fill opacity=0.20] ( 83.31, 59.48) circle (  2.13);

\path[fill=fillColor,fill opacity=0.20] ( 84.32, 72.15) circle (  2.13);

\path[fill=fillColor,fill opacity=0.20] ( 91.34, 68.72) circle (  2.13);

\path[fill=fillColor,fill opacity=0.20] ( 91.34, 71.21) circle (  2.13);

\path[fill=fillColor,fill opacity=0.20] (101.37,105.58) circle (  2.13);

\path[fill=fillColor,fill opacity=0.20] ( 78.30, 99.35) circle (  2.13);

\path[fill=fillColor,fill opacity=0.20] ( 92.34, 72.46) circle (  2.13);

\path[fill=fillColor,fill opacity=0.20] ( 77.29, 77.23) circle (  2.13);

\path[fill=fillColor,fill opacity=0.20] ( 75.29, 83.77) circle (  2.13);

\path[fill=fillColor,fill opacity=0.20] ( 73.28, 65.71) circle (  2.13);

\path[fill=fillColor,fill opacity=0.20] ( 72.28, 48.89) circle (  2.13);

\path[fill=fillColor,fill opacity=0.20] ( 67.46, 40.38) circle (  2.13);

\path[fill=fillColor,fill opacity=0.20] ( 79.30, 46.40) circle (  2.13);

\path[fill=fillColor,fill opacity=0.20] ( 81.31, 42.14) circle (  2.13);

\path[fill=fillColor,fill opacity=0.20] ( 55.12, 65.81) circle (  2.13);

\path[fill=fillColor,fill opacity=0.20] ( 75.29, 61.14) circle (  2.13);

\path[fill=fillColor,fill opacity=0.20] ( 81.31, 63.74) circle (  2.13);

\path[fill=fillColor,fill opacity=0.20] ( 86.32, 64.46) circle (  2.13);

\path[fill=fillColor,fill opacity=0.20] ( 79.30, 64.15) circle (  2.13);

\path[fill=fillColor,fill opacity=0.20] ( 82.31, 61.66) circle (  2.13);

\path[fill=fillColor,fill opacity=0.20] ( 93.34, 61.14) circle (  2.13);

\path[fill=fillColor,fill opacity=0.20] ( 94.35, 95.19) circle (  2.13);

\path[fill=fillColor,fill opacity=0.20] (104.38,105.58) circle (  2.13);

\path[fill=fillColor,fill opacity=0.20] ( 93.34, 96.23) circle (  2.13);

\path[fill=fillColor,fill opacity=0.20] ( 88.33, 75.47) circle (  2.13);

\path[fill=fillColor,fill opacity=0.20] ( 76.29, 66.95) circle (  2.13);

\path[fill=fillColor,fill opacity=0.20] ( 75.29, 79.52) circle (  2.13);

\path[fill=fillColor,fill opacity=0.20] ( 71.27, 79.00) circle (  2.13);

\path[fill=fillColor,fill opacity=0.20] ( 66.76, 60.10) circle (  2.13);

\path[fill=fillColor,fill opacity=0.20] ( 58.53, 47.33) circle (  2.13);

\path[fill=fillColor,fill opacity=0.20] ( 84.32, 37.99) circle (  2.13);

\path[fill=fillColor,fill opacity=0.20] ( 84.32, 68.51) circle (  2.13);

\path[fill=fillColor,fill opacity=0.20] ( 77.29, 59.06) circle (  2.13);

\path[fill=fillColor,fill opacity=0.20] ( 74.28, 56.68) circle (  2.13);

\path[fill=fillColor,fill opacity=0.20] ( 77.29, 64.05) circle (  2.13);

\path[fill=fillColor,fill opacity=0.20] ( 80.30, 67.16) circle (  2.13);

\path[fill=fillColor,fill opacity=0.20] ( 84.32, 71.32) circle (  2.13);

\path[fill=fillColor,fill opacity=0.20] ( 79.30, 75.68) circle (  2.13);

\path[fill=fillColor,fill opacity=0.20] ( 86.32, 64.67) circle (  2.13);

\path[fill=fillColor,fill opacity=0.20] ( 90.33, 62.91) circle (  2.13);

\path[fill=fillColor,fill opacity=0.20] (100.37, 87.93) circle (  2.13);

\path[fill=fillColor,fill opacity=0.20] ( 86.32, 82.74) circle (  2.13);

\path[fill=fillColor,fill opacity=0.20] ( 89.33, 83.77) circle (  2.13);

\path[fill=fillColor,fill opacity=0.20] ( 91.34, 81.70) circle (  2.13);

\path[fill=fillColor,fill opacity=0.20] ( 81.31, 69.24) circle (  2.13);

\path[fill=fillColor,fill opacity=0.20] ( 78.30, 63.32) circle (  2.13);

\path[fill=fillColor,fill opacity=0.20] ( 75.29, 64.36) circle (  2.13);

\path[fill=fillColor,fill opacity=0.20] ( 69.27, 68.10) circle (  2.13);

\path[fill=fillColor,fill opacity=0.20] ( 74.28, 61.45) circle (  2.13);

\path[fill=fillColor,fill opacity=0.20] ( 84.32, 53.35) circle (  2.13);

\path[fill=fillColor,fill opacity=0.20] ( 78.30, 53.25) circle (  2.13);

\path[fill=fillColor,fill opacity=0.20] ( 72.28, 61.66) circle (  2.13);

\path[fill=fillColor,fill opacity=0.20] ( 79.30, 69.86) circle (  2.13);

\path[fill=fillColor,fill opacity=0.20] ( 83.31, 77.13) circle (  2.13);

\path[fill=fillColor,fill opacity=0.20] ( 85.32, 73.08) circle (  2.13);

\path[fill=fillColor,fill opacity=0.20] ( 58.93, 54.81) circle (  2.13);

\path[fill=fillColor,fill opacity=0.20] ( 91.34, 54.50) circle (  2.13);

\path[fill=fillColor,fill opacity=0.20] ( 96.35, 73.60) circle (  2.13);

\path[fill=fillColor,fill opacity=0.20] (104.38, 92.08) circle (  2.13);

\path[fill=fillColor,fill opacity=0.20] (106.39, 99.35) circle (  2.13);

\path[fill=fillColor,fill opacity=0.20] (121.43,111.81) circle (  2.13);

\path[fill=fillColor,fill opacity=0.20] ( 93.34, 87.93) circle (  2.13);

\path[fill=fillColor,fill opacity=0.20] ( 89.33, 67.99) circle (  2.13);

\path[fill=fillColor,fill opacity=0.20] ( 85.32, 80.24) circle (  2.13);

\path[fill=fillColor,fill opacity=0.20] ( 84.32, 72.87) circle (  2.13);

\path[fill=fillColor,fill opacity=0.20] ( 77.29, 63.74) circle (  2.13);

\path[fill=fillColor,fill opacity=0.20] ( 71.27, 73.29) circle (  2.13);

\path[fill=fillColor,fill opacity=0.20] ( 65.15, 70.90) circle (  2.13);

\path[fill=fillColor,fill opacity=0.20] ( 82.31, 50.97) circle (  2.13);

\path[fill=fillColor,fill opacity=0.20] ( 90.33, 63.84) circle (  2.13);

\path[fill=fillColor,fill opacity=0.20] ( 82.31, 46.92) circle (  2.13);

\path[fill=fillColor,fill opacity=0.20] ( 83.31, 45.26) circle (  2.13);

\path[fill=fillColor,fill opacity=0.20] ( 85.32, 62.49) circle (  2.13);

\path[fill=fillColor,fill opacity=0.20] ( 85.32, 64.88) circle (  2.13);

\path[fill=fillColor,fill opacity=0.20] ( 90.33, 50.03) circle (  2.13);

\path[fill=fillColor,fill opacity=0.20] ( 94.35, 53.87) circle (  2.13);

\path[fill=fillColor,fill opacity=0.20] ( 86.32, 66.44) circle (  2.13);

\path[fill=fillColor,fill opacity=0.20] ( 91.34, 69.24) circle (  2.13);

\path[fill=fillColor,fill opacity=0.20] ( 99.36, 72.25) circle (  2.13);

\path[fill=fillColor,fill opacity=0.20] ( 98.36, 71.94) circle (  2.13);

\path[fill=fillColor,fill opacity=0.20] (107.39, 83.77) circle (  2.13);

\path[fill=fillColor,fill opacity=0.20] (101.37,112.84) circle (  2.13);

\path[fill=fillColor,fill opacity=0.20] ( 89.33,110.77) circle (  2.13);

\path[fill=fillColor,fill opacity=0.20] ( 83.31,111.81) circle (  2.13);

\path[fill=fillColor,fill opacity=0.20] ( 90.33, 87.93) circle (  2.13);

\path[fill=fillColor,fill opacity=0.20] ( 88.33, 61.35) circle (  2.13);

\path[fill=fillColor,fill opacity=0.20] ( 79.30, 68.20) circle (  2.13);

\path[fill=fillColor,fill opacity=0.20] ( 81.31, 73.50) circle (  2.13);

\path[fill=fillColor,fill opacity=0.20] ( 71.27, 66.12) circle (  2.13);

\path[fill=fillColor,fill opacity=0.20] ( 65.25, 66.12) circle (  2.13);

\path[fill=fillColor,fill opacity=0.20] ( 72.28, 56.36) circle (  2.13);

\path[fill=fillColor,fill opacity=0.20] ( 81.31, 44.94) circle (  2.13);

\path[fill=fillColor,fill opacity=0.20] ( 79.30, 55.33) circle (  2.13);

\path[fill=fillColor,fill opacity=0.20] ( 82.31, 62.80) circle (  2.13);

\path[fill=fillColor,fill opacity=0.20] ( 87.33, 63.74) circle (  2.13);

\path[fill=fillColor,fill opacity=0.20] ( 82.31, 69.34) circle (  2.13);

\path[fill=fillColor,fill opacity=0.20] ( 91.34, 68.30) circle (  2.13);

\path[fill=fillColor,fill opacity=0.20] ( 97.36, 62.91) circle (  2.13);

\path[fill=fillColor,fill opacity=0.20] ( 95.35, 64.98) circle (  2.13);

\path[fill=fillColor,fill opacity=0.20] ( 96.35, 64.67) circle (  2.13);

\path[fill=fillColor,fill opacity=0.20] (104.38, 61.35) circle (  2.13);

\path[fill=fillColor,fill opacity=0.20] ( 98.36, 68.82) circle (  2.13);

\path[fill=fillColor,fill opacity=0.20] ( 96.35, 87.93) circle (  2.13);

\path[fill=fillColor,fill opacity=0.20] ( 94.35,108.69) circle (  2.13);

\path[fill=fillColor,fill opacity=0.20] ( 49.00,111.81) circle (  2.13);

\path[fill=fillColor,fill opacity=0.20] (109.40,115.96) circle (  2.13);

\path[fill=fillColor,fill opacity=0.20] ( 92.34,106.61) circle (  2.13);

\path[fill=fillColor,fill opacity=0.20] ( 99.36, 92.08) circle (  2.13);

\path[fill=fillColor,fill opacity=0.20] ( 94.35,100.39) circle (  2.13);

\path[fill=fillColor,fill opacity=0.20] ( 86.32,102.46) circle (  2.13);

\path[fill=fillColor,fill opacity=0.20] ( 92.34, 75.47) circle (  2.13);

\path[fill=fillColor,fill opacity=0.20] ( 99.36, 43.28) circle (  2.13);

\path[fill=fillColor,fill opacity=0.20] (101.37, 47.64) circle (  2.13);

\path[fill=fillColor,fill opacity=0.20] ( 91.34, 56.78) circle (  2.13);

\path[fill=fillColor,fill opacity=0.20] ( 89.33, 53.87) circle (  2.13);

\path[fill=fillColor,fill opacity=0.20] ( 91.34, 65.61) circle (  2.13);

\path[fill=fillColor,fill opacity=0.20] ( 87.33, 75.68) circle (  2.13);

\path[fill=fillColor,fill opacity=0.20] ( 79.30, 58.86) circle (  2.13);

\path[fill=fillColor,fill opacity=0.20] ( 74.28, 41.93) circle (  2.13);

\path[fill=fillColor,fill opacity=0.20] ( 81.31, 40.69) circle (  2.13);

\path[fill=fillColor,fill opacity=0.20] ( 81.31, 58.75) circle (  2.13);

\path[fill=fillColor,fill opacity=0.20] ( 82.31, 63.63) circle (  2.13);

\path[fill=fillColor,fill opacity=0.20] ( 81.31, 56.99) circle (  2.13);

\path[fill=fillColor,fill opacity=0.20] ( 83.31, 63.42) circle (  2.13);

\path[fill=fillColor,fill opacity=0.20] ( 93.34, 67.99) circle (  2.13);

\path[fill=fillColor,fill opacity=0.20] ( 92.34, 60.83) circle (  2.13);

\path[fill=fillColor,fill opacity=0.20] ( 95.35, 65.09) circle (  2.13);

\path[fill=fillColor,fill opacity=0.20] (101.37, 63.32) circle (  2.13);

\path[fill=fillColor,fill opacity=0.20] (109.40, 51.28) circle (  2.13);

\path[fill=fillColor,fill opacity=0.20] ( 96.35, 54.70) circle (  2.13);

\path[fill=fillColor,fill opacity=0.20] ( 85.32, 60.41) circle (  2.13);

\path[fill=fillColor,fill opacity=0.20] ( 90.33, 58.96) circle (  2.13);

\path[fill=fillColor,fill opacity=0.20] ( 87.33, 61.04) circle (  2.13);

\path[fill=fillColor,fill opacity=0.20] ( 78.30, 67.16) circle (  2.13);

\path[fill=fillColor,fill opacity=0.20] ( 89.33, 70.38) circle (  2.13);

\path[fill=fillColor,fill opacity=0.20] ( 96.35, 68.51) circle (  2.13);

\path[fill=fillColor,fill opacity=0.20] (108.39, 61.35) circle (  2.13);

\path[fill=fillColor,fill opacity=0.20] (101.37, 59.06) circle (  2.13);

\path[fill=fillColor,fill opacity=0.20] ( 83.31, 66.64) circle (  2.13);

\path[fill=fillColor,fill opacity=0.20] ( 90.33, 69.55) circle (  2.13);

\path[fill=fillColor,fill opacity=0.20] ( 96.35, 65.92) circle (  2.13);

\path[fill=fillColor,fill opacity=0.20] ( 96.35, 66.23) circle (  2.13);

\path[fill=fillColor,fill opacity=0.20] ( 91.34, 66.23) circle (  2.13);

\path[fill=fillColor,fill opacity=0.20] ( 97.36, 59.79) circle (  2.13);

\path[fill=fillColor,fill opacity=0.20] (100.37, 52.84) circle (  2.13);

\path[fill=fillColor,fill opacity=0.20] ( 90.33, 57.20) circle (  2.13);

\path[fill=fillColor,fill opacity=0.20] ( 89.33, 62.08) circle (  2.13);

\path[fill=fillColor,fill opacity=0.20] (102.37, 57.30) circle (  2.13);

\path[fill=fillColor,fill opacity=0.20] (101.37, 55.43) circle (  2.13);

\path[fill=fillColor,fill opacity=0.20] (101.37, 55.02) circle (  2.13);

\path[fill=fillColor,fill opacity=0.20] ( 91.34, 53.87) circle (  2.13);

\path[fill=fillColor,fill opacity=0.20] ( 87.33, 65.50) circle (  2.13);

\path[fill=fillColor,fill opacity=0.20] ( 82.31, 72.87) circle (  2.13);

\path[fill=fillColor,fill opacity=0.20] ( 85.32, 51.28) circle (  2.13);

\path[fill=fillColor,fill opacity=0.20] ( 92.34, 56.57) circle (  2.13);

\path[fill=fillColor,fill opacity=0.20] ( 87.33, 48.27) circle (  2.13);

\path[fill=fillColor,fill opacity=0.20] ( 81.31, 54.39) circle (  2.13);

\path[fill=fillColor,fill opacity=0.20] ( 83.31, 59.06) circle (  2.13);

\path[fill=fillColor,fill opacity=0.20] ( 89.33, 64.67) circle (  2.13);

\path[fill=fillColor,fill opacity=0.20] ( 84.32, 68.20) circle (  2.13);

\path[fill=fillColor,fill opacity=0.20] ( 87.33, 58.75) circle (  2.13);

\path[fill=fillColor,fill opacity=0.20] ( 94.35, 54.18) circle (  2.13);

\path[fill=fillColor,fill opacity=0.20] ( 97.36, 65.61) circle (  2.13);

\path[fill=fillColor,fill opacity=0.20] ( 97.36, 67.58) circle (  2.13);

\path[fill=fillColor,fill opacity=0.20] (103.38, 64.15) circle (  2.13);

\path[fill=fillColor,fill opacity=0.20] ( 91.34, 61.97) circle (  2.13);

\path[fill=fillColor,fill opacity=0.20] ( 82.31, 57.82) circle (  2.13);

\path[fill=fillColor,fill opacity=0.20] ( 87.33, 68.30) circle (  2.13);

\path[fill=fillColor,fill opacity=0.20] ( 94.35, 79.83) circle (  2.13);

\path[fill=fillColor,fill opacity=0.20] ( 97.36, 68.72) circle (  2.13);

\path[fill=fillColor,fill opacity=0.20] ( 95.35, 57.20) circle (  2.13);

\path[fill=fillColor,fill opacity=0.20] ( 93.34, 61.04) circle (  2.13);

\path[fill=fillColor,fill opacity=0.20] ( 99.36, 57.51) circle (  2.13);

\path[fill=fillColor,fill opacity=0.20] ( 95.35, 49.10) circle (  2.13);

\path[fill=fillColor,fill opacity=0.20] ( 86.32, 54.91) circle (  2.13);

\path[fill=fillColor,fill opacity=0.20] ( 85.32, 60.83) circle (  2.13);

\path[fill=fillColor,fill opacity=0.20] ( 84.32, 60.00) circle (  2.13);

\path[fill=fillColor,fill opacity=0.20] ( 97.36, 65.81) circle (  2.13);

\path[fill=fillColor,fill opacity=0.20] ( 97.36, 75.68) circle (  2.13);

\path[fill=fillColor,fill opacity=0.20] ( 95.35, 77.23) circle (  2.13);

\path[fill=fillColor,fill opacity=0.20] ( 94.35, 75.68) circle (  2.13);

\path[fill=fillColor,fill opacity=0.20] ( 94.35, 74.74) circle (  2.13);

\path[fill=fillColor,fill opacity=0.20] ( 91.34, 63.53) circle (  2.13);

\path[fill=fillColor,fill opacity=0.20] ( 87.33, 54.08) circle (  2.13);

\path[fill=fillColor,fill opacity=0.20] ( 84.32, 61.45) circle (  2.13);

\path[fill=fillColor,fill opacity=0.20] ( 90.33, 56.16) circle (  2.13);

\path[fill=fillColor,fill opacity=0.20] ( 80.30, 59.38) circle (  2.13);

\path[fill=fillColor,fill opacity=0.20] ( 84.32, 58.96) circle (  2.13);

\path[fill=fillColor,fill opacity=0.20] ( 87.33, 52.21) circle (  2.13);

\path[fill=fillColor,fill opacity=0.20] ( 87.33, 50.76) circle (  2.13);

\path[fill=fillColor,fill opacity=0.20] ( 87.33, 57.09) circle (  2.13);

\path[fill=fillColor,fill opacity=0.20] ( 93.34, 64.57) circle (  2.13);

\path[fill=fillColor,fill opacity=0.20] ( 95.35, 66.12) circle (  2.13);

\path[fill=fillColor,fill opacity=0.20] ( 88.33, 61.66) circle (  2.13);

\path[fill=fillColor,fill opacity=0.20] ( 90.33, 54.29) circle (  2.13);

\path[fill=fillColor,fill opacity=0.20] ( 91.34, 60.31) circle (  2.13);

\path[fill=fillColor,fill opacity=0.20] ( 84.32, 72.77) circle (  2.13);

\path[fill=fillColor,fill opacity=0.20] ( 82.31, 70.59) circle (  2.13);

\path[fill=fillColor,fill opacity=0.20] ( 87.33, 65.40) circle (  2.13);

\path[fill=fillColor,fill opacity=0.20] ( 89.33, 63.22) circle (  2.13);

\path[fill=fillColor,fill opacity=0.20] ( 95.35, 56.36) circle (  2.13);

\path[fill=fillColor,fill opacity=0.20] ( 86.32, 50.86) circle (  2.13);

\path[fill=fillColor,fill opacity=0.20] ( 70.27, 60.00) circle (  2.13);

\path[fill=fillColor,fill opacity=0.20] ( 88.33, 69.24) circle (  2.13);

\path[fill=fillColor,fill opacity=0.20] ( 91.34, 64.67) circle (  2.13);

\path[fill=fillColor,fill opacity=0.20] ( 89.33, 60.62) circle (  2.13);

\path[fill=fillColor,fill opacity=0.20] ( 87.33, 64.67) circle (  2.13);

\path[fill=fillColor,fill opacity=0.20] ( 89.33, 65.71) circle (  2.13);

\path[fill=fillColor,fill opacity=0.20] ( 86.32, 63.01) circle (  2.13);

\path[fill=fillColor,fill opacity=0.20] ( 82.31, 56.68) circle (  2.13);

\path[fill=fillColor,fill opacity=0.20] ( 91.34, 49.10) circle (  2.13);

\path[fill=fillColor,fill opacity=0.20] ( 93.34, 50.45) circle (  2.13);

\path[fill=fillColor,fill opacity=0.20] (106.39, 51.07) circle (  2.13);

\path[fill=fillColor,fill opacity=0.20] ( 88.33, 53.67) circle (  2.13);

\path[fill=fillColor,fill opacity=0.20] ( 93.34, 54.70) circle (  2.13);

\path[fill=fillColor,fill opacity=0.20] ( 87.33, 48.68) circle (  2.13);

\path[fill=fillColor,fill opacity=0.20] ( 88.33, 47.02) circle (  2.13);

\path[fill=fillColor,fill opacity=0.20] ( 86.32, 42.56) circle (  2.13);

\path[fill=fillColor,fill opacity=0.20] ( 84.32, 42.87) circle (  2.13);

\path[fill=fillColor,fill opacity=0.20] ( 93.34, 48.99) circle (  2.13);

\path[fill=fillColor,fill opacity=0.20] ( 94.35, 47.75) circle (  2.13);

\path[fill=fillColor,fill opacity=0.20] ( 97.36, 46.19) circle (  2.13);

\path[fill=fillColor,fill opacity=0.20] ( 97.36, 49.62) circle (  2.13);

\path[fill=fillColor,fill opacity=0.20] ( 84.32, 49.51) circle (  2.13);

\path[fill=fillColor,fill opacity=0.20] ( 97.36, 49.82) circle (  2.13);

\path[fill=fillColor,fill opacity=0.20] ( 99.36, 52.63) circle (  2.13);

\path[fill=fillColor,fill opacity=0.20] ( 89.33, 48.06) circle (  2.13);

\path[fill=fillColor,fill opacity=0.20] ( 92.34, 39.86) circle (  2.13);

\path[fill=fillColor,fill opacity=0.20] ( 99.36, 39.13) circle (  2.13);

\path[fill=fillColor,fill opacity=0.20] (127.45,108.69) circle (  2.13);

\path[fill=fillColor,fill opacity=0.20] (118.43,104.54) circle (  2.13);

\path[fill=fillColor,fill opacity=0.20] ( 71.27,106.61) circle (  2.13);

\path[fill=fillColor,fill opacity=0.20] (107.39, 99.35) circle (  2.13);

\path[fill=fillColor,fill opacity=0.20] (110.40, 90.00) circle (  2.13);

\path[fill=fillColor,fill opacity=0.20] ( 87.33, 99.35) circle (  2.13);

\path[fill=fillColor,fill opacity=0.20] (123.44,103.50) circle (  2.13);

\path[fill=fillColor,fill opacity=0.20] (110.40, 99.35) circle (  2.13);

\path[fill=fillColor,fill opacity=0.20] (126.45,105.58) circle (  2.13);

\path[fill=fillColor,fill opacity=0.20] (151.53,115.96) circle (  2.13);

\path[fill=fillColor,fill opacity=0.20] (107.39,110.77) circle (  2.13);

\path[fill=fillColor,fill opacity=0.20] ( 68.16,109.73) circle (  2.13);

\path[fill=fillColor,fill opacity=0.20] ( 82.31,112.84) circle (  2.13);

\path[fill=fillColor,fill opacity=0.20] ( 82.31,102.46) circle (  2.13);

\path[fill=fillColor,fill opacity=0.20] ( 87.33, 85.85) circle (  2.13);

\path[fill=fillColor,fill opacity=0.20] ( 92.34, 77.13) circle (  2.13);

\path[fill=fillColor,fill opacity=0.20] ( 86.32, 73.08) circle (  2.13);

\path[fill=fillColor,fill opacity=0.20] ( 90.33, 69.55) circle (  2.13);

\path[fill=fillColor,fill opacity=0.20] ( 65.05,106.61) circle (  2.13);

\path[fill=fillColor,fill opacity=0.20] ( 74.28,104.54) circle (  2.13);

\path[fill=fillColor,fill opacity=0.20] ( 68.16, 94.16) circle (  2.13);

\path[fill=fillColor,fill opacity=0.20] ( 65.25, 87.93) circle (  2.13);

\path[fill=fillColor,fill opacity=0.20] ( 62.75, 77.65) circle (  2.13);

\path[fill=fillColor,fill opacity=0.20] ( 77.29, 67.79) circle (  2.13);

\path[fill=fillColor,fill opacity=0.20] ( 81.31, 54.08) circle (  2.13);

\path[fill=fillColor,fill opacity=0.20] ( 85.32, 47.96) circle (  2.13);

\path[fill=fillColor,fill opacity=0.20] ( 83.31, 61.35) circle (  2.13);

\path[fill=fillColor,fill opacity=0.20] ( 83.31, 63.63) circle (  2.13);

\path[fill=fillColor,fill opacity=0.20] ( 96.35,108.69) circle (  2.13);

\path[fill=fillColor,fill opacity=0.20] ( 71.27, 91.04) circle (  2.13);

\path[fill=fillColor,fill opacity=0.20] ( 67.06, 90.00) circle (  2.13);

\path[fill=fillColor,fill opacity=0.20] ( 50.91, 77.34) circle (  2.13);

\path[fill=fillColor,fill opacity=0.20] ( 51.81, 70.48) circle (  2.13);

\path[fill=fillColor,fill opacity=0.20] ( 73.28, 64.88) circle (  2.13);

\path[fill=fillColor,fill opacity=0.20] ( 75.29, 70.48) circle (  2.13);

\path[fill=fillColor,fill opacity=0.20] ( 79.30, 77.13) circle (  2.13);

\path[fill=fillColor,fill opacity=0.20] ( 85.32, 59.17) circle (  2.13);

\path[fill=fillColor,fill opacity=0.20] ( 77.29, 44.84) circle (  2.13);

\path[fill=fillColor,fill opacity=0.20] ( 90.33, 59.58) circle (  2.13);

\path[fill=fillColor,fill opacity=0.20] ( 97.36, 43.91) circle (  2.13);

\path[fill=fillColor,fill opacity=0.20] ( 83.31, 90.00) circle (  2.13);

\path[fill=fillColor,fill opacity=0.20] ( 52.11, 71.11) circle (  2.13);

\path[fill=fillColor,fill opacity=0.20] ( 64.45, 73.70) circle (  2.13);

\path[fill=fillColor,fill opacity=0.20] ( 80.30, 76.61) circle (  2.13);

\path[fill=fillColor,fill opacity=0.20] ( 80.30, 78.48) circle (  2.13);

\path[fill=fillColor,fill opacity=0.20] ( 80.30, 69.55) circle (  2.13);

\path[fill=fillColor,fill opacity=0.20] ( 80.30, 66.85) circle (  2.13);

\path[fill=fillColor,fill opacity=0.20] ( 84.32, 80.45) circle (  2.13);

\path[fill=fillColor,fill opacity=0.20] ( 87.33, 73.29) circle (  2.13);

\path[fill=fillColor,fill opacity=0.20] ( 91.34, 52.94) circle (  2.13);

\path[fill=fillColor,fill opacity=0.20] ( 91.34, 57.40) circle (  2.13);

\path[fill=fillColor,fill opacity=0.20] ( 82.31, 50.97) circle (  2.13);

\path[fill=fillColor,fill opacity=0.20] ( 71.27, 54.50) circle (  2.13);

\path[fill=fillColor,fill opacity=0.20] ( 66.36, 40.27) circle (  2.13);

\path[fill=fillColor,fill opacity=0.20] ( 70.27, 81.18) circle (  2.13);

\path[fill=fillColor,fill opacity=0.20] ( 75.29, 57.71) circle (  2.13);

\path[fill=fillColor,fill opacity=0.20] ( 63.65, 66.64) circle (  2.13);

\path[fill=fillColor,fill opacity=0.20] ( 77.29, 75.16) circle (  2.13);

\path[fill=fillColor,fill opacity=0.20] ( 80.30, 80.04) circle (  2.13);

\path[fill=fillColor,fill opacity=0.20] ( 86.32, 70.69) circle (  2.13);

\path[fill=fillColor,fill opacity=0.20] ( 82.31, 63.01) circle (  2.13);

\path[fill=fillColor,fill opacity=0.20] ( 85.32, 80.76) circle (  2.13);

\path[fill=fillColor,fill opacity=0.20] ( 89.33, 67.89) circle (  2.13);

\path[fill=fillColor,fill opacity=0.20] ( 75.29, 42.76) circle (  2.13);

\path[fill=fillColor,fill opacity=0.20] ( 79.30, 51.69) circle (  2.13);

\path[fill=fillColor,fill opacity=0.20] ( 59.84, 58.86) circle (  2.13);

\path[fill=fillColor,fill opacity=0.20] ( 58.83, 66.54) circle (  2.13);

\path[fill=fillColor,fill opacity=0.20] ( 53.12, 64.57) circle (  2.13);

\path[fill=fillColor,fill opacity=0.20] ( 56.63, 49.62) circle (  2.13);

\path[fill=fillColor,fill opacity=0.20] ( 79.30, 81.70) circle (  2.13);

\path[fill=fillColor,fill opacity=0.20] ( 76.29, 50.03) circle (  2.13);

\path[fill=fillColor,fill opacity=0.20] ( 79.30, 59.38) circle (  2.13);

\path[fill=fillColor,fill opacity=0.20] ( 76.29, 74.53) circle (  2.13);

\path[fill=fillColor,fill opacity=0.20] ( 83.31, 66.02) circle (  2.13);

\path[fill=fillColor,fill opacity=0.20] ( 76.29, 59.58) circle (  2.13);

\path[fill=fillColor,fill opacity=0.20] ( 83.31, 60.52) circle (  2.13);

\path[fill=fillColor,fill opacity=0.20] ( 91.34, 77.54) circle (  2.13);

\path[fill=fillColor,fill opacity=0.20] (110.40, 77.96) circle (  2.13);

\path[fill=fillColor,fill opacity=0.20] ( 87.33, 45.15) circle (  2.13);

\path[fill=fillColor,fill opacity=0.20] ( 73.28, 57.71) circle (  2.13);

\path[fill=fillColor,fill opacity=0.20] ( 70.27, 61.87) circle (  2.13);

\path[fill=fillColor,fill opacity=0.20] ( 67.56, 58.23) circle (  2.13);

\path[fill=fillColor,fill opacity=0.20] ( 59.94, 68.82) circle (  2.13);

\path[fill=fillColor,fill opacity=0.20] ( 45.09, 80.87) circle (  2.13);

\path[fill=fillColor,fill opacity=0.20] ( 61.64, 66.02) circle (  2.13);

\path[fill=fillColor,fill opacity=0.20] ( 69.27, 83.77) circle (  2.13);

\path[fill=fillColor,fill opacity=0.20] ( 59.54, 42.76) circle (  2.13);

\path[fill=fillColor,fill opacity=0.20] ( 85.32, 55.95) circle (  2.13);

\path[fill=fillColor,fill opacity=0.20] ( 86.32, 70.17) circle (  2.13);

\path[fill=fillColor,fill opacity=0.20] ( 87.33, 53.04) circle (  2.13);

\path[fill=fillColor,fill opacity=0.20] ( 85.32, 50.03) circle (  2.13);

\path[fill=fillColor,fill opacity=0.20] ( 91.34, 65.92) circle (  2.13);

\path[fill=fillColor,fill opacity=0.20] ( 67.36, 68.62) circle (  2.13);

\path[fill=fillColor,fill opacity=0.20] ( 99.36, 63.63) circle (  2.13);

\path[fill=fillColor,fill opacity=0.20] ( 79.30, 50.97) circle (  2.13);

\path[fill=fillColor,fill opacity=0.20] ( 70.27, 54.81) circle (  2.13);

\path[fill=fillColor,fill opacity=0.20] ( 77.29, 52.63) circle (  2.13);

\path[fill=fillColor,fill opacity=0.20] ( 76.29, 52.73) circle (  2.13);

\path[fill=fillColor,fill opacity=0.20] ( 68.16, 66.54) circle (  2.13);

\path[fill=fillColor,fill opacity=0.20] ( 65.25, 72.66) circle (  2.13);

\path[fill=fillColor,fill opacity=0.20] ( 60.74, 51.69) circle (  2.13);

\path[fill=fillColor,fill opacity=0.20] ( 67.96, 42.04) circle (  2.13);

\path[fill=fillColor,fill opacity=0.20] ( 83.31, 54.60) circle (  2.13);

\path[fill=fillColor,fill opacity=0.20] ( 91.34, 64.88) circle (  2.13);

\path[fill=fillColor,fill opacity=0.20] ( 93.34, 48.37) circle (  2.13);

\path[fill=fillColor,fill opacity=0.20] ( 96.35, 46.50) circle (  2.13);

\path[fill=fillColor,fill opacity=0.20] ( 96.35, 62.18) circle (  2.13);

\path[fill=fillColor,fill opacity=0.20] ( 99.36, 66.23) circle (  2.13);

\path[fill=fillColor,fill opacity=0.20] (100.37, 52.52) circle (  2.13);

\path[fill=fillColor,fill opacity=0.20] (101.37, 47.64) circle (  2.13);

\path[fill=fillColor,fill opacity=0.20] ( 94.35, 76.19) circle (  2.13);

\path[fill=fillColor,fill opacity=0.20] ( 75.29, 51.59) circle (  2.13);

\path[fill=fillColor,fill opacity=0.20] ( 80.30, 45.98) circle (  2.13);

\path[fill=fillColor,fill opacity=0.20] ( 77.29, 43.28) circle (  2.13);

\path[fill=fillColor,fill opacity=0.20] ( 74.28, 61.04) circle (  2.13);

\path[fill=fillColor,fill opacity=0.20] ( 70.27, 73.50) circle (  2.13);

\path[fill=fillColor,fill opacity=0.20] ( 59.34, 59.58) circle (  2.13);

\path[fill=fillColor,fill opacity=0.20] ( 58.33, 42.66) circle (  2.13);

\path[fill=fillColor,fill opacity=0.20] ( 46.29, 56.26) circle (  2.13);

\path[fill=fillColor,fill opacity=0.20] ( 51.91, 55.43) circle (  2.13);

\path[fill=fillColor,fill opacity=0.20] ( 72.28, 60.00) circle (  2.13);

\path[fill=fillColor,fill opacity=0.20] ( 82.31, 56.78) circle (  2.13);

\path[fill=fillColor,fill opacity=0.20] ( 86.32, 64.05) circle (  2.13);

\path[fill=fillColor,fill opacity=0.20] ( 88.33, 59.48) circle (  2.13);

\path[fill=fillColor,fill opacity=0.20] ( 85.32, 56.68) circle (  2.13);

\path[fill=fillColor,fill opacity=0.20] ( 93.34, 57.82) circle (  2.13);

\path[fill=fillColor,fill opacity=0.20] ( 92.34, 56.26) circle (  2.13);

\path[fill=fillColor,fill opacity=0.20] ( 94.35, 48.58) circle (  2.13);

\path[fill=fillColor,fill opacity=0.20] ( 84.32, 52.11) circle (  2.13);

\path[fill=fillColor,fill opacity=0.20] (104.38, 64.98) circle (  2.13);

\path[fill=fillColor,fill opacity=0.20] (107.39, 94.16) circle (  2.13);

\path[fill=fillColor,fill opacity=0.20] ( 95.35, 60.52) circle (  2.13);

\path[fill=fillColor,fill opacity=0.20] ( 78.30, 54.70) circle (  2.13);

\path[fill=fillColor,fill opacity=0.20] ( 73.28, 51.38) circle (  2.13);

\path[fill=fillColor,fill opacity=0.20] ( 73.28, 49.20) circle (  2.13);

\path[fill=fillColor,fill opacity=0.20] ( 77.29, 58.96) circle (  2.13);

\path[fill=fillColor,fill opacity=0.20] ( 71.27, 57.30) circle (  2.13);

\path[fill=fillColor,fill opacity=0.20] ( 73.28, 47.33) circle (  2.13);

\path[fill=fillColor,fill opacity=0.20] ( 71.27, 50.34) circle (  2.13);

\path[fill=fillColor,fill opacity=0.20] ( 66.66, 55.74) circle (  2.13);

\path[fill=fillColor,fill opacity=0.20] ( 56.63, 58.75) circle (  2.13);

\path[fill=fillColor,fill opacity=0.20] ( 60.24, 63.84) circle (  2.13);

\path[fill=fillColor,fill opacity=0.20] ( 67.06, 46.09) circle (  2.13);

\path[fill=fillColor,fill opacity=0.20] ( 72.28, 82.74) circle (  2.13);

\path[fill=fillColor,fill opacity=0.20] ( 73.28, 58.34) circle (  2.13);

\path[fill=fillColor,fill opacity=0.20] ( 82.31, 58.44) circle (  2.13);

\path[fill=fillColor,fill opacity=0.20] ( 81.31, 69.03) circle (  2.13);

\path[fill=fillColor,fill opacity=0.20] ( 91.34, 71.63) circle (  2.13);

\path[fill=fillColor,fill opacity=0.20] ( 95.35, 64.57) circle (  2.13);

\path[fill=fillColor,fill opacity=0.20] ( 91.34, 50.45) circle (  2.13);

\path[fill=fillColor,fill opacity=0.20] ( 95.35, 53.87) circle (  2.13);

\path[fill=fillColor,fill opacity=0.20] (100.37, 72.15) circle (  2.13);

\path[fill=fillColor,fill opacity=0.20] ( 99.36, 75.26) circle (  2.13);

\path[fill=fillColor,fill opacity=0.20] (106.39, 75.47) circle (  2.13);

\path[fill=fillColor,fill opacity=0.20] (107.39, 97.27) circle (  2.13);

\path[fill=fillColor,fill opacity=0.20] ( 84.32, 50.55) circle (  2.13);

\path[fill=fillColor,fill opacity=0.20] ( 73.28, 45.98) circle (  2.13);

\path[fill=fillColor,fill opacity=0.20] ( 73.28, 47.23) circle (  2.13);

\path[fill=fillColor,fill opacity=0.20] ( 78.30, 57.92) circle (  2.13);

\path[fill=fillColor,fill opacity=0.20] ( 77.29, 55.22) circle (  2.13);

\path[fill=fillColor,fill opacity=0.20] ( 74.28, 45.57) circle (  2.13);

\path[fill=fillColor,fill opacity=0.20] ( 71.27, 53.46) circle (  2.13);

\path[fill=fillColor,fill opacity=0.20] ( 70.27, 66.85) circle (  2.13);

\path[fill=fillColor,fill opacity=0.20] ( 71.27, 60.00) circle (  2.13);

\path[fill=fillColor,fill opacity=0.20] ( 64.05, 51.49) circle (  2.13);

\path[fill=fillColor,fill opacity=0.20] ( 64.05, 67.79) circle (  2.13);

\path[fill=fillColor,fill opacity=0.20] ( 68.26, 68.72) circle (  2.13);

\path[fill=fillColor,fill opacity=0.20] ( 71.27, 66.44) circle (  2.13);

\path[fill=fillColor,fill opacity=0.20] ( 80.30, 47.75) circle (  2.13);

\path[fill=fillColor,fill opacity=0.20] ( 88.33, 53.15) circle (  2.13);

\path[fill=fillColor,fill opacity=0.20] ( 94.35, 65.40) circle (  2.13);

\path[fill=fillColor,fill opacity=0.20] ( 99.36, 58.65) circle (  2.13);

\path[fill=fillColor,fill opacity=0.20] ( 97.36, 50.45) circle (  2.13);

\path[fill=fillColor,fill opacity=0.20] ( 99.36, 64.67) circle (  2.13);

\path[fill=fillColor,fill opacity=0.20] ( 97.36, 79.21) circle (  2.13);

\path[fill=fillColor,fill opacity=0.20] ( 91.34, 77.65) circle (  2.13);

\path[fill=fillColor,fill opacity=0.20] ( 93.34, 75.78) circle (  2.13);

\path[fill=fillColor,fill opacity=0.20] (101.37, 78.89) circle (  2.13);

\path[fill=fillColor,fill opacity=0.20] (103.38, 88.96) circle (  2.13);

\path[fill=fillColor,fill opacity=0.20] ( 73.28,113.88) circle (  2.13);

\path[fill=fillColor,fill opacity=0.20] ( 88.33, 71.83) circle (  2.13);

\path[fill=fillColor,fill opacity=0.20] ( 70.27, 58.75) circle (  2.13);

\path[fill=fillColor,fill opacity=0.20] ( 76.29, 52.84) circle (  2.13);

\path[fill=fillColor,fill opacity=0.20] ( 76.29, 55.43) circle (  2.13);

\path[fill=fillColor,fill opacity=0.20] ( 76.29, 52.63) circle (  2.13);

\path[fill=fillColor,fill opacity=0.20] ( 65.86, 58.44) circle (  2.13);

\path[fill=fillColor,fill opacity=0.20] ( 60.64, 75.99) circle (  2.13);

\path[fill=fillColor,fill opacity=0.20] ( 72.28, 72.87) circle (  2.13);

\path[fill=fillColor,fill opacity=0.20] ( 72.28, 51.38) circle (  2.13);

\path[fill=fillColor,fill opacity=0.20] ( 64.65, 51.90) circle (  2.13);

\path[fill=fillColor,fill opacity=0.20] ( 68.26, 74.12) circle (  2.13);

\path[fill=fillColor,fill opacity=0.20] ( 64.25, 77.03) circle (  2.13);

\path[fill=fillColor,fill opacity=0.20] ( 77.29, 60.21) circle (  2.13);

\path[fill=fillColor,fill opacity=0.20] ( 58.43, 61.14) circle (  2.13);

\path[fill=fillColor,fill opacity=0.20] ( 90.33, 41.10) circle (  2.13);

\path[fill=fillColor,fill opacity=0.20] ( 92.34, 46.92) circle (  2.13);

\path[fill=fillColor,fill opacity=0.20] ( 90.33, 44.43) circle (  2.13);

\path[fill=fillColor,fill opacity=0.20] ( 90.33, 50.45) circle (  2.13);

\path[fill=fillColor,fill opacity=0.20] ( 97.36, 64.46) circle (  2.13);

\path[fill=fillColor,fill opacity=0.20] ( 92.34, 67.27) circle (  2.13);

\path[fill=fillColor,fill opacity=0.20] ( 76.29, 56.57) circle (  2.13);

\path[fill=fillColor,fill opacity=0.20] ( 91.34, 63.94) circle (  2.13);

\path[fill=fillColor,fill opacity=0.20] ( 98.36, 71.11) circle (  2.13);

\path[fill=fillColor,fill opacity=0.20] (103.38, 68.62) circle (  2.13);

\path[fill=fillColor,fill opacity=0.20] (102.37, 84.81) circle (  2.13);

\path[fill=fillColor,fill opacity=0.20] ( 82.31, 96.23) circle (  2.13);

\path[fill=fillColor,fill opacity=0.20] ( 85.32, 72.35) circle (  2.13);

\path[fill=fillColor,fill opacity=0.20] ( 78.30, 79.21) circle (  2.13);

\path[fill=fillColor,fill opacity=0.20] ( 76.29, 70.90) circle (  2.13);

\path[fill=fillColor,fill opacity=0.20] ( 72.28, 52.94) circle (  2.13);

\path[fill=fillColor,fill opacity=0.20] ( 69.27, 53.77) circle (  2.13);

\path[fill=fillColor,fill opacity=0.20] ( 66.66, 71.11) circle (  2.13);

\path[fill=fillColor,fill opacity=0.20] ( 69.27, 79.72) circle (  2.13);

\path[fill=fillColor,fill opacity=0.20] ( 71.27, 67.16) circle (  2.13);

\path[fill=fillColor,fill opacity=0.20] ( 70.27, 57.51) circle (  2.13);

\path[fill=fillColor,fill opacity=0.20] ( 68.16, 63.53) circle (  2.13);

\path[fill=fillColor,fill opacity=0.20] ( 67.36, 68.10) circle (  2.13);

\path[fill=fillColor,fill opacity=0.20] ( 71.27, 72.56) circle (  2.13);

\path[fill=fillColor,fill opacity=0.20] ( 67.86, 81.70) circle (  2.13);

\path[fill=fillColor,fill opacity=0.20] ( 62.24, 62.28) circle (  2.13);

\path[fill=fillColor,fill opacity=0.20] ( 86.32, 52.21) circle (  2.13);

\path[fill=fillColor,fill opacity=0.20] ( 89.33, 51.38) circle (  2.13);

\path[fill=fillColor,fill opacity=0.20] ( 89.33, 53.77) circle (  2.13);

\path[fill=fillColor,fill opacity=0.20] ( 92.34, 52.11) circle (  2.13);

\path[fill=fillColor,fill opacity=0.20] ( 86.32, 50.34) circle (  2.13);

\path[fill=fillColor,fill opacity=0.20] ( 82.31, 47.96) circle (  2.13);

\path[fill=fillColor,fill opacity=0.20] ( 81.31, 53.87) circle (  2.13);

\path[fill=fillColor,fill opacity=0.20] ( 83.31, 59.27) circle (  2.13);

\path[fill=fillColor,fill opacity=0.20] ( 97.36, 57.51) circle (  2.13);

\path[fill=fillColor,fill opacity=0.20] ( 97.36, 62.08) circle (  2.13);

\path[fill=fillColor,fill opacity=0.20] ( 99.36, 69.76) circle (  2.13);

\path[fill=fillColor,fill opacity=0.20] (109.40, 71.63) circle (  2.13);

\path[fill=fillColor,fill opacity=0.20] (117.42, 87.93) circle (  2.13);

\path[fill=fillColor,fill opacity=0.20] ( 81.31,108.69) circle (  2.13);

\path[fill=fillColor,fill opacity=0.20] ( 63.15, 97.27) circle (  2.13);

\path[fill=fillColor,fill opacity=0.20] ( 86.32,103.50) circle (  2.13);

\path[fill=fillColor,fill opacity=0.20] ( 77.29, 83.77) circle (  2.13);

\path[fill=fillColor,fill opacity=0.20] ( 61.34, 62.08) circle (  2.13);

\path[fill=fillColor,fill opacity=0.20] ( 73.28, 59.48) circle (  2.13);

\path[fill=fillColor,fill opacity=0.20] ( 77.29, 73.29) circle (  2.13);

\path[fill=fillColor,fill opacity=0.20] ( 75.29, 69.13) circle (  2.13);

\path[fill=fillColor,fill opacity=0.20] ( 77.29, 49.72) circle (  2.13);

\path[fill=fillColor,fill opacity=0.20] ( 73.28, 49.31) circle (  2.13);

\path[fill=fillColor,fill opacity=0.20] ( 73.28, 66.12) circle (  2.13);

\path[fill=fillColor,fill opacity=0.20] ( 69.27, 65.19) circle (  2.13);

\path[fill=fillColor,fill opacity=0.20] ( 70.27, 53.35) circle (  2.13);

\path[fill=fillColor,fill opacity=0.20] ( 71.27, 66.64) circle (  2.13);

\path[fill=fillColor,fill opacity=0.20] ( 66.76, 74.43) circle (  2.13);

\path[fill=fillColor,fill opacity=0.20] ( 70.27, 51.07) circle (  2.13);

\path[fill=fillColor,fill opacity=0.20] ( 77.29, 57.51) circle (  2.13);

\path[fill=fillColor,fill opacity=0.20] ( 64.15, 94.16) circle (  2.13);

\path[fill=fillColor,fill opacity=0.20] ( 82.31, 70.38) circle (  2.13);

\path[fill=fillColor,fill opacity=0.20] ( 82.31, 66.75) circle (  2.13);

\path[fill=fillColor,fill opacity=0.20] ( 83.31, 58.23) circle (  2.13);

\path[fill=fillColor,fill opacity=0.20] ( 89.33, 60.52) circle (  2.13);

\path[fill=fillColor,fill opacity=0.20] ( 86.32, 63.53) circle (  2.13);

\path[fill=fillColor,fill opacity=0.20] ( 99.36, 56.16) circle (  2.13);

\path[fill=fillColor,fill opacity=0.20] ( 92.34, 47.85) circle (  2.13);

\path[fill=fillColor,fill opacity=0.20] ( 95.35, 47.75) circle (  2.13);

\path[fill=fillColor,fill opacity=0.20] (100.37, 53.25) circle (  2.13);

\path[fill=fillColor,fill opacity=0.20] (105.38, 49.93) circle (  2.13);

\path[fill=fillColor,fill opacity=0.20] ( 97.36, 56.47) circle (  2.13);

\path[fill=fillColor,fill opacity=0.20] ( 96.35, 68.93) circle (  2.13);

\path[fill=fillColor,fill opacity=0.20] ( 74.28, 72.46) circle (  2.13);

\path[fill=fillColor,fill opacity=0.20] ( 94.35, 78.48) circle (  2.13);

\path[fill=fillColor,fill opacity=0.20] ( 97.36, 88.96) circle (  2.13);

\path[fill=fillColor,fill opacity=0.20] ( 58.13, 84.81) circle (  2.13);

\path[fill=fillColor,fill opacity=0.20] ( 87.33, 80.14) circle (  2.13);

\path[fill=fillColor,fill opacity=0.20] ( 83.31, 63.74) circle (  2.13);

\path[fill=fillColor,fill opacity=0.20] ( 86.32, 67.06) circle (  2.13);

\path[fill=fillColor,fill opacity=0.20] ( 81.31, 80.66) circle (  2.13);

\path[fill=fillColor,fill opacity=0.20] ( 80.30, 68.93) circle (  2.13);

\path[fill=fillColor,fill opacity=0.20] ( 80.30, 61.04) circle (  2.13);

\path[fill=fillColor,fill opacity=0.20] ( 75.29, 53.77) circle (  2.13);

\path[fill=fillColor,fill opacity=0.20] ( 79.30, 46.40) circle (  2.13);

\path[fill=fillColor,fill opacity=0.20] ( 77.29, 53.77) circle (  2.13);

\path[fill=fillColor,fill opacity=0.20] ( 88.33, 50.65) circle (  2.13);

\path[fill=fillColor,fill opacity=0.20] ( 79.30, 49.93) circle (  2.13);

\path[fill=fillColor,fill opacity=0.20] ( 78.30, 61.14) circle (  2.13);

\path[fill=fillColor,fill opacity=0.20] ( 73.28, 63.84) circle (  2.13);

\path[fill=fillColor,fill opacity=0.20] ( 72.28, 50.65) circle (  2.13);

\path[fill=fillColor,fill opacity=0.20] ( 70.27, 52.84) circle (  2.13);

\path[fill=fillColor,fill opacity=0.20] ( 69.27, 61.87) circle (  2.13);

\path[fill=fillColor,fill opacity=0.20] ( 68.26, 42.87) circle (  2.13);

\path[fill=fillColor,fill opacity=0.20] ( 69.27, 44.94) circle (  2.13);

\path[fill=fillColor,fill opacity=0.20] ( 67.06, 87.93) circle (  2.13);

\path[fill=fillColor,fill opacity=0.20] ( 87.33, 65.40) circle (  2.13);

\path[fill=fillColor,fill opacity=0.20] ( 96.35, 59.27) circle (  2.13);

\path[fill=fillColor,fill opacity=0.20] ( 97.36, 49.10) circle (  2.13);

\path[fill=fillColor,fill opacity=0.20] ( 97.36, 42.97) circle (  2.13);

\path[fill=fillColor,fill opacity=0.20] ( 96.35, 51.07) circle (  2.13);

\path[fill=fillColor,fill opacity=0.20] ( 92.34, 59.79) circle (  2.13);

\path[fill=fillColor,fill opacity=0.20] ( 87.33, 56.36) circle (  2.13);

\path[fill=fillColor,fill opacity=0.20] ( 92.34, 49.93) circle (  2.13);

\path[fill=fillColor,fill opacity=0.20] ( 99.36, 57.40) circle (  2.13);

\path[fill=fillColor,fill opacity=0.20] ( 78.30, 63.84) circle (  2.13);

\path[fill=fillColor,fill opacity=0.20] ( 95.35, 70.38) circle (  2.13);

\path[fill=fillColor,fill opacity=0.20] ( 97.36, 82.74) circle (  2.13);

\path[fill=fillColor,fill opacity=0.20] (117.42, 95.19) circle (  2.13);

\path[fill=fillColor,fill opacity=0.20] ( 97.36, 75.78) circle (  2.13);

\path[fill=fillColor,fill opacity=0.20] ( 89.33, 72.04) circle (  2.13);

\path[fill=fillColor,fill opacity=0.20] ( 91.34, 52.21) circle (  2.13);

\path[fill=fillColor,fill opacity=0.20] ( 81.31, 48.68) circle (  2.13);

\path[fill=fillColor,fill opacity=0.20] ( 80.30, 73.81) circle (  2.13);

\path[fill=fillColor,fill opacity=0.20] ( 77.29, 80.56) circle (  2.13);

\path[fill=fillColor,fill opacity=0.20] ( 85.32, 66.12) circle (  2.13);

\path[fill=fillColor,fill opacity=0.20] ( 83.31, 63.01) circle (  2.13);

\path[fill=fillColor,fill opacity=0.20] ( 66.26, 69.03) circle (  2.13);

\path[fill=fillColor,fill opacity=0.20] ( 79.30, 52.21) circle (  2.13);

\path[fill=fillColor,fill opacity=0.20] ( 83.31, 39.44) circle (  2.13);

\path[fill=fillColor,fill opacity=0.20] ( 73.28, 50.65) circle (  2.13);

\path[fill=fillColor,fill opacity=0.20] ( 82.31, 63.11) circle (  2.13);

\path[fill=fillColor,fill opacity=0.20] ( 84.32, 64.88) circle (  2.13);

\path[fill=fillColor,fill opacity=0.20] ( 78.30, 58.96) circle (  2.13);

\path[fill=fillColor,fill opacity=0.20] ( 75.29, 60.93) circle (  2.13);

\path[fill=fillColor,fill opacity=0.20] ( 75.29, 57.71) circle (  2.13);

\path[fill=fillColor,fill opacity=0.20] ( 64.25, 45.05) circle (  2.13);

\path[fill=fillColor,fill opacity=0.20] ( 72.28, 48.99) circle (  2.13);

\path[fill=fillColor,fill opacity=0.20] ( 67.96, 52.84) circle (  2.13);

\path[fill=fillColor,fill opacity=0.20] ( 65.35, 44.84) circle (  2.13);

\path[fill=fillColor,fill opacity=0.20] ( 66.16, 85.85) circle (  2.13);

\path[fill=fillColor,fill opacity=0.20] ( 82.31, 74.74) circle (  2.13);

\path[fill=fillColor,fill opacity=0.20] ( 89.33, 67.37) circle (  2.13);

\path[fill=fillColor,fill opacity=0.20] ( 90.33, 57.40) circle (  2.13);

\path[fill=fillColor,fill opacity=0.20] ( 94.35, 56.47) circle (  2.13);

\path[fill=fillColor,fill opacity=0.20] ( 95.35, 54.39) circle (  2.13);

\path[fill=fillColor,fill opacity=0.20] ( 98.36, 48.27) circle (  2.13);

\path[fill=fillColor,fill opacity=0.20] ( 98.36, 46.71) circle (  2.13);

\path[fill=fillColor,fill opacity=0.20] ( 92.34, 56.16) circle (  2.13);

\path[fill=fillColor,fill opacity=0.20] ( 97.36, 53.98) circle (  2.13);

\path[fill=fillColor,fill opacity=0.20] ( 96.35, 58.44) circle (  2.13);

\path[fill=fillColor,fill opacity=0.20] ( 95.35, 68.10) circle (  2.13);

\path[fill=fillColor,fill opacity=0.20] ( 90.33, 68.51) circle (  2.13);

\path[fill=fillColor,fill opacity=0.20] ( 97.36, 65.61) circle (  2.13);

\path[fill=fillColor,fill opacity=0.20] ( 96.35, 79.41) circle (  2.13);

\path[fill=fillColor,fill opacity=0.20] (100.37,100.39) circle (  2.13);

\path[fill=fillColor,fill opacity=0.20] ( 87.33, 82.74) circle (  2.13);

\path[fill=fillColor,fill opacity=0.20] ( 92.34, 65.81) circle (  2.13);

\path[fill=fillColor,fill opacity=0.20] ( 94.35, 74.12) circle (  2.13);

\path[fill=fillColor,fill opacity=0.20] ( 86.32, 74.12) circle (  2.13);

\path[fill=fillColor,fill opacity=0.20] ( 98.36, 53.87) circle (  2.13);

\path[fill=fillColor,fill opacity=0.20] ( 92.34, 52.11) circle (  2.13);

\path[fill=fillColor,fill opacity=0.20] ( 88.33, 73.08) circle (  2.13);

\path[fill=fillColor,fill opacity=0.20] ( 71.27, 76.51) circle (  2.13);

\path[fill=fillColor,fill opacity=0.20] ( 87.33, 68.20) circle (  2.13);

\path[fill=fillColor,fill opacity=0.20] ( 85.32, 56.47) circle (  2.13);

\path[fill=fillColor,fill opacity=0.20] ( 84.32, 48.47) circle (  2.13);

\path[fill=fillColor,fill opacity=0.20] ( 83.31, 53.15) circle (  2.13);

\path[fill=fillColor,fill opacity=0.20] ( 83.31, 57.20) circle (  2.13);

\path[fill=fillColor,fill opacity=0.20] ( 78.30, 61.45) circle (  2.13);

\path[fill=fillColor,fill opacity=0.20] ( 72.28, 72.66) circle (  2.13);

\path[fill=fillColor,fill opacity=0.20] ( 79.30, 69.13) circle (  2.13);

\path[fill=fillColor,fill opacity=0.20] ( 80.30, 54.29) circle (  2.13);

\path[fill=fillColor,fill opacity=0.20] ( 74.28, 55.12) circle (  2.13);

\path[fill=fillColor,fill opacity=0.20] ( 74.28, 55.85) circle (  2.13);

\path[fill=fillColor,fill opacity=0.20] ( 75.29, 51.69) circle (  2.13);

\path[fill=fillColor,fill opacity=0.20] ( 79.30, 55.95) circle (  2.13);

\path[fill=fillColor,fill opacity=0.20] ( 69.27, 54.60) circle (  2.13);

\path[fill=fillColor,fill opacity=0.20] ( 63.45, 52.11) circle (  2.13);

\path[fill=fillColor,fill opacity=0.20] ( 68.16, 93.12) circle (  2.13);

\path[fill=fillColor,fill opacity=0.20] ( 94.35, 52.84) circle (  2.13);

\path[fill=fillColor,fill opacity=0.20] ( 96.35, 59.17) circle (  2.13);

\path[fill=fillColor,fill opacity=0.20] ( 96.35, 55.02) circle (  2.13);

\path[fill=fillColor,fill opacity=0.20] ( 94.35, 54.70) circle (  2.13);

\path[fill=fillColor,fill opacity=0.20] ( 98.36, 55.64) circle (  2.13);

\path[fill=fillColor,fill opacity=0.20] (100.37, 58.03) circle (  2.13);

\path[fill=fillColor,fill opacity=0.20] ( 97.36, 57.61) circle (  2.13);

\path[fill=fillColor,fill opacity=0.20] ( 97.36, 58.96) circle (  2.13);

\path[fill=fillColor,fill opacity=0.20] (103.38, 58.96) circle (  2.13);

\path[fill=fillColor,fill opacity=0.20] (100.37, 55.85) circle (  2.13);

\path[fill=fillColor,fill opacity=0.20] ( 97.36, 54.39) circle (  2.13);

\path[fill=fillColor,fill opacity=0.20] ( 96.35, 58.55) circle (  2.13);

\path[fill=fillColor,fill opacity=0.20] ( 96.35, 80.35) circle (  2.13);

\path[fill=fillColor,fill opacity=0.20] ( 95.35, 96.23) circle (  2.13);

\path[fill=fillColor,fill opacity=0.20] ( 91.34, 79.00) circle (  2.13);

\path[fill=fillColor,fill opacity=0.20] ( 86.32, 72.77) circle (  2.13);

\path[fill=fillColor,fill opacity=0.20] ( 83.31, 77.75) circle (  2.13);

\path[fill=fillColor,fill opacity=0.20] ( 90.33, 86.89) circle (  2.13);

\path[fill=fillColor,fill opacity=0.20] ( 89.33, 69.65) circle (  2.13);

\path[fill=fillColor,fill opacity=0.20] ( 97.36, 70.48) circle (  2.13);

\path[fill=fillColor,fill opacity=0.20] ( 94.35, 73.39) circle (  2.13);

\path[fill=fillColor,fill opacity=0.20] ( 92.34, 69.34) circle (  2.13);

\path[fill=fillColor,fill opacity=0.20] ( 92.34, 66.44) circle (  2.13);

\path[fill=fillColor,fill opacity=0.20] ( 89.33, 67.99) circle (  2.13);

\path[fill=fillColor,fill opacity=0.20] ( 88.33, 71.73) circle (  2.13);

\path[fill=fillColor,fill opacity=0.20] ( 92.34, 60.83) circle (  2.13);

\path[fill=fillColor,fill opacity=0.20] ( 88.33, 55.12) circle (  2.13);

\path[fill=fillColor,fill opacity=0.20] ( 90.33, 61.97) circle (  2.13);

\path[fill=fillColor,fill opacity=0.20] ( 88.33, 68.72) circle (  2.13);

\path[fill=fillColor,fill opacity=0.20] ( 81.31, 69.45) circle (  2.13);

\path[fill=fillColor,fill opacity=0.20] ( 80.30, 57.82) circle (  2.13);

\path[fill=fillColor,fill opacity=0.20] ( 83.31, 47.33) circle (  2.13);

\path[fill=fillColor,fill opacity=0.20] ( 87.33, 59.48) circle (  2.13);

\path[fill=fillColor,fill opacity=0.20] ( 82.31, 70.48) circle (  2.13);

\path[fill=fillColor,fill opacity=0.20] ( 77.29, 69.03) circle (  2.13);

\path[fill=fillColor,fill opacity=0.20] ( 73.28, 64.15) circle (  2.13);

\path[fill=fillColor,fill opacity=0.20] ( 78.30, 51.38) circle (  2.13);

\path[fill=fillColor,fill opacity=0.20] ( 86.32, 53.77) circle (  2.13);

\path[fill=fillColor,fill opacity=0.20] ( 79.30, 58.65) circle (  2.13);

\path[fill=fillColor,fill opacity=0.20] ( 80.30, 49.93) circle (  2.13);

\path[fill=fillColor,fill opacity=0.20] ( 83.31, 54.91) circle (  2.13);

\path[fill=fillColor,fill opacity=0.20] ( 84.32, 56.68) circle (  2.13);

\path[fill=fillColor,fill opacity=0.20] ( 68.26, 48.79) circle (  2.13);

\path[fill=fillColor,fill opacity=0.20] ( 68.26, 58.44) circle (  2.13);

\path[fill=fillColor,fill opacity=0.20] ( 83.31, 64.36) circle (  2.13);

\path[fill=fillColor,fill opacity=0.20] ( 72.28, 64.26) circle (  2.13);

\path[fill=fillColor,fill opacity=0.20] ( 94.35, 53.77) circle (  2.13);

\path[fill=fillColor,fill opacity=0.20] ( 94.35, 44.43) circle (  2.13);

\path[fill=fillColor,fill opacity=0.20] ( 99.36, 54.39) circle (  2.13);

\path[fill=fillColor,fill opacity=0.20] (105.38, 58.65) circle (  2.13);

\path[fill=fillColor,fill opacity=0.20] (107.39, 54.18) circle (  2.13);

\path[fill=fillColor,fill opacity=0.20] ( 98.36, 58.65) circle (  2.13);

\path[fill=fillColor,fill opacity=0.20] ( 90.33, 58.75) circle (  2.13);

\path[fill=fillColor,fill opacity=0.20] ( 88.33, 55.74) circle (  2.13);

\path[fill=fillColor,fill opacity=0.20] ( 90.33, 63.53) circle (  2.13);

\path[fill=fillColor,fill opacity=0.20] ( 95.35, 88.96) circle (  2.13);

\path[fill=fillColor,fill opacity=0.20] ( 91.34, 72.77) circle (  2.13);

\path[fill=fillColor,fill opacity=0.20] (105.38,109.73) circle (  2.13);

\path[fill=fillColor,fill opacity=0.20] ( 65.25,102.46) circle (  2.13);

\path[fill=fillColor,fill opacity=0.20] ( 92.34, 87.93) circle (  2.13);

\path[fill=fillColor,fill opacity=0.20] ( 81.31, 83.77) circle (  2.13);

\path[fill=fillColor,fill opacity=0.20] ( 88.33, 82.74) circle (  2.13);

\path[fill=fillColor,fill opacity=0.20] ( 95.35, 50.34) circle (  2.13);

\path[fill=fillColor,fill opacity=0.20] ( 82.31, 58.65) circle (  2.13);

\path[fill=fillColor,fill opacity=0.20] ( 91.34, 65.81) circle (  2.13);

\path[fill=fillColor,fill opacity=0.20] ( 91.34, 64.15) circle (  2.13);

\path[fill=fillColor,fill opacity=0.20] ( 83.31, 81.59) circle (  2.13);

\path[fill=fillColor,fill opacity=0.20] ( 56.73, 83.77) circle (  2.13);

\path[fill=fillColor,fill opacity=0.20] ( 89.33, 78.89) circle (  2.13);

\path[fill=fillColor,fill opacity=0.20] ( 80.30, 65.19) circle (  2.13);

\path[fill=fillColor,fill opacity=0.20] ( 82.31, 45.15) circle (  2.13);

\path[fill=fillColor,fill opacity=0.20] ( 82.31, 46.09) circle (  2.13);

\path[fill=fillColor,fill opacity=0.20] ( 82.31, 47.12) circle (  2.13);

\path[fill=fillColor,fill opacity=0.20] ( 78.30, 53.77) circle (  2.13);

\path[fill=fillColor,fill opacity=0.20] ( 79.30, 66.12) circle (  2.13);

\path[fill=fillColor,fill opacity=0.20] ( 76.29, 73.91) circle (  2.13);

\path[fill=fillColor,fill opacity=0.20] ( 86.32, 56.05) circle (  2.13);

\path[fill=fillColor,fill opacity=0.20] ( 81.31, 64.26) circle (  2.13);

\path[fill=fillColor,fill opacity=0.20] ( 78.30, 69.34) circle (  2.13);

\path[fill=fillColor,fill opacity=0.20] ( 81.31, 62.80) circle (  2.13);

\path[fill=fillColor,fill opacity=0.20] ( 80.30, 52.00) circle (  2.13);

\path[fill=fillColor,fill opacity=0.20] ( 90.33, 45.26) circle (  2.13);

\path[fill=fillColor,fill opacity=0.20] ( 84.32, 59.58) circle (  2.13);

\path[fill=fillColor,fill opacity=0.20] ( 81.31, 67.99) circle (  2.13);

\path[fill=fillColor,fill opacity=0.20] ( 87.33, 55.53) circle (  2.13);

\path[fill=fillColor,fill opacity=0.20] ( 91.34, 57.09) circle (  2.13);

\path[fill=fillColor,fill opacity=0.20] ( 84.32, 56.47) circle (  2.13);

\path[fill=fillColor,fill opacity=0.20] ( 86.32, 50.97) circle (  2.13);

\path[fill=fillColor,fill opacity=0.20] ( 81.31, 62.80) circle (  2.13);

\path[fill=fillColor,fill opacity=0.20] ( 90.33, 47.64) circle (  2.13);

\path[fill=fillColor,fill opacity=0.20] ( 98.36, 55.02) circle (  2.13);

\path[fill=fillColor,fill opacity=0.20] ( 94.35, 56.05) circle (  2.13);

\path[fill=fillColor,fill opacity=0.20] ( 94.35, 72.77) circle (  2.13);

\path[fill=fillColor,fill opacity=0.20] (104.38, 59.79) circle (  2.13);

\path[fill=fillColor,fill opacity=0.20] (100.37, 42.87) circle (  2.13);

\path[fill=fillColor,fill opacity=0.20] (100.37, 49.31) circle (  2.13);

\path[fill=fillColor,fill opacity=0.20] (102.37, 57.20) circle (  2.13);

\path[fill=fillColor,fill opacity=0.20] ( 99.36, 64.26) circle (  2.13);

\path[fill=fillColor,fill opacity=0.20] ( 92.34, 72.56) circle (  2.13);

\path[fill=fillColor,fill opacity=0.20] ( 90.33, 56.16) circle (  2.13);

\path[fill=fillColor,fill opacity=0.20] ( 97.36, 40.69) circle (  2.13);

\path[fill=fillColor,fill opacity=0.20] ( 89.33, 68.62) circle (  2.13);

\path[fill=fillColor,fill opacity=0.20] ( 81.31, 76.51) circle (  2.13);

\path[fill=fillColor,fill opacity=0.20] ( 87.33, 59.58) circle (  2.13);

\path[fill=fillColor,fill opacity=0.20] ( 86.32, 63.74) circle (  2.13);

\path[fill=fillColor,fill opacity=0.20] ( 87.33, 69.24) circle (  2.13);

\path[fill=fillColor,fill opacity=0.20] ( 86.32, 73.60) circle (  2.13);

\path[fill=fillColor,fill opacity=0.20] ( 92.34, 78.06) circle (  2.13);

\path[fill=fillColor,fill opacity=0.20] ( 81.31, 68.41) circle (  2.13);

\path[fill=fillColor,fill opacity=0.20] ( 76.29, 61.14) circle (  2.13);

\path[fill=fillColor,fill opacity=0.20] ( 48.40, 63.63) circle (  2.13);

\path[fill=fillColor,fill opacity=0.20] ( 81.31, 57.30) circle (  2.13);

\path[fill=fillColor,fill opacity=0.20] ( 87.33, 47.33) circle (  2.13);

\path[fill=fillColor,fill opacity=0.20] ( 64.65, 55.02) circle (  2.13);

\path[fill=fillColor,fill opacity=0.20] ( 88.33, 72.46) circle (  2.13);

\path[fill=fillColor,fill opacity=0.20] ( 84.32, 72.56) circle (  2.13);

\path[fill=fillColor,fill opacity=0.20] ( 73.28, 49.31) circle (  2.13);

\path[fill=fillColor,fill opacity=0.20] ( 83.31, 40.69) circle (  2.13);

\path[fill=fillColor,fill opacity=0.20] ( 86.32, 67.79) circle (  2.13);

\path[fill=fillColor,fill opacity=0.20] ( 82.31, 67.79) circle (  2.13);

\path[fill=fillColor,fill opacity=0.20] ( 76.29, 63.84) circle (  2.13);

\path[fill=fillColor,fill opacity=0.20] ( 81.31, 55.95) circle (  2.13);

\path[fill=fillColor,fill opacity=0.20] ( 78.30, 60.83) circle (  2.13);

\path[fill=fillColor,fill opacity=0.20] ( 70.27, 61.66) circle (  2.13);

\path[fill=fillColor,fill opacity=0.20] ( 76.29, 70.28) circle (  2.13);

\path[fill=fillColor,fill opacity=0.20] ( 82.31, 62.80) circle (  2.13);

\path[fill=fillColor,fill opacity=0.20] ( 80.30, 59.69) circle (  2.13);

\path[fill=fillColor,fill opacity=0.20] ( 79.30, 60.62) circle (  2.13);

\path[fill=fillColor,fill opacity=0.20] ( 86.32, 57.30) circle (  2.13);

\path[fill=fillColor,fill opacity=0.20] ( 80.30, 53.46) circle (  2.13);

\path[fill=fillColor,fill opacity=0.20] ( 82.31, 58.55) circle (  2.13);

\path[fill=fillColor,fill opacity=0.20] ( 85.32, 57.61) circle (  2.13);

\path[fill=fillColor,fill opacity=0.20] ( 87.33, 62.91) circle (  2.13);

\path[fill=fillColor,fill opacity=0.20] ( 84.32, 66.33) circle (  2.13);

\path[fill=fillColor,fill opacity=0.20] ( 90.33, 62.59) circle (  2.13);

\path[fill=fillColor,fill opacity=0.20] ( 92.34, 65.92) circle (  2.13);

\path[fill=fillColor,fill opacity=0.20] ( 86.32, 61.66) circle (  2.13);

\path[fill=fillColor,fill opacity=0.20] ( 85.32, 55.12) circle (  2.13);

\path[fill=fillColor,fill opacity=0.20] ( 75.29, 71.42) circle (  2.13);

\path[fill=fillColor,fill opacity=0.20] ( 89.33, 54.29) circle (  2.13);

\path[fill=fillColor,fill opacity=0.20] ( 88.33, 55.43) circle (  2.13);

\path[fill=fillColor,fill opacity=0.20] (101.37, 59.48) circle (  2.13);

\path[fill=fillColor,fill opacity=0.20] ( 91.34, 55.64) circle (  2.13);

\path[fill=fillColor,fill opacity=0.20] ( 99.36, 53.77) circle (  2.13);

\path[fill=fillColor,fill opacity=0.20] ( 94.35, 50.14) circle (  2.13);

\path[fill=fillColor,fill opacity=0.20] (101.37, 44.32) circle (  2.13);

\path[fill=fillColor,fill opacity=0.20] ( 98.36, 49.82) circle (  2.13);

\path[fill=fillColor,fill opacity=0.20] ( 94.35, 52.63) circle (  2.13);

\path[fill=fillColor,fill opacity=0.20] ( 87.33, 48.79) circle (  2.13);

\path[fill=fillColor,fill opacity=0.20] ( 92.34, 47.85) circle (  2.13);

\path[fill=fillColor,fill opacity=0.20] ( 80.30, 56.78) circle (  2.13);

\path[fill=fillColor,fill opacity=0.20] ( 88.33, 68.62) circle (  2.13);

\path[fill=fillColor,fill opacity=0.20] ( 93.34, 61.45) circle (  2.13);

\path[fill=fillColor,fill opacity=0.20] ( 87.33, 55.85) circle (  2.13);

\path[fill=fillColor,fill opacity=0.20] ( 85.32, 56.05) circle (  2.13);

\path[fill=fillColor,fill opacity=0.20] ( 86.32, 57.71) circle (  2.13);

\path[fill=fillColor,fill opacity=0.20] ( 79.30, 60.93) circle (  2.13);

\path[fill=fillColor,fill opacity=0.20] ( 53.22, 57.92) circle (  2.13);

\path[fill=fillColor,fill opacity=0.20] ( 78.30, 46.61) circle (  2.13);

\path[fill=fillColor,fill opacity=0.20] ( 78.30, 51.80) circle (  2.13);

\path[fill=fillColor,fill opacity=0.20] ( 72.28, 69.24) circle (  2.13);

\path[fill=fillColor,fill opacity=0.20] ( 76.29, 69.13) circle (  2.13);

\path[fill=fillColor,fill opacity=0.20] ( 74.28, 58.65) circle (  2.13);

\path[fill=fillColor,fill opacity=0.20] ( 83.31, 48.99) circle (  2.13);

\path[fill=fillColor,fill opacity=0.20] ( 82.31, 61.97) circle (  2.13);

\path[fill=fillColor,fill opacity=0.20] ( 77.29, 71.83) circle (  2.13);

\path[fill=fillColor,fill opacity=0.20] ( 84.32, 60.21) circle (  2.13);

\path[fill=fillColor,fill opacity=0.20] ( 85.32, 55.53) circle (  2.13);

\path[fill=fillColor,fill opacity=0.20] ( 85.32, 53.25) circle (  2.13);

\path[fill=fillColor,fill opacity=0.20] ( 87.33, 52.73) circle (  2.13);

\path[fill=fillColor,fill opacity=0.20] ( 88.33, 65.19) circle (  2.13);

\path[fill=fillColor,fill opacity=0.20] ( 81.31, 60.62) circle (  2.13);

\path[fill=fillColor,fill opacity=0.20] ( 70.27, 60.93) circle (  2.13);

\path[fill=fillColor,fill opacity=0.20] ( 94.35, 49.31) circle (  2.13);

\path[fill=fillColor,fill opacity=0.20] ( 88.33, 48.58) circle (  2.13);

\path[fill=fillColor,fill opacity=0.20] ( 88.33, 53.46) circle (  2.13);

\path[fill=fillColor,fill opacity=0.20] ( 89.33, 58.55) circle (  2.13);

\path[fill=fillColor,fill opacity=0.20] ( 95.35, 61.56) circle (  2.13);

\path[fill=fillColor,fill opacity=0.20] ( 96.35, 55.02) circle (  2.13);

\path[fill=fillColor,fill opacity=0.20] ( 99.36, 46.50) circle (  2.13);

\path[fill=fillColor,fill opacity=0.20] (102.37, 39.03) circle (  2.13);

\path[fill=fillColor,fill opacity=0.20] ( 98.36, 39.13) circle (  2.13);

\path[fill=fillColor,fill opacity=0.20] ( 99.36, 58.96) circle (  2.13);

\path[fill=fillColor,fill opacity=0.20] ( 94.35, 69.55) circle (  2.13);

\path[fill=fillColor,fill opacity=0.20] ( 87.33, 55.64) circle (  2.13);

\path[fill=fillColor,fill opacity=0.20] ( 84.32, 49.62) circle (  2.13);

\path[fill=fillColor,fill opacity=0.20] ( 88.33, 54.60) circle (  2.13);

\path[fill=fillColor,fill opacity=0.20] ( 70.27, 56.16) circle (  2.13);

\path[fill=fillColor,fill opacity=0.20] ( 84.32, 59.38) circle (  2.13);

\path[fill=fillColor,fill opacity=0.20] ( 81.31, 61.87) circle (  2.13);

\path[fill=fillColor,fill opacity=0.20] ( 80.30, 63.53) circle (  2.13);

\path[fill=fillColor,fill opacity=0.20] ( 72.28, 62.28) circle (  2.13);

\path[fill=fillColor,fill opacity=0.20] ( 69.27, 57.71) circle (  2.13);

\path[fill=fillColor,fill opacity=0.20] ( 88.33, 67.37) circle (  2.13);

\path[fill=fillColor,fill opacity=0.20] ( 77.29, 82.74) circle (  2.13);

\path[fill=fillColor,fill opacity=0.20] ( 66.66, 82.74) circle (  2.13);

\path[fill=fillColor,fill opacity=0.20] ( 86.32, 48.27) circle (  2.13);

\path[fill=fillColor,fill opacity=0.20] ( 68.26, 56.99) circle (  2.13);

\path[fill=fillColor,fill opacity=0.20] ( 88.33, 56.88) circle (  2.13);

\path[fill=fillColor,fill opacity=0.20] ( 82.31, 47.44) circle (  2.13);

\path[fill=fillColor,fill opacity=0.20] ( 91.34, 43.28) circle (  2.13);

\path[fill=fillColor,fill opacity=0.20] ( 84.32, 57.82) circle (  2.13);

\path[fill=fillColor,fill opacity=0.20] ( 87.33, 74.01) circle (  2.13);

\path[fill=fillColor,fill opacity=0.20] ( 71.27, 61.35) circle (  2.13);

\path[fill=fillColor,fill opacity=0.20] ( 85.32, 46.71) circle (  2.13);

\path[fill=fillColor,fill opacity=0.20] ( 79.30, 58.55) circle (  2.13);

\path[fill=fillColor,fill opacity=0.20] ( 74.28, 66.44) circle (  2.13);

\path[fill=fillColor,fill opacity=0.20] ( 77.29, 64.15) circle (  2.13);

\path[fill=fillColor,fill opacity=0.20] ( 77.29, 63.32) circle (  2.13);

\path[fill=fillColor,fill opacity=0.20] ( 80.30, 56.68) circle (  2.13);

\path[fill=fillColor,fill opacity=0.20] ( 72.28, 69.76) circle (  2.13);

\path[fill=fillColor,fill opacity=0.20] ( 72.28, 79.52) circle (  2.13);

\path[fill=fillColor,fill opacity=0.20] ( 80.30, 70.48) circle (  2.13);

\path[fill=fillColor,fill opacity=0.20] ( 76.29, 99.35) circle (  2.13);

\path[fill=fillColor,fill opacity=0.20] ( 76.29, 81.70) circle (  2.13);

\path[fill=fillColor,fill opacity=0.20] ( 74.28, 78.38) circle (  2.13);

\path[fill=fillColor,fill opacity=0.20] ( 82.31, 68.62) circle (  2.13);

\path[fill=fillColor,fill opacity=0.20] ( 79.30, 94.16) circle (  2.13);

\path[fill=fillColor,fill opacity=0.20] ( 71.27, 60.10) circle (  2.13);

\path[fill=fillColor,fill opacity=0.20] ( 58.23, 51.28) circle (  2.13);

\path[fill=fillColor,fill opacity=0.20] ( 75.29, 52.21) circle (  2.13);

\path[fill=fillColor,fill opacity=0.20] ( 76.29, 62.49) circle (  2.13);

\path[fill=fillColor,fill opacity=0.20] ( 75.29, 74.95) circle (  2.13);

\path[fill=fillColor,fill opacity=0.20] ( 95.35,107.65) circle (  2.13);

\path[fill=fillColor,fill opacity=0.20] (102.37,111.81) circle (  2.13);

\path[fill=fillColor,fill opacity=0.20] ( 95.35,106.61) circle (  2.13);

\path[fill=fillColor,fill opacity=0.20] ( 98.36, 88.96) circle (  2.13);

\path[fill=fillColor,fill opacity=0.20] (100.37, 84.81) circle (  2.13);

\path[fill=fillColor,fill opacity=0.20] ( 88.33, 95.19) circle (  2.13);

\path[fill=fillColor,fill opacity=0.20] ( 71.27, 55.43) circle (  2.13);

\path[fill=fillColor,fill opacity=0.20] ( 63.95, 62.39) circle (  2.13);

\path[fill=fillColor,fill opacity=0.20] ( 54.52, 64.15) circle (  2.13);

\path[fill=fillColor,fill opacity=0.20] ( 67.16, 51.38) circle (  2.13);

\path[fill=fillColor,fill opacity=0.20] ( 74.28, 61.24) circle (  2.13);

\path[fill=fillColor,fill opacity=0.20] ( 57.13, 78.38) circle (  2.13);

\path[fill=fillColor,fill opacity=0.20] ( 89.33, 71.63) circle (  2.13);

\path[fill=fillColor,fill opacity=0.20] ( 89.33, 73.39) circle (  2.13);

\path[fill=fillColor,fill opacity=0.20] ( 94.35,114.92) circle (  2.13);

\path[fill=fillColor,fill opacity=0.20] ( 89.33, 84.81) circle (  2.13);

\path[fill=fillColor,fill opacity=0.20] ( 95.35, 69.97) circle (  2.13);

\path[fill=fillColor,fill opacity=0.20] ( 93.34, 69.45) circle (  2.13);

\path[fill=fillColor,fill opacity=0.20] ( 91.34, 70.38) circle (  2.13);

\path[fill=fillColor,fill opacity=0.20] ( 87.33, 61.76) circle (  2.13);

\path[fill=fillColor,fill opacity=0.20] ( 85.32, 52.21) circle (  2.13);

\path[fill=fillColor,fill opacity=0.20] ( 86.32, 57.20) circle (  2.13);

\path[fill=fillColor,fill opacity=0.20] ( 95.35, 70.48) circle (  2.13);

\path[fill=fillColor,fill opacity=0.20] (108.39, 79.41) circle (  2.13);

\path[fill=fillColor,fill opacity=0.20] ( 74.28, 78.06) circle (  2.13);

\path[fill=fillColor,fill opacity=0.20] ( 67.46, 46.40) circle (  2.13);

\path[fill=fillColor,fill opacity=0.20] ( 58.93, 78.06) circle (  2.13);

\path[fill=fillColor,fill opacity=0.20] ( 61.04, 73.39) circle (  2.13);

\path[fill=fillColor,fill opacity=0.20] ( 54.02, 63.11) circle (  2.13);

\path[fill=fillColor,fill opacity=0.20] ( 76.29, 71.94) circle (  2.13);

\path[fill=fillColor,fill opacity=0.20] ( 78.30, 74.12) circle (  2.13);

\path[fill=fillColor,fill opacity=0.20] ( 77.29, 64.05) circle (  2.13);

\path[fill=fillColor,fill opacity=0.20] ( 82.31, 56.05) circle (  2.13);

\path[fill=fillColor,fill opacity=0.20] ( 92.34, 57.92) circle (  2.13);

\path[fill=fillColor,fill opacity=0.20] ( 92.34, 96.23) circle (  2.13);

\path[fill=fillColor,fill opacity=0.20] ( 88.33, 47.23) circle (  2.13);

\path[fill=fillColor,fill opacity=0.20] ( 84.32, 40.69) circle (  2.13);

\path[fill=fillColor,fill opacity=0.20] ( 84.32, 51.49) circle (  2.13);

\path[fill=fillColor,fill opacity=0.20] ( 89.33, 63.53) circle (  2.13);

\path[fill=fillColor,fill opacity=0.20] ( 88.33, 63.84) circle (  2.13);

\path[fill=fillColor,fill opacity=0.20] ( 87.33, 52.73) circle (  2.13);

\path[fill=fillColor,fill opacity=0.20] ( 58.93, 46.40) circle (  2.13);

\path[fill=fillColor,fill opacity=0.20] (100.37, 47.85) circle (  2.13);

\path[fill=fillColor,fill opacity=0.20] (115.42, 51.17) circle (  2.13);

\path[fill=fillColor,fill opacity=0.20] ( 76.29, 75.68) circle (  2.13);

\path[fill=fillColor,fill opacity=0.20] ( 68.26, 52.94) circle (  2.13);

\path[fill=fillColor,fill opacity=0.20] ( 64.45, 81.70) circle (  2.13);

\path[fill=fillColor,fill opacity=0.20] ( 48.00, 74.12) circle (  2.13);

\path[fill=fillColor,fill opacity=0.20] ( 61.54, 82.74) circle (  2.13);

\path[fill=fillColor,fill opacity=0.20] ( 79.30, 62.80) circle (  2.13);

\path[fill=fillColor,fill opacity=0.20] ( 82.31, 47.33) circle (  2.13);

\path[fill=fillColor,fill opacity=0.20] ( 84.32, 64.77) circle (  2.13);

\path[fill=fillColor,fill opacity=0.20] ( 86.32, 69.13) circle (  2.13);

\path[fill=fillColor,fill opacity=0.20] ( 89.33, 58.13) circle (  2.13);

\path[fill=fillColor,fill opacity=0.20] ( 86.32, 91.04) circle (  2.13);

\path[fill=fillColor,fill opacity=0.20] ( 81.31, 44.22) circle (  2.13);

\path[fill=fillColor,fill opacity=0.20] ( 77.29, 41.00) circle (  2.13);

\path[fill=fillColor,fill opacity=0.20] ( 71.27, 54.39) circle (  2.13);

\path[fill=fillColor,fill opacity=0.20] ( 65.66, 60.52) circle (  2.13);

\path[fill=fillColor,fill opacity=0.20] ( 71.27, 61.76) circle (  2.13);

\path[fill=fillColor,fill opacity=0.20] ( 75.29, 58.03) circle (  2.13);

\path[fill=fillColor,fill opacity=0.20] ( 81.31, 51.90) circle (  2.13);

\path[fill=fillColor,fill opacity=0.20] ( 95.35, 57.71) circle (  2.13);

\path[fill=fillColor,fill opacity=0.20] (102.37, 66.33) circle (  2.13);

\path[fill=fillColor,fill opacity=0.20] ( 91.34, 53.87) circle (  2.13);

\path[fill=fillColor,fill opacity=0.20] (118.43, 54.18) circle (  2.13);

\path[fill=fillColor,fill opacity=0.20] ( 78.30, 83.77) circle (  2.13);

\path[fill=fillColor,fill opacity=0.20] ( 67.36, 50.34) circle (  2.13);

\path[fill=fillColor,fill opacity=0.20] ( 60.84, 72.46) circle (  2.13);

\path[fill=fillColor,fill opacity=0.20] ( 47.20, 66.75) circle (  2.13);

\path[fill=fillColor,fill opacity=0.20] ( 61.64, 64.67) circle (  2.13);

\path[fill=fillColor,fill opacity=0.20] ( 51.91, 72.04) circle (  2.13);

\path[fill=fillColor,fill opacity=0.20] ( 70.27, 52.94) circle (  2.13);

\path[fill=fillColor,fill opacity=0.20] ( 83.31, 43.39) circle (  2.13);

\path[fill=fillColor,fill opacity=0.20] ( 83.31, 65.92) circle (  2.13);

\path[fill=fillColor,fill opacity=0.20] ( 83.31, 66.33) circle (  2.13);

\path[fill=fillColor,fill opacity=0.20] ( 91.34, 55.43) circle (  2.13);

\path[fill=fillColor,fill opacity=0.20] ( 80.30, 54.39) circle (  2.13);

\path[fill=fillColor,fill opacity=0.20] ( 75.29, 38.20) circle (  2.13);

\path[fill=fillColor,fill opacity=0.20] ( 65.45, 51.38) circle (  2.13);

\path[fill=fillColor,fill opacity=0.20] ( 61.54, 61.66) circle (  2.13);

\path[fill=fillColor,fill opacity=0.20] ( 52.51, 65.40) circle (  2.13);

\path[fill=fillColor,fill opacity=0.20] ( 69.27, 50.65) circle (  2.13);

\path[fill=fillColor,fill opacity=0.20] ( 81.31, 38.09) circle (  2.13);

\path[fill=fillColor,fill opacity=0.20] ( 94.35, 47.02) circle (  2.13);

\path[fill=fillColor,fill opacity=0.20] (106.39, 61.97) circle (  2.13);

\path[fill=fillColor,fill opacity=0.20] ( 89.33, 70.69) circle (  2.13);

\path[fill=fillColor,fill opacity=0.20] ( 75.29, 39.44) circle (  2.13);

\path[fill=fillColor,fill opacity=0.20] ( 69.27, 62.08) circle (  2.13);

\path[fill=fillColor,fill opacity=0.20] ( 66.56, 59.06) circle (  2.13);

\path[fill=fillColor,fill opacity=0.20] ( 69.27, 48.89) circle (  2.13);

\path[fill=fillColor,fill opacity=0.20] ( 70.27, 53.46) circle (  2.13);

\path[fill=fillColor,fill opacity=0.20] ( 73.28, 53.25) circle (  2.13);

\path[fill=fillColor,fill opacity=0.20] ( 80.30, 55.12) circle (  2.13);

\path[fill=fillColor,fill opacity=0.20] ( 86.32, 57.92) circle (  2.13);

\path[fill=fillColor,fill opacity=0.20] ( 70.27, 50.34) circle (  2.13);

\path[fill=fillColor,fill opacity=0.20] ( 93.34, 93.12) circle (  2.13);

\path[fill=fillColor,fill opacity=0.20] ( 82.31, 42.97) circle (  2.13);

\path[fill=fillColor,fill opacity=0.20] ( 72.28, 45.15) circle (  2.13);

\path[fill=fillColor,fill opacity=0.20] ( 63.65, 51.17) circle (  2.13);

\path[fill=fillColor,fill opacity=0.20] ( 59.54, 49.10) circle (  2.13);

\path[fill=fillColor,fill opacity=0.20] ( 48.50, 59.48) circle (  2.13);

\path[fill=fillColor,fill opacity=0.20] ( 77.29, 61.66) circle (  2.13);

\path[fill=fillColor,fill opacity=0.20] ( 91.34, 56.16) circle (  2.13);

\path[fill=fillColor,fill opacity=0.20] ( 98.36, 42.45) circle (  2.13);

\path[fill=fillColor,fill opacity=0.20] (111.40, 44.43) circle (  2.13);

\path[fill=fillColor,fill opacity=0.20] ( 83.31, 63.53) circle (  2.13);

\path[fill=fillColor,fill opacity=0.20] ( 82.31, 51.90) circle (  2.13);

\path[fill=fillColor,fill opacity=0.20] ( 79.30, 65.29) circle (  2.13);

\path[fill=fillColor,fill opacity=0.20] ( 61.74, 62.49) circle (  2.13);

\path[fill=fillColor,fill opacity=0.20] ( 72.28, 52.42) circle (  2.13);

\path[fill=fillColor,fill opacity=0.20] ( 73.28, 55.53) circle (  2.13);

\path[fill=fillColor,fill opacity=0.20] ( 73.28, 62.08) circle (  2.13);

\path[fill=fillColor,fill opacity=0.20] ( 73.28, 56.78) circle (  2.13);

\path[fill=fillColor,fill opacity=0.20] ( 83.31, 50.45) circle (  2.13);

\path[fill=fillColor,fill opacity=0.20] ( 85.32, 47.96) circle (  2.13);

\path[fill=fillColor,fill opacity=0.20] ( 84.32, 73.81) circle (  2.13);

\path[fill=fillColor,fill opacity=0.20] ( 72.28, 51.38) circle (  2.13);

\path[fill=fillColor,fill opacity=0.20] ( 65.66, 60.10) circle (  2.13);

\path[fill=fillColor,fill opacity=0.20] ( 58.03, 52.52) circle (  2.13);

\path[fill=fillColor,fill opacity=0.20] ( 73.28, 57.82) circle (  2.13);

\path[fill=fillColor,fill opacity=0.20] ( 75.29, 64.57) circle (  2.13);

\path[fill=fillColor,fill opacity=0.20] ( 75.29, 68.10) circle (  2.13);

\path[fill=fillColor,fill opacity=0.20] ( 95.35, 64.88) circle (  2.13);

\path[fill=fillColor,fill opacity=0.20] (107.39, 53.87) circle (  2.13);

\path[fill=fillColor,fill opacity=0.20] ( 87.33, 53.77) circle (  2.13);

\path[fill=fillColor,fill opacity=0.20] ( 83.31, 60.93) circle (  2.13);

\path[fill=fillColor,fill opacity=0.20] ( 78.30, 57.92) circle (  2.13);

\path[fill=fillColor,fill opacity=0.20] ( 83.31, 61.97) circle (  2.13);

\path[fill=fillColor,fill opacity=0.20] ( 83.31, 61.35) circle (  2.13);

\path[fill=fillColor,fill opacity=0.20] ( 79.30, 61.87) circle (  2.13);

\path[fill=fillColor,fill opacity=0.20] ( 79.30, 63.11) circle (  2.13);

\path[fill=fillColor,fill opacity=0.20] ( 66.56, 55.95) circle (  2.13);

\path[fill=fillColor,fill opacity=0.20] ( 73.28, 49.10) circle (  2.13);

\path[fill=fillColor,fill opacity=0.20] ( 77.29, 50.55) circle (  2.13);

\path[fill=fillColor,fill opacity=0.20] ( 87.33, 51.90) circle (  2.13);

\path[fill=fillColor,fill opacity=0.20] ( 98.36, 52.84) circle (  2.13);

\path[fill=fillColor,fill opacity=0.20] (115.42, 57.61) circle (  2.13);

\path[fill=fillColor,fill opacity=0.20] ( 83.31, 69.13) circle (  2.13);

\path[fill=fillColor,fill opacity=0.20] ( 69.27, 54.81) circle (  2.13);

\path[fill=fillColor,fill opacity=0.20] ( 67.46, 69.97) circle (  2.13);

\path[fill=fillColor,fill opacity=0.20] ( 65.76, 73.91) circle (  2.13);

\path[fill=fillColor,fill opacity=0.20] ( 75.29, 74.01) circle (  2.13);

\path[fill=fillColor,fill opacity=0.20] ( 87.33, 63.42) circle (  2.13);

\path[fill=fillColor,fill opacity=0.20] ( 88.33, 55.12) circle (  2.13);

\path[fill=fillColor,fill opacity=0.20] ( 90.33, 55.95) circle (  2.13);

\path[fill=fillColor,fill opacity=0.20] ( 88.33, 67.06) circle (  2.13);

\path[fill=fillColor,fill opacity=0.20] ( 84.32, 44.32) circle (  2.13);

\path[fill=fillColor,fill opacity=0.20] ( 86.32, 48.47) circle (  2.13);

\path[fill=fillColor,fill opacity=0.20] ( 76.29, 55.85) circle (  2.13);

\path[fill=fillColor,fill opacity=0.20] ( 75.29, 67.68) circle (  2.13);

\path[fill=fillColor,fill opacity=0.20] ( 72.28, 60.52) circle (  2.13);

\path[fill=fillColor,fill opacity=0.20] ( 82.31, 46.40) circle (  2.13);

\path[fill=fillColor,fill opacity=0.20] ( 83.31, 47.44) circle (  2.13);

\path[fill=fillColor,fill opacity=0.20] ( 92.34, 49.20) circle (  2.13);

\path[fill=fillColor,fill opacity=0.20] ( 92.34, 47.02) circle (  2.13);

\path[fill=fillColor,fill opacity=0.20] ( 84.32, 76.82) circle (  2.13);

\path[fill=fillColor,fill opacity=0.20] ( 77.29, 48.06) circle (  2.13);

\path[fill=fillColor,fill opacity=0.20] ( 68.26, 54.08) circle (  2.13);

\path[fill=fillColor,fill opacity=0.20] ( 68.26, 61.97) circle (  2.13);

\path[fill=fillColor,fill opacity=0.20] ( 73.28, 79.00) circle (  2.13);

\path[fill=fillColor,fill opacity=0.20] ( 78.30, 81.49) circle (  2.13);

\path[fill=fillColor,fill opacity=0.20] ( 87.33, 64.77) circle (  2.13);

\path[fill=fillColor,fill opacity=0.20] ( 87.33, 59.48) circle (  2.13);

\path[fill=fillColor,fill opacity=0.20] ( 88.33, 59.17) circle (  2.13);

\path[fill=fillColor,fill opacity=0.20] (109.40, 63.94) circle (  2.13);

\path[fill=fillColor,fill opacity=0.20] ( 93.34,101.42) circle (  2.13);

\path[fill=fillColor,fill opacity=0.20] ( 86.32, 62.49) circle (  2.13);

\path[fill=fillColor,fill opacity=0.20] ( 82.31, 44.74) circle (  2.13);

\path[fill=fillColor,fill opacity=0.20] ( 79.30, 55.22) circle (  2.13);

\path[fill=fillColor,fill opacity=0.20] ( 80.30, 61.14) circle (  2.13);

\path[fill=fillColor,fill opacity=0.20] ( 79.30, 55.33) circle (  2.13);

\path[fill=fillColor,fill opacity=0.20] ( 83.31, 60.00) circle (  2.13);

\path[fill=fillColor,fill opacity=0.20] ( 79.30, 60.21) circle (  2.13);

\path[fill=fillColor,fill opacity=0.20] ( 78.30, 54.91) circle (  2.13);

\path[fill=fillColor,fill opacity=0.20] ( 86.32, 53.35) circle (  2.13);

\path[fill=fillColor,fill opacity=0.20] ( 93.34, 46.19) circle (  2.13);

\path[fill=fillColor,fill opacity=0.20] ( 67.86, 59.69) circle (  2.13);

\path[fill=fillColor,fill opacity=0.20] ( 70.27, 55.64) circle (  2.13);

\path[fill=fillColor,fill opacity=0.20] ( 66.86, 49.82) circle (  2.13);

\path[fill=fillColor,fill opacity=0.20] ( 80.30, 59.48) circle (  2.13);

\path[fill=fillColor,fill opacity=0.20] ( 85.32, 71.63) circle (  2.13);

\path[fill=fillColor,fill opacity=0.20] ( 89.33, 70.48) circle (  2.13);

\path[fill=fillColor,fill opacity=0.20] ( 89.33, 69.03) circle (  2.13);

\path[fill=fillColor,fill opacity=0.20] ( 91.34, 64.15) circle (  2.13);

\path[fill=fillColor,fill opacity=0.20] ( 97.36, 58.75) circle (  2.13);

\path[fill=fillColor,fill opacity=0.20] ( 82.31, 90.00) circle (  2.13);

\path[fill=fillColor,fill opacity=0.20] ( 84.32, 73.81) circle (  2.13);

\path[fill=fillColor,fill opacity=0.20] ( 77.29, 66.12) circle (  2.13);

\path[fill=fillColor,fill opacity=0.20] ( 71.27, 54.39) circle (  2.13);

\path[fill=fillColor,fill opacity=0.20] ( 77.29, 39.55) circle (  2.13);

\path[fill=fillColor,fill opacity=0.20] ( 80.30, 46.40) circle (  2.13);

\path[fill=fillColor,fill opacity=0.20] ( 72.28, 61.76) circle (  2.13);

\path[fill=fillColor,fill opacity=0.20] ( 77.29, 65.09) circle (  2.13);

\path[fill=fillColor,fill opacity=0.20] ( 81.31, 53.35) circle (  2.13);

\path[fill=fillColor,fill opacity=0.20] ( 89.33, 45.15) circle (  2.13);

\path[fill=fillColor,fill opacity=0.20] (104.38, 57.61) circle (  2.13);

\path[fill=fillColor,fill opacity=0.20] ( 82.31, 67.68) circle (  2.13);

\path[fill=fillColor,fill opacity=0.20] ( 67.76, 60.62) circle (  2.13);

\path[fill=fillColor,fill opacity=0.20] ( 57.93, 47.44) circle (  2.13);

\path[fill=fillColor,fill opacity=0.20] ( 82.31, 39.34) circle (  2.13);

\path[fill=fillColor,fill opacity=0.20] ( 81.31, 56.16) circle (  2.13);

\path[fill=fillColor,fill opacity=0.20] ( 90.33, 64.98) circle (  2.13);

\path[fill=fillColor,fill opacity=0.20] ( 98.36, 63.01) circle (  2.13);

\path[fill=fillColor,fill opacity=0.20] ( 99.36, 62.18) circle (  2.13);

\path[fill=fillColor,fill opacity=0.20] (103.38, 56.99) circle (  2.13);

\path[fill=fillColor,fill opacity=0.20] (122.44, 55.12) circle (  2.13);

\path[fill=fillColor,fill opacity=0.20] ( 76.29, 77.03) circle (  2.13);

\path[fill=fillColor,fill opacity=0.20] ( 74.28, 71.63) circle (  2.13);

\path[fill=fillColor,fill opacity=0.20] ( 72.28, 64.88) circle (  2.13);

\path[fill=fillColor,fill opacity=0.20] ( 65.15, 49.72) circle (  2.13);

\path[fill=fillColor,fill opacity=0.20] ( 74.28, 41.62) circle (  2.13);

\path[fill=fillColor,fill opacity=0.20] ( 70.27, 48.89) circle (  2.13);

\path[fill=fillColor,fill opacity=0.20] ( 77.29, 55.95) circle (  2.13);

\path[fill=fillColor,fill opacity=0.20] ( 84.32, 53.98) circle (  2.13);

\path[fill=fillColor,fill opacity=0.20] (112.41, 61.45) circle (  2.13);

\path[fill=fillColor,fill opacity=0.20] (131.47, 60.10) circle (  2.13);

\path[fill=fillColor,fill opacity=0.20] ( 90.33, 81.28) circle (  2.13);

\path[fill=fillColor,fill opacity=0.20] ( 77.29, 60.52) circle (  2.13);

\path[fill=fillColor,fill opacity=0.20] ( 81.31, 56.99) circle (  2.13);

\path[fill=fillColor,fill opacity=0.20] ( 81.31, 50.45) circle (  2.13);

\path[fill=fillColor,fill opacity=0.20] ( 85.32, 52.52) circle (  2.13);

\path[fill=fillColor,fill opacity=0.20] ( 90.33, 56.68) circle (  2.13);

\path[fill=fillColor,fill opacity=0.20] ( 92.34, 52.63) circle (  2.13);

\path[fill=fillColor,fill opacity=0.20] ( 85.32, 53.56) circle (  2.13);

\path[fill=fillColor,fill opacity=0.20] ( 95.35, 53.35) circle (  2.13);

\path[fill=fillColor,fill opacity=0.20] (108.39, 49.10) circle (  2.13);

\path[fill=fillColor,fill opacity=0.20] ( 79.30, 72.56) circle (  2.13);

\path[fill=fillColor,fill opacity=0.20] ( 73.28, 66.85) circle (  2.13);

\path[fill=fillColor,fill opacity=0.20] ( 72.28, 56.99) circle (  2.13);

\path[fill=fillColor,fill opacity=0.20] ( 71.27, 56.26) circle (  2.13);

\path[fill=fillColor,fill opacity=0.20] ( 63.45, 64.46) circle (  2.13);

\path[fill=fillColor,fill opacity=0.20] ( 65.66, 66.33) circle (  2.13);

\path[fill=fillColor,fill opacity=0.20] ( 78.30, 56.36) circle (  2.13);

\path[fill=fillColor,fill opacity=0.20] ( 83.31, 43.49) circle (  2.13);

\path[fill=fillColor,fill opacity=0.20] ( 87.33, 41.52) circle (  2.13);

\path[fill=fillColor,fill opacity=0.20] ( 81.31, 64.77) circle (  2.13);

\path[fill=fillColor,fill opacity=0.20] ( 83.31, 60.93) circle (  2.13);

\path[fill=fillColor,fill opacity=0.20] ( 88.33, 51.80) circle (  2.13);

\path[fill=fillColor,fill opacity=0.20] ( 88.33, 54.18) circle (  2.13);

\path[fill=fillColor,fill opacity=0.20] ( 89.33, 52.21) circle (  2.13);

\path[fill=fillColor,fill opacity=0.20] ( 93.34, 43.49) circle (  2.13);

\path[fill=fillColor,fill opacity=0.20] ( 98.36, 56.26) circle (  2.13);

\path[fill=fillColor,fill opacity=0.20] (104.38, 71.52) circle (  2.13);

\path[fill=fillColor,fill opacity=0.20] ( 91.34, 76.82) circle (  2.13);

\path[fill=fillColor,fill opacity=0.20] ( 81.31, 60.00) circle (  2.13);

\path[fill=fillColor,fill opacity=0.20] ( 85.32, 77.75) circle (  2.13);

\path[fill=fillColor,fill opacity=0.20] ( 83.31, 66.33) circle (  2.13);

\path[fill=fillColor,fill opacity=0.20] ( 75.29, 63.11) circle (  2.13);

\path[fill=fillColor,fill opacity=0.20] ( 71.27, 61.45) circle (  2.13);

\path[fill=fillColor,fill opacity=0.20] ( 69.27, 62.80) circle (  2.13);

\path[fill=fillColor,fill opacity=0.20] ( 71.27, 75.26) circle (  2.13);

\path[fill=fillColor,fill opacity=0.20] ( 59.64, 72.56) circle (  2.13);

\path[fill=fillColor,fill opacity=0.20] ( 82.31, 48.27) circle (  2.13);

\path[fill=fillColor,fill opacity=0.20] ( 92.34, 39.34) circle (  2.13);

\path[fill=fillColor,fill opacity=0.20] ( 85.32, 61.14) circle (  2.13);

\path[fill=fillColor,fill opacity=0.20] ( 95.35, 68.82) circle (  2.13);

\path[fill=fillColor,fill opacity=0.20] (105.38, 49.93) circle (  2.13);

\path[fill=fillColor,fill opacity=0.20] ( 75.29, 67.89) circle (  2.13);

\path[fill=fillColor,fill opacity=0.20] ( 82.31, 54.91) circle (  2.13);

\path[fill=fillColor,fill opacity=0.20] ( 87.33, 52.94) circle (  2.13);

\path[fill=fillColor,fill opacity=0.20] ( 92.34, 54.50) circle (  2.13);

\path[fill=fillColor,fill opacity=0.20] ( 91.34, 50.76) circle (  2.13);

\path[fill=fillColor,fill opacity=0.20] ( 97.36, 49.31) circle (  2.13);

\path[fill=fillColor,fill opacity=0.20] ( 85.32, 60.62) circle (  2.13);

\path[fill=fillColor,fill opacity=0.20] (101.37, 76.40) circle (  2.13);

\path[fill=fillColor,fill opacity=0.20] ( 67.16, 71.83) circle (  2.13);

\path[fill=fillColor,fill opacity=0.20] ( 89.33, 57.40) circle (  2.13);

\path[fill=fillColor,fill opacity=0.20] ( 89.33, 61.14) circle (  2.13);

\path[fill=fillColor,fill opacity=0.20] ( 84.32, 53.04) circle (  2.13);

\path[fill=fillColor,fill opacity=0.20] ( 83.31, 50.55) circle (  2.13);

\path[fill=fillColor,fill opacity=0.20] ( 80.30, 66.85) circle (  2.13);

\path[fill=fillColor,fill opacity=0.20] ( 77.29, 74.12) circle (  2.13);

\path[fill=fillColor,fill opacity=0.20] ( 72.28, 71.32) circle (  2.13);

\path[fill=fillColor,fill opacity=0.20] ( 81.31, 67.16) circle (  2.13);

\path[fill=fillColor,fill opacity=0.20] ( 86.32, 57.09) circle (  2.13);

\path[fill=fillColor,fill opacity=0.20] ( 88.33, 44.53) circle (  2.13);

\path[fill=fillColor,fill opacity=0.20] ( 96.35, 54.60) circle (  2.13);

\path[fill=fillColor,fill opacity=0.20] (115.42, 40.38) circle (  2.13);

\path[fill=fillColor,fill opacity=0.20] ( 81.31, 67.79) circle (  2.13);

\path[fill=fillColor,fill opacity=0.20] ( 77.29, 56.68) circle (  2.13);

\path[fill=fillColor,fill opacity=0.20] ( 82.31, 68.62) circle (  2.13);

\path[fill=fillColor,fill opacity=0.20] ( 89.33, 60.73) circle (  2.13);

\path[fill=fillColor,fill opacity=0.20] ( 93.34, 54.91) circle (  2.13);

\path[fill=fillColor,fill opacity=0.20] (100.37, 64.36) circle (  2.13);

\path[fill=fillColor,fill opacity=0.20] (100.37, 67.89) circle (  2.13);

\path[fill=fillColor,fill opacity=0.20] ( 95.35, 58.03) circle (  2.13);

\path[fill=fillColor,fill opacity=0.20] ( 90.33, 66.02) circle (  2.13);

\path[fill=fillColor,fill opacity=0.20] (116.42, 80.76) circle (  2.13);

\path[fill=fillColor,fill opacity=0.20] ( 89.33, 59.06) circle (  2.13);

\path[fill=fillColor,fill opacity=0.20] ( 86.32, 65.09) circle (  2.13);

\path[fill=fillColor,fill opacity=0.20] ( 87.33, 66.54) circle (  2.13);

\path[fill=fillColor,fill opacity=0.20] ( 85.32, 47.64) circle (  2.13);

\path[fill=fillColor,fill opacity=0.20] ( 79.30, 38.09) circle (  2.13);

\path[fill=fillColor,fill opacity=0.20] ( 74.28, 51.28) circle (  2.13);

\path[fill=fillColor,fill opacity=0.20] ( 82.31, 61.35) circle (  2.13);

\path[fill=fillColor,fill opacity=0.20] ( 77.29, 68.41) circle (  2.13);

\path[fill=fillColor,fill opacity=0.20] ( 80.30, 73.50) circle (  2.13);

\path[fill=fillColor,fill opacity=0.20] ( 87.33, 67.16) circle (  2.13);

\path[fill=fillColor,fill opacity=0.20] ( 98.36, 58.96) circle (  2.13);

\path[fill=fillColor,fill opacity=0.20] ( 92.34, 59.38) circle (  2.13);

\path[fill=fillColor,fill opacity=0.20] ( 96.35, 65.61) circle (  2.13);

\path[fill=fillColor,fill opacity=0.20] ( 63.65, 88.96) circle (  2.13);

\path[fill=fillColor,fill opacity=0.20] ( 82.31, 70.48) circle (  2.13);

\path[fill=fillColor,fill opacity=0.20] ( 80.30, 69.03) circle (  2.13);

\path[fill=fillColor,fill opacity=0.20] ( 88.33, 66.75) circle (  2.13);

\path[fill=fillColor,fill opacity=0.20] ( 94.35, 69.65) circle (  2.13);

\path[fill=fillColor,fill opacity=0.20] ( 88.33, 65.50) circle (  2.13);

\path[fill=fillColor,fill opacity=0.20] ( 95.35, 45.67) circle (  2.13);

\path[fill=fillColor,fill opacity=0.20] ( 97.36, 54.39) circle (  2.13);

\path[fill=fillColor,fill opacity=0.20] (103.38, 69.86) circle (  2.13);

\path[fill=fillColor,fill opacity=0.20] ( 96.35, 61.97) circle (  2.13);

\path[fill=fillColor,fill opacity=0.20] (104.38, 67.37) circle (  2.13);

\path[fill=fillColor,fill opacity=0.20] (118.43, 79.00) circle (  2.13);

\path[fill=fillColor,fill opacity=0.20] ( 86.32, 66.64) circle (  2.13);

\path[fill=fillColor,fill opacity=0.20] ( 83.31, 61.14) circle (  2.13);

\path[fill=fillColor,fill opacity=0.20] ( 81.31, 64.46) circle (  2.13);

\path[fill=fillColor,fill opacity=0.20] ( 79.30, 61.04) circle (  2.13);

\path[fill=fillColor,fill opacity=0.20] ( 81.31, 48.99) circle (  2.13);

\path[fill=fillColor,fill opacity=0.20] ( 77.29, 52.00) circle (  2.13);

\path[fill=fillColor,fill opacity=0.20] ( 79.30, 63.42) circle (  2.13);

\path[fill=fillColor,fill opacity=0.20] ( 75.29, 60.21) circle (  2.13);

\path[fill=fillColor,fill opacity=0.20] ( 82.31, 57.71) circle (  2.13);

\path[fill=fillColor,fill opacity=0.20] ( 84.32, 66.85) circle (  2.13);

\path[fill=fillColor,fill opacity=0.20] ( 88.33, 68.72) circle (  2.13);

\path[fill=fillColor,fill opacity=0.20] ( 95.35, 67.16) circle (  2.13);

\path[fill=fillColor,fill opacity=0.20] ( 97.36, 63.94) circle (  2.13);

\path[fill=fillColor,fill opacity=0.20] ( 98.36, 58.34) circle (  2.13);

\path[fill=fillColor,fill opacity=0.20] ( 66.36, 84.81) circle (  2.13);

\path[fill=fillColor,fill opacity=0.20] ( 80.30, 55.43) circle (  2.13);

\path[fill=fillColor,fill opacity=0.20] ( 77.29, 63.01) circle (  2.13);

\path[fill=fillColor,fill opacity=0.20] ( 84.32, 83.77) circle (  2.13);

\path[fill=fillColor,fill opacity=0.20] ( 78.30, 60.21) circle (  2.13);

\path[fill=fillColor,fill opacity=0.20] ( 85.32, 69.45) circle (  2.13);

\path[fill=fillColor,fill opacity=0.20] ( 91.34, 64.77) circle (  2.13);

\path[fill=fillColor,fill opacity=0.20] ( 92.34, 67.06) circle (  2.13);

\path[fill=fillColor,fill opacity=0.20] ( 92.34, 65.71) circle (  2.13);

\path[fill=fillColor,fill opacity=0.20] ( 99.36, 61.45) circle (  2.13);

\path[fill=fillColor,fill opacity=0.20] (103.38, 59.69) circle (  2.13);

\path[fill=fillColor,fill opacity=0.20] ( 99.36, 65.19) circle (  2.13);

\path[fill=fillColor,fill opacity=0.20] ( 83.31, 63.84) circle (  2.13);

\path[fill=fillColor,fill opacity=0.20] ( 82.31, 54.39) circle (  2.13);

\path[fill=fillColor,fill opacity=0.20] ( 77.29, 50.76) circle (  2.13);

\path[fill=fillColor,fill opacity=0.20] ( 75.29, 56.47) circle (  2.13);

\path[fill=fillColor,fill opacity=0.20] ( 77.29, 64.98) circle (  2.13);

\path[fill=fillColor,fill opacity=0.20] ( 76.29, 65.09) circle (  2.13);

\path[fill=fillColor,fill opacity=0.20] ( 77.29, 63.01) circle (  2.13);

\path[fill=fillColor,fill opacity=0.20] ( 82.31, 58.86) circle (  2.13);

\path[fill=fillColor,fill opacity=0.20] ( 83.31, 53.04) circle (  2.13);

\path[fill=fillColor,fill opacity=0.20] ( 88.33, 55.95) circle (  2.13);

\path[fill=fillColor,fill opacity=0.20] ( 97.36, 64.57) circle (  2.13);

\path[fill=fillColor,fill opacity=0.20] ( 92.34, 57.61) circle (  2.13);

\path[fill=fillColor,fill opacity=0.20] ( 94.35, 55.64) circle (  2.13);

\path[fill=fillColor,fill opacity=0.20] (102.37, 61.87) circle (  2.13);

\path[fill=fillColor,fill opacity=0.20] ( 69.27, 66.33) circle (  2.13);

\path[fill=fillColor,fill opacity=0.20] ( 77.29, 63.53) circle (  2.13);

\path[fill=fillColor,fill opacity=0.20] ( 88.33, 69.45) circle (  2.13);

\path[fill=fillColor,fill opacity=0.20] ( 83.31, 70.80) circle (  2.13);

\path[fill=fillColor,fill opacity=0.20] ( 83.31, 59.48) circle (  2.13);

\path[fill=fillColor,fill opacity=0.20] ( 81.31, 49.82) circle (  2.13);

\path[fill=fillColor,fill opacity=0.20] ( 87.33, 66.02) circle (  2.13);

\path[fill=fillColor,fill opacity=0.20] ( 84.32, 73.70) circle (  2.13);

\path[fill=fillColor,fill opacity=0.20] ( 98.36, 74.01) circle (  2.13);

\path[fill=fillColor,fill opacity=0.20] ( 94.35, 63.32) circle (  2.13);

\path[fill=fillColor,fill opacity=0.20] ( 96.35, 59.48) circle (  2.13);

\path[fill=fillColor,fill opacity=0.20] ( 88.33, 68.51) circle (  2.13);

\path[fill=fillColor,fill opacity=0.20] ( 84.32, 56.57) circle (  2.13);

\path[fill=fillColor,fill opacity=0.20] ( 85.32, 52.84) circle (  2.13);

\path[fill=fillColor,fill opacity=0.20] ( 65.96, 52.94) circle (  2.13);

\path[fill=fillColor,fill opacity=0.20] ( 76.29, 54.29) circle (  2.13);

\path[fill=fillColor,fill opacity=0.20] ( 74.28, 66.33) circle (  2.13);

\path[fill=fillColor,fill opacity=0.20] ( 74.28, 77.96) circle (  2.13);

\path[fill=fillColor,fill opacity=0.20] ( 79.30, 72.46) circle (  2.13);

\path[fill=fillColor,fill opacity=0.20] ( 75.29, 57.61) circle (  2.13);

\path[fill=fillColor,fill opacity=0.20] ( 88.33, 42.35) circle (  2.13);

\path[fill=fillColor,fill opacity=0.20] ( 97.36, 56.05) circle (  2.13);

\path[fill=fillColor,fill opacity=0.20] (102.37, 68.93) circle (  2.13);

\path[fill=fillColor,fill opacity=0.20] (109.40, 51.80) circle (  2.13);

\path[fill=fillColor,fill opacity=0.20] (126.45, 51.38) circle (  2.13);

\path[fill=fillColor,fill opacity=0.20] ( 71.27, 66.02) circle (  2.13);

\path[fill=fillColor,fill opacity=0.20] ( 85.32, 68.10) circle (  2.13);

\path[fill=fillColor,fill opacity=0.20] ( 84.32, 75.57) circle (  2.13);

\path[fill=fillColor,fill opacity=0.20] ( 83.31, 65.71) circle (  2.13);

\path[fill=fillColor,fill opacity=0.20] ( 81.31, 62.39) circle (  2.13);

\path[fill=fillColor,fill opacity=0.20] ( 87.33, 76.82) circle (  2.13);

\path[fill=fillColor,fill opacity=0.20] ( 86.32, 75.47) circle (  2.13);

\path[fill=fillColor,fill opacity=0.20] ( 93.34, 66.02) circle (  2.13);

\path[fill=fillColor,fill opacity=0.20] ( 93.34, 56.05) circle (  2.13);

\path[fill=fillColor,fill opacity=0.20] (101.37, 59.17) circle (  2.13);

\path[fill=fillColor,fill opacity=0.20] (104.38, 75.99) circle (  2.13);

\path[fill=fillColor,fill opacity=0.20] ( 98.36, 81.39) circle (  2.13);

\path[fill=fillColor,fill opacity=0.20] ( 99.36, 71.52) circle (  2.13);

\path[fill=fillColor,fill opacity=0.20] ( 93.34, 86.89) circle (  2.13);

\path[fill=fillColor,fill opacity=0.20] ( 79.30, 66.54) circle (  2.13);

\path[fill=fillColor,fill opacity=0.20] ( 80.30, 66.85) circle (  2.13);

\path[fill=fillColor,fill opacity=0.20] ( 81.31, 74.43) circle (  2.13);

\path[fill=fillColor,fill opacity=0.20] ( 82.31, 73.50) circle (  2.13);

\path[fill=fillColor,fill opacity=0.20] ( 78.30, 70.07) circle (  2.13);

\path[fill=fillColor,fill opacity=0.20] ( 82.31, 64.67) circle (  2.13);

\path[fill=fillColor,fill opacity=0.20] ( 76.29, 67.27) circle (  2.13);

\path[fill=fillColor,fill opacity=0.20] ( 80.30, 73.60) circle (  2.13);

\path[fill=fillColor,fill opacity=0.20] ( 88.33, 60.62) circle (  2.13);

\path[fill=fillColor,fill opacity=0.20] (102.37, 48.27) circle (  2.13);

\path[fill=fillColor,fill opacity=0.20] ( 97.36, 52.84) circle (  2.13);

\path[fill=fillColor,fill opacity=0.20] (101.37, 60.62) circle (  2.13);

\path[fill=fillColor,fill opacity=0.20] (101.37, 69.97) circle (  2.13);

\path[fill=fillColor,fill opacity=0.20] (139.49, 56.68) circle (  2.13);

\path[fill=fillColor,fill opacity=0.20] ( 88.33, 69.97) circle (  2.13);

\path[fill=fillColor,fill opacity=0.20] ( 74.28, 63.01) circle (  2.13);

\path[fill=fillColor,fill opacity=0.20] ( 79.30, 77.13) circle (  2.13);

\path[fill=fillColor,fill opacity=0.20] ( 87.33, 81.70) circle (  2.13);

\path[fill=fillColor,fill opacity=0.20] ( 79.30, 63.11) circle (  2.13);

\path[fill=fillColor,fill opacity=0.20] ( 75.29, 41.52) circle (  2.13);

\path[fill=fillColor,fill opacity=0.20] ( 87.33, 41.73) circle (  2.13);

\path[fill=fillColor,fill opacity=0.20] ( 95.35, 51.69) circle (  2.13);

\path[fill=fillColor,fill opacity=0.20] ( 88.33, 59.89) circle (  2.13);

\path[fill=fillColor,fill opacity=0.20] ( 94.35, 64.15) circle (  2.13);

\path[fill=fillColor,fill opacity=0.20] ( 91.34, 69.55) circle (  2.13);

\path[fill=fillColor,fill opacity=0.20] (100.37, 72.77) circle (  2.13);

\path[fill=fillColor,fill opacity=0.20] (106.39, 71.63) circle (  2.13);

\path[fill=fillColor,fill opacity=0.20] ( 88.33, 94.16) circle (  2.13);

\path[fill=fillColor,fill opacity=0.20] ( 78.30, 77.13) circle (  2.13);

\path[fill=fillColor,fill opacity=0.20] ( 68.06, 67.06) circle (  2.13);

\path[fill=fillColor,fill opacity=0.20] ( 73.28, 71.21) circle (  2.13);

\path[fill=fillColor,fill opacity=0.20] ( 73.28, 67.99) circle (  2.13);

\path[fill=fillColor,fill opacity=0.20] ( 84.32, 69.86) circle (  2.13);

\path[fill=fillColor,fill opacity=0.20] ( 85.32, 74.85) circle (  2.13);

\path[fill=fillColor,fill opacity=0.20] ( 82.31, 60.52) circle (  2.13);

\path[fill=fillColor,fill opacity=0.20] ( 94.35, 54.91) circle (  2.13);

\path[fill=fillColor,fill opacity=0.20] (102.37, 61.97) circle (  2.13);

\path[fill=fillColor,fill opacity=0.20] (112.41, 56.57) circle (  2.13);

\path[fill=fillColor,fill opacity=0.20] (105.38, 60.41) circle (  2.13);

\path[fill=fillColor,fill opacity=0.20] (120.43, 81.18) circle (  2.13);

\path[fill=fillColor,fill opacity=0.20] (127.45, 87.93) circle (  2.13);

\path[fill=fillColor,fill opacity=0.20] ( 46.39, 63.42) circle (  2.13);

\path[fill=fillColor,fill opacity=0.20] ( 73.28, 66.75) circle (  2.13);

\path[fill=fillColor,fill opacity=0.20] ( 73.28, 56.36) circle (  2.13);

\path[fill=fillColor,fill opacity=0.20] ( 72.28, 49.62) circle (  2.13);

\path[fill=fillColor,fill opacity=0.20] ( 82.31, 52.32) circle (  2.13);

\path[fill=fillColor,fill opacity=0.20] ( 85.32, 49.62) circle (  2.13);

\path[fill=fillColor,fill opacity=0.20] ( 88.33, 45.36) circle (  2.13);

\path[fill=fillColor,fill opacity=0.20] ( 96.35, 53.56) circle (  2.13);

\path[fill=fillColor,fill opacity=0.20] ( 91.34, 69.03) circle (  2.13);

\path[fill=fillColor,fill opacity=0.20] (102.37, 77.13) circle (  2.13);

\path[fill=fillColor,fill opacity=0.20] (108.39, 62.49) circle (  2.13);

\path[fill=fillColor,fill opacity=0.20] (110.40, 55.64) circle (  2.13);

\path[fill=fillColor,fill opacity=0.20] (108.39, 67.58) circle (  2.13);

\path[fill=fillColor,fill opacity=0.20] ( 71.27, 81.39) circle (  2.13);

\path[fill=fillColor,fill opacity=0.20] ( 84.32, 86.89) circle (  2.13);

\path[fill=fillColor,fill opacity=0.20] ( 88.33, 87.93) circle (  2.13);

\path[fill=fillColor,fill opacity=0.20] ( 77.29, 77.86) circle (  2.13);

\path[fill=fillColor,fill opacity=0.20] ( 81.31, 68.20) circle (  2.13);

\path[fill=fillColor,fill opacity=0.20] ( 84.32, 57.82) circle (  2.13);

\path[fill=fillColor,fill opacity=0.20] ( 67.46, 39.65) circle (  2.13);

\path[fill=fillColor,fill opacity=0.20] ( 89.33, 40.07) circle (  2.13);

\path[fill=fillColor,fill opacity=0.20] ( 66.06, 69.34) circle (  2.13);

\path[fill=fillColor,fill opacity=0.20] ( 99.36, 86.89) circle (  2.13);

\path[fill=fillColor,fill opacity=0.20] (111.40, 82.74) circle (  2.13);

\path[fill=fillColor,fill opacity=0.20] ( 75.29, 58.75) circle (  2.13);

\path[fill=fillColor,fill opacity=0.20] ( 77.29, 71.94) circle (  2.13);

\path[fill=fillColor,fill opacity=0.20] ( 81.31, 67.58) circle (  2.13);

\path[fill=fillColor,fill opacity=0.20] ( 90.33, 58.55) circle (  2.13);

\path[fill=fillColor,fill opacity=0.20] ( 89.33, 52.84) circle (  2.13);

\path[fill=fillColor,fill opacity=0.20] ( 86.32, 58.03) circle (  2.13);

\path[fill=fillColor,fill opacity=0.20] ( 96.35, 66.75) circle (  2.13);

\path[fill=fillColor,fill opacity=0.20] ( 96.35, 70.90) circle (  2.13);

\path[fill=fillColor,fill opacity=0.20] (103.38, 68.30) circle (  2.13);

\path[fill=fillColor,fill opacity=0.20] (109.40, 55.85) circle (  2.13);

\path[fill=fillColor,fill opacity=0.20] ( 96.35, 66.44) circle (  2.13);

\path[fill=fillColor,fill opacity=0.20] ( 88.33, 74.12) circle (  2.13);

\path[fill=fillColor,fill opacity=0.20] ( 85.32, 78.27) circle (  2.13);

\path[fill=fillColor,fill opacity=0.20] ( 89.33, 68.30) circle (  2.13);

\path[fill=fillColor,fill opacity=0.20] ( 91.34, 69.76) circle (  2.13);

\path[fill=fillColor,fill opacity=0.20] ( 99.36, 69.13) circle (  2.13);

\path[fill=fillColor,fill opacity=0.20] ( 95.35, 56.68) circle (  2.13);

\path[fill=fillColor,fill opacity=0.20] ( 94.35, 53.35) circle (  2.13);

\path[fill=fillColor,fill opacity=0.20] ( 97.36, 52.21) circle (  2.13);

\path[fill=fillColor,fill opacity=0.20] ( 83.31, 63.74) circle (  2.13);

\path[fill=fillColor,fill opacity=0.20] ( 98.36, 81.49) circle (  2.13);

\path[fill=fillColor,fill opacity=0.20] (116.42, 84.81) circle (  2.13);

\path[fill=fillColor,fill opacity=0.20] ( 55.92,100.39) circle (  2.13);

\path[fill=fillColor,fill opacity=0.20] ( 90.33, 58.34) circle (  2.13);

\path[fill=fillColor,fill opacity=0.20] ( 77.29, 60.00) circle (  2.13);

\path[fill=fillColor,fill opacity=0.20] ( 78.30, 62.91) circle (  2.13);

\path[fill=fillColor,fill opacity=0.20] ( 88.33, 66.44) circle (  2.13);

\path[fill=fillColor,fill opacity=0.20] ( 88.33, 60.93) circle (  2.13);

\path[fill=fillColor,fill opacity=0.20] ( 90.33, 54.29) circle (  2.13);

\path[fill=fillColor,fill opacity=0.20] ( 97.36, 60.31) circle (  2.13);

\path[fill=fillColor,fill opacity=0.20] ( 87.33, 68.20) circle (  2.13);

\path[fill=fillColor,fill opacity=0.20] ( 88.33, 71.21) circle (  2.13);

\path[fill=fillColor,fill opacity=0.20] ( 98.36, 75.47) circle (  2.13);

\path[fill=fillColor,fill opacity=0.20] (101.37, 74.12) circle (  2.13);

\path[fill=fillColor,fill opacity=0.20] (107.39, 54.60) circle (  2.13);

\path[fill=fillColor,fill opacity=0.20] (108.39, 41.93) circle (  2.13);

\path[fill=fillColor,fill opacity=0.20] (107.39, 54.91) circle (  2.13);

\path[fill=fillColor,fill opacity=0.20] (107.39, 75.88) circle (  2.13);

\path[fill=fillColor,fill opacity=0.20] ( 98.36, 91.04) circle (  2.13);

\path[fill=fillColor,fill opacity=0.20] ( 89.33, 74.12) circle (  2.13);

\path[fill=fillColor,fill opacity=0.20] ( 87.33, 77.44) circle (  2.13);

\path[fill=fillColor,fill opacity=0.20] ( 86.32, 77.65) circle (  2.13);

\path[fill=fillColor,fill opacity=0.20] ( 88.33, 54.91) circle (  2.13);

\path[fill=fillColor,fill opacity=0.20] ( 80.30, 55.64) circle (  2.13);

\path[fill=fillColor,fill opacity=0.20] ( 83.31, 64.05) circle (  2.13);

\path[fill=fillColor,fill opacity=0.20] ( 92.34, 65.92) circle (  2.13);

\path[fill=fillColor,fill opacity=0.20] (102.37, 64.36) circle (  2.13);

\path[fill=fillColor,fill opacity=0.20] ( 74.28, 55.43) circle (  2.13);

\path[fill=fillColor,fill opacity=0.20] ( 75.29, 61.97) circle (  2.13);

\path[fill=fillColor,fill opacity=0.20] ( 73.28, 59.06) circle (  2.13);

\path[fill=fillColor,fill opacity=0.20] ( 83.31, 67.68) circle (  2.13);

\path[fill=fillColor,fill opacity=0.20] ( 85.32, 62.39) circle (  2.13);

\path[fill=fillColor,fill opacity=0.20] ( 87.33, 49.72) circle (  2.13);

\path[fill=fillColor,fill opacity=0.20] ( 89.33, 51.59) circle (  2.13);

\path[fill=fillColor,fill opacity=0.20] ( 86.32, 64.05) circle (  2.13);

\path[fill=fillColor,fill opacity=0.20] ( 83.31, 68.72) circle (  2.13);

\path[fill=fillColor,fill opacity=0.20] ( 89.33, 71.32) circle (  2.13);

\path[fill=fillColor,fill opacity=0.20] ( 97.36, 65.09) circle (  2.13);

\path[fill=fillColor,fill opacity=0.20] (100.37, 63.22) circle (  2.13);

\path[fill=fillColor,fill opacity=0.20] ( 98.36, 64.77) circle (  2.13);

\path[fill=fillColor,fill opacity=0.20] ( 98.36, 59.06) circle (  2.13);

\path[fill=fillColor,fill opacity=0.20] ( 87.33, 54.29) circle (  2.13);

\path[fill=fillColor,fill opacity=0.20] (104.38, 55.64) circle (  2.13);

\path[fill=fillColor,fill opacity=0.20] (104.38, 55.85) circle (  2.13);

\path[fill=fillColor,fill opacity=0.20] ( 97.36, 60.21) circle (  2.13);

\path[fill=fillColor,fill opacity=0.20] ( 95.35, 70.28) circle (  2.13);

\path[fill=fillColor,fill opacity=0.20] (103.38, 77.03) circle (  2.13);

\path[fill=fillColor,fill opacity=0.20] (100.37, 82.74) circle (  2.13);

\path[fill=fillColor,fill opacity=0.20] (104.38, 79.72) circle (  2.13);

\path[fill=fillColor,fill opacity=0.20] (102.37, 65.50) circle (  2.13);

\path[fill=fillColor,fill opacity=0.20] (102.37, 59.06) circle (  2.13);

\path[fill=fillColor,fill opacity=0.20] ( 97.36, 68.72) circle (  2.13);

\path[fill=fillColor,fill opacity=0.20] ( 97.36, 72.25) circle (  2.13);

\path[fill=fillColor,fill opacity=0.20] ( 93.34, 68.62) circle (  2.13);

\path[fill=fillColor,fill opacity=0.20] ( 95.35, 70.80) circle (  2.13);

\path[fill=fillColor,fill opacity=0.20] ( 98.36, 72.04) circle (  2.13);

\path[fill=fillColor,fill opacity=0.20] ( 93.34, 66.02) circle (  2.13);

\path[fill=fillColor,fill opacity=0.20] ( 91.34, 56.99) circle (  2.13);

\path[fill=fillColor,fill opacity=0.20] ( 79.30, 44.53) circle (  2.13);

\path[fill=fillColor,fill opacity=0.20] ( 79.30, 40.58) circle (  2.13);

\path[fill=fillColor,fill opacity=0.20] ( 78.30, 60.52) circle (  2.13);

\path[fill=fillColor,fill opacity=0.20] ( 86.32, 75.68) circle (  2.13);

\path[fill=fillColor,fill opacity=0.20] ( 86.32, 59.17) circle (  2.13);

\path[fill=fillColor,fill opacity=0.20] ( 86.32, 38.92) circle (  2.13);

\path[fill=fillColor,fill opacity=0.20] ( 84.32, 45.78) circle (  2.13);

\path[fill=fillColor,fill opacity=0.20] ( 92.34, 54.91) circle (  2.13);

\path[fill=fillColor,fill opacity=0.20] ( 95.35, 63.53) circle (  2.13);

\path[fill=fillColor,fill opacity=0.20] ( 57.33, 91.04) circle (  2.13);

\path[fill=fillColor,fill opacity=0.20] ( 87.33, 68.93) circle (  2.13);

\path[fill=fillColor,fill opacity=0.20] ( 77.29, 73.29) circle (  2.13);

\path[fill=fillColor,fill opacity=0.20] ( 93.34, 70.38) circle (  2.13);

\path[fill=fillColor,fill opacity=0.20] ( 92.34, 59.27) circle (  2.13);

\path[fill=fillColor,fill opacity=0.20] ( 84.32, 57.61) circle (  2.13);

\path[fill=fillColor,fill opacity=0.20] ( 76.29, 59.89) circle (  2.13);

\path[fill=fillColor,fill opacity=0.20] ( 87.33, 54.50) circle (  2.13);

\path[fill=fillColor,fill opacity=0.20] ( 90.33, 57.30) circle (  2.13);

\path[fill=fillColor,fill opacity=0.20] ( 88.33, 74.43) circle (  2.13);

\path[fill=fillColor,fill opacity=0.20] ( 94.35, 81.49) circle (  2.13);

\path[fill=fillColor,fill opacity=0.20] ( 95.35, 72.04) circle (  2.13);

\path[fill=fillColor,fill opacity=0.20] ( 95.35, 64.77) circle (  2.13);

\path[fill=fillColor,fill opacity=0.20] ( 91.34, 56.16) circle (  2.13);

\path[fill=fillColor,fill opacity=0.20] ( 93.34, 71.94) circle (  2.13);

\path[fill=fillColor,fill opacity=0.20] ( 97.36, 66.85) circle (  2.13);

\path[fill=fillColor,fill opacity=0.20] (102.37, 53.35) circle (  2.13);

\path[fill=fillColor,fill opacity=0.20] ( 98.36, 50.55) circle (  2.13);

\path[fill=fillColor,fill opacity=0.20] ( 97.36, 63.22) circle (  2.13);

\path[fill=fillColor,fill opacity=0.20] ( 95.35, 73.60) circle (  2.13);

\path[fill=fillColor,fill opacity=0.20] ( 98.36, 65.40) circle (  2.13);

\path[fill=fillColor,fill opacity=0.20] ( 93.34, 58.55) circle (  2.13);

\path[fill=fillColor,fill opacity=0.20] ( 88.33, 67.99) circle (  2.13);

\path[fill=fillColor,fill opacity=0.20] ( 97.36, 66.02) circle (  2.13);

\path[fill=fillColor,fill opacity=0.20] ( 95.35, 51.17) circle (  2.13);

\path[fill=fillColor,fill opacity=0.20] ( 95.35, 52.00) circle (  2.13);

\path[fill=fillColor,fill opacity=0.20] ( 88.33, 62.39) circle (  2.13);

\path[fill=fillColor,fill opacity=0.20] ( 83.31, 60.00) circle (  2.13);

\path[fill=fillColor,fill opacity=0.20] ( 84.32, 51.49) circle (  2.13);

\path[fill=fillColor,fill opacity=0.20] ( 88.33, 58.96) circle (  2.13);

\path[fill=fillColor,fill opacity=0.20] ( 96.35, 67.06) circle (  2.13);

\path[fill=fillColor,fill opacity=0.20] ( 92.34, 53.87) circle (  2.13);

\path[fill=fillColor,fill opacity=0.20] ( 80.30, 48.06) circle (  2.13);

\path[fill=fillColor,fill opacity=0.20] ( 66.66, 64.15) circle (  2.13);

\path[fill=fillColor,fill opacity=0.20] (104.38, 85.85) circle (  2.13);

\path[fill=fillColor,fill opacity=0.20] ( 91.34, 60.93) circle (  2.13);

\path[fill=fillColor,fill opacity=0.20] ( 85.32, 50.34) circle (  2.13);

\path[fill=fillColor,fill opacity=0.20] ( 96.35, 39.65) circle (  2.13);

\path[fill=fillColor,fill opacity=0.20] ( 92.34, 48.27) circle (  2.13);

\path[fill=fillColor,fill opacity=0.20] ( 88.33, 72.04) circle (  2.13);

\path[fill=fillColor,fill opacity=0.20] ( 90.33, 73.91) circle (  2.13);

\path[fill=fillColor,fill opacity=0.20] ( 88.33, 64.46) circle (  2.13);

\path[fill=fillColor,fill opacity=0.20] ( 89.33, 68.51) circle (  2.13);

\path[fill=fillColor,fill opacity=0.20] ( 87.33, 64.46) circle (  2.13);

\path[fill=fillColor,fill opacity=0.20] ( 83.31, 65.09) circle (  2.13);

\path[fill=fillColor,fill opacity=0.20] ( 89.33, 79.00) circle (  2.13);

\path[fill=fillColor,fill opacity=0.20] ( 90.33, 67.68) circle (  2.13);

\path[fill=fillColor,fill opacity=0.20] ( 93.34, 50.97) circle (  2.13);

\path[fill=fillColor,fill opacity=0.20] ( 90.33, 61.76) circle (  2.13);

\path[fill=fillColor,fill opacity=0.20] ( 88.33, 72.87) circle (  2.13);

\path[fill=fillColor,fill opacity=0.20] ( 88.33, 68.30) circle (  2.13);

\path[fill=fillColor,fill opacity=0.20] ( 76.29, 80.45) circle (  2.13);

\path[fill=fillColor,fill opacity=0.20] ( 92.34, 72.98) circle (  2.13);

\path[fill=fillColor,fill opacity=0.20] ( 92.34, 48.06) circle (  2.13);

\path[fill=fillColor,fill opacity=0.20] ( 92.34, 45.88) circle (  2.13);

\path[fill=fillColor,fill opacity=0.20] ( 87.33, 59.69) circle (  2.13);

\path[fill=fillColor,fill opacity=0.20] ( 86.32, 63.84) circle (  2.13);

\path[fill=fillColor,fill opacity=0.20] ( 90.33, 58.65) circle (  2.13);

\path[fill=fillColor,fill opacity=0.20] ( 94.35, 58.44) circle (  2.13);

\path[fill=fillColor,fill opacity=0.20] ( 98.36, 69.03) circle (  2.13);

\path[fill=fillColor,fill opacity=0.20] (102.37, 81.28) circle (  2.13);

\path[fill=fillColor,fill opacity=0.20] (109.40, 92.08) circle (  2.13);

\path[fill=fillColor,fill opacity=0.20] (117.42,103.50) circle (  2.13);

\path[fill=fillColor,fill opacity=0.20] ( 97.36, 71.42) circle (  2.13);

\path[fill=fillColor,fill opacity=0.20] ( 96.35, 53.25) circle (  2.13);

\path[fill=fillColor,fill opacity=0.20] ( 98.36, 41.83) circle (  2.13);

\path[fill=fillColor,fill opacity=0.20] ( 87.33, 48.68) circle (  2.13);

\path[fill=fillColor,fill opacity=0.20] ( 89.33, 52.11) circle (  2.13);

\path[fill=fillColor,fill opacity=0.20] (105.38, 49.31) circle (  2.13);

\path[fill=fillColor,fill opacity=0.20] ( 95.35, 56.26) circle (  2.13);

\path[fill=fillColor,fill opacity=0.20] ( 87.33, 51.69) circle (  2.13);

\path[fill=fillColor,fill opacity=0.20] ( 91.34, 42.04) circle (  2.13);

\path[fill=fillColor,fill opacity=0.20] ( 95.35, 52.21) circle (  2.13);

\path[fill=fillColor,fill opacity=0.20] ( 71.27, 56.05) circle (  2.13);

\path[fill=fillColor,fill opacity=0.20] ( 83.31, 44.63) circle (  2.13);

\path[fill=fillColor,fill opacity=0.20] ( 88.33, 49.93) circle (  2.13);

\path[fill=fillColor,fill opacity=0.20] ( 90.33, 68.30) circle (  2.13);

\path[fill=fillColor,fill opacity=0.20] ( 85.32, 75.88) circle (  2.13);

\path[fill=fillColor,fill opacity=0.20] ( 89.33, 68.82) circle (  2.13);

\path[fill=fillColor,fill opacity=0.20] ( 80.30, 64.88) circle (  2.13);

\path[fill=fillColor,fill opacity=0.20] ( 78.30, 66.64) circle (  2.13);

\path[fill=fillColor,fill opacity=0.20] ( 84.32, 67.47) circle (  2.13);

\path[fill=fillColor,fill opacity=0.20] ( 90.33, 57.92) circle (  2.13);

\path[fill=fillColor,fill opacity=0.20] ( 94.35, 47.33) circle (  2.13);

\path[fill=fillColor,fill opacity=0.20] ( 97.36, 45.78) circle (  2.13);

\path[fill=fillColor,fill opacity=0.20] ( 95.35, 49.72) circle (  2.13);

\path[fill=fillColor,fill opacity=0.20] ( 95.35, 55.53) circle (  2.13);

\path[fill=fillColor,fill opacity=0.20] ( 93.34, 58.03) circle (  2.13);

\path[fill=fillColor,fill opacity=0.20] ( 98.36, 69.34) circle (  2.13);

\path[fill=fillColor,fill opacity=0.20] (105.38, 68.41) circle (  2.13);

\path[fill=fillColor,fill opacity=0.20] (108.39, 61.87) circle (  2.13);

\path[fill=fillColor,fill opacity=0.20] ( 96.35, 58.75) circle (  2.13);

\path[fill=fillColor,fill opacity=0.20] (103.38, 53.77) circle (  2.13);

\path[fill=fillColor,fill opacity=0.20] (112.41, 46.92) circle (  2.13);

\path[fill=fillColor,fill opacity=0.20] (109.40, 49.72) circle (  2.13);

\path[fill=fillColor,fill opacity=0.20] ( 51.61, 46.50) circle (  2.13);

\path[fill=fillColor,fill opacity=0.20] ( 97.36, 57.71) circle (  2.13);

\path[fill=fillColor,fill opacity=0.20] ( 97.36, 54.29) circle (  2.13);

\path[fill=fillColor,fill opacity=0.20] ( 97.36, 46.71) circle (  2.13);

\path[fill=fillColor,fill opacity=0.20] ( 90.33, 56.05) circle (  2.13);

\path[fill=fillColor,fill opacity=0.20] ( 95.35, 54.50) circle (  2.13);

\path[fill=fillColor,fill opacity=0.20] ( 99.36, 54.29) circle (  2.13);

\path[fill=fillColor,fill opacity=0.20] ( 51.31, 67.27) circle (  2.13);

\path[fill=fillColor,fill opacity=0.20] ( 89.33, 66.64) circle (  2.13);

\path[fill=fillColor,fill opacity=0.20] (123.44, 92.08) circle (  2.13);

\path[fill=fillColor,fill opacity=0.20] ( 98.36, 69.13) circle (  2.13);

\path[fill=fillColor,fill opacity=0.20] (101.37, 48.27) circle (  2.13);

\path[fill=fillColor,fill opacity=0.20] (100.37, 62.59) circle (  2.13);

\path[fill=fillColor,fill opacity=0.20] (102.37, 64.05) circle (  2.13);

\path[fill=fillColor,fill opacity=0.20] ( 99.36, 65.92) circle (  2.13);

\path[fill=fillColor,fill opacity=0.20] ( 89.33, 76.92) circle (  2.13);

\path[fill=fillColor,fill opacity=0.20] ( 74.28, 72.25) circle (  2.13);

\path[fill=fillColor,fill opacity=0.20] (109.40, 60.73) circle (  2.13);

\path[fill=fillColor,fill opacity=0.20] (121.43, 64.46) circle (  2.13);

\path[fill=fillColor,fill opacity=0.20] (117.42, 77.44) circle (  2.13);

\path[fill=fillColor,fill opacity=0.20] (109.40, 84.81) circle (  2.13);

\path[fill=fillColor,fill opacity=0.20] ( 69.27, 81.70) circle (  2.13);

\path[fill=fillColor,fill opacity=0.20] (120.43, 84.81) circle (  2.13);

\path[fill=fillColor,fill opacity=0.20] ( 69.27, 88.96) circle (  2.13);

\path[fill=fillColor,fill opacity=0.20] ( 70.27, 83.77) circle (  2.13);

\path[fill=fillColor,fill opacity=0.20] ( 84.32, 76.51) circle (  2.13);

\path[fill=fillColor,fill opacity=0.20] ( 85.32, 77.34) circle (  2.13);

\path[fill=fillColor,fill opacity=0.20] ( 79.30,105.58) circle (  2.13);

\path[fill=fillColor,fill opacity=0.20] ( 71.27, 91.04) circle (  2.13);

\path[fill=fillColor,fill opacity=0.20] ( 73.28, 84.81) circle (  2.13);

\path[fill=fillColor,fill opacity=0.20] ( 78.30, 74.85) circle (  2.13);

\path[fill=fillColor,fill opacity=0.20] ( 79.30, 65.29) circle (  2.13);

\path[fill=fillColor,fill opacity=0.20] ( 78.30, 67.89) circle (  2.13);

\path[fill=fillColor,fill opacity=0.20] ( 89.33, 66.44) circle (  2.13);

\path[fill=fillColor,fill opacity=0.20] ( 90.33, 61.56) circle (  2.13);

\path[fill=fillColor,fill opacity=0.20] ( 84.32, 66.95) circle (  2.13);

\path[fill=fillColor,fill opacity=0.20] (106.39, 65.40) circle (  2.13);

\path[fill=fillColor,fill opacity=0.20] (110.40, 54.39) circle (  2.13);

\path[fill=fillColor,fill opacity=0.20] ( 83.31, 98.31) circle (  2.13);

\path[fill=fillColor,fill opacity=0.20] ( 84.32, 76.61) circle (  2.13);

\path[fill=fillColor,fill opacity=0.20] ( 80.30, 73.81) circle (  2.13);

\path[fill=fillColor,fill opacity=0.20] ( 77.29, 75.26) circle (  2.13);

\path[fill=fillColor,fill opacity=0.20] ( 81.31, 66.23) circle (  2.13);

\path[fill=fillColor,fill opacity=0.20] ( 83.31, 60.62) circle (  2.13);

\path[fill=fillColor,fill opacity=0.20] ( 86.32, 58.75) circle (  2.13);

\path[fill=fillColor,fill opacity=0.20] ( 92.34, 59.27) circle (  2.13);

\path[fill=fillColor,fill opacity=0.20] ( 90.33, 56.68) circle (  2.13);

\path[fill=fillColor,fill opacity=0.20] ( 96.35, 50.34) circle (  2.13);

\path[fill=fillColor,fill opacity=0.20] ( 90.33, 43.70) circle (  2.13);

\path[fill=fillColor,fill opacity=0.20] ( 88.33, 51.49) circle (  2.13);

\path[fill=fillColor,fill opacity=0.20] ( 96.35, 71.63) circle (  2.13);

\path[fill=fillColor,fill opacity=0.20] ( 95.35,110.77) circle (  2.13);

\path[fill=fillColor,fill opacity=0.20] ( 85.32, 75.05) circle (  2.13);

\path[fill=fillColor,fill opacity=0.20] ( 78.30, 66.02) circle (  2.13);

\path[fill=fillColor,fill opacity=0.20] ( 75.29, 65.40) circle (  2.13);

\path[fill=fillColor,fill opacity=0.20] ( 73.28, 70.48) circle (  2.13);

\path[fill=fillColor,fill opacity=0.20] ( 81.31, 67.06) circle (  2.13);

\path[fill=fillColor,fill opacity=0.20] ( 83.31, 65.81) circle (  2.13);

\path[fill=fillColor,fill opacity=0.20] ( 88.33, 61.24) circle (  2.13);

\path[fill=fillColor,fill opacity=0.20] ( 94.35, 58.34) circle (  2.13);

\path[fill=fillColor,fill opacity=0.20] ( 95.35, 60.00) circle (  2.13);

\path[fill=fillColor,fill opacity=0.20] ( 97.36, 64.88) circle (  2.13);

\path[fill=fillColor,fill opacity=0.20] ( 89.33, 73.08) circle (  2.13);

\path[fill=fillColor,fill opacity=0.20] (105.38, 73.70) circle (  2.13);

\path[fill=fillColor,fill opacity=0.20] (100.37, 66.85) circle (  2.13);

\path[fill=fillColor,fill opacity=0.20] ( 76.29, 70.59) circle (  2.13);

\path[fill=fillColor,fill opacity=0.20] (148.52, 80.87) circle (  2.13);

\path[fill=fillColor,fill opacity=0.20] ( 79.30, 73.39) circle (  2.13);

\path[fill=fillColor,fill opacity=0.20] ( 82.31, 75.99) circle (  2.13);

\path[fill=fillColor,fill opacity=0.20] ( 85.32, 68.93) circle (  2.13);

\path[fill=fillColor,fill opacity=0.20] ( 82.31, 71.83) circle (  2.13);

\path[fill=fillColor,fill opacity=0.20] ( 84.32, 76.19) circle (  2.13);

\path[fill=fillColor,fill opacity=0.20] ( 89.33, 70.90) circle (  2.13);

\path[fill=fillColor,fill opacity=0.20] ( 96.35, 67.27) circle (  2.13);

\path[fill=fillColor,fill opacity=0.20] ( 95.35, 61.97) circle (  2.13);

\path[fill=fillColor,fill opacity=0.20] ( 99.36, 54.39) circle (  2.13);

\path[fill=fillColor,fill opacity=0.20] (106.39, 60.10) circle (  2.13);

\path[fill=fillColor,fill opacity=0.20] (102.37, 76.71) circle (  2.13);

\path[fill=fillColor,fill opacity=0.20] ( 94.35, 79.10) circle (  2.13);

\path[fill=fillColor,fill opacity=0.20] ( 92.34, 84.81) circle (  2.13);

\path[fill=fillColor,fill opacity=0.20] ( 81.31, 66.02) circle (  2.13);

\path[fill=fillColor,fill opacity=0.20] ( 86.32, 74.64) circle (  2.13);

\path[fill=fillColor,fill opacity=0.20] ( 85.32, 63.01) circle (  2.13);

\path[fill=fillColor,fill opacity=0.20] ( 92.34, 67.58) circle (  2.13);

\path[fill=fillColor,fill opacity=0.20] ( 94.35, 81.70) circle (  2.13);

\path[fill=fillColor,fill opacity=0.20] ( 92.34, 76.61) circle (  2.13);

\path[fill=fillColor,fill opacity=0.20] ( 96.35, 67.27) circle (  2.13);

\path[fill=fillColor,fill opacity=0.20] ( 98.36, 64.57) circle (  2.13);

\path[fill=fillColor,fill opacity=0.20] (101.37, 53.04) circle (  2.13);

\path[fill=fillColor,fill opacity=0.20] (121.43, 50.97) circle (  2.13);

\path[fill=fillColor,fill opacity=0.20] ( 82.31, 88.96) circle (  2.13);

\path[fill=fillColor,fill opacity=0.20] ( 85.32, 74.22) circle (  2.13);

\path[fill=fillColor,fill opacity=0.20] ( 82.31, 57.20) circle (  2.13);

\path[fill=fillColor,fill opacity=0.20] ( 83.31, 62.80) circle (  2.13);

\path[fill=fillColor,fill opacity=0.20] ( 75.29, 58.96) circle (  2.13);

\path[fill=fillColor,fill opacity=0.20] ( 93.34, 64.15) circle (  2.13);

\path[fill=fillColor,fill opacity=0.20] ( 98.36, 75.99) circle (  2.13);

\path[fill=fillColor,fill opacity=0.20] ( 99.36, 77.86) circle (  2.13);

\path[fill=fillColor,fill opacity=0.20] (104.38, 65.71) circle (  2.13);

\path[fill=fillColor,fill opacity=0.20] (104.38, 63.84) circle (  2.13);

\path[fill=fillColor,fill opacity=0.20] (119.43, 70.48) circle (  2.13);

\path[fill=fillColor,fill opacity=0.20] ( 92.34, 55.95) circle (  2.13);

\path[fill=fillColor,fill opacity=0.20] ( 88.33, 54.18) circle (  2.13);

\path[fill=fillColor,fill opacity=0.20] ( 89.33, 56.57) circle (  2.13);

\path[fill=fillColor,fill opacity=0.20] ( 96.35, 63.32) circle (  2.13);

\path[fill=fillColor,fill opacity=0.20] ( 92.34, 88.96) circle (  2.13);

\path[fill=fillColor,fill opacity=0.20] ( 86.32, 70.17) circle (  2.13);

\path[fill=fillColor,fill opacity=0.20] ( 83.31, 68.82) circle (  2.13);

\path[fill=fillColor,fill opacity=0.20] ( 85.32, 64.67) circle (  2.13);

\path[fill=fillColor,fill opacity=0.20] ( 93.34, 66.23) circle (  2.13);

\path[fill=fillColor,fill opacity=0.20] ( 96.35, 66.95) circle (  2.13);

\path[fill=fillColor,fill opacity=0.20] (101.37, 69.13) circle (  2.13);

\path[fill=fillColor,fill opacity=0.20] (110.40, 67.79) circle (  2.13);

\path[fill=fillColor,fill opacity=0.20] (115.42, 70.07) circle (  2.13);

\path[fill=fillColor,fill opacity=0.20] (135.48, 83.77) circle (  2.13);

\path[fill=fillColor,fill opacity=0.20] ( 90.33, 69.24) circle (  2.13);

\path[fill=fillColor,fill opacity=0.20] ( 99.36, 60.62) circle (  2.13);

\path[fill=fillColor,fill opacity=0.20] ( 89.33, 77.34) circle (  2.13);

\path[fill=fillColor,fill opacity=0.20] ( 83.31, 74.53) circle (  2.13);

\path[fill=fillColor,fill opacity=0.20] ( 91.34, 70.28) circle (  2.13);

\path[fill=fillColor,fill opacity=0.20] ( 97.36, 62.28) circle (  2.13);

\path[fill=fillColor,fill opacity=0.20] ( 98.36, 45.36) circle (  2.13);

\path[fill=fillColor,fill opacity=0.20] ( 91.34, 81.28) circle (  2.13);

\path[fill=fillColor,fill opacity=0.20] (102.37,100.39) circle (  2.13);

\path[fill=fillColor,fill opacity=0.20] ( 75.29, 76.19) circle (  2.13);

\path[fill=fillColor,fill opacity=0.20] ( 83.31, 71.63) circle (  2.13);

\path[fill=fillColor,fill opacity=0.20] ( 91.34, 60.73) circle (  2.13);

\path[fill=fillColor,fill opacity=0.20] ( 93.34, 63.53) circle (  2.13);

\path[fill=fillColor,fill opacity=0.20] ( 88.33, 64.26) circle (  2.13);

\path[fill=fillColor,fill opacity=0.20] (104.38, 61.66) circle (  2.13);

\path[fill=fillColor,fill opacity=0.20] (111.40, 74.33) circle (  2.13);

\path[fill=fillColor,fill opacity=0.20] ( 87.33, 62.39) circle (  2.13);

\path[fill=fillColor,fill opacity=0.20] ( 89.33, 67.47) circle (  2.13);

\path[fill=fillColor,fill opacity=0.20] ( 87.33, 71.94) circle (  2.13);

\path[fill=fillColor,fill opacity=0.20] ( 89.33, 63.74) circle (  2.13);

\path[fill=fillColor,fill opacity=0.20] ( 84.32, 62.28) circle (  2.13);

\path[fill=fillColor,fill opacity=0.20] ( 89.33, 65.61) circle (  2.13);

\path[fill=fillColor,fill opacity=0.20] (101.37, 58.96) circle (  2.13);

\path[fill=fillColor,fill opacity=0.20] ( 94.35, 60.83) circle (  2.13);

\path[fill=fillColor,fill opacity=0.20] (102.37, 97.27) circle (  2.13);

\path[fill=fillColor,fill opacity=0.20] ( 75.29, 53.98) circle (  2.13);

\path[fill=fillColor,fill opacity=0.20] ( 69.27, 59.38) circle (  2.13);

\path[fill=fillColor,fill opacity=0.20] ( 92.34, 51.07) circle (  2.13);

\path[fill=fillColor,fill opacity=0.20] ( 87.33, 58.65) circle (  2.13);

\path[fill=fillColor,fill opacity=0.20] ( 88.33, 66.64) circle (  2.13);

\path[fill=fillColor,fill opacity=0.20] (104.38, 56.47) circle (  2.13);

\path[fill=fillColor,fill opacity=0.20] ( 99.36, 60.62) circle (  2.13);

\path[fill=fillColor,fill opacity=0.20] (106.39, 67.79) circle (  2.13);

\path[fill=fillColor,fill opacity=0.20] ( 84.32, 76.40) circle (  2.13);

\path[fill=fillColor,fill opacity=0.20] ( 78.30, 56.68) circle (  2.13);

\path[fill=fillColor,fill opacity=0.20] ( 81.31, 60.62) circle (  2.13);

\path[fill=fillColor,fill opacity=0.20] ( 88.33, 55.85) circle (  2.13);

\path[fill=fillColor,fill opacity=0.20] ( 97.36, 52.00) circle (  2.13);

\path[fill=fillColor,fill opacity=0.20] ( 92.34, 45.57) circle (  2.13);

\path[fill=fillColor,fill opacity=0.20] ( 91.34, 50.86) circle (  2.13);

\path[fill=fillColor,fill opacity=0.20] ( 97.36, 57.71) circle (  2.13);

\path[fill=fillColor,fill opacity=0.20] ( 92.34, 54.60) circle (  2.13);

\path[fill=fillColor,fill opacity=0.20] ( 82.31, 47.85) circle (  2.13);

\path[fill=fillColor,fill opacity=0.20] ( 83.31, 53.77) circle (  2.13);

\path[fill=fillColor,fill opacity=0.20] ( 92.34, 59.79) circle (  2.13);

\path[fill=fillColor,fill opacity=0.20] ( 94.35, 62.91) circle (  2.13);

\path[fill=fillColor,fill opacity=0.20] ( 97.36, 67.27) circle (  2.13);

\path[fill=fillColor,fill opacity=0.20] (104.38, 56.88) circle (  2.13);

\path[fill=fillColor,fill opacity=0.20] (103.38, 48.68) circle (  2.13);

\path[fill=fillColor,fill opacity=0.20] (104.38, 51.07) circle (  2.13);

\path[fill=fillColor,fill opacity=0.20] ( 83.31, 58.55) circle (  2.13);

\path[fill=fillColor,fill opacity=0.20] ( 75.29, 49.62) circle (  2.13);

\path[fill=fillColor,fill opacity=0.20] ( 82.31, 66.33) circle (  2.13);

\path[fill=fillColor,fill opacity=0.20] ( 86.32, 60.52) circle (  2.13);

\path[fill=fillColor,fill opacity=0.20] ( 86.32, 52.00) circle (  2.13);

\path[fill=fillColor,fill opacity=0.20] ( 92.34, 39.23) circle (  2.13);

\path[fill=fillColor,fill opacity=0.20] ( 89.33, 43.28) circle (  2.13);

\path[fill=fillColor,fill opacity=0.20] ( 96.35, 52.11) circle (  2.13);

\path[fill=fillColor,fill opacity=0.20] (101.37, 45.26) circle (  2.13);

\path[fill=fillColor,fill opacity=0.20] ( 83.31, 66.85) circle (  2.13);

\path[fill=fillColor,fill opacity=0.20] (115.42, 86.89) circle (  2.13);

\path[fill=fillColor,fill opacity=0.20] ( 94.35, 60.10) circle (  2.13);

\path[fill=fillColor,fill opacity=0.20] ( 85.32, 69.86) circle (  2.13);

\path[fill=fillColor,fill opacity=0.20] ( 86.32, 70.17) circle (  2.13);

\path[fill=fillColor,fill opacity=0.20] ( 93.34, 62.28) circle (  2.13);

\path[fill=fillColor,fill opacity=0.20] ( 97.36, 56.05) circle (  2.13);

\path[fill=fillColor,fill opacity=0.20] (106.39, 51.17) circle (  2.13);

\path[fill=fillColor,fill opacity=0.20] (108.39, 52.00) circle (  2.13);

\path[fill=fillColor,fill opacity=0.20] ( 77.29, 57.71) circle (  2.13);

\path[fill=fillColor,fill opacity=0.20] ( 78.30, 62.59) circle (  2.13);

\path[fill=fillColor,fill opacity=0.20] ( 82.31, 77.44) circle (  2.13);

\path[fill=fillColor,fill opacity=0.20] ( 83.31, 63.74) circle (  2.13);

\path[fill=fillColor,fill opacity=0.20] ( 82.31, 50.34) circle (  2.13);

\path[fill=fillColor,fill opacity=0.20] ( 88.33, 42.76) circle (  2.13);

\path[fill=fillColor,fill opacity=0.20] ( 83.31, 50.24) circle (  2.13);

\path[fill=fillColor,fill opacity=0.20] ( 91.34, 56.57) circle (  2.13);

\path[fill=fillColor,fill opacity=0.20] ( 99.36, 42.25) circle (  2.13);

\path[fill=fillColor,fill opacity=0.20] (101.37, 49.62) circle (  2.13);

\path[fill=fillColor,fill opacity=0.20] ( 95.35, 66.23) circle (  2.13);

\path[fill=fillColor,fill opacity=0.20] ( 84.32, 55.85) circle (  2.13);

\path[fill=fillColor,fill opacity=0.20] ( 84.32, 60.31) circle (  2.13);

\path[fill=fillColor,fill opacity=0.20] ( 93.34, 56.78) circle (  2.13);

\path[fill=fillColor,fill opacity=0.20] ( 95.35, 60.31) circle (  2.13);

\path[fill=fillColor,fill opacity=0.20] ( 97.36, 71.42) circle (  2.13);

\path[fill=fillColor,fill opacity=0.20] (106.39, 74.43) circle (  2.13);

\path[fill=fillColor,fill opacity=0.20] (110.40, 66.23) circle (  2.13);

\path[fill=fillColor,fill opacity=0.20] ( 82.31, 60.10) circle (  2.13);

\path[fill=fillColor,fill opacity=0.20] ( 86.32, 68.62) circle (  2.13);

\path[fill=fillColor,fill opacity=0.20] ( 84.32, 69.65) circle (  2.13);

\path[fill=fillColor,fill opacity=0.20] ( 87.33, 57.92) circle (  2.13);

\path[fill=fillColor,fill opacity=0.20] ( 87.33, 57.30) circle (  2.13);

\path[fill=fillColor,fill opacity=0.20] ( 90.33, 56.36) circle (  2.13);

\path[fill=fillColor,fill opacity=0.20] ( 83.31, 56.47) circle (  2.13);

\path[fill=fillColor,fill opacity=0.20] ( 83.31, 58.34) circle (  2.13);

\path[fill=fillColor,fill opacity=0.20] ( 96.35, 47.75) circle (  2.13);

\path[fill=fillColor,fill opacity=0.20] (147.52, 87.93) circle (  2.13);

\path[fill=fillColor,fill opacity=0.20] (108.39, 56.99) circle (  2.13);

\path[fill=fillColor,fill opacity=0.20] ( 96.35, 44.01) circle (  2.13);

\path[fill=fillColor,fill opacity=0.20] ( 95.35, 60.21) circle (  2.13);

\path[fill=fillColor,fill opacity=0.20] ( 94.35, 71.73) circle (  2.13);

\path[fill=fillColor,fill opacity=0.20] ( 89.33, 75.47) circle (  2.13);

\path[fill=fillColor,fill opacity=0.20] ( 96.35, 77.34) circle (  2.13);

\path[fill=fillColor,fill opacity=0.20] (104.38, 69.97) circle (  2.13);

\path[fill=fillColor,fill opacity=0.20] (111.40, 63.63) circle (  2.13);

\path[fill=fillColor,fill opacity=0.20] ( 91.34, 78.38) circle (  2.13);

\path[fill=fillColor,fill opacity=0.20] ( 91.34, 55.22) circle (  2.13);

\path[fill=fillColor,fill opacity=0.20] ( 91.34, 49.93) circle (  2.13);

\path[fill=fillColor,fill opacity=0.20] ( 87.33, 53.46) circle (  2.13);

\path[fill=fillColor,fill opacity=0.20] ( 86.32, 62.18) circle (  2.13);

\path[fill=fillColor,fill opacity=0.20] ( 77.29, 67.06) circle (  2.13);

\path[fill=fillColor,fill opacity=0.20] ( 79.30, 63.42) circle (  2.13);

\path[fill=fillColor,fill opacity=0.20] ( 81.31, 56.99) circle (  2.13);

\path[fill=fillColor,fill opacity=0.20] ( 79.30, 57.71) circle (  2.13);

\path[fill=fillColor,fill opacity=0.20] ( 91.34, 55.53) circle (  2.13);

\path[fill=fillColor,fill opacity=0.20] ( 95.35, 49.62) circle (  2.13);

\path[fill=fillColor,fill opacity=0.20] ( 86.32, 69.13) circle (  2.13);

\path[fill=fillColor,fill opacity=0.20] (143.51, 50.14) circle (  2.13);

\path[fill=fillColor,fill opacity=0.20] (109.40, 56.26) circle (  2.13);

\path[fill=fillColor,fill opacity=0.20] ( 94.35, 71.63) circle (  2.13);

\path[fill=fillColor,fill opacity=0.20] ( 86.32, 64.57) circle (  2.13);

\path[fill=fillColor,fill opacity=0.20] ( 95.35, 60.83) circle (  2.13);

\path[fill=fillColor,fill opacity=0.20] (100.37, 70.59) circle (  2.13);

\path[fill=fillColor,fill opacity=0.20] (101.37, 73.39) circle (  2.13);

\path[fill=fillColor,fill opacity=0.20] (101.37, 74.85) circle (  2.13);

\path[fill=fillColor,fill opacity=0.20] ( 89.33, 62.59) circle (  2.13);

\path[fill=fillColor,fill opacity=0.20] ( 87.33, 63.94) circle (  2.13);

\path[fill=fillColor,fill opacity=0.20] ( 83.31, 49.10) circle (  2.13);

\path[fill=fillColor,fill opacity=0.20] ( 84.32, 39.96) circle (  2.13);

\path[fill=fillColor,fill opacity=0.20] ( 83.31, 54.60) circle (  2.13);

\path[fill=fillColor,fill opacity=0.20] ( 80.30, 67.58) circle (  2.13);

\path[fill=fillColor,fill opacity=0.20] ( 75.29, 64.26) circle (  2.13);

\path[fill=fillColor,fill opacity=0.20] ( 76.29, 57.09) circle (  2.13);

\path[fill=fillColor,fill opacity=0.20] ( 80.30, 55.74) circle (  2.13);

\path[fill=fillColor,fill opacity=0.20] ( 78.30, 61.04) circle (  2.13);

\path[fill=fillColor,fill opacity=0.20] ( 84.32, 60.62) circle (  2.13);

\path[fill=fillColor,fill opacity=0.20] ( 97.36, 60.21) circle (  2.13);

\path[fill=fillColor,fill opacity=0.20] (129.46, 81.18) circle (  2.13);

\path[fill=fillColor,fill opacity=0.20] (136.48, 51.17) circle (  2.13);

\path[fill=fillColor,fill opacity=0.20] (100.37, 54.29) circle (  2.13);

\path[fill=fillColor,fill opacity=0.20] ( 88.33, 59.58) circle (  2.13);

\path[fill=fillColor,fill opacity=0.20] ( 95.35, 60.41) circle (  2.13);

\path[fill=fillColor,fill opacity=0.20] ( 97.36, 67.27) circle (  2.13);

\path[fill=fillColor,fill opacity=0.20] ( 97.36, 77.44) circle (  2.13);

\path[fill=fillColor,fill opacity=0.20] (105.38, 69.03) circle (  2.13);

\path[fill=fillColor,fill opacity=0.20] ( 84.32, 68.62) circle (  2.13);

\path[fill=fillColor,fill opacity=0.20] ( 80.30, 61.04) circle (  2.13);

\path[fill=fillColor,fill opacity=0.20] ( 79.30, 62.70) circle (  2.13);

\path[fill=fillColor,fill opacity=0.20] ( 75.29, 56.47) circle (  2.13);

\path[fill=fillColor,fill opacity=0.20] ( 76.29, 64.15) circle (  2.13);

\path[fill=fillColor,fill opacity=0.20] ( 71.27, 70.17) circle (  2.13);

\path[fill=fillColor,fill opacity=0.20] ( 75.29, 64.46) circle (  2.13);

\path[fill=fillColor,fill opacity=0.20] ( 77.29, 57.92) circle (  2.13);

\path[fill=fillColor,fill opacity=0.20] ( 75.29, 50.97) circle (  2.13);

\path[fill=fillColor,fill opacity=0.20] ( 77.29, 52.11) circle (  2.13);

\path[fill=fillColor,fill opacity=0.20] ( 73.28, 62.80) circle (  2.13);

\path[fill=fillColor,fill opacity=0.20] ( 84.32, 56.47) circle (  2.13);

\path[fill=fillColor,fill opacity=0.20] ( 94.35, 51.07) circle (  2.13);

\path[fill=fillColor,fill opacity=0.20] (130.46, 78.79) circle (  2.13);

\path[fill=fillColor,fill opacity=0.20] ( 82.31, 52.84) circle (  2.13);

\path[fill=fillColor,fill opacity=0.20] (100.37, 57.20) circle (  2.13);

\path[fill=fillColor,fill opacity=0.20] ( 93.34, 68.30) circle (  2.13);

\path[fill=fillColor,fill opacity=0.20] ( 98.36, 66.33) circle (  2.13);

\path[fill=fillColor,fill opacity=0.20] ( 99.36, 68.20) circle (  2.13);

\path[fill=fillColor,fill opacity=0.20] (101.37, 67.06) circle (  2.13);

\path[fill=fillColor,fill opacity=0.20] (104.38, 64.36) circle (  2.13);

\path[fill=fillColor,fill opacity=0.20] ( 87.33, 65.29) circle (  2.13);

\path[fill=fillColor,fill opacity=0.20] ( 86.32, 65.19) circle (  2.13);

\path[fill=fillColor,fill opacity=0.20] ( 82.31, 63.42) circle (  2.13);

\path[fill=fillColor,fill opacity=0.20] ( 80.30, 53.77) circle (  2.13);

\path[fill=fillColor,fill opacity=0.20] ( 78.30, 64.67) circle (  2.13);

\path[fill=fillColor,fill opacity=0.20] ( 75.29, 84.81) circle (  2.13);

\path[fill=fillColor,fill opacity=0.20] ( 69.27, 80.76) circle (  2.13);

\path[fill=fillColor,fill opacity=0.20] ( 74.28, 63.22) circle (  2.13);

\path[fill=fillColor,fill opacity=0.20] ( 83.31, 51.69) circle (  2.13);

\path[fill=fillColor,fill opacity=0.20] ( 78.30, 45.98) circle (  2.13);

\path[fill=fillColor,fill opacity=0.20] ( 77.29, 54.39) circle (  2.13);

\path[fill=fillColor,fill opacity=0.20] ( 74.28, 64.88) circle (  2.13);

\path[fill=fillColor,fill opacity=0.20] ( 89.33, 54.29) circle (  2.13);

\path[fill=fillColor,fill opacity=0.20] (102.37, 51.28) circle (  2.13);

\path[fill=fillColor,fill opacity=0.20] (150.53, 79.31) circle (  2.13);

\path[fill=fillColor,fill opacity=0.20] (108.39, 57.92) circle (  2.13);

\path[fill=fillColor,fill opacity=0.20] (107.39, 62.59) circle (  2.13);

\path[fill=fillColor,fill opacity=0.20] ( 89.33, 71.52) circle (  2.13);

\path[fill=fillColor,fill opacity=0.20] ( 94.35, 67.16) circle (  2.13);

\path[fill=fillColor,fill opacity=0.20] ( 93.34, 65.50) circle (  2.13);

\path[fill=fillColor,fill opacity=0.20] ( 93.34, 69.65) circle (  2.13);

\path[fill=fillColor,fill opacity=0.20] (100.37, 69.76) circle (  2.13);

\path[fill=fillColor,fill opacity=0.20] (108.39, 73.50) circle (  2.13);

\path[fill=fillColor,fill opacity=0.20] ( 96.35, 67.89) circle (  2.13);

\path[fill=fillColor,fill opacity=0.20] ( 91.34, 59.89) circle (  2.13);

\path[fill=fillColor,fill opacity=0.20] ( 80.30, 60.31) circle (  2.13);

\path[fill=fillColor,fill opacity=0.20] ( 82.31, 55.74) circle (  2.13);

\path[fill=fillColor,fill opacity=0.20] ( 78.30, 55.33) circle (  2.13);

\path[fill=fillColor,fill opacity=0.20] ( 79.30, 65.92) circle (  2.13);

\path[fill=fillColor,fill opacity=0.20] ( 75.29, 75.78) circle (  2.13);

\path[fill=fillColor,fill opacity=0.20] ( 71.27, 76.09) circle (  2.13);

\path[fill=fillColor,fill opacity=0.20] ( 74.28, 63.22) circle (  2.13);

\path[fill=fillColor,fill opacity=0.20] ( 81.31, 47.96) circle (  2.13);

\path[fill=fillColor,fill opacity=0.20] ( 78.30, 53.25) circle (  2.13);

\path[fill=fillColor,fill opacity=0.20] ( 71.27, 67.37) circle (  2.13);

\path[fill=fillColor,fill opacity=0.20] ( 74.28, 66.33) circle (  2.13);

\path[fill=fillColor,fill opacity=0.20] ( 88.33, 59.79) circle (  2.13);

\path[fill=fillColor,fill opacity=0.20] (101.37, 68.62) circle (  2.13);

\path[fill=fillColor,fill opacity=0.20] ( 69.27, 67.68) circle (  2.13);

\path[fill=fillColor,fill opacity=0.20] ( 90.33, 63.11) circle (  2.13);

\path[fill=fillColor,fill opacity=0.20] ( 92.34, 61.76) circle (  2.13);

\path[fill=fillColor,fill opacity=0.20] ( 89.33, 71.11) circle (  2.13);

\path[fill=fillColor,fill opacity=0.20] (101.37, 65.81) circle (  2.13);

\path[fill=fillColor,fill opacity=0.20] (104.38, 61.76) circle (  2.13);

\path[fill=fillColor,fill opacity=0.20] (107.39, 84.81) circle (  2.13);

\path[fill=fillColor,fill opacity=0.20] ( 97.36, 70.28) circle (  2.13);

\path[fill=fillColor,fill opacity=0.20] ( 95.35, 74.12) circle (  2.13);

\path[fill=fillColor,fill opacity=0.20] ( 78.30, 65.61) circle (  2.13);

\path[fill=fillColor,fill opacity=0.20] ( 79.30, 61.66) circle (  2.13);

\path[fill=fillColor,fill opacity=0.20] ( 79.30, 65.19) circle (  2.13);

\path[fill=fillColor,fill opacity=0.20] ( 76.29, 62.49) circle (  2.13);

\path[fill=fillColor,fill opacity=0.20] ( 79.30, 60.83) circle (  2.13);

\path[fill=fillColor,fill opacity=0.20] ( 75.29, 62.70) circle (  2.13);

\path[fill=fillColor,fill opacity=0.20] ( 76.29, 64.98) circle (  2.13);

\path[fill=fillColor,fill opacity=0.20] ( 78.30, 59.27) circle (  2.13);

\path[fill=fillColor,fill opacity=0.20] ( 76.29, 54.39) circle (  2.13);

\path[fill=fillColor,fill opacity=0.20] ( 58.23, 65.29) circle (  2.13);

\path[fill=fillColor,fill opacity=0.20] ( 70.27, 66.54) circle (  2.13);

\path[fill=fillColor,fill opacity=0.20] ( 82.31, 59.48) circle (  2.13);

\path[fill=fillColor,fill opacity=0.20] (106.39, 68.93) circle (  2.13);

\path[fill=fillColor,fill opacity=0.20] (104.38, 63.22) circle (  2.13);

\path[fill=fillColor,fill opacity=0.20] (108.39, 47.23) circle (  2.13);

\path[fill=fillColor,fill opacity=0.20] ( 99.36, 57.20) circle (  2.13);

\path[fill=fillColor,fill opacity=0.20] ( 91.34, 69.86) circle (  2.13);

\path[fill=fillColor,fill opacity=0.20] ( 97.36, 57.92) circle (  2.13);

\path[fill=fillColor,fill opacity=0.20] (100.37, 59.17) circle (  2.13);

\path[fill=fillColor,fill opacity=0.20] (102.37, 73.70) circle (  2.13);

\path[fill=fillColor,fill opacity=0.20] (101.37, 77.54) circle (  2.13);

\path[fill=fillColor,fill opacity=0.20] ( 96.35, 68.93) circle (  2.13);

\path[fill=fillColor,fill opacity=0.20] ( 90.33, 66.12) circle (  2.13);

\path[fill=fillColor,fill opacity=0.20] ( 84.32, 76.40) circle (  2.13);

\path[fill=fillColor,fill opacity=0.20] ( 77.29, 71.73) circle (  2.13);

\path[fill=fillColor,fill opacity=0.20] ( 76.29, 68.62) circle (  2.13);

\path[fill=fillColor,fill opacity=0.20] ( 76.29, 72.35) circle (  2.13);

\path[fill=fillColor,fill opacity=0.20] ( 75.29, 60.10) circle (  2.13);

\path[fill=fillColor,fill opacity=0.20] ( 75.29, 52.32) circle (  2.13);

\path[fill=fillColor,fill opacity=0.20] ( 75.29, 63.84) circle (  2.13);

\path[fill=fillColor,fill opacity=0.20] ( 51.61, 64.67) circle (  2.13);

\path[fill=fillColor,fill opacity=0.20] ( 77.29, 56.57) circle (  2.13);

\path[fill=fillColor,fill opacity=0.20] ( 76.29, 57.51) circle (  2.13);

\path[fill=fillColor,fill opacity=0.20] ( 81.31, 57.71) circle (  2.13);

\path[fill=fillColor,fill opacity=0.20] ( 81.31, 49.72) circle (  2.13);

\path[fill=fillColor,fill opacity=0.20] ( 76.29, 57.92) circle (  2.13);

\path[fill=fillColor,fill opacity=0.20] (151.53, 95.19) circle (  2.13);

\path[fill=fillColor,fill opacity=0.20] (129.46, 67.99) circle (  2.13);

\path[fill=fillColor,fill opacity=0.20] (107.39, 59.69) circle (  2.13);

\path[fill=fillColor,fill opacity=0.20] ( 97.36, 57.92) circle (  2.13);

\path[fill=fillColor,fill opacity=0.20] ( 93.34, 55.43) circle (  2.13);

\path[fill=fillColor,fill opacity=0.20] ( 90.33, 52.52) circle (  2.13);

\path[fill=fillColor,fill opacity=0.20] ( 97.36, 51.90) circle (  2.13);

\path[fill=fillColor,fill opacity=0.20] (100.37, 56.36) circle (  2.13);

\path[fill=fillColor,fill opacity=0.20] (103.38, 63.11) circle (  2.13);

\path[fill=fillColor,fill opacity=0.20] ( 87.33, 60.73) circle (  2.13);

\path[fill=fillColor,fill opacity=0.20] ( 82.31, 55.43) circle (  2.13);

\path[fill=fillColor,fill opacity=0.20] ( 79.30, 70.38) circle (  2.13);

\path[fill=fillColor,fill opacity=0.20] ( 79.30, 72.35) circle (  2.13);

\path[fill=fillColor,fill opacity=0.20] ( 74.28, 66.02) circle (  2.13);

\path[fill=fillColor,fill opacity=0.20] ( 73.28, 66.44) circle (  2.13);

\path[fill=fillColor,fill opacity=0.20] ( 66.96, 53.87) circle (  2.13);

\path[fill=fillColor,fill opacity=0.20] ( 71.27, 47.02) circle (  2.13);

\path[fill=fillColor,fill opacity=0.20] ( 72.28, 62.39) circle (  2.13);

\path[fill=fillColor,fill opacity=0.20] ( 71.27, 62.91) circle (  2.13);

\path[fill=fillColor,fill opacity=0.20] ( 77.29, 54.50) circle (  2.13);

\path[fill=fillColor,fill opacity=0.20] ( 81.31, 59.69) circle (  2.13);

\path[fill=fillColor,fill opacity=0.20] ( 91.34, 58.03) circle (  2.13);

\path[fill=fillColor,fill opacity=0.20] (128.46, 57.09) circle (  2.13);

\path[fill=fillColor,fill opacity=0.20] (140.50, 79.72) circle (  2.13);

\path[fill=fillColor,fill opacity=0.20] ( 85.32, 52.52) circle (  2.13);

\path[fill=fillColor,fill opacity=0.20] ( 99.36, 53.46) circle (  2.13);

\path[fill=fillColor,fill opacity=0.20] ( 94.35, 56.05) circle (  2.13);

\path[fill=fillColor,fill opacity=0.20] (100.37, 49.10) circle (  2.13);

\path[fill=fillColor,fill opacity=0.20] (103.38, 52.52) circle (  2.13);

\path[fill=fillColor,fill opacity=0.20] (112.41, 56.36) circle (  2.13);

\path[fill=fillColor,fill opacity=0.20] (113.41, 57.71) circle (  2.13);

\path[fill=fillColor,fill opacity=0.20] (100.37, 58.44) circle (  2.13);

\path[fill=fillColor,fill opacity=0.20] ( 94.35, 66.64) circle (  2.13);

\path[fill=fillColor,fill opacity=0.20] ( 85.32, 70.07) circle (  2.13);

\path[fill=fillColor,fill opacity=0.20] ( 81.31, 60.73) circle (  2.13);

\path[fill=fillColor,fill opacity=0.20] ( 81.31, 60.93) circle (  2.13);

\path[fill=fillColor,fill opacity=0.20] ( 80.30, 67.37) circle (  2.13);

\path[fill=fillColor,fill opacity=0.20] ( 78.30, 64.98) circle (  2.13);

\path[fill=fillColor,fill opacity=0.20] ( 76.29, 64.05) circle (  2.13);

\path[fill=fillColor,fill opacity=0.20] ( 74.28, 65.61) circle (  2.13);

\path[fill=fillColor,fill opacity=0.20] ( 64.95, 60.21) circle (  2.13);

\path[fill=fillColor,fill opacity=0.20] ( 45.29, 53.46) circle (  2.13);

\path[fill=fillColor,fill opacity=0.20] ( 75.29, 52.84) circle (  2.13);

\path[fill=fillColor,fill opacity=0.20] ( 79.30, 53.46) circle (  2.13);

\path[fill=fillColor,fill opacity=0.20] ( 86.32, 55.02) circle (  2.13);

\path[fill=fillColor,fill opacity=0.20] ( 95.35, 74.95) circle (  2.13);

\path[fill=fillColor,fill opacity=0.20] ( 95.35, 52.84) circle (  2.13);

\path[fill=fillColor,fill opacity=0.20] ( 95.35, 57.40) circle (  2.13);

\path[fill=fillColor,fill opacity=0.20] (101.37, 65.29) circle (  2.13);

\path[fill=fillColor,fill opacity=0.20] (104.38, 64.57) circle (  2.13);

\path[fill=fillColor,fill opacity=0.20] (103.38, 68.10) circle (  2.13);

\path[fill=fillColor,fill opacity=0.20] (106.39, 66.44) circle (  2.13);

\path[fill=fillColor,fill opacity=0.20] (107.39, 64.88) circle (  2.13);

\path[fill=fillColor,fill opacity=0.20] (103.38, 61.56) circle (  2.13);

\path[fill=fillColor,fill opacity=0.20] (101.37, 66.95) circle (  2.13);

\path[fill=fillColor,fill opacity=0.20] ( 92.34, 60.41) circle (  2.13);

\path[fill=fillColor,fill opacity=0.20] ( 88.33, 55.43) circle (  2.13);

\path[fill=fillColor,fill opacity=0.20] ( 82.31, 64.05) circle (  2.13);

\path[fill=fillColor,fill opacity=0.20] ( 76.29, 64.15) circle (  2.13);

\path[fill=fillColor,fill opacity=0.20] ( 77.29, 55.02) circle (  2.13);

\path[fill=fillColor,fill opacity=0.20] ( 81.31, 58.03) circle (  2.13);

\path[fill=fillColor,fill opacity=0.20] ( 75.29, 64.26) circle (  2.13);

\path[fill=fillColor,fill opacity=0.20] ( 74.28, 64.36) circle (  2.13);

\path[fill=fillColor,fill opacity=0.20] ( 73.28, 67.06) circle (  2.13);

\path[fill=fillColor,fill opacity=0.20] ( 78.30, 64.26) circle (  2.13);

\path[fill=fillColor,fill opacity=0.20] ( 80.30, 62.39) circle (  2.13);

\path[fill=fillColor,fill opacity=0.20] ( 79.30, 64.05) circle (  2.13);

\path[fill=fillColor,fill opacity=0.20] ( 79.30, 58.13) circle (  2.13);

\path[fill=fillColor,fill opacity=0.20] ( 91.34, 55.43) circle (  2.13);

\path[fill=fillColor,fill opacity=0.20] (139.49, 75.57) circle (  2.13);

\path[fill=fillColor,fill opacity=0.20] (131.47, 69.65) circle (  2.13);

\path[fill=fillColor,fill opacity=0.20] (113.41, 53.77) circle (  2.13);

\path[fill=fillColor,fill opacity=0.20] ( 88.33, 66.64) circle (  2.13);

\path[fill=fillColor,fill opacity=0.20] ( 92.34, 71.32) circle (  2.13);

\path[fill=fillColor,fill opacity=0.20] ( 95.35, 74.22) circle (  2.13);

\path[fill=fillColor,fill opacity=0.20] ( 89.33, 69.24) circle (  2.13);

\path[fill=fillColor,fill opacity=0.20] ( 96.35, 61.66) circle (  2.13);

\path[fill=fillColor,fill opacity=0.20] (102.37, 55.43) circle (  2.13);

\path[fill=fillColor,fill opacity=0.20] (102.37, 77.54) circle (  2.13);

\path[fill=fillColor,fill opacity=0.20] (101.37,101.42) circle (  2.13);

\path[fill=fillColor,fill opacity=0.20] ( 99.36, 55.85) circle (  2.13);

\path[fill=fillColor,fill opacity=0.20] ( 97.36, 55.43) circle (  2.13);

\path[fill=fillColor,fill opacity=0.20] ( 89.33, 61.45) circle (  2.13);

\path[fill=fillColor,fill opacity=0.20] ( 85.32, 56.99) circle (  2.13);

\path[fill=fillColor,fill opacity=0.20] ( 77.29, 60.93) circle (  2.13);

\path[fill=fillColor,fill opacity=0.20] ( 72.28, 71.11) circle (  2.13);

\path[fill=fillColor,fill opacity=0.20] ( 72.28, 63.63) circle (  2.13);

\path[fill=fillColor,fill opacity=0.20] ( 69.27, 48.99) circle (  2.13);

\path[fill=fillColor,fill opacity=0.20] ( 79.30, 47.02) circle (  2.13);

\path[fill=fillColor,fill opacity=0.20] ( 77.29, 53.56) circle (  2.13);

\path[fill=fillColor,fill opacity=0.20] ( 82.31, 58.75) circle (  2.13);

\path[fill=fillColor,fill opacity=0.20] ( 83.31, 63.32) circle (  2.13);

\path[fill=fillColor,fill opacity=0.20] ( 88.33, 69.65) circle (  2.13);

\path[fill=fillColor,fill opacity=0.20] ( 95.35, 81.28) circle (  2.13);

\path[fill=fillColor,fill opacity=0.20] (116.42, 82.74) circle (  2.13);

\path[fill=fillColor,fill opacity=0.20] (113.41, 70.80) circle (  2.13);

\path[fill=fillColor,fill opacity=0.20] (102.37, 59.48) circle (  2.13);

\path[fill=fillColor,fill opacity=0.20] (100.37, 60.41) circle (  2.13);

\path[fill=fillColor,fill opacity=0.20] ( 88.33, 63.74) circle (  2.13);

\path[fill=fillColor,fill opacity=0.20] ( 92.34, 66.23) circle (  2.13);

\path[fill=fillColor,fill opacity=0.20] ( 94.35, 66.02) circle (  2.13);

\path[fill=fillColor,fill opacity=0.20] ( 94.35, 60.52) circle (  2.13);

\path[fill=fillColor,fill opacity=0.20] ( 93.34, 51.49) circle (  2.13);

\path[fill=fillColor,fill opacity=0.20] (105.38, 53.87) circle (  2.13);

\path[fill=fillColor,fill opacity=0.20] (103.38, 68.82) circle (  2.13);

\path[fill=fillColor,fill opacity=0.20] (104.38, 69.97) circle (  2.13);

\path[fill=fillColor,fill opacity=0.20] (108.39, 61.97) circle (  2.13);

\path[fill=fillColor,fill opacity=0.20] ( 96.35, 72.98) circle (  2.13);

\path[fill=fillColor,fill opacity=0.20] ( 99.36, 53.87) circle (  2.13);

\path[fill=fillColor,fill opacity=0.20] ( 93.34, 58.86) circle (  2.13);

\path[fill=fillColor,fill opacity=0.20] ( 91.34, 72.35) circle (  2.13);

\path[fill=fillColor,fill opacity=0.20] ( 84.32, 73.50) circle (  2.13);

\path[fill=fillColor,fill opacity=0.20] ( 76.29, 68.93) circle (  2.13);

\path[fill=fillColor,fill opacity=0.20] ( 79.30, 67.16) circle (  2.13);

\path[fill=fillColor,fill opacity=0.20] ( 76.29, 68.93) circle (  2.13);

\path[fill=fillColor,fill opacity=0.20] ( 71.27, 61.87) circle (  2.13);

\path[fill=fillColor,fill opacity=0.20] ( 75.29, 56.68) circle (  2.13);

\path[fill=fillColor,fill opacity=0.20] ( 81.31, 57.20) circle (  2.13);

\path[fill=fillColor,fill opacity=0.20] ( 87.33, 56.68) circle (  2.13);

\path[fill=fillColor,fill opacity=0.20] ( 95.35, 60.73) circle (  2.13);

\path[fill=fillColor,fill opacity=0.20] (101.37, 68.93) circle (  2.13);

\path[fill=fillColor,fill opacity=0.20] (110.40, 73.60) circle (  2.13);

\path[fill=fillColor,fill opacity=0.20] ( 99.36, 88.96) circle (  2.13);

\path[fill=fillColor,fill opacity=0.20] (103.38, 59.38) circle (  2.13);

\path[fill=fillColor,fill opacity=0.20] ( 92.34, 63.42) circle (  2.13);

\path[fill=fillColor,fill opacity=0.20] ( 98.36, 70.28) circle (  2.13);

\path[fill=fillColor,fill opacity=0.20] ( 88.33, 67.89) circle (  2.13);

\path[fill=fillColor,fill opacity=0.20] ( 96.35, 57.51) circle (  2.13);

\path[fill=fillColor,fill opacity=0.20] ( 99.36, 58.96) circle (  2.13);

\path[fill=fillColor,fill opacity=0.20] (104.38, 63.01) circle (  2.13);

\path[fill=fillColor,fill opacity=0.20] ( 93.34, 61.14) circle (  2.13);

\path[fill=fillColor,fill opacity=0.20] ( 94.35, 63.11) circle (  2.13);

\path[fill=fillColor,fill opacity=0.20] ( 99.36, 57.09) circle (  2.13);

\path[fill=fillColor,fill opacity=0.20] ( 98.36, 67.68) circle (  2.13);

\path[fill=fillColor,fill opacity=0.20] ( 93.34, 69.45) circle (  2.13);

\path[fill=fillColor,fill opacity=0.20] ( 88.33, 68.93) circle (  2.13);

\path[fill=fillColor,fill opacity=0.20] ( 79.30, 74.22) circle (  2.13);

\path[fill=fillColor,fill opacity=0.20] ( 74.28, 77.34) circle (  2.13);

\path[fill=fillColor,fill opacity=0.20] ( 79.30, 69.13) circle (  2.13);

\path[fill=fillColor,fill opacity=0.20] ( 80.30, 60.21) circle (  2.13);

\path[fill=fillColor,fill opacity=0.20] ( 77.29, 58.75) circle (  2.13);

\path[fill=fillColor,fill opacity=0.20] ( 84.32, 55.74) circle (  2.13);

\path[fill=fillColor,fill opacity=0.20] ( 91.34, 56.26) circle (  2.13);

\path[fill=fillColor,fill opacity=0.20] ( 58.43, 68.51) circle (  2.13);

\path[fill=fillColor,fill opacity=0.20] (110.40, 82.74) circle (  2.13);

\path[fill=fillColor,fill opacity=0.20] (105.38, 86.89) circle (  2.13);

\path[fill=fillColor,fill opacity=0.20] ( 87.33, 68.20) circle (  2.13);

\path[fill=fillColor,fill opacity=0.20] (104.38, 62.59) circle (  2.13);

\path[fill=fillColor,fill opacity=0.20] (100.37, 62.49) circle (  2.13);

\path[fill=fillColor,fill opacity=0.20] ( 97.36, 62.59) circle (  2.13);

\path[fill=fillColor,fill opacity=0.20] (102.37, 67.27) circle (  2.13);

\path[fill=fillColor,fill opacity=0.20] (103.38, 67.47) circle (  2.13);

\path[fill=fillColor,fill opacity=0.20] (104.38, 55.85) circle (  2.13);

\path[fill=fillColor,fill opacity=0.20] ( 96.35, 60.73) circle (  2.13);

\path[fill=fillColor,fill opacity=0.20] ( 94.35, 72.35) circle (  2.13);

\path[fill=fillColor,fill opacity=0.20] (101.37, 65.40) circle (  2.13);

\path[fill=fillColor,fill opacity=0.20] (103.38, 66.12) circle (  2.13);

\path[fill=fillColor,fill opacity=0.20] (102.37, 74.43) circle (  2.13);

\path[fill=fillColor,fill opacity=0.20] ( 89.33, 58.13) circle (  2.13);

\path[fill=fillColor,fill opacity=0.20] ( 96.35, 60.93) circle (  2.13);

\path[fill=fillColor,fill opacity=0.20] ( 96.35, 63.94) circle (  2.13);

\path[fill=fillColor,fill opacity=0.20] (108.39, 65.92) circle (  2.13);

\path[fill=fillColor,fill opacity=0.20] (103.38, 68.41) circle (  2.13);

\path[fill=fillColor,fill opacity=0.20] (111.40, 61.24) circle (  2.13);

\path[fill=fillColor,fill opacity=0.20] (111.40, 61.45) circle (  2.13);

\path[fill=fillColor,fill opacity=0.20] (104.38, 62.49) circle (  2.13);

\path[fill=fillColor,fill opacity=0.20] (101.37, 61.35) circle (  2.13);

\path[fill=fillColor,fill opacity=0.20] ( 90.33, 59.79) circle (  2.13);

\path[fill=fillColor,fill opacity=0.20] ( 83.31, 70.80) circle (  2.13);

\path[fill=fillColor,fill opacity=0.20] ( 84.32, 62.08) circle (  2.13);

\path[fill=fillColor,fill opacity=0.20] ( 91.34, 63.84) circle (  2.13);

\path[fill=fillColor,fill opacity=0.20] ( 89.33, 55.85) circle (  2.13);

\path[fill=fillColor,fill opacity=0.20] ( 87.33, 43.60) circle (  2.13);

\path[fill=fillColor,fill opacity=0.20] ( 56.73, 50.14) circle (  2.13);

\path[fill=fillColor,fill opacity=0.20] (101.37, 62.80) circle (  2.13);

\path[fill=fillColor,fill opacity=0.20] (111.40, 74.53) circle (  2.13);

\path[fill=fillColor,fill opacity=0.20] (114.41, 95.19) circle (  2.13);

\path[fill=fillColor,fill opacity=0.20] (108.39, 70.80) circle (  2.13);

\path[fill=fillColor,fill opacity=0.20] (112.41, 57.61) circle (  2.13);

\path[fill=fillColor,fill opacity=0.20] ( 99.36, 62.28) circle (  2.13);

\path[fill=fillColor,fill opacity=0.20] (102.37, 62.28) circle (  2.13);

\path[fill=fillColor,fill opacity=0.20] ( 98.36, 53.87) circle (  2.13);

\path[fill=fillColor,fill opacity=0.20] ( 97.36, 61.76) circle (  2.13);

\path[fill=fillColor,fill opacity=0.20] ( 95.35, 76.40) circle (  2.13);

\path[fill=fillColor,fill opacity=0.20] ( 93.34, 74.53) circle (  2.13);

\path[fill=fillColor,fill opacity=0.20] (101.37, 63.42) circle (  2.13);

\path[fill=fillColor,fill opacity=0.20] (100.37, 67.58) circle (  2.13);

\path[fill=fillColor,fill opacity=0.20] (107.39, 70.69) circle (  2.13);

\path[fill=fillColor,fill opacity=0.20] (104.38, 58.23) circle (  2.13);

\path[fill=fillColor,fill opacity=0.20] ( 98.36, 50.14) circle (  2.13);

\path[fill=fillColor,fill opacity=0.20] ( 99.36, 52.00) circle (  2.13);

\path[fill=fillColor,fill opacity=0.20] (103.38, 52.00) circle (  2.13);

\path[fill=fillColor,fill opacity=0.20] ( 96.35, 65.09) circle (  2.13);

\path[fill=fillColor,fill opacity=0.20] ( 79.30, 72.87) circle (  2.13);

\path[fill=fillColor,fill opacity=0.20] ( 86.32, 53.98) circle (  2.13);

\path[fill=fillColor,fill opacity=0.20] ( 89.33, 43.60) circle (  2.13);

\path[fill=fillColor,fill opacity=0.20] ( 98.36, 57.92) circle (  2.13);

\path[fill=fillColor,fill opacity=0.20] (104.38, 61.97) circle (  2.13);

\path[fill=fillColor,fill opacity=0.20] (106.39, 60.62) circle (  2.13);

\path[fill=fillColor,fill opacity=0.20] (109.40, 64.36) circle (  2.13);

\path[fill=fillColor,fill opacity=0.20] (101.37, 62.80) circle (  2.13);

\path[fill=fillColor,fill opacity=0.20] ( 98.36, 63.01) circle (  2.13);

\path[fill=fillColor,fill opacity=0.20] ( 79.30, 68.51) circle (  2.13);

\path[fill=fillColor,fill opacity=0.20] ( 78.30, 64.46) circle (  2.13);

\path[fill=fillColor,fill opacity=0.20] ( 89.33, 47.44) circle (  2.13);

\path[fill=fillColor,fill opacity=0.20] ( 84.32, 48.27) circle (  2.13);

\path[fill=fillColor,fill opacity=0.20] ( 83.31, 61.24) circle (  2.13);

\path[fill=fillColor,fill opacity=0.20] (129.46, 75.26) circle (  2.13);

\path[fill=fillColor,fill opacity=0.20] (109.40, 94.16) circle (  2.13);

\path[fill=fillColor,fill opacity=0.20] (125.45, 66.23) circle (  2.13);

\path[fill=fillColor,fill opacity=0.20] (101.37, 57.40) circle (  2.13);

\path[fill=fillColor,fill opacity=0.20] (105.38, 69.45) circle (  2.13);

\path[fill=fillColor,fill opacity=0.20] (104.38, 74.85) circle (  2.13);

\path[fill=fillColor,fill opacity=0.20] (111.40, 66.54) circle (  2.13);

\path[fill=fillColor,fill opacity=0.20] (114.41, 53.15) circle (  2.13);

\path[fill=fillColor,fill opacity=0.20] (107.39, 62.39) circle (  2.13);

\path[fill=fillColor,fill opacity=0.20] ( 95.35, 74.74) circle (  2.13);

\path[fill=fillColor,fill opacity=0.20] (104.38, 55.22) circle (  2.13);

\path[fill=fillColor,fill opacity=0.20] ( 99.36, 45.15) circle (  2.13);

\path[fill=fillColor,fill opacity=0.20] ( 89.33, 62.70) circle (  2.13);

\path[fill=fillColor,fill opacity=0.20] (102.37, 65.29) circle (  2.13);

\path[fill=fillColor,fill opacity=0.20] (111.40, 66.54) circle (  2.13);

\path[fill=fillColor,fill opacity=0.20] (102.37, 74.74) circle (  2.13);

\path[fill=fillColor,fill opacity=0.20] (103.38, 62.59) circle (  2.13);

\path[fill=fillColor,fill opacity=0.20] (106.39, 56.47) circle (  2.13);

\path[fill=fillColor,fill opacity=0.20] (103.38, 73.39) circle (  2.13);

\path[fill=fillColor,fill opacity=0.20] (100.37, 65.40) circle (  2.13);

\path[fill=fillColor,fill opacity=0.20] ( 92.34, 77.54) circle (  2.13);

\path[fill=fillColor,fill opacity=0.20] ( 89.33, 85.85) circle (  2.13);

\path[fill=fillColor,fill opacity=0.20] ( 89.33, 64.46) circle (  2.13);

\path[fill=fillColor,fill opacity=0.20] ( 83.31, 48.99) circle (  2.13);

\path[fill=fillColor,fill opacity=0.20] ( 77.29, 39.23) circle (  2.13);

\path[fill=fillColor,fill opacity=0.20] ( 92.34, 56.36) circle (  2.13);

\path[fill=fillColor,fill opacity=0.20] (110.40, 63.53) circle (  2.13);

\path[fill=fillColor,fill opacity=0.20] ( 93.34, 56.26) circle (  2.13);

\path[fill=fillColor,fill opacity=0.20] (109.40, 85.85) circle (  2.13);

\path[fill=fillColor,fill opacity=0.20] (115.42, 77.13) circle (  2.13);

\path[fill=fillColor,fill opacity=0.20] (111.40, 66.44) circle (  2.13);

\path[fill=fillColor,fill opacity=0.20] (111.40, 51.49) circle (  2.13);

\path[fill=fillColor,fill opacity=0.20] (113.41, 40.17) circle (  2.13);

\path[fill=fillColor,fill opacity=0.20] (104.38, 45.15) circle (  2.13);

\path[fill=fillColor,fill opacity=0.20] (105.38, 64.88) circle (  2.13);

\path[fill=fillColor,fill opacity=0.20] (103.38, 64.88) circle (  2.13);

\path[fill=fillColor,fill opacity=0.20] (105.38, 60.73) circle (  2.13);

\path[fill=fillColor,fill opacity=0.20] (101.37, 75.68) circle (  2.13);

\path[fill=fillColor,fill opacity=0.20] ( 99.36, 76.51) circle (  2.13);

\path[fill=fillColor,fill opacity=0.20] ( 95.35, 74.22) circle (  2.13);

\path[fill=fillColor,fill opacity=0.20] ( 90.33, 60.73) circle (  2.13);

\path[fill=fillColor,fill opacity=0.20] ( 74.28, 70.90) circle (  2.13);

\path[fill=fillColor,fill opacity=0.20] ( 83.31, 59.89) circle (  2.13);

\path[fill=fillColor,fill opacity=0.20] ( 80.30, 46.81) circle (  2.13);

\path[fill=fillColor,fill opacity=0.20] ( 94.35, 42.87) circle (  2.13);

\path[fill=fillColor,fill opacity=0.20] ( 86.32, 51.17) circle (  2.13);

\path[fill=fillColor,fill opacity=0.20] ( 94.35, 66.02) circle (  2.13);

\path[fill=fillColor,fill opacity=0.20] (110.40, 66.23) circle (  2.13);

\path[fill=fillColor,fill opacity=0.20] (115.42, 56.36) circle (  2.13);

\path[fill=fillColor,fill opacity=0.20] (134.48, 68.62) circle (  2.13);

\path[fill=fillColor,fill opacity=0.20] (123.44, 58.75) circle (  2.13);

\path[fill=fillColor,fill opacity=0.20] (114.41, 56.68) circle (  2.13);

\path[fill=fillColor,fill opacity=0.20] (102.37, 51.49) circle (  2.13);

\path[fill=fillColor,fill opacity=0.20] (103.38, 59.48) circle (  2.13);

\path[fill=fillColor,fill opacity=0.20] ( 89.33, 54.81) circle (  2.13);

\path[fill=fillColor,fill opacity=0.20] ( 96.35, 70.17) circle (  2.13);

\path[fill=fillColor,fill opacity=0.20] ( 88.33, 43.49) circle (  2.13);

\path[fill=fillColor,fill opacity=0.20] ( 97.36, 46.19) circle (  2.13);

\path[fill=fillColor,fill opacity=0.20] (101.37, 56.16) circle (  2.13);

\path[fill=fillColor,fill opacity=0.20] ( 83.31, 49.10) circle (  2.13);

\path[fill=fillColor,fill opacity=0.20] ( 98.36, 44.01) circle (  2.13);

\path[fill=fillColor,fill opacity=0.20] (119.43, 57.20) circle (  2.13);

\path[fill=fillColor,fill opacity=0.20] (133.47, 75.16) circle (  2.13);

\path[fill=fillColor,fill opacity=0.20] (134.48, 91.04) circle (  2.13);

\path[fill=fillColor,fill opacity=0.20] (128.46, 70.38) circle (  2.13);

\path[fill=fillColor,fill opacity=0.20] ( 99.36, 60.21) circle (  2.13);

\path[fill=fillColor,fill opacity=0.20] (101.37, 45.88) circle (  2.13);

\path[fill=fillColor,fill opacity=0.20] ( 85.32, 41.31) circle (  2.13);

\path[fill=fillColor,fill opacity=0.20] ( 90.33, 54.08) circle (  2.13);

\path[fill=fillColor,fill opacity=0.20] (113.41, 41.93) circle (  2.13);

\path[fill=fillColor,fill opacity=0.20] (130.46, 53.56) circle (  2.13);

\path[fill=fillColor,fill opacity=0.20] (127.45, 66.75) circle (  2.13);

\path[fill=fillColor,fill opacity=0.20] (124.44, 71.94) circle (  2.13);

\path[fill=fillColor,fill opacity=0.20] (118.43, 58.03) circle (  2.13);

\path[fill=fillColor,fill opacity=0.20] ( 88.33, 57.71) circle (  2.13);

\path[fill=fillColor,fill opacity=0.20] (112.41, 69.45) circle (  2.13);

\path[fill=fillColor,fill opacity=0.20] (105.38, 70.28) circle (  2.13);

\path[fill=fillColor,fill opacity=0.20] ( 94.35, 58.96) circle (  2.13);

\path[fill=fillColor,fill opacity=0.20] (147.52, 68.82) circle (  2.13);

\path[fill=fillColor,fill opacity=0.20] ( 49.30,104.54) circle (  2.13);
\end{scope}
\begin{scope}
\path[clip] (159.87, 34.04) rectangle (277.03,119.86);
\definecolor[named]{fillColor}{rgb}{0.90,0.90,0.90}

\path[fill=fillColor] (159.87, 34.04) rectangle (277.03,119.86);
\definecolor[named]{drawColor}{rgb}{0.95,0.95,0.95}

\path[draw=drawColor,line width= 0.3pt,line join=round] (159.87, 50.55) --
	(277.03, 50.55);

\path[draw=drawColor,line width= 0.3pt,line join=round] (159.87, 71.32) --
	(277.03, 71.32);

\path[draw=drawColor,line width= 0.3pt,line join=round] (159.87, 92.08) --
	(277.03, 92.08);

\path[draw=drawColor,line width= 0.3pt,line join=round] (159.87,112.84) --
	(277.03,112.84);

\path[draw=drawColor,line width= 0.3pt,line join=round] (175.90, 34.04) --
	(175.90,119.86);

\path[draw=drawColor,line width= 0.3pt,line join=round] (200.98, 34.04) --
	(200.98,119.86);

\path[draw=drawColor,line width= 0.3pt,line join=round] (226.06, 34.04) --
	(226.06,119.86);

\path[draw=drawColor,line width= 0.3pt,line join=round] (251.14, 34.04) --
	(251.14,119.86);

\path[draw=drawColor,line width= 0.3pt,line join=round] (276.22, 34.04) --
	(276.22,119.86);
\definecolor[named]{drawColor}{rgb}{1.00,1.00,1.00}

\path[draw=drawColor,line width= 0.6pt,line join=round] (159.87, 40.17) --
	(277.03, 40.17);

\path[draw=drawColor,line width= 0.6pt,line join=round] (159.87, 60.93) --
	(277.03, 60.93);

\path[draw=drawColor,line width= 0.6pt,line join=round] (159.87, 81.70) --
	(277.03, 81.70);

\path[draw=drawColor,line width= 0.6pt,line join=round] (159.87,102.46) --
	(277.03,102.46);

\path[draw=drawColor,line width= 0.6pt,line join=round] (163.36, 34.04) --
	(163.36,119.86);

\path[draw=drawColor,line width= 0.6pt,line join=round] (188.44, 34.04) --
	(188.44,119.86);

\path[draw=drawColor,line width= 0.6pt,line join=round] (213.52, 34.04) --
	(213.52,119.86);

\path[draw=drawColor,line width= 0.6pt,line join=round] (238.60, 34.04) --
	(238.60,119.86);

\path[draw=drawColor,line width= 0.6pt,line join=round] (263.68, 34.04) --
	(263.68,119.86);
\definecolor[named]{fillColor}{rgb}{0.00,0.00,0.00}

\path[fill=fillColor,fill opacity=0.20] (218.54, 70.28) circle (  2.13);

\path[fill=fillColor,fill opacity=0.20] (213.52, 72.25) circle (  2.13);

\path[fill=fillColor,fill opacity=0.20] (221.55, 68.62) circle (  2.13);

\path[fill=fillColor,fill opacity=0.20] (213.52, 59.06) circle (  2.13);

\path[fill=fillColor,fill opacity=0.20] (223.55, 56.16) circle (  2.13);

\path[fill=fillColor,fill opacity=0.20] (235.59, 69.13) circle (  2.13);

\path[fill=fillColor,fill opacity=0.20] (216.53, 72.98) circle (  2.13);

\path[fill=fillColor,fill opacity=0.20] (206.50, 65.29) circle (  2.13);

\path[fill=fillColor,fill opacity=0.20] (201.48, 70.17) circle (  2.13);

\path[fill=fillColor,fill opacity=0.20] (201.48, 71.11) circle (  2.13);

\path[fill=fillColor,fill opacity=0.20] (207.50, 60.10) circle (  2.13);

\path[fill=fillColor,fill opacity=0.20] (204.49, 51.90) circle (  2.13);

\path[fill=fillColor,fill opacity=0.20] (205.50, 58.96) circle (  2.13);

\path[fill=fillColor,fill opacity=0.20] (216.53, 68.62) circle (  2.13);

\path[fill=fillColor,fill opacity=0.20] (228.57, 69.03) circle (  2.13);

\path[fill=fillColor,fill opacity=0.20] (238.60, 65.50) circle (  2.13);

\path[fill=fillColor,fill opacity=0.20] (263.68, 66.02) circle (  2.13);

\path[fill=fillColor,fill opacity=0.20] (210.51, 82.74) circle (  2.13);

\path[fill=fillColor,fill opacity=0.20] (201.48, 77.03) circle (  2.13);

\path[fill=fillColor,fill opacity=0.20] (196.47, 70.07) circle (  2.13);

\path[fill=fillColor,fill opacity=0.20] (196.47, 66.12) circle (  2.13);

\path[fill=fillColor,fill opacity=0.20] (203.49, 59.89) circle (  2.13);

\path[fill=fillColor,fill opacity=0.20] (207.50, 57.09) circle (  2.13);

\path[fill=fillColor,fill opacity=0.20] (210.51, 62.39) circle (  2.13);

\path[fill=fillColor,fill opacity=0.20] (209.51, 69.03) circle (  2.13);

\path[fill=fillColor,fill opacity=0.20] (205.50, 72.04) circle (  2.13);

\path[fill=fillColor,fill opacity=0.20] (217.53, 73.08) circle (  2.13);

\path[fill=fillColor,fill opacity=0.20] (222.55, 68.82) circle (  2.13);

\path[fill=fillColor,fill opacity=0.20] (224.56, 67.16) circle (  2.13);

\path[fill=fillColor,fill opacity=0.20] (207.50, 72.25) circle (  2.13);

\path[fill=fillColor,fill opacity=0.20] (198.47, 68.20) circle (  2.13);

\path[fill=fillColor,fill opacity=0.20] (191.45, 69.24) circle (  2.13);

\path[fill=fillColor,fill opacity=0.20] (191.45, 73.81) circle (  2.13);

\path[fill=fillColor,fill opacity=0.20] (196.47, 69.13) circle (  2.13);

\path[fill=fillColor,fill opacity=0.20] (200.48, 62.28) circle (  2.13);

\path[fill=fillColor,fill opacity=0.20] (201.48, 64.36) circle (  2.13);

\path[fill=fillColor,fill opacity=0.20] (206.50, 70.59) circle (  2.13);

\path[fill=fillColor,fill opacity=0.20] (205.50, 71.21) circle (  2.13);

\path[fill=fillColor,fill opacity=0.20] (198.47, 64.15) circle (  2.13);

\path[fill=fillColor,fill opacity=0.20] (200.48, 71.83) circle (  2.13);

\path[fill=fillColor,fill opacity=0.20] (210.51, 83.77) circle (  2.13);

\path[fill=fillColor,fill opacity=0.20] (217.53, 81.70) circle (  2.13);

\path[fill=fillColor,fill opacity=0.20] (206.50, 64.77) circle (  2.13);

\path[fill=fillColor,fill opacity=0.20] (194.46, 49.72) circle (  2.13);

\path[fill=fillColor,fill opacity=0.20] (191.45, 55.22) circle (  2.13);

\path[fill=fillColor,fill opacity=0.20] (194.46, 73.81) circle (  2.13);

\path[fill=fillColor,fill opacity=0.20] (196.47, 78.06) circle (  2.13);

\path[fill=fillColor,fill opacity=0.20] (198.47, 72.87) circle (  2.13);

\path[fill=fillColor,fill opacity=0.20] (201.48, 67.99) circle (  2.13);

\path[fill=fillColor,fill opacity=0.20] (202.49, 68.41) circle (  2.13);

\path[fill=fillColor,fill opacity=0.20] (208.51, 73.08) circle (  2.13);

\path[fill=fillColor,fill opacity=0.20] (201.48, 69.97) circle (  2.13);

\path[fill=fillColor,fill opacity=0.20] (197.47, 77.65) circle (  2.13);

\path[fill=fillColor,fill opacity=0.20] (203.49, 79.83) circle (  2.13);

\path[fill=fillColor,fill opacity=0.20] (207.50, 73.70) circle (  2.13);

\path[fill=fillColor,fill opacity=0.20] (212.52, 66.33) circle (  2.13);

\path[fill=fillColor,fill opacity=0.20] (216.53, 70.48) circle (  2.13);

\path[fill=fillColor,fill opacity=0.20] (214.53, 83.77) circle (  2.13);

\path[fill=fillColor,fill opacity=0.20] (209.51, 65.50) circle (  2.13);

\path[fill=fillColor,fill opacity=0.20] (200.48, 56.36) circle (  2.13);

\path[fill=fillColor,fill opacity=0.20] (197.47, 60.21) circle (  2.13);

\path[fill=fillColor,fill opacity=0.20] (197.47, 71.83) circle (  2.13);

\path[fill=fillColor,fill opacity=0.20] (199.48, 77.65) circle (  2.13);

\path[fill=fillColor,fill opacity=0.20] (201.48, 74.74) circle (  2.13);

\path[fill=fillColor,fill opacity=0.20] (204.49, 65.92) circle (  2.13);

\path[fill=fillColor,fill opacity=0.20] (207.50, 66.02) circle (  2.13);

\path[fill=fillColor,fill opacity=0.20] (220.54, 79.41) circle (  2.13);

\path[fill=fillColor,fill opacity=0.20] (197.47, 75.99) circle (  2.13);

\path[fill=fillColor,fill opacity=0.20] (193.46, 74.01) circle (  2.13);

\path[fill=fillColor,fill opacity=0.20] (195.46, 83.77) circle (  2.13);

\path[fill=fillColor,fill opacity=0.20] (196.47, 78.06) circle (  2.13);

\path[fill=fillColor,fill opacity=0.20] (196.47, 68.20) circle (  2.13);

\path[fill=fillColor,fill opacity=0.20] (204.49, 62.91) circle (  2.13);

\path[fill=fillColor,fill opacity=0.20] (204.49, 57.82) circle (  2.13);

\path[fill=fillColor,fill opacity=0.20] (208.51, 57.40) circle (  2.13);

\path[fill=fillColor,fill opacity=0.20] (213.52, 92.08) circle (  2.13);

\path[fill=fillColor,fill opacity=0.20] (211.52, 72.56) circle (  2.13);

\path[fill=fillColor,fill opacity=0.20] (201.48, 73.70) circle (  2.13);

\path[fill=fillColor,fill opacity=0.20] (199.48, 74.01) circle (  2.13);

\path[fill=fillColor,fill opacity=0.20] (198.47, 72.56) circle (  2.13);

\path[fill=fillColor,fill opacity=0.20] (199.48, 72.46) circle (  2.13);

\path[fill=fillColor,fill opacity=0.20] (204.49, 72.35) circle (  2.13);

\path[fill=fillColor,fill opacity=0.20] (207.50, 67.47) circle (  2.13);

\path[fill=fillColor,fill opacity=0.20] (212.52, 65.40) circle (  2.13);

\path[fill=fillColor,fill opacity=0.20] (200.48, 67.99) circle (  2.13);

\path[fill=fillColor,fill opacity=0.20] (193.46, 71.83) circle (  2.13);

\path[fill=fillColor,fill opacity=0.20] (196.47, 72.98) circle (  2.13);

\path[fill=fillColor,fill opacity=0.20] (198.47, 61.14) circle (  2.13);

\path[fill=fillColor,fill opacity=0.20] (200.48, 56.57) circle (  2.13);

\path[fill=fillColor,fill opacity=0.20] (201.48, 59.17) circle (  2.13);

\path[fill=fillColor,fill opacity=0.20] (201.48, 55.43) circle (  2.13);

\path[fill=fillColor,fill opacity=0.20] (211.52, 49.41) circle (  2.13);

\path[fill=fillColor,fill opacity=0.20] (209.51, 78.17) circle (  2.13);

\path[fill=fillColor,fill opacity=0.20] (199.48, 77.96) circle (  2.13);

\path[fill=fillColor,fill opacity=0.20] (199.48, 78.79) circle (  2.13);

\path[fill=fillColor,fill opacity=0.20] (202.49, 76.51) circle (  2.13);

\path[fill=fillColor,fill opacity=0.20] (200.48, 72.77) circle (  2.13);

\path[fill=fillColor,fill opacity=0.20] (203.49, 69.65) circle (  2.13);

\path[fill=fillColor,fill opacity=0.20] (206.50, 66.75) circle (  2.13);

\path[fill=fillColor,fill opacity=0.20] (212.52, 63.32) circle (  2.13);

\path[fill=fillColor,fill opacity=0.20] (202.49, 79.72) circle (  2.13);

\path[fill=fillColor,fill opacity=0.20] (196.47, 55.02) circle (  2.13);

\path[fill=fillColor,fill opacity=0.20] (198.47, 57.51) circle (  2.13);

\path[fill=fillColor,fill opacity=0.20] (198.47, 50.45) circle (  2.13);

\path[fill=fillColor,fill opacity=0.20] (199.48, 48.16) circle (  2.13);

\path[fill=fillColor,fill opacity=0.20] (203.49, 58.13) circle (  2.13);

\path[fill=fillColor,fill opacity=0.20] (205.50, 56.05) circle (  2.13);

\path[fill=fillColor,fill opacity=0.20] (211.52, 49.10) circle (  2.13);

\path[fill=fillColor,fill opacity=0.20] (215.53, 49.31) circle (  2.13);

\path[fill=fillColor,fill opacity=0.20] (220.54, 47.02) circle (  2.13);

\path[fill=fillColor,fill opacity=0.20] (208.51, 82.74) circle (  2.13);

\path[fill=fillColor,fill opacity=0.20] (201.48, 75.05) circle (  2.13);

\path[fill=fillColor,fill opacity=0.20] (201.48, 75.26) circle (  2.13);

\path[fill=fillColor,fill opacity=0.20] (204.49, 76.61) circle (  2.13);

\path[fill=fillColor,fill opacity=0.20] (202.49, 69.86) circle (  2.13);

\path[fill=fillColor,fill opacity=0.20] (203.49, 58.13) circle (  2.13);

\path[fill=fillColor,fill opacity=0.20] (207.50, 56.57) circle (  2.13);

\path[fill=fillColor,fill opacity=0.20] (210.51, 59.17) circle (  2.13);

\path[fill=fillColor,fill opacity=0.20] (217.53, 62.39) circle (  2.13);

\path[fill=fillColor,fill opacity=0.20] (198.47, 49.10) circle (  2.13);

\path[fill=fillColor,fill opacity=0.20] (195.46, 41.21) circle (  2.13);

\path[fill=fillColor,fill opacity=0.20] (201.48, 50.55) circle (  2.13);

\path[fill=fillColor,fill opacity=0.20] (202.49, 42.66) circle (  2.13);

\path[fill=fillColor,fill opacity=0.20] (203.49, 47.44) circle (  2.13);

\path[fill=fillColor,fill opacity=0.20] (206.50, 57.61) circle (  2.13);

\path[fill=fillColor,fill opacity=0.20] (208.51, 46.50) circle (  2.13);

\path[fill=fillColor,fill opacity=0.20] (214.53, 41.93) circle (  2.13);

\path[fill=fillColor,fill opacity=0.20] (219.54, 52.11) circle (  2.13);

\path[fill=fillColor,fill opacity=0.20] (223.55, 50.65) circle (  2.13);

\path[fill=fillColor,fill opacity=0.20] (209.51, 87.93) circle (  2.13);

\path[fill=fillColor,fill opacity=0.20] (202.49, 70.80) circle (  2.13);

\path[fill=fillColor,fill opacity=0.20] (199.48, 68.10) circle (  2.13);

\path[fill=fillColor,fill opacity=0.20] (198.47, 66.75) circle (  2.13);

\path[fill=fillColor,fill opacity=0.20] (200.48, 57.82) circle (  2.13);

\path[fill=fillColor,fill opacity=0.20] (203.49, 49.62) circle (  2.13);

\path[fill=fillColor,fill opacity=0.20] (209.51, 53.04) circle (  2.13);

\path[fill=fillColor,fill opacity=0.20] (213.52, 56.26) circle (  2.13);

\path[fill=fillColor,fill opacity=0.20] (215.53, 51.80) circle (  2.13);

\path[fill=fillColor,fill opacity=0.20] (226.56, 67.37) circle (  2.13);

\path[fill=fillColor,fill opacity=0.20] (205.50,102.46) circle (  2.13);

\path[fill=fillColor,fill opacity=0.20] (199.48, 41.83) circle (  2.13);

\path[fill=fillColor,fill opacity=0.20] (205.50, 49.51) circle (  2.13);

\path[fill=fillColor,fill opacity=0.20] (203.49, 46.61) circle (  2.13);

\path[fill=fillColor,fill opacity=0.20] (202.49, 51.17) circle (  2.13);

\path[fill=fillColor,fill opacity=0.20] (204.49, 48.89) circle (  2.13);

\path[fill=fillColor,fill opacity=0.20] (210.51, 39.75) circle (  2.13);

\path[fill=fillColor,fill opacity=0.20] (218.54, 48.68) circle (  2.13);

\path[fill=fillColor,fill opacity=0.20] (215.53, 46.50) circle (  2.13);

\path[fill=fillColor,fill opacity=0.20] (239.61, 57.20) circle (  2.13);

\path[fill=fillColor,fill opacity=0.20] (202.49, 67.37) circle (  2.13);

\path[fill=fillColor,fill opacity=0.20] (197.47, 60.00) circle (  2.13);

\path[fill=fillColor,fill opacity=0.20] (195.46, 50.55) circle (  2.13);

\path[fill=fillColor,fill opacity=0.20] (198.47, 45.05) circle (  2.13);

\path[fill=fillColor,fill opacity=0.20] (204.49, 53.25) circle (  2.13);

\path[fill=fillColor,fill opacity=0.20] (208.51, 57.30) circle (  2.13);

\path[fill=fillColor,fill opacity=0.20] (215.53, 54.50) circle (  2.13);

\path[fill=fillColor,fill opacity=0.20] (213.52, 51.90) circle (  2.13);

\path[fill=fillColor,fill opacity=0.20] (220.54, 57.51) circle (  2.13);

\path[fill=fillColor,fill opacity=0.20] (205.50, 62.49) circle (  2.13);

\path[fill=fillColor,fill opacity=0.20] (199.48, 45.98) circle (  2.13);

\path[fill=fillColor,fill opacity=0.20] (200.48, 47.12) circle (  2.13);

\path[fill=fillColor,fill opacity=0.20] (204.49, 42.25) circle (  2.13);

\path[fill=fillColor,fill opacity=0.20] (202.49, 46.09) circle (  2.13);

\path[fill=fillColor,fill opacity=0.20] (204.49, 52.42) circle (  2.13);

\path[fill=fillColor,fill opacity=0.20] (209.51, 46.40) circle (  2.13);

\path[fill=fillColor,fill opacity=0.20] (211.52, 42.45) circle (  2.13);

\path[fill=fillColor,fill opacity=0.20] (212.52, 45.46) circle (  2.13);

\path[fill=fillColor,fill opacity=0.20] (211.52, 45.15) circle (  2.13);

\path[fill=fillColor,fill opacity=0.20] (221.55, 44.84) circle (  2.13);

\path[fill=fillColor,fill opacity=0.20] (240.61, 64.46) circle (  2.13);

\path[fill=fillColor,fill opacity=0.20] (211.52, 69.34) circle (  2.13);

\path[fill=fillColor,fill opacity=0.20] (205.50, 56.78) circle (  2.13);

\path[fill=fillColor,fill opacity=0.20] (198.47, 46.81) circle (  2.13);

\path[fill=fillColor,fill opacity=0.20] (202.49, 47.64) circle (  2.13);

\path[fill=fillColor,fill opacity=0.20] (205.50, 56.68) circle (  2.13);

\path[fill=fillColor,fill opacity=0.20] (203.49, 54.60) circle (  2.13);

\path[fill=fillColor,fill opacity=0.20] (209.51, 52.84) circle (  2.13);

\path[fill=fillColor,fill opacity=0.20] (214.53, 58.13) circle (  2.13);

\path[fill=fillColor,fill opacity=0.20] (220.54, 61.56) circle (  2.13);

\path[fill=fillColor,fill opacity=0.20] (211.52, 85.85) circle (  2.13);

\path[fill=fillColor,fill opacity=0.20] (205.50, 46.19) circle (  2.13);

\path[fill=fillColor,fill opacity=0.20] (204.49, 50.86) circle (  2.13);

\path[fill=fillColor,fill opacity=0.20] (206.50, 45.26) circle (  2.13);

\path[fill=fillColor,fill opacity=0.20] (199.48, 39.44) circle (  2.13);

\path[fill=fillColor,fill opacity=0.20] (203.49, 46.71) circle (  2.13);

\path[fill=fillColor,fill opacity=0.20] (202.49, 51.38) circle (  2.13);

\path[fill=fillColor,fill opacity=0.20] (203.49, 52.63) circle (  2.13);

\path[fill=fillColor,fill opacity=0.20] (210.51, 55.02) circle (  2.13);

\path[fill=fillColor,fill opacity=0.20] (212.52, 50.97) circle (  2.13);

\path[fill=fillColor,fill opacity=0.20] (214.53, 45.98) circle (  2.13);

\path[fill=fillColor,fill opacity=0.20] (223.55, 56.57) circle (  2.13);

\path[fill=fillColor,fill opacity=0.20] (210.51, 65.40) circle (  2.13);

\path[fill=fillColor,fill opacity=0.20] (203.49, 51.17) circle (  2.13);

\path[fill=fillColor,fill opacity=0.20] (204.49, 53.87) circle (  2.13);

\path[fill=fillColor,fill opacity=0.20] (206.50, 54.70) circle (  2.13);

\path[fill=fillColor,fill opacity=0.20] (204.49, 52.00) circle (  2.13);

\path[fill=fillColor,fill opacity=0.20] (204.49, 55.95) circle (  2.13);

\path[fill=fillColor,fill opacity=0.20] (210.51, 61.35) circle (  2.13);

\path[fill=fillColor,fill opacity=0.20] (212.52, 63.01) circle (  2.13);

\path[fill=fillColor,fill opacity=0.20] (219.54, 65.40) circle (  2.13);

\path[fill=fillColor,fill opacity=0.20] (202.49, 59.27) circle (  2.13);

\path[fill=fillColor,fill opacity=0.20] (208.51, 50.65) circle (  2.13);

\path[fill=fillColor,fill opacity=0.20] (208.51, 51.38) circle (  2.13);

\path[fill=fillColor,fill opacity=0.20] (205.50, 44.74) circle (  2.13);

\path[fill=fillColor,fill opacity=0.20] (200.48, 44.94) circle (  2.13);

\path[fill=fillColor,fill opacity=0.20] (199.48, 52.63) circle (  2.13);

\path[fill=fillColor,fill opacity=0.20] (210.51, 54.91) circle (  2.13);

\path[fill=fillColor,fill opacity=0.20] (203.49, 58.03) circle (  2.13);

\path[fill=fillColor,fill opacity=0.20] (208.51, 59.48) circle (  2.13);

\path[fill=fillColor,fill opacity=0.20] (211.52, 49.51) circle (  2.13);

\path[fill=fillColor,fill opacity=0.20] (217.53, 46.61) circle (  2.13);

\path[fill=fillColor,fill opacity=0.20] (210.51, 61.35) circle (  2.13);

\path[fill=fillColor,fill opacity=0.20] (207.50, 57.09) circle (  2.13);

\path[fill=fillColor,fill opacity=0.20] (207.50, 58.13) circle (  2.13);

\path[fill=fillColor,fill opacity=0.20] (207.50, 60.73) circle (  2.13);

\path[fill=fillColor,fill opacity=0.20] (206.50, 63.42) circle (  2.13);

\path[fill=fillColor,fill opacity=0.20] (203.49, 58.65) circle (  2.13);

\path[fill=fillColor,fill opacity=0.20] (206.50, 57.92) circle (  2.13);

\path[fill=fillColor,fill opacity=0.20] (210.51, 60.52) circle (  2.13);

\path[fill=fillColor,fill opacity=0.20] (220.54, 58.86) circle (  2.13);

\path[fill=fillColor,fill opacity=0.20] (230.58, 68.93) circle (  2.13);

\path[fill=fillColor,fill opacity=0.20] (199.48, 62.28) circle (  2.13);

\path[fill=fillColor,fill opacity=0.20] (200.48, 55.33) circle (  2.13);

\path[fill=fillColor,fill opacity=0.20] (206.50, 56.78) circle (  2.13);

\path[fill=fillColor,fill opacity=0.20] (205.50, 59.48) circle (  2.13);

\path[fill=fillColor,fill opacity=0.20] (197.47, 51.69) circle (  2.13);

\path[fill=fillColor,fill opacity=0.20] (200.48, 46.29) circle (  2.13);

\path[fill=fillColor,fill opacity=0.20] (195.46, 53.87) circle (  2.13);

\path[fill=fillColor,fill opacity=0.20] (209.51, 58.55) circle (  2.13);

\path[fill=fillColor,fill opacity=0.20] (206.50, 55.85) circle (  2.13);

\path[fill=fillColor,fill opacity=0.20] (212.52, 53.15) circle (  2.13);

\path[fill=fillColor,fill opacity=0.20] (214.53, 48.79) circle (  2.13);

\path[fill=fillColor,fill opacity=0.20] (221.55, 52.00) circle (  2.13);

\path[fill=fillColor,fill opacity=0.20] (212.52, 69.97) circle (  2.13);

\path[fill=fillColor,fill opacity=0.20] (208.51, 67.89) circle (  2.13);

\path[fill=fillColor,fill opacity=0.20] (207.50, 65.29) circle (  2.13);

\path[fill=fillColor,fill opacity=0.20] (205.50, 62.49) circle (  2.13);

\path[fill=fillColor,fill opacity=0.20] (206.50, 55.74) circle (  2.13);

\path[fill=fillColor,fill opacity=0.20] (207.50, 57.09) circle (  2.13);

\path[fill=fillColor,fill opacity=0.20] (212.52, 63.01) circle (  2.13);

\path[fill=fillColor,fill opacity=0.20] (216.53, 59.89) circle (  2.13);

\path[fill=fillColor,fill opacity=0.20] (221.55, 56.68) circle (  2.13);

\path[fill=fillColor,fill opacity=0.20] (201.48, 59.79) circle (  2.13);

\path[fill=fillColor,fill opacity=0.20] (195.46, 59.38) circle (  2.13);

\path[fill=fillColor,fill opacity=0.20] (201.48, 58.86) circle (  2.13);

\path[fill=fillColor,fill opacity=0.20] (197.47, 59.58) circle (  2.13);

\path[fill=fillColor,fill opacity=0.20] (196.47, 63.63) circle (  2.13);

\path[fill=fillColor,fill opacity=0.20] (196.47, 57.09) circle (  2.13);

\path[fill=fillColor,fill opacity=0.20] (201.48, 47.85) circle (  2.13);

\path[fill=fillColor,fill opacity=0.20] (206.50, 50.65) circle (  2.13);

\path[fill=fillColor,fill opacity=0.20] (206.50, 53.56) circle (  2.13);

\path[fill=fillColor,fill opacity=0.20] (208.51, 47.23) circle (  2.13);

\path[fill=fillColor,fill opacity=0.20] (216.53, 47.23) circle (  2.13);

\path[fill=fillColor,fill opacity=0.20] (223.55, 55.33) circle (  2.13);

\path[fill=fillColor,fill opacity=0.20] (212.52, 81.18) circle (  2.13);

\path[fill=fillColor,fill opacity=0.20] (203.49, 62.80) circle (  2.13);

\path[fill=fillColor,fill opacity=0.20] (205.50, 54.81) circle (  2.13);

\path[fill=fillColor,fill opacity=0.20] (210.51, 53.77) circle (  2.13);

\path[fill=fillColor,fill opacity=0.20] (207.50, 55.74) circle (  2.13);

\path[fill=fillColor,fill opacity=0.20] (207.50, 59.58) circle (  2.13);

\path[fill=fillColor,fill opacity=0.20] (208.51, 58.03) circle (  2.13);

\path[fill=fillColor,fill opacity=0.20] (205.50, 55.22) circle (  2.13);

\path[fill=fillColor,fill opacity=0.20] (214.53, 64.05) circle (  2.13);

\path[fill=fillColor,fill opacity=0.20] (197.47, 61.97) circle (  2.13);

\path[fill=fillColor,fill opacity=0.20] (199.48, 60.83) circle (  2.13);

\path[fill=fillColor,fill opacity=0.20] (196.47, 57.71) circle (  2.13);

\path[fill=fillColor,fill opacity=0.20] (195.46, 55.22) circle (  2.13);

\path[fill=fillColor,fill opacity=0.20] (198.47, 53.67) circle (  2.13);

\path[fill=fillColor,fill opacity=0.20] (196.47, 57.30) circle (  2.13);

\path[fill=fillColor,fill opacity=0.20] (202.49, 59.38) circle (  2.13);

\path[fill=fillColor,fill opacity=0.20] (202.49, 55.02) circle (  2.13);

\path[fill=fillColor,fill opacity=0.20] (205.50, 48.37) circle (  2.13);

\path[fill=fillColor,fill opacity=0.20] (206.50, 41.83) circle (  2.13);

\path[fill=fillColor,fill opacity=0.20] (217.53, 45.88) circle (  2.13);

\path[fill=fillColor,fill opacity=0.20] (226.56, 63.01) circle (  2.13);

\path[fill=fillColor,fill opacity=0.20] (205.50, 56.88) circle (  2.13);

\path[fill=fillColor,fill opacity=0.20] (208.51, 53.15) circle (  2.13);

\path[fill=fillColor,fill opacity=0.20] (209.51, 47.33) circle (  2.13);

\path[fill=fillColor,fill opacity=0.20] (212.52, 45.98) circle (  2.13);

\path[fill=fillColor,fill opacity=0.20] (212.52, 50.97) circle (  2.13);

\path[fill=fillColor,fill opacity=0.20] (210.51, 50.03) circle (  2.13);

\path[fill=fillColor,fill opacity=0.20] (205.50, 53.25) circle (  2.13);

\path[fill=fillColor,fill opacity=0.20] (211.52, 65.92) circle (  2.13);

\path[fill=fillColor,fill opacity=0.20] (222.55, 77.23) circle (  2.13);

\path[fill=fillColor,fill opacity=0.20] (210.51, 59.17) circle (  2.13);

\path[fill=fillColor,fill opacity=0.20] (198.47, 65.29) circle (  2.13);

\path[fill=fillColor,fill opacity=0.20] (198.47, 62.49) circle (  2.13);

\path[fill=fillColor,fill opacity=0.20] (201.48, 53.67) circle (  2.13);

\path[fill=fillColor,fill opacity=0.20] (197.47, 51.07) circle (  2.13);

\path[fill=fillColor,fill opacity=0.20] (199.48, 46.29) circle (  2.13);

\path[fill=fillColor,fill opacity=0.20] (200.48, 44.53) circle (  2.13);

\path[fill=fillColor,fill opacity=0.20] (200.48, 58.03) circle (  2.13);

\path[fill=fillColor,fill opacity=0.20] (199.48, 68.62) circle (  2.13);

\path[fill=fillColor,fill opacity=0.20] (203.49, 60.93) circle (  2.13);

\path[fill=fillColor,fill opacity=0.20] (212.52, 47.54) circle (  2.13);

\path[fill=fillColor,fill opacity=0.20] (217.53, 44.43) circle (  2.13);

\path[fill=fillColor,fill opacity=0.20] (211.52, 51.69) circle (  2.13);

\path[fill=fillColor,fill opacity=0.20] (214.53, 69.24) circle (  2.13);

\path[fill=fillColor,fill opacity=0.20] (206.50, 52.00) circle (  2.13);

\path[fill=fillColor,fill opacity=0.20] (205.50, 45.57) circle (  2.13);

\path[fill=fillColor,fill opacity=0.20] (206.50, 47.96) circle (  2.13);

\path[fill=fillColor,fill opacity=0.20] (210.51, 52.94) circle (  2.13);

\path[fill=fillColor,fill opacity=0.20] (214.53, 49.72) circle (  2.13);

\path[fill=fillColor,fill opacity=0.20] (215.53, 45.26) circle (  2.13);

\path[fill=fillColor,fill opacity=0.20] (211.52, 56.57) circle (  2.13);

\path[fill=fillColor,fill opacity=0.20] (218.54, 70.90) circle (  2.13);

\path[fill=fillColor,fill opacity=0.20] (217.53, 76.51) circle (  2.13);

\path[fill=fillColor,fill opacity=0.20] (207.50, 42.66) circle (  2.13);

\path[fill=fillColor,fill opacity=0.20] (203.49, 57.82) circle (  2.13);

\path[fill=fillColor,fill opacity=0.20] (194.46, 64.26) circle (  2.13);

\path[fill=fillColor,fill opacity=0.20] (196.47, 53.98) circle (  2.13);

\path[fill=fillColor,fill opacity=0.20] (196.47, 51.69) circle (  2.13);

\path[fill=fillColor,fill opacity=0.20] (198.47, 53.25) circle (  2.13);

\path[fill=fillColor,fill opacity=0.20] (199.48, 48.16) circle (  2.13);

\path[fill=fillColor,fill opacity=0.20] (200.48, 49.31) circle (  2.13);

\path[fill=fillColor,fill opacity=0.20] (206.50, 60.52) circle (  2.13);

\path[fill=fillColor,fill opacity=0.20] (207.50, 63.11) circle (  2.13);

\path[fill=fillColor,fill opacity=0.20] (216.53, 51.80) circle (  2.13);

\path[fill=fillColor,fill opacity=0.20] (212.52, 45.78) circle (  2.13);

\path[fill=fillColor,fill opacity=0.20] (229.57, 58.13) circle (  2.13);

\path[fill=fillColor,fill opacity=0.20] (209.51, 61.87) circle (  2.13);

\path[fill=fillColor,fill opacity=0.20] (200.48, 59.48) circle (  2.13);

\path[fill=fillColor,fill opacity=0.20] (202.49, 67.37) circle (  2.13);

\path[fill=fillColor,fill opacity=0.20] (201.48, 58.55) circle (  2.13);

\path[fill=fillColor,fill opacity=0.20] (209.51, 52.11) circle (  2.13);

\path[fill=fillColor,fill opacity=0.20] (213.52, 56.99) circle (  2.13);

\path[fill=fillColor,fill opacity=0.20] (218.54, 55.64) circle (  2.13);

\path[fill=fillColor,fill opacity=0.20] (215.53, 57.71) circle (  2.13);

\path[fill=fillColor,fill opacity=0.20] (217.53, 61.87) circle (  2.13);

\path[fill=fillColor,fill opacity=0.20] (218.54, 53.46) circle (  2.13);

\path[fill=fillColor,fill opacity=0.20] (226.56, 51.80) circle (  2.13);

\path[fill=fillColor,fill opacity=0.20] (213.52, 55.22) circle (  2.13);

\path[fill=fillColor,fill opacity=0.20] (204.49, 55.74) circle (  2.13);

\path[fill=fillColor,fill opacity=0.20] (201.48, 63.53) circle (  2.13);

\path[fill=fillColor,fill opacity=0.20] (197.47, 63.42) circle (  2.13);

\path[fill=fillColor,fill opacity=0.20] (197.47, 56.05) circle (  2.13);

\path[fill=fillColor,fill opacity=0.20] (198.47, 51.90) circle (  2.13);

\path[fill=fillColor,fill opacity=0.20] (197.47, 58.44) circle (  2.13);

\path[fill=fillColor,fill opacity=0.20] (196.47, 62.28) circle (  2.13);

\path[fill=fillColor,fill opacity=0.20] (198.47, 55.33) circle (  2.13);

\path[fill=fillColor,fill opacity=0.20] (206.50, 51.49) circle (  2.13);

\path[fill=fillColor,fill opacity=0.20] (210.51, 54.91) circle (  2.13);

\path[fill=fillColor,fill opacity=0.20] (214.53, 50.24) circle (  2.13);

\path[fill=fillColor,fill opacity=0.20] (219.54, 42.76) circle (  2.13);

\path[fill=fillColor,fill opacity=0.20] (234.59, 54.70) circle (  2.13);

\path[fill=fillColor,fill opacity=0.20] (205.50, 80.87) circle (  2.13);

\path[fill=fillColor,fill opacity=0.20] (205.50, 64.67) circle (  2.13);

\path[fill=fillColor,fill opacity=0.20] (203.49, 51.28) circle (  2.13);

\path[fill=fillColor,fill opacity=0.20] (206.50, 59.38) circle (  2.13);

\path[fill=fillColor,fill opacity=0.20] (217.53, 56.26) circle (  2.13);

\path[fill=fillColor,fill opacity=0.20] (218.54, 60.83) circle (  2.13);

\path[fill=fillColor,fill opacity=0.20] (215.53, 62.39) circle (  2.13);

\path[fill=fillColor,fill opacity=0.20] (218.54, 58.34) circle (  2.13);

\path[fill=fillColor,fill opacity=0.20] (219.54, 62.49) circle (  2.13);

\path[fill=fillColor,fill opacity=0.20] (223.55, 64.88) circle (  2.13);

\path[fill=fillColor,fill opacity=0.20] (228.57, 60.41) circle (  2.13);

\path[fill=fillColor,fill opacity=0.20] (211.52, 61.14) circle (  2.13);

\path[fill=fillColor,fill opacity=0.20] (208.51, 58.65) circle (  2.13);

\path[fill=fillColor,fill opacity=0.20] (203.49, 61.56) circle (  2.13);

\path[fill=fillColor,fill opacity=0.20] (200.48, 71.21) circle (  2.13);

\path[fill=fillColor,fill opacity=0.20] (199.48, 74.95) circle (  2.13);

\path[fill=fillColor,fill opacity=0.20] (200.48, 63.11) circle (  2.13);

\path[fill=fillColor,fill opacity=0.20] (202.49, 53.04) circle (  2.13);

\path[fill=fillColor,fill opacity=0.20] (203.49, 55.74) circle (  2.13);

\path[fill=fillColor,fill opacity=0.20] (199.48, 60.83) circle (  2.13);

\path[fill=fillColor,fill opacity=0.20] (196.47, 59.79) circle (  2.13);

\path[fill=fillColor,fill opacity=0.20] (198.47, 53.87) circle (  2.13);

\path[fill=fillColor,fill opacity=0.20] (207.50, 47.44) circle (  2.13);

\path[fill=fillColor,fill opacity=0.20] (211.52, 47.64) circle (  2.13);

\path[fill=fillColor,fill opacity=0.20] (224.56, 53.25) circle (  2.13);

\path[fill=fillColor,fill opacity=0.20] (220.54, 64.15) circle (  2.13);

\path[fill=fillColor,fill opacity=0.20] (206.50, 49.20) circle (  2.13);

\path[fill=fillColor,fill opacity=0.20] (202.49, 51.38) circle (  2.13);

\path[fill=fillColor,fill opacity=0.20] (210.51, 55.74) circle (  2.13);

\path[fill=fillColor,fill opacity=0.20] (215.53, 58.65) circle (  2.13);

\path[fill=fillColor,fill opacity=0.20] (213.52, 62.18) circle (  2.13);

\path[fill=fillColor,fill opacity=0.20] (219.54, 64.26) circle (  2.13);

\path[fill=fillColor,fill opacity=0.20] (221.55, 64.67) circle (  2.13);

\path[fill=fillColor,fill opacity=0.20] (219.54, 62.80) circle (  2.13);

\path[fill=fillColor,fill opacity=0.20] (218.54, 57.82) circle (  2.13);

\path[fill=fillColor,fill opacity=0.20] (223.55, 56.05) circle (  2.13);

\path[fill=fillColor,fill opacity=0.20] (238.60, 64.88) circle (  2.13);

\path[fill=fillColor,fill opacity=0.20] (213.52, 69.55) circle (  2.13);

\path[fill=fillColor,fill opacity=0.20] (182.02, 61.87) circle (  2.13);

\path[fill=fillColor,fill opacity=0.20] (201.48, 64.57) circle (  2.13);

\path[fill=fillColor,fill opacity=0.20] (202.49, 72.56) circle (  2.13);

\path[fill=fillColor,fill opacity=0.20] (201.48, 69.24) circle (  2.13);

\path[fill=fillColor,fill opacity=0.20] (203.49, 63.42) circle (  2.13);

\path[fill=fillColor,fill opacity=0.20] (201.48, 62.70) circle (  2.13);

\path[fill=fillColor,fill opacity=0.20] (200.48, 57.20) circle (  2.13);

\path[fill=fillColor,fill opacity=0.20] (201.48, 50.14) circle (  2.13);

\path[fill=fillColor,fill opacity=0.20] (203.49, 50.03) circle (  2.13);

\path[fill=fillColor,fill opacity=0.20] (205.50, 52.32) circle (  2.13);

\path[fill=fillColor,fill opacity=0.20] (208.51, 55.22) circle (  2.13);

\path[fill=fillColor,fill opacity=0.20] (207.50, 58.75) circle (  2.13);

\path[fill=fillColor,fill opacity=0.20] (209.51, 58.34) circle (  2.13);

\path[fill=fillColor,fill opacity=0.20] (226.56, 62.39) circle (  2.13);

\path[fill=fillColor,fill opacity=0.20] (210.51, 66.12) circle (  2.13);

\path[fill=fillColor,fill opacity=0.20] (201.48, 58.55) circle (  2.13);

\path[fill=fillColor,fill opacity=0.20] (211.52, 59.17) circle (  2.13);

\path[fill=fillColor,fill opacity=0.20] (211.52, 54.70) circle (  2.13);

\path[fill=fillColor,fill opacity=0.20] (215.53, 54.39) circle (  2.13);

\path[fill=fillColor,fill opacity=0.20] (219.54, 58.34) circle (  2.13);

\path[fill=fillColor,fill opacity=0.20] (215.53, 60.93) circle (  2.13);

\path[fill=fillColor,fill opacity=0.20] (218.54, 68.41) circle (  2.13);

\path[fill=fillColor,fill opacity=0.20] (219.54, 73.18) circle (  2.13);

\path[fill=fillColor,fill opacity=0.20] (216.53, 70.59) circle (  2.13);

\path[fill=fillColor,fill opacity=0.20] (218.54, 64.36) circle (  2.13);

\path[fill=fillColor,fill opacity=0.20] (216.53, 68.93) circle (  2.13);

\path[fill=fillColor,fill opacity=0.20] (219.54, 70.80) circle (  2.13);

\path[fill=fillColor,fill opacity=0.20] (216.53, 63.84) circle (  2.13);

\path[fill=fillColor,fill opacity=0.20] (225.56, 61.04) circle (  2.13);

\path[fill=fillColor,fill opacity=0.20] (223.55, 67.79) circle (  2.13);

\path[fill=fillColor,fill opacity=0.20] (229.57, 71.00) circle (  2.13);

\path[fill=fillColor,fill opacity=0.20] (227.57, 67.89) circle (  2.13);

\path[fill=fillColor,fill opacity=0.20] (210.51, 70.07) circle (  2.13);

\path[fill=fillColor,fill opacity=0.20] (224.56, 76.09) circle (  2.13);

\path[fill=fillColor,fill opacity=0.20] (224.56, 75.16) circle (  2.13);

\path[fill=fillColor,fill opacity=0.20] (219.54, 73.39) circle (  2.13);

\path[fill=fillColor,fill opacity=0.20] (217.53, 77.13) circle (  2.13);

\path[fill=fillColor,fill opacity=0.20] (221.55, 76.40) circle (  2.13);

\path[fill=fillColor,fill opacity=0.20] (216.53, 66.44) circle (  2.13);

\path[fill=fillColor,fill opacity=0.20] (207.50, 64.36) circle (  2.13);

\path[fill=fillColor,fill opacity=0.20] (213.52, 71.11) circle (  2.13);

\path[fill=fillColor,fill opacity=0.20] (213.52, 68.82) circle (  2.13);

\path[fill=fillColor,fill opacity=0.20] (214.53, 63.11) circle (  2.13);

\path[fill=fillColor,fill opacity=0.20] (211.52, 64.36) circle (  2.13);

\path[fill=fillColor,fill opacity=0.20] (206.50, 65.19) circle (  2.13);

\path[fill=fillColor,fill opacity=0.20] (207.50, 66.02) circle (  2.13);

\path[fill=fillColor,fill opacity=0.20] (207.50, 68.51) circle (  2.13);

\path[fill=fillColor,fill opacity=0.20] (201.48, 64.36) circle (  2.13);

\path[fill=fillColor,fill opacity=0.20] (200.48, 59.89) circle (  2.13);

\path[fill=fillColor,fill opacity=0.20] (200.48, 58.34) circle (  2.13);

\path[fill=fillColor,fill opacity=0.20] (203.49, 52.73) circle (  2.13);

\path[fill=fillColor,fill opacity=0.20] (203.49, 48.06) circle (  2.13);

\path[fill=fillColor,fill opacity=0.20] (202.49, 47.02) circle (  2.13);

\path[fill=fillColor,fill opacity=0.20] (199.48, 46.61) circle (  2.13);

\path[fill=fillColor,fill opacity=0.20] (210.51, 47.85) circle (  2.13);

\path[fill=fillColor,fill opacity=0.20] (209.51, 48.89) circle (  2.13);

\path[fill=fillColor,fill opacity=0.20] (212.52, 50.97) circle (  2.13);

\path[fill=fillColor,fill opacity=0.20] (220.54, 62.59) circle (  2.13);

\path[fill=fillColor,fill opacity=0.20] (220.54, 77.23) circle (  2.13);

\path[fill=fillColor,fill opacity=0.20] (213.52, 69.45) circle (  2.13);

\path[fill=fillColor,fill opacity=0.20] (206.50, 58.13) circle (  2.13);

\path[fill=fillColor,fill opacity=0.20] (214.53, 52.21) circle (  2.13);

\path[fill=fillColor,fill opacity=0.20] (213.52, 51.69) circle (  2.13);

\path[fill=fillColor,fill opacity=0.20] (211.52, 54.70) circle (  2.13);

\path[fill=fillColor,fill opacity=0.20] (210.51, 63.53) circle (  2.13);

\path[fill=fillColor,fill opacity=0.20] (212.52, 71.73) circle (  2.13);

\path[fill=fillColor,fill opacity=0.20] (211.52, 71.32) circle (  2.13);

\path[fill=fillColor,fill opacity=0.20] (212.52, 69.97) circle (  2.13);

\path[fill=fillColor,fill opacity=0.20] (213.52, 72.25) circle (  2.13);

\path[fill=fillColor,fill opacity=0.20] (213.52, 74.12) circle (  2.13);

\path[fill=fillColor,fill opacity=0.20] (212.52, 64.88) circle (  2.13);

\path[fill=fillColor,fill opacity=0.20] (205.50, 54.91) circle (  2.13);

\path[fill=fillColor,fill opacity=0.20] (215.53, 55.95) circle (  2.13);

\path[fill=fillColor,fill opacity=0.20] (217.53, 57.51) circle (  2.13);

\path[fill=fillColor,fill opacity=0.20] (214.53, 53.67) circle (  2.13);

\path[fill=fillColor,fill opacity=0.20] (213.52, 55.02) circle (  2.13);

\path[fill=fillColor,fill opacity=0.20] (214.53, 63.63) circle (  2.13);

\path[fill=fillColor,fill opacity=0.20] (213.52, 66.64) circle (  2.13);

\path[fill=fillColor,fill opacity=0.20] (208.51, 58.44) circle (  2.13);

\path[fill=fillColor,fill opacity=0.20] (215.53, 49.41) circle (  2.13);

\path[fill=fillColor,fill opacity=0.20] (217.53, 52.00) circle (  2.13);

\path[fill=fillColor,fill opacity=0.20] (213.52, 59.58) circle (  2.13);

\path[fill=fillColor,fill opacity=0.20] (211.52, 58.55) circle (  2.13);

\path[fill=fillColor,fill opacity=0.20] (209.51, 56.78) circle (  2.13);

\path[fill=fillColor,fill opacity=0.20] (213.52, 57.61) circle (  2.13);

\path[fill=fillColor,fill opacity=0.20] (209.51, 52.42) circle (  2.13);

\path[fill=fillColor,fill opacity=0.20] (208.51, 52.52) circle (  2.13);

\path[fill=fillColor,fill opacity=0.20] (208.51, 59.27) circle (  2.13);

\path[fill=fillColor,fill opacity=0.20] (207.50, 55.95) circle (  2.13);

\path[fill=fillColor,fill opacity=0.20] (205.50, 50.97) circle (  2.13);

\path[fill=fillColor,fill opacity=0.20] (206.50, 55.95) circle (  2.13);

\path[fill=fillColor,fill opacity=0.20] (205.50, 57.09) circle (  2.13);

\path[fill=fillColor,fill opacity=0.20] (204.49, 50.14) circle (  2.13);

\path[fill=fillColor,fill opacity=0.20] (205.50, 46.19) circle (  2.13);

\path[fill=fillColor,fill opacity=0.20] (205.50, 50.03) circle (  2.13);

\path[fill=fillColor,fill opacity=0.20] (206.50, 56.99) circle (  2.13);

\path[fill=fillColor,fill opacity=0.20] (210.51, 60.52) circle (  2.13);

\path[fill=fillColor,fill opacity=0.20] (211.52, 63.11) circle (  2.13);

\path[fill=fillColor,fill opacity=0.20] (220.54, 67.68) circle (  2.13);

\path[fill=fillColor,fill opacity=0.20] (221.55, 69.65) circle (  2.13);

\path[fill=fillColor,fill opacity=0.20] (224.56, 68.62) circle (  2.13);

\path[fill=fillColor,fill opacity=0.20] (213.52, 54.29) circle (  2.13);

\path[fill=fillColor,fill opacity=0.20] (205.50, 52.52) circle (  2.13);

\path[fill=fillColor,fill opacity=0.20] (210.51, 54.60) circle (  2.13);

\path[fill=fillColor,fill opacity=0.20] (205.50, 56.99) circle (  2.13);

\path[fill=fillColor,fill opacity=0.20] (211.52, 56.26) circle (  2.13);

\path[fill=fillColor,fill opacity=0.20] (213.52, 51.69) circle (  2.13);

\path[fill=fillColor,fill opacity=0.20] (215.53, 54.29) circle (  2.13);

\path[fill=fillColor,fill opacity=0.20] (210.51, 61.14) circle (  2.13);

\path[fill=fillColor,fill opacity=0.20] (216.53, 63.94) circle (  2.13);

\path[fill=fillColor,fill opacity=0.20] (206.50, 66.85) circle (  2.13);

\path[fill=fillColor,fill opacity=0.20] (213.52, 67.37) circle (  2.13);

\path[fill=fillColor,fill opacity=0.20] (211.52, 63.22) circle (  2.13);

\path[fill=fillColor,fill opacity=0.20] (216.53, 57.51) circle (  2.13);

\path[fill=fillColor,fill opacity=0.20] (213.52, 55.85) circle (  2.13);

\path[fill=fillColor,fill opacity=0.20] (212.52, 62.08) circle (  2.13);

\path[fill=fillColor,fill opacity=0.20] (211.52, 67.37) circle (  2.13);

\path[fill=fillColor,fill opacity=0.20] (213.52, 62.28) circle (  2.13);

\path[fill=fillColor,fill opacity=0.20] (215.53, 56.88) circle (  2.13);

\path[fill=fillColor,fill opacity=0.20] (206.50, 54.18) circle (  2.13);

\path[fill=fillColor,fill opacity=0.20] (213.52, 53.67) circle (  2.13);

\path[fill=fillColor,fill opacity=0.20] (213.52, 58.65) circle (  2.13);

\path[fill=fillColor,fill opacity=0.20] (213.52, 60.62) circle (  2.13);

\path[fill=fillColor,fill opacity=0.20] (214.53, 55.22) circle (  2.13);

\path[fill=fillColor,fill opacity=0.20] (211.52, 48.68) circle (  2.13);

\path[fill=fillColor,fill opacity=0.20] (209.51, 45.57) circle (  2.13);

\path[fill=fillColor,fill opacity=0.20] (209.51, 48.37) circle (  2.13);

\path[fill=fillColor,fill opacity=0.20] (209.51, 54.50) circle (  2.13);

\path[fill=fillColor,fill opacity=0.20] (209.51, 56.36) circle (  2.13);

\path[fill=fillColor,fill opacity=0.20] (212.52, 56.78) circle (  2.13);

\path[fill=fillColor,fill opacity=0.20] (215.53, 62.28) circle (  2.13);

\path[fill=fillColor,fill opacity=0.20] (216.53, 67.27) circle (  2.13);

\path[fill=fillColor,fill opacity=0.20] (214.53, 69.45) circle (  2.13);

\path[fill=fillColor,fill opacity=0.20] (213.52, 75.47) circle (  2.13);

\path[fill=fillColor,fill opacity=0.20] (216.53, 84.81) circle (  2.13);

\path[fill=fillColor,fill opacity=0.20] (223.55, 90.00) circle (  2.13);

\path[fill=fillColor,fill opacity=0.20] (230.58, 90.00) circle (  2.13);

\path[fill=fillColor,fill opacity=0.20] (215.53, 60.52) circle (  2.13);

\path[fill=fillColor,fill opacity=0.20] (213.52, 58.03) circle (  2.13);

\path[fill=fillColor,fill opacity=0.20] (211.52, 60.10) circle (  2.13);

\path[fill=fillColor,fill opacity=0.20] (208.51, 54.39) circle (  2.13);

\path[fill=fillColor,fill opacity=0.20] (214.53, 41.00) circle (  2.13);

\path[fill=fillColor,fill opacity=0.20] (214.53, 38.30) circle (  2.13);

\path[fill=fillColor,fill opacity=0.20] (216.53, 44.63) circle (  2.13);

\path[fill=fillColor,fill opacity=0.20] (211.52, 51.28) circle (  2.13);

\path[fill=fillColor,fill opacity=0.20] (213.52, 64.77) circle (  2.13);

\path[fill=fillColor,fill opacity=0.20] (211.52, 69.03) circle (  2.13);

\path[fill=fillColor,fill opacity=0.20] (216.53, 59.69) circle (  2.13);

\path[fill=fillColor,fill opacity=0.20] (216.53, 51.90) circle (  2.13);

\path[fill=fillColor,fill opacity=0.20] (215.53, 52.32) circle (  2.13);

\path[fill=fillColor,fill opacity=0.20] (214.53, 57.61) circle (  2.13);

\path[fill=fillColor,fill opacity=0.20] (208.51, 60.00) circle (  2.13);

\path[fill=fillColor,fill opacity=0.20] (217.53, 52.42) circle (  2.13);

\path[fill=fillColor,fill opacity=0.20] (217.53, 49.20) circle (  2.13);

\path[fill=fillColor,fill opacity=0.20] (213.52, 53.98) circle (  2.13);

\path[fill=fillColor,fill opacity=0.20] (213.52, 56.88) circle (  2.13);

\path[fill=fillColor,fill opacity=0.20] (210.51, 58.65) circle (  2.13);

\path[fill=fillColor,fill opacity=0.20] (213.52, 58.86) circle (  2.13);

\path[fill=fillColor,fill opacity=0.20] (214.53, 55.64) circle (  2.13);

\path[fill=fillColor,fill opacity=0.20] (210.51, 54.08) circle (  2.13);

\path[fill=fillColor,fill opacity=0.20] (214.53, 59.69) circle (  2.13);

\path[fill=fillColor,fill opacity=0.20] (215.53, 66.95) circle (  2.13);

\path[fill=fillColor,fill opacity=0.20] (218.54, 74.85) circle (  2.13);

\path[fill=fillColor,fill opacity=0.20] (222.55, 75.57) circle (  2.13);

\path[fill=fillColor,fill opacity=0.20] (217.53, 60.41) circle (  2.13);

\path[fill=fillColor,fill opacity=0.20] (212.52, 55.85) circle (  2.13);

\path[fill=fillColor,fill opacity=0.20] (215.53, 54.91) circle (  2.13);

\path[fill=fillColor,fill opacity=0.20] (209.51, 57.09) circle (  2.13);

\path[fill=fillColor,fill opacity=0.20] (212.52, 63.84) circle (  2.13);

\path[fill=fillColor,fill opacity=0.20] (211.52, 63.11) circle (  2.13);

\path[fill=fillColor,fill opacity=0.20] (214.53, 56.78) circle (  2.13);

\path[fill=fillColor,fill opacity=0.20] (216.53, 53.77) circle (  2.13);

\path[fill=fillColor,fill opacity=0.20] (218.54, 52.73) circle (  2.13);

\path[fill=fillColor,fill opacity=0.20] (217.53, 54.91) circle (  2.13);

\path[fill=fillColor,fill opacity=0.20] (220.54, 57.51) circle (  2.13);

\path[fill=fillColor,fill opacity=0.20] (218.54, 53.56) circle (  2.13);

\path[fill=fillColor,fill opacity=0.20] (217.53, 51.90) circle (  2.13);

\path[fill=fillColor,fill opacity=0.20] (215.53, 58.44) circle (  2.13);

\path[fill=fillColor,fill opacity=0.20] (221.55, 65.29) circle (  2.13);

\path[fill=fillColor,fill opacity=0.20] (213.52, 70.38) circle (  2.13);

\path[fill=fillColor,fill opacity=0.20] (221.55, 79.31) circle (  2.13);

\path[fill=fillColor,fill opacity=0.20] (212.52, 88.96) circle (  2.13);

\path[fill=fillColor,fill opacity=0.20] (221.55, 76.09) circle (  2.13);

\path[fill=fillColor,fill opacity=0.20] (222.55, 69.13) circle (  2.13);

\path[fill=fillColor,fill opacity=0.20] (221.55, 71.63) circle (  2.13);

\path[fill=fillColor,fill opacity=0.20] (224.56, 69.76) circle (  2.13);

\path[fill=fillColor,fill opacity=0.20] (185.23, 81.39) circle (  2.13);

\path[fill=fillColor,fill opacity=0.20] (189.44, 66.64) circle (  2.13);

\path[fill=fillColor,fill opacity=0.20] (188.44, 83.77) circle (  2.13);

\path[fill=fillColor,fill opacity=0.20] (187.84, 74.64) circle (  2.13);

\path[fill=fillColor,fill opacity=0.20] (188.44, 72.04) circle (  2.13);

\path[fill=fillColor,fill opacity=0.20] (189.44, 69.86) circle (  2.13);

\path[fill=fillColor,fill opacity=0.20] (187.34, 54.08) circle (  2.13);

\path[fill=fillColor,fill opacity=0.20] (194.46, 58.44) circle (  2.13);

\path[fill=fillColor,fill opacity=0.20] (184.53, 44.32) circle (  2.13);

\path[fill=fillColor,fill opacity=0.20] (189.44, 50.86) circle (  2.13);

\path[fill=fillColor,fill opacity=0.20] (187.94, 57.92) circle (  2.13);

\path[fill=fillColor,fill opacity=0.20] (183.83, 49.51) circle (  2.13);

\path[fill=fillColor,fill opacity=0.20] (184.03, 54.60) circle (  2.13);

\path[fill=fillColor,fill opacity=0.20] (181.32, 58.13) circle (  2.13);

\path[fill=fillColor,fill opacity=0.20] (180.62, 54.50) circle (  2.13);

\path[fill=fillColor,fill opacity=0.20] (190.45, 62.80) circle (  2.13);

\path[fill=fillColor,fill opacity=0.20] (204.49, 79.31) circle (  2.13);

\path[fill=fillColor,fill opacity=0.20] (175.00, 56.68) circle (  2.13);

\path[fill=fillColor,fill opacity=0.20] (179.61, 56.88) circle (  2.13);

\path[fill=fillColor,fill opacity=0.20] (180.11, 39.96) circle (  2.13);

\path[fill=fillColor,fill opacity=0.20] (183.32, 44.63) circle (  2.13);

\path[fill=fillColor,fill opacity=0.20] (182.22, 50.03) circle (  2.13);

\path[fill=fillColor,fill opacity=0.20] (175.20, 49.10) circle (  2.13);

\path[fill=fillColor,fill opacity=0.20] (183.43, 60.41) circle (  2.13);

\path[fill=fillColor,fill opacity=0.20] (205.50, 68.41) circle (  2.13);

\path[fill=fillColor,fill opacity=0.20] (185.73, 45.98) circle (  2.13);

\path[fill=fillColor,fill opacity=0.20] (172.49, 49.93) circle (  2.13);

\path[fill=fillColor,fill opacity=0.20] (171.19, 51.28) circle (  2.13);

\path[fill=fillColor,fill opacity=0.20] (167.37, 52.84) circle (  2.13);

\path[fill=fillColor,fill opacity=0.20] (169.48, 53.77) circle (  2.13);

\path[fill=fillColor,fill opacity=0.20] (167.37, 55.33) circle (  2.13);

\path[fill=fillColor,fill opacity=0.20] (184.83, 53.67) circle (  2.13);

\path[fill=fillColor,fill opacity=0.20] (195.46, 51.07) circle (  2.13);

\path[fill=fillColor,fill opacity=0.20] (186.84, 43.70) circle (  2.13);

\path[fill=fillColor,fill opacity=0.20] (175.60, 54.39) circle (  2.13);

\path[fill=fillColor,fill opacity=0.20] (180.42, 57.20) circle (  2.13);

\path[fill=fillColor,fill opacity=0.20] (182.32, 59.48) circle (  2.13);

\path[fill=fillColor,fill opacity=0.20] (178.01, 58.34) circle (  2.13);

\path[fill=fillColor,fill opacity=0.20] (177.21, 55.53) circle (  2.13);

\path[fill=fillColor,fill opacity=0.20] (178.81, 51.59) circle (  2.13);

\path[fill=fillColor,fill opacity=0.20] (195.46, 49.10) circle (  2.13);

\path[fill=fillColor,fill opacity=0.20] (210.51, 61.56) circle (  2.13);

\path[fill=fillColor,fill opacity=0.20] (192.45, 77.65) circle (  2.13);

\path[fill=fillColor,fill opacity=0.20] (193.46, 59.69) circle (  2.13);

\path[fill=fillColor,fill opacity=0.20] (201.48, 60.73) circle (  2.13);

\path[fill=fillColor,fill opacity=0.20] (210.51, 52.32) circle (  2.13);

\path[fill=fillColor,fill opacity=0.20] (214.53, 49.72) circle (  2.13);

\path[fill=fillColor,fill opacity=0.20] (212.52, 58.13) circle (  2.13);

\path[fill=fillColor,fill opacity=0.20] (213.52, 64.77) circle (  2.13);

\path[fill=fillColor,fill opacity=0.20] (233.59, 46.19) circle (  2.13);

\path[fill=fillColor,fill opacity=0.20] (198.47, 62.70) circle (  2.13);

\path[fill=fillColor,fill opacity=0.20] (187.94, 57.92) circle (  2.13);

\path[fill=fillColor,fill opacity=0.20] (186.13, 62.18) circle (  2.13);

\path[fill=fillColor,fill opacity=0.20] (186.33, 60.93) circle (  2.13);

\path[fill=fillColor,fill opacity=0.20] (185.43, 59.58) circle (  2.13);

\path[fill=fillColor,fill opacity=0.20] (181.52, 61.45) circle (  2.13);

\path[fill=fillColor,fill opacity=0.20] (179.21, 52.21) circle (  2.13);

\path[fill=fillColor,fill opacity=0.20] (183.93, 42.66) circle (  2.13);

\path[fill=fillColor,fill opacity=0.20] (205.50, 47.02) circle (  2.13);

\path[fill=fillColor,fill opacity=0.20] (247.63, 58.13) circle (  2.13);

\path[fill=fillColor,fill opacity=0.20] (196.47, 74.74) circle (  2.13);

\path[fill=fillColor,fill opacity=0.20] (193.46, 59.17) circle (  2.13);

\path[fill=fillColor,fill opacity=0.20] (191.45, 53.46) circle (  2.13);

\path[fill=fillColor,fill opacity=0.20] (200.48, 58.13) circle (  2.13);

\path[fill=fillColor,fill opacity=0.20] (208.51, 57.40) circle (  2.13);

\path[fill=fillColor,fill opacity=0.20] (202.49, 50.76) circle (  2.13);

\path[fill=fillColor,fill opacity=0.20] (203.49, 57.82) circle (  2.13);

\path[fill=fillColor,fill opacity=0.20] (224.56, 62.28) circle (  2.13);

\path[fill=fillColor,fill opacity=0.20] (201.48, 57.09) circle (  2.13);

\path[fill=fillColor,fill opacity=0.20] (190.45, 62.28) circle (  2.13);

\path[fill=fillColor,fill opacity=0.20] (177.21, 62.39) circle (  2.13);

\path[fill=fillColor,fill opacity=0.20] (181.32, 55.33) circle (  2.13);

\path[fill=fillColor,fill opacity=0.20] (183.53, 53.56) circle (  2.13);

\path[fill=fillColor,fill opacity=0.20] (182.92, 53.04) circle (  2.13);

\path[fill=fillColor,fill opacity=0.20] (183.12, 55.02) circle (  2.13);

\path[fill=fillColor,fill opacity=0.20] (195.46, 49.72) circle (  2.13);

\path[fill=fillColor,fill opacity=0.20] (207.50, 57.40) circle (  2.13);

\path[fill=fillColor,fill opacity=0.20] (199.48, 60.10) circle (  2.13);

\path[fill=fillColor,fill opacity=0.20] (195.46, 52.11) circle (  2.13);

\path[fill=fillColor,fill opacity=0.20] (195.46, 68.72) circle (  2.13);

\path[fill=fillColor,fill opacity=0.20] (198.47, 70.28) circle (  2.13);

\path[fill=fillColor,fill opacity=0.20] (198.47, 71.21) circle (  2.13);

\path[fill=fillColor,fill opacity=0.20] (204.49, 70.59) circle (  2.13);

\path[fill=fillColor,fill opacity=0.20] (206.50, 58.96) circle (  2.13);

\path[fill=fillColor,fill opacity=0.20] (204.49, 60.31) circle (  2.13);

\path[fill=fillColor,fill opacity=0.20] (209.51, 72.56) circle (  2.13);

\path[fill=fillColor,fill opacity=0.20] (192.45, 54.08) circle (  2.13);

\path[fill=fillColor,fill opacity=0.20] (194.46, 37.99) circle (  2.13);

\path[fill=fillColor,fill opacity=0.20] (184.63, 53.77) circle (  2.13);

\path[fill=fillColor,fill opacity=0.20] (167.57, 66.02) circle (  2.13);

\path[fill=fillColor,fill opacity=0.20] (165.37, 55.12) circle (  2.13);

\path[fill=fillColor,fill opacity=0.20] (179.31, 48.27) circle (  2.13);

\path[fill=fillColor,fill opacity=0.20] (172.79, 53.87) circle (  2.13);

\path[fill=fillColor,fill opacity=0.20] (181.32, 53.15) circle (  2.13);

\path[fill=fillColor,fill opacity=0.20] (185.83, 46.81) circle (  2.13);

\path[fill=fillColor,fill opacity=0.20] (208.51, 52.32) circle (  2.13);

\path[fill=fillColor,fill opacity=0.20] (206.50, 68.82) circle (  2.13);

\path[fill=fillColor,fill opacity=0.20] (198.47, 52.84) circle (  2.13);

\path[fill=fillColor,fill opacity=0.20] (187.14, 55.74) circle (  2.13);

\path[fill=fillColor,fill opacity=0.20] (193.46, 60.41) circle (  2.13);

\path[fill=fillColor,fill opacity=0.20] (200.48, 66.33) circle (  2.13);

\path[fill=fillColor,fill opacity=0.20] (203.49, 76.09) circle (  2.13);

\path[fill=fillColor,fill opacity=0.20] (206.50, 73.08) circle (  2.13);

\path[fill=fillColor,fill opacity=0.20] (208.51, 60.52) circle (  2.13);

\path[fill=fillColor,fill opacity=0.20] (211.52, 60.21) circle (  2.13);

\path[fill=fillColor,fill opacity=0.20] (219.54, 65.81) circle (  2.13);

\path[fill=fillColor,fill opacity=0.20] (179.91, 65.50) circle (  2.13);

\path[fill=fillColor,fill opacity=0.20] (190.45, 63.63) circle (  2.13);

\path[fill=fillColor,fill opacity=0.20] (189.44, 59.17) circle (  2.13);

\path[fill=fillColor,fill opacity=0.20] (184.53, 70.38) circle (  2.13);

\path[fill=fillColor,fill opacity=0.20] (181.62, 61.24) circle (  2.13);

\path[fill=fillColor,fill opacity=0.20] (180.62, 50.34) circle (  2.13);

\path[fill=fillColor,fill opacity=0.20] (177.00, 59.17) circle (  2.13);

\path[fill=fillColor,fill opacity=0.20] (195.46, 45.67) circle (  2.13);

\path[fill=fillColor,fill opacity=0.20] (228.57, 48.16) circle (  2.13);

\path[fill=fillColor,fill opacity=0.20] (213.52, 64.15) circle (  2.13);

\path[fill=fillColor,fill opacity=0.20] (197.47, 54.50) circle (  2.13);

\path[fill=fillColor,fill opacity=0.20] (191.45, 62.39) circle (  2.13);

\path[fill=fillColor,fill opacity=0.20] (198.47, 55.33) circle (  2.13);

\path[fill=fillColor,fill opacity=0.20] (202.49, 53.56) circle (  2.13);

\path[fill=fillColor,fill opacity=0.20] (208.51, 65.92) circle (  2.13);

\path[fill=fillColor,fill opacity=0.20] (213.52, 63.11) circle (  2.13);

\path[fill=fillColor,fill opacity=0.20] (215.53, 54.08) circle (  2.13);

\path[fill=fillColor,fill opacity=0.20] (211.52, 56.47) circle (  2.13);

\path[fill=fillColor,fill opacity=0.20] (243.62, 60.62) circle (  2.13);

\path[fill=fillColor,fill opacity=0.20] (198.47, 73.60) circle (  2.13);

\path[fill=fillColor,fill opacity=0.20] (194.46, 60.83) circle (  2.13);

\path[fill=fillColor,fill opacity=0.20] (191.45, 70.90) circle (  2.13);

\path[fill=fillColor,fill opacity=0.20] (186.94, 77.23) circle (  2.13);

\path[fill=fillColor,fill opacity=0.20] (189.44, 69.03) circle (  2.13);

\path[fill=fillColor,fill opacity=0.20] (188.34, 58.65) circle (  2.13);

\path[fill=fillColor,fill opacity=0.20] (181.32, 54.50) circle (  2.13);

\path[fill=fillColor,fill opacity=0.20] (183.32, 56.47) circle (  2.13);

\path[fill=fillColor,fill opacity=0.20] (184.33, 65.50) circle (  2.13);

\path[fill=fillColor,fill opacity=0.20] (180.42, 65.19) circle (  2.13);

\path[fill=fillColor,fill opacity=0.20] (209.51, 56.57) circle (  2.13);

\path[fill=fillColor,fill opacity=0.20] (271.71, 60.52) circle (  2.13);

\path[fill=fillColor,fill opacity=0.20] (220.54, 70.38) circle (  2.13);

\path[fill=fillColor,fill opacity=0.20] (203.49, 52.94) circle (  2.13);

\path[fill=fillColor,fill opacity=0.20] (192.45, 61.76) circle (  2.13);

\path[fill=fillColor,fill opacity=0.20] (196.47, 62.18) circle (  2.13);

\path[fill=fillColor,fill opacity=0.20] (203.49, 53.46) circle (  2.13);

\path[fill=fillColor,fill opacity=0.20] (201.48, 57.30) circle (  2.13);

\path[fill=fillColor,fill opacity=0.20] (207.50, 56.05) circle (  2.13);

\path[fill=fillColor,fill opacity=0.20] (213.52, 47.44) circle (  2.13);

\path[fill=fillColor,fill opacity=0.20] (227.57, 56.36) circle (  2.13);

\path[fill=fillColor,fill opacity=0.20] (200.48, 64.67) circle (  2.13);

\path[fill=fillColor,fill opacity=0.20] (191.45, 64.26) circle (  2.13);

\path[fill=fillColor,fill opacity=0.20] (191.45, 61.56) circle (  2.13);

\path[fill=fillColor,fill opacity=0.20] (188.34, 79.93) circle (  2.13);

\path[fill=fillColor,fill opacity=0.20] (187.14, 80.04) circle (  2.13);

\path[fill=fillColor,fill opacity=0.20] (189.44, 58.23) circle (  2.13);

\path[fill=fillColor,fill opacity=0.20] (187.44, 53.98) circle (  2.13);

\path[fill=fillColor,fill opacity=0.20] (188.24, 59.38) circle (  2.13);

\path[fill=fillColor,fill opacity=0.20] (190.45, 60.83) circle (  2.13);

\path[fill=fillColor,fill opacity=0.20] (213.52, 63.74) circle (  2.13);

\path[fill=fillColor,fill opacity=0.20] (260.67, 68.82) circle (  2.13);

\path[fill=fillColor,fill opacity=0.20] (215.53, 54.39) circle (  2.13);

\path[fill=fillColor,fill opacity=0.20] (190.45, 53.87) circle (  2.13);

\path[fill=fillColor,fill opacity=0.20] (188.44, 60.93) circle (  2.13);

\path[fill=fillColor,fill opacity=0.20] (192.45, 54.81) circle (  2.13);

\path[fill=fillColor,fill opacity=0.20] (192.45, 53.98) circle (  2.13);

\path[fill=fillColor,fill opacity=0.20] (190.45, 62.49) circle (  2.13);

\path[fill=fillColor,fill opacity=0.20] (200.48, 58.96) circle (  2.13);

\path[fill=fillColor,fill opacity=0.20] (211.52, 57.71) circle (  2.13);

\path[fill=fillColor,fill opacity=0.20] (196.47, 68.62) circle (  2.13);

\path[fill=fillColor,fill opacity=0.20] (181.32, 67.79) circle (  2.13);

\path[fill=fillColor,fill opacity=0.20] (182.62, 55.64) circle (  2.13);

\path[fill=fillColor,fill opacity=0.20] (189.44, 62.08) circle (  2.13);

\path[fill=fillColor,fill opacity=0.20] (189.44, 78.17) circle (  2.13);

\path[fill=fillColor,fill opacity=0.20] (187.74, 83.77) circle (  2.13);

\path[fill=fillColor,fill opacity=0.20] (191.45, 71.42) circle (  2.13);

\path[fill=fillColor,fill opacity=0.20] (195.46, 62.18) circle (  2.13);

\path[fill=fillColor,fill opacity=0.20] (203.49, 54.08) circle (  2.13);

\path[fill=fillColor,fill opacity=0.20] (224.56, 47.12) circle (  2.13);

\path[fill=fillColor,fill opacity=0.20] (202.49, 51.69) circle (  2.13);

\path[fill=fillColor,fill opacity=0.20] (194.46, 54.50) circle (  2.13);

\path[fill=fillColor,fill opacity=0.20] (194.46, 51.38) circle (  2.13);

\path[fill=fillColor,fill opacity=0.20] (185.83, 50.76) circle (  2.13);

\path[fill=fillColor,fill opacity=0.20] (186.64, 66.33) circle (  2.13);

\path[fill=fillColor,fill opacity=0.20] (195.46, 75.36) circle (  2.13);

\path[fill=fillColor,fill opacity=0.20] (201.48, 64.88) circle (  2.13);

\path[fill=fillColor,fill opacity=0.20] (209.51, 84.81) circle (  2.13);

\path[fill=fillColor,fill opacity=0.20] (199.48, 64.26) circle (  2.13);

\path[fill=fillColor,fill opacity=0.20] (183.43, 73.50) circle (  2.13);

\path[fill=fillColor,fill opacity=0.20] (185.23, 68.72) circle (  2.13);

\path[fill=fillColor,fill opacity=0.20] (190.45, 65.29) circle (  2.13);

\path[fill=fillColor,fill opacity=0.20] (191.45, 61.97) circle (  2.13);

\path[fill=fillColor,fill opacity=0.20] (193.46, 68.51) circle (  2.13);

\path[fill=fillColor,fill opacity=0.20] (206.50, 67.27) circle (  2.13);

\path[fill=fillColor,fill opacity=0.20] (213.52, 59.89) circle (  2.13);

\path[fill=fillColor,fill opacity=0.20] (221.55, 52.63) circle (  2.13);

\path[fill=fillColor,fill opacity=0.20] (210.51, 66.44) circle (  2.13);

\path[fill=fillColor,fill opacity=0.20] (199.48, 56.99) circle (  2.13);

\path[fill=fillColor,fill opacity=0.20] (199.48, 51.49) circle (  2.13);

\path[fill=fillColor,fill opacity=0.20] (193.46, 56.78) circle (  2.13);

\path[fill=fillColor,fill opacity=0.20] (193.46, 65.09) circle (  2.13);

\path[fill=fillColor,fill opacity=0.20] (198.47, 70.38) circle (  2.13);

\path[fill=fillColor,fill opacity=0.20] (199.48, 71.42) circle (  2.13);

\path[fill=fillColor,fill opacity=0.20] (204.49, 79.83) circle (  2.13);

\path[fill=fillColor,fill opacity=0.20] (196.47, 57.20) circle (  2.13);

\path[fill=fillColor,fill opacity=0.20] (192.45, 65.81) circle (  2.13);

\path[fill=fillColor,fill opacity=0.20] (180.21, 73.91) circle (  2.13);

\path[fill=fillColor,fill opacity=0.20] (181.22, 69.65) circle (  2.13);

\path[fill=fillColor,fill opacity=0.20] (189.44, 60.10) circle (  2.13);

\path[fill=fillColor,fill opacity=0.20] (201.48, 53.25) circle (  2.13);

\path[fill=fillColor,fill opacity=0.20] (212.52, 52.42) circle (  2.13);

\path[fill=fillColor,fill opacity=0.20] (213.52, 62.59) circle (  2.13);

\path[fill=fillColor,fill opacity=0.20] (220.54, 60.93) circle (  2.13);

\path[fill=fillColor,fill opacity=0.20] (223.55, 53.46) circle (  2.13);

\path[fill=fillColor,fill opacity=0.20] (211.52, 67.37) circle (  2.13);

\path[fill=fillColor,fill opacity=0.20] (202.49, 51.49) circle (  2.13);

\path[fill=fillColor,fill opacity=0.20] (197.47, 58.86) circle (  2.13);

\path[fill=fillColor,fill opacity=0.20] (194.46, 62.39) circle (  2.13);

\path[fill=fillColor,fill opacity=0.20] (196.47, 58.75) circle (  2.13);

\path[fill=fillColor,fill opacity=0.20] (196.47, 65.29) circle (  2.13);

\path[fill=fillColor,fill opacity=0.20] (199.48, 75.88) circle (  2.13);

\path[fill=fillColor,fill opacity=0.20] (201.48, 79.21) circle (  2.13);

\path[fill=fillColor,fill opacity=0.20] (191.45, 78.27) circle (  2.13);

\path[fill=fillColor,fill opacity=0.20] (184.13, 78.58) circle (  2.13);

\path[fill=fillColor,fill opacity=0.20] (186.23, 68.62) circle (  2.13);

\path[fill=fillColor,fill opacity=0.20] (174.40, 58.65) circle (  2.13);

\path[fill=fillColor,fill opacity=0.20] (186.33, 45.78) circle (  2.13);

\path[fill=fillColor,fill opacity=0.20] (178.01, 44.53) circle (  2.13);

\path[fill=fillColor,fill opacity=0.20] (230.58, 60.83) circle (  2.13);

\path[fill=fillColor,fill opacity=0.20] (219.54, 72.15) circle (  2.13);

\path[fill=fillColor,fill opacity=0.20] (207.50, 41.41) circle (  2.13);

\path[fill=fillColor,fill opacity=0.20] (197.47, 51.17) circle (  2.13);

\path[fill=fillColor,fill opacity=0.20] (194.46, 61.66) circle (  2.13);

\path[fill=fillColor,fill opacity=0.20] (196.47, 58.86) circle (  2.13);

\path[fill=fillColor,fill opacity=0.20] (198.47, 57.71) circle (  2.13);

\path[fill=fillColor,fill opacity=0.20] (197.47, 64.05) circle (  2.13);

\path[fill=fillColor,fill opacity=0.20] (198.47, 74.22) circle (  2.13);

\path[fill=fillColor,fill opacity=0.20] (203.49, 75.36) circle (  2.13);

\path[fill=fillColor,fill opacity=0.20] (189.44, 80.04) circle (  2.13);

\path[fill=fillColor,fill opacity=0.20] (190.45, 80.56) circle (  2.13);

\path[fill=fillColor,fill opacity=0.20] (195.46, 66.02) circle (  2.13);

\path[fill=fillColor,fill opacity=0.20] (189.44, 52.42) circle (  2.13);

\path[fill=fillColor,fill opacity=0.20] (213.52, 45.26) circle (  2.13);

\path[fill=fillColor,fill opacity=0.20] (232.58, 72.56) circle (  2.13);

\path[fill=fillColor,fill opacity=0.20] (218.54, 38.82) circle (  2.13);

\path[fill=fillColor,fill opacity=0.20] (203.49, 48.99) circle (  2.13);

\path[fill=fillColor,fill opacity=0.20] (198.47, 57.09) circle (  2.13);

\path[fill=fillColor,fill opacity=0.20] (197.47, 56.16) circle (  2.13);

\path[fill=fillColor,fill opacity=0.20] (198.47, 59.38) circle (  2.13);

\path[fill=fillColor,fill opacity=0.20] (203.49, 63.01) circle (  2.13);

\path[fill=fillColor,fill opacity=0.20] (204.49, 69.55) circle (  2.13);

\path[fill=fillColor,fill opacity=0.20] (205.50, 61.56) circle (  2.13);

\path[fill=fillColor,fill opacity=0.20] (207.50, 55.64) circle (  2.13);

\path[fill=fillColor,fill opacity=0.20] (187.84, 75.05) circle (  2.13);

\path[fill=fillColor,fill opacity=0.20] (190.45, 77.86) circle (  2.13);

\path[fill=fillColor,fill opacity=0.20] (196.47, 84.81) circle (  2.13);

\path[fill=fillColor,fill opacity=0.20] (204.49, 65.29) circle (  2.13);

\path[fill=fillColor,fill opacity=0.20] (190.45, 48.79) circle (  2.13);

\path[fill=fillColor,fill opacity=0.20] (226.56, 64.36) circle (  2.13);

\path[fill=fillColor,fill opacity=0.20] (209.51, 52.11) circle (  2.13);

\path[fill=fillColor,fill opacity=0.20] (196.47, 55.12) circle (  2.13);

\path[fill=fillColor,fill opacity=0.20] (204.49, 60.52) circle (  2.13);

\path[fill=fillColor,fill opacity=0.20] (203.49, 56.16) circle (  2.13);

\path[fill=fillColor,fill opacity=0.20] (193.46, 59.17) circle (  2.13);

\path[fill=fillColor,fill opacity=0.20] (199.48, 65.29) circle (  2.13);

\path[fill=fillColor,fill opacity=0.20] (199.48, 61.56) circle (  2.13);

\path[fill=fillColor,fill opacity=0.20] (202.49, 78.17) circle (  2.13);

\path[fill=fillColor,fill opacity=0.20] (197.47, 58.55) circle (  2.13);

\path[fill=fillColor,fill opacity=0.20] (187.94, 68.82) circle (  2.13);

\path[fill=fillColor,fill opacity=0.20] (194.46, 72.98) circle (  2.13);

\path[fill=fillColor,fill opacity=0.20] (203.49, 49.62) circle (  2.13);

\path[fill=fillColor,fill opacity=0.20] (242.62, 63.53) circle (  2.13);

\path[fill=fillColor,fill opacity=0.20] (208.51, 52.11) circle (  2.13);

\path[fill=fillColor,fill opacity=0.20] (207.50, 44.53) circle (  2.13);

\path[fill=fillColor,fill opacity=0.20] (197.47, 42.45) circle (  2.13);

\path[fill=fillColor,fill opacity=0.20] (196.47, 61.14) circle (  2.13);

\path[fill=fillColor,fill opacity=0.20] (196.47, 72.98) circle (  2.13);

\path[fill=fillColor,fill opacity=0.20] (199.48, 68.41) circle (  2.13);

\path[fill=fillColor,fill opacity=0.20] (206.50, 63.22) circle (  2.13);

\path[fill=fillColor,fill opacity=0.20] (201.48, 72.87) circle (  2.13);

\path[fill=fillColor,fill opacity=0.20] (204.49, 87.93) circle (  2.13);

\path[fill=fillColor,fill opacity=0.20] (202.49, 64.77) circle (  2.13);

\path[fill=fillColor,fill opacity=0.20] (201.48, 72.35) circle (  2.13);

\path[fill=fillColor,fill opacity=0.20] (198.47, 66.12) circle (  2.13);

\path[fill=fillColor,fill opacity=0.20] (196.47, 58.86) circle (  2.13);

\path[fill=fillColor,fill opacity=0.20] (196.47, 63.74) circle (  2.13);

\path[fill=fillColor,fill opacity=0.20] (229.57, 68.62) circle (  2.13);

\path[fill=fillColor,fill opacity=0.20] (205.50, 52.32) circle (  2.13);

\path[fill=fillColor,fill opacity=0.20] (197.47, 51.17) circle (  2.13);

\path[fill=fillColor,fill opacity=0.20] (191.45, 59.17) circle (  2.13);

\path[fill=fillColor,fill opacity=0.20] (194.46, 63.32) circle (  2.13);

\path[fill=fillColor,fill opacity=0.20] (197.47, 65.81) circle (  2.13);

\path[fill=fillColor,fill opacity=0.20] (205.50, 62.08) circle (  2.13);

\path[fill=fillColor,fill opacity=0.20] (202.49, 74.22) circle (  2.13);

\path[fill=fillColor,fill opacity=0.20] (202.49, 84.81) circle (  2.13);

\path[fill=fillColor,fill opacity=0.20] (209.51, 74.74) circle (  2.13);

\path[fill=fillColor,fill opacity=0.20] (215.53, 65.19) circle (  2.13);

\path[fill=fillColor,fill opacity=0.20] (218.54, 73.50) circle (  2.13);

\path[fill=fillColor,fill opacity=0.20] (214.53, 81.70) circle (  2.13);

\path[fill=fillColor,fill opacity=0.20] (210.51, 66.12) circle (  2.13);

\path[fill=fillColor,fill opacity=0.20] (206.50, 59.06) circle (  2.13);

\path[fill=fillColor,fill opacity=0.20] (203.49, 57.20) circle (  2.13);

\path[fill=fillColor,fill opacity=0.20] (203.49, 65.92) circle (  2.13);

\path[fill=fillColor,fill opacity=0.20] (197.47, 69.65) circle (  2.13);

\path[fill=fillColor,fill opacity=0.20] (195.46, 61.97) circle (  2.13);

\path[fill=fillColor,fill opacity=0.20] (198.47, 54.08) circle (  2.13);

\path[fill=fillColor,fill opacity=0.20] (215.53, 49.51) circle (  2.13);

\path[fill=fillColor,fill opacity=0.20] (214.53, 81.39) circle (  2.13);

\path[fill=fillColor,fill opacity=0.20] (230.58, 74.33) circle (  2.13);

\path[fill=fillColor,fill opacity=0.20] (210.51, 59.89) circle (  2.13);

\path[fill=fillColor,fill opacity=0.20] (197.47, 53.56) circle (  2.13);

\path[fill=fillColor,fill opacity=0.20] (198.47, 55.74) circle (  2.13);

\path[fill=fillColor,fill opacity=0.20] (200.48, 53.56) circle (  2.13);

\path[fill=fillColor,fill opacity=0.20] (202.49, 65.40) circle (  2.13);

\path[fill=fillColor,fill opacity=0.20] (203.49, 83.77) circle (  2.13);

\path[fill=fillColor,fill opacity=0.20] (205.50, 80.97) circle (  2.13);

\path[fill=fillColor,fill opacity=0.20] (206.50, 69.24) circle (  2.13);

\path[fill=fillColor,fill opacity=0.20] (209.51, 64.77) circle (  2.13);

\path[fill=fillColor,fill opacity=0.20] (217.53, 63.01) circle (  2.13);

\path[fill=fillColor,fill opacity=0.20] (203.49, 71.11) circle (  2.13);

\path[fill=fillColor,fill opacity=0.20] (216.53, 78.58) circle (  2.13);

\path[fill=fillColor,fill opacity=0.20] (212.52, 73.70) circle (  2.13);

\path[fill=fillColor,fill opacity=0.20] (209.51, 60.00) circle (  2.13);

\path[fill=fillColor,fill opacity=0.20] (214.53, 49.41) circle (  2.13);

\path[fill=fillColor,fill opacity=0.20] (213.52, 49.51) circle (  2.13);

\path[fill=fillColor,fill opacity=0.20] (212.52, 56.68) circle (  2.13);

\path[fill=fillColor,fill opacity=0.20] (200.48, 57.40) circle (  2.13);

\path[fill=fillColor,fill opacity=0.20] (196.47, 52.42) circle (  2.13);

\path[fill=fillColor,fill opacity=0.20] (200.48, 61.35) circle (  2.13);

\path[fill=fillColor,fill opacity=0.20] (204.49, 62.59) circle (  2.13);

\path[fill=fillColor,fill opacity=0.20] (203.49, 49.10) circle (  2.13);

\path[fill=fillColor,fill opacity=0.20] (222.55, 48.16) circle (  2.13);

\path[fill=fillColor,fill opacity=0.20] (199.48, 64.57) circle (  2.13);

\path[fill=fillColor,fill opacity=0.20] (231.58, 76.09) circle (  2.13);

\path[fill=fillColor,fill opacity=0.20] (203.49, 59.58) circle (  2.13);

\path[fill=fillColor,fill opacity=0.20] (202.49, 48.37) circle (  2.13);

\path[fill=fillColor,fill opacity=0.20] (202.49, 46.92) circle (  2.13);

\path[fill=fillColor,fill opacity=0.20] (200.48, 55.33) circle (  2.13);

\path[fill=fillColor,fill opacity=0.20] (204.49, 64.57) circle (  2.13);

\path[fill=fillColor,fill opacity=0.20] (200.48, 69.86) circle (  2.13);

\path[fill=fillColor,fill opacity=0.20] (198.47, 67.27) circle (  2.13);

\path[fill=fillColor,fill opacity=0.20] (206.50, 63.42) circle (  2.13);

\path[fill=fillColor,fill opacity=0.20] (211.52, 69.24) circle (  2.13);

\path[fill=fillColor,fill opacity=0.20] (212.52, 71.32) circle (  2.13);

\path[fill=fillColor,fill opacity=0.20] (212.52, 62.91) circle (  2.13);

\path[fill=fillColor,fill opacity=0.20] (210.51, 56.99) circle (  2.13);

\path[fill=fillColor,fill opacity=0.20] (214.53, 55.22) circle (  2.13);

\path[fill=fillColor,fill opacity=0.20] (210.51, 54.18) circle (  2.13);

\path[fill=fillColor,fill opacity=0.20] (213.52, 60.73) circle (  2.13);

\path[fill=fillColor,fill opacity=0.20] (211.52, 58.86) circle (  2.13);

\path[fill=fillColor,fill opacity=0.20] (209.51, 52.11) circle (  2.13);

\path[fill=fillColor,fill opacity=0.20] (208.51, 57.20) circle (  2.13);

\path[fill=fillColor,fill opacity=0.20] (214.53, 62.18) circle (  2.13);

\path[fill=fillColor,fill opacity=0.20] (216.53, 65.92) circle (  2.13);

\path[fill=fillColor,fill opacity=0.20] (211.52, 68.41) circle (  2.13);

\path[fill=fillColor,fill opacity=0.20] (211.52, 60.83) circle (  2.13);

\path[fill=fillColor,fill opacity=0.20] (213.52, 54.91) circle (  2.13);

\path[fill=fillColor,fill opacity=0.20] (212.52, 58.13) circle (  2.13);

\path[fill=fillColor,fill opacity=0.20] (209.51, 61.04) circle (  2.13);

\path[fill=fillColor,fill opacity=0.20] (205.50, 64.88) circle (  2.13);

\path[fill=fillColor,fill opacity=0.20] (204.49, 61.87) circle (  2.13);

\path[fill=fillColor,fill opacity=0.20] (202.49, 61.66) circle (  2.13);

\path[fill=fillColor,fill opacity=0.20] (200.48, 68.82) circle (  2.13);

\path[fill=fillColor,fill opacity=0.20] (200.48, 64.15) circle (  2.13);

\path[fill=fillColor,fill opacity=0.20] (197.47, 52.42) circle (  2.13);

\path[fill=fillColor,fill opacity=0.20] (204.49, 50.24) circle (  2.13);

\path[fill=fillColor,fill opacity=0.20] (213.52, 47.54) circle (  2.13);

\path[fill=fillColor,fill opacity=0.20] (206.50, 60.93) circle (  2.13);

\path[fill=fillColor,fill opacity=0.20] (222.55, 51.59) circle (  2.13);

\path[fill=fillColor,fill opacity=0.20] (200.48, 41.00) circle (  2.13);

\path[fill=fillColor,fill opacity=0.20] (201.48, 63.22) circle (  2.13);

\path[fill=fillColor,fill opacity=0.20] (199.48, 68.93) circle (  2.13);

\path[fill=fillColor,fill opacity=0.20] (201.48, 60.41) circle (  2.13);

\path[fill=fillColor,fill opacity=0.20] (202.49, 62.49) circle (  2.13);

\path[fill=fillColor,fill opacity=0.20] (206.50, 71.00) circle (  2.13);

\path[fill=fillColor,fill opacity=0.20] (201.48, 73.81) circle (  2.13);

\path[fill=fillColor,fill opacity=0.20] (199.48, 68.82) circle (  2.13);

\path[fill=fillColor,fill opacity=0.20] (201.48, 58.44) circle (  2.13);

\path[fill=fillColor,fill opacity=0.20] (206.50, 54.60) circle (  2.13);

\path[fill=fillColor,fill opacity=0.20] (205.50, 59.79) circle (  2.13);

\path[fill=fillColor,fill opacity=0.20] (205.50, 61.56) circle (  2.13);

\path[fill=fillColor,fill opacity=0.20] (207.50, 58.86) circle (  2.13);

\path[fill=fillColor,fill opacity=0.20] (205.50, 57.82) circle (  2.13);

\path[fill=fillColor,fill opacity=0.20] (208.51, 57.71) circle (  2.13);

\path[fill=fillColor,fill opacity=0.20] (210.51, 59.38) circle (  2.13);

\path[fill=fillColor,fill opacity=0.20] (203.49, 61.04) circle (  2.13);

\path[fill=fillColor,fill opacity=0.20] (206.50, 62.80) circle (  2.13);

\path[fill=fillColor,fill opacity=0.20] (203.49, 62.70) circle (  2.13);

\path[fill=fillColor,fill opacity=0.20] (201.48, 62.18) circle (  2.13);

\path[fill=fillColor,fill opacity=0.20] (201.48, 67.89) circle (  2.13);

\path[fill=fillColor,fill opacity=0.20] (195.46, 71.32) circle (  2.13);

\path[fill=fillColor,fill opacity=0.20] (198.47, 60.52) circle (  2.13);

\path[fill=fillColor,fill opacity=0.20] (201.48, 54.50) circle (  2.13);

\path[fill=fillColor,fill opacity=0.20] (191.45, 58.96) circle (  2.13);

\path[fill=fillColor,fill opacity=0.20] (215.53, 52.00) circle (  2.13);

\path[fill=fillColor,fill opacity=0.20] (223.55, 50.86) circle (  2.13);

\path[fill=fillColor,fill opacity=0.20] (190.45, 68.41) circle (  2.13);

\path[fill=fillColor,fill opacity=0.20] (264.69, 59.89) circle (  2.13);

\path[fill=fillColor,fill opacity=0.20] (228.57, 55.22) circle (  2.13);

\path[fill=fillColor,fill opacity=0.20] (186.64, 65.71) circle (  2.13);

\path[fill=fillColor,fill opacity=0.20] (202.49, 71.52) circle (  2.13);

\path[fill=fillColor,fill opacity=0.20] (214.53, 60.10) circle (  2.13);

\path[fill=fillColor,fill opacity=0.20] (207.50, 48.47) circle (  2.13);

\path[fill=fillColor,fill opacity=0.20] (206.50, 54.70) circle (  2.13);

\path[fill=fillColor,fill opacity=0.20] (193.46, 61.97) circle (  2.13);

\path[fill=fillColor,fill opacity=0.20] (197.47, 56.57) circle (  2.13);

\path[fill=fillColor,fill opacity=0.20] (199.48, 46.40) circle (  2.13);

\path[fill=fillColor,fill opacity=0.20] (204.49, 43.39) circle (  2.13);

\path[fill=fillColor,fill opacity=0.20] (200.48, 47.75) circle (  2.13);

\path[fill=fillColor,fill opacity=0.20] (203.49, 55.33) circle (  2.13);

\path[fill=fillColor,fill opacity=0.20] (202.49, 60.10) circle (  2.13);

\path[fill=fillColor,fill opacity=0.20] (207.50, 59.89) circle (  2.13);

\path[fill=fillColor,fill opacity=0.20] (211.52, 58.13) circle (  2.13);

\path[fill=fillColor,fill opacity=0.20] (212.52, 57.40) circle (  2.13);

\path[fill=fillColor,fill opacity=0.20] (200.48, 52.21) circle (  2.13);

\path[fill=fillColor,fill opacity=0.20] (206.50, 45.88) circle (  2.13);

\path[fill=fillColor,fill opacity=0.20] (215.53, 46.09) circle (  2.13);

\path[fill=fillColor,fill opacity=0.20] (191.45, 50.76) circle (  2.13);

\path[fill=fillColor,fill opacity=0.20] (208.51, 48.58) circle (  2.13);

\path[fill=fillColor,fill opacity=0.20] (209.51, 43.18) circle (  2.13);

\path[fill=fillColor,fill opacity=0.20] (195.46, 42.35) circle (  2.13);

\path[fill=fillColor,fill opacity=0.20] (228.57, 42.56) circle (  2.13);

\path[fill=fillColor,fill opacity=0.20] (236.60, 48.68) circle (  2.13);

\path[fill=fillColor,fill opacity=0.20] (257.66, 63.42) circle (  2.13);

\path[fill=fillColor,fill opacity=0.20] (251.64, 54.81) circle (  2.13);

\path[fill=fillColor,fill opacity=0.20] (234.59, 46.29) circle (  2.13);

\path[fill=fillColor,fill opacity=0.20] (240.61, 40.17) circle (  2.13);

\path[fill=fillColor,fill opacity=0.20] (238.60, 39.23) circle (  2.13);

\path[fill=fillColor,fill opacity=0.20] (216.53, 38.09) circle (  2.13);

\path[fill=fillColor,fill opacity=0.20] (226.56, 39.03) circle (  2.13);

\path[fill=fillColor,fill opacity=0.20] (229.57, 46.81) circle (  2.13);

\path[fill=fillColor,fill opacity=0.20] (239.61, 55.12) circle (  2.13);

\path[fill=fillColor,fill opacity=0.20] (223.55, 61.04) circle (  2.13);

\path[fill=fillColor,fill opacity=0.20] (220.54, 63.94) circle (  2.13);

\path[fill=fillColor,fill opacity=0.20] (222.55, 69.13) circle (  2.13);

\path[fill=fillColor,fill opacity=0.20] (226.56, 73.50) circle (  2.13);

\path[fill=fillColor,fill opacity=0.20] (203.49, 49.72) circle (  2.13);

\path[fill=fillColor,fill opacity=0.20] (218.54, 60.21) circle (  2.13);

\path[fill=fillColor,fill opacity=0.20] (250.64, 69.03) circle (  2.13);

\path[fill=fillColor,fill opacity=0.20] (209.51, 70.59) circle (  2.13);

\path[fill=fillColor,fill opacity=0.20] (197.47, 79.62) circle (  2.13);

\path[fill=fillColor,fill opacity=0.20] (187.84, 76.19) circle (  2.13);

\path[fill=fillColor,fill opacity=0.20] (191.45, 76.71) circle (  2.13);

\path[fill=fillColor,fill opacity=0.20] (197.47, 76.30) circle (  2.13);

\path[fill=fillColor,fill opacity=0.20] (205.50, 69.13) circle (  2.13);

\path[fill=fillColor,fill opacity=0.20] (208.51, 60.00) circle (  2.13);

\path[fill=fillColor,fill opacity=0.20] (208.51, 58.55) circle (  2.13);

\path[fill=fillColor,fill opacity=0.20] (209.51, 64.26) circle (  2.13);

\path[fill=fillColor,fill opacity=0.20] (214.53, 70.90) circle (  2.13);

\path[fill=fillColor,fill opacity=0.20] (217.53, 73.29) circle (  2.13);

\path[fill=fillColor,fill opacity=0.20] (210.51, 74.01) circle (  2.13);

\path[fill=fillColor,fill opacity=0.20] (192.45, 72.66) circle (  2.13);

\path[fill=fillColor,fill opacity=0.20] (200.48, 60.73) circle (  2.13);

\path[fill=fillColor,fill opacity=0.20] (173.29, 69.86) circle (  2.13);

\path[fill=fillColor,fill opacity=0.20] (197.47, 80.14) circle (  2.13);

\path[fill=fillColor,fill opacity=0.20] (199.48, 78.06) circle (  2.13);

\path[fill=fillColor,fill opacity=0.20] (204.49, 70.80) circle (  2.13);

\path[fill=fillColor,fill opacity=0.20] (209.51, 65.09) circle (  2.13);

\path[fill=fillColor,fill opacity=0.20] (209.51, 60.21) circle (  2.13);

\path[fill=fillColor,fill opacity=0.20] (209.51, 59.48) circle (  2.13);

\path[fill=fillColor,fill opacity=0.20] (217.53, 70.48) circle (  2.13);

\path[fill=fillColor,fill opacity=0.20] (219.54, 77.86) circle (  2.13);

\path[fill=fillColor,fill opacity=0.20] (218.54, 75.05) circle (  2.13);

\path[fill=fillColor,fill opacity=0.20] (196.47, 86.89) circle (  2.13);

\path[fill=fillColor,fill opacity=0.20] (195.46, 73.29) circle (  2.13);

\path[fill=fillColor,fill opacity=0.20] (199.48, 65.29) circle (  2.13);

\path[fill=fillColor,fill opacity=0.20] (200.48, 70.48) circle (  2.13);

\path[fill=fillColor,fill opacity=0.20] (201.48, 76.51) circle (  2.13);

\path[fill=fillColor,fill opacity=0.20] (203.49, 73.50) circle (  2.13);

\path[fill=fillColor,fill opacity=0.20] (205.50, 70.48) circle (  2.13);

\path[fill=fillColor,fill opacity=0.20] (209.51, 72.35) circle (  2.13);

\path[fill=fillColor,fill opacity=0.20] (211.52, 70.28) circle (  2.13);

\path[fill=fillColor,fill opacity=0.20] (215.53, 67.68) circle (  2.13);

\path[fill=fillColor,fill opacity=0.20] (224.56, 74.33) circle (  2.13);

\path[fill=fillColor,fill opacity=0.20] (168.38, 75.68) circle (  2.13);

\path[fill=fillColor,fill opacity=0.20] (195.46, 88.96) circle (  2.13);

\path[fill=fillColor,fill opacity=0.20] (200.48, 79.10) circle (  2.13);

\path[fill=fillColor,fill opacity=0.20] (202.49, 76.82) circle (  2.13);

\path[fill=fillColor,fill opacity=0.20] (203.49, 74.53) circle (  2.13);

\path[fill=fillColor,fill opacity=0.20] (205.50, 72.04) circle (  2.13);

\path[fill=fillColor,fill opacity=0.20] (206.50, 68.82) circle (  2.13);

\path[fill=fillColor,fill opacity=0.20] (209.51, 67.89) circle (  2.13);

\path[fill=fillColor,fill opacity=0.20] (211.52, 73.70) circle (  2.13);

\path[fill=fillColor,fill opacity=0.20] (215.53, 76.82) circle (  2.13);

\path[fill=fillColor,fill opacity=0.20] (223.55, 73.50) circle (  2.13);

\path[fill=fillColor,fill opacity=0.20] (238.60, 75.99) circle (  2.13);

\path[fill=fillColor,fill opacity=0.20] (207.50, 62.91) circle (  2.13);

\path[fill=fillColor,fill opacity=0.20] (204.49, 65.61) circle (  2.13);

\path[fill=fillColor,fill opacity=0.20] (188.44, 65.81) circle (  2.13);

\path[fill=fillColor,fill opacity=0.20] (210.51, 94.16) circle (  2.13);

\path[fill=fillColor,fill opacity=0.20] (198.47, 79.52) circle (  2.13);

\path[fill=fillColor,fill opacity=0.20] (201.48, 74.12) circle (  2.13);

\path[fill=fillColor,fill opacity=0.20] (205.50, 79.10) circle (  2.13);

\path[fill=fillColor,fill opacity=0.20] (205.50, 75.88) circle (  2.13);

\path[fill=fillColor,fill opacity=0.20] (208.51, 70.28) circle (  2.13);

\path[fill=fillColor,fill opacity=0.20] (210.51, 70.38) circle (  2.13);

\path[fill=fillColor,fill opacity=0.20] (212.52, 70.07) circle (  2.13);

\path[fill=fillColor,fill opacity=0.20] (213.52, 72.77) circle (  2.13);

\path[fill=fillColor,fill opacity=0.20] (222.55, 76.82) circle (  2.13);

\path[fill=fillColor,fill opacity=0.20] (209.51, 62.08) circle (  2.13);

\path[fill=fillColor,fill opacity=0.20] (211.52, 60.10) circle (  2.13);

\path[fill=fillColor,fill opacity=0.20] (201.48, 65.40) circle (  2.13);

\path[fill=fillColor,fill opacity=0.20] (193.46, 59.48) circle (  2.13);

\path[fill=fillColor,fill opacity=0.20] (189.44, 57.92) circle (  2.13);

\path[fill=fillColor,fill opacity=0.20] (189.44, 60.62) circle (  2.13);

\path[fill=fillColor,fill opacity=0.20] (196.47, 67.27) circle (  2.13);

\path[fill=fillColor,fill opacity=0.20] (204.49, 76.30) circle (  2.13);

\path[fill=fillColor,fill opacity=0.20] (207.50, 88.96) circle (  2.13);

\path[fill=fillColor,fill opacity=0.20] (196.47, 68.72) circle (  2.13);

\path[fill=fillColor,fill opacity=0.20] (203.49, 66.44) circle (  2.13);

\path[fill=fillColor,fill opacity=0.20] (208.51, 76.82) circle (  2.13);

\path[fill=fillColor,fill opacity=0.20] (208.51, 72.87) circle (  2.13);

\path[fill=fillColor,fill opacity=0.20] (214.53, 66.44) circle (  2.13);

\path[fill=fillColor,fill opacity=0.20] (214.53, 69.86) circle (  2.13);

\path[fill=fillColor,fill opacity=0.20] (212.52, 72.87) circle (  2.13);

\path[fill=fillColor,fill opacity=0.20] (214.53, 74.64) circle (  2.13);

\path[fill=fillColor,fill opacity=0.20] (206.50, 61.56) circle (  2.13);

\path[fill=fillColor,fill opacity=0.20] (199.48, 68.10) circle (  2.13);

\path[fill=fillColor,fill opacity=0.20] (198.47, 67.37) circle (  2.13);

\path[fill=fillColor,fill opacity=0.20] (201.48, 59.58) circle (  2.13);

\path[fill=fillColor,fill opacity=0.20] (189.44, 53.67) circle (  2.13);

\path[fill=fillColor,fill opacity=0.20] (189.44, 52.42) circle (  2.13);

\path[fill=fillColor,fill opacity=0.20] (194.46, 60.10) circle (  2.13);

\path[fill=fillColor,fill opacity=0.20] (198.47, 68.30) circle (  2.13);

\path[fill=fillColor,fill opacity=0.20] (198.47, 70.69) circle (  2.13);

\path[fill=fillColor,fill opacity=0.20] (165.87, 73.29) circle (  2.13);

\path[fill=fillColor,fill opacity=0.20] (210.51, 93.12) circle (  2.13);

\path[fill=fillColor,fill opacity=0.20] (198.47, 75.99) circle (  2.13);

\path[fill=fillColor,fill opacity=0.20] (206.50, 66.33) circle (  2.13);

\path[fill=fillColor,fill opacity=0.20] (209.51, 74.95) circle (  2.13);

\path[fill=fillColor,fill opacity=0.20] (206.50, 76.71) circle (  2.13);

\path[fill=fillColor,fill opacity=0.20] (210.51, 67.99) circle (  2.13);

\path[fill=fillColor,fill opacity=0.20] (215.53, 64.67) circle (  2.13);

\path[fill=fillColor,fill opacity=0.20] (212.52, 70.59) circle (  2.13);

\path[fill=fillColor,fill opacity=0.20] (213.52, 79.52) circle (  2.13);

\path[fill=fillColor,fill opacity=0.20] (206.50, 83.77) circle (  2.13);

\path[fill=fillColor,fill opacity=0.20] (200.48, 45.46) circle (  2.13);

\path[fill=fillColor,fill opacity=0.20] (202.49, 59.48) circle (  2.13);

\path[fill=fillColor,fill opacity=0.20] (199.48, 62.80) circle (  2.13);

\path[fill=fillColor,fill opacity=0.20] (191.45, 53.98) circle (  2.13);

\path[fill=fillColor,fill opacity=0.20] (189.44, 49.82) circle (  2.13);

\path[fill=fillColor,fill opacity=0.20] (189.44, 53.25) circle (  2.13);

\path[fill=fillColor,fill opacity=0.20] (192.45, 59.17) circle (  2.13);

\path[fill=fillColor,fill opacity=0.20] (195.46, 56.05) circle (  2.13);

\path[fill=fillColor,fill opacity=0.20] (198.47, 52.11) circle (  2.13);

\path[fill=fillColor,fill opacity=0.20] (199.48, 66.75) circle (  2.13);

\path[fill=fillColor,fill opacity=0.20] (193.46, 83.77) circle (  2.13);

\path[fill=fillColor,fill opacity=0.20] (203.49, 81.18) circle (  2.13);

\path[fill=fillColor,fill opacity=0.20] (203.49, 63.84) circle (  2.13);

\path[fill=fillColor,fill opacity=0.20] (206.50, 71.42) circle (  2.13);

\path[fill=fillColor,fill opacity=0.20] (207.50, 83.77) circle (  2.13);

\path[fill=fillColor,fill opacity=0.20] (204.49, 73.91) circle (  2.13);

\path[fill=fillColor,fill opacity=0.20] (209.51, 58.65) circle (  2.13);

\path[fill=fillColor,fill opacity=0.20] (215.53, 62.28) circle (  2.13);

\path[fill=fillColor,fill opacity=0.20] (217.53, 76.51) circle (  2.13);

\path[fill=fillColor,fill opacity=0.20] (218.54, 79.21) circle (  2.13);

\path[fill=fillColor,fill opacity=0.20] (223.55, 75.16) circle (  2.13);

\path[fill=fillColor,fill opacity=0.20] (202.49, 59.58) circle (  2.13);

\path[fill=fillColor,fill opacity=0.20] (199.48, 49.10) circle (  2.13);

\path[fill=fillColor,fill opacity=0.20] (199.48, 50.24) circle (  2.13);

\path[fill=fillColor,fill opacity=0.20] (195.46, 43.49) circle (  2.13);

\path[fill=fillColor,fill opacity=0.20] (195.46, 51.17) circle (  2.13);

\path[fill=fillColor,fill opacity=0.20] (192.45, 58.55) circle (  2.13);

\path[fill=fillColor,fill opacity=0.20] (189.44, 58.96) circle (  2.13);

\path[fill=fillColor,fill opacity=0.20] (190.45, 52.63) circle (  2.13);

\path[fill=fillColor,fill opacity=0.20] (193.46, 43.49) circle (  2.13);

\path[fill=fillColor,fill opacity=0.20] (188.44, 56.99) circle (  2.13);

\path[fill=fillColor,fill opacity=0.20] (201.48, 77.23) circle (  2.13);

\path[fill=fillColor,fill opacity=0.20] (191.45, 69.24) circle (  2.13);

\path[fill=fillColor,fill opacity=0.20] (202.49, 71.00) circle (  2.13);

\path[fill=fillColor,fill opacity=0.20] (198.47, 54.91) circle (  2.13);

\path[fill=fillColor,fill opacity=0.20] (204.49, 64.67) circle (  2.13);

\path[fill=fillColor,fill opacity=0.20] (208.51, 77.54) circle (  2.13);

\path[fill=fillColor,fill opacity=0.20] (213.52, 68.20) circle (  2.13);

\path[fill=fillColor,fill opacity=0.20] (218.54, 54.81) circle (  2.13);

\path[fill=fillColor,fill opacity=0.20] (217.53, 58.13) circle (  2.13);

\path[fill=fillColor,fill opacity=0.20] (216.53, 66.44) circle (  2.13);

\path[fill=fillColor,fill opacity=0.20] (219.54, 67.16) circle (  2.13);

\path[fill=fillColor,fill opacity=0.20] (221.55, 64.05) circle (  2.13);

\path[fill=fillColor,fill opacity=0.20] (199.48, 57.40) circle (  2.13);

\path[fill=fillColor,fill opacity=0.20] (194.46, 38.51) circle (  2.13);

\path[fill=fillColor,fill opacity=0.20] (202.49, 51.59) circle (  2.13);

\path[fill=fillColor,fill opacity=0.20] (200.48, 44.94) circle (  2.13);

\path[fill=fillColor,fill opacity=0.20] (198.47, 46.19) circle (  2.13);

\path[fill=fillColor,fill opacity=0.20] (196.47, 57.71) circle (  2.13);

\path[fill=fillColor,fill opacity=0.20] (194.46, 58.86) circle (  2.13);

\path[fill=fillColor,fill opacity=0.20] (191.45, 54.29) circle (  2.13);

\path[fill=fillColor,fill opacity=0.20] (191.45, 53.67) circle (  2.13);

\path[fill=fillColor,fill opacity=0.20] (190.45, 47.64) circle (  2.13);

\path[fill=fillColor,fill opacity=0.20] (190.45, 48.58) circle (  2.13);

\path[fill=fillColor,fill opacity=0.20] (202.49, 62.28) circle (  2.13);

\path[fill=fillColor,fill opacity=0.20] (210.51, 68.72) circle (  2.13);

\path[fill=fillColor,fill opacity=0.20] (205.50, 70.38) circle (  2.13);

\path[fill=fillColor,fill opacity=0.20] (198.47, 53.46) circle (  2.13);

\path[fill=fillColor,fill opacity=0.20] (205.50, 58.96) circle (  2.13);

\path[fill=fillColor,fill opacity=0.20] (211.52, 66.75) circle (  2.13);

\path[fill=fillColor,fill opacity=0.20] (213.52, 59.06) circle (  2.13);

\path[fill=fillColor,fill opacity=0.20] (216.53, 54.91) circle (  2.13);

\path[fill=fillColor,fill opacity=0.20] (213.52, 61.87) circle (  2.13);

\path[fill=fillColor,fill opacity=0.20] (214.53, 65.92) circle (  2.13);

\path[fill=fillColor,fill opacity=0.20] (214.53, 62.80) circle (  2.13);

\path[fill=fillColor,fill opacity=0.20] (217.53, 55.85) circle (  2.13);

\path[fill=fillColor,fill opacity=0.20] (223.55, 56.88) circle (  2.13);

\path[fill=fillColor,fill opacity=0.20] (200.48, 68.41) circle (  2.13);

\path[fill=fillColor,fill opacity=0.20] (193.46, 50.65) circle (  2.13);

\path[fill=fillColor,fill opacity=0.20] (198.47, 57.09) circle (  2.13);

\path[fill=fillColor,fill opacity=0.20] (199.48, 53.56) circle (  2.13);

\path[fill=fillColor,fill opacity=0.20] (200.48, 61.56) circle (  2.13);

\path[fill=fillColor,fill opacity=0.20] (195.46, 60.83) circle (  2.13);

\path[fill=fillColor,fill opacity=0.20] (194.46, 51.69) circle (  2.13);

\path[fill=fillColor,fill opacity=0.20] (193.46, 55.53) circle (  2.13);

\path[fill=fillColor,fill opacity=0.20] (192.45, 58.03) circle (  2.13);

\path[fill=fillColor,fill opacity=0.20] (190.45, 48.47) circle (  2.13);

\path[fill=fillColor,fill opacity=0.20] (192.45, 42.25) circle (  2.13);

\path[fill=fillColor,fill opacity=0.20] (198.47, 48.68) circle (  2.13);

\path[fill=fillColor,fill opacity=0.20] (208.51, 61.76) circle (  2.13);

\path[fill=fillColor,fill opacity=0.20] (225.56, 87.93) circle (  2.13);

\path[fill=fillColor,fill opacity=0.20] (207.50, 63.01) circle (  2.13);

\path[fill=fillColor,fill opacity=0.20] (207.50, 57.61) circle (  2.13);

\path[fill=fillColor,fill opacity=0.20] (210.51, 63.74) circle (  2.13);

\path[fill=fillColor,fill opacity=0.20] (211.52, 61.97) circle (  2.13);

\path[fill=fillColor,fill opacity=0.20] (210.51, 57.30) circle (  2.13);

\path[fill=fillColor,fill opacity=0.20] (212.52, 62.08) circle (  2.13);

\path[fill=fillColor,fill opacity=0.20] (212.52, 71.21) circle (  2.13);

\path[fill=fillColor,fill opacity=0.20] (215.53, 70.90) circle (  2.13);

\path[fill=fillColor,fill opacity=0.20] (215.53, 56.26) circle (  2.13);

\path[fill=fillColor,fill opacity=0.20] (216.53, 46.61) circle (  2.13);

\path[fill=fillColor,fill opacity=0.20] (224.56, 58.44) circle (  2.13);

\path[fill=fillColor,fill opacity=0.20] (228.57, 80.87) circle (  2.13);

\path[fill=fillColor,fill opacity=0.20] (202.49, 57.20) circle (  2.13);

\path[fill=fillColor,fill opacity=0.20] (197.47, 59.06) circle (  2.13);

\path[fill=fillColor,fill opacity=0.20] (197.47, 58.96) circle (  2.13);

\path[fill=fillColor,fill opacity=0.20] (194.46, 69.24) circle (  2.13);

\path[fill=fillColor,fill opacity=0.20] (190.45, 61.45) circle (  2.13);

\path[fill=fillColor,fill opacity=0.20] (190.45, 47.75) circle (  2.13);

\path[fill=fillColor,fill opacity=0.20] (190.45, 61.56) circle (  2.13);

\path[fill=fillColor,fill opacity=0.20] (191.45, 64.57) circle (  2.13);

\path[fill=fillColor,fill opacity=0.20] (192.45, 46.92) circle (  2.13);

\path[fill=fillColor,fill opacity=0.20] (196.47, 42.66) circle (  2.13);

\path[fill=fillColor,fill opacity=0.20] (203.49, 47.75) circle (  2.13);

\path[fill=fillColor,fill opacity=0.20] (205.50, 52.52) circle (  2.13);

\path[fill=fillColor,fill opacity=0.20] (211.52, 81.28) circle (  2.13);

\path[fill=fillColor,fill opacity=0.20] (228.57, 76.71) circle (  2.13);

\path[fill=fillColor,fill opacity=0.20] (209.51, 59.89) circle (  2.13);

\path[fill=fillColor,fill opacity=0.20] (209.51, 57.61) circle (  2.13);

\path[fill=fillColor,fill opacity=0.20] (207.50, 60.83) circle (  2.13);

\path[fill=fillColor,fill opacity=0.20] (209.51, 59.89) circle (  2.13);

\path[fill=fillColor,fill opacity=0.20] (209.51, 59.27) circle (  2.13);

\path[fill=fillColor,fill opacity=0.20] (211.52, 66.23) circle (  2.13);

\path[fill=fillColor,fill opacity=0.20] (211.52, 70.07) circle (  2.13);

\path[fill=fillColor,fill opacity=0.20] (210.51, 60.62) circle (  2.13);

\path[fill=fillColor,fill opacity=0.20] (213.52, 52.63) circle (  2.13);

\path[fill=fillColor,fill opacity=0.20] (222.55, 59.89) circle (  2.13);

\path[fill=fillColor,fill opacity=0.20] (219.54, 69.24) circle (  2.13);

\path[fill=fillColor,fill opacity=0.20] (223.55, 75.57) circle (  2.13);

\path[fill=fillColor,fill opacity=0.20] (209.51, 72.77) circle (  2.13);

\path[fill=fillColor,fill opacity=0.20] (201.48, 59.06) circle (  2.13);

\path[fill=fillColor,fill opacity=0.20] (200.48, 59.58) circle (  2.13);

\path[fill=fillColor,fill opacity=0.20] (194.46, 57.40) circle (  2.13);

\path[fill=fillColor,fill opacity=0.20] (191.45, 62.08) circle (  2.13);

\path[fill=fillColor,fill opacity=0.20] (191.45, 60.83) circle (  2.13);

\path[fill=fillColor,fill opacity=0.20] (191.45, 54.18) circle (  2.13);

\path[fill=fillColor,fill opacity=0.20] (189.44, 62.18) circle (  2.13);

\path[fill=fillColor,fill opacity=0.20] (194.46, 61.66) circle (  2.13);

\path[fill=fillColor,fill opacity=0.20] (196.47, 49.93) circle (  2.13);

\path[fill=fillColor,fill opacity=0.20] (194.46, 52.63) circle (  2.13);

\path[fill=fillColor,fill opacity=0.20] (198.47, 52.73) circle (  2.13);

\path[fill=fillColor,fill opacity=0.20] (207.50, 47.75) circle (  2.13);

\path[fill=fillColor,fill opacity=0.20] (184.73, 82.74) circle (  2.13);

\path[fill=fillColor,fill opacity=0.20] (218.54, 66.75) circle (  2.13);

\path[fill=fillColor,fill opacity=0.20] (200.48, 46.71) circle (  2.13);

\path[fill=fillColor,fill opacity=0.20] (210.51, 47.54) circle (  2.13);

\path[fill=fillColor,fill opacity=0.20] (211.52, 56.99) circle (  2.13);

\path[fill=fillColor,fill opacity=0.20] (208.51, 58.55) circle (  2.13);

\path[fill=fillColor,fill opacity=0.20] (213.52, 57.61) circle (  2.13);

\path[fill=fillColor,fill opacity=0.20] (193.46, 60.41) circle (  2.13);

\path[fill=fillColor,fill opacity=0.20] (210.51, 61.45) circle (  2.13);

\path[fill=fillColor,fill opacity=0.20] (216.53, 62.59) circle (  2.13);

\path[fill=fillColor,fill opacity=0.20] (218.54, 65.29) circle (  2.13);

\path[fill=fillColor,fill opacity=0.20] (217.53, 67.79) circle (  2.13);

\path[fill=fillColor,fill opacity=0.20] (219.54, 72.15) circle (  2.13);

\path[fill=fillColor,fill opacity=0.20] (221.55, 68.82) circle (  2.13);

\path[fill=fillColor,fill opacity=0.20] (224.56, 67.68) circle (  2.13);

\path[fill=fillColor,fill opacity=0.20] (205.50, 84.81) circle (  2.13);

\path[fill=fillColor,fill opacity=0.20] (200.48, 65.81) circle (  2.13);

\path[fill=fillColor,fill opacity=0.20] (200.48, 58.23) circle (  2.13);

\path[fill=fillColor,fill opacity=0.20] (200.48, 60.83) circle (  2.13);

\path[fill=fillColor,fill opacity=0.20] (200.48, 56.57) circle (  2.13);

\path[fill=fillColor,fill opacity=0.20] (196.47, 49.93) circle (  2.13);

\path[fill=fillColor,fill opacity=0.20] (189.44, 51.28) circle (  2.13);

\path[fill=fillColor,fill opacity=0.20] (192.45, 57.61) circle (  2.13);

\path[fill=fillColor,fill opacity=0.20] (191.45, 58.96) circle (  2.13);

\path[fill=fillColor,fill opacity=0.20] (193.46, 51.28) circle (  2.13);

\path[fill=fillColor,fill opacity=0.20] (198.47, 55.12) circle (  2.13);

\path[fill=fillColor,fill opacity=0.20] (193.46, 63.94) circle (  2.13);

\path[fill=fillColor,fill opacity=0.20] (192.45, 53.77) circle (  2.13);

\path[fill=fillColor,fill opacity=0.20] (209.51, 50.65) circle (  2.13);

\path[fill=fillColor,fill opacity=0.20] (239.61, 58.13) circle (  2.13);

\path[fill=fillColor,fill opacity=0.20] (224.56, 49.93) circle (  2.13);

\path[fill=fillColor,fill opacity=0.20] (215.53, 51.80) circle (  2.13);

\path[fill=fillColor,fill opacity=0.20] (206.50, 52.84) circle (  2.13);

\path[fill=fillColor,fill opacity=0.20] (211.52, 53.98) circle (  2.13);

\path[fill=fillColor,fill opacity=0.20] (214.53, 54.70) circle (  2.13);

\path[fill=fillColor,fill opacity=0.20] (205.50, 56.68) circle (  2.13);

\path[fill=fillColor,fill opacity=0.20] (214.53, 65.71) circle (  2.13);

\path[fill=fillColor,fill opacity=0.20] (214.53, 68.72) circle (  2.13);

\path[fill=fillColor,fill opacity=0.20] (217.53, 66.02) circle (  2.13);

\path[fill=fillColor,fill opacity=0.20] (220.54, 72.25) circle (  2.13);

\path[fill=fillColor,fill opacity=0.20] (215.53, 69.03) circle (  2.13);

\path[fill=fillColor,fill opacity=0.20] (216.53, 59.69) circle (  2.13);

\path[fill=fillColor,fill opacity=0.20] (218.54, 66.85) circle (  2.13);

\path[fill=fillColor,fill opacity=0.20] (215.53, 79.93) circle (  2.13);

\path[fill=fillColor,fill opacity=0.20] (210.51, 70.69) circle (  2.13);

\path[fill=fillColor,fill opacity=0.20] (209.51, 72.15) circle (  2.13);

\path[fill=fillColor,fill opacity=0.20] (207.50, 71.63) circle (  2.13);

\path[fill=fillColor,fill opacity=0.20] (203.49, 74.33) circle (  2.13);

\path[fill=fillColor,fill opacity=0.20] (199.48, 71.32) circle (  2.13);

\path[fill=fillColor,fill opacity=0.20] (196.47, 63.84) circle (  2.13);

\path[fill=fillColor,fill opacity=0.20] (196.47, 59.69) circle (  2.13);

\path[fill=fillColor,fill opacity=0.20] (198.47, 60.10) circle (  2.13);

\path[fill=fillColor,fill opacity=0.20] (202.49, 57.82) circle (  2.13);

\path[fill=fillColor,fill opacity=0.20] (196.47, 49.62) circle (  2.13);

\path[fill=fillColor,fill opacity=0.20] (196.47, 44.84) circle (  2.13);

\path[fill=fillColor,fill opacity=0.20] (189.44, 51.90) circle (  2.13);

\path[fill=fillColor,fill opacity=0.20] (193.46, 55.85) circle (  2.13);

\path[fill=fillColor,fill opacity=0.20] (194.46, 52.94) circle (  2.13);

\path[fill=fillColor,fill opacity=0.20] (194.46, 59.58) circle (  2.13);

\path[fill=fillColor,fill opacity=0.20] (192.45, 63.84) circle (  2.13);

\path[fill=fillColor,fill opacity=0.20] (202.49, 53.56) circle (  2.13);

\path[fill=fillColor,fill opacity=0.20] (208.51, 59.06) circle (  2.13);

\path[fill=fillColor,fill opacity=0.20] (168.08, 86.89) circle (  2.13);

\path[fill=fillColor,fill opacity=0.20] (233.59, 78.58) circle (  2.13);

\path[fill=fillColor,fill opacity=0.20] (224.56, 64.46) circle (  2.13);

\path[fill=fillColor,fill opacity=0.20] (216.53, 53.46) circle (  2.13);

\path[fill=fillColor,fill opacity=0.20] (218.54, 52.32) circle (  2.13);

\path[fill=fillColor,fill opacity=0.20] (209.51, 49.72) circle (  2.13);

\path[fill=fillColor,fill opacity=0.20] (209.51, 49.72) circle (  2.13);

\path[fill=fillColor,fill opacity=0.20] (210.51, 63.94) circle (  2.13);

\path[fill=fillColor,fill opacity=0.20] (210.51, 73.91) circle (  2.13);

\path[fill=fillColor,fill opacity=0.20] (217.53, 68.30) circle (  2.13);

\path[fill=fillColor,fill opacity=0.20] (218.54, 69.13) circle (  2.13);

\path[fill=fillColor,fill opacity=0.20] (212.52, 70.69) circle (  2.13);

\path[fill=fillColor,fill opacity=0.20] (213.52, 67.89) circle (  2.13);

\path[fill=fillColor,fill opacity=0.20] (213.52, 67.37) circle (  2.13);

\path[fill=fillColor,fill opacity=0.20] (217.53, 58.55) circle (  2.13);

\path[fill=fillColor,fill opacity=0.20] (225.56, 50.24) circle (  2.13);

\path[fill=fillColor,fill opacity=0.20] (226.56, 61.24) circle (  2.13);

\path[fill=fillColor,fill opacity=0.20] (215.53, 75.26) circle (  2.13);

\path[fill=fillColor,fill opacity=0.20] (216.53, 73.08) circle (  2.13);

\path[fill=fillColor,fill opacity=0.20] (215.53, 73.91) circle (  2.13);

\path[fill=fillColor,fill opacity=0.20] (197.47, 70.69) circle (  2.13);

\path[fill=fillColor,fill opacity=0.20] (200.48, 72.46) circle (  2.13);

\path[fill=fillColor,fill opacity=0.20] (203.49, 73.70) circle (  2.13);

\path[fill=fillColor,fill opacity=0.20] (203.49, 68.72) circle (  2.13);

\path[fill=fillColor,fill opacity=0.20] (199.48, 72.15) circle (  2.13);

\path[fill=fillColor,fill opacity=0.20] (199.48, 71.83) circle (  2.13);

\path[fill=fillColor,fill opacity=0.20] (198.47, 63.11) circle (  2.13);

\path[fill=fillColor,fill opacity=0.20] (199.48, 57.30) circle (  2.13);

\path[fill=fillColor,fill opacity=0.20] (198.47, 59.27) circle (  2.13);

\path[fill=fillColor,fill opacity=0.20] (198.47, 59.48) circle (  2.13);

\path[fill=fillColor,fill opacity=0.20] (199.48, 59.38) circle (  2.13);

\path[fill=fillColor,fill opacity=0.20] (196.47, 56.16) circle (  2.13);

\path[fill=fillColor,fill opacity=0.20] (194.46, 51.17) circle (  2.13);

\path[fill=fillColor,fill opacity=0.20] (192.45, 54.50) circle (  2.13);

\path[fill=fillColor,fill opacity=0.20] (196.47, 63.94) circle (  2.13);

\path[fill=fillColor,fill opacity=0.20] (196.47, 62.49) circle (  2.13);

\path[fill=fillColor,fill opacity=0.20] (193.46, 53.98) circle (  2.13);

\path[fill=fillColor,fill opacity=0.20] (195.46, 51.28) circle (  2.13);

\path[fill=fillColor,fill opacity=0.20] (213.52, 60.62) circle (  2.13);

\path[fill=fillColor,fill opacity=0.20] (208.51, 81.70) circle (  2.13);

\path[fill=fillColor,fill opacity=0.20] (182.72, 81.70) circle (  2.13);

\path[fill=fillColor,fill opacity=0.20] (229.57, 68.72) circle (  2.13);

\path[fill=fillColor,fill opacity=0.20] (223.55, 58.34) circle (  2.13);

\path[fill=fillColor,fill opacity=0.20] (207.50, 54.08) circle (  2.13);

\path[fill=fillColor,fill opacity=0.20] (209.51, 56.05) circle (  2.13);

\path[fill=fillColor,fill opacity=0.20] (209.51, 64.15) circle (  2.13);

\path[fill=fillColor,fill opacity=0.20] (210.51, 64.77) circle (  2.13);

\path[fill=fillColor,fill opacity=0.20] (209.51, 62.18) circle (  2.13);

\path[fill=fillColor,fill opacity=0.20] (212.52, 63.32) circle (  2.13);

\path[fill=fillColor,fill opacity=0.20] (210.51, 65.29) circle (  2.13);

\path[fill=fillColor,fill opacity=0.20] (214.53, 60.62) circle (  2.13);

\path[fill=fillColor,fill opacity=0.20] (218.54, 52.42) circle (  2.13);

\path[fill=fillColor,fill opacity=0.20] (220.54, 56.57) circle (  2.13);

\path[fill=fillColor,fill opacity=0.20] (220.54, 64.36) circle (  2.13);

\path[fill=fillColor,fill opacity=0.20] (218.54, 58.65) circle (  2.13);

\path[fill=fillColor,fill opacity=0.20] (219.54, 60.21) circle (  2.13);

\path[fill=fillColor,fill opacity=0.20] (212.52, 80.24) circle (  2.13);

\path[fill=fillColor,fill opacity=0.20] (212.52, 80.97) circle (  2.13);

\path[fill=fillColor,fill opacity=0.20] (211.52, 72.87) circle (  2.13);

\path[fill=fillColor,fill opacity=0.20] (200.48, 61.66) circle (  2.13);

\path[fill=fillColor,fill opacity=0.20] (200.48, 61.87) circle (  2.13);

\path[fill=fillColor,fill opacity=0.20] (206.50, 65.92) circle (  2.13);

\path[fill=fillColor,fill opacity=0.20] (206.50, 61.66) circle (  2.13);

\path[fill=fillColor,fill opacity=0.20] (196.47, 63.42) circle (  2.13);

\path[fill=fillColor,fill opacity=0.20] (193.46, 64.67) circle (  2.13);

\path[fill=fillColor,fill opacity=0.20] (200.48, 53.56) circle (  2.13);

\path[fill=fillColor,fill opacity=0.20] (198.47, 49.41) circle (  2.13);

\path[fill=fillColor,fill opacity=0.20] (197.47, 58.65) circle (  2.13);

\path[fill=fillColor,fill opacity=0.20] (199.48, 61.97) circle (  2.13);

\path[fill=fillColor,fill opacity=0.20] (198.47, 64.26) circle (  2.13);

\path[fill=fillColor,fill opacity=0.20] (199.48, 65.61) circle (  2.13);

\path[fill=fillColor,fill opacity=0.20] (199.48, 56.99) circle (  2.13);

\path[fill=fillColor,fill opacity=0.20] (199.48, 56.68) circle (  2.13);

\path[fill=fillColor,fill opacity=0.20] (202.49, 70.07) circle (  2.13);

\path[fill=fillColor,fill opacity=0.20] (201.48, 66.02) circle (  2.13);

\path[fill=fillColor,fill opacity=0.20] (194.46, 49.41) circle (  2.13);

\path[fill=fillColor,fill opacity=0.20] (197.47, 44.01) circle (  2.13);

\path[fill=fillColor,fill opacity=0.20] (216.53, 49.62) circle (  2.13);

\path[fill=fillColor,fill opacity=0.20] (223.55, 75.26) circle (  2.13);

\path[fill=fillColor,fill opacity=0.20] (243.62, 90.00) circle (  2.13);

\path[fill=fillColor,fill opacity=0.20] (232.58, 74.43) circle (  2.13);

\path[fill=fillColor,fill opacity=0.20] (219.54, 55.02) circle (  2.13);

\path[fill=fillColor,fill opacity=0.20] (219.54, 51.07) circle (  2.13);

\path[fill=fillColor,fill opacity=0.20] (217.53, 55.74) circle (  2.13);

\path[fill=fillColor,fill opacity=0.20] (208.51, 50.65) circle (  2.13);

\path[fill=fillColor,fill opacity=0.20] (212.52, 54.91) circle (  2.13);

\path[fill=fillColor,fill opacity=0.20] (214.53, 64.05) circle (  2.13);

\path[fill=fillColor,fill opacity=0.20] (206.50, 54.60) circle (  2.13);

\path[fill=fillColor,fill opacity=0.20] (216.53, 51.28) circle (  2.13);

\path[fill=fillColor,fill opacity=0.20] (214.53, 69.76) circle (  2.13);

\path[fill=fillColor,fill opacity=0.20] (214.53, 70.80) circle (  2.13);

\path[fill=fillColor,fill opacity=0.20] (214.53, 47.33) circle (  2.13);

\path[fill=fillColor,fill opacity=0.20] (214.53, 45.67) circle (  2.13);

\path[fill=fillColor,fill opacity=0.20] (218.54, 70.80) circle (  2.13);

\path[fill=fillColor,fill opacity=0.20] (210.51, 81.07) circle (  2.13);

\path[fill=fillColor,fill opacity=0.20] (207.50, 59.38) circle (  2.13);

\path[fill=fillColor,fill opacity=0.20] (209.51, 41.31) circle (  2.13);

\path[fill=fillColor,fill opacity=0.20] (215.53, 52.32) circle (  2.13);

\path[fill=fillColor,fill opacity=0.20] (215.53, 66.95) circle (  2.13);

\path[fill=fillColor,fill opacity=0.20] (207.50, 69.86) circle (  2.13);

\path[fill=fillColor,fill opacity=0.20] (212.52, 63.01) circle (  2.13);

\path[fill=fillColor,fill opacity=0.20] (210.51, 49.20) circle (  2.13);

\path[fill=fillColor,fill opacity=0.20] (208.51, 43.80) circle (  2.13);

\path[fill=fillColor,fill opacity=0.20] (208.51, 53.35) circle (  2.13);

\path[fill=fillColor,fill opacity=0.20] (206.50, 61.04) circle (  2.13);

\path[fill=fillColor,fill opacity=0.20] (200.48, 59.69) circle (  2.13);

\path[fill=fillColor,fill opacity=0.20] (200.48, 53.98) circle (  2.13);

\path[fill=fillColor,fill opacity=0.20] (202.49, 46.81) circle (  2.13);

\path[fill=fillColor,fill opacity=0.20] (193.46, 47.23) circle (  2.13);

\path[fill=fillColor,fill opacity=0.20] (201.48, 56.16) circle (  2.13);

\path[fill=fillColor,fill opacity=0.20] (200.48, 60.00) circle (  2.13);

\path[fill=fillColor,fill opacity=0.20] (202.49, 62.28) circle (  2.13);

\path[fill=fillColor,fill opacity=0.20] (203.49, 63.63) circle (  2.13);

\path[fill=fillColor,fill opacity=0.20] (203.49, 58.23) circle (  2.13);

\path[fill=fillColor,fill opacity=0.20] (200.48, 56.99) circle (  2.13);

\path[fill=fillColor,fill opacity=0.20] (199.48, 66.33) circle (  2.13);

\path[fill=fillColor,fill opacity=0.20] (197.47, 65.81) circle (  2.13);

\path[fill=fillColor,fill opacity=0.20] (202.49, 49.51) circle (  2.13);

\path[fill=fillColor,fill opacity=0.20] (219.54, 48.58) circle (  2.13);

\path[fill=fillColor,fill opacity=0.20] (243.62, 74.33) circle (  2.13);

\path[fill=fillColor,fill opacity=0.20] (232.58, 70.07) circle (  2.13);

\path[fill=fillColor,fill opacity=0.20] (233.59, 63.01) circle (  2.13);

\path[fill=fillColor,fill opacity=0.20] (213.52, 52.63) circle (  2.13);

\path[fill=fillColor,fill opacity=0.20] (219.54, 59.38) circle (  2.13);

\path[fill=fillColor,fill opacity=0.20] (217.53, 64.05) circle (  2.13);

\path[fill=fillColor,fill opacity=0.20] (214.53, 50.86) circle (  2.13);

\path[fill=fillColor,fill opacity=0.20] (209.51, 51.28) circle (  2.13);

\path[fill=fillColor,fill opacity=0.20] (212.52, 69.34) circle (  2.13);

\path[fill=fillColor,fill opacity=0.20] (215.53, 64.46) circle (  2.13);

\path[fill=fillColor,fill opacity=0.20] (215.53, 46.09) circle (  2.13);

\path[fill=fillColor,fill opacity=0.20] (208.51, 51.17) circle (  2.13);

\path[fill=fillColor,fill opacity=0.20] (213.52, 69.76) circle (  2.13);

\path[fill=fillColor,fill opacity=0.20] (217.53, 72.56) circle (  2.13);

\path[fill=fillColor,fill opacity=0.20] (212.52, 61.45) circle (  2.13);

\path[fill=fillColor,fill opacity=0.20] (216.53, 55.64) circle (  2.13);

\path[fill=fillColor,fill opacity=0.20] (219.54, 64.46) circle (  2.13);

\path[fill=fillColor,fill opacity=0.20] (211.52, 65.09) circle (  2.13);

\path[fill=fillColor,fill opacity=0.20] (213.52, 50.97) circle (  2.13);

\path[fill=fillColor,fill opacity=0.20] (216.53, 58.96) circle (  2.13);

\path[fill=fillColor,fill opacity=0.20] (224.56, 77.86) circle (  2.13);

\path[fill=fillColor,fill opacity=0.20] (220.54, 83.77) circle (  2.13);

\path[fill=fillColor,fill opacity=0.20] (216.53, 76.19) circle (  2.13);

\path[fill=fillColor,fill opacity=0.20] (215.53, 60.93) circle (  2.13);

\path[fill=fillColor,fill opacity=0.20] (208.51, 40.48) circle (  2.13);

\path[fill=fillColor,fill opacity=0.20] (212.52, 38.51) circle (  2.13);

\path[fill=fillColor,fill opacity=0.20] (210.51, 41.52) circle (  2.13);

\path[fill=fillColor,fill opacity=0.20] (205.50, 44.74) circle (  2.13);

\path[fill=fillColor,fill opacity=0.20] (205.50, 53.04) circle (  2.13);

\path[fill=fillColor,fill opacity=0.20] (199.48, 55.22) circle (  2.13);

\path[fill=fillColor,fill opacity=0.20] (204.49, 52.21) circle (  2.13);

\path[fill=fillColor,fill opacity=0.20] (201.48, 50.86) circle (  2.13);

\path[fill=fillColor,fill opacity=0.20] (200.48, 52.42) circle (  2.13);

\path[fill=fillColor,fill opacity=0.20] (200.48, 53.04) circle (  2.13);

\path[fill=fillColor,fill opacity=0.20] (205.50, 53.98) circle (  2.13);

\path[fill=fillColor,fill opacity=0.20] (204.49, 58.03) circle (  2.13);

\path[fill=fillColor,fill opacity=0.20] (205.50, 59.38) circle (  2.13);

\path[fill=fillColor,fill opacity=0.20] (206.50, 55.64) circle (  2.13);

\path[fill=fillColor,fill opacity=0.20] (204.49, 52.73) circle (  2.13);

\path[fill=fillColor,fill opacity=0.20] (196.47, 56.57) circle (  2.13);

\path[fill=fillColor,fill opacity=0.20] (193.46, 56.26) circle (  2.13);

\path[fill=fillColor,fill opacity=0.20] (208.51, 43.91) circle (  2.13);

\path[fill=fillColor,fill opacity=0.20] (217.53, 44.43) circle (  2.13);

\path[fill=fillColor,fill opacity=0.20] (217.53, 75.99) circle (  2.13);

\path[fill=fillColor,fill opacity=0.20] (259.67, 68.82) circle (  2.13);

\path[fill=fillColor,fill opacity=0.20] (229.57, 70.59) circle (  2.13);

\path[fill=fillColor,fill opacity=0.20] (226.56, 66.95) circle (  2.13);

\path[fill=fillColor,fill opacity=0.20] (220.54, 57.61) circle (  2.13);

\path[fill=fillColor,fill opacity=0.20] (209.51, 56.16) circle (  2.13);

\path[fill=fillColor,fill opacity=0.20] (210.51, 56.16) circle (  2.13);

\path[fill=fillColor,fill opacity=0.20] (217.53, 51.28) circle (  2.13);

\path[fill=fillColor,fill opacity=0.20] (216.53, 48.06) circle (  2.13);

\path[fill=fillColor,fill opacity=0.20] (213.52, 47.85) circle (  2.13);

\path[fill=fillColor,fill opacity=0.20] (214.53, 54.18) circle (  2.13);

\path[fill=fillColor,fill opacity=0.20] (216.53, 61.56) circle (  2.13);

\path[fill=fillColor,fill opacity=0.20] (205.50, 57.09) circle (  2.13);

\path[fill=fillColor,fill opacity=0.20] (222.55, 55.12) circle (  2.13);

\path[fill=fillColor,fill opacity=0.20] (224.56, 61.35) circle (  2.13);

\path[fill=fillColor,fill opacity=0.20] (221.55, 58.55) circle (  2.13);

\path[fill=fillColor,fill opacity=0.20] (216.53, 52.52) circle (  2.13);

\path[fill=fillColor,fill opacity=0.20] (216.53, 60.52) circle (  2.13);

\path[fill=fillColor,fill opacity=0.20] (211.52, 93.12) circle (  2.13);

\path[fill=fillColor,fill opacity=0.20] (216.53, 78.17) circle (  2.13);

\path[fill=fillColor,fill opacity=0.20] (221.55, 67.27) circle (  2.13);

\path[fill=fillColor,fill opacity=0.20] (208.51, 72.35) circle (  2.13);

\path[fill=fillColor,fill opacity=0.20] (218.54, 71.32) circle (  2.13);

\path[fill=fillColor,fill opacity=0.20] (213.52, 55.02) circle (  2.13);

\path[fill=fillColor,fill opacity=0.20] (218.54, 49.51) circle (  2.13);

\path[fill=fillColor,fill opacity=0.20] (211.52, 57.40) circle (  2.13);

\path[fill=fillColor,fill opacity=0.20] (211.52, 56.47) circle (  2.13);

\path[fill=fillColor,fill opacity=0.20] (209.51, 54.39) circle (  2.13);

\path[fill=fillColor,fill opacity=0.20] (217.53, 63.11) circle (  2.13);

\path[fill=fillColor,fill opacity=0.20] (226.56, 66.12) circle (  2.13);

\path[fill=fillColor,fill opacity=0.20] (233.59, 75.99) circle (  2.13);

\path[fill=fillColor,fill opacity=0.20] (238.60,105.58) circle (  2.13);

\path[fill=fillColor,fill opacity=0.20] (217.53, 77.54) circle (  2.13);

\path[fill=fillColor,fill opacity=0.20] (212.52, 71.52) circle (  2.13);

\path[fill=fillColor,fill opacity=0.20] (211.52, 68.93) circle (  2.13);

\path[fill=fillColor,fill opacity=0.20] (207.50, 58.96) circle (  2.13);

\path[fill=fillColor,fill opacity=0.20] (195.46, 49.82) circle (  2.13);

\path[fill=fillColor,fill opacity=0.20] (206.50, 51.69) circle (  2.13);

\path[fill=fillColor,fill opacity=0.20] (200.48, 53.56) circle (  2.13);

\path[fill=fillColor,fill opacity=0.20] (201.48, 53.56) circle (  2.13);

\path[fill=fillColor,fill opacity=0.20] (203.49, 54.60) circle (  2.13);

\path[fill=fillColor,fill opacity=0.20] (204.49, 53.35) circle (  2.13);

\path[fill=fillColor,fill opacity=0.20] (204.49, 52.00) circle (  2.13);

\path[fill=fillColor,fill opacity=0.20] (202.49, 53.87) circle (  2.13);

\path[fill=fillColor,fill opacity=0.20] (202.49, 55.53) circle (  2.13);

\path[fill=fillColor,fill opacity=0.20] (197.47, 55.33) circle (  2.13);

\path[fill=fillColor,fill opacity=0.20] (204.49, 52.32) circle (  2.13);

\path[fill=fillColor,fill opacity=0.20] (200.48, 49.72) circle (  2.13);

\path[fill=fillColor,fill opacity=0.20] (208.51, 48.68) circle (  2.13);

\path[fill=fillColor,fill opacity=0.20] (221.55, 53.15) circle (  2.13);

\path[fill=fillColor,fill opacity=0.20] (230.58, 71.42) circle (  2.13);

\path[fill=fillColor,fill opacity=0.20] (222.55, 73.29) circle (  2.13);

\path[fill=fillColor,fill opacity=0.20] (217.53, 69.55) circle (  2.13);

\path[fill=fillColor,fill opacity=0.20] (221.55, 62.18) circle (  2.13);

\path[fill=fillColor,fill opacity=0.20] (222.55, 59.69) circle (  2.13);

\path[fill=fillColor,fill opacity=0.20] (216.53, 58.23) circle (  2.13);

\path[fill=fillColor,fill opacity=0.20] (209.51, 45.15) circle (  2.13);

\path[fill=fillColor,fill opacity=0.20] (217.53, 42.87) circle (  2.13);

\path[fill=fillColor,fill opacity=0.20] (218.54, 48.79) circle (  2.13);

\path[fill=fillColor,fill opacity=0.20] (218.54, 58.34) circle (  2.13);

\path[fill=fillColor,fill opacity=0.20] (212.52, 68.10) circle (  2.13);

\path[fill=fillColor,fill opacity=0.20] (214.53, 63.01) circle (  2.13);

\path[fill=fillColor,fill opacity=0.20] (224.56, 51.17) circle (  2.13);

\path[fill=fillColor,fill opacity=0.20] (214.53, 51.49) circle (  2.13);

\path[fill=fillColor,fill opacity=0.20] (218.54, 58.65) circle (  2.13);

\path[fill=fillColor,fill opacity=0.20] (222.55, 66.85) circle (  2.13);

\path[fill=fillColor,fill opacity=0.20] (224.56, 72.66) circle (  2.13);

\path[fill=fillColor,fill opacity=0.20] (224.56, 70.17) circle (  2.13);

\path[fill=fillColor,fill opacity=0.20] (221.55, 64.46) circle (  2.13);

\path[fill=fillColor,fill opacity=0.20] (220.54, 63.01) circle (  2.13);

\path[fill=fillColor,fill opacity=0.20] (218.54, 67.99) circle (  2.13);

\path[fill=fillColor,fill opacity=0.20] (218.54, 77.23) circle (  2.13);

\path[fill=fillColor,fill opacity=0.20] (220.54, 86.89) circle (  2.13);

\path[fill=fillColor,fill opacity=0.20] (220.54, 77.86) circle (  2.13);

\path[fill=fillColor,fill opacity=0.20] (217.53, 68.82) circle (  2.13);

\path[fill=fillColor,fill opacity=0.20] (210.51, 66.12) circle (  2.13);

\path[fill=fillColor,fill opacity=0.20] (214.53, 63.84) circle (  2.13);

\path[fill=fillColor,fill opacity=0.20] (210.51, 60.73) circle (  2.13);

\path[fill=fillColor,fill opacity=0.20] (209.51, 58.75) circle (  2.13);

\path[fill=fillColor,fill opacity=0.20] (215.53, 53.04) circle (  2.13);

\path[fill=fillColor,fill opacity=0.20] (212.52, 51.69) circle (  2.13);

\path[fill=fillColor,fill opacity=0.20] (210.51, 61.97) circle (  2.13);

\path[fill=fillColor,fill opacity=0.20] (211.52, 64.88) circle (  2.13);

\path[fill=fillColor,fill opacity=0.20] (211.52, 52.63) circle (  2.13);

\path[fill=fillColor,fill opacity=0.20] (215.53, 47.96) circle (  2.13);

\path[fill=fillColor,fill opacity=0.20] (214.53, 57.51) circle (  2.13);

\path[fill=fillColor,fill opacity=0.20] (206.50, 64.26) circle (  2.13);

\path[fill=fillColor,fill opacity=0.20] (218.54, 74.53) circle (  2.13);

\path[fill=fillColor,fill opacity=0.20] (219.54, 93.12) circle (  2.13);

\path[fill=fillColor,fill opacity=0.20] (219.54,107.65) circle (  2.13);

\path[fill=fillColor,fill opacity=0.20] (217.53, 97.27) circle (  2.13);

\path[fill=fillColor,fill opacity=0.20] (206.50, 85.85) circle (  2.13);

\path[fill=fillColor,fill opacity=0.20] (200.48, 70.59) circle (  2.13);

\path[fill=fillColor,fill opacity=0.20] (202.49, 61.97) circle (  2.13);

\path[fill=fillColor,fill opacity=0.20] (207.50, 60.83) circle (  2.13);

\path[fill=fillColor,fill opacity=0.20] (206.50, 60.52) circle (  2.13);

\path[fill=fillColor,fill opacity=0.20] (201.48, 60.62) circle (  2.13);

\path[fill=fillColor,fill opacity=0.20] (203.49, 59.48) circle (  2.13);

\path[fill=fillColor,fill opacity=0.20] (207.50, 57.82) circle (  2.13);

\path[fill=fillColor,fill opacity=0.20] (207.50, 63.32) circle (  2.13);

\path[fill=fillColor,fill opacity=0.20] (208.51, 66.54) circle (  2.13);

\path[fill=fillColor,fill opacity=0.20] (212.52, 62.70) circle (  2.13);

\path[fill=fillColor,fill opacity=0.20] (215.53, 71.32) circle (  2.13);

\path[fill=fillColor,fill opacity=0.20] (229.57, 88.96) circle (  2.13);

\path[fill=fillColor,fill opacity=0.20] (224.56, 68.72) circle (  2.13);

\path[fill=fillColor,fill opacity=0.20] (230.58, 46.71) circle (  2.13);

\path[fill=fillColor,fill opacity=0.20] (222.55, 42.66) circle (  2.13);

\path[fill=fillColor,fill opacity=0.20] (216.53, 48.68) circle (  2.13);

\path[fill=fillColor,fill opacity=0.20] (213.52, 54.50) circle (  2.13);

\path[fill=fillColor,fill opacity=0.20] (213.52, 56.88) circle (  2.13);

\path[fill=fillColor,fill opacity=0.20] (221.55, 55.53) circle (  2.13);

\path[fill=fillColor,fill opacity=0.20] (214.53, 53.77) circle (  2.13);

\path[fill=fillColor,fill opacity=0.20] (219.54, 53.15) circle (  2.13);

\path[fill=fillColor,fill opacity=0.20] (217.53, 55.43) circle (  2.13);

\path[fill=fillColor,fill opacity=0.20] (222.55, 63.53) circle (  2.13);

\path[fill=fillColor,fill opacity=0.20] (220.54, 68.72) circle (  2.13);

\path[fill=fillColor,fill opacity=0.20] (216.53, 65.40) circle (  2.13);

\path[fill=fillColor,fill opacity=0.20] (216.53, 61.87) circle (  2.13);

\path[fill=fillColor,fill opacity=0.20] (214.53, 58.13) circle (  2.13);

\path[fill=fillColor,fill opacity=0.20] (219.54, 53.35) circle (  2.13);

\path[fill=fillColor,fill opacity=0.20] (219.54, 54.81) circle (  2.13);

\path[fill=fillColor,fill opacity=0.20] (214.53, 56.99) circle (  2.13);

\path[fill=fillColor,fill opacity=0.20] (216.53, 58.03) circle (  2.13);

\path[fill=fillColor,fill opacity=0.20] (213.52, 63.32) circle (  2.13);

\path[fill=fillColor,fill opacity=0.20] (212.52, 67.99) circle (  2.13);

\path[fill=fillColor,fill opacity=0.20] (197.47, 68.41) circle (  2.13);

\path[fill=fillColor,fill opacity=0.20] (208.51, 63.01) circle (  2.13);

\path[fill=fillColor,fill opacity=0.20] (212.52, 56.78) circle (  2.13);

\path[fill=fillColor,fill opacity=0.20] (208.51, 57.92) circle (  2.13);

\path[fill=fillColor,fill opacity=0.20] (207.50, 61.04) circle (  2.13);

\path[fill=fillColor,fill opacity=0.20] (207.50, 54.08) circle (  2.13);

\path[fill=fillColor,fill opacity=0.20] (207.50, 47.64) circle (  2.13);

\path[fill=fillColor,fill opacity=0.20] (210.51, 56.57) circle (  2.13);

\path[fill=fillColor,fill opacity=0.20] (210.51, 69.76) circle (  2.13);

\path[fill=fillColor,fill opacity=0.20] (216.53, 76.51) circle (  2.13);

\path[fill=fillColor,fill opacity=0.20] (226.56, 82.74) circle (  2.13);

\path[fill=fillColor,fill opacity=0.20] (231.58,112.84) circle (  2.13);

\path[fill=fillColor,fill opacity=0.20] (212.52, 94.16) circle (  2.13);

\path[fill=fillColor,fill opacity=0.20] (214.53, 88.96) circle (  2.13);

\path[fill=fillColor,fill opacity=0.20] (212.52, 81.39) circle (  2.13);

\path[fill=fillColor,fill opacity=0.20] (206.50, 79.00) circle (  2.13);

\path[fill=fillColor,fill opacity=0.20] (205.50, 77.44) circle (  2.13);

\path[fill=fillColor,fill opacity=0.20] (207.50, 75.78) circle (  2.13);

\path[fill=fillColor,fill opacity=0.20] (210.51, 83.77) circle (  2.13);

\path[fill=fillColor,fill opacity=0.20] (232.58, 82.74) circle (  2.13);

\path[fill=fillColor,fill opacity=0.20] (220.54, 64.26) circle (  2.13);

\path[fill=fillColor,fill opacity=0.20] (217.53, 50.55) circle (  2.13);

\path[fill=fillColor,fill opacity=0.20] (214.53, 47.23) circle (  2.13);

\path[fill=fillColor,fill opacity=0.20] (209.51, 46.19) circle (  2.13);

\path[fill=fillColor,fill opacity=0.20] (212.52, 50.34) circle (  2.13);

\path[fill=fillColor,fill opacity=0.20] (215.53, 57.51) circle (  2.13);

\path[fill=fillColor,fill opacity=0.20] (219.54, 58.03) circle (  2.13);

\path[fill=fillColor,fill opacity=0.20] (217.53, 57.40) circle (  2.13);

\path[fill=fillColor,fill opacity=0.20] (214.53, 60.52) circle (  2.13);

\path[fill=fillColor,fill opacity=0.20] (210.51, 61.45) circle (  2.13);

\path[fill=fillColor,fill opacity=0.20] (207.50, 64.15) circle (  2.13);

\path[fill=fillColor,fill opacity=0.20] (214.53, 68.20) circle (  2.13);

\path[fill=fillColor,fill opacity=0.20] (217.53, 66.02) circle (  2.13);

\path[fill=fillColor,fill opacity=0.20] (214.53, 61.97) circle (  2.13);

\path[fill=fillColor,fill opacity=0.20] (212.52, 60.83) circle (  2.13);

\path[fill=fillColor,fill opacity=0.20] (209.51, 60.21) circle (  2.13);

\path[fill=fillColor,fill opacity=0.20] (211.52, 60.52) circle (  2.13);

\path[fill=fillColor,fill opacity=0.20] (204.49, 64.46) circle (  2.13);

\path[fill=fillColor,fill opacity=0.20] (201.48, 71.21) circle (  2.13);

\path[fill=fillColor,fill opacity=0.20] (204.49, 69.24) circle (  2.13);

\path[fill=fillColor,fill opacity=0.20] (205.50, 57.30) circle (  2.13);

\path[fill=fillColor,fill opacity=0.20] (204.49, 52.21) circle (  2.13);

\path[fill=fillColor,fill opacity=0.20] (204.49, 57.82) circle (  2.13);

\path[fill=fillColor,fill opacity=0.20] (206.50, 59.79) circle (  2.13);

\path[fill=fillColor,fill opacity=0.20] (215.53, 58.96) circle (  2.13);

\path[fill=fillColor,fill opacity=0.20] (217.53, 69.76) circle (  2.13);

\path[fill=fillColor,fill opacity=0.20] (220.54, 87.93) circle (  2.13);

\path[fill=fillColor,fill opacity=0.20] (232.58, 73.50) circle (  2.13);

\path[fill=fillColor,fill opacity=0.20] (207.50, 62.18) circle (  2.13);

\path[fill=fillColor,fill opacity=0.20] (216.53, 52.73) circle (  2.13);

\path[fill=fillColor,fill opacity=0.20] (218.54, 57.09) circle (  2.13);

\path[fill=fillColor,fill opacity=0.20] (219.54, 57.61) circle (  2.13);

\path[fill=fillColor,fill opacity=0.20] (215.53, 48.79) circle (  2.13);

\path[fill=fillColor,fill opacity=0.20] (213.52, 48.99) circle (  2.13);

\path[fill=fillColor,fill opacity=0.20] (210.51, 58.65) circle (  2.13);

\path[fill=fillColor,fill opacity=0.20] (211.52, 62.49) circle (  2.13);

\path[fill=fillColor,fill opacity=0.20] (207.50, 62.28) circle (  2.13);

\path[fill=fillColor,fill opacity=0.20] (209.51, 64.26) circle (  2.13);

\path[fill=fillColor,fill opacity=0.20] (209.51, 64.26) circle (  2.13);

\path[fill=fillColor,fill opacity=0.20] (204.49, 60.31) circle (  2.13);

\path[fill=fillColor,fill opacity=0.20] (205.50, 57.92) circle (  2.13);

\path[fill=fillColor,fill opacity=0.20] (206.50, 57.30) circle (  2.13);

\path[fill=fillColor,fill opacity=0.20] (199.48, 53.98) circle (  2.13);

\path[fill=fillColor,fill opacity=0.20] (198.47, 55.74) circle (  2.13);

\path[fill=fillColor,fill opacity=0.20] (208.51, 60.21) circle (  2.13);

\path[fill=fillColor,fill opacity=0.20] (209.51, 56.16) circle (  2.13);

\path[fill=fillColor,fill opacity=0.20] (208.51, 50.24) circle (  2.13);

\path[fill=fillColor,fill opacity=0.20] (208.51, 54.91) circle (  2.13);

\path[fill=fillColor,fill opacity=0.20] (207.50, 66.95) circle (  2.13);

\path[fill=fillColor,fill opacity=0.20] (222.55, 65.40) circle (  2.13);

\path[fill=fillColor,fill opacity=0.20] (220.54, 61.66) circle (  2.13);

\path[fill=fillColor,fill opacity=0.20] (212.52, 58.96) circle (  2.13);

\path[fill=fillColor,fill opacity=0.20] (196.47, 64.88) circle (  2.13);

\path[fill=fillColor,fill opacity=0.20] (218.54, 61.66) circle (  2.13);

\path[fill=fillColor,fill opacity=0.20] (211.52, 46.71) circle (  2.13);

\path[fill=fillColor,fill opacity=0.20] (205.50, 43.39) circle (  2.13);

\path[fill=fillColor,fill opacity=0.20] (207.50, 52.21) circle (  2.13);

\path[fill=fillColor,fill opacity=0.20] (205.50, 52.42) circle (  2.13);

\path[fill=fillColor,fill opacity=0.20] (205.50, 51.38) circle (  2.13);

\path[fill=fillColor,fill opacity=0.20] (207.50, 57.61) circle (  2.13);

\path[fill=fillColor,fill opacity=0.20] (197.47, 58.86) circle (  2.13);

\path[fill=fillColor,fill opacity=0.20] (211.52, 53.56) circle (  2.13);

\path[fill=fillColor,fill opacity=0.20] (215.53, 54.39) circle (  2.13);

\path[fill=fillColor,fill opacity=0.20] (208.51, 60.10) circle (  2.13);

\path[fill=fillColor,fill opacity=0.20] (220.54, 64.15) circle (  2.13);

\path[fill=fillColor,fill opacity=0.20] (230.58, 67.37) circle (  2.13);

\path[fill=fillColor,fill opacity=0.20] (211.52, 58.03) circle (  2.13);

\path[fill=fillColor,fill opacity=0.20] (210.51, 47.75) circle (  2.13);

\path[fill=fillColor,fill opacity=0.20] (215.53, 50.65) circle (  2.13);

\path[fill=fillColor,fill opacity=0.20] (213.52, 58.44) circle (  2.13);

\path[fill=fillColor,fill opacity=0.20] (211.52, 64.46) circle (  2.13);

\path[fill=fillColor,fill opacity=0.20] (218.54, 70.90) circle (  2.13);

\path[fill=fillColor,fill opacity=0.20] (220.54, 77.44) circle (  2.13);

\path[fill=fillColor,fill opacity=0.20] (224.56, 76.09) circle (  2.13);

\path[fill=fillColor,fill opacity=0.20] (237.60, 69.65) circle (  2.13);

\path[fill=fillColor,fill opacity=0.20] (187.54,108.69) circle (  2.13);

\path[fill=fillColor,fill opacity=0.20] (204.49,112.84) circle (  2.13);

\path[fill=fillColor,fill opacity=0.20] (203.49, 64.26) circle (  2.13);

\path[fill=fillColor,fill opacity=0.20] (201.48, 63.11) circle (  2.13);

\path[fill=fillColor,fill opacity=0.20] (200.48, 72.15) circle (  2.13);

\path[fill=fillColor,fill opacity=0.20] (201.48, 82.74) circle (  2.13);

\path[fill=fillColor,fill opacity=0.20] (203.49, 79.72) circle (  2.13);

\path[fill=fillColor,fill opacity=0.20] (242.62, 90.00) circle (  2.13);

\path[fill=fillColor,fill opacity=0.20] (245.63, 69.76) circle (  2.13);

\path[fill=fillColor,fill opacity=0.20] (241.61, 58.96) circle (  2.13);

\path[fill=fillColor,fill opacity=0.20] (237.60, 69.24) circle (  2.13);

\path[fill=fillColor,fill opacity=0.20] (238.60, 63.94) circle (  2.13);

\path[fill=fillColor,fill opacity=0.20] (241.61, 50.97) circle (  2.13);

\path[fill=fillColor,fill opacity=0.20] (208.51, 53.87) circle (  2.13);

\path[fill=fillColor,fill opacity=0.20] (203.49, 52.42) circle (  2.13);

\path[fill=fillColor,fill opacity=0.20] (200.48, 66.75) circle (  2.13);

\path[fill=fillColor,fill opacity=0.20] (196.47, 70.59) circle (  2.13);

\path[fill=fillColor,fill opacity=0.20] (196.47, 71.00) circle (  2.13);

\path[fill=fillColor,fill opacity=0.20] (199.48, 76.51) circle (  2.13);

\path[fill=fillColor,fill opacity=0.20] (202.49, 85.85) circle (  2.13);

\path[fill=fillColor,fill opacity=0.20] (205.50, 96.23) circle (  2.13);

\path[fill=fillColor,fill opacity=0.20] (240.61, 92.08) circle (  2.13);

\path[fill=fillColor,fill opacity=0.20] (235.59, 78.89) circle (  2.13);

\path[fill=fillColor,fill opacity=0.20] (239.61, 81.70) circle (  2.13);

\path[fill=fillColor,fill opacity=0.20] (230.58, 68.20) circle (  2.13);

\path[fill=fillColor,fill opacity=0.20] (220.54, 56.47) circle (  2.13);

\path[fill=fillColor,fill opacity=0.20] (217.53, 62.28) circle (  2.13);

\path[fill=fillColor,fill opacity=0.20] (221.55, 60.10) circle (  2.13);

\path[fill=fillColor,fill opacity=0.20] (224.56, 55.02) circle (  2.13);

\path[fill=fillColor,fill opacity=0.20] (228.57, 54.18) circle (  2.13);

\path[fill=fillColor,fill opacity=0.20] (232.58, 59.06) circle (  2.13);

\path[fill=fillColor,fill opacity=0.20] (208.51, 74.85) circle (  2.13);

\path[fill=fillColor,fill opacity=0.20] (201.48, 47.12) circle (  2.13);

\path[fill=fillColor,fill opacity=0.20] (198.47, 71.52) circle (  2.13);

\path[fill=fillColor,fill opacity=0.20] (194.46, 68.41) circle (  2.13);

\path[fill=fillColor,fill opacity=0.20] (191.45, 63.22) circle (  2.13);

\path[fill=fillColor,fill opacity=0.20] (193.46, 58.34) circle (  2.13);

\path[fill=fillColor,fill opacity=0.20] (195.46, 63.11) circle (  2.13);

\path[fill=fillColor,fill opacity=0.20] (200.48, 68.51) circle (  2.13);

\path[fill=fillColor,fill opacity=0.20] (206.50, 68.41) circle (  2.13);

\path[fill=fillColor,fill opacity=0.20] (210.51, 85.85) circle (  2.13);

\path[fill=fillColor,fill opacity=0.20] (202.49, 81.49) circle (  2.13);

\path[fill=fillColor,fill opacity=0.20] (228.57, 73.81) circle (  2.13);

\path[fill=fillColor,fill opacity=0.20] (218.54, 71.83) circle (  2.13);

\path[fill=fillColor,fill opacity=0.20] (223.55, 60.10) circle (  2.13);

\path[fill=fillColor,fill opacity=0.20] (218.54, 52.11) circle (  2.13);

\path[fill=fillColor,fill opacity=0.20] (214.53, 65.19) circle (  2.13);

\path[fill=fillColor,fill opacity=0.20] (212.52, 75.05) circle (  2.13);

\path[fill=fillColor,fill opacity=0.20] (213.52, 61.04) circle (  2.13);

\path[fill=fillColor,fill opacity=0.20] (218.54, 55.22) circle (  2.13);

\path[fill=fillColor,fill opacity=0.20] (214.53, 65.40) circle (  2.13);

\path[fill=fillColor,fill opacity=0.20] (205.50, 49.82) circle (  2.13);

\path[fill=fillColor,fill opacity=0.20] (197.47, 62.91) circle (  2.13);

\path[fill=fillColor,fill opacity=0.20] (195.46, 82.74) circle (  2.13);

\path[fill=fillColor,fill opacity=0.20] (191.45, 71.11) circle (  2.13);

\path[fill=fillColor,fill opacity=0.20] (192.45, 56.68) circle (  2.13);

\path[fill=fillColor,fill opacity=0.20] (193.46, 46.92) circle (  2.13);

\path[fill=fillColor,fill opacity=0.20] (194.46, 46.50) circle (  2.13);

\path[fill=fillColor,fill opacity=0.20] (199.48, 43.28) circle (  2.13);

\path[fill=fillColor,fill opacity=0.20] (203.49, 40.48) circle (  2.13);

\path[fill=fillColor,fill opacity=0.20] (207.50, 59.48) circle (  2.13);

\path[fill=fillColor,fill opacity=0.20] (241.61, 79.93) circle (  2.13);

\path[fill=fillColor,fill opacity=0.20] (233.59, 49.72) circle (  2.13);

\path[fill=fillColor,fill opacity=0.20] (227.57, 70.69) circle (  2.13);

\path[fill=fillColor,fill opacity=0.20] (223.55, 84.81) circle (  2.13);

\path[fill=fillColor,fill opacity=0.20] (220.54, 69.34) circle (  2.13);

\path[fill=fillColor,fill opacity=0.20] (214.53, 63.74) circle (  2.13);

\path[fill=fillColor,fill opacity=0.20] (210.51, 67.89) circle (  2.13);

\path[fill=fillColor,fill opacity=0.20] (211.52, 70.59) circle (  2.13);

\path[fill=fillColor,fill opacity=0.20] (213.52, 66.33) circle (  2.13);

\path[fill=fillColor,fill opacity=0.20] (218.54, 55.43) circle (  2.13);

\path[fill=fillColor,fill opacity=0.20] (228.57, 55.22) circle (  2.13);

\path[fill=fillColor,fill opacity=0.20] (234.59, 68.10) circle (  2.13);

\path[fill=fillColor,fill opacity=0.20] (204.49, 67.47) circle (  2.13);

\path[fill=fillColor,fill opacity=0.20] (197.47, 55.33) circle (  2.13);

\path[fill=fillColor,fill opacity=0.20] (195.46, 75.26) circle (  2.13);

\path[fill=fillColor,fill opacity=0.20] (192.45, 76.51) circle (  2.13);

\path[fill=fillColor,fill opacity=0.20] (192.45, 55.64) circle (  2.13);

\path[fill=fillColor,fill opacity=0.20] (191.45, 44.11) circle (  2.13);

\path[fill=fillColor,fill opacity=0.20] (193.46, 47.54) circle (  2.13);

\path[fill=fillColor,fill opacity=0.20] (197.47, 41.52) circle (  2.13);

\path[fill=fillColor,fill opacity=0.20] (202.49, 42.04) circle (  2.13);

\path[fill=fillColor,fill opacity=0.20] (207.50, 67.16) circle (  2.13);

\path[fill=fillColor,fill opacity=0.20] (256.66, 72.66) circle (  2.13);

\path[fill=fillColor,fill opacity=0.20] (223.55, 57.30) circle (  2.13);

\path[fill=fillColor,fill opacity=0.20] (215.53, 49.93) circle (  2.13);

\path[fill=fillColor,fill opacity=0.20] (219.54, 66.12) circle (  2.13);

\path[fill=fillColor,fill opacity=0.20] (217.53, 75.78) circle (  2.13);

\path[fill=fillColor,fill opacity=0.20] (213.52, 71.83) circle (  2.13);

\path[fill=fillColor,fill opacity=0.20] (209.51, 69.97) circle (  2.13);

\path[fill=fillColor,fill opacity=0.20] (208.51, 62.08) circle (  2.13);

\path[fill=fillColor,fill opacity=0.20] (208.51, 58.65) circle (  2.13);

\path[fill=fillColor,fill opacity=0.20] (206.50, 69.86) circle (  2.13);

\path[fill=fillColor,fill opacity=0.20] (214.53, 61.66) circle (  2.13);

\path[fill=fillColor,fill opacity=0.20] (233.59, 56.05) circle (  2.13);

\path[fill=fillColor,fill opacity=0.20] (205.50, 58.23) circle (  2.13);

\path[fill=fillColor,fill opacity=0.20] (199.48, 45.36) circle (  2.13);

\path[fill=fillColor,fill opacity=0.20] (197.47, 61.76) circle (  2.13);

\path[fill=fillColor,fill opacity=0.20] (190.45, 65.40) circle (  2.13);

\path[fill=fillColor,fill opacity=0.20] (190.45, 53.35) circle (  2.13);

\path[fill=fillColor,fill opacity=0.20] (193.46, 43.60) circle (  2.13);

\path[fill=fillColor,fill opacity=0.20] (193.46, 60.52) circle (  2.13);

\path[fill=fillColor,fill opacity=0.20] (194.46, 65.71) circle (  2.13);

\path[fill=fillColor,fill opacity=0.20] (196.47, 51.28) circle (  2.13);

\path[fill=fillColor,fill opacity=0.20] (201.48, 43.08) circle (  2.13);

\path[fill=fillColor,fill opacity=0.20] (206.50, 54.18) circle (  2.13);

\path[fill=fillColor,fill opacity=0.20] (210.51, 82.74) circle (  2.13);

\path[fill=fillColor,fill opacity=0.20] (240.61, 55.02) circle (  2.13);

\path[fill=fillColor,fill opacity=0.20] (220.54, 40.48) circle (  2.13);

\path[fill=fillColor,fill opacity=0.20] (215.53, 53.35) circle (  2.13);

\path[fill=fillColor,fill opacity=0.20] (208.51, 46.40) circle (  2.13);

\path[fill=fillColor,fill opacity=0.20] (206.50, 48.89) circle (  2.13);

\path[fill=fillColor,fill opacity=0.20] (202.49, 64.05) circle (  2.13);

\path[fill=fillColor,fill opacity=0.20] (198.47, 65.50) circle (  2.13);

\path[fill=fillColor,fill opacity=0.20] (202.49, 57.51) circle (  2.13);

\path[fill=fillColor,fill opacity=0.20] (203.49, 55.74) circle (  2.13);

\path[fill=fillColor,fill opacity=0.20] (202.49, 63.53) circle (  2.13);

\path[fill=fillColor,fill opacity=0.20] (210.51, 71.73) circle (  2.13);

\path[fill=fillColor,fill opacity=0.20] (200.48, 48.89) circle (  2.13);

\path[fill=fillColor,fill opacity=0.20] (199.48, 57.51) circle (  2.13);

\path[fill=fillColor,fill opacity=0.20] (191.45, 56.68) circle (  2.13);

\path[fill=fillColor,fill opacity=0.20] (191.45, 54.60) circle (  2.13);

\path[fill=fillColor,fill opacity=0.20] (195.46, 50.14) circle (  2.13);

\path[fill=fillColor,fill opacity=0.20] (195.46, 64.77) circle (  2.13);

\path[fill=fillColor,fill opacity=0.20] (195.46, 73.91) circle (  2.13);

\path[fill=fillColor,fill opacity=0.20] (197.47, 55.95) circle (  2.13);

\path[fill=fillColor,fill opacity=0.20] (202.49, 41.52) circle (  2.13);

\path[fill=fillColor,fill opacity=0.20] (208.51, 49.51) circle (  2.13);

\path[fill=fillColor,fill opacity=0.20] (211.52, 77.75) circle (  2.13);

\path[fill=fillColor,fill opacity=0.20] (248.64, 72.87) circle (  2.13);

\path[fill=fillColor,fill opacity=0.20] (222.55, 62.08) circle (  2.13);

\path[fill=fillColor,fill opacity=0.20] (214.53, 41.52) circle (  2.13);

\path[fill=fillColor,fill opacity=0.20] (211.52, 44.63) circle (  2.13);

\path[fill=fillColor,fill opacity=0.20] (205.50, 40.07) circle (  2.13);

\path[fill=fillColor,fill opacity=0.20] (194.46, 45.46) circle (  2.13);

\path[fill=fillColor,fill opacity=0.20] (194.46, 66.75) circle (  2.13);

\path[fill=fillColor,fill opacity=0.20] (196.47, 61.45) circle (  2.13);

\path[fill=fillColor,fill opacity=0.20] (198.47, 47.85) circle (  2.13);

\path[fill=fillColor,fill opacity=0.20] (199.48, 56.26) circle (  2.13);

\path[fill=fillColor,fill opacity=0.20] (202.49, 66.23) circle (  2.13);

\path[fill=fillColor,fill opacity=0.20] (214.53, 87.93) circle (  2.13);

\path[fill=fillColor,fill opacity=0.20] (214.53, 43.18) circle (  2.13);

\path[fill=fillColor,fill opacity=0.20] (204.49, 46.29) circle (  2.13);

\path[fill=fillColor,fill opacity=0.20] (200.48, 59.79) circle (  2.13);

\path[fill=fillColor,fill opacity=0.20] (199.48, 62.49) circle (  2.13);

\path[fill=fillColor,fill opacity=0.20] (197.47, 65.71) circle (  2.13);

\path[fill=fillColor,fill opacity=0.20] (197.47, 56.05) circle (  2.13);

\path[fill=fillColor,fill opacity=0.20] (195.46, 49.41) circle (  2.13);

\path[fill=fillColor,fill opacity=0.20] (193.46, 58.13) circle (  2.13);

\path[fill=fillColor,fill opacity=0.20] (197.47, 59.48) circle (  2.13);

\path[fill=fillColor,fill opacity=0.20] (200.48, 46.81) circle (  2.13);

\path[fill=fillColor,fill opacity=0.20] (203.49, 44.22) circle (  2.13);

\path[fill=fillColor,fill opacity=0.20] (210.51, 51.90) circle (  2.13);

\path[fill=fillColor,fill opacity=0.20] (246.63, 69.45) circle (  2.13);

\path[fill=fillColor,fill opacity=0.20] (218.54, 63.74) circle (  2.13);

\path[fill=fillColor,fill opacity=0.20] (210.51, 51.38) circle (  2.13);

\path[fill=fillColor,fill opacity=0.20] (210.51, 51.07) circle (  2.13);

\path[fill=fillColor,fill opacity=0.20] (205.50, 55.33) circle (  2.13);

\path[fill=fillColor,fill opacity=0.20] (203.49, 65.09) circle (  2.13);

\path[fill=fillColor,fill opacity=0.20] (200.48, 75.16) circle (  2.13);

\path[fill=fillColor,fill opacity=0.20] (199.48, 71.21) circle (  2.13);

\path[fill=fillColor,fill opacity=0.20] (200.48, 51.80) circle (  2.13);

\path[fill=fillColor,fill opacity=0.20] (205.50, 54.29) circle (  2.13);

\path[fill=fillColor,fill opacity=0.20] (212.52, 73.91) circle (  2.13);

\path[fill=fillColor,fill opacity=0.20] (222.55, 64.98) circle (  2.13);

\path[fill=fillColor,fill opacity=0.20] (210.51, 54.29) circle (  2.13);

\path[fill=fillColor,fill opacity=0.20] (202.49, 61.66) circle (  2.13);

\path[fill=fillColor,fill opacity=0.20] (199.48, 60.21) circle (  2.13);

\path[fill=fillColor,fill opacity=0.20] (199.48, 64.98) circle (  2.13);

\path[fill=fillColor,fill opacity=0.20] (200.48, 65.61) circle (  2.13);

\path[fill=fillColor,fill opacity=0.20] (199.48, 56.36) circle (  2.13);

\path[fill=fillColor,fill opacity=0.20] (196.47, 49.51) circle (  2.13);

\path[fill=fillColor,fill opacity=0.20] (195.46, 51.49) circle (  2.13);

\path[fill=fillColor,fill opacity=0.20] (200.48, 51.69) circle (  2.13);

\path[fill=fillColor,fill opacity=0.20] (201.48, 56.68) circle (  2.13);

\path[fill=fillColor,fill opacity=0.20] (207.50, 57.71) circle (  2.13);

\path[fill=fillColor,fill opacity=0.20] (214.53, 58.75) circle (  2.13);

\path[fill=fillColor,fill opacity=0.20] (240.61, 64.05) circle (  2.13);

\path[fill=fillColor,fill opacity=0.20] (218.54, 56.47) circle (  2.13);

\path[fill=fillColor,fill opacity=0.20] (212.52, 58.03) circle (  2.13);

\path[fill=fillColor,fill opacity=0.20] (206.50, 58.96) circle (  2.13);

\path[fill=fillColor,fill opacity=0.20] (203.49, 58.13) circle (  2.13);

\path[fill=fillColor,fill opacity=0.20] (201.48, 66.75) circle (  2.13);

\path[fill=fillColor,fill opacity=0.20] (203.49, 71.83) circle (  2.13);

\path[fill=fillColor,fill opacity=0.20] (204.49, 71.52) circle (  2.13);

\path[fill=fillColor,fill opacity=0.20] (206.50, 66.23) circle (  2.13);

\path[fill=fillColor,fill opacity=0.20] (212.52, 59.38) circle (  2.13);

\path[fill=fillColor,fill opacity=0.20] (214.53, 65.09) circle (  2.13);

\path[fill=fillColor,fill opacity=0.20] (208.51, 67.37) circle (  2.13);

\path[fill=fillColor,fill opacity=0.20] (202.49, 51.59) circle (  2.13);

\path[fill=fillColor,fill opacity=0.20] (199.48, 48.16) circle (  2.13);

\path[fill=fillColor,fill opacity=0.20] (198.47, 59.06) circle (  2.13);

\path[fill=fillColor,fill opacity=0.20] (194.46, 56.68) circle (  2.13);

\path[fill=fillColor,fill opacity=0.20] (199.48, 52.84) circle (  2.13);

\path[fill=fillColor,fill opacity=0.20] (195.46, 56.16) circle (  2.13);

\path[fill=fillColor,fill opacity=0.20] (196.47, 55.22) circle (  2.13);

\path[fill=fillColor,fill opacity=0.20] (200.48, 61.04) circle (  2.13);

\path[fill=fillColor,fill opacity=0.20] (204.49, 72.56) circle (  2.13);

\path[fill=fillColor,fill opacity=0.20] (209.51, 65.09) circle (  2.13);

\path[fill=fillColor,fill opacity=0.20] (216.53, 70.69) circle (  2.13);

\path[fill=fillColor,fill opacity=0.20] (215.53, 53.77) circle (  2.13);

\path[fill=fillColor,fill opacity=0.20] (206.50, 52.52) circle (  2.13);

\path[fill=fillColor,fill opacity=0.20] (201.48, 61.24) circle (  2.13);

\path[fill=fillColor,fill opacity=0.20] (199.48, 64.57) circle (  2.13);

\path[fill=fillColor,fill opacity=0.20] (200.48, 67.68) circle (  2.13);

\path[fill=fillColor,fill opacity=0.20] (203.49, 64.77) circle (  2.13);

\path[fill=fillColor,fill opacity=0.20] (205.50, 60.21) circle (  2.13);

\path[fill=fillColor,fill opacity=0.20] (207.50, 69.55) circle (  2.13);

\path[fill=fillColor,fill opacity=0.20] (215.53, 70.59) circle (  2.13);

\path[fill=fillColor,fill opacity=0.20] (198.47, 64.36) circle (  2.13);

\path[fill=fillColor,fill opacity=0.20] (212.52, 66.33) circle (  2.13);

\path[fill=fillColor,fill opacity=0.20] (204.49, 61.56) circle (  2.13);

\path[fill=fillColor,fill opacity=0.20] (200.48, 44.32) circle (  2.13);

\path[fill=fillColor,fill opacity=0.20] (197.47, 38.40) circle (  2.13);

\path[fill=fillColor,fill opacity=0.20] (194.46, 54.18) circle (  2.13);

\path[fill=fillColor,fill opacity=0.20] (194.46, 51.59) circle (  2.13);

\path[fill=fillColor,fill opacity=0.20] (195.46, 48.99) circle (  2.13);

\path[fill=fillColor,fill opacity=0.20] (195.46, 62.80) circle (  2.13);

\path[fill=fillColor,fill opacity=0.20] (198.47, 66.95) circle (  2.13);

\path[fill=fillColor,fill opacity=0.20] (200.48, 64.26) circle (  2.13);

\path[fill=fillColor,fill opacity=0.20] (205.50, 63.32) circle (  2.13);

\path[fill=fillColor,fill opacity=0.20] (213.52, 61.66) circle (  2.13);

\path[fill=fillColor,fill opacity=0.20] (217.53, 85.85) circle (  2.13);

\path[fill=fillColor,fill opacity=0.20] (223.55, 57.20) circle (  2.13);

\path[fill=fillColor,fill opacity=0.20] (208.51, 47.23) circle (  2.13);

\path[fill=fillColor,fill opacity=0.20] (201.48, 60.31) circle (  2.13);

\path[fill=fillColor,fill opacity=0.20] (200.48, 65.92) circle (  2.13);

\path[fill=fillColor,fill opacity=0.20] (204.49, 63.01) circle (  2.13);

\path[fill=fillColor,fill opacity=0.20] (208.51, 61.97) circle (  2.13);

\path[fill=fillColor,fill opacity=0.20] (206.50, 61.04) circle (  2.13);

\path[fill=fillColor,fill opacity=0.20] (208.51, 65.92) circle (  2.13);

\path[fill=fillColor,fill opacity=0.20] (215.53, 75.47) circle (  2.13);

\path[fill=fillColor,fill opacity=0.20] (217.53, 59.06) circle (  2.13);

\path[fill=fillColor,fill opacity=0.20] (210.51, 57.71) circle (  2.13);

\path[fill=fillColor,fill opacity=0.20] (201.48, 61.87) circle (  2.13);

\path[fill=fillColor,fill opacity=0.20] (200.48, 67.68) circle (  2.13);

\path[fill=fillColor,fill opacity=0.20] (199.48, 57.92) circle (  2.13);

\path[fill=fillColor,fill opacity=0.20] (193.46, 54.08) circle (  2.13);

\path[fill=fillColor,fill opacity=0.20] (191.45, 54.50) circle (  2.13);

\path[fill=fillColor,fill opacity=0.20] (192.45, 56.88) circle (  2.13);

\path[fill=fillColor,fill opacity=0.20] (194.46, 65.71) circle (  2.13);

\path[fill=fillColor,fill opacity=0.20] (196.47, 71.42) circle (  2.13);

\path[fill=fillColor,fill opacity=0.20] (197.47, 59.48) circle (  2.13);

\path[fill=fillColor,fill opacity=0.20] (209.51, 49.20) circle (  2.13);

\path[fill=fillColor,fill opacity=0.20] (218.54, 60.10) circle (  2.13);

\path[fill=fillColor,fill opacity=0.20] (234.59, 53.87) circle (  2.13);

\path[fill=fillColor,fill opacity=0.20] (212.52, 47.75) circle (  2.13);

\path[fill=fillColor,fill opacity=0.20] (205.50, 57.51) circle (  2.13);

\path[fill=fillColor,fill opacity=0.20] (205.50, 58.13) circle (  2.13);

\path[fill=fillColor,fill opacity=0.20] (206.50, 50.76) circle (  2.13);

\path[fill=fillColor,fill opacity=0.20] (203.49, 62.39) circle (  2.13);

\path[fill=fillColor,fill opacity=0.20] (209.51, 69.24) circle (  2.13);

\path[fill=fillColor,fill opacity=0.20] (212.52, 60.52) circle (  2.13);

\path[fill=fillColor,fill opacity=0.20] (213.52, 59.58) circle (  2.13);

\path[fill=fillColor,fill opacity=0.20] (219.54, 76.30) circle (  2.13);

\path[fill=fillColor,fill opacity=0.20] (221.55, 60.00) circle (  2.13);

\path[fill=fillColor,fill opacity=0.20] (213.52, 45.88) circle (  2.13);

\path[fill=fillColor,fill opacity=0.20] (203.49, 50.86) circle (  2.13);

\path[fill=fillColor,fill opacity=0.20] (202.49, 64.98) circle (  2.13);

\path[fill=fillColor,fill opacity=0.20] (201.48, 75.88) circle (  2.13);

\path[fill=fillColor,fill opacity=0.20] (200.48, 70.48) circle (  2.13);

\path[fill=fillColor,fill opacity=0.20] (194.46, 58.34) circle (  2.13);

\path[fill=fillColor,fill opacity=0.20] (190.45, 60.31) circle (  2.13);

\path[fill=fillColor,fill opacity=0.20] (189.44, 69.97) circle (  2.13);

\path[fill=fillColor,fill opacity=0.20] (189.44, 69.86) circle (  2.13);

\path[fill=fillColor,fill opacity=0.20] (191.45, 63.32) circle (  2.13);

\path[fill=fillColor,fill opacity=0.20] (195.46, 58.55) circle (  2.13);

\path[fill=fillColor,fill opacity=0.20] (209.51, 56.05) circle (  2.13);

\path[fill=fillColor,fill opacity=0.20] (223.55, 57.40) circle (  2.13);

\path[fill=fillColor,fill opacity=0.20] (212.52, 53.56) circle (  2.13);

\path[fill=fillColor,fill opacity=0.20] (208.51, 47.54) circle (  2.13);

\path[fill=fillColor,fill opacity=0.20] (206.50, 49.31) circle (  2.13);

\path[fill=fillColor,fill opacity=0.20] (207.50, 64.26) circle (  2.13);

\path[fill=fillColor,fill opacity=0.20] (214.53, 66.44) circle (  2.13);

\path[fill=fillColor,fill opacity=0.20] (214.53, 53.15) circle (  2.13);

\path[fill=fillColor,fill opacity=0.20] (211.52, 48.27) circle (  2.13);

\path[fill=fillColor,fill opacity=0.20] (215.53, 55.02) circle (  2.13);

\path[fill=fillColor,fill opacity=0.20] (228.57, 76.30) circle (  2.13);

\path[fill=fillColor,fill opacity=0.20] (217.53, 68.72) circle (  2.13);

\path[fill=fillColor,fill opacity=0.20] (210.51, 53.25) circle (  2.13);

\path[fill=fillColor,fill opacity=0.20] (208.51, 48.68) circle (  2.13);

\path[fill=fillColor,fill opacity=0.20] (207.50, 57.20) circle (  2.13);

\path[fill=fillColor,fill opacity=0.20] (203.49, 58.75) circle (  2.13);

\path[fill=fillColor,fill opacity=0.20] (199.48, 57.40) circle (  2.13);

\path[fill=fillColor,fill opacity=0.20] (198.47, 64.46) circle (  2.13);

\path[fill=fillColor,fill opacity=0.20] (194.46, 63.22) circle (  2.13);

\path[fill=fillColor,fill opacity=0.20] (191.45, 61.24) circle (  2.13);

\path[fill=fillColor,fill opacity=0.20] (193.46, 68.20) circle (  2.13);

\path[fill=fillColor,fill opacity=0.20] (190.45, 64.36) circle (  2.13);

\path[fill=fillColor,fill opacity=0.20] (194.46, 54.70) circle (  2.13);

\path[fill=fillColor,fill opacity=0.20] (207.50, 59.79) circle (  2.13);

\path[fill=fillColor,fill opacity=0.20] (219.54, 57.92) circle (  2.13);

\path[fill=fillColor,fill opacity=0.20] (209.51, 53.77) circle (  2.13);

\path[fill=fillColor,fill opacity=0.20] (208.51, 56.26) circle (  2.13);

\path[fill=fillColor,fill opacity=0.20] (209.51, 57.82) circle (  2.13);

\path[fill=fillColor,fill opacity=0.20] (210.51, 56.47) circle (  2.13);

\path[fill=fillColor,fill opacity=0.20] (212.52, 56.26) circle (  2.13);

\path[fill=fillColor,fill opacity=0.20] (209.51, 56.26) circle (  2.13);

\path[fill=fillColor,fill opacity=0.20] (211.52, 56.88) circle (  2.13);

\path[fill=fillColor,fill opacity=0.20] (219.54, 56.68) circle (  2.13);

\path[fill=fillColor,fill opacity=0.20] (230.58, 70.28) circle (  2.13);

\path[fill=fillColor,fill opacity=0.20] (209.51, 83.77) circle (  2.13);

\path[fill=fillColor,fill opacity=0.20] (212.52, 53.87) circle (  2.13);

\path[fill=fillColor,fill opacity=0.20] (208.51, 55.95) circle (  2.13);

\path[fill=fillColor,fill opacity=0.20] (208.51, 70.07) circle (  2.13);

\path[fill=fillColor,fill opacity=0.20] (203.49, 66.33) circle (  2.13);

\path[fill=fillColor,fill opacity=0.20] (200.48, 51.80) circle (  2.13);

\path[fill=fillColor,fill opacity=0.20] (195.46, 44.32) circle (  2.13);

\path[fill=fillColor,fill opacity=0.20] (194.46, 56.47) circle (  2.13);

\path[fill=fillColor,fill opacity=0.20] (189.44, 63.74) circle (  2.13);

\path[fill=fillColor,fill opacity=0.20] (192.45, 56.57) circle (  2.13);

\path[fill=fillColor,fill opacity=0.20] (199.48, 49.72) circle (  2.13);

\path[fill=fillColor,fill opacity=0.20] (197.47, 50.86) circle (  2.13);

\path[fill=fillColor,fill opacity=0.20] (204.49, 59.79) circle (  2.13);

\path[fill=fillColor,fill opacity=0.20] (215.53, 61.45) circle (  2.13);

\path[fill=fillColor,fill opacity=0.20] (213.52, 61.35) circle (  2.13);

\path[fill=fillColor,fill opacity=0.20] (209.51, 55.33) circle (  2.13);

\path[fill=fillColor,fill opacity=0.20] (210.51, 43.70) circle (  2.13);

\path[fill=fillColor,fill opacity=0.20] (210.51, 45.67) circle (  2.13);

\path[fill=fillColor,fill opacity=0.20] (210.51, 59.17) circle (  2.13);

\path[fill=fillColor,fill opacity=0.20] (208.51, 65.92) circle (  2.13);

\path[fill=fillColor,fill opacity=0.20] (210.51, 63.01) circle (  2.13);

\path[fill=fillColor,fill opacity=0.20] (218.54, 55.53) circle (  2.13);

\path[fill=fillColor,fill opacity=0.20] (223.55, 50.97) circle (  2.13);

\path[fill=fillColor,fill opacity=0.20] (233.59, 61.76) circle (  2.13);

\path[fill=fillColor,fill opacity=0.20] (220.54, 83.77) circle (  2.13);

\path[fill=fillColor,fill opacity=0.20] (209.51, 75.88) circle (  2.13);

\path[fill=fillColor,fill opacity=0.20] (206.50, 64.36) circle (  2.13);

\path[fill=fillColor,fill opacity=0.20] (210.51, 62.91) circle (  2.13);

\path[fill=fillColor,fill opacity=0.20] (210.51, 73.18) circle (  2.13);

\path[fill=fillColor,fill opacity=0.20] (200.48, 68.82) circle (  2.13);

\path[fill=fillColor,fill opacity=0.20] (197.47, 55.22) circle (  2.13);

\path[fill=fillColor,fill opacity=0.20] (195.46, 55.95) circle (  2.13);

\path[fill=fillColor,fill opacity=0.20] (193.46, 65.19) circle (  2.13);

\path[fill=fillColor,fill opacity=0.20] (189.44, 65.19) circle (  2.13);

\path[fill=fillColor,fill opacity=0.20] (166.37, 60.31) circle (  2.13);

\path[fill=fillColor,fill opacity=0.20] (202.49, 47.96) circle (  2.13);

\path[fill=fillColor,fill opacity=0.20] (206.50, 45.46) circle (  2.13);

\path[fill=fillColor,fill opacity=0.20] (212.52, 60.62) circle (  2.13);

\path[fill=fillColor,fill opacity=0.20] (223.55, 61.87) circle (  2.13);

\path[fill=fillColor,fill opacity=0.20] (215.53, 52.42) circle (  2.13);

\path[fill=fillColor,fill opacity=0.20] (209.51, 39.55) circle (  2.13);

\path[fill=fillColor,fill opacity=0.20] (210.51, 41.00) circle (  2.13);

\path[fill=fillColor,fill opacity=0.20] (211.52, 58.86) circle (  2.13);

\path[fill=fillColor,fill opacity=0.20] (210.51, 65.50) circle (  2.13);

\path[fill=fillColor,fill opacity=0.20] (215.53, 59.38) circle (  2.13);

\path[fill=fillColor,fill opacity=0.20] (211.52, 56.36) circle (  2.13);

\path[fill=fillColor,fill opacity=0.20] (218.54, 60.62) circle (  2.13);

\path[fill=fillColor,fill opacity=0.20] (225.56, 62.18) circle (  2.13);

\path[fill=fillColor,fill opacity=0.20] (234.59, 61.04) circle (  2.13);

\path[fill=fillColor,fill opacity=0.20] (222.55, 72.77) circle (  2.13);

\path[fill=fillColor,fill opacity=0.20] (214.53, 51.59) circle (  2.13);

\path[fill=fillColor,fill opacity=0.20] (210.51, 68.72) circle (  2.13);

\path[fill=fillColor,fill opacity=0.20] (204.49, 74.64) circle (  2.13);

\path[fill=fillColor,fill opacity=0.20] (200.48, 59.48) circle (  2.13);

\path[fill=fillColor,fill opacity=0.20] (204.49, 57.71) circle (  2.13);

\path[fill=fillColor,fill opacity=0.20] (201.48, 67.58) circle (  2.13);

\path[fill=fillColor,fill opacity=0.20] (199.48, 65.19) circle (  2.13);

\path[fill=fillColor,fill opacity=0.20] (198.47, 62.91) circle (  2.13);

\path[fill=fillColor,fill opacity=0.20] (196.47, 65.40) circle (  2.13);

\path[fill=fillColor,fill opacity=0.20] (198.47, 66.44) circle (  2.13);

\path[fill=fillColor,fill opacity=0.20] (201.48, 67.58) circle (  2.13);

\path[fill=fillColor,fill opacity=0.20] (207.50, 63.94) circle (  2.13);

\path[fill=fillColor,fill opacity=0.20] (215.53, 51.07) circle (  2.13);

\path[fill=fillColor,fill opacity=0.20] (216.53, 67.79) circle (  2.13);

\path[fill=fillColor,fill opacity=0.20] (207.50, 45.78) circle (  2.13);

\path[fill=fillColor,fill opacity=0.20] (208.51, 40.58) circle (  2.13);

\path[fill=fillColor,fill opacity=0.20] (211.52, 53.56) circle (  2.13);

\path[fill=fillColor,fill opacity=0.20] (216.53, 60.10) circle (  2.13);

\path[fill=fillColor,fill opacity=0.20] (214.53, 55.22) circle (  2.13);

\path[fill=fillColor,fill opacity=0.20] (211.52, 57.82) circle (  2.13);

\path[fill=fillColor,fill opacity=0.20] (212.52, 67.68) circle (  2.13);

\path[fill=fillColor,fill opacity=0.20] (220.54, 71.42) circle (  2.13);

\path[fill=fillColor,fill opacity=0.20] (223.55, 61.87) circle (  2.13);

\path[fill=fillColor,fill opacity=0.20] (222.55, 57.92) circle (  2.13);

\path[fill=fillColor,fill opacity=0.20] (227.57, 84.81) circle (  2.13);

\path[fill=fillColor,fill opacity=0.20] (223.55, 71.52) circle (  2.13);

\path[fill=fillColor,fill opacity=0.20] (219.54, 47.23) circle (  2.13);

\path[fill=fillColor,fill opacity=0.20] (209.51, 59.17) circle (  2.13);

\path[fill=fillColor,fill opacity=0.20] (207.50, 72.15) circle (  2.13);

\path[fill=fillColor,fill opacity=0.20] (195.46, 62.18) circle (  2.13);

\path[fill=fillColor,fill opacity=0.20] (200.48, 56.05) circle (  2.13);

\path[fill=fillColor,fill opacity=0.20] (200.48, 67.58) circle (  2.13);

\path[fill=fillColor,fill opacity=0.20] (203.49, 67.99) circle (  2.13);

\path[fill=fillColor,fill opacity=0.20] (201.48, 61.66) circle (  2.13);

\path[fill=fillColor,fill opacity=0.20] (194.46, 58.23) circle (  2.13);

\path[fill=fillColor,fill opacity=0.20] (206.50, 52.11) circle (  2.13);

\path[fill=fillColor,fill opacity=0.20] (210.51, 58.86) circle (  2.13);

\path[fill=fillColor,fill opacity=0.20] (213.52, 69.65) circle (  2.13);

\path[fill=fillColor,fill opacity=0.20] (222.55, 68.20) circle (  2.13);

\path[fill=fillColor,fill opacity=0.20] (223.55, 74.74) circle (  2.13);

\path[fill=fillColor,fill opacity=0.20] (207.50, 75.36) circle (  2.13);

\path[fill=fillColor,fill opacity=0.20] (209.51, 53.67) circle (  2.13);

\path[fill=fillColor,fill opacity=0.20] (212.52, 41.83) circle (  2.13);

\path[fill=fillColor,fill opacity=0.20] (211.52, 44.32) circle (  2.13);

\path[fill=fillColor,fill opacity=0.20] (212.52, 48.89) circle (  2.13);

\path[fill=fillColor,fill opacity=0.20] (214.53, 51.69) circle (  2.13);

\path[fill=fillColor,fill opacity=0.20] (217.53, 61.76) circle (  2.13);

\path[fill=fillColor,fill opacity=0.20] (215.53, 72.87) circle (  2.13);

\path[fill=fillColor,fill opacity=0.20] (218.54, 78.89) circle (  2.13);

\path[fill=fillColor,fill opacity=0.20] (217.53, 67.16) circle (  2.13);

\path[fill=fillColor,fill opacity=0.20] (224.56, 49.51) circle (  2.13);

\path[fill=fillColor,fill opacity=0.20] (231.58, 60.21) circle (  2.13);

\path[fill=fillColor,fill opacity=0.20] (220.54, 59.89) circle (  2.13);

\path[fill=fillColor,fill opacity=0.20] (211.52, 50.65) circle (  2.13);

\path[fill=fillColor,fill opacity=0.20] (214.53, 46.40) circle (  2.13);

\path[fill=fillColor,fill opacity=0.20] (215.53, 46.29) circle (  2.13);

\path[fill=fillColor,fill opacity=0.20] (209.51, 56.16) circle (  2.13);

\path[fill=fillColor,fill opacity=0.20] (205.50, 63.94) circle (  2.13);

\path[fill=fillColor,fill opacity=0.20] (203.49, 60.62) circle (  2.13);

\path[fill=fillColor,fill opacity=0.20] (200.48, 57.92) circle (  2.13);

\path[fill=fillColor,fill opacity=0.20] (194.46, 59.17) circle (  2.13);

\path[fill=fillColor,fill opacity=0.20] (207.50, 59.79) circle (  2.13);

\path[fill=fillColor,fill opacity=0.20] (206.50, 67.16) circle (  2.13);

\path[fill=fillColor,fill opacity=0.20] (205.50, 70.28) circle (  2.13);

\path[fill=fillColor,fill opacity=0.20] (213.52, 57.71) circle (  2.13);

\path[fill=fillColor,fill opacity=0.20] (219.54, 60.00) circle (  2.13);

\path[fill=fillColor,fill opacity=0.20] (212.52, 71.32) circle (  2.13);

\path[fill=fillColor,fill opacity=0.20] (210.51, 56.05) circle (  2.13);

\path[fill=fillColor,fill opacity=0.20] (208.51, 48.58) circle (  2.13);

\path[fill=fillColor,fill opacity=0.20] (211.52, 47.33) circle (  2.13);

\path[fill=fillColor,fill opacity=0.20] (205.50, 52.00) circle (  2.13);

\path[fill=fillColor,fill opacity=0.20] (214.53, 67.47) circle (  2.13);

\path[fill=fillColor,fill opacity=0.20] (209.51, 91.04) circle (  2.13);

\path[fill=fillColor,fill opacity=0.20] (208.51, 85.85) circle (  2.13);

\path[fill=fillColor,fill opacity=0.20] (216.53, 49.72) circle (  2.13);

\path[fill=fillColor,fill opacity=0.20] (222.55, 64.26) circle (  2.13);

\path[fill=fillColor,fill opacity=0.20] (227.57, 63.84) circle (  2.13);

\path[fill=fillColor,fill opacity=0.20] (218.54, 58.65) circle (  2.13);

\path[fill=fillColor,fill opacity=0.20] (211.52, 66.85) circle (  2.13);

\path[fill=fillColor,fill opacity=0.20] (207.50, 65.71) circle (  2.13);

\path[fill=fillColor,fill opacity=0.20] (209.51, 60.00) circle (  2.13);

\path[fill=fillColor,fill opacity=0.20] (210.51, 56.99) circle (  2.13);

\path[fill=fillColor,fill opacity=0.20] (212.52, 57.40) circle (  2.13);

\path[fill=fillColor,fill opacity=0.20] (210.51, 54.91) circle (  2.13);

\path[fill=fillColor,fill opacity=0.20] (210.51, 50.45) circle (  2.13);

\path[fill=fillColor,fill opacity=0.20] (215.53, 55.43) circle (  2.13);

\path[fill=fillColor,fill opacity=0.20] (215.53, 68.41) circle (  2.13);

\path[fill=fillColor,fill opacity=0.20] (211.52, 65.71) circle (  2.13);

\path[fill=fillColor,fill opacity=0.20] (205.50, 41.93) circle (  2.13);

\path[fill=fillColor,fill opacity=0.20] (209.51, 55.53) circle (  2.13);

\path[fill=fillColor,fill opacity=0.20] (208.51, 87.93) circle (  2.13);

\path[fill=fillColor,fill opacity=0.20] (205.50, 86.89) circle (  2.13);

\path[fill=fillColor,fill opacity=0.20] (209.51, 53.98) circle (  2.13);

\path[fill=fillColor,fill opacity=0.20] (209.51, 50.86) circle (  2.13);

\path[fill=fillColor,fill opacity=0.20] (210.51, 74.43) circle (  2.13);

\path[fill=fillColor,fill opacity=0.20] (211.52, 77.96) circle (  2.13);

\path[fill=fillColor,fill opacity=0.20] (222.55, 80.66) circle (  2.13);

\path[fill=fillColor,fill opacity=0.20] (220.54, 71.42) circle (  2.13);

\path[fill=fillColor,fill opacity=0.20] (218.54, 61.14) circle (  2.13);

\path[fill=fillColor,fill opacity=0.20] (213.52, 59.38) circle (  2.13);

\path[fill=fillColor,fill opacity=0.20] (207.50, 60.10) circle (  2.13);

\path[fill=fillColor,fill opacity=0.20] (205.50, 61.24) circle (  2.13);

\path[fill=fillColor,fill opacity=0.20] (204.49, 63.94) circle (  2.13);

\path[fill=fillColor,fill opacity=0.20] (209.51, 67.99) circle (  2.13);

\path[fill=fillColor,fill opacity=0.20] (214.53, 72.15) circle (  2.13);

\path[fill=fillColor,fill opacity=0.20] (217.53, 74.95) circle (  2.13);

\path[fill=fillColor,fill opacity=0.20] (223.55, 75.16) circle (  2.13);

\path[fill=fillColor,fill opacity=0.20] (225.56, 77.13) circle (  2.13);

\path[fill=fillColor,fill opacity=0.20] (206.50, 70.17) circle (  2.13);

\path[fill=fillColor,fill opacity=0.20] (209.51, 49.10) circle (  2.13);

\path[fill=fillColor,fill opacity=0.20] (205.50, 52.21) circle (  2.13);

\path[fill=fillColor,fill opacity=0.20] (205.50, 76.40) circle (  2.13);

\path[fill=fillColor,fill opacity=0.20] (206.50, 71.94) circle (  2.13);

\path[fill=fillColor,fill opacity=0.20] (207.50, 48.47) circle (  2.13);

\path[fill=fillColor,fill opacity=0.20] (208.51, 61.35) circle (  2.13);

\path[fill=fillColor,fill opacity=0.20] (205.50, 83.77) circle (  2.13);

\path[fill=fillColor,fill opacity=0.20] (215.53, 70.69) circle (  2.13);

\path[fill=fillColor,fill opacity=0.20] (215.53, 53.04) circle (  2.13);

\path[fill=fillColor,fill opacity=0.20] (213.52, 90.00) circle (  2.13);

\path[fill=fillColor,fill opacity=0.20] (212.52, 65.29) circle (  2.13);

\path[fill=fillColor,fill opacity=0.20] (212.52, 55.02) circle (  2.13);

\path[fill=fillColor,fill opacity=0.20] (212.52, 50.65) circle (  2.13);

\path[fill=fillColor,fill opacity=0.20] (206.50, 46.09) circle (  2.13);

\path[fill=fillColor,fill opacity=0.20] (206.50, 58.23) circle (  2.13);

\path[fill=fillColor,fill opacity=0.20] (214.53, 73.50) circle (  2.13);

\path[fill=fillColor,fill opacity=0.20] (209.51, 86.89) circle (  2.13);

\path[fill=fillColor,fill opacity=0.20] (214.53, 86.89) circle (  2.13);

\path[fill=fillColor,fill opacity=0.20] (209.51, 72.56) circle (  2.13);

\path[fill=fillColor,fill opacity=0.20] (208.51, 67.89) circle (  2.13);

\path[fill=fillColor,fill opacity=0.20] (209.51, 54.29) circle (  2.13);

\path[fill=fillColor,fill opacity=0.20] (208.51, 43.91) circle (  2.13);

\path[fill=fillColor,fill opacity=0.20] (205.50, 56.36) circle (  2.13);

\path[fill=fillColor,fill opacity=0.20] (205.50, 68.82) circle (  2.13);

\path[fill=fillColor,fill opacity=0.20] (209.51, 68.72) circle (  2.13);

\path[fill=fillColor,fill opacity=0.20] (213.52, 71.42) circle (  2.13);

\path[fill=fillColor,fill opacity=0.20] (210.51, 66.75) circle (  2.13);

\path[fill=fillColor,fill opacity=0.20] (218.54, 42.04) circle (  2.13);

\path[fill=fillColor,fill opacity=0.20] (223.55, 61.04) circle (  2.13);

\path[fill=fillColor,fill opacity=0.20] (221.55, 85.85) circle (  2.13);

\path[fill=fillColor,fill opacity=0.20] (218.54, 91.04) circle (  2.13);

\path[fill=fillColor,fill opacity=0.20] (212.52, 59.06) circle (  2.13);

\path[fill=fillColor,fill opacity=0.20] (212.52, 39.96) circle (  2.13);

\path[fill=fillColor,fill opacity=0.20] (211.52, 53.87) circle (  2.13);

\path[fill=fillColor,fill opacity=0.20] (213.52, 61.45) circle (  2.13);

\path[fill=fillColor,fill opacity=0.20] (214.53, 68.82) circle (  2.13);

\path[fill=fillColor,fill opacity=0.20] (213.52, 86.89) circle (  2.13);

\path[fill=fillColor,fill opacity=0.20] (215.53,107.65) circle (  2.13);

\path[fill=fillColor,fill opacity=0.20] (216.53,106.61) circle (  2.13);

\path[fill=fillColor,fill opacity=0.20] (213.52, 82.74) circle (  2.13);

\path[fill=fillColor,fill opacity=0.20] (209.51, 53.87) circle (  2.13);

\path[fill=fillColor,fill opacity=0.20] (209.51, 47.96) circle (  2.13);

\path[fill=fillColor,fill opacity=0.20] (206.50, 49.10) circle (  2.13);

\path[fill=fillColor,fill opacity=0.20] (210.51, 47.64) circle (  2.13);

\path[fill=fillColor,fill opacity=0.20] (207.50, 56.57) circle (  2.13);

\path[fill=fillColor,fill opacity=0.20] (209.51, 77.34) circle (  2.13);

\path[fill=fillColor,fill opacity=0.20] (211.52, 84.81) circle (  2.13);

\path[fill=fillColor,fill opacity=0.20] (212.52, 66.64) circle (  2.13);

\path[fill=fillColor,fill opacity=0.20] (214.53, 55.43) circle (  2.13);

\path[fill=fillColor,fill opacity=0.20] (221.55, 64.57) circle (  2.13);

\path[fill=fillColor,fill opacity=0.20] (218.54, 64.36) circle (  2.13);

\path[fill=fillColor,fill opacity=0.20] (221.55, 56.99) circle (  2.13);

\path[fill=fillColor,fill opacity=0.20] (223.55, 68.20) circle (  2.13);

\path[fill=fillColor,fill opacity=0.20] (218.54, 91.04) circle (  2.13);

\path[fill=fillColor,fill opacity=0.20] (213.52, 90.00) circle (  2.13);

\path[fill=fillColor,fill opacity=0.20] (212.52, 78.48) circle (  2.13);

\path[fill=fillColor,fill opacity=0.20] (214.53, 65.40) circle (  2.13);

\path[fill=fillColor,fill opacity=0.20] (216.53, 59.38) circle (  2.13);

\path[fill=fillColor,fill opacity=0.20] (213.52, 57.30) circle (  2.13);

\path[fill=fillColor,fill opacity=0.20] (190.45, 46.81) circle (  2.13);

\path[fill=fillColor,fill opacity=0.20] (224.56, 67.16) circle (  2.13);

\path[fill=fillColor,fill opacity=0.20] (217.53,108.69) circle (  2.13);

\path[fill=fillColor,fill opacity=0.20] (214.53, 74.12) circle (  2.13);

\path[fill=fillColor,fill opacity=0.20] (205.50, 57.82) circle (  2.13);

\path[fill=fillColor,fill opacity=0.20] (208.51, 45.67) circle (  2.13);

\path[fill=fillColor,fill opacity=0.20] (209.51, 38.40) circle (  2.13);

\path[fill=fillColor,fill opacity=0.20] (210.51, 48.37) circle (  2.13);

\path[fill=fillColor,fill opacity=0.20] (209.51, 68.30) circle (  2.13);

\path[fill=fillColor,fill opacity=0.20] (207.50, 80.76) circle (  2.13);

\path[fill=fillColor,fill opacity=0.20] (213.52, 80.35) circle (  2.13);

\path[fill=fillColor,fill opacity=0.20] (213.52, 85.85) circle (  2.13);

\path[fill=fillColor,fill opacity=0.20] (212.52, 84.81) circle (  2.13);

\path[fill=fillColor,fill opacity=0.20] (214.53, 62.59) circle (  2.13);

\path[fill=fillColor,fill opacity=0.20] (220.54, 49.93) circle (  2.13);

\path[fill=fillColor,fill opacity=0.20] (219.54, 63.42) circle (  2.13);

\path[fill=fillColor,fill opacity=0.20] (215.53, 69.03) circle (  2.13);

\path[fill=fillColor,fill opacity=0.20] (218.54, 57.20) circle (  2.13);

\path[fill=fillColor,fill opacity=0.20] (221.55, 56.36) circle (  2.13);

\path[fill=fillColor,fill opacity=0.20] (223.55, 74.22) circle (  2.13);

\path[fill=fillColor,fill opacity=0.20] (222.55, 95.19) circle (  2.13);

\path[fill=fillColor,fill opacity=0.20] (218.54,103.50) circle (  2.13);

\path[fill=fillColor,fill opacity=0.20] (216.53,101.42) circle (  2.13);

\path[fill=fillColor,fill opacity=0.20] (214.53, 88.96) circle (  2.13);

\path[fill=fillColor,fill opacity=0.20] (216.53, 95.19) circle (  2.13);

\path[fill=fillColor,fill opacity=0.20] (221.55, 98.31) circle (  2.13);

\path[fill=fillColor,fill opacity=0.20] (215.53, 78.48) circle (  2.13);

\path[fill=fillColor,fill opacity=0.20] (216.53, 70.48) circle (  2.13);

\path[fill=fillColor,fill opacity=0.20] (214.53, 64.77) circle (  2.13);

\path[fill=fillColor,fill opacity=0.20] (212.52, 71.32) circle (  2.13);

\path[fill=fillColor,fill opacity=0.20] (211.52, 69.65) circle (  2.13);

\path[fill=fillColor,fill opacity=0.20] (215.53, 54.50) circle (  2.13);

\path[fill=fillColor,fill opacity=0.20] (213.52, 52.84) circle (  2.13);

\path[fill=fillColor,fill opacity=0.20] (214.53, 45.57) circle (  2.13);

\path[fill=fillColor,fill opacity=0.20] (219.54, 44.84) circle (  2.13);

\path[fill=fillColor,fill opacity=0.20] (221.55, 77.23) circle (  2.13);

\path[fill=fillColor,fill opacity=0.20] (208.51, 58.23) circle (  2.13);

\path[fill=fillColor,fill opacity=0.20] (206.50, 55.02) circle (  2.13);

\path[fill=fillColor,fill opacity=0.20] (209.51, 62.70) circle (  2.13);

\path[fill=fillColor,fill opacity=0.20] (213.52, 64.36) circle (  2.13);

\path[fill=fillColor,fill opacity=0.20] (215.53, 61.97) circle (  2.13);

\path[fill=fillColor,fill opacity=0.20] (214.53, 64.26) circle (  2.13);

\path[fill=fillColor,fill opacity=0.20] (210.51, 71.11) circle (  2.13);

\path[fill=fillColor,fill opacity=0.20] (209.51, 67.27) circle (  2.13);

\path[fill=fillColor,fill opacity=0.20] (213.52, 55.64) circle (  2.13);

\path[fill=fillColor,fill opacity=0.20] (216.53, 55.02) circle (  2.13);

\path[fill=fillColor,fill opacity=0.20] (216.53, 66.85) circle (  2.13);

\path[fill=fillColor,fill opacity=0.20] (214.53, 69.45) circle (  2.13);

\path[fill=fillColor,fill opacity=0.20] (215.53, 58.96) circle (  2.13);

\path[fill=fillColor,fill opacity=0.20] (220.54, 54.08) circle (  2.13);

\path[fill=fillColor,fill opacity=0.20] (217.53, 65.61) circle (  2.13);

\path[fill=fillColor,fill opacity=0.20] (215.53, 73.18) circle (  2.13);

\path[fill=fillColor,fill opacity=0.20] (213.52, 70.69) circle (  2.13);

\path[fill=fillColor,fill opacity=0.20] (216.53, 58.65) circle (  2.13);

\path[fill=fillColor,fill opacity=0.20] (216.53, 57.51) circle (  2.13);

\path[fill=fillColor,fill opacity=0.20] (214.53, 69.24) circle (  2.13);

\path[fill=fillColor,fill opacity=0.20] (210.51, 69.86) circle (  2.13);

\path[fill=fillColor,fill opacity=0.20] (214.53, 66.44) circle (  2.13);

\path[fill=fillColor,fill opacity=0.20] (215.53, 68.41) circle (  2.13);

\path[fill=fillColor,fill opacity=0.20] (214.53, 59.69) circle (  2.13);

\path[fill=fillColor,fill opacity=0.20] (214.53, 54.29) circle (  2.13);

\path[fill=fillColor,fill opacity=0.20] (213.52, 65.61) circle (  2.13);

\path[fill=fillColor,fill opacity=0.20] (208.51, 62.08) circle (  2.13);

\path[fill=fillColor,fill opacity=0.20] (216.53, 49.41) circle (  2.13);

\path[fill=fillColor,fill opacity=0.20] (211.52, 49.93) circle (  2.13);

\path[fill=fillColor,fill opacity=0.20] (213.52, 43.70) circle (  2.13);

\path[fill=fillColor,fill opacity=0.20] (200.48, 39.34) circle (  2.13);

\path[fill=fillColor,fill opacity=0.20] (212.52, 51.38) circle (  2.13);

\path[fill=fillColor,fill opacity=0.20] (214.53, 58.44) circle (  2.13);

\path[fill=fillColor,fill opacity=0.20] (217.53, 61.66) circle (  2.13);

\path[fill=fillColor,fill opacity=0.20] (217.53, 71.63) circle (  2.13);

\path[fill=fillColor,fill opacity=0.20] (220.54, 96.23) circle (  2.13);

\path[fill=fillColor,fill opacity=0.20] (215.53, 61.87) circle (  2.13);

\path[fill=fillColor,fill opacity=0.20] (213.52, 60.31) circle (  2.13);

\path[fill=fillColor,fill opacity=0.20] (210.51, 54.50) circle (  2.13);

\path[fill=fillColor,fill opacity=0.20] (208.51, 57.09) circle (  2.13);

\path[fill=fillColor,fill opacity=0.20] (209.51, 57.30) circle (  2.13);

\path[fill=fillColor,fill opacity=0.20] (212.52, 56.26) circle (  2.13);

\path[fill=fillColor,fill opacity=0.20] (214.53, 63.63) circle (  2.13);

\path[fill=fillColor,fill opacity=0.20] (215.53, 65.92) circle (  2.13);

\path[fill=fillColor,fill opacity=0.20] (213.52, 57.51) circle (  2.13);

\path[fill=fillColor,fill opacity=0.20] (211.52, 58.34) circle (  2.13);

\path[fill=fillColor,fill opacity=0.20] (215.53, 66.23) circle (  2.13);

\path[fill=fillColor,fill opacity=0.20] (211.52, 70.28) circle (  2.13);

\path[fill=fillColor,fill opacity=0.20] (207.50, 74.12) circle (  2.13);

\path[fill=fillColor,fill opacity=0.20] (211.52, 63.11) circle (  2.13);

\path[fill=fillColor,fill opacity=0.20] (215.53, 50.65) circle (  2.13);

\path[fill=fillColor,fill opacity=0.20] (214.53, 60.00) circle (  2.13);

\path[fill=fillColor,fill opacity=0.20] (212.52, 67.47) circle (  2.13);

\path[fill=fillColor,fill opacity=0.20] (212.52, 60.83) circle (  2.13);

\path[fill=fillColor,fill opacity=0.20] (214.53, 63.42) circle (  2.13);

\path[fill=fillColor,fill opacity=0.20] (213.52, 63.53) circle (  2.13);

\path[fill=fillColor,fill opacity=0.20] (212.52, 54.81) circle (  2.13);

\path[fill=fillColor,fill opacity=0.20] (210.51, 56.47) circle (  2.13);

\path[fill=fillColor,fill opacity=0.20] (209.51, 54.91) circle (  2.13);

\path[fill=fillColor,fill opacity=0.20] (208.51, 45.88) circle (  2.13);

\path[fill=fillColor,fill opacity=0.20] (209.51, 49.41) circle (  2.13);

\path[fill=fillColor,fill opacity=0.20] (210.51, 50.76) circle (  2.13);

\path[fill=fillColor,fill opacity=0.20] (215.53, 54.08) circle (  2.13);

\path[fill=fillColor,fill opacity=0.20] (217.53, 77.23) circle (  2.13);

\path[fill=fillColor,fill opacity=0.20] (207.50, 76.71) circle (  2.13);

\path[fill=fillColor,fill opacity=0.20] (208.51, 81.18) circle (  2.13);

\path[fill=fillColor,fill opacity=0.20] (209.51, 75.57) circle (  2.13);

\path[fill=fillColor,fill opacity=0.20] (213.52, 56.16) circle (  2.13);

\path[fill=fillColor,fill opacity=0.20] (217.53, 63.84) circle (  2.13);

\path[fill=fillColor,fill opacity=0.20] (214.53, 73.50) circle (  2.13);

\path[fill=fillColor,fill opacity=0.20] (211.52, 63.94) circle (  2.13);

\path[fill=fillColor,fill opacity=0.20] (212.52, 53.56) circle (  2.13);

\path[fill=fillColor,fill opacity=0.20] (213.52, 52.73) circle (  2.13);

\path[fill=fillColor,fill opacity=0.20] (209.51, 54.39) circle (  2.13);

\path[fill=fillColor,fill opacity=0.20] (209.51, 56.05) circle (  2.13);

\path[fill=fillColor,fill opacity=0.20] (209.51, 58.13) circle (  2.13);

\path[fill=fillColor,fill opacity=0.20] (210.51, 49.41) circle (  2.13);

\path[fill=fillColor,fill opacity=0.20] (212.52, 42.97) circle (  2.13);

\path[fill=fillColor,fill opacity=0.20] (211.52, 45.78) circle (  2.13);

\path[fill=fillColor,fill opacity=0.20] (211.52, 49.72) circle (  2.13);

\path[fill=fillColor,fill opacity=0.20] (211.52, 52.73) circle (  2.13);

\path[fill=fillColor,fill opacity=0.20] (209.51, 60.41) circle (  2.13);

\path[fill=fillColor,fill opacity=0.20] (209.51, 66.02) circle (  2.13);

\path[fill=fillColor,fill opacity=0.20] (209.51, 60.73) circle (  2.13);

\path[fill=fillColor,fill opacity=0.20] (208.51, 56.99) circle (  2.13);

\path[fill=fillColor,fill opacity=0.20] (207.50, 60.93) circle (  2.13);

\path[fill=fillColor,fill opacity=0.20] (209.51, 65.81) circle (  2.13);

\path[fill=fillColor,fill opacity=0.20] (212.52, 76.30) circle (  2.13);

\path[fill=fillColor,fill opacity=0.20] (220.54, 70.38) circle (  2.13);

\path[fill=fillColor,fill opacity=0.20] (215.53, 96.23) circle (  2.13);

\path[fill=fillColor,fill opacity=0.20] (212.52, 93.12) circle (  2.13);

\path[fill=fillColor,fill opacity=0.20] (215.53, 76.82) circle (  2.13);

\path[fill=fillColor,fill opacity=0.20] (215.53, 73.18) circle (  2.13);

\path[fill=fillColor,fill opacity=0.20] (212.52, 67.27) circle (  2.13);

\path[fill=fillColor,fill opacity=0.20] (211.52, 50.24) circle (  2.13);

\path[fill=fillColor,fill opacity=0.20] (209.51, 45.78) circle (  2.13);

\path[fill=fillColor,fill opacity=0.20] (209.51, 52.94) circle (  2.13);

\path[fill=fillColor,fill opacity=0.20] (211.52, 52.52) circle (  2.13);

\path[fill=fillColor,fill opacity=0.20] (212.52, 52.32) circle (  2.13);

\path[fill=fillColor,fill opacity=0.20] (214.53, 55.43) circle (  2.13);

\path[fill=fillColor,fill opacity=0.20] (212.52, 58.23) circle (  2.13);

\path[fill=fillColor,fill opacity=0.20] (209.51, 70.59) circle (  2.13);

\path[fill=fillColor,fill opacity=0.20] (210.51, 79.00) circle (  2.13);

\path[fill=fillColor,fill opacity=0.20] (215.53, 76.09) circle (  2.13);

\path[fill=fillColor,fill opacity=0.20] (215.53, 86.89) circle (  2.13);

\path[fill=fillColor,fill opacity=0.20] (217.53, 93.12) circle (  2.13);

\path[fill=fillColor,fill opacity=0.20] (216.53, 80.66) circle (  2.13);

\path[fill=fillColor,fill opacity=0.20] (211.52, 72.35) circle (  2.13);

\path[fill=fillColor,fill opacity=0.20] (210.51, 79.31) circle (  2.13);

\path[fill=fillColor,fill opacity=0.20] (203.49, 92.08) circle (  2.13);

\path[fill=fillColor,fill opacity=0.20] (213.52, 97.27) circle (  2.13);

\path[fill=fillColor,fill opacity=0.20] (215.53, 91.04) circle (  2.13);

\path[fill=fillColor,fill opacity=0.20] (202.49, 96.23) circle (  2.13);

\path[fill=fillColor,fill opacity=0.20] (200.48,111.81) circle (  2.13);

\path[fill=fillColor,fill opacity=0.20] (217.53,107.65) circle (  2.13);

\path[fill=fillColor,fill opacity=0.20] (222.55,103.50) circle (  2.13);

\path[fill=fillColor,fill opacity=0.20] (208.51, 65.09) circle (  2.13);

\path[fill=fillColor,fill opacity=0.20] (212.52, 65.29) circle (  2.13);

\path[fill=fillColor,fill opacity=0.20] (212.52, 64.05) circle (  2.13);

\path[fill=fillColor,fill opacity=0.20] (214.53, 69.76) circle (  2.13);

\path[fill=fillColor,fill opacity=0.20] (210.51, 71.73) circle (  2.13);

\path[fill=fillColor,fill opacity=0.20] (205.50, 71.00) circle (  2.13);

\path[fill=fillColor,fill opacity=0.20] (204.49, 69.13) circle (  2.13);

\path[fill=fillColor,fill opacity=0.20] (209.51, 63.74) circle (  2.13);

\path[fill=fillColor,fill opacity=0.20] (211.52, 57.82) circle (  2.13);

\path[fill=fillColor,fill opacity=0.20] (214.53, 58.44) circle (  2.13);

\path[fill=fillColor,fill opacity=0.20] (221.55, 63.42) circle (  2.13);

\path[fill=fillColor,fill opacity=0.20] (228.57, 66.02) circle (  2.13);

\path[fill=fillColor,fill opacity=0.20] (206.50, 73.39) circle (  2.13);

\path[fill=fillColor,fill opacity=0.20] (207.50, 74.12) circle (  2.13);

\path[fill=fillColor,fill opacity=0.20] (198.47, 75.57) circle (  2.13);

\path[fill=fillColor,fill opacity=0.20] (203.49, 73.18) circle (  2.13);

\path[fill=fillColor,fill opacity=0.20] (204.49, 66.75) circle (  2.13);

\path[fill=fillColor,fill opacity=0.20] (208.51, 61.87) circle (  2.13);

\path[fill=fillColor,fill opacity=0.20] (211.52, 61.35) circle (  2.13);

\path[fill=fillColor,fill opacity=0.20] (214.53, 63.22) circle (  2.13);

\path[fill=fillColor,fill opacity=0.20] (220.54, 67.68) circle (  2.13);

\path[fill=fillColor,fill opacity=0.20] (239.61, 70.69) circle (  2.13);

\path[fill=fillColor,fill opacity=0.20] (247.63, 78.38) circle (  2.13);

\path[fill=fillColor,fill opacity=0.20] (212.52, 86.89) circle (  2.13);

\path[fill=fillColor,fill opacity=0.20] (207.50, 75.78) circle (  2.13);

\path[fill=fillColor,fill opacity=0.20] (201.48, 68.20) circle (  2.13);

\path[fill=fillColor,fill opacity=0.20] (204.49, 70.28) circle (  2.13);

\path[fill=fillColor,fill opacity=0.20] (204.49, 69.03) circle (  2.13);

\path[fill=fillColor,fill opacity=0.20] (206.50, 64.88) circle (  2.13);

\path[fill=fillColor,fill opacity=0.20] (212.52, 64.36) circle (  2.13);

\path[fill=fillColor,fill opacity=0.20] (211.52, 66.33) circle (  2.13);

\path[fill=fillColor,fill opacity=0.20] (217.53, 68.20) circle (  2.13);

\path[fill=fillColor,fill opacity=0.20] (223.55, 75.78) circle (  2.13);

\path[fill=fillColor,fill opacity=0.20] (244.62, 86.89) circle (  2.13);

\path[fill=fillColor,fill opacity=0.20] (226.56, 84.81) circle (  2.13);

\path[fill=fillColor,fill opacity=0.20] (211.52, 81.07) circle (  2.13);

\path[fill=fillColor,fill opacity=0.20] (209.51, 71.52) circle (  2.13);

\path[fill=fillColor,fill opacity=0.20] (209.51, 62.28) circle (  2.13);

\path[fill=fillColor,fill opacity=0.20] (208.51, 60.21) circle (  2.13);

\path[fill=fillColor,fill opacity=0.20] (208.51, 62.59) circle (  2.13);

\path[fill=fillColor,fill opacity=0.20] (210.51, 66.64) circle (  2.13);

\path[fill=fillColor,fill opacity=0.20] (214.53, 69.24) circle (  2.13);

\path[fill=fillColor,fill opacity=0.20] (221.55, 66.64) circle (  2.13);

\path[fill=fillColor,fill opacity=0.20] (220.54, 66.12) circle (  2.13);

\path[fill=fillColor,fill opacity=0.20] (235.59, 78.06) circle (  2.13);

\path[fill=fillColor,fill opacity=0.20] (213.52, 64.98) circle (  2.13);

\path[fill=fillColor,fill opacity=0.20] (212.52, 64.46) circle (  2.13);

\path[fill=fillColor,fill opacity=0.20] (212.52, 63.63) circle (  2.13);

\path[fill=fillColor,fill opacity=0.20] (211.52, 54.70) circle (  2.13);

\path[fill=fillColor,fill opacity=0.20] (209.51, 56.05) circle (  2.13);

\path[fill=fillColor,fill opacity=0.20] (211.52, 65.19) circle (  2.13);

\path[fill=fillColor,fill opacity=0.20] (215.53, 68.41) circle (  2.13);

\path[fill=fillColor,fill opacity=0.20] (225.56, 65.50) circle (  2.13);

\path[fill=fillColor,fill opacity=0.20] (231.58, 67.89) circle (  2.13);

\path[fill=fillColor,fill opacity=0.20] (244.62, 82.74) circle (  2.13);

\path[fill=fillColor,fill opacity=0.20] (210.51, 58.44) circle (  2.13);

\path[fill=fillColor,fill opacity=0.20] (209.51, 54.70) circle (  2.13);

\path[fill=fillColor,fill opacity=0.20] (213.52, 56.99) circle (  2.13);

\path[fill=fillColor,fill opacity=0.20] (213.52, 59.06) circle (  2.13);

\path[fill=fillColor,fill opacity=0.20] (210.51, 63.94) circle (  2.13);

\path[fill=fillColor,fill opacity=0.20] (208.51, 53.87) circle (  2.13);

\path[fill=fillColor,fill opacity=0.20] (213.52, 54.39) circle (  2.13);

\path[fill=fillColor,fill opacity=0.20] (215.53, 63.11) circle (  2.13);

\path[fill=fillColor,fill opacity=0.20] (216.53, 64.98) circle (  2.13);

\path[fill=fillColor,fill opacity=0.20] (226.56, 67.68) circle (  2.13);

\path[fill=fillColor,fill opacity=0.20] (228.57, 86.89) circle (  2.13);

\path[fill=fillColor,fill opacity=0.20] (208.51, 54.08) circle (  2.13);

\path[fill=fillColor,fill opacity=0.20] (207.50, 51.90) circle (  2.13);

\path[fill=fillColor,fill opacity=0.20] (207.50, 54.39) circle (  2.13);

\path[fill=fillColor,fill opacity=0.20] (210.51, 58.03) circle (  2.13);

\path[fill=fillColor,fill opacity=0.20] (210.51, 60.41) circle (  2.13);

\path[fill=fillColor,fill opacity=0.20] (208.51, 59.27) circle (  2.13);

\path[fill=fillColor,fill opacity=0.20] (210.51, 62.08) circle (  2.13);

\path[fill=fillColor,fill opacity=0.20] (209.51, 57.20) circle (  2.13);

\path[fill=fillColor,fill opacity=0.20] (209.51, 60.00) circle (  2.13);

\path[fill=fillColor,fill opacity=0.20] (218.54, 66.33) circle (  2.13);

\path[fill=fillColor,fill opacity=0.20] (219.54, 65.50) circle (  2.13);

\path[fill=fillColor,fill opacity=0.20] (225.56, 72.98) circle (  2.13);

\path[fill=fillColor,fill opacity=0.20] (204.49, 54.18) circle (  2.13);

\path[fill=fillColor,fill opacity=0.20] (209.51, 53.25) circle (  2.13);

\path[fill=fillColor,fill opacity=0.20] (212.52, 50.97) circle (  2.13);

\path[fill=fillColor,fill opacity=0.20] (206.50, 46.29) circle (  2.13);

\path[fill=fillColor,fill opacity=0.20] (212.52, 58.96) circle (  2.13);

\path[fill=fillColor,fill opacity=0.20] (204.49, 58.55) circle (  2.13);

\path[fill=fillColor,fill opacity=0.20] (206.50, 61.56) circle (  2.13);

\path[fill=fillColor,fill opacity=0.20] (208.51, 64.05) circle (  2.13);

\path[fill=fillColor,fill opacity=0.20] (214.53, 65.81) circle (  2.13);

\path[fill=fillColor,fill opacity=0.20] (220.54, 65.81) circle (  2.13);

\path[fill=fillColor,fill opacity=0.20] (220.54, 65.19) circle (  2.13);

\path[fill=fillColor,fill opacity=0.20] (221.55, 74.74) circle (  2.13);

\path[fill=fillColor,fill opacity=0.20] (214.53, 59.69) circle (  2.13);

\path[fill=fillColor,fill opacity=0.20] (205.50, 57.20) circle (  2.13);

\path[fill=fillColor,fill opacity=0.20] (209.51, 54.50) circle (  2.13);

\path[fill=fillColor,fill opacity=0.20] (215.53, 55.64) circle (  2.13);

\path[fill=fillColor,fill opacity=0.20] (212.52, 53.56) circle (  2.13);

\path[fill=fillColor,fill opacity=0.20] (209.51, 47.33) circle (  2.13);

\path[fill=fillColor,fill opacity=0.20] (210.51, 50.55) circle (  2.13);

\path[fill=fillColor,fill opacity=0.20] (223.55, 58.34) circle (  2.13);

\path[fill=fillColor,fill opacity=0.20] (204.49, 54.50) circle (  2.13);

\path[fill=fillColor,fill opacity=0.20] (209.51, 61.45) circle (  2.13);

\path[fill=fillColor,fill opacity=0.20] (208.51, 67.16) circle (  2.13);

\path[fill=fillColor,fill opacity=0.20] (212.52, 67.89) circle (  2.13);

\path[fill=fillColor,fill opacity=0.20] (215.53, 58.34) circle (  2.13);

\path[fill=fillColor,fill opacity=0.20] (219.54, 56.47) circle (  2.13);

\path[fill=fillColor,fill opacity=0.20] (219.54, 72.04) circle (  2.13);

\path[fill=fillColor,fill opacity=0.20] (220.54, 69.65) circle (  2.13);

\path[fill=fillColor,fill opacity=0.20] (217.53, 62.28) circle (  2.13);

\path[fill=fillColor,fill opacity=0.20] (218.54, 54.39) circle (  2.13);

\path[fill=fillColor,fill opacity=0.20] (216.53, 53.56) circle (  2.13);

\path[fill=fillColor,fill opacity=0.20] (219.54, 56.36) circle (  2.13);

\path[fill=fillColor,fill opacity=0.20] (208.51, 58.13) circle (  2.13);

\path[fill=fillColor,fill opacity=0.20] (216.53, 54.50) circle (  2.13);

\path[fill=fillColor,fill opacity=0.20] (215.53, 53.35) circle (  2.13);

\path[fill=fillColor,fill opacity=0.20] (229.57, 60.73) circle (  2.13);

\path[fill=fillColor,fill opacity=0.20] (213.52, 75.57) circle (  2.13);

\path[fill=fillColor,fill opacity=0.20] (212.52, 56.68) circle (  2.13);

\path[fill=fillColor,fill opacity=0.20] (209.51, 57.51) circle (  2.13);

\path[fill=fillColor,fill opacity=0.20] (212.52, 63.22) circle (  2.13);

\path[fill=fillColor,fill opacity=0.20] (214.53, 69.34) circle (  2.13);

\path[fill=fillColor,fill opacity=0.20] (215.53, 61.14) circle (  2.13);

\path[fill=fillColor,fill opacity=0.20] (215.53, 53.87) circle (  2.13);

\path[fill=fillColor,fill opacity=0.20] (214.53, 70.80) circle (  2.13);

\path[fill=fillColor,fill opacity=0.20] (223.55,100.39) circle (  2.13);

\path[fill=fillColor,fill opacity=0.20] (213.52, 80.24) circle (  2.13);

\path[fill=fillColor,fill opacity=0.20] (217.53, 70.69) circle (  2.13);

\path[fill=fillColor,fill opacity=0.20] (217.53, 60.73) circle (  2.13);

\path[fill=fillColor,fill opacity=0.20] (222.55, 53.67) circle (  2.13);

\path[fill=fillColor,fill opacity=0.20] (221.55, 52.11) circle (  2.13);

\path[fill=fillColor,fill opacity=0.20] (223.55, 57.30) circle (  2.13);

\path[fill=fillColor,fill opacity=0.20] (223.55, 53.77) circle (  2.13);

\path[fill=fillColor,fill opacity=0.20] (215.53, 47.75) circle (  2.13);

\path[fill=fillColor,fill opacity=0.20] (232.58, 58.65) circle (  2.13);

\path[fill=fillColor,fill opacity=0.20] (234.59, 61.35) circle (  2.13);

\path[fill=fillColor,fill opacity=0.20] (213.52, 53.87) circle (  2.13);

\path[fill=fillColor,fill opacity=0.20] (210.51, 59.38) circle (  2.13);

\path[fill=fillColor,fill opacity=0.20] (203.49, 69.24) circle (  2.13);

\path[fill=fillColor,fill opacity=0.20] (217.53, 70.69) circle (  2.13);

\path[fill=fillColor,fill opacity=0.20] (218.54, 61.45) circle (  2.13);

\path[fill=fillColor,fill opacity=0.20] (215.53, 67.16) circle (  2.13);

\path[fill=fillColor,fill opacity=0.20] (218.54, 84.81) circle (  2.13);

\path[fill=fillColor,fill opacity=0.20] (210.51, 79.00) circle (  2.13);

\path[fill=fillColor,fill opacity=0.20] (214.53, 70.80) circle (  2.13);

\path[fill=fillColor,fill opacity=0.20] (214.53, 60.83) circle (  2.13);

\path[fill=fillColor,fill opacity=0.20] (220.54, 52.52) circle (  2.13);

\path[fill=fillColor,fill opacity=0.20] (216.53, 49.72) circle (  2.13);

\path[fill=fillColor,fill opacity=0.20] (217.53, 55.64) circle (  2.13);

\path[fill=fillColor,fill opacity=0.20] (222.55, 54.08) circle (  2.13);

\path[fill=fillColor,fill opacity=0.20] (218.54, 47.23) circle (  2.13);

\path[fill=fillColor,fill opacity=0.20] (224.56, 57.71) circle (  2.13);

\path[fill=fillColor,fill opacity=0.20] (269.70, 60.31) circle (  2.13);

\path[fill=fillColor,fill opacity=0.20] (239.61, 54.91) circle (  2.13);

\path[fill=fillColor,fill opacity=0.20] (218.54, 58.86) circle (  2.13);

\path[fill=fillColor,fill opacity=0.20] (199.48, 59.06) circle (  2.13);

\path[fill=fillColor,fill opacity=0.20] (216.53, 65.50) circle (  2.13);

\path[fill=fillColor,fill opacity=0.20] (218.54, 65.50) circle (  2.13);

\path[fill=fillColor,fill opacity=0.20] (213.52, 59.27) circle (  2.13);

\path[fill=fillColor,fill opacity=0.20] (217.53, 66.75) circle (  2.13);

\path[fill=fillColor,fill opacity=0.20] (222.55,100.39) circle (  2.13);

\path[fill=fillColor,fill opacity=0.20] (216.53, 69.86) circle (  2.13);

\path[fill=fillColor,fill opacity=0.20] (210.51, 59.17) circle (  2.13);

\path[fill=fillColor,fill opacity=0.20] (209.51, 55.43) circle (  2.13);

\path[fill=fillColor,fill opacity=0.20] (217.53, 53.25) circle (  2.13);

\path[fill=fillColor,fill opacity=0.20] (217.53, 57.40) circle (  2.13);

\path[fill=fillColor,fill opacity=0.20] (212.52, 60.10) circle (  2.13);

\path[fill=fillColor,fill opacity=0.20] (218.54, 56.68) circle (  2.13);

\path[fill=fillColor,fill opacity=0.20] (231.58, 60.10) circle (  2.13);

\path[fill=fillColor,fill opacity=0.20] (235.59, 54.60) circle (  2.13);

\path[fill=fillColor,fill opacity=0.20] (217.53, 44.53) circle (  2.13);

\path[fill=fillColor,fill opacity=0.20] (214.53, 51.17) circle (  2.13);

\path[fill=fillColor,fill opacity=0.20] (211.52, 60.83) circle (  2.13);

\path[fill=fillColor,fill opacity=0.20] (216.53, 57.82) circle (  2.13);

\path[fill=fillColor,fill opacity=0.20] (215.53, 61.45) circle (  2.13);

\path[fill=fillColor,fill opacity=0.20] (215.53, 80.87) circle (  2.13);

\path[fill=fillColor,fill opacity=0.20] (209.51, 86.89) circle (  2.13);

\path[fill=fillColor,fill opacity=0.20] (207.50, 71.73) circle (  2.13);

\path[fill=fillColor,fill opacity=0.20] (206.50, 57.71) circle (  2.13);

\path[fill=fillColor,fill opacity=0.20] (207.50, 58.86) circle (  2.13);

\path[fill=fillColor,fill opacity=0.20] (213.52, 56.57) circle (  2.13);

\path[fill=fillColor,fill opacity=0.20] (210.51, 52.00) circle (  2.13);

\path[fill=fillColor,fill opacity=0.20] (210.51, 57.20) circle (  2.13);

\path[fill=fillColor,fill opacity=0.20] (213.52, 60.10) circle (  2.13);

\path[fill=fillColor,fill opacity=0.20] (216.53, 55.53) circle (  2.13);

\path[fill=fillColor,fill opacity=0.20] (226.56, 59.48) circle (  2.13);

\path[fill=fillColor,fill opacity=0.20] (264.69, 74.01) circle (  2.13);

\path[fill=fillColor,fill opacity=0.20] (232.58, 45.88) circle (  2.13);

\path[fill=fillColor,fill opacity=0.20] (212.52, 48.58) circle (  2.13);

\path[fill=fillColor,fill opacity=0.20] (205.50, 57.92) circle (  2.13);

\path[fill=fillColor,fill opacity=0.20] (210.51, 59.69) circle (  2.13);

\path[fill=fillColor,fill opacity=0.20] (214.53, 60.62) circle (  2.13);

\path[fill=fillColor,fill opacity=0.20] (213.52, 65.50) circle (  2.13);

\path[fill=fillColor,fill opacity=0.20] (214.53, 78.17) circle (  2.13);

\path[fill=fillColor,fill opacity=0.20] (200.48, 96.23) circle (  2.13);

\path[fill=fillColor,fill opacity=0.20] (202.49, 79.41) circle (  2.13);

\path[fill=fillColor,fill opacity=0.20] (202.49, 67.06) circle (  2.13);

\path[fill=fillColor,fill opacity=0.20] (204.49, 63.42) circle (  2.13);

\path[fill=fillColor,fill opacity=0.20] (204.49, 56.16) circle (  2.13);

\path[fill=fillColor,fill opacity=0.20] (208.51, 50.76) circle (  2.13);

\path[fill=fillColor,fill opacity=0.20] (211.52, 52.63) circle (  2.13);

\path[fill=fillColor,fill opacity=0.20] (217.53, 49.72) circle (  2.13);

\path[fill=fillColor,fill opacity=0.20] (217.53, 43.49) circle (  2.13);

\path[fill=fillColor,fill opacity=0.20] (225.56, 54.60) circle (  2.13);

\path[fill=fillColor,fill opacity=0.20] (244.62, 64.67) circle (  2.13);

\path[fill=fillColor,fill opacity=0.20] (218.54, 56.26) circle (  2.13);

\path[fill=fillColor,fill opacity=0.20] (212.52, 55.53) circle (  2.13);

\path[fill=fillColor,fill opacity=0.20] (210.51, 55.85) circle (  2.13);

\path[fill=fillColor,fill opacity=0.20] (211.52, 55.43) circle (  2.13);

\path[fill=fillColor,fill opacity=0.20] (211.52, 59.27) circle (  2.13);

\path[fill=fillColor,fill opacity=0.20] (212.52, 69.76) circle (  2.13);

\path[fill=fillColor,fill opacity=0.20] (215.53, 76.61) circle (  2.13);

\path[fill=fillColor,fill opacity=0.20] (221.55,105.58) circle (  2.13);

\path[fill=fillColor,fill opacity=0.20] (201.48, 92.08) circle (  2.13);

\path[fill=fillColor,fill opacity=0.20] (200.48, 84.81) circle (  2.13);

\path[fill=fillColor,fill opacity=0.20] (202.49, 69.86) circle (  2.13);

\path[fill=fillColor,fill opacity=0.20] (202.49, 63.74) circle (  2.13);

\path[fill=fillColor,fill opacity=0.20] (203.49, 60.41) circle (  2.13);

\path[fill=fillColor,fill opacity=0.20] (201.48, 56.16) circle (  2.13);

\path[fill=fillColor,fill opacity=0.20] (202.49, 54.81) circle (  2.13);

\path[fill=fillColor,fill opacity=0.20] (204.49, 52.94) circle (  2.13);

\path[fill=fillColor,fill opacity=0.20] (211.52, 45.98) circle (  2.13);

\path[fill=fillColor,fill opacity=0.20] (219.54, 41.10) circle (  2.13);

\path[fill=fillColor,fill opacity=0.20] (241.61, 49.10) circle (  2.13);

\path[fill=fillColor,fill opacity=0.20] (266.69, 66.75) circle (  2.13);

\path[fill=fillColor,fill opacity=0.20] (251.64, 57.71) circle (  2.13);

\path[fill=fillColor,fill opacity=0.20] (222.55, 52.94) circle (  2.13);

\path[fill=fillColor,fill opacity=0.20] (212.52, 53.98) circle (  2.13);

\path[fill=fillColor,fill opacity=0.20] (212.52, 60.93) circle (  2.13);

\path[fill=fillColor,fill opacity=0.20] (212.52, 70.69) circle (  2.13);

\path[fill=fillColor,fill opacity=0.20] (213.52, 72.04) circle (  2.13);

\path[fill=fillColor,fill opacity=0.20] (218.54, 75.47) circle (  2.13);

\path[fill=fillColor,fill opacity=0.20] (200.48, 83.77) circle (  2.13);

\path[fill=fillColor,fill opacity=0.20] (197.47, 73.70) circle (  2.13);

\path[fill=fillColor,fill opacity=0.20] (199.48, 64.67) circle (  2.13);

\path[fill=fillColor,fill opacity=0.20] (203.49, 53.67) circle (  2.13);

\path[fill=fillColor,fill opacity=0.20] (203.49, 55.12) circle (  2.13);

\path[fill=fillColor,fill opacity=0.20] (203.49, 60.73) circle (  2.13);

\path[fill=fillColor,fill opacity=0.20] (207.50, 56.99) circle (  2.13);

\path[fill=fillColor,fill opacity=0.20] (206.50, 55.53) circle (  2.13);

\path[fill=fillColor,fill opacity=0.20] (207.50, 57.51) circle (  2.13);

\path[fill=fillColor,fill opacity=0.20] (209.51, 56.68) circle (  2.13);

\path[fill=fillColor,fill opacity=0.20] (215.53, 52.21) circle (  2.13);

\path[fill=fillColor,fill opacity=0.20] (251.64, 52.52) circle (  2.13);

\path[fill=fillColor,fill opacity=0.20] (266.69, 74.22) circle (  2.13);

\path[fill=fillColor,fill opacity=0.20] (239.61, 62.59) circle (  2.13);

\path[fill=fillColor,fill opacity=0.20] (226.56, 56.57) circle (  2.13);

\path[fill=fillColor,fill opacity=0.20] (216.53, 60.31) circle (  2.13);

\path[fill=fillColor,fill opacity=0.20] (212.52, 64.77) circle (  2.13);

\path[fill=fillColor,fill opacity=0.20] (213.52, 71.32) circle (  2.13);

\path[fill=fillColor,fill opacity=0.20] (212.52, 71.94) circle (  2.13);

\path[fill=fillColor,fill opacity=0.20] (212.52, 76.40) circle (  2.13);

\path[fill=fillColor,fill opacity=0.20] (196.47, 83.77) circle (  2.13);

\path[fill=fillColor,fill opacity=0.20] (194.46, 70.28) circle (  2.13);

\path[fill=fillColor,fill opacity=0.20] (189.44, 66.02) circle (  2.13);

\path[fill=fillColor,fill opacity=0.20] (198.47, 61.24) circle (  2.13);

\path[fill=fillColor,fill opacity=0.20] (202.49, 53.98) circle (  2.13);

\path[fill=fillColor,fill opacity=0.20] (200.48, 59.69) circle (  2.13);

\path[fill=fillColor,fill opacity=0.20] (204.49, 66.85) circle (  2.13);

\path[fill=fillColor,fill opacity=0.20] (210.51, 60.10) circle (  2.13);

\path[fill=fillColor,fill opacity=0.20] (208.51, 56.36) circle (  2.13);

\path[fill=fillColor,fill opacity=0.20] (206.50, 59.58) circle (  2.13);

\path[fill=fillColor,fill opacity=0.20] (212.52, 60.21) circle (  2.13);

\path[fill=fillColor,fill opacity=0.20] (247.63, 62.28) circle (  2.13);

\path[fill=fillColor,fill opacity=0.20] (227.57, 59.58) circle (  2.13);

\path[fill=fillColor,fill opacity=0.20] (207.50, 60.00) circle (  2.13);

\path[fill=fillColor,fill opacity=0.20] (208.51, 66.54) circle (  2.13);

\path[fill=fillColor,fill opacity=0.20] (211.52, 74.74) circle (  2.13);

\path[fill=fillColor,fill opacity=0.20] (217.53, 73.29) circle (  2.13);

\path[fill=fillColor,fill opacity=0.20] (216.53, 80.14) circle (  2.13);

\path[fill=fillColor,fill opacity=0.20] (195.46, 91.04) circle (  2.13);

\path[fill=fillColor,fill opacity=0.20] (192.45, 73.81) circle (  2.13);

\path[fill=fillColor,fill opacity=0.20] (192.45, 70.38) circle (  2.13);

\path[fill=fillColor,fill opacity=0.20] (187.94, 69.03) circle (  2.13);

\path[fill=fillColor,fill opacity=0.20] (198.47, 63.32) circle (  2.13);

\path[fill=fillColor,fill opacity=0.20] (201.48, 60.62) circle (  2.13);

\path[fill=fillColor,fill opacity=0.20] (199.48, 64.26) circle (  2.13);

\path[fill=fillColor,fill opacity=0.20] (205.50, 63.22) circle (  2.13);

\path[fill=fillColor,fill opacity=0.20] (211.52, 59.48) circle (  2.13);

\path[fill=fillColor,fill opacity=0.20] (213.52, 62.59) circle (  2.13);

\path[fill=fillColor,fill opacity=0.20] (215.53, 60.41) circle (  2.13);

\path[fill=fillColor,fill opacity=0.20] (230.58, 55.64) circle (  2.13);

\path[fill=fillColor,fill opacity=0.20] (246.63, 61.66) circle (  2.13);

\path[fill=fillColor,fill opacity=0.20] (235.59, 58.96) circle (  2.13);

\path[fill=fillColor,fill opacity=0.20] (215.53, 58.13) circle (  2.13);

\path[fill=fillColor,fill opacity=0.20] (217.53, 68.82) circle (  2.13);

\path[fill=fillColor,fill opacity=0.20] (210.51, 72.66) circle (  2.13);

\path[fill=fillColor,fill opacity=0.20] (210.51, 67.27) circle (  2.13);

\path[fill=fillColor,fill opacity=0.20] (211.52, 78.89) circle (  2.13);

\path[fill=fillColor,fill opacity=0.20] (193.46,110.77) circle (  2.13);

\path[fill=fillColor,fill opacity=0.20] (197.47, 83.77) circle (  2.13);

\path[fill=fillColor,fill opacity=0.20] (195.46, 68.10) circle (  2.13);

\path[fill=fillColor,fill opacity=0.20] (191.45, 65.40) circle (  2.13);

\path[fill=fillColor,fill opacity=0.20] (192.45, 65.92) circle (  2.13);

\path[fill=fillColor,fill opacity=0.20] (196.47, 64.98) circle (  2.13);

\path[fill=fillColor,fill opacity=0.20] (201.48, 59.17) circle (  2.13);

\path[fill=fillColor,fill opacity=0.20] (199.48, 61.24) circle (  2.13);

\path[fill=fillColor,fill opacity=0.20] (201.48, 64.77) circle (  2.13);

\path[fill=fillColor,fill opacity=0.20] (207.50, 54.39) circle (  2.13);

\path[fill=fillColor,fill opacity=0.20] (220.54, 51.59) circle (  2.13);

\path[fill=fillColor,fill opacity=0.20] (222.55, 61.76) circle (  2.13);

\path[fill=fillColor,fill opacity=0.20] (255.66, 63.01) circle (  2.13);

\path[fill=fillColor,fill opacity=0.20] (240.61, 56.16) circle (  2.13);

\path[fill=fillColor,fill opacity=0.20] (240.61, 54.70) circle (  2.13);

\path[fill=fillColor,fill opacity=0.20] (222.55, 59.69) circle (  2.13);

\path[fill=fillColor,fill opacity=0.20] (212.52, 64.36) circle (  2.13);

\path[fill=fillColor,fill opacity=0.20] (211.52, 64.98) circle (  2.13);

\path[fill=fillColor,fill opacity=0.20] (207.50, 65.92) circle (  2.13);

\path[fill=fillColor,fill opacity=0.20] (204.49, 69.86) circle (  2.13);

\path[fill=fillColor,fill opacity=0.20] (211.52, 90.00) circle (  2.13);

\path[fill=fillColor,fill opacity=0.20] (193.46,108.69) circle (  2.13);

\path[fill=fillColor,fill opacity=0.20] (191.45, 88.96) circle (  2.13);

\path[fill=fillColor,fill opacity=0.20] (191.45, 78.58) circle (  2.13);

\path[fill=fillColor,fill opacity=0.20] (195.46, 61.04) circle (  2.13);

\path[fill=fillColor,fill opacity=0.20] (195.46, 56.05) circle (  2.13);

\path[fill=fillColor,fill opacity=0.20] (192.45, 62.59) circle (  2.13);

\path[fill=fillColor,fill opacity=0.20] (194.46, 58.55) circle (  2.13);

\path[fill=fillColor,fill opacity=0.20] (187.64, 56.68) circle (  2.13);

\path[fill=fillColor,fill opacity=0.20] (199.48, 54.91) circle (  2.13);

\path[fill=fillColor,fill opacity=0.20] (200.48, 57.61) circle (  2.13);

\path[fill=fillColor,fill opacity=0.20] (209.51, 63.01) circle (  2.13);

\path[fill=fillColor,fill opacity=0.20] (218.54, 52.00) circle (  2.13);

\path[fill=fillColor,fill opacity=0.20] (236.60, 45.05) circle (  2.13);

\path[fill=fillColor,fill opacity=0.20] (258.67, 57.71) circle (  2.13);

\path[fill=fillColor,fill opacity=0.20] (238.60, 66.85) circle (  2.13);

\path[fill=fillColor,fill opacity=0.20] (251.64, 55.02) circle (  2.13);

\path[fill=fillColor,fill opacity=0.20] (224.56, 51.38) circle (  2.13);

\path[fill=fillColor,fill opacity=0.20] (212.52, 54.70) circle (  2.13);

\path[fill=fillColor,fill opacity=0.20] (210.51, 60.00) circle (  2.13);

\path[fill=fillColor,fill opacity=0.20] (208.51, 60.93) circle (  2.13);

\path[fill=fillColor,fill opacity=0.20] (207.50, 65.71) circle (  2.13);

\path[fill=fillColor,fill opacity=0.20] (212.52, 82.74) circle (  2.13);

\path[fill=fillColor,fill opacity=0.20] (189.44, 92.08) circle (  2.13);

\path[fill=fillColor,fill opacity=0.20] (187.54, 76.40) circle (  2.13);

\path[fill=fillColor,fill opacity=0.20] (184.33, 72.35) circle (  2.13);

\path[fill=fillColor,fill opacity=0.20] (195.46, 69.24) circle (  2.13);

\path[fill=fillColor,fill opacity=0.20] (194.46, 62.49) circle (  2.13);

\path[fill=fillColor,fill opacity=0.20] (196.47, 62.59) circle (  2.13);

\path[fill=fillColor,fill opacity=0.20] (195.46, 62.18) circle (  2.13);

\path[fill=fillColor,fill opacity=0.20] (194.46, 60.62) circle (  2.13);

\path[fill=fillColor,fill opacity=0.20] (200.48, 60.83) circle (  2.13);

\path[fill=fillColor,fill opacity=0.20] (207.50, 53.98) circle (  2.13);

\path[fill=fillColor,fill opacity=0.20] (219.54, 54.18) circle (  2.13);

\path[fill=fillColor,fill opacity=0.20] (227.57, 60.73) circle (  2.13);

\path[fill=fillColor,fill opacity=0.20] (242.62, 57.30) circle (  2.13);

\path[fill=fillColor,fill opacity=0.20] (248.64, 56.16) circle (  2.13);

\path[fill=fillColor,fill opacity=0.20] (233.59, 68.20) circle (  2.13);

\path[fill=fillColor,fill opacity=0.20] (255.66, 47.96) circle (  2.13);

\path[fill=fillColor,fill opacity=0.20] (216.53, 49.31) circle (  2.13);

\path[fill=fillColor,fill opacity=0.20] (209.51, 61.04) circle (  2.13);

\path[fill=fillColor,fill opacity=0.20] (207.50, 73.50) circle (  2.13);

\path[fill=fillColor,fill opacity=0.20] (209.51, 74.95) circle (  2.13);

\path[fill=fillColor,fill opacity=0.20] (211.52, 70.90) circle (  2.13);

\path[fill=fillColor,fill opacity=0.20] (207.50, 74.74) circle (  2.13);

\path[fill=fillColor,fill opacity=0.20] (210.51, 95.19) circle (  2.13);

\path[fill=fillColor,fill opacity=0.20] (190.45, 99.35) circle (  2.13);

\path[fill=fillColor,fill opacity=0.20] (191.45, 74.95) circle (  2.13);

\path[fill=fillColor,fill opacity=0.20] (189.44, 69.03) circle (  2.13);

\path[fill=fillColor,fill opacity=0.20] (177.51, 70.69) circle (  2.13);

\path[fill=fillColor,fill opacity=0.20] (194.46, 67.47) circle (  2.13);

\path[fill=fillColor,fill opacity=0.20] (195.46, 64.15) circle (  2.13);

\path[fill=fillColor,fill opacity=0.20] (191.45, 63.94) circle (  2.13);

\path[fill=fillColor,fill opacity=0.20] (192.45, 62.70) circle (  2.13);

\path[fill=fillColor,fill opacity=0.20] (195.46, 64.98) circle (  2.13);

\path[fill=fillColor,fill opacity=0.20] (204.49, 66.33) circle (  2.13);

\path[fill=fillColor,fill opacity=0.20] (220.54, 62.18) circle (  2.13);

\path[fill=fillColor,fill opacity=0.20] (232.58, 58.96) circle (  2.13);

\path[fill=fillColor,fill opacity=0.20] (258.67, 65.40) circle (  2.13);

\path[fill=fillColor,fill opacity=0.20] (243.62, 70.28) circle (  2.13);

\path[fill=fillColor,fill opacity=0.20] (234.59, 66.02) circle (  2.13);

\path[fill=fillColor,fill opacity=0.20] (222.55, 64.15) circle (  2.13);

\path[fill=fillColor,fill opacity=0.20] (216.53, 68.72) circle (  2.13);

\path[fill=fillColor,fill opacity=0.20] (211.52, 69.13) circle (  2.13);

\path[fill=fillColor,fill opacity=0.20] (209.51, 68.62) circle (  2.13);

\path[fill=fillColor,fill opacity=0.20] (209.51, 66.85) circle (  2.13);

\path[fill=fillColor,fill opacity=0.20] (207.50, 66.33) circle (  2.13);

\path[fill=fillColor,fill opacity=0.20] (212.52, 79.31) circle (  2.13);

\path[fill=fillColor,fill opacity=0.20] (216.53, 99.35) circle (  2.13);

\path[fill=fillColor,fill opacity=0.20] (190.45,101.42) circle (  2.13);

\path[fill=fillColor,fill opacity=0.20] (188.04, 86.89) circle (  2.13);

\path[fill=fillColor,fill opacity=0.20] (191.45, 79.31) circle (  2.13);

\path[fill=fillColor,fill opacity=0.20] (195.46, 70.07) circle (  2.13);

\path[fill=fillColor,fill opacity=0.20] (195.46, 64.46) circle (  2.13);

\path[fill=fillColor,fill opacity=0.20] (192.45, 66.12) circle (  2.13);

\path[fill=fillColor,fill opacity=0.20] (196.47, 61.56) circle (  2.13);

\path[fill=fillColor,fill opacity=0.20] (203.49, 52.94) circle (  2.13);

\path[fill=fillColor,fill opacity=0.20] (208.51, 52.73) circle (  2.13);

\path[fill=fillColor,fill opacity=0.20] (209.51, 59.06) circle (  2.13);

\path[fill=fillColor,fill opacity=0.20] (213.52, 65.40) circle (  2.13);

\path[fill=fillColor,fill opacity=0.20] (209.51, 67.37) circle (  2.13);

\path[fill=fillColor,fill opacity=0.20] (233.59, 65.50) circle (  2.13);

\path[fill=fillColor,fill opacity=0.20] (247.63, 70.07) circle (  2.13);

\path[fill=fillColor,fill opacity=0.20] (217.53, 53.98) circle (  2.13);

\path[fill=fillColor,fill opacity=0.20] (223.55, 50.14) circle (  2.13);

\path[fill=fillColor,fill opacity=0.20] (212.52, 56.68) circle (  2.13);

\path[fill=fillColor,fill opacity=0.20] (208.51, 69.55) circle (  2.13);

\path[fill=fillColor,fill opacity=0.20] (206.50, 74.33) circle (  2.13);

\path[fill=fillColor,fill opacity=0.20] (209.51, 74.22) circle (  2.13);

\path[fill=fillColor,fill opacity=0.20] (214.53, 77.23) circle (  2.13);

\path[fill=fillColor,fill opacity=0.20] (213.52, 81.70) circle (  2.13);

\path[fill=fillColor,fill opacity=0.20] (210.51, 91.04) circle (  2.13);

\path[fill=fillColor,fill opacity=0.20] (193.46, 95.19) circle (  2.13);

\path[fill=fillColor,fill opacity=0.20] (190.45, 77.23) circle (  2.13);

\path[fill=fillColor,fill opacity=0.20] (187.44, 74.12) circle (  2.13);

\path[fill=fillColor,fill opacity=0.20] (193.46, 75.99) circle (  2.13);

\path[fill=fillColor,fill opacity=0.20] (201.48, 71.94) circle (  2.13);

\path[fill=fillColor,fill opacity=0.20] (207.50, 62.70) circle (  2.13);

\path[fill=fillColor,fill opacity=0.20] (196.47, 58.55) circle (  2.13);

\path[fill=fillColor,fill opacity=0.20] (211.52, 54.39) circle (  2.13);

\path[fill=fillColor,fill opacity=0.20] (226.56, 49.51) circle (  2.13);

\path[fill=fillColor,fill opacity=0.20] (243.62, 55.74) circle (  2.13);

\path[fill=fillColor,fill opacity=0.20] (249.64, 75.26) circle (  2.13);

\path[fill=fillColor,fill opacity=0.20] (230.58, 76.09) circle (  2.13);

\path[fill=fillColor,fill opacity=0.20] (259.67, 76.51) circle (  2.13);

\path[fill=fillColor,fill opacity=0.20] (269.70, 60.21) circle (  2.13);

\path[fill=fillColor,fill opacity=0.20] (238.60, 51.38) circle (  2.13);

\path[fill=fillColor,fill opacity=0.20] (224.56, 50.86) circle (  2.13);

\path[fill=fillColor,fill opacity=0.20] (211.52, 63.94) circle (  2.13);

\path[fill=fillColor,fill opacity=0.20] (207.50, 75.26) circle (  2.13);

\path[fill=fillColor,fill opacity=0.20] (214.53, 66.95) circle (  2.13);

\path[fill=fillColor,fill opacity=0.20] (214.53, 66.95) circle (  2.13);

\path[fill=fillColor,fill opacity=0.20] (214.53, 76.30) circle (  2.13);

\path[fill=fillColor,fill opacity=0.20] (212.52, 79.21) circle (  2.13);

\path[fill=fillColor,fill opacity=0.20] (210.51, 81.49) circle (  2.13);

\path[fill=fillColor,fill opacity=0.20] (207.50, 93.12) circle (  2.13);

\path[fill=fillColor,fill opacity=0.20] (197.47,113.88) circle (  2.13);

\path[fill=fillColor,fill opacity=0.20] (196.47, 94.16) circle (  2.13);

\path[fill=fillColor,fill opacity=0.20] (197.47, 91.04) circle (  2.13);

\path[fill=fillColor,fill opacity=0.20] (198.47, 84.81) circle (  2.13);

\path[fill=fillColor,fill opacity=0.20] (198.47, 73.18) circle (  2.13);

\path[fill=fillColor,fill opacity=0.20] (199.48, 66.33) circle (  2.13);

\path[fill=fillColor,fill opacity=0.20] (214.53, 67.79) circle (  2.13);

\path[fill=fillColor,fill opacity=0.20] (213.52, 66.23) circle (  2.13);

\path[fill=fillColor,fill opacity=0.20] (236.60, 61.24) circle (  2.13);

\path[fill=fillColor,fill opacity=0.20] (245.63, 61.35) circle (  2.13);

\path[fill=fillColor,fill opacity=0.20] (237.60, 65.19) circle (  2.13);

\path[fill=fillColor,fill opacity=0.20] (237.60, 69.03) circle (  2.13);

\path[fill=fillColor,fill opacity=0.20] (219.54, 75.16) circle (  2.13);

\path[fill=fillColor,fill opacity=0.20] (265.69, 69.45) circle (  2.13);

\path[fill=fillColor,fill opacity=0.20] (233.59, 55.22) circle (  2.13);

\path[fill=fillColor,fill opacity=0.20] (221.55, 55.33) circle (  2.13);

\path[fill=fillColor,fill opacity=0.20] (212.52, 48.79) circle (  2.13);

\path[fill=fillColor,fill opacity=0.20] (210.51, 53.77) circle (  2.13);

\path[fill=fillColor,fill opacity=0.20] (213.52, 70.59) circle (  2.13);

\path[fill=fillColor,fill opacity=0.20] (212.52, 76.09) circle (  2.13);

\path[fill=fillColor,fill opacity=0.20] (211.52, 73.50) circle (  2.13);

\path[fill=fillColor,fill opacity=0.20] (217.53, 75.78) circle (  2.13);

\path[fill=fillColor,fill opacity=0.20] (218.54, 78.27) circle (  2.13);

\path[fill=fillColor,fill opacity=0.20] (215.53, 87.93) circle (  2.13);

\path[fill=fillColor,fill opacity=0.20] (194.46,104.54) circle (  2.13);

\path[fill=fillColor,fill opacity=0.20] (196.47, 92.08) circle (  2.13);

\path[fill=fillColor,fill opacity=0.20] (199.48, 79.83) circle (  2.13);

\path[fill=fillColor,fill opacity=0.20] (204.49, 71.63) circle (  2.13);

\path[fill=fillColor,fill opacity=0.20] (206.50, 65.81) circle (  2.13);

\path[fill=fillColor,fill opacity=0.20] (208.51, 68.62) circle (  2.13);

\path[fill=fillColor,fill opacity=0.20] (219.54, 74.01) circle (  2.13);

\path[fill=fillColor,fill opacity=0.20] (224.56, 69.65) circle (  2.13);

\path[fill=fillColor,fill opacity=0.20] (227.57, 64.46) circle (  2.13);

\path[fill=fillColor,fill opacity=0.20] (235.59, 66.54) circle (  2.13);

\path[fill=fillColor,fill opacity=0.20] (242.62, 65.81) circle (  2.13);

\path[fill=fillColor,fill opacity=0.20] (231.58, 64.15) circle (  2.13);

\path[fill=fillColor,fill opacity=0.20] (243.62, 72.77) circle (  2.13);

\path[fill=fillColor,fill opacity=0.20] (264.69, 74.64) circle (  2.13);

\path[fill=fillColor,fill opacity=0.20] (257.66, 66.02) circle (  2.13);

\path[fill=fillColor,fill opacity=0.20] (255.66, 57.71) circle (  2.13);

\path[fill=fillColor,fill opacity=0.20] (209.51, 52.21) circle (  2.13);

\path[fill=fillColor,fill opacity=0.20] (224.56, 55.53) circle (  2.13);

\path[fill=fillColor,fill opacity=0.20] (214.53, 63.84) circle (  2.13);

\path[fill=fillColor,fill opacity=0.20] (210.51, 63.53) circle (  2.13);

\path[fill=fillColor,fill opacity=0.20] (213.52, 64.98) circle (  2.13);

\path[fill=fillColor,fill opacity=0.20] (216.53, 72.87) circle (  2.13);

\path[fill=fillColor,fill opacity=0.20] (218.54, 70.69) circle (  2.13);

\path[fill=fillColor,fill opacity=0.20] (218.54, 63.63) circle (  2.13);

\path[fill=fillColor,fill opacity=0.20] (219.54, 69.65) circle (  2.13);

\path[fill=fillColor,fill opacity=0.20] (223.55, 78.27) circle (  2.13);

\path[fill=fillColor,fill opacity=0.20] (227.57, 79.21) circle (  2.13);

\path[fill=fillColor,fill opacity=0.20] (216.53, 83.77) circle (  2.13);

\path[fill=fillColor,fill opacity=0.20] (208.51, 91.04) circle (  2.13);

\path[fill=fillColor,fill opacity=0.20] (223.55, 91.04) circle (  2.13);

\path[fill=fillColor,fill opacity=0.20] (213.52, 87.93) circle (  2.13);

\path[fill=fillColor,fill opacity=0.20] (216.53, 88.96) circle (  2.13);

\path[fill=fillColor,fill opacity=0.20] (216.53, 92.08) circle (  2.13);

\path[fill=fillColor,fill opacity=0.20] (217.53, 95.19) circle (  2.13);

\path[fill=fillColor,fill opacity=0.20] (221.55, 94.16) circle (  2.13);

\path[fill=fillColor,fill opacity=0.20] (204.49, 91.04) circle (  2.13);

\path[fill=fillColor,fill opacity=0.20] (203.49, 88.96) circle (  2.13);

\path[fill=fillColor,fill opacity=0.20] (198.47, 86.89) circle (  2.13);

\path[fill=fillColor,fill opacity=0.20] (201.48, 85.85) circle (  2.13);

\path[fill=fillColor,fill opacity=0.20] (201.48, 78.38) circle (  2.13);

\path[fill=fillColor,fill opacity=0.20] (198.47, 67.58) circle (  2.13);

\path[fill=fillColor,fill opacity=0.20] (199.48, 67.99) circle (  2.13);

\path[fill=fillColor,fill opacity=0.20] (194.46, 73.18) circle (  2.13);

\path[fill=fillColor,fill opacity=0.20] (213.52, 69.55) circle (  2.13);

\path[fill=fillColor,fill opacity=0.20] (217.53, 65.61) circle (  2.13);

\path[fill=fillColor,fill opacity=0.20] (231.58, 68.82) circle (  2.13);

\path[fill=fillColor,fill opacity=0.20] (233.59, 70.80) circle (  2.13);

\path[fill=fillColor,fill opacity=0.20] (223.55, 72.77) circle (  2.13);

\path[fill=fillColor,fill opacity=0.20] (271.71, 76.51) circle (  2.13);

\path[fill=fillColor,fill opacity=0.20] (247.63, 79.62) circle (  2.13);

\path[fill=fillColor,fill opacity=0.20] (233.59, 79.83) circle (  2.13);

\path[fill=fillColor,fill opacity=0.20] (231.58, 75.78) circle (  2.13);

\path[fill=fillColor,fill opacity=0.20] (244.62, 81.28) circle (  2.13);

\path[fill=fillColor,fill opacity=0.20] (251.64, 72.66) circle (  2.13);

\path[fill=fillColor,fill opacity=0.20] (237.60, 67.58) circle (  2.13);

\path[fill=fillColor,fill opacity=0.20] (217.53, 62.49) circle (  2.13);

\path[fill=fillColor,fill opacity=0.20] (218.54, 59.79) circle (  2.13);

\path[fill=fillColor,fill opacity=0.20] (218.54, 63.42) circle (  2.13);

\path[fill=fillColor,fill opacity=0.20] (219.54, 59.58) circle (  2.13);

\path[fill=fillColor,fill opacity=0.20] (217.53, 55.53) circle (  2.13);

\path[fill=fillColor,fill opacity=0.20] (210.51, 61.97) circle (  2.13);

\path[fill=fillColor,fill opacity=0.20] (218.54, 65.40) circle (  2.13);

\path[fill=fillColor,fill opacity=0.20] (220.54, 64.26) circle (  2.13);

\path[fill=fillColor,fill opacity=0.20] (218.54, 65.29) circle (  2.13);

\path[fill=fillColor,fill opacity=0.20] (215.53, 70.28) circle (  2.13);

\path[fill=fillColor,fill opacity=0.20] (224.56, 73.50) circle (  2.13);

\path[fill=fillColor,fill opacity=0.20] (219.54, 72.98) circle (  2.13);

\path[fill=fillColor,fill opacity=0.20] (213.52, 72.87) circle (  2.13);

\path[fill=fillColor,fill opacity=0.20] (215.53, 72.87) circle (  2.13);

\path[fill=fillColor,fill opacity=0.20] (209.51, 70.80) circle (  2.13);

\path[fill=fillColor,fill opacity=0.20] (210.51, 70.17) circle (  2.13);

\path[fill=fillColor,fill opacity=0.20] (213.52, 67.79) circle (  2.13);

\path[fill=fillColor,fill opacity=0.20] (214.53, 67.16) circle (  2.13);

\path[fill=fillColor,fill opacity=0.20] (204.49, 71.42) circle (  2.13);

\path[fill=fillColor,fill opacity=0.20] (204.49, 75.88) circle (  2.13);

\path[fill=fillColor,fill opacity=0.20] (202.49, 76.82) circle (  2.13);

\path[fill=fillColor,fill opacity=0.20] (201.48, 70.48) circle (  2.13);

\path[fill=fillColor,fill opacity=0.20] (201.48, 61.04) circle (  2.13);

\path[fill=fillColor,fill opacity=0.20] (209.51, 55.95) circle (  2.13);

\path[fill=fillColor,fill opacity=0.20] (215.53, 54.08) circle (  2.13);

\path[fill=fillColor,fill opacity=0.20] (218.54, 60.21) circle (  2.13);

\path[fill=fillColor,fill opacity=0.20] (237.60, 70.59) circle (  2.13);

\path[fill=fillColor,fill opacity=0.20] (244.62, 73.60) circle (  2.13);

\path[fill=fillColor,fill opacity=0.20] (247.63, 77.13) circle (  2.13);

\path[fill=fillColor,fill opacity=0.20] (262.68, 88.96) circle (  2.13);

\path[fill=fillColor,fill opacity=0.20] (249.64, 99.35) circle (  2.13);

\path[fill=fillColor,fill opacity=0.20] (243.62, 77.44) circle (  2.13);

\path[fill=fillColor,fill opacity=0.20] (246.63, 79.10) circle (  2.13);

\path[fill=fillColor,fill opacity=0.20] (246.63, 74.74) circle (  2.13);

\path[fill=fillColor,fill opacity=0.20] (263.68, 62.70) circle (  2.13);

\path[fill=fillColor,fill opacity=0.20] (233.59, 52.73) circle (  2.13);

\path[fill=fillColor,fill opacity=0.20] (235.59, 47.64) circle (  2.13);

\path[fill=fillColor,fill opacity=0.20] (221.55, 50.14) circle (  2.13);

\path[fill=fillColor,fill opacity=0.20] (218.54, 58.34) circle (  2.13);

\path[fill=fillColor,fill opacity=0.20] (215.53, 60.31) circle (  2.13);

\path[fill=fillColor,fill opacity=0.20] (213.52, 58.75) circle (  2.13);

\path[fill=fillColor,fill opacity=0.20] (209.51, 59.27) circle (  2.13);

\path[fill=fillColor,fill opacity=0.20] (215.53, 60.21) circle (  2.13);

\path[fill=fillColor,fill opacity=0.20] (214.53, 62.80) circle (  2.13);

\path[fill=fillColor,fill opacity=0.20] (208.51, 67.99) circle (  2.13);

\path[fill=fillColor,fill opacity=0.20] (208.51, 73.70) circle (  2.13);

\path[fill=fillColor,fill opacity=0.20] (209.51, 71.42) circle (  2.13);

\path[fill=fillColor,fill opacity=0.20] (207.50, 64.36) circle (  2.13);

\path[fill=fillColor,fill opacity=0.20] (208.51, 61.56) circle (  2.13);

\path[fill=fillColor,fill opacity=0.20] (216.53, 60.52) circle (  2.13);

\path[fill=fillColor,fill opacity=0.20] (212.52, 59.79) circle (  2.13);

\path[fill=fillColor,fill opacity=0.20] (212.52, 63.63) circle (  2.13);

\path[fill=fillColor,fill opacity=0.20] (220.54, 67.68) circle (  2.13);

\path[fill=fillColor,fill opacity=0.20] (216.53, 68.30) circle (  2.13);

\path[fill=fillColor,fill opacity=0.20] (219.54, 63.94) circle (  2.13);

\path[fill=fillColor,fill opacity=0.20] (230.58, 55.53) circle (  2.13);

\path[fill=fillColor,fill opacity=0.20] (245.63, 54.08) circle (  2.13);

\path[fill=fillColor,fill opacity=0.20] (246.63, 63.84) circle (  2.13);

\path[fill=fillColor,fill opacity=0.20] (241.61, 80.87) circle (  2.13);

\path[fill=fillColor,fill opacity=0.20] (243.62, 68.41) circle (  2.13);

\path[fill=fillColor,fill opacity=0.20] (243.62, 61.56) circle (  2.13);

\path[fill=fillColor,fill opacity=0.20] (258.67, 63.01) circle (  2.13);

\path[fill=fillColor,fill opacity=0.20] (224.56, 60.83) circle (  2.13);

\path[fill=fillColor,fill opacity=0.20] (227.57, 55.02) circle (  2.13);

\path[fill=fillColor,fill opacity=0.20] (220.54, 56.05) circle (  2.13);

\path[fill=fillColor,fill opacity=0.20] (217.53, 59.89) circle (  2.13);

\path[fill=fillColor,fill opacity=0.20] (215.53, 56.78) circle (  2.13);

\path[fill=fillColor,fill opacity=0.20] (212.52, 50.14) circle (  2.13);

\path[fill=fillColor,fill opacity=0.20] (207.50, 53.87) circle (  2.13);

\path[fill=fillColor,fill opacity=0.20] (208.51, 64.15) circle (  2.13);

\path[fill=fillColor,fill opacity=0.20] (217.53, 62.08) circle (  2.13);

\path[fill=fillColor,fill opacity=0.20] (221.55, 56.16) circle (  2.13);

\path[fill=fillColor,fill opacity=0.20] (225.56, 55.02) circle (  2.13);

\path[fill=fillColor,fill opacity=0.20] (241.61, 56.05) circle (  2.13);

\path[fill=fillColor,fill opacity=0.20] (232.58, 53.04) circle (  2.13);

\path[fill=fillColor,fill opacity=0.20] (261.68, 50.76) circle (  2.13);

\path[fill=fillColor,fill opacity=0.20] (268.70, 49.93) circle (  2.13);

\path[fill=fillColor,fill opacity=0.20] (243.62, 55.12) circle (  2.13);

\path[fill=fillColor,fill opacity=0.20] (236.60, 67.47) circle (  2.13);

\path[fill=fillColor,fill opacity=0.20] (238.60, 70.69) circle (  2.13);

\path[fill=fillColor,fill opacity=0.20] (236.60, 71.63) circle (  2.13);

\path[fill=fillColor,fill opacity=0.20] (248.64, 84.81) circle (  2.13);

\path[fill=fillColor,fill opacity=0.20] (244.62, 85.85) circle (  2.13);

\path[fill=fillColor,fill opacity=0.20] (253.65, 73.50) circle (  2.13);

\path[fill=fillColor,fill opacity=0.20] (252.65, 61.24) circle (  2.13);

\path[fill=fillColor,fill opacity=0.20] (259.67, 63.53) circle (  2.13);

\path[fill=fillColor,fill opacity=0.20] (236.60, 66.95) circle (  2.13);

\path[fill=fillColor,fill opacity=0.20] (208.51, 62.28) circle (  2.13);

\path[fill=fillColor,fill opacity=0.20] (234.59, 54.50) circle (  2.13);

\path[fill=fillColor,fill opacity=0.20] (231.58, 50.14) circle (  2.13);

\path[fill=fillColor,fill opacity=0.20] (219.54, 52.84) circle (  2.13);

\path[fill=fillColor,fill opacity=0.20] (265.69, 55.22) circle (  2.13);

\path[fill=fillColor,fill opacity=0.20] (252.65, 57.71) circle (  2.13);

\path[fill=fillColor,fill opacity=0.20] (270.71, 64.57) circle (  2.13);

\path[fill=fillColor,fill opacity=0.20] (244.62, 71.83) circle (  2.13);

\path[fill=fillColor,fill opacity=0.20] (244.62, 69.03) circle (  2.13);

\path[fill=fillColor,fill opacity=0.20] (243.62, 60.62) circle (  2.13);

\path[fill=fillColor,fill opacity=0.20] (230.58, 60.31) circle (  2.13);

\path[fill=fillColor,fill opacity=0.20] (209.51, 66.85) circle (  2.13);

\path[fill=fillColor,fill opacity=0.20] (228.57, 74.64) circle (  2.13);

\path[fill=fillColor,fill opacity=0.20] (236.60, 73.08) circle (  2.13);

\path[fill=fillColor,fill opacity=0.20] (238.60, 68.72) circle (  2.13);

\path[fill=fillColor,fill opacity=0.20] (235.59, 67.27) circle (  2.13);

\path[fill=fillColor,fill opacity=0.20] (232.58, 61.66) circle (  2.13);

\path[fill=fillColor,fill opacity=0.20] (253.65, 57.40) circle (  2.13);

\path[fill=fillColor,fill opacity=0.20] (249.64, 63.63) circle (  2.13);

\path[fill=fillColor,fill opacity=0.20] (242.62, 72.56) circle (  2.13);
\end{scope}
\begin{scope}
\path[clip] (  0.00,  0.00) rectangle (289.08,144.54);
\definecolor[named]{drawColor}{rgb}{0.50,0.50,0.50}

\node[text=drawColor,anchor=base east,inner sep=0pt, outer sep=0pt, scale=  0.96] at ( 32.58, 36.86) {0.6};

\node[text=drawColor,anchor=base east,inner sep=0pt, outer sep=0pt, scale=  0.96] at ( 32.58, 57.63) {0.8};

\node[text=drawColor,anchor=base east,inner sep=0pt, outer sep=0pt, scale=  0.96] at ( 32.58, 78.39) {1.0};

\node[text=drawColor,anchor=base east,inner sep=0pt, outer sep=0pt, scale=  0.96] at ( 32.58, 99.15) {1.2};
\end{scope}
\begin{scope}
\path[clip] (  0.00,  0.00) rectangle (289.08,144.54);
\definecolor[named]{drawColor}{rgb}{0.50,0.50,0.50}

\path[draw=drawColor,line width= 0.6pt,line join=round] ( 35.42, 40.17) --
	( 39.69, 40.17);

\path[draw=drawColor,line width= 0.6pt,line join=round] ( 35.42, 60.93) --
	( 39.69, 60.93);

\path[draw=drawColor,line width= 0.6pt,line join=round] ( 35.42, 81.70) --
	( 39.69, 81.70);

\path[draw=drawColor,line width= 0.6pt,line join=round] ( 35.42,102.46) --
	( 39.69,102.46);
\end{scope}
\begin{scope}
\path[clip] (  0.00,  0.00) rectangle (289.08,144.54);
\definecolor[named]{drawColor}{rgb}{0.50,0.50,0.50}

\path[draw=drawColor,line width= 0.6pt,line join=round] ( 43.18, 29.77) --
	( 43.18, 34.04);

\path[draw=drawColor,line width= 0.6pt,line join=round] ( 68.26, 29.77) --
	( 68.26, 34.04);

\path[draw=drawColor,line width= 0.6pt,line join=round] ( 93.34, 29.77) --
	( 93.34, 34.04);

\path[draw=drawColor,line width= 0.6pt,line join=round] (118.43, 29.77) --
	(118.43, 34.04);

\path[draw=drawColor,line width= 0.6pt,line join=round] (143.51, 29.77) --
	(143.51, 34.04);
\end{scope}
\begin{scope}
\path[clip] (  0.00,  0.00) rectangle (289.08,144.54);
\definecolor[named]{drawColor}{rgb}{0.50,0.50,0.50}

\node[text=drawColor,anchor=base,inner sep=0pt, outer sep=0pt, scale=  0.96] at ( 43.18, 20.31) {7.5};

\node[text=drawColor,anchor=base,inner sep=0pt, outer sep=0pt, scale=  0.96] at ( 68.26, 20.31) {10.0};

\node[text=drawColor,anchor=base,inner sep=0pt, outer sep=0pt, scale=  0.96] at ( 93.34, 20.31) {12.5};

\node[text=drawColor,anchor=base,inner sep=0pt, outer sep=0pt, scale=  0.96] at (118.43, 20.31) {15.0};

\node[text=drawColor,anchor=base,inner sep=0pt, outer sep=0pt, scale=  0.96] at (143.51, 20.31) {17.5};
\end{scope}
\begin{scope}
\path[clip] (  0.00,  0.00) rectangle (289.08,144.54);
\definecolor[named]{drawColor}{rgb}{0.50,0.50,0.50}

\path[draw=drawColor,line width= 0.6pt,line join=round] (163.36, 29.77) --
	(163.36, 34.04);

\path[draw=drawColor,line width= 0.6pt,line join=round] (188.44, 29.77) --
	(188.44, 34.04);

\path[draw=drawColor,line width= 0.6pt,line join=round] (213.52, 29.77) --
	(213.52, 34.04);

\path[draw=drawColor,line width= 0.6pt,line join=round] (238.60, 29.77) --
	(238.60, 34.04);

\path[draw=drawColor,line width= 0.6pt,line join=round] (263.68, 29.77) --
	(263.68, 34.04);
\end{scope}
\begin{scope}
\path[clip] (  0.00,  0.00) rectangle (289.08,144.54);
\definecolor[named]{drawColor}{rgb}{0.50,0.50,0.50}

\node[text=drawColor,anchor=base,inner sep=0pt, outer sep=0pt, scale=  0.96] at (163.36, 20.31) {7.5};

\node[text=drawColor,anchor=base,inner sep=0pt, outer sep=0pt, scale=  0.96] at (188.44, 20.31) {10.0};

\node[text=drawColor,anchor=base,inner sep=0pt, outer sep=0pt, scale=  0.96] at (213.52, 20.31) {12.5};

\node[text=drawColor,anchor=base,inner sep=0pt, outer sep=0pt, scale=  0.96] at (238.60, 20.31) {15.0};

\node[text=drawColor,anchor=base,inner sep=0pt, outer sep=0pt, scale=  0.96] at (263.68, 20.31) {17.5};
\end{scope}
\begin{scope}
\path[clip] (  0.00,  0.00) rectangle (289.08,144.54);
\definecolor[named]{drawColor}{rgb}{0.00,0.00,0.00}

\node[text=drawColor,anchor=base,inner sep=0pt, outer sep=0pt, scale=  1.20] at (158.36,  9.03) {$a$ $[\mu m]$};
\end{scope}
\begin{scope}
\path[clip] (  0.00,  0.00) rectangle (289.08,144.54);
\definecolor[named]{drawColor}{rgb}{0.00,0.00,0.00}

\node[text=drawColor,rotate= 90.00,anchor=base,inner sep=0pt, outer sep=0pt, scale=  1.20] at ( 17.30, 76.95) {MD $[\times 10^{-9}mm^2/s]$};
\end{scope}
\end{tikzpicture}

						\end{adjustbox}\\
						\begin{adjustbox}{width={\textwidth},totalheight=\textheight,keepaspectratio}
							\strut
							% Created by tikzDevice version 0.6.2-92-0ad2792 on 2012-09-27 18:25:18
% !TEX encoding = UTF-8 Unicode
\begin{tikzpicture}[x=1pt,y=1pt]
\definecolor[named]{fillColor}{rgb}{1.00,1.00,1.00}
\path[use as bounding box,fill=fillColor,fill opacity=0.00] (0,0) rectangle (289.08,144.54);
\begin{scope}
\path[clip] (  0.00,  0.00) rectangle (289.08,144.54);
\definecolor[named]{drawColor}{rgb}{1.00,1.00,1.00}
\definecolor[named]{fillColor}{rgb}{1.00,1.00,1.00}

\path[draw=drawColor,line width= 0.6pt,line join=round,line cap=round,fill=fillColor] ( -0.00,  0.00) rectangle (289.08,144.54);
\end{scope}
\begin{scope}
\path[clip] ( 39.69,119.86) rectangle (156.86,132.50);
\definecolor[named]{fillColor}{rgb}{0.80,0.80,0.80}

\path[fill=fillColor] ( 39.69,119.86) rectangle (156.86,132.50);
\definecolor[named]{drawColor}{rgb}{0.00,0.00,0.00}

\node[text=drawColor,anchor=base,inner sep=0pt, outer sep=0pt, scale=  0.96] at ( 98.27,122.87) {Scan (r=-0.208)};
\end{scope}
\begin{scope}
\path[clip] (159.87,119.86) rectangle (277.03,132.50);
\definecolor[named]{fillColor}{rgb}{0.80,0.80,0.80}

\path[fill=fillColor] (159.87,119.86) rectangle (277.03,132.50);
\definecolor[named]{drawColor}{rgb}{0.00,0.00,0.00}

\node[text=drawColor,anchor=base,inner sep=0pt, outer sep=0pt, scale=  0.96] at (218.45,122.87) {Rescan (r=-0.233)};
\end{scope}
\begin{scope}
\path[clip] ( 39.69, 34.04) rectangle (156.86,119.86);
\definecolor[named]{fillColor}{rgb}{0.90,0.90,0.90}

\path[fill=fillColor] ( 39.69, 34.04) rectangle (156.86,119.86);
\definecolor[named]{drawColor}{rgb}{0.95,0.95,0.95}

\path[draw=drawColor,line width= 0.3pt,line join=round] ( 39.69, 50.55) --
	(156.86, 50.55);

\path[draw=drawColor,line width= 0.3pt,line join=round] ( 39.69, 71.32) --
	(156.86, 71.32);

\path[draw=drawColor,line width= 0.3pt,line join=round] ( 39.69, 92.08) --
	(156.86, 92.08);

\path[draw=drawColor,line width= 0.3pt,line join=round] ( 39.69,112.84) --
	(156.86,112.84);

\path[draw=drawColor,line width= 0.3pt,line join=round] ( 49.61, 34.04) --
	( 49.61,119.86);

\path[draw=drawColor,line width= 0.3pt,line join=round] ( 71.46, 34.04) --
	( 71.46,119.86);

\path[draw=drawColor,line width= 0.3pt,line join=round] ( 93.31, 34.04) --
	( 93.31,119.86);

\path[draw=drawColor,line width= 0.3pt,line join=round] (115.16, 34.04) --
	(115.16,119.86);

\path[draw=drawColor,line width= 0.3pt,line join=round] (137.01, 34.04) --
	(137.01,119.86);
\definecolor[named]{drawColor}{rgb}{1.00,1.00,1.00}

\path[draw=drawColor,line width= 0.6pt,line join=round] ( 39.69, 40.17) --
	(156.86, 40.17);

\path[draw=drawColor,line width= 0.6pt,line join=round] ( 39.69, 60.93) --
	(156.86, 60.93);

\path[draw=drawColor,line width= 0.6pt,line join=round] ( 39.69, 81.70) --
	(156.86, 81.70);

\path[draw=drawColor,line width= 0.6pt,line join=round] ( 39.69,102.46) --
	(156.86,102.46);

\path[draw=drawColor,line width= 0.6pt,line join=round] ( 60.53, 34.04) --
	( 60.53,119.86);

\path[draw=drawColor,line width= 0.6pt,line join=round] ( 82.38, 34.04) --
	( 82.38,119.86);

\path[draw=drawColor,line width= 0.6pt,line join=round] (104.23, 34.04) --
	(104.23,119.86);

\path[draw=drawColor,line width= 0.6pt,line join=round] (126.08, 34.04) --
	(126.08,119.86);

\path[draw=drawColor,line width= 0.6pt,line join=round] (147.93, 34.04) --
	(147.93,119.86);
\definecolor[named]{fillColor}{rgb}{0.00,0.00,0.00}

\path[fill=fillColor,fill opacity=0.20] ( 52.23, 72.46) circle (  2.13);

\path[fill=fillColor,fill opacity=0.20] ( 64.25, 80.76) circle (  2.13);

\path[fill=fillColor,fill opacity=0.20] ( 76.26, 68.82) circle (  2.13);

\path[fill=fillColor,fill opacity=0.20] ( 86.97, 56.47) circle (  2.13);

\path[fill=fillColor,fill opacity=0.20] ( 81.07, 53.67) circle (  2.13);

\path[fill=fillColor,fill opacity=0.20] ( 75.17, 53.46) circle (  2.13);

\path[fill=fillColor,fill opacity=0.20] ( 71.02, 69.65) circle (  2.13);

\path[fill=fillColor,fill opacity=0.20] ( 59.88, 93.12) circle (  2.13);

\path[fill=fillColor,fill opacity=0.20] ( 75.17, 83.77) circle (  2.13);

\path[fill=fillColor,fill opacity=0.20] ( 85.88, 58.65) circle (  2.13);

\path[fill=fillColor,fill opacity=0.20] ( 97.68, 51.49) circle (  2.13);

\path[fill=fillColor,fill opacity=0.20] (108.60, 57.71) circle (  2.13);

\path[fill=fillColor,fill opacity=0.20] ( 96.36, 57.92) circle (  2.13);

\path[fill=fillColor,fill opacity=0.20] (103.79, 50.45) circle (  2.13);

\path[fill=fillColor,fill opacity=0.20] (100.30, 44.94) circle (  2.13);

\path[fill=fillColor,fill opacity=0.20] ( 91.78, 56.68) circle (  2.13);

\path[fill=fillColor,fill opacity=0.20] ( 69.49, 59.79) circle (  2.13);

\path[fill=fillColor,fill opacity=0.20] ( 58.13, 57.09) circle (  2.13);

\path[fill=fillColor,fill opacity=0.20] ( 69.05,101.42) circle (  2.13);

\path[fill=fillColor,fill opacity=0.20] ( 85.22, 73.18) circle (  2.13);

\path[fill=fillColor,fill opacity=0.20] ( 90.68, 60.10) circle (  2.13);

\path[fill=fillColor,fill opacity=0.20] ( 94.84, 60.52) circle (  2.13);

\path[fill=fillColor,fill opacity=0.20] ( 98.33, 59.48) circle (  2.13);

\path[fill=fillColor,fill opacity=0.20] (102.92, 59.27) circle (  2.13);

\path[fill=fillColor,fill opacity=0.20] (100.08, 64.46) circle (  2.13);

\path[fill=fillColor,fill opacity=0.20] ( 95.27, 65.50) circle (  2.13);

\path[fill=fillColor,fill opacity=0.20] ( 93.52, 55.33) circle (  2.13);

\path[fill=fillColor,fill opacity=0.20] ( 88.06, 42.35) circle (  2.13);

\path[fill=fillColor,fill opacity=0.20] ( 76.04, 47.12) circle (  2.13);

\path[fill=fillColor,fill opacity=0.20] ( 74.95, 56.26) circle (  2.13);

\path[fill=fillColor,fill opacity=0.20] ( 71.46, 59.79) circle (  2.13);

\path[fill=fillColor,fill opacity=0.20] ( 58.56, 85.85) circle (  2.13);

\path[fill=fillColor,fill opacity=0.20] ( 81.51, 76.30) circle (  2.13);

\path[fill=fillColor,fill opacity=0.20] ( 98.11, 64.05) circle (  2.13);

\path[fill=fillColor,fill opacity=0.20] (111.66, 55.85) circle (  2.13);

\path[fill=fillColor,fill opacity=0.20] (106.42, 72.66) circle (  2.13);

\path[fill=fillColor,fill opacity=0.20] ( 99.21, 72.04) circle (  2.13);

\path[fill=fillColor,fill opacity=0.20] ( 92.87, 59.48) circle (  2.13);

\path[fill=fillColor,fill opacity=0.20] ( 97.68, 63.53) circle (  2.13);

\path[fill=fillColor,fill opacity=0.20] ( 99.86, 67.37) circle (  2.13);

\path[fill=fillColor,fill opacity=0.20] ( 88.94, 60.00) circle (  2.13);

\path[fill=fillColor,fill opacity=0.20] ( 79.76, 54.50) circle (  2.13);

\path[fill=fillColor,fill opacity=0.20] ( 77.14, 58.44) circle (  2.13);

\path[fill=fillColor,fill opacity=0.20] ( 77.79, 59.58) circle (  2.13);

\path[fill=fillColor,fill opacity=0.20] ( 74.30, 62.08) circle (  2.13);

\path[fill=fillColor,fill opacity=0.20] ( 60.97, 80.35) circle (  2.13);

\path[fill=fillColor,fill opacity=0.20] ( 76.48, 63.84) circle (  2.13);

\path[fill=fillColor,fill opacity=0.20] ( 98.77, 64.67) circle (  2.13);

\path[fill=fillColor,fill opacity=0.20] (107.95, 58.23) circle (  2.13);

\path[fill=fillColor,fill opacity=0.20] (105.10, 61.76) circle (  2.13);

\path[fill=fillColor,fill opacity=0.20] (103.58, 67.89) circle (  2.13);

\path[fill=fillColor,fill opacity=0.20] ( 95.27, 70.69) circle (  2.13);

\path[fill=fillColor,fill opacity=0.20] ( 83.25, 67.89) circle (  2.13);

\path[fill=fillColor,fill opacity=0.20] ( 89.37, 59.17) circle (  2.13);

\path[fill=fillColor,fill opacity=0.20] ( 85.22, 56.26) circle (  2.13);

\path[fill=fillColor,fill opacity=0.20] ( 76.92, 63.42) circle (  2.13);

\path[fill=fillColor,fill opacity=0.20] ( 74.30, 71.94) circle (  2.13);

\path[fill=fillColor,fill opacity=0.20] ( 65.56, 81.70) circle (  2.13);

\path[fill=fillColor,fill opacity=0.20] ( 54.19,106.61) circle (  2.13);

\path[fill=fillColor,fill opacity=0.20] ( 70.80, 75.68) circle (  2.13);

\path[fill=fillColor,fill opacity=0.20] ( 90.68, 45.05) circle (  2.13);

\path[fill=fillColor,fill opacity=0.20] ( 81.51, 76.71) circle (  2.13);

\path[fill=fillColor,fill opacity=0.20] ( 81.29, 76.71) circle (  2.13);

\path[fill=fillColor,fill opacity=0.20] ( 83.47, 64.88) circle (  2.13);

\path[fill=fillColor,fill opacity=0.20] ( 64.25, 85.85) circle (  2.13);

\path[fill=fillColor,fill opacity=0.20] ( 78.23, 58.34) circle (  2.13);

\path[fill=fillColor,fill opacity=0.20] ( 95.93, 60.00) circle (  2.13);

\path[fill=fillColor,fill opacity=0.20] (102.48, 58.03) circle (  2.13);

\path[fill=fillColor,fill opacity=0.20] ( 98.77, 46.50) circle (  2.13);

\path[fill=fillColor,fill opacity=0.20] (102.05, 48.89) circle (  2.13);

\path[fill=fillColor,fill opacity=0.20] ( 99.21, 69.97) circle (  2.13);

\path[fill=fillColor,fill opacity=0.20] ( 87.84, 76.19) circle (  2.13);

\path[fill=fillColor,fill opacity=0.20] ( 73.86, 63.22) circle (  2.13);

\path[fill=fillColor,fill opacity=0.20] ( 71.67, 57.51) circle (  2.13);

\path[fill=fillColor,fill opacity=0.20] ( 64.25, 63.01) circle (  2.13);

\path[fill=fillColor,fill opacity=0.20] ( 66.65, 86.89) circle (  2.13);

\path[fill=fillColor,fill opacity=0.20] (105.32, 52.63) circle (  2.13);

\path[fill=fillColor,fill opacity=0.20] ( 97.02, 64.15) circle (  2.13);

\path[fill=fillColor,fill opacity=0.20] ( 94.84, 60.41) circle (  2.13);

\path[fill=fillColor,fill opacity=0.20] ( 99.86, 61.14) circle (  2.13);

\path[fill=fillColor,fill opacity=0.20] ( 99.42, 62.18) circle (  2.13);

\path[fill=fillColor,fill opacity=0.20] ( 95.71, 54.81) circle (  2.13);

\path[fill=fillColor,fill opacity=0.20] ( 81.29, 65.19) circle (  2.13);

\path[fill=fillColor,fill opacity=0.20] ( 55.94,114.92) circle (  2.13);

\path[fill=fillColor,fill opacity=0.20] ( 70.36, 79.93) circle (  2.13);

\path[fill=fillColor,fill opacity=0.20] ( 77.57, 56.68) circle (  2.13);

\path[fill=fillColor,fill opacity=0.20] ( 88.06, 51.28) circle (  2.13);

\path[fill=fillColor,fill opacity=0.20] ( 90.47, 61.04) circle (  2.13);

\path[fill=fillColor,fill opacity=0.20] ( 87.62, 63.74) circle (  2.13);

\path[fill=fillColor,fill opacity=0.20] ( 89.59, 51.17) circle (  2.13);

\path[fill=fillColor,fill opacity=0.20] ( 91.99, 57.30) circle (  2.13);

\path[fill=fillColor,fill opacity=0.20] ( 85.44, 79.21) circle (  2.13);

\path[fill=fillColor,fill opacity=0.20] ( 64.46, 64.77) circle (  2.13);

\path[fill=fillColor,fill opacity=0.20] ( 77.57, 79.41) circle (  2.13);

\path[fill=fillColor,fill opacity=0.20] (111.88, 50.76) circle (  2.13);

\path[fill=fillColor,fill opacity=0.20] (110.35, 67.47) circle (  2.13);

\path[fill=fillColor,fill opacity=0.20] (116.03, 60.93) circle (  2.13);

\path[fill=fillColor,fill opacity=0.20] (126.95, 47.33) circle (  2.13);

\path[fill=fillColor,fill opacity=0.20] (146.84, 44.74) circle (  2.13);

\path[fill=fillColor,fill opacity=0.20] (113.41, 50.24) circle (  2.13);

\path[fill=fillColor,fill opacity=0.20] (105.54, 44.22) circle (  2.13);

\path[fill=fillColor,fill opacity=0.20] (101.39, 49.31) circle (  2.13);

\path[fill=fillColor,fill opacity=0.20] ( 65.34, 90.00) circle (  2.13);

\path[fill=fillColor,fill opacity=0.20] ( 67.96, 71.94) circle (  2.13);

\path[fill=fillColor,fill opacity=0.20] ( 78.01, 55.53) circle (  2.13);

\path[fill=fillColor,fill opacity=0.20] ( 88.94, 60.93) circle (  2.13);

\path[fill=fillColor,fill opacity=0.20] ( 81.94, 63.63) circle (  2.13);

\path[fill=fillColor,fill opacity=0.20] ( 80.41, 72.15) circle (  2.13);

\path[fill=fillColor,fill opacity=0.20] ( 83.47, 69.45) circle (  2.13);

\path[fill=fillColor,fill opacity=0.20] ( 80.20, 62.18) circle (  2.13);

\path[fill=fillColor,fill opacity=0.20] ( 78.67, 73.81) circle (  2.13);

\path[fill=fillColor,fill opacity=0.20] ( 77.79, 78.58) circle (  2.13);

\path[fill=fillColor,fill opacity=0.20] ( 67.96, 70.38) circle (  2.13);

\path[fill=fillColor,fill opacity=0.20] (111.66, 49.82) circle (  2.13);

\path[fill=fillColor,fill opacity=0.20] (115.59, 65.29) circle (  2.13);

\path[fill=fillColor,fill opacity=0.20] (119.09, 65.81) circle (  2.13);

\path[fill=fillColor,fill opacity=0.20] (133.73, 44.43) circle (  2.13);

\path[fill=fillColor,fill opacity=0.20] (120.40, 43.49) circle (  2.13);

\path[fill=fillColor,fill opacity=0.20] (122.58, 52.00) circle (  2.13);

\path[fill=fillColor,fill opacity=0.20] (100.95, 51.80) circle (  2.13);

\path[fill=fillColor,fill opacity=0.20] ( 96.15, 45.78) circle (  2.13);

\path[fill=fillColor,fill opacity=0.20] ( 79.98, 56.05) circle (  2.13);

\path[fill=fillColor,fill opacity=0.20] ( 60.31, 74.12) circle (  2.13);

\path[fill=fillColor,fill opacity=0.20] ( 81.07, 62.39) circle (  2.13);

\path[fill=fillColor,fill opacity=0.20] ( 81.94, 74.64) circle (  2.13);

\path[fill=fillColor,fill opacity=0.20] ( 83.91, 58.34) circle (  2.13);

\path[fill=fillColor,fill opacity=0.20] ( 86.97, 53.04) circle (  2.13);

\path[fill=fillColor,fill opacity=0.20] ( 83.04, 70.28) circle (  2.13);

\path[fill=fillColor,fill opacity=0.20] ( 86.10, 75.68) circle (  2.13);

\path[fill=fillColor,fill opacity=0.20] ( 80.63, 67.47) circle (  2.13);

\path[fill=fillColor,fill opacity=0.20] ( 82.82, 60.52) circle (  2.13);

\path[fill=fillColor,fill opacity=0.20] ( 71.24, 65.40) circle (  2.13);

\path[fill=fillColor,fill opacity=0.20] ( 89.15, 78.06) circle (  2.13);

\path[fill=fillColor,fill opacity=0.20] (117.78, 58.75) circle (  2.13);

\path[fill=fillColor,fill opacity=0.20] (109.26, 58.55) circle (  2.13);

\path[fill=fillColor,fill opacity=0.20] (101.17, 48.68) circle (  2.13);

\path[fill=fillColor,fill opacity=0.20] (100.08, 58.23) circle (  2.13);

\path[fill=fillColor,fill opacity=0.20] (102.48, 61.04) circle (  2.13);

\path[fill=fillColor,fill opacity=0.20] ( 94.40, 55.22) circle (  2.13);

\path[fill=fillColor,fill opacity=0.20] ( 82.60, 54.91) circle (  2.13);

\path[fill=fillColor,fill opacity=0.20] ( 56.38, 84.81) circle (  2.13);

\path[fill=fillColor,fill opacity=0.20] ( 75.39, 68.93) circle (  2.13);

\path[fill=fillColor,fill opacity=0.20] ( 81.73, 71.52) circle (  2.13);

\path[fill=fillColor,fill opacity=0.20] ( 91.78, 52.11) circle (  2.13);

\path[fill=fillColor,fill opacity=0.20] ( 93.09, 38.82) circle (  2.13);

\path[fill=fillColor,fill opacity=0.20] ( 86.53, 56.99) circle (  2.13);

\path[fill=fillColor,fill opacity=0.20] ( 83.47, 70.59) circle (  2.13);

\path[fill=fillColor,fill opacity=0.20] ( 81.94, 67.16) circle (  2.13);

\path[fill=fillColor,fill opacity=0.20] ( 80.41, 56.57) circle (  2.13);

\path[fill=fillColor,fill opacity=0.20] ( 78.23, 53.15) circle (  2.13);

\path[fill=fillColor,fill opacity=0.20] (129.14, 53.15) circle (  2.13);

\path[fill=fillColor,fill opacity=0.20] (114.72, 48.99) circle (  2.13);

\path[fill=fillColor,fill opacity=0.20] (102.70, 61.35) circle (  2.13);

\path[fill=fillColor,fill opacity=0.20] (119.96, 53.98) circle (  2.13);

\path[fill=fillColor,fill opacity=0.20] (100.95, 56.05) circle (  2.13);

\path[fill=fillColor,fill opacity=0.20] ( 93.31, 62.39) circle (  2.13);

\path[fill=fillColor,fill opacity=0.20] (100.73, 60.41) circle (  2.13);

\path[fill=fillColor,fill opacity=0.20] ( 94.18, 60.10) circle (  2.13);

\path[fill=fillColor,fill opacity=0.20] ( 86.31, 60.41) circle (  2.13);

\path[fill=fillColor,fill opacity=0.20] ( 88.28, 42.56) circle (  2.13);

\path[fill=fillColor,fill opacity=0.20] ( 72.33, 70.28) circle (  2.13);

\path[fill=fillColor,fill opacity=0.20] ( 90.90, 60.52) circle (  2.13);

\path[fill=fillColor,fill opacity=0.20] ( 99.64, 56.05) circle (  2.13);

\path[fill=fillColor,fill opacity=0.20] ( 97.46, 53.77) circle (  2.13);

\path[fill=fillColor,fill opacity=0.20] ( 88.06, 53.15) circle (  2.13);

\path[fill=fillColor,fill opacity=0.20] ( 84.78, 53.15) circle (  2.13);

\path[fill=fillColor,fill opacity=0.20] ( 83.04, 60.10) circle (  2.13);

\path[fill=fillColor,fill opacity=0.20] ( 76.48, 71.11) circle (  2.13);

\path[fill=fillColor,fill opacity=0.20] ( 79.98, 58.96) circle (  2.13);

\path[fill=fillColor,fill opacity=0.20] ( 66.21,108.69) circle (  2.13);

\path[fill=fillColor,fill opacity=0.20] (117.34, 46.29) circle (  2.13);

\path[fill=fillColor,fill opacity=0.20] (103.36, 72.25) circle (  2.13);

\path[fill=fillColor,fill opacity=0.20] (109.69, 71.21) circle (  2.13);

\path[fill=fillColor,fill opacity=0.20] (122.15, 54.18) circle (  2.13);

\path[fill=fillColor,fill opacity=0.20] (106.20, 55.53) circle (  2.13);

\path[fill=fillColor,fill opacity=0.20] ( 96.58, 61.04) circle (  2.13);

\path[fill=fillColor,fill opacity=0.20] (106.63, 61.14) circle (  2.13);

\path[fill=fillColor,fill opacity=0.20] ( 83.25, 65.71) circle (  2.13);

\path[fill=fillColor,fill opacity=0.20] ( 88.06, 58.34) circle (  2.13);

\path[fill=fillColor,fill opacity=0.20] ( 87.62, 39.86) circle (  2.13);

\path[fill=fillColor,fill opacity=0.20] ( 74.30, 52.94) circle (  2.13);

\path[fill=fillColor,fill opacity=0.20] ( 82.60, 51.69) circle (  2.13);

\path[fill=fillColor,fill opacity=0.20] ( 90.68, 60.83) circle (  2.13);

\path[fill=fillColor,fill opacity=0.20] ( 92.43, 66.33) circle (  2.13);

\path[fill=fillColor,fill opacity=0.20] ( 90.90, 59.17) circle (  2.13);

\path[fill=fillColor,fill opacity=0.20] ( 88.28, 47.02) circle (  2.13);

\path[fill=fillColor,fill opacity=0.20] ( 90.68, 50.65) circle (  2.13);

\path[fill=fillColor,fill opacity=0.20] ( 81.29, 75.47) circle (  2.13);

\path[fill=fillColor,fill opacity=0.20] ( 89.81, 74.33) circle (  2.13);

\path[fill=fillColor,fill opacity=0.20] ( 82.60, 60.31) circle (  2.13);

\path[fill=fillColor,fill opacity=0.20] ( 95.93, 62.08) circle (  2.13);

\path[fill=fillColor,fill opacity=0.20] ( 89.81, 62.18) circle (  2.13);

\path[fill=fillColor,fill opacity=0.20] ( 88.28, 77.86) circle (  2.13);

\path[fill=fillColor,fill opacity=0.20] ( 97.89, 68.93) circle (  2.13);

\path[fill=fillColor,fill opacity=0.20] (107.73, 57.20) circle (  2.13);

\path[fill=fillColor,fill opacity=0.20] (100.52, 60.73) circle (  2.13);

\path[fill=fillColor,fill opacity=0.20] ( 98.33, 59.89) circle (  2.13);

\path[fill=fillColor,fill opacity=0.20] (101.17, 53.67) circle (  2.13);

\path[fill=fillColor,fill opacity=0.20] ( 92.87, 56.68) circle (  2.13);

\path[fill=fillColor,fill opacity=0.20] ( 85.88, 51.59) circle (  2.13);

\path[fill=fillColor,fill opacity=0.20] ( 85.66, 40.17) circle (  2.13);

\path[fill=fillColor,fill opacity=0.20] ( 66.43, 64.98) circle (  2.13);

\path[fill=fillColor,fill opacity=0.20] ( 72.11, 60.00) circle (  2.13);

\path[fill=fillColor,fill opacity=0.20] ( 80.63, 63.32) circle (  2.13);

\path[fill=fillColor,fill opacity=0.20] ( 82.38, 64.15) circle (  2.13);

\path[fill=fillColor,fill opacity=0.20] ( 91.99, 57.92) circle (  2.13);

\path[fill=fillColor,fill opacity=0.20] ( 86.75, 55.53) circle (  2.13);

\path[fill=fillColor,fill opacity=0.20] ( 87.62, 55.22) circle (  2.13);

\path[fill=fillColor,fill opacity=0.20] ( 88.50, 64.15) circle (  2.13);

\path[fill=fillColor,fill opacity=0.20] ( 79.32, 67.58) circle (  2.13);

\path[fill=fillColor,fill opacity=0.20] ( 75.83, 56.05) circle (  2.13);

\path[fill=fillColor,fill opacity=0.20] ( 64.90, 69.24) circle (  2.13);

\path[fill=fillColor,fill opacity=0.20] ( 86.31, 78.69) circle (  2.13);

\path[fill=fillColor,fill opacity=0.20] (110.13, 59.06) circle (  2.13);

\path[fill=fillColor,fill opacity=0.20] ( 83.25, 71.83) circle (  2.13);

\path[fill=fillColor,fill opacity=0.20] ( 87.62, 70.28) circle (  2.13);

\path[fill=fillColor,fill opacity=0.20] ( 95.05, 60.31) circle (  2.13);

\path[fill=fillColor,fill opacity=0.20] ( 98.33, 55.85) circle (  2.13);

\path[fill=fillColor,fill opacity=0.20] (101.17, 60.52) circle (  2.13);

\path[fill=fillColor,fill opacity=0.20] ( 91.56, 55.43) circle (  2.13);

\path[fill=fillColor,fill opacity=0.20] (106.63, 44.94) circle (  2.13);

\path[fill=fillColor,fill opacity=0.20] (102.92, 50.45) circle (  2.13);

\path[fill=fillColor,fill opacity=0.20] ( 90.47, 50.14) circle (  2.13);

\path[fill=fillColor,fill opacity=0.20] ( 83.25, 47.33) circle (  2.13);

\path[fill=fillColor,fill opacity=0.20] ( 71.46, 73.50) circle (  2.13);

\path[fill=fillColor,fill opacity=0.20] ( 89.59, 58.55) circle (  2.13);

\path[fill=fillColor,fill opacity=0.20] ( 90.68, 49.82) circle (  2.13);

\path[fill=fillColor,fill opacity=0.20] ( 85.00, 58.03) circle (  2.13);

\path[fill=fillColor,fill opacity=0.20] ( 89.59, 59.89) circle (  2.13);

\path[fill=fillColor,fill opacity=0.20] ( 90.68, 50.34) circle (  2.13);

\path[fill=fillColor,fill opacity=0.20] ( 86.75, 43.49) circle (  2.13);

\path[fill=fillColor,fill opacity=0.20] ( 77.79, 43.08) circle (  2.13);

\path[fill=fillColor,fill opacity=0.20] ( 76.26, 49.62) circle (  2.13);

\path[fill=fillColor,fill opacity=0.20] ( 67.52, 68.93) circle (  2.13);

\path[fill=fillColor,fill opacity=0.20] ( 72.99, 96.23) circle (  2.13);

\path[fill=fillColor,fill opacity=0.20] ( 97.24, 53.46) circle (  2.13);

\path[fill=fillColor,fill opacity=0.20] ( 96.15, 63.74) circle (  2.13);

\path[fill=fillColor,fill opacity=0.20] ( 96.80, 60.00) circle (  2.13);

\path[fill=fillColor,fill opacity=0.20] (105.32, 57.09) circle (  2.13);

\path[fill=fillColor,fill opacity=0.20] (102.26, 59.06) circle (  2.13);

\path[fill=fillColor,fill opacity=0.20] ( 99.64, 58.34) circle (  2.13);

\path[fill=fillColor,fill opacity=0.20] (100.95, 53.56) circle (  2.13);

\path[fill=fillColor,fill opacity=0.20] ( 98.99, 47.64) circle (  2.13);

\path[fill=fillColor,fill opacity=0.20] (103.36, 48.79) circle (  2.13);

\path[fill=fillColor,fill opacity=0.20] ( 95.93, 58.23) circle (  2.13);

\path[fill=fillColor,fill opacity=0.20] ( 94.84, 58.96) circle (  2.13);

\path[fill=fillColor,fill opacity=0.20] ( 78.67, 62.08) circle (  2.13);

\path[fill=fillColor,fill opacity=0.20] ( 88.06, 68.30) circle (  2.13);

\path[fill=fillColor,fill opacity=0.20] ( 99.42, 44.94) circle (  2.13);

\path[fill=fillColor,fill opacity=0.20] ( 84.35, 47.64) circle (  2.13);

\path[fill=fillColor,fill opacity=0.20] ( 89.37, 53.35) circle (  2.13);

\path[fill=fillColor,fill opacity=0.20] ( 89.81, 47.54) circle (  2.13);

\path[fill=fillColor,fill opacity=0.20] ( 85.22, 44.43) circle (  2.13);

\path[fill=fillColor,fill opacity=0.20] ( 86.97, 45.05) circle (  2.13);

\path[fill=fillColor,fill opacity=0.20] ( 78.45, 50.86) circle (  2.13);

\path[fill=fillColor,fill opacity=0.20] ( 75.17, 64.46) circle (  2.13);

\path[fill=fillColor,fill opacity=0.20] ( 60.31, 79.10) circle (  2.13);

\path[fill=fillColor,fill opacity=0.20] ( 75.17,107.65) circle (  2.13);

\path[fill=fillColor,fill opacity=0.20] ( 97.46, 60.83) circle (  2.13);

\path[fill=fillColor,fill opacity=0.20] ( 94.18, 69.24) circle (  2.13);

\path[fill=fillColor,fill opacity=0.20] ( 97.46, 58.96) circle (  2.13);

\path[fill=fillColor,fill opacity=0.20] (107.29, 49.51) circle (  2.13);

\path[fill=fillColor,fill opacity=0.20] (108.38, 54.81) circle (  2.13);

\path[fill=fillColor,fill opacity=0.20] (102.70, 64.36) circle (  2.13);

\path[fill=fillColor,fill opacity=0.20] ( 98.33, 67.68) circle (  2.13);

\path[fill=fillColor,fill opacity=0.20] (107.29, 58.65) circle (  2.13);

\path[fill=fillColor,fill opacity=0.20] ( 98.33, 51.69) circle (  2.13);

\path[fill=fillColor,fill opacity=0.20] (100.52, 55.22) circle (  2.13);

\path[fill=fillColor,fill opacity=0.20] (100.52, 58.23) circle (  2.13);

\path[fill=fillColor,fill opacity=0.20] ( 86.10, 67.16) circle (  2.13);

\path[fill=fillColor,fill opacity=0.20] ( 57.69, 88.96) circle (  2.13);

\path[fill=fillColor,fill opacity=0.20] ( 55.51, 95.19) circle (  2.13);

\path[fill=fillColor,fill opacity=0.20] ( 75.61, 60.10) circle (  2.13);

\path[fill=fillColor,fill opacity=0.20] ( 88.28, 48.16) circle (  2.13);

\path[fill=fillColor,fill opacity=0.20] ( 86.53, 48.79) circle (  2.13);

\path[fill=fillColor,fill opacity=0.20] ( 81.73, 56.57) circle (  2.13);

\path[fill=fillColor,fill opacity=0.20] ( 89.37, 58.55) circle (  2.13);

\path[fill=fillColor,fill opacity=0.20] ( 86.97, 54.18) circle (  2.13);

\path[fill=fillColor,fill opacity=0.20] ( 86.10, 56.99) circle (  2.13);

\path[fill=fillColor,fill opacity=0.20] ( 77.36, 68.62) circle (  2.13);

\path[fill=fillColor,fill opacity=0.20] ( 72.77, 75.68) circle (  2.13);

\path[fill=fillColor,fill opacity=0.20] ( 65.77, 80.87) circle (  2.13);

\path[fill=fillColor,fill opacity=0.20] ( 68.40,111.81) circle (  2.13);

\path[fill=fillColor,fill opacity=0.20] ( 97.68, 58.13) circle (  2.13);

\path[fill=fillColor,fill opacity=0.20] ( 94.18, 72.87) circle (  2.13);

\path[fill=fillColor,fill opacity=0.20] ( 95.05, 69.55) circle (  2.13);

\path[fill=fillColor,fill opacity=0.20] (100.95, 59.17) circle (  2.13);

\path[fill=fillColor,fill opacity=0.20] (102.05, 63.11) circle (  2.13);

\path[fill=fillColor,fill opacity=0.20] ( 99.21, 64.98) circle (  2.13);

\path[fill=fillColor,fill opacity=0.20] (100.08, 63.74) circle (  2.13);

\path[fill=fillColor,fill opacity=0.20] (103.58, 64.77) circle (  2.13);

\path[fill=fillColor,fill opacity=0.20] (100.08, 66.02) circle (  2.13);

\path[fill=fillColor,fill opacity=0.20] (104.23, 68.62) circle (  2.13);

\path[fill=fillColor,fill opacity=0.20] (102.26, 59.17) circle (  2.13);

\path[fill=fillColor,fill opacity=0.20] ( 88.06, 47.75) circle (  2.13);

\path[fill=fillColor,fill opacity=0.20] ( 55.94, 94.16) circle (  2.13);

\path[fill=fillColor,fill opacity=0.20] ( 85.00, 71.32) circle (  2.13);

\path[fill=fillColor,fill opacity=0.20] ( 93.74, 53.04) circle (  2.13);

\path[fill=fillColor,fill opacity=0.20] ( 85.22, 56.88) circle (  2.13);

\path[fill=fillColor,fill opacity=0.20] ( 91.34, 53.56) circle (  2.13);

\path[fill=fillColor,fill opacity=0.20] ( 98.11, 45.57) circle (  2.13);

\path[fill=fillColor,fill opacity=0.20] ( 86.75, 53.56) circle (  2.13);

\path[fill=fillColor,fill opacity=0.20] ( 84.78, 59.58) circle (  2.13);

\path[fill=fillColor,fill opacity=0.20] ( 83.25, 61.87) circle (  2.13);

\path[fill=fillColor,fill opacity=0.20] ( 83.47, 62.91) circle (  2.13);

\path[fill=fillColor,fill opacity=0.20] ( 68.83, 66.95) circle (  2.13);

\path[fill=fillColor,fill opacity=0.20] ( 92.21, 64.88) circle (  2.13);

\path[fill=fillColor,fill opacity=0.20] ( 93.09, 74.85) circle (  2.13);

\path[fill=fillColor,fill opacity=0.20] (103.14, 69.65) circle (  2.13);

\path[fill=fillColor,fill opacity=0.20] (103.36, 55.74) circle (  2.13);

\path[fill=fillColor,fill opacity=0.20] (105.76, 55.64) circle (  2.13);

\path[fill=fillColor,fill opacity=0.20] (102.48, 63.22) circle (  2.13);

\path[fill=fillColor,fill opacity=0.20] (100.73, 67.27) circle (  2.13);

\path[fill=fillColor,fill opacity=0.20] (103.79, 60.31) circle (  2.13);

\path[fill=fillColor,fill opacity=0.20] (110.35, 57.61) circle (  2.13);

\path[fill=fillColor,fill opacity=0.20] (104.89, 66.33) circle (  2.13);

\path[fill=fillColor,fill opacity=0.20] ( 98.11, 60.41) circle (  2.13);

\path[fill=fillColor,fill opacity=0.20] ( 76.70, 50.45) circle (  2.13);

\path[fill=fillColor,fill opacity=0.20] ( 55.51,101.42) circle (  2.13);

\path[fill=fillColor,fill opacity=0.20] ( 91.12, 66.12) circle (  2.13);

\path[fill=fillColor,fill opacity=0.20] (101.39, 49.93) circle (  2.13);

\path[fill=fillColor,fill opacity=0.20] ( 98.33, 41.41) circle (  2.13);

\path[fill=fillColor,fill opacity=0.20] (102.48, 39.55) circle (  2.13);

\path[fill=fillColor,fill opacity=0.20] ( 91.78, 49.82) circle (  2.13);

\path[fill=fillColor,fill opacity=0.20] ( 83.69, 53.87) circle (  2.13);

\path[fill=fillColor,fill opacity=0.20] ( 83.91, 59.48) circle (  2.13);

\path[fill=fillColor,fill opacity=0.20] ( 76.26, 64.88) circle (  2.13);

\path[fill=fillColor,fill opacity=0.20] ( 72.77, 63.74) circle (  2.13);

\path[fill=fillColor,fill opacity=0.20] ( 66.87, 75.99) circle (  2.13);

\path[fill=fillColor,fill opacity=0.20] ( 62.72,108.69) circle (  2.13);

\path[fill=fillColor,fill opacity=0.20] ( 91.34, 71.94) circle (  2.13);

\path[fill=fillColor,fill opacity=0.20] ( 86.31, 80.24) circle (  2.13);

\path[fill=fillColor,fill opacity=0.20] ( 98.11, 71.52) circle (  2.13);

\path[fill=fillColor,fill opacity=0.20] (113.41, 57.71) circle (  2.13);

\path[fill=fillColor,fill opacity=0.20] (116.69, 51.38) circle (  2.13);

\path[fill=fillColor,fill opacity=0.20] (117.34, 49.20) circle (  2.13);

\path[fill=fillColor,fill opacity=0.20] (109.04, 51.38) circle (  2.13);

\path[fill=fillColor,fill opacity=0.20] (106.63, 57.92) circle (  2.13);

\path[fill=fillColor,fill opacity=0.20] (101.61, 61.97) circle (  2.13);

\path[fill=fillColor,fill opacity=0.20] (100.08, 61.87) circle (  2.13);

\path[fill=fillColor,fill opacity=0.20] (111.88, 60.62) circle (  2.13);

\path[fill=fillColor,fill opacity=0.20] (104.67, 56.57) circle (  2.13);

\path[fill=fillColor,fill opacity=0.20] ( 74.73, 63.32) circle (  2.13);

\path[fill=fillColor,fill opacity=0.20] ( 53.98, 95.19) circle (  2.13);

\path[fill=fillColor,fill opacity=0.20] ( 98.77, 62.59) circle (  2.13);

\path[fill=fillColor,fill opacity=0.20] (102.92, 51.80) circle (  2.13);

\path[fill=fillColor,fill opacity=0.20] ( 97.02, 52.63) circle (  2.13);

\path[fill=fillColor,fill opacity=0.20] ( 97.68, 53.35) circle (  2.13);

\path[fill=fillColor,fill opacity=0.20] ( 84.78, 59.58) circle (  2.13);

\path[fill=fillColor,fill opacity=0.20] ( 82.82, 68.51) circle (  2.13);

\path[fill=fillColor,fill opacity=0.20] ( 81.51, 68.10) circle (  2.13);

\path[fill=fillColor,fill opacity=0.20] ( 72.33, 67.58) circle (  2.13);

\path[fill=fillColor,fill opacity=0.20] ( 72.33, 69.03) circle (  2.13);

\path[fill=fillColor,fill opacity=0.20] ( 68.18, 65.92) circle (  2.13);

\path[fill=fillColor,fill opacity=0.20] ( 70.36, 80.14) circle (  2.13);

\path[fill=fillColor,fill opacity=0.20] ( 97.24, 54.81) circle (  2.13);

\path[fill=fillColor,fill opacity=0.20] (102.70, 80.14) circle (  2.13);

\path[fill=fillColor,fill opacity=0.20] ( 95.71, 81.39) circle (  2.13);

\path[fill=fillColor,fill opacity=0.20] (112.10, 59.89) circle (  2.13);

\path[fill=fillColor,fill opacity=0.20] (112.10, 59.06) circle (  2.13);

\path[fill=fillColor,fill opacity=0.20] (116.90, 64.15) circle (  2.13);

\path[fill=fillColor,fill opacity=0.20] (114.72, 57.92) circle (  2.13);

\path[fill=fillColor,fill opacity=0.20] (110.79, 49.72) circle (  2.13);

\path[fill=fillColor,fill opacity=0.20] (109.47, 51.38) circle (  2.13);

\path[fill=fillColor,fill opacity=0.20] (100.95, 58.75) circle (  2.13);

\path[fill=fillColor,fill opacity=0.20] ( 98.33, 56.26) circle (  2.13);

\path[fill=fillColor,fill opacity=0.20] ( 97.68, 52.00) circle (  2.13);

\path[fill=fillColor,fill opacity=0.20] ( 70.14, 60.10) circle (  2.13);

\path[fill=fillColor,fill opacity=0.20] ( 91.34, 74.64) circle (  2.13);

\path[fill=fillColor,fill opacity=0.20] (100.30, 68.41) circle (  2.13);

\path[fill=fillColor,fill opacity=0.20] ( 91.34, 53.87) circle (  2.13);

\path[fill=fillColor,fill opacity=0.20] ( 87.41, 52.52) circle (  2.13);

\path[fill=fillColor,fill opacity=0.20] ( 84.78, 66.95) circle (  2.13);

\path[fill=fillColor,fill opacity=0.20] ( 82.60, 66.33) circle (  2.13);

\path[fill=fillColor,fill opacity=0.20] ( 80.41, 61.56) circle (  2.13);

\path[fill=fillColor,fill opacity=0.20] ( 72.77, 62.80) circle (  2.13);

\path[fill=fillColor,fill opacity=0.20] ( 70.80, 57.09) circle (  2.13);

\path[fill=fillColor,fill opacity=0.20] ( 81.73, 54.08) circle (  2.13);

\path[fill=fillColor,fill opacity=0.20] ( 70.58, 60.73) circle (  2.13);

\path[fill=fillColor,fill opacity=0.20] ( 58.35, 82.74) circle (  2.13);

\path[fill=fillColor,fill opacity=0.20] ( 78.01, 52.52) circle (  2.13);

\path[fill=fillColor,fill opacity=0.20] ( 87.19, 53.77) circle (  2.13);

\path[fill=fillColor,fill opacity=0.20] ( 98.33, 63.94) circle (  2.13);

\path[fill=fillColor,fill opacity=0.20] ( 92.65, 67.58) circle (  2.13);

\path[fill=fillColor,fill opacity=0.20] (102.92, 68.10) circle (  2.13);

\path[fill=fillColor,fill opacity=0.20] (105.98, 65.29) circle (  2.13);

\path[fill=fillColor,fill opacity=0.20] (106.42, 68.20) circle (  2.13);

\path[fill=fillColor,fill opacity=0.20] (110.79, 74.95) circle (  2.13);

\path[fill=fillColor,fill opacity=0.20] (111.00, 68.72) circle (  2.13);

\path[fill=fillColor,fill opacity=0.20] (111.66, 54.70) circle (  2.13);

\path[fill=fillColor,fill opacity=0.20] (106.85, 50.86) circle (  2.13);

\path[fill=fillColor,fill opacity=0.20] (100.08, 50.97) circle (  2.13);

\path[fill=fillColor,fill opacity=0.20] ( 85.88, 46.40) circle (  2.13);

\path[fill=fillColor,fill opacity=0.20] ( 71.24, 57.20) circle (  2.13);

\path[fill=fillColor,fill opacity=0.20] ( 68.62, 82.74) circle (  2.13);

\path[fill=fillColor,fill opacity=0.20] ( 85.44, 59.48) circle (  2.13);

\path[fill=fillColor,fill opacity=0.20] ( 94.40, 41.93) circle (  2.13);

\path[fill=fillColor,fill opacity=0.20] ( 92.65, 49.41) circle (  2.13);

\path[fill=fillColor,fill opacity=0.20] ( 85.88, 60.52) circle (  2.13);

\path[fill=fillColor,fill opacity=0.20] ( 84.13, 59.06) circle (  2.13);

\path[fill=fillColor,fill opacity=0.20] ( 76.70, 61.66) circle (  2.13);

\path[fill=fillColor,fill opacity=0.20] ( 70.36, 72.35) circle (  2.13);

\path[fill=fillColor,fill opacity=0.20] ( 69.27, 74.22) circle (  2.13);

\path[fill=fillColor,fill opacity=0.20] ( 73.86, 60.73) circle (  2.13);

\path[fill=fillColor,fill opacity=0.20] ( 71.89, 48.79) circle (  2.13);

\path[fill=fillColor,fill opacity=0.20] ( 72.33, 52.21) circle (  2.13);

\path[fill=fillColor,fill opacity=0.20] ( 64.90, 70.38) circle (  2.13);

\path[fill=fillColor,fill opacity=0.20] ( 67.52, 65.40) circle (  2.13);

\path[fill=fillColor,fill opacity=0.20] ( 76.04, 63.74) circle (  2.13);

\path[fill=fillColor,fill opacity=0.20] ( 85.66, 63.42) circle (  2.13);

\path[fill=fillColor,fill opacity=0.20] ( 86.75, 64.46) circle (  2.13);

\path[fill=fillColor,fill opacity=0.20] (116.90, 68.62) circle (  2.13);

\path[fill=fillColor,fill opacity=0.20] ( 93.09, 68.72) circle (  2.13);

\path[fill=fillColor,fill opacity=0.20] ( 93.96, 58.03) circle (  2.13);

\path[fill=fillColor,fill opacity=0.20] (103.58, 58.34) circle (  2.13);

\path[fill=fillColor,fill opacity=0.20] (105.10, 67.68) circle (  2.13);

\path[fill=fillColor,fill opacity=0.20] (105.54, 70.80) circle (  2.13);

\path[fill=fillColor,fill opacity=0.20] (108.38, 71.83) circle (  2.13);

\path[fill=fillColor,fill opacity=0.20] (109.47, 74.43) circle (  2.13);

\path[fill=fillColor,fill opacity=0.20] ( 99.86, 67.79) circle (  2.13);

\path[fill=fillColor,fill opacity=0.20] ( 94.62, 55.43) circle (  2.13);

\path[fill=fillColor,fill opacity=0.20] ( 80.85, 58.13) circle (  2.13);

\path[fill=fillColor,fill opacity=0.20] ( 67.96, 80.97) circle (  2.13);

\path[fill=fillColor,fill opacity=0.20] ( 78.67, 58.75) circle (  2.13);

\path[fill=fillColor,fill opacity=0.20] ( 91.34, 47.23) circle (  2.13);

\path[fill=fillColor,fill opacity=0.20] ( 93.74, 54.50) circle (  2.13);

\path[fill=fillColor,fill opacity=0.20] ( 87.84, 57.61) circle (  2.13);

\path[fill=fillColor,fill opacity=0.20] ( 89.15, 56.05) circle (  2.13);

\path[fill=fillColor,fill opacity=0.20] ( 80.41, 68.51) circle (  2.13);

\path[fill=fillColor,fill opacity=0.20] ( 77.57, 74.53) circle (  2.13);

\path[fill=fillColor,fill opacity=0.20] ( 75.61, 65.71) circle (  2.13);

\path[fill=fillColor,fill opacity=0.20] ( 75.17, 55.12) circle (  2.13);

\path[fill=fillColor,fill opacity=0.20] ( 75.39, 52.63) circle (  2.13);

\path[fill=fillColor,fill opacity=0.20] ( 77.57, 61.76) circle (  2.13);

\path[fill=fillColor,fill opacity=0.20] ( 70.80, 69.24) circle (  2.13);

\path[fill=fillColor,fill opacity=0.20] ( 70.58, 60.93) circle (  2.13);

\path[fill=fillColor,fill opacity=0.20] ( 65.99, 68.10) circle (  2.13);

\path[fill=fillColor,fill opacity=0.20] ( 66.65, 66.95) circle (  2.13);

\path[fill=fillColor,fill opacity=0.20] ( 66.21, 69.86) circle (  2.13);

\path[fill=fillColor,fill opacity=0.20] ( 90.68, 73.81) circle (  2.13);

\path[fill=fillColor,fill opacity=0.20] ( 85.66, 62.39) circle (  2.13);

\path[fill=fillColor,fill opacity=0.20] ( 86.10, 62.91) circle (  2.13);

\path[fill=fillColor,fill opacity=0.20] ( 95.71, 78.79) circle (  2.13);

\path[fill=fillColor,fill opacity=0.20] ( 89.59, 76.30) circle (  2.13);

\path[fill=fillColor,fill opacity=0.20] ( 97.68, 64.15) circle (  2.13);

\path[fill=fillColor,fill opacity=0.20] ( 90.68, 64.67) circle (  2.13);

\path[fill=fillColor,fill opacity=0.20] (101.83, 60.62) circle (  2.13);

\path[fill=fillColor,fill opacity=0.20] (107.95, 60.31) circle (  2.13);

\path[fill=fillColor,fill opacity=0.20] (119.31, 67.16) circle (  2.13);

\path[fill=fillColor,fill opacity=0.20] (105.98, 63.84) circle (  2.13);

\path[fill=fillColor,fill opacity=0.20] ( 96.80, 63.84) circle (  2.13);

\path[fill=fillColor,fill opacity=0.20] ( 92.21, 73.81) circle (  2.13);

\path[fill=fillColor,fill opacity=0.20] ( 78.67, 78.48) circle (  2.13);

\path[fill=fillColor,fill opacity=0.20] ( 75.61, 72.46) circle (  2.13);

\path[fill=fillColor,fill opacity=0.20] ( 67.30, 72.77) circle (  2.13);

\path[fill=fillColor,fill opacity=0.20] ( 80.63, 69.45) circle (  2.13);

\path[fill=fillColor,fill opacity=0.20] ( 91.34, 61.14) circle (  2.13);

\path[fill=fillColor,fill opacity=0.20] ( 97.89, 46.19) circle (  2.13);

\path[fill=fillColor,fill opacity=0.20] ( 98.11, 45.15) circle (  2.13);

\path[fill=fillColor,fill opacity=0.20] ( 87.84, 56.68) circle (  2.13);

\path[fill=fillColor,fill opacity=0.20] ( 79.10, 59.06) circle (  2.13);

\path[fill=fillColor,fill opacity=0.20] ( 78.01, 62.08) circle (  2.13);

\path[fill=fillColor,fill opacity=0.20] ( 75.83, 67.89) circle (  2.13);

\path[fill=fillColor,fill opacity=0.20] ( 76.70, 67.37) circle (  2.13);

\path[fill=fillColor,fill opacity=0.20] ( 73.20, 68.93) circle (  2.13);

\path[fill=fillColor,fill opacity=0.20] ( 71.46, 67.89) circle (  2.13);

\path[fill=fillColor,fill opacity=0.20] ( 71.67, 61.56) circle (  2.13);

\path[fill=fillColor,fill opacity=0.20] ( 72.33, 58.96) circle (  2.13);

\path[fill=fillColor,fill opacity=0.20] ( 74.95, 59.79) circle (  2.13);

\path[fill=fillColor,fill opacity=0.20] ( 71.46, 62.59) circle (  2.13);

\path[fill=fillColor,fill opacity=0.20] ( 65.99, 70.38) circle (  2.13);

\path[fill=fillColor,fill opacity=0.20] ( 76.92, 58.44) circle (  2.13);

\path[fill=fillColor,fill opacity=0.20] ( 71.24, 45.67) circle (  2.13);

\path[fill=fillColor,fill opacity=0.20] ( 67.30, 60.83) circle (  2.13);

\path[fill=fillColor,fill opacity=0.20] ( 68.62, 76.19) circle (  2.13);

\path[fill=fillColor,fill opacity=0.20] ( 74.30, 69.13) circle (  2.13);

\path[fill=fillColor,fill opacity=0.20] ( 72.11, 64.15) circle (  2.13);

\path[fill=fillColor,fill opacity=0.20] ( 71.46, 69.45) circle (  2.13);

\path[fill=fillColor,fill opacity=0.20] ( 69.71, 68.62) circle (  2.13);

\path[fill=fillColor,fill opacity=0.20] ( 77.57, 58.75) circle (  2.13);

\path[fill=fillColor,fill opacity=0.20] ( 74.08, 53.35) circle (  2.13);

\path[fill=fillColor,fill opacity=0.20] ( 76.26, 55.74) circle (  2.13);

\path[fill=fillColor,fill opacity=0.20] ( 78.23, 66.12) circle (  2.13);

\path[fill=fillColor,fill opacity=0.20] ( 74.95, 70.80) circle (  2.13);

\path[fill=fillColor,fill opacity=0.20] ( 76.70, 64.26) circle (  2.13);

\path[fill=fillColor,fill opacity=0.20] ( 76.26, 60.21) circle (  2.13);

\path[fill=fillColor,fill opacity=0.20] ( 82.82, 57.40) circle (  2.13);

\path[fill=fillColor,fill opacity=0.20] ( 80.20, 57.20) circle (  2.13);

\path[fill=fillColor,fill opacity=0.20] ( 88.94, 68.10) circle (  2.13);

\path[fill=fillColor,fill opacity=0.20] ( 89.81, 71.00) circle (  2.13);

\path[fill=fillColor,fill opacity=0.20] ( 84.13, 64.36) circle (  2.13);

\path[fill=fillColor,fill opacity=0.20] ( 88.72, 67.37) circle (  2.13);

\path[fill=fillColor,fill opacity=0.20] ( 99.42, 66.02) circle (  2.13);

\path[fill=fillColor,fill opacity=0.20] (101.39, 59.89) circle (  2.13);

\path[fill=fillColor,fill opacity=0.20] (100.30, 56.36) circle (  2.13);

\path[fill=fillColor,fill opacity=0.20] ( 97.68, 55.85) circle (  2.13);

\path[fill=fillColor,fill opacity=0.20] ( 97.02, 64.26) circle (  2.13);

\path[fill=fillColor,fill opacity=0.20] ( 88.72, 77.23) circle (  2.13);

\path[fill=fillColor,fill opacity=0.20] ( 80.20, 71.73) circle (  2.13);

\path[fill=fillColor,fill opacity=0.20] ( 89.15, 72.56) circle (  2.13);

\path[fill=fillColor,fill opacity=0.20] ( 74.73, 78.38) circle (  2.13);

\path[fill=fillColor,fill opacity=0.20] ( 73.20, 70.38) circle (  2.13);

\path[fill=fillColor,fill opacity=0.20] ( 74.30, 75.47) circle (  2.13);

\path[fill=fillColor,fill opacity=0.20] ( 82.82, 60.21) circle (  2.13);

\path[fill=fillColor,fill opacity=0.20] ( 90.90, 48.37) circle (  2.13);

\path[fill=fillColor,fill opacity=0.20] ( 91.78, 51.17) circle (  2.13);

\path[fill=fillColor,fill opacity=0.20] ( 87.19, 57.09) circle (  2.13);

\path[fill=fillColor,fill opacity=0.20] ( 82.82, 53.98) circle (  2.13);

\path[fill=fillColor,fill opacity=0.20] ( 79.76, 56.99) circle (  2.13);

\path[fill=fillColor,fill opacity=0.20] ( 74.95, 60.73) circle (  2.13);

\path[fill=fillColor,fill opacity=0.20] ( 74.51, 60.73) circle (  2.13);

\path[fill=fillColor,fill opacity=0.20] ( 74.73, 65.50) circle (  2.13);

\path[fill=fillColor,fill opacity=0.20] ( 71.02, 68.20) circle (  2.13);

\path[fill=fillColor,fill opacity=0.20] ( 75.61, 65.19) circle (  2.13);

\path[fill=fillColor,fill opacity=0.20] ( 82.82, 66.85) circle (  2.13);

\path[fill=fillColor,fill opacity=0.20] ( 73.20, 68.41) circle (  2.13);

\path[fill=fillColor,fill opacity=0.20] ( 76.48, 64.88) circle (  2.13);

\path[fill=fillColor,fill opacity=0.20] ( 74.08, 56.78) circle (  2.13);

\path[fill=fillColor,fill opacity=0.20] ( 76.26, 46.61) circle (  2.13);

\path[fill=fillColor,fill opacity=0.20] ( 75.83, 49.93) circle (  2.13);

\path[fill=fillColor,fill opacity=0.20] ( 79.32, 56.99) circle (  2.13);

\path[fill=fillColor,fill opacity=0.20] ( 72.77, 52.32) circle (  2.13);

\path[fill=fillColor,fill opacity=0.20] ( 79.32, 52.42) circle (  2.13);

\path[fill=fillColor,fill opacity=0.20] ( 77.79, 59.17) circle (  2.13);

\path[fill=fillColor,fill opacity=0.20] ( 79.76, 60.41) circle (  2.13);

\path[fill=fillColor,fill opacity=0.20] ( 85.22, 61.14) circle (  2.13);

\path[fill=fillColor,fill opacity=0.20] ( 82.60, 63.01) circle (  2.13);

\path[fill=fillColor,fill opacity=0.20] ( 83.69, 62.39) circle (  2.13);

\path[fill=fillColor,fill opacity=0.20] ( 82.38, 66.85) circle (  2.13);

\path[fill=fillColor,fill opacity=0.20] ( 79.54, 71.21) circle (  2.13);

\path[fill=fillColor,fill opacity=0.20] ( 86.97, 67.06) circle (  2.13);

\path[fill=fillColor,fill opacity=0.20] ( 89.15, 59.69) circle (  2.13);

\path[fill=fillColor,fill opacity=0.20] ( 92.21, 57.92) circle (  2.13);

\path[fill=fillColor,fill opacity=0.20] ( 95.71, 61.14) circle (  2.13);

\path[fill=fillColor,fill opacity=0.20] ( 90.68, 50.76) circle (  2.13);

\path[fill=fillColor,fill opacity=0.20] ( 90.68, 48.58) circle (  2.13);

\path[fill=fillColor,fill opacity=0.20] ( 90.68, 58.96) circle (  2.13);

\path[fill=fillColor,fill opacity=0.20] ( 95.27, 55.12) circle (  2.13);

\path[fill=fillColor,fill opacity=0.20] ( 87.19, 62.28) circle (  2.13);

\path[fill=fillColor,fill opacity=0.20] ( 82.16, 66.44) circle (  2.13);

\path[fill=fillColor,fill opacity=0.20] ( 81.51, 68.20) circle (  2.13);

\path[fill=fillColor,fill opacity=0.20] ( 75.61, 83.77) circle (  2.13);

\path[fill=fillColor,fill opacity=0.20] ( 67.09, 86.89) circle (  2.13);

\path[fill=fillColor,fill opacity=0.20] ( 83.25, 79.31) circle (  2.13);

\path[fill=fillColor,fill opacity=0.20] ( 61.19, 86.89) circle (  2.13);

\path[fill=fillColor,fill opacity=0.20] ( 76.92, 75.57) circle (  2.13);

\path[fill=fillColor,fill opacity=0.20] ( 79.32, 66.23) circle (  2.13);

\path[fill=fillColor,fill opacity=0.20] ( 85.00, 63.63) circle (  2.13);

\path[fill=fillColor,fill opacity=0.20] ( 90.68, 56.78) circle (  2.13);

\path[fill=fillColor,fill opacity=0.20] ( 86.75, 46.71) circle (  2.13);

\path[fill=fillColor,fill opacity=0.20] ( 85.44, 45.88) circle (  2.13);

\path[fill=fillColor,fill opacity=0.20] ( 80.41, 55.22) circle (  2.13);

\path[fill=fillColor,fill opacity=0.20] ( 76.48, 65.19) circle (  2.13);

\path[fill=fillColor,fill opacity=0.20] ( 78.45, 66.12) circle (  2.13);

\path[fill=fillColor,fill opacity=0.20] ( 76.04, 67.58) circle (  2.13);

\path[fill=fillColor,fill opacity=0.20] ( 72.55, 72.77) circle (  2.13);

\path[fill=fillColor,fill opacity=0.20] ( 75.17, 71.63) circle (  2.13);

\path[fill=fillColor,fill opacity=0.20] ( 77.36, 65.71) circle (  2.13);

\path[fill=fillColor,fill opacity=0.20] ( 73.86, 64.98) circle (  2.13);

\path[fill=fillColor,fill opacity=0.20] ( 76.92, 65.09) circle (  2.13);

\path[fill=fillColor,fill opacity=0.20] ( 78.23, 62.91) circle (  2.13);

\path[fill=fillColor,fill opacity=0.20] ( 71.67, 60.93) circle (  2.13);

\path[fill=fillColor,fill opacity=0.20] ( 74.51, 59.58) circle (  2.13);

\path[fill=fillColor,fill opacity=0.20] ( 78.01, 60.62) circle (  2.13);

\path[fill=fillColor,fill opacity=0.20] ( 77.79, 60.73) circle (  2.13);

\path[fill=fillColor,fill opacity=0.20] ( 78.01, 62.28) circle (  2.13);

\path[fill=fillColor,fill opacity=0.20] ( 74.73, 69.76) circle (  2.13);

\path[fill=fillColor,fill opacity=0.20] ( 74.73, 72.25) circle (  2.13);

\path[fill=fillColor,fill opacity=0.20] ( 77.36, 68.10) circle (  2.13);

\path[fill=fillColor,fill opacity=0.20] ( 80.85, 63.84) circle (  2.13);

\path[fill=fillColor,fill opacity=0.20] ( 83.25, 59.58) circle (  2.13);

\path[fill=fillColor,fill opacity=0.20] ( 88.94, 57.82) circle (  2.13);

\path[fill=fillColor,fill opacity=0.20] ( 88.06, 54.70) circle (  2.13);

\path[fill=fillColor,fill opacity=0.20] ( 98.77, 50.76) circle (  2.13);

\path[fill=fillColor,fill opacity=0.20] ( 96.36, 58.23) circle (  2.13);

\path[fill=fillColor,fill opacity=0.20] ( 98.33, 65.92) circle (  2.13);

\path[fill=fillColor,fill opacity=0.20] ( 81.07, 62.49) circle (  2.13);

\path[fill=fillColor,fill opacity=0.20] ( 85.00, 63.11) circle (  2.13);

\path[fill=fillColor,fill opacity=0.20] ( 76.70, 72.56) circle (  2.13);

\path[fill=fillColor,fill opacity=0.20] ( 70.58, 81.70) circle (  2.13);

\path[fill=fillColor,fill opacity=0.20] ( 54.63,104.54) circle (  2.13);

\path[fill=fillColor,fill opacity=0.20] ( 75.17, 80.24) circle (  2.13);

\path[fill=fillColor,fill opacity=0.20] ( 81.73, 59.69) circle (  2.13);

\path[fill=fillColor,fill opacity=0.20] ( 84.57, 49.31) circle (  2.13);

\path[fill=fillColor,fill opacity=0.20] ( 90.47, 58.03) circle (  2.13);

\path[fill=fillColor,fill opacity=0.20] ( 89.37, 66.12) circle (  2.13);

\path[fill=fillColor,fill opacity=0.20] ( 87.19, 62.18) circle (  2.13);

\path[fill=fillColor,fill opacity=0.20] ( 89.37, 60.62) circle (  2.13);

\path[fill=fillColor,fill opacity=0.20] ( 88.94, 63.22) circle (  2.13);

\path[fill=fillColor,fill opacity=0.20] ( 81.73, 60.41) circle (  2.13);

\path[fill=fillColor,fill opacity=0.20] ( 79.76, 55.53) circle (  2.13);

\path[fill=fillColor,fill opacity=0.20] ( 82.60, 50.55) circle (  2.13);

\path[fill=fillColor,fill opacity=0.20] ( 82.38, 52.00) circle (  2.13);

\path[fill=fillColor,fill opacity=0.20] ( 77.79, 58.44) circle (  2.13);

\path[fill=fillColor,fill opacity=0.20] ( 79.32, 58.96) circle (  2.13);

\path[fill=fillColor,fill opacity=0.20] ( 78.67, 52.52) circle (  2.13);

\path[fill=fillColor,fill opacity=0.20] ( 78.23, 47.75) circle (  2.13);

\path[fill=fillColor,fill opacity=0.20] ( 79.32, 44.94) circle (  2.13);

\path[fill=fillColor,fill opacity=0.20] ( 83.47, 50.14) circle (  2.13);

\path[fill=fillColor,fill opacity=0.20] ( 89.37, 58.86) circle (  2.13);

\path[fill=fillColor,fill opacity=0.20] ( 83.04, 59.38) circle (  2.13);

\path[fill=fillColor,fill opacity=0.20] ( 87.62, 57.51) circle (  2.13);

\path[fill=fillColor,fill opacity=0.20] ( 88.50, 57.51) circle (  2.13);

\path[fill=fillColor,fill opacity=0.20] ( 92.87, 54.29) circle (  2.13);

\path[fill=fillColor,fill opacity=0.20] ( 89.81, 57.82) circle (  2.13);

\path[fill=fillColor,fill opacity=0.20] ( 83.69, 70.48) circle (  2.13);

\path[fill=fillColor,fill opacity=0.20] ( 71.67, 77.65) circle (  2.13);

\path[fill=fillColor,fill opacity=0.20] ( 67.52, 85.85) circle (  2.13);

\path[fill=fillColor,fill opacity=0.20] ( 52.01,109.73) circle (  2.13);

\path[fill=fillColor,fill opacity=0.20] ( 53.54,113.88) circle (  2.13);

\path[fill=fillColor,fill opacity=0.20] ( 67.52, 80.24) circle (  2.13);

\path[fill=fillColor,fill opacity=0.20] ( 81.07, 75.36) circle (  2.13);

\path[fill=fillColor,fill opacity=0.20] ( 88.28, 70.69) circle (  2.13);

\path[fill=fillColor,fill opacity=0.20] ( 79.32, 69.55) circle (  2.13);

\path[fill=fillColor,fill opacity=0.20] ( 93.96, 60.93) circle (  2.13);

\path[fill=fillColor,fill opacity=0.20] ( 92.65, 46.92) circle (  2.13);

\path[fill=fillColor,fill opacity=0.20] ( 97.89, 52.63) circle (  2.13);

\path[fill=fillColor,fill opacity=0.20] ( 90.68, 62.59) circle (  2.13);

\path[fill=fillColor,fill opacity=0.20] ( 93.96, 48.37) circle (  2.13);

\path[fill=fillColor,fill opacity=0.20] ( 85.44, 51.28) circle (  2.13);

\path[fill=fillColor,fill opacity=0.20] ( 82.82, 57.61) circle (  2.13);

\path[fill=fillColor,fill opacity=0.20] ( 85.22, 52.94) circle (  2.13);

\path[fill=fillColor,fill opacity=0.20] ( 85.88, 49.93) circle (  2.13);

\path[fill=fillColor,fill opacity=0.20] ( 87.62, 49.62) circle (  2.13);

\path[fill=fillColor,fill opacity=0.20] ( 83.91, 54.70) circle (  2.13);

\path[fill=fillColor,fill opacity=0.20] ( 89.37, 61.76) circle (  2.13);

\path[fill=fillColor,fill opacity=0.20] ( 86.97, 65.71) circle (  2.13);

\path[fill=fillColor,fill opacity=0.20] ( 75.17, 76.71) circle (  2.13);

\path[fill=fillColor,fill opacity=0.20] ( 74.30, 84.81) circle (  2.13);

\path[fill=fillColor,fill opacity=0.20] ( 76.26, 81.70) circle (  2.13);

\path[fill=fillColor,fill opacity=0.20] ( 52.01,101.42) circle (  2.13);

\path[fill=fillColor,fill opacity=0.20] ( 57.69, 93.12) circle (  2.13);

\path[fill=fillColor,fill opacity=0.20] ( 66.21, 74.95) circle (  2.13);

\path[fill=fillColor,fill opacity=0.20] ( 80.85, 78.48) circle (  2.13);

\path[fill=fillColor,fill opacity=0.20] ( 79.76, 84.81) circle (  2.13);

\path[fill=fillColor,fill opacity=0.20] ( 77.14, 77.86) circle (  2.13);

\path[fill=fillColor,fill opacity=0.20] ( 72.55, 74.85) circle (  2.13);

\path[fill=fillColor,fill opacity=0.20] ( 69.27, 74.12) circle (  2.13);

\path[fill=fillColor,fill opacity=0.20] ( 71.67, 74.53) circle (  2.13);

\path[fill=fillColor,fill opacity=0.20] ( 71.24, 77.23) circle (  2.13);

\path[fill=fillColor,fill opacity=0.20] ( 77.36, 77.96) circle (  2.13);

\path[fill=fillColor,fill opacity=0.20] ( 74.95,102.46) circle (  2.13);

\path[fill=fillColor,fill opacity=0.20] ( 73.20, 95.19) circle (  2.13);

\path[fill=fillColor,fill opacity=0.20] ( 91.99,113.88) circle (  2.13);

\path[fill=fillColor,fill opacity=0.20] (111.66, 95.19) circle (  2.13);

\path[fill=fillColor,fill opacity=0.20] ( 97.68, 68.72) circle (  2.13);

\path[fill=fillColor,fill opacity=0.20] ( 96.58, 59.38) circle (  2.13);

\path[fill=fillColor,fill opacity=0.20] ( 89.15, 71.11) circle (  2.13);

\path[fill=fillColor,fill opacity=0.20] ( 94.18, 71.11) circle (  2.13);

\path[fill=fillColor,fill opacity=0.20] ( 88.72, 61.35) circle (  2.13);

\path[fill=fillColor,fill opacity=0.20] ( 86.75, 77.13) circle (  2.13);

\path[fill=fillColor,fill opacity=0.20] (130.89, 58.96) circle (  2.13);

\path[fill=fillColor,fill opacity=0.20] (112.10, 65.71) circle (  2.13);

\path[fill=fillColor,fill opacity=0.20] (109.91, 55.95) circle (  2.13);

\path[fill=fillColor,fill opacity=0.20] (118.21, 54.29) circle (  2.13);

\path[fill=fillColor,fill opacity=0.20] (108.82, 71.73) circle (  2.13);

\path[fill=fillColor,fill opacity=0.20] (108.38, 66.64) circle (  2.13);

\path[fill=fillColor,fill opacity=0.20] (109.26, 52.84) circle (  2.13);

\path[fill=fillColor,fill opacity=0.20] ( 90.25, 65.09) circle (  2.13);

\path[fill=fillColor,fill opacity=0.20] ( 62.28, 99.35) circle (  2.13);

\path[fill=fillColor,fill opacity=0.20] (128.48, 70.90) circle (  2.13);

\path[fill=fillColor,fill opacity=0.20] (106.20, 55.53) circle (  2.13);

\path[fill=fillColor,fill opacity=0.20] (101.83, 79.10) circle (  2.13);

\path[fill=fillColor,fill opacity=0.20] (124.99, 65.71) circle (  2.13);

\path[fill=fillColor,fill opacity=0.20] (122.80, 63.74) circle (  2.13);

\path[fill=fillColor,fill opacity=0.20] (114.28, 72.25) circle (  2.13);

\path[fill=fillColor,fill opacity=0.20] (119.96, 68.62) circle (  2.13);

\path[fill=fillColor,fill opacity=0.20] (130.67, 58.34) circle (  2.13);

\path[fill=fillColor,fill opacity=0.20] (114.94, 61.04) circle (  2.13);

\path[fill=fillColor,fill opacity=0.20] ( 75.39, 79.41) circle (  2.13);

\path[fill=fillColor,fill opacity=0.20] (124.99, 52.52) circle (  2.13);

\path[fill=fillColor,fill opacity=0.20] (106.20, 61.14) circle (  2.13);

\path[fill=fillColor,fill opacity=0.20] (116.90, 71.52) circle (  2.13);

\path[fill=fillColor,fill opacity=0.20] (146.40, 63.74) circle (  2.13);

\path[fill=fillColor,fill opacity=0.20] (136.13, 63.42) circle (  2.13);

\path[fill=fillColor,fill opacity=0.20] (113.63, 55.43) circle (  2.13);

\path[fill=fillColor,fill opacity=0.20] (122.15, 53.67) circle (  2.13);

\path[fill=fillColor,fill opacity=0.20] (139.41, 61.04) circle (  2.13);

\path[fill=fillColor,fill opacity=0.20] (125.43, 55.43) circle (  2.13);

\path[fill=fillColor,fill opacity=0.20] ( 85.00, 62.18) circle (  2.13);

\path[fill=fillColor,fill opacity=0.20] ( 45.89, 91.04) circle (  2.13);

\path[fill=fillColor,fill opacity=0.20] ( 45.45, 84.81) circle (  2.13);

\path[fill=fillColor,fill opacity=0.20] ( 52.23, 88.96) circle (  2.13);

\path[fill=fillColor,fill opacity=0.20] (112.53, 57.20) circle (  2.13);

\path[fill=fillColor,fill opacity=0.20] (120.40, 46.19) circle (  2.13);

\path[fill=fillColor,fill opacity=0.20] (123.90, 53.46) circle (  2.13);

\path[fill=fillColor,fill opacity=0.20] (125.21, 65.29) circle (  2.13);

\path[fill=fillColor,fill opacity=0.20] (126.08, 67.16) circle (  2.13);

\path[fill=fillColor,fill opacity=0.20] (125.43, 49.20) circle (  2.13);

\path[fill=fillColor,fill opacity=0.20] (132.20, 42.87) circle (  2.13);

\path[fill=fillColor,fill opacity=0.20] (135.04, 54.91) circle (  2.13);

\path[fill=fillColor,fill opacity=0.20] (119.31, 55.12) circle (  2.13);

\path[fill=fillColor,fill opacity=0.20] ( 59.00,104.54) circle (  2.13);

\path[fill=fillColor,fill opacity=0.20] ( 71.02, 61.24) circle (  2.13);

\path[fill=fillColor,fill opacity=0.20] ( 79.98, 60.31) circle (  2.13);

\path[fill=fillColor,fill opacity=0.20] ( 92.21, 67.79) circle (  2.13);

\path[fill=fillColor,fill opacity=0.20] ( 93.52, 50.86) circle (  2.13);

\path[fill=fillColor,fill opacity=0.20] ( 94.18, 43.70) circle (  2.13);

\path[fill=fillColor,fill opacity=0.20] ( 74.73, 58.44) circle (  2.13);

\path[fill=fillColor,fill opacity=0.20] ( 79.10, 53.56) circle (  2.13);

\path[fill=fillColor,fill opacity=0.20] ( 76.26, 52.21) circle (  2.13);

\path[fill=fillColor,fill opacity=0.20] ( 93.09, 70.17) circle (  2.13);

\path[fill=fillColor,fill opacity=0.20] (104.89, 58.03) circle (  2.13);

\path[fill=fillColor,fill opacity=0.20] (114.06, 53.46) circle (  2.13);

\path[fill=fillColor,fill opacity=0.20] (118.87, 63.22) circle (  2.13);

\path[fill=fillColor,fill opacity=0.20] (126.95, 67.27) circle (  2.13);

\path[fill=fillColor,fill opacity=0.20] (134.38, 60.31) circle (  2.13);

\path[fill=fillColor,fill opacity=0.20] (138.32, 54.39) circle (  2.13);

\path[fill=fillColor,fill opacity=0.20] (144.22, 49.82) circle (  2.13);

\path[fill=fillColor,fill opacity=0.20] (122.37, 54.60) circle (  2.13);

\path[fill=fillColor,fill opacity=0.20] ( 84.57, 97.27) circle (  2.13);

\path[fill=fillColor,fill opacity=0.20] (104.23, 60.21) circle (  2.13);

\path[fill=fillColor,fill opacity=0.20] (111.44, 45.88) circle (  2.13);

\path[fill=fillColor,fill opacity=0.20] ( 80.85, 40.38) circle (  2.13);

\path[fill=fillColor,fill opacity=0.20] ( 78.45, 52.52) circle (  2.13);

\path[fill=fillColor,fill opacity=0.20] ( 72.11, 69.03) circle (  2.13);

\path[fill=fillColor,fill opacity=0.20] ( 73.20, 69.45) circle (  2.13);

\path[fill=fillColor,fill opacity=0.20] ( 87.41, 81.70) circle (  2.13);

\path[fill=fillColor,fill opacity=0.20] ( 92.21, 64.15) circle (  2.13);

\path[fill=fillColor,fill opacity=0.20] ( 97.02, 69.45) circle (  2.13);

\path[fill=fillColor,fill opacity=0.20] (106.42, 58.86) circle (  2.13);

\path[fill=fillColor,fill opacity=0.20] (112.32, 47.02) circle (  2.13);

\path[fill=fillColor,fill opacity=0.20] (118.87, 48.16) circle (  2.13);

\path[fill=fillColor,fill opacity=0.20] (126.08, 59.06) circle (  2.13);

\path[fill=fillColor,fill opacity=0.20] (135.04, 61.45) circle (  2.13);

\path[fill=fillColor,fill opacity=0.20] (142.91, 48.79) circle (  2.13);

\path[fill=fillColor,fill opacity=0.20] (104.67, 46.81) circle (  2.13);

\path[fill=fillColor,fill opacity=0.20] (112.53, 41.52) circle (  2.13);

\path[fill=fillColor,fill opacity=0.20] (111.44, 57.92) circle (  2.13);

\path[fill=fillColor,fill opacity=0.20] (106.42, 60.41) circle (  2.13);

\path[fill=fillColor,fill opacity=0.20] ( 94.62, 57.40) circle (  2.13);

\path[fill=fillColor,fill opacity=0.20] ( 89.15, 58.86) circle (  2.13);

\path[fill=fillColor,fill opacity=0.20] ( 78.01, 57.30) circle (  2.13);

\path[fill=fillColor,fill opacity=0.20] ( 82.82, 59.58) circle (  2.13);

\path[fill=fillColor,fill opacity=0.20] ( 80.85, 65.40) circle (  2.13);

\path[fill=fillColor,fill opacity=0.20] (109.26, 51.38) circle (  2.13);

\path[fill=fillColor,fill opacity=0.20] (106.42, 43.49) circle (  2.13);

\path[fill=fillColor,fill opacity=0.20] (107.95, 55.53) circle (  2.13);

\path[fill=fillColor,fill opacity=0.20] (114.28, 64.67) circle (  2.13);

\path[fill=fillColor,fill opacity=0.20] (111.22, 44.94) circle (  2.13);

\path[fill=fillColor,fill opacity=0.20] (119.31, 37.99) circle (  2.13);

\path[fill=fillColor,fill opacity=0.20] (135.69, 53.98) circle (  2.13);

\path[fill=fillColor,fill opacity=0.20] (135.04, 56.47) circle (  2.13);

\path[fill=fillColor,fill opacity=0.20] (119.96, 48.16) circle (  2.13);

\path[fill=fillColor,fill opacity=0.20] ( 88.94, 46.50) circle (  2.13);

\path[fill=fillColor,fill opacity=0.20] ( 85.00, 85.85) circle (  2.13);

\path[fill=fillColor,fill opacity=0.20] (109.47, 44.32) circle (  2.13);

\path[fill=fillColor,fill opacity=0.20] (107.29, 60.41) circle (  2.13);

\path[fill=fillColor,fill opacity=0.20] (105.10, 73.81) circle (  2.13);

\path[fill=fillColor,fill opacity=0.20] (104.23, 64.05) circle (  2.13);

\path[fill=fillColor,fill opacity=0.20] (101.17, 65.81) circle (  2.13);

\path[fill=fillColor,fill opacity=0.20] ( 87.19, 67.58) circle (  2.13);

\path[fill=fillColor,fill opacity=0.20] ( 86.10, 53.87) circle (  2.13);

\path[fill=fillColor,fill opacity=0.20] ( 92.87, 47.75) circle (  2.13);

\path[fill=fillColor,fill opacity=0.20] ( 89.81, 53.67) circle (  2.13);

\path[fill=fillColor,fill opacity=0.20] ( 73.42, 87.93) circle (  2.13);

\path[fill=fillColor,fill opacity=0.20] ( 88.28, 53.35) circle (  2.13);

\path[fill=fillColor,fill opacity=0.20] (105.32, 58.75) circle (  2.13);

\path[fill=fillColor,fill opacity=0.20] (114.28, 47.02) circle (  2.13);

\path[fill=fillColor,fill opacity=0.20] (113.84, 56.36) circle (  2.13);

\path[fill=fillColor,fill opacity=0.20] (126.08, 72.56) circle (  2.13);

\path[fill=fillColor,fill opacity=0.20] (119.31, 68.20) circle (  2.13);

\path[fill=fillColor,fill opacity=0.20] (119.96, 64.05) circle (  2.13);

\path[fill=fillColor,fill opacity=0.20] (137.01, 63.01) circle (  2.13);

\path[fill=fillColor,fill opacity=0.20] (131.32, 58.13) circle (  2.13);

\path[fill=fillColor,fill opacity=0.20] ( 94.40, 50.65) circle (  2.13);

\path[fill=fillColor,fill opacity=0.20] ( 95.49, 54.39) circle (  2.13);

\path[fill=fillColor,fill opacity=0.20] (117.34, 42.87) circle (  2.13);

\path[fill=fillColor,fill opacity=0.20] (138.97, 67.99) circle (  2.13);

\path[fill=fillColor,fill opacity=0.20] (104.45, 63.74) circle (  2.13);

\path[fill=fillColor,fill opacity=0.20] (110.35, 54.29) circle (  2.13);

\path[fill=fillColor,fill opacity=0.20] ( 98.77, 68.51) circle (  2.13);

\path[fill=fillColor,fill opacity=0.20] ( 91.34, 69.03) circle (  2.13);

\path[fill=fillColor,fill opacity=0.20] ( 93.52, 57.09) circle (  2.13);

\path[fill=fillColor,fill opacity=0.20] ( 87.62, 54.81) circle (  2.13);

\path[fill=fillColor,fill opacity=0.20] ( 69.27,106.61) circle (  2.13);

\path[fill=fillColor,fill opacity=0.20] ( 89.15, 55.43) circle (  2.13);

\path[fill=fillColor,fill opacity=0.20] ( 95.05, 70.38) circle (  2.13);

\path[fill=fillColor,fill opacity=0.20] (127.83, 68.62) circle (  2.13);

\path[fill=fillColor,fill opacity=0.20] (118.21, 59.38) circle (  2.13);

\path[fill=fillColor,fill opacity=0.20] (118.43, 62.91) circle (  2.13);

\path[fill=fillColor,fill opacity=0.20] (137.44, 63.22) circle (  2.13);

\path[fill=fillColor,fill opacity=0.20] (138.54, 61.04) circle (  2.13);

\path[fill=fillColor,fill opacity=0.20] (119.31, 53.77) circle (  2.13);

\path[fill=fillColor,fill opacity=0.20] ( 72.77,114.92) circle (  2.13);

\path[fill=fillColor,fill opacity=0.20] ( 98.55, 57.92) circle (  2.13);

\path[fill=fillColor,fill opacity=0.20] (121.49, 48.99) circle (  2.13);

\path[fill=fillColor,fill opacity=0.20] (124.77, 53.56) circle (  2.13);

\path[fill=fillColor,fill opacity=0.20] (105.32, 39.65) circle (  2.13);

\path[fill=fillColor,fill opacity=0.20] (102.92, 42.56) circle (  2.13);

\path[fill=fillColor,fill opacity=0.20] ( 96.58, 64.46) circle (  2.13);

\path[fill=fillColor,fill opacity=0.20] ( 83.25, 66.12) circle (  2.13);

\path[fill=fillColor,fill opacity=0.20] ( 80.41, 60.93) circle (  2.13);

\path[fill=fillColor,fill opacity=0.20] ( 76.70, 64.67) circle (  2.13);

\path[fill=fillColor,fill opacity=0.20] ( 74.73,108.69) circle (  2.13);

\path[fill=fillColor,fill opacity=0.20] ( 81.07, 61.87) circle (  2.13);

\path[fill=fillColor,fill opacity=0.20] ( 89.37, 70.48) circle (  2.13);

\path[fill=fillColor,fill opacity=0.20] (104.67, 70.59) circle (  2.13);

\path[fill=fillColor,fill opacity=0.20] (111.00, 62.70) circle (  2.13);

\path[fill=fillColor,fill opacity=0.20] (108.38, 61.66) circle (  2.13);

\path[fill=fillColor,fill opacity=0.20] (123.24, 56.26) circle (  2.13);

\path[fill=fillColor,fill opacity=0.20] (135.91, 41.93) circle (  2.13);

\path[fill=fillColor,fill opacity=0.20] (147.28, 61.35) circle (  2.13);

\path[fill=fillColor,fill opacity=0.20] (138.97, 49.72) circle (  2.13);

\path[fill=fillColor,fill opacity=0.20] ( 82.60, 73.18) circle (  2.13);

\path[fill=fillColor,fill opacity=0.20] (122.80, 59.27) circle (  2.13);

\path[fill=fillColor,fill opacity=0.20] (131.76, 53.25) circle (  2.13);

\path[fill=fillColor,fill opacity=0.20] (107.73, 45.57) circle (  2.13);

\path[fill=fillColor,fill opacity=0.20] (103.14, 50.45) circle (  2.13);

\path[fill=fillColor,fill opacity=0.20] ( 93.96, 59.38) circle (  2.13);

\path[fill=fillColor,fill opacity=0.20] ( 80.63, 61.45) circle (  2.13);

\path[fill=fillColor,fill opacity=0.20] ( 75.17, 60.21) circle (  2.13);

\path[fill=fillColor,fill opacity=0.20] ( 72.99, 56.47) circle (  2.13);

\path[fill=fillColor,fill opacity=0.20] ( 62.93,108.69) circle (  2.13);

\path[fill=fillColor,fill opacity=0.20] ( 88.50, 63.94) circle (  2.13);

\path[fill=fillColor,fill opacity=0.20] ( 89.15, 78.48) circle (  2.13);

\path[fill=fillColor,fill opacity=0.20] ( 95.05, 78.06) circle (  2.13);

\path[fill=fillColor,fill opacity=0.20] (108.60, 63.22) circle (  2.13);

\path[fill=fillColor,fill opacity=0.20] (100.08, 61.76) circle (  2.13);

\path[fill=fillColor,fill opacity=0.20] (103.58, 64.15) circle (  2.13);

\path[fill=fillColor,fill opacity=0.20] (121.27, 57.09) circle (  2.13);

\path[fill=fillColor,fill opacity=0.20] (123.90, 45.46) circle (  2.13);

\path[fill=fillColor,fill opacity=0.20] (125.64, 59.69) circle (  2.13);

\path[fill=fillColor,fill opacity=0.20] ( 99.42, 55.64) circle (  2.13);

\path[fill=fillColor,fill opacity=0.20] ( 74.30, 85.85) circle (  2.13);

\path[fill=fillColor,fill opacity=0.20] (107.73, 58.86) circle (  2.13);

\path[fill=fillColor,fill opacity=0.20] (135.91, 59.79) circle (  2.13);

\path[fill=fillColor,fill opacity=0.20] (123.02, 68.62) circle (  2.13);

\path[fill=fillColor,fill opacity=0.20] (102.92, 66.54) circle (  2.13);

\path[fill=fillColor,fill opacity=0.20] (100.30, 61.76) circle (  2.13);

\path[fill=fillColor,fill opacity=0.20] ( 88.50, 64.88) circle (  2.13);

\path[fill=fillColor,fill opacity=0.20] ( 89.37, 61.76) circle (  2.13);

\path[fill=fillColor,fill opacity=0.20] ( 80.41, 51.28) circle (  2.13);

\path[fill=fillColor,fill opacity=0.20] ( 61.40, 73.60) circle (  2.13);

\path[fill=fillColor,fill opacity=0.20] ( 63.15, 99.35) circle (  2.13);

\path[fill=fillColor,fill opacity=0.20] ( 78.88, 66.95) circle (  2.13);

\path[fill=fillColor,fill opacity=0.20] ( 88.72, 71.73) circle (  2.13);

\path[fill=fillColor,fill opacity=0.20] ( 99.64, 68.72) circle (  2.13);

\path[fill=fillColor,fill opacity=0.20] (112.75, 70.48) circle (  2.13);

\path[fill=fillColor,fill opacity=0.20] (118.65, 66.12) circle (  2.13);

\path[fill=fillColor,fill opacity=0.20] ( 99.42, 61.87) circle (  2.13);

\path[fill=fillColor,fill opacity=0.20] (102.92, 62.49) circle (  2.13);

\path[fill=fillColor,fill opacity=0.20] (120.84, 60.93) circle (  2.13);

\path[fill=fillColor,fill opacity=0.20] (114.06, 60.21) circle (  2.13);

\path[fill=fillColor,fill opacity=0.20] (107.73, 61.76) circle (  2.13);

\path[fill=fillColor,fill opacity=0.20] ( 97.24, 58.86) circle (  2.13);

\path[fill=fillColor,fill opacity=0.20] ( 81.73, 67.06) circle (  2.13);

\path[fill=fillColor,fill opacity=0.20] (108.82, 61.35) circle (  2.13);

\path[fill=fillColor,fill opacity=0.20] (116.90, 65.61) circle (  2.13);

\path[fill=fillColor,fill opacity=0.20] (114.50, 59.79) circle (  2.13);

\path[fill=fillColor,fill opacity=0.20] (109.26, 64.67) circle (  2.13);

\path[fill=fillColor,fill opacity=0.20] (108.82, 71.00) circle (  2.13);

\path[fill=fillColor,fill opacity=0.20] (102.92, 62.70) circle (  2.13);

\path[fill=fillColor,fill opacity=0.20] ( 88.50, 61.66) circle (  2.13);

\path[fill=fillColor,fill opacity=0.20] ( 73.42, 74.43) circle (  2.13);

\path[fill=fillColor,fill opacity=0.20] ( 63.37, 94.16) circle (  2.13);

\path[fill=fillColor,fill opacity=0.20] ( 82.38, 60.93) circle (  2.13);

\path[fill=fillColor,fill opacity=0.20] ( 94.40, 75.47) circle (  2.13);

\path[fill=fillColor,fill opacity=0.20] ( 95.27, 67.27) circle (  2.13);

\path[fill=fillColor,fill opacity=0.20] (102.70, 53.15) circle (  2.13);

\path[fill=fillColor,fill opacity=0.20] (116.47, 59.89) circle (  2.13);

\path[fill=fillColor,fill opacity=0.20] (116.90, 55.33) circle (  2.13);

\path[fill=fillColor,fill opacity=0.20] (109.04, 45.57) circle (  2.13);

\path[fill=fillColor,fill opacity=0.20] (112.97, 47.64) circle (  2.13);

\path[fill=fillColor,fill opacity=0.20] (105.54, 54.50) circle (  2.13);

\path[fill=fillColor,fill opacity=0.20] ( 94.62, 61.14) circle (  2.13);

\path[fill=fillColor,fill opacity=0.20] ( 81.29, 61.76) circle (  2.13);

\path[fill=fillColor,fill opacity=0.20] ( 66.21, 84.81) circle (  2.13);

\path[fill=fillColor,fill opacity=0.20] ( 91.34, 78.89) circle (  2.13);

\path[fill=fillColor,fill opacity=0.20] (113.41, 61.04) circle (  2.13);

\path[fill=fillColor,fill opacity=0.20] (111.22, 48.06) circle (  2.13);

\path[fill=fillColor,fill opacity=0.20] (111.66, 63.84) circle (  2.13);

\path[fill=fillColor,fill opacity=0.20] (108.82, 67.79) circle (  2.13);

\path[fill=fillColor,fill opacity=0.20] (104.01, 55.33) circle (  2.13);

\path[fill=fillColor,fill opacity=0.20] ( 94.84, 61.24) circle (  2.13);

\path[fill=fillColor,fill opacity=0.20] ( 80.85, 71.94) circle (  2.13);

\path[fill=fillColor,fill opacity=0.20] ( 69.05, 77.03) circle (  2.13);

\path[fill=fillColor,fill opacity=0.20] ( 50.48,113.88) circle (  2.13);

\path[fill=fillColor,fill opacity=0.20] ( 70.80, 86.89) circle (  2.13);

\path[fill=fillColor,fill opacity=0.20] ( 78.23, 68.93) circle (  2.13);

\path[fill=fillColor,fill opacity=0.20] ( 87.62, 75.47) circle (  2.13);

\path[fill=fillColor,fill opacity=0.20] ( 99.42, 73.60) circle (  2.13);

\path[fill=fillColor,fill opacity=0.20] (108.38, 63.63) circle (  2.13);

\path[fill=fillColor,fill opacity=0.20] (114.28, 48.58) circle (  2.13);

\path[fill=fillColor,fill opacity=0.20] (110.13, 39.55) circle (  2.13);

\path[fill=fillColor,fill opacity=0.20] (104.45, 47.23) circle (  2.13);

\path[fill=fillColor,fill opacity=0.20] ( 77.57, 52.63) circle (  2.13);

\path[fill=fillColor,fill opacity=0.20] ( 77.14, 62.70) circle (  2.13);

\path[fill=fillColor,fill opacity=0.20] ( 83.47, 72.77) circle (  2.13);

\path[fill=fillColor,fill opacity=0.20] ( 80.85, 92.08) circle (  2.13);

\path[fill=fillColor,fill opacity=0.20] (124.77, 52.11) circle (  2.13);

\path[fill=fillColor,fill opacity=0.20] (112.75, 39.55) circle (  2.13);

\path[fill=fillColor,fill opacity=0.20] (112.10, 64.98) circle (  2.13);

\path[fill=fillColor,fill opacity=0.20] (107.51, 69.34) circle (  2.13);

\path[fill=fillColor,fill opacity=0.20] ( 97.02, 54.60) circle (  2.13);

\path[fill=fillColor,fill opacity=0.20] ( 96.36, 59.48) circle (  2.13);

\path[fill=fillColor,fill opacity=0.20] ( 88.72, 72.15) circle (  2.13);

\path[fill=fillColor,fill opacity=0.20] ( 83.47, 68.72) circle (  2.13);

\path[fill=fillColor,fill opacity=0.20] ( 79.98, 71.21) circle (  2.13);

\path[fill=fillColor,fill opacity=0.20] ( 57.91,105.58) circle (  2.13);

\path[fill=fillColor,fill opacity=0.20] ( 77.36, 99.35) circle (  2.13);

\path[fill=fillColor,fill opacity=0.20] ( 74.73, 72.46) circle (  2.13);

\path[fill=fillColor,fill opacity=0.20] ( 93.96, 77.23) circle (  2.13);

\path[fill=fillColor,fill opacity=0.20] ( 95.93, 83.77) circle (  2.13);

\path[fill=fillColor,fill opacity=0.20] (103.14, 65.71) circle (  2.13);

\path[fill=fillColor,fill opacity=0.20] (110.79, 48.89) circle (  2.13);

\path[fill=fillColor,fill opacity=0.20] (117.78, 40.38) circle (  2.13);

\path[fill=fillColor,fill opacity=0.20] (102.05, 46.40) circle (  2.13);

\path[fill=fillColor,fill opacity=0.20] (102.26, 42.14) circle (  2.13);

\path[fill=fillColor,fill opacity=0.20] (134.82, 65.81) circle (  2.13);

\path[fill=fillColor,fill opacity=0.20] (103.79, 61.14) circle (  2.13);

\path[fill=fillColor,fill opacity=0.20] ( 96.58, 63.74) circle (  2.13);

\path[fill=fillColor,fill opacity=0.20] ( 92.65, 64.46) circle (  2.13);

\path[fill=fillColor,fill opacity=0.20] ( 95.05, 64.15) circle (  2.13);

\path[fill=fillColor,fill opacity=0.20] ( 84.35, 61.66) circle (  2.13);

\path[fill=fillColor,fill opacity=0.20] ( 75.39, 61.14) circle (  2.13);

\path[fill=fillColor,fill opacity=0.20] ( 63.37, 95.19) circle (  2.13);

\path[fill=fillColor,fill opacity=0.20] ( 53.98,105.58) circle (  2.13);

\path[fill=fillColor,fill opacity=0.20] ( 64.90, 96.23) circle (  2.13);

\path[fill=fillColor,fill opacity=0.20] ( 78.88, 75.47) circle (  2.13);

\path[fill=fillColor,fill opacity=0.20] ( 95.05, 66.95) circle (  2.13);

\path[fill=fillColor,fill opacity=0.20] ( 95.93, 79.52) circle (  2.13);

\path[fill=fillColor,fill opacity=0.20] (106.42, 79.00) circle (  2.13);

\path[fill=fillColor,fill opacity=0.20] (115.37, 60.10) circle (  2.13);

\path[fill=fillColor,fill opacity=0.20] (127.83, 47.33) circle (  2.13);

\path[fill=fillColor,fill opacity=0.20] ( 83.25, 37.99) circle (  2.13);

\path[fill=fillColor,fill opacity=0.20] ( 87.84, 68.51) circle (  2.13);

\path[fill=fillColor,fill opacity=0.20] (102.05, 59.06) circle (  2.13);

\path[fill=fillColor,fill opacity=0.20] (107.51, 56.68) circle (  2.13);

\path[fill=fillColor,fill opacity=0.20] (100.73, 64.05) circle (  2.13);

\path[fill=fillColor,fill opacity=0.20] ( 97.46, 67.16) circle (  2.13);

\path[fill=fillColor,fill opacity=0.20] ( 88.28, 71.32) circle (  2.13);

\path[fill=fillColor,fill opacity=0.20] ( 87.41, 75.68) circle (  2.13);

\path[fill=fillColor,fill opacity=0.20] ( 82.82, 64.67) circle (  2.13);

\path[fill=fillColor,fill opacity=0.20] ( 79.98, 62.91) circle (  2.13);

\path[fill=fillColor,fill opacity=0.20] ( 64.68, 87.93) circle (  2.13);

\path[fill=fillColor,fill opacity=0.20] ( 64.90, 82.74) circle (  2.13);

\path[fill=fillColor,fill opacity=0.20] ( 65.12, 83.77) circle (  2.13);

\path[fill=fillColor,fill opacity=0.20] ( 67.52, 81.70) circle (  2.13);

\path[fill=fillColor,fill opacity=0.20] ( 80.20, 69.24) circle (  2.13);

\path[fill=fillColor,fill opacity=0.20] ( 89.81, 63.32) circle (  2.13);

\path[fill=fillColor,fill opacity=0.20] ( 95.27, 64.36) circle (  2.13);

\path[fill=fillColor,fill opacity=0.20] (106.63, 68.10) circle (  2.13);

\path[fill=fillColor,fill opacity=0.20] ( 97.24, 61.45) circle (  2.13);

\path[fill=fillColor,fill opacity=0.20] ( 91.34, 53.35) circle (  2.13);

\path[fill=fillColor,fill opacity=0.20] ( 96.36, 53.25) circle (  2.13);

\path[fill=fillColor,fill opacity=0.20] (101.17, 61.66) circle (  2.13);

\path[fill=fillColor,fill opacity=0.20] ( 93.96, 69.86) circle (  2.13);

\path[fill=fillColor,fill opacity=0.20] ( 89.15, 77.13) circle (  2.13);

\path[fill=fillColor,fill opacity=0.20] ( 87.62, 73.08) circle (  2.13);

\path[fill=fillColor,fill opacity=0.20] (124.11, 54.81) circle (  2.13);

\path[fill=fillColor,fill opacity=0.20] ( 83.91, 54.50) circle (  2.13);

\path[fill=fillColor,fill opacity=0.20] ( 78.67, 73.60) circle (  2.13);

\path[fill=fillColor,fill opacity=0.20] ( 63.15, 92.08) circle (  2.13);

\path[fill=fillColor,fill opacity=0.20] ( 55.72, 99.35) circle (  2.13);

\path[fill=fillColor,fill opacity=0.20] ( 46.98,111.81) circle (  2.13);

\path[fill=fillColor,fill opacity=0.20] ( 64.46, 87.93) circle (  2.13);

\path[fill=fillColor,fill opacity=0.20] ( 78.01, 67.99) circle (  2.13);

\path[fill=fillColor,fill opacity=0.20] ( 78.67, 80.24) circle (  2.13);

\path[fill=fillColor,fill opacity=0.20] ( 85.44, 72.87) circle (  2.13);

\path[fill=fillColor,fill opacity=0.20] ( 95.93, 63.74) circle (  2.13);

\path[fill=fillColor,fill opacity=0.20] (103.58, 73.29) circle (  2.13);

\path[fill=fillColor,fill opacity=0.20] (115.37, 70.90) circle (  2.13);

\path[fill=fillColor,fill opacity=0.20] ( 81.94, 50.97) circle (  2.13);

\path[fill=fillColor,fill opacity=0.20] ( 79.10, 63.84) circle (  2.13);

\path[fill=fillColor,fill opacity=0.20] ( 91.78, 46.92) circle (  2.13);

\path[fill=fillColor,fill opacity=0.20] ( 93.96, 45.26) circle (  2.13);

\path[fill=fillColor,fill opacity=0.20] ( 92.65, 62.49) circle (  2.13);

\path[fill=fillColor,fill opacity=0.20] ( 92.43, 64.88) circle (  2.13);

\path[fill=fillColor,fill opacity=0.20] ( 88.72, 50.03) circle (  2.13);

\path[fill=fillColor,fill opacity=0.20] ( 83.47, 53.87) circle (  2.13);

\path[fill=fillColor,fill opacity=0.20] ( 86.31, 66.44) circle (  2.13);

\path[fill=fillColor,fill opacity=0.20] ( 81.51, 69.24) circle (  2.13);

\path[fill=fillColor,fill opacity=0.20] ( 71.89, 72.25) circle (  2.13);

\path[fill=fillColor,fill opacity=0.20] ( 70.58, 71.94) circle (  2.13);

\path[fill=fillColor,fill opacity=0.20] ( 63.37, 83.77) circle (  2.13);

\path[fill=fillColor,fill opacity=0.20] ( 55.94,112.84) circle (  2.13);

\path[fill=fillColor,fill opacity=0.20] ( 65.77,110.77) circle (  2.13);

\path[fill=fillColor,fill opacity=0.20] ( 68.40,111.81) circle (  2.13);

\path[fill=fillColor,fill opacity=0.20] ( 74.51, 87.93) circle (  2.13);

\path[fill=fillColor,fill opacity=0.20] ( 84.78, 61.35) circle (  2.13);

\path[fill=fillColor,fill opacity=0.20] ( 97.89, 68.20) circle (  2.13);

\path[fill=fillColor,fill opacity=0.20] ( 99.86, 73.50) circle (  2.13);

\path[fill=fillColor,fill opacity=0.20] (108.60, 66.12) circle (  2.13);

\path[fill=fillColor,fill opacity=0.20] (118.21, 66.12) circle (  2.13);

\path[fill=fillColor,fill opacity=0.20] (102.92, 56.36) circle (  2.13);

\path[fill=fillColor,fill opacity=0.20] ( 97.46, 44.94) circle (  2.13);

\path[fill=fillColor,fill opacity=0.20] ( 98.33, 55.33) circle (  2.13);

\path[fill=fillColor,fill opacity=0.20] ( 90.68, 62.80) circle (  2.13);

\path[fill=fillColor,fill opacity=0.20] ( 87.62, 63.74) circle (  2.13);

\path[fill=fillColor,fill opacity=0.20] ( 87.62, 69.34) circle (  2.13);

\path[fill=fillColor,fill opacity=0.20] ( 78.23, 68.30) circle (  2.13);

\path[fill=fillColor,fill opacity=0.20] ( 75.83, 62.91) circle (  2.13);

\path[fill=fillColor,fill opacity=0.20] ( 78.45, 64.98) circle (  2.13);

\path[fill=fillColor,fill opacity=0.20] ( 79.10, 64.67) circle (  2.13);

\path[fill=fillColor,fill opacity=0.20] ( 70.58, 61.35) circle (  2.13);

\path[fill=fillColor,fill opacity=0.20] ( 73.86, 68.82) circle (  2.13);

\path[fill=fillColor,fill opacity=0.20] ( 70.80, 87.93) circle (  2.13);

\path[fill=fillColor,fill opacity=0.20] ( 63.37,108.69) circle (  2.13);

\path[fill=fillColor,fill opacity=0.20] ( 97.02,111.81) circle (  2.13);

\path[fill=fillColor,fill opacity=0.20] ( 52.01,115.96) circle (  2.13);

\path[fill=fillColor,fill opacity=0.20] ( 61.62,106.61) circle (  2.13);

\path[fill=fillColor,fill opacity=0.20] ( 62.72, 92.08) circle (  2.13);

\path[fill=fillColor,fill opacity=0.20] ( 65.77,100.39) circle (  2.13);

\path[fill=fillColor,fill opacity=0.20] ( 69.27,102.46) circle (  2.13);

\path[fill=fillColor,fill opacity=0.20] ( 75.17, 75.47) circle (  2.13);

\path[fill=fillColor,fill opacity=0.20] ( 76.70, 43.28) circle (  2.13);

\path[fill=fillColor,fill opacity=0.20] ( 74.95, 47.64) circle (  2.13);

\path[fill=fillColor,fill opacity=0.20] ( 82.38, 56.78) circle (  2.13);

\path[fill=fillColor,fill opacity=0.20] ( 84.78, 53.87) circle (  2.13);

\path[fill=fillColor,fill opacity=0.20] ( 83.47, 65.61) circle (  2.13);

\path[fill=fillColor,fill opacity=0.20] ( 90.68, 75.68) circle (  2.13);

\path[fill=fillColor,fill opacity=0.20] (104.23, 58.86) circle (  2.13);

\path[fill=fillColor,fill opacity=0.20] (108.60, 41.93) circle (  2.13);

\path[fill=fillColor,fill opacity=0.20] ( 89.59, 40.69) circle (  2.13);

\path[fill=fillColor,fill opacity=0.20] ( 90.68, 58.75) circle (  2.13);

\path[fill=fillColor,fill opacity=0.20] ( 94.84, 63.63) circle (  2.13);

\path[fill=fillColor,fill opacity=0.20] ( 98.55, 56.99) circle (  2.13);

\path[fill=fillColor,fill opacity=0.20] ( 93.52, 63.42) circle (  2.13);

\path[fill=fillColor,fill opacity=0.20] ( 78.67, 67.99) circle (  2.13);

\path[fill=fillColor,fill opacity=0.20] ( 81.07, 60.83) circle (  2.13);

\path[fill=fillColor,fill opacity=0.20] ( 77.14, 65.09) circle (  2.13);

\path[fill=fillColor,fill opacity=0.20] ( 76.26, 63.32) circle (  2.13);

\path[fill=fillColor,fill opacity=0.20] ( 71.02, 51.28) circle (  2.13);

\path[fill=fillColor,fill opacity=0.20] ( 81.51, 54.70) circle (  2.13);

\path[fill=fillColor,fill opacity=0.20] ( 90.25, 60.41) circle (  2.13);

\path[fill=fillColor,fill opacity=0.20] ( 85.88, 58.96) circle (  2.13);

\path[fill=fillColor,fill opacity=0.20] ( 83.69, 61.04) circle (  2.13);

\path[fill=fillColor,fill opacity=0.20] ( 85.88, 67.16) circle (  2.13);

\path[fill=fillColor,fill opacity=0.20] ( 77.57, 70.38) circle (  2.13);

\path[fill=fillColor,fill opacity=0.20] ( 76.70, 68.51) circle (  2.13);

\path[fill=fillColor,fill opacity=0.20] ( 70.14, 61.35) circle (  2.13);

\path[fill=fillColor,fill opacity=0.20] ( 70.80, 59.06) circle (  2.13);

\path[fill=fillColor,fill opacity=0.20] ( 77.57, 66.64) circle (  2.13);

\path[fill=fillColor,fill opacity=0.20] ( 74.95, 69.55) circle (  2.13);

\path[fill=fillColor,fill opacity=0.20] ( 75.83, 65.92) circle (  2.13);

\path[fill=fillColor,fill opacity=0.20] ( 76.26, 66.23) circle (  2.13);

\path[fill=fillColor,fill opacity=0.20] ( 81.29, 66.23) circle (  2.13);

\path[fill=fillColor,fill opacity=0.20] ( 75.83, 59.79) circle (  2.13);

\path[fill=fillColor,fill opacity=0.20] ( 72.77, 52.84) circle (  2.13);

\path[fill=fillColor,fill opacity=0.20] ( 79.10, 57.20) circle (  2.13);

\path[fill=fillColor,fill opacity=0.20] ( 80.85, 62.08) circle (  2.13);

\path[fill=fillColor,fill opacity=0.20] ( 75.83, 57.30) circle (  2.13);

\path[fill=fillColor,fill opacity=0.20] ( 75.61, 55.43) circle (  2.13);

\path[fill=fillColor,fill opacity=0.20] ( 77.36, 55.02) circle (  2.13);

\path[fill=fillColor,fill opacity=0.20] ( 88.06, 53.87) circle (  2.13);

\path[fill=fillColor,fill opacity=0.20] ( 91.99, 65.50) circle (  2.13);

\path[fill=fillColor,fill opacity=0.20] ( 92.21, 72.87) circle (  2.13);

\path[fill=fillColor,fill opacity=0.20] ( 92.21, 51.28) circle (  2.13);

\path[fill=fillColor,fill opacity=0.20] ( 85.66, 56.57) circle (  2.13);

\path[fill=fillColor,fill opacity=0.20] ( 93.09, 48.27) circle (  2.13);

\path[fill=fillColor,fill opacity=0.20] (102.48, 54.39) circle (  2.13);

\path[fill=fillColor,fill opacity=0.20] ( 98.77, 59.06) circle (  2.13);

\path[fill=fillColor,fill opacity=0.20] ( 86.75, 64.67) circle (  2.13);

\path[fill=fillColor,fill opacity=0.20] ( 87.62, 68.20) circle (  2.13);

\path[fill=fillColor,fill opacity=0.20] ( 88.28, 58.75) circle (  2.13);

\path[fill=fillColor,fill opacity=0.20] ( 82.16, 54.18) circle (  2.13);

\path[fill=fillColor,fill opacity=0.20] ( 77.14, 65.61) circle (  2.13);

\path[fill=fillColor,fill opacity=0.20] ( 77.57, 67.58) circle (  2.13);

\path[fill=fillColor,fill opacity=0.20] ( 74.30, 64.15) circle (  2.13);

\path[fill=fillColor,fill opacity=0.20] ( 84.13, 61.97) circle (  2.13);

\path[fill=fillColor,fill opacity=0.20] ( 88.94, 57.82) circle (  2.13);

\path[fill=fillColor,fill opacity=0.20] ( 78.88, 68.30) circle (  2.13);

\path[fill=fillColor,fill opacity=0.20] ( 75.17, 79.83) circle (  2.13);

\path[fill=fillColor,fill opacity=0.20] ( 77.36, 68.72) circle (  2.13);

\path[fill=fillColor,fill opacity=0.20] ( 79.54, 57.20) circle (  2.13);

\path[fill=fillColor,fill opacity=0.20] ( 79.10, 61.04) circle (  2.13);

\path[fill=fillColor,fill opacity=0.20] ( 79.54, 57.51) circle (  2.13);

\path[fill=fillColor,fill opacity=0.20] ( 83.25, 49.10) circle (  2.13);

\path[fill=fillColor,fill opacity=0.20] ( 83.69, 54.91) circle (  2.13);

\path[fill=fillColor,fill opacity=0.20] ( 82.60, 60.83) circle (  2.13);

\path[fill=fillColor,fill opacity=0.20] ( 87.19, 60.00) circle (  2.13);

\path[fill=fillColor,fill opacity=0.20] ( 76.04, 65.81) circle (  2.13);

\path[fill=fillColor,fill opacity=0.20] ( 74.51, 75.68) circle (  2.13);

\path[fill=fillColor,fill opacity=0.20] ( 76.92, 77.23) circle (  2.13);

\path[fill=fillColor,fill opacity=0.20] ( 80.41, 75.68) circle (  2.13);

\path[fill=fillColor,fill opacity=0.20] ( 78.67, 74.74) circle (  2.13);

\path[fill=fillColor,fill opacity=0.20] ( 87.19, 63.53) circle (  2.13);

\path[fill=fillColor,fill opacity=0.20] ( 94.18, 54.08) circle (  2.13);

\path[fill=fillColor,fill opacity=0.20] ( 91.99, 61.45) circle (  2.13);

\path[fill=fillColor,fill opacity=0.20] ( 81.51, 56.16) circle (  2.13);

\path[fill=fillColor,fill opacity=0.20] ( 92.65, 59.38) circle (  2.13);

\path[fill=fillColor,fill opacity=0.20] ( 91.56, 58.96) circle (  2.13);

\path[fill=fillColor,fill opacity=0.20] ( 91.99, 52.21) circle (  2.13);

\path[fill=fillColor,fill opacity=0.20] ( 91.56, 50.76) circle (  2.13);

\path[fill=fillColor,fill opacity=0.20] ( 90.90, 57.09) circle (  2.13);

\path[fill=fillColor,fill opacity=0.20] ( 83.25, 64.57) circle (  2.13);

\path[fill=fillColor,fill opacity=0.20] ( 82.60, 66.12) circle (  2.13);

\path[fill=fillColor,fill opacity=0.20] ( 88.94, 61.66) circle (  2.13);

\path[fill=fillColor,fill opacity=0.20] ( 88.28, 54.29) circle (  2.13);

\path[fill=fillColor,fill opacity=0.20] ( 84.57, 60.31) circle (  2.13);

\path[fill=fillColor,fill opacity=0.20] ( 84.35, 72.77) circle (  2.13);

\path[fill=fillColor,fill opacity=0.20] ( 90.03, 70.59) circle (  2.13);

\path[fill=fillColor,fill opacity=0.20] ( 90.03, 65.40) circle (  2.13);

\path[fill=fillColor,fill opacity=0.20] ( 88.72, 63.22) circle (  2.13);

\path[fill=fillColor,fill opacity=0.20] ( 81.73, 56.36) circle (  2.13);

\path[fill=fillColor,fill opacity=0.20] ( 90.47, 50.86) circle (  2.13);

\path[fill=fillColor,fill opacity=0.20] (103.58, 60.00) circle (  2.13);

\path[fill=fillColor,fill opacity=0.20] ( 81.73, 69.24) circle (  2.13);

\path[fill=fillColor,fill opacity=0.20] ( 80.20, 64.67) circle (  2.13);

\path[fill=fillColor,fill opacity=0.20] ( 88.72, 60.62) circle (  2.13);

\path[fill=fillColor,fill opacity=0.20] ( 87.84, 64.67) circle (  2.13);

\path[fill=fillColor,fill opacity=0.20] ( 85.00, 65.71) circle (  2.13);

\path[fill=fillColor,fill opacity=0.20] ( 91.34, 63.01) circle (  2.13);

\path[fill=fillColor,fill opacity=0.20] ( 91.34, 56.68) circle (  2.13);

\path[fill=fillColor,fill opacity=0.20] ( 83.69, 49.10) circle (  2.13);

\path[fill=fillColor,fill opacity=0.20] ( 85.44, 50.45) circle (  2.13);

\path[fill=fillColor,fill opacity=0.20] ( 74.30, 51.07) circle (  2.13);

\path[fill=fillColor,fill opacity=0.20] ( 81.73, 53.67) circle (  2.13);

\path[fill=fillColor,fill opacity=0.20] ( 83.04, 54.70) circle (  2.13);

\path[fill=fillColor,fill opacity=0.20] ( 92.65, 48.68) circle (  2.13);

\path[fill=fillColor,fill opacity=0.20] ( 91.78, 47.02) circle (  2.13);

\path[fill=fillColor,fill opacity=0.20] ( 94.40, 42.56) circle (  2.13);

\path[fill=fillColor,fill opacity=0.20] ( 94.40, 42.87) circle (  2.13);

\path[fill=fillColor,fill opacity=0.20] ( 86.75, 48.99) circle (  2.13);

\path[fill=fillColor,fill opacity=0.20] ( 84.78, 47.75) circle (  2.13);

\path[fill=fillColor,fill opacity=0.20] ( 78.23, 46.19) circle (  2.13);

\path[fill=fillColor,fill opacity=0.20] ( 74.95, 49.62) circle (  2.13);

\path[fill=fillColor,fill opacity=0.20] ( 83.04, 49.51) circle (  2.13);

\path[fill=fillColor,fill opacity=0.20] ( 78.67, 49.82) circle (  2.13);

\path[fill=fillColor,fill opacity=0.20] ( 79.32, 52.63) circle (  2.13);

\path[fill=fillColor,fill opacity=0.20] ( 86.10, 48.06) circle (  2.13);

\path[fill=fillColor,fill opacity=0.20] ( 79.98, 39.86) circle (  2.13);

\path[fill=fillColor,fill opacity=0.20] ( 78.23, 39.13) circle (  2.13);

\path[fill=fillColor,fill opacity=0.20] ( 52.23,108.69) circle (  2.13);

\path[fill=fillColor,fill opacity=0.20] ( 54.19,104.54) circle (  2.13);

\path[fill=fillColor,fill opacity=0.20] ( 86.97,106.61) circle (  2.13);

\path[fill=fillColor,fill opacity=0.20] ( 59.66, 99.35) circle (  2.13);

\path[fill=fillColor,fill opacity=0.20] ( 58.35, 90.00) circle (  2.13);

\path[fill=fillColor,fill opacity=0.20] ( 72.55, 99.35) circle (  2.13);

\path[fill=fillColor,fill opacity=0.20] ( 55.29,103.50) circle (  2.13);

\path[fill=fillColor,fill opacity=0.20] ( 55.51, 99.35) circle (  2.13);

\path[fill=fillColor,fill opacity=0.20] ( 51.79,105.58) circle (  2.13);

\path[fill=fillColor,fill opacity=0.20] ( 45.67,115.96) circle (  2.13);

\path[fill=fillColor,fill opacity=0.20] ( 54.63,110.77) circle (  2.13);

\path[fill=fillColor,fill opacity=0.20] ( 77.57,109.73) circle (  2.13);

\path[fill=fillColor,fill opacity=0.20] ( 64.46,112.84) circle (  2.13);

\path[fill=fillColor,fill opacity=0.20] ( 71.02,102.46) circle (  2.13);

\path[fill=fillColor,fill opacity=0.20] ( 73.42, 85.85) circle (  2.13);

\path[fill=fillColor,fill opacity=0.20] ( 68.62, 77.13) circle (  2.13);

\path[fill=fillColor,fill opacity=0.20] ( 73.20, 73.08) circle (  2.13);

\path[fill=fillColor,fill opacity=0.20] ( 74.08, 69.55) circle (  2.13);

\path[fill=fillColor,fill opacity=0.20] (140.72, 77.13) circle (  2.13);

\path[fill=fillColor,fill opacity=0.20] ( 89.37,106.61) circle (  2.13);

\path[fill=fillColor,fill opacity=0.20] ( 75.83,104.54) circle (  2.13);

\path[fill=fillColor,fill opacity=0.20] ( 84.78, 94.16) circle (  2.13);

\path[fill=fillColor,fill opacity=0.20] ( 95.71, 87.93) circle (  2.13);

\path[fill=fillColor,fill opacity=0.20] (103.36, 77.65) circle (  2.13);

\path[fill=fillColor,fill opacity=0.20] ( 88.06, 67.79) circle (  2.13);

\path[fill=fillColor,fill opacity=0.20] ( 84.57, 54.08) circle (  2.13);

\path[fill=fillColor,fill opacity=0.20] ( 83.04, 47.96) circle (  2.13);

\path[fill=fillColor,fill opacity=0.20] ( 81.73, 61.35) circle (  2.13);

\path[fill=fillColor,fill opacity=0.20] ( 78.01, 63.63) circle (  2.13);

\path[fill=fillColor,fill opacity=0.20] ( 57.25,108.69) circle (  2.13);

\path[fill=fillColor,fill opacity=0.20] ( 79.76, 91.04) circle (  2.13);

\path[fill=fillColor,fill opacity=0.20] ( 93.31, 90.00) circle (  2.13);

\path[fill=fillColor,fill opacity=0.20] (126.52, 77.34) circle (  2.13);

\path[fill=fillColor,fill opacity=0.20] (134.82, 70.48) circle (  2.13);

\path[fill=fillColor,fill opacity=0.20] (103.14, 64.88) circle (  2.13);

\path[fill=fillColor,fill opacity=0.20] ( 93.09, 70.48) circle (  2.13);

\path[fill=fillColor,fill opacity=0.20] ( 85.00, 77.13) circle (  2.13);

\path[fill=fillColor,fill opacity=0.20] ( 86.31, 59.17) circle (  2.13);

\path[fill=fillColor,fill opacity=0.20] (102.05, 44.84) circle (  2.13);

\path[fill=fillColor,fill opacity=0.20] ( 83.91, 59.58) circle (  2.13);

\path[fill=fillColor,fill opacity=0.20] ( 74.73, 43.91) circle (  2.13);

\path[fill=fillColor,fill opacity=0.20] ( 67.09, 90.00) circle (  2.13);

\path[fill=fillColor,fill opacity=0.20] (124.99, 71.11) circle (  2.13);

\path[fill=fillColor,fill opacity=0.20] (112.53, 73.70) circle (  2.13);

\path[fill=fillColor,fill opacity=0.20] ( 93.31, 76.61) circle (  2.13);

\path[fill=fillColor,fill opacity=0.20] ( 91.78, 78.48) circle (  2.13);

\path[fill=fillColor,fill opacity=0.20] ( 93.09, 69.55) circle (  2.13);

\path[fill=fillColor,fill opacity=0.20] ( 94.40, 66.85) circle (  2.13);

\path[fill=fillColor,fill opacity=0.20] ( 83.91, 80.45) circle (  2.13);

\path[fill=fillColor,fill opacity=0.20] ( 81.51, 73.29) circle (  2.13);

\path[fill=fillColor,fill opacity=0.20] ( 85.44, 52.94) circle (  2.13);

\path[fill=fillColor,fill opacity=0.20] ( 79.98, 57.40) circle (  2.13);

\path[fill=fillColor,fill opacity=0.20] ( 91.34, 50.97) circle (  2.13);

\path[fill=fillColor,fill opacity=0.20] (102.92, 54.50) circle (  2.13);

\path[fill=fillColor,fill opacity=0.20] (105.76, 40.27) circle (  2.13);

\path[fill=fillColor,fill opacity=0.20] ( 86.53, 81.18) circle (  2.13);

\path[fill=fillColor,fill opacity=0.20] (103.36, 57.71) circle (  2.13);

\path[fill=fillColor,fill opacity=0.20] (119.31, 66.64) circle (  2.13);

\path[fill=fillColor,fill opacity=0.20] ( 99.64, 75.16) circle (  2.13);

\path[fill=fillColor,fill opacity=0.20] ( 93.31, 80.04) circle (  2.13);

\path[fill=fillColor,fill opacity=0.20] ( 84.78, 70.69) circle (  2.13);

\path[fill=fillColor,fill opacity=0.20] ( 95.05, 63.01) circle (  2.13);

\path[fill=fillColor,fill opacity=0.20] ( 78.45, 80.76) circle (  2.13);

\path[fill=fillColor,fill opacity=0.20] ( 79.98, 67.89) circle (  2.13);

\path[fill=fillColor,fill opacity=0.20] ( 91.34, 42.76) circle (  2.13);

\path[fill=fillColor,fill opacity=0.20] ( 98.33, 51.69) circle (  2.13);

\path[fill=fillColor,fill opacity=0.20] (123.24, 58.86) circle (  2.13);

\path[fill=fillColor,fill opacity=0.20] (123.24, 66.54) circle (  2.13);

\path[fill=fillColor,fill opacity=0.20] (128.27, 64.57) circle (  2.13);

\path[fill=fillColor,fill opacity=0.20] (138.32, 49.62) circle (  2.13);

\path[fill=fillColor,fill opacity=0.20] ( 88.50, 81.70) circle (  2.13);

\path[fill=fillColor,fill opacity=0.20] (107.07, 50.03) circle (  2.13);

\path[fill=fillColor,fill opacity=0.20] ( 95.71, 59.38) circle (  2.13);

\path[fill=fillColor,fill opacity=0.20] (100.52, 74.53) circle (  2.13);

\path[fill=fillColor,fill opacity=0.20] ( 95.27, 66.02) circle (  2.13);

\path[fill=fillColor,fill opacity=0.20] ( 96.80, 59.58) circle (  2.13);

\path[fill=fillColor,fill opacity=0.20] ( 92.43, 60.52) circle (  2.13);

\path[fill=fillColor,fill opacity=0.20] ( 78.23, 77.54) circle (  2.13);

\path[fill=fillColor,fill opacity=0.20] ( 63.81, 77.96) circle (  2.13);

\path[fill=fillColor,fill opacity=0.20] ( 86.97, 45.15) circle (  2.13);

\path[fill=fillColor,fill opacity=0.20] ( 94.62, 57.71) circle (  2.13);

\path[fill=fillColor,fill opacity=0.20] (102.92, 61.87) circle (  2.13);

\path[fill=fillColor,fill opacity=0.20] (120.18, 58.23) circle (  2.13);

\path[fill=fillColor,fill opacity=0.20] (131.54, 68.82) circle (  2.13);

\path[fill=fillColor,fill opacity=0.20] (121.27, 66.02) circle (  2.13);

\path[fill=fillColor,fill opacity=0.20] ( 98.55, 83.77) circle (  2.13);

\path[fill=fillColor,fill opacity=0.20] (131.76, 42.76) circle (  2.13);

\path[fill=fillColor,fill opacity=0.20] ( 85.88, 55.95) circle (  2.13);

\path[fill=fillColor,fill opacity=0.20] ( 87.84, 70.17) circle (  2.13);

\path[fill=fillColor,fill opacity=0.20] ( 92.43, 53.04) circle (  2.13);

\path[fill=fillColor,fill opacity=0.20] ( 91.34, 50.03) circle (  2.13);

\path[fill=fillColor,fill opacity=0.20] ( 84.13, 65.92) circle (  2.13);

\path[fill=fillColor,fill opacity=0.20] (112.10, 68.62) circle (  2.13);

\path[fill=fillColor,fill opacity=0.20] ( 77.14, 63.63) circle (  2.13);

\path[fill=fillColor,fill opacity=0.20] ( 97.46, 50.97) circle (  2.13);

\path[fill=fillColor,fill opacity=0.20] ( 99.86, 54.81) circle (  2.13);

\path[fill=fillColor,fill opacity=0.20] ( 96.36, 52.63) circle (  2.13);

\path[fill=fillColor,fill opacity=0.20] (105.98, 52.73) circle (  2.13);

\path[fill=fillColor,fill opacity=0.20] (123.02, 66.54) circle (  2.13);

\path[fill=fillColor,fill opacity=0.20] (125.43, 72.66) circle (  2.13);

\path[fill=fillColor,fill opacity=0.20] (118.43, 51.69) circle (  2.13);

\path[fill=fillColor,fill opacity=0.20] (139.63, 86.89) circle (  2.13);

\path[fill=fillColor,fill opacity=0.20] (112.75, 42.04) circle (  2.13);

\path[fill=fillColor,fill opacity=0.20] ( 91.99, 54.60) circle (  2.13);

\path[fill=fillColor,fill opacity=0.20] ( 84.35, 64.88) circle (  2.13);

\path[fill=fillColor,fill opacity=0.20] ( 85.44, 48.37) circle (  2.13);

\path[fill=fillColor,fill opacity=0.20] ( 82.16, 46.50) circle (  2.13);

\path[fill=fillColor,fill opacity=0.20] ( 80.85, 62.18) circle (  2.13);

\path[fill=fillColor,fill opacity=0.20] ( 78.01, 66.23) circle (  2.13);

\path[fill=fillColor,fill opacity=0.20] ( 74.73, 52.52) circle (  2.13);

\path[fill=fillColor,fill opacity=0.20] ( 73.64, 47.64) circle (  2.13);

\path[fill=fillColor,fill opacity=0.20] ( 69.93, 76.19) circle (  2.13);

\path[fill=fillColor,fill opacity=0.20] (102.70, 51.59) circle (  2.13);

\path[fill=fillColor,fill opacity=0.20] (100.73, 45.98) circle (  2.13);

\path[fill=fillColor,fill opacity=0.20] (103.58, 43.28) circle (  2.13);

\path[fill=fillColor,fill opacity=0.20] (109.47, 61.04) circle (  2.13);

\path[fill=fillColor,fill opacity=0.20] (118.21, 73.50) circle (  2.13);

\path[fill=fillColor,fill opacity=0.20] (139.63, 59.58) circle (  2.13);

\path[fill=fillColor,fill opacity=0.20] (147.49, 42.66) circle (  2.13);

\path[fill=fillColor,fill opacity=0.20] (145.53, 55.43) circle (  2.13);

\path[fill=fillColor,fill opacity=0.20] (102.26, 60.00) circle (  2.13);

\path[fill=fillColor,fill opacity=0.20] ( 97.02, 56.78) circle (  2.13);

\path[fill=fillColor,fill opacity=0.20] ( 91.12, 64.05) circle (  2.13);

\path[fill=fillColor,fill opacity=0.20] ( 86.53, 59.48) circle (  2.13);

\path[fill=fillColor,fill opacity=0.20] ( 92.87, 56.68) circle (  2.13);

\path[fill=fillColor,fill opacity=0.20] ( 84.57, 57.82) circle (  2.13);

\path[fill=fillColor,fill opacity=0.20] ( 85.00, 56.26) circle (  2.13);

\path[fill=fillColor,fill opacity=0.20] ( 79.98, 48.58) circle (  2.13);

\path[fill=fillColor,fill opacity=0.20] ( 88.06, 52.11) circle (  2.13);

\path[fill=fillColor,fill opacity=0.20] ( 69.71, 64.98) circle (  2.13);

\path[fill=fillColor,fill opacity=0.20] ( 55.07, 94.16) circle (  2.13);

\path[fill=fillColor,fill opacity=0.20] ( 75.61, 60.52) circle (  2.13);

\path[fill=fillColor,fill opacity=0.20] (101.83, 54.70) circle (  2.13);

\path[fill=fillColor,fill opacity=0.20] (110.57, 51.38) circle (  2.13);

\path[fill=fillColor,fill opacity=0.20] (111.22, 49.20) circle (  2.13);

\path[fill=fillColor,fill opacity=0.20] (105.32, 58.96) circle (  2.13);

\path[fill=fillColor,fill opacity=0.20] (115.16, 57.30) circle (  2.13);

\path[fill=fillColor,fill opacity=0.20] (112.97, 47.33) circle (  2.13);

\path[fill=fillColor,fill opacity=0.20] (117.78, 50.34) circle (  2.13);

\path[fill=fillColor,fill opacity=0.20] (125.64, 55.74) circle (  2.13);

\path[fill=fillColor,fill opacity=0.20] (141.38, 58.75) circle (  2.13);

\path[fill=fillColor,fill opacity=0.20] (129.58, 63.84) circle (  2.13);

\path[fill=fillColor,fill opacity=0.20] (123.46, 46.09) circle (  2.13);

\path[fill=fillColor,fill opacity=0.20] ( 98.33, 82.74) circle (  2.13);

\path[fill=fillColor,fill opacity=0.20] (102.48, 58.34) circle (  2.13);

\path[fill=fillColor,fill opacity=0.20] ( 96.58, 58.44) circle (  2.13);

\path[fill=fillColor,fill opacity=0.20] ( 92.21, 69.03) circle (  2.13);

\path[fill=fillColor,fill opacity=0.20] ( 79.76, 71.63) circle (  2.13);

\path[fill=fillColor,fill opacity=0.20] ( 81.07, 64.57) circle (  2.13);

\path[fill=fillColor,fill opacity=0.20] ( 86.97, 50.45) circle (  2.13);

\path[fill=fillColor,fill opacity=0.20] ( 83.25, 53.87) circle (  2.13);

\path[fill=fillColor,fill opacity=0.20] ( 77.36, 72.15) circle (  2.13);

\path[fill=fillColor,fill opacity=0.20] ( 71.24, 75.26) circle (  2.13);

\path[fill=fillColor,fill opacity=0.20] ( 61.40, 75.47) circle (  2.13);

\path[fill=fillColor,fill opacity=0.20] ( 53.76, 97.27) circle (  2.13);

\path[fill=fillColor,fill opacity=0.20] ( 97.68, 50.55) circle (  2.13);

\path[fill=fillColor,fill opacity=0.20] (109.04, 45.98) circle (  2.13);

\path[fill=fillColor,fill opacity=0.20] (107.95, 47.23) circle (  2.13);

\path[fill=fillColor,fill opacity=0.20] (102.26, 57.92) circle (  2.13);

\path[fill=fillColor,fill opacity=0.20] (106.20, 55.22) circle (  2.13);

\path[fill=fillColor,fill opacity=0.20] (114.28, 45.57) circle (  2.13);

\path[fill=fillColor,fill opacity=0.20] (116.69, 53.46) circle (  2.13);

\path[fill=fillColor,fill opacity=0.20] (110.57, 66.85) circle (  2.13);

\path[fill=fillColor,fill opacity=0.20] (117.78, 60.00) circle (  2.13);

\path[fill=fillColor,fill opacity=0.20] (136.57, 51.49) circle (  2.13);

\path[fill=fillColor,fill opacity=0.20] (125.43, 67.79) circle (  2.13);

\path[fill=fillColor,fill opacity=0.20] (109.04, 68.72) circle (  2.13);

\path[fill=fillColor,fill opacity=0.20] (101.83, 66.44) circle (  2.13);

\path[fill=fillColor,fill opacity=0.20] ( 96.58, 47.75) circle (  2.13);

\path[fill=fillColor,fill opacity=0.20] ( 88.06, 53.15) circle (  2.13);

\path[fill=fillColor,fill opacity=0.20] ( 77.79, 65.40) circle (  2.13);

\path[fill=fillColor,fill opacity=0.20] ( 77.57, 58.65) circle (  2.13);

\path[fill=fillColor,fill opacity=0.20] ( 81.73, 50.45) circle (  2.13);

\path[fill=fillColor,fill opacity=0.20] ( 76.26, 64.67) circle (  2.13);

\path[fill=fillColor,fill opacity=0.20] ( 74.95, 79.21) circle (  2.13);

\path[fill=fillColor,fill opacity=0.20] ( 82.60, 77.65) circle (  2.13);

\path[fill=fillColor,fill opacity=0.20] ( 74.30, 75.78) circle (  2.13);

\path[fill=fillColor,fill opacity=0.20] ( 64.46, 78.89) circle (  2.13);

\path[fill=fillColor,fill opacity=0.20] ( 56.82, 88.96) circle (  2.13);

\path[fill=fillColor,fill opacity=0.20] ( 64.90,113.88) circle (  2.13);

\path[fill=fillColor,fill opacity=0.20] ( 91.12, 71.83) circle (  2.13);

\path[fill=fillColor,fill opacity=0.20] (114.28, 58.75) circle (  2.13);

\path[fill=fillColor,fill opacity=0.20] (104.23, 52.84) circle (  2.13);

\path[fill=fillColor,fill opacity=0.20] (103.58, 55.43) circle (  2.13);

\path[fill=fillColor,fill opacity=0.20] (103.58, 52.63) circle (  2.13);

\path[fill=fillColor,fill opacity=0.20] (127.39, 58.44) circle (  2.13);

\path[fill=fillColor,fill opacity=0.20] (129.80, 75.99) circle (  2.13);

\path[fill=fillColor,fill opacity=0.20] (109.47, 72.87) circle (  2.13);

\path[fill=fillColor,fill opacity=0.20] (116.90, 51.38) circle (  2.13);

\path[fill=fillColor,fill opacity=0.20] (135.26, 51.90) circle (  2.13);

\path[fill=fillColor,fill opacity=0.20] (120.40, 74.12) circle (  2.13);

\path[fill=fillColor,fill opacity=0.20] (116.69, 77.03) circle (  2.13);

\path[fill=fillColor,fill opacity=0.20] (100.08, 60.21) circle (  2.13);

\path[fill=fillColor,fill opacity=0.20] (116.90, 61.14) circle (  2.13);

\path[fill=fillColor,fill opacity=0.20] ( 87.19, 41.10) circle (  2.13);

\path[fill=fillColor,fill opacity=0.20] ( 85.00, 46.92) circle (  2.13);

\path[fill=fillColor,fill opacity=0.20] ( 83.69, 44.43) circle (  2.13);

\path[fill=fillColor,fill opacity=0.20] ( 84.57, 50.45) circle (  2.13);

\path[fill=fillColor,fill opacity=0.20] ( 72.11, 64.46) circle (  2.13);

\path[fill=fillColor,fill opacity=0.20] ( 75.83, 67.27) circle (  2.13);

\path[fill=fillColor,fill opacity=0.20] ( 96.80, 56.57) circle (  2.13);

\path[fill=fillColor,fill opacity=0.20] ( 78.45, 63.94) circle (  2.13);

\path[fill=fillColor,fill opacity=0.20] ( 72.99, 71.11) circle (  2.13);

\path[fill=fillColor,fill opacity=0.20] ( 63.81, 68.62) circle (  2.13);

\path[fill=fillColor,fill opacity=0.20] ( 61.84, 84.81) circle (  2.13);

\path[fill=fillColor,fill opacity=0.20] ( 70.36, 96.23) circle (  2.13);

\path[fill=fillColor,fill opacity=0.20] ( 83.04, 72.35) circle (  2.13);

\path[fill=fillColor,fill opacity=0.20] ( 93.31, 79.21) circle (  2.13);

\path[fill=fillColor,fill opacity=0.20] (104.23, 70.90) circle (  2.13);

\path[fill=fillColor,fill opacity=0.20] (113.41, 52.94) circle (  2.13);

\path[fill=fillColor,fill opacity=0.20] (116.25, 53.77) circle (  2.13);

\path[fill=fillColor,fill opacity=0.20] (115.37, 71.11) circle (  2.13);

\path[fill=fillColor,fill opacity=0.20] (109.47, 79.72) circle (  2.13);

\path[fill=fillColor,fill opacity=0.20] (116.03, 67.16) circle (  2.13);

\path[fill=fillColor,fill opacity=0.20] (118.43, 57.51) circle (  2.13);

\path[fill=fillColor,fill opacity=0.20] (119.74, 63.53) circle (  2.13);

\path[fill=fillColor,fill opacity=0.20] (122.80, 68.10) circle (  2.13);

\path[fill=fillColor,fill opacity=0.20] (116.25, 72.56) circle (  2.13);

\path[fill=fillColor,fill opacity=0.20] (114.50, 81.70) circle (  2.13);

\path[fill=fillColor,fill opacity=0.20] (107.73, 62.28) circle (  2.13);

\path[fill=fillColor,fill opacity=0.20] ( 90.03, 52.21) circle (  2.13);

\path[fill=fillColor,fill opacity=0.20] ( 81.51, 51.38) circle (  2.13);

\path[fill=fillColor,fill opacity=0.20] ( 84.13, 53.77) circle (  2.13);

\path[fill=fillColor,fill opacity=0.20] ( 82.60, 52.11) circle (  2.13);

\path[fill=fillColor,fill opacity=0.20] ( 86.75, 50.34) circle (  2.13);

\path[fill=fillColor,fill opacity=0.20] ( 88.72, 47.96) circle (  2.13);

\path[fill=fillColor,fill opacity=0.20] ( 90.25, 53.87) circle (  2.13);

\path[fill=fillColor,fill opacity=0.20] ( 87.62, 59.27) circle (  2.13);

\path[fill=fillColor,fill opacity=0.20] ( 77.57, 57.51) circle (  2.13);

\path[fill=fillColor,fill opacity=0.20] ( 73.64, 62.08) circle (  2.13);

\path[fill=fillColor,fill opacity=0.20] ( 67.52, 69.76) circle (  2.13);

\path[fill=fillColor,fill opacity=0.20] ( 60.75, 71.63) circle (  2.13);

\path[fill=fillColor,fill opacity=0.20] ( 54.41, 87.93) circle (  2.13);

\path[fill=fillColor,fill opacity=0.20] ( 56.82,108.69) circle (  2.13);

\path[fill=fillColor,fill opacity=0.20] ( 64.90, 97.27) circle (  2.13);

\path[fill=fillColor,fill opacity=0.20] ( 56.16,103.50) circle (  2.13);

\path[fill=fillColor,fill opacity=0.20] ( 72.55, 83.77) circle (  2.13);

\path[fill=fillColor,fill opacity=0.20] (103.58, 62.08) circle (  2.13);

\path[fill=fillColor,fill opacity=0.20] ( 93.09, 59.48) circle (  2.13);

\path[fill=fillColor,fill opacity=0.20] ( 91.78, 73.29) circle (  2.13);

\path[fill=fillColor,fill opacity=0.20] (104.89, 69.13) circle (  2.13);

\path[fill=fillColor,fill opacity=0.20] (108.82, 49.72) circle (  2.13);

\path[fill=fillColor,fill opacity=0.20] (114.72, 49.31) circle (  2.13);

\path[fill=fillColor,fill opacity=0.20] (109.47, 66.12) circle (  2.13);

\path[fill=fillColor,fill opacity=0.20] (116.03, 65.19) circle (  2.13);

\path[fill=fillColor,fill opacity=0.20] (118.65, 53.35) circle (  2.13);

\path[fill=fillColor,fill opacity=0.20] (112.75, 66.64) circle (  2.13);

\path[fill=fillColor,fill opacity=0.20] (115.16, 74.43) circle (  2.13);

\path[fill=fillColor,fill opacity=0.20] (113.63, 51.07) circle (  2.13);

\path[fill=fillColor,fill opacity=0.20] (106.42, 57.51) circle (  2.13);

\path[fill=fillColor,fill opacity=0.20] (115.16, 94.16) circle (  2.13);

\path[fill=fillColor,fill opacity=0.20] ( 84.35, 70.38) circle (  2.13);

\path[fill=fillColor,fill opacity=0.20] ( 91.34, 66.75) circle (  2.13);

\path[fill=fillColor,fill opacity=0.20] ( 97.24, 58.23) circle (  2.13);

\path[fill=fillColor,fill opacity=0.20] ( 85.88, 60.52) circle (  2.13);

\path[fill=fillColor,fill opacity=0.20] ( 85.44, 63.53) circle (  2.13);

\path[fill=fillColor,fill opacity=0.20] ( 78.23, 56.16) circle (  2.13);

\path[fill=fillColor,fill opacity=0.20] ( 83.25, 47.85) circle (  2.13);

\path[fill=fillColor,fill opacity=0.20] ( 78.88, 47.75) circle (  2.13);

\path[fill=fillColor,fill opacity=0.20] ( 72.33, 53.25) circle (  2.13);

\path[fill=fillColor,fill opacity=0.20] ( 69.27, 49.93) circle (  2.13);

\path[fill=fillColor,fill opacity=0.20] ( 71.24, 56.47) circle (  2.13);

\path[fill=fillColor,fill opacity=0.20] ( 68.40, 68.93) circle (  2.13);

\path[fill=fillColor,fill opacity=0.20] ( 75.17, 72.46) circle (  2.13);

\path[fill=fillColor,fill opacity=0.20] ( 60.31, 78.48) circle (  2.13);

\path[fill=fillColor,fill opacity=0.20] ( 55.51, 88.96) circle (  2.13);

\path[fill=fillColor,fill opacity=0.20] ( 83.91, 84.81) circle (  2.13);

\path[fill=fillColor,fill opacity=0.20] ( 66.21, 80.14) circle (  2.13);

\path[fill=fillColor,fill opacity=0.20] ( 74.51, 63.74) circle (  2.13);

\path[fill=fillColor,fill opacity=0.20] ( 74.73, 67.06) circle (  2.13);

\path[fill=fillColor,fill opacity=0.20] ( 74.08, 80.66) circle (  2.13);

\path[fill=fillColor,fill opacity=0.20] ( 78.23, 68.93) circle (  2.13);

\path[fill=fillColor,fill opacity=0.20] ( 83.91, 61.04) circle (  2.13);

\path[fill=fillColor,fill opacity=0.20] ( 95.71, 53.77) circle (  2.13);

\path[fill=fillColor,fill opacity=0.20] ( 92.21, 46.40) circle (  2.13);

\path[fill=fillColor,fill opacity=0.20] ( 99.42, 53.77) circle (  2.13);

\path[fill=fillColor,fill opacity=0.20] ( 91.12, 50.65) circle (  2.13);

\path[fill=fillColor,fill opacity=0.20] (104.23, 49.93) circle (  2.13);

\path[fill=fillColor,fill opacity=0.20] (104.67, 61.14) circle (  2.13);

\path[fill=fillColor,fill opacity=0.20] (109.04, 63.84) circle (  2.13);

\path[fill=fillColor,fill opacity=0.20] (110.57, 50.65) circle (  2.13);

\path[fill=fillColor,fill opacity=0.20] (119.09, 52.84) circle (  2.13);

\path[fill=fillColor,fill opacity=0.20] (115.81, 61.87) circle (  2.13);

\path[fill=fillColor,fill opacity=0.20] (115.59, 42.87) circle (  2.13);

\path[fill=fillColor,fill opacity=0.20] (115.37, 44.94) circle (  2.13);

\path[fill=fillColor,fill opacity=0.20] (105.32, 87.93) circle (  2.13);

\path[fill=fillColor,fill opacity=0.20] ( 85.88, 65.40) circle (  2.13);

\path[fill=fillColor,fill opacity=0.20] ( 79.54, 59.27) circle (  2.13);

\path[fill=fillColor,fill opacity=0.20] ( 76.70, 49.10) circle (  2.13);

\path[fill=fillColor,fill opacity=0.20] ( 79.32, 42.97) circle (  2.13);

\path[fill=fillColor,fill opacity=0.20] ( 79.76, 51.07) circle (  2.13);

\path[fill=fillColor,fill opacity=0.20] ( 78.67, 59.79) circle (  2.13);

\path[fill=fillColor,fill opacity=0.20] ( 81.51, 56.36) circle (  2.13);

\path[fill=fillColor,fill opacity=0.20] ( 77.79, 49.93) circle (  2.13);

\path[fill=fillColor,fill opacity=0.20] ( 65.34, 57.40) circle (  2.13);

\path[fill=fillColor,fill opacity=0.20] ( 77.14, 63.84) circle (  2.13);

\path[fill=fillColor,fill opacity=0.20] ( 64.90, 70.38) circle (  2.13);

\path[fill=fillColor,fill opacity=0.20] ( 57.25, 82.74) circle (  2.13);

\path[fill=fillColor,fill opacity=0.20] ( 52.01, 95.19) circle (  2.13);

\path[fill=fillColor,fill opacity=0.20] ( 66.43, 75.78) circle (  2.13);

\path[fill=fillColor,fill opacity=0.20] ( 76.04, 72.04) circle (  2.13);

\path[fill=fillColor,fill opacity=0.20] ( 78.67, 52.21) circle (  2.13);

\path[fill=fillColor,fill opacity=0.20] ( 89.59, 48.68) circle (  2.13);

\path[fill=fillColor,fill opacity=0.20] ( 94.84, 73.81) circle (  2.13);

\path[fill=fillColor,fill opacity=0.20] ( 94.40, 80.56) circle (  2.13);

\path[fill=fillColor,fill opacity=0.20] ( 84.57, 66.12) circle (  2.13);

\path[fill=fillColor,fill opacity=0.20] ( 86.10, 63.01) circle (  2.13);

\path[fill=fillColor,fill opacity=0.20] (106.42, 69.03) circle (  2.13);

\path[fill=fillColor,fill opacity=0.20] ( 98.77, 52.21) circle (  2.13);

\path[fill=fillColor,fill opacity=0.20] ( 90.47, 39.44) circle (  2.13);

\path[fill=fillColor,fill opacity=0.20] ( 95.49, 50.65) circle (  2.13);

\path[fill=fillColor,fill opacity=0.20] ( 92.21, 63.11) circle (  2.13);

\path[fill=fillColor,fill opacity=0.20] ( 96.58, 64.88) circle (  2.13);

\path[fill=fillColor,fill opacity=0.20] (103.58, 58.96) circle (  2.13);

\path[fill=fillColor,fill opacity=0.20] (100.95, 60.93) circle (  2.13);

\path[fill=fillColor,fill opacity=0.20] (100.52, 57.71) circle (  2.13);

\path[fill=fillColor,fill opacity=0.20] (129.36, 45.05) circle (  2.13);

\path[fill=fillColor,fill opacity=0.20] (114.50, 48.99) circle (  2.13);

\path[fill=fillColor,fill opacity=0.20] (121.06, 52.84) circle (  2.13);

\path[fill=fillColor,fill opacity=0.20] (128.70, 44.84) circle (  2.13);

\path[fill=fillColor,fill opacity=0.20] (104.89, 85.85) circle (  2.13);

\path[fill=fillColor,fill opacity=0.20] ( 86.10, 74.74) circle (  2.13);

\path[fill=fillColor,fill opacity=0.20] ( 77.79, 67.37) circle (  2.13);

\path[fill=fillColor,fill opacity=0.20] ( 79.10, 57.40) circle (  2.13);

\path[fill=fillColor,fill opacity=0.20] ( 82.16, 56.47) circle (  2.13);

\path[fill=fillColor,fill opacity=0.20] ( 77.36, 54.39) circle (  2.13);

\path[fill=fillColor,fill opacity=0.20] ( 76.70, 48.27) circle (  2.13);

\path[fill=fillColor,fill opacity=0.20] ( 77.79, 46.71) circle (  2.13);

\path[fill=fillColor,fill opacity=0.20] ( 79.32, 56.16) circle (  2.13);

\path[fill=fillColor,fill opacity=0.20] ( 75.61, 53.98) circle (  2.13);

\path[fill=fillColor,fill opacity=0.20] ( 74.51, 58.44) circle (  2.13);

\path[fill=fillColor,fill opacity=0.20] ( 70.80, 68.10) circle (  2.13);

\path[fill=fillColor,fill opacity=0.20] ( 68.62, 68.51) circle (  2.13);

\path[fill=fillColor,fill opacity=0.20] ( 62.50, 65.61) circle (  2.13);

\path[fill=fillColor,fill opacity=0.20] ( 59.66, 79.41) circle (  2.13);

\path[fill=fillColor,fill opacity=0.20] ( 52.66,100.39) circle (  2.13);

\path[fill=fillColor,fill opacity=0.20] ( 69.71, 82.74) circle (  2.13);

\path[fill=fillColor,fill opacity=0.20] ( 73.86, 65.81) circle (  2.13);

\path[fill=fillColor,fill opacity=0.20] ( 76.04, 74.12) circle (  2.13);

\path[fill=fillColor,fill opacity=0.20] ( 85.66, 74.12) circle (  2.13);

\path[fill=fillColor,fill opacity=0.20] ( 77.36, 53.87) circle (  2.13);

\path[fill=fillColor,fill opacity=0.20] ( 85.88, 52.11) circle (  2.13);

\path[fill=fillColor,fill opacity=0.20] ( 87.84, 73.08) circle (  2.13);

\path[fill=fillColor,fill opacity=0.20] (101.17, 76.51) circle (  2.13);

\path[fill=fillColor,fill opacity=0.20] ( 85.66, 68.20) circle (  2.13);

\path[fill=fillColor,fill opacity=0.20] ( 90.90, 56.47) circle (  2.13);

\path[fill=fillColor,fill opacity=0.20] ( 90.68, 48.47) circle (  2.13);

\path[fill=fillColor,fill opacity=0.20] ( 91.99, 53.15) circle (  2.13);

\path[fill=fillColor,fill opacity=0.20] ( 89.81, 57.20) circle (  2.13);

\path[fill=fillColor,fill opacity=0.20] ( 86.97, 61.45) circle (  2.13);

\path[fill=fillColor,fill opacity=0.20] ( 98.99, 72.66) circle (  2.13);

\path[fill=fillColor,fill opacity=0.20] ( 99.86, 69.13) circle (  2.13);

\path[fill=fillColor,fill opacity=0.20] ( 99.64, 54.29) circle (  2.13);

\path[fill=fillColor,fill opacity=0.20] (100.08, 55.12) circle (  2.13);

\path[fill=fillColor,fill opacity=0.20] (101.39, 55.85) circle (  2.13);

\path[fill=fillColor,fill opacity=0.20] (108.16, 51.69) circle (  2.13);

\path[fill=fillColor,fill opacity=0.20] ( 97.02, 55.95) circle (  2.13);

\path[fill=fillColor,fill opacity=0.20] (117.34, 54.60) circle (  2.13);

\path[fill=fillColor,fill opacity=0.20] (136.79, 52.11) circle (  2.13);

\path[fill=fillColor,fill opacity=0.20] (103.36, 93.12) circle (  2.13);

\path[fill=fillColor,fill opacity=0.20] ( 77.57, 52.84) circle (  2.13);

\path[fill=fillColor,fill opacity=0.20] ( 73.42, 59.17) circle (  2.13);

\path[fill=fillColor,fill opacity=0.20] ( 80.41, 55.02) circle (  2.13);

\path[fill=fillColor,fill opacity=0.20] ( 79.98, 54.70) circle (  2.13);

\path[fill=fillColor,fill opacity=0.20] ( 78.88, 55.64) circle (  2.13);

\path[fill=fillColor,fill opacity=0.20] ( 76.04, 58.03) circle (  2.13);

\path[fill=fillColor,fill opacity=0.20] ( 76.92, 57.61) circle (  2.13);

\path[fill=fillColor,fill opacity=0.20] ( 71.89, 58.96) circle (  2.13);

\path[fill=fillColor,fill opacity=0.20] ( 68.62, 58.96) circle (  2.13);

\path[fill=fillColor,fill opacity=0.20] ( 71.46, 55.85) circle (  2.13);

\path[fill=fillColor,fill opacity=0.20] ( 68.18, 54.39) circle (  2.13);

\path[fill=fillColor,fill opacity=0.20] ( 65.56, 58.55) circle (  2.13);

\path[fill=fillColor,fill opacity=0.20] ( 59.44, 80.35) circle (  2.13);

\path[fill=fillColor,fill opacity=0.20] ( 53.54, 96.23) circle (  2.13);

\path[fill=fillColor,fill opacity=0.20] ( 58.35, 79.00) circle (  2.13);

\path[fill=fillColor,fill opacity=0.20] ( 64.03, 72.77) circle (  2.13);

\path[fill=fillColor,fill opacity=0.20] ( 72.11, 77.75) circle (  2.13);

\path[fill=fillColor,fill opacity=0.20] ( 64.68, 86.89) circle (  2.13);

\path[fill=fillColor,fill opacity=0.20] ( 70.14, 69.65) circle (  2.13);

\path[fill=fillColor,fill opacity=0.20] ( 69.49, 70.48) circle (  2.13);

\path[fill=fillColor,fill opacity=0.20] ( 70.58, 73.39) circle (  2.13);

\path[fill=fillColor,fill opacity=0.20] ( 71.46, 69.34) circle (  2.13);

\path[fill=fillColor,fill opacity=0.20] ( 76.04, 66.44) circle (  2.13);

\path[fill=fillColor,fill opacity=0.20] ( 80.20, 67.99) circle (  2.13);

\path[fill=fillColor,fill opacity=0.20] ( 79.76, 71.73) circle (  2.13);

\path[fill=fillColor,fill opacity=0.20] ( 81.94, 60.83) circle (  2.13);

\path[fill=fillColor,fill opacity=0.20] ( 86.53, 55.12) circle (  2.13);

\path[fill=fillColor,fill opacity=0.20] ( 83.25, 61.97) circle (  2.13);

\path[fill=fillColor,fill opacity=0.20] ( 84.13, 68.72) circle (  2.13);

\path[fill=fillColor,fill opacity=0.20] ( 91.34, 69.45) circle (  2.13);

\path[fill=fillColor,fill opacity=0.20] ( 98.55, 57.82) circle (  2.13);

\path[fill=fillColor,fill opacity=0.20] ( 93.09, 47.33) circle (  2.13);

\path[fill=fillColor,fill opacity=0.20] ( 84.57, 59.48) circle (  2.13);

\path[fill=fillColor,fill opacity=0.20] ( 87.62, 70.48) circle (  2.13);

\path[fill=fillColor,fill opacity=0.20] ( 95.71, 69.03) circle (  2.13);

\path[fill=fillColor,fill opacity=0.20] (100.95, 64.15) circle (  2.13);

\path[fill=fillColor,fill opacity=0.20] (100.73, 51.38) circle (  2.13);

\path[fill=fillColor,fill opacity=0.20] ( 92.87, 53.77) circle (  2.13);

\path[fill=fillColor,fill opacity=0.20] ( 99.64, 58.65) circle (  2.13);

\path[fill=fillColor,fill opacity=0.20] ( 98.99, 49.93) circle (  2.13);

\path[fill=fillColor,fill opacity=0.20] ( 95.93, 54.91) circle (  2.13);

\path[fill=fillColor,fill opacity=0.20] ( 91.78, 56.68) circle (  2.13);

\path[fill=fillColor,fill opacity=0.20] (115.37, 48.79) circle (  2.13);

\path[fill=fillColor,fill opacity=0.20] (123.24, 58.44) circle (  2.13);

\path[fill=fillColor,fill opacity=0.20] ( 82.60, 64.36) circle (  2.13);

\path[fill=fillColor,fill opacity=0.20] ( 92.65, 64.26) circle (  2.13);

\path[fill=fillColor,fill opacity=0.20] ( 75.17, 53.77) circle (  2.13);

\path[fill=fillColor,fill opacity=0.20] ( 79.10, 44.43) circle (  2.13);

\path[fill=fillColor,fill opacity=0.20] ( 77.36, 54.39) circle (  2.13);

\path[fill=fillColor,fill opacity=0.20] ( 72.99, 58.65) circle (  2.13);

\path[fill=fillColor,fill opacity=0.20] ( 69.71, 54.18) circle (  2.13);

\path[fill=fillColor,fill opacity=0.20] ( 76.48, 58.65) circle (  2.13);

\path[fill=fillColor,fill opacity=0.20] ( 79.32, 58.75) circle (  2.13);

\path[fill=fillColor,fill opacity=0.20] ( 76.26, 55.74) circle (  2.13);

\path[fill=fillColor,fill opacity=0.20] ( 71.24, 63.53) circle (  2.13);

\path[fill=fillColor,fill opacity=0.20] ( 55.29, 88.96) circle (  2.13);

\path[fill=fillColor,fill opacity=0.20] ( 56.82, 72.77) circle (  2.13);

\path[fill=fillColor,fill opacity=0.20] ( 47.20,109.73) circle (  2.13);

\path[fill=fillColor,fill opacity=0.20] ( 69.27,102.46) circle (  2.13);

\path[fill=fillColor,fill opacity=0.20] ( 54.85, 87.93) circle (  2.13);

\path[fill=fillColor,fill opacity=0.20] ( 63.81, 83.77) circle (  2.13);

\path[fill=fillColor,fill opacity=0.20] ( 65.77, 82.74) circle (  2.13);

\path[fill=fillColor,fill opacity=0.20] ( 62.72, 50.34) circle (  2.13);

\path[fill=fillColor,fill opacity=0.20] ( 81.29, 58.65) circle (  2.13);

\path[fill=fillColor,fill opacity=0.20] ( 75.17, 65.81) circle (  2.13);

\path[fill=fillColor,fill opacity=0.20] ( 80.85, 64.15) circle (  2.13);

\path[fill=fillColor,fill opacity=0.20] ( 78.23, 81.59) circle (  2.13);

\path[fill=fillColor,fill opacity=0.20] (109.69, 83.77) circle (  2.13);

\path[fill=fillColor,fill opacity=0.20] ( 72.77, 78.89) circle (  2.13);

\path[fill=fillColor,fill opacity=0.20] ( 82.82, 65.19) circle (  2.13);

\path[fill=fillColor,fill opacity=0.20] ( 83.91, 45.15) circle (  2.13);

\path[fill=fillColor,fill opacity=0.20] ( 93.96, 46.09) circle (  2.13);

\path[fill=fillColor,fill opacity=0.20] ( 92.43, 47.12) circle (  2.13);

\path[fill=fillColor,fill opacity=0.20] ( 93.52, 53.77) circle (  2.13);

\path[fill=fillColor,fill opacity=0.20] ( 97.02, 66.12) circle (  2.13);

\path[fill=fillColor,fill opacity=0.20] ( 95.49, 73.91) circle (  2.13);

\path[fill=fillColor,fill opacity=0.20] ( 89.81, 56.05) circle (  2.13);

\path[fill=fillColor,fill opacity=0.20] ( 91.34, 64.26) circle (  2.13);

\path[fill=fillColor,fill opacity=0.20] ( 94.62, 69.34) circle (  2.13);

\path[fill=fillColor,fill opacity=0.20] ( 93.31, 62.80) circle (  2.13);

\path[fill=fillColor,fill opacity=0.20] ( 95.05, 52.00) circle (  2.13);

\path[fill=fillColor,fill opacity=0.20] ( 86.10, 45.26) circle (  2.13);

\path[fill=fillColor,fill opacity=0.20] ( 93.52, 59.58) circle (  2.13);

\path[fill=fillColor,fill opacity=0.20] ( 93.31, 67.99) circle (  2.13);

\path[fill=fillColor,fill opacity=0.20] ( 90.03, 55.53) circle (  2.13);

\path[fill=fillColor,fill opacity=0.20] ( 85.00, 57.09) circle (  2.13);

\path[fill=fillColor,fill opacity=0.20] ( 92.87, 56.47) circle (  2.13);

\path[fill=fillColor,fill opacity=0.20] ( 93.09, 50.97) circle (  2.13);

\path[fill=fillColor,fill opacity=0.20] ( 99.64, 62.80) circle (  2.13);

\path[fill=fillColor,fill opacity=0.20] ( 79.32, 47.64) circle (  2.13);

\path[fill=fillColor,fill opacity=0.20] ( 77.14, 55.02) circle (  2.13);

\path[fill=fillColor,fill opacity=0.20] ( 81.94, 56.05) circle (  2.13);

\path[fill=fillColor,fill opacity=0.20] ( 77.36, 72.77) circle (  2.13);

\path[fill=fillColor,fill opacity=0.20] ( 71.67, 59.79) circle (  2.13);

\path[fill=fillColor,fill opacity=0.20] ( 76.04, 42.87) circle (  2.13);

\path[fill=fillColor,fill opacity=0.20] ( 74.08, 49.31) circle (  2.13);

\path[fill=fillColor,fill opacity=0.20] ( 73.42, 57.20) circle (  2.13);

\path[fill=fillColor,fill opacity=0.20] ( 70.58, 64.26) circle (  2.13);

\path[fill=fillColor,fill opacity=0.20] ( 69.93, 72.56) circle (  2.13);

\path[fill=fillColor,fill opacity=0.20] ( 75.39, 56.16) circle (  2.13);

\path[fill=fillColor,fill opacity=0.20] ( 69.49, 40.69) circle (  2.13);

\path[fill=fillColor,fill opacity=0.20] ( 68.62, 68.62) circle (  2.13);

\path[fill=fillColor,fill opacity=0.20] ( 68.83, 76.51) circle (  2.13);

\path[fill=fillColor,fill opacity=0.20] ( 63.81, 59.58) circle (  2.13);

\path[fill=fillColor,fill opacity=0.20] ( 64.25, 63.74) circle (  2.13);

\path[fill=fillColor,fill opacity=0.20] ( 64.03, 69.24) circle (  2.13);

\path[fill=fillColor,fill opacity=0.20] ( 71.24, 73.60) circle (  2.13);

\path[fill=fillColor,fill opacity=0.20] ( 64.25, 78.06) circle (  2.13);

\path[fill=fillColor,fill opacity=0.20] ( 68.18, 68.41) circle (  2.13);

\path[fill=fillColor,fill opacity=0.20] ( 75.61, 61.14) circle (  2.13);

\path[fill=fillColor,fill opacity=0.20] (115.81, 63.63) circle (  2.13);

\path[fill=fillColor,fill opacity=0.20] ( 78.01, 57.30) circle (  2.13);

\path[fill=fillColor,fill opacity=0.20] ( 71.67, 47.33) circle (  2.13);

\path[fill=fillColor,fill opacity=0.20] (101.83, 55.02) circle (  2.13);

\path[fill=fillColor,fill opacity=0.20] ( 82.16, 72.46) circle (  2.13);

\path[fill=fillColor,fill opacity=0.20] ( 75.61, 72.56) circle (  2.13);

\path[fill=fillColor,fill opacity=0.20] ( 82.16, 49.31) circle (  2.13);

\path[fill=fillColor,fill opacity=0.20] ( 77.36, 40.69) circle (  2.13);

\path[fill=fillColor,fill opacity=0.20] ( 81.73, 67.79) circle (  2.13);

\path[fill=fillColor,fill opacity=0.20] ( 80.63, 67.79) circle (  2.13);

\path[fill=fillColor,fill opacity=0.20] ( 87.19, 63.84) circle (  2.13);

\path[fill=fillColor,fill opacity=0.20] ( 85.22, 55.95) circle (  2.13);

\path[fill=fillColor,fill opacity=0.20] ( 92.87, 60.83) circle (  2.13);

\path[fill=fillColor,fill opacity=0.20] (104.45, 61.66) circle (  2.13);

\path[fill=fillColor,fill opacity=0.20] ( 94.62, 70.28) circle (  2.13);

\path[fill=fillColor,fill opacity=0.20] ( 84.57, 62.80) circle (  2.13);

\path[fill=fillColor,fill opacity=0.20] ( 89.15, 59.69) circle (  2.13);

\path[fill=fillColor,fill opacity=0.20] ( 93.52, 60.62) circle (  2.13);

\path[fill=fillColor,fill opacity=0.20] ( 90.03, 57.30) circle (  2.13);

\path[fill=fillColor,fill opacity=0.20] ( 92.65, 53.46) circle (  2.13);

\path[fill=fillColor,fill opacity=0.20] ( 94.40, 58.55) circle (  2.13);

\path[fill=fillColor,fill opacity=0.20] ( 92.21, 57.61) circle (  2.13);

\path[fill=fillColor,fill opacity=0.20] ( 87.41, 62.91) circle (  2.13);

\path[fill=fillColor,fill opacity=0.20] ( 89.37, 66.33) circle (  2.13);

\path[fill=fillColor,fill opacity=0.20] ( 85.22, 62.59) circle (  2.13);

\path[fill=fillColor,fill opacity=0.20] ( 78.88, 65.92) circle (  2.13);

\path[fill=fillColor,fill opacity=0.20] ( 90.68, 61.66) circle (  2.13);

\path[fill=fillColor,fill opacity=0.20] ( 94.40, 55.12) circle (  2.13);

\path[fill=fillColor,fill opacity=0.20] ( 97.24, 71.42) circle (  2.13);

\path[fill=fillColor,fill opacity=0.20] ( 80.20, 54.29) circle (  2.13);

\path[fill=fillColor,fill opacity=0.20] ( 77.36, 55.43) circle (  2.13);

\path[fill=fillColor,fill opacity=0.20] ( 71.46, 59.48) circle (  2.13);

\path[fill=fillColor,fill opacity=0.20] ( 81.73, 55.64) circle (  2.13);

\path[fill=fillColor,fill opacity=0.20] ( 77.79, 53.77) circle (  2.13);

\path[fill=fillColor,fill opacity=0.20] ( 80.85, 50.14) circle (  2.13);

\path[fill=fillColor,fill opacity=0.20] ( 74.73, 44.32) circle (  2.13);

\path[fill=fillColor,fill opacity=0.20] ( 75.39, 49.82) circle (  2.13);

\path[fill=fillColor,fill opacity=0.20] ( 76.04, 52.63) circle (  2.13);

\path[fill=fillColor,fill opacity=0.20] ( 84.78, 48.79) circle (  2.13);

\path[fill=fillColor,fill opacity=0.20] ( 78.45, 47.85) circle (  2.13);

\path[fill=fillColor,fill opacity=0.20] ( 81.73, 56.78) circle (  2.13);

\path[fill=fillColor,fill opacity=0.20] ( 76.04, 68.62) circle (  2.13);

\path[fill=fillColor,fill opacity=0.20] ( 74.30, 61.45) circle (  2.13);

\path[fill=fillColor,fill opacity=0.20] ( 77.79, 55.85) circle (  2.13);

\path[fill=fillColor,fill opacity=0.20] ( 79.76, 56.05) circle (  2.13);

\path[fill=fillColor,fill opacity=0.20] ( 81.29, 57.71) circle (  2.13);

\path[fill=fillColor,fill opacity=0.20] ( 83.91, 60.93) circle (  2.13);

\path[fill=fillColor,fill opacity=0.20] (123.02, 57.92) circle (  2.13);

\path[fill=fillColor,fill opacity=0.20] ( 91.78, 46.61) circle (  2.13);

\path[fill=fillColor,fill opacity=0.20] ( 88.06, 51.80) circle (  2.13);

\path[fill=fillColor,fill opacity=0.20] ( 86.10, 69.24) circle (  2.13);

\path[fill=fillColor,fill opacity=0.20] ( 94.18, 69.13) circle (  2.13);

\path[fill=fillColor,fill opacity=0.20] ( 98.99, 58.65) circle (  2.13);

\path[fill=fillColor,fill opacity=0.20] ( 93.96, 48.99) circle (  2.13);

\path[fill=fillColor,fill opacity=0.20] ( 90.03, 61.97) circle (  2.13);

\path[fill=fillColor,fill opacity=0.20] ( 96.80, 71.83) circle (  2.13);

\path[fill=fillColor,fill opacity=0.20] ( 94.18, 60.21) circle (  2.13);

\path[fill=fillColor,fill opacity=0.20] ( 87.84, 55.53) circle (  2.13);

\path[fill=fillColor,fill opacity=0.20] ( 91.99, 53.25) circle (  2.13);

\path[fill=fillColor,fill opacity=0.20] ( 86.75, 52.73) circle (  2.13);

\path[fill=fillColor,fill opacity=0.20] ( 86.75, 65.19) circle (  2.13);

\path[fill=fillColor,fill opacity=0.20] ( 93.52, 60.62) circle (  2.13);

\path[fill=fillColor,fill opacity=0.20] (102.48, 60.93) circle (  2.13);

\path[fill=fillColor,fill opacity=0.20] ( 78.67, 49.31) circle (  2.13);

\path[fill=fillColor,fill opacity=0.20] ( 84.13, 48.58) circle (  2.13);

\path[fill=fillColor,fill opacity=0.20] ( 84.35, 53.46) circle (  2.13);

\path[fill=fillColor,fill opacity=0.20] ( 84.35, 58.55) circle (  2.13);

\path[fill=fillColor,fill opacity=0.20] ( 73.42, 61.56) circle (  2.13);

\path[fill=fillColor,fill opacity=0.20] ( 75.61, 55.02) circle (  2.13);

\path[fill=fillColor,fill opacity=0.20] ( 79.32, 46.50) circle (  2.13);

\path[fill=fillColor,fill opacity=0.20] ( 76.26, 39.03) circle (  2.13);

\path[fill=fillColor,fill opacity=0.20] ( 79.98, 39.13) circle (  2.13);

\path[fill=fillColor,fill opacity=0.20] ( 75.39, 58.96) circle (  2.13);

\path[fill=fillColor,fill opacity=0.20] ( 76.70, 69.55) circle (  2.13);

\path[fill=fillColor,fill opacity=0.20] ( 83.91, 55.64) circle (  2.13);

\path[fill=fillColor,fill opacity=0.20] ( 82.82, 49.62) circle (  2.13);

\path[fill=fillColor,fill opacity=0.20] ( 82.82, 54.60) circle (  2.13);

\path[fill=fillColor,fill opacity=0.20] (100.30, 56.16) circle (  2.13);

\path[fill=fillColor,fill opacity=0.20] ( 81.29, 59.38) circle (  2.13);

\path[fill=fillColor,fill opacity=0.20] ( 84.35, 61.87) circle (  2.13);

\path[fill=fillColor,fill opacity=0.20] ( 85.00, 63.53) circle (  2.13);

\path[fill=fillColor,fill opacity=0.20] ( 92.65, 62.28) circle (  2.13);

\path[fill=fillColor,fill opacity=0.20] (100.52, 57.71) circle (  2.13);

\path[fill=fillColor,fill opacity=0.20] ( 84.57, 67.37) circle (  2.13);

\path[fill=fillColor,fill opacity=0.20] ( 94.18, 82.74) circle (  2.13);

\path[fill=fillColor,fill opacity=0.20] (105.54, 82.74) circle (  2.13);

\path[fill=fillColor,fill opacity=0.20] ( 88.50, 48.27) circle (  2.13);

\path[fill=fillColor,fill opacity=0.20] (108.16, 56.99) circle (  2.13);

\path[fill=fillColor,fill opacity=0.20] ( 82.38, 56.88) circle (  2.13);

\path[fill=fillColor,fill opacity=0.20] ( 89.59, 47.44) circle (  2.13);

\path[fill=fillColor,fill opacity=0.20] ( 80.41, 43.28) circle (  2.13);

\path[fill=fillColor,fill opacity=0.20] ( 86.75, 57.82) circle (  2.13);

\path[fill=fillColor,fill opacity=0.20] ( 82.82, 74.01) circle (  2.13);

\path[fill=fillColor,fill opacity=0.20] ( 97.02, 61.35) circle (  2.13);

\path[fill=fillColor,fill opacity=0.20] ( 87.41, 46.71) circle (  2.13);

\path[fill=fillColor,fill opacity=0.20] ( 90.68, 58.55) circle (  2.13);

\path[fill=fillColor,fill opacity=0.20] ( 94.18, 66.44) circle (  2.13);

\path[fill=fillColor,fill opacity=0.20] ( 89.59, 64.15) circle (  2.13);

\path[fill=fillColor,fill opacity=0.20] ( 88.72, 63.32) circle (  2.13);

\path[fill=fillColor,fill opacity=0.20] ( 89.37, 56.68) circle (  2.13);

\path[fill=fillColor,fill opacity=0.20] ( 97.46, 69.76) circle (  2.13);

\path[fill=fillColor,fill opacity=0.20] ( 95.93, 79.52) circle (  2.13);

\path[fill=fillColor,fill opacity=0.20] ( 88.94, 70.48) circle (  2.13);

\path[fill=fillColor,fill opacity=0.20] (100.08, 99.35) circle (  2.13);

\path[fill=fillColor,fill opacity=0.20] (101.83, 81.70) circle (  2.13);

\path[fill=fillColor,fill opacity=0.20] (100.95, 78.38) circle (  2.13);

\path[fill=fillColor,fill opacity=0.20] ( 90.47, 68.62) circle (  2.13);

\path[fill=fillColor,fill opacity=0.20] ( 93.52, 94.16) circle (  2.13);

\path[fill=fillColor,fill opacity=0.20] (108.60, 60.10) circle (  2.13);

\path[fill=fillColor,fill opacity=0.20] (130.23, 51.28) circle (  2.13);

\path[fill=fillColor,fill opacity=0.20] (102.92, 52.21) circle (  2.13);

\path[fill=fillColor,fill opacity=0.20] ( 93.52, 62.49) circle (  2.13);

\path[fill=fillColor,fill opacity=0.20] ( 89.15, 74.95) circle (  2.13);

\path[fill=fillColor,fill opacity=0.20] ( 79.10,107.65) circle (  2.13);

\path[fill=fillColor,fill opacity=0.20] ( 75.83,111.81) circle (  2.13);

\path[fill=fillColor,fill opacity=0.20] ( 82.82,106.61) circle (  2.13);

\path[fill=fillColor,fill opacity=0.20] ( 78.67, 88.96) circle (  2.13);

\path[fill=fillColor,fill opacity=0.20] ( 76.70, 84.81) circle (  2.13);

\path[fill=fillColor,fill opacity=0.20] ( 85.44, 95.19) circle (  2.13);

\path[fill=fillColor,fill opacity=0.20] (100.95, 55.43) circle (  2.13);

\path[fill=fillColor,fill opacity=0.20] (110.13, 62.39) circle (  2.13);

\path[fill=fillColor,fill opacity=0.20] (136.57, 64.15) circle (  2.13);

\path[fill=fillColor,fill opacity=0.20] (110.57, 51.38) circle (  2.13);

\path[fill=fillColor,fill opacity=0.20] ( 97.02, 61.24) circle (  2.13);

\path[fill=fillColor,fill opacity=0.20] (119.53, 78.38) circle (  2.13);

\path[fill=fillColor,fill opacity=0.20] ( 85.88, 71.63) circle (  2.13);

\path[fill=fillColor,fill opacity=0.20] ( 72.77, 73.39) circle (  2.13);

\path[fill=fillColor,fill opacity=0.20] ( 86.75,114.92) circle (  2.13);

\path[fill=fillColor,fill opacity=0.20] ( 92.21, 84.81) circle (  2.13);

\path[fill=fillColor,fill opacity=0.20] ( 84.57, 69.97) circle (  2.13);

\path[fill=fillColor,fill opacity=0.20] ( 85.66, 69.45) circle (  2.13);

\path[fill=fillColor,fill opacity=0.20] ( 87.19, 70.38) circle (  2.13);

\path[fill=fillColor,fill opacity=0.20] ( 90.25, 61.76) circle (  2.13);

\path[fill=fillColor,fill opacity=0.20] ( 95.27, 52.21) circle (  2.13);

\path[fill=fillColor,fill opacity=0.20] ( 94.84, 57.20) circle (  2.13);

\path[fill=fillColor,fill opacity=0.20] ( 81.29, 70.48) circle (  2.13);

\path[fill=fillColor,fill opacity=0.20] ( 67.52, 79.41) circle (  2.13);

\path[fill=fillColor,fill opacity=0.20] (105.10, 78.06) circle (  2.13);

\path[fill=fillColor,fill opacity=0.20] (107.73, 46.40) circle (  2.13);

\path[fill=fillColor,fill opacity=0.20] (122.58, 78.06) circle (  2.13);

\path[fill=fillColor,fill opacity=0.20] (123.90, 73.39) circle (  2.13);

\path[fill=fillColor,fill opacity=0.20] (136.35, 63.11) circle (  2.13);

\path[fill=fillColor,fill opacity=0.20] (100.08, 71.94) circle (  2.13);

\path[fill=fillColor,fill opacity=0.20] ( 98.33, 74.12) circle (  2.13);

\path[fill=fillColor,fill opacity=0.20] ( 98.55, 64.05) circle (  2.13);

\path[fill=fillColor,fill opacity=0.20] ( 98.11, 56.05) circle (  2.13);

\path[fill=fillColor,fill opacity=0.20] ( 77.36, 57.92) circle (  2.13);

\path[fill=fillColor,fill opacity=0.20] ( 88.50, 96.23) circle (  2.13);

\path[fill=fillColor,fill opacity=0.20] ( 92.87, 47.23) circle (  2.13);

\path[fill=fillColor,fill opacity=0.20] ( 96.36, 40.69) circle (  2.13);

\path[fill=fillColor,fill opacity=0.20] ( 96.36, 51.49) circle (  2.13);

\path[fill=fillColor,fill opacity=0.20] ( 87.84, 63.53) circle (  2.13);

\path[fill=fillColor,fill opacity=0.20] ( 90.90, 63.84) circle (  2.13);

\path[fill=fillColor,fill opacity=0.20] ( 91.34, 52.73) circle (  2.13);

\path[fill=fillColor,fill opacity=0.20] (126.52, 46.40) circle (  2.13);

\path[fill=fillColor,fill opacity=0.20] ( 74.51, 47.85) circle (  2.13);

\path[fill=fillColor,fill opacity=0.20] ( 60.97, 51.17) circle (  2.13);

\path[fill=fillColor,fill opacity=0.20] (106.85, 75.68) circle (  2.13);

\path[fill=fillColor,fill opacity=0.20] (113.84, 52.94) circle (  2.13);

\path[fill=fillColor,fill opacity=0.20] (115.37, 81.70) circle (  2.13);

\path[fill=fillColor,fill opacity=0.20] (151.43, 74.12) circle (  2.13);

\path[fill=fillColor,fill opacity=0.20] (123.02, 82.74) circle (  2.13);

\path[fill=fillColor,fill opacity=0.20] (101.17, 62.80) circle (  2.13);

\path[fill=fillColor,fill opacity=0.20] ( 91.56, 47.33) circle (  2.13);

\path[fill=fillColor,fill opacity=0.20] ( 89.37, 64.77) circle (  2.13);

\path[fill=fillColor,fill opacity=0.20] ( 85.44, 69.13) circle (  2.13);

\path[fill=fillColor,fill opacity=0.20] ( 76.48, 58.13) circle (  2.13);

\path[fill=fillColor,fill opacity=0.20] ( 94.84, 91.04) circle (  2.13);

\path[fill=fillColor,fill opacity=0.20] (102.05, 44.22) circle (  2.13);

\path[fill=fillColor,fill opacity=0.20] (106.85, 41.00) circle (  2.13);

\path[fill=fillColor,fill opacity=0.20] (113.41, 54.39) circle (  2.13);

\path[fill=fillColor,fill opacity=0.20] (123.24, 60.52) circle (  2.13);

\path[fill=fillColor,fill opacity=0.20] (114.50, 61.76) circle (  2.13);

\path[fill=fillColor,fill opacity=0.20] (106.63, 58.03) circle (  2.13);

\path[fill=fillColor,fill opacity=0.20] ( 97.46, 51.90) circle (  2.13);

\path[fill=fillColor,fill opacity=0.20] ( 81.51, 57.71) circle (  2.13);

\path[fill=fillColor,fill opacity=0.20] ( 73.64, 66.33) circle (  2.13);

\path[fill=fillColor,fill opacity=0.20] ( 76.48, 53.87) circle (  2.13);

\path[fill=fillColor,fill opacity=0.20] ( 52.01, 54.18) circle (  2.13);

\path[fill=fillColor,fill opacity=0.20] (102.70, 83.77) circle (  2.13);

\path[fill=fillColor,fill opacity=0.20] (112.53, 50.34) circle (  2.13);

\path[fill=fillColor,fill opacity=0.20] (124.55, 72.46) circle (  2.13);

\path[fill=fillColor,fill opacity=0.20] (124.11, 64.67) circle (  2.13);

\path[fill=fillColor,fill opacity=0.20] (145.53, 72.04) circle (  2.13);

\path[fill=fillColor,fill opacity=0.20] (117.78, 52.94) circle (  2.13);

\path[fill=fillColor,fill opacity=0.20] ( 94.18, 43.39) circle (  2.13);

\path[fill=fillColor,fill opacity=0.20] ( 88.50, 65.92) circle (  2.13);

\path[fill=fillColor,fill opacity=0.20] ( 92.21, 66.33) circle (  2.13);

\path[fill=fillColor,fill opacity=0.20] ( 80.20, 55.43) circle (  2.13);

\path[fill=fillColor,fill opacity=0.20] (103.14, 54.39) circle (  2.13);

\path[fill=fillColor,fill opacity=0.20] (107.73, 38.20) circle (  2.13);

\path[fill=fillColor,fill opacity=0.20] (125.43, 51.38) circle (  2.13);

\path[fill=fillColor,fill opacity=0.20] (131.11, 61.66) circle (  2.13);

\path[fill=fillColor,fill opacity=0.20] (148.80, 65.40) circle (  2.13);

\path[fill=fillColor,fill opacity=0.20] (109.47, 50.65) circle (  2.13);

\path[fill=fillColor,fill opacity=0.20] ( 95.93, 38.09) circle (  2.13);

\path[fill=fillColor,fill opacity=0.20] ( 83.69, 47.02) circle (  2.13);

\path[fill=fillColor,fill opacity=0.20] ( 66.21, 61.97) circle (  2.13);

\path[fill=fillColor,fill opacity=0.20] ( 83.69, 70.69) circle (  2.13);

\path[fill=fillColor,fill opacity=0.20] ( 96.15, 39.44) circle (  2.13);

\path[fill=fillColor,fill opacity=0.20] (110.35, 62.08) circle (  2.13);

\path[fill=fillColor,fill opacity=0.20] (124.11, 59.06) circle (  2.13);

\path[fill=fillColor,fill opacity=0.20] (114.28, 48.89) circle (  2.13);

\path[fill=fillColor,fill opacity=0.20] (111.22, 53.46) circle (  2.13);

\path[fill=fillColor,fill opacity=0.20] (112.32, 53.25) circle (  2.13);

\path[fill=fillColor,fill opacity=0.20] ( 99.64, 55.12) circle (  2.13);

\path[fill=fillColor,fill opacity=0.20] ( 89.59, 57.92) circle (  2.13);

\path[fill=fillColor,fill opacity=0.20] (114.28, 50.34) circle (  2.13);

\path[fill=fillColor,fill opacity=0.20] ( 87.62, 93.12) circle (  2.13);

\path[fill=fillColor,fill opacity=0.20] ( 99.42, 42.97) circle (  2.13);

\path[fill=fillColor,fill opacity=0.20] (113.19, 45.15) circle (  2.13);

\path[fill=fillColor,fill opacity=0.20] (128.27, 51.17) circle (  2.13);

\path[fill=fillColor,fill opacity=0.20] (132.85, 49.10) circle (  2.13);

\path[fill=fillColor,fill opacity=0.20] ( 95.49, 61.66) circle (  2.13);

\path[fill=fillColor,fill opacity=0.20] ( 83.91, 56.16) circle (  2.13);

\path[fill=fillColor,fill opacity=0.20] ( 77.79, 42.45) circle (  2.13);

\path[fill=fillColor,fill opacity=0.20] ( 63.15, 44.43) circle (  2.13);

\path[fill=fillColor,fill opacity=0.20] ( 91.78, 63.53) circle (  2.13);

\path[fill=fillColor,fill opacity=0.20] ( 90.03, 51.90) circle (  2.13);

\path[fill=fillColor,fill opacity=0.20] ( 94.62, 65.29) circle (  2.13);

\path[fill=fillColor,fill opacity=0.20] (121.49, 62.49) circle (  2.13);

\path[fill=fillColor,fill opacity=0.20] (106.20, 52.42) circle (  2.13);

\path[fill=fillColor,fill opacity=0.20] (107.29, 55.53) circle (  2.13);

\path[fill=fillColor,fill opacity=0.20] (111.22, 62.08) circle (  2.13);

\path[fill=fillColor,fill opacity=0.20] (110.13, 56.78) circle (  2.13);

\path[fill=fillColor,fill opacity=0.20] ( 95.27, 50.45) circle (  2.13);

\path[fill=fillColor,fill opacity=0.20] ( 91.34, 47.96) circle (  2.13);

\path[fill=fillColor,fill opacity=0.20] ( 98.77, 73.81) circle (  2.13);

\path[fill=fillColor,fill opacity=0.20] (112.97, 51.38) circle (  2.13);

\path[fill=fillColor,fill opacity=0.20] (120.84, 60.10) circle (  2.13);

\path[fill=fillColor,fill opacity=0.20] (132.42, 52.52) circle (  2.13);

\path[fill=fillColor,fill opacity=0.20] (105.76, 57.82) circle (  2.13);

\path[fill=fillColor,fill opacity=0.20] (105.32, 64.57) circle (  2.13);

\path[fill=fillColor,fill opacity=0.20] (101.61, 68.10) circle (  2.13);

\path[fill=fillColor,fill opacity=0.20] ( 78.01, 64.88) circle (  2.13);

\path[fill=fillColor,fill opacity=0.20] ( 68.18, 53.87) circle (  2.13);

\path[fill=fillColor,fill opacity=0.20] ( 76.26, 53.77) circle (  2.13);

\path[fill=fillColor,fill opacity=0.20] ( 90.25, 60.93) circle (  2.13);

\path[fill=fillColor,fill opacity=0.20] ( 99.21, 57.92) circle (  2.13);

\path[fill=fillColor,fill opacity=0.20] ( 92.65, 61.97) circle (  2.13);

\path[fill=fillColor,fill opacity=0.20] ( 88.06, 61.35) circle (  2.13);

\path[fill=fillColor,fill opacity=0.20] ( 94.84, 61.87) circle (  2.13);

\path[fill=fillColor,fill opacity=0.20] ( 99.64, 63.11) circle (  2.13);

\path[fill=fillColor,fill opacity=0.20] (117.56, 55.95) circle (  2.13);

\path[fill=fillColor,fill opacity=0.20] (109.47, 49.10) circle (  2.13);

\path[fill=fillColor,fill opacity=0.20] ( 98.77, 50.55) circle (  2.13);

\path[fill=fillColor,fill opacity=0.20] ( 82.38, 51.90) circle (  2.13);

\path[fill=fillColor,fill opacity=0.20] ( 65.77, 52.84) circle (  2.13);

\path[fill=fillColor,fill opacity=0.20] ( 59.88, 57.61) circle (  2.13);

\path[fill=fillColor,fill opacity=0.20] ( 97.02, 69.13) circle (  2.13);

\path[fill=fillColor,fill opacity=0.20] (118.65, 54.81) circle (  2.13);

\path[fill=fillColor,fill opacity=0.20] (119.96, 69.97) circle (  2.13);

\path[fill=fillColor,fill opacity=0.20] (114.94, 73.91) circle (  2.13);

\path[fill=fillColor,fill opacity=0.20] (100.52, 74.01) circle (  2.13);

\path[fill=fillColor,fill opacity=0.20] ( 87.19, 63.42) circle (  2.13);

\path[fill=fillColor,fill opacity=0.20] ( 86.97, 55.12) circle (  2.13);

\path[fill=fillColor,fill opacity=0.20] ( 80.85, 55.95) circle (  2.13);

\path[fill=fillColor,fill opacity=0.20] ( 82.38, 67.06) circle (  2.13);

\path[fill=fillColor,fill opacity=0.20] ( 88.50, 44.32) circle (  2.13);

\path[fill=fillColor,fill opacity=0.20] ( 89.37, 48.47) circle (  2.13);

\path[fill=fillColor,fill opacity=0.20] (102.26, 55.85) circle (  2.13);

\path[fill=fillColor,fill opacity=0.20] ( 96.58, 67.68) circle (  2.13);

\path[fill=fillColor,fill opacity=0.20] (102.05, 60.52) circle (  2.13);

\path[fill=fillColor,fill opacity=0.20] ( 93.96, 46.40) circle (  2.13);

\path[fill=fillColor,fill opacity=0.20] ( 91.99, 47.44) circle (  2.13);

\path[fill=fillColor,fill opacity=0.20] ( 84.78, 49.20) circle (  2.13);

\path[fill=fillColor,fill opacity=0.20] ( 73.86, 47.02) circle (  2.13);

\path[fill=fillColor,fill opacity=0.20] ( 94.84, 76.82) circle (  2.13);

\path[fill=fillColor,fill opacity=0.20] (104.89, 48.06) circle (  2.13);

\path[fill=fillColor,fill opacity=0.20] (119.96, 54.08) circle (  2.13);

\path[fill=fillColor,fill opacity=0.20] (121.49, 61.97) circle (  2.13);

\path[fill=fillColor,fill opacity=0.20] (107.95, 79.00) circle (  2.13);

\path[fill=fillColor,fill opacity=0.20] ( 96.15, 81.49) circle (  2.13);

\path[fill=fillColor,fill opacity=0.20] ( 83.25, 64.77) circle (  2.13);

\path[fill=fillColor,fill opacity=0.20] ( 82.38, 59.48) circle (  2.13);

\path[fill=fillColor,fill opacity=0.20] ( 79.76, 59.17) circle (  2.13);

\path[fill=fillColor,fill opacity=0.20] ( 62.50, 63.94) circle (  2.13);

\path[fill=fillColor,fill opacity=0.20] ( 77.79,101.42) circle (  2.13);

\path[fill=fillColor,fill opacity=0.20] ( 82.82, 62.49) circle (  2.13);

\path[fill=fillColor,fill opacity=0.20] ( 85.66, 44.74) circle (  2.13);

\path[fill=fillColor,fill opacity=0.20] ( 94.18, 55.22) circle (  2.13);

\path[fill=fillColor,fill opacity=0.20] ( 98.11, 61.14) circle (  2.13);

\path[fill=fillColor,fill opacity=0.20] ( 98.77, 55.33) circle (  2.13);

\path[fill=fillColor,fill opacity=0.20] ( 92.87, 60.00) circle (  2.13);

\path[fill=fillColor,fill opacity=0.20] ( 93.09, 60.21) circle (  2.13);

\path[fill=fillColor,fill opacity=0.20] ( 93.31, 54.91) circle (  2.13);

\path[fill=fillColor,fill opacity=0.20] ( 88.94, 53.35) circle (  2.13);

\path[fill=fillColor,fill opacity=0.20] ( 83.69, 46.19) circle (  2.13);

\path[fill=fillColor,fill opacity=0.20] (120.62, 59.69) circle (  2.13);

\path[fill=fillColor,fill opacity=0.20] (114.50, 55.64) circle (  2.13);

\path[fill=fillColor,fill opacity=0.20] (124.77, 49.82) circle (  2.13);

\path[fill=fillColor,fill opacity=0.20] (101.61, 59.48) circle (  2.13);

\path[fill=fillColor,fill opacity=0.20] ( 92.87, 71.63) circle (  2.13);

\path[fill=fillColor,fill opacity=0.20] ( 83.47, 70.48) circle (  2.13);

\path[fill=fillColor,fill opacity=0.20] ( 77.14, 69.03) circle (  2.13);

\path[fill=fillColor,fill opacity=0.20] ( 78.67, 64.15) circle (  2.13);

\path[fill=fillColor,fill opacity=0.20] ( 72.11, 58.75) circle (  2.13);

\path[fill=fillColor,fill opacity=0.20] ( 84.13, 90.00) circle (  2.13);

\path[fill=fillColor,fill opacity=0.20] ( 83.25, 73.81) circle (  2.13);

\path[fill=fillColor,fill opacity=0.20] ( 92.21, 66.12) circle (  2.13);

\path[fill=fillColor,fill opacity=0.20] (108.16, 54.39) circle (  2.13);

\path[fill=fillColor,fill opacity=0.20] (101.61, 39.55) circle (  2.13);

\path[fill=fillColor,fill opacity=0.20] ( 94.40, 46.40) circle (  2.13);

\path[fill=fillColor,fill opacity=0.20] (101.83, 61.76) circle (  2.13);

\path[fill=fillColor,fill opacity=0.20] ( 95.93, 65.09) circle (  2.13);

\path[fill=fillColor,fill opacity=0.20] ( 95.27, 53.35) circle (  2.13);

\path[fill=fillColor,fill opacity=0.20] ( 88.06, 45.15) circle (  2.13);

\path[fill=fillColor,fill opacity=0.20] ( 68.40, 57.61) circle (  2.13);

\path[fill=fillColor,fill opacity=0.20] ( 99.42, 67.68) circle (  2.13);

\path[fill=fillColor,fill opacity=0.20] (120.84, 60.62) circle (  2.13);

\path[fill=fillColor,fill opacity=0.20] (138.97, 47.44) circle (  2.13);

\path[fill=fillColor,fill opacity=0.20] ( 99.21, 39.34) circle (  2.13);

\path[fill=fillColor,fill opacity=0.20] ( 99.86, 56.16) circle (  2.13);

\path[fill=fillColor,fill opacity=0.20] ( 88.94, 64.98) circle (  2.13);

\path[fill=fillColor,fill opacity=0.20] ( 75.17, 63.01) circle (  2.13);

\path[fill=fillColor,fill opacity=0.20] ( 74.95, 62.18) circle (  2.13);

\path[fill=fillColor,fill opacity=0.20] ( 73.42, 56.99) circle (  2.13);

\path[fill=fillColor,fill opacity=0.20] ( 57.03, 55.12) circle (  2.13);

\path[fill=fillColor,fill opacity=0.20] ( 94.18, 77.03) circle (  2.13);

\path[fill=fillColor,fill opacity=0.20] (105.32, 71.63) circle (  2.13);

\path[fill=fillColor,fill opacity=0.20] (108.60, 64.88) circle (  2.13);

\path[fill=fillColor,fill opacity=0.20] (118.00, 49.72) circle (  2.13);

\path[fill=fillColor,fill opacity=0.20] (101.17, 41.62) circle (  2.13);

\path[fill=fillColor,fill opacity=0.20] (109.26, 48.89) circle (  2.13);

\path[fill=fillColor,fill opacity=0.20] ( 96.80, 55.95) circle (  2.13);

\path[fill=fillColor,fill opacity=0.20] ( 94.18, 53.98) circle (  2.13);

\path[fill=fillColor,fill opacity=0.20] ( 64.25, 61.45) circle (  2.13);

\path[fill=fillColor,fill opacity=0.20] ( 57.47, 60.10) circle (  2.13);

\path[fill=fillColor,fill opacity=0.20] ( 83.69, 81.28) circle (  2.13);

\path[fill=fillColor,fill opacity=0.20] (103.14, 60.52) circle (  2.13);

\path[fill=fillColor,fill opacity=0.20] ( 99.86, 56.99) circle (  2.13);

\path[fill=fillColor,fill opacity=0.20] ( 99.21, 50.45) circle (  2.13);

\path[fill=fillColor,fill opacity=0.20] ( 95.49, 52.52) circle (  2.13);

\path[fill=fillColor,fill opacity=0.20] ( 90.25, 56.68) circle (  2.13);

\path[fill=fillColor,fill opacity=0.20] ( 86.75, 52.63) circle (  2.13);

\path[fill=fillColor,fill opacity=0.20] ( 86.75, 53.56) circle (  2.13);

\path[fill=fillColor,fill opacity=0.20] ( 74.08, 53.35) circle (  2.13);

\path[fill=fillColor,fill opacity=0.20] ( 65.12, 49.10) circle (  2.13);

\path[fill=fillColor,fill opacity=0.20] ( 88.94, 72.56) circle (  2.13);

\path[fill=fillColor,fill opacity=0.20] (102.26, 66.85) circle (  2.13);

\path[fill=fillColor,fill opacity=0.20] (111.44, 56.99) circle (  2.13);

\path[fill=fillColor,fill opacity=0.20] (108.60, 56.26) circle (  2.13);

\path[fill=fillColor,fill opacity=0.20] (115.59, 64.46) circle (  2.13);

\path[fill=fillColor,fill opacity=0.20] (113.41, 66.33) circle (  2.13);

\path[fill=fillColor,fill opacity=0.20] ( 98.77, 56.36) circle (  2.13);

\path[fill=fillColor,fill opacity=0.20] ( 89.15, 43.49) circle (  2.13);

\path[fill=fillColor,fill opacity=0.20] ( 88.28, 41.52) circle (  2.13);

\path[fill=fillColor,fill opacity=0.20] ( 99.86, 64.77) circle (  2.13);

\path[fill=fillColor,fill opacity=0.20] ( 95.93, 60.93) circle (  2.13);

\path[fill=fillColor,fill opacity=0.20] ( 88.28, 51.80) circle (  2.13);

\path[fill=fillColor,fill opacity=0.20] ( 88.94, 54.18) circle (  2.13);

\path[fill=fillColor,fill opacity=0.20] ( 87.62, 52.21) circle (  2.13);

\path[fill=fillColor,fill opacity=0.20] ( 78.45, 43.49) circle (  2.13);

\path[fill=fillColor,fill opacity=0.20] ( 71.02, 56.26) circle (  2.13);

\path[fill=fillColor,fill opacity=0.20] ( 68.62, 71.52) circle (  2.13);

\path[fill=fillColor,fill opacity=0.20] ( 74.30, 76.82) circle (  2.13);

\path[fill=fillColor,fill opacity=0.20] ( 85.22, 60.00) circle (  2.13);

\path[fill=fillColor,fill opacity=0.20] ( 86.31, 77.75) circle (  2.13);

\path[fill=fillColor,fill opacity=0.20] ( 90.47, 66.33) circle (  2.13);

\path[fill=fillColor,fill opacity=0.20] ( 97.46, 63.11) circle (  2.13);

\path[fill=fillColor,fill opacity=0.20] (105.32, 61.45) circle (  2.13);

\path[fill=fillColor,fill opacity=0.20] (107.73, 62.80) circle (  2.13);

\path[fill=fillColor,fill opacity=0.20] (104.67, 75.26) circle (  2.13);

\path[fill=fillColor,fill opacity=0.20] (121.27, 72.56) circle (  2.13);

\path[fill=fillColor,fill opacity=0.20] ( 88.06, 48.27) circle (  2.13);

\path[fill=fillColor,fill opacity=0.20] ( 77.79, 39.34) circle (  2.13);

\path[fill=fillColor,fill opacity=0.20] ( 83.25, 61.14) circle (  2.13);

\path[fill=fillColor,fill opacity=0.20] ( 74.30, 68.82) circle (  2.13);

\path[fill=fillColor,fill opacity=0.20] ( 64.03, 49.93) circle (  2.13);

\path[fill=fillColor,fill opacity=0.20] (106.85, 67.89) circle (  2.13);

\path[fill=fillColor,fill opacity=0.20] ( 97.02, 54.91) circle (  2.13);

\path[fill=fillColor,fill opacity=0.20] ( 81.29, 52.94) circle (  2.13);

\path[fill=fillColor,fill opacity=0.20] ( 79.76, 54.50) circle (  2.13);

\path[fill=fillColor,fill opacity=0.20] ( 80.63, 50.76) circle (  2.13);

\path[fill=fillColor,fill opacity=0.20] ( 76.04, 49.31) circle (  2.13);

\path[fill=fillColor,fill opacity=0.20] ( 85.22, 60.62) circle (  2.13);

\path[fill=fillColor,fill opacity=0.20] ( 70.36, 76.40) circle (  2.13);

\path[fill=fillColor,fill opacity=0.20] ( 99.21, 71.83) circle (  2.13);

\path[fill=fillColor,fill opacity=0.20] ( 81.07, 57.40) circle (  2.13);

\path[fill=fillColor,fill opacity=0.20] ( 85.88, 61.14) circle (  2.13);

\path[fill=fillColor,fill opacity=0.20] ( 90.47, 53.04) circle (  2.13);

\path[fill=fillColor,fill opacity=0.20] ( 91.34, 50.55) circle (  2.13);

\path[fill=fillColor,fill opacity=0.20] ( 92.65, 66.85) circle (  2.13);

\path[fill=fillColor,fill opacity=0.20] ( 91.78, 74.12) circle (  2.13);

\path[fill=fillColor,fill opacity=0.20] ( 98.33, 71.32) circle (  2.13);

\path[fill=fillColor,fill opacity=0.20] ( 92.65, 67.16) circle (  2.13);

\path[fill=fillColor,fill opacity=0.20] ( 83.91, 57.09) circle (  2.13);

\path[fill=fillColor,fill opacity=0.20] ( 83.25, 44.53) circle (  2.13);

\path[fill=fillColor,fill opacity=0.20] ( 74.08, 54.60) circle (  2.13);

\path[fill=fillColor,fill opacity=0.20] ( 53.76, 40.38) circle (  2.13);

\path[fill=fillColor,fill opacity=0.20] ( 90.47, 67.79) circle (  2.13);

\path[fill=fillColor,fill opacity=0.20] (102.92, 56.68) circle (  2.13);

\path[fill=fillColor,fill opacity=0.20] ( 92.21, 68.62) circle (  2.13);

\path[fill=fillColor,fill opacity=0.20] ( 78.23, 60.73) circle (  2.13);

\path[fill=fillColor,fill opacity=0.20] ( 77.14, 54.91) circle (  2.13);

\path[fill=fillColor,fill opacity=0.20] ( 77.36, 64.36) circle (  2.13);

\path[fill=fillColor,fill opacity=0.20] ( 78.01, 67.89) circle (  2.13);

\path[fill=fillColor,fill opacity=0.20] ( 71.89, 58.03) circle (  2.13);

\path[fill=fillColor,fill opacity=0.20] ( 76.04, 66.02) circle (  2.13);

\path[fill=fillColor,fill opacity=0.20] ( 60.75, 80.76) circle (  2.13);

\path[fill=fillColor,fill opacity=0.20] ( 84.13, 59.06) circle (  2.13);

\path[fill=fillColor,fill opacity=0.20] ( 89.37, 65.09) circle (  2.13);

\path[fill=fillColor,fill opacity=0.20] ( 90.68, 66.54) circle (  2.13);

\path[fill=fillColor,fill opacity=0.20] ( 87.41, 47.64) circle (  2.13);

\path[fill=fillColor,fill opacity=0.20] ( 91.12, 38.09) circle (  2.13);

\path[fill=fillColor,fill opacity=0.20] (104.45, 51.28) circle (  2.13);

\path[fill=fillColor,fill opacity=0.20] ( 92.43, 61.35) circle (  2.13);

\path[fill=fillColor,fill opacity=0.20] ( 91.78, 68.41) circle (  2.13);

\path[fill=fillColor,fill opacity=0.20] ( 88.50, 73.50) circle (  2.13);

\path[fill=fillColor,fill opacity=0.20] ( 86.97, 67.16) circle (  2.13);

\path[fill=fillColor,fill opacity=0.20] ( 75.39, 58.96) circle (  2.13);

\path[fill=fillColor,fill opacity=0.20] ( 83.47, 59.38) circle (  2.13);

\path[fill=fillColor,fill opacity=0.20] ( 82.82, 65.61) circle (  2.13);

\path[fill=fillColor,fill opacity=0.20] ( 84.57, 88.96) circle (  2.13);

\path[fill=fillColor,fill opacity=0.20] ( 89.37, 70.48) circle (  2.13);

\path[fill=fillColor,fill opacity=0.20] ( 99.64, 69.03) circle (  2.13);

\path[fill=fillColor,fill opacity=0.20] ( 82.82, 66.75) circle (  2.13);

\path[fill=fillColor,fill opacity=0.20] ( 80.20, 69.65) circle (  2.13);

\path[fill=fillColor,fill opacity=0.20] ( 89.37, 65.50) circle (  2.13);

\path[fill=fillColor,fill opacity=0.20] ( 80.41, 45.67) circle (  2.13);

\path[fill=fillColor,fill opacity=0.20] ( 74.30, 54.39) circle (  2.13);

\path[fill=fillColor,fill opacity=0.20] ( 73.20, 69.86) circle (  2.13);

\path[fill=fillColor,fill opacity=0.20] ( 75.39, 61.97) circle (  2.13);

\path[fill=fillColor,fill opacity=0.20] ( 67.52, 67.37) circle (  2.13);

\path[fill=fillColor,fill opacity=0.20] ( 60.09, 79.00) circle (  2.13);

\path[fill=fillColor,fill opacity=0.20] ( 85.66, 66.64) circle (  2.13);

\path[fill=fillColor,fill opacity=0.20] ( 89.81, 61.14) circle (  2.13);

\path[fill=fillColor,fill opacity=0.20] ( 91.12, 64.46) circle (  2.13);

\path[fill=fillColor,fill opacity=0.20] ( 92.43, 61.04) circle (  2.13);

\path[fill=fillColor,fill opacity=0.20] ( 89.15, 48.99) circle (  2.13);

\path[fill=fillColor,fill opacity=0.20] ( 94.18, 52.00) circle (  2.13);

\path[fill=fillColor,fill opacity=0.20] ( 96.80, 63.42) circle (  2.13);

\path[fill=fillColor,fill opacity=0.20] (102.26, 60.21) circle (  2.13);

\path[fill=fillColor,fill opacity=0.20] ( 86.10, 57.71) circle (  2.13);

\path[fill=fillColor,fill opacity=0.20] ( 91.12, 66.85) circle (  2.13);

\path[fill=fillColor,fill opacity=0.20] ( 85.22, 68.72) circle (  2.13);

\path[fill=fillColor,fill opacity=0.20] ( 81.94, 67.16) circle (  2.13);

\path[fill=fillColor,fill opacity=0.20] ( 81.94, 63.94) circle (  2.13);

\path[fill=fillColor,fill opacity=0.20] ( 75.83, 58.34) circle (  2.13);

\path[fill=fillColor,fill opacity=0.20] ( 79.98, 84.81) circle (  2.13);

\path[fill=fillColor,fill opacity=0.20] ( 91.78, 55.43) circle (  2.13);

\path[fill=fillColor,fill opacity=0.20] (103.79, 63.01) circle (  2.13);

\path[fill=fillColor,fill opacity=0.20] ( 91.34, 83.77) circle (  2.13);

\path[fill=fillColor,fill opacity=0.20] ( 93.09, 60.21) circle (  2.13);

\path[fill=fillColor,fill opacity=0.20] ( 87.19, 69.45) circle (  2.13);

\path[fill=fillColor,fill opacity=0.20] ( 82.16, 64.77) circle (  2.13);

\path[fill=fillColor,fill opacity=0.20] ( 79.10, 67.06) circle (  2.13);

\path[fill=fillColor,fill opacity=0.20] ( 78.23, 65.71) circle (  2.13);

\path[fill=fillColor,fill opacity=0.20] ( 75.39, 61.45) circle (  2.13);

\path[fill=fillColor,fill opacity=0.20] ( 71.02, 59.69) circle (  2.13);

\path[fill=fillColor,fill opacity=0.20] ( 72.11, 65.19) circle (  2.13);

\path[fill=fillColor,fill opacity=0.20] ( 86.97, 63.84) circle (  2.13);

\path[fill=fillColor,fill opacity=0.20] ( 90.47, 54.39) circle (  2.13);

\path[fill=fillColor,fill opacity=0.20] ( 94.18, 50.76) circle (  2.13);

\path[fill=fillColor,fill opacity=0.20] ( 93.74, 56.47) circle (  2.13);

\path[fill=fillColor,fill opacity=0.20] ( 94.62, 64.98) circle (  2.13);

\path[fill=fillColor,fill opacity=0.20] ( 97.24, 65.09) circle (  2.13);

\path[fill=fillColor,fill opacity=0.20] ( 97.68, 63.01) circle (  2.13);

\path[fill=fillColor,fill opacity=0.20] ( 93.31, 58.86) circle (  2.13);

\path[fill=fillColor,fill opacity=0.20] ( 91.56, 53.04) circle (  2.13);

\path[fill=fillColor,fill opacity=0.20] ( 82.38, 55.95) circle (  2.13);

\path[fill=fillColor,fill opacity=0.20] ( 77.79, 64.57) circle (  2.13);

\path[fill=fillColor,fill opacity=0.20] ( 79.10, 57.61) circle (  2.13);

\path[fill=fillColor,fill opacity=0.20] ( 75.17, 55.64) circle (  2.13);

\path[fill=fillColor,fill opacity=0.20] ( 70.36, 61.87) circle (  2.13);

\path[fill=fillColor,fill opacity=0.20] ( 90.47, 66.33) circle (  2.13);

\path[fill=fillColor,fill opacity=0.20] ( 97.46, 63.53) circle (  2.13);

\path[fill=fillColor,fill opacity=0.20] ( 82.38, 69.45) circle (  2.13);

\path[fill=fillColor,fill opacity=0.20] ( 90.03, 70.80) circle (  2.13);

\path[fill=fillColor,fill opacity=0.20] ( 89.59, 59.48) circle (  2.13);

\path[fill=fillColor,fill opacity=0.20] ( 90.03, 49.82) circle (  2.13);

\path[fill=fillColor,fill opacity=0.20] ( 86.31, 66.02) circle (  2.13);

\path[fill=fillColor,fill opacity=0.20] ( 91.99, 73.70) circle (  2.13);

\path[fill=fillColor,fill opacity=0.20] ( 79.76, 74.01) circle (  2.13);

\path[fill=fillColor,fill opacity=0.20] ( 76.92, 63.32) circle (  2.13);

\path[fill=fillColor,fill opacity=0.20] ( 73.64, 59.48) circle (  2.13);

\path[fill=fillColor,fill opacity=0.20] ( 77.57, 68.51) circle (  2.13);

\path[fill=fillColor,fill opacity=0.20] ( 82.60, 56.57) circle (  2.13);

\path[fill=fillColor,fill opacity=0.20] ( 83.69, 52.84) circle (  2.13);

\path[fill=fillColor,fill opacity=0.20] (104.23, 52.94) circle (  2.13);

\path[fill=fillColor,fill opacity=0.20] ( 99.86, 54.29) circle (  2.13);

\path[fill=fillColor,fill opacity=0.20] (100.73, 66.33) circle (  2.13);

\path[fill=fillColor,fill opacity=0.20] (101.61, 77.96) circle (  2.13);

\path[fill=fillColor,fill opacity=0.20] ( 97.24, 72.46) circle (  2.13);

\path[fill=fillColor,fill opacity=0.20] (102.05, 57.61) circle (  2.13);

\path[fill=fillColor,fill opacity=0.20] ( 86.75, 42.35) circle (  2.13);

\path[fill=fillColor,fill opacity=0.20] ( 71.89, 56.05) circle (  2.13);

\path[fill=fillColor,fill opacity=0.20] ( 69.93, 68.93) circle (  2.13);

\path[fill=fillColor,fill opacity=0.20] ( 66.21, 51.80) circle (  2.13);

\path[fill=fillColor,fill opacity=0.20] ( 55.72, 51.38) circle (  2.13);

\path[fill=fillColor,fill opacity=0.20] ( 64.03, 66.02) circle (  2.13);

\path[fill=fillColor,fill opacity=0.20] ( 81.29, 68.10) circle (  2.13);

\path[fill=fillColor,fill opacity=0.20] ( 95.49, 75.57) circle (  2.13);

\path[fill=fillColor,fill opacity=0.20] ( 95.49, 65.71) circle (  2.13);

\path[fill=fillColor,fill opacity=0.20] ( 94.40, 62.39) circle (  2.13);

\path[fill=fillColor,fill opacity=0.20] ( 87.84, 76.82) circle (  2.13);

\path[fill=fillColor,fill opacity=0.20] ( 92.21, 75.47) circle (  2.13);

\path[fill=fillColor,fill opacity=0.20] ( 85.66, 66.02) circle (  2.13);

\path[fill=fillColor,fill opacity=0.20] ( 82.60, 56.05) circle (  2.13);

\path[fill=fillColor,fill opacity=0.20] ( 76.04, 59.17) circle (  2.13);

\path[fill=fillColor,fill opacity=0.20] ( 74.51, 75.99) circle (  2.13);

\path[fill=fillColor,fill opacity=0.20] ( 71.02, 81.39) circle (  2.13);

\path[fill=fillColor,fill opacity=0.20] ( 69.27, 71.52) circle (  2.13);

\path[fill=fillColor,fill opacity=0.20] ( 69.27, 86.89) circle (  2.13);

\path[fill=fillColor,fill opacity=0.20] ( 83.25, 66.54) circle (  2.13);

\path[fill=fillColor,fill opacity=0.20] ( 85.00, 66.85) circle (  2.13);

\path[fill=fillColor,fill opacity=0.20] ( 89.59, 74.43) circle (  2.13);

\path[fill=fillColor,fill opacity=0.20] ( 92.87, 73.50) circle (  2.13);

\path[fill=fillColor,fill opacity=0.20] (100.30, 70.07) circle (  2.13);

\path[fill=fillColor,fill opacity=0.20] ( 97.02, 64.67) circle (  2.13);

\path[fill=fillColor,fill opacity=0.20] ( 97.24, 67.27) circle (  2.13);

\path[fill=fillColor,fill opacity=0.20] ( 90.68, 73.60) circle (  2.13);

\path[fill=fillColor,fill opacity=0.20] ( 83.47, 60.62) circle (  2.13);

\path[fill=fillColor,fill opacity=0.20] ( 69.05, 48.27) circle (  2.13);

\path[fill=fillColor,fill opacity=0.20] ( 68.40, 52.84) circle (  2.13);

\path[fill=fillColor,fill opacity=0.20] ( 65.77, 60.62) circle (  2.13);

\path[fill=fillColor,fill opacity=0.20] ( 67.09, 69.97) circle (  2.13);

\path[fill=fillColor,fill opacity=0.20] ( 53.76, 56.68) circle (  2.13);

\path[fill=fillColor,fill opacity=0.20] ( 73.20, 69.97) circle (  2.13);

\path[fill=fillColor,fill opacity=0.20] (106.20, 63.01) circle (  2.13);

\path[fill=fillColor,fill opacity=0.20] ( 99.42, 77.13) circle (  2.13);

\path[fill=fillColor,fill opacity=0.20] ( 87.62, 81.70) circle (  2.13);

\path[fill=fillColor,fill opacity=0.20] ( 95.49, 63.11) circle (  2.13);

\path[fill=fillColor,fill opacity=0.20] ( 99.64, 41.52) circle (  2.13);

\path[fill=fillColor,fill opacity=0.20] ( 90.47, 41.73) circle (  2.13);

\path[fill=fillColor,fill opacity=0.20] ( 81.73, 51.69) circle (  2.13);

\path[fill=fillColor,fill opacity=0.20] ( 87.84, 59.89) circle (  2.13);

\path[fill=fillColor,fill opacity=0.20] ( 76.48, 64.15) circle (  2.13);

\path[fill=fillColor,fill opacity=0.20] ( 75.39, 69.55) circle (  2.13);

\path[fill=fillColor,fill opacity=0.20] ( 72.77, 72.77) circle (  2.13);

\path[fill=fillColor,fill opacity=0.20] ( 70.80, 71.63) circle (  2.13);

\path[fill=fillColor,fill opacity=0.20] ( 78.23, 94.16) circle (  2.13);

\path[fill=fillColor,fill opacity=0.20] ( 88.72, 77.13) circle (  2.13);

\path[fill=fillColor,fill opacity=0.20] (110.35, 67.06) circle (  2.13);

\path[fill=fillColor,fill opacity=0.20] (114.72, 71.21) circle (  2.13);

\path[fill=fillColor,fill opacity=0.20] (111.88, 67.99) circle (  2.13);

\path[fill=fillColor,fill opacity=0.20] ( 89.15, 69.86) circle (  2.13);

\path[fill=fillColor,fill opacity=0.20] ( 81.94, 74.85) circle (  2.13);

\path[fill=fillColor,fill opacity=0.20] ( 78.67, 60.52) circle (  2.13);

\path[fill=fillColor,fill opacity=0.20] ( 68.18, 54.91) circle (  2.13);

\path[fill=fillColor,fill opacity=0.20] ( 68.62, 61.97) circle (  2.13);

\path[fill=fillColor,fill opacity=0.20] ( 61.62, 56.57) circle (  2.13);

\path[fill=fillColor,fill opacity=0.20] ( 59.22, 60.41) circle (  2.13);

\path[fill=fillColor,fill opacity=0.20] ( 53.10, 81.18) circle (  2.13);

\path[fill=fillColor,fill opacity=0.20] ( 55.29, 87.93) circle (  2.13);

\path[fill=fillColor,fill opacity=0.20] (142.25, 74.12) circle (  2.13);

\path[fill=fillColor,fill opacity=0.20] (145.09, 63.42) circle (  2.13);

\path[fill=fillColor,fill opacity=0.20] (107.29, 66.75) circle (  2.13);

\path[fill=fillColor,fill opacity=0.20] (100.08, 56.36) circle (  2.13);

\path[fill=fillColor,fill opacity=0.20] (102.05, 49.62) circle (  2.13);

\path[fill=fillColor,fill opacity=0.20] ( 92.87, 52.32) circle (  2.13);

\path[fill=fillColor,fill opacity=0.20] ( 85.00, 49.62) circle (  2.13);

\path[fill=fillColor,fill opacity=0.20] ( 83.25, 45.36) circle (  2.13);

\path[fill=fillColor,fill opacity=0.20] ( 79.54, 53.56) circle (  2.13);

\path[fill=fillColor,fill opacity=0.20] ( 80.20, 69.03) circle (  2.13);

\path[fill=fillColor,fill opacity=0.20] ( 72.11, 77.13) circle (  2.13);

\path[fill=fillColor,fill opacity=0.20] ( 69.49, 62.49) circle (  2.13);

\path[fill=fillColor,fill opacity=0.20] ( 67.74, 55.64) circle (  2.13);

\path[fill=fillColor,fill opacity=0.20] ( 65.99, 67.58) circle (  2.13);

\path[fill=fillColor,fill opacity=0.20] ( 91.56, 81.39) circle (  2.13);

\path[fill=fillColor,fill opacity=0.20] ( 84.13, 86.89) circle (  2.13);

\path[fill=fillColor,fill opacity=0.20] ( 86.75, 87.93) circle (  2.13);

\path[fill=fillColor,fill opacity=0.20] ( 97.02, 77.86) circle (  2.13);

\path[fill=fillColor,fill opacity=0.20] ( 95.71, 68.20) circle (  2.13);

\path[fill=fillColor,fill opacity=0.20] ( 87.62, 57.82) circle (  2.13);

\path[fill=fillColor,fill opacity=0.20] ( 88.50, 39.65) circle (  2.13);

\path[fill=fillColor,fill opacity=0.20] ( 59.66, 40.07) circle (  2.13);

\path[fill=fillColor,fill opacity=0.20] ( 62.50, 69.34) circle (  2.13);

\path[fill=fillColor,fill opacity=0.20] ( 47.42, 86.89) circle (  2.13);

\path[fill=fillColor,fill opacity=0.20] ( 54.41, 82.74) circle (  2.13);

\path[fill=fillColor,fill opacity=0.20] (103.79, 58.75) circle (  2.13);

\path[fill=fillColor,fill opacity=0.20] (101.61, 71.94) circle (  2.13);

\path[fill=fillColor,fill opacity=0.20] ( 93.74, 67.58) circle (  2.13);

\path[fill=fillColor,fill opacity=0.20] ( 83.69, 58.55) circle (  2.13);

\path[fill=fillColor,fill opacity=0.20] ( 85.66, 52.84) circle (  2.13);

\path[fill=fillColor,fill opacity=0.20] ( 86.10, 58.03) circle (  2.13);

\path[fill=fillColor,fill opacity=0.20] ( 76.48, 66.75) circle (  2.13);

\path[fill=fillColor,fill opacity=0.20] ( 76.26, 70.90) circle (  2.13);

\path[fill=fillColor,fill opacity=0.20] ( 70.36, 68.30) circle (  2.13);

\path[fill=fillColor,fill opacity=0.20] ( 70.80, 55.85) circle (  2.13);

\path[fill=fillColor,fill opacity=0.20] ( 79.98, 66.44) circle (  2.13);

\path[fill=fillColor,fill opacity=0.20] ( 82.16, 74.12) circle (  2.13);

\path[fill=fillColor,fill opacity=0.20] ( 86.97, 78.27) circle (  2.13);

\path[fill=fillColor,fill opacity=0.20] ( 83.25, 68.30) circle (  2.13);

\path[fill=fillColor,fill opacity=0.20] ( 83.47, 69.76) circle (  2.13);

\path[fill=fillColor,fill opacity=0.20] ( 74.51, 69.13) circle (  2.13);

\path[fill=fillColor,fill opacity=0.20] ( 71.89, 56.68) circle (  2.13);

\path[fill=fillColor,fill opacity=0.20] ( 67.74, 53.35) circle (  2.13);

\path[fill=fillColor,fill opacity=0.20] ( 58.13, 52.21) circle (  2.13);

\path[fill=fillColor,fill opacity=0.20] ( 51.79, 63.74) circle (  2.13);

\path[fill=fillColor,fill opacity=0.20] ( 66.65,100.39) circle (  2.13);

\path[fill=fillColor,fill opacity=0.20] ( 70.58, 58.34) circle (  2.13);

\path[fill=fillColor,fill opacity=0.20] ( 97.24, 60.00) circle (  2.13);

\path[fill=fillColor,fill opacity=0.20] (100.30, 62.91) circle (  2.13);

\path[fill=fillColor,fill opacity=0.20] ( 91.34, 66.44) circle (  2.13);

\path[fill=fillColor,fill opacity=0.20] ( 91.56, 60.93) circle (  2.13);

\path[fill=fillColor,fill opacity=0.20] ( 87.84, 54.29) circle (  2.13);

\path[fill=fillColor,fill opacity=0.20] ( 79.54, 60.31) circle (  2.13);

\path[fill=fillColor,fill opacity=0.20] ( 84.13, 68.20) circle (  2.13);

\path[fill=fillColor,fill opacity=0.20] ( 83.69, 71.21) circle (  2.13);

\path[fill=fillColor,fill opacity=0.20] ( 78.45, 75.47) circle (  2.13);

\path[fill=fillColor,fill opacity=0.20] ( 77.36, 74.12) circle (  2.13);

\path[fill=fillColor,fill opacity=0.20] ( 67.74, 54.60) circle (  2.13);

\path[fill=fillColor,fill opacity=0.20] ( 68.83, 41.93) circle (  2.13);

\path[fill=fillColor,fill opacity=0.20] ( 69.71, 54.91) circle (  2.13);

\path[fill=fillColor,fill opacity=0.20] ( 70.14, 75.88) circle (  2.13);

\path[fill=fillColor,fill opacity=0.20] ( 75.39, 91.04) circle (  2.13);

\path[fill=fillColor,fill opacity=0.20] ( 81.07, 74.12) circle (  2.13);

\path[fill=fillColor,fill opacity=0.20] ( 77.57, 77.44) circle (  2.13);

\path[fill=fillColor,fill opacity=0.20] ( 87.41, 77.65) circle (  2.13);

\path[fill=fillColor,fill opacity=0.20] ( 89.37, 54.91) circle (  2.13);

\path[fill=fillColor,fill opacity=0.20] ( 94.40, 55.64) circle (  2.13);

\path[fill=fillColor,fill opacity=0.20] ( 93.31, 64.05) circle (  2.13);

\path[fill=fillColor,fill opacity=0.20] ( 82.82, 65.92) circle (  2.13);

\path[fill=fillColor,fill opacity=0.20] ( 64.90, 64.36) circle (  2.13);

\path[fill=fillColor,fill opacity=0.20] ( 63.81, 55.43) circle (  2.13);

\path[fill=fillColor,fill opacity=0.20] ( 70.14, 61.97) circle (  2.13);

\path[fill=fillColor,fill opacity=0.20] ( 94.18, 59.06) circle (  2.13);

\path[fill=fillColor,fill opacity=0.20] ( 91.12, 67.68) circle (  2.13);

\path[fill=fillColor,fill opacity=0.20] ( 89.37, 62.39) circle (  2.13);

\path[fill=fillColor,fill opacity=0.20] ( 91.34, 49.72) circle (  2.13);

\path[fill=fillColor,fill opacity=0.20] ( 90.25, 51.59) circle (  2.13);

\path[fill=fillColor,fill opacity=0.20] ( 94.62, 64.05) circle (  2.13);

\path[fill=fillColor,fill opacity=0.20] ( 95.93, 68.72) circle (  2.13);

\path[fill=fillColor,fill opacity=0.20] ( 88.72, 71.32) circle (  2.13);

\path[fill=fillColor,fill opacity=0.20] ( 81.51, 65.09) circle (  2.13);

\path[fill=fillColor,fill opacity=0.20] ( 75.83, 63.22) circle (  2.13);

\path[fill=fillColor,fill opacity=0.20] ( 74.73, 64.77) circle (  2.13);

\path[fill=fillColor,fill opacity=0.20] ( 74.51, 59.06) circle (  2.13);

\path[fill=fillColor,fill opacity=0.20] ( 81.29, 54.29) circle (  2.13);

\path[fill=fillColor,fill opacity=0.20] ( 70.80, 55.64) circle (  2.13);

\path[fill=fillColor,fill opacity=0.20] ( 71.02, 55.85) circle (  2.13);

\path[fill=fillColor,fill opacity=0.20] ( 71.24, 60.21) circle (  2.13);

\path[fill=fillColor,fill opacity=0.20] ( 72.55, 70.28) circle (  2.13);

\path[fill=fillColor,fill opacity=0.20] ( 75.17, 77.03) circle (  2.13);

\path[fill=fillColor,fill opacity=0.20] ( 78.88, 82.74) circle (  2.13);

\path[fill=fillColor,fill opacity=0.20] ( 72.77, 79.72) circle (  2.13);

\path[fill=fillColor,fill opacity=0.20] ( 72.99, 65.50) circle (  2.13);

\path[fill=fillColor,fill opacity=0.20] ( 77.14, 59.06) circle (  2.13);

\path[fill=fillColor,fill opacity=0.20] ( 79.98, 68.72) circle (  2.13);

\path[fill=fillColor,fill opacity=0.20] ( 77.36, 72.25) circle (  2.13);

\path[fill=fillColor,fill opacity=0.20] ( 79.10, 68.62) circle (  2.13);

\path[fill=fillColor,fill opacity=0.20] ( 80.20, 70.80) circle (  2.13);

\path[fill=fillColor,fill opacity=0.20] ( 78.01, 72.04) circle (  2.13);

\path[fill=fillColor,fill opacity=0.20] ( 78.23, 66.02) circle (  2.13);

\path[fill=fillColor,fill opacity=0.20] ( 83.25, 56.99) circle (  2.13);

\path[fill=fillColor,fill opacity=0.20] ( 92.65, 44.53) circle (  2.13);

\path[fill=fillColor,fill opacity=0.20] ( 90.25, 40.58) circle (  2.13);

\path[fill=fillColor,fill opacity=0.20] ( 93.52, 60.52) circle (  2.13);

\path[fill=fillColor,fill opacity=0.20] ( 91.34, 75.68) circle (  2.13);

\path[fill=fillColor,fill opacity=0.20] ( 93.52, 59.17) circle (  2.13);

\path[fill=fillColor,fill opacity=0.20] ( 90.68, 38.92) circle (  2.13);

\path[fill=fillColor,fill opacity=0.20] ( 82.82, 45.78) circle (  2.13);

\path[fill=fillColor,fill opacity=0.20] ( 61.19, 54.91) circle (  2.13);

\path[fill=fillColor,fill opacity=0.20] ( 46.55, 63.53) circle (  2.13);

\path[fill=fillColor,fill opacity=0.20] ( 67.09, 91.04) circle (  2.13);

\path[fill=fillColor,fill opacity=0.20] ( 61.19, 68.93) circle (  2.13);

\path[fill=fillColor,fill opacity=0.20] ( 72.99, 73.29) circle (  2.13);

\path[fill=fillColor,fill opacity=0.20] ( 71.24, 70.38) circle (  2.13);

\path[fill=fillColor,fill opacity=0.20] ( 82.82, 59.27) circle (  2.13);

\path[fill=fillColor,fill opacity=0.20] ( 98.11, 57.61) circle (  2.13);

\path[fill=fillColor,fill opacity=0.20] (109.69, 59.89) circle (  2.13);

\path[fill=fillColor,fill opacity=0.20] ( 93.74, 54.50) circle (  2.13);

\path[fill=fillColor,fill opacity=0.20] ( 87.62, 57.30) circle (  2.13);

\path[fill=fillColor,fill opacity=0.20] ( 90.25, 74.43) circle (  2.13);

\path[fill=fillColor,fill opacity=0.20] ( 85.66, 81.49) circle (  2.13);

\path[fill=fillColor,fill opacity=0.20] ( 80.85, 72.04) circle (  2.13);

\path[fill=fillColor,fill opacity=0.20] ( 78.67, 64.77) circle (  2.13);

\path[fill=fillColor,fill opacity=0.20] ( 76.26, 56.16) circle (  2.13);

\path[fill=fillColor,fill opacity=0.20] ( 76.48, 71.94) circle (  2.13);

\path[fill=fillColor,fill opacity=0.20] ( 79.10, 66.85) circle (  2.13);

\path[fill=fillColor,fill opacity=0.20] ( 73.42, 53.35) circle (  2.13);

\path[fill=fillColor,fill opacity=0.20] ( 69.93, 50.55) circle (  2.13);

\path[fill=fillColor,fill opacity=0.20] ( 71.89, 63.22) circle (  2.13);

\path[fill=fillColor,fill opacity=0.20] ( 77.79, 73.60) circle (  2.13);

\path[fill=fillColor,fill opacity=0.20] ( 74.73, 65.40) circle (  2.13);

\path[fill=fillColor,fill opacity=0.20] ( 75.17, 58.55) circle (  2.13);

\path[fill=fillColor,fill opacity=0.20] ( 80.41, 67.99) circle (  2.13);

\path[fill=fillColor,fill opacity=0.20] ( 78.45, 66.02) circle (  2.13);

\path[fill=fillColor,fill opacity=0.20] ( 80.63, 51.17) circle (  2.13);

\path[fill=fillColor,fill opacity=0.20] ( 80.41, 52.00) circle (  2.13);

\path[fill=fillColor,fill opacity=0.20] ( 89.81, 62.39) circle (  2.13);

\path[fill=fillColor,fill opacity=0.20] ( 95.71, 60.00) circle (  2.13);

\path[fill=fillColor,fill opacity=0.20] ( 90.90, 51.49) circle (  2.13);

\path[fill=fillColor,fill opacity=0.20] ( 78.01, 58.96) circle (  2.13);

\path[fill=fillColor,fill opacity=0.20] ( 68.40, 67.06) circle (  2.13);

\path[fill=fillColor,fill opacity=0.20] ( 66.43, 53.87) circle (  2.13);

\path[fill=fillColor,fill opacity=0.20] ( 67.96, 48.06) circle (  2.13);

\path[fill=fillColor,fill opacity=0.20] ( 69.71, 64.15) circle (  2.13);

\path[fill=fillColor,fill opacity=0.20] ( 78.88, 73.08) circle (  2.13);

\path[fill=fillColor,fill opacity=0.20] ( 55.94, 85.85) circle (  2.13);

\path[fill=fillColor,fill opacity=0.20] ( 74.51, 60.93) circle (  2.13);

\path[fill=fillColor,fill opacity=0.20] ( 90.47, 50.34) circle (  2.13);

\path[fill=fillColor,fill opacity=0.20] ( 79.76, 39.65) circle (  2.13);

\path[fill=fillColor,fill opacity=0.20] ( 82.38, 48.27) circle (  2.13);

\path[fill=fillColor,fill opacity=0.20] ( 88.28, 72.04) circle (  2.13);

\path[fill=fillColor,fill opacity=0.20] ( 88.72, 73.91) circle (  2.13);

\path[fill=fillColor,fill opacity=0.20] ( 92.87, 64.46) circle (  2.13);

\path[fill=fillColor,fill opacity=0.20] ( 91.56, 68.51) circle (  2.13);

\path[fill=fillColor,fill opacity=0.20] ( 94.40, 64.46) circle (  2.13);

\path[fill=fillColor,fill opacity=0.20] ( 96.80, 65.09) circle (  2.13);

\path[fill=fillColor,fill opacity=0.20] ( 90.90, 79.00) circle (  2.13);

\path[fill=fillColor,fill opacity=0.20] ( 88.06, 67.68) circle (  2.13);

\path[fill=fillColor,fill opacity=0.20] ( 79.32, 50.97) circle (  2.13);

\path[fill=fillColor,fill opacity=0.20] ( 79.32, 61.76) circle (  2.13);

\path[fill=fillColor,fill opacity=0.20] ( 83.69, 72.87) circle (  2.13);

\path[fill=fillColor,fill opacity=0.20] ( 85.44, 68.30) circle (  2.13);

\path[fill=fillColor,fill opacity=0.20] (103.14, 80.45) circle (  2.13);

\path[fill=fillColor,fill opacity=0.20] ( 86.75, 72.98) circle (  2.13);

\path[fill=fillColor,fill opacity=0.20] ( 86.31, 48.06) circle (  2.13);

\path[fill=fillColor,fill opacity=0.20] ( 85.00, 45.88) circle (  2.13);

\path[fill=fillColor,fill opacity=0.20] ( 90.03, 59.69) circle (  2.13);

\path[fill=fillColor,fill opacity=0.20] ( 91.99, 63.84) circle (  2.13);

\path[fill=fillColor,fill opacity=0.20] ( 78.45, 58.65) circle (  2.13);

\path[fill=fillColor,fill opacity=0.20] ( 60.53, 58.44) circle (  2.13);

\path[fill=fillColor,fill opacity=0.20] ( 52.01, 69.03) circle (  2.13);

\path[fill=fillColor,fill opacity=0.20] ( 48.51, 81.28) circle (  2.13);

\path[fill=fillColor,fill opacity=0.20] ( 45.02, 92.08) circle (  2.13);

\path[fill=fillColor,fill opacity=0.20] ( 45.24,103.50) circle (  2.13);

\path[fill=fillColor,fill opacity=0.20] ( 55.29, 71.42) circle (  2.13);

\path[fill=fillColor,fill opacity=0.20] ( 59.44, 53.25) circle (  2.13);

\path[fill=fillColor,fill opacity=0.20] ( 61.19, 41.83) circle (  2.13);

\path[fill=fillColor,fill opacity=0.20] ( 69.49, 48.68) circle (  2.13);

\path[fill=fillColor,fill opacity=0.20] ( 74.73, 52.11) circle (  2.13);

\path[fill=fillColor,fill opacity=0.20] ( 71.02, 49.31) circle (  2.13);

\path[fill=fillColor,fill opacity=0.20] ( 80.41, 56.26) circle (  2.13);

\path[fill=fillColor,fill opacity=0.20] ( 91.99, 51.69) circle (  2.13);

\path[fill=fillColor,fill opacity=0.20] ( 84.35, 42.04) circle (  2.13);

\path[fill=fillColor,fill opacity=0.20] ( 83.69, 52.21) circle (  2.13);

\path[fill=fillColor,fill opacity=0.20] (118.87, 56.05) circle (  2.13);

\path[fill=fillColor,fill opacity=0.20] ( 99.21, 44.63) circle (  2.13);

\path[fill=fillColor,fill opacity=0.20] ( 91.34, 49.93) circle (  2.13);

\path[fill=fillColor,fill opacity=0.20] ( 90.25, 68.30) circle (  2.13);

\path[fill=fillColor,fill opacity=0.20] ( 95.93, 75.88) circle (  2.13);

\path[fill=fillColor,fill opacity=0.20] ( 88.50, 68.82) circle (  2.13);

\path[fill=fillColor,fill opacity=0.20] (101.17, 64.88) circle (  2.13);

\path[fill=fillColor,fill opacity=0.20] (105.98, 66.64) circle (  2.13);

\path[fill=fillColor,fill opacity=0.20] ( 96.36, 67.47) circle (  2.13);

\path[fill=fillColor,fill opacity=0.20] ( 85.22, 57.92) circle (  2.13);

\path[fill=fillColor,fill opacity=0.20] ( 80.20, 47.33) circle (  2.13);

\path[fill=fillColor,fill opacity=0.20] ( 75.17, 45.78) circle (  2.13);

\path[fill=fillColor,fill opacity=0.20] ( 73.64, 49.72) circle (  2.13);

\path[fill=fillColor,fill opacity=0.20] ( 69.05, 55.53) circle (  2.13);

\path[fill=fillColor,fill opacity=0.20] ( 62.50, 58.03) circle (  2.13);

\path[fill=fillColor,fill opacity=0.20] ( 52.23, 69.34) circle (  2.13);

\path[fill=fillColor,fill opacity=0.20] ( 50.48, 68.41) circle (  2.13);

\path[fill=fillColor,fill opacity=0.20] ( 50.92, 61.87) circle (  2.13);

\path[fill=fillColor,fill opacity=0.20] ( 57.25, 58.75) circle (  2.13);

\path[fill=fillColor,fill opacity=0.20] ( 54.85, 53.77) circle (  2.13);

\path[fill=fillColor,fill opacity=0.20] ( 49.17, 46.92) circle (  2.13);

\path[fill=fillColor,fill opacity=0.20] ( 51.79, 49.72) circle (  2.13);

\path[fill=fillColor,fill opacity=0.20] (122.15, 46.50) circle (  2.13);

\path[fill=fillColor,fill opacity=0.20] ( 78.45, 57.71) circle (  2.13);

\path[fill=fillColor,fill opacity=0.20] ( 74.95, 54.29) circle (  2.13);

\path[fill=fillColor,fill opacity=0.20] ( 71.67, 46.71) circle (  2.13);

\path[fill=fillColor,fill opacity=0.20] ( 78.01, 56.05) circle (  2.13);

\path[fill=fillColor,fill opacity=0.20] ( 74.30, 54.50) circle (  2.13);

\path[fill=fillColor,fill opacity=0.20] ( 65.12, 54.29) circle (  2.13);

\path[fill=fillColor,fill opacity=0.20] (105.10, 67.27) circle (  2.13);

\path[fill=fillColor,fill opacity=0.20] ( 66.43, 66.64) circle (  2.13);

\path[fill=fillColor,fill opacity=0.20] ( 58.78, 69.13) circle (  2.13);

\path[fill=fillColor,fill opacity=0.20] ( 59.22, 48.27) circle (  2.13);

\path[fill=fillColor,fill opacity=0.20] ( 56.82, 62.59) circle (  2.13);

\path[fill=fillColor,fill opacity=0.20] ( 54.85, 64.05) circle (  2.13);

\path[fill=fillColor,fill opacity=0.20] ( 54.85, 65.92) circle (  2.13);

\path[fill=fillColor,fill opacity=0.20] ( 56.38, 76.92) circle (  2.13);

\path[fill=fillColor,fill opacity=0.20] ( 59.22, 72.25) circle (  2.13);

\path[fill=fillColor,fill opacity=0.20] ( 46.98, 60.73) circle (  2.13);

\path[fill=fillColor,fill opacity=0.20] ( 48.08, 64.46) circle (  2.13);

\path[fill=fillColor,fill opacity=0.20] ( 50.70, 77.44) circle (  2.13);

\path[fill=fillColor,fill opacity=0.20] ( 54.41, 84.81) circle (  2.13);

\path[fill=fillColor,fill opacity=0.20] ( 81.51, 81.70) circle (  2.13);

\path[fill=fillColor,fill opacity=0.20] ( 53.98, 84.81) circle (  2.13);

\path[fill=fillColor,fill opacity=0.20] ( 85.66, 88.96) circle (  2.13);

\path[fill=fillColor,fill opacity=0.20] (133.07, 91.04) circle (  2.13);

\path[fill=fillColor,fill opacity=0.20] ( 95.93, 83.77) circle (  2.13);

\path[fill=fillColor,fill opacity=0.20] ( 81.51, 76.51) circle (  2.13);

\path[fill=fillColor,fill opacity=0.20] ( 78.23, 77.34) circle (  2.13);

\path[fill=fillColor,fill opacity=0.20] ( 73.42,105.58) circle (  2.13);

\path[fill=fillColor,fill opacity=0.20] ( 93.52, 91.04) circle (  2.13);

\path[fill=fillColor,fill opacity=0.20] (101.17, 84.81) circle (  2.13);

\path[fill=fillColor,fill opacity=0.20] ( 97.24, 74.85) circle (  2.13);

\path[fill=fillColor,fill opacity=0.20] ( 98.11, 65.29) circle (  2.13);

\path[fill=fillColor,fill opacity=0.20] ( 94.84, 67.89) circle (  2.13);

\path[fill=fillColor,fill opacity=0.20] ( 81.73, 66.44) circle (  2.13);

\path[fill=fillColor,fill opacity=0.20] ( 79.76, 61.56) circle (  2.13);

\path[fill=fillColor,fill opacity=0.20] ( 85.66, 66.95) circle (  2.13);

\path[fill=fillColor,fill opacity=0.20] ( 70.14, 65.40) circle (  2.13);

\path[fill=fillColor,fill opacity=0.20] ( 69.49, 54.39) circle (  2.13);

\path[fill=fillColor,fill opacity=0.20] ( 72.77, 98.31) circle (  2.13);

\path[fill=fillColor,fill opacity=0.20] ( 86.31, 76.61) circle (  2.13);

\path[fill=fillColor,fill opacity=0.20] ( 92.87, 73.81) circle (  2.13);

\path[fill=fillColor,fill opacity=0.20] ( 99.64, 75.26) circle (  2.13);

\path[fill=fillColor,fill opacity=0.20] ( 95.93, 66.23) circle (  2.13);

\path[fill=fillColor,fill opacity=0.20] ( 92.21, 60.62) circle (  2.13);

\path[fill=fillColor,fill opacity=0.20] ( 89.15, 58.75) circle (  2.13);

\path[fill=fillColor,fill opacity=0.20] ( 80.20, 59.27) circle (  2.13);

\path[fill=fillColor,fill opacity=0.20] ( 81.29, 56.68) circle (  2.13);

\path[fill=fillColor,fill opacity=0.20] ( 77.14, 50.34) circle (  2.13);

\path[fill=fillColor,fill opacity=0.20] ( 79.32, 43.70) circle (  2.13);

\path[fill=fillColor,fill opacity=0.20] ( 78.23, 51.49) circle (  2.13);

\path[fill=fillColor,fill opacity=0.20] ( 66.43, 71.63) circle (  2.13);

\path[fill=fillColor,fill opacity=0.20] ( 60.31,110.77) circle (  2.13);

\path[fill=fillColor,fill opacity=0.20] ( 84.35, 75.05) circle (  2.13);

\path[fill=fillColor,fill opacity=0.20] ( 95.05, 66.02) circle (  2.13);

\path[fill=fillColor,fill opacity=0.20] ( 96.58, 65.40) circle (  2.13);

\path[fill=fillColor,fill opacity=0.20] ( 96.15, 70.48) circle (  2.13);

\path[fill=fillColor,fill opacity=0.20] ( 91.12, 67.06) circle (  2.13);

\path[fill=fillColor,fill opacity=0.20] ( 87.62, 65.81) circle (  2.13);

\path[fill=fillColor,fill opacity=0.20] ( 85.00, 61.24) circle (  2.13);

\path[fill=fillColor,fill opacity=0.20] ( 82.38, 58.34) circle (  2.13);

\path[fill=fillColor,fill opacity=0.20] ( 77.57, 60.00) circle (  2.13);

\path[fill=fillColor,fill opacity=0.20] ( 73.64, 64.88) circle (  2.13);

\path[fill=fillColor,fill opacity=0.20] ( 74.51, 73.08) circle (  2.13);

\path[fill=fillColor,fill opacity=0.20] ( 62.93, 73.70) circle (  2.13);

\path[fill=fillColor,fill opacity=0.20] ( 62.28, 66.85) circle (  2.13);

\path[fill=fillColor,fill opacity=0.20] ( 71.02, 70.59) circle (  2.13);

\path[fill=fillColor,fill opacity=0.20] ( 95.93, 73.39) circle (  2.13);

\path[fill=fillColor,fill opacity=0.20] ( 94.84, 75.99) circle (  2.13);

\path[fill=fillColor,fill opacity=0.20] ( 90.90, 68.93) circle (  2.13);

\path[fill=fillColor,fill opacity=0.20] ( 87.19, 71.83) circle (  2.13);

\path[fill=fillColor,fill opacity=0.20] ( 85.44, 76.19) circle (  2.13);

\path[fill=fillColor,fill opacity=0.20] ( 81.73, 70.90) circle (  2.13);

\path[fill=fillColor,fill opacity=0.20] ( 76.26, 67.27) circle (  2.13);

\path[fill=fillColor,fill opacity=0.20] ( 78.67, 61.97) circle (  2.13);

\path[fill=fillColor,fill opacity=0.20] ( 76.70, 54.39) circle (  2.13);

\path[fill=fillColor,fill opacity=0.20] ( 64.90, 60.10) circle (  2.13);

\path[fill=fillColor,fill opacity=0.20] ( 60.53, 76.71) circle (  2.13);

\path[fill=fillColor,fill opacity=0.20] ( 61.84, 79.10) circle (  2.13);

\path[fill=fillColor,fill opacity=0.20] ( 74.08, 84.81) circle (  2.13);

\path[fill=fillColor,fill opacity=0.20] ( 96.36, 66.02) circle (  2.13);

\path[fill=fillColor,fill opacity=0.20] ( 92.43, 74.64) circle (  2.13);

\path[fill=fillColor,fill opacity=0.20] ( 95.49, 63.01) circle (  2.13);

\path[fill=fillColor,fill opacity=0.20] ( 83.69, 67.58) circle (  2.13);

\path[fill=fillColor,fill opacity=0.20] ( 76.70, 81.70) circle (  2.13);

\path[fill=fillColor,fill opacity=0.20] ( 79.32, 76.61) circle (  2.13);

\path[fill=fillColor,fill opacity=0.20] ( 76.92, 67.27) circle (  2.13);

\path[fill=fillColor,fill opacity=0.20] ( 73.42, 64.57) circle (  2.13);

\path[fill=fillColor,fill opacity=0.20] ( 67.74, 53.04) circle (  2.13);

\path[fill=fillColor,fill opacity=0.20] ( 53.98, 50.97) circle (  2.13);

\path[fill=fillColor,fill opacity=0.20] ( 68.62, 88.96) circle (  2.13);

\path[fill=fillColor,fill opacity=0.20] ( 80.20, 74.22) circle (  2.13);

\path[fill=fillColor,fill opacity=0.20] ( 91.78, 57.20) circle (  2.13);

\path[fill=fillColor,fill opacity=0.20] ( 91.12, 62.80) circle (  2.13);

\path[fill=fillColor,fill opacity=0.20] (104.01, 58.96) circle (  2.13);

\path[fill=fillColor,fill opacity=0.20] ( 84.13, 64.15) circle (  2.13);

\path[fill=fillColor,fill opacity=0.20] ( 77.14, 75.99) circle (  2.13);

\path[fill=fillColor,fill opacity=0.20] ( 74.08, 77.86) circle (  2.13);

\path[fill=fillColor,fill opacity=0.20] ( 71.89, 65.71) circle (  2.13);

\path[fill=fillColor,fill opacity=0.20] ( 67.96, 63.84) circle (  2.13);

\path[fill=fillColor,fill opacity=0.20] ( 56.16, 70.48) circle (  2.13);

\path[fill=fillColor,fill opacity=0.20] ( 72.77, 55.95) circle (  2.13);

\path[fill=fillColor,fill opacity=0.20] ( 79.10, 54.18) circle (  2.13);

\path[fill=fillColor,fill opacity=0.20] ( 77.57, 56.57) circle (  2.13);

\path[fill=fillColor,fill opacity=0.20] ( 72.33, 63.32) circle (  2.13);

\path[fill=fillColor,fill opacity=0.20] ( 76.04, 88.96) circle (  2.13);

\path[fill=fillColor,fill opacity=0.20] ( 93.31, 70.17) circle (  2.13);

\path[fill=fillColor,fill opacity=0.20] ( 89.59, 68.82) circle (  2.13);

\path[fill=fillColor,fill opacity=0.20] ( 85.44, 64.67) circle (  2.13);

\path[fill=fillColor,fill opacity=0.20] ( 78.45, 66.23) circle (  2.13);

\path[fill=fillColor,fill opacity=0.20] ( 78.45, 66.95) circle (  2.13);

\path[fill=fillColor,fill opacity=0.20] ( 73.86, 69.13) circle (  2.13);

\path[fill=fillColor,fill opacity=0.20] ( 68.62, 67.79) circle (  2.13);

\path[fill=fillColor,fill opacity=0.20] ( 61.62, 70.07) circle (  2.13);

\path[fill=fillColor,fill opacity=0.20] ( 50.04, 83.77) circle (  2.13);

\path[fill=fillColor,fill opacity=0.20] ( 76.92, 69.24) circle (  2.13);

\path[fill=fillColor,fill opacity=0.20] ( 74.30, 60.62) circle (  2.13);

\path[fill=fillColor,fill opacity=0.20] ( 79.98, 77.34) circle (  2.13);

\path[fill=fillColor,fill opacity=0.20] ( 88.06, 74.53) circle (  2.13);

\path[fill=fillColor,fill opacity=0.20] ( 79.76, 70.28) circle (  2.13);

\path[fill=fillColor,fill opacity=0.20] ( 78.67, 62.28) circle (  2.13);

\path[fill=fillColor,fill opacity=0.20] ( 77.14, 45.36) circle (  2.13);

\path[fill=fillColor,fill opacity=0.20] ( 66.87, 81.28) circle (  2.13);

\path[fill=fillColor,fill opacity=0.20] ( 64.68,100.39) circle (  2.13);

\path[fill=fillColor,fill opacity=0.20] (112.10, 76.19) circle (  2.13);

\path[fill=fillColor,fill opacity=0.20] ( 94.84, 71.63) circle (  2.13);

\path[fill=fillColor,fill opacity=0.20] ( 84.13, 60.73) circle (  2.13);

\path[fill=fillColor,fill opacity=0.20] ( 82.60, 63.53) circle (  2.13);

\path[fill=fillColor,fill opacity=0.20] ( 83.25, 64.26) circle (  2.13);

\path[fill=fillColor,fill opacity=0.20] ( 72.77, 61.66) circle (  2.13);

\path[fill=fillColor,fill opacity=0.20] ( 66.21, 74.33) circle (  2.13);

\path[fill=fillColor,fill opacity=0.20] ( 86.97, 62.39) circle (  2.13);

\path[fill=fillColor,fill opacity=0.20] ( 84.13, 67.47) circle (  2.13);

\path[fill=fillColor,fill opacity=0.20] ( 88.72, 71.94) circle (  2.13);

\path[fill=fillColor,fill opacity=0.20] ( 86.97, 63.74) circle (  2.13);

\path[fill=fillColor,fill opacity=0.20] ( 86.10, 62.28) circle (  2.13);

\path[fill=fillColor,fill opacity=0.20] ( 81.94, 65.61) circle (  2.13);

\path[fill=fillColor,fill opacity=0.20] ( 75.83, 58.96) circle (  2.13);

\path[fill=fillColor,fill opacity=0.20] ( 78.88, 60.83) circle (  2.13);

\path[fill=fillColor,fill opacity=0.20] ( 59.88, 97.27) circle (  2.13);

\path[fill=fillColor,fill opacity=0.20] (110.57, 53.98) circle (  2.13);

\path[fill=fillColor,fill opacity=0.20] (114.28, 59.38) circle (  2.13);

\path[fill=fillColor,fill opacity=0.20] ( 88.06, 51.07) circle (  2.13);

\path[fill=fillColor,fill opacity=0.20] ( 91.99, 58.65) circle (  2.13);

\path[fill=fillColor,fill opacity=0.20] ( 89.15, 66.64) circle (  2.13);

\path[fill=fillColor,fill opacity=0.20] ( 72.33, 56.47) circle (  2.13);

\path[fill=fillColor,fill opacity=0.20] ( 75.83, 60.62) circle (  2.13);

\path[fill=fillColor,fill opacity=0.20] ( 68.40, 67.79) circle (  2.13);

\path[fill=fillColor,fill opacity=0.20] ( 86.53, 76.40) circle (  2.13);

\path[fill=fillColor,fill opacity=0.20] ( 94.84, 56.68) circle (  2.13);

\path[fill=fillColor,fill opacity=0.20] ( 94.84, 60.62) circle (  2.13);

\path[fill=fillColor,fill opacity=0.20] ( 90.90, 55.85) circle (  2.13);

\path[fill=fillColor,fill opacity=0.20] ( 81.51, 52.00) circle (  2.13);

\path[fill=fillColor,fill opacity=0.20] ( 85.44, 45.57) circle (  2.13);

\path[fill=fillColor,fill opacity=0.20] ( 86.53, 50.86) circle (  2.13);

\path[fill=fillColor,fill opacity=0.20] ( 77.36, 57.71) circle (  2.13);

\path[fill=fillColor,fill opacity=0.20] ( 84.78, 54.60) circle (  2.13);

\path[fill=fillColor,fill opacity=0.20] ( 97.68, 47.85) circle (  2.13);

\path[fill=fillColor,fill opacity=0.20] ( 98.77, 53.77) circle (  2.13);

\path[fill=fillColor,fill opacity=0.20] ( 84.78, 59.79) circle (  2.13);

\path[fill=fillColor,fill opacity=0.20] ( 85.66, 62.91) circle (  2.13);

\path[fill=fillColor,fill opacity=0.20] ( 82.16, 67.27) circle (  2.13);

\path[fill=fillColor,fill opacity=0.20] ( 74.95, 56.88) circle (  2.13);

\path[fill=fillColor,fill opacity=0.20] ( 74.51, 48.68) circle (  2.13);

\path[fill=fillColor,fill opacity=0.20] ( 68.62, 51.07) circle (  2.13);

\path[fill=fillColor,fill opacity=0.20] ( 94.62, 58.55) circle (  2.13);

\path[fill=fillColor,fill opacity=0.20] (102.26, 49.62) circle (  2.13);

\path[fill=fillColor,fill opacity=0.20] ( 94.62, 66.33) circle (  2.13);

\path[fill=fillColor,fill opacity=0.20] ( 92.21, 60.52) circle (  2.13);

\path[fill=fillColor,fill opacity=0.20] ( 92.43, 52.00) circle (  2.13);

\path[fill=fillColor,fill opacity=0.20] ( 87.41, 39.23) circle (  2.13);

\path[fill=fillColor,fill opacity=0.20] ( 91.12, 43.28) circle (  2.13);

\path[fill=fillColor,fill opacity=0.20] ( 81.94, 52.11) circle (  2.13);

\path[fill=fillColor,fill opacity=0.20] ( 78.67, 45.26) circle (  2.13);

\path[fill=fillColor,fill opacity=0.20] ( 91.56, 66.85) circle (  2.13);

\path[fill=fillColor,fill opacity=0.20] ( 65.56, 86.89) circle (  2.13);

\path[fill=fillColor,fill opacity=0.20] ( 82.60, 60.10) circle (  2.13);

\path[fill=fillColor,fill opacity=0.20] ( 89.81, 69.86) circle (  2.13);

\path[fill=fillColor,fill opacity=0.20] ( 85.88, 70.17) circle (  2.13);

\path[fill=fillColor,fill opacity=0.20] ( 82.60, 62.28) circle (  2.13);

\path[fill=fillColor,fill opacity=0.20] ( 81.29, 56.05) circle (  2.13);

\path[fill=fillColor,fill opacity=0.20] ( 71.89, 51.17) circle (  2.13);

\path[fill=fillColor,fill opacity=0.20] ( 66.65, 52.00) circle (  2.13);

\path[fill=fillColor,fill opacity=0.20] (104.67, 57.71) circle (  2.13);

\path[fill=fillColor,fill opacity=0.20] ( 98.11, 62.59) circle (  2.13);

\path[fill=fillColor,fill opacity=0.20] ( 91.99, 77.44) circle (  2.13);

\path[fill=fillColor,fill opacity=0.20] ( 96.58, 63.74) circle (  2.13);

\path[fill=fillColor,fill opacity=0.20] ( 99.21, 50.34) circle (  2.13);

\path[fill=fillColor,fill opacity=0.20] ( 92.87, 42.76) circle (  2.13);

\path[fill=fillColor,fill opacity=0.20] ( 99.21, 50.24) circle (  2.13);

\path[fill=fillColor,fill opacity=0.20] ( 88.06, 56.57) circle (  2.13);

\path[fill=fillColor,fill opacity=0.20] ( 80.41, 42.25) circle (  2.13);

\path[fill=fillColor,fill opacity=0.20] ( 78.23, 49.62) circle (  2.13);

\path[fill=fillColor,fill opacity=0.20] ( 76.04, 66.23) circle (  2.13);

\path[fill=fillColor,fill opacity=0.20] ( 87.84, 55.85) circle (  2.13);

\path[fill=fillColor,fill opacity=0.20] ( 90.47, 60.31) circle (  2.13);

\path[fill=fillColor,fill opacity=0.20] ( 77.79, 56.78) circle (  2.13);

\path[fill=fillColor,fill opacity=0.20] ( 79.32, 60.31) circle (  2.13);

\path[fill=fillColor,fill opacity=0.20] ( 75.17, 71.42) circle (  2.13);

\path[fill=fillColor,fill opacity=0.20] ( 65.34, 74.43) circle (  2.13);

\path[fill=fillColor,fill opacity=0.20] ( 67.74, 66.23) circle (  2.13);

\path[fill=fillColor,fill opacity=0.20] ( 96.36, 60.10) circle (  2.13);

\path[fill=fillColor,fill opacity=0.20] ( 87.41, 68.62) circle (  2.13);

\path[fill=fillColor,fill opacity=0.20] ( 89.59, 69.65) circle (  2.13);

\path[fill=fillColor,fill opacity=0.20] ( 92.43, 57.92) circle (  2.13);

\path[fill=fillColor,fill opacity=0.20] ( 93.52, 57.30) circle (  2.13);

\path[fill=fillColor,fill opacity=0.20] ( 87.41, 56.36) circle (  2.13);

\path[fill=fillColor,fill opacity=0.20] ( 93.31, 56.47) circle (  2.13);

\path[fill=fillColor,fill opacity=0.20] ( 92.65, 58.34) circle (  2.13);

\path[fill=fillColor,fill opacity=0.20] ( 80.41, 47.75) circle (  2.13);

\path[fill=fillColor,fill opacity=0.20] ( 46.55, 87.93) circle (  2.13);

\path[fill=fillColor,fill opacity=0.20] ( 66.43, 56.99) circle (  2.13);

\path[fill=fillColor,fill opacity=0.20] ( 83.04, 44.01) circle (  2.13);

\path[fill=fillColor,fill opacity=0.20] ( 82.16, 60.21) circle (  2.13);

\path[fill=fillColor,fill opacity=0.20] ( 76.70, 71.73) circle (  2.13);

\path[fill=fillColor,fill opacity=0.20] ( 78.67, 75.47) circle (  2.13);

\path[fill=fillColor,fill opacity=0.20] ( 72.77, 77.34) circle (  2.13);

\path[fill=fillColor,fill opacity=0.20] ( 69.71, 69.97) circle (  2.13);

\path[fill=fillColor,fill opacity=0.20] ( 66.87, 63.63) circle (  2.13);

\path[fill=fillColor,fill opacity=0.20] ( 78.88, 78.38) circle (  2.13);

\path[fill=fillColor,fill opacity=0.20] ( 84.78, 55.22) circle (  2.13);

\path[fill=fillColor,fill opacity=0.20] ( 83.91, 49.93) circle (  2.13);

\path[fill=fillColor,fill opacity=0.20] ( 91.12, 53.46) circle (  2.13);

\path[fill=fillColor,fill opacity=0.20] ( 93.31, 62.18) circle (  2.13);

\path[fill=fillColor,fill opacity=0.20] (100.08, 67.06) circle (  2.13);

\path[fill=fillColor,fill opacity=0.20] ( 98.55, 63.42) circle (  2.13);

\path[fill=fillColor,fill opacity=0.20] ( 95.49, 56.99) circle (  2.13);

\path[fill=fillColor,fill opacity=0.20] ( 91.12, 57.71) circle (  2.13);

\path[fill=fillColor,fill opacity=0.20] ( 84.13, 55.53) circle (  2.13);

\path[fill=fillColor,fill opacity=0.20] ( 82.60, 49.62) circle (  2.13);

\path[fill=fillColor,fill opacity=0.20] ( 74.51, 69.13) circle (  2.13);

\path[fill=fillColor,fill opacity=0.20] ( 52.45, 50.14) circle (  2.13);

\path[fill=fillColor,fill opacity=0.20] ( 71.89, 56.26) circle (  2.13);

\path[fill=fillColor,fill opacity=0.20] ( 80.63, 71.63) circle (  2.13);

\path[fill=fillColor,fill opacity=0.20] ( 86.10, 64.57) circle (  2.13);

\path[fill=fillColor,fill opacity=0.20] ( 78.01, 60.83) circle (  2.13);

\path[fill=fillColor,fill opacity=0.20] ( 72.33, 70.59) circle (  2.13);

\path[fill=fillColor,fill opacity=0.20] ( 68.62, 73.39) circle (  2.13);

\path[fill=fillColor,fill opacity=0.20] ( 70.80, 74.85) circle (  2.13);

\path[fill=fillColor,fill opacity=0.20] ( 76.26, 62.59) circle (  2.13);

\path[fill=fillColor,fill opacity=0.20] ( 81.07, 63.94) circle (  2.13);

\path[fill=fillColor,fill opacity=0.20] ( 90.90, 49.10) circle (  2.13);

\path[fill=fillColor,fill opacity=0.20] ( 95.05, 39.96) circle (  2.13);

\path[fill=fillColor,fill opacity=0.20] ( 94.18, 54.60) circle (  2.13);

\path[fill=fillColor,fill opacity=0.20] ( 97.89, 67.58) circle (  2.13);

\path[fill=fillColor,fill opacity=0.20] (104.45, 64.26) circle (  2.13);

\path[fill=fillColor,fill opacity=0.20] (107.07, 57.09) circle (  2.13);

\path[fill=fillColor,fill opacity=0.20] ( 99.42, 55.74) circle (  2.13);

\path[fill=fillColor,fill opacity=0.20] ( 93.96, 61.04) circle (  2.13);

\path[fill=fillColor,fill opacity=0.20] ( 91.12, 60.62) circle (  2.13);

\path[fill=fillColor,fill opacity=0.20] ( 81.07, 60.21) circle (  2.13);

\path[fill=fillColor,fill opacity=0.20] ( 52.88, 81.18) circle (  2.13);

\path[fill=fillColor,fill opacity=0.20] ( 54.19, 51.17) circle (  2.13);

\path[fill=fillColor,fill opacity=0.20] ( 77.36, 54.29) circle (  2.13);

\path[fill=fillColor,fill opacity=0.20] ( 90.90, 59.58) circle (  2.13);

\path[fill=fillColor,fill opacity=0.20] ( 79.98, 60.41) circle (  2.13);

\path[fill=fillColor,fill opacity=0.20] ( 75.61, 67.27) circle (  2.13);

\path[fill=fillColor,fill opacity=0.20] ( 73.42, 77.44) circle (  2.13);

\path[fill=fillColor,fill opacity=0.20] ( 69.49, 69.03) circle (  2.13);

\path[fill=fillColor,fill opacity=0.20] ( 82.82, 68.62) circle (  2.13);

\path[fill=fillColor,fill opacity=0.20] ( 89.15, 61.04) circle (  2.13);

\path[fill=fillColor,fill opacity=0.20] ( 86.75, 62.70) circle (  2.13);

\path[fill=fillColor,fill opacity=0.20] ( 97.02, 56.47) circle (  2.13);

\path[fill=fillColor,fill opacity=0.20] (101.83, 64.15) circle (  2.13);

\path[fill=fillColor,fill opacity=0.20] (109.91, 70.17) circle (  2.13);

\path[fill=fillColor,fill opacity=0.20] (105.10, 64.46) circle (  2.13);

\path[fill=fillColor,fill opacity=0.20] (104.45, 57.92) circle (  2.13);

\path[fill=fillColor,fill opacity=0.20] (100.52, 50.97) circle (  2.13);

\path[fill=fillColor,fill opacity=0.20] (100.08, 52.11) circle (  2.13);

\path[fill=fillColor,fill opacity=0.20] (101.39, 62.80) circle (  2.13);

\path[fill=fillColor,fill opacity=0.20] ( 94.18, 56.47) circle (  2.13);

\path[fill=fillColor,fill opacity=0.20] ( 83.04, 51.07) circle (  2.13);

\path[fill=fillColor,fill opacity=0.20] ( 53.32, 78.79) circle (  2.13);

\path[fill=fillColor,fill opacity=0.20] ( 80.85, 52.84) circle (  2.13);

\path[fill=fillColor,fill opacity=0.20] ( 76.26, 57.20) circle (  2.13);

\path[fill=fillColor,fill opacity=0.20] ( 85.66, 68.30) circle (  2.13);

\path[fill=fillColor,fill opacity=0.20] ( 79.54, 66.33) circle (  2.13);

\path[fill=fillColor,fill opacity=0.20] ( 76.70, 68.20) circle (  2.13);

\path[fill=fillColor,fill opacity=0.20] ( 74.51, 67.06) circle (  2.13);

\path[fill=fillColor,fill opacity=0.20] ( 72.99, 64.36) circle (  2.13);

\path[fill=fillColor,fill opacity=0.20] ( 80.85, 65.29) circle (  2.13);

\path[fill=fillColor,fill opacity=0.20] ( 88.50, 65.19) circle (  2.13);

\path[fill=fillColor,fill opacity=0.20] ( 90.68, 63.42) circle (  2.13);

\path[fill=fillColor,fill opacity=0.20] ( 92.65, 53.77) circle (  2.13);

\path[fill=fillColor,fill opacity=0.20] ( 97.46, 64.67) circle (  2.13);

\path[fill=fillColor,fill opacity=0.20] ( 96.15, 84.81) circle (  2.13);

\path[fill=fillColor,fill opacity=0.20] (106.85, 80.76) circle (  2.13);

\path[fill=fillColor,fill opacity=0.20] (105.76, 63.22) circle (  2.13);

\path[fill=fillColor,fill opacity=0.20] ( 97.02, 51.69) circle (  2.13);

\path[fill=fillColor,fill opacity=0.20] ( 99.42, 45.98) circle (  2.13);

\path[fill=fillColor,fill opacity=0.20] ( 99.86, 54.39) circle (  2.13);

\path[fill=fillColor,fill opacity=0.20] (102.48, 64.88) circle (  2.13);

\path[fill=fillColor,fill opacity=0.20] ( 89.37, 54.29) circle (  2.13);

\path[fill=fillColor,fill opacity=0.20] ( 74.30, 51.28) circle (  2.13);

\path[fill=fillColor,fill opacity=0.20] ( 48.08, 79.31) circle (  2.13);

\path[fill=fillColor,fill opacity=0.20] ( 62.72, 57.92) circle (  2.13);

\path[fill=fillColor,fill opacity=0.20] ( 71.89, 62.59) circle (  2.13);

\path[fill=fillColor,fill opacity=0.20] ( 87.19, 71.52) circle (  2.13);

\path[fill=fillColor,fill opacity=0.20] ( 81.07, 67.16) circle (  2.13);

\path[fill=fillColor,fill opacity=0.20] ( 78.01, 65.50) circle (  2.13);

\path[fill=fillColor,fill opacity=0.20] ( 82.60, 69.65) circle (  2.13);

\path[fill=fillColor,fill opacity=0.20] ( 73.20, 69.76) circle (  2.13);

\path[fill=fillColor,fill opacity=0.20] ( 64.90, 73.50) circle (  2.13);

\path[fill=fillColor,fill opacity=0.20] ( 73.20, 67.89) circle (  2.13);

\path[fill=fillColor,fill opacity=0.20] ( 86.97, 59.89) circle (  2.13);

\path[fill=fillColor,fill opacity=0.20] ( 97.24, 60.31) circle (  2.13);

\path[fill=fillColor,fill opacity=0.20] ( 91.34, 55.74) circle (  2.13);

\path[fill=fillColor,fill opacity=0.20] (101.17, 55.33) circle (  2.13);

\path[fill=fillColor,fill opacity=0.20] (100.73, 65.92) circle (  2.13);

\path[fill=fillColor,fill opacity=0.20] (102.05, 75.78) circle (  2.13);

\path[fill=fillColor,fill opacity=0.20] (104.45, 76.09) circle (  2.13);

\path[fill=fillColor,fill opacity=0.20] (105.32, 63.22) circle (  2.13);

\path[fill=fillColor,fill opacity=0.20] (100.08, 47.96) circle (  2.13);

\path[fill=fillColor,fill opacity=0.20] (103.79, 53.25) circle (  2.13);

\path[fill=fillColor,fill opacity=0.20] (105.76, 67.37) circle (  2.13);

\path[fill=fillColor,fill opacity=0.20] (102.48, 66.33) circle (  2.13);

\path[fill=fillColor,fill opacity=0.20] ( 86.53, 59.79) circle (  2.13);

\path[fill=fillColor,fill opacity=0.20] ( 69.49, 68.62) circle (  2.13);

\path[fill=fillColor,fill opacity=0.20] (104.23, 67.68) circle (  2.13);

\path[fill=fillColor,fill opacity=0.20] ( 86.75, 63.11) circle (  2.13);

\path[fill=fillColor,fill opacity=0.20] ( 84.13, 61.76) circle (  2.13);

\path[fill=fillColor,fill opacity=0.20] ( 81.29, 71.11) circle (  2.13);

\path[fill=fillColor,fill opacity=0.20] ( 72.11, 65.81) circle (  2.13);

\path[fill=fillColor,fill opacity=0.20] ( 73.20, 61.76) circle (  2.13);

\path[fill=fillColor,fill opacity=0.20] ( 63.37, 84.81) circle (  2.13);

\path[fill=fillColor,fill opacity=0.20] ( 72.77, 70.28) circle (  2.13);

\path[fill=fillColor,fill opacity=0.20] ( 75.83, 74.12) circle (  2.13);

\path[fill=fillColor,fill opacity=0.20] ( 95.71, 65.61) circle (  2.13);

\path[fill=fillColor,fill opacity=0.20] ( 98.77, 61.66) circle (  2.13);

\path[fill=fillColor,fill opacity=0.20] ( 90.68, 65.19) circle (  2.13);

\path[fill=fillColor,fill opacity=0.20] ( 96.80, 62.49) circle (  2.13);

\path[fill=fillColor,fill opacity=0.20] ( 99.86, 60.83) circle (  2.13);

\path[fill=fillColor,fill opacity=0.20] (107.29, 62.70) circle (  2.13);

\path[fill=fillColor,fill opacity=0.20] (101.83, 64.98) circle (  2.13);

\path[fill=fillColor,fill opacity=0.20] ( 98.77, 59.27) circle (  2.13);

\path[fill=fillColor,fill opacity=0.20] (105.32, 54.39) circle (  2.13);

\path[fill=fillColor,fill opacity=0.20] (130.45, 65.29) circle (  2.13);

\path[fill=fillColor,fill opacity=0.20] (100.30, 66.54) circle (  2.13);

\path[fill=fillColor,fill opacity=0.20] ( 88.50, 59.48) circle (  2.13);

\path[fill=fillColor,fill opacity=0.20] ( 68.62, 68.93) circle (  2.13);

\path[fill=fillColor,fill opacity=0.20] ( 68.62, 63.22) circle (  2.13);

\path[fill=fillColor,fill opacity=0.20] ( 72.33, 47.23) circle (  2.13);

\path[fill=fillColor,fill opacity=0.20] ( 78.88, 57.20) circle (  2.13);

\path[fill=fillColor,fill opacity=0.20] ( 76.26, 69.86) circle (  2.13);

\path[fill=fillColor,fill opacity=0.20] ( 74.08, 57.92) circle (  2.13);

\path[fill=fillColor,fill opacity=0.20] ( 73.42, 59.17) circle (  2.13);

\path[fill=fillColor,fill opacity=0.20] ( 69.49, 73.70) circle (  2.13);

\path[fill=fillColor,fill opacity=0.20] ( 69.27, 77.54) circle (  2.13);

\path[fill=fillColor,fill opacity=0.20] ( 77.14, 68.93) circle (  2.13);

\path[fill=fillColor,fill opacity=0.20] ( 80.20, 66.12) circle (  2.13);

\path[fill=fillColor,fill opacity=0.20] ( 87.19, 76.40) circle (  2.13);

\path[fill=fillColor,fill opacity=0.20] ( 96.58, 71.73) circle (  2.13);

\path[fill=fillColor,fill opacity=0.20] (101.17, 68.62) circle (  2.13);

\path[fill=fillColor,fill opacity=0.20] ( 98.99, 72.35) circle (  2.13);

\path[fill=fillColor,fill opacity=0.20] ( 97.24, 60.10) circle (  2.13);

\path[fill=fillColor,fill opacity=0.20] ( 99.86, 52.32) circle (  2.13);

\path[fill=fillColor,fill opacity=0.20] (101.83, 63.84) circle (  2.13);

\path[fill=fillColor,fill opacity=0.20] (138.75, 64.67) circle (  2.13);

\path[fill=fillColor,fill opacity=0.20] (101.17, 56.57) circle (  2.13);

\path[fill=fillColor,fill opacity=0.20] ( 93.96, 57.51) circle (  2.13);

\path[fill=fillColor,fill opacity=0.20] ( 83.69, 57.71) circle (  2.13);

\path[fill=fillColor,fill opacity=0.20] ( 83.47, 49.72) circle (  2.13);

\path[fill=fillColor,fill opacity=0.20] ( 88.28, 57.92) circle (  2.13);

\path[fill=fillColor,fill opacity=0.20] ( 55.51, 67.99) circle (  2.13);

\path[fill=fillColor,fill opacity=0.20] ( 68.83, 59.69) circle (  2.13);

\path[fill=fillColor,fill opacity=0.20] ( 77.36, 57.92) circle (  2.13);

\path[fill=fillColor,fill opacity=0.20] ( 77.79, 55.43) circle (  2.13);

\path[fill=fillColor,fill opacity=0.20] ( 81.94, 52.52) circle (  2.13);

\path[fill=fillColor,fill opacity=0.20] ( 78.88, 51.90) circle (  2.13);

\path[fill=fillColor,fill opacity=0.20] ( 75.61, 56.36) circle (  2.13);

\path[fill=fillColor,fill opacity=0.20] ( 71.02, 63.11) circle (  2.13);

\path[fill=fillColor,fill opacity=0.20] ( 85.22, 60.73) circle (  2.13);

\path[fill=fillColor,fill opacity=0.20] ( 91.78, 55.43) circle (  2.13);

\path[fill=fillColor,fill opacity=0.20] ( 97.68, 70.38) circle (  2.13);

\path[fill=fillColor,fill opacity=0.20] ( 98.11, 72.35) circle (  2.13);

\path[fill=fillColor,fill opacity=0.20] (102.92, 66.02) circle (  2.13);

\path[fill=fillColor,fill opacity=0.20] (104.23, 66.44) circle (  2.13);

\path[fill=fillColor,fill opacity=0.20] (111.66, 53.87) circle (  2.13);

\path[fill=fillColor,fill opacity=0.20] (105.32, 47.02) circle (  2.13);

\path[fill=fillColor,fill opacity=0.20] ( 98.77, 62.39) circle (  2.13);

\path[fill=fillColor,fill opacity=0.20] ( 99.86, 62.91) circle (  2.13);

\path[fill=fillColor,fill opacity=0.20] ( 94.40, 54.50) circle (  2.13);

\path[fill=fillColor,fill opacity=0.20] ( 81.07, 59.69) circle (  2.13);

\path[fill=fillColor,fill opacity=0.20] ( 73.42, 58.03) circle (  2.13);

\path[fill=fillColor,fill opacity=0.20] ( 57.91, 57.09) circle (  2.13);

\path[fill=fillColor,fill opacity=0.20] ( 51.57, 79.72) circle (  2.13);

\path[fill=fillColor,fill opacity=0.20] ( 93.09, 52.52) circle (  2.13);

\path[fill=fillColor,fill opacity=0.20] ( 80.20, 53.46) circle (  2.13);

\path[fill=fillColor,fill opacity=0.20] ( 81.73, 56.05) circle (  2.13);

\path[fill=fillColor,fill opacity=0.20] ( 75.83, 49.10) circle (  2.13);

\path[fill=fillColor,fill opacity=0.20] ( 74.51, 52.52) circle (  2.13);

\path[fill=fillColor,fill opacity=0.20] ( 69.05, 56.36) circle (  2.13);

\path[fill=fillColor,fill opacity=0.20] ( 68.18, 57.71) circle (  2.13);

\path[fill=fillColor,fill opacity=0.20] ( 74.73, 58.44) circle (  2.13);

\path[fill=fillColor,fill opacity=0.20] ( 74.51, 66.64) circle (  2.13);

\path[fill=fillColor,fill opacity=0.20] ( 83.47, 70.07) circle (  2.13);

\path[fill=fillColor,fill opacity=0.20] ( 93.96, 60.73) circle (  2.13);

\path[fill=fillColor,fill opacity=0.20] ( 97.24, 60.93) circle (  2.13);

\path[fill=fillColor,fill opacity=0.20] ( 95.49, 67.37) circle (  2.13);

\path[fill=fillColor,fill opacity=0.20] (100.08, 64.98) circle (  2.13);

\path[fill=fillColor,fill opacity=0.20] ( 98.33, 64.05) circle (  2.13);

\path[fill=fillColor,fill opacity=0.20] ( 99.21, 65.61) circle (  2.13);

\path[fill=fillColor,fill opacity=0.20] (111.22, 60.21) circle (  2.13);

\path[fill=fillColor,fill opacity=0.20] ( 92.87, 52.84) circle (  2.13);

\path[fill=fillColor,fill opacity=0.20] ( 86.10, 53.46) circle (  2.13);

\path[fill=fillColor,fill opacity=0.20] ( 77.14, 55.02) circle (  2.13);

\path[fill=fillColor,fill opacity=0.20] ( 70.36, 74.95) circle (  2.13);

\path[fill=fillColor,fill opacity=0.20] ( 46.98, 71.63) circle (  2.13);

\path[fill=fillColor,fill opacity=0.20] ( 82.82, 52.84) circle (  2.13);

\path[fill=fillColor,fill opacity=0.20] ( 82.60, 57.40) circle (  2.13);

\path[fill=fillColor,fill opacity=0.20] ( 76.70, 65.29) circle (  2.13);

\path[fill=fillColor,fill opacity=0.20] ( 74.73, 64.57) circle (  2.13);

\path[fill=fillColor,fill opacity=0.20] ( 71.89, 68.10) circle (  2.13);

\path[fill=fillColor,fill opacity=0.20] ( 72.11, 66.44) circle (  2.13);

\path[fill=fillColor,fill opacity=0.20] ( 71.67, 64.88) circle (  2.13);

\path[fill=fillColor,fill opacity=0.20] ( 69.49, 61.56) circle (  2.13);

\path[fill=fillColor,fill opacity=0.20] ( 70.14, 66.95) circle (  2.13);

\path[fill=fillColor,fill opacity=0.20] ( 79.32, 60.41) circle (  2.13);

\path[fill=fillColor,fill opacity=0.20] ( 85.22, 55.43) circle (  2.13);

\path[fill=fillColor,fill opacity=0.20] ( 88.94, 64.05) circle (  2.13);

\path[fill=fillColor,fill opacity=0.20] ( 96.80, 64.15) circle (  2.13);

\path[fill=fillColor,fill opacity=0.20] ( 98.55, 55.02) circle (  2.13);

\path[fill=fillColor,fill opacity=0.20] ( 96.80, 58.03) circle (  2.13);

\path[fill=fillColor,fill opacity=0.20] (100.52, 64.26) circle (  2.13);

\path[fill=fillColor,fill opacity=0.20] (102.48, 64.36) circle (  2.13);

\path[fill=fillColor,fill opacity=0.20] (104.67, 67.06) circle (  2.13);

\path[fill=fillColor,fill opacity=0.20] ( 95.49, 64.26) circle (  2.13);

\path[fill=fillColor,fill opacity=0.20] ( 88.94, 62.39) circle (  2.13);

\path[fill=fillColor,fill opacity=0.20] ( 89.59, 64.05) circle (  2.13);

\path[fill=fillColor,fill opacity=0.20] ( 87.62, 58.13) circle (  2.13);

\path[fill=fillColor,fill opacity=0.20] ( 74.95, 55.43) circle (  2.13);

\path[fill=fillColor,fill opacity=0.20] ( 52.01, 75.57) circle (  2.13);

\path[fill=fillColor,fill opacity=0.20] ( 52.01, 69.65) circle (  2.13);

\path[fill=fillColor,fill opacity=0.20] ( 69.05, 53.77) circle (  2.13);

\path[fill=fillColor,fill opacity=0.20] ( 90.25, 66.64) circle (  2.13);

\path[fill=fillColor,fill opacity=0.20] ( 78.23, 71.32) circle (  2.13);

\path[fill=fillColor,fill opacity=0.20] ( 74.51, 74.22) circle (  2.13);

\path[fill=fillColor,fill opacity=0.20] ( 82.16, 69.24) circle (  2.13);

\path[fill=fillColor,fill opacity=0.20] ( 79.98, 61.66) circle (  2.13);

\path[fill=fillColor,fill opacity=0.20] ( 70.14, 55.43) circle (  2.13);

\path[fill=fillColor,fill opacity=0.20] ( 62.50, 77.54) circle (  2.13);

\path[fill=fillColor,fill opacity=0.20] ( 58.56,101.42) circle (  2.13);

\path[fill=fillColor,fill opacity=0.20] ( 71.89, 55.85) circle (  2.13);

\path[fill=fillColor,fill opacity=0.20] ( 77.14, 55.43) circle (  2.13);

\path[fill=fillColor,fill opacity=0.20] ( 78.88, 61.45) circle (  2.13);

\path[fill=fillColor,fill opacity=0.20] ( 89.59, 56.99) circle (  2.13);

\path[fill=fillColor,fill opacity=0.20] ( 98.99, 60.93) circle (  2.13);

\path[fill=fillColor,fill opacity=0.20] (101.61, 71.11) circle (  2.13);

\path[fill=fillColor,fill opacity=0.20] (105.10, 63.63) circle (  2.13);

\path[fill=fillColor,fill opacity=0.20] (115.16, 48.99) circle (  2.13);

\path[fill=fillColor,fill opacity=0.20] ( 93.09, 47.02) circle (  2.13);

\path[fill=fillColor,fill opacity=0.20] ( 91.12, 53.56) circle (  2.13);

\path[fill=fillColor,fill opacity=0.20] ( 83.91, 58.75) circle (  2.13);

\path[fill=fillColor,fill opacity=0.20] ( 83.04, 63.32) circle (  2.13);

\path[fill=fillColor,fill opacity=0.20] ( 78.67, 69.65) circle (  2.13);

\path[fill=fillColor,fill opacity=0.20] ( 70.36, 81.28) circle (  2.13);

\path[fill=fillColor,fill opacity=0.20] ( 61.62, 82.74) circle (  2.13);

\path[fill=fillColor,fill opacity=0.20] ( 59.88, 70.80) circle (  2.13);

\path[fill=fillColor,fill opacity=0.20] ( 45.67, 67.68) circle (  2.13);

\path[fill=fillColor,fill opacity=0.20] ( 75.17, 59.48) circle (  2.13);

\path[fill=fillColor,fill opacity=0.20] ( 75.83, 60.41) circle (  2.13);

\path[fill=fillColor,fill opacity=0.20] ( 77.57, 63.74) circle (  2.13);

\path[fill=fillColor,fill opacity=0.20] ( 77.57, 66.23) circle (  2.13);

\path[fill=fillColor,fill opacity=0.20] ( 76.26, 66.02) circle (  2.13);

\path[fill=fillColor,fill opacity=0.20] ( 80.41, 60.52) circle (  2.13);

\path[fill=fillColor,fill opacity=0.20] ( 80.63, 51.49) circle (  2.13);

\path[fill=fillColor,fill opacity=0.20] ( 70.58, 53.87) circle (  2.13);

\path[fill=fillColor,fill opacity=0.20] ( 66.43, 68.82) circle (  2.13);

\path[fill=fillColor,fill opacity=0.20] ( 68.40, 69.97) circle (  2.13);

\path[fill=fillColor,fill opacity=0.20] ( 66.87, 61.97) circle (  2.13);

\path[fill=fillColor,fill opacity=0.20] ( 71.89, 72.98) circle (  2.13);

\path[fill=fillColor,fill opacity=0.20] ( 72.11, 53.87) circle (  2.13);

\path[fill=fillColor,fill opacity=0.20] ( 74.95, 58.86) circle (  2.13);

\path[fill=fillColor,fill opacity=0.20] ( 78.01, 72.35) circle (  2.13);

\path[fill=fillColor,fill opacity=0.20] ( 87.41, 73.50) circle (  2.13);

\path[fill=fillColor,fill opacity=0.20] ( 97.24, 68.93) circle (  2.13);

\path[fill=fillColor,fill opacity=0.20] ( 93.52, 67.16) circle (  2.13);

\path[fill=fillColor,fill opacity=0.20] ( 91.78, 68.93) circle (  2.13);

\path[fill=fillColor,fill opacity=0.20] ( 92.21, 61.87) circle (  2.13);

\path[fill=fillColor,fill opacity=0.20] ( 92.87, 56.68) circle (  2.13);

\path[fill=fillColor,fill opacity=0.20] ( 87.84, 57.20) circle (  2.13);

\path[fill=fillColor,fill opacity=0.20] ( 78.45, 56.68) circle (  2.13);

\path[fill=fillColor,fill opacity=0.20] ( 68.18, 60.73) circle (  2.13);

\path[fill=fillColor,fill opacity=0.20] ( 65.12, 68.93) circle (  2.13);

\path[fill=fillColor,fill opacity=0.20] ( 59.22, 73.60) circle (  2.13);

\path[fill=fillColor,fill opacity=0.20] ( 59.66, 88.96) circle (  2.13);

\path[fill=fillColor,fill opacity=0.20] ( 74.08, 59.38) circle (  2.13);

\path[fill=fillColor,fill opacity=0.20] ( 76.26, 63.42) circle (  2.13);

\path[fill=fillColor,fill opacity=0.20] ( 70.36, 70.28) circle (  2.13);

\path[fill=fillColor,fill opacity=0.20] ( 79.98, 67.89) circle (  2.13);

\path[fill=fillColor,fill opacity=0.20] ( 83.04, 57.51) circle (  2.13);

\path[fill=fillColor,fill opacity=0.20] ( 79.54, 58.96) circle (  2.13);

\path[fill=fillColor,fill opacity=0.20] ( 75.39, 63.01) circle (  2.13);

\path[fill=fillColor,fill opacity=0.20] ( 75.39, 61.14) circle (  2.13);

\path[fill=fillColor,fill opacity=0.20] ( 77.79, 63.11) circle (  2.13);

\path[fill=fillColor,fill opacity=0.20] ( 74.73, 57.09) circle (  2.13);

\path[fill=fillColor,fill opacity=0.20] ( 66.43, 67.68) circle (  2.13);

\path[fill=fillColor,fill opacity=0.20] ( 77.36, 69.45) circle (  2.13);

\path[fill=fillColor,fill opacity=0.20] ( 79.76, 68.93) circle (  2.13);

\path[fill=fillColor,fill opacity=0.20] ( 88.94, 74.22) circle (  2.13);

\path[fill=fillColor,fill opacity=0.20] ( 98.11, 77.34) circle (  2.13);

\path[fill=fillColor,fill opacity=0.20] ( 89.59, 69.13) circle (  2.13);

\path[fill=fillColor,fill opacity=0.20] ( 89.81, 60.21) circle (  2.13);

\path[fill=fillColor,fill opacity=0.20] ( 91.12, 58.75) circle (  2.13);

\path[fill=fillColor,fill opacity=0.20] ( 75.83, 55.74) circle (  2.13);

\path[fill=fillColor,fill opacity=0.20] ( 67.96, 56.26) circle (  2.13);

\path[fill=fillColor,fill opacity=0.20] (103.36, 68.51) circle (  2.13);

\path[fill=fillColor,fill opacity=0.20] ( 60.31, 82.74) circle (  2.13);

\path[fill=fillColor,fill opacity=0.20] ( 58.13, 86.89) circle (  2.13);

\path[fill=fillColor,fill opacity=0.20] ( 83.04, 68.20) circle (  2.13);

\path[fill=fillColor,fill opacity=0.20] ( 69.71, 62.59) circle (  2.13);

\path[fill=fillColor,fill opacity=0.20] ( 71.02, 62.49) circle (  2.13);

\path[fill=fillColor,fill opacity=0.20] ( 78.01, 62.59) circle (  2.13);

\path[fill=fillColor,fill opacity=0.20] ( 75.39, 67.27) circle (  2.13);

\path[fill=fillColor,fill opacity=0.20] ( 74.95, 67.47) circle (  2.13);

\path[fill=fillColor,fill opacity=0.20] ( 73.20, 55.85) circle (  2.13);

\path[fill=fillColor,fill opacity=0.20] ( 75.61, 60.73) circle (  2.13);

\path[fill=fillColor,fill opacity=0.20] ( 76.92, 72.35) circle (  2.13);

\path[fill=fillColor,fill opacity=0.20] ( 77.36, 65.40) circle (  2.13);

\path[fill=fillColor,fill opacity=0.20] ( 72.77, 66.12) circle (  2.13);

\path[fill=fillColor,fill opacity=0.20] ( 68.62, 74.43) circle (  2.13);

\path[fill=fillColor,fill opacity=0.20] ( 78.88, 58.13) circle (  2.13);

\path[fill=fillColor,fill opacity=0.20] ( 72.77, 60.93) circle (  2.13);

\path[fill=fillColor,fill opacity=0.20] ( 71.89, 63.94) circle (  2.13);

\path[fill=fillColor,fill opacity=0.20] ( 65.56, 65.92) circle (  2.13);

\path[fill=fillColor,fill opacity=0.20] ( 68.40, 68.41) circle (  2.13);

\path[fill=fillColor,fill opacity=0.20] ( 67.30, 61.24) circle (  2.13);

\path[fill=fillColor,fill opacity=0.20] ( 62.06, 61.45) circle (  2.13);

\path[fill=fillColor,fill opacity=0.20] ( 63.59, 62.49) circle (  2.13);

\path[fill=fillColor,fill opacity=0.20] ( 68.83, 61.35) circle (  2.13);

\path[fill=fillColor,fill opacity=0.20] ( 78.67, 59.79) circle (  2.13);

\path[fill=fillColor,fill opacity=0.20] ( 80.85, 70.80) circle (  2.13);

\path[fill=fillColor,fill opacity=0.20] ( 78.45, 62.08) circle (  2.13);

\path[fill=fillColor,fill opacity=0.20] ( 74.51, 63.84) circle (  2.13);

\path[fill=fillColor,fill opacity=0.20] ( 72.77, 55.85) circle (  2.13);

\path[fill=fillColor,fill opacity=0.20] ( 70.14, 43.60) circle (  2.13);

\path[fill=fillColor,fill opacity=0.20] (106.85, 50.14) circle (  2.13);

\path[fill=fillColor,fill opacity=0.20] ( 64.68, 62.80) circle (  2.13);

\path[fill=fillColor,fill opacity=0.20] ( 57.03, 74.53) circle (  2.13);

\path[fill=fillColor,fill opacity=0.20] ( 51.35, 95.19) circle (  2.13);

\path[fill=fillColor,fill opacity=0.20] ( 66.65, 70.80) circle (  2.13);

\path[fill=fillColor,fill opacity=0.20] ( 61.84, 57.61) circle (  2.13);

\path[fill=fillColor,fill opacity=0.20] ( 67.52, 62.28) circle (  2.13);

\path[fill=fillColor,fill opacity=0.20] ( 69.49, 62.28) circle (  2.13);

\path[fill=fillColor,fill opacity=0.20] ( 71.46, 53.87) circle (  2.13);

\path[fill=fillColor,fill opacity=0.20] ( 74.08, 61.76) circle (  2.13);

\path[fill=fillColor,fill opacity=0.20] ( 77.57, 76.40) circle (  2.13);

\path[fill=fillColor,fill opacity=0.20] ( 78.01, 74.53) circle (  2.13);

\path[fill=fillColor,fill opacity=0.20] ( 76.70, 63.42) circle (  2.13);

\path[fill=fillColor,fill opacity=0.20] ( 79.32, 67.58) circle (  2.13);

\path[fill=fillColor,fill opacity=0.20] ( 71.46, 70.69) circle (  2.13);

\path[fill=fillColor,fill opacity=0.20] ( 72.99, 58.23) circle (  2.13);

\path[fill=fillColor,fill opacity=0.20] ( 81.07, 50.14) circle (  2.13);

\path[fill=fillColor,fill opacity=0.20] ( 78.01, 52.00) circle (  2.13);

\path[fill=fillColor,fill opacity=0.20] ( 75.61, 52.00) circle (  2.13);

\path[fill=fillColor,fill opacity=0.20] ( 76.92, 65.09) circle (  2.13);

\path[fill=fillColor,fill opacity=0.20] ( 88.28, 72.87) circle (  2.13);

\path[fill=fillColor,fill opacity=0.20] ( 86.75, 53.98) circle (  2.13);

\path[fill=fillColor,fill opacity=0.20] ( 87.19, 43.60) circle (  2.13);

\path[fill=fillColor,fill opacity=0.20] ( 80.41, 57.92) circle (  2.13);

\path[fill=fillColor,fill opacity=0.20] ( 74.08, 61.97) circle (  2.13);

\path[fill=fillColor,fill opacity=0.20] ( 71.89, 60.62) circle (  2.13);

\path[fill=fillColor,fill opacity=0.20] ( 71.24, 64.36) circle (  2.13);

\path[fill=fillColor,fill opacity=0.20] ( 70.14, 62.80) circle (  2.13);

\path[fill=fillColor,fill opacity=0.20] ( 72.55, 63.01) circle (  2.13);

\path[fill=fillColor,fill opacity=0.20] ( 91.56, 68.51) circle (  2.13);

\path[fill=fillColor,fill opacity=0.20] ( 85.00, 64.46) circle (  2.13);

\path[fill=fillColor,fill opacity=0.20] ( 66.43, 47.44) circle (  2.13);

\path[fill=fillColor,fill opacity=0.20] ( 69.05, 48.27) circle (  2.13);

\path[fill=fillColor,fill opacity=0.20] ( 69.05, 61.24) circle (  2.13);

\path[fill=fillColor,fill opacity=0.20] ( 49.17, 75.26) circle (  2.13);

\path[fill=fillColor,fill opacity=0.20] ( 55.72, 94.16) circle (  2.13);

\path[fill=fillColor,fill opacity=0.20] ( 54.63, 66.23) circle (  2.13);

\path[fill=fillColor,fill opacity=0.20] ( 67.52, 57.40) circle (  2.13);

\path[fill=fillColor,fill opacity=0.20] ( 65.56, 69.45) circle (  2.13);

\path[fill=fillColor,fill opacity=0.20] ( 69.27, 74.85) circle (  2.13);

\path[fill=fillColor,fill opacity=0.20] ( 68.83, 66.54) circle (  2.13);

\path[fill=fillColor,fill opacity=0.20] ( 67.09, 53.15) circle (  2.13);

\path[fill=fillColor,fill opacity=0.20] ( 73.20, 62.39) circle (  2.13);

\path[fill=fillColor,fill opacity=0.20] ( 83.04, 74.74) circle (  2.13);

\path[fill=fillColor,fill opacity=0.20] ( 75.17, 55.22) circle (  2.13);

\path[fill=fillColor,fill opacity=0.20] ( 79.54, 45.15) circle (  2.13);

\path[fill=fillColor,fill opacity=0.20] ( 87.41, 62.70) circle (  2.13);

\path[fill=fillColor,fill opacity=0.20] ( 74.95, 65.29) circle (  2.13);

\path[fill=fillColor,fill opacity=0.20] ( 70.58, 66.54) circle (  2.13);

\path[fill=fillColor,fill opacity=0.20] ( 76.48, 74.74) circle (  2.13);

\path[fill=fillColor,fill opacity=0.20] ( 76.48, 62.59) circle (  2.13);

\path[fill=fillColor,fill opacity=0.20] ( 74.30, 56.47) circle (  2.13);

\path[fill=fillColor,fill opacity=0.20] ( 72.99, 73.39) circle (  2.13);

\path[fill=fillColor,fill opacity=0.20] ( 75.61, 65.40) circle (  2.13);

\path[fill=fillColor,fill opacity=0.20] ( 78.01, 77.54) circle (  2.13);

\path[fill=fillColor,fill opacity=0.20] ( 78.67, 85.85) circle (  2.13);

\path[fill=fillColor,fill opacity=0.20] ( 76.04, 64.46) circle (  2.13);

\path[fill=fillColor,fill opacity=0.20] ( 78.23, 48.99) circle (  2.13);

\path[fill=fillColor,fill opacity=0.20] ( 80.85, 39.23) circle (  2.13);

\path[fill=fillColor,fill opacity=0.20] ( 66.43, 56.36) circle (  2.13);

\path[fill=fillColor,fill opacity=0.20] ( 55.51, 63.53) circle (  2.13);

\path[fill=fillColor,fill opacity=0.20] ( 62.93, 56.26) circle (  2.13);

\path[fill=fillColor,fill opacity=0.20] ( 59.22, 85.85) circle (  2.13);

\path[fill=fillColor,fill opacity=0.20] ( 60.75, 77.13) circle (  2.13);

\path[fill=fillColor,fill opacity=0.20] ( 62.50, 66.44) circle (  2.13);

\path[fill=fillColor,fill opacity=0.20] ( 64.90, 51.49) circle (  2.13);

\path[fill=fillColor,fill opacity=0.20] ( 62.93, 40.17) circle (  2.13);

\path[fill=fillColor,fill opacity=0.20] ( 67.74, 45.15) circle (  2.13);

\path[fill=fillColor,fill opacity=0.20] ( 70.80, 64.88) circle (  2.13);

\path[fill=fillColor,fill opacity=0.20] ( 73.20, 64.88) circle (  2.13);

\path[fill=fillColor,fill opacity=0.20] ( 71.46, 60.73) circle (  2.13);

\path[fill=fillColor,fill opacity=0.20] ( 75.17, 75.68) circle (  2.13);

\path[fill=fillColor,fill opacity=0.20] ( 75.39, 76.51) circle (  2.13);

\path[fill=fillColor,fill opacity=0.20] ( 80.41, 74.22) circle (  2.13);

\path[fill=fillColor,fill opacity=0.20] ( 77.36, 60.73) circle (  2.13);

\path[fill=fillColor,fill opacity=0.20] ( 85.88, 70.90) circle (  2.13);

\path[fill=fillColor,fill opacity=0.20] ( 84.57, 59.89) circle (  2.13);

\path[fill=fillColor,fill opacity=0.20] ( 88.50, 46.81) circle (  2.13);

\path[fill=fillColor,fill opacity=0.20] ( 81.51, 42.87) circle (  2.13);

\path[fill=fillColor,fill opacity=0.20] ( 88.72, 51.17) circle (  2.13);

\path[fill=fillColor,fill opacity=0.20] ( 74.51, 66.02) circle (  2.13);

\path[fill=fillColor,fill opacity=0.20] ( 58.13, 66.23) circle (  2.13);

\path[fill=fillColor,fill opacity=0.20] ( 53.10, 56.36) circle (  2.13);

\path[fill=fillColor,fill opacity=0.20] ( 53.76, 68.62) circle (  2.13);

\path[fill=fillColor,fill opacity=0.20] ( 59.00, 58.75) circle (  2.13);

\path[fill=fillColor,fill opacity=0.20] ( 61.62, 56.68) circle (  2.13);

\path[fill=fillColor,fill opacity=0.20] ( 69.05, 51.49) circle (  2.13);

\path[fill=fillColor,fill opacity=0.20] ( 67.74, 59.48) circle (  2.13);

\path[fill=fillColor,fill opacity=0.20] ( 82.38, 54.81) circle (  2.13);

\path[fill=fillColor,fill opacity=0.20] ( 76.70, 70.17) circle (  2.13);

\path[fill=fillColor,fill opacity=0.20] ( 81.29, 43.49) circle (  2.13);

\path[fill=fillColor,fill opacity=0.20] ( 70.36, 46.19) circle (  2.13);

\path[fill=fillColor,fill opacity=0.20] ( 64.68, 56.16) circle (  2.13);

\path[fill=fillColor,fill opacity=0.20] ( 79.54, 49.10) circle (  2.13);

\path[fill=fillColor,fill opacity=0.20] ( 67.74, 44.01) circle (  2.13);

\path[fill=fillColor,fill opacity=0.20] ( 60.31, 57.20) circle (  2.13);

\path[fill=fillColor,fill opacity=0.20] ( 55.29, 75.16) circle (  2.13);

\path[fill=fillColor,fill opacity=0.20] ( 53.98, 91.04) circle (  2.13);

\path[fill=fillColor,fill opacity=0.20] ( 55.29, 70.38) circle (  2.13);

\path[fill=fillColor,fill opacity=0.20] ( 69.27, 60.21) circle (  2.13);

\path[fill=fillColor,fill opacity=0.20] ( 68.18, 45.88) circle (  2.13);

\path[fill=fillColor,fill opacity=0.20] ( 86.97, 41.31) circle (  2.13);

\path[fill=fillColor,fill opacity=0.20] ( 83.91, 54.08) circle (  2.13);

\path[fill=fillColor,fill opacity=0.20] ( 58.56, 41.93) circle (  2.13);

\path[fill=fillColor,fill opacity=0.20] ( 49.61, 53.56) circle (  2.13);

\path[fill=fillColor,fill opacity=0.20] ( 46.77, 66.75) circle (  2.13);

\path[fill=fillColor,fill opacity=0.20] ( 58.35, 71.94) circle (  2.13);

\path[fill=fillColor,fill opacity=0.20] ( 62.50, 58.03) circle (  2.13);

\path[fill=fillColor,fill opacity=0.20] ( 78.01, 57.71) circle (  2.13);

\path[fill=fillColor,fill opacity=0.20] ( 65.99, 69.45) circle (  2.13);

\path[fill=fillColor,fill opacity=0.20] ( 69.49, 70.28) circle (  2.13);

\path[fill=fillColor,fill opacity=0.20] ( 78.45, 58.96) circle (  2.13);

\path[fill=fillColor,fill opacity=0.20] ( 48.95, 68.82) circle (  2.13);

\path[fill=fillColor,fill opacity=0.20] (104.45,104.54) circle (  2.13);
\end{scope}
\begin{scope}
\path[clip] (159.87, 34.04) rectangle (277.03,119.86);
\definecolor[named]{fillColor}{rgb}{0.90,0.90,0.90}

\path[fill=fillColor] (159.87, 34.04) rectangle (277.03,119.86);
\definecolor[named]{drawColor}{rgb}{0.95,0.95,0.95}

\path[draw=drawColor,line width= 0.3pt,line join=round] (159.87, 50.55) --
	(277.03, 50.55);

\path[draw=drawColor,line width= 0.3pt,line join=round] (159.87, 71.32) --
	(277.03, 71.32);

\path[draw=drawColor,line width= 0.3pt,line join=round] (159.87, 92.08) --
	(277.03, 92.08);

\path[draw=drawColor,line width= 0.3pt,line join=round] (159.87,112.84) --
	(277.03,112.84);

\path[draw=drawColor,line width= 0.3pt,line join=round] (169.78, 34.04) --
	(169.78,119.86);

\path[draw=drawColor,line width= 0.3pt,line join=round] (191.63, 34.04) --
	(191.63,119.86);

\path[draw=drawColor,line width= 0.3pt,line join=round] (213.48, 34.04) --
	(213.48,119.86);

\path[draw=drawColor,line width= 0.3pt,line join=round] (235.33, 34.04) --
	(235.33,119.86);

\path[draw=drawColor,line width= 0.3pt,line join=round] (257.18, 34.04) --
	(257.18,119.86);
\definecolor[named]{drawColor}{rgb}{1.00,1.00,1.00}

\path[draw=drawColor,line width= 0.6pt,line join=round] (159.87, 40.17) --
	(277.03, 40.17);

\path[draw=drawColor,line width= 0.6pt,line join=round] (159.87, 60.93) --
	(277.03, 60.93);

\path[draw=drawColor,line width= 0.6pt,line join=round] (159.87, 81.70) --
	(277.03, 81.70);

\path[draw=drawColor,line width= 0.6pt,line join=round] (159.87,102.46) --
	(277.03,102.46);

\path[draw=drawColor,line width= 0.6pt,line join=round] (180.71, 34.04) --
	(180.71,119.86);

\path[draw=drawColor,line width= 0.6pt,line join=round] (202.56, 34.04) --
	(202.56,119.86);

\path[draw=drawColor,line width= 0.6pt,line join=round] (224.41, 34.04) --
	(224.41,119.86);

\path[draw=drawColor,line width= 0.6pt,line join=round] (246.26, 34.04) --
	(246.26,119.86);

\path[draw=drawColor,line width= 0.6pt,line join=round] (268.11, 34.04) --
	(268.11,119.86);
\definecolor[named]{fillColor}{rgb}{0.00,0.00,0.00}

\path[fill=fillColor,fill opacity=0.20] (185.95, 70.28) circle (  2.13);

\path[fill=fillColor,fill opacity=0.20] (193.16, 72.25) circle (  2.13);

\path[fill=fillColor,fill opacity=0.20] (191.41, 68.62) circle (  2.13);

\path[fill=fillColor,fill opacity=0.20] (196.88, 59.06) circle (  2.13);

\path[fill=fillColor,fill opacity=0.20] (186.61, 56.16) circle (  2.13);

\path[fill=fillColor,fill opacity=0.20] (177.65, 69.13) circle (  2.13);

\path[fill=fillColor,fill opacity=0.20] (187.48, 72.98) circle (  2.13);

\path[fill=fillColor,fill opacity=0.20] (202.56, 65.29) circle (  2.13);

\path[fill=fillColor,fill opacity=0.20] (213.05, 70.17) circle (  2.13);

\path[fill=fillColor,fill opacity=0.20] (214.14, 71.11) circle (  2.13);

\path[fill=fillColor,fill opacity=0.20] (207.58, 60.10) circle (  2.13);

\path[fill=fillColor,fill opacity=0.20] (209.99, 51.90) circle (  2.13);

\path[fill=fillColor,fill opacity=0.20] (210.42, 58.96) circle (  2.13);

\path[fill=fillColor,fill opacity=0.20] (200.15, 68.62) circle (  2.13);

\path[fill=fillColor,fill opacity=0.20] (186.83, 69.03) circle (  2.13);

\path[fill=fillColor,fill opacity=0.20] (177.21, 65.50) circle (  2.13);

\path[fill=fillColor,fill opacity=0.20] (170.66, 66.02) circle (  2.13);

\path[fill=fillColor,fill opacity=0.20] (194.69, 82.74) circle (  2.13);

\path[fill=fillColor,fill opacity=0.20] (211.30, 77.03) circle (  2.13);

\path[fill=fillColor,fill opacity=0.20] (222.00, 70.07) circle (  2.13);

\path[fill=fillColor,fill opacity=0.20] (219.38, 66.12) circle (  2.13);

\path[fill=fillColor,fill opacity=0.20] (211.52, 59.89) circle (  2.13);

\path[fill=fillColor,fill opacity=0.20] (206.93, 57.09) circle (  2.13);

\path[fill=fillColor,fill opacity=0.20] (202.12, 62.39) circle (  2.13);

\path[fill=fillColor,fill opacity=0.20] (202.56, 69.03) circle (  2.13);

\path[fill=fillColor,fill opacity=0.20] (201.47, 72.04) circle (  2.13);

\path[fill=fillColor,fill opacity=0.20] (194.91, 73.08) circle (  2.13);

\path[fill=fillColor,fill opacity=0.20] (188.14, 68.82) circle (  2.13);

\path[fill=fillColor,fill opacity=0.20] (183.77, 67.16) circle (  2.13);

\path[fill=fillColor,fill opacity=0.20] (189.67, 72.25) circle (  2.13);

\path[fill=fillColor,fill opacity=0.20] (206.27, 68.20) circle (  2.13);

\path[fill=fillColor,fill opacity=0.20] (220.69, 69.24) circle (  2.13);

\path[fill=fillColor,fill opacity=0.20] (221.13, 73.81) circle (  2.13);

\path[fill=fillColor,fill opacity=0.20] (216.54, 69.13) circle (  2.13);

\path[fill=fillColor,fill opacity=0.20] (216.54, 62.28) circle (  2.13);

\path[fill=fillColor,fill opacity=0.20] (213.70, 64.36) circle (  2.13);

\path[fill=fillColor,fill opacity=0.20] (205.62, 70.59) circle (  2.13);

\path[fill=fillColor,fill opacity=0.20] (203.00, 71.21) circle (  2.13);

\path[fill=fillColor,fill opacity=0.20] (214.79, 64.15) circle (  2.13);

\path[fill=fillColor,fill opacity=0.20] (214.79, 71.83) circle (  2.13);

\path[fill=fillColor,fill opacity=0.20] (200.81, 83.77) circle (  2.13);

\path[fill=fillColor,fill opacity=0.20] (182.24, 81.70) circle (  2.13);

\path[fill=fillColor,fill opacity=0.20] (195.57, 64.77) circle (  2.13);

\path[fill=fillColor,fill opacity=0.20] (215.67, 49.72) circle (  2.13);

\path[fill=fillColor,fill opacity=0.20] (222.22, 55.22) circle (  2.13);

\path[fill=fillColor,fill opacity=0.20] (215.67, 73.81) circle (  2.13);

\path[fill=fillColor,fill opacity=0.20] (218.07, 78.06) circle (  2.13);

\path[fill=fillColor,fill opacity=0.20] (219.60, 72.87) circle (  2.13);

\path[fill=fillColor,fill opacity=0.20] (209.33, 67.99) circle (  2.13);

\path[fill=fillColor,fill opacity=0.20] (204.96, 68.41) circle (  2.13);

\path[fill=fillColor,fill opacity=0.20] (199.50, 73.08) circle (  2.13);

\path[fill=fillColor,fill opacity=0.20] (215.01, 69.97) circle (  2.13);

\path[fill=fillColor,fill opacity=0.20] (215.23, 77.65) circle (  2.13);

\path[fill=fillColor,fill opacity=0.20] (208.46, 79.83) circle (  2.13);

\path[fill=fillColor,fill opacity=0.20] (204.96, 73.70) circle (  2.13);

\path[fill=fillColor,fill opacity=0.20] (201.90, 66.33) circle (  2.13);

\path[fill=fillColor,fill opacity=0.20] (195.57, 70.48) circle (  2.13);

\path[fill=fillColor,fill opacity=0.20] (184.64, 83.77) circle (  2.13);

\path[fill=fillColor,fill opacity=0.20] (197.97, 65.50) circle (  2.13);

\path[fill=fillColor,fill opacity=0.20] (215.89, 56.36) circle (  2.13);

\path[fill=fillColor,fill opacity=0.20] (218.73, 60.21) circle (  2.13);

\path[fill=fillColor,fill opacity=0.20] (213.92, 71.83) circle (  2.13);

\path[fill=fillColor,fill opacity=0.20] (211.74, 77.65) circle (  2.13);

\path[fill=fillColor,fill opacity=0.20] (210.86, 74.74) circle (  2.13);

\path[fill=fillColor,fill opacity=0.20] (206.71, 65.92) circle (  2.13);

\path[fill=fillColor,fill opacity=0.20] (203.65, 66.02) circle (  2.13);

\path[fill=fillColor,fill opacity=0.20] (190.32, 79.41) circle (  2.13);

\path[fill=fillColor,fill opacity=0.20] (209.77, 75.99) circle (  2.13);

\path[fill=fillColor,fill opacity=0.20] (221.13, 74.01) circle (  2.13);

\path[fill=fillColor,fill opacity=0.20] (216.54, 83.77) circle (  2.13);

\path[fill=fillColor,fill opacity=0.20] (215.45, 78.06) circle (  2.13);

\path[fill=fillColor,fill opacity=0.20] (219.60, 68.20) circle (  2.13);

\path[fill=fillColor,fill opacity=0.20] (213.05, 62.91) circle (  2.13);

\path[fill=fillColor,fill opacity=0.20] (210.64, 57.82) circle (  2.13);

\path[fill=fillColor,fill opacity=0.20] (208.68, 57.40) circle (  2.13);

\path[fill=fillColor,fill opacity=0.20] (185.52, 92.08) circle (  2.13);

\path[fill=fillColor,fill opacity=0.20] (199.50, 72.56) circle (  2.13);

\path[fill=fillColor,fill opacity=0.20] (215.45, 73.70) circle (  2.13);

\path[fill=fillColor,fill opacity=0.20] (215.89, 74.01) circle (  2.13);

\path[fill=fillColor,fill opacity=0.20] (213.48, 72.56) circle (  2.13);

\path[fill=fillColor,fill opacity=0.20] (211.74, 72.46) circle (  2.13);

\path[fill=fillColor,fill opacity=0.20] (206.71, 72.35) circle (  2.13);

\path[fill=fillColor,fill opacity=0.20] (206.93, 67.47) circle (  2.13);

\path[fill=fillColor,fill opacity=0.20] (201.68, 65.40) circle (  2.13);

\path[fill=fillColor,fill opacity=0.20] (214.79, 67.99) circle (  2.13);

\path[fill=fillColor,fill opacity=0.20] (220.26, 71.83) circle (  2.13);

\path[fill=fillColor,fill opacity=0.20] (218.95, 72.98) circle (  2.13);

\path[fill=fillColor,fill opacity=0.20] (222.00, 61.14) circle (  2.13);

\path[fill=fillColor,fill opacity=0.20] (218.95, 56.57) circle (  2.13);

\path[fill=fillColor,fill opacity=0.20] (217.42, 59.17) circle (  2.13);

\path[fill=fillColor,fill opacity=0.20] (218.51, 55.43) circle (  2.13);

\path[fill=fillColor,fill opacity=0.20] (206.93, 49.41) circle (  2.13);

\path[fill=fillColor,fill opacity=0.20] (199.72, 78.17) circle (  2.13);

\path[fill=fillColor,fill opacity=0.20] (213.26, 77.96) circle (  2.13);

\path[fill=fillColor,fill opacity=0.20] (213.70, 78.79) circle (  2.13);

\path[fill=fillColor,fill opacity=0.20] (211.74, 76.51) circle (  2.13);

\path[fill=fillColor,fill opacity=0.20] (215.23, 72.77) circle (  2.13);

\path[fill=fillColor,fill opacity=0.20] (210.64, 69.65) circle (  2.13);

\path[fill=fillColor,fill opacity=0.20] (208.46, 66.75) circle (  2.13);

\path[fill=fillColor,fill opacity=0.20] (201.25, 63.32) circle (  2.13);

\path[fill=fillColor,fill opacity=0.20] (206.05, 79.72) circle (  2.13);

\path[fill=fillColor,fill opacity=0.20] (227.03, 55.02) circle (  2.13);

\path[fill=fillColor,fill opacity=0.20] (219.38, 57.51) circle (  2.13);

\path[fill=fillColor,fill opacity=0.20] (224.19, 50.45) circle (  2.13);

\path[fill=fillColor,fill opacity=0.20] (223.10, 48.16) circle (  2.13);

\path[fill=fillColor,fill opacity=0.20] (213.05, 58.13) circle (  2.13);

\path[fill=fillColor,fill opacity=0.20] (212.61, 56.05) circle (  2.13);

\path[fill=fillColor,fill opacity=0.20] (206.71, 49.10) circle (  2.13);

\path[fill=fillColor,fill opacity=0.20] (203.21, 49.31) circle (  2.13);

\path[fill=fillColor,fill opacity=0.20] (196.88, 47.02) circle (  2.13);

\path[fill=fillColor,fill opacity=0.20] (199.94, 82.74) circle (  2.13);

\path[fill=fillColor,fill opacity=0.20] (209.33, 75.05) circle (  2.13);

\path[fill=fillColor,fill opacity=0.20] (212.39, 75.26) circle (  2.13);

\path[fill=fillColor,fill opacity=0.20] (211.30, 76.61) circle (  2.13);

\path[fill=fillColor,fill opacity=0.20] (215.23, 69.86) circle (  2.13);

\path[fill=fillColor,fill opacity=0.20] (214.14, 58.13) circle (  2.13);

\path[fill=fillColor,fill opacity=0.20] (206.49, 56.57) circle (  2.13);

\path[fill=fillColor,fill opacity=0.20] (201.68, 59.17) circle (  2.13);

\path[fill=fillColor,fill opacity=0.20] (189.89, 62.39) circle (  2.13);

\path[fill=fillColor,fill opacity=0.20] (224.85, 49.10) circle (  2.13);

\path[fill=fillColor,fill opacity=0.20] (232.27, 41.21) circle (  2.13);

\path[fill=fillColor,fill opacity=0.20] (218.51, 50.55) circle (  2.13);

\path[fill=fillColor,fill opacity=0.20] (219.82, 42.66) circle (  2.13);

\path[fill=fillColor,fill opacity=0.20] (215.01, 47.44) circle (  2.13);

\path[fill=fillColor,fill opacity=0.20] (207.37, 57.61) circle (  2.13);

\path[fill=fillColor,fill opacity=0.20] (211.30, 46.50) circle (  2.13);

\path[fill=fillColor,fill opacity=0.20] (204.52, 41.93) circle (  2.13);

\path[fill=fillColor,fill opacity=0.20] (199.94, 52.11) circle (  2.13);

\path[fill=fillColor,fill opacity=0.20] (194.04, 50.65) circle (  2.13);

\path[fill=fillColor,fill opacity=0.20] (195.78, 87.93) circle (  2.13);

\path[fill=fillColor,fill opacity=0.20] (210.42, 70.80) circle (  2.13);

\path[fill=fillColor,fill opacity=0.20] (217.85, 68.10) circle (  2.13);

\path[fill=fillColor,fill opacity=0.20] (215.89, 66.75) circle (  2.13);

\path[fill=fillColor,fill opacity=0.20] (216.98, 57.82) circle (  2.13);

\path[fill=fillColor,fill opacity=0.20] (215.23, 49.62) circle (  2.13);

\path[fill=fillColor,fill opacity=0.20] (205.62, 53.04) circle (  2.13);

\path[fill=fillColor,fill opacity=0.20] (199.06, 56.26) circle (  2.13);

\path[fill=fillColor,fill opacity=0.20] (195.57, 51.80) circle (  2.13);

\path[fill=fillColor,fill opacity=0.20] (176.99, 67.37) circle (  2.13);

\path[fill=fillColor,fill opacity=0.20] (185.73,102.46) circle (  2.13);

\path[fill=fillColor,fill opacity=0.20] (222.00, 41.83) circle (  2.13);

\path[fill=fillColor,fill opacity=0.20] (214.36, 49.51) circle (  2.13);

\path[fill=fillColor,fill opacity=0.20] (217.20, 46.61) circle (  2.13);

\path[fill=fillColor,fill opacity=0.20] (216.76, 51.17) circle (  2.13);

\path[fill=fillColor,fill opacity=0.20] (215.01, 48.89) circle (  2.13);

\path[fill=fillColor,fill opacity=0.20] (209.33, 39.75) circle (  2.13);

\path[fill=fillColor,fill opacity=0.20] (201.03, 48.68) circle (  2.13);

\path[fill=fillColor,fill opacity=0.20] (204.09, 46.50) circle (  2.13);

\path[fill=fillColor,fill opacity=0.20] (183.11, 57.20) circle (  2.13);

\path[fill=fillColor,fill opacity=0.20] (210.86, 67.37) circle (  2.13);

\path[fill=fillColor,fill opacity=0.20] (221.35, 60.00) circle (  2.13);

\path[fill=fillColor,fill opacity=0.20] (221.79, 50.55) circle (  2.13);

\path[fill=fillColor,fill opacity=0.20] (216.76, 45.05) circle (  2.13);

\path[fill=fillColor,fill opacity=0.20] (213.26, 53.25) circle (  2.13);

\path[fill=fillColor,fill opacity=0.20] (208.89, 57.30) circle (  2.13);

\path[fill=fillColor,fill opacity=0.20] (200.59, 54.50) circle (  2.13);

\path[fill=fillColor,fill opacity=0.20] (201.03, 51.90) circle (  2.13);

\path[fill=fillColor,fill opacity=0.20] (190.76, 57.51) circle (  2.13);

\path[fill=fillColor,fill opacity=0.20] (204.96, 62.49) circle (  2.13);

\path[fill=fillColor,fill opacity=0.20] (221.79, 45.98) circle (  2.13);

\path[fill=fillColor,fill opacity=0.20] (213.48, 47.12) circle (  2.13);

\path[fill=fillColor,fill opacity=0.20] (214.79, 42.25) circle (  2.13);

\path[fill=fillColor,fill opacity=0.20] (218.07, 46.09) circle (  2.13);

\path[fill=fillColor,fill opacity=0.20] (214.36, 52.42) circle (  2.13);

\path[fill=fillColor,fill opacity=0.20] (211.30, 46.40) circle (  2.13);

\path[fill=fillColor,fill opacity=0.20] (209.77, 42.45) circle (  2.13);

\path[fill=fillColor,fill opacity=0.20] (207.15, 45.46) circle (  2.13);

\path[fill=fillColor,fill opacity=0.20] (208.68, 45.15) circle (  2.13);

\path[fill=fillColor,fill opacity=0.20] (198.19, 44.84) circle (  2.13);

\path[fill=fillColor,fill opacity=0.20] (180.71, 64.46) circle (  2.13);

\path[fill=fillColor,fill opacity=0.20] (198.63, 69.34) circle (  2.13);

\path[fill=fillColor,fill opacity=0.20] (208.89, 56.78) circle (  2.13);

\path[fill=fillColor,fill opacity=0.20] (218.95, 46.81) circle (  2.13);

\path[fill=fillColor,fill opacity=0.20] (214.14, 47.64) circle (  2.13);

\path[fill=fillColor,fill opacity=0.20] (210.64, 56.68) circle (  2.13);

\path[fill=fillColor,fill opacity=0.20] (215.89, 54.60) circle (  2.13);

\path[fill=fillColor,fill opacity=0.20] (209.33, 52.84) circle (  2.13);

\path[fill=fillColor,fill opacity=0.20] (204.09, 58.13) circle (  2.13);

\path[fill=fillColor,fill opacity=0.20] (197.31, 61.56) circle (  2.13);

\path[fill=fillColor,fill opacity=0.20] (186.83, 85.85) circle (  2.13);

\path[fill=fillColor,fill opacity=0.20] (211.52, 46.19) circle (  2.13);

\path[fill=fillColor,fill opacity=0.20] (208.89, 50.86) circle (  2.13);

\path[fill=fillColor,fill opacity=0.20] (209.33, 45.26) circle (  2.13);

\path[fill=fillColor,fill opacity=0.20] (221.35, 39.44) circle (  2.13);

\path[fill=fillColor,fill opacity=0.20] (214.79, 46.71) circle (  2.13);

\path[fill=fillColor,fill opacity=0.20] (218.07, 51.38) circle (  2.13);

\path[fill=fillColor,fill opacity=0.20] (218.73, 52.63) circle (  2.13);

\path[fill=fillColor,fill opacity=0.20] (210.21, 55.02) circle (  2.13);

\path[fill=fillColor,fill opacity=0.20] (206.71, 50.97) circle (  2.13);

\path[fill=fillColor,fill opacity=0.20] (205.84, 45.98) circle (  2.13);

\path[fill=fillColor,fill opacity=0.20] (194.91, 56.57) circle (  2.13);

\path[fill=fillColor,fill opacity=0.20] (198.41, 65.40) circle (  2.13);

\path[fill=fillColor,fill opacity=0.20] (213.05, 51.17) circle (  2.13);

\path[fill=fillColor,fill opacity=0.20] (212.83, 53.87) circle (  2.13);

\path[fill=fillColor,fill opacity=0.20] (209.33, 54.70) circle (  2.13);

\path[fill=fillColor,fill opacity=0.20] (214.58, 52.00) circle (  2.13);

\path[fill=fillColor,fill opacity=0.20] (216.76, 55.95) circle (  2.13);

\path[fill=fillColor,fill opacity=0.20] (206.93, 61.35) circle (  2.13);

\path[fill=fillColor,fill opacity=0.20] (201.25, 63.01) circle (  2.13);

\path[fill=fillColor,fill opacity=0.20] (187.48, 65.40) circle (  2.13);

\path[fill=fillColor,fill opacity=0.20] (208.46, 59.27) circle (  2.13);

\path[fill=fillColor,fill opacity=0.20] (206.49, 50.65) circle (  2.13);

\path[fill=fillColor,fill opacity=0.20] (209.11, 51.38) circle (  2.13);

\path[fill=fillColor,fill opacity=0.20] (213.48, 44.74) circle (  2.13);

\path[fill=fillColor,fill opacity=0.20] (217.42, 44.94) circle (  2.13);

\path[fill=fillColor,fill opacity=0.20] (215.01, 52.63) circle (  2.13);

\path[fill=fillColor,fill opacity=0.20] (207.15, 54.91) circle (  2.13);

\path[fill=fillColor,fill opacity=0.20] (215.45, 58.03) circle (  2.13);

\path[fill=fillColor,fill opacity=0.20] (210.86, 59.48) circle (  2.13);

\path[fill=fillColor,fill opacity=0.20] (208.24, 49.51) circle (  2.13);

\path[fill=fillColor,fill opacity=0.20] (199.94, 46.61) circle (  2.13);

\path[fill=fillColor,fill opacity=0.20] (203.00, 61.35) circle (  2.13);

\path[fill=fillColor,fill opacity=0.20] (209.33, 57.09) circle (  2.13);

\path[fill=fillColor,fill opacity=0.20] (208.02, 58.13) circle (  2.13);

\path[fill=fillColor,fill opacity=0.20] (208.68, 60.73) circle (  2.13);

\path[fill=fillColor,fill opacity=0.20] (211.95, 63.42) circle (  2.13);

\path[fill=fillColor,fill opacity=0.20] (209.11, 58.65) circle (  2.13);

\path[fill=fillColor,fill opacity=0.20] (204.74, 57.92) circle (  2.13);

\path[fill=fillColor,fill opacity=0.20] (197.97, 60.52) circle (  2.13);

\path[fill=fillColor,fill opacity=0.20] (189.01, 58.86) circle (  2.13);

\path[fill=fillColor,fill opacity=0.20] (176.12, 68.93) circle (  2.13);

\path[fill=fillColor,fill opacity=0.20] (210.64, 62.28) circle (  2.13);

\path[fill=fillColor,fill opacity=0.20] (212.83, 55.33) circle (  2.13);

\path[fill=fillColor,fill opacity=0.20] (208.89, 56.78) circle (  2.13);

\path[fill=fillColor,fill opacity=0.20] (213.48, 59.48) circle (  2.13);

\path[fill=fillColor,fill opacity=0.20] (222.22, 51.69) circle (  2.13);

\path[fill=fillColor,fill opacity=0.20] (217.85, 46.29) circle (  2.13);

\path[fill=fillColor,fill opacity=0.20] (219.82, 53.87) circle (  2.13);

\path[fill=fillColor,fill opacity=0.20] (205.40, 58.55) circle (  2.13);

\path[fill=fillColor,fill opacity=0.20] (208.02, 55.85) circle (  2.13);

\path[fill=fillColor,fill opacity=0.20] (205.84, 53.15) circle (  2.13);

\path[fill=fillColor,fill opacity=0.20] (204.74, 48.79) circle (  2.13);

\path[fill=fillColor,fill opacity=0.20] (196.88, 52.00) circle (  2.13);

\path[fill=fillColor,fill opacity=0.20] (200.15, 69.97) circle (  2.13);

\path[fill=fillColor,fill opacity=0.20] (208.24, 67.89) circle (  2.13);

\path[fill=fillColor,fill opacity=0.20] (209.55, 65.29) circle (  2.13);

\path[fill=fillColor,fill opacity=0.20] (210.42, 62.49) circle (  2.13);

\path[fill=fillColor,fill opacity=0.20] (207.58, 55.74) circle (  2.13);

\path[fill=fillColor,fill opacity=0.20] (207.58, 57.09) circle (  2.13);

\path[fill=fillColor,fill opacity=0.20] (202.56, 63.01) circle (  2.13);

\path[fill=fillColor,fill opacity=0.20] (200.59, 59.89) circle (  2.13);

\path[fill=fillColor,fill opacity=0.20] (190.98, 56.68) circle (  2.13);

\path[fill=fillColor,fill opacity=0.20] (208.89, 59.79) circle (  2.13);

\path[fill=fillColor,fill opacity=0.20] (217.63, 59.38) circle (  2.13);

\path[fill=fillColor,fill opacity=0.20] (210.86, 58.86) circle (  2.13);

\path[fill=fillColor,fill opacity=0.20] (216.32, 59.58) circle (  2.13);

\path[fill=fillColor,fill opacity=0.20] (217.63, 63.63) circle (  2.13);

\path[fill=fillColor,fill opacity=0.20] (222.00, 57.09) circle (  2.13);

\path[fill=fillColor,fill opacity=0.20] (215.45, 47.85) circle (  2.13);

\path[fill=fillColor,fill opacity=0.20] (210.42, 50.65) circle (  2.13);

\path[fill=fillColor,fill opacity=0.20] (211.08, 53.56) circle (  2.13);

\path[fill=fillColor,fill opacity=0.20] (210.42, 47.23) circle (  2.13);

\path[fill=fillColor,fill opacity=0.20] (202.78, 47.23) circle (  2.13);

\path[fill=fillColor,fill opacity=0.20] (196.22, 55.33) circle (  2.13);

\path[fill=fillColor,fill opacity=0.20] (201.03, 81.18) circle (  2.13);

\path[fill=fillColor,fill opacity=0.20] (217.42, 62.80) circle (  2.13);

\path[fill=fillColor,fill opacity=0.20] (211.52, 54.81) circle (  2.13);

\path[fill=fillColor,fill opacity=0.20] (204.09, 53.77) circle (  2.13);

\path[fill=fillColor,fill opacity=0.20] (211.52, 55.74) circle (  2.13);

\path[fill=fillColor,fill opacity=0.20] (209.77, 59.58) circle (  2.13);

\path[fill=fillColor,fill opacity=0.20] (206.93, 58.03) circle (  2.13);

\path[fill=fillColor,fill opacity=0.20] (204.96, 55.22) circle (  2.13);

\path[fill=fillColor,fill opacity=0.20] (192.94, 64.05) circle (  2.13);

\path[fill=fillColor,fill opacity=0.20] (214.14, 61.97) circle (  2.13);

\path[fill=fillColor,fill opacity=0.20] (211.74, 60.83) circle (  2.13);

\path[fill=fillColor,fill opacity=0.20] (219.38, 57.71) circle (  2.13);

\path[fill=fillColor,fill opacity=0.20] (221.79, 55.22) circle (  2.13);

\path[fill=fillColor,fill opacity=0.20] (218.07, 53.67) circle (  2.13);

\path[fill=fillColor,fill opacity=0.20] (218.29, 57.30) circle (  2.13);

\path[fill=fillColor,fill opacity=0.20] (215.23, 59.38) circle (  2.13);

\path[fill=fillColor,fill opacity=0.20] (218.51, 55.02) circle (  2.13);

\path[fill=fillColor,fill opacity=0.20] (213.92, 48.37) circle (  2.13);

\path[fill=fillColor,fill opacity=0.20] (213.26, 41.83) circle (  2.13);

\path[fill=fillColor,fill opacity=0.20] (202.12, 45.88) circle (  2.13);

\path[fill=fillColor,fill opacity=0.20] (190.54, 63.01) circle (  2.13);

\path[fill=fillColor,fill opacity=0.20] (215.45, 56.88) circle (  2.13);

\path[fill=fillColor,fill opacity=0.20] (209.99, 53.15) circle (  2.13);

\path[fill=fillColor,fill opacity=0.20] (208.02, 47.33) circle (  2.13);

\path[fill=fillColor,fill opacity=0.20] (206.05, 45.98) circle (  2.13);

\path[fill=fillColor,fill opacity=0.20] (203.65, 50.97) circle (  2.13);

\path[fill=fillColor,fill opacity=0.20] (203.87, 50.03) circle (  2.13);

\path[fill=fillColor,fill opacity=0.20] (207.15, 53.25) circle (  2.13);

\path[fill=fillColor,fill opacity=0.20] (200.15, 65.92) circle (  2.13);

\path[fill=fillColor,fill opacity=0.20] (182.24, 77.23) circle (  2.13);

\path[fill=fillColor,fill opacity=0.20] (202.56, 59.17) circle (  2.13);

\path[fill=fillColor,fill opacity=0.20] (212.17, 65.29) circle (  2.13);

\path[fill=fillColor,fill opacity=0.20] (215.01, 62.49) circle (  2.13);

\path[fill=fillColor,fill opacity=0.20] (215.89, 53.67) circle (  2.13);

\path[fill=fillColor,fill opacity=0.20] (222.22, 51.07) circle (  2.13);

\path[fill=fillColor,fill opacity=0.20] (219.60, 46.29) circle (  2.13);

\path[fill=fillColor,fill opacity=0.20] (220.48, 44.53) circle (  2.13);

\path[fill=fillColor,fill opacity=0.20] (216.98, 58.03) circle (  2.13);

\path[fill=fillColor,fill opacity=0.20] (215.89, 68.62) circle (  2.13);

\path[fill=fillColor,fill opacity=0.20] (212.61, 60.93) circle (  2.13);

\path[fill=fillColor,fill opacity=0.20] (205.62, 47.54) circle (  2.13);

\path[fill=fillColor,fill opacity=0.20] (201.90, 44.43) circle (  2.13);

\path[fill=fillColor,fill opacity=0.20] (202.78, 51.69) circle (  2.13);

\path[fill=fillColor,fill opacity=0.20] (201.25, 69.24) circle (  2.13);

\path[fill=fillColor,fill opacity=0.20] (212.39, 52.00) circle (  2.13);

\path[fill=fillColor,fill opacity=0.20] (213.48, 45.57) circle (  2.13);

\path[fill=fillColor,fill opacity=0.20] (214.36, 47.96) circle (  2.13);

\path[fill=fillColor,fill opacity=0.20] (207.58, 52.94) circle (  2.13);

\path[fill=fillColor,fill opacity=0.20] (203.87, 49.72) circle (  2.13);

\path[fill=fillColor,fill opacity=0.20] (200.59, 45.26) circle (  2.13);

\path[fill=fillColor,fill opacity=0.20] (199.94, 56.57) circle (  2.13);

\path[fill=fillColor,fill opacity=0.20] (193.60, 70.90) circle (  2.13);

\path[fill=fillColor,fill opacity=0.20] (172.84, 76.51) circle (  2.13);

\path[fill=fillColor,fill opacity=0.20] (205.62, 42.66) circle (  2.13);

\path[fill=fillColor,fill opacity=0.20] (206.49, 57.82) circle (  2.13);

\path[fill=fillColor,fill opacity=0.20] (216.98, 64.26) circle (  2.13);

\path[fill=fillColor,fill opacity=0.20] (221.79, 53.98) circle (  2.13);

\path[fill=fillColor,fill opacity=0.20] (225.50, 51.69) circle (  2.13);

\path[fill=fillColor,fill opacity=0.20] (223.32, 53.25) circle (  2.13);

\path[fill=fillColor,fill opacity=0.20] (222.66, 48.16) circle (  2.13);

\path[fill=fillColor,fill opacity=0.20] (219.16, 49.31) circle (  2.13);

\path[fill=fillColor,fill opacity=0.20] (209.77, 60.52) circle (  2.13);

\path[fill=fillColor,fill opacity=0.20] (205.18, 63.11) circle (  2.13);

\path[fill=fillColor,fill opacity=0.20] (202.12, 51.80) circle (  2.13);

\path[fill=fillColor,fill opacity=0.20] (204.74, 45.78) circle (  2.13);

\path[fill=fillColor,fill opacity=0.20] (187.92, 58.13) circle (  2.13);

\path[fill=fillColor,fill opacity=0.20] (206.71, 61.87) circle (  2.13);

\path[fill=fillColor,fill opacity=0.20] (220.91, 59.48) circle (  2.13);

\path[fill=fillColor,fill opacity=0.20] (218.51, 67.37) circle (  2.13);

\path[fill=fillColor,fill opacity=0.20] (218.51, 58.55) circle (  2.13);

\path[fill=fillColor,fill opacity=0.20] (208.24, 52.11) circle (  2.13);

\path[fill=fillColor,fill opacity=0.20] (201.90, 56.99) circle (  2.13);

\path[fill=fillColor,fill opacity=0.20] (197.75, 55.64) circle (  2.13);

\path[fill=fillColor,fill opacity=0.20] (198.19, 57.71) circle (  2.13);

\path[fill=fillColor,fill opacity=0.20] (198.41, 61.87) circle (  2.13);

\path[fill=fillColor,fill opacity=0.20] (193.16, 53.46) circle (  2.13);

\path[fill=fillColor,fill opacity=0.20] (180.05, 51.80) circle (  2.13);

\path[fill=fillColor,fill opacity=0.20] (193.60, 55.22) circle (  2.13);

\path[fill=fillColor,fill opacity=0.20] (206.93, 55.74) circle (  2.13);

\path[fill=fillColor,fill opacity=0.20] (208.46, 63.53) circle (  2.13);

\path[fill=fillColor,fill opacity=0.20] (210.86, 63.42) circle (  2.13);

\path[fill=fillColor,fill opacity=0.20] (217.20, 56.05) circle (  2.13);

\path[fill=fillColor,fill opacity=0.20] (218.07, 51.90) circle (  2.13);

\path[fill=fillColor,fill opacity=0.20] (220.48, 58.44) circle (  2.13);

\path[fill=fillColor,fill opacity=0.20] (221.35, 62.28) circle (  2.13);

\path[fill=fillColor,fill opacity=0.20] (222.00, 55.33) circle (  2.13);

\path[fill=fillColor,fill opacity=0.20] (209.55, 51.49) circle (  2.13);

\path[fill=fillColor,fill opacity=0.20] (205.84, 54.91) circle (  2.13);

\path[fill=fillColor,fill opacity=0.20] (201.90, 50.24) circle (  2.13);

\path[fill=fillColor,fill opacity=0.20] (200.15, 42.76) circle (  2.13);

\path[fill=fillColor,fill opacity=0.20] (185.30, 54.70) circle (  2.13);

\path[fill=fillColor,fill opacity=0.20] (207.37, 80.87) circle (  2.13);

\path[fill=fillColor,fill opacity=0.20] (212.83, 64.67) circle (  2.13);

\path[fill=fillColor,fill opacity=0.20] (211.08, 51.28) circle (  2.13);

\path[fill=fillColor,fill opacity=0.20] (203.65, 59.38) circle (  2.13);

\path[fill=fillColor,fill opacity=0.20] (197.75, 56.26) circle (  2.13);

\path[fill=fillColor,fill opacity=0.20] (199.06, 60.83) circle (  2.13);

\path[fill=fillColor,fill opacity=0.20] (200.81, 62.39) circle (  2.13);

\path[fill=fillColor,fill opacity=0.20] (196.88, 58.34) circle (  2.13);

\path[fill=fillColor,fill opacity=0.20] (190.98, 62.49) circle (  2.13);

\path[fill=fillColor,fill opacity=0.20] (183.99, 64.88) circle (  2.13);

\path[fill=fillColor,fill opacity=0.20] (178.30, 60.41) circle (  2.13);

\path[fill=fillColor,fill opacity=0.20] (193.60, 61.14) circle (  2.13);

\path[fill=fillColor,fill opacity=0.20] (203.87, 58.65) circle (  2.13);

\path[fill=fillColor,fill opacity=0.20] (207.58, 61.56) circle (  2.13);

\path[fill=fillColor,fill opacity=0.20] (206.71, 71.21) circle (  2.13);

\path[fill=fillColor,fill opacity=0.20] (207.37, 74.95) circle (  2.13);

\path[fill=fillColor,fill opacity=0.20] (211.74, 63.11) circle (  2.13);

\path[fill=fillColor,fill opacity=0.20] (213.48, 53.04) circle (  2.13);

\path[fill=fillColor,fill opacity=0.20] (213.05, 55.74) circle (  2.13);

\path[fill=fillColor,fill opacity=0.20] (215.89, 60.83) circle (  2.13);

\path[fill=fillColor,fill opacity=0.20] (219.16, 59.79) circle (  2.13);

\path[fill=fillColor,fill opacity=0.20] (222.22, 53.87) circle (  2.13);

\path[fill=fillColor,fill opacity=0.20] (210.86, 47.44) circle (  2.13);

\path[fill=fillColor,fill opacity=0.20] (206.49, 47.64) circle (  2.13);

\path[fill=fillColor,fill opacity=0.20] (194.04, 53.25) circle (  2.13);

\path[fill=fillColor,fill opacity=0.20] (190.54, 64.15) circle (  2.13);

\path[fill=fillColor,fill opacity=0.20] (208.68, 49.20) circle (  2.13);

\path[fill=fillColor,fill opacity=0.20] (211.74, 51.38) circle (  2.13);

\path[fill=fillColor,fill opacity=0.20] (204.52, 55.74) circle (  2.13);

\path[fill=fillColor,fill opacity=0.20] (198.19, 58.65) circle (  2.13);

\path[fill=fillColor,fill opacity=0.20] (201.03, 62.18) circle (  2.13);

\path[fill=fillColor,fill opacity=0.20] (194.47, 64.26) circle (  2.13);

\path[fill=fillColor,fill opacity=0.20] (191.85, 64.67) circle (  2.13);

\path[fill=fillColor,fill opacity=0.20] (194.47, 62.80) circle (  2.13);

\path[fill=fillColor,fill opacity=0.20] (196.66, 57.82) circle (  2.13);

\path[fill=fillColor,fill opacity=0.20] (189.67, 56.05) circle (  2.13);

\path[fill=fillColor,fill opacity=0.20] (171.97, 64.88) circle (  2.13);

\path[fill=fillColor,fill opacity=0.20] (193.16, 69.55) circle (  2.13);

\path[fill=fillColor,fill opacity=0.20] (224.63, 61.87) circle (  2.13);

\path[fill=fillColor,fill opacity=0.20] (207.37, 64.57) circle (  2.13);

\path[fill=fillColor,fill opacity=0.20] (206.05, 72.56) circle (  2.13);

\path[fill=fillColor,fill opacity=0.20] (205.18, 69.24) circle (  2.13);

\path[fill=fillColor,fill opacity=0.20] (207.15, 63.42) circle (  2.13);

\path[fill=fillColor,fill opacity=0.20] (211.74, 62.70) circle (  2.13);

\path[fill=fillColor,fill opacity=0.20] (213.05, 57.20) circle (  2.13);

\path[fill=fillColor,fill opacity=0.20] (216.98, 50.14) circle (  2.13);

\path[fill=fillColor,fill opacity=0.20] (217.42, 50.03) circle (  2.13);

\path[fill=fillColor,fill opacity=0.20] (211.30, 52.32) circle (  2.13);

\path[fill=fillColor,fill opacity=0.20] (209.55, 55.22) circle (  2.13);

\path[fill=fillColor,fill opacity=0.20] (209.11, 58.75) circle (  2.13);

\path[fill=fillColor,fill opacity=0.20] (204.31, 58.34) circle (  2.13);

\path[fill=fillColor,fill opacity=0.20] (189.89, 62.39) circle (  2.13);

\path[fill=fillColor,fill opacity=0.20] (206.27, 66.12) circle (  2.13);

\path[fill=fillColor,fill opacity=0.20] (217.42, 58.55) circle (  2.13);

\path[fill=fillColor,fill opacity=0.20] (203.65, 59.17) circle (  2.13);

\path[fill=fillColor,fill opacity=0.20] (203.65, 54.70) circle (  2.13);

\path[fill=fillColor,fill opacity=0.20] (199.50, 54.39) circle (  2.13);

\path[fill=fillColor,fill opacity=0.20] (196.44, 58.34) circle (  2.13);

\path[fill=fillColor,fill opacity=0.20] (197.97, 60.93) circle (  2.13);

\path[fill=fillColor,fill opacity=0.20] (194.47, 68.41) circle (  2.13);

\path[fill=fillColor,fill opacity=0.20] (194.04, 73.18) circle (  2.13);

\path[fill=fillColor,fill opacity=0.20] (198.19, 70.59) circle (  2.13);

\path[fill=fillColor,fill opacity=0.20] (190.98, 64.36) circle (  2.13);

\path[fill=fillColor,fill opacity=0.20] (190.98, 68.93) circle (  2.13);

\path[fill=fillColor,fill opacity=0.20] (184.86, 70.80) circle (  2.13);

\path[fill=fillColor,fill opacity=0.20] (180.49, 63.84) circle (  2.13);

\path[fill=fillColor,fill opacity=0.20] (179.83, 61.04) circle (  2.13);

\path[fill=fillColor,fill opacity=0.20] (181.15, 67.79) circle (  2.13);

\path[fill=fillColor,fill opacity=0.20] (176.12, 71.00) circle (  2.13);

\path[fill=fillColor,fill opacity=0.20] (176.34, 67.89) circle (  2.13);

\path[fill=fillColor,fill opacity=0.20] (185.52, 70.07) circle (  2.13);

\path[fill=fillColor,fill opacity=0.20] (179.40, 76.09) circle (  2.13);

\path[fill=fillColor,fill opacity=0.20] (180.05, 75.16) circle (  2.13);

\path[fill=fillColor,fill opacity=0.20] (180.27, 73.39) circle (  2.13);

\path[fill=fillColor,fill opacity=0.20] (180.93, 77.13) circle (  2.13);

\path[fill=fillColor,fill opacity=0.20] (180.05, 76.40) circle (  2.13);

\path[fill=fillColor,fill opacity=0.20] (186.61, 66.44) circle (  2.13);

\path[fill=fillColor,fill opacity=0.20] (196.00, 64.36) circle (  2.13);

\path[fill=fillColor,fill opacity=0.20] (187.70, 71.11) circle (  2.13);

\path[fill=fillColor,fill opacity=0.20] (189.23, 68.82) circle (  2.13);

\path[fill=fillColor,fill opacity=0.20] (191.85, 63.11) circle (  2.13);

\path[fill=fillColor,fill opacity=0.20] (194.47, 64.36) circle (  2.13);

\path[fill=fillColor,fill opacity=0.20] (200.15, 65.19) circle (  2.13);

\path[fill=fillColor,fill opacity=0.20] (198.19, 66.02) circle (  2.13);

\path[fill=fillColor,fill opacity=0.20] (197.53, 68.51) circle (  2.13);

\path[fill=fillColor,fill opacity=0.20] (206.93, 64.36) circle (  2.13);

\path[fill=fillColor,fill opacity=0.20] (208.46, 59.89) circle (  2.13);

\path[fill=fillColor,fill opacity=0.20] (211.74, 58.34) circle (  2.13);

\path[fill=fillColor,fill opacity=0.20] (215.67, 52.73) circle (  2.13);

\path[fill=fillColor,fill opacity=0.20] (215.45, 48.06) circle (  2.13);

\path[fill=fillColor,fill opacity=0.20] (213.70, 47.02) circle (  2.13);

\path[fill=fillColor,fill opacity=0.20] (217.20, 46.61) circle (  2.13);

\path[fill=fillColor,fill opacity=0.20] (206.93, 47.85) circle (  2.13);

\path[fill=fillColor,fill opacity=0.20] (208.46, 48.89) circle (  2.13);

\path[fill=fillColor,fill opacity=0.20] (204.09, 50.97) circle (  2.13);

\path[fill=fillColor,fill opacity=0.20] (196.00, 62.59) circle (  2.13);

\path[fill=fillColor,fill opacity=0.20] (190.76, 77.23) circle (  2.13);

\path[fill=fillColor,fill opacity=0.20] (202.12, 69.45) circle (  2.13);

\path[fill=fillColor,fill opacity=0.20] (212.39, 58.13) circle (  2.13);

\path[fill=fillColor,fill opacity=0.20] (205.18, 52.21) circle (  2.13);

\path[fill=fillColor,fill opacity=0.20] (204.74, 51.69) circle (  2.13);

\path[fill=fillColor,fill opacity=0.20] (203.65, 54.70) circle (  2.13);

\path[fill=fillColor,fill opacity=0.20] (200.59, 63.53) circle (  2.13);

\path[fill=fillColor,fill opacity=0.20] (195.78, 71.73) circle (  2.13);

\path[fill=fillColor,fill opacity=0.20] (199.50, 71.32) circle (  2.13);

\path[fill=fillColor,fill opacity=0.20] (199.50, 69.97) circle (  2.13);

\path[fill=fillColor,fill opacity=0.20] (197.31, 72.25) circle (  2.13);

\path[fill=fillColor,fill opacity=0.20] (196.66, 74.12) circle (  2.13);

\path[fill=fillColor,fill opacity=0.20] (196.66, 64.88) circle (  2.13);

\path[fill=fillColor,fill opacity=0.20] (205.40, 54.91) circle (  2.13);

\path[fill=fillColor,fill opacity=0.20] (198.84, 55.95) circle (  2.13);

\path[fill=fillColor,fill opacity=0.20] (194.26, 57.51) circle (  2.13);

\path[fill=fillColor,fill opacity=0.20] (198.41, 53.67) circle (  2.13);

\path[fill=fillColor,fill opacity=0.20] (200.59, 55.02) circle (  2.13);

\path[fill=fillColor,fill opacity=0.20] (201.25, 63.63) circle (  2.13);

\path[fill=fillColor,fill opacity=0.20] (198.84, 66.64) circle (  2.13);

\path[fill=fillColor,fill opacity=0.20] (200.81, 58.44) circle (  2.13);

\path[fill=fillColor,fill opacity=0.20] (199.72, 49.41) circle (  2.13);

\path[fill=fillColor,fill opacity=0.20] (196.44, 52.00) circle (  2.13);

\path[fill=fillColor,fill opacity=0.20] (197.97, 59.58) circle (  2.13);

\path[fill=fillColor,fill opacity=0.20] (199.28, 58.55) circle (  2.13);

\path[fill=fillColor,fill opacity=0.20] (201.90, 56.78) circle (  2.13);

\path[fill=fillColor,fill opacity=0.20] (200.37, 57.61) circle (  2.13);

\path[fill=fillColor,fill opacity=0.20] (202.78, 52.42) circle (  2.13);

\path[fill=fillColor,fill opacity=0.20] (203.43, 52.52) circle (  2.13);

\path[fill=fillColor,fill opacity=0.20] (206.71, 59.27) circle (  2.13);

\path[fill=fillColor,fill opacity=0.20] (206.93, 55.95) circle (  2.13);

\path[fill=fillColor,fill opacity=0.20] (209.11, 50.97) circle (  2.13);

\path[fill=fillColor,fill opacity=0.20] (208.02, 55.95) circle (  2.13);

\path[fill=fillColor,fill opacity=0.20] (209.99, 57.09) circle (  2.13);

\path[fill=fillColor,fill opacity=0.20] (212.83, 50.14) circle (  2.13);

\path[fill=fillColor,fill opacity=0.20] (214.14, 46.19) circle (  2.13);

\path[fill=fillColor,fill opacity=0.20] (215.67, 50.03) circle (  2.13);

\path[fill=fillColor,fill opacity=0.20] (209.33, 56.99) circle (  2.13);

\path[fill=fillColor,fill opacity=0.20] (204.09, 60.52) circle (  2.13);

\path[fill=fillColor,fill opacity=0.20] (201.47, 63.11) circle (  2.13);

\path[fill=fillColor,fill opacity=0.20] (189.67, 67.68) circle (  2.13);

\path[fill=fillColor,fill opacity=0.20] (187.70, 69.65) circle (  2.13);

\path[fill=fillColor,fill opacity=0.20] (182.46, 68.62) circle (  2.13);

\path[fill=fillColor,fill opacity=0.20] (205.40, 54.29) circle (  2.13);

\path[fill=fillColor,fill opacity=0.20] (212.83, 52.52) circle (  2.13);

\path[fill=fillColor,fill opacity=0.20] (204.31, 54.60) circle (  2.13);

\path[fill=fillColor,fill opacity=0.20] (205.62, 56.99) circle (  2.13);

\path[fill=fillColor,fill opacity=0.20] (204.31, 56.26) circle (  2.13);

\path[fill=fillColor,fill opacity=0.20] (203.43, 51.69) circle (  2.13);

\path[fill=fillColor,fill opacity=0.20] (199.06, 54.29) circle (  2.13);

\path[fill=fillColor,fill opacity=0.20] (198.84, 61.14) circle (  2.13);

\path[fill=fillColor,fill opacity=0.20] (197.31, 63.94) circle (  2.13);

\path[fill=fillColor,fill opacity=0.20] (204.96, 66.85) circle (  2.13);

\path[fill=fillColor,fill opacity=0.20] (199.06, 67.37) circle (  2.13);

\path[fill=fillColor,fill opacity=0.20] (201.25, 63.22) circle (  2.13);

\path[fill=fillColor,fill opacity=0.20] (197.97, 57.51) circle (  2.13);

\path[fill=fillColor,fill opacity=0.20] (201.25, 55.85) circle (  2.13);

\path[fill=fillColor,fill opacity=0.20] (203.00, 62.08) circle (  2.13);

\path[fill=fillColor,fill opacity=0.20] (200.59, 67.37) circle (  2.13);

\path[fill=fillColor,fill opacity=0.20] (199.72, 62.28) circle (  2.13);

\path[fill=fillColor,fill opacity=0.20] (199.50, 56.88) circle (  2.13);

\path[fill=fillColor,fill opacity=0.20] (206.49, 54.18) circle (  2.13);

\path[fill=fillColor,fill opacity=0.20] (199.50, 53.67) circle (  2.13);

\path[fill=fillColor,fill opacity=0.20] (201.03, 58.65) circle (  2.13);

\path[fill=fillColor,fill opacity=0.20] (201.47, 60.62) circle (  2.13);

\path[fill=fillColor,fill opacity=0.20] (203.87, 55.22) circle (  2.13);

\path[fill=fillColor,fill opacity=0.20] (208.02, 48.68) circle (  2.13);

\path[fill=fillColor,fill opacity=0.20] (208.02, 45.57) circle (  2.13);

\path[fill=fillColor,fill opacity=0.20] (208.68, 48.37) circle (  2.13);

\path[fill=fillColor,fill opacity=0.20] (208.89, 54.50) circle (  2.13);

\path[fill=fillColor,fill opacity=0.20] (208.46, 56.36) circle (  2.13);

\path[fill=fillColor,fill opacity=0.20] (205.18, 56.78) circle (  2.13);

\path[fill=fillColor,fill opacity=0.20] (200.81, 62.28) circle (  2.13);

\path[fill=fillColor,fill opacity=0.20] (199.28, 67.27) circle (  2.13);

\path[fill=fillColor,fill opacity=0.20] (199.72, 69.45) circle (  2.13);

\path[fill=fillColor,fill opacity=0.20] (195.78, 75.47) circle (  2.13);

\path[fill=fillColor,fill opacity=0.20] (187.70, 84.81) circle (  2.13);

\path[fill=fillColor,fill opacity=0.20] (179.83, 90.00) circle (  2.13);

\path[fill=fillColor,fill opacity=0.20] (178.52, 90.00) circle (  2.13);

\path[fill=fillColor,fill opacity=0.20] (198.19, 60.52) circle (  2.13);

\path[fill=fillColor,fill opacity=0.20] (201.47, 58.03) circle (  2.13);

\path[fill=fillColor,fill opacity=0.20] (202.78, 60.10) circle (  2.13);

\path[fill=fillColor,fill opacity=0.20] (206.27, 54.39) circle (  2.13);

\path[fill=fillColor,fill opacity=0.20] (200.37, 41.00) circle (  2.13);

\path[fill=fillColor,fill opacity=0.20] (201.68, 38.30) circle (  2.13);

\path[fill=fillColor,fill opacity=0.20] (201.90, 44.63) circle (  2.13);

\path[fill=fillColor,fill opacity=0.20] (204.96, 51.28) circle (  2.13);

\path[fill=fillColor,fill opacity=0.20] (203.00, 64.77) circle (  2.13);

\path[fill=fillColor,fill opacity=0.20] (201.68, 69.03) circle (  2.13);

\path[fill=fillColor,fill opacity=0.20] (202.34, 59.69) circle (  2.13);

\path[fill=fillColor,fill opacity=0.20] (202.12, 51.90) circle (  2.13);

\path[fill=fillColor,fill opacity=0.20] (202.34, 52.32) circle (  2.13);

\path[fill=fillColor,fill opacity=0.20] (200.81, 57.61) circle (  2.13);

\path[fill=fillColor,fill opacity=0.20] (204.74, 60.00) circle (  2.13);

\path[fill=fillColor,fill opacity=0.20] (198.41, 52.42) circle (  2.13);

\path[fill=fillColor,fill opacity=0.20] (199.72, 49.20) circle (  2.13);

\path[fill=fillColor,fill opacity=0.20] (204.74, 53.98) circle (  2.13);

\path[fill=fillColor,fill opacity=0.20] (202.34, 56.88) circle (  2.13);

\path[fill=fillColor,fill opacity=0.20] (204.74, 58.65) circle (  2.13);

\path[fill=fillColor,fill opacity=0.20] (203.65, 58.86) circle (  2.13);

\path[fill=fillColor,fill opacity=0.20] (204.31, 55.64) circle (  2.13);

\path[fill=fillColor,fill opacity=0.20] (208.46, 54.08) circle (  2.13);

\path[fill=fillColor,fill opacity=0.20] (201.90, 59.69) circle (  2.13);

\path[fill=fillColor,fill opacity=0.20] (198.84, 66.95) circle (  2.13);

\path[fill=fillColor,fill opacity=0.20] (193.60, 74.85) circle (  2.13);

\path[fill=fillColor,fill opacity=0.20] (187.92, 75.57) circle (  2.13);

\path[fill=fillColor,fill opacity=0.20] (194.04, 60.41) circle (  2.13);

\path[fill=fillColor,fill opacity=0.20] (200.37, 55.85) circle (  2.13);

\path[fill=fillColor,fill opacity=0.20] (202.12, 54.91) circle (  2.13);

\path[fill=fillColor,fill opacity=0.20] (209.11, 57.09) circle (  2.13);

\path[fill=fillColor,fill opacity=0.20] (205.18, 63.84) circle (  2.13);

\path[fill=fillColor,fill opacity=0.20] (204.74, 63.11) circle (  2.13);

\path[fill=fillColor,fill opacity=0.20] (204.52, 56.78) circle (  2.13);

\path[fill=fillColor,fill opacity=0.20] (203.43, 53.77) circle (  2.13);

\path[fill=fillColor,fill opacity=0.20] (200.37, 52.73) circle (  2.13);

\path[fill=fillColor,fill opacity=0.20] (200.59, 54.91) circle (  2.13);

\path[fill=fillColor,fill opacity=0.20] (197.10, 57.51) circle (  2.13);

\path[fill=fillColor,fill opacity=0.20] (198.84, 53.56) circle (  2.13);

\path[fill=fillColor,fill opacity=0.20] (199.50, 51.90) circle (  2.13);

\path[fill=fillColor,fill opacity=0.20] (199.94, 58.44) circle (  2.13);

\path[fill=fillColor,fill opacity=0.20] (197.31, 65.29) circle (  2.13);

\path[fill=fillColor,fill opacity=0.20] (198.84, 70.38) circle (  2.13);

\path[fill=fillColor,fill opacity=0.20] (191.20, 79.31) circle (  2.13);

\path[fill=fillColor,fill opacity=0.20] (191.63, 88.96) circle (  2.13);

\path[fill=fillColor,fill opacity=0.20] (191.41, 76.09) circle (  2.13);

\path[fill=fillColor,fill opacity=0.20] (192.94, 69.13) circle (  2.13);

\path[fill=fillColor,fill opacity=0.20] (191.20, 71.63) circle (  2.13);

\path[fill=fillColor,fill opacity=0.20] (190.98, 69.76) circle (  2.13);

\path[fill=fillColor,fill opacity=0.20] (250.19, 81.39) circle (  2.13);

\path[fill=fillColor,fill opacity=0.20] (237.52, 66.64) circle (  2.13);

\path[fill=fillColor,fill opacity=0.20] (240.36, 83.77) circle (  2.13);

\path[fill=fillColor,fill opacity=0.20] (240.14, 74.64) circle (  2.13);

\path[fill=fillColor,fill opacity=0.20] (238.17, 72.04) circle (  2.13);

\path[fill=fillColor,fill opacity=0.20] (239.70, 69.86) circle (  2.13);

\path[fill=fillColor,fill opacity=0.20] (239.92, 54.08) circle (  2.13);

\path[fill=fillColor,fill opacity=0.20] (214.14, 58.44) circle (  2.13);

\path[fill=fillColor,fill opacity=0.20] (235.55, 44.32) circle (  2.13);

\path[fill=fillColor,fill opacity=0.20] (234.24, 50.86) circle (  2.13);

\path[fill=fillColor,fill opacity=0.20] (243.20, 57.92) circle (  2.13);

\path[fill=fillColor,fill opacity=0.20] (247.79, 49.51) circle (  2.13);

\path[fill=fillColor,fill opacity=0.20] (250.85, 54.60) circle (  2.13);

\path[fill=fillColor,fill opacity=0.20] (258.28, 58.13) circle (  2.13);

\path[fill=fillColor,fill opacity=0.20] (260.24, 54.50) circle (  2.13);

\path[fill=fillColor,fill opacity=0.20] (230.96, 62.80) circle (  2.13);

\path[fill=fillColor,fill opacity=0.20] (196.44, 79.31) circle (  2.13);

\path[fill=fillColor,fill opacity=0.20] (266.14, 56.68) circle (  2.13);

\path[fill=fillColor,fill opacity=0.20] (253.03, 56.88) circle (  2.13);

\path[fill=fillColor,fill opacity=0.20] (256.75, 39.96) circle (  2.13);

\path[fill=fillColor,fill opacity=0.20] (251.72, 44.63) circle (  2.13);

\path[fill=fillColor,fill opacity=0.20] (256.96, 50.03) circle (  2.13);

\path[fill=fillColor,fill opacity=0.20] (252.81, 60.41) circle (  2.13);

\path[fill=fillColor,fill opacity=0.20] (200.81, 68.41) circle (  2.13);

\path[fill=fillColor,fill opacity=0.20] (242.33, 45.98) circle (  2.13);

\path[fill=fillColor,fill opacity=0.20] (269.86, 49.93) circle (  2.13);

\path[fill=fillColor,fill opacity=0.20] (243.85, 53.67) circle (  2.13);

\path[fill=fillColor,fill opacity=0.20] (217.63, 51.07) circle (  2.13);

\path[fill=fillColor,fill opacity=0.20] (231.40, 43.70) circle (  2.13);

\path[fill=fillColor,fill opacity=0.20] (261.12, 54.39) circle (  2.13);

\path[fill=fillColor,fill opacity=0.20] (253.25, 57.20) circle (  2.13);

\path[fill=fillColor,fill opacity=0.20] (247.57, 59.48) circle (  2.13);

\path[fill=fillColor,fill opacity=0.20] (256.09, 58.34) circle (  2.13);

\path[fill=fillColor,fill opacity=0.20] (264.83, 55.53) circle (  2.13);

\path[fill=fillColor,fill opacity=0.20] (259.15, 51.59) circle (  2.13);

\path[fill=fillColor,fill opacity=0.20] (222.88, 49.10) circle (  2.13);

\path[fill=fillColor,fill opacity=0.20] (194.47, 61.56) circle (  2.13);

\path[fill=fillColor,fill opacity=0.20] (232.06, 77.65) circle (  2.13);

\path[fill=fillColor,fill opacity=0.20] (229.87, 59.69) circle (  2.13);

\path[fill=fillColor,fill opacity=0.20] (217.85, 60.73) circle (  2.13);

\path[fill=fillColor,fill opacity=0.20] (207.80, 52.32) circle (  2.13);

\path[fill=fillColor,fill opacity=0.20] (203.65, 49.72) circle (  2.13);

\path[fill=fillColor,fill opacity=0.20] (203.43, 58.13) circle (  2.13);

\path[fill=fillColor,fill opacity=0.20] (196.44, 64.77) circle (  2.13);

\path[fill=fillColor,fill opacity=0.20] (172.19, 46.19) circle (  2.13);

\path[fill=fillColor,fill opacity=0.20] (216.54, 62.70) circle (  2.13);

\path[fill=fillColor,fill opacity=0.20] (232.06, 57.92) circle (  2.13);

\path[fill=fillColor,fill opacity=0.20] (235.77, 62.18) circle (  2.13);

\path[fill=fillColor,fill opacity=0.20] (240.36, 60.93) circle (  2.13);

\path[fill=fillColor,fill opacity=0.20] (244.07, 59.58) circle (  2.13);

\path[fill=fillColor,fill opacity=0.20] (251.94, 61.45) circle (  2.13);

\path[fill=fillColor,fill opacity=0.20] (256.31, 52.21) circle (  2.13);

\path[fill=fillColor,fill opacity=0.20] (246.48, 42.66) circle (  2.13);

\path[fill=fillColor,fill opacity=0.20] (207.58, 47.02) circle (  2.13);

\path[fill=fillColor,fill opacity=0.20] (175.03, 58.13) circle (  2.13);

\path[fill=fillColor,fill opacity=0.20] (225.06, 74.74) circle (  2.13);

\path[fill=fillColor,fill opacity=0.20] (228.12, 59.17) circle (  2.13);

\path[fill=fillColor,fill opacity=0.20] (228.78, 53.46) circle (  2.13);

\path[fill=fillColor,fill opacity=0.20] (214.58, 58.13) circle (  2.13);

\path[fill=fillColor,fill opacity=0.20] (207.58, 57.40) circle (  2.13);

\path[fill=fillColor,fill opacity=0.20] (215.89, 50.76) circle (  2.13);

\path[fill=fillColor,fill opacity=0.20] (208.89, 57.82) circle (  2.13);

\path[fill=fillColor,fill opacity=0.20] (180.05, 62.28) circle (  2.13);

\path[fill=fillColor,fill opacity=0.20] (217.85, 57.09) circle (  2.13);

\path[fill=fillColor,fill opacity=0.20] (233.80, 62.28) circle (  2.13);

\path[fill=fillColor,fill opacity=0.20] (260.46, 62.39) circle (  2.13);

\path[fill=fillColor,fill opacity=0.20] (252.16, 55.33) circle (  2.13);

\path[fill=fillColor,fill opacity=0.20] (247.79, 53.56) circle (  2.13);

\path[fill=fillColor,fill opacity=0.20] (253.69, 53.04) circle (  2.13);

\path[fill=fillColor,fill opacity=0.20] (255.65, 55.02) circle (  2.13);

\path[fill=fillColor,fill opacity=0.20] (228.12, 49.72) circle (  2.13);

\path[fill=fillColor,fill opacity=0.20] (198.84, 57.40) circle (  2.13);

\path[fill=fillColor,fill opacity=0.20] (220.69, 60.10) circle (  2.13);

\path[fill=fillColor,fill opacity=0.20] (226.59, 52.11) circle (  2.13);

\path[fill=fillColor,fill opacity=0.20] (229.00, 68.72) circle (  2.13);

\path[fill=fillColor,fill opacity=0.20] (222.88, 70.28) circle (  2.13);

\path[fill=fillColor,fill opacity=0.20] (218.29, 71.21) circle (  2.13);

\path[fill=fillColor,fill opacity=0.20] (216.11, 70.59) circle (  2.13);

\path[fill=fillColor,fill opacity=0.20] (213.48, 58.96) circle (  2.13);

\path[fill=fillColor,fill opacity=0.20] (210.86, 60.31) circle (  2.13);

\path[fill=fillColor,fill opacity=0.20] (201.25, 72.56) circle (  2.13);

\path[fill=fillColor,fill opacity=0.20] (217.63, 54.08) circle (  2.13);

\path[fill=fillColor,fill opacity=0.20] (226.16, 37.99) circle (  2.13);

\path[fill=fillColor,fill opacity=0.20] (242.33, 53.77) circle (  2.13);

\path[fill=fillColor,fill opacity=0.20] (258.06, 48.27) circle (  2.13);

\path[fill=fillColor,fill opacity=0.20] (271.39, 53.87) circle (  2.13);

\path[fill=fillColor,fill opacity=0.20] (258.28, 53.15) circle (  2.13);

\path[fill=fillColor,fill opacity=0.20] (251.28, 46.81) circle (  2.13);

\path[fill=fillColor,fill opacity=0.20] (209.77, 52.32) circle (  2.13);

\path[fill=fillColor,fill opacity=0.20] (212.17, 68.82) circle (  2.13);

\path[fill=fillColor,fill opacity=0.20] (221.57, 52.84) circle (  2.13);

\path[fill=fillColor,fill opacity=0.20] (242.33, 55.74) circle (  2.13);

\path[fill=fillColor,fill opacity=0.20] (233.80, 60.41) circle (  2.13);

\path[fill=fillColor,fill opacity=0.20] (222.88, 66.33) circle (  2.13);

\path[fill=fillColor,fill opacity=0.20] (216.98, 76.09) circle (  2.13);

\path[fill=fillColor,fill opacity=0.20] (213.26, 73.08) circle (  2.13);

\path[fill=fillColor,fill opacity=0.20] (211.95, 60.52) circle (  2.13);

\path[fill=fillColor,fill opacity=0.20] (207.15, 60.21) circle (  2.13);

\path[fill=fillColor,fill opacity=0.20] (186.39, 65.81) circle (  2.13);

\path[fill=fillColor,fill opacity=0.20] (235.33, 65.50) circle (  2.13);

\path[fill=fillColor,fill opacity=0.20] (228.12, 63.63) circle (  2.13);

\path[fill=fillColor,fill opacity=0.20] (233.59, 59.17) circle (  2.13);

\path[fill=fillColor,fill opacity=0.20] (242.33, 70.38) circle (  2.13);

\path[fill=fillColor,fill opacity=0.20] (252.16, 61.24) circle (  2.13);

\path[fill=fillColor,fill opacity=0.20] (249.75, 50.34) circle (  2.13);

\path[fill=fillColor,fill opacity=0.20] (260.24, 59.17) circle (  2.13);

\path[fill=fillColor,fill opacity=0.20] (221.79, 45.67) circle (  2.13);

\path[fill=fillColor,fill opacity=0.20] (185.08, 48.16) circle (  2.13);

\path[fill=fillColor,fill opacity=0.20] (204.74, 64.15) circle (  2.13);

\path[fill=fillColor,fill opacity=0.20] (227.90, 54.50) circle (  2.13);

\path[fill=fillColor,fill opacity=0.20] (235.11, 62.39) circle (  2.13);

\path[fill=fillColor,fill opacity=0.20] (224.19, 55.33) circle (  2.13);

\path[fill=fillColor,fill opacity=0.20] (219.38, 53.56) circle (  2.13);

\path[fill=fillColor,fill opacity=0.20] (209.77, 65.92) circle (  2.13);

\path[fill=fillColor,fill opacity=0.20] (204.09, 63.11) circle (  2.13);

\path[fill=fillColor,fill opacity=0.20] (204.09, 54.08) circle (  2.13);

\path[fill=fillColor,fill opacity=0.20] (201.25, 56.47) circle (  2.13);

\path[fill=fillColor,fill opacity=0.20] (168.47, 60.62) circle (  2.13);

\path[fill=fillColor,fill opacity=0.20] (214.36, 73.60) circle (  2.13);

\path[fill=fillColor,fill opacity=0.20] (219.38, 60.83) circle (  2.13);

\path[fill=fillColor,fill opacity=0.20] (223.32, 70.90) circle (  2.13);

\path[fill=fillColor,fill opacity=0.20] (230.09, 77.23) circle (  2.13);

\path[fill=fillColor,fill opacity=0.20] (233.59, 69.03) circle (  2.13);

\path[fill=fillColor,fill opacity=0.20] (240.80, 58.65) circle (  2.13);

\path[fill=fillColor,fill opacity=0.20] (249.10, 54.50) circle (  2.13);

\path[fill=fillColor,fill opacity=0.20] (240.36, 56.47) circle (  2.13);

\path[fill=fillColor,fill opacity=0.20] (242.54, 65.50) circle (  2.13);

\path[fill=fillColor,fill opacity=0.20] (246.48, 65.19) circle (  2.13);

\path[fill=fillColor,fill opacity=0.20] (196.00, 56.57) circle (  2.13);

\path[fill=fillColor,fill opacity=0.20] (197.10, 70.38) circle (  2.13);

\path[fill=fillColor,fill opacity=0.20] (218.51, 52.94) circle (  2.13);

\path[fill=fillColor,fill opacity=0.20] (235.11, 61.76) circle (  2.13);

\path[fill=fillColor,fill opacity=0.20] (227.69, 62.18) circle (  2.13);

\path[fill=fillColor,fill opacity=0.20] (218.95, 53.46) circle (  2.13);

\path[fill=fillColor,fill opacity=0.20] (218.95, 57.30) circle (  2.13);

\path[fill=fillColor,fill opacity=0.20] (206.93, 56.05) circle (  2.13);

\path[fill=fillColor,fill opacity=0.20] (201.68, 47.44) circle (  2.13);

\path[fill=fillColor,fill opacity=0.20] (191.41, 56.36) circle (  2.13);

\path[fill=fillColor,fill opacity=0.20] (217.42, 64.67) circle (  2.13);

\path[fill=fillColor,fill opacity=0.20] (230.09, 64.26) circle (  2.13);

\path[fill=fillColor,fill opacity=0.20] (225.28, 61.56) circle (  2.13);

\path[fill=fillColor,fill opacity=0.20] (223.97, 79.93) circle (  2.13);

\path[fill=fillColor,fill opacity=0.20] (235.11, 80.04) circle (  2.13);

\path[fill=fillColor,fill opacity=0.20] (237.74, 58.23) circle (  2.13);

\path[fill=fillColor,fill opacity=0.20] (235.55, 53.98) circle (  2.13);

\path[fill=fillColor,fill opacity=0.20] (233.37, 59.38) circle (  2.13);

\path[fill=fillColor,fill opacity=0.20] (234.90, 60.83) circle (  2.13);

\path[fill=fillColor,fill opacity=0.20] (197.53, 63.74) circle (  2.13);

\path[fill=fillColor,fill opacity=0.20] (169.78, 68.82) circle (  2.13);

\path[fill=fillColor,fill opacity=0.20] (202.12, 54.39) circle (  2.13);

\path[fill=fillColor,fill opacity=0.20] (237.96, 53.87) circle (  2.13);

\path[fill=fillColor,fill opacity=0.20] (241.89, 60.93) circle (  2.13);

\path[fill=fillColor,fill opacity=0.20] (234.24, 54.81) circle (  2.13);

\path[fill=fillColor,fill opacity=0.20] (232.71, 53.98) circle (  2.13);

\path[fill=fillColor,fill opacity=0.20] (231.40, 62.49) circle (  2.13);

\path[fill=fillColor,fill opacity=0.20] (214.36, 58.96) circle (  2.13);

\path[fill=fillColor,fill opacity=0.20] (203.65, 57.71) circle (  2.13);

\path[fill=fillColor,fill opacity=0.20] (216.76, 68.62) circle (  2.13);

\path[fill=fillColor,fill opacity=0.20] (245.60, 67.79) circle (  2.13);

\path[fill=fillColor,fill opacity=0.20] (245.17, 55.64) circle (  2.13);

\path[fill=fillColor,fill opacity=0.20] (233.15, 62.08) circle (  2.13);

\path[fill=fillColor,fill opacity=0.20] (230.31, 78.17) circle (  2.13);

\path[fill=fillColor,fill opacity=0.20] (234.46, 83.77) circle (  2.13);

\path[fill=fillColor,fill opacity=0.20] (234.90, 71.42) circle (  2.13);

\path[fill=fillColor,fill opacity=0.20] (227.90, 62.18) circle (  2.13);

\path[fill=fillColor,fill opacity=0.20] (216.98, 54.08) circle (  2.13);

\path[fill=fillColor,fill opacity=0.20] (191.85, 47.12) circle (  2.13);

\path[fill=fillColor,fill opacity=0.20] (167.38, 60.52) circle (  2.13);

\path[fill=fillColor,fill opacity=0.20] (212.61, 51.69) circle (  2.13);

\path[fill=fillColor,fill opacity=0.20] (227.03, 54.50) circle (  2.13);

\path[fill=fillColor,fill opacity=0.20] (231.62, 51.38) circle (  2.13);

\path[fill=fillColor,fill opacity=0.20] (244.73, 50.76) circle (  2.13);

\path[fill=fillColor,fill opacity=0.20] (240.58, 66.33) circle (  2.13);

\path[fill=fillColor,fill opacity=0.20] (223.10, 75.36) circle (  2.13);

\path[fill=fillColor,fill opacity=0.20] (216.76, 64.88) circle (  2.13);

\path[fill=fillColor,fill opacity=0.20] (201.47, 84.81) circle (  2.13);

\path[fill=fillColor,fill opacity=0.20] (212.83, 64.26) circle (  2.13);

\path[fill=fillColor,fill opacity=0.20] (245.38, 73.50) circle (  2.13);

\path[fill=fillColor,fill opacity=0.20] (248.01, 68.72) circle (  2.13);

\path[fill=fillColor,fill opacity=0.20] (237.52, 65.29) circle (  2.13);

\path[fill=fillColor,fill opacity=0.20] (231.62, 61.97) circle (  2.13);

\path[fill=fillColor,fill opacity=0.20] (221.57, 68.51) circle (  2.13);

\path[fill=fillColor,fill opacity=0.20] (211.08, 67.27) circle (  2.13);

\path[fill=fillColor,fill opacity=0.20] (204.52, 59.89) circle (  2.13);

\path[fill=fillColor,fill opacity=0.20] (189.23, 52.63) circle (  2.13);

\path[fill=fillColor,fill opacity=0.20] (198.84, 66.44) circle (  2.13);

\path[fill=fillColor,fill opacity=0.20] (213.05, 56.99) circle (  2.13);

\path[fill=fillColor,fill opacity=0.20] (224.19, 51.49) circle (  2.13);

\path[fill=fillColor,fill opacity=0.20] (231.62, 56.78) circle (  2.13);

\path[fill=fillColor,fill opacity=0.20] (226.81, 65.09) circle (  2.13);

\path[fill=fillColor,fill opacity=0.20] (221.57, 70.38) circle (  2.13);

\path[fill=fillColor,fill opacity=0.20] (216.98, 71.42) circle (  2.13);

\path[fill=fillColor,fill opacity=0.20] (206.27, 79.83) circle (  2.13);

\path[fill=fillColor,fill opacity=0.20] (220.04, 57.20) circle (  2.13);

\path[fill=fillColor,fill opacity=0.20] (232.93, 65.81) circle (  2.13);

\path[fill=fillColor,fill opacity=0.20] (265.49, 73.91) circle (  2.13);

\path[fill=fillColor,fill opacity=0.20] (260.46, 69.65) circle (  2.13);

\path[fill=fillColor,fill opacity=0.20] (228.78, 60.10) circle (  2.13);

\path[fill=fillColor,fill opacity=0.20] (203.00, 53.25) circle (  2.13);

\path[fill=fillColor,fill opacity=0.20] (185.73, 52.42) circle (  2.13);

\path[fill=fillColor,fill opacity=0.20] (183.77, 62.59) circle (  2.13);

\path[fill=fillColor,fill opacity=0.20] (185.08, 60.93) circle (  2.13);

\path[fill=fillColor,fill opacity=0.20] (181.80, 53.46) circle (  2.13);

\path[fill=fillColor,fill opacity=0.20] (206.05, 67.37) circle (  2.13);

\path[fill=fillColor,fill opacity=0.20] (220.04, 51.49) circle (  2.13);

\path[fill=fillColor,fill opacity=0.20] (223.10, 58.86) circle (  2.13);

\path[fill=fillColor,fill opacity=0.20] (224.85, 62.39) circle (  2.13);

\path[fill=fillColor,fill opacity=0.20] (223.97, 58.75) circle (  2.13);

\path[fill=fillColor,fill opacity=0.20] (220.04, 65.29) circle (  2.13);

\path[fill=fillColor,fill opacity=0.20] (211.95, 75.88) circle (  2.13);

\path[fill=fillColor,fill opacity=0.20] (210.64, 79.21) circle (  2.13);

\path[fill=fillColor,fill opacity=0.20] (230.96, 78.27) circle (  2.13);

\path[fill=fillColor,fill opacity=0.20] (254.56, 78.58) circle (  2.13);

\path[fill=fillColor,fill opacity=0.20] (249.10, 68.62) circle (  2.13);

\path[fill=fillColor,fill opacity=0.20] (251.50, 58.65) circle (  2.13);

\path[fill=fillColor,fill opacity=0.20] (203.43, 45.78) circle (  2.13);

\path[fill=fillColor,fill opacity=0.20] (201.25, 44.53) circle (  2.13);

\path[fill=fillColor,fill opacity=0.20] (168.04, 60.83) circle (  2.13);

\path[fill=fillColor,fill opacity=0.20] (192.51, 72.15) circle (  2.13);

\path[fill=fillColor,fill opacity=0.20] (210.21, 41.41) circle (  2.13);

\path[fill=fillColor,fill opacity=0.20] (217.42, 51.17) circle (  2.13);

\path[fill=fillColor,fill opacity=0.20] (228.56, 61.66) circle (  2.13);

\path[fill=fillColor,fill opacity=0.20] (223.75, 58.86) circle (  2.13);

\path[fill=fillColor,fill opacity=0.20] (218.29, 57.71) circle (  2.13);

\path[fill=fillColor,fill opacity=0.20] (218.07, 64.05) circle (  2.13);

\path[fill=fillColor,fill opacity=0.20] (215.01, 74.22) circle (  2.13);

\path[fill=fillColor,fill opacity=0.20] (210.64, 75.36) circle (  2.13);

\path[fill=fillColor,fill opacity=0.20] (232.71, 80.04) circle (  2.13);

\path[fill=fillColor,fill opacity=0.20] (239.70, 80.56) circle (  2.13);

\path[fill=fillColor,fill opacity=0.20] (224.63, 66.02) circle (  2.13);

\path[fill=fillColor,fill opacity=0.20] (211.95, 52.42) circle (  2.13);

\path[fill=fillColor,fill opacity=0.20] (177.65, 45.26) circle (  2.13);

\path[fill=fillColor,fill opacity=0.20] (183.55, 38.82) circle (  2.13);

\path[fill=fillColor,fill opacity=0.20] (199.72, 48.99) circle (  2.13);

\path[fill=fillColor,fill opacity=0.20] (217.63, 57.09) circle (  2.13);

\path[fill=fillColor,fill opacity=0.20] (224.85, 56.16) circle (  2.13);

\path[fill=fillColor,fill opacity=0.20] (222.88, 59.38) circle (  2.13);

\path[fill=fillColor,fill opacity=0.20] (213.92, 63.01) circle (  2.13);

\path[fill=fillColor,fill opacity=0.20] (211.52, 69.55) circle (  2.13);

\path[fill=fillColor,fill opacity=0.20] (210.21, 61.56) circle (  2.13);

\path[fill=fillColor,fill opacity=0.20] (208.02, 55.64) circle (  2.13);

\path[fill=fillColor,fill opacity=0.20] (231.62, 75.05) circle (  2.13);

\path[fill=fillColor,fill opacity=0.20] (233.59, 77.86) circle (  2.13);

\path[fill=fillColor,fill opacity=0.20] (204.31, 84.81) circle (  2.13);

\path[fill=fillColor,fill opacity=0.20] (185.95, 65.29) circle (  2.13);

\path[fill=fillColor,fill opacity=0.20] (170.66, 48.79) circle (  2.13);

\path[fill=fillColor,fill opacity=0.20] (182.89, 64.36) circle (  2.13);

\path[fill=fillColor,fill opacity=0.20] (202.56, 52.11) circle (  2.13);

\path[fill=fillColor,fill opacity=0.20] (224.85, 55.12) circle (  2.13);

\path[fill=fillColor,fill opacity=0.20] (216.11, 60.52) circle (  2.13);

\path[fill=fillColor,fill opacity=0.20] (216.98, 56.16) circle (  2.13);

\path[fill=fillColor,fill opacity=0.20] (224.63, 59.17) circle (  2.13);

\path[fill=fillColor,fill opacity=0.20] (216.98, 65.29) circle (  2.13);

\path[fill=fillColor,fill opacity=0.20] (217.85, 61.56) circle (  2.13);

\path[fill=fillColor,fill opacity=0.20] (218.07, 78.17) circle (  2.13);

\path[fill=fillColor,fill opacity=0.20] (217.63, 58.55) circle (  2.13);

\path[fill=fillColor,fill opacity=0.20] (245.82, 68.82) circle (  2.13);

\path[fill=fillColor,fill opacity=0.20] (229.22, 72.98) circle (  2.13);

\path[fill=fillColor,fill opacity=0.20] (182.89, 49.62) circle (  2.13);

\path[fill=fillColor,fill opacity=0.20] (208.89, 48.79) circle (  2.13);

\path[fill=fillColor,fill opacity=0.20] (174.37, 63.53) circle (  2.13);

\path[fill=fillColor,fill opacity=0.20] (194.26, 52.11) circle (  2.13);

\path[fill=fillColor,fill opacity=0.20] (209.99, 44.53) circle (  2.13);

\path[fill=fillColor,fill opacity=0.20] (224.85, 42.45) circle (  2.13);

\path[fill=fillColor,fill opacity=0.20] (225.50, 61.14) circle (  2.13);

\path[fill=fillColor,fill opacity=0.20] (223.10, 72.98) circle (  2.13);

\path[fill=fillColor,fill opacity=0.20] (218.07, 68.41) circle (  2.13);

\path[fill=fillColor,fill opacity=0.20] (209.33, 63.22) circle (  2.13);

\path[fill=fillColor,fill opacity=0.20] (213.70, 72.87) circle (  2.13);

\path[fill=fillColor,fill opacity=0.20] (209.55, 87.93) circle (  2.13);

\path[fill=fillColor,fill opacity=0.20] (211.52, 64.77) circle (  2.13);

\path[fill=fillColor,fill opacity=0.20] (216.98, 72.35) circle (  2.13);

\path[fill=fillColor,fill opacity=0.20] (223.53, 66.12) circle (  2.13);

\path[fill=fillColor,fill opacity=0.20] (230.53, 58.86) circle (  2.13);

\path[fill=fillColor,fill opacity=0.20] (215.01, 63.74) circle (  2.13);

\path[fill=fillColor,fill opacity=0.20] (211.52, 50.03) circle (  2.13);

\path[fill=fillColor,fill opacity=0.20] (177.65, 68.62) circle (  2.13);

\path[fill=fillColor,fill opacity=0.20] (188.57, 52.32) circle (  2.13);

\path[fill=fillColor,fill opacity=0.20] (208.89, 51.17) circle (  2.13);

\path[fill=fillColor,fill opacity=0.20] (229.43, 59.17) circle (  2.13);

\path[fill=fillColor,fill opacity=0.20] (229.00, 63.32) circle (  2.13);

\path[fill=fillColor,fill opacity=0.20] (222.00, 65.81) circle (  2.13);

\path[fill=fillColor,fill opacity=0.20] (212.39, 62.08) circle (  2.13);

\path[fill=fillColor,fill opacity=0.20] (209.33, 74.22) circle (  2.13);

\path[fill=fillColor,fill opacity=0.20] (207.58, 84.81) circle (  2.13);

\path[fill=fillColor,fill opacity=0.20] (199.94, 74.74) circle (  2.13);

\path[fill=fillColor,fill opacity=0.20] (197.97, 65.19) circle (  2.13);

\path[fill=fillColor,fill opacity=0.20] (198.63, 73.50) circle (  2.13);

\path[fill=fillColor,fill opacity=0.20] (203.00, 81.70) circle (  2.13);

\path[fill=fillColor,fill opacity=0.20] (206.93, 66.12) circle (  2.13);

\path[fill=fillColor,fill opacity=0.20] (208.24, 59.06) circle (  2.13);

\path[fill=fillColor,fill opacity=0.20] (211.74, 57.20) circle (  2.13);

\path[fill=fillColor,fill opacity=0.20] (216.32, 65.92) circle (  2.13);

\path[fill=fillColor,fill opacity=0.20] (229.00, 69.65) circle (  2.13);

\path[fill=fillColor,fill opacity=0.20] (223.53, 61.97) circle (  2.13);

\path[fill=fillColor,fill opacity=0.20] (197.31, 54.08) circle (  2.13);

\path[fill=fillColor,fill opacity=0.20] (166.51, 49.51) circle (  2.13);

\path[fill=fillColor,fill opacity=0.20] (174.15, 74.33) circle (  2.13);

\path[fill=fillColor,fill opacity=0.20] (189.67, 59.89) circle (  2.13);

\path[fill=fillColor,fill opacity=0.20] (213.26, 53.56) circle (  2.13);

\path[fill=fillColor,fill opacity=0.20] (221.35, 55.74) circle (  2.13);

\path[fill=fillColor,fill opacity=0.20] (221.13, 53.56) circle (  2.13);

\path[fill=fillColor,fill opacity=0.20] (217.63, 65.40) circle (  2.13);

\path[fill=fillColor,fill opacity=0.20] (212.17, 83.77) circle (  2.13);

\path[fill=fillColor,fill opacity=0.20] (206.27, 80.97) circle (  2.13);

\path[fill=fillColor,fill opacity=0.20] (205.84, 69.24) circle (  2.13);

\path[fill=fillColor,fill opacity=0.20] (201.25, 64.77) circle (  2.13);

\path[fill=fillColor,fill opacity=0.20] (195.57, 63.01) circle (  2.13);

\path[fill=fillColor,fill opacity=0.20] (210.64, 71.11) circle (  2.13);

\path[fill=fillColor,fill opacity=0.20] (199.72, 78.58) circle (  2.13);

\path[fill=fillColor,fill opacity=0.20] (199.94, 73.70) circle (  2.13);

\path[fill=fillColor,fill opacity=0.20] (201.68, 60.00) circle (  2.13);

\path[fill=fillColor,fill opacity=0.20] (199.06, 49.41) circle (  2.13);

\path[fill=fillColor,fill opacity=0.20] (200.37, 49.51) circle (  2.13);

\path[fill=fillColor,fill opacity=0.20] (203.43, 56.68) circle (  2.13);

\path[fill=fillColor,fill opacity=0.20] (220.69, 57.40) circle (  2.13);

\path[fill=fillColor,fill opacity=0.20] (224.85, 52.42) circle (  2.13);

\path[fill=fillColor,fill opacity=0.20] (221.35, 61.35) circle (  2.13);

\path[fill=fillColor,fill opacity=0.20] (213.70, 62.59) circle (  2.13);

\path[fill=fillColor,fill opacity=0.20] (197.31, 49.10) circle (  2.13);

\path[fill=fillColor,fill opacity=0.20] (171.09, 48.16) circle (  2.13);

\path[fill=fillColor,fill opacity=0.20] (167.38, 64.57) circle (  2.13);

\path[fill=fillColor,fill opacity=0.20] (175.68, 76.09) circle (  2.13);

\path[fill=fillColor,fill opacity=0.20] (202.56, 59.58) circle (  2.13);

\path[fill=fillColor,fill opacity=0.20] (211.52, 48.37) circle (  2.13);

\path[fill=fillColor,fill opacity=0.20] (215.89, 46.92) circle (  2.13);

\path[fill=fillColor,fill opacity=0.20] (221.57, 55.33) circle (  2.13);

\path[fill=fillColor,fill opacity=0.20] (217.42, 64.57) circle (  2.13);

\path[fill=fillColor,fill opacity=0.20] (219.60, 69.86) circle (  2.13);

\path[fill=fillColor,fill opacity=0.20] (217.85, 67.27) circle (  2.13);

\path[fill=fillColor,fill opacity=0.20] (208.68, 63.42) circle (  2.13);

\path[fill=fillColor,fill opacity=0.20] (206.05, 69.24) circle (  2.13);

\path[fill=fillColor,fill opacity=0.20] (205.40, 71.32) circle (  2.13);

\path[fill=fillColor,fill opacity=0.20] (201.47, 62.91) circle (  2.13);

\path[fill=fillColor,fill opacity=0.20] (201.68, 56.99) circle (  2.13);

\path[fill=fillColor,fill opacity=0.20] (201.25, 55.22) circle (  2.13);

\path[fill=fillColor,fill opacity=0.20] (203.00, 54.18) circle (  2.13);

\path[fill=fillColor,fill opacity=0.20] (198.63, 60.73) circle (  2.13);

\path[fill=fillColor,fill opacity=0.20] (202.56, 58.86) circle (  2.13);

\path[fill=fillColor,fill opacity=0.20] (203.65, 52.11) circle (  2.13);

\path[fill=fillColor,fill opacity=0.20] (203.00, 57.20) circle (  2.13);

\path[fill=fillColor,fill opacity=0.20] (196.22, 62.18) circle (  2.13);

\path[fill=fillColor,fill opacity=0.20] (199.50, 65.92) circle (  2.13);

\path[fill=fillColor,fill opacity=0.20] (201.25, 68.41) circle (  2.13);

\path[fill=fillColor,fill opacity=0.20] (198.41, 60.83) circle (  2.13);

\path[fill=fillColor,fill opacity=0.20] (200.81, 54.91) circle (  2.13);

\path[fill=fillColor,fill opacity=0.20] (203.21, 58.13) circle (  2.13);

\path[fill=fillColor,fill opacity=0.20] (203.87, 61.04) circle (  2.13);

\path[fill=fillColor,fill opacity=0.20] (208.68, 64.88) circle (  2.13);

\path[fill=fillColor,fill opacity=0.20] (214.36, 61.87) circle (  2.13);

\path[fill=fillColor,fill opacity=0.20] (216.98, 61.66) circle (  2.13);

\path[fill=fillColor,fill opacity=0.20] (219.38, 68.82) circle (  2.13);

\path[fill=fillColor,fill opacity=0.20] (217.63, 64.15) circle (  2.13);

\path[fill=fillColor,fill opacity=0.20] (210.42, 52.42) circle (  2.13);

\path[fill=fillColor,fill opacity=0.20] (192.73, 50.24) circle (  2.13);

\path[fill=fillColor,fill opacity=0.20] (175.46, 47.54) circle (  2.13);

\path[fill=fillColor,fill opacity=0.20] (215.89, 56.36) circle (  2.13);

\path[fill=fillColor,fill opacity=0.20] (190.76, 60.93) circle (  2.13);

\path[fill=fillColor,fill opacity=0.20] (186.39, 51.59) circle (  2.13);

\path[fill=fillColor,fill opacity=0.20] (211.52, 41.00) circle (  2.13);

\path[fill=fillColor,fill opacity=0.20] (215.45, 63.22) circle (  2.13);

\path[fill=fillColor,fill opacity=0.20] (217.20, 68.93) circle (  2.13);

\path[fill=fillColor,fill opacity=0.20] (219.38, 60.41) circle (  2.13);

\path[fill=fillColor,fill opacity=0.20] (218.95, 62.49) circle (  2.13);

\path[fill=fillColor,fill opacity=0.20] (215.01, 71.00) circle (  2.13);

\path[fill=fillColor,fill opacity=0.20] (220.04, 73.81) circle (  2.13);

\path[fill=fillColor,fill opacity=0.20] (218.95, 68.82) circle (  2.13);

\path[fill=fillColor,fill opacity=0.20] (215.45, 58.44) circle (  2.13);

\path[fill=fillColor,fill opacity=0.20] (211.08, 54.60) circle (  2.13);

\path[fill=fillColor,fill opacity=0.20] (214.14, 59.79) circle (  2.13);

\path[fill=fillColor,fill opacity=0.20] (213.05, 61.56) circle (  2.13);

\path[fill=fillColor,fill opacity=0.20] (211.95, 58.86) circle (  2.13);

\path[fill=fillColor,fill opacity=0.20] (212.39, 57.82) circle (  2.13);

\path[fill=fillColor,fill opacity=0.20] (208.24, 57.71) circle (  2.13);

\path[fill=fillColor,fill opacity=0.20] (208.68, 59.38) circle (  2.13);

\path[fill=fillColor,fill opacity=0.20] (216.32, 61.04) circle (  2.13);

\path[fill=fillColor,fill opacity=0.20] (213.70, 62.80) circle (  2.13);

\path[fill=fillColor,fill opacity=0.20] (217.42, 62.70) circle (  2.13);

\path[fill=fillColor,fill opacity=0.20] (218.29, 62.18) circle (  2.13);

\path[fill=fillColor,fill opacity=0.20] (221.57, 67.89) circle (  2.13);

\path[fill=fillColor,fill opacity=0.20] (231.40, 71.32) circle (  2.13);

\path[fill=fillColor,fill opacity=0.20] (225.06, 60.52) circle (  2.13);

\path[fill=fillColor,fill opacity=0.20] (212.61, 54.50) circle (  2.13);

\path[fill=fillColor,fill opacity=0.20] (211.74, 58.96) circle (  2.13);

\path[fill=fillColor,fill opacity=0.20] (179.83, 52.00) circle (  2.13);

\path[fill=fillColor,fill opacity=0.20] (169.56, 50.86) circle (  2.13);

\path[fill=fillColor,fill opacity=0.20] (179.40, 68.41) circle (  2.13);

\path[fill=fillColor,fill opacity=0.20] (166.07, 59.89) circle (  2.13);

\path[fill=fillColor,fill opacity=0.20] (175.68, 55.22) circle (  2.13);

\path[fill=fillColor,fill opacity=0.20] (208.46, 65.71) circle (  2.13);

\path[fill=fillColor,fill opacity=0.20] (196.88, 71.52) circle (  2.13);

\path[fill=fillColor,fill opacity=0.20] (192.29, 60.10) circle (  2.13);

\path[fill=fillColor,fill opacity=0.20] (198.84, 48.47) circle (  2.13);

\path[fill=fillColor,fill opacity=0.20] (205.84, 54.70) circle (  2.13);

\path[fill=fillColor,fill opacity=0.20] (224.85, 61.97) circle (  2.13);

\path[fill=fillColor,fill opacity=0.20] (217.85, 56.57) circle (  2.13);

\path[fill=fillColor,fill opacity=0.20] (216.54, 46.40) circle (  2.13);

\path[fill=fillColor,fill opacity=0.20] (215.23, 43.39) circle (  2.13);

\path[fill=fillColor,fill opacity=0.20] (221.57, 47.75) circle (  2.13);

\path[fill=fillColor,fill opacity=0.20] (213.48, 55.33) circle (  2.13);

\path[fill=fillColor,fill opacity=0.20] (214.14, 60.10) circle (  2.13);

\path[fill=fillColor,fill opacity=0.20] (206.27, 59.89) circle (  2.13);

\path[fill=fillColor,fill opacity=0.20] (203.43, 58.13) circle (  2.13);

\path[fill=fillColor,fill opacity=0.20] (192.73, 57.40) circle (  2.13);

\path[fill=fillColor,fill opacity=0.20] (198.41, 52.21) circle (  2.13);

\path[fill=fillColor,fill opacity=0.20] (202.34, 45.88) circle (  2.13);

\path[fill=fillColor,fill opacity=0.20] (197.31, 46.09) circle (  2.13);

\path[fill=fillColor,fill opacity=0.20] (212.83, 50.76) circle (  2.13);

\path[fill=fillColor,fill opacity=0.20] (193.38, 48.58) circle (  2.13);

\path[fill=fillColor,fill opacity=0.20] (192.07, 43.18) circle (  2.13);

\path[fill=fillColor,fill opacity=0.20] (192.94, 42.35) circle (  2.13);

\path[fill=fillColor,fill opacity=0.20] (174.81, 42.56) circle (  2.13);

\path[fill=fillColor,fill opacity=0.20] (168.47, 48.68) circle (  2.13);

\path[fill=fillColor,fill opacity=0.20] (165.19, 63.42) circle (  2.13);

\path[fill=fillColor,fill opacity=0.20] (166.94, 54.81) circle (  2.13);

\path[fill=fillColor,fill opacity=0.20] (172.41, 46.29) circle (  2.13);

\path[fill=fillColor,fill opacity=0.20] (169.35, 40.17) circle (  2.13);

\path[fill=fillColor,fill opacity=0.20] (169.35, 39.23) circle (  2.13);

\path[fill=fillColor,fill opacity=0.20] (177.21, 38.09) circle (  2.13);

\path[fill=fillColor,fill opacity=0.20] (172.41, 39.03) circle (  2.13);

\path[fill=fillColor,fill opacity=0.20] (173.50, 46.81) circle (  2.13);

\path[fill=fillColor,fill opacity=0.20] (169.78, 55.12) circle (  2.13);

\path[fill=fillColor,fill opacity=0.20] (175.25, 61.04) circle (  2.13);

\path[fill=fillColor,fill opacity=0.20] (177.65, 63.94) circle (  2.13);

\path[fill=fillColor,fill opacity=0.20] (170.00, 69.13) circle (  2.13);

\path[fill=fillColor,fill opacity=0.20] (165.19, 73.50) circle (  2.13);

\path[fill=fillColor,fill opacity=0.20] (233.59, 61.35) circle (  2.13);

\path[fill=fillColor,fill opacity=0.20] (183.77, 49.72) circle (  2.13);

\path[fill=fillColor,fill opacity=0.20] (172.62, 60.21) circle (  2.13);

\path[fill=fillColor,fill opacity=0.20] (204.09, 70.59) circle (  2.13);

\path[fill=fillColor,fill opacity=0.20] (215.01, 79.62) circle (  2.13);

\path[fill=fillColor,fill opacity=0.20] (230.74, 76.19) circle (  2.13);

\path[fill=fillColor,fill opacity=0.20] (224.85, 76.71) circle (  2.13);

\path[fill=fillColor,fill opacity=0.20] (215.89, 76.30) circle (  2.13);

\path[fill=fillColor,fill opacity=0.20] (206.05, 69.13) circle (  2.13);

\path[fill=fillColor,fill opacity=0.20] (202.56, 60.00) circle (  2.13);

\path[fill=fillColor,fill opacity=0.20] (201.90, 58.55) circle (  2.13);

\path[fill=fillColor,fill opacity=0.20] (202.78, 64.26) circle (  2.13);

\path[fill=fillColor,fill opacity=0.20] (196.66, 70.90) circle (  2.13);

\path[fill=fillColor,fill opacity=0.20] (191.85, 73.29) circle (  2.13);

\path[fill=fillColor,fill opacity=0.20] (191.63, 74.01) circle (  2.13);

\path[fill=fillColor,fill opacity=0.20] (223.10, 72.66) circle (  2.13);

\path[fill=fillColor,fill opacity=0.20] (212.39, 60.73) circle (  2.13);

\path[fill=fillColor,fill opacity=0.20] (258.71, 69.86) circle (  2.13);

\path[fill=fillColor,fill opacity=0.20] (216.98, 80.14) circle (  2.13);

\path[fill=fillColor,fill opacity=0.20] (211.74, 78.06) circle (  2.13);

\path[fill=fillColor,fill opacity=0.20] (204.52, 70.80) circle (  2.13);

\path[fill=fillColor,fill opacity=0.20] (199.06, 65.09) circle (  2.13);

\path[fill=fillColor,fill opacity=0.20] (198.41, 60.21) circle (  2.13);

\path[fill=fillColor,fill opacity=0.20] (196.22, 59.48) circle (  2.13);

\path[fill=fillColor,fill opacity=0.20] (189.23, 70.48) circle (  2.13);

\path[fill=fillColor,fill opacity=0.20] (180.71, 77.86) circle (  2.13);

\path[fill=fillColor,fill opacity=0.20] (174.37, 75.05) circle (  2.13);

\path[fill=fillColor,fill opacity=0.20] (214.58, 86.89) circle (  2.13);

\path[fill=fillColor,fill opacity=0.20] (218.29, 73.29) circle (  2.13);

\path[fill=fillColor,fill opacity=0.20] (209.33, 65.29) circle (  2.13);

\path[fill=fillColor,fill opacity=0.20] (211.08, 70.48) circle (  2.13);

\path[fill=fillColor,fill opacity=0.20] (211.08, 76.51) circle (  2.13);

\path[fill=fillColor,fill opacity=0.20] (207.58, 73.50) circle (  2.13);

\path[fill=fillColor,fill opacity=0.20] (204.52, 70.48) circle (  2.13);

\path[fill=fillColor,fill opacity=0.20] (198.19, 72.35) circle (  2.13);

\path[fill=fillColor,fill opacity=0.20] (197.75, 70.28) circle (  2.13);

\path[fill=fillColor,fill opacity=0.20] (192.73, 67.68) circle (  2.13);

\path[fill=fillColor,fill opacity=0.20] (180.05, 74.33) circle (  2.13);

\path[fill=fillColor,fill opacity=0.20] (213.70, 75.68) circle (  2.13);

\path[fill=fillColor,fill opacity=0.20] (252.16, 77.03) circle (  2.13);

\path[fill=fillColor,fill opacity=0.20] (214.79, 88.96) circle (  2.13);

\path[fill=fillColor,fill opacity=0.20] (213.26, 79.10) circle (  2.13);

\path[fill=fillColor,fill opacity=0.20] (208.68, 76.82) circle (  2.13);

\path[fill=fillColor,fill opacity=0.20] (205.84, 74.53) circle (  2.13);

\path[fill=fillColor,fill opacity=0.20] (205.18, 72.04) circle (  2.13);

\path[fill=fillColor,fill opacity=0.20] (203.65, 68.82) circle (  2.13);

\path[fill=fillColor,fill opacity=0.20] (201.47, 67.89) circle (  2.13);

\path[fill=fillColor,fill opacity=0.20] (197.53, 73.70) circle (  2.13);

\path[fill=fillColor,fill opacity=0.20] (191.20, 76.82) circle (  2.13);

\path[fill=fillColor,fill opacity=0.20] (183.11, 73.50) circle (  2.13);

\path[fill=fillColor,fill opacity=0.20] (168.25, 75.99) circle (  2.13);

\path[fill=fillColor,fill opacity=0.20] (199.06, 62.91) circle (  2.13);

\path[fill=fillColor,fill opacity=0.20] (203.65, 65.61) circle (  2.13);

\path[fill=fillColor,fill opacity=0.20] (212.39, 65.81) circle (  2.13);

\path[fill=fillColor,fill opacity=0.20] (193.60, 94.16) circle (  2.13);

\path[fill=fillColor,fill opacity=0.20] (215.23, 79.52) circle (  2.13);

\path[fill=fillColor,fill opacity=0.20] (215.67, 74.12) circle (  2.13);

\path[fill=fillColor,fill opacity=0.20] (206.71, 79.10) circle (  2.13);

\path[fill=fillColor,fill opacity=0.20] (204.96, 75.88) circle (  2.13);

\path[fill=fillColor,fill opacity=0.20] (202.34, 70.28) circle (  2.13);

\path[fill=fillColor,fill opacity=0.20] (200.15, 70.38) circle (  2.13);

\path[fill=fillColor,fill opacity=0.20] (199.94, 70.07) circle (  2.13);

\path[fill=fillColor,fill opacity=0.20] (197.10, 72.77) circle (  2.13);

\path[fill=fillColor,fill opacity=0.20] (187.26, 76.82) circle (  2.13);

\path[fill=fillColor,fill opacity=0.20] (205.62, 62.08) circle (  2.13);

\path[fill=fillColor,fill opacity=0.20] (205.40, 60.10) circle (  2.13);

\path[fill=fillColor,fill opacity=0.20] (214.36, 65.40) circle (  2.13);

\path[fill=fillColor,fill opacity=0.20] (224.19, 59.48) circle (  2.13);

\path[fill=fillColor,fill opacity=0.20] (225.06, 57.92) circle (  2.13);

\path[fill=fillColor,fill opacity=0.20] (219.60, 60.62) circle (  2.13);

\path[fill=fillColor,fill opacity=0.20] (213.05, 67.27) circle (  2.13);

\path[fill=fillColor,fill opacity=0.20] (207.37, 76.30) circle (  2.13);

\path[fill=fillColor,fill opacity=0.20] (193.82, 88.96) circle (  2.13);

\path[fill=fillColor,fill opacity=0.20] (218.07, 68.72) circle (  2.13);

\path[fill=fillColor,fill opacity=0.20] (212.83, 66.44) circle (  2.13);

\path[fill=fillColor,fill opacity=0.20] (206.27, 76.82) circle (  2.13);

\path[fill=fillColor,fill opacity=0.20] (205.18, 72.87) circle (  2.13);

\path[fill=fillColor,fill opacity=0.20] (200.15, 66.44) circle (  2.13);

\path[fill=fillColor,fill opacity=0.20] (197.10, 69.86) circle (  2.13);

\path[fill=fillColor,fill opacity=0.20] (197.97, 72.87) circle (  2.13);

\path[fill=fillColor,fill opacity=0.20] (194.91, 74.64) circle (  2.13);

\path[fill=fillColor,fill opacity=0.20] (208.89, 61.56) circle (  2.13);

\path[fill=fillColor,fill opacity=0.20] (218.29, 68.10) circle (  2.13);

\path[fill=fillColor,fill opacity=0.20] (222.22, 67.37) circle (  2.13);

\path[fill=fillColor,fill opacity=0.20] (219.38, 59.58) circle (  2.13);

\path[fill=fillColor,fill opacity=0.20] (234.46, 53.67) circle (  2.13);

\path[fill=fillColor,fill opacity=0.20] (229.87, 52.42) circle (  2.13);

\path[fill=fillColor,fill opacity=0.20] (219.60, 60.10) circle (  2.13);

\path[fill=fillColor,fill opacity=0.20] (215.89, 68.30) circle (  2.13);

\path[fill=fillColor,fill opacity=0.20] (210.64, 70.69) circle (  2.13);

\path[fill=fillColor,fill opacity=0.20] (227.90, 73.29) circle (  2.13);

\path[fill=fillColor,fill opacity=0.20] (191.41, 93.12) circle (  2.13);

\path[fill=fillColor,fill opacity=0.20] (217.85, 75.99) circle (  2.13);

\path[fill=fillColor,fill opacity=0.20] (207.80, 66.33) circle (  2.13);

\path[fill=fillColor,fill opacity=0.20] (202.12, 74.95) circle (  2.13);

\path[fill=fillColor,fill opacity=0.20] (203.00, 76.71) circle (  2.13);

\path[fill=fillColor,fill opacity=0.20] (200.81, 67.99) circle (  2.13);

\path[fill=fillColor,fill opacity=0.20] (198.63, 64.67) circle (  2.13);

\path[fill=fillColor,fill opacity=0.20] (197.75, 70.59) circle (  2.13);

\path[fill=fillColor,fill opacity=0.20] (195.13, 79.52) circle (  2.13);

\path[fill=fillColor,fill opacity=0.20] (206.71, 83.77) circle (  2.13);

\path[fill=fillColor,fill opacity=0.20] (211.30, 45.46) circle (  2.13);

\path[fill=fillColor,fill opacity=0.20] (211.95, 59.48) circle (  2.13);

\path[fill=fillColor,fill opacity=0.20] (219.60, 62.80) circle (  2.13);

\path[fill=fillColor,fill opacity=0.20] (234.68, 53.98) circle (  2.13);

\path[fill=fillColor,fill opacity=0.20] (238.61, 49.82) circle (  2.13);

\path[fill=fillColor,fill opacity=0.20] (237.52, 53.25) circle (  2.13);

\path[fill=fillColor,fill opacity=0.20] (230.09, 59.17) circle (  2.13);

\path[fill=fillColor,fill opacity=0.20] (224.41, 56.05) circle (  2.13);

\path[fill=fillColor,fill opacity=0.20] (213.05, 52.11) circle (  2.13);

\path[fill=fillColor,fill opacity=0.20] (201.68, 66.75) circle (  2.13);

\path[fill=fillColor,fill opacity=0.20] (196.22, 83.77) circle (  2.13);

\path[fill=fillColor,fill opacity=0.20] (212.83, 81.18) circle (  2.13);

\path[fill=fillColor,fill opacity=0.20] (215.23, 63.84) circle (  2.13);

\path[fill=fillColor,fill opacity=0.20] (203.21, 71.42) circle (  2.13);

\path[fill=fillColor,fill opacity=0.20] (203.43, 83.77) circle (  2.13);

\path[fill=fillColor,fill opacity=0.20] (210.42, 73.91) circle (  2.13);

\path[fill=fillColor,fill opacity=0.20] (203.65, 58.65) circle (  2.13);

\path[fill=fillColor,fill opacity=0.20] (196.00, 62.28) circle (  2.13);

\path[fill=fillColor,fill opacity=0.20] (194.04, 76.51) circle (  2.13);

\path[fill=fillColor,fill opacity=0.20] (194.91, 79.21) circle (  2.13);

\path[fill=fillColor,fill opacity=0.20] (188.57, 75.16) circle (  2.13);

\path[fill=fillColor,fill opacity=0.20] (219.16, 59.58) circle (  2.13);

\path[fill=fillColor,fill opacity=0.20] (217.42, 49.10) circle (  2.13);

\path[fill=fillColor,fill opacity=0.20] (221.35, 50.24) circle (  2.13);

\path[fill=fillColor,fill opacity=0.20] (227.03, 43.49) circle (  2.13);

\path[fill=fillColor,fill opacity=0.20] (229.22, 51.17) circle (  2.13);

\path[fill=fillColor,fill opacity=0.20] (234.02, 58.55) circle (  2.13);

\path[fill=fillColor,fill opacity=0.20] (239.27, 58.96) circle (  2.13);

\path[fill=fillColor,fill opacity=0.20] (237.96, 52.63) circle (  2.13);

\path[fill=fillColor,fill opacity=0.20] (226.16, 43.49) circle (  2.13);

\path[fill=fillColor,fill opacity=0.20] (218.07, 56.99) circle (  2.13);

\path[fill=fillColor,fill opacity=0.20] (196.00, 77.23) circle (  2.13);

\path[fill=fillColor,fill opacity=0.20] (196.00, 69.24) circle (  2.13);

\path[fill=fillColor,fill opacity=0.20] (208.46, 71.00) circle (  2.13);

\path[fill=fillColor,fill opacity=0.20] (222.44, 54.91) circle (  2.13);

\path[fill=fillColor,fill opacity=0.20] (211.74, 64.67) circle (  2.13);

\path[fill=fillColor,fill opacity=0.20] (207.15, 77.54) circle (  2.13);

\path[fill=fillColor,fill opacity=0.20] (205.40, 68.20) circle (  2.13);

\path[fill=fillColor,fill opacity=0.20] (198.84, 54.81) circle (  2.13);

\path[fill=fillColor,fill opacity=0.20] (192.94, 58.13) circle (  2.13);

\path[fill=fillColor,fill opacity=0.20] (194.91, 66.44) circle (  2.13);

\path[fill=fillColor,fill opacity=0.20] (196.44, 67.16) circle (  2.13);

\path[fill=fillColor,fill opacity=0.20] (193.82, 64.05) circle (  2.13);

\path[fill=fillColor,fill opacity=0.20] (225.28, 57.40) circle (  2.13);

\path[fill=fillColor,fill opacity=0.20] (228.56, 38.51) circle (  2.13);

\path[fill=fillColor,fill opacity=0.20] (218.07, 51.59) circle (  2.13);

\path[fill=fillColor,fill opacity=0.20] (222.88, 44.94) circle (  2.13);

\path[fill=fillColor,fill opacity=0.20] (225.72, 46.19) circle (  2.13);

\path[fill=fillColor,fill opacity=0.20] (226.16, 57.71) circle (  2.13);

\path[fill=fillColor,fill opacity=0.20] (230.09, 58.86) circle (  2.13);

\path[fill=fillColor,fill opacity=0.20] (235.99, 54.29) circle (  2.13);

\path[fill=fillColor,fill opacity=0.20] (238.39, 53.67) circle (  2.13);

\path[fill=fillColor,fill opacity=0.20] (234.68, 47.64) circle (  2.13);

\path[fill=fillColor,fill opacity=0.20] (224.41, 48.58) circle (  2.13);

\path[fill=fillColor,fill opacity=0.20] (204.74, 62.28) circle (  2.13);

\path[fill=fillColor,fill opacity=0.20] (196.00, 68.72) circle (  2.13);

\path[fill=fillColor,fill opacity=0.20] (196.00, 70.38) circle (  2.13);

\path[fill=fillColor,fill opacity=0.20] (219.16, 53.46) circle (  2.13);

\path[fill=fillColor,fill opacity=0.20] (214.14, 58.96) circle (  2.13);

\path[fill=fillColor,fill opacity=0.20] (205.84, 66.75) circle (  2.13);

\path[fill=fillColor,fill opacity=0.20] (204.31, 59.06) circle (  2.13);

\path[fill=fillColor,fill opacity=0.20] (200.15, 54.91) circle (  2.13);

\path[fill=fillColor,fill opacity=0.20] (199.28, 61.87) circle (  2.13);

\path[fill=fillColor,fill opacity=0.20] (197.53, 65.92) circle (  2.13);

\path[fill=fillColor,fill opacity=0.20] (200.15, 62.80) circle (  2.13);

\path[fill=fillColor,fill opacity=0.20] (198.84, 55.85) circle (  2.13);

\path[fill=fillColor,fill opacity=0.20] (194.04, 56.88) circle (  2.13);

\path[fill=fillColor,fill opacity=0.20] (222.88, 68.41) circle (  2.13);

\path[fill=fillColor,fill opacity=0.20] (231.62, 50.65) circle (  2.13);

\path[fill=fillColor,fill opacity=0.20] (226.16, 57.09) circle (  2.13);

\path[fill=fillColor,fill opacity=0.20] (224.41, 53.56) circle (  2.13);

\path[fill=fillColor,fill opacity=0.20] (220.04, 61.56) circle (  2.13);

\path[fill=fillColor,fill opacity=0.20] (226.16, 60.83) circle (  2.13);

\path[fill=fillColor,fill opacity=0.20] (232.06, 51.69) circle (  2.13);

\path[fill=fillColor,fill opacity=0.20] (234.02, 55.53) circle (  2.13);

\path[fill=fillColor,fill opacity=0.20] (234.02, 58.03) circle (  2.13);

\path[fill=fillColor,fill opacity=0.20] (238.17, 48.47) circle (  2.13);

\path[fill=fillColor,fill opacity=0.20] (231.40, 42.25) circle (  2.13);

\path[fill=fillColor,fill opacity=0.20] (215.23, 48.68) circle (  2.13);

\path[fill=fillColor,fill opacity=0.20] (196.00, 61.76) circle (  2.13);

\path[fill=fillColor,fill opacity=0.20] (177.87, 87.93) circle (  2.13);

\path[fill=fillColor,fill opacity=0.20] (203.65, 63.01) circle (  2.13);

\path[fill=fillColor,fill opacity=0.20] (209.99, 57.61) circle (  2.13);

\path[fill=fillColor,fill opacity=0.20] (206.71, 63.74) circle (  2.13);

\path[fill=fillColor,fill opacity=0.20] (205.40, 61.97) circle (  2.13);

\path[fill=fillColor,fill opacity=0.20] (207.15, 57.30) circle (  2.13);

\path[fill=fillColor,fill opacity=0.20] (204.52, 62.08) circle (  2.13);

\path[fill=fillColor,fill opacity=0.20] (203.43, 71.21) circle (  2.13);

\path[fill=fillColor,fill opacity=0.20] (199.28, 70.90) circle (  2.13);

\path[fill=fillColor,fill opacity=0.20] (201.90, 56.26) circle (  2.13);

\path[fill=fillColor,fill opacity=0.20] (198.84, 46.61) circle (  2.13);

\path[fill=fillColor,fill opacity=0.20] (192.73, 58.44) circle (  2.13);

\path[fill=fillColor,fill opacity=0.20] (186.83, 80.87) circle (  2.13);

\path[fill=fillColor,fill opacity=0.20] (219.16, 57.20) circle (  2.13);

\path[fill=fillColor,fill opacity=0.20] (227.03, 59.06) circle (  2.13);

\path[fill=fillColor,fill opacity=0.20] (225.50, 58.96) circle (  2.13);

\path[fill=fillColor,fill opacity=0.20] (223.75, 69.24) circle (  2.13);

\path[fill=fillColor,fill opacity=0.20] (232.71, 61.45) circle (  2.13);

\path[fill=fillColor,fill opacity=0.20] (236.21, 47.75) circle (  2.13);

\path[fill=fillColor,fill opacity=0.20] (237.08, 61.56) circle (  2.13);

\path[fill=fillColor,fill opacity=0.20] (235.33, 64.57) circle (  2.13);

\path[fill=fillColor,fill opacity=0.20] (236.21, 46.92) circle (  2.13);

\path[fill=fillColor,fill opacity=0.20] (228.56, 42.66) circle (  2.13);

\path[fill=fillColor,fill opacity=0.20] (215.23, 47.75) circle (  2.13);

\path[fill=fillColor,fill opacity=0.20] (199.50, 52.52) circle (  2.13);

\path[fill=fillColor,fill opacity=0.20] (181.58, 81.28) circle (  2.13);

\path[fill=fillColor,fill opacity=0.20] (179.18, 76.71) circle (  2.13);

\path[fill=fillColor,fill opacity=0.20] (203.43, 59.89) circle (  2.13);

\path[fill=fillColor,fill opacity=0.20] (208.89, 57.61) circle (  2.13);

\path[fill=fillColor,fill opacity=0.20] (209.77, 60.83) circle (  2.13);

\path[fill=fillColor,fill opacity=0.20] (204.52, 59.89) circle (  2.13);

\path[fill=fillColor,fill opacity=0.20] (206.05, 59.27) circle (  2.13);

\path[fill=fillColor,fill opacity=0.20] (205.18, 66.23) circle (  2.13);

\path[fill=fillColor,fill opacity=0.20] (203.87, 70.07) circle (  2.13);

\path[fill=fillColor,fill opacity=0.20] (206.71, 60.62) circle (  2.13);

\path[fill=fillColor,fill opacity=0.20] (202.34, 52.63) circle (  2.13);

\path[fill=fillColor,fill opacity=0.20] (194.47, 59.89) circle (  2.13);

\path[fill=fillColor,fill opacity=0.20] (194.69, 69.24) circle (  2.13);

\path[fill=fillColor,fill opacity=0.20] (190.54, 75.57) circle (  2.13);

\path[fill=fillColor,fill opacity=0.20] (204.52, 72.77) circle (  2.13);

\path[fill=fillColor,fill opacity=0.20] (217.20, 59.06) circle (  2.13);

\path[fill=fillColor,fill opacity=0.20] (222.22, 59.58) circle (  2.13);

\path[fill=fillColor,fill opacity=0.20] (228.12, 57.40) circle (  2.13);

\path[fill=fillColor,fill opacity=0.20] (229.00, 62.08) circle (  2.13);

\path[fill=fillColor,fill opacity=0.20] (233.15, 60.83) circle (  2.13);

\path[fill=fillColor,fill opacity=0.20] (231.40, 54.18) circle (  2.13);

\path[fill=fillColor,fill opacity=0.20] (234.02, 62.18) circle (  2.13);

\path[fill=fillColor,fill opacity=0.20] (232.49, 61.66) circle (  2.13);

\path[fill=fillColor,fill opacity=0.20] (230.74, 49.93) circle (  2.13);

\path[fill=fillColor,fill opacity=0.20] (232.27, 52.63) circle (  2.13);

\path[fill=fillColor,fill opacity=0.20] (222.88, 52.73) circle (  2.13);

\path[fill=fillColor,fill opacity=0.20] (199.28, 47.75) circle (  2.13);

\path[fill=fillColor,fill opacity=0.20] (197.31, 82.74) circle (  2.13);

\path[fill=fillColor,fill opacity=0.20] (183.99, 66.75) circle (  2.13);

\path[fill=fillColor,fill opacity=0.20] (203.21, 46.71) circle (  2.13);

\path[fill=fillColor,fill opacity=0.20] (204.74, 47.54) circle (  2.13);

\path[fill=fillColor,fill opacity=0.20] (205.84, 56.99) circle (  2.13);

\path[fill=fillColor,fill opacity=0.20] (208.68, 58.55) circle (  2.13);

\path[fill=fillColor,fill opacity=0.20] (205.18, 57.61) circle (  2.13);

\path[fill=fillColor,fill opacity=0.20] (224.41, 60.41) circle (  2.13);

\path[fill=fillColor,fill opacity=0.20] (203.87, 61.45) circle (  2.13);

\path[fill=fillColor,fill opacity=0.20] (200.37, 62.59) circle (  2.13);

\path[fill=fillColor,fill opacity=0.20] (198.63, 65.29) circle (  2.13);

\path[fill=fillColor,fill opacity=0.20] (198.63, 67.79) circle (  2.13);

\path[fill=fillColor,fill opacity=0.20] (196.44, 72.15) circle (  2.13);

\path[fill=fillColor,fill opacity=0.20] (194.26, 68.82) circle (  2.13);

\path[fill=fillColor,fill opacity=0.20] (191.41, 67.68) circle (  2.13);

\path[fill=fillColor,fill opacity=0.20] (196.88, 84.81) circle (  2.13);

\path[fill=fillColor,fill opacity=0.20] (211.95, 65.81) circle (  2.13);

\path[fill=fillColor,fill opacity=0.20] (217.63, 58.23) circle (  2.13);

\path[fill=fillColor,fill opacity=0.20] (219.82, 60.83) circle (  2.13);

\path[fill=fillColor,fill opacity=0.20] (220.48, 56.57) circle (  2.13);

\path[fill=fillColor,fill opacity=0.20] (226.59, 49.93) circle (  2.13);

\path[fill=fillColor,fill opacity=0.20] (239.92, 51.28) circle (  2.13);

\path[fill=fillColor,fill opacity=0.20] (228.12, 57.61) circle (  2.13);

\path[fill=fillColor,fill opacity=0.20] (231.84, 58.96) circle (  2.13);

\path[fill=fillColor,fill opacity=0.20] (233.59, 51.28) circle (  2.13);

\path[fill=fillColor,fill opacity=0.20] (224.19, 55.12) circle (  2.13);

\path[fill=fillColor,fill opacity=0.20] (232.49, 63.94) circle (  2.13);

\path[fill=fillColor,fill opacity=0.20] (236.64, 53.77) circle (  2.13);

\path[fill=fillColor,fill opacity=0.20] (200.15, 50.65) circle (  2.13);

\path[fill=fillColor,fill opacity=0.20] (218.73, 85.85) circle (  2.13);

\path[fill=fillColor,fill opacity=0.20] (173.72, 58.13) circle (  2.13);

\path[fill=fillColor,fill opacity=0.20] (186.61, 49.93) circle (  2.13);

\path[fill=fillColor,fill opacity=0.20] (199.28, 51.80) circle (  2.13);

\path[fill=fillColor,fill opacity=0.20] (205.62, 52.84) circle (  2.13);

\path[fill=fillColor,fill opacity=0.20] (205.62, 53.98) circle (  2.13);

\path[fill=fillColor,fill opacity=0.20] (203.00, 54.70) circle (  2.13);

\path[fill=fillColor,fill opacity=0.20] (208.89, 56.68) circle (  2.13);

\path[fill=fillColor,fill opacity=0.20] (199.50, 65.71) circle (  2.13);

\path[fill=fillColor,fill opacity=0.20] (198.84, 68.72) circle (  2.13);

\path[fill=fillColor,fill opacity=0.20] (197.75, 66.02) circle (  2.13);

\path[fill=fillColor,fill opacity=0.20] (195.35, 72.25) circle (  2.13);

\path[fill=fillColor,fill opacity=0.20] (195.13, 69.03) circle (  2.13);

\path[fill=fillColor,fill opacity=0.20] (196.00, 59.69) circle (  2.13);

\path[fill=fillColor,fill opacity=0.20] (193.60, 66.85) circle (  2.13);

\path[fill=fillColor,fill opacity=0.20] (189.01, 79.93) circle (  2.13);

\path[fill=fillColor,fill opacity=0.20] (193.16, 70.69) circle (  2.13);

\path[fill=fillColor,fill opacity=0.20] (192.51, 72.15) circle (  2.13);

\path[fill=fillColor,fill opacity=0.20] (197.53, 71.63) circle (  2.13);

\path[fill=fillColor,fill opacity=0.20] (205.84, 74.33) circle (  2.13);

\path[fill=fillColor,fill opacity=0.20] (214.79, 71.32) circle (  2.13);

\path[fill=fillColor,fill opacity=0.20] (217.42, 63.84) circle (  2.13);

\path[fill=fillColor,fill opacity=0.20] (216.11, 59.69) circle (  2.13);

\path[fill=fillColor,fill opacity=0.20] (214.58, 60.10) circle (  2.13);

\path[fill=fillColor,fill opacity=0.20] (217.42, 57.82) circle (  2.13);

\path[fill=fillColor,fill opacity=0.20] (228.78, 49.62) circle (  2.13);

\path[fill=fillColor,fill opacity=0.20] (226.37, 44.84) circle (  2.13);

\path[fill=fillColor,fill opacity=0.20] (232.93, 51.90) circle (  2.13);

\path[fill=fillColor,fill opacity=0.20] (230.96, 55.85) circle (  2.13);

\path[fill=fillColor,fill opacity=0.20] (229.65, 52.94) circle (  2.13);

\path[fill=fillColor,fill opacity=0.20] (223.53, 59.58) circle (  2.13);

\path[fill=fillColor,fill opacity=0.20] (233.15, 63.84) circle (  2.13);

\path[fill=fillColor,fill opacity=0.20] (221.13, 53.56) circle (  2.13);

\path[fill=fillColor,fill opacity=0.20] (199.50, 59.06) circle (  2.13);

\path[fill=fillColor,fill opacity=0.20] (204.31, 86.89) circle (  2.13);

\path[fill=fillColor,fill opacity=0.20] (173.50, 78.58) circle (  2.13);

\path[fill=fillColor,fill opacity=0.20] (177.43, 64.46) circle (  2.13);

\path[fill=fillColor,fill opacity=0.20] (183.55, 53.46) circle (  2.13);

\path[fill=fillColor,fill opacity=0.20] (190.54, 52.32) circle (  2.13);

\path[fill=fillColor,fill opacity=0.20] (204.31, 49.72) circle (  2.13);

\path[fill=fillColor,fill opacity=0.20] (202.56, 49.72) circle (  2.13);

\path[fill=fillColor,fill opacity=0.20] (202.78, 63.94) circle (  2.13);

\path[fill=fillColor,fill opacity=0.20] (200.81, 73.91) circle (  2.13);

\path[fill=fillColor,fill opacity=0.20] (198.84, 68.30) circle (  2.13);

\path[fill=fillColor,fill opacity=0.20] (195.35, 69.13) circle (  2.13);

\path[fill=fillColor,fill opacity=0.20] (197.97, 70.69) circle (  2.13);

\path[fill=fillColor,fill opacity=0.20] (202.12, 67.89) circle (  2.13);

\path[fill=fillColor,fill opacity=0.20] (198.84, 67.37) circle (  2.13);

\path[fill=fillColor,fill opacity=0.20] (196.88, 58.55) circle (  2.13);

\path[fill=fillColor,fill opacity=0.20] (192.94, 50.24) circle (  2.13);

\path[fill=fillColor,fill opacity=0.20] (190.98, 61.24) circle (  2.13);

\path[fill=fillColor,fill opacity=0.20] (187.26, 75.26) circle (  2.13);

\path[fill=fillColor,fill opacity=0.20] (192.73, 73.08) circle (  2.13);

\path[fill=fillColor,fill opacity=0.20] (197.97, 73.91) circle (  2.13);

\path[fill=fillColor,fill opacity=0.20] (213.26, 70.69) circle (  2.13);

\path[fill=fillColor,fill opacity=0.20] (209.33, 72.46) circle (  2.13);

\path[fill=fillColor,fill opacity=0.20] (206.71, 73.70) circle (  2.13);

\path[fill=fillColor,fill opacity=0.20] (204.31, 68.72) circle (  2.13);

\path[fill=fillColor,fill opacity=0.20] (213.70, 72.15) circle (  2.13);

\path[fill=fillColor,fill opacity=0.20] (218.51, 71.83) circle (  2.13);

\path[fill=fillColor,fill opacity=0.20] (214.79, 63.11) circle (  2.13);

\path[fill=fillColor,fill opacity=0.20] (210.64, 57.30) circle (  2.13);

\path[fill=fillColor,fill opacity=0.20] (213.70, 59.27) circle (  2.13);

\path[fill=fillColor,fill opacity=0.20] (216.98, 59.48) circle (  2.13);

\path[fill=fillColor,fill opacity=0.20] (220.69, 59.38) circle (  2.13);

\path[fill=fillColor,fill opacity=0.20] (223.53, 56.16) circle (  2.13);

\path[fill=fillColor,fill opacity=0.20] (226.16, 51.17) circle (  2.13);

\path[fill=fillColor,fill opacity=0.20] (230.31, 54.50) circle (  2.13);

\path[fill=fillColor,fill opacity=0.20] (224.85, 63.94) circle (  2.13);

\path[fill=fillColor,fill opacity=0.20] (223.10, 62.49) circle (  2.13);

\path[fill=fillColor,fill opacity=0.20] (232.27, 53.98) circle (  2.13);

\path[fill=fillColor,fill opacity=0.20] (230.53, 51.28) circle (  2.13);

\path[fill=fillColor,fill opacity=0.20] (194.47, 60.62) circle (  2.13);

\path[fill=fillColor,fill opacity=0.20] (171.53, 81.70) circle (  2.13);

\path[fill=fillColor,fill opacity=0.20] (196.88, 81.70) circle (  2.13);

\path[fill=fillColor,fill opacity=0.20] (174.37, 68.72) circle (  2.13);

\path[fill=fillColor,fill opacity=0.20] (184.42, 58.34) circle (  2.13);

\path[fill=fillColor,fill opacity=0.20] (198.84, 54.08) circle (  2.13);

\path[fill=fillColor,fill opacity=0.20] (205.40, 56.05) circle (  2.13);

\path[fill=fillColor,fill opacity=0.20] (205.40, 64.15) circle (  2.13);

\path[fill=fillColor,fill opacity=0.20] (205.62, 64.77) circle (  2.13);

\path[fill=fillColor,fill opacity=0.20] (204.96, 62.18) circle (  2.13);

\path[fill=fillColor,fill opacity=0.20] (204.96, 63.32) circle (  2.13);

\path[fill=fillColor,fill opacity=0.20] (205.62, 65.29) circle (  2.13);

\path[fill=fillColor,fill opacity=0.20] (201.03, 60.62) circle (  2.13);

\path[fill=fillColor,fill opacity=0.20] (196.44, 52.42) circle (  2.13);

\path[fill=fillColor,fill opacity=0.20] (194.91, 56.57) circle (  2.13);

\path[fill=fillColor,fill opacity=0.20] (194.91, 64.36) circle (  2.13);

\path[fill=fillColor,fill opacity=0.20] (198.84, 58.65) circle (  2.13);

\path[fill=fillColor,fill opacity=0.20] (193.60, 60.21) circle (  2.13);

\path[fill=fillColor,fill opacity=0.20] (194.26, 80.24) circle (  2.13);

\path[fill=fillColor,fill opacity=0.20] (196.44, 80.97) circle (  2.13);

\path[fill=fillColor,fill opacity=0.20] (201.03, 72.87) circle (  2.13);

\path[fill=fillColor,fill opacity=0.20] (209.55, 61.66) circle (  2.13);

\path[fill=fillColor,fill opacity=0.20] (210.86, 61.87) circle (  2.13);

\path[fill=fillColor,fill opacity=0.20] (206.27, 65.92) circle (  2.13);

\path[fill=fillColor,fill opacity=0.20] (205.40, 61.66) circle (  2.13);

\path[fill=fillColor,fill opacity=0.20] (216.98, 63.42) circle (  2.13);

\path[fill=fillColor,fill opacity=0.20] (226.81, 64.67) circle (  2.13);

\path[fill=fillColor,fill opacity=0.20] (216.54, 53.56) circle (  2.13);

\path[fill=fillColor,fill opacity=0.20] (218.51, 49.41) circle (  2.13);

\path[fill=fillColor,fill opacity=0.20] (221.35, 58.65) circle (  2.13);

\path[fill=fillColor,fill opacity=0.20] (217.42, 61.97) circle (  2.13);

\path[fill=fillColor,fill opacity=0.20] (216.98, 64.26) circle (  2.13);

\path[fill=fillColor,fill opacity=0.20] (218.51, 65.61) circle (  2.13);

\path[fill=fillColor,fill opacity=0.20] (221.35, 56.99) circle (  2.13);

\path[fill=fillColor,fill opacity=0.20] (222.44, 56.68) circle (  2.13);

\path[fill=fillColor,fill opacity=0.20] (217.63, 70.07) circle (  2.13);

\path[fill=fillColor,fill opacity=0.20] (220.69, 66.02) circle (  2.13);

\path[fill=fillColor,fill opacity=0.20] (230.74, 49.41) circle (  2.13);

\path[fill=fillColor,fill opacity=0.20] (222.44, 44.01) circle (  2.13);

\path[fill=fillColor,fill opacity=0.20] (190.76, 49.62) circle (  2.13);

\path[fill=fillColor,fill opacity=0.20] (168.04, 75.26) circle (  2.13);

\path[fill=fillColor,fill opacity=0.20] (172.62, 90.00) circle (  2.13);

\path[fill=fillColor,fill opacity=0.20] (176.99, 74.43) circle (  2.13);

\path[fill=fillColor,fill opacity=0.20] (189.01, 55.02) circle (  2.13);

\path[fill=fillColor,fill opacity=0.20] (194.69, 51.07) circle (  2.13);

\path[fill=fillColor,fill opacity=0.20] (197.53, 55.74) circle (  2.13);

\path[fill=fillColor,fill opacity=0.20] (204.96, 50.65) circle (  2.13);

\path[fill=fillColor,fill opacity=0.20] (204.31, 54.91) circle (  2.13);

\path[fill=fillColor,fill opacity=0.20] (203.43, 64.05) circle (  2.13);

\path[fill=fillColor,fill opacity=0.20] (211.52, 54.60) circle (  2.13);

\path[fill=fillColor,fill opacity=0.20] (201.68, 51.28) circle (  2.13);

\path[fill=fillColor,fill opacity=0.20] (199.06, 69.76) circle (  2.13);

\path[fill=fillColor,fill opacity=0.20] (197.75, 70.80) circle (  2.13);

\path[fill=fillColor,fill opacity=0.20] (198.19, 47.33) circle (  2.13);

\path[fill=fillColor,fill opacity=0.20] (196.66, 45.67) circle (  2.13);

\path[fill=fillColor,fill opacity=0.20] (192.94, 70.80) circle (  2.13);

\path[fill=fillColor,fill opacity=0.20] (200.15, 81.07) circle (  2.13);

\path[fill=fillColor,fill opacity=0.20] (208.68, 59.38) circle (  2.13);

\path[fill=fillColor,fill opacity=0.20] (205.84, 41.31) circle (  2.13);

\path[fill=fillColor,fill opacity=0.20] (193.82, 52.32) circle (  2.13);

\path[fill=fillColor,fill opacity=0.20] (193.38, 66.95) circle (  2.13);

\path[fill=fillColor,fill opacity=0.20] (201.03, 69.86) circle (  2.13);

\path[fill=fillColor,fill opacity=0.20] (201.25, 63.01) circle (  2.13);

\path[fill=fillColor,fill opacity=0.20] (203.21, 49.20) circle (  2.13);

\path[fill=fillColor,fill opacity=0.20] (204.09, 43.80) circle (  2.13);

\path[fill=fillColor,fill opacity=0.20] (206.71, 53.35) circle (  2.13);

\path[fill=fillColor,fill opacity=0.20] (207.80, 61.04) circle (  2.13);

\path[fill=fillColor,fill opacity=0.20] (214.58, 59.69) circle (  2.13);

\path[fill=fillColor,fill opacity=0.20] (217.20, 53.98) circle (  2.13);

\path[fill=fillColor,fill opacity=0.20] (217.42, 46.81) circle (  2.13);

\path[fill=fillColor,fill opacity=0.20] (231.18, 47.23) circle (  2.13);

\path[fill=fillColor,fill opacity=0.20] (217.42, 56.16) circle (  2.13);

\path[fill=fillColor,fill opacity=0.20] (214.36, 60.00) circle (  2.13);

\path[fill=fillColor,fill opacity=0.20] (212.39, 62.28) circle (  2.13);

\path[fill=fillColor,fill opacity=0.20] (215.01, 63.63) circle (  2.13);

\path[fill=fillColor,fill opacity=0.20] (217.42, 58.23) circle (  2.13);

\path[fill=fillColor,fill opacity=0.20] (217.42, 56.99) circle (  2.13);

\path[fill=fillColor,fill opacity=0.20] (218.51, 66.33) circle (  2.13);

\path[fill=fillColor,fill opacity=0.20] (225.50, 65.81) circle (  2.13);

\path[fill=fillColor,fill opacity=0.20] (221.57, 49.51) circle (  2.13);

\path[fill=fillColor,fill opacity=0.20] (185.08, 48.58) circle (  2.13);

\path[fill=fillColor,fill opacity=0.20] (175.25, 74.33) circle (  2.13);

\path[fill=fillColor,fill opacity=0.20] (179.62, 70.07) circle (  2.13);

\path[fill=fillColor,fill opacity=0.20] (181.58, 63.01) circle (  2.13);

\path[fill=fillColor,fill opacity=0.20] (193.16, 52.63) circle (  2.13);

\path[fill=fillColor,fill opacity=0.20] (194.26, 59.38) circle (  2.13);

\path[fill=fillColor,fill opacity=0.20] (199.06, 64.05) circle (  2.13);

\path[fill=fillColor,fill opacity=0.20] (204.31, 50.86) circle (  2.13);

\path[fill=fillColor,fill opacity=0.20] (209.33, 51.28) circle (  2.13);

\path[fill=fillColor,fill opacity=0.20] (201.68, 69.34) circle (  2.13);

\path[fill=fillColor,fill opacity=0.20] (200.15, 64.46) circle (  2.13);

\path[fill=fillColor,fill opacity=0.20] (200.37, 46.09) circle (  2.13);

\path[fill=fillColor,fill opacity=0.20] (204.96, 51.17) circle (  2.13);

\path[fill=fillColor,fill opacity=0.20] (193.82, 69.76) circle (  2.13);

\path[fill=fillColor,fill opacity=0.20] (192.29, 72.56) circle (  2.13);

\path[fill=fillColor,fill opacity=0.20] (199.72, 61.45) circle (  2.13);

\path[fill=fillColor,fill opacity=0.20] (199.50, 55.64) circle (  2.13);

\path[fill=fillColor,fill opacity=0.20] (198.19, 64.46) circle (  2.13);

\path[fill=fillColor,fill opacity=0.20] (205.40, 65.09) circle (  2.13);

\path[fill=fillColor,fill opacity=0.20] (204.09, 50.97) circle (  2.13);

\path[fill=fillColor,fill opacity=0.20] (196.00, 58.96) circle (  2.13);

\path[fill=fillColor,fill opacity=0.20] (189.23, 77.86) circle (  2.13);

\path[fill=fillColor,fill opacity=0.20] (195.35, 83.77) circle (  2.13);

\path[fill=fillColor,fill opacity=0.20] (202.12, 76.19) circle (  2.13);

\path[fill=fillColor,fill opacity=0.20] (202.78, 60.93) circle (  2.13);

\path[fill=fillColor,fill opacity=0.20] (206.05, 40.48) circle (  2.13);

\path[fill=fillColor,fill opacity=0.20] (207.58, 38.51) circle (  2.13);

\path[fill=fillColor,fill opacity=0.20] (207.58, 41.52) circle (  2.13);

\path[fill=fillColor,fill opacity=0.20] (207.15, 44.74) circle (  2.13);

\path[fill=fillColor,fill opacity=0.20] (210.21, 53.04) circle (  2.13);

\path[fill=fillColor,fill opacity=0.20] (217.20, 55.22) circle (  2.13);

\path[fill=fillColor,fill opacity=0.20] (211.30, 52.21) circle (  2.13);

\path[fill=fillColor,fill opacity=0.20] (216.32, 50.86) circle (  2.13);

\path[fill=fillColor,fill opacity=0.20] (218.29, 52.42) circle (  2.13);

\path[fill=fillColor,fill opacity=0.20] (215.45, 53.04) circle (  2.13);

\path[fill=fillColor,fill opacity=0.20] (209.11, 53.98) circle (  2.13);

\path[fill=fillColor,fill opacity=0.20] (211.08, 58.03) circle (  2.13);

\path[fill=fillColor,fill opacity=0.20] (210.64, 59.38) circle (  2.13);

\path[fill=fillColor,fill opacity=0.20] (209.33, 55.64) circle (  2.13);

\path[fill=fillColor,fill opacity=0.20] (210.64, 52.73) circle (  2.13);

\path[fill=fillColor,fill opacity=0.20] (217.42, 56.57) circle (  2.13);

\path[fill=fillColor,fill opacity=0.20] (226.16, 56.26) circle (  2.13);

\path[fill=fillColor,fill opacity=0.20] (212.17, 43.91) circle (  2.13);

\path[fill=fillColor,fill opacity=0.20] (200.37, 44.43) circle (  2.13);

\path[fill=fillColor,fill opacity=0.20] (179.40, 75.99) circle (  2.13);

\path[fill=fillColor,fill opacity=0.20] (172.41, 68.82) circle (  2.13);

\path[fill=fillColor,fill opacity=0.20] (181.80, 70.59) circle (  2.13);

\path[fill=fillColor,fill opacity=0.20] (184.42, 66.95) circle (  2.13);

\path[fill=fillColor,fill opacity=0.20] (191.85, 57.61) circle (  2.13);

\path[fill=fillColor,fill opacity=0.20] (200.37, 56.16) circle (  2.13);

\path[fill=fillColor,fill opacity=0.20] (202.12, 56.16) circle (  2.13);

\path[fill=fillColor,fill opacity=0.20] (199.94, 51.28) circle (  2.13);

\path[fill=fillColor,fill opacity=0.20] (201.25, 48.06) circle (  2.13);

\path[fill=fillColor,fill opacity=0.20] (206.27, 47.85) circle (  2.13);

\path[fill=fillColor,fill opacity=0.20] (201.47, 54.18) circle (  2.13);

\path[fill=fillColor,fill opacity=0.20] (197.75, 61.56) circle (  2.13);

\path[fill=fillColor,fill opacity=0.20] (206.93, 57.09) circle (  2.13);

\path[fill=fillColor,fill opacity=0.20] (194.69, 55.12) circle (  2.13);

\path[fill=fillColor,fill opacity=0.20] (192.94, 61.35) circle (  2.13);

\path[fill=fillColor,fill opacity=0.20] (196.66, 58.55) circle (  2.13);

\path[fill=fillColor,fill opacity=0.20] (200.37, 52.52) circle (  2.13);

\path[fill=fillColor,fill opacity=0.20] (197.10, 60.52) circle (  2.13);

\path[fill=fillColor,fill opacity=0.20] (190.10, 93.12) circle (  2.13);

\path[fill=fillColor,fill opacity=0.20] (190.76, 78.17) circle (  2.13);

\path[fill=fillColor,fill opacity=0.20] (189.01, 67.27) circle (  2.13);

\path[fill=fillColor,fill opacity=0.20] (198.84, 72.35) circle (  2.13);

\path[fill=fillColor,fill opacity=0.20] (194.26, 71.32) circle (  2.13);

\path[fill=fillColor,fill opacity=0.20] (200.81, 55.02) circle (  2.13);

\path[fill=fillColor,fill opacity=0.20] (196.22, 49.51) circle (  2.13);

\path[fill=fillColor,fill opacity=0.20] (204.09, 57.40) circle (  2.13);

\path[fill=fillColor,fill opacity=0.20] (204.52, 56.47) circle (  2.13);

\path[fill=fillColor,fill opacity=0.20] (204.52, 54.39) circle (  2.13);

\path[fill=fillColor,fill opacity=0.20] (198.19, 63.11) circle (  2.13);

\path[fill=fillColor,fill opacity=0.20] (188.57, 66.12) circle (  2.13);

\path[fill=fillColor,fill opacity=0.20] (176.34, 75.99) circle (  2.13);

\path[fill=fillColor,fill opacity=0.20] (166.29,105.58) circle (  2.13);

\path[fill=fillColor,fill opacity=0.20] (200.59, 77.54) circle (  2.13);

\path[fill=fillColor,fill opacity=0.20] (208.68, 71.52) circle (  2.13);

\path[fill=fillColor,fill opacity=0.20] (208.68, 68.93) circle (  2.13);

\path[fill=fillColor,fill opacity=0.20] (210.64, 58.96) circle (  2.13);

\path[fill=fillColor,fill opacity=0.20] (225.50, 49.82) circle (  2.13);

\path[fill=fillColor,fill opacity=0.20] (210.64, 51.69) circle (  2.13);

\path[fill=fillColor,fill opacity=0.20] (215.01, 53.56) circle (  2.13);

\path[fill=fillColor,fill opacity=0.20] (213.70, 53.56) circle (  2.13);

\path[fill=fillColor,fill opacity=0.20] (212.17, 54.60) circle (  2.13);

\path[fill=fillColor,fill opacity=0.20] (212.61, 53.35) circle (  2.13);

\path[fill=fillColor,fill opacity=0.20] (213.05, 52.00) circle (  2.13);

\path[fill=fillColor,fill opacity=0.20] (211.95, 53.87) circle (  2.13);

\path[fill=fillColor,fill opacity=0.20] (211.95, 55.53) circle (  2.13);

\path[fill=fillColor,fill opacity=0.20] (216.32, 55.33) circle (  2.13);

\path[fill=fillColor,fill opacity=0.20] (209.11, 52.32) circle (  2.13);

\path[fill=fillColor,fill opacity=0.20] (216.54, 49.72) circle (  2.13);

\path[fill=fillColor,fill opacity=0.20] (211.52, 48.68) circle (  2.13);

\path[fill=fillColor,fill opacity=0.20] (198.19, 53.15) circle (  2.13);

\path[fill=fillColor,fill opacity=0.20] (182.24, 71.42) circle (  2.13);

\path[fill=fillColor,fill opacity=0.20] (182.24, 73.29) circle (  2.13);

\path[fill=fillColor,fill opacity=0.20] (184.86, 69.55) circle (  2.13);

\path[fill=fillColor,fill opacity=0.20] (183.11, 62.18) circle (  2.13);

\path[fill=fillColor,fill opacity=0.20] (187.04, 59.69) circle (  2.13);

\path[fill=fillColor,fill opacity=0.20] (194.26, 58.23) circle (  2.13);

\path[fill=fillColor,fill opacity=0.20] (206.27, 45.15) circle (  2.13);

\path[fill=fillColor,fill opacity=0.20] (201.25, 42.87) circle (  2.13);

\path[fill=fillColor,fill opacity=0.20] (200.81, 48.79) circle (  2.13);

\path[fill=fillColor,fill opacity=0.20] (200.15, 58.34) circle (  2.13);

\path[fill=fillColor,fill opacity=0.20] (199.28, 68.10) circle (  2.13);

\path[fill=fillColor,fill opacity=0.20] (199.06, 63.01) circle (  2.13);

\path[fill=fillColor,fill opacity=0.20] (192.29, 51.17) circle (  2.13);

\path[fill=fillColor,fill opacity=0.20] (201.25, 51.49) circle (  2.13);

\path[fill=fillColor,fill opacity=0.20] (197.53, 58.65) circle (  2.13);

\path[fill=fillColor,fill opacity=0.20] (193.82, 66.85) circle (  2.13);

\path[fill=fillColor,fill opacity=0.20] (191.85, 72.66) circle (  2.13);

\path[fill=fillColor,fill opacity=0.20] (191.41, 70.17) circle (  2.13);

\path[fill=fillColor,fill opacity=0.20] (193.38, 64.46) circle (  2.13);

\path[fill=fillColor,fill opacity=0.20] (198.84, 63.01) circle (  2.13);

\path[fill=fillColor,fill opacity=0.20] (196.66, 67.99) circle (  2.13);

\path[fill=fillColor,fill opacity=0.20] (193.82, 77.23) circle (  2.13);

\path[fill=fillColor,fill opacity=0.20] (191.41, 86.89) circle (  2.13);

\path[fill=fillColor,fill opacity=0.20] (192.94, 77.86) circle (  2.13);

\path[fill=fillColor,fill opacity=0.20] (197.10, 68.82) circle (  2.13);

\path[fill=fillColor,fill opacity=0.20] (200.59, 66.12) circle (  2.13);

\path[fill=fillColor,fill opacity=0.20] (201.25, 63.84) circle (  2.13);

\path[fill=fillColor,fill opacity=0.20] (204.31, 60.73) circle (  2.13);

\path[fill=fillColor,fill opacity=0.20] (205.40, 58.75) circle (  2.13);

\path[fill=fillColor,fill opacity=0.20] (203.00, 53.04) circle (  2.13);

\path[fill=fillColor,fill opacity=0.20] (204.09, 51.69) circle (  2.13);

\path[fill=fillColor,fill opacity=0.20] (206.05, 61.97) circle (  2.13);

\path[fill=fillColor,fill opacity=0.20] (205.62, 64.88) circle (  2.13);

\path[fill=fillColor,fill opacity=0.20] (204.74, 52.63) circle (  2.13);

\path[fill=fillColor,fill opacity=0.20] (201.47, 47.96) circle (  2.13);

\path[fill=fillColor,fill opacity=0.20] (200.59, 57.51) circle (  2.13);

\path[fill=fillColor,fill opacity=0.20] (203.00, 64.26) circle (  2.13);

\path[fill=fillColor,fill opacity=0.20] (184.86, 74.53) circle (  2.13);

\path[fill=fillColor,fill opacity=0.20] (175.25, 93.12) circle (  2.13);

\path[fill=fillColor,fill opacity=0.20] (190.32,107.65) circle (  2.13);

\path[fill=fillColor,fill opacity=0.20] (196.00, 97.27) circle (  2.13);

\path[fill=fillColor,fill opacity=0.20] (207.58, 85.85) circle (  2.13);

\path[fill=fillColor,fill opacity=0.20] (216.32, 70.59) circle (  2.13);

\path[fill=fillColor,fill opacity=0.20] (212.17, 61.97) circle (  2.13);

\path[fill=fillColor,fill opacity=0.20] (207.37, 60.83) circle (  2.13);

\path[fill=fillColor,fill opacity=0.20] (209.11, 60.52) circle (  2.13);

\path[fill=fillColor,fill opacity=0.20] (216.98, 60.62) circle (  2.13);

\path[fill=fillColor,fill opacity=0.20] (214.14, 59.48) circle (  2.13);

\path[fill=fillColor,fill opacity=0.20] (207.58, 57.82) circle (  2.13);

\path[fill=fillColor,fill opacity=0.20] (209.77, 63.32) circle (  2.13);

\path[fill=fillColor,fill opacity=0.20] (209.55, 66.54) circle (  2.13);

\path[fill=fillColor,fill opacity=0.20] (203.87, 62.70) circle (  2.13);

\path[fill=fillColor,fill opacity=0.20] (198.84, 71.32) circle (  2.13);

\path[fill=fillColor,fill opacity=0.20] (178.74, 88.96) circle (  2.13);

\path[fill=fillColor,fill opacity=0.20] (181.36, 68.72) circle (  2.13);

\path[fill=fillColor,fill opacity=0.20] (183.33, 46.71) circle (  2.13);

\path[fill=fillColor,fill opacity=0.20] (192.51, 42.66) circle (  2.13);

\path[fill=fillColor,fill opacity=0.20] (202.34, 48.68) circle (  2.13);

\path[fill=fillColor,fill opacity=0.20] (206.71, 54.50) circle (  2.13);

\path[fill=fillColor,fill opacity=0.20] (204.96, 56.88) circle (  2.13);

\path[fill=fillColor,fill opacity=0.20] (196.44, 55.53) circle (  2.13);

\path[fill=fillColor,fill opacity=0.20] (201.25, 53.77) circle (  2.13);

\path[fill=fillColor,fill opacity=0.20] (198.84, 53.15) circle (  2.13);

\path[fill=fillColor,fill opacity=0.20] (200.59, 55.43) circle (  2.13);

\path[fill=fillColor,fill opacity=0.20] (196.66, 63.53) circle (  2.13);

\path[fill=fillColor,fill opacity=0.20] (196.44, 68.72) circle (  2.13);

\path[fill=fillColor,fill opacity=0.20] (198.84, 65.40) circle (  2.13);

\path[fill=fillColor,fill opacity=0.20] (197.97, 61.87) circle (  2.13);

\path[fill=fillColor,fill opacity=0.20] (201.90, 58.13) circle (  2.13);

\path[fill=fillColor,fill opacity=0.20] (199.94, 53.35) circle (  2.13);

\path[fill=fillColor,fill opacity=0.20] (198.41, 54.81) circle (  2.13);

\path[fill=fillColor,fill opacity=0.20] (203.43, 56.99) circle (  2.13);

\path[fill=fillColor,fill opacity=0.20] (200.37, 58.03) circle (  2.13);

\path[fill=fillColor,fill opacity=0.20] (199.94, 63.32) circle (  2.13);

\path[fill=fillColor,fill opacity=0.20] (201.03, 67.99) circle (  2.13);

\path[fill=fillColor,fill opacity=0.20] (217.20, 68.41) circle (  2.13);

\path[fill=fillColor,fill opacity=0.20] (209.33, 63.01) circle (  2.13);

\path[fill=fillColor,fill opacity=0.20] (206.05, 56.78) circle (  2.13);

\path[fill=fillColor,fill opacity=0.20] (207.58, 57.92) circle (  2.13);

\path[fill=fillColor,fill opacity=0.20] (211.08, 61.04) circle (  2.13);

\path[fill=fillColor,fill opacity=0.20] (213.05, 54.08) circle (  2.13);

\path[fill=fillColor,fill opacity=0.20] (209.11, 47.64) circle (  2.13);

\path[fill=fillColor,fill opacity=0.20] (202.56, 56.57) circle (  2.13);

\path[fill=fillColor,fill opacity=0.20] (196.88, 69.76) circle (  2.13);

\path[fill=fillColor,fill opacity=0.20] (187.92, 76.51) circle (  2.13);

\path[fill=fillColor,fill opacity=0.20] (178.52, 82.74) circle (  2.13);

\path[fill=fillColor,fill opacity=0.20] (167.38,112.84) circle (  2.13);

\path[fill=fillColor,fill opacity=0.20] (199.28, 94.16) circle (  2.13);

\path[fill=fillColor,fill opacity=0.20] (195.57, 88.96) circle (  2.13);

\path[fill=fillColor,fill opacity=0.20] (198.41, 81.39) circle (  2.13);

\path[fill=fillColor,fill opacity=0.20] (205.40, 79.00) circle (  2.13);

\path[fill=fillColor,fill opacity=0.20] (206.05, 77.44) circle (  2.13);

\path[fill=fillColor,fill opacity=0.20] (200.37, 75.78) circle (  2.13);

\path[fill=fillColor,fill opacity=0.20] (199.94, 83.77) circle (  2.13);

\path[fill=fillColor,fill opacity=0.20] (176.12, 82.74) circle (  2.13);

\path[fill=fillColor,fill opacity=0.20] (185.95, 64.26) circle (  2.13);

\path[fill=fillColor,fill opacity=0.20] (193.82, 50.55) circle (  2.13);

\path[fill=fillColor,fill opacity=0.20] (199.28, 47.23) circle (  2.13);

\path[fill=fillColor,fill opacity=0.20] (209.11, 46.19) circle (  2.13);

\path[fill=fillColor,fill opacity=0.20] (206.05, 50.34) circle (  2.13);

\path[fill=fillColor,fill opacity=0.20] (202.34, 57.51) circle (  2.13);

\path[fill=fillColor,fill opacity=0.20] (199.28, 58.03) circle (  2.13);

\path[fill=fillColor,fill opacity=0.20] (201.68, 57.40) circle (  2.13);

\path[fill=fillColor,fill opacity=0.20] (203.43, 60.52) circle (  2.13);

\path[fill=fillColor,fill opacity=0.20] (202.34, 61.45) circle (  2.13);

\path[fill=fillColor,fill opacity=0.20] (205.62, 64.15) circle (  2.13);

\path[fill=fillColor,fill opacity=0.20] (202.12, 68.20) circle (  2.13);

\path[fill=fillColor,fill opacity=0.20] (201.03, 66.02) circle (  2.13);

\path[fill=fillColor,fill opacity=0.20] (200.59, 61.97) circle (  2.13);

\path[fill=fillColor,fill opacity=0.20] (204.74, 60.83) circle (  2.13);

\path[fill=fillColor,fill opacity=0.20] (205.40, 60.21) circle (  2.13);

\path[fill=fillColor,fill opacity=0.20] (203.65, 60.52) circle (  2.13);

\path[fill=fillColor,fill opacity=0.20] (210.42, 64.46) circle (  2.13);

\path[fill=fillColor,fill opacity=0.20] (212.17, 71.21) circle (  2.13);

\path[fill=fillColor,fill opacity=0.20] (211.95, 69.24) circle (  2.13);

\path[fill=fillColor,fill opacity=0.20] (213.48, 57.30) circle (  2.13);

\path[fill=fillColor,fill opacity=0.20] (212.17, 52.21) circle (  2.13);

\path[fill=fillColor,fill opacity=0.20] (211.08, 57.82) circle (  2.13);

\path[fill=fillColor,fill opacity=0.20] (204.52, 59.79) circle (  2.13);

\path[fill=fillColor,fill opacity=0.20] (191.41, 58.96) circle (  2.13);

\path[fill=fillColor,fill opacity=0.20] (181.36, 69.76) circle (  2.13);

\path[fill=fillColor,fill opacity=0.20] (175.68, 87.93) circle (  2.13);

\path[fill=fillColor,fill opacity=0.20] (180.93, 73.50) circle (  2.13);

\path[fill=fillColor,fill opacity=0.20] (203.00, 62.18) circle (  2.13);

\path[fill=fillColor,fill opacity=0.20] (194.69, 52.73) circle (  2.13);

\path[fill=fillColor,fill opacity=0.20] (194.47, 57.09) circle (  2.13);

\path[fill=fillColor,fill opacity=0.20] (196.66, 57.61) circle (  2.13);

\path[fill=fillColor,fill opacity=0.20] (200.59, 48.79) circle (  2.13);

\path[fill=fillColor,fill opacity=0.20] (203.43, 48.99) circle (  2.13);

\path[fill=fillColor,fill opacity=0.20] (205.84, 58.65) circle (  2.13);

\path[fill=fillColor,fill opacity=0.20] (206.71, 62.49) circle (  2.13);

\path[fill=fillColor,fill opacity=0.20] (209.11, 62.28) circle (  2.13);

\path[fill=fillColor,fill opacity=0.20] (208.68, 64.26) circle (  2.13);

\path[fill=fillColor,fill opacity=0.20] (207.80, 64.26) circle (  2.13);

\path[fill=fillColor,fill opacity=0.20] (212.17, 60.31) circle (  2.13);

\path[fill=fillColor,fill opacity=0.20] (213.48, 57.92) circle (  2.13);

\path[fill=fillColor,fill opacity=0.20] (211.30, 57.30) circle (  2.13);

\path[fill=fillColor,fill opacity=0.20] (218.29, 53.98) circle (  2.13);

\path[fill=fillColor,fill opacity=0.20] (219.16, 55.74) circle (  2.13);

\path[fill=fillColor,fill opacity=0.20] (208.89, 60.21) circle (  2.13);

\path[fill=fillColor,fill opacity=0.20] (206.27, 56.16) circle (  2.13);

\path[fill=fillColor,fill opacity=0.20] (202.34, 50.24) circle (  2.13);

\path[fill=fillColor,fill opacity=0.20] (197.75, 54.91) circle (  2.13);

\path[fill=fillColor,fill opacity=0.20] (188.57, 66.95) circle (  2.13);

\path[fill=fillColor,fill opacity=0.20] (184.86, 65.40) circle (  2.13);

\path[fill=fillColor,fill opacity=0.20] (188.79, 61.66) circle (  2.13);

\path[fill=fillColor,fill opacity=0.20] (197.10, 58.96) circle (  2.13);

\path[fill=fillColor,fill opacity=0.20] (219.60, 64.88) circle (  2.13);

\path[fill=fillColor,fill opacity=0.20] (199.06, 61.66) circle (  2.13);

\path[fill=fillColor,fill opacity=0.20] (206.93, 46.71) circle (  2.13);

\path[fill=fillColor,fill opacity=0.20] (212.17, 43.39) circle (  2.13);

\path[fill=fillColor,fill opacity=0.20] (211.95, 52.21) circle (  2.13);

\path[fill=fillColor,fill opacity=0.20] (215.23, 52.42) circle (  2.13);

\path[fill=fillColor,fill opacity=0.20] (213.05, 51.38) circle (  2.13);

\path[fill=fillColor,fill opacity=0.20] (208.89, 57.61) circle (  2.13);

\path[fill=fillColor,fill opacity=0.20] (214.58, 58.86) circle (  2.13);

\path[fill=fillColor,fill opacity=0.20] (199.94, 53.56) circle (  2.13);

\path[fill=fillColor,fill opacity=0.20] (194.47, 54.39) circle (  2.13);

\path[fill=fillColor,fill opacity=0.20] (195.13, 60.10) circle (  2.13);

\path[fill=fillColor,fill opacity=0.20] (183.99, 64.15) circle (  2.13);

\path[fill=fillColor,fill opacity=0.20] (179.62, 67.37) circle (  2.13);

\path[fill=fillColor,fill opacity=0.20] (196.66, 58.03) circle (  2.13);

\path[fill=fillColor,fill opacity=0.20] (200.59, 47.75) circle (  2.13);

\path[fill=fillColor,fill opacity=0.20] (195.57, 50.65) circle (  2.13);

\path[fill=fillColor,fill opacity=0.20] (196.22, 58.44) circle (  2.13);

\path[fill=fillColor,fill opacity=0.20] (196.44, 64.46) circle (  2.13);

\path[fill=fillColor,fill opacity=0.20] (189.89, 70.90) circle (  2.13);

\path[fill=fillColor,fill opacity=0.20] (185.30, 77.44) circle (  2.13);

\path[fill=fillColor,fill opacity=0.20] (182.46, 76.09) circle (  2.13);

\path[fill=fillColor,fill opacity=0.20] (177.65, 69.65) circle (  2.13);

\path[fill=fillColor,fill opacity=0.20] (197.31,108.69) circle (  2.13);

\path[fill=fillColor,fill opacity=0.20] (185.08,112.84) circle (  2.13);

\path[fill=fillColor,fill opacity=0.20] (204.74, 64.26) circle (  2.13);

\path[fill=fillColor,fill opacity=0.20] (209.99, 63.11) circle (  2.13);

\path[fill=fillColor,fill opacity=0.20] (210.21, 72.15) circle (  2.13);

\path[fill=fillColor,fill opacity=0.20] (209.55, 82.74) circle (  2.13);

\path[fill=fillColor,fill opacity=0.20] (205.62, 79.72) circle (  2.13);

\path[fill=fillColor,fill opacity=0.20] (180.05, 90.00) circle (  2.13);

\path[fill=fillColor,fill opacity=0.20] (177.21, 69.76) circle (  2.13);

\path[fill=fillColor,fill opacity=0.20] (182.24, 58.96) circle (  2.13);

\path[fill=fillColor,fill opacity=0.20] (183.99, 69.24) circle (  2.13);

\path[fill=fillColor,fill opacity=0.20] (183.11, 63.94) circle (  2.13);

\path[fill=fillColor,fill opacity=0.20] (180.49, 50.97) circle (  2.13);

\path[fill=fillColor,fill opacity=0.20] (201.90, 53.87) circle (  2.13);

\path[fill=fillColor,fill opacity=0.20] (215.67, 52.42) circle (  2.13);

\path[fill=fillColor,fill opacity=0.20] (217.20, 66.75) circle (  2.13);

\path[fill=fillColor,fill opacity=0.20] (223.32, 70.59) circle (  2.13);

\path[fill=fillColor,fill opacity=0.20] (225.28, 71.00) circle (  2.13);

\path[fill=fillColor,fill opacity=0.20] (220.04, 76.51) circle (  2.13);

\path[fill=fillColor,fill opacity=0.20] (212.61, 85.85) circle (  2.13);

\path[fill=fillColor,fill opacity=0.20] (202.34, 96.23) circle (  2.13);

\path[fill=fillColor,fill opacity=0.20] (180.49, 92.08) circle (  2.13);

\path[fill=fillColor,fill opacity=0.20] (181.15, 78.89) circle (  2.13);

\path[fill=fillColor,fill opacity=0.20] (182.46, 81.70) circle (  2.13);

\path[fill=fillColor,fill opacity=0.20] (190.54, 68.20) circle (  2.13);

\path[fill=fillColor,fill opacity=0.20] (199.06, 56.47) circle (  2.13);

\path[fill=fillColor,fill opacity=0.20] (201.90, 62.28) circle (  2.13);

\path[fill=fillColor,fill opacity=0.20] (198.63, 60.10) circle (  2.13);

\path[fill=fillColor,fill opacity=0.20] (194.47, 55.02) circle (  2.13);

\path[fill=fillColor,fill opacity=0.20] (190.10, 54.18) circle (  2.13);

\path[fill=fillColor,fill opacity=0.20] (185.52, 59.06) circle (  2.13);

\path[fill=fillColor,fill opacity=0.20] (185.73, 74.85) circle (  2.13);

\path[fill=fillColor,fill opacity=0.20] (219.38, 47.12) circle (  2.13);

\path[fill=fillColor,fill opacity=0.20] (220.91, 71.52) circle (  2.13);

\path[fill=fillColor,fill opacity=0.20] (224.19, 68.41) circle (  2.13);

\path[fill=fillColor,fill opacity=0.20] (232.93, 63.22) circle (  2.13);

\path[fill=fillColor,fill opacity=0.20] (234.02, 58.34) circle (  2.13);

\path[fill=fillColor,fill opacity=0.20] (229.00, 63.11) circle (  2.13);

\path[fill=fillColor,fill opacity=0.20] (217.42, 68.51) circle (  2.13);

\path[fill=fillColor,fill opacity=0.20] (206.93, 68.41) circle (  2.13);

\path[fill=fillColor,fill opacity=0.20] (197.75, 85.85) circle (  2.13);

\path[fill=fillColor,fill opacity=0.20] (206.71, 81.49) circle (  2.13);

\path[fill=fillColor,fill opacity=0.20] (190.10, 73.81) circle (  2.13);

\path[fill=fillColor,fill opacity=0.20] (197.53, 71.83) circle (  2.13);

\path[fill=fillColor,fill opacity=0.20] (196.66, 60.10) circle (  2.13);

\path[fill=fillColor,fill opacity=0.20] (201.68, 52.11) circle (  2.13);

\path[fill=fillColor,fill opacity=0.20] (204.96, 65.19) circle (  2.13);

\path[fill=fillColor,fill opacity=0.20] (206.71, 75.05) circle (  2.13);

\path[fill=fillColor,fill opacity=0.20] (205.40, 61.04) circle (  2.13);

\path[fill=fillColor,fill opacity=0.20] (197.97, 55.22) circle (  2.13);

\path[fill=fillColor,fill opacity=0.20] (195.57, 65.40) circle (  2.13);

\path[fill=fillColor,fill opacity=0.20] (198.84, 49.82) circle (  2.13);

\path[fill=fillColor,fill opacity=0.20] (227.25, 62.91) circle (  2.13);

\path[fill=fillColor,fill opacity=0.20] (225.72, 82.74) circle (  2.13);

\path[fill=fillColor,fill opacity=0.20] (233.59, 71.11) circle (  2.13);

\path[fill=fillColor,fill opacity=0.20] (234.68, 56.68) circle (  2.13);

\path[fill=fillColor,fill opacity=0.20] (233.15, 46.92) circle (  2.13);

\path[fill=fillColor,fill opacity=0.20] (232.27, 46.50) circle (  2.13);

\path[fill=fillColor,fill opacity=0.20] (222.44, 43.28) circle (  2.13);

\path[fill=fillColor,fill opacity=0.20] (217.20, 40.48) circle (  2.13);

\path[fill=fillColor,fill opacity=0.20] (206.27, 59.48) circle (  2.13);

\path[fill=fillColor,fill opacity=0.20] (179.18, 79.93) circle (  2.13);

\path[fill=fillColor,fill opacity=0.20] (188.14, 49.72) circle (  2.13);

\path[fill=fillColor,fill opacity=0.20] (192.29, 70.69) circle (  2.13);

\path[fill=fillColor,fill opacity=0.20] (196.44, 84.81) circle (  2.13);

\path[fill=fillColor,fill opacity=0.20] (199.28, 69.34) circle (  2.13);

\path[fill=fillColor,fill opacity=0.20] (204.74, 63.74) circle (  2.13);

\path[fill=fillColor,fill opacity=0.20] (208.02, 67.89) circle (  2.13);

\path[fill=fillColor,fill opacity=0.20] (206.27, 70.59) circle (  2.13);

\path[fill=fillColor,fill opacity=0.20] (205.18, 66.33) circle (  2.13);

\path[fill=fillColor,fill opacity=0.20] (194.69, 55.43) circle (  2.13);

\path[fill=fillColor,fill opacity=0.20] (185.73, 55.22) circle (  2.13);

\path[fill=fillColor,fill opacity=0.20] (177.21, 68.10) circle (  2.13);

\path[fill=fillColor,fill opacity=0.20] (197.97, 67.47) circle (  2.13);

\path[fill=fillColor,fill opacity=0.20] (225.06, 55.33) circle (  2.13);

\path[fill=fillColor,fill opacity=0.20] (225.50, 75.26) circle (  2.13);

\path[fill=fillColor,fill opacity=0.20] (234.24, 76.51) circle (  2.13);

\path[fill=fillColor,fill opacity=0.20] (235.33, 55.64) circle (  2.13);

\path[fill=fillColor,fill opacity=0.20] (235.11, 44.11) circle (  2.13);

\path[fill=fillColor,fill opacity=0.20] (233.80, 47.54) circle (  2.13);

\path[fill=fillColor,fill opacity=0.20] (226.37, 41.52) circle (  2.13);

\path[fill=fillColor,fill opacity=0.20] (216.11, 42.04) circle (  2.13);

\path[fill=fillColor,fill opacity=0.20] (205.62, 67.16) circle (  2.13);

\path[fill=fillColor,fill opacity=0.20] (175.25, 72.66) circle (  2.13);

\path[fill=fillColor,fill opacity=0.20] (192.94, 57.30) circle (  2.13);

\path[fill=fillColor,fill opacity=0.20] (201.68, 49.93) circle (  2.13);

\path[fill=fillColor,fill opacity=0.20] (199.94, 66.12) circle (  2.13);

\path[fill=fillColor,fill opacity=0.20] (201.25, 75.78) circle (  2.13);

\path[fill=fillColor,fill opacity=0.20] (204.31, 71.83) circle (  2.13);

\path[fill=fillColor,fill opacity=0.20] (207.80, 69.97) circle (  2.13);

\path[fill=fillColor,fill opacity=0.20] (209.33, 62.08) circle (  2.13);

\path[fill=fillColor,fill opacity=0.20] (206.93, 58.65) circle (  2.13);

\path[fill=fillColor,fill opacity=0.20] (204.31, 69.86) circle (  2.13);

\path[fill=fillColor,fill opacity=0.20] (196.22, 61.66) circle (  2.13);

\path[fill=fillColor,fill opacity=0.20] (173.28, 56.05) circle (  2.13);

\path[fill=fillColor,fill opacity=0.20] (197.10, 58.23) circle (  2.13);

\path[fill=fillColor,fill opacity=0.20] (220.91, 45.36) circle (  2.13);

\path[fill=fillColor,fill opacity=0.20] (223.53, 61.76) circle (  2.13);

\path[fill=fillColor,fill opacity=0.20] (234.46, 65.40) circle (  2.13);

\path[fill=fillColor,fill opacity=0.20] (238.17, 53.35) circle (  2.13);

\path[fill=fillColor,fill opacity=0.20] (233.15, 43.60) circle (  2.13);

\path[fill=fillColor,fill opacity=0.20] (230.31, 60.52) circle (  2.13);

\path[fill=fillColor,fill opacity=0.20] (229.22, 65.71) circle (  2.13);

\path[fill=fillColor,fill opacity=0.20] (226.37, 51.28) circle (  2.13);

\path[fill=fillColor,fill opacity=0.20] (216.54, 43.08) circle (  2.13);

\path[fill=fillColor,fill opacity=0.20] (208.02, 54.18) circle (  2.13);

\path[fill=fillColor,fill opacity=0.20] (194.26, 82.74) circle (  2.13);

\path[fill=fillColor,fill opacity=0.20] (183.11, 55.02) circle (  2.13);

\path[fill=fillColor,fill opacity=0.20] (198.63, 40.48) circle (  2.13);

\path[fill=fillColor,fill opacity=0.20] (202.56, 53.35) circle (  2.13);

\path[fill=fillColor,fill opacity=0.20] (208.46, 46.40) circle (  2.13);

\path[fill=fillColor,fill opacity=0.20] (210.42, 48.89) circle (  2.13);

\path[fill=fillColor,fill opacity=0.20] (213.05, 64.05) circle (  2.13);

\path[fill=fillColor,fill opacity=0.20] (215.45, 65.50) circle (  2.13);

\path[fill=fillColor,fill opacity=0.20] (212.17, 57.51) circle (  2.13);

\path[fill=fillColor,fill opacity=0.20] (208.89, 55.74) circle (  2.13);

\path[fill=fillColor,fill opacity=0.20] (203.43, 63.53) circle (  2.13);

\path[fill=fillColor,fill opacity=0.20] (189.67, 71.73) circle (  2.13);

\path[fill=fillColor,fill opacity=0.20] (219.38, 48.89) circle (  2.13);

\path[fill=fillColor,fill opacity=0.20] (220.91, 57.51) circle (  2.13);

\path[fill=fillColor,fill opacity=0.20] (234.68, 56.68) circle (  2.13);

\path[fill=fillColor,fill opacity=0.20] (237.30, 54.60) circle (  2.13);

\path[fill=fillColor,fill opacity=0.20] (231.18, 50.14) circle (  2.13);

\path[fill=fillColor,fill opacity=0.20] (227.90, 64.77) circle (  2.13);

\path[fill=fillColor,fill opacity=0.20] (225.28, 73.91) circle (  2.13);

\path[fill=fillColor,fill opacity=0.20] (222.44, 55.95) circle (  2.13);

\path[fill=fillColor,fill opacity=0.20] (216.32, 41.52) circle (  2.13);

\path[fill=fillColor,fill opacity=0.20] (204.52, 49.51) circle (  2.13);

\path[fill=fillColor,fill opacity=0.20] (191.63, 77.75) circle (  2.13);

\path[fill=fillColor,fill opacity=0.20] (177.43, 72.87) circle (  2.13);

\path[fill=fillColor,fill opacity=0.20] (194.04, 62.08) circle (  2.13);

\path[fill=fillColor,fill opacity=0.20] (204.31, 41.52) circle (  2.13);

\path[fill=fillColor,fill opacity=0.20] (208.46, 44.63) circle (  2.13);

\path[fill=fillColor,fill opacity=0.20] (213.48, 40.07) circle (  2.13);

\path[fill=fillColor,fill opacity=0.20] (223.53, 45.46) circle (  2.13);

\path[fill=fillColor,fill opacity=0.20] (220.04, 66.75) circle (  2.13);

\path[fill=fillColor,fill opacity=0.20] (217.42, 61.45) circle (  2.13);

\path[fill=fillColor,fill opacity=0.20] (214.14, 47.85) circle (  2.13);

\path[fill=fillColor,fill opacity=0.20] (209.55, 56.26) circle (  2.13);

\path[fill=fillColor,fill opacity=0.20] (196.22, 66.23) circle (  2.13);

\path[fill=fillColor,fill opacity=0.20] (174.81, 87.93) circle (  2.13);

\path[fill=fillColor,fill opacity=0.20] (187.92, 43.18) circle (  2.13);

\path[fill=fillColor,fill opacity=0.20] (213.05, 46.29) circle (  2.13);

\path[fill=fillColor,fill opacity=0.20] (217.85, 59.79) circle (  2.13);

\path[fill=fillColor,fill opacity=0.20] (216.98, 62.49) circle (  2.13);

\path[fill=fillColor,fill opacity=0.20] (222.88, 65.71) circle (  2.13);

\path[fill=fillColor,fill opacity=0.20] (227.25, 56.05) circle (  2.13);

\path[fill=fillColor,fill opacity=0.20] (231.18, 49.41) circle (  2.13);

\path[fill=fillColor,fill opacity=0.20] (232.27, 58.13) circle (  2.13);

\path[fill=fillColor,fill opacity=0.20] (224.63, 59.48) circle (  2.13);

\path[fill=fillColor,fill opacity=0.20] (220.04, 46.81) circle (  2.13);

\path[fill=fillColor,fill opacity=0.20] (216.76, 44.22) circle (  2.13);

\path[fill=fillColor,fill opacity=0.20] (204.74, 51.90) circle (  2.13);

\path[fill=fillColor,fill opacity=0.20] (177.43, 69.45) circle (  2.13);

\path[fill=fillColor,fill opacity=0.20] (196.66, 63.74) circle (  2.13);

\path[fill=fillColor,fill opacity=0.20] (206.49, 51.38) circle (  2.13);

\path[fill=fillColor,fill opacity=0.20] (209.77, 51.07) circle (  2.13);

\path[fill=fillColor,fill opacity=0.20] (215.23, 55.33) circle (  2.13);

\path[fill=fillColor,fill opacity=0.20] (215.45, 65.09) circle (  2.13);

\path[fill=fillColor,fill opacity=0.20] (215.23, 75.16) circle (  2.13);

\path[fill=fillColor,fill opacity=0.20] (215.01, 71.21) circle (  2.13);

\path[fill=fillColor,fill opacity=0.20] (211.95, 51.80) circle (  2.13);

\path[fill=fillColor,fill opacity=0.20] (202.78, 54.29) circle (  2.13);

\path[fill=fillColor,fill opacity=0.20] (184.20, 73.91) circle (  2.13);

\path[fill=fillColor,fill opacity=0.20] (178.96, 64.98) circle (  2.13);

\path[fill=fillColor,fill opacity=0.20] (203.00, 54.29) circle (  2.13);

\path[fill=fillColor,fill opacity=0.20] (213.70, 61.66) circle (  2.13);

\path[fill=fillColor,fill opacity=0.20] (216.32, 60.21) circle (  2.13);

\path[fill=fillColor,fill opacity=0.20] (213.92, 64.98) circle (  2.13);

\path[fill=fillColor,fill opacity=0.20] (218.95, 65.61) circle (  2.13);

\path[fill=fillColor,fill opacity=0.20] (223.53, 56.36) circle (  2.13);

\path[fill=fillColor,fill opacity=0.20] (228.78, 49.51) circle (  2.13);

\path[fill=fillColor,fill opacity=0.20] (227.47, 51.49) circle (  2.13);

\path[fill=fillColor,fill opacity=0.20] (219.60, 51.69) circle (  2.13);

\path[fill=fillColor,fill opacity=0.20] (219.38, 56.68) circle (  2.13);

\path[fill=fillColor,fill opacity=0.20] (211.74, 57.71) circle (  2.13);

\path[fill=fillColor,fill opacity=0.20] (198.84, 58.75) circle (  2.13);

\path[fill=fillColor,fill opacity=0.20] (179.62, 64.05) circle (  2.13);

\path[fill=fillColor,fill opacity=0.20] (194.26, 56.47) circle (  2.13);

\path[fill=fillColor,fill opacity=0.20] (204.52, 58.03) circle (  2.13);

\path[fill=fillColor,fill opacity=0.20] (212.61, 58.96) circle (  2.13);

\path[fill=fillColor,fill opacity=0.20] (217.85, 58.13) circle (  2.13);

\path[fill=fillColor,fill opacity=0.20] (218.73, 66.75) circle (  2.13);

\path[fill=fillColor,fill opacity=0.20] (215.23, 71.83) circle (  2.13);

\path[fill=fillColor,fill opacity=0.20] (211.95, 71.52) circle (  2.13);

\path[fill=fillColor,fill opacity=0.20] (207.15, 66.23) circle (  2.13);

\path[fill=fillColor,fill opacity=0.20] (194.91, 59.38) circle (  2.13);

\path[fill=fillColor,fill opacity=0.20] (191.41, 65.09) circle (  2.13);

\path[fill=fillColor,fill opacity=0.20] (202.12, 67.37) circle (  2.13);

\path[fill=fillColor,fill opacity=0.20] (209.11, 51.59) circle (  2.13);

\path[fill=fillColor,fill opacity=0.20] (213.92, 48.16) circle (  2.13);

\path[fill=fillColor,fill opacity=0.20] (218.07, 59.06) circle (  2.13);

\path[fill=fillColor,fill opacity=0.20] (226.16, 56.68) circle (  2.13);

\path[fill=fillColor,fill opacity=0.20] (220.26, 52.84) circle (  2.13);

\path[fill=fillColor,fill opacity=0.20] (224.41, 56.16) circle (  2.13);

\path[fill=fillColor,fill opacity=0.20] (223.10, 55.22) circle (  2.13);

\path[fill=fillColor,fill opacity=0.20] (217.85, 61.04) circle (  2.13);

\path[fill=fillColor,fill opacity=0.20] (213.26, 72.56) circle (  2.13);

\path[fill=fillColor,fill opacity=0.20] (207.37, 65.09) circle (  2.13);

\path[fill=fillColor,fill opacity=0.20] (193.60, 70.69) circle (  2.13);

\path[fill=fillColor,fill opacity=0.20] (194.26, 53.77) circle (  2.13);

\path[fill=fillColor,fill opacity=0.20] (210.86, 52.52) circle (  2.13);

\path[fill=fillColor,fill opacity=0.20] (215.67, 61.24) circle (  2.13);

\path[fill=fillColor,fill opacity=0.20] (219.16, 64.57) circle (  2.13);

\path[fill=fillColor,fill opacity=0.20] (217.85, 67.68) circle (  2.13);

\path[fill=fillColor,fill opacity=0.20] (212.39, 64.77) circle (  2.13);

\path[fill=fillColor,fill opacity=0.20] (208.46, 60.21) circle (  2.13);

\path[fill=fillColor,fill opacity=0.20] (204.52, 69.55) circle (  2.13);

\path[fill=fillColor,fill opacity=0.20] (191.63, 70.59) circle (  2.13);

\path[fill=fillColor,fill opacity=0.20] (194.47, 64.36) circle (  2.13);

\path[fill=fillColor,fill opacity=0.20] (196.00, 66.33) circle (  2.13);

\path[fill=fillColor,fill opacity=0.20] (203.65, 61.56) circle (  2.13);

\path[fill=fillColor,fill opacity=0.20] (208.46, 44.32) circle (  2.13);

\path[fill=fillColor,fill opacity=0.20] (215.67, 38.40) circle (  2.13);

\path[fill=fillColor,fill opacity=0.20] (223.10, 54.18) circle (  2.13);

\path[fill=fillColor,fill opacity=0.20] (226.37, 51.59) circle (  2.13);

\path[fill=fillColor,fill opacity=0.20] (223.10, 48.99) circle (  2.13);

\path[fill=fillColor,fill opacity=0.20] (220.48, 62.80) circle (  2.13);

\path[fill=fillColor,fill opacity=0.20] (218.73, 66.95) circle (  2.13);

\path[fill=fillColor,fill opacity=0.20] (216.54, 64.26) circle (  2.13);

\path[fill=fillColor,fill opacity=0.20] (209.11, 63.32) circle (  2.13);

\path[fill=fillColor,fill opacity=0.20] (198.84, 61.66) circle (  2.13);

\path[fill=fillColor,fill opacity=0.20] (184.86, 85.85) circle (  2.13);

\path[fill=fillColor,fill opacity=0.20] (191.20, 57.20) circle (  2.13);

\path[fill=fillColor,fill opacity=0.20] (208.46, 47.23) circle (  2.13);

\path[fill=fillColor,fill opacity=0.20] (214.58, 60.31) circle (  2.13);

\path[fill=fillColor,fill opacity=0.20] (214.58, 65.92) circle (  2.13);

\path[fill=fillColor,fill opacity=0.20] (209.55, 63.01) circle (  2.13);

\path[fill=fillColor,fill opacity=0.20] (204.31, 61.97) circle (  2.13);

\path[fill=fillColor,fill opacity=0.20] (205.62, 61.04) circle (  2.13);

\path[fill=fillColor,fill opacity=0.20] (203.87, 65.92) circle (  2.13);

\path[fill=fillColor,fill opacity=0.20] (194.26, 75.47) circle (  2.13);

\path[fill=fillColor,fill opacity=0.20] (187.92, 59.06) circle (  2.13);

\path[fill=fillColor,fill opacity=0.20] (199.06, 57.71) circle (  2.13);

\path[fill=fillColor,fill opacity=0.20] (209.11, 61.87) circle (  2.13);

\path[fill=fillColor,fill opacity=0.20] (211.74, 67.68) circle (  2.13);

\path[fill=fillColor,fill opacity=0.20] (213.26, 57.92) circle (  2.13);

\path[fill=fillColor,fill opacity=0.20] (223.53, 54.08) circle (  2.13);

\path[fill=fillColor,fill opacity=0.20] (228.12, 54.50) circle (  2.13);

\path[fill=fillColor,fill opacity=0.20] (225.28, 56.88) circle (  2.13);

\path[fill=fillColor,fill opacity=0.20] (222.22, 65.71) circle (  2.13);

\path[fill=fillColor,fill opacity=0.20] (220.69, 71.42) circle (  2.13);

\path[fill=fillColor,fill opacity=0.20] (222.22, 59.48) circle (  2.13);

\path[fill=fillColor,fill opacity=0.20] (206.49, 49.20) circle (  2.13);

\path[fill=fillColor,fill opacity=0.20] (190.32, 60.10) circle (  2.13);

\path[fill=fillColor,fill opacity=0.20] (184.42, 53.87) circle (  2.13);

\path[fill=fillColor,fill opacity=0.20] (201.03, 47.75) circle (  2.13);

\path[fill=fillColor,fill opacity=0.20] (210.86, 57.51) circle (  2.13);

\path[fill=fillColor,fill opacity=0.20] (210.42, 58.13) circle (  2.13);

\path[fill=fillColor,fill opacity=0.20] (209.11, 50.76) circle (  2.13);

\path[fill=fillColor,fill opacity=0.20] (209.11, 62.39) circle (  2.13);

\path[fill=fillColor,fill opacity=0.20] (203.43, 69.24) circle (  2.13);

\path[fill=fillColor,fill opacity=0.20] (201.90, 60.52) circle (  2.13);

\path[fill=fillColor,fill opacity=0.20] (199.06, 59.58) circle (  2.13);

\path[fill=fillColor,fill opacity=0.20] (183.55, 76.30) circle (  2.13);

\path[fill=fillColor,fill opacity=0.20] (185.08, 60.00) circle (  2.13);

\path[fill=fillColor,fill opacity=0.20] (197.31, 45.88) circle (  2.13);

\path[fill=fillColor,fill opacity=0.20] (210.64, 50.86) circle (  2.13);

\path[fill=fillColor,fill opacity=0.20] (214.58, 64.98) circle (  2.13);

\path[fill=fillColor,fill opacity=0.20] (213.92, 75.88) circle (  2.13);

\path[fill=fillColor,fill opacity=0.20] (213.70, 70.48) circle (  2.13);

\path[fill=fillColor,fill opacity=0.20] (223.97, 58.34) circle (  2.13);

\path[fill=fillColor,fill opacity=0.20] (231.84, 60.31) circle (  2.13);

\path[fill=fillColor,fill opacity=0.20] (231.40, 69.97) circle (  2.13);

\path[fill=fillColor,fill opacity=0.20] (231.84, 69.86) circle (  2.13);

\path[fill=fillColor,fill opacity=0.20] (229.22, 63.32) circle (  2.13);

\path[fill=fillColor,fill opacity=0.20] (220.69, 58.55) circle (  2.13);

\path[fill=fillColor,fill opacity=0.20] (201.68, 56.05) circle (  2.13);

\path[fill=fillColor,fill opacity=0.20] (191.63, 57.40) circle (  2.13);

\path[fill=fillColor,fill opacity=0.20] (204.09, 53.56) circle (  2.13);

\path[fill=fillColor,fill opacity=0.20] (208.89, 47.54) circle (  2.13);

\path[fill=fillColor,fill opacity=0.20] (209.99, 49.31) circle (  2.13);

\path[fill=fillColor,fill opacity=0.20] (207.58, 64.26) circle (  2.13);

\path[fill=fillColor,fill opacity=0.20] (201.90, 66.44) circle (  2.13);

\path[fill=fillColor,fill opacity=0.20] (201.90, 53.15) circle (  2.13);

\path[fill=fillColor,fill opacity=0.20] (204.96, 48.27) circle (  2.13);

\path[fill=fillColor,fill opacity=0.20] (198.41, 55.02) circle (  2.13);

\path[fill=fillColor,fill opacity=0.20] (177.87, 76.30) circle (  2.13);

\path[fill=fillColor,fill opacity=0.20] (185.52, 68.72) circle (  2.13);

\path[fill=fillColor,fill opacity=0.20] (197.10, 53.25) circle (  2.13);

\path[fill=fillColor,fill opacity=0.20] (203.21, 48.68) circle (  2.13);

\path[fill=fillColor,fill opacity=0.20] (209.99, 57.20) circle (  2.13);

\path[fill=fillColor,fill opacity=0.20] (215.45, 58.75) circle (  2.13);

\path[fill=fillColor,fill opacity=0.20] (217.42, 57.40) circle (  2.13);

\path[fill=fillColor,fill opacity=0.20] (218.51, 64.46) circle (  2.13);

\path[fill=fillColor,fill opacity=0.20] (224.63, 63.22) circle (  2.13);

\path[fill=fillColor,fill opacity=0.20] (230.53, 61.24) circle (  2.13);

\path[fill=fillColor,fill opacity=0.20] (227.90, 68.20) circle (  2.13);

\path[fill=fillColor,fill opacity=0.20] (232.49, 64.36) circle (  2.13);

\path[fill=fillColor,fill opacity=0.20] (222.88, 54.70) circle (  2.13);

\path[fill=fillColor,fill opacity=0.20] (203.87, 59.79) circle (  2.13);

\path[fill=fillColor,fill opacity=0.20] (198.63, 57.92) circle (  2.13);

\path[fill=fillColor,fill opacity=0.20] (207.58, 53.77) circle (  2.13);

\path[fill=fillColor,fill opacity=0.20] (208.68, 56.26) circle (  2.13);

\path[fill=fillColor,fill opacity=0.20] (207.37, 57.82) circle (  2.13);

\path[fill=fillColor,fill opacity=0.20] (207.37, 56.47) circle (  2.13);

\path[fill=fillColor,fill opacity=0.20] (203.65, 56.26) circle (  2.13);

\path[fill=fillColor,fill opacity=0.20] (205.40, 56.26) circle (  2.13);

\path[fill=fillColor,fill opacity=0.20] (202.12, 56.88) circle (  2.13);

\path[fill=fillColor,fill opacity=0.20] (190.10, 56.68) circle (  2.13);

\path[fill=fillColor,fill opacity=0.20] (176.56, 70.28) circle (  2.13);

\path[fill=fillColor,fill opacity=0.20] (186.61, 83.77) circle (  2.13);

\path[fill=fillColor,fill opacity=0.20] (193.16, 53.87) circle (  2.13);

\path[fill=fillColor,fill opacity=0.20] (201.90, 55.95) circle (  2.13);

\path[fill=fillColor,fill opacity=0.20] (202.56, 70.07) circle (  2.13);

\path[fill=fillColor,fill opacity=0.20] (208.68, 66.33) circle (  2.13);

\path[fill=fillColor,fill opacity=0.20] (215.89, 51.80) circle (  2.13);

\path[fill=fillColor,fill opacity=0.20] (225.72, 44.32) circle (  2.13);

\path[fill=fillColor,fill opacity=0.20] (223.75, 56.47) circle (  2.13);

\path[fill=fillColor,fill opacity=0.20] (227.90, 63.74) circle (  2.13);

\path[fill=fillColor,fill opacity=0.20] (222.66, 56.57) circle (  2.13);

\path[fill=fillColor,fill opacity=0.20] (219.16, 49.72) circle (  2.13);

\path[fill=fillColor,fill opacity=0.20] (224.41, 50.86) circle (  2.13);

\path[fill=fillColor,fill opacity=0.20] (210.64, 59.79) circle (  2.13);

\path[fill=fillColor,fill opacity=0.20] (195.35, 61.45) circle (  2.13);

\path[fill=fillColor,fill opacity=0.20] (203.21, 61.35) circle (  2.13);

\path[fill=fillColor,fill opacity=0.20] (208.02, 55.33) circle (  2.13);

\path[fill=fillColor,fill opacity=0.20] (208.68, 43.70) circle (  2.13);

\path[fill=fillColor,fill opacity=0.20] (209.11, 45.67) circle (  2.13);

\path[fill=fillColor,fill opacity=0.20] (207.80, 59.17) circle (  2.13);

\path[fill=fillColor,fill opacity=0.20] (206.27, 65.92) circle (  2.13);

\path[fill=fillColor,fill opacity=0.20] (202.34, 63.01) circle (  2.13);

\path[fill=fillColor,fill opacity=0.20] (195.78, 55.53) circle (  2.13);

\path[fill=fillColor,fill opacity=0.20] (191.85, 50.97) circle (  2.13);

\path[fill=fillColor,fill opacity=0.20] (180.49, 61.76) circle (  2.13);

\path[fill=fillColor,fill opacity=0.20] (182.02, 83.77) circle (  2.13);

\path[fill=fillColor,fill opacity=0.20] (194.91, 75.88) circle (  2.13);

\path[fill=fillColor,fill opacity=0.20] (203.00, 64.36) circle (  2.13);

\path[fill=fillColor,fill opacity=0.20] (201.47, 62.91) circle (  2.13);

\path[fill=fillColor,fill opacity=0.20] (200.15, 73.18) circle (  2.13);

\path[fill=fillColor,fill opacity=0.20] (210.21, 68.82) circle (  2.13);

\path[fill=fillColor,fill opacity=0.20] (218.29, 55.22) circle (  2.13);

\path[fill=fillColor,fill opacity=0.20] (225.06, 55.95) circle (  2.13);

\path[fill=fillColor,fill opacity=0.20] (225.28, 65.19) circle (  2.13);

\path[fill=fillColor,fill opacity=0.20] (225.06, 65.19) circle (  2.13);

\path[fill=fillColor,fill opacity=0.20] (270.51, 60.31) circle (  2.13);

\path[fill=fillColor,fill opacity=0.20] (212.61, 47.96) circle (  2.13);

\path[fill=fillColor,fill opacity=0.20] (209.33, 45.46) circle (  2.13);

\path[fill=fillColor,fill opacity=0.20] (200.37, 60.62) circle (  2.13);

\path[fill=fillColor,fill opacity=0.20] (190.32, 61.87) circle (  2.13);

\path[fill=fillColor,fill opacity=0.20] (201.03, 52.42) circle (  2.13);

\path[fill=fillColor,fill opacity=0.20] (209.99, 39.55) circle (  2.13);

\path[fill=fillColor,fill opacity=0.20] (208.02, 41.00) circle (  2.13);

\path[fill=fillColor,fill opacity=0.20] (206.27, 58.86) circle (  2.13);

\path[fill=fillColor,fill opacity=0.20] (206.27, 65.50) circle (  2.13);

\path[fill=fillColor,fill opacity=0.20] (201.47, 59.38) circle (  2.13);

\path[fill=fillColor,fill opacity=0.20] (204.52, 56.36) circle (  2.13);

\path[fill=fillColor,fill opacity=0.20] (199.28, 60.62) circle (  2.13);

\path[fill=fillColor,fill opacity=0.20] (193.38, 62.18) circle (  2.13);

\path[fill=fillColor,fill opacity=0.20] (181.36, 61.04) circle (  2.13);

\path[fill=fillColor,fill opacity=0.20] (181.36, 72.77) circle (  2.13);

\path[fill=fillColor,fill opacity=0.20] (195.57, 51.59) circle (  2.13);

\path[fill=fillColor,fill opacity=0.20] (204.31, 68.72) circle (  2.13);

\path[fill=fillColor,fill opacity=0.20] (208.68, 74.64) circle (  2.13);

\path[fill=fillColor,fill opacity=0.20] (214.14, 59.48) circle (  2.13);

\path[fill=fillColor,fill opacity=0.20] (208.46, 57.71) circle (  2.13);

\path[fill=fillColor,fill opacity=0.20] (213.70, 67.58) circle (  2.13);

\path[fill=fillColor,fill opacity=0.20] (216.76, 65.19) circle (  2.13);

\path[fill=fillColor,fill opacity=0.20] (218.29, 62.91) circle (  2.13);

\path[fill=fillColor,fill opacity=0.20] (219.82, 65.40) circle (  2.13);

\path[fill=fillColor,fill opacity=0.20] (214.79, 66.44) circle (  2.13);

\path[fill=fillColor,fill opacity=0.20] (211.30, 67.58) circle (  2.13);

\path[fill=fillColor,fill opacity=0.20] (208.89, 63.94) circle (  2.13);

\path[fill=fillColor,fill opacity=0.20] (198.63, 51.07) circle (  2.13);

\path[fill=fillColor,fill opacity=0.20] (194.69, 67.79) circle (  2.13);

\path[fill=fillColor,fill opacity=0.20] (208.02, 45.78) circle (  2.13);

\path[fill=fillColor,fill opacity=0.20] (209.55, 40.58) circle (  2.13);

\path[fill=fillColor,fill opacity=0.20] (206.71, 53.56) circle (  2.13);

\path[fill=fillColor,fill opacity=0.20] (200.81, 60.10) circle (  2.13);

\path[fill=fillColor,fill opacity=0.20] (202.78, 55.22) circle (  2.13);

\path[fill=fillColor,fill opacity=0.20] (205.18, 57.82) circle (  2.13);

\path[fill=fillColor,fill opacity=0.20] (201.25, 67.68) circle (  2.13);

\path[fill=fillColor,fill opacity=0.20] (193.60, 71.42) circle (  2.13);

\path[fill=fillColor,fill opacity=0.20] (191.20, 61.87) circle (  2.13);

\path[fill=fillColor,fill opacity=0.20] (188.57, 57.92) circle (  2.13);

\path[fill=fillColor,fill opacity=0.20] (177.65, 84.81) circle (  2.13);

\path[fill=fillColor,fill opacity=0.20] (180.27, 71.52) circle (  2.13);

\path[fill=fillColor,fill opacity=0.20] (191.41, 47.23) circle (  2.13);

\path[fill=fillColor,fill opacity=0.20] (204.52, 59.17) circle (  2.13);

\path[fill=fillColor,fill opacity=0.20] (205.84, 72.15) circle (  2.13);

\path[fill=fillColor,fill opacity=0.20] (217.63, 62.18) circle (  2.13);

\path[fill=fillColor,fill opacity=0.20] (215.45, 56.05) circle (  2.13);

\path[fill=fillColor,fill opacity=0.20] (216.32, 67.58) circle (  2.13);

\path[fill=fillColor,fill opacity=0.20] (212.17, 67.99) circle (  2.13);

\path[fill=fillColor,fill opacity=0.20] (211.52, 61.66) circle (  2.13);

\path[fill=fillColor,fill opacity=0.20] (220.91, 58.23) circle (  2.13);

\path[fill=fillColor,fill opacity=0.20] (210.86, 52.11) circle (  2.13);

\path[fill=fillColor,fill opacity=0.20] (202.56, 58.86) circle (  2.13);

\path[fill=fillColor,fill opacity=0.20] (200.37, 69.65) circle (  2.13);

\path[fill=fillColor,fill opacity=0.20] (186.17, 68.20) circle (  2.13);

\path[fill=fillColor,fill opacity=0.20] (185.73, 74.74) circle (  2.13);

\path[fill=fillColor,fill opacity=0.20] (198.63, 75.36) circle (  2.13);

\path[fill=fillColor,fill opacity=0.20] (204.09, 53.67) circle (  2.13);

\path[fill=fillColor,fill opacity=0.20] (204.09, 41.83) circle (  2.13);

\path[fill=fillColor,fill opacity=0.20] (205.84, 44.32) circle (  2.13);

\path[fill=fillColor,fill opacity=0.20] (204.96, 48.89) circle (  2.13);

\path[fill=fillColor,fill opacity=0.20] (202.56, 51.69) circle (  2.13);

\path[fill=fillColor,fill opacity=0.20] (196.88, 61.76) circle (  2.13);

\path[fill=fillColor,fill opacity=0.20] (198.41, 72.87) circle (  2.13);

\path[fill=fillColor,fill opacity=0.20] (195.35, 78.89) circle (  2.13);

\path[fill=fillColor,fill opacity=0.20] (197.53, 67.16) circle (  2.13);

\path[fill=fillColor,fill opacity=0.20] (191.85, 49.51) circle (  2.13);

\path[fill=fillColor,fill opacity=0.20] (180.93, 60.21) circle (  2.13);

\path[fill=fillColor,fill opacity=0.20] (183.55, 59.89) circle (  2.13);

\path[fill=fillColor,fill opacity=0.20] (197.75, 50.65) circle (  2.13);

\path[fill=fillColor,fill opacity=0.20] (199.28, 46.40) circle (  2.13);

\path[fill=fillColor,fill opacity=0.20] (195.78, 46.29) circle (  2.13);

\path[fill=fillColor,fill opacity=0.20] (203.87, 56.16) circle (  2.13);

\path[fill=fillColor,fill opacity=0.20] (209.33, 63.94) circle (  2.13);

\path[fill=fillColor,fill opacity=0.20] (212.83, 60.62) circle (  2.13);

\path[fill=fillColor,fill opacity=0.20] (216.76, 57.92) circle (  2.13);

\path[fill=fillColor,fill opacity=0.20] (224.19, 59.17) circle (  2.13);

\path[fill=fillColor,fill opacity=0.20] (207.58, 59.79) circle (  2.13);

\path[fill=fillColor,fill opacity=0.20] (207.80, 67.16) circle (  2.13);

\path[fill=fillColor,fill opacity=0.20] (208.02, 70.28) circle (  2.13);

\path[fill=fillColor,fill opacity=0.20] (201.03, 57.71) circle (  2.13);

\path[fill=fillColor,fill opacity=0.20] (192.29, 60.00) circle (  2.13);

\path[fill=fillColor,fill opacity=0.20] (191.20, 71.32) circle (  2.13);

\path[fill=fillColor,fill opacity=0.20] (199.94, 56.05) circle (  2.13);

\path[fill=fillColor,fill opacity=0.20] (208.89, 48.58) circle (  2.13);

\path[fill=fillColor,fill opacity=0.20] (205.18, 47.33) circle (  2.13);

\path[fill=fillColor,fill opacity=0.20] (211.52, 52.00) circle (  2.13);

\path[fill=fillColor,fill opacity=0.20] (203.00, 67.47) circle (  2.13);

\path[fill=fillColor,fill opacity=0.20] (203.87, 91.04) circle (  2.13);

\path[fill=fillColor,fill opacity=0.20] (204.52, 85.85) circle (  2.13);

\path[fill=fillColor,fill opacity=0.20] (201.90, 49.72) circle (  2.13);

\path[fill=fillColor,fill opacity=0.20] (187.70, 64.26) circle (  2.13);

\path[fill=fillColor,fill opacity=0.20] (182.46, 63.84) circle (  2.13);

\path[fill=fillColor,fill opacity=0.20] (192.94, 58.65) circle (  2.13);

\path[fill=fillColor,fill opacity=0.20] (201.25, 66.85) circle (  2.13);

\path[fill=fillColor,fill opacity=0.20] (203.21, 65.71) circle (  2.13);

\path[fill=fillColor,fill opacity=0.20] (203.21, 60.00) circle (  2.13);

\path[fill=fillColor,fill opacity=0.20] (205.40, 56.99) circle (  2.13);

\path[fill=fillColor,fill opacity=0.20] (204.74, 57.40) circle (  2.13);

\path[fill=fillColor,fill opacity=0.20] (205.40, 54.91) circle (  2.13);

\path[fill=fillColor,fill opacity=0.20] (204.96, 50.45) circle (  2.13);

\path[fill=fillColor,fill opacity=0.20] (198.41, 55.43) circle (  2.13);

\path[fill=fillColor,fill opacity=0.20] (197.53, 68.41) circle (  2.13);

\path[fill=fillColor,fill opacity=0.20] (197.31, 65.71) circle (  2.13);

\path[fill=fillColor,fill opacity=0.20] (211.74, 41.93) circle (  2.13);

\path[fill=fillColor,fill opacity=0.20] (207.37, 55.53) circle (  2.13);

\path[fill=fillColor,fill opacity=0.20] (201.47, 87.93) circle (  2.13);

\path[fill=fillColor,fill opacity=0.20] (204.31, 86.89) circle (  2.13);

\path[fill=fillColor,fill opacity=0.20] (208.46, 53.98) circle (  2.13);

\path[fill=fillColor,fill opacity=0.20] (209.99, 50.86) circle (  2.13);

\path[fill=fillColor,fill opacity=0.20] (203.00, 74.43) circle (  2.13);

\path[fill=fillColor,fill opacity=0.20] (192.94, 77.96) circle (  2.13);

\path[fill=fillColor,fill opacity=0.20] (182.02, 80.66) circle (  2.13);

\path[fill=fillColor,fill opacity=0.20] (185.08, 71.42) circle (  2.13);

\path[fill=fillColor,fill opacity=0.20] (192.51, 61.14) circle (  2.13);

\path[fill=fillColor,fill opacity=0.20] (199.94, 59.38) circle (  2.13);

\path[fill=fillColor,fill opacity=0.20] (208.46, 60.10) circle (  2.13);

\path[fill=fillColor,fill opacity=0.20] (209.99, 61.24) circle (  2.13);

\path[fill=fillColor,fill opacity=0.20] (207.80, 63.94) circle (  2.13);

\path[fill=fillColor,fill opacity=0.20] (201.47, 67.99) circle (  2.13);

\path[fill=fillColor,fill opacity=0.20] (197.75, 72.15) circle (  2.13);

\path[fill=fillColor,fill opacity=0.20] (194.26, 74.95) circle (  2.13);

\path[fill=fillColor,fill opacity=0.20] (190.10, 75.16) circle (  2.13);

\path[fill=fillColor,fill opacity=0.20] (185.95, 77.13) circle (  2.13);

\path[fill=fillColor,fill opacity=0.20] (197.75, 70.17) circle (  2.13);

\path[fill=fillColor,fill opacity=0.20] (207.37, 49.10) circle (  2.13);

\path[fill=fillColor,fill opacity=0.20] (211.95, 52.21) circle (  2.13);

\path[fill=fillColor,fill opacity=0.20] (208.46, 76.40) circle (  2.13);

\path[fill=fillColor,fill opacity=0.20] (206.93, 71.94) circle (  2.13);

\path[fill=fillColor,fill opacity=0.20] (209.33, 48.47) circle (  2.13);

\path[fill=fillColor,fill opacity=0.20] (208.02, 61.35) circle (  2.13);

\path[fill=fillColor,fill opacity=0.20] (205.18, 83.77) circle (  2.13);

\path[fill=fillColor,fill opacity=0.20] (195.35, 70.69) circle (  2.13);

\path[fill=fillColor,fill opacity=0.20] (197.31, 53.04) circle (  2.13);

\path[fill=fillColor,fill opacity=0.20] (186.83, 90.00) circle (  2.13);

\path[fill=fillColor,fill opacity=0.20] (194.04, 65.29) circle (  2.13);

\path[fill=fillColor,fill opacity=0.20] (200.59, 55.02) circle (  2.13);

\path[fill=fillColor,fill opacity=0.20] (201.03, 50.65) circle (  2.13);

\path[fill=fillColor,fill opacity=0.20] (208.24, 46.09) circle (  2.13);

\path[fill=fillColor,fill opacity=0.20] (204.09, 58.23) circle (  2.13);

\path[fill=fillColor,fill opacity=0.20] (192.73, 73.50) circle (  2.13);

\path[fill=fillColor,fill opacity=0.20] (188.79, 86.89) circle (  2.13);

\path[fill=fillColor,fill opacity=0.20] (185.08, 86.89) circle (  2.13);

\path[fill=fillColor,fill opacity=0.20] (199.72, 72.56) circle (  2.13);

\path[fill=fillColor,fill opacity=0.20] (209.11, 67.89) circle (  2.13);

\path[fill=fillColor,fill opacity=0.20] (208.46, 54.29) circle (  2.13);

\path[fill=fillColor,fill opacity=0.20] (207.58, 43.91) circle (  2.13);

\path[fill=fillColor,fill opacity=0.20] (207.80, 56.36) circle (  2.13);

\path[fill=fillColor,fill opacity=0.20] (208.68, 68.82) circle (  2.13);

\path[fill=fillColor,fill opacity=0.20] (203.00, 68.72) circle (  2.13);

\path[fill=fillColor,fill opacity=0.20] (201.25, 71.42) circle (  2.13);

\path[fill=fillColor,fill opacity=0.20] (202.78, 66.75) circle (  2.13);

\path[fill=fillColor,fill opacity=0.20] (196.88, 42.04) circle (  2.13);

\path[fill=fillColor,fill opacity=0.20] (187.04, 61.04) circle (  2.13);

\path[fill=fillColor,fill opacity=0.20] (181.36, 85.85) circle (  2.13);

\path[fill=fillColor,fill opacity=0.20] (183.11, 91.04) circle (  2.13);

\path[fill=fillColor,fill opacity=0.20] (191.20, 59.06) circle (  2.13);

\path[fill=fillColor,fill opacity=0.20] (194.26, 39.96) circle (  2.13);

\path[fill=fillColor,fill opacity=0.20] (200.37, 53.87) circle (  2.13);

\path[fill=fillColor,fill opacity=0.20] (196.44, 61.45) circle (  2.13);

\path[fill=fillColor,fill opacity=0.20] (189.45, 68.82) circle (  2.13);

\path[fill=fillColor,fill opacity=0.20] (183.33, 86.89) circle (  2.13);

\path[fill=fillColor,fill opacity=0.20] (175.90,107.65) circle (  2.13);

\path[fill=fillColor,fill opacity=0.20] (179.62,106.61) circle (  2.13);

\path[fill=fillColor,fill opacity=0.20] (192.07, 82.74) circle (  2.13);

\path[fill=fillColor,fill opacity=0.20] (203.21, 53.87) circle (  2.13);

\path[fill=fillColor,fill opacity=0.20] (206.49, 47.96) circle (  2.13);

\path[fill=fillColor,fill opacity=0.20] (209.77, 49.10) circle (  2.13);

\path[fill=fillColor,fill opacity=0.20] (207.15, 47.64) circle (  2.13);

\path[fill=fillColor,fill opacity=0.20] (206.71, 56.57) circle (  2.13);

\path[fill=fillColor,fill opacity=0.20] (203.00, 77.34) circle (  2.13);

\path[fill=fillColor,fill opacity=0.20] (200.37, 84.81) circle (  2.13);

\path[fill=fillColor,fill opacity=0.20] (199.94, 66.64) circle (  2.13);

\path[fill=fillColor,fill opacity=0.20] (198.19, 55.43) circle (  2.13);

\path[fill=fillColor,fill opacity=0.20] (192.94, 64.57) circle (  2.13);

\path[fill=fillColor,fill opacity=0.20] (193.38, 64.36) circle (  2.13);

\path[fill=fillColor,fill opacity=0.20] (190.10, 56.99) circle (  2.13);

\path[fill=fillColor,fill opacity=0.20] (186.83, 68.20) circle (  2.13);

\path[fill=fillColor,fill opacity=0.20] (182.02, 91.04) circle (  2.13);

\path[fill=fillColor,fill opacity=0.20] (186.39, 90.00) circle (  2.13);

\path[fill=fillColor,fill opacity=0.20] (192.07, 78.48) circle (  2.13);

\path[fill=fillColor,fill opacity=0.20] (194.91, 65.40) circle (  2.13);

\path[fill=fillColor,fill opacity=0.20] (194.26, 59.38) circle (  2.13);

\path[fill=fillColor,fill opacity=0.20] (198.63, 57.30) circle (  2.13);

\path[fill=fillColor,fill opacity=0.20] (220.48, 46.81) circle (  2.13);

\path[fill=fillColor,fill opacity=0.20] (182.24, 67.16) circle (  2.13);

\path[fill=fillColor,fill opacity=0.20] (175.68,108.69) circle (  2.13);

\path[fill=fillColor,fill opacity=0.20] (184.86, 74.12) circle (  2.13);

\path[fill=fillColor,fill opacity=0.20] (203.21, 57.82) circle (  2.13);

\path[fill=fillColor,fill opacity=0.20] (209.99, 45.67) circle (  2.13);

\path[fill=fillColor,fill opacity=0.20] (206.05, 38.40) circle (  2.13);

\path[fill=fillColor,fill opacity=0.20] (205.18, 48.37) circle (  2.13);

\path[fill=fillColor,fill opacity=0.20] (206.27, 68.30) circle (  2.13);

\path[fill=fillColor,fill opacity=0.20] (205.40, 80.76) circle (  2.13);

\path[fill=fillColor,fill opacity=0.20] (199.94, 80.35) circle (  2.13);

\path[fill=fillColor,fill opacity=0.20] (199.94, 85.85) circle (  2.13);

\path[fill=fillColor,fill opacity=0.20] (197.97, 84.81) circle (  2.13);

\path[fill=fillColor,fill opacity=0.20] (197.31, 62.59) circle (  2.13);

\path[fill=fillColor,fill opacity=0.20] (196.00, 49.93) circle (  2.13);

\path[fill=fillColor,fill opacity=0.20] (196.22, 63.42) circle (  2.13);

\path[fill=fillColor,fill opacity=0.20] (193.38, 69.03) circle (  2.13);

\path[fill=fillColor,fill opacity=0.20] (190.10, 57.20) circle (  2.13);

\path[fill=fillColor,fill opacity=0.20] (187.70, 56.36) circle (  2.13);

\path[fill=fillColor,fill opacity=0.20] (183.33, 74.22) circle (  2.13);

\path[fill=fillColor,fill opacity=0.20] (178.74, 95.19) circle (  2.13);

\path[fill=fillColor,fill opacity=0.20] (178.09,103.50) circle (  2.13);

\path[fill=fillColor,fill opacity=0.20] (178.74,101.42) circle (  2.13);

\path[fill=fillColor,fill opacity=0.20] (181.36, 88.96) circle (  2.13);

\path[fill=fillColor,fill opacity=0.20] (180.93, 95.19) circle (  2.13);

\path[fill=fillColor,fill opacity=0.20] (179.18, 98.31) circle (  2.13);

\path[fill=fillColor,fill opacity=0.20] (188.79, 78.48) circle (  2.13);

\path[fill=fillColor,fill opacity=0.20] (189.45, 70.48) circle (  2.13);

\path[fill=fillColor,fill opacity=0.20] (192.73, 64.77) circle (  2.13);

\path[fill=fillColor,fill opacity=0.20] (197.31, 71.32) circle (  2.13);

\path[fill=fillColor,fill opacity=0.20] (199.94, 69.65) circle (  2.13);

\path[fill=fillColor,fill opacity=0.20] (198.63, 54.50) circle (  2.13);

\path[fill=fillColor,fill opacity=0.20] (199.72, 52.84) circle (  2.13);

\path[fill=fillColor,fill opacity=0.20] (198.84, 45.57) circle (  2.13);

\path[fill=fillColor,fill opacity=0.20] (192.07, 44.84) circle (  2.13);

\path[fill=fillColor,fill opacity=0.20] (182.02, 77.23) circle (  2.13);

\path[fill=fillColor,fill opacity=0.20] (195.57, 58.23) circle (  2.13);

\path[fill=fillColor,fill opacity=0.20] (202.12, 55.02) circle (  2.13);

\path[fill=fillColor,fill opacity=0.20] (202.56, 62.70) circle (  2.13);

\path[fill=fillColor,fill opacity=0.20] (201.25, 64.36) circle (  2.13);

\path[fill=fillColor,fill opacity=0.20] (201.68, 61.97) circle (  2.13);

\path[fill=fillColor,fill opacity=0.20] (204.09, 64.26) circle (  2.13);

\path[fill=fillColor,fill opacity=0.20] (208.02, 71.11) circle (  2.13);

\path[fill=fillColor,fill opacity=0.20] (206.71, 67.27) circle (  2.13);

\path[fill=fillColor,fill opacity=0.20] (200.59, 55.64) circle (  2.13);

\path[fill=fillColor,fill opacity=0.20] (199.50, 55.02) circle (  2.13);

\path[fill=fillColor,fill opacity=0.20] (198.19, 66.85) circle (  2.13);

\path[fill=fillColor,fill opacity=0.20] (195.78, 69.45) circle (  2.13);

\path[fill=fillColor,fill opacity=0.20] (195.35, 58.96) circle (  2.13);

\path[fill=fillColor,fill opacity=0.20] (194.91, 54.08) circle (  2.13);

\path[fill=fillColor,fill opacity=0.20] (194.47, 65.61) circle (  2.13);

\path[fill=fillColor,fill opacity=0.20] (195.35, 73.18) circle (  2.13);

\path[fill=fillColor,fill opacity=0.20] (195.78, 70.69) circle (  2.13);

\path[fill=fillColor,fill opacity=0.20] (194.91, 58.65) circle (  2.13);

\path[fill=fillColor,fill opacity=0.20] (196.22, 57.51) circle (  2.13);

\path[fill=fillColor,fill opacity=0.20] (195.13, 69.24) circle (  2.13);

\path[fill=fillColor,fill opacity=0.20] (197.31, 69.86) circle (  2.13);

\path[fill=fillColor,fill opacity=0.20] (195.13, 66.44) circle (  2.13);

\path[fill=fillColor,fill opacity=0.20] (195.35, 68.41) circle (  2.13);

\path[fill=fillColor,fill opacity=0.20] (195.78, 59.69) circle (  2.13);

\path[fill=fillColor,fill opacity=0.20] (195.78, 54.29) circle (  2.13);

\path[fill=fillColor,fill opacity=0.20] (195.35, 65.61) circle (  2.13);

\path[fill=fillColor,fill opacity=0.20] (200.81, 62.08) circle (  2.13);

\path[fill=fillColor,fill opacity=0.20] (196.88, 49.41) circle (  2.13);

\path[fill=fillColor,fill opacity=0.20] (200.59, 49.93) circle (  2.13);

\path[fill=fillColor,fill opacity=0.20] (200.59, 43.70) circle (  2.13);

\path[fill=fillColor,fill opacity=0.20] (214.14, 39.34) circle (  2.13);

\path[fill=fillColor,fill opacity=0.20] (201.68, 51.38) circle (  2.13);

\path[fill=fillColor,fill opacity=0.20] (197.97, 58.44) circle (  2.13);

\path[fill=fillColor,fill opacity=0.20] (194.04, 61.66) circle (  2.13);

\path[fill=fillColor,fill opacity=0.20] (186.17, 71.63) circle (  2.13);

\path[fill=fillColor,fill opacity=0.20] (175.46, 96.23) circle (  2.13);

\path[fill=fillColor,fill opacity=0.20] (193.60, 61.87) circle (  2.13);

\path[fill=fillColor,fill opacity=0.20] (198.41, 60.31) circle (  2.13);

\path[fill=fillColor,fill opacity=0.20] (206.49, 54.50) circle (  2.13);

\path[fill=fillColor,fill opacity=0.20] (208.24, 57.09) circle (  2.13);

\path[fill=fillColor,fill opacity=0.20] (208.24, 57.30) circle (  2.13);

\path[fill=fillColor,fill opacity=0.20] (205.62, 56.26) circle (  2.13);

\path[fill=fillColor,fill opacity=0.20] (203.00, 63.63) circle (  2.13);

\path[fill=fillColor,fill opacity=0.20] (198.41, 65.92) circle (  2.13);

\path[fill=fillColor,fill opacity=0.20] (201.47, 57.51) circle (  2.13);

\path[fill=fillColor,fill opacity=0.20] (203.43, 58.34) circle (  2.13);

\path[fill=fillColor,fill opacity=0.20] (199.50, 66.23) circle (  2.13);

\path[fill=fillColor,fill opacity=0.20] (201.25, 70.28) circle (  2.13);

\path[fill=fillColor,fill opacity=0.20] (203.21, 74.12) circle (  2.13);

\path[fill=fillColor,fill opacity=0.20] (198.41, 63.11) circle (  2.13);

\path[fill=fillColor,fill opacity=0.20] (196.22, 50.65) circle (  2.13);

\path[fill=fillColor,fill opacity=0.20] (198.19, 60.00) circle (  2.13);

\path[fill=fillColor,fill opacity=0.20] (199.28, 67.47) circle (  2.13);

\path[fill=fillColor,fill opacity=0.20] (201.68, 60.83) circle (  2.13);

\path[fill=fillColor,fill opacity=0.20] (201.47, 63.42) circle (  2.13);

\path[fill=fillColor,fill opacity=0.20] (199.06, 63.53) circle (  2.13);

\path[fill=fillColor,fill opacity=0.20] (200.37, 54.81) circle (  2.13);

\path[fill=fillColor,fill opacity=0.20] (201.68, 56.47) circle (  2.13);

\path[fill=fillColor,fill opacity=0.20] (203.43, 54.91) circle (  2.13);

\path[fill=fillColor,fill opacity=0.20] (204.31, 45.88) circle (  2.13);

\path[fill=fillColor,fill opacity=0.20] (206.71, 49.41) circle (  2.13);

\path[fill=fillColor,fill opacity=0.20] (204.74, 50.76) circle (  2.13);

\path[fill=fillColor,fill opacity=0.20] (195.57, 54.08) circle (  2.13);

\path[fill=fillColor,fill opacity=0.20] (184.20, 77.23) circle (  2.13);

\path[fill=fillColor,fill opacity=0.20] (190.76, 76.71) circle (  2.13);

\path[fill=fillColor,fill opacity=0.20] (195.13, 81.18) circle (  2.13);

\path[fill=fillColor,fill opacity=0.20] (198.41, 75.57) circle (  2.13);

\path[fill=fillColor,fill opacity=0.20] (199.72, 56.16) circle (  2.13);

\path[fill=fillColor,fill opacity=0.20] (197.53, 63.84) circle (  2.13);

\path[fill=fillColor,fill opacity=0.20] (197.75, 73.50) circle (  2.13);

\path[fill=fillColor,fill opacity=0.20] (201.90, 63.94) circle (  2.13);

\path[fill=fillColor,fill opacity=0.20] (200.59, 53.56) circle (  2.13);

\path[fill=fillColor,fill opacity=0.20] (202.34, 52.73) circle (  2.13);

\path[fill=fillColor,fill opacity=0.20] (206.71, 54.39) circle (  2.13);

\path[fill=fillColor,fill opacity=0.20] (207.15, 56.05) circle (  2.13);

\path[fill=fillColor,fill opacity=0.20] (206.05, 58.13) circle (  2.13);

\path[fill=fillColor,fill opacity=0.20] (204.52, 49.41) circle (  2.13);

\path[fill=fillColor,fill opacity=0.20] (203.65, 42.97) circle (  2.13);

\path[fill=fillColor,fill opacity=0.20] (205.40, 45.78) circle (  2.13);

\path[fill=fillColor,fill opacity=0.20] (207.37, 49.72) circle (  2.13);

\path[fill=fillColor,fill opacity=0.20] (206.93, 52.73) circle (  2.13);

\path[fill=fillColor,fill opacity=0.20] (205.18, 60.41) circle (  2.13);

\path[fill=fillColor,fill opacity=0.20] (203.00, 66.02) circle (  2.13);

\path[fill=fillColor,fill opacity=0.20] (205.18, 60.73) circle (  2.13);

\path[fill=fillColor,fill opacity=0.20] (206.49, 56.99) circle (  2.13);

\path[fill=fillColor,fill opacity=0.20] (205.84, 60.93) circle (  2.13);

\path[fill=fillColor,fill opacity=0.20] (201.90, 65.81) circle (  2.13);

\path[fill=fillColor,fill opacity=0.20] (194.04, 76.30) circle (  2.13);

\path[fill=fillColor,fill opacity=0.20] (184.86, 70.38) circle (  2.13);

\path[fill=fillColor,fill opacity=0.20] (186.17, 96.23) circle (  2.13);

\path[fill=fillColor,fill opacity=0.20] (188.79, 93.12) circle (  2.13);

\path[fill=fillColor,fill opacity=0.20] (187.26, 76.82) circle (  2.13);

\path[fill=fillColor,fill opacity=0.20] (187.92, 73.18) circle (  2.13);

\path[fill=fillColor,fill opacity=0.20] (196.44, 67.27) circle (  2.13);

\path[fill=fillColor,fill opacity=0.20] (203.87, 50.24) circle (  2.13);

\path[fill=fillColor,fill opacity=0.20] (205.18, 45.78) circle (  2.13);

\path[fill=fillColor,fill opacity=0.20] (204.96, 52.94) circle (  2.13);

\path[fill=fillColor,fill opacity=0.20] (203.43, 52.52) circle (  2.13);

\path[fill=fillColor,fill opacity=0.20] (202.56, 52.32) circle (  2.13);

\path[fill=fillColor,fill opacity=0.20] (200.59, 55.43) circle (  2.13);

\path[fill=fillColor,fill opacity=0.20] (201.90, 58.23) circle (  2.13);

\path[fill=fillColor,fill opacity=0.20] (201.25, 70.59) circle (  2.13);

\path[fill=fillColor,fill opacity=0.20] (198.41, 79.00) circle (  2.13);

\path[fill=fillColor,fill opacity=0.20] (194.91, 76.09) circle (  2.13);

\path[fill=fillColor,fill opacity=0.20] (189.01, 86.89) circle (  2.13);

\path[fill=fillColor,fill opacity=0.20] (182.89, 93.12) circle (  2.13);

\path[fill=fillColor,fill opacity=0.20] (186.61, 80.66) circle (  2.13);

\path[fill=fillColor,fill opacity=0.20] (191.41, 72.35) circle (  2.13);

\path[fill=fillColor,fill opacity=0.20] (191.41, 79.31) circle (  2.13);

\path[fill=fillColor,fill opacity=0.20] (196.66, 92.08) circle (  2.13);

\path[fill=fillColor,fill opacity=0.20] (187.70, 97.27) circle (  2.13);

\path[fill=fillColor,fill opacity=0.20] (183.77, 91.04) circle (  2.13);

\path[fill=fillColor,fill opacity=0.20] (190.98, 96.23) circle (  2.13);

\path[fill=fillColor,fill opacity=0.20] (189.67,111.81) circle (  2.13);

\path[fill=fillColor,fill opacity=0.20] (180.27,107.65) circle (  2.13);

\path[fill=fillColor,fill opacity=0.20] (178.52,103.50) circle (  2.13);

\path[fill=fillColor,fill opacity=0.20] (204.31, 65.09) circle (  2.13);

\path[fill=fillColor,fill opacity=0.20] (203.21, 65.29) circle (  2.13);

\path[fill=fillColor,fill opacity=0.20] (203.43, 64.05) circle (  2.13);

\path[fill=fillColor,fill opacity=0.20] (198.84, 69.76) circle (  2.13);

\path[fill=fillColor,fill opacity=0.20] (202.34, 71.73) circle (  2.13);

\path[fill=fillColor,fill opacity=0.20] (207.37, 71.00) circle (  2.13);

\path[fill=fillColor,fill opacity=0.20] (207.58, 69.13) circle (  2.13);

\path[fill=fillColor,fill opacity=0.20] (206.05, 63.74) circle (  2.13);

\path[fill=fillColor,fill opacity=0.20] (205.18, 57.82) circle (  2.13);

\path[fill=fillColor,fill opacity=0.20] (200.15, 58.44) circle (  2.13);

\path[fill=fillColor,fill opacity=0.20] (192.73, 63.42) circle (  2.13);

\path[fill=fillColor,fill opacity=0.20] (181.58, 66.02) circle (  2.13);

\path[fill=fillColor,fill opacity=0.20] (207.37, 73.39) circle (  2.13);

\path[fill=fillColor,fill opacity=0.20] (206.49, 74.12) circle (  2.13);

\path[fill=fillColor,fill opacity=0.20] (213.92, 75.57) circle (  2.13);

\path[fill=fillColor,fill opacity=0.20] (210.86, 73.18) circle (  2.13);

\path[fill=fillColor,fill opacity=0.20] (210.86, 66.75) circle (  2.13);

\path[fill=fillColor,fill opacity=0.20] (206.71, 61.87) circle (  2.13);

\path[fill=fillColor,fill opacity=0.20] (203.87, 61.35) circle (  2.13);

\path[fill=fillColor,fill opacity=0.20] (201.03, 63.22) circle (  2.13);

\path[fill=fillColor,fill opacity=0.20] (190.10, 67.68) circle (  2.13);

\path[fill=fillColor,fill opacity=0.20] (171.97, 70.69) circle (  2.13);

\path[fill=fillColor,fill opacity=0.20] (200.59, 86.89) circle (  2.13);

\path[fill=fillColor,fill opacity=0.20] (207.58, 75.78) circle (  2.13);

\path[fill=fillColor,fill opacity=0.20] (212.39, 68.20) circle (  2.13);

\path[fill=fillColor,fill opacity=0.20] (215.01, 70.28) circle (  2.13);

\path[fill=fillColor,fill opacity=0.20] (211.74, 69.03) circle (  2.13);

\path[fill=fillColor,fill opacity=0.20] (209.11, 64.88) circle (  2.13);

\path[fill=fillColor,fill opacity=0.20] (204.31, 64.36) circle (  2.13);

\path[fill=fillColor,fill opacity=0.20] (200.37, 66.33) circle (  2.13);

\path[fill=fillColor,fill opacity=0.20] (193.82, 68.20) circle (  2.13);

\path[fill=fillColor,fill opacity=0.20] (181.58, 75.78) circle (  2.13);

\path[fill=fillColor,fill opacity=0.20] (166.94, 86.89) circle (  2.13);

\path[fill=fillColor,fill opacity=0.20] (180.05, 84.81) circle (  2.13);

\path[fill=fillColor,fill opacity=0.20] (205.18, 81.07) circle (  2.13);

\path[fill=fillColor,fill opacity=0.20] (208.24, 71.52) circle (  2.13);

\path[fill=fillColor,fill opacity=0.20] (208.68, 62.28) circle (  2.13);

\path[fill=fillColor,fill opacity=0.20] (207.37, 60.21) circle (  2.13);

\path[fill=fillColor,fill opacity=0.20] (207.80, 62.59) circle (  2.13);

\path[fill=fillColor,fill opacity=0.20] (204.96, 66.64) circle (  2.13);

\path[fill=fillColor,fill opacity=0.20] (199.06, 69.24) circle (  2.13);

\path[fill=fillColor,fill opacity=0.20] (192.29, 66.64) circle (  2.13);

\path[fill=fillColor,fill opacity=0.20] (183.77, 66.12) circle (  2.13);

\path[fill=fillColor,fill opacity=0.20] (168.69, 78.06) circle (  2.13);

\path[fill=fillColor,fill opacity=0.20] (204.31, 64.98) circle (  2.13);

\path[fill=fillColor,fill opacity=0.20] (205.40, 64.46) circle (  2.13);

\path[fill=fillColor,fill opacity=0.20] (206.71, 63.63) circle (  2.13);

\path[fill=fillColor,fill opacity=0.20] (205.18, 54.70) circle (  2.13);

\path[fill=fillColor,fill opacity=0.20] (205.40, 56.05) circle (  2.13);

\path[fill=fillColor,fill opacity=0.20] (201.03, 65.19) circle (  2.13);

\path[fill=fillColor,fill opacity=0.20] (198.84, 68.41) circle (  2.13);

\path[fill=fillColor,fill opacity=0.20] (187.48, 65.50) circle (  2.13);

\path[fill=fillColor,fill opacity=0.20] (172.41, 67.89) circle (  2.13);

\path[fill=fillColor,fill opacity=0.20] (207.58, 58.44) circle (  2.13);

\path[fill=fillColor,fill opacity=0.20] (208.89, 54.70) circle (  2.13);

\path[fill=fillColor,fill opacity=0.20] (203.87, 56.99) circle (  2.13);

\path[fill=fillColor,fill opacity=0.20] (205.40, 59.06) circle (  2.13);

\path[fill=fillColor,fill opacity=0.20] (207.37, 63.94) circle (  2.13);

\path[fill=fillColor,fill opacity=0.20] (209.77, 53.87) circle (  2.13);

\path[fill=fillColor,fill opacity=0.20] (201.03, 54.39) circle (  2.13);

\path[fill=fillColor,fill opacity=0.20] (200.81, 63.11) circle (  2.13);

\path[fill=fillColor,fill opacity=0.20] (199.06, 64.98) circle (  2.13);

\path[fill=fillColor,fill opacity=0.20] (186.17, 67.68) circle (  2.13);

\path[fill=fillColor,fill opacity=0.20] (170.00, 86.89) circle (  2.13);

\path[fill=fillColor,fill opacity=0.20] (209.55, 54.08) circle (  2.13);

\path[fill=fillColor,fill opacity=0.20] (210.86, 51.90) circle (  2.13);

\path[fill=fillColor,fill opacity=0.20] (210.64, 54.39) circle (  2.13);

\path[fill=fillColor,fill opacity=0.20] (207.37, 58.03) circle (  2.13);

\path[fill=fillColor,fill opacity=0.20] (205.18, 60.41) circle (  2.13);

\path[fill=fillColor,fill opacity=0.20] (209.11, 59.27) circle (  2.13);

\path[fill=fillColor,fill opacity=0.20] (207.37, 62.08) circle (  2.13);

\path[fill=fillColor,fill opacity=0.20] (208.02, 57.20) circle (  2.13);

\path[fill=fillColor,fill opacity=0.20] (204.09, 60.00) circle (  2.13);

\path[fill=fillColor,fill opacity=0.20] (200.15, 66.33) circle (  2.13);

\path[fill=fillColor,fill opacity=0.20] (199.72, 65.50) circle (  2.13);

\path[fill=fillColor,fill opacity=0.20] (188.57, 72.98) circle (  2.13);

\path[fill=fillColor,fill opacity=0.20] (212.17, 54.18) circle (  2.13);

\path[fill=fillColor,fill opacity=0.20] (207.37, 53.25) circle (  2.13);

\path[fill=fillColor,fill opacity=0.20] (206.49, 50.97) circle (  2.13);

\path[fill=fillColor,fill opacity=0.20] (208.89, 46.29) circle (  2.13);

\path[fill=fillColor,fill opacity=0.20] (196.44, 58.96) circle (  2.13);

\path[fill=fillColor,fill opacity=0.20] (211.74, 58.55) circle (  2.13);

\path[fill=fillColor,fill opacity=0.20] (208.02, 61.56) circle (  2.13);

\path[fill=fillColor,fill opacity=0.20] (208.46, 64.05) circle (  2.13);

\path[fill=fillColor,fill opacity=0.20] (202.78, 65.81) circle (  2.13);

\path[fill=fillColor,fill opacity=0.20] (196.00, 65.81) circle (  2.13);

\path[fill=fillColor,fill opacity=0.20] (197.10, 65.19) circle (  2.13);

\path[fill=fillColor,fill opacity=0.20] (192.29, 74.74) circle (  2.13);

\path[fill=fillColor,fill opacity=0.20] (196.88, 59.69) circle (  2.13);

\path[fill=fillColor,fill opacity=0.20] (208.46, 57.20) circle (  2.13);

\path[fill=fillColor,fill opacity=0.20] (206.05, 54.50) circle (  2.13);

\path[fill=fillColor,fill opacity=0.20] (202.34, 55.64) circle (  2.13);

\path[fill=fillColor,fill opacity=0.20] (201.03, 53.56) circle (  2.13);

\path[fill=fillColor,fill opacity=0.20] (209.11, 47.33) circle (  2.13);

\path[fill=fillColor,fill opacity=0.20] (204.31, 50.55) circle (  2.13);

\path[fill=fillColor,fill opacity=0.20] (182.67, 58.34) circle (  2.13);

\path[fill=fillColor,fill opacity=0.20] (206.93, 54.50) circle (  2.13);

\path[fill=fillColor,fill opacity=0.20] (206.71, 61.45) circle (  2.13);

\path[fill=fillColor,fill opacity=0.20] (205.40, 67.16) circle (  2.13);

\path[fill=fillColor,fill opacity=0.20] (203.00, 67.89) circle (  2.13);

\path[fill=fillColor,fill opacity=0.20] (197.31, 58.34) circle (  2.13);

\path[fill=fillColor,fill opacity=0.20] (197.53, 56.47) circle (  2.13);

\path[fill=fillColor,fill opacity=0.20] (197.53, 72.04) circle (  2.13);

\path[fill=fillColor,fill opacity=0.20] (188.14, 69.65) circle (  2.13);

\path[fill=fillColor,fill opacity=0.20] (192.73, 62.28) circle (  2.13);

\path[fill=fillColor,fill opacity=0.20] (195.35, 54.39) circle (  2.13);

\path[fill=fillColor,fill opacity=0.20] (199.28, 53.56) circle (  2.13);

\path[fill=fillColor,fill opacity=0.20] (197.53, 56.36) circle (  2.13);

\path[fill=fillColor,fill opacity=0.20] (204.31, 58.13) circle (  2.13);

\path[fill=fillColor,fill opacity=0.20] (197.10, 54.50) circle (  2.13);

\path[fill=fillColor,fill opacity=0.20] (200.37, 53.35) circle (  2.13);

\path[fill=fillColor,fill opacity=0.20] (186.83, 60.73) circle (  2.13);

\path[fill=fillColor,fill opacity=0.20] (185.73, 75.57) circle (  2.13);

\path[fill=fillColor,fill opacity=0.20] (199.94, 56.68) circle (  2.13);

\path[fill=fillColor,fill opacity=0.20] (208.02, 57.51) circle (  2.13);

\path[fill=fillColor,fill opacity=0.20] (202.78, 63.22) circle (  2.13);

\path[fill=fillColor,fill opacity=0.20] (200.15, 69.34) circle (  2.13);

\path[fill=fillColor,fill opacity=0.20] (200.59, 61.14) circle (  2.13);

\path[fill=fillColor,fill opacity=0.20] (199.28, 53.87) circle (  2.13);

\path[fill=fillColor,fill opacity=0.20] (201.90, 70.80) circle (  2.13);

\path[fill=fillColor,fill opacity=0.20] (191.41,100.39) circle (  2.13);

\path[fill=fillColor,fill opacity=0.20] (192.29, 80.24) circle (  2.13);

\path[fill=fillColor,fill opacity=0.20] (196.88, 70.69) circle (  2.13);

\path[fill=fillColor,fill opacity=0.20] (201.03, 60.73) circle (  2.13);

\path[fill=fillColor,fill opacity=0.20] (194.26, 53.67) circle (  2.13);

\path[fill=fillColor,fill opacity=0.20] (194.69, 52.11) circle (  2.13);

\path[fill=fillColor,fill opacity=0.20] (194.69, 57.30) circle (  2.13);

\path[fill=fillColor,fill opacity=0.20] (194.91, 53.77) circle (  2.13);

\path[fill=fillColor,fill opacity=0.20] (201.47, 47.75) circle (  2.13);

\path[fill=fillColor,fill opacity=0.20] (186.17, 58.65) circle (  2.13);

\path[fill=fillColor,fill opacity=0.20] (181.36, 61.35) circle (  2.13);

\path[fill=fillColor,fill opacity=0.20] (199.06, 53.87) circle (  2.13);

\path[fill=fillColor,fill opacity=0.20] (204.31, 59.38) circle (  2.13);

\path[fill=fillColor,fill opacity=0.20] (216.76, 69.24) circle (  2.13);

\path[fill=fillColor,fill opacity=0.20] (201.68, 70.69) circle (  2.13);

\path[fill=fillColor,fill opacity=0.20] (199.28, 61.45) circle (  2.13);

\path[fill=fillColor,fill opacity=0.20] (203.21, 67.16) circle (  2.13);

\path[fill=fillColor,fill opacity=0.20] (198.63, 84.81) circle (  2.13);

\path[fill=fillColor,fill opacity=0.20] (200.15, 79.00) circle (  2.13);

\path[fill=fillColor,fill opacity=0.20] (200.37, 70.80) circle (  2.13);

\path[fill=fillColor,fill opacity=0.20] (201.47, 60.83) circle (  2.13);

\path[fill=fillColor,fill opacity=0.20] (197.31, 52.52) circle (  2.13);

\path[fill=fillColor,fill opacity=0.20] (200.59, 49.72) circle (  2.13);

\path[fill=fillColor,fill opacity=0.20] (199.28, 55.64) circle (  2.13);

\path[fill=fillColor,fill opacity=0.20] (194.47, 54.08) circle (  2.13);

\path[fill=fillColor,fill opacity=0.20] (199.72, 47.23) circle (  2.13);

\path[fill=fillColor,fill opacity=0.20] (191.20, 57.71) circle (  2.13);

\path[fill=fillColor,fill opacity=0.20] (167.82, 60.31) circle (  2.13);

\path[fill=fillColor,fill opacity=0.20] (175.46, 54.91) circle (  2.13);

\path[fill=fillColor,fill opacity=0.20] (194.47, 58.86) circle (  2.13);

\path[fill=fillColor,fill opacity=0.20] (214.79, 59.06) circle (  2.13);

\path[fill=fillColor,fill opacity=0.20] (202.34, 65.50) circle (  2.13);

\path[fill=fillColor,fill opacity=0.20] (200.59, 65.50) circle (  2.13);

\path[fill=fillColor,fill opacity=0.20] (204.09, 59.27) circle (  2.13);

\path[fill=fillColor,fill opacity=0.20] (200.59, 66.75) circle (  2.13);

\path[fill=fillColor,fill opacity=0.20] (195.13,100.39) circle (  2.13);

\path[fill=fillColor,fill opacity=0.20] (196.88, 69.86) circle (  2.13);

\path[fill=fillColor,fill opacity=0.20] (201.03, 59.17) circle (  2.13);

\path[fill=fillColor,fill opacity=0.20] (206.27, 55.43) circle (  2.13);

\path[fill=fillColor,fill opacity=0.20] (198.63, 53.25) circle (  2.13);

\path[fill=fillColor,fill opacity=0.20] (197.10, 57.40) circle (  2.13);

\path[fill=fillColor,fill opacity=0.20] (204.52, 60.10) circle (  2.13);

\path[fill=fillColor,fill opacity=0.20] (198.63, 56.68) circle (  2.13);

\path[fill=fillColor,fill opacity=0.20] (185.73, 60.10) circle (  2.13);

\path[fill=fillColor,fill opacity=0.20] (175.68, 54.60) circle (  2.13);

\path[fill=fillColor,fill opacity=0.20] (193.16, 44.53) circle (  2.13);

\path[fill=fillColor,fill opacity=0.20] (202.56, 51.17) circle (  2.13);

\path[fill=fillColor,fill opacity=0.20] (206.71, 60.83) circle (  2.13);

\path[fill=fillColor,fill opacity=0.20] (201.03, 57.82) circle (  2.13);

\path[fill=fillColor,fill opacity=0.20] (200.37, 61.45) circle (  2.13);

\path[fill=fillColor,fill opacity=0.20] (199.94, 80.87) circle (  2.13);

\path[fill=fillColor,fill opacity=0.20] (197.97, 86.89) circle (  2.13);

\path[fill=fillColor,fill opacity=0.20] (205.18, 71.73) circle (  2.13);

\path[fill=fillColor,fill opacity=0.20] (210.21, 57.71) circle (  2.13);

\path[fill=fillColor,fill opacity=0.20] (209.55, 58.86) circle (  2.13);

\path[fill=fillColor,fill opacity=0.20] (200.59, 56.57) circle (  2.13);

\path[fill=fillColor,fill opacity=0.20] (199.72, 52.00) circle (  2.13);

\path[fill=fillColor,fill opacity=0.20] (204.31, 57.20) circle (  2.13);

\path[fill=fillColor,fill opacity=0.20] (201.90, 60.10) circle (  2.13);

\path[fill=fillColor,fill opacity=0.20] (199.06, 55.53) circle (  2.13);

\path[fill=fillColor,fill opacity=0.20] (191.63, 59.48) circle (  2.13);

\path[fill=fillColor,fill opacity=0.20] (171.97, 74.01) circle (  2.13);

\path[fill=fillColor,fill opacity=0.20] (167.38, 59.48) circle (  2.13);

\path[fill=fillColor,fill opacity=0.20] (177.43, 45.88) circle (  2.13);

\path[fill=fillColor,fill opacity=0.20] (192.73, 48.58) circle (  2.13);

\path[fill=fillColor,fill opacity=0.20] (205.62, 57.92) circle (  2.13);

\path[fill=fillColor,fill opacity=0.20] (203.43, 59.69) circle (  2.13);

\path[fill=fillColor,fill opacity=0.20] (199.94, 60.62) circle (  2.13);

\path[fill=fillColor,fill opacity=0.20] (200.15, 65.50) circle (  2.13);

\path[fill=fillColor,fill opacity=0.20] (196.22, 78.17) circle (  2.13);

\path[fill=fillColor,fill opacity=0.20] (211.95, 96.23) circle (  2.13);

\path[fill=fillColor,fill opacity=0.20] (211.08, 79.41) circle (  2.13);

\path[fill=fillColor,fill opacity=0.20] (210.64, 67.06) circle (  2.13);

\path[fill=fillColor,fill opacity=0.20] (211.74, 63.42) circle (  2.13);

\path[fill=fillColor,fill opacity=0.20] (207.80, 56.16) circle (  2.13);

\path[fill=fillColor,fill opacity=0.20] (207.58, 50.76) circle (  2.13);

\path[fill=fillColor,fill opacity=0.20] (202.12, 52.63) circle (  2.13);

\path[fill=fillColor,fill opacity=0.20] (199.06, 49.72) circle (  2.13);

\path[fill=fillColor,fill opacity=0.20] (199.50, 43.49) circle (  2.13);

\path[fill=fillColor,fill opacity=0.20] (189.23, 54.60) circle (  2.13);

\path[fill=fillColor,fill opacity=0.20] (178.96, 64.67) circle (  2.13);

\path[fill=fillColor,fill opacity=0.20] (183.77, 56.26) circle (  2.13);

\path[fill=fillColor,fill opacity=0.20] (186.83, 55.53) circle (  2.13);

\path[fill=fillColor,fill opacity=0.20] (200.59, 55.85) circle (  2.13);

\path[fill=fillColor,fill opacity=0.20] (204.74, 55.43) circle (  2.13);

\path[fill=fillColor,fill opacity=0.20] (202.56, 59.27) circle (  2.13);

\path[fill=fillColor,fill opacity=0.20] (198.19, 69.76) circle (  2.13);

\path[fill=fillColor,fill opacity=0.20] (194.91, 76.61) circle (  2.13);

\path[fill=fillColor,fill opacity=0.20] (190.54,105.58) circle (  2.13);

\path[fill=fillColor,fill opacity=0.20] (210.21, 92.08) circle (  2.13);

\path[fill=fillColor,fill opacity=0.20] (217.20, 84.81) circle (  2.13);

\path[fill=fillColor,fill opacity=0.20] (213.48, 69.86) circle (  2.13);

\path[fill=fillColor,fill opacity=0.20] (209.99, 63.74) circle (  2.13);

\path[fill=fillColor,fill opacity=0.20] (209.33, 60.41) circle (  2.13);

\path[fill=fillColor,fill opacity=0.20] (213.70, 56.16) circle (  2.13);

\path[fill=fillColor,fill opacity=0.20] (212.61, 54.81) circle (  2.13);

\path[fill=fillColor,fill opacity=0.20] (208.68, 52.94) circle (  2.13);

\path[fill=fillColor,fill opacity=0.20] (203.00, 45.98) circle (  2.13);

\path[fill=fillColor,fill opacity=0.20] (192.51, 41.10) circle (  2.13);

\path[fill=fillColor,fill opacity=0.20] (181.36, 49.10) circle (  2.13);

\path[fill=fillColor,fill opacity=0.20] (170.88, 66.75) circle (  2.13);

\path[fill=fillColor,fill opacity=0.20] (169.56, 57.71) circle (  2.13);

\path[fill=fillColor,fill opacity=0.20] (180.27, 52.94) circle (  2.13);

\path[fill=fillColor,fill opacity=0.20] (198.19, 53.98) circle (  2.13);

\path[fill=fillColor,fill opacity=0.20] (203.21, 60.93) circle (  2.13);

\path[fill=fillColor,fill opacity=0.20] (201.25, 70.69) circle (  2.13);

\path[fill=fillColor,fill opacity=0.20] (201.03, 72.04) circle (  2.13);

\path[fill=fillColor,fill opacity=0.20] (194.04, 75.47) circle (  2.13);

\path[fill=fillColor,fill opacity=0.20] (210.21, 83.77) circle (  2.13);

\path[fill=fillColor,fill opacity=0.20] (220.04, 73.70) circle (  2.13);

\path[fill=fillColor,fill opacity=0.20] (219.38, 64.67) circle (  2.13);

\path[fill=fillColor,fill opacity=0.20] (212.83, 53.67) circle (  2.13);

\path[fill=fillColor,fill opacity=0.20] (211.08, 55.12) circle (  2.13);

\path[fill=fillColor,fill opacity=0.20] (206.93, 60.73) circle (  2.13);

\path[fill=fillColor,fill opacity=0.20] (205.84, 56.99) circle (  2.13);

\path[fill=fillColor,fill opacity=0.20] (206.71, 55.53) circle (  2.13);

\path[fill=fillColor,fill opacity=0.20] (204.31, 57.51) circle (  2.13);

\path[fill=fillColor,fill opacity=0.20] (200.15, 56.68) circle (  2.13);

\path[fill=fillColor,fill opacity=0.20] (188.57, 52.21) circle (  2.13);

\path[fill=fillColor,fill opacity=0.20] (177.87, 52.52) circle (  2.13);

\path[fill=fillColor,fill opacity=0.20] (170.88, 74.22) circle (  2.13);

\path[fill=fillColor,fill opacity=0.20] (175.90, 62.59) circle (  2.13);

\path[fill=fillColor,fill opacity=0.20] (181.15, 56.57) circle (  2.13);

\path[fill=fillColor,fill opacity=0.20] (194.69, 60.31) circle (  2.13);

\path[fill=fillColor,fill opacity=0.20] (203.87, 64.77) circle (  2.13);

\path[fill=fillColor,fill opacity=0.20] (203.87, 71.32) circle (  2.13);

\path[fill=fillColor,fill opacity=0.20] (202.34, 71.94) circle (  2.13);

\path[fill=fillColor,fill opacity=0.20] (198.41, 76.40) circle (  2.13);

\path[fill=fillColor,fill opacity=0.20] (214.14, 83.77) circle (  2.13);

\path[fill=fillColor,fill opacity=0.20] (218.29, 70.28) circle (  2.13);

\path[fill=fillColor,fill opacity=0.20] (227.69, 66.02) circle (  2.13);

\path[fill=fillColor,fill opacity=0.20] (218.07, 61.24) circle (  2.13);

\path[fill=fillColor,fill opacity=0.20] (213.05, 53.98) circle (  2.13);

\path[fill=fillColor,fill opacity=0.20] (215.23, 59.69) circle (  2.13);

\path[fill=fillColor,fill opacity=0.20] (208.68, 66.85) circle (  2.13);

\path[fill=fillColor,fill opacity=0.20] (202.78, 60.10) circle (  2.13);

\path[fill=fillColor,fill opacity=0.20] (202.78, 56.36) circle (  2.13);

\path[fill=fillColor,fill opacity=0.20] (201.90, 59.58) circle (  2.13);

\path[fill=fillColor,fill opacity=0.20] (196.22, 60.21) circle (  2.13);

\path[fill=fillColor,fill opacity=0.20] (177.65, 62.28) circle (  2.13);

\path[fill=fillColor,fill opacity=0.20] (166.07, 78.69) circle (  2.13);

\path[fill=fillColor,fill opacity=0.20] (182.02, 59.58) circle (  2.13);

\path[fill=fillColor,fill opacity=0.20] (201.90, 60.00) circle (  2.13);

\path[fill=fillColor,fill opacity=0.20] (208.24, 66.54) circle (  2.13);

\path[fill=fillColor,fill opacity=0.20] (206.05, 74.74) circle (  2.13);

\path[fill=fillColor,fill opacity=0.20] (201.03, 73.29) circle (  2.13);

\path[fill=fillColor,fill opacity=0.20] (200.15, 80.14) circle (  2.13);

\path[fill=fillColor,fill opacity=0.20] (217.42, 91.04) circle (  2.13);

\path[fill=fillColor,fill opacity=0.20] (222.44, 73.81) circle (  2.13);

\path[fill=fillColor,fill opacity=0.20] (220.69, 70.38) circle (  2.13);

\path[fill=fillColor,fill opacity=0.20] (223.10, 69.03) circle (  2.13);

\path[fill=fillColor,fill opacity=0.20] (215.45, 63.32) circle (  2.13);

\path[fill=fillColor,fill opacity=0.20] (213.92, 60.62) circle (  2.13);

\path[fill=fillColor,fill opacity=0.20] (214.14, 64.26) circle (  2.13);

\path[fill=fillColor,fill opacity=0.20] (212.17, 63.22) circle (  2.13);

\path[fill=fillColor,fill opacity=0.20] (207.37, 59.48) circle (  2.13);

\path[fill=fillColor,fill opacity=0.20] (196.88, 62.59) circle (  2.13);

\path[fill=fillColor,fill opacity=0.20] (190.98, 60.41) circle (  2.13);

\path[fill=fillColor,fill opacity=0.20] (184.20, 55.64) circle (  2.13);

\path[fill=fillColor,fill opacity=0.20] (180.71, 61.66) circle (  2.13);

\path[fill=fillColor,fill opacity=0.20] (165.41, 67.06) circle (  2.13);

\path[fill=fillColor,fill opacity=0.20] (178.74, 58.96) circle (  2.13);

\path[fill=fillColor,fill opacity=0.20] (193.82, 58.13) circle (  2.13);

\path[fill=fillColor,fill opacity=0.20] (198.63, 68.82) circle (  2.13);

\path[fill=fillColor,fill opacity=0.20] (208.46, 72.66) circle (  2.13);

\path[fill=fillColor,fill opacity=0.20] (207.37, 67.27) circle (  2.13);

\path[fill=fillColor,fill opacity=0.20] (204.52, 78.89) circle (  2.13);

\path[fill=fillColor,fill opacity=0.20] (219.82,110.77) circle (  2.13);

\path[fill=fillColor,fill opacity=0.20] (214.36, 83.77) circle (  2.13);

\path[fill=fillColor,fill opacity=0.20] (219.82, 68.10) circle (  2.13);

\path[fill=fillColor,fill opacity=0.20] (223.97, 65.40) circle (  2.13);

\path[fill=fillColor,fill opacity=0.20] (219.82, 65.92) circle (  2.13);

\path[fill=fillColor,fill opacity=0.20] (212.39, 64.98) circle (  2.13);

\path[fill=fillColor,fill opacity=0.20] (212.61, 59.17) circle (  2.13);

\path[fill=fillColor,fill opacity=0.20] (217.20, 61.24) circle (  2.13);

\path[fill=fillColor,fill opacity=0.20] (208.89, 64.77) circle (  2.13);

\path[fill=fillColor,fill opacity=0.20] (203.43, 54.39) circle (  2.13);

\path[fill=fillColor,fill opacity=0.20] (192.94, 51.59) circle (  2.13);

\path[fill=fillColor,fill opacity=0.20] (185.08, 61.76) circle (  2.13);

\path[fill=fillColor,fill opacity=0.20] (175.68, 63.01) circle (  2.13);

\path[fill=fillColor,fill opacity=0.20] (183.99, 56.16) circle (  2.13);

\path[fill=fillColor,fill opacity=0.20] (166.94, 62.49) circle (  2.13);

\path[fill=fillColor,fill opacity=0.20] (175.46, 54.70) circle (  2.13);

\path[fill=fillColor,fill opacity=0.20] (187.48, 59.69) circle (  2.13);

\path[fill=fillColor,fill opacity=0.20] (201.90, 64.36) circle (  2.13);

\path[fill=fillColor,fill opacity=0.20] (206.71, 64.98) circle (  2.13);

\path[fill=fillColor,fill opacity=0.20] (204.09, 65.92) circle (  2.13);

\path[fill=fillColor,fill opacity=0.20] (204.52, 69.86) circle (  2.13);

\path[fill=fillColor,fill opacity=0.20] (198.41, 90.00) circle (  2.13);

\path[fill=fillColor,fill opacity=0.20] (216.76,108.69) circle (  2.13);

\path[fill=fillColor,fill opacity=0.20] (221.57, 88.96) circle (  2.13);

\path[fill=fillColor,fill opacity=0.20] (225.72, 78.58) circle (  2.13);

\path[fill=fillColor,fill opacity=0.20] (219.60, 61.04) circle (  2.13);

\path[fill=fillColor,fill opacity=0.20] (220.48, 56.05) circle (  2.13);

\path[fill=fillColor,fill opacity=0.20] (227.03, 62.59) circle (  2.13);

\path[fill=fillColor,fill opacity=0.20] (219.38, 58.55) circle (  2.13);

\path[fill=fillColor,fill opacity=0.20] (232.06, 56.68) circle (  2.13);

\path[fill=fillColor,fill opacity=0.20] (219.38, 54.91) circle (  2.13);

\path[fill=fillColor,fill opacity=0.20] (208.46, 57.61) circle (  2.13);

\path[fill=fillColor,fill opacity=0.20] (198.41, 63.01) circle (  2.13);

\path[fill=fillColor,fill opacity=0.20] (192.29, 52.00) circle (  2.13);

\path[fill=fillColor,fill opacity=0.20] (178.52, 45.05) circle (  2.13);

\path[fill=fillColor,fill opacity=0.20] (174.15, 57.71) circle (  2.13);

\path[fill=fillColor,fill opacity=0.20] (185.30, 66.85) circle (  2.13);

\path[fill=fillColor,fill opacity=0.20] (169.56, 55.02) circle (  2.13);

\path[fill=fillColor,fill opacity=0.20] (183.99, 51.38) circle (  2.13);

\path[fill=fillColor,fill opacity=0.20] (197.75, 54.70) circle (  2.13);

\path[fill=fillColor,fill opacity=0.20] (207.15, 60.00) circle (  2.13);

\path[fill=fillColor,fill opacity=0.20] (205.18, 60.93) circle (  2.13);

\path[fill=fillColor,fill opacity=0.20] (203.87, 65.71) circle (  2.13);

\path[fill=fillColor,fill opacity=0.20] (197.97, 82.74) circle (  2.13);

\path[fill=fillColor,fill opacity=0.20] (225.72, 92.08) circle (  2.13);

\path[fill=fillColor,fill opacity=0.20] (228.56, 76.40) circle (  2.13);

\path[fill=fillColor,fill opacity=0.20] (234.02, 72.35) circle (  2.13);

\path[fill=fillColor,fill opacity=0.20] (222.88, 69.24) circle (  2.13);

\path[fill=fillColor,fill opacity=0.20] (223.97, 62.49) circle (  2.13);

\path[fill=fillColor,fill opacity=0.20] (216.76, 62.59) circle (  2.13);

\path[fill=fillColor,fill opacity=0.20] (221.35, 62.18) circle (  2.13);

\path[fill=fillColor,fill opacity=0.20] (219.38, 60.62) circle (  2.13);

\path[fill=fillColor,fill opacity=0.20] (209.77, 60.83) circle (  2.13);

\path[fill=fillColor,fill opacity=0.20] (202.12, 53.98) circle (  2.13);

\path[fill=fillColor,fill opacity=0.20] (187.48, 54.18) circle (  2.13);

\path[fill=fillColor,fill opacity=0.20] (183.11, 60.73) circle (  2.13);

\path[fill=fillColor,fill opacity=0.20] (182.67, 57.30) circle (  2.13);

\path[fill=fillColor,fill opacity=0.20] (179.83, 56.16) circle (  2.13);

\path[fill=fillColor,fill opacity=0.20] (188.79, 68.20) circle (  2.13);

\path[fill=fillColor,fill opacity=0.20] (166.94, 47.96) circle (  2.13);

\path[fill=fillColor,fill opacity=0.20] (185.30, 49.31) circle (  2.13);

\path[fill=fillColor,fill opacity=0.20] (197.31, 61.04) circle (  2.13);

\path[fill=fillColor,fill opacity=0.20] (205.40, 73.50) circle (  2.13);

\path[fill=fillColor,fill opacity=0.20] (206.27, 74.95) circle (  2.13);

\path[fill=fillColor,fill opacity=0.20] (200.37, 70.90) circle (  2.13);

\path[fill=fillColor,fill opacity=0.20] (200.59, 74.74) circle (  2.13);

\path[fill=fillColor,fill opacity=0.20] (198.41, 95.19) circle (  2.13);

\path[fill=fillColor,fill opacity=0.20] (223.53, 99.35) circle (  2.13);

\path[fill=fillColor,fill opacity=0.20] (225.50, 74.95) circle (  2.13);

\path[fill=fillColor,fill opacity=0.20] (230.74, 69.03) circle (  2.13);

\path[fill=fillColor,fill opacity=0.20] (247.35, 70.69) circle (  2.13);

\path[fill=fillColor,fill opacity=0.20] (219.38, 67.47) circle (  2.13);

\path[fill=fillColor,fill opacity=0.20] (219.60, 64.15) circle (  2.13);

\path[fill=fillColor,fill opacity=0.20] (221.13, 63.94) circle (  2.13);

\path[fill=fillColor,fill opacity=0.20] (214.36, 62.70) circle (  2.13);

\path[fill=fillColor,fill opacity=0.20] (206.49, 64.98) circle (  2.13);

\path[fill=fillColor,fill opacity=0.20] (196.66, 66.33) circle (  2.13);

\path[fill=fillColor,fill opacity=0.20] (182.24, 62.18) circle (  2.13);

\path[fill=fillColor,fill opacity=0.20] (182.02, 58.96) circle (  2.13);

\path[fill=fillColor,fill opacity=0.20] (174.81, 65.40) circle (  2.13);

\path[fill=fillColor,fill opacity=0.20] (182.67, 70.28) circle (  2.13);

\path[fill=fillColor,fill opacity=0.20] (188.14, 66.02) circle (  2.13);

\path[fill=fillColor,fill opacity=0.20] (180.27, 64.15) circle (  2.13);

\path[fill=fillColor,fill opacity=0.20] (189.89, 68.72) circle (  2.13);

\path[fill=fillColor,fill opacity=0.20] (198.41, 69.13) circle (  2.13);

\path[fill=fillColor,fill opacity=0.20] (204.96, 68.62) circle (  2.13);

\path[fill=fillColor,fill opacity=0.20] (201.25, 66.85) circle (  2.13);

\path[fill=fillColor,fill opacity=0.20] (200.81, 66.33) circle (  2.13);

\path[fill=fillColor,fill opacity=0.20] (201.03, 79.31) circle (  2.13);

\path[fill=fillColor,fill opacity=0.20] (196.88, 99.35) circle (  2.13);

\path[fill=fillColor,fill opacity=0.20] (222.44,101.42) circle (  2.13);

\path[fill=fillColor,fill opacity=0.20] (230.09, 86.89) circle (  2.13);

\path[fill=fillColor,fill opacity=0.20] (225.28, 79.31) circle (  2.13);

\path[fill=fillColor,fill opacity=0.20] (220.26, 70.07) circle (  2.13);

\path[fill=fillColor,fill opacity=0.20] (219.82, 64.46) circle (  2.13);

\path[fill=fillColor,fill opacity=0.20] (217.63, 66.12) circle (  2.13);

\path[fill=fillColor,fill opacity=0.20] (208.89, 61.56) circle (  2.13);

\path[fill=fillColor,fill opacity=0.20] (200.37, 52.94) circle (  2.13);

\path[fill=fillColor,fill opacity=0.20] (191.41, 52.73) circle (  2.13);

\path[fill=fillColor,fill opacity=0.20] (190.32, 59.06) circle (  2.13);

\path[fill=fillColor,fill opacity=0.20] (183.99, 65.40) circle (  2.13);

\path[fill=fillColor,fill opacity=0.20] (191.41, 67.37) circle (  2.13);

\path[fill=fillColor,fill opacity=0.20] (184.64, 65.50) circle (  2.13);

\path[fill=fillColor,fill opacity=0.20] (179.62, 70.07) circle (  2.13);

\path[fill=fillColor,fill opacity=0.20] (168.04, 78.89) circle (  2.13);

\path[fill=fillColor,fill opacity=0.20] (183.77, 53.98) circle (  2.13);

\path[fill=fillColor,fill opacity=0.20] (183.55, 50.14) circle (  2.13);

\path[fill=fillColor,fill opacity=0.20] (198.41, 56.68) circle (  2.13);

\path[fill=fillColor,fill opacity=0.20] (201.25, 69.55) circle (  2.13);

\path[fill=fillColor,fill opacity=0.20] (203.87, 74.33) circle (  2.13);

\path[fill=fillColor,fill opacity=0.20] (205.84, 74.22) circle (  2.13);

\path[fill=fillColor,fill opacity=0.20] (200.81, 77.23) circle (  2.13);

\path[fill=fillColor,fill opacity=0.20] (197.10, 81.70) circle (  2.13);

\path[fill=fillColor,fill opacity=0.20] (199.06, 91.04) circle (  2.13);

\path[fill=fillColor,fill opacity=0.20] (217.42, 95.19) circle (  2.13);

\path[fill=fillColor,fill opacity=0.20] (224.41, 77.23) circle (  2.13);

\path[fill=fillColor,fill opacity=0.20] (227.25, 74.12) circle (  2.13);

\path[fill=fillColor,fill opacity=0.20] (212.61, 75.99) circle (  2.13);

\path[fill=fillColor,fill opacity=0.20] (205.84, 71.94) circle (  2.13);

\path[fill=fillColor,fill opacity=0.20] (200.59, 62.70) circle (  2.13);

\path[fill=fillColor,fill opacity=0.20] (208.46, 58.55) circle (  2.13);

\path[fill=fillColor,fill opacity=0.20] (189.23, 54.39) circle (  2.13);

\path[fill=fillColor,fill opacity=0.20] (185.73, 49.51) circle (  2.13);

\path[fill=fillColor,fill opacity=0.20] (182.02, 55.74) circle (  2.13);

\path[fill=fillColor,fill opacity=0.20] (165.19, 69.65) circle (  2.13);

\path[fill=fillColor,fill opacity=0.20] (178.96, 75.26) circle (  2.13);

\path[fill=fillColor,fill opacity=0.20] (190.98, 76.09) circle (  2.13);

\path[fill=fillColor,fill opacity=0.20] (173.72, 76.51) circle (  2.13);

\path[fill=fillColor,fill opacity=0.20] (170.00, 60.21) circle (  2.13);

\path[fill=fillColor,fill opacity=0.20] (175.68, 51.38) circle (  2.13);

\path[fill=fillColor,fill opacity=0.20] (180.05, 50.86) circle (  2.13);

\path[fill=fillColor,fill opacity=0.20] (193.60, 63.94) circle (  2.13);

\path[fill=fillColor,fill opacity=0.20] (200.59, 75.26) circle (  2.13);

\path[fill=fillColor,fill opacity=0.20] (198.63, 66.95) circle (  2.13);

\path[fill=fillColor,fill opacity=0.20] (202.34, 66.95) circle (  2.13);

\path[fill=fillColor,fill opacity=0.20] (198.63, 76.30) circle (  2.13);

\path[fill=fillColor,fill opacity=0.20] (198.84, 79.21) circle (  2.13);

\path[fill=fillColor,fill opacity=0.20] (200.37, 81.49) circle (  2.13);

\path[fill=fillColor,fill opacity=0.20] (200.59, 93.12) circle (  2.13);

\path[fill=fillColor,fill opacity=0.20] (212.39,113.88) circle (  2.13);

\path[fill=fillColor,fill opacity=0.20] (209.77, 94.16) circle (  2.13);

\path[fill=fillColor,fill opacity=0.20] (209.11, 91.04) circle (  2.13);

\path[fill=fillColor,fill opacity=0.20] (205.84, 84.81) circle (  2.13);

\path[fill=fillColor,fill opacity=0.20] (201.90, 73.18) circle (  2.13);

\path[fill=fillColor,fill opacity=0.20] (198.84, 66.33) circle (  2.13);

\path[fill=fillColor,fill opacity=0.20] (186.17, 67.79) circle (  2.13);

\path[fill=fillColor,fill opacity=0.20] (192.07, 66.23) circle (  2.13);

\path[fill=fillColor,fill opacity=0.20] (183.55, 61.24) circle (  2.13);

\path[fill=fillColor,fill opacity=0.20] (180.27, 61.35) circle (  2.13);

\path[fill=fillColor,fill opacity=0.20] (185.73, 65.19) circle (  2.13);

\path[fill=fillColor,fill opacity=0.20] (186.39, 69.03) circle (  2.13);

\path[fill=fillColor,fill opacity=0.20] (201.03, 75.16) circle (  2.13);

\path[fill=fillColor,fill opacity=0.20] (171.31, 69.45) circle (  2.13);

\path[fill=fillColor,fill opacity=0.20] (176.34, 55.22) circle (  2.13);

\path[fill=fillColor,fill opacity=0.20] (184.86, 55.33) circle (  2.13);

\path[fill=fillColor,fill opacity=0.20] (194.04, 48.79) circle (  2.13);

\path[fill=fillColor,fill opacity=0.20] (197.75, 53.77) circle (  2.13);

\path[fill=fillColor,fill opacity=0.20] (195.78, 70.59) circle (  2.13);

\path[fill=fillColor,fill opacity=0.20] (201.03, 76.09) circle (  2.13);

\path[fill=fillColor,fill opacity=0.20] (203.87, 73.50) circle (  2.13);

\path[fill=fillColor,fill opacity=0.20] (199.06, 75.78) circle (  2.13);

\path[fill=fillColor,fill opacity=0.20] (193.82, 78.27) circle (  2.13);

\path[fill=fillColor,fill opacity=0.20] (196.44, 87.93) circle (  2.13);

\path[fill=fillColor,fill opacity=0.20] (215.89,104.54) circle (  2.13);

\path[fill=fillColor,fill opacity=0.20] (209.33, 92.08) circle (  2.13);

\path[fill=fillColor,fill opacity=0.20] (202.12, 79.83) circle (  2.13);

\path[fill=fillColor,fill opacity=0.20] (192.51, 71.63) circle (  2.13);

\path[fill=fillColor,fill opacity=0.20] (185.95, 65.81) circle (  2.13);

\path[fill=fillColor,fill opacity=0.20] (185.30, 68.62) circle (  2.13);

\path[fill=fillColor,fill opacity=0.20] (179.18, 74.01) circle (  2.13);

\path[fill=fillColor,fill opacity=0.20] (179.40, 69.65) circle (  2.13);

\path[fill=fillColor,fill opacity=0.20] (180.27, 64.46) circle (  2.13);

\path[fill=fillColor,fill opacity=0.20] (185.08, 66.54) circle (  2.13);

\path[fill=fillColor,fill opacity=0.20] (183.11, 65.81) circle (  2.13);

\path[fill=fillColor,fill opacity=0.20] (190.32, 64.15) circle (  2.13);

\path[fill=fillColor,fill opacity=0.20] (182.24, 72.77) circle (  2.13);

\path[fill=fillColor,fill opacity=0.20] (172.19, 74.64) circle (  2.13);

\path[fill=fillColor,fill opacity=0.20] (174.59, 66.02) circle (  2.13);

\path[fill=fillColor,fill opacity=0.20] (175.46, 57.71) circle (  2.13);

\path[fill=fillColor,fill opacity=0.20] (192.07, 52.21) circle (  2.13);

\path[fill=fillColor,fill opacity=0.20] (182.67, 55.53) circle (  2.13);

\path[fill=fillColor,fill opacity=0.20] (187.70, 63.84) circle (  2.13);

\path[fill=fillColor,fill opacity=0.20] (196.22, 63.53) circle (  2.13);

\path[fill=fillColor,fill opacity=0.20] (200.37, 64.98) circle (  2.13);

\path[fill=fillColor,fill opacity=0.20] (199.28, 72.87) circle (  2.13);

\path[fill=fillColor,fill opacity=0.20] (194.91, 70.69) circle (  2.13);

\path[fill=fillColor,fill opacity=0.20] (193.60, 63.63) circle (  2.13);

\path[fill=fillColor,fill opacity=0.20] (195.35, 69.65) circle (  2.13);

\path[fill=fillColor,fill opacity=0.20] (192.94, 78.27) circle (  2.13);

\path[fill=fillColor,fill opacity=0.20] (190.54, 79.21) circle (  2.13);

\path[fill=fillColor,fill opacity=0.20] (196.88, 83.77) circle (  2.13);

\path[fill=fillColor,fill opacity=0.20] (204.09, 91.04) circle (  2.13);

\path[fill=fillColor,fill opacity=0.20] (190.54, 91.04) circle (  2.13);

\path[fill=fillColor,fill opacity=0.20] (199.06, 87.93) circle (  2.13);

\path[fill=fillColor,fill opacity=0.20] (198.41, 88.96) circle (  2.13);

\path[fill=fillColor,fill opacity=0.20] (199.50, 92.08) circle (  2.13);

\path[fill=fillColor,fill opacity=0.20] (198.41, 95.19) circle (  2.13);

\path[fill=fillColor,fill opacity=0.20] (195.78, 94.16) circle (  2.13);

\path[fill=fillColor,fill opacity=0.20] (206.93, 91.04) circle (  2.13);

\path[fill=fillColor,fill opacity=0.20] (209.33, 88.96) circle (  2.13);

\path[fill=fillColor,fill opacity=0.20] (215.67, 86.89) circle (  2.13);

\path[fill=fillColor,fill opacity=0.20] (212.61, 85.85) circle (  2.13);

\path[fill=fillColor,fill opacity=0.20] (209.55, 78.38) circle (  2.13);

\path[fill=fillColor,fill opacity=0.20] (209.55, 67.58) circle (  2.13);

\path[fill=fillColor,fill opacity=0.20] (201.90, 67.99) circle (  2.13);

\path[fill=fillColor,fill opacity=0.20] (199.50, 73.18) circle (  2.13);

\path[fill=fillColor,fill opacity=0.20] (185.08, 69.55) circle (  2.13);

\path[fill=fillColor,fill opacity=0.20] (182.02, 65.61) circle (  2.13);

\path[fill=fillColor,fill opacity=0.20] (175.03, 68.82) circle (  2.13);

\path[fill=fillColor,fill opacity=0.20] (173.28, 70.80) circle (  2.13);

\path[fill=fillColor,fill opacity=0.20] (182.89, 72.77) circle (  2.13);

\path[fill=fillColor,fill opacity=0.20] (168.91, 76.51) circle (  2.13);

\path[fill=fillColor,fill opacity=0.20] (179.83, 79.62) circle (  2.13);

\path[fill=fillColor,fill opacity=0.20] (188.57, 79.83) circle (  2.13);

\path[fill=fillColor,fill opacity=0.20] (190.10, 75.78) circle (  2.13);

\path[fill=fillColor,fill opacity=0.20] (181.58, 81.28) circle (  2.13);

\path[fill=fillColor,fill opacity=0.20] (177.87, 72.66) circle (  2.13);

\path[fill=fillColor,fill opacity=0.20] (169.13, 67.58) circle (  2.13);

\path[fill=fillColor,fill opacity=0.20] (175.25, 67.58) circle (  2.13);

\path[fill=fillColor,fill opacity=0.20] (185.52, 62.49) circle (  2.13);

\path[fill=fillColor,fill opacity=0.20] (188.79, 59.79) circle (  2.13);

\path[fill=fillColor,fill opacity=0.20] (189.01, 63.42) circle (  2.13);

\path[fill=fillColor,fill opacity=0.20] (188.14, 59.58) circle (  2.13);

\path[fill=fillColor,fill opacity=0.20] (190.54, 55.53) circle (  2.13);

\path[fill=fillColor,fill opacity=0.20] (199.94, 61.97) circle (  2.13);

\path[fill=fillColor,fill opacity=0.20] (196.44, 65.40) circle (  2.13);

\path[fill=fillColor,fill opacity=0.20] (196.66, 64.26) circle (  2.13);

\path[fill=fillColor,fill opacity=0.20] (198.41, 65.29) circle (  2.13);

\path[fill=fillColor,fill opacity=0.20] (197.97, 70.28) circle (  2.13);

\path[fill=fillColor,fill opacity=0.20] (192.07, 73.50) circle (  2.13);

\path[fill=fillColor,fill opacity=0.20] (196.00, 72.98) circle (  2.13);

\path[fill=fillColor,fill opacity=0.20] (201.47, 72.87) circle (  2.13);

\path[fill=fillColor,fill opacity=0.20] (202.34, 72.87) circle (  2.13);

\path[fill=fillColor,fill opacity=0.20] (208.24, 70.80) circle (  2.13);

\path[fill=fillColor,fill opacity=0.20] (202.12, 70.17) circle (  2.13);

\path[fill=fillColor,fill opacity=0.20] (202.34, 67.79) circle (  2.13);

\path[fill=fillColor,fill opacity=0.20] (199.94, 67.16) circle (  2.13);

\path[fill=fillColor,fill opacity=0.20] (206.27, 71.42) circle (  2.13);

\path[fill=fillColor,fill opacity=0.20] (206.27, 75.88) circle (  2.13);

\path[fill=fillColor,fill opacity=0.20] (206.93, 76.82) circle (  2.13);

\path[fill=fillColor,fill opacity=0.20] (205.18, 70.48) circle (  2.13);

\path[fill=fillColor,fill opacity=0.20] (201.68, 61.04) circle (  2.13);

\path[fill=fillColor,fill opacity=0.20] (192.29, 55.95) circle (  2.13);

\path[fill=fillColor,fill opacity=0.20] (182.02, 54.08) circle (  2.13);

\path[fill=fillColor,fill opacity=0.20] (175.68, 60.21) circle (  2.13);

\path[fill=fillColor,fill opacity=0.20] (168.04, 70.59) circle (  2.13);

\path[fill=fillColor,fill opacity=0.20] (175.46, 73.60) circle (  2.13);

\path[fill=fillColor,fill opacity=0.20] (178.74, 77.13) circle (  2.13);

\path[fill=fillColor,fill opacity=0.20] (172.84, 88.96) circle (  2.13);

\path[fill=fillColor,fill opacity=0.20] (179.40, 99.35) circle (  2.13);

\path[fill=fillColor,fill opacity=0.20] (182.46, 77.44) circle (  2.13);

\path[fill=fillColor,fill opacity=0.20] (180.05, 79.10) circle (  2.13);

\path[fill=fillColor,fill opacity=0.20] (179.83, 74.74) circle (  2.13);

\path[fill=fillColor,fill opacity=0.20] (172.19, 62.70) circle (  2.13);

\path[fill=fillColor,fill opacity=0.20] (178.09, 52.73) circle (  2.13);

\path[fill=fillColor,fill opacity=0.20] (175.46, 47.64) circle (  2.13);

\path[fill=fillColor,fill opacity=0.20] (182.02, 50.14) circle (  2.13);

\path[fill=fillColor,fill opacity=0.20] (189.01, 58.34) circle (  2.13);

\path[fill=fillColor,fill opacity=0.20] (192.94, 60.31) circle (  2.13);

\path[fill=fillColor,fill opacity=0.20] (194.91, 58.75) circle (  2.13);

\path[fill=fillColor,fill opacity=0.20] (199.94, 59.27) circle (  2.13);

\path[fill=fillColor,fill opacity=0.20] (197.31, 60.21) circle (  2.13);

\path[fill=fillColor,fill opacity=0.20] (200.15, 62.80) circle (  2.13);

\path[fill=fillColor,fill opacity=0.20] (204.74, 67.99) circle (  2.13);

\path[fill=fillColor,fill opacity=0.20] (208.02, 73.70) circle (  2.13);

\path[fill=fillColor,fill opacity=0.20] (203.43, 71.42) circle (  2.13);

\path[fill=fillColor,fill opacity=0.20] (203.65, 64.36) circle (  2.13);

\path[fill=fillColor,fill opacity=0.20] (196.66, 61.56) circle (  2.13);

\path[fill=fillColor,fill opacity=0.20] (188.14, 60.52) circle (  2.13);

\path[fill=fillColor,fill opacity=0.20] (190.54, 59.79) circle (  2.13);

\path[fill=fillColor,fill opacity=0.20] (189.45, 63.63) circle (  2.13);

\path[fill=fillColor,fill opacity=0.20] (183.55, 67.68) circle (  2.13);

\path[fill=fillColor,fill opacity=0.20] (185.30, 68.30) circle (  2.13);

\path[fill=fillColor,fill opacity=0.20] (183.99, 63.94) circle (  2.13);

\path[fill=fillColor,fill opacity=0.20] (178.96, 55.53) circle (  2.13);

\path[fill=fillColor,fill opacity=0.20] (174.59, 54.08) circle (  2.13);

\path[fill=fillColor,fill opacity=0.20] (173.50, 63.84) circle (  2.13);

\path[fill=fillColor,fill opacity=0.20] (183.55, 80.87) circle (  2.13);

\path[fill=fillColor,fill opacity=0.20] (182.46, 68.41) circle (  2.13);

\path[fill=fillColor,fill opacity=0.20] (182.24, 61.56) circle (  2.13);

\path[fill=fillColor,fill opacity=0.20] (174.15, 63.01) circle (  2.13);

\path[fill=fillColor,fill opacity=0.20] (167.82, 64.67) circle (  2.13);

\path[fill=fillColor,fill opacity=0.20] (180.27, 60.83) circle (  2.13);

\path[fill=fillColor,fill opacity=0.20] (179.40, 55.02) circle (  2.13);

\path[fill=fillColor,fill opacity=0.20] (184.20, 56.05) circle (  2.13);

\path[fill=fillColor,fill opacity=0.20] (188.14, 59.89) circle (  2.13);

\path[fill=fillColor,fill opacity=0.20] (193.38, 56.78) circle (  2.13);

\path[fill=fillColor,fill opacity=0.20] (195.35, 50.14) circle (  2.13);

\path[fill=fillColor,fill opacity=0.20] (199.50, 53.87) circle (  2.13);

\path[fill=fillColor,fill opacity=0.20] (196.88, 64.15) circle (  2.13);

\path[fill=fillColor,fill opacity=0.20] (188.57, 62.08) circle (  2.13);

\path[fill=fillColor,fill opacity=0.20] (183.55, 56.16) circle (  2.13);

\path[fill=fillColor,fill opacity=0.20] (178.09, 55.02) circle (  2.13);

\path[fill=fillColor,fill opacity=0.20] (174.15, 56.05) circle (  2.13);

\path[fill=fillColor,fill opacity=0.20] (176.78, 53.04) circle (  2.13);

\path[fill=fillColor,fill opacity=0.20] (171.09, 50.76) circle (  2.13);

\path[fill=fillColor,fill opacity=0.20] (170.66, 49.93) circle (  2.13);

\path[fill=fillColor,fill opacity=0.20] (181.36, 55.12) circle (  2.13);

\path[fill=fillColor,fill opacity=0.20] (185.73, 67.47) circle (  2.13);

\path[fill=fillColor,fill opacity=0.20] (185.08, 70.69) circle (  2.13);

\path[fill=fillColor,fill opacity=0.20] (186.61, 71.63) circle (  2.13);

\path[fill=fillColor,fill opacity=0.20] (179.40, 84.81) circle (  2.13);

\path[fill=fillColor,fill opacity=0.20] (181.80, 85.85) circle (  2.13);

\path[fill=fillColor,fill opacity=0.20] (176.56, 73.50) circle (  2.13);

\path[fill=fillColor,fill opacity=0.20] (177.21, 61.24) circle (  2.13);

\path[fill=fillColor,fill opacity=0.20] (171.97, 63.53) circle (  2.13);

\path[fill=fillColor,fill opacity=0.20] (180.49, 66.95) circle (  2.13);

\path[fill=fillColor,fill opacity=0.20] (193.38, 62.28) circle (  2.13);

\path[fill=fillColor,fill opacity=0.20] (181.36, 54.50) circle (  2.13);

\path[fill=fillColor,fill opacity=0.20] (178.96, 50.14) circle (  2.13);

\path[fill=fillColor,fill opacity=0.20] (180.05, 52.84) circle (  2.13);

\path[fill=fillColor,fill opacity=0.20] (170.66, 55.22) circle (  2.13);

\path[fill=fillColor,fill opacity=0.20] (177.43, 57.71) circle (  2.13);

\path[fill=fillColor,fill opacity=0.20] (169.56, 64.57) circle (  2.13);

\path[fill=fillColor,fill opacity=0.20] (181.36, 71.83) circle (  2.13);

\path[fill=fillColor,fill opacity=0.20] (181.80, 69.03) circle (  2.13);

\path[fill=fillColor,fill opacity=0.20] (182.24, 60.62) circle (  2.13);

\path[fill=fillColor,fill opacity=0.20] (190.76, 60.31) circle (  2.13);

\path[fill=fillColor,fill opacity=0.20] (210.64, 66.85) circle (  2.13);

\path[fill=fillColor,fill opacity=0.20] (192.07, 74.64) circle (  2.13);

\path[fill=fillColor,fill opacity=0.20] (186.83, 73.08) circle (  2.13);

\path[fill=fillColor,fill opacity=0.20] (185.30, 68.72) circle (  2.13);

\path[fill=fillColor,fill opacity=0.20] (187.26, 67.27) circle (  2.13);

\path[fill=fillColor,fill opacity=0.20] (189.67, 61.66) circle (  2.13);

\path[fill=fillColor,fill opacity=0.20] (176.56, 57.40) circle (  2.13);

\path[fill=fillColor,fill opacity=0.20] (178.74, 63.63) circle (  2.13);

\path[fill=fillColor,fill opacity=0.20] (182.89, 72.56) circle (  2.13);
\end{scope}
\begin{scope}
\path[clip] (  0.00,  0.00) rectangle (289.08,144.54);
\definecolor[named]{drawColor}{rgb}{0.50,0.50,0.50}

\node[text=drawColor,anchor=base east,inner sep=0pt, outer sep=0pt, scale=  0.96] at ( 32.58, 36.86) {0.6};

\node[text=drawColor,anchor=base east,inner sep=0pt, outer sep=0pt, scale=  0.96] at ( 32.58, 57.63) {0.8};

\node[text=drawColor,anchor=base east,inner sep=0pt, outer sep=0pt, scale=  0.96] at ( 32.58, 78.39) {1.0};

\node[text=drawColor,anchor=base east,inner sep=0pt, outer sep=0pt, scale=  0.96] at ( 32.58, 99.15) {1.2};
\end{scope}
\begin{scope}
\path[clip] (  0.00,  0.00) rectangle (289.08,144.54);
\definecolor[named]{drawColor}{rgb}{0.50,0.50,0.50}

\path[draw=drawColor,line width= 0.6pt,line join=round] ( 35.42, 40.17) --
	( 39.69, 40.17);

\path[draw=drawColor,line width= 0.6pt,line join=round] ( 35.42, 60.93) --
	( 39.69, 60.93);

\path[draw=drawColor,line width= 0.6pt,line join=round] ( 35.42, 81.70) --
	( 39.69, 81.70);

\path[draw=drawColor,line width= 0.6pt,line join=round] ( 35.42,102.46) --
	( 39.69,102.46);
\end{scope}
\begin{scope}
\path[clip] (  0.00,  0.00) rectangle (289.08,144.54);
\definecolor[named]{drawColor}{rgb}{0.50,0.50,0.50}

\path[draw=drawColor,line width= 0.6pt,line join=round] ( 60.53, 29.77) --
	( 60.53, 34.04);

\path[draw=drawColor,line width= 0.6pt,line join=round] ( 82.38, 29.77) --
	( 82.38, 34.04);

\path[draw=drawColor,line width= 0.6pt,line join=round] (104.23, 29.77) --
	(104.23, 34.04);

\path[draw=drawColor,line width= 0.6pt,line join=round] (126.08, 29.77) --
	(126.08, 34.04);

\path[draw=drawColor,line width= 0.6pt,line join=round] (147.93, 29.77) --
	(147.93, 34.04);
\end{scope}
\begin{scope}
\path[clip] (  0.00,  0.00) rectangle (289.08,144.54);
\definecolor[named]{drawColor}{rgb}{0.50,0.50,0.50}

\node[text=drawColor,anchor=base,inner sep=0pt, outer sep=0pt, scale=  0.96] at ( 60.53, 20.31) {0.02};

\node[text=drawColor,anchor=base,inner sep=0pt, outer sep=0pt, scale=  0.96] at ( 82.38, 20.31) {0.03};

\node[text=drawColor,anchor=base,inner sep=0pt, outer sep=0pt, scale=  0.96] at (104.23, 20.31) {0.04};

\node[text=drawColor,anchor=base,inner sep=0pt, outer sep=0pt, scale=  0.96] at (126.08, 20.31) {0.05};

\node[text=drawColor,anchor=base,inner sep=0pt, outer sep=0pt, scale=  0.96] at (147.93, 20.31) {0.06};
\end{scope}
\begin{scope}
\path[clip] (  0.00,  0.00) rectangle (289.08,144.54);
\definecolor[named]{drawColor}{rgb}{0.50,0.50,0.50}

\path[draw=drawColor,line width= 0.6pt,line join=round] (180.71, 29.77) --
	(180.71, 34.04);

\path[draw=drawColor,line width= 0.6pt,line join=round] (202.56, 29.77) --
	(202.56, 34.04);

\path[draw=drawColor,line width= 0.6pt,line join=round] (224.41, 29.77) --
	(224.41, 34.04);

\path[draw=drawColor,line width= 0.6pt,line join=round] (246.26, 29.77) --
	(246.26, 34.04);

\path[draw=drawColor,line width= 0.6pt,line join=round] (268.11, 29.77) --
	(268.11, 34.04);
\end{scope}
\begin{scope}
\path[clip] (  0.00,  0.00) rectangle (289.08,144.54);
\definecolor[named]{drawColor}{rgb}{0.50,0.50,0.50}

\node[text=drawColor,anchor=base,inner sep=0pt, outer sep=0pt, scale=  0.96] at (180.71, 20.31) {0.02};

\node[text=drawColor,anchor=base,inner sep=0pt, outer sep=0pt, scale=  0.96] at (202.56, 20.31) {0.03};

\node[text=drawColor,anchor=base,inner sep=0pt, outer sep=0pt, scale=  0.96] at (224.41, 20.31) {0.04};

\node[text=drawColor,anchor=base,inner sep=0pt, outer sep=0pt, scale=  0.96] at (246.26, 20.31) {0.05};

\node[text=drawColor,anchor=base,inner sep=0pt, outer sep=0pt, scale=  0.96] at (268.11, 20.31) {0.06};
\end{scope}
\begin{scope}
\path[clip] (  0.00,  0.00) rectangle (289.08,144.54);
\definecolor[named]{drawColor}{rgb}{0.00,0.00,0.00}

\node[text=drawColor,anchor=base,inner sep=0pt, outer sep=0pt, scale=  1.20] at (158.36,  9.03) {$\rho$ $[\mu m^{-2}]$};
\end{scope}
\begin{scope}
\path[clip] (  0.00,  0.00) rectangle (289.08,144.54);
\definecolor[named]{drawColor}{rgb}{0.00,0.00,0.00}

\node[text=drawColor,rotate= 90.00,anchor=base,inner sep=0pt, outer sep=0pt, scale=  1.20] at ( 17.30, 76.95) {MD $[\times 10^{-9}mm^2/s]$};
\end{scope}
\end{tikzpicture}

						\end{adjustbox}
						\end{minipage}
					}\\
					\subfloat[AD]{
						\begin{minipage}{0.5\textwidth}						
						\begin{adjustbox}{width={\textwidth},totalheight=\textheight,keepaspectratio}
							\strut
							% Created by tikzDevice version 0.6.2-92-0ad2792 on 2012-09-27 18:25:25
% !TEX encoding = UTF-8 Unicode
\begin{tikzpicture}[x=1pt,y=1pt]
\definecolor[named]{fillColor}{rgb}{1.00,1.00,1.00}
\path[use as bounding box,fill=fillColor,fill opacity=0.00] (0,0) rectangle (289.08,144.54);
\begin{scope}
\path[clip] (  0.00,  0.00) rectangle (289.08,144.54);
\definecolor[named]{drawColor}{rgb}{1.00,1.00,1.00}
\definecolor[named]{fillColor}{rgb}{1.00,1.00,1.00}

\path[draw=drawColor,line width= 0.6pt,line join=round,line cap=round,fill=fillColor] ( -0.00,  0.00) rectangle (289.08,144.54);
\end{scope}
\begin{scope}
\path[clip] ( 44.49,119.86) rectangle (159.26,132.50);
\definecolor[named]{fillColor}{rgb}{0.80,0.80,0.80}

\path[fill=fillColor] ( 44.49,119.86) rectangle (159.26,132.50);
\definecolor[named]{drawColor}{rgb}{0.00,0.00,0.00}

\node[text=drawColor,anchor=base,inner sep=0pt, outer sep=0pt, scale=  0.96] at (101.87,122.87) {Scan (r=-0.600)};
\end{scope}
\begin{scope}
\path[clip] (162.27,119.86) rectangle (277.03,132.50);
\definecolor[named]{fillColor}{rgb}{0.80,0.80,0.80}

\path[fill=fillColor] (162.27,119.86) rectangle (277.03,132.50);
\definecolor[named]{drawColor}{rgb}{0.00,0.00,0.00}

\node[text=drawColor,anchor=base,inner sep=0pt, outer sep=0pt, scale=  0.96] at (219.65,122.87) {Rescan (r=-0.479)};
\end{scope}
\begin{scope}
\path[clip] ( 44.49, 34.04) rectangle (159.26,119.86);
\definecolor[named]{fillColor}{rgb}{0.90,0.90,0.90}

\path[fill=fillColor] ( 44.49, 34.04) rectangle (159.26,119.86);
\definecolor[named]{drawColor}{rgb}{0.95,0.95,0.95}

\path[draw=drawColor,line width= 0.3pt,line join=round] ( 44.49, 42.41) --
	(159.26, 42.41);

\path[draw=drawColor,line width= 0.3pt,line join=round] ( 44.49, 62.73) --
	(159.26, 62.73);

\path[draw=drawColor,line width= 0.3pt,line join=round] ( 44.49, 83.04) --
	(159.26, 83.04);

\path[draw=drawColor,line width= 0.3pt,line join=round] ( 44.49,103.36) --
	(159.26,103.36);

\path[draw=drawColor,line width= 0.3pt,line join=round] ( 60.20, 34.04) --
	( 60.20,119.86);

\path[draw=drawColor,line width= 0.3pt,line join=round] ( 84.76, 34.04) --
	( 84.76,119.86);

\path[draw=drawColor,line width= 0.3pt,line join=round] (109.33, 34.04) --
	(109.33,119.86);

\path[draw=drawColor,line width= 0.3pt,line join=round] (133.90, 34.04) --
	(133.90,119.86);

\path[draw=drawColor,line width= 0.3pt,line join=round] (158.46, 34.04) --
	(158.46,119.86);
\definecolor[named]{drawColor}{rgb}{1.00,1.00,1.00}

\path[draw=drawColor,line width= 0.6pt,line join=round] ( 44.49, 52.57) --
	(159.26, 52.57);

\path[draw=drawColor,line width= 0.6pt,line join=round] ( 44.49, 72.89) --
	(159.26, 72.89);

\path[draw=drawColor,line width= 0.6pt,line join=round] ( 44.49, 93.20) --
	(159.26, 93.20);

\path[draw=drawColor,line width= 0.6pt,line join=round] ( 44.49,113.52) --
	(159.26,113.52);

\path[draw=drawColor,line width= 0.6pt,line join=round] ( 47.91, 34.04) --
	( 47.91,119.86);

\path[draw=drawColor,line width= 0.6pt,line join=round] ( 72.48, 34.04) --
	( 72.48,119.86);

\path[draw=drawColor,line width= 0.6pt,line join=round] ( 97.05, 34.04) --
	( 97.05,119.86);

\path[draw=drawColor,line width= 0.6pt,line join=round] (121.61, 34.04) --
	(121.61,119.86);

\path[draw=drawColor,line width= 0.6pt,line join=round] (146.18, 34.04) --
	(146.18,119.86);
\definecolor[named]{fillColor}{rgb}{0.00,0.00,0.00}

\path[fill=fillColor,fill opacity=0.20] (144.21, 55.82) circle (  2.13);

\path[fill=fillColor,fill opacity=0.20] (102.94, 61.51) circle (  2.13);

\path[fill=fillColor,fill opacity=0.20] ( 98.03, 59.07) circle (  2.13);

\path[fill=fillColor,fill opacity=0.20] ( 93.11, 55.82) circle (  2.13);

\path[fill=fillColor,fill opacity=0.20] ( 95.08, 55.01) circle (  2.13);

\path[fill=fillColor,fill opacity=0.20] (105.89, 52.57) circle (  2.13);

\path[fill=fillColor,fill opacity=0.20] (112.77, 60.69) circle (  2.13);

\path[fill=fillColor,fill opacity=0.20] (110.80, 68.01) circle (  2.13);

\path[fill=fillColor,fill opacity=0.20] ( 90.17, 69.63) circle (  2.13);

\path[fill=fillColor,fill opacity=0.20] ( 92.13, 59.07) circle (  2.13);

\path[fill=fillColor,fill opacity=0.20] ( 87.22, 62.32) circle (  2.13);

\path[fill=fillColor,fill opacity=0.20] ( 81.32, 73.70) circle (  2.13);

\path[fill=fillColor,fill opacity=0.20] ( 87.22, 72.89) circle (  2.13);

\path[fill=fillColor,fill opacity=0.20] ( 78.37, 61.51) circle (  2.13);

\path[fill=fillColor,fill opacity=0.20] ( 86.24, 50.94) circle (  2.13);

\path[fill=fillColor,fill opacity=0.20] ( 91.15, 53.38) circle (  2.13);

\path[fill=fillColor,fill opacity=0.20] (108.84, 47.69) circle (  2.13);

\path[fill=fillColor,fill opacity=0.20] (130.46, 38.75) circle (  2.13);

\path[fill=fillColor,fill opacity=0.20] ( 85.25, 76.14) circle (  2.13);

\path[fill=fillColor,fill opacity=0.20] ( 86.24, 65.57) circle (  2.13);

\path[fill=fillColor,fill opacity=0.20] ( 87.22, 67.20) circle (  2.13);

\path[fill=fillColor,fill opacity=0.20] ( 86.24, 78.57) circle (  2.13);

\path[fill=fillColor,fill opacity=0.20] ( 87.22, 85.89) circle (  2.13);

\path[fill=fillColor,fill opacity=0.20] ( 84.27, 89.95) circle (  2.13);

\path[fill=fillColor,fill opacity=0.20] ( 85.25, 91.58) circle (  2.13);

\path[fill=fillColor,fill opacity=0.20] ( 87.22, 84.26) circle (  2.13);

\path[fill=fillColor,fill opacity=0.20] ( 88.20, 68.01) circle (  2.13);

\path[fill=fillColor,fill opacity=0.20] ( 90.17, 50.13) circle (  2.13);

\path[fill=fillColor,fill opacity=0.20] (103.92, 45.25) circle (  2.13);

\path[fill=fillColor,fill opacity=0.20] (101.96, 42.82) circle (  2.13);

\path[fill=fillColor,fill opacity=0.20] (108.84, 38.75) circle (  2.13);

\path[fill=fillColor,fill opacity=0.20] (102.94, 61.51) circle (  2.13);

\path[fill=fillColor,fill opacity=0.20] ( 85.25, 66.38) circle (  2.13);

\path[fill=fillColor,fill opacity=0.20] ( 85.25, 72.89) circle (  2.13);

\path[fill=fillColor,fill opacity=0.20] ( 77.39, 81.01) circle (  2.13);

\path[fill=fillColor,fill opacity=0.20] ( 79.36,102.96) circle (  2.13);

\path[fill=fillColor,fill opacity=0.20] ( 84.27,107.02) circle (  2.13);

\path[fill=fillColor,fill opacity=0.20] ( 90.17, 98.08) circle (  2.13);

\path[fill=fillColor,fill opacity=0.20] ( 87.22, 96.45) circle (  2.13);

\path[fill=fillColor,fill opacity=0.20] ( 85.25, 90.76) circle (  2.13);

\path[fill=fillColor,fill opacity=0.20] ( 90.17, 77.76) circle (  2.13);

\path[fill=fillColor,fill opacity=0.20] ( 97.05, 66.38) circle (  2.13);

\path[fill=fillColor,fill opacity=0.20] (100.98, 60.69) circle (  2.13);

\path[fill=fillColor,fill opacity=0.20] ( 98.03, 51.75) circle (  2.13);

\path[fill=fillColor,fill opacity=0.20] (102.94, 45.25) circle (  2.13);

\path[fill=fillColor,fill opacity=0.20] ( 90.17, 41.19) circle (  2.13);

\path[fill=fillColor,fill opacity=0.20] (104.91, 63.13) circle (  2.13);

\path[fill=fillColor,fill opacity=0.20] ( 98.03, 68.01) circle (  2.13);

\path[fill=fillColor,fill opacity=0.20] ( 85.25, 86.70) circle (  2.13);

\path[fill=fillColor,fill opacity=0.20] ( 80.34, 94.02) circle (  2.13);

\path[fill=fillColor,fill opacity=0.20] ( 81.32,101.33) circle (  2.13);

\path[fill=fillColor,fill opacity=0.20] ( 81.32,106.21) circle (  2.13);

\path[fill=fillColor,fill opacity=0.20] ( 85.25,107.02) circle (  2.13);

\path[fill=fillColor,fill opacity=0.20] ( 89.18, 99.70) circle (  2.13);

\path[fill=fillColor,fill opacity=0.20] ( 89.18, 85.89) circle (  2.13);

\path[fill=fillColor,fill opacity=0.20] ( 96.06, 76.95) circle (  2.13);

\path[fill=fillColor,fill opacity=0.20] (103.92, 77.76) circle (  2.13);

\path[fill=fillColor,fill opacity=0.20] ( 99.01, 76.95) circle (  2.13);

\path[fill=fillColor,fill opacity=0.20] (101.96, 76.95) circle (  2.13);

\path[fill=fillColor,fill opacity=0.20] (105.89, 85.89) circle (  2.13);

\path[fill=fillColor,fill opacity=0.20] ( 99.99, 85.89) circle (  2.13);

\path[fill=fillColor,fill opacity=0.20] ( 89.18, 65.57) circle (  2.13);

\path[fill=fillColor,fill opacity=0.20] ( 88.20, 93.20) circle (  2.13);

\path[fill=fillColor,fill opacity=0.20] ( 96.06, 94.02) circle (  2.13);

\path[fill=fillColor,fill opacity=0.20] ( 96.06, 85.08) circle (  2.13);

\path[fill=fillColor,fill opacity=0.20] (105.89, 72.89) circle (  2.13);

\path[fill=fillColor,fill opacity=0.20] ( 99.99, 71.26) circle (  2.13);

\path[fill=fillColor,fill opacity=0.20] ( 84.27, 92.39) circle (  2.13);

\path[fill=fillColor,fill opacity=0.20] ( 82.31, 98.89) circle (  2.13);

\path[fill=fillColor,fill opacity=0.20] ( 86.24, 90.76) circle (  2.13);

\path[fill=fillColor,fill opacity=0.20] ( 83.29, 90.76) circle (  2.13);

\path[fill=fillColor,fill opacity=0.20] ( 84.27,103.77) circle (  2.13);

\path[fill=fillColor,fill opacity=0.20] ( 82.31,104.58) circle (  2.13);

\path[fill=fillColor,fill opacity=0.20] ( 97.05, 86.70) circle (  2.13);

\path[fill=fillColor,fill opacity=0.20] (106.87, 76.14) circle (  2.13);

\path[fill=fillColor,fill opacity=0.20] (117.68, 76.95) circle (  2.13);

\path[fill=fillColor,fill opacity=0.20] (102.94, 94.83) circle (  2.13);

\path[fill=fillColor,fill opacity=0.20] ( 81.32, 72.89) circle (  2.13);

\path[fill=fillColor,fill opacity=0.20] ( 83.29, 86.70) circle (  2.13);

\path[fill=fillColor,fill opacity=0.20] ( 86.24, 86.70) circle (  2.13);

\path[fill=fillColor,fill opacity=0.20] ( 84.27, 89.14) circle (  2.13);

\path[fill=fillColor,fill opacity=0.20] ( 86.24, 89.14) circle (  2.13);

\path[fill=fillColor,fill opacity=0.20] ( 90.17, 81.82) circle (  2.13);

\path[fill=fillColor,fill opacity=0.20] (100.98, 89.14) circle (  2.13);

\path[fill=fillColor,fill opacity=0.20] ( 99.01, 69.63) circle (  2.13);

\path[fill=fillColor,fill opacity=0.20] (103.92, 72.89) circle (  2.13);

\path[fill=fillColor,fill opacity=0.20] ( 95.08, 87.51) circle (  2.13);

\path[fill=fillColor,fill opacity=0.20] ( 92.13,100.52) circle (  2.13);

\path[fill=fillColor,fill opacity=0.20] ( 91.15,101.33) circle (  2.13);

\path[fill=fillColor,fill opacity=0.20] ( 92.13, 89.14) circle (  2.13);

\path[fill=fillColor,fill opacity=0.20] ( 90.17, 90.76) circle (  2.13);

\path[fill=fillColor,fill opacity=0.20] ( 86.24,102.96) circle (  2.13);

\path[fill=fillColor,fill opacity=0.20] (109.82, 78.57) circle (  2.13);

\path[fill=fillColor,fill opacity=0.20] ( 91.15, 91.58) circle (  2.13);

\path[fill=fillColor,fill opacity=0.20] ( 76.41, 73.70) circle (  2.13);

\path[fill=fillColor,fill opacity=0.20] ( 73.46, 92.39) circle (  2.13);

\path[fill=fillColor,fill opacity=0.20] ( 74.44, 89.95) circle (  2.13);

\path[fill=fillColor,fill opacity=0.20] ( 71.59, 81.01) circle (  2.13);

\path[fill=fillColor,fill opacity=0.20] ( 64.42, 79.39) circle (  2.13);

\path[fill=fillColor,fill opacity=0.20] ( 78.37, 81.82) circle (  2.13);

\path[fill=fillColor,fill opacity=0.20] ( 83.29, 75.32) circle (  2.13);

\path[fill=fillColor,fill opacity=0.20] ( 86.24, 76.95) circle (  2.13);

\path[fill=fillColor,fill opacity=0.20] (112.77,104.58) circle (  2.13);

\path[fill=fillColor,fill opacity=0.20] (109.82, 62.32) circle (  2.13);

\path[fill=fillColor,fill opacity=0.20] (103.92, 68.82) circle (  2.13);

\path[fill=fillColor,fill opacity=0.20] ( 93.11, 91.58) circle (  2.13);

\path[fill=fillColor,fill opacity=0.20] ( 97.05, 99.70) circle (  2.13);

\path[fill=fillColor,fill opacity=0.20] ( 96.06,105.39) circle (  2.13);

\path[fill=fillColor,fill opacity=0.20] ( 94.10,100.52) circle (  2.13);

\path[fill=fillColor,fill opacity=0.20] ( 99.99, 91.58) circle (  2.13);

\path[fill=fillColor,fill opacity=0.20] ( 96.06, 94.83) circle (  2.13);

\path[fill=fillColor,fill opacity=0.20] ( 97.05, 91.58) circle (  2.13);

\path[fill=fillColor,fill opacity=0.20] (107.86, 78.57) circle (  2.13);

\path[fill=fillColor,fill opacity=0.20] ( 79.36, 75.32) circle (  2.13);

\path[fill=fillColor,fill opacity=0.20] ( 73.46, 91.58) circle (  2.13);

\path[fill=fillColor,fill opacity=0.20] ( 70.32, 94.83) circle (  2.13);

\path[fill=fillColor,fill opacity=0.20] ( 68.84, 80.20) circle (  2.13);

\path[fill=fillColor,fill opacity=0.20] ( 74.44, 80.20) circle (  2.13);

\path[fill=fillColor,fill opacity=0.20] ( 72.48, 86.70) circle (  2.13);

\path[fill=fillColor,fill opacity=0.20] ( 85.25, 83.45) circle (  2.13);

\path[fill=fillColor,fill opacity=0.20] ( 88.20, 76.14) circle (  2.13);

\path[fill=fillColor,fill opacity=0.20] ( 77.39, 65.57) circle (  2.13);

\path[fill=fillColor,fill opacity=0.20] (103.92, 75.32) circle (  2.13);

\path[fill=fillColor,fill opacity=0.20] (118.66, 59.88) circle (  2.13);

\path[fill=fillColor,fill opacity=0.20] (100.98, 67.20) circle (  2.13);

\path[fill=fillColor,fill opacity=0.20] ( 97.05, 95.64) circle (  2.13);

\path[fill=fillColor,fill opacity=0.20] ( 93.11, 91.58) circle (  2.13);

\path[fill=fillColor,fill opacity=0.20] ( 94.10, 87.51) circle (  2.13);

\path[fill=fillColor,fill opacity=0.20] ( 96.06, 99.70) circle (  2.13);

\path[fill=fillColor,fill opacity=0.20] ( 91.15, 99.70) circle (  2.13);

\path[fill=fillColor,fill opacity=0.20] ( 97.05, 88.33) circle (  2.13);

\path[fill=fillColor,fill opacity=0.20] ( 94.10, 75.32) circle (  2.13);

\path[fill=fillColor,fill opacity=0.20] (105.89, 72.89) circle (  2.13);

\path[fill=fillColor,fill opacity=0.20] (136.35,102.14) circle (  2.13);

\path[fill=fillColor,fill opacity=0.20] ( 81.32, 98.89) circle (  2.13);

\path[fill=fillColor,fill opacity=0.20] ( 73.46, 70.45) circle (  2.13);

\path[fill=fillColor,fill opacity=0.20] ( 73.46, 90.76) circle (  2.13);

\path[fill=fillColor,fill opacity=0.20] ( 76.41, 92.39) circle (  2.13);

\path[fill=fillColor,fill opacity=0.20] ( 85.25, 84.26) circle (  2.13);

\path[fill=fillColor,fill opacity=0.20] ( 82.31, 91.58) circle (  2.13);

\path[fill=fillColor,fill opacity=0.20] ( 81.32, 92.39) circle (  2.13);

\path[fill=fillColor,fill opacity=0.20] ( 90.17, 85.08) circle (  2.13);

\path[fill=fillColor,fill opacity=0.20] ( 96.06, 82.64) circle (  2.13);

\path[fill=fillColor,fill opacity=0.20] ( 94.10, 65.57) circle (  2.13);

\path[fill=fillColor,fill opacity=0.20] ( 98.03, 49.32) circle (  2.13);

\path[fill=fillColor,fill opacity=0.20] (112.77, 63.95) circle (  2.13);

\path[fill=fillColor,fill opacity=0.20] (104.91, 66.38) circle (  2.13);

\path[fill=fillColor,fill opacity=0.20] ( 96.06, 85.08) circle (  2.13);

\path[fill=fillColor,fill opacity=0.20] ( 88.20, 82.64) circle (  2.13);

\path[fill=fillColor,fill opacity=0.20] ( 90.17, 75.32) circle (  2.13);

\path[fill=fillColor,fill opacity=0.20] ( 95.08, 88.33) circle (  2.13);

\path[fill=fillColor,fill opacity=0.20] ( 97.05, 95.64) circle (  2.13);

\path[fill=fillColor,fill opacity=0.20] ( 99.01, 87.51) circle (  2.13);

\path[fill=fillColor,fill opacity=0.20] ( 99.99, 73.70) circle (  2.13);

\path[fill=fillColor,fill opacity=0.20] (100.98, 64.76) circle (  2.13);

\path[fill=fillColor,fill opacity=0.20] ( 70.91, 84.26) circle (  2.13);

\path[fill=fillColor,fill opacity=0.20] ( 76.41, 82.64) circle (  2.13);

\path[fill=fillColor,fill opacity=0.20] ( 80.34, 94.83) circle (  2.13);

\path[fill=fillColor,fill opacity=0.20] ( 72.18, 89.95) circle (  2.13);

\path[fill=fillColor,fill opacity=0.20] ( 83.29, 90.76) circle (  2.13);

\path[fill=fillColor,fill opacity=0.20] ( 85.25, 94.83) circle (  2.13);

\path[fill=fillColor,fill opacity=0.20] ( 82.31, 91.58) circle (  2.13);

\path[fill=fillColor,fill opacity=0.20] ( 86.24, 88.33) circle (  2.13);

\path[fill=fillColor,fill opacity=0.20] ( 92.13, 85.08) circle (  2.13);

\path[fill=fillColor,fill opacity=0.20] ( 95.08, 66.38) circle (  2.13);

\path[fill=fillColor,fill opacity=0.20] (103.92, 54.19) circle (  2.13);

\path[fill=fillColor,fill opacity=0.20] (106.87, 59.88) circle (  2.13);

\path[fill=fillColor,fill opacity=0.20] ( 89.18, 68.01) circle (  2.13);

\path[fill=fillColor,fill opacity=0.20] ( 83.29, 79.39) circle (  2.13);

\path[fill=fillColor,fill opacity=0.20] ( 85.25, 84.26) circle (  2.13);

\path[fill=fillColor,fill opacity=0.20] ( 95.08, 84.26) circle (  2.13);

\path[fill=fillColor,fill opacity=0.20] ( 98.03, 81.82) circle (  2.13);

\path[fill=fillColor,fill opacity=0.20] ( 98.03, 84.26) circle (  2.13);

\path[fill=fillColor,fill opacity=0.20] (101.96, 87.51) circle (  2.13);

\path[fill=fillColor,fill opacity=0.20] ( 99.01, 72.07) circle (  2.13);

\path[fill=fillColor,fill opacity=0.20] ( 73.46, 80.20) circle (  2.13);

\path[fill=fillColor,fill opacity=0.20] ( 76.41,102.14) circle (  2.13);

\path[fill=fillColor,fill opacity=0.20] ( 78.37,102.14) circle (  2.13);

\path[fill=fillColor,fill opacity=0.20] ( 73.46, 89.14) circle (  2.13);

\path[fill=fillColor,fill opacity=0.20] ( 81.32, 89.14) circle (  2.13);

\path[fill=fillColor,fill opacity=0.20] ( 87.22, 92.39) circle (  2.13);

\path[fill=fillColor,fill opacity=0.20] ( 80.34, 90.76) circle (  2.13);

\path[fill=fillColor,fill opacity=0.20] ( 95.08, 91.58) circle (  2.13);

\path[fill=fillColor,fill opacity=0.20] ( 92.13, 81.01) circle (  2.13);

\path[fill=fillColor,fill opacity=0.20] ( 97.05, 59.07) circle (  2.13);

\path[fill=fillColor,fill opacity=0.20] (109.82, 59.88) circle (  2.13);

\path[fill=fillColor,fill opacity=0.20] ( 99.01, 51.75) circle (  2.13);

\path[fill=fillColor,fill opacity=0.20] ( 90.17, 73.70) circle (  2.13);

\path[fill=fillColor,fill opacity=0.20] ( 88.20, 88.33) circle (  2.13);

\path[fill=fillColor,fill opacity=0.20] ( 92.13, 85.89) circle (  2.13);

\path[fill=fillColor,fill opacity=0.20] ( 95.08, 76.14) circle (  2.13);

\path[fill=fillColor,fill opacity=0.20] ( 93.11, 76.95) circle (  2.13);

\path[fill=fillColor,fill opacity=0.20] ( 97.05, 93.20) circle (  2.13);

\path[fill=fillColor,fill opacity=0.20] ( 86.24, 86.70) circle (  2.13);

\path[fill=fillColor,fill opacity=0.20] ( 97.05, 70.45) circle (  2.13);

\path[fill=fillColor,fill opacity=0.20] ( 84.27, 89.95) circle (  2.13);

\path[fill=fillColor,fill opacity=0.20] ( 83.29, 91.58) circle (  2.13);

\path[fill=fillColor,fill opacity=0.20] ( 84.27,105.39) circle (  2.13);

\path[fill=fillColor,fill opacity=0.20] ( 85.25, 98.89) circle (  2.13);

\path[fill=fillColor,fill opacity=0.20] ( 81.32, 89.95) circle (  2.13);

\path[fill=fillColor,fill opacity=0.20] ( 83.29, 91.58) circle (  2.13);

\path[fill=fillColor,fill opacity=0.20] ( 84.27, 89.14) circle (  2.13);

\path[fill=fillColor,fill opacity=0.20] ( 83.29, 83.45) circle (  2.13);

\path[fill=fillColor,fill opacity=0.20] ( 86.24, 81.82) circle (  2.13);

\path[fill=fillColor,fill opacity=0.20] ( 97.05, 71.26) circle (  2.13);

\path[fill=fillColor,fill opacity=0.20] ( 98.03, 53.38) circle (  2.13);

\path[fill=fillColor,fill opacity=0.20] (113.75, 61.51) circle (  2.13);

\path[fill=fillColor,fill opacity=0.20] (103.92, 49.32) circle (  2.13);

\path[fill=fillColor,fill opacity=0.20] ( 99.01, 66.38) circle (  2.13);

\path[fill=fillColor,fill opacity=0.20] ( 99.01, 78.57) circle (  2.13);

\path[fill=fillColor,fill opacity=0.20] ( 91.15, 79.39) circle (  2.13);

\path[fill=fillColor,fill opacity=0.20] ( 95.08, 80.20) circle (  2.13);

\path[fill=fillColor,fill opacity=0.20] ( 94.10, 79.39) circle (  2.13);

\path[fill=fillColor,fill opacity=0.20] ( 91.15, 84.26) circle (  2.13);

\path[fill=fillColor,fill opacity=0.20] ( 99.99, 81.82) circle (  2.13);

\path[fill=fillColor,fill opacity=0.20] (104.91, 68.01) circle (  2.13);

\path[fill=fillColor,fill opacity=0.20] (117.68, 73.70) circle (  2.13);

\path[fill=fillColor,fill opacity=0.20] ( 84.27,102.14) circle (  2.13);

\path[fill=fillColor,fill opacity=0.20] ( 73.46, 88.33) circle (  2.13);

\path[fill=fillColor,fill opacity=0.20] ( 90.17, 99.70) circle (  2.13);

\path[fill=fillColor,fill opacity=0.20] ( 88.20, 98.08) circle (  2.13);

\path[fill=fillColor,fill opacity=0.20] ( 84.27, 91.58) circle (  2.13);

\path[fill=fillColor,fill opacity=0.20] ( 84.27, 87.51) circle (  2.13);

\path[fill=fillColor,fill opacity=0.20] ( 79.36, 90.76) circle (  2.13);

\path[fill=fillColor,fill opacity=0.20] ( 87.22, 85.08) circle (  2.13);

\path[fill=fillColor,fill opacity=0.20] ( 81.32, 74.51) circle (  2.13);

\path[fill=fillColor,fill opacity=0.20] ( 81.32, 73.70) circle (  2.13);

\path[fill=fillColor,fill opacity=0.20] ( 93.11, 65.57) circle (  2.13);

\path[fill=fillColor,fill opacity=0.20] ( 99.01, 51.75) circle (  2.13);

\path[fill=fillColor,fill opacity=0.20] (101.96, 64.76) circle (  2.13);

\path[fill=fillColor,fill opacity=0.20] ( 90.17, 65.57) circle (  2.13);

\path[fill=fillColor,fill opacity=0.20] ( 91.15, 66.38) circle (  2.13);

\path[fill=fillColor,fill opacity=0.20] ( 93.11, 77.76) circle (  2.13);

\path[fill=fillColor,fill opacity=0.20] ( 91.15, 82.64) circle (  2.13);

\path[fill=fillColor,fill opacity=0.20] ( 92.13, 74.51) circle (  2.13);

\path[fill=fillColor,fill opacity=0.20] ( 94.10, 65.57) circle (  2.13);

\path[fill=fillColor,fill opacity=0.20] (100.98, 60.69) circle (  2.13);

\path[fill=fillColor,fill opacity=0.20] (100.98, 61.51) circle (  2.13);

\path[fill=fillColor,fill opacity=0.20] (108.84, 72.07) circle (  2.13);

\path[fill=fillColor,fill opacity=0.20] ( 88.20,113.52) circle (  2.13);

\path[fill=fillColor,fill opacity=0.20] ( 80.34, 83.45) circle (  2.13);

\path[fill=fillColor,fill opacity=0.20] ( 81.32, 92.39) circle (  2.13);

\path[fill=fillColor,fill opacity=0.20] ( 86.24, 89.95) circle (  2.13);

\path[fill=fillColor,fill opacity=0.20] ( 79.36, 87.51) circle (  2.13);

\path[fill=fillColor,fill opacity=0.20] ( 79.36, 89.14) circle (  2.13);

\path[fill=fillColor,fill opacity=0.20] ( 79.36, 89.14) circle (  2.13);

\path[fill=fillColor,fill opacity=0.20] ( 78.37, 84.26) circle (  2.13);

\path[fill=fillColor,fill opacity=0.20] ( 82.31, 76.95) circle (  2.13);

\path[fill=fillColor,fill opacity=0.20] ( 83.29, 74.51) circle (  2.13);

\path[fill=fillColor,fill opacity=0.20] ( 88.20, 76.14) circle (  2.13);

\path[fill=fillColor,fill opacity=0.20] ( 85.25, 66.38) circle (  2.13);

\path[fill=fillColor,fill opacity=0.20] ( 94.10, 56.63) circle (  2.13);

\path[fill=fillColor,fill opacity=0.20] ( 87.22, 62.32) circle (  2.13);

\path[fill=fillColor,fill opacity=0.20] ( 81.32, 53.38) circle (  2.13);

\path[fill=fillColor,fill opacity=0.20] ( 94.10, 63.13) circle (  2.13);

\path[fill=fillColor,fill opacity=0.20] ( 90.17, 73.70) circle (  2.13);

\path[fill=fillColor,fill opacity=0.20] ( 93.11, 71.26) circle (  2.13);

\path[fill=fillColor,fill opacity=0.20] ( 97.05, 68.01) circle (  2.13);

\path[fill=fillColor,fill opacity=0.20] ( 91.15, 64.76) circle (  2.13);

\path[fill=fillColor,fill opacity=0.20] ( 96.06, 65.57) circle (  2.13);

\path[fill=fillColor,fill opacity=0.20] ( 99.99, 72.89) circle (  2.13);

\path[fill=fillColor,fill opacity=0.20] (114.73, 81.01) circle (  2.13);

\path[fill=fillColor,fill opacity=0.20] ( 82.31, 86.70) circle (  2.13);

\path[fill=fillColor,fill opacity=0.20] ( 80.34, 95.64) circle (  2.13);

\path[fill=fillColor,fill opacity=0.20] ( 84.27, 88.33) circle (  2.13);

\path[fill=fillColor,fill opacity=0.20] ( 80.34, 81.01) circle (  2.13);

\path[fill=fillColor,fill opacity=0.20] ( 79.36, 85.08) circle (  2.13);

\path[fill=fillColor,fill opacity=0.20] ( 79.36, 92.39) circle (  2.13);

\path[fill=fillColor,fill opacity=0.20] ( 79.36, 94.83) circle (  2.13);

\path[fill=fillColor,fill opacity=0.20] ( 76.41, 85.89) circle (  2.13);

\path[fill=fillColor,fill opacity=0.20] ( 83.29, 77.76) circle (  2.13);

\path[fill=fillColor,fill opacity=0.20] ( 82.31, 75.32) circle (  2.13);

\path[fill=fillColor,fill opacity=0.20] ( 84.27, 69.63) circle (  2.13);

\path[fill=fillColor,fill opacity=0.20] ( 87.22, 64.76) circle (  2.13);

\path[fill=fillColor,fill opacity=0.20] (103.92, 69.63) circle (  2.13);

\path[fill=fillColor,fill opacity=0.20] (115.72, 72.89) circle (  2.13);

\path[fill=fillColor,fill opacity=0.20] ( 98.03, 55.01) circle (  2.13);

\path[fill=fillColor,fill opacity=0.20] ( 92.13, 55.01) circle (  2.13);

\path[fill=fillColor,fill opacity=0.20] ( 94.10, 63.95) circle (  2.13);

\path[fill=fillColor,fill opacity=0.20] ( 97.05, 75.32) circle (  2.13);

\path[fill=fillColor,fill opacity=0.20] ( 90.17, 78.57) circle (  2.13);

\path[fill=fillColor,fill opacity=0.20] ( 94.10, 73.70) circle (  2.13);

\path[fill=fillColor,fill opacity=0.20] ( 93.11, 72.89) circle (  2.13);

\path[fill=fillColor,fill opacity=0.20] ( 99.01, 79.39) circle (  2.13);

\path[fill=fillColor,fill opacity=0.20] ( 99.99, 82.64) circle (  2.13);

\path[fill=fillColor,fill opacity=0.20] (108.84, 81.82) circle (  2.13);

\path[fill=fillColor,fill opacity=0.20] ( 84.27, 85.08) circle (  2.13);

\path[fill=fillColor,fill opacity=0.20] ( 81.32, 96.45) circle (  2.13);

\path[fill=fillColor,fill opacity=0.20] ( 84.27, 96.45) circle (  2.13);

\path[fill=fillColor,fill opacity=0.20] ( 82.31, 88.33) circle (  2.13);

\path[fill=fillColor,fill opacity=0.20] ( 77.39, 92.39) circle (  2.13);

\path[fill=fillColor,fill opacity=0.20] ( 83.29, 94.02) circle (  2.13);

\path[fill=fillColor,fill opacity=0.20] ( 83.29, 92.39) circle (  2.13);

\path[fill=fillColor,fill opacity=0.20] ( 79.36, 91.58) circle (  2.13);

\path[fill=fillColor,fill opacity=0.20] ( 80.34, 89.95) circle (  2.13);

\path[fill=fillColor,fill opacity=0.20] ( 78.37, 87.51) circle (  2.13);

\path[fill=fillColor,fill opacity=0.20] ( 81.32, 72.89) circle (  2.13);

\path[fill=fillColor,fill opacity=0.20] ( 94.10, 53.38) circle (  2.13);

\path[fill=fillColor,fill opacity=0.20] (114.73, 72.89) circle (  2.13);

\path[fill=fillColor,fill opacity=0.20] ( 90.17, 63.95) circle (  2.13);

\path[fill=fillColor,fill opacity=0.20] ( 89.18, 59.07) circle (  2.13);

\path[fill=fillColor,fill opacity=0.20] ( 95.08, 69.63) circle (  2.13);

\path[fill=fillColor,fill opacity=0.20] ( 90.17, 72.07) circle (  2.13);

\path[fill=fillColor,fill opacity=0.20] ( 87.22, 68.01) circle (  2.13);

\path[fill=fillColor,fill opacity=0.20] ( 94.10, 72.89) circle (  2.13);

\path[fill=fillColor,fill opacity=0.20] ( 95.08, 76.14) circle (  2.13);

\path[fill=fillColor,fill opacity=0.20] ( 95.08, 75.32) circle (  2.13);

\path[fill=fillColor,fill opacity=0.20] ( 97.05, 72.89) circle (  2.13);

\path[fill=fillColor,fill opacity=0.20] (106.87, 72.07) circle (  2.13);

\path[fill=fillColor,fill opacity=0.20] ( 88.20, 92.39) circle (  2.13);

\path[fill=fillColor,fill opacity=0.20] ( 83.29,100.52) circle (  2.13);

\path[fill=fillColor,fill opacity=0.20] ( 78.37, 96.45) circle (  2.13);

\path[fill=fillColor,fill opacity=0.20] ( 81.32, 87.51) circle (  2.13);

\path[fill=fillColor,fill opacity=0.20] ( 76.41, 88.33) circle (  2.13);

\path[fill=fillColor,fill opacity=0.20] ( 77.39, 94.02) circle (  2.13);

\path[fill=fillColor,fill opacity=0.20] ( 80.34, 96.45) circle (  2.13);

\path[fill=fillColor,fill opacity=0.20] ( 79.36, 89.14) circle (  2.13);

\path[fill=fillColor,fill opacity=0.20] ( 75.43, 85.08) circle (  2.13);

\path[fill=fillColor,fill opacity=0.20] ( 76.41, 88.33) circle (  2.13);

\path[fill=fillColor,fill opacity=0.20] ( 50.37, 84.26) circle (  2.13);

\path[fill=fillColor,fill opacity=0.20] ( 82.31, 66.38) circle (  2.13);

\path[fill=fillColor,fill opacity=0.20] (101.96, 46.88) circle (  2.13);

\path[fill=fillColor,fill opacity=0.20] (115.72, 78.57) circle (  2.13);

\path[fill=fillColor,fill opacity=0.20] ( 89.18, 59.88) circle (  2.13);

\path[fill=fillColor,fill opacity=0.20] ( 85.25, 55.82) circle (  2.13);

\path[fill=fillColor,fill opacity=0.20] ( 86.24, 57.44) circle (  2.13);

\path[fill=fillColor,fill opacity=0.20] ( 80.34, 61.51) circle (  2.13);

\path[fill=fillColor,fill opacity=0.20] ( 89.18, 70.45) circle (  2.13);

\path[fill=fillColor,fill opacity=0.20] ( 93.11, 72.89) circle (  2.13);

\path[fill=fillColor,fill opacity=0.20] ( 93.11, 76.14) circle (  2.13);

\path[fill=fillColor,fill opacity=0.20] (100.98, 77.76) circle (  2.13);

\path[fill=fillColor,fill opacity=0.20] (105.89, 74.51) circle (  2.13);

\path[fill=fillColor,fill opacity=0.20] (110.80, 81.82) circle (  2.13);

\path[fill=fillColor,fill opacity=0.20] ( 85.25, 95.64) circle (  2.13);

\path[fill=fillColor,fill opacity=0.20] ( 87.22,105.39) circle (  2.13);

\path[fill=fillColor,fill opacity=0.20] ( 81.32, 98.89) circle (  2.13);

\path[fill=fillColor,fill opacity=0.20] ( 76.41, 88.33) circle (  2.13);

\path[fill=fillColor,fill opacity=0.20] ( 73.46, 85.89) circle (  2.13);

\path[fill=fillColor,fill opacity=0.20] ( 70.91, 84.26) circle (  2.13);

\path[fill=fillColor,fill opacity=0.20] ( 76.41, 85.89) circle (  2.13);

\path[fill=fillColor,fill opacity=0.20] ( 77.39, 89.14) circle (  2.13);

\path[fill=fillColor,fill opacity=0.20] ( 75.43, 89.14) circle (  2.13);

\path[fill=fillColor,fill opacity=0.20] ( 77.39, 85.89) circle (  2.13);

\path[fill=fillColor,fill opacity=0.20] ( 72.18, 80.20) circle (  2.13);

\path[fill=fillColor,fill opacity=0.20] ( 79.36, 66.38) circle (  2.13);

\path[fill=fillColor,fill opacity=0.20] ( 99.01, 59.88) circle (  2.13);

\path[fill=fillColor,fill opacity=0.20] (113.75, 72.07) circle (  2.13);

\path[fill=fillColor,fill opacity=0.20] ( 78.37, 55.82) circle (  2.13);

\path[fill=fillColor,fill opacity=0.20] ( 83.29, 56.63) circle (  2.13);

\path[fill=fillColor,fill opacity=0.20] ( 82.31, 63.95) circle (  2.13);

\path[fill=fillColor,fill opacity=0.20] ( 84.27, 68.82) circle (  2.13);

\path[fill=fillColor,fill opacity=0.20] ( 93.11, 75.32) circle (  2.13);

\path[fill=fillColor,fill opacity=0.20] ( 91.15, 83.45) circle (  2.13);

\path[fill=fillColor,fill opacity=0.20] ( 95.08, 81.82) circle (  2.13);

\path[fill=fillColor,fill opacity=0.20] (105.89, 79.39) circle (  2.13);

\path[fill=fillColor,fill opacity=0.20] (102.94, 79.39) circle (  2.13);

\path[fill=fillColor,fill opacity=0.20] (107.86, 75.32) circle (  2.13);

\path[fill=fillColor,fill opacity=0.20] (102.94, 90.76) circle (  2.13);

\path[fill=fillColor,fill opacity=0.20] ( 84.27, 76.14) circle (  2.13);

\path[fill=fillColor,fill opacity=0.20] ( 70.51,101.33) circle (  2.13);

\path[fill=fillColor,fill opacity=0.20] ( 79.36,105.39) circle (  2.13);

\path[fill=fillColor,fill opacity=0.20] ( 75.43, 89.14) circle (  2.13);

\path[fill=fillColor,fill opacity=0.20] ( 73.46, 89.14) circle (  2.13);

\path[fill=fillColor,fill opacity=0.20] ( 70.91, 94.02) circle (  2.13);

\path[fill=fillColor,fill opacity=0.20] ( 72.48, 89.95) circle (  2.13);

\path[fill=fillColor,fill opacity=0.20] ( 76.41, 81.82) circle (  2.13);

\path[fill=fillColor,fill opacity=0.20] ( 76.41, 81.01) circle (  2.13);

\path[fill=fillColor,fill opacity=0.20] ( 77.39, 83.45) circle (  2.13);

\path[fill=fillColor,fill opacity=0.20] ( 80.34, 76.14) circle (  2.13);

\path[fill=fillColor,fill opacity=0.20] ( 79.36, 64.76) circle (  2.13);

\path[fill=fillColor,fill opacity=0.20] ( 99.99, 60.69) circle (  2.13);

\path[fill=fillColor,fill opacity=0.20] ( 79.36, 63.13) circle (  2.13);

\path[fill=fillColor,fill opacity=0.20] ( 80.34, 66.38) circle (  2.13);

\path[fill=fillColor,fill opacity=0.20] ( 90.17, 60.69) circle (  2.13);

\path[fill=fillColor,fill opacity=0.20] ( 93.11, 63.95) circle (  2.13);

\path[fill=fillColor,fill opacity=0.20] ( 89.18, 78.57) circle (  2.13);

\path[fill=fillColor,fill opacity=0.20] ( 93.11, 80.20) circle (  2.13);

\path[fill=fillColor,fill opacity=0.20] ( 99.99, 76.14) circle (  2.13);

\path[fill=fillColor,fill opacity=0.20] (102.94, 76.95) circle (  2.13);

\path[fill=fillColor,fill opacity=0.20] (108.84, 71.26) circle (  2.13);

\path[fill=fillColor,fill opacity=0.20] ( 99.99, 66.38) circle (  2.13);

\path[fill=fillColor,fill opacity=0.20] (108.84, 68.82) circle (  2.13);

\path[fill=fillColor,fill opacity=0.20] (115.72, 82.64) circle (  2.13);

\path[fill=fillColor,fill opacity=0.20] ( 97.05, 67.20) circle (  2.13);

\path[fill=fillColor,fill opacity=0.20] ( 89.18, 72.07) circle (  2.13);

\path[fill=fillColor,fill opacity=0.20] ( 76.41, 84.26) circle (  2.13);

\path[fill=fillColor,fill opacity=0.20] ( 81.32, 90.76) circle (  2.13);

\path[fill=fillColor,fill opacity=0.20] ( 78.37, 93.20) circle (  2.13);

\path[fill=fillColor,fill opacity=0.20] ( 77.39, 91.58) circle (  2.13);

\path[fill=fillColor,fill opacity=0.20] ( 74.44, 94.02) circle (  2.13);

\path[fill=fillColor,fill opacity=0.20] ( 71.89, 99.70) circle (  2.13);

\path[fill=fillColor,fill opacity=0.20] ( 71.89, 94.02) circle (  2.13);

\path[fill=fillColor,fill opacity=0.20] ( 73.46, 81.01) circle (  2.13);

\path[fill=fillColor,fill opacity=0.20] ( 77.39, 75.32) circle (  2.13);

\path[fill=fillColor,fill opacity=0.20] ( 80.34, 71.26) circle (  2.13);

\path[fill=fillColor,fill opacity=0.20] ( 94.10, 60.69) circle (  2.13);

\path[fill=fillColor,fill opacity=0.20] ( 99.01, 59.88) circle (  2.13);

\path[fill=fillColor,fill opacity=0.20] ( 97.05, 66.38) circle (  2.13);

\path[fill=fillColor,fill opacity=0.20] ( 92.13, 55.01) circle (  2.13);

\path[fill=fillColor,fill opacity=0.20] ( 90.17, 47.69) circle (  2.13);

\path[fill=fillColor,fill opacity=0.20] ( 87.22, 58.26) circle (  2.13);

\path[fill=fillColor,fill opacity=0.20] ( 90.17, 72.07) circle (  2.13);

\path[fill=fillColor,fill opacity=0.20] ( 97.05, 73.70) circle (  2.13);

\path[fill=fillColor,fill opacity=0.20] (104.91, 76.95) circle (  2.13);

\path[fill=fillColor,fill opacity=0.20] (108.84, 85.08) circle (  2.13);

\path[fill=fillColor,fill opacity=0.20] (107.86, 85.08) circle (  2.13);

\path[fill=fillColor,fill opacity=0.20] (106.87, 73.70) circle (  2.13);

\path[fill=fillColor,fill opacity=0.20] (110.80, 61.51) circle (  2.13);

\path[fill=fillColor,fill opacity=0.20] (111.79, 61.51) circle (  2.13);

\path[fill=fillColor,fill opacity=0.20] (118.66, 72.07) circle (  2.13);

\path[fill=fillColor,fill opacity=0.20] (109.82, 63.95) circle (  2.13);

\path[fill=fillColor,fill opacity=0.20] (100.98, 68.82) circle (  2.13);

\path[fill=fillColor,fill opacity=0.20] ( 90.17, 74.51) circle (  2.13);

\path[fill=fillColor,fill opacity=0.20] ( 85.25, 78.57) circle (  2.13);

\path[fill=fillColor,fill opacity=0.20] ( 61.57, 85.08) circle (  2.13);

\path[fill=fillColor,fill opacity=0.20] ( 80.34, 88.33) circle (  2.13);

\path[fill=fillColor,fill opacity=0.20] ( 86.24, 82.64) circle (  2.13);

\path[fill=fillColor,fill opacity=0.20] ( 77.39, 84.26) circle (  2.13);

\path[fill=fillColor,fill opacity=0.20] ( 75.43, 90.76) circle (  2.13);

\path[fill=fillColor,fill opacity=0.20] ( 75.43, 92.39) circle (  2.13);

\path[fill=fillColor,fill opacity=0.20] ( 73.46, 93.20) circle (  2.13);

\path[fill=fillColor,fill opacity=0.20] ( 70.71, 94.02) circle (  2.13);

\path[fill=fillColor,fill opacity=0.20] ( 78.37, 85.89) circle (  2.13);

\path[fill=fillColor,fill opacity=0.20] ( 82.31, 72.07) circle (  2.13);

\path[fill=fillColor,fill opacity=0.20] ( 96.06, 68.01) circle (  2.13);

\path[fill=fillColor,fill opacity=0.20] (105.89, 61.51) circle (  2.13);

\path[fill=fillColor,fill opacity=0.20] ( 94.10, 50.94) circle (  2.13);

\path[fill=fillColor,fill opacity=0.20] ( 87.22, 48.50) circle (  2.13);

\path[fill=fillColor,fill opacity=0.20] ( 89.18, 60.69) circle (  2.13);

\path[fill=fillColor,fill opacity=0.20] ( 95.08, 68.82) circle (  2.13);

\path[fill=fillColor,fill opacity=0.20] ( 94.10, 71.26) circle (  2.13);

\path[fill=fillColor,fill opacity=0.20] ( 97.05, 82.64) circle (  2.13);

\path[fill=fillColor,fill opacity=0.20] ( 99.01, 87.51) circle (  2.13);

\path[fill=fillColor,fill opacity=0.20] (103.92, 81.82) circle (  2.13);

\path[fill=fillColor,fill opacity=0.20] (104.91, 72.07) circle (  2.13);

\path[fill=fillColor,fill opacity=0.20] (103.92, 67.20) circle (  2.13);

\path[fill=fillColor,fill opacity=0.20] (102.94, 72.07) circle (  2.13);

\path[fill=fillColor,fill opacity=0.20] (108.84, 74.51) circle (  2.13);

\path[fill=fillColor,fill opacity=0.20] (112.77, 64.76) circle (  2.13);

\path[fill=fillColor,fill opacity=0.20] (114.73, 68.01) circle (  2.13);

\path[fill=fillColor,fill opacity=0.20] (110.80, 58.26) circle (  2.13);

\path[fill=fillColor,fill opacity=0.20] (109.82, 64.76) circle (  2.13);

\path[fill=fillColor,fill opacity=0.20] ( 82.31, 72.07) circle (  2.13);

\path[fill=fillColor,fill opacity=0.20] ( 91.15, 67.20) circle (  2.13);

\path[fill=fillColor,fill opacity=0.20] ( 87.22, 72.07) circle (  2.13);

\path[fill=fillColor,fill opacity=0.20] ( 77.39, 89.14) circle (  2.13);

\path[fill=fillColor,fill opacity=0.20] ( 81.32, 89.14) circle (  2.13);

\path[fill=fillColor,fill opacity=0.20] ( 80.34, 81.01) circle (  2.13);

\path[fill=fillColor,fill opacity=0.20] ( 88.20, 83.45) circle (  2.13);

\path[fill=fillColor,fill opacity=0.20] ( 80.34, 82.64) circle (  2.13);

\path[fill=fillColor,fill opacity=0.20] ( 73.46, 82.64) circle (  2.13);

\path[fill=fillColor,fill opacity=0.20] ( 65.60, 86.70) circle (  2.13);

\path[fill=fillColor,fill opacity=0.20] ( 77.39, 81.82) circle (  2.13);

\path[fill=fillColor,fill opacity=0.20] ( 81.32, 80.20) circle (  2.13);

\path[fill=fillColor,fill opacity=0.20] ( 79.36, 84.26) circle (  2.13);

\path[fill=fillColor,fill opacity=0.20] ( 90.17, 83.45) circle (  2.13);

\path[fill=fillColor,fill opacity=0.20] ( 93.11, 73.70) circle (  2.13);

\path[fill=fillColor,fill opacity=0.20] (103.92, 60.69) circle (  2.13);

\path[fill=fillColor,fill opacity=0.20] ( 93.11, 63.95) circle (  2.13);

\path[fill=fillColor,fill opacity=0.20] ( 91.15, 64.76) circle (  2.13);

\path[fill=fillColor,fill opacity=0.20] ( 86.24, 58.26) circle (  2.13);

\path[fill=fillColor,fill opacity=0.20] ( 86.24, 61.51) circle (  2.13);

\path[fill=fillColor,fill opacity=0.20] ( 94.10, 72.89) circle (  2.13);

\path[fill=fillColor,fill opacity=0.20] (100.98, 76.95) circle (  2.13);

\path[fill=fillColor,fill opacity=0.20] ( 99.99, 79.39) circle (  2.13);

\path[fill=fillColor,fill opacity=0.20] (104.91, 82.64) circle (  2.13);

\path[fill=fillColor,fill opacity=0.20] (104.91, 79.39) circle (  2.13);

\path[fill=fillColor,fill opacity=0.20] (106.87, 79.39) circle (  2.13);

\path[fill=fillColor,fill opacity=0.20] (110.80, 76.14) circle (  2.13);

\path[fill=fillColor,fill opacity=0.20] (109.82, 69.63) circle (  2.13);

\path[fill=fillColor,fill opacity=0.20] (106.87, 65.57) circle (  2.13);

\path[fill=fillColor,fill opacity=0.20] (105.89, 64.76) circle (  2.13);

\path[fill=fillColor,fill opacity=0.20] (105.89, 66.38) circle (  2.13);

\path[fill=fillColor,fill opacity=0.20] (107.86, 71.26) circle (  2.13);

\path[fill=fillColor,fill opacity=0.20] (101.96, 60.69) circle (  2.13);

\path[fill=fillColor,fill opacity=0.20] (111.79, 48.50) circle (  2.13);

\path[fill=fillColor,fill opacity=0.20] (112.77, 59.07) circle (  2.13);

\path[fill=fillColor,fill opacity=0.20] (102.94, 70.45) circle (  2.13);

\path[fill=fillColor,fill opacity=0.20] (101.96, 64.76) circle (  2.13);

\path[fill=fillColor,fill opacity=0.20] (104.91, 59.07) circle (  2.13);

\path[fill=fillColor,fill opacity=0.20] (105.89, 62.32) circle (  2.13);

\path[fill=fillColor,fill opacity=0.20] (106.87, 59.88) circle (  2.13);

\path[fill=fillColor,fill opacity=0.20] ( 99.99, 53.38) circle (  2.13);

\path[fill=fillColor,fill opacity=0.20] (103.92, 48.50) circle (  2.13);

\path[fill=fillColor,fill opacity=0.20] (104.91, 49.32) circle (  2.13);

\path[fill=fillColor,fill opacity=0.20] (102.94, 55.82) circle (  2.13);

\path[fill=fillColor,fill opacity=0.20] (101.96, 59.88) circle (  2.13);

\path[fill=fillColor,fill opacity=0.20] (103.92, 56.63) circle (  2.13);

\path[fill=fillColor,fill opacity=0.20] (103.92, 55.01) circle (  2.13);

\path[fill=fillColor,fill opacity=0.20] ( 98.03, 55.01) circle (  2.13);

\path[fill=fillColor,fill opacity=0.20] ( 92.13, 57.44) circle (  2.13);

\path[fill=fillColor,fill opacity=0.20] ( 85.25, 69.63) circle (  2.13);

\path[fill=fillColor,fill opacity=0.20] ( 83.29, 76.14) circle (  2.13);

\path[fill=fillColor,fill opacity=0.20] ( 89.18, 73.70) circle (  2.13);

\path[fill=fillColor,fill opacity=0.20] ( 85.25, 78.57) circle (  2.13);

\path[fill=fillColor,fill opacity=0.20] ( 80.34, 80.20) circle (  2.13);

\path[fill=fillColor,fill opacity=0.20] ( 83.29, 76.95) circle (  2.13);

\path[fill=fillColor,fill opacity=0.20] ( 84.27, 73.70) circle (  2.13);

\path[fill=fillColor,fill opacity=0.20] ( 85.25, 73.70) circle (  2.13);

\path[fill=fillColor,fill opacity=0.20] ( 82.31, 80.20) circle (  2.13);

\path[fill=fillColor,fill opacity=0.20] ( 86.24, 89.14) circle (  2.13);

\path[fill=fillColor,fill opacity=0.20] ( 94.10, 82.64) circle (  2.13);

\path[fill=fillColor,fill opacity=0.20] ( 80.34, 79.39) circle (  2.13);

\path[fill=fillColor,fill opacity=0.20] ( 92.13, 80.20) circle (  2.13);

\path[fill=fillColor,fill opacity=0.20] ( 99.01, 69.63) circle (  2.13);

\path[fill=fillColor,fill opacity=0.20] ( 93.11, 64.76) circle (  2.13);

\path[fill=fillColor,fill opacity=0.20] ( 95.08, 58.26) circle (  2.13);

\path[fill=fillColor,fill opacity=0.20] ( 92.13, 54.19) circle (  2.13);

\path[fill=fillColor,fill opacity=0.20] ( 91.15, 61.51) circle (  2.13);

\path[fill=fillColor,fill opacity=0.20] ( 93.11, 68.82) circle (  2.13);

\path[fill=fillColor,fill opacity=0.20] ( 96.06, 68.82) circle (  2.13);

\path[fill=fillColor,fill opacity=0.20] (100.98, 71.26) circle (  2.13);

\path[fill=fillColor,fill opacity=0.20] (105.89, 74.51) circle (  2.13);

\path[fill=fillColor,fill opacity=0.20] (105.89, 74.51) circle (  2.13);

\path[fill=fillColor,fill opacity=0.20] (104.91, 77.76) circle (  2.13);

\path[fill=fillColor,fill opacity=0.20] (106.87, 78.57) circle (  2.13);

\path[fill=fillColor,fill opacity=0.20] (100.98, 75.32) circle (  2.13);

\path[fill=fillColor,fill opacity=0.20] ( 92.13, 76.14) circle (  2.13);

\path[fill=fillColor,fill opacity=0.20] (103.92, 76.14) circle (  2.13);

\path[fill=fillColor,fill opacity=0.20] (102.94, 72.89) circle (  2.13);

\path[fill=fillColor,fill opacity=0.20] (105.89, 64.76) circle (  2.13);

\path[fill=fillColor,fill opacity=0.20] (102.94, 55.01) circle (  2.13);

\path[fill=fillColor,fill opacity=0.20] (105.89, 56.63) circle (  2.13);

\path[fill=fillColor,fill opacity=0.20] (101.96, 62.32) circle (  2.13);

\path[fill=fillColor,fill opacity=0.20] (106.87, 58.26) circle (  2.13);

\path[fill=fillColor,fill opacity=0.20] ( 99.99, 59.07) circle (  2.13);

\path[fill=fillColor,fill opacity=0.20] ( 97.05, 63.95) circle (  2.13);

\path[fill=fillColor,fill opacity=0.20] ( 95.08, 65.57) circle (  2.13);

\path[fill=fillColor,fill opacity=0.20] ( 93.11, 67.20) circle (  2.13);

\path[fill=fillColor,fill opacity=0.20] ( 96.06, 68.01) circle (  2.13);

\path[fill=fillColor,fill opacity=0.20] ( 96.06, 65.57) circle (  2.13);

\path[fill=fillColor,fill opacity=0.20] ( 94.10, 67.20) circle (  2.13);

\path[fill=fillColor,fill opacity=0.20] ( 94.10, 70.45) circle (  2.13);

\path[fill=fillColor,fill opacity=0.20] ( 90.17, 68.01) circle (  2.13);

\path[fill=fillColor,fill opacity=0.20] ( 90.17, 62.32) circle (  2.13);

\path[fill=fillColor,fill opacity=0.20] ( 86.24, 61.51) circle (  2.13);

\path[fill=fillColor,fill opacity=0.20] ( 85.25, 66.38) circle (  2.13);

\path[fill=fillColor,fill opacity=0.20] ( 91.15, 60.69) circle (  2.13);

\path[fill=fillColor,fill opacity=0.20] ( 87.22, 52.57) circle (  2.13);

\path[fill=fillColor,fill opacity=0.20] ( 91.15, 63.13) circle (  2.13);

\path[fill=fillColor,fill opacity=0.20] ( 88.20, 72.89) circle (  2.13);

\path[fill=fillColor,fill opacity=0.20] ( 87.22, 70.45) circle (  2.13);

\path[fill=fillColor,fill opacity=0.20] ( 90.17, 75.32) circle (  2.13);

\path[fill=fillColor,fill opacity=0.20] ( 92.13, 76.14) circle (  2.13);

\path[fill=fillColor,fill opacity=0.20] ( 88.20, 75.32) circle (  2.13);

\path[fill=fillColor,fill opacity=0.20] ( 90.17, 85.08) circle (  2.13);

\path[fill=fillColor,fill opacity=0.20] (101.96, 85.08) circle (  2.13);

\path[fill=fillColor,fill opacity=0.20] ( 77.39, 75.32) circle (  2.13);

\path[fill=fillColor,fill opacity=0.20] (114.73, 67.20) circle (  2.13);

\path[fill=fillColor,fill opacity=0.20] ( 93.11, 63.13) circle (  2.13);

\path[fill=fillColor,fill opacity=0.20] ( 92.13, 61.51) circle (  2.13);

\path[fill=fillColor,fill opacity=0.20] ( 95.08, 64.76) circle (  2.13);

\path[fill=fillColor,fill opacity=0.20] ( 91.15, 63.13) circle (  2.13);

\path[fill=fillColor,fill opacity=0.20] ( 95.08, 58.26) circle (  2.13);

\path[fill=fillColor,fill opacity=0.20] ( 96.06, 59.07) circle (  2.13);

\path[fill=fillColor,fill opacity=0.20] ( 99.99, 68.01) circle (  2.13);

\path[fill=fillColor,fill opacity=0.20] (100.98, 76.14) circle (  2.13);

\path[fill=fillColor,fill opacity=0.20] ( 99.01, 77.76) circle (  2.13);

\path[fill=fillColor,fill opacity=0.20] (100.98, 79.39) circle (  2.13);

\path[fill=fillColor,fill opacity=0.20] (102.94, 82.64) circle (  2.13);

\path[fill=fillColor,fill opacity=0.20] ( 99.99, 81.82) circle (  2.13);

\path[fill=fillColor,fill opacity=0.20] ( 99.01, 76.14) circle (  2.13);

\path[fill=fillColor,fill opacity=0.20] (100.98, 75.32) circle (  2.13);

\path[fill=fillColor,fill opacity=0.20] ( 99.99, 73.70) circle (  2.13);

\path[fill=fillColor,fill opacity=0.20] ( 99.99, 72.07) circle (  2.13);

\path[fill=fillColor,fill opacity=0.20] (105.89, 70.45) circle (  2.13);

\path[fill=fillColor,fill opacity=0.20] (102.94, 69.63) circle (  2.13);

\path[fill=fillColor,fill opacity=0.20] ( 93.11, 71.26) circle (  2.13);

\path[fill=fillColor,fill opacity=0.20] ( 99.01, 70.45) circle (  2.13);

\path[fill=fillColor,fill opacity=0.20] (103.92, 72.89) circle (  2.13);

\path[fill=fillColor,fill opacity=0.20] (103.92, 79.39) circle (  2.13);

\path[fill=fillColor,fill opacity=0.20] ( 99.01, 81.01) circle (  2.13);

\path[fill=fillColor,fill opacity=0.20] (100.98, 76.14) circle (  2.13);

\path[fill=fillColor,fill opacity=0.20] ( 97.05, 71.26) circle (  2.13);

\path[fill=fillColor,fill opacity=0.20] ( 96.06, 66.38) circle (  2.13);

\path[fill=fillColor,fill opacity=0.20] ( 93.11, 64.76) circle (  2.13);

\path[fill=fillColor,fill opacity=0.20] ( 95.08, 62.32) circle (  2.13);

\path[fill=fillColor,fill opacity=0.20] ( 86.24, 59.07) circle (  2.13);

\path[fill=fillColor,fill opacity=0.20] ( 85.25, 65.57) circle (  2.13);

\path[fill=fillColor,fill opacity=0.20] ( 77.39, 72.07) circle (  2.13);

\path[fill=fillColor,fill opacity=0.20] ( 93.11, 69.63) circle (  2.13);

\path[fill=fillColor,fill opacity=0.20] ( 86.24, 70.45) circle (  2.13);

\path[fill=fillColor,fill opacity=0.20] ( 91.15, 77.76) circle (  2.13);

\path[fill=fillColor,fill opacity=0.20] ( 91.15, 83.45) circle (  2.13);

\path[fill=fillColor,fill opacity=0.20] (106.87, 91.58) circle (  2.13);

\path[fill=fillColor,fill opacity=0.20] ( 96.06, 71.26) circle (  2.13);

\path[fill=fillColor,fill opacity=0.20] ( 94.10, 58.26) circle (  2.13);

\path[fill=fillColor,fill opacity=0.20] ( 94.10, 53.38) circle (  2.13);

\path[fill=fillColor,fill opacity=0.20] ( 88.20, 62.32) circle (  2.13);

\path[fill=fillColor,fill opacity=0.20] ( 90.17, 71.26) circle (  2.13);

\path[fill=fillColor,fill opacity=0.20] ( 93.11, 69.63) circle (  2.13);

\path[fill=fillColor,fill opacity=0.20] ( 91.15, 70.45) circle (  2.13);

\path[fill=fillColor,fill opacity=0.20] ( 89.18, 73.70) circle (  2.13);

\path[fill=fillColor,fill opacity=0.20] ( 94.10, 71.26) circle (  2.13);

\path[fill=fillColor,fill opacity=0.20] ( 97.05, 67.20) circle (  2.13);

\path[fill=fillColor,fill opacity=0.20] ( 98.03, 63.13) circle (  2.13);

\path[fill=fillColor,fill opacity=0.20] ( 99.01, 63.95) circle (  2.13);

\path[fill=fillColor,fill opacity=0.20] (103.92, 68.01) circle (  2.13);

\path[fill=fillColor,fill opacity=0.20] (100.98, 68.01) circle (  2.13);

\path[fill=fillColor,fill opacity=0.20] (101.96, 63.13) circle (  2.13);

\path[fill=fillColor,fill opacity=0.20] (102.94, 59.07) circle (  2.13);

\path[fill=fillColor,fill opacity=0.20] (102.94, 56.63) circle (  2.13);

\path[fill=fillColor,fill opacity=0.20] ( 98.03, 60.69) circle (  2.13);

\path[fill=fillColor,fill opacity=0.20] ( 90.17, 67.20) circle (  2.13);

\path[fill=fillColor,fill opacity=0.20] ( 95.08, 67.20) circle (  2.13);

\path[fill=fillColor,fill opacity=0.20] ( 90.17, 65.57) circle (  2.13);

\path[fill=fillColor,fill opacity=0.20] ( 93.11, 63.95) circle (  2.13);

\path[fill=fillColor,fill opacity=0.20] ( 88.20, 59.88) circle (  2.13);

\path[fill=fillColor,fill opacity=0.20] ( 88.20, 61.51) circle (  2.13);

\path[fill=fillColor,fill opacity=0.20] ( 89.18, 71.26) circle (  2.13);

\path[fill=fillColor,fill opacity=0.20] ( 99.99, 76.14) circle (  2.13);

\path[fill=fillColor,fill opacity=0.20] ( 92.13, 81.01) circle (  2.13);

\path[fill=fillColor,fill opacity=0.20] (101.96, 95.64) circle (  2.13);

\path[fill=fillColor,fill opacity=0.20] ( 97.05, 98.89) circle (  2.13);

\path[fill=fillColor,fill opacity=0.20] ( 99.99, 65.57) circle (  2.13);

\path[fill=fillColor,fill opacity=0.20] ( 85.25, 64.76) circle (  2.13);

\path[fill=fillColor,fill opacity=0.20] ( 88.20, 64.76) circle (  2.13);

\path[fill=fillColor,fill opacity=0.20] ( 96.06, 66.38) circle (  2.13);

\path[fill=fillColor,fill opacity=0.20] ( 86.24, 62.32) circle (  2.13);

\path[fill=fillColor,fill opacity=0.20] ( 91.15, 52.57) circle (  2.13);

\path[fill=fillColor,fill opacity=0.20] ( 82.31, 58.26) circle (  2.13);

\path[fill=fillColor,fill opacity=0.20] ( 87.22, 66.38) circle (  2.13);

\path[fill=fillColor,fill opacity=0.20] ( 89.18, 54.19) circle (  2.13);

\path[fill=fillColor,fill opacity=0.20] ( 98.03, 45.25) circle (  2.13);

\path[fill=fillColor,fill opacity=0.20] ( 97.05, 55.01) circle (  2.13);

\path[fill=fillColor,fill opacity=0.20] ( 99.01, 58.26) circle (  2.13);

\path[fill=fillColor,fill opacity=0.20] ( 96.06, 54.19) circle (  2.13);

\path[fill=fillColor,fill opacity=0.20] ( 97.05, 50.94) circle (  2.13);

\path[fill=fillColor,fill opacity=0.20] ( 94.10, 50.13) circle (  2.13);

\path[fill=fillColor,fill opacity=0.20] ( 95.08, 53.38) circle (  2.13);

\path[fill=fillColor,fill opacity=0.20] ( 87.22, 59.07) circle (  2.13);

\path[fill=fillColor,fill opacity=0.20] ( 86.24, 61.51) circle (  2.13);

\path[fill=fillColor,fill opacity=0.20] ( 94.10, 69.63) circle (  2.13);

\path[fill=fillColor,fill opacity=0.20] ( 90.17, 75.32) circle (  2.13);

\path[fill=fillColor,fill opacity=0.20] ( 84.27, 72.07) circle (  2.13);

\path[fill=fillColor,fill opacity=0.20] (118.66, 81.01) circle (  2.13);

\path[fill=fillColor,fill opacity=0.20] (108.84, 76.14) circle (  2.13);

\path[fill=fillColor,fill opacity=0.20] (103.92, 63.95) circle (  2.13);

\path[fill=fillColor,fill opacity=0.20] ( 85.25, 68.01) circle (  2.13);

\path[fill=fillColor,fill opacity=0.20] ( 83.29, 73.70) circle (  2.13);

\path[fill=fillColor,fill opacity=0.20] ( 93.11, 67.20) circle (  2.13);

\path[fill=fillColor,fill opacity=0.20] ( 99.99, 63.95) circle (  2.13);

\path[fill=fillColor,fill opacity=0.20] (102.94, 62.32) circle (  2.13);

\path[fill=fillColor,fill opacity=0.20] ( 99.01, 61.51) circle (  2.13);

\path[fill=fillColor,fill opacity=0.20] ( 99.99, 62.32) circle (  2.13);

\path[fill=fillColor,fill opacity=0.20] ( 91.15, 61.51) circle (  2.13);

\path[fill=fillColor,fill opacity=0.20] ( 76.41,101.33) circle (  2.13);

\path[fill=fillColor,fill opacity=0.20] ( 79.36, 97.27) circle (  2.13);

\path[fill=fillColor,fill opacity=0.20] ( 79.36,106.21) circle (  2.13);

\path[fill=fillColor,fill opacity=0.20] ( 70.22,102.96) circle (  2.13);

\path[fill=fillColor,fill opacity=0.20] ( 74.44, 91.58) circle (  2.13);

\path[fill=fillColor,fill opacity=0.20] ( 63.24,101.33) circle (  2.13);

\path[fill=fillColor,fill opacity=0.20] ( 66.29, 89.95) circle (  2.13);

\path[fill=fillColor,fill opacity=0.20] ( 70.12, 98.89) circle (  2.13);

\path[fill=fillColor,fill opacity=0.20] ( 75.43, 94.02) circle (  2.13);

\path[fill=fillColor,fill opacity=0.20] ( 73.46, 94.83) circle (  2.13);

\path[fill=fillColor,fill opacity=0.20] ( 74.44,108.64) circle (  2.13);

\path[fill=fillColor,fill opacity=0.20] ( 72.48,101.33) circle (  2.13);

\path[fill=fillColor,fill opacity=0.20] ( 72.48, 87.51) circle (  2.13);

\path[fill=fillColor,fill opacity=0.20] ( 76.41, 94.02) circle (  2.13);

\path[fill=fillColor,fill opacity=0.20] ( 87.22,115.96) circle (  2.13);

\path[fill=fillColor,fill opacity=0.20] ( 65.21, 98.89) circle (  2.13);

\path[fill=fillColor,fill opacity=0.20] ( 78.37, 89.95) circle (  2.13);

\path[fill=fillColor,fill opacity=0.20] ( 72.48,110.27) circle (  2.13);

\path[fill=fillColor,fill opacity=0.20] ( 64.81,102.14) circle (  2.13);

\path[fill=fillColor,fill opacity=0.20] ( 67.27,102.14) circle (  2.13);

\path[fill=fillColor,fill opacity=0.20] ( 72.48,109.46) circle (  2.13);

\path[fill=fillColor,fill opacity=0.20] ( 70.02,103.77) circle (  2.13);

\path[fill=fillColor,fill opacity=0.20] ( 67.07, 91.58) circle (  2.13);

\path[fill=fillColor,fill opacity=0.20] ( 71.99, 89.95) circle (  2.13);

\path[fill=fillColor,fill opacity=0.20] ( 89.18, 99.70) circle (  2.13);

\path[fill=fillColor,fill opacity=0.20] ( 72.18, 85.89) circle (  2.13);

\path[fill=fillColor,fill opacity=0.20] ( 79.36, 95.64) circle (  2.13);

\path[fill=fillColor,fill opacity=0.20] ( 69.14,105.39) circle (  2.13);

\path[fill=fillColor,fill opacity=0.20] ( 57.94,102.14) circle (  2.13);

\path[fill=fillColor,fill opacity=0.20] ( 60.59,103.77) circle (  2.13);

\path[fill=fillColor,fill opacity=0.20] ( 77.39, 97.27) circle (  2.13);

\path[fill=fillColor,fill opacity=0.20] ( 73.46, 92.39) circle (  2.13);

\path[fill=fillColor,fill opacity=0.20] ( 63.34, 93.20) circle (  2.13);

\path[fill=fillColor,fill opacity=0.20] ( 70.61, 85.08) circle (  2.13);

\path[fill=fillColor,fill opacity=0.20] ( 94.10, 84.26) circle (  2.13);

\path[fill=fillColor,fill opacity=0.20] (145.20, 59.88) circle (  2.13);

\path[fill=fillColor,fill opacity=0.20] ( 79.36, 88.33) circle (  2.13);

\path[fill=fillColor,fill opacity=0.20] ( 73.46, 83.45) circle (  2.13);

\path[fill=fillColor,fill opacity=0.20] ( 70.12, 92.39) circle (  2.13);

\path[fill=fillColor,fill opacity=0.20] ( 69.24,105.39) circle (  2.13);

\path[fill=fillColor,fill opacity=0.20] ( 69.14,108.64) circle (  2.13);

\path[fill=fillColor,fill opacity=0.20] ( 72.28, 94.02) circle (  2.13);

\path[fill=fillColor,fill opacity=0.20] ( 70.02, 85.08) circle (  2.13);

\path[fill=fillColor,fill opacity=0.20] ( 67.27, 88.33) circle (  2.13);

\path[fill=fillColor,fill opacity=0.20] ( 72.48, 83.45) circle (  2.13);

\path[fill=fillColor,fill opacity=0.20] ( 70.22,107.02) circle (  2.13);

\path[fill=fillColor,fill opacity=0.20] ( 93.11, 81.82) circle (  2.13);

\path[fill=fillColor,fill opacity=0.20] ( 97.05, 50.13) circle (  2.13);

\path[fill=fillColor,fill opacity=0.20] ( 98.03, 48.50) circle (  2.13);

\path[fill=fillColor,fill opacity=0.20] ( 92.13, 51.75) circle (  2.13);

\path[fill=fillColor,fill opacity=0.20] ( 85.25, 96.45) circle (  2.13);

\path[fill=fillColor,fill opacity=0.20] ( 78.37, 91.58) circle (  2.13);

\path[fill=fillColor,fill opacity=0.20] ( 74.44, 91.58) circle (  2.13);

\path[fill=fillColor,fill opacity=0.20] ( 74.44,102.96) circle (  2.13);

\path[fill=fillColor,fill opacity=0.20] ( 70.81,110.27) circle (  2.13);

\path[fill=fillColor,fill opacity=0.20] ( 65.99,104.58) circle (  2.13);

\path[fill=fillColor,fill opacity=0.20] ( 66.09, 94.02) circle (  2.13);

\path[fill=fillColor,fill opacity=0.20] ( 63.24, 83.45) circle (  2.13);

\path[fill=fillColor,fill opacity=0.20] ( 68.94, 81.82) circle (  2.13);

\path[fill=fillColor,fill opacity=0.20] ( 69.14,115.96) circle (  2.13);

\path[fill=fillColor,fill opacity=0.20] ( 76.41, 86.70) circle (  2.13);

\path[fill=fillColor,fill opacity=0.20] ( 73.46, 62.32) circle (  2.13);

\path[fill=fillColor,fill opacity=0.20] ( 74.44, 51.75) circle (  2.13);

\path[fill=fillColor,fill opacity=0.20] ( 86.24, 42.00) circle (  2.13);

\path[fill=fillColor,fill opacity=0.20] ( 98.03, 38.75) circle (  2.13);

\path[fill=fillColor,fill opacity=0.20] ( 94.10, 39.56) circle (  2.13);

\path[fill=fillColor,fill opacity=0.20] ( 98.03, 47.69) circle (  2.13);

\path[fill=fillColor,fill opacity=0.20] ( 99.99, 43.63) circle (  2.13);

\path[fill=fillColor,fill opacity=0.20] ( 85.25, 99.70) circle (  2.13);

\path[fill=fillColor,fill opacity=0.20] ( 82.31, 91.58) circle (  2.13);

\path[fill=fillColor,fill opacity=0.20] ( 78.37,101.33) circle (  2.13);

\path[fill=fillColor,fill opacity=0.20] ( 78.37, 97.27) circle (  2.13);

\path[fill=fillColor,fill opacity=0.20] ( 78.37, 90.76) circle (  2.13);

\path[fill=fillColor,fill opacity=0.20] ( 74.44, 94.02) circle (  2.13);

\path[fill=fillColor,fill opacity=0.20] ( 69.73,101.33) circle (  2.13);

\path[fill=fillColor,fill opacity=0.20] ( 63.63, 97.27) circle (  2.13);

\path[fill=fillColor,fill opacity=0.20] ( 63.14, 80.20) circle (  2.13);

\path[fill=fillColor,fill opacity=0.20] ( 79.36, 72.07) circle (  2.13);

\path[fill=fillColor,fill opacity=0.20] ( 76.41, 63.13) circle (  2.13);

\path[fill=fillColor,fill opacity=0.20] ( 78.37, 79.39) circle (  2.13);

\path[fill=fillColor,fill opacity=0.20] ( 81.32, 80.20) circle (  2.13);

\path[fill=fillColor,fill opacity=0.20] ( 83.29, 73.70) circle (  2.13);

\path[fill=fillColor,fill opacity=0.20] ( 86.24, 69.63) circle (  2.13);

\path[fill=fillColor,fill opacity=0.20] ( 99.99, 62.32) circle (  2.13);

\path[fill=fillColor,fill opacity=0.20] ( 94.10, 56.63) circle (  2.13);

\path[fill=fillColor,fill opacity=0.20] ( 93.11, 55.01) circle (  2.13);

\path[fill=fillColor,fill opacity=0.20] ( 76.41, 78.57) circle (  2.13);

\path[fill=fillColor,fill opacity=0.20] ( 77.39, 77.76) circle (  2.13);

\path[fill=fillColor,fill opacity=0.20] ( 73.46, 91.58) circle (  2.13);

\path[fill=fillColor,fill opacity=0.20] ( 72.18,102.96) circle (  2.13);

\path[fill=fillColor,fill opacity=0.20] ( 75.43, 89.95) circle (  2.13);

\path[fill=fillColor,fill opacity=0.20] ( 71.30, 85.08) circle (  2.13);

\path[fill=fillColor,fill opacity=0.20] ( 65.60, 94.83) circle (  2.13);

\path[fill=fillColor,fill opacity=0.20] ( 65.80, 90.76) circle (  2.13);

\path[fill=fillColor,fill opacity=0.20] ( 73.46, 76.95) circle (  2.13);

\path[fill=fillColor,fill opacity=0.20] ( 92.13, 66.38) circle (  2.13);

\path[fill=fillColor,fill opacity=0.20] ( 76.41, 98.08) circle (  2.13);

\path[fill=fillColor,fill opacity=0.20] ( 76.41, 78.57) circle (  2.13);

\path[fill=fillColor,fill opacity=0.20] ( 76.41, 96.45) circle (  2.13);

\path[fill=fillColor,fill opacity=0.20] ( 76.41,109.46) circle (  2.13);

\path[fill=fillColor,fill opacity=0.20] ( 81.32,100.52) circle (  2.13);

\path[fill=fillColor,fill opacity=0.20] ( 81.32, 96.45) circle (  2.13);

\path[fill=fillColor,fill opacity=0.20] ( 95.08, 91.58) circle (  2.13);

\path[fill=fillColor,fill opacity=0.20] ( 95.08, 72.89) circle (  2.13);

\path[fill=fillColor,fill opacity=0.20] ( 90.17, 60.69) circle (  2.13);

\path[fill=fillColor,fill opacity=0.20] ( 94.10, 57.44) circle (  2.13);

\path[fill=fillColor,fill opacity=0.20] ( 91.15, 95.64) circle (  2.13);

\path[fill=fillColor,fill opacity=0.20] ( 82.31, 76.14) circle (  2.13);

\path[fill=fillColor,fill opacity=0.20] ( 71.10, 85.89) circle (  2.13);

\path[fill=fillColor,fill opacity=0.20] ( 74.44, 81.82) circle (  2.13);

\path[fill=fillColor,fill opacity=0.20] ( 74.44, 92.39) circle (  2.13);

\path[fill=fillColor,fill opacity=0.20] ( 64.62,109.46) circle (  2.13);

\path[fill=fillColor,fill opacity=0.20] ( 68.25,107.83) circle (  2.13);

\path[fill=fillColor,fill opacity=0.20] ( 68.06,104.58) circle (  2.13);

\path[fill=fillColor,fill opacity=0.20] ( 64.22, 98.89) circle (  2.13);

\path[fill=fillColor,fill opacity=0.20] ( 66.88, 88.33) circle (  2.13);

\path[fill=fillColor,fill opacity=0.20] ( 88.20, 75.32) circle (  2.13);

\path[fill=fillColor,fill opacity=0.20] ( 83.29, 82.64) circle (  2.13);

\path[fill=fillColor,fill opacity=0.20] ( 74.44, 87.51) circle (  2.13);

\path[fill=fillColor,fill opacity=0.20] ( 60.49,112.71) circle (  2.13);

\path[fill=fillColor,fill opacity=0.20] ( 81.32,109.46) circle (  2.13);

\path[fill=fillColor,fill opacity=0.20] ( 79.36,101.33) circle (  2.13);

\path[fill=fillColor,fill opacity=0.20] ( 86.24,107.83) circle (  2.13);

\path[fill=fillColor,fill opacity=0.20] ( 92.13,100.52) circle (  2.13);

\path[fill=fillColor,fill opacity=0.20] ( 90.17, 81.82) circle (  2.13);

\path[fill=fillColor,fill opacity=0.20] ( 95.08, 70.45) circle (  2.13);

\path[fill=fillColor,fill opacity=0.20] ( 90.17,106.21) circle (  2.13);

\path[fill=fillColor,fill opacity=0.20] ( 87.22, 72.07) circle (  2.13);

\path[fill=fillColor,fill opacity=0.20] ( 77.39, 89.95) circle (  2.13);

\path[fill=fillColor,fill opacity=0.20] ( 59.90, 94.02) circle (  2.13);

\path[fill=fillColor,fill opacity=0.20] ( 70.61, 91.58) circle (  2.13);

\path[fill=fillColor,fill opacity=0.20] ( 72.48, 98.08) circle (  2.13);

\path[fill=fillColor,fill opacity=0.20] ( 63.54,101.33) circle (  2.13);

\path[fill=fillColor,fill opacity=0.20] ( 53.71,107.02) circle (  2.13);

\path[fill=fillColor,fill opacity=0.20] ( 64.13, 94.83) circle (  2.13);

\path[fill=fillColor,fill opacity=0.20] ( 72.48, 81.82) circle (  2.13);

\path[fill=fillColor,fill opacity=0.20] ( 72.48,107.83) circle (  2.13);

\path[fill=fillColor,fill opacity=0.20] ( 78.37, 84.26) circle (  2.13);

\path[fill=fillColor,fill opacity=0.20] ( 72.48, 93.20) circle (  2.13);

\path[fill=fillColor,fill opacity=0.20] ( 70.22,101.33) circle (  2.13);

\path[fill=fillColor,fill opacity=0.20] ( 83.29, 92.39) circle (  2.13);

\path[fill=fillColor,fill opacity=0.20] ( 85.25, 92.39) circle (  2.13);

\path[fill=fillColor,fill opacity=0.20] ( 88.20,105.39) circle (  2.13);

\path[fill=fillColor,fill opacity=0.20] ( 97.05, 98.08) circle (  2.13);

\path[fill=fillColor,fill opacity=0.20] (100.98, 84.26) circle (  2.13);

\path[fill=fillColor,fill opacity=0.20] (103.92, 76.95) circle (  2.13);

\path[fill=fillColor,fill opacity=0.20] ( 82.31,106.21) circle (  2.13);

\path[fill=fillColor,fill opacity=0.20] ( 92.13, 75.32) circle (  2.13);

\path[fill=fillColor,fill opacity=0.20] ( 85.25, 87.51) circle (  2.13);

\path[fill=fillColor,fill opacity=0.20] ( 75.43, 93.20) circle (  2.13);

\path[fill=fillColor,fill opacity=0.20] ( 72.48, 91.58) circle (  2.13);

\path[fill=fillColor,fill opacity=0.20] ( 73.46, 94.02) circle (  2.13);

\path[fill=fillColor,fill opacity=0.20] ( 68.35, 93.20) circle (  2.13);

\path[fill=fillColor,fill opacity=0.20] ( 66.68, 83.45) circle (  2.13);

\path[fill=fillColor,fill opacity=0.20] ( 51.06, 85.89) circle (  2.13);

\path[fill=fillColor,fill opacity=0.20] ( 61.47, 96.45) circle (  2.13);

\path[fill=fillColor,fill opacity=0.20] ( 64.91, 82.64) circle (  2.13);

\path[fill=fillColor,fill opacity=0.20] ( 90.17, 63.95) circle (  2.13);

\path[fill=fillColor,fill opacity=0.20] ( 93.11, 89.95) circle (  2.13);

\path[fill=fillColor,fill opacity=0.20] ( 71.89, 95.64) circle (  2.13);

\path[fill=fillColor,fill opacity=0.20] ( 67.07, 98.89) circle (  2.13);

\path[fill=fillColor,fill opacity=0.20] ( 80.34, 95.64) circle (  2.13);

\path[fill=fillColor,fill opacity=0.20] ( 84.27, 98.08) circle (  2.13);

\path[fill=fillColor,fill opacity=0.20] ( 89.18,101.33) circle (  2.13);

\path[fill=fillColor,fill opacity=0.20] ( 98.03, 94.83) circle (  2.13);

\path[fill=fillColor,fill opacity=0.20] (101.96, 83.45) circle (  2.13);

\path[fill=fillColor,fill opacity=0.20] (104.91, 70.45) circle (  2.13);

\path[fill=fillColor,fill opacity=0.20] ( 96.06,105.39) circle (  2.13);

\path[fill=fillColor,fill opacity=0.20] ( 86.24, 77.76) circle (  2.13);

\path[fill=fillColor,fill opacity=0.20] ( 84.27, 94.83) circle (  2.13);

\path[fill=fillColor,fill opacity=0.20] ( 81.32, 98.89) circle (  2.13);

\path[fill=fillColor,fill opacity=0.20] ( 78.37, 92.39) circle (  2.13);

\path[fill=fillColor,fill opacity=0.20] ( 76.41, 95.64) circle (  2.13);

\path[fill=fillColor,fill opacity=0.20] ( 74.44, 98.89) circle (  2.13);

\path[fill=fillColor,fill opacity=0.20] ( 69.24, 94.02) circle (  2.13);

\path[fill=fillColor,fill opacity=0.20] ( 70.81, 83.45) circle (  2.13);

\path[fill=fillColor,fill opacity=0.20] ( 70.42, 90.76) circle (  2.13);

\path[fill=fillColor,fill opacity=0.20] ( 81.32, 80.20) circle (  2.13);

\path[fill=fillColor,fill opacity=0.20] (104.91, 89.95) circle (  2.13);

\path[fill=fillColor,fill opacity=0.20] ( 82.31, 86.70) circle (  2.13);

\path[fill=fillColor,fill opacity=0.20] ( 65.50,100.52) circle (  2.13);

\path[fill=fillColor,fill opacity=0.20] ( 68.15,112.71) circle (  2.13);

\path[fill=fillColor,fill opacity=0.20] ( 83.29,111.08) circle (  2.13);

\path[fill=fillColor,fill opacity=0.20] ( 83.29,103.77) circle (  2.13);

\path[fill=fillColor,fill opacity=0.20] ( 91.15, 98.89) circle (  2.13);

\path[fill=fillColor,fill opacity=0.20] ( 90.17, 86.70) circle (  2.13);

\path[fill=fillColor,fill opacity=0.20] ( 98.03, 68.82) circle (  2.13);

\path[fill=fillColor,fill opacity=0.20] (116.70, 78.57) circle (  2.13);

\path[fill=fillColor,fill opacity=0.20] ( 99.99, 94.83) circle (  2.13);

\path[fill=fillColor,fill opacity=0.20] ( 93.11, 78.57) circle (  2.13);

\path[fill=fillColor,fill opacity=0.20] ( 85.25, 89.14) circle (  2.13);

\path[fill=fillColor,fill opacity=0.20] ( 78.37, 91.58) circle (  2.13);

\path[fill=fillColor,fill opacity=0.20] ( 73.46, 97.27) circle (  2.13);

\path[fill=fillColor,fill opacity=0.20] ( 71.59, 98.08) circle (  2.13);

\path[fill=fillColor,fill opacity=0.20] ( 80.34, 98.08) circle (  2.13);

\path[fill=fillColor,fill opacity=0.20] ( 76.41, 98.89) circle (  2.13);

\path[fill=fillColor,fill opacity=0.20] ( 70.32, 96.45) circle (  2.13);

\path[fill=fillColor,fill opacity=0.20] ( 74.44, 93.20) circle (  2.13);

\path[fill=fillColor,fill opacity=0.20] ( 77.39, 91.58) circle (  2.13);

\path[fill=fillColor,fill opacity=0.20] ( 82.31, 82.64) circle (  2.13);

\path[fill=fillColor,fill opacity=0.20] (100.98, 82.64) circle (  2.13);

\path[fill=fillColor,fill opacity=0.20] ( 76.41, 94.02) circle (  2.13);

\path[fill=fillColor,fill opacity=0.20] ( 74.44,106.21) circle (  2.13);

\path[fill=fillColor,fill opacity=0.20] ( 77.39,105.39) circle (  2.13);

\path[fill=fillColor,fill opacity=0.20] ( 77.39,106.21) circle (  2.13);

\path[fill=fillColor,fill opacity=0.20] ( 77.39,104.58) circle (  2.13);

\path[fill=fillColor,fill opacity=0.20] ( 82.31, 90.76) circle (  2.13);

\path[fill=fillColor,fill opacity=0.20] ( 91.15, 81.82) circle (  2.13);

\path[fill=fillColor,fill opacity=0.20] (101.96, 83.45) circle (  2.13);

\path[fill=fillColor,fill opacity=0.20] (105.89, 88.33) circle (  2.13);

\path[fill=fillColor,fill opacity=0.20] ( 95.08, 72.07) circle (  2.13);

\path[fill=fillColor,fill opacity=0.20] ( 79.36, 89.95) circle (  2.13);

\path[fill=fillColor,fill opacity=0.20] ( 81.32, 88.33) circle (  2.13);

\path[fill=fillColor,fill opacity=0.20] ( 82.31, 81.01) circle (  2.13);

\path[fill=fillColor,fill opacity=0.20] ( 72.48, 89.14) circle (  2.13);

\path[fill=fillColor,fill opacity=0.20] ( 71.50, 89.14) circle (  2.13);

\path[fill=fillColor,fill opacity=0.20] ( 77.39, 82.64) circle (  2.13);

\path[fill=fillColor,fill opacity=0.20] ( 73.46, 81.82) circle (  2.13);

\path[fill=fillColor,fill opacity=0.20] ( 79.36, 85.08) circle (  2.13);

\path[fill=fillColor,fill opacity=0.20] ( 88.20, 85.89) circle (  2.13);

\path[fill=fillColor,fill opacity=0.20] ( 95.08, 81.01) circle (  2.13);

\path[fill=fillColor,fill opacity=0.20] (116.70, 84.26) circle (  2.13);

\path[fill=fillColor,fill opacity=0.20] ( 87.22, 94.83) circle (  2.13);

\path[fill=fillColor,fill opacity=0.20] ( 71.89, 94.83) circle (  2.13);

\path[fill=fillColor,fill opacity=0.20] ( 78.37, 90.76) circle (  2.13);

\path[fill=fillColor,fill opacity=0.20] ( 76.41,103.77) circle (  2.13);

\path[fill=fillColor,fill opacity=0.20] ( 76.41,102.96) circle (  2.13);

\path[fill=fillColor,fill opacity=0.20] ( 81.32, 86.70) circle (  2.13);

\path[fill=fillColor,fill opacity=0.20] ( 86.24, 85.08) circle (  2.13);

\path[fill=fillColor,fill opacity=0.20] ( 93.11, 86.70) circle (  2.13);

\path[fill=fillColor,fill opacity=0.20] (107.86, 82.64) circle (  2.13);

\path[fill=fillColor,fill opacity=0.20] (120.63,102.14) circle (  2.13);

\path[fill=fillColor,fill opacity=0.20] (113.75,100.52) circle (  2.13);

\path[fill=fillColor,fill opacity=0.20] ( 98.03, 81.01) circle (  2.13);

\path[fill=fillColor,fill opacity=0.20] ( 93.11, 75.32) circle (  2.13);

\path[fill=fillColor,fill opacity=0.20] ( 86.24, 86.70) circle (  2.13);

\path[fill=fillColor,fill opacity=0.20] ( 79.36, 90.76) circle (  2.13);

\path[fill=fillColor,fill opacity=0.20] ( 75.43, 85.89) circle (  2.13);

\path[fill=fillColor,fill opacity=0.20] ( 74.44, 75.32) circle (  2.13);

\path[fill=fillColor,fill opacity=0.20] ( 75.43, 68.01) circle (  2.13);

\path[fill=fillColor,fill opacity=0.20] ( 74.44, 74.51) circle (  2.13);

\path[fill=fillColor,fill opacity=0.20] ( 92.13, 77.76) circle (  2.13);

\path[fill=fillColor,fill opacity=0.20] ( 93.11, 82.64) circle (  2.13);

\path[fill=fillColor,fill opacity=0.20] ( 87.22, 85.89) circle (  2.13);

\path[fill=fillColor,fill opacity=0.20] ( 95.08, 92.39) circle (  2.13);

\path[fill=fillColor,fill opacity=0.20] ( 71.89, 74.51) circle (  2.13);

\path[fill=fillColor,fill opacity=0.20] ( 78.37, 74.51) circle (  2.13);

\path[fill=fillColor,fill opacity=0.20] ( 71.89, 98.89) circle (  2.13);

\path[fill=fillColor,fill opacity=0.20] ( 75.43,103.77) circle (  2.13);

\path[fill=fillColor,fill opacity=0.20] ( 87.22, 88.33) circle (  2.13);

\path[fill=fillColor,fill opacity=0.20] ( 87.22, 86.70) circle (  2.13);

\path[fill=fillColor,fill opacity=0.20] ( 88.20, 91.58) circle (  2.13);

\path[fill=fillColor,fill opacity=0.20] ( 95.08, 82.64) circle (  2.13);

\path[fill=fillColor,fill opacity=0.20] ( 95.08, 76.95) circle (  2.13);

\path[fill=fillColor,fill opacity=0.20] (104.91, 95.64) circle (  2.13);

\path[fill=fillColor,fill opacity=0.20] ( 82.31, 85.08) circle (  2.13);

\path[fill=fillColor,fill opacity=0.20] ( 96.06, 71.26) circle (  2.13);

\path[fill=fillColor,fill opacity=0.20] ( 81.32, 80.20) circle (  2.13);

\path[fill=fillColor,fill opacity=0.20] ( 79.36, 91.58) circle (  2.13);

\path[fill=fillColor,fill opacity=0.20] ( 77.39, 81.01) circle (  2.13);

\path[fill=fillColor,fill opacity=0.20] ( 76.41, 70.45) circle (  2.13);

\path[fill=fillColor,fill opacity=0.20] ( 71.69, 64.76) circle (  2.13);

\path[fill=fillColor,fill opacity=0.20] ( 83.29, 55.82) circle (  2.13);

\path[fill=fillColor,fill opacity=0.20] ( 85.25, 63.95) circle (  2.13);

\path[fill=fillColor,fill opacity=0.20] ( 59.61, 90.76) circle (  2.13);

\path[fill=fillColor,fill opacity=0.20] ( 79.36, 94.02) circle (  2.13);

\path[fill=fillColor,fill opacity=0.20] ( 85.25, 93.20) circle (  2.13);

\path[fill=fillColor,fill opacity=0.20] ( 90.17, 89.14) circle (  2.13);

\path[fill=fillColor,fill opacity=0.20] ( 83.29, 84.26) circle (  2.13);

\path[fill=fillColor,fill opacity=0.20] ( 86.24, 76.14) circle (  2.13);

\path[fill=fillColor,fill opacity=0.20] ( 97.05, 69.63) circle (  2.13);

\path[fill=fillColor,fill opacity=0.20] ( 98.03, 89.95) circle (  2.13);

\path[fill=fillColor,fill opacity=0.20] (107.86, 77.76) circle (  2.13);

\path[fill=fillColor,fill opacity=0.20] ( 97.05, 76.95) circle (  2.13);

\path[fill=fillColor,fill opacity=0.20] ( 92.13, 66.38) circle (  2.13);

\path[fill=fillColor,fill opacity=0.20] ( 80.34, 66.38) circle (  2.13);

\path[fill=fillColor,fill opacity=0.20] ( 79.36, 81.82) circle (  2.13);

\path[fill=fillColor,fill opacity=0.20] ( 75.43, 85.89) circle (  2.13);

\path[fill=fillColor,fill opacity=0.20] ( 71.00, 74.51) circle (  2.13);

\path[fill=fillColor,fill opacity=0.20] ( 62.95, 66.38) circle (  2.13);

\path[fill=fillColor,fill opacity=0.20] ( 88.20, 59.07) circle (  2.13);

\path[fill=fillColor,fill opacity=0.20] ( 88.20, 79.39) circle (  2.13);

\path[fill=fillColor,fill opacity=0.20] ( 81.32, 81.01) circle (  2.13);

\path[fill=fillColor,fill opacity=0.20] ( 78.37, 84.26) circle (  2.13);

\path[fill=fillColor,fill opacity=0.20] ( 81.32, 91.58) circle (  2.13);

\path[fill=fillColor,fill opacity=0.20] ( 84.27, 92.39) circle (  2.13);

\path[fill=fillColor,fill opacity=0.20] ( 88.20, 92.39) circle (  2.13);

\path[fill=fillColor,fill opacity=0.20] ( 83.29, 91.58) circle (  2.13);

\path[fill=fillColor,fill opacity=0.20] ( 90.17, 76.95) circle (  2.13);

\path[fill=fillColor,fill opacity=0.20] ( 94.10, 70.45) circle (  2.13);

\path[fill=fillColor,fill opacity=0.20] (103.92, 85.08) circle (  2.13);

\path[fill=fillColor,fill opacity=0.20] ( 90.17, 55.01) circle (  2.13);

\path[fill=fillColor,fill opacity=0.20] ( 93.11, 57.44) circle (  2.13);

\path[fill=fillColor,fill opacity=0.20] ( 95.08, 61.51) circle (  2.13);

\path[fill=fillColor,fill opacity=0.20] ( 85.25, 56.63) circle (  2.13);

\path[fill=fillColor,fill opacity=0.20] ( 82.31, 57.44) circle (  2.13);

\path[fill=fillColor,fill opacity=0.20] ( 79.36, 63.13) circle (  2.13);

\path[fill=fillColor,fill opacity=0.20] ( 73.46, 71.26) circle (  2.13);

\path[fill=fillColor,fill opacity=0.20] ( 78.37, 70.45) circle (  2.13);

\path[fill=fillColor,fill opacity=0.20] ( 88.20, 68.82) circle (  2.13);

\path[fill=fillColor,fill opacity=0.20] ( 82.31, 73.70) circle (  2.13);

\path[fill=fillColor,fill opacity=0.20] ( 76.41, 82.64) circle (  2.13);

\path[fill=fillColor,fill opacity=0.20] ( 83.29, 89.95) circle (  2.13);

\path[fill=fillColor,fill opacity=0.20] ( 87.22, 94.02) circle (  2.13);

\path[fill=fillColor,fill opacity=0.20] ( 89.18, 87.51) circle (  2.13);

\path[fill=fillColor,fill opacity=0.20] ( 63.34, 69.63) circle (  2.13);

\path[fill=fillColor,fill opacity=0.20] ( 95.08, 65.57) circle (  2.13);

\path[fill=fillColor,fill opacity=0.20] ( 99.99, 76.95) circle (  2.13);

\path[fill=fillColor,fill opacity=0.20] (107.86, 87.51) circle (  2.13);

\path[fill=fillColor,fill opacity=0.20] (109.82, 89.95) circle (  2.13);

\path[fill=fillColor,fill opacity=0.20] (124.56, 97.27) circle (  2.13);

\path[fill=fillColor,fill opacity=0.20] ( 97.05, 90.76) circle (  2.13);

\path[fill=fillColor,fill opacity=0.20] ( 97.05, 59.88) circle (  2.13);

\path[fill=fillColor,fill opacity=0.20] ( 93.11, 49.32) circle (  2.13);

\path[fill=fillColor,fill opacity=0.20] ( 89.18, 63.95) circle (  2.13);

\path[fill=fillColor,fill opacity=0.20] ( 88.20, 63.13) circle (  2.13);

\path[fill=fillColor,fill opacity=0.20] ( 81.32, 61.51) circle (  2.13);

\path[fill=fillColor,fill opacity=0.20] ( 75.43, 72.89) circle (  2.13);

\path[fill=fillColor,fill opacity=0.20] ( 69.43, 74.51) circle (  2.13);

\path[fill=fillColor,fill opacity=0.20] ( 86.24, 62.32) circle (  2.13);

\path[fill=fillColor,fill opacity=0.20] ( 94.10, 68.82) circle (  2.13);

\path[fill=fillColor,fill opacity=0.20] ( 86.24, 62.32) circle (  2.13);

\path[fill=fillColor,fill opacity=0.20] ( 87.22, 66.38) circle (  2.13);

\path[fill=fillColor,fill opacity=0.20] ( 89.18, 82.64) circle (  2.13);

\path[fill=fillColor,fill opacity=0.20] ( 89.18, 83.45) circle (  2.13);

\path[fill=fillColor,fill opacity=0.20] ( 94.10, 69.63) circle (  2.13);

\path[fill=fillColor,fill opacity=0.20] ( 98.03, 70.45) circle (  2.13);

\path[fill=fillColor,fill opacity=0.20] ( 90.17, 77.76) circle (  2.13);

\path[fill=fillColor,fill opacity=0.20] ( 95.08, 77.76) circle (  2.13);

\path[fill=fillColor,fill opacity=0.20] (102.94, 76.95) circle (  2.13);

\path[fill=fillColor,fill opacity=0.20] (101.96, 72.89) circle (  2.13);

\path[fill=fillColor,fill opacity=0.20] (110.80, 80.20) circle (  2.13);

\path[fill=fillColor,fill opacity=0.20] (104.91,102.14) circle (  2.13);

\path[fill=fillColor,fill opacity=0.20] ( 99.01,113.52) circle (  2.13);

\path[fill=fillColor,fill opacity=0.20] (101.96, 89.95) circle (  2.13);

\path[fill=fillColor,fill opacity=0.20] ( 93.11, 79.39) circle (  2.13);

\path[fill=fillColor,fill opacity=0.20] ( 87.22, 82.64) circle (  2.13);

\path[fill=fillColor,fill opacity=0.20] ( 94.10, 68.82) circle (  2.13);

\path[fill=fillColor,fill opacity=0.20] ( 92.13, 51.75) circle (  2.13);

\path[fill=fillColor,fill opacity=0.20] ( 83.29, 61.51) circle (  2.13);

\path[fill=fillColor,fill opacity=0.20] ( 85.25, 69.63) circle (  2.13);

\path[fill=fillColor,fill opacity=0.20] ( 75.43, 68.01) circle (  2.13);

\path[fill=fillColor,fill opacity=0.20] ( 69.53, 72.07) circle (  2.13);

\path[fill=fillColor,fill opacity=0.20] ( 76.41, 67.20) circle (  2.13);

\path[fill=fillColor,fill opacity=0.20] ( 85.25, 63.95) circle (  2.13);

\path[fill=fillColor,fill opacity=0.20] ( 83.29, 74.51) circle (  2.13);

\path[fill=fillColor,fill opacity=0.20] ( 86.24, 80.20) circle (  2.13);

\path[fill=fillColor,fill opacity=0.20] ( 91.15, 80.20) circle (  2.13);

\path[fill=fillColor,fill opacity=0.20] ( 86.24, 83.45) circle (  2.13);

\path[fill=fillColor,fill opacity=0.20] ( 95.08, 81.82) circle (  2.13);

\path[fill=fillColor,fill opacity=0.20] (100.98, 76.14) circle (  2.13);

\path[fill=fillColor,fill opacity=0.20] ( 99.01, 75.32) circle (  2.13);

\path[fill=fillColor,fill opacity=0.20] ( 99.99, 72.89) circle (  2.13);

\path[fill=fillColor,fill opacity=0.20] (107.86, 68.82) circle (  2.13);

\path[fill=fillColor,fill opacity=0.20] (101.96, 73.70) circle (  2.13);

\path[fill=fillColor,fill opacity=0.20] ( 99.99, 86.70) circle (  2.13);

\path[fill=fillColor,fill opacity=0.20] ( 98.03,100.52) circle (  2.13);

\path[fill=fillColor,fill opacity=0.20] ( 86.24,110.27) circle (  2.13);

\path[fill=fillColor,fill opacity=0.20] ( 82.31,112.71) circle (  2.13);

\path[fill=fillColor,fill opacity=0.20] ( 53.61, 99.70) circle (  2.13);

\path[fill=fillColor,fill opacity=0.20] (112.77,102.14) circle (  2.13);

\path[fill=fillColor,fill opacity=0.20] (108.84,114.33) circle (  2.13);

\path[fill=fillColor,fill opacity=0.20] (104.91,109.46) circle (  2.13);

\path[fill=fillColor,fill opacity=0.20] ( 96.06, 89.95) circle (  2.13);

\path[fill=fillColor,fill opacity=0.20] (102.94, 78.57) circle (  2.13);

\path[fill=fillColor,fill opacity=0.20] ( 98.03, 84.26) circle (  2.13);

\path[fill=fillColor,fill opacity=0.20] ( 90.17, 86.70) circle (  2.13);

\path[fill=fillColor,fill opacity=0.20] ( 96.06, 65.57) circle (  2.13);

\path[fill=fillColor,fill opacity=0.20] (102.94, 41.19) circle (  2.13);

\path[fill=fillColor,fill opacity=0.20] (104.91, 45.25) circle (  2.13);

\path[fill=fillColor,fill opacity=0.20] ( 95.08, 54.19) circle (  2.13);

\path[fill=fillColor,fill opacity=0.20] ( 93.11, 55.01) circle (  2.13);

\path[fill=fillColor,fill opacity=0.20] ( 95.08, 68.01) circle (  2.13);

\path[fill=fillColor,fill opacity=0.20] ( 91.15, 77.76) circle (  2.13);

\path[fill=fillColor,fill opacity=0.20] ( 83.29, 68.01) circle (  2.13);

\path[fill=fillColor,fill opacity=0.20] ( 78.37, 57.44) circle (  2.13);

\path[fill=fillColor,fill opacity=0.20] ( 85.25, 58.26) circle (  2.13);

\path[fill=fillColor,fill opacity=0.20] ( 85.25, 69.63) circle (  2.13);

\path[fill=fillColor,fill opacity=0.20] ( 86.24, 76.14) circle (  2.13);

\path[fill=fillColor,fill opacity=0.20] ( 85.25, 72.89) circle (  2.13);

\path[fill=fillColor,fill opacity=0.20] ( 87.22, 80.20) circle (  2.13);

\path[fill=fillColor,fill opacity=0.20] ( 97.05, 85.89) circle (  2.13);

\path[fill=fillColor,fill opacity=0.20] ( 96.06, 79.39) circle (  2.13);

\path[fill=fillColor,fill opacity=0.20] ( 99.01, 81.82) circle (  2.13);

\path[fill=fillColor,fill opacity=0.20] (104.91, 79.39) circle (  2.13);

\path[fill=fillColor,fill opacity=0.20] (112.77, 69.63) circle (  2.13);

\path[fill=fillColor,fill opacity=0.20] ( 99.99, 72.07) circle (  2.13);

\path[fill=fillColor,fill opacity=0.20] ( 89.18, 74.51) circle (  2.13);

\path[fill=fillColor,fill opacity=0.20] ( 94.10, 71.26) circle (  2.13);

\path[fill=fillColor,fill opacity=0.20] ( 91.15, 70.45) circle (  2.13);

\path[fill=fillColor,fill opacity=0.20] ( 82.31, 73.70) circle (  2.13);

\path[fill=fillColor,fill opacity=0.20] ( 93.11, 75.32) circle (  2.13);

\path[fill=fillColor,fill opacity=0.20] ( 99.99, 74.51) circle (  2.13);

\path[fill=fillColor,fill opacity=0.20] (111.79, 67.20) circle (  2.13);

\path[fill=fillColor,fill opacity=0.20] (104.91, 64.76) circle (  2.13);

\path[fill=fillColor,fill opacity=0.20] ( 87.22, 68.82) circle (  2.13);

\path[fill=fillColor,fill opacity=0.20] ( 94.10, 70.45) circle (  2.13);

\path[fill=fillColor,fill opacity=0.20] ( 99.99, 68.01) circle (  2.13);

\path[fill=fillColor,fill opacity=0.20] ( 99.99, 68.01) circle (  2.13);

\path[fill=fillColor,fill opacity=0.20] ( 95.08, 67.20) circle (  2.13);

\path[fill=fillColor,fill opacity=0.20] (100.98, 62.32) circle (  2.13);

\path[fill=fillColor,fill opacity=0.20] (103.92, 57.44) circle (  2.13);

\path[fill=fillColor,fill opacity=0.20] ( 94.10, 62.32) circle (  2.13);

\path[fill=fillColor,fill opacity=0.20] ( 93.11, 68.82) circle (  2.13);

\path[fill=fillColor,fill opacity=0.20] (105.89, 66.38) circle (  2.13);

\path[fill=fillColor,fill opacity=0.20] (104.91, 65.57) circle (  2.13);

\path[fill=fillColor,fill opacity=0.20] (104.91, 66.38) circle (  2.13);

\path[fill=fillColor,fill opacity=0.20] ( 95.08, 65.57) circle (  2.13);

\path[fill=fillColor,fill opacity=0.20] ( 91.15, 75.32) circle (  2.13);

\path[fill=fillColor,fill opacity=0.20] ( 86.24, 83.45) circle (  2.13);

\path[fill=fillColor,fill opacity=0.20] ( 89.18, 66.38) circle (  2.13);

\path[fill=fillColor,fill opacity=0.20] ( 86.24, 49.32) circle (  2.13);

\path[fill=fillColor,fill opacity=0.20] ( 96.06, 59.88) circle (  2.13);

\path[fill=fillColor,fill opacity=0.20] ( 91.15, 58.26) circle (  2.13);

\path[fill=fillColor,fill opacity=0.20] ( 85.25, 68.82) circle (  2.13);

\path[fill=fillColor,fill opacity=0.20] ( 87.22, 76.14) circle (  2.13);

\path[fill=fillColor,fill opacity=0.20] ( 93.11, 82.64) circle (  2.13);

\path[fill=fillColor,fill opacity=0.20] ( 88.20, 86.70) circle (  2.13);

\path[fill=fillColor,fill opacity=0.20] ( 91.15, 80.20) circle (  2.13);

\path[fill=fillColor,fill opacity=0.20] ( 98.03, 76.95) circle (  2.13);

\path[fill=fillColor,fill opacity=0.20] (100.98, 85.08) circle (  2.13);

\path[fill=fillColor,fill opacity=0.20] (100.98, 84.26) circle (  2.13);

\path[fill=fillColor,fill opacity=0.20] (106.87, 80.20) circle (  2.13);

\path[fill=fillColor,fill opacity=0.20] ( 95.08, 76.95) circle (  2.13);

\path[fill=fillColor,fill opacity=0.20] ( 86.24, 72.07) circle (  2.13);

\path[fill=fillColor,fill opacity=0.20] ( 91.15, 79.39) circle (  2.13);

\path[fill=fillColor,fill opacity=0.20] ( 98.03, 88.33) circle (  2.13);

\path[fill=fillColor,fill opacity=0.20] (100.98, 79.39) circle (  2.13);

\path[fill=fillColor,fill opacity=0.20] ( 99.01, 68.82) circle (  2.13);

\path[fill=fillColor,fill opacity=0.20] ( 97.05, 69.63) circle (  2.13);

\path[fill=fillColor,fill opacity=0.20] (102.94, 67.20) circle (  2.13);

\path[fill=fillColor,fill opacity=0.20] ( 99.01, 60.69) circle (  2.13);

\path[fill=fillColor,fill opacity=0.20] ( 90.17, 64.76) circle (  2.13);

\path[fill=fillColor,fill opacity=0.20] ( 89.18, 69.63) circle (  2.13);

\path[fill=fillColor,fill opacity=0.20] ( 88.20, 70.45) circle (  2.13);

\path[fill=fillColor,fill opacity=0.20] (100.98, 76.14) circle (  2.13);

\path[fill=fillColor,fill opacity=0.20] (100.98, 85.89) circle (  2.13);

\path[fill=fillColor,fill opacity=0.20] ( 99.01, 88.33) circle (  2.13);

\path[fill=fillColor,fill opacity=0.20] ( 98.03, 88.33) circle (  2.13);

\path[fill=fillColor,fill opacity=0.20] ( 98.03, 88.33) circle (  2.13);

\path[fill=fillColor,fill opacity=0.20] ( 95.08, 79.39) circle (  2.13);

\path[fill=fillColor,fill opacity=0.20] ( 91.15, 71.26) circle (  2.13);

\path[fill=fillColor,fill opacity=0.20] ( 88.20, 76.95) circle (  2.13);

\path[fill=fillColor,fill opacity=0.20] ( 94.10, 71.26) circle (  2.13);

\path[fill=fillColor,fill opacity=0.20] ( 84.27, 71.26) circle (  2.13);

\path[fill=fillColor,fill opacity=0.20] ( 88.20, 74.51) circle (  2.13);

\path[fill=fillColor,fill opacity=0.20] ( 91.15, 71.26) circle (  2.13);

\path[fill=fillColor,fill opacity=0.20] ( 91.15, 72.07) circle (  2.13);

\path[fill=fillColor,fill opacity=0.20] ( 91.15, 76.14) circle (  2.13);

\path[fill=fillColor,fill opacity=0.20] ( 97.05, 81.01) circle (  2.13);

\path[fill=fillColor,fill opacity=0.20] ( 99.01, 81.01) circle (  2.13);

\path[fill=fillColor,fill opacity=0.20] ( 92.13, 76.14) circle (  2.13);

\path[fill=fillColor,fill opacity=0.20] ( 94.10, 70.45) circle (  2.13);

\path[fill=fillColor,fill opacity=0.20] ( 95.08, 73.70) circle (  2.13);

\path[fill=fillColor,fill opacity=0.20] ( 88.20, 83.45) circle (  2.13);

\path[fill=fillColor,fill opacity=0.20] ( 86.24, 81.82) circle (  2.13);

\path[fill=fillColor,fill opacity=0.20] ( 91.15, 76.95) circle (  2.13);

\path[fill=fillColor,fill opacity=0.20] ( 93.11, 74.51) circle (  2.13);

\path[fill=fillColor,fill opacity=0.20] ( 99.01, 69.63) circle (  2.13);

\path[fill=fillColor,fill opacity=0.20] ( 90.17, 65.57) circle (  2.13);

\path[fill=fillColor,fill opacity=0.20] ( 74.44, 73.70) circle (  2.13);

\path[fill=fillColor,fill opacity=0.20] ( 92.13, 81.82) circle (  2.13);

\path[fill=fillColor,fill opacity=0.20] ( 95.08, 78.57) circle (  2.13);

\path[fill=fillColor,fill opacity=0.20] ( 93.11, 76.95) circle (  2.13);

\path[fill=fillColor,fill opacity=0.20] ( 91.15, 81.01) circle (  2.13);

\path[fill=fillColor,fill opacity=0.20] ( 93.11, 81.82) circle (  2.13);

\path[fill=fillColor,fill opacity=0.20] ( 90.17, 79.39) circle (  2.13);

\path[fill=fillColor,fill opacity=0.20] ( 86.24, 73.70) circle (  2.13);

\path[fill=fillColor,fill opacity=0.20] ( 95.08, 65.57) circle (  2.13);

\path[fill=fillColor,fill opacity=0.20] ( 97.05, 59.07) circle (  2.13);

\path[fill=fillColor,fill opacity=0.20] (109.82, 62.32) circle (  2.13);

\path[fill=fillColor,fill opacity=0.20] ( 92.13, 64.76) circle (  2.13);

\path[fill=fillColor,fill opacity=0.20] ( 97.05, 66.38) circle (  2.13);

\path[fill=fillColor,fill opacity=0.20] ( 91.15, 62.32) circle (  2.13);

\path[fill=fillColor,fill opacity=0.20] ( 92.13, 60.69) circle (  2.13);

\path[fill=fillColor,fill opacity=0.20] ( 90.17, 56.63) circle (  2.13);

\path[fill=fillColor,fill opacity=0.20] ( 83.29, 48.50) circle (  2.13);

\path[fill=fillColor,fill opacity=0.20] ( 88.20, 50.13) circle (  2.13);

\path[fill=fillColor,fill opacity=0.20] ( 88.20, 58.26) circle (  2.13);

\path[fill=fillColor,fill opacity=0.20] ( 97.05, 63.13) circle (  2.13);

\path[fill=fillColor,fill opacity=0.20] ( 98.03, 60.69) circle (  2.13);

\path[fill=fillColor,fill opacity=0.20] (100.98, 60.69) circle (  2.13);

\path[fill=fillColor,fill opacity=0.20] (100.98, 63.95) circle (  2.13);

\path[fill=fillColor,fill opacity=0.20] ( 88.20, 65.57) circle (  2.13);

\path[fill=fillColor,fill opacity=0.20] (100.98, 66.38) circle (  2.13);

\path[fill=fillColor,fill opacity=0.20] (102.94, 68.01) circle (  2.13);

\path[fill=fillColor,fill opacity=0.20] ( 93.11, 65.57) circle (  2.13);

\path[fill=fillColor,fill opacity=0.20] ( 96.06, 59.07) circle (  2.13);

\path[fill=fillColor,fill opacity=0.20] (102.94, 57.44) circle (  2.13);

\path[fill=fillColor,fill opacity=0.20] (136.35, 99.70) circle (  2.13);

\path[fill=fillColor,fill opacity=0.20] (135.37, 98.08) circle (  2.13);

\path[fill=fillColor,fill opacity=0.20] (122.60,115.15) circle (  2.13);

\path[fill=fillColor,fill opacity=0.20] (130.46,108.64) circle (  2.13);

\path[fill=fillColor,fill opacity=0.20] (121.61,105.39) circle (  2.13);

\path[fill=fillColor,fill opacity=0.20] ( 75.43,106.21) circle (  2.13);

\path[fill=fillColor,fill opacity=0.20] (110.80, 97.27) circle (  2.13);

\path[fill=fillColor,fill opacity=0.20] (113.75, 86.70) circle (  2.13);

\path[fill=fillColor,fill opacity=0.20] ( 91.15, 89.95) circle (  2.13);

\path[fill=fillColor,fill opacity=0.20] (126.53, 89.95) circle (  2.13);

\path[fill=fillColor,fill opacity=0.20] (113.75, 82.64) circle (  2.13);

\path[fill=fillColor,fill opacity=0.20] (129.47, 81.82) circle (  2.13);

\path[fill=fillColor,fill opacity=0.20] (154.04,103.77) circle (  2.13);

\path[fill=fillColor,fill opacity=0.20] (110.80,111.89) circle (  2.13);

\path[fill=fillColor,fill opacity=0.20] ( 91.15,103.77) circle (  2.13);

\path[fill=fillColor,fill opacity=0.20] ( 96.06, 94.83) circle (  2.13);

\path[fill=fillColor,fill opacity=0.20] ( 90.17, 85.89) circle (  2.13);

\path[fill=fillColor,fill opacity=0.20] ( 94.10, 76.95) circle (  2.13);

\path[fill=fillColor,fill opacity=0.20] ( 69.33,105.39) circle (  2.13);

\path[fill=fillColor,fill opacity=0.20] ( 69.53,115.15) circle (  2.13);

\path[fill=fillColor,fill opacity=0.20] ( 67.07,102.96) circle (  2.13);

\path[fill=fillColor,fill opacity=0.20] ( 81.32, 91.58) circle (  2.13);

\path[fill=fillColor,fill opacity=0.20] ( 85.25, 72.89) circle (  2.13);

\path[fill=fillColor,fill opacity=0.20] ( 89.18, 61.51) circle (  2.13);

\path[fill=fillColor,fill opacity=0.20] ( 87.22, 65.57) circle (  2.13);

\path[fill=fillColor,fill opacity=0.20] ( 87.22, 61.51) circle (  2.13);

\path[fill=fillColor,fill opacity=0.20] ( 99.99, 96.45) circle (  2.13);

\path[fill=fillColor,fill opacity=0.20] ( 75.43, 99.70) circle (  2.13);

\path[fill=fillColor,fill opacity=0.20] ( 71.30,113.52) circle (  2.13);

\path[fill=fillColor,fill opacity=0.20] ( 55.48,109.46) circle (  2.13);

\path[fill=fillColor,fill opacity=0.20] ( 56.36,103.77) circle (  2.13);

\path[fill=fillColor,fill opacity=0.20] ( 77.39, 94.83) circle (  2.13);

\path[fill=fillColor,fill opacity=0.20] ( 79.36, 96.45) circle (  2.13);

\path[fill=fillColor,fill opacity=0.20] ( 83.29, 97.27) circle (  2.13);

\path[fill=fillColor,fill opacity=0.20] ( 89.18, 74.51) circle (  2.13);

\path[fill=fillColor,fill opacity=0.20] ( 81.32, 55.01) circle (  2.13);

\path[fill=fillColor,fill opacity=0.20] ( 94.10, 59.07) circle (  2.13);

\path[fill=fillColor,fill opacity=0.20] (100.98, 63.95) circle (  2.13);

\path[fill=fillColor,fill opacity=0.20] ( 95.08, 48.50) circle (  2.13);

\path[fill=fillColor,fill opacity=0.20] ( 87.22, 84.26) circle (  2.13);

\path[fill=fillColor,fill opacity=0.20] ( 56.66, 86.70) circle (  2.13);

\path[fill=fillColor,fill opacity=0.20] ( 68.74,102.14) circle (  2.13);

\path[fill=fillColor,fill opacity=0.20] ( 84.27,109.46) circle (  2.13);

\path[fill=fillColor,fill opacity=0.20] ( 84.27,108.64) circle (  2.13);

\path[fill=fillColor,fill opacity=0.20] ( 84.27, 96.45) circle (  2.13);

\path[fill=fillColor,fill opacity=0.20] ( 84.27, 89.95) circle (  2.13);

\path[fill=fillColor,fill opacity=0.20] ( 88.20, 95.64) circle (  2.13);

\path[fill=fillColor,fill opacity=0.20] ( 91.15, 81.82) circle (  2.13);

\path[fill=fillColor,fill opacity=0.20] ( 95.08, 57.44) circle (  2.13);

\path[fill=fillColor,fill opacity=0.20] ( 95.08, 71.26) circle (  2.13);

\path[fill=fillColor,fill opacity=0.20] ( 86.24, 72.89) circle (  2.13);

\path[fill=fillColor,fill opacity=0.20] ( 75.43, 81.01) circle (  2.13);

\path[fill=fillColor,fill opacity=0.20] ( 70.61, 75.32) circle (  2.13);

\path[fill=fillColor,fill opacity=0.20] ( 61.08, 56.63) circle (  2.13);

\path[fill=fillColor,fill opacity=0.20] ( 74.44, 76.14) circle (  2.13);

\path[fill=fillColor,fill opacity=0.20] ( 79.36, 74.51) circle (  2.13);

\path[fill=fillColor,fill opacity=0.20] ( 67.96, 94.83) circle (  2.13);

\path[fill=fillColor,fill opacity=0.20] ( 81.32,106.21) circle (  2.13);

\path[fill=fillColor,fill opacity=0.20] ( 84.27,108.64) circle (  2.13);

\path[fill=fillColor,fill opacity=0.20] ( 90.17, 96.45) circle (  2.13);

\path[fill=fillColor,fill opacity=0.20] ( 86.24, 85.08) circle (  2.13);

\path[fill=fillColor,fill opacity=0.20] ( 89.18, 85.89) circle (  2.13);

\path[fill=fillColor,fill opacity=0.20] ( 93.11, 68.82) circle (  2.13);

\path[fill=fillColor,fill opacity=0.20] ( 79.36, 61.51) circle (  2.13);

\path[fill=fillColor,fill opacity=0.20] ( 84.27, 62.32) circle (  2.13);

\path[fill=fillColor,fill opacity=0.20] ( 83.29, 77.76) circle (  2.13);

\path[fill=fillColor,fill opacity=0.20] ( 64.22, 89.14) circle (  2.13);

\path[fill=fillColor,fill opacity=0.20] ( 63.24, 99.70) circle (  2.13);

\path[fill=fillColor,fill opacity=0.20] ( 57.64,101.33) circle (  2.13);

\path[fill=fillColor,fill opacity=0.20] ( 61.08, 91.58) circle (  2.13);

\path[fill=fillColor,fill opacity=0.20] ( 83.29, 73.70) circle (  2.13);

\path[fill=fillColor,fill opacity=0.20] ( 80.34, 65.57) circle (  2.13);

\path[fill=fillColor,fill opacity=0.20] ( 83.29, 85.89) circle (  2.13);

\path[fill=fillColor,fill opacity=0.20] ( 80.34,102.96) circle (  2.13);

\path[fill=fillColor,fill opacity=0.20] ( 87.22, 94.83) circle (  2.13);

\path[fill=fillColor,fill opacity=0.20] ( 80.34, 85.89) circle (  2.13);

\path[fill=fillColor,fill opacity=0.20] ( 87.22, 81.82) circle (  2.13);

\path[fill=fillColor,fill opacity=0.20] ( 95.08, 77.76) circle (  2.13);

\path[fill=fillColor,fill opacity=0.20] (113.75, 68.82) circle (  2.13);

\path[fill=fillColor,fill opacity=0.20] ( 91.15, 68.01) circle (  2.13);

\path[fill=fillColor,fill opacity=0.20] ( 77.39, 83.45) circle (  2.13);

\path[fill=fillColor,fill opacity=0.20] ( 74.44, 91.58) circle (  2.13);

\path[fill=fillColor,fill opacity=0.20] ( 71.79, 94.83) circle (  2.13);

\path[fill=fillColor,fill opacity=0.20] ( 64.32,107.02) circle (  2.13);

\path[fill=fillColor,fill opacity=0.20] ( 65.99,108.64) circle (  2.13);

\path[fill=fillColor,fill opacity=0.20] ( 76.41, 72.89) circle (  2.13);

\path[fill=fillColor,fill opacity=0.20] ( 73.46, 71.26) circle (  2.13);

\path[fill=fillColor,fill opacity=0.20] ( 63.93, 55.82) circle (  2.13);

\path[fill=fillColor,fill opacity=0.20] ( 89.18, 80.20) circle (  2.13);

\path[fill=fillColor,fill opacity=0.20] ( 90.17, 97.27) circle (  2.13);

\path[fill=fillColor,fill opacity=0.20] ( 91.15, 82.64) circle (  2.13);

\path[fill=fillColor,fill opacity=0.20] ( 89.18, 76.14) circle (  2.13);

\path[fill=fillColor,fill opacity=0.20] ( 95.08, 85.08) circle (  2.13);

\path[fill=fillColor,fill opacity=0.20] ( 71.59, 83.45) circle (  2.13);

\path[fill=fillColor,fill opacity=0.20] (102.94, 74.51) circle (  2.13);

\path[fill=fillColor,fill opacity=0.20] ( 83.29, 72.89) circle (  2.13);

\path[fill=fillColor,fill opacity=0.20] ( 74.44, 81.82) circle (  2.13);

\path[fill=fillColor,fill opacity=0.20] ( 81.32, 87.51) circle (  2.13);

\path[fill=fillColor,fill opacity=0.20] ( 80.34, 94.02) circle (  2.13);

\path[fill=fillColor,fill opacity=0.20] ( 72.38,108.64) circle (  2.13);

\path[fill=fillColor,fill opacity=0.20] ( 69.53,115.15) circle (  2.13);

\path[fill=fillColor,fill opacity=0.20] ( 65.11, 98.89) circle (  2.13);

\path[fill=fillColor,fill opacity=0.20] ( 69.14, 76.14) circle (  2.13);

\path[fill=fillColor,fill opacity=0.20] ( 87.22, 55.01) circle (  2.13);

\path[fill=fillColor,fill opacity=0.20] ( 72.18, 49.32) circle (  2.13);

\path[fill=fillColor,fill opacity=0.20] ( 87.22, 73.70) circle (  2.13);

\path[fill=fillColor,fill opacity=0.20] ( 95.08, 89.14) circle (  2.13);

\path[fill=fillColor,fill opacity=0.20] ( 97.05, 76.14) circle (  2.13);

\path[fill=fillColor,fill opacity=0.20] ( 99.99, 72.07) circle (  2.13);

\path[fill=fillColor,fill opacity=0.20] ( 99.99, 81.01) circle (  2.13);

\path[fill=fillColor,fill opacity=0.20] (102.94, 81.01) circle (  2.13);

\path[fill=fillColor,fill opacity=0.20] (103.92, 67.20) circle (  2.13);

\path[fill=fillColor,fill opacity=0.20] (104.91, 59.07) circle (  2.13);

\path[fill=fillColor,fill opacity=0.20] ( 98.03, 84.26) circle (  2.13);

\path[fill=fillColor,fill opacity=0.20] ( 76.41, 60.69) circle (  2.13);

\path[fill=fillColor,fill opacity=0.20] ( 79.36, 78.57) circle (  2.13);

\path[fill=fillColor,fill opacity=0.20] ( 84.27, 81.01) circle (  2.13);

\path[fill=fillColor,fill opacity=0.20] ( 81.32, 85.89) circle (  2.13);

\path[fill=fillColor,fill opacity=0.20] ( 78.37,105.39) circle (  2.13);

\path[fill=fillColor,fill opacity=0.20] ( 63.73,105.39) circle (  2.13);

\path[fill=fillColor,fill opacity=0.20] ( 62.75, 93.20) circle (  2.13);

\path[fill=fillColor,fill opacity=0.20] ( 50.96,105.39) circle (  2.13);

\path[fill=fillColor,fill opacity=0.20] ( 56.46,101.33) circle (  2.13);

\path[fill=fillColor,fill opacity=0.20] ( 90.17, 63.13) circle (  2.13);

\path[fill=fillColor,fill opacity=0.20] ( 76.41, 55.01) circle (  2.13);

\path[fill=fillColor,fill opacity=0.20] ( 86.24, 67.20) circle (  2.13);

\path[fill=fillColor,fill opacity=0.20] ( 90.17, 81.82) circle (  2.13);

\path[fill=fillColor,fill opacity=0.20] ( 92.13, 81.82) circle (  2.13);

\path[fill=fillColor,fill opacity=0.20] ( 89.18, 78.57) circle (  2.13);

\path[fill=fillColor,fill opacity=0.20] ( 97.05, 76.14) circle (  2.13);

\path[fill=fillColor,fill opacity=0.20] ( 96.06, 73.70) circle (  2.13);

\path[fill=fillColor,fill opacity=0.20] ( 98.03, 65.57) circle (  2.13);

\path[fill=fillColor,fill opacity=0.20] ( 88.20, 64.76) circle (  2.13);

\path[fill=fillColor,fill opacity=0.20] (107.86, 71.26) circle (  2.13);

\path[fill=fillColor,fill opacity=0.20] (110.80, 89.95) circle (  2.13);

\path[fill=fillColor,fill opacity=0.20] ( 99.01, 74.51) circle (  2.13);

\path[fill=fillColor,fill opacity=0.20] ( 80.34, 59.07) circle (  2.13);

\path[fill=fillColor,fill opacity=0.20] ( 82.31, 81.01) circle (  2.13);

\path[fill=fillColor,fill opacity=0.20] ( 77.39, 84.26) circle (  2.13);

\path[fill=fillColor,fill opacity=0.20] ( 77.39, 89.14) circle (  2.13);

\path[fill=fillColor,fill opacity=0.20] ( 81.32,101.33) circle (  2.13);

\path[fill=fillColor,fill opacity=0.20] ( 75.43,102.96) circle (  2.13);

\path[fill=fillColor,fill opacity=0.20] ( 77.39, 95.64) circle (  2.13);

\path[fill=fillColor,fill opacity=0.20] ( 75.43, 98.89) circle (  2.13);

\path[fill=fillColor,fill opacity=0.20] ( 70.91,105.39) circle (  2.13);

\path[fill=fillColor,fill opacity=0.20] ( 61.08,107.02) circle (  2.13);

\path[fill=fillColor,fill opacity=0.20] ( 64.62,109.46) circle (  2.13);

\path[fill=fillColor,fill opacity=0.20] ( 71.30, 93.20) circle (  2.13);

\path[fill=fillColor,fill opacity=0.20] ( 87.22, 54.19) circle (  2.13);

\path[fill=fillColor,fill opacity=0.20] ( 76.41, 62.32) circle (  2.13);

\path[fill=fillColor,fill opacity=0.20] ( 77.39, 55.82) circle (  2.13);

\path[fill=fillColor,fill opacity=0.20] ( 86.24, 68.01) circle (  2.13);

\path[fill=fillColor,fill opacity=0.20] ( 85.25, 82.64) circle (  2.13);

\path[fill=fillColor,fill opacity=0.20] ( 95.08, 85.89) circle (  2.13);

\path[fill=fillColor,fill opacity=0.20] ( 99.01, 80.20) circle (  2.13);

\path[fill=fillColor,fill opacity=0.20] ( 95.08, 68.82) circle (  2.13);

\path[fill=fillColor,fill opacity=0.20] ( 99.01, 72.07) circle (  2.13);

\path[fill=fillColor,fill opacity=0.20] (103.92, 83.45) circle (  2.13);

\path[fill=fillColor,fill opacity=0.20] (102.94, 82.64) circle (  2.13);

\path[fill=fillColor,fill opacity=0.20] (109.82, 79.39) circle (  2.13);

\path[fill=fillColor,fill opacity=0.20] (110.80, 92.39) circle (  2.13);

\path[fill=fillColor,fill opacity=0.20] ( 88.20, 73.70) circle (  2.13);

\path[fill=fillColor,fill opacity=0.20] ( 77.39, 75.32) circle (  2.13);

\path[fill=fillColor,fill opacity=0.20] ( 77.39, 81.01) circle (  2.13);

\path[fill=fillColor,fill opacity=0.20] ( 82.31, 94.83) circle (  2.13);

\path[fill=fillColor,fill opacity=0.20] ( 81.32, 96.45) circle (  2.13);

\path[fill=fillColor,fill opacity=0.20] ( 78.37, 90.76) circle (  2.13);

\path[fill=fillColor,fill opacity=0.20] ( 75.43, 98.08) circle (  2.13);

\path[fill=fillColor,fill opacity=0.20] ( 74.44,111.08) circle (  2.13);

\path[fill=fillColor,fill opacity=0.20] ( 75.43,107.02) circle (  2.13);

\path[fill=fillColor,fill opacity=0.20] ( 68.35,100.52) circle (  2.13);

\path[fill=fillColor,fill opacity=0.20] ( 68.35,111.89) circle (  2.13);

\path[fill=fillColor,fill opacity=0.20] ( 72.48,111.89) circle (  2.13);

\path[fill=fillColor,fill opacity=0.20] ( 69.92, 84.26) circle (  2.13);

\path[fill=fillColor,fill opacity=0.20] ( 57.44, 55.01) circle (  2.13);

\path[fill=fillColor,fill opacity=0.20] ( 75.43, 49.32) circle (  2.13);

\path[fill=fillColor,fill opacity=0.20] ( 84.27, 45.25) circle (  2.13);

\path[fill=fillColor,fill opacity=0.20] ( 92.13, 59.07) circle (  2.13);

\path[fill=fillColor,fill opacity=0.20] ( 98.03, 74.51) circle (  2.13);

\path[fill=fillColor,fill opacity=0.20] (102.94, 72.07) circle (  2.13);

\path[fill=fillColor,fill opacity=0.20] (100.98, 68.82) circle (  2.13);

\path[fill=fillColor,fill opacity=0.20] (102.94, 81.01) circle (  2.13);

\path[fill=fillColor,fill opacity=0.20] (100.98, 92.39) circle (  2.13);

\path[fill=fillColor,fill opacity=0.20] ( 95.08, 88.33) circle (  2.13);

\path[fill=fillColor,fill opacity=0.20] ( 97.05, 84.26) circle (  2.13);

\path[fill=fillColor,fill opacity=0.20] (104.91, 83.45) circle (  2.13);

\path[fill=fillColor,fill opacity=0.20] (106.87, 88.33) circle (  2.13);

\path[fill=fillColor,fill opacity=0.20] ( 77.39,102.14) circle (  2.13);

\path[fill=fillColor,fill opacity=0.20] ( 92.13, 90.76) circle (  2.13);

\path[fill=fillColor,fill opacity=0.20] ( 74.44, 85.08) circle (  2.13);

\path[fill=fillColor,fill opacity=0.20] ( 80.34, 85.08) circle (  2.13);

\path[fill=fillColor,fill opacity=0.20] ( 80.34, 92.39) circle (  2.13);

\path[fill=fillColor,fill opacity=0.20] ( 80.34, 93.20) circle (  2.13);

\path[fill=fillColor,fill opacity=0.20] ( 70.12, 99.70) circle (  2.13);

\path[fill=fillColor,fill opacity=0.20] ( 65.01,115.15) circle (  2.13);

\path[fill=fillColor,fill opacity=0.20] ( 76.41,113.52) circle (  2.13);

\path[fill=fillColor,fill opacity=0.20] ( 76.41, 98.89) circle (  2.13);

\path[fill=fillColor,fill opacity=0.20] ( 68.94, 99.70) circle (  2.13);

\path[fill=fillColor,fill opacity=0.20] ( 72.48,115.96) circle (  2.13);

\path[fill=fillColor,fill opacity=0.20] ( 81.32,105.39) circle (  2.13);

\path[fill=fillColor,fill opacity=0.20] ( 62.85, 43.63) circle (  2.13);

\path[fill=fillColor,fill opacity=0.20] ( 96.06, 48.50) circle (  2.13);

\path[fill=fillColor,fill opacity=0.20] ( 94.10, 52.57) circle (  2.13);

\path[fill=fillColor,fill opacity=0.20] ( 94.10, 61.51) circle (  2.13);

\path[fill=fillColor,fill opacity=0.20] (100.98, 76.95) circle (  2.13);

\path[fill=fillColor,fill opacity=0.20] ( 96.06, 81.01) circle (  2.13);

\path[fill=fillColor,fill opacity=0.20] ( 80.34, 72.07) circle (  2.13);

\path[fill=fillColor,fill opacity=0.20] ( 95.08, 76.95) circle (  2.13);

\path[fill=fillColor,fill opacity=0.20] (101.96, 81.82) circle (  2.13);

\path[fill=fillColor,fill opacity=0.20] (106.87, 76.14) circle (  2.13);

\path[fill=fillColor,fill opacity=0.20] (105.89, 85.89) circle (  2.13);

\path[fill=fillColor,fill opacity=0.20] ( 86.24,102.96) circle (  2.13);

\path[fill=fillColor,fill opacity=0.20] ( 89.18, 89.95) circle (  2.13);

\path[fill=fillColor,fill opacity=0.20] ( 82.31, 98.89) circle (  2.13);

\path[fill=fillColor,fill opacity=0.20] ( 80.34, 97.27) circle (  2.13);

\path[fill=fillColor,fill opacity=0.20] ( 76.41, 87.51) circle (  2.13);

\path[fill=fillColor,fill opacity=0.20] ( 73.46, 92.39) circle (  2.13);

\path[fill=fillColor,fill opacity=0.20] ( 70.91,107.02) circle (  2.13);

\path[fill=fillColor,fill opacity=0.20] ( 73.46,114.33) circle (  2.13);

\path[fill=fillColor,fill opacity=0.20] ( 75.43,106.21) circle (  2.13);

\path[fill=fillColor,fill opacity=0.20] ( 74.44,101.33) circle (  2.13);

\path[fill=fillColor,fill opacity=0.20] ( 72.38,105.39) circle (  2.13);

\path[fill=fillColor,fill opacity=0.20] ( 71.59,108.64) circle (  2.13);

\path[fill=fillColor,fill opacity=0.20] ( 75.43,111.08) circle (  2.13);

\path[fill=fillColor,fill opacity=0.20] ( 66.58, 41.19) circle (  2.13);

\path[fill=fillColor,fill opacity=0.20] ( 90.17, 40.38) circle (  2.13);

\path[fill=fillColor,fill opacity=0.20] ( 93.11, 45.25) circle (  2.13);

\path[fill=fillColor,fill opacity=0.20] ( 93.11, 53.38) circle (  2.13);

\path[fill=fillColor,fill opacity=0.20] ( 96.06, 56.63) circle (  2.13);

\path[fill=fillColor,fill opacity=0.20] ( 90.17, 59.07) circle (  2.13);

\path[fill=fillColor,fill opacity=0.20] ( 86.24, 59.07) circle (  2.13);

\path[fill=fillColor,fill opacity=0.20] ( 85.25, 66.38) circle (  2.13);

\path[fill=fillColor,fill opacity=0.20] ( 87.22, 71.26) circle (  2.13);

\path[fill=fillColor,fill opacity=0.20] (100.98, 69.63) circle (  2.13);

\path[fill=fillColor,fill opacity=0.20] (100.98, 72.07) circle (  2.13);

\path[fill=fillColor,fill opacity=0.20] (102.94, 74.51) circle (  2.13);

\path[fill=fillColor,fill opacity=0.20] (112.77, 71.26) circle (  2.13);

\path[fill=fillColor,fill opacity=0.20] (120.63, 80.20) circle (  2.13);

\path[fill=fillColor,fill opacity=0.20] ( 78.37,108.64) circle (  2.13);

\path[fill=fillColor,fill opacity=0.20] ( 85.25,103.77) circle (  2.13);

\path[fill=fillColor,fill opacity=0.20] ( 67.47, 97.27) circle (  2.13);

\path[fill=fillColor,fill opacity=0.20] ( 90.17,104.58) circle (  2.13);

\path[fill=fillColor,fill opacity=0.20] ( 81.32, 90.76) circle (  2.13);

\path[fill=fillColor,fill opacity=0.20] ( 65.70, 77.76) circle (  2.13);

\path[fill=fillColor,fill opacity=0.20] ( 77.39, 79.39) circle (  2.13);

\path[fill=fillColor,fill opacity=0.20] ( 81.32, 93.20) circle (  2.13);

\path[fill=fillColor,fill opacity=0.20] ( 79.36, 92.39) circle (  2.13);

\path[fill=fillColor,fill opacity=0.20] ( 81.32, 81.01) circle (  2.13);

\path[fill=fillColor,fill opacity=0.20] ( 77.39, 83.45) circle (  2.13);

\path[fill=fillColor,fill opacity=0.20] ( 77.39, 98.08) circle (  2.13);

\path[fill=fillColor,fill opacity=0.20] ( 73.46, 98.89) circle (  2.13);

\path[fill=fillColor,fill opacity=0.20] ( 74.44, 91.58) circle (  2.13);

\path[fill=fillColor,fill opacity=0.20] ( 75.43,104.58) circle (  2.13);

\path[fill=fillColor,fill opacity=0.20] ( 71.00,110.27) circle (  2.13);

\path[fill=fillColor,fill opacity=0.20] ( 74.44, 90.76) circle (  2.13);

\path[fill=fillColor,fill opacity=0.20] ( 81.32, 96.45) circle (  2.13);

\path[fill=fillColor,fill opacity=0.20] ( 86.24, 49.32) circle (  2.13);

\path[fill=fillColor,fill opacity=0.20] ( 86.24, 50.94) circle (  2.13);

\path[fill=fillColor,fill opacity=0.20] ( 87.22, 49.32) circle (  2.13);

\path[fill=fillColor,fill opacity=0.20] ( 93.11, 54.19) circle (  2.13);

\path[fill=fillColor,fill opacity=0.20] ( 90.17, 59.88) circle (  2.13);

\path[fill=fillColor,fill opacity=0.20] (102.94, 56.63) circle (  2.13);

\path[fill=fillColor,fill opacity=0.20] ( 96.06, 54.19) circle (  2.13);

\path[fill=fillColor,fill opacity=0.20] ( 99.01, 57.44) circle (  2.13);

\path[fill=fillColor,fill opacity=0.20] (103.92, 64.76) circle (  2.13);

\path[fill=fillColor,fill opacity=0.20] (108.84, 60.69) circle (  2.13);

\path[fill=fillColor,fill opacity=0.20] (100.98, 64.76) circle (  2.13);

\path[fill=fillColor,fill opacity=0.20] ( 99.99, 72.89) circle (  2.13);

\path[fill=fillColor,fill opacity=0.20] ( 78.37, 73.70) circle (  2.13);

\path[fill=fillColor,fill opacity=0.20] ( 98.03, 75.32) circle (  2.13);

\path[fill=fillColor,fill opacity=0.20] (100.98, 84.26) circle (  2.13);

\path[fill=fillColor,fill opacity=0.20] ( 62.55, 83.45) circle (  2.13);

\path[fill=fillColor,fill opacity=0.20] ( 91.15, 81.82) circle (  2.13);

\path[fill=fillColor,fill opacity=0.20] ( 87.22, 72.89) circle (  2.13);

\path[fill=fillColor,fill opacity=0.20] ( 90.17, 79.39) circle (  2.13);

\path[fill=fillColor,fill opacity=0.20] ( 85.25, 89.95) circle (  2.13);

\path[fill=fillColor,fill opacity=0.20] ( 84.27, 81.82) circle (  2.13);

\path[fill=fillColor,fill opacity=0.20] ( 84.27, 76.95) circle (  2.13);

\path[fill=fillColor,fill opacity=0.20] ( 79.36, 76.14) circle (  2.13);

\path[fill=fillColor,fill opacity=0.20] ( 83.29, 72.07) circle (  2.13);

\path[fill=fillColor,fill opacity=0.20] ( 81.32, 79.39) circle (  2.13);

\path[fill=fillColor,fill opacity=0.20] ( 92.13, 77.76) circle (  2.13);

\path[fill=fillColor,fill opacity=0.20] ( 83.29, 78.57) circle (  2.13);

\path[fill=fillColor,fill opacity=0.20] ( 82.31, 89.14) circle (  2.13);

\path[fill=fillColor,fill opacity=0.20] ( 77.39, 94.83) circle (  2.13);

\path[fill=fillColor,fill opacity=0.20] ( 76.41, 85.89) circle (  2.13);

\path[fill=fillColor,fill opacity=0.20] ( 74.44, 89.14) circle (  2.13);

\path[fill=fillColor,fill opacity=0.20] ( 73.46, 96.45) circle (  2.13);

\path[fill=fillColor,fill opacity=0.20] ( 72.48, 81.82) circle (  2.13);

\path[fill=fillColor,fill opacity=0.20] ( 73.46, 82.64) circle (  2.13);

\path[fill=fillColor,fill opacity=0.20] ( 71.30,115.96) circle (  2.13);

\path[fill=fillColor,fill opacity=0.20] ( 91.15, 52.57) circle (  2.13);

\path[fill=fillColor,fill opacity=0.20] ( 99.99, 50.94) circle (  2.13);

\path[fill=fillColor,fill opacity=0.20] (100.98, 46.88) circle (  2.13);

\path[fill=fillColor,fill opacity=0.20] (100.98, 46.07) circle (  2.13);

\path[fill=fillColor,fill opacity=0.20] ( 99.99, 55.01) circle (  2.13);

\path[fill=fillColor,fill opacity=0.20] ( 96.06, 63.13) circle (  2.13);

\path[fill=fillColor,fill opacity=0.20] ( 91.15, 61.51) circle (  2.13);

\path[fill=fillColor,fill opacity=0.20] ( 96.06, 58.26) circle (  2.13);

\path[fill=fillColor,fill opacity=0.20] (102.94, 63.95) circle (  2.13);

\path[fill=fillColor,fill opacity=0.20] ( 82.31, 68.01) circle (  2.13);

\path[fill=fillColor,fill opacity=0.20] ( 99.01, 69.63) circle (  2.13);

\path[fill=fillColor,fill opacity=0.20] (100.98, 76.14) circle (  2.13);

\path[fill=fillColor,fill opacity=0.20] (120.63, 79.39) circle (  2.13);

\path[fill=fillColor,fill opacity=0.20] (100.98, 79.39) circle (  2.13);

\path[fill=fillColor,fill opacity=0.20] ( 93.11, 79.39) circle (  2.13);

\path[fill=fillColor,fill opacity=0.20] ( 95.08, 66.38) circle (  2.13);

\path[fill=fillColor,fill opacity=0.20] ( 85.25, 67.20) circle (  2.13);

\path[fill=fillColor,fill opacity=0.20] ( 84.27, 89.14) circle (  2.13);

\path[fill=fillColor,fill opacity=0.20] ( 81.32, 94.02) circle (  2.13);

\path[fill=fillColor,fill opacity=0.20] ( 89.18, 82.64) circle (  2.13);

\path[fill=fillColor,fill opacity=0.20] ( 87.22, 80.20) circle (  2.13);

\path[fill=fillColor,fill opacity=0.20] ( 70.51, 85.89) circle (  2.13);

\path[fill=fillColor,fill opacity=0.20] ( 83.29, 73.70) circle (  2.13);

\path[fill=fillColor,fill opacity=0.20] ( 87.22, 65.57) circle (  2.13);

\path[fill=fillColor,fill opacity=0.20] ( 77.39, 76.14) circle (  2.13);

\path[fill=fillColor,fill opacity=0.20] ( 86.24, 87.51) circle (  2.13);

\path[fill=fillColor,fill opacity=0.20] ( 88.20, 88.33) circle (  2.13);

\path[fill=fillColor,fill opacity=0.20] ( 82.31, 85.08) circle (  2.13);

\path[fill=fillColor,fill opacity=0.20] ( 79.36, 89.95) circle (  2.13);

\path[fill=fillColor,fill opacity=0.20] ( 79.36, 89.95) circle (  2.13);

\path[fill=fillColor,fill opacity=0.20] ( 68.55, 80.20) circle (  2.13);

\path[fill=fillColor,fill opacity=0.20] ( 76.41, 83.45) circle (  2.13);

\path[fill=fillColor,fill opacity=0.20] ( 72.18, 85.89) circle (  2.13);

\path[fill=fillColor,fill opacity=0.20] ( 69.63, 80.20) circle (  2.13);

\path[fill=fillColor,fill opacity=0.20] ( 70.42,111.89) circle (  2.13);

\path[fill=fillColor,fill opacity=0.20] ( 86.24, 53.38) circle (  2.13);

\path[fill=fillColor,fill opacity=0.20] ( 93.11, 49.32) circle (  2.13);

\path[fill=fillColor,fill opacity=0.20] ( 94.10, 44.44) circle (  2.13);

\path[fill=fillColor,fill opacity=0.20] ( 98.03, 46.88) circle (  2.13);

\path[fill=fillColor,fill opacity=0.20] ( 99.01, 46.07) circle (  2.13);

\path[fill=fillColor,fill opacity=0.20] (101.96, 43.63) circle (  2.13);

\path[fill=fillColor,fill opacity=0.20] (101.96, 46.07) circle (  2.13);

\path[fill=fillColor,fill opacity=0.20] ( 96.06, 57.44) circle (  2.13);

\path[fill=fillColor,fill opacity=0.20] (100.98, 57.44) circle (  2.13);

\path[fill=fillColor,fill opacity=0.20] ( 99.99, 61.51) circle (  2.13);

\path[fill=fillColor,fill opacity=0.20] ( 99.01, 68.82) circle (  2.13);

\path[fill=fillColor,fill opacity=0.20] ( 94.10, 68.82) circle (  2.13);

\path[fill=fillColor,fill opacity=0.20] (100.98, 65.57) circle (  2.13);

\path[fill=fillColor,fill opacity=0.20] ( 99.99, 74.51) circle (  2.13);

\path[fill=fillColor,fill opacity=0.20] (103.92, 89.14) circle (  2.13);

\path[fill=fillColor,fill opacity=0.20] ( 91.15, 72.07) circle (  2.13);

\path[fill=fillColor,fill opacity=0.20] ( 96.06, 63.95) circle (  2.13);

\path[fill=fillColor,fill opacity=0.20] ( 98.03, 78.57) circle (  2.13);

\path[fill=fillColor,fill opacity=0.20] ( 90.17, 81.01) circle (  2.13);

\path[fill=fillColor,fill opacity=0.20] (101.96, 68.82) circle (  2.13);

\path[fill=fillColor,fill opacity=0.20] ( 96.06, 71.26) circle (  2.13);

\path[fill=fillColor,fill opacity=0.20] ( 92.13, 89.14) circle (  2.13);

\path[fill=fillColor,fill opacity=0.20] ( 75.43, 90.76) circle (  2.13);

\path[fill=fillColor,fill opacity=0.20] ( 91.15, 85.08) circle (  2.13);

\path[fill=fillColor,fill opacity=0.20] ( 89.18, 76.14) circle (  2.13);

\path[fill=fillColor,fill opacity=0.20] ( 88.20, 69.63) circle (  2.13);

\path[fill=fillColor,fill opacity=0.20] ( 87.22, 74.51) circle (  2.13);

\path[fill=fillColor,fill opacity=0.20] ( 87.22, 78.57) circle (  2.13);

\path[fill=fillColor,fill opacity=0.20] ( 82.31, 85.08) circle (  2.13);

\path[fill=fillColor,fill opacity=0.20] ( 76.41, 94.83) circle (  2.13);

\path[fill=fillColor,fill opacity=0.20] ( 83.29, 92.39) circle (  2.13);

\path[fill=fillColor,fill opacity=0.20] ( 84.27, 81.82) circle (  2.13);

\path[fill=fillColor,fill opacity=0.20] ( 78.37, 85.08) circle (  2.13);

\path[fill=fillColor,fill opacity=0.20] ( 78.37, 85.89) circle (  2.13);

\path[fill=fillColor,fill opacity=0.20] ( 79.36, 82.64) circle (  2.13);

\path[fill=fillColor,fill opacity=0.20] ( 83.29, 85.89) circle (  2.13);

\path[fill=fillColor,fill opacity=0.20] ( 73.46, 83.45) circle (  2.13);

\path[fill=fillColor,fill opacity=0.20] ( 67.76, 81.01) circle (  2.13);

\path[fill=fillColor,fill opacity=0.20] ( 72.38,113.52) circle (  2.13);

\path[fill=fillColor,fill opacity=0.20] ( 99.99, 40.38) circle (  2.13);

\path[fill=fillColor,fill opacity=0.20] ( 99.99, 40.38) circle (  2.13);

\path[fill=fillColor,fill opacity=0.20] ( 98.03, 44.44) circle (  2.13);

\path[fill=fillColor,fill opacity=0.20] (101.96, 48.50) circle (  2.13);

\path[fill=fillColor,fill opacity=0.20] (103.92, 53.38) circle (  2.13);

\path[fill=fillColor,fill opacity=0.20] (100.98, 55.01) circle (  2.13);

\path[fill=fillColor,fill opacity=0.20] (100.98, 59.07) circle (  2.13);

\path[fill=fillColor,fill opacity=0.20] (106.87, 60.69) circle (  2.13);

\path[fill=fillColor,fill opacity=0.20] (103.92, 59.07) circle (  2.13);

\path[fill=fillColor,fill opacity=0.20] (100.98, 56.63) circle (  2.13);

\path[fill=fillColor,fill opacity=0.20] ( 99.99, 59.07) circle (  2.13);

\path[fill=fillColor,fill opacity=0.20] ( 99.99, 75.32) circle (  2.13);

\path[fill=fillColor,fill opacity=0.20] ( 99.01, 77.76) circle (  2.13);

\path[fill=fillColor,fill opacity=0.20] ( 95.08, 66.38) circle (  2.13);

\path[fill=fillColor,fill opacity=0.20] ( 90.17, 63.95) circle (  2.13);

\path[fill=fillColor,fill opacity=0.20] ( 87.22, 67.20) circle (  2.13);

\path[fill=fillColor,fill opacity=0.20] ( 94.10, 74.51) circle (  2.13);

\path[fill=fillColor,fill opacity=0.20] ( 93.11, 59.07) circle (  2.13);

\path[fill=fillColor,fill opacity=0.20] (100.98, 59.88) circle (  2.13);

\path[fill=fillColor,fill opacity=0.20] ( 98.03, 59.07) circle (  2.13);

\path[fill=fillColor,fill opacity=0.20] ( 96.06, 59.88) circle (  2.13);

\path[fill=fillColor,fill opacity=0.20] ( 96.06, 61.51) circle (  2.13);

\path[fill=fillColor,fill opacity=0.20] ( 93.11, 66.38) circle (  2.13);

\path[fill=fillColor,fill opacity=0.20] ( 92.13, 71.26) circle (  2.13);

\path[fill=fillColor,fill opacity=0.20] ( 96.06, 67.20) circle (  2.13);

\path[fill=fillColor,fill opacity=0.20] ( 92.13, 67.20) circle (  2.13);

\path[fill=fillColor,fill opacity=0.20] ( 94.10, 74.51) circle (  2.13);

\path[fill=fillColor,fill opacity=0.20] ( 92.13, 81.01) circle (  2.13);

\path[fill=fillColor,fill opacity=0.20] ( 85.25, 82.64) circle (  2.13);

\path[fill=fillColor,fill opacity=0.20] ( 84.27, 75.32) circle (  2.13);

\path[fill=fillColor,fill opacity=0.20] ( 87.22, 68.01) circle (  2.13);

\path[fill=fillColor,fill opacity=0.20] ( 91.15, 79.39) circle (  2.13);

\path[fill=fillColor,fill opacity=0.20] ( 86.24, 89.95) circle (  2.13);

\path[fill=fillColor,fill opacity=0.20] ( 81.32, 90.76) circle (  2.13);

\path[fill=fillColor,fill opacity=0.20] ( 77.39, 88.33) circle (  2.13);

\path[fill=fillColor,fill opacity=0.20] ( 82.31, 77.76) circle (  2.13);

\path[fill=fillColor,fill opacity=0.20] ( 90.17, 80.20) circle (  2.13);

\path[fill=fillColor,fill opacity=0.20] ( 83.29, 85.08) circle (  2.13);

\path[fill=fillColor,fill opacity=0.20] ( 84.27, 77.76) circle (  2.13);

\path[fill=fillColor,fill opacity=0.20] ( 87.22, 81.01) circle (  2.13);

\path[fill=fillColor,fill opacity=0.20] ( 88.20, 81.82) circle (  2.13);

\path[fill=fillColor,fill opacity=0.20] ( 72.48, 74.51) circle (  2.13);

\path[fill=fillColor,fill opacity=0.20] ( 72.48, 81.01) circle (  2.13);

\path[fill=fillColor,fill opacity=0.20] ( 87.22, 41.19) circle (  2.13);

\path[fill=fillColor,fill opacity=0.20] ( 76.41, 42.82) circle (  2.13);

\path[fill=fillColor,fill opacity=0.20] (102.94, 44.44) circle (  2.13);

\path[fill=fillColor,fill opacity=0.20] (108.84, 50.13) circle (  2.13);

\path[fill=fillColor,fill opacity=0.20] (110.80, 50.94) circle (  2.13);

\path[fill=fillColor,fill opacity=0.20] (101.96, 55.01) circle (  2.13);

\path[fill=fillColor,fill opacity=0.20] ( 94.10, 57.44) circle (  2.13);

\path[fill=fillColor,fill opacity=0.20] ( 92.13, 56.63) circle (  2.13);

\path[fill=fillColor,fill opacity=0.20] ( 94.10, 63.95) circle (  2.13);

\path[fill=fillColor,fill opacity=0.20] ( 99.01, 79.39) circle (  2.13);

\path[fill=fillColor,fill opacity=0.20] ( 95.08, 66.38) circle (  2.13);

\path[fill=fillColor,fill opacity=0.20] (108.84, 97.27) circle (  2.13);

\path[fill=fillColor,fill opacity=0.20] ( 69.53, 91.58) circle (  2.13);

\path[fill=fillColor,fill opacity=0.20] ( 96.06, 80.20) circle (  2.13);

\path[fill=fillColor,fill opacity=0.20] ( 85.25, 76.14) circle (  2.13);

\path[fill=fillColor,fill opacity=0.20] ( 92.13, 72.07) circle (  2.13);

\path[fill=fillColor,fill opacity=0.20] ( 99.01, 46.07) circle (  2.13);

\path[fill=fillColor,fill opacity=0.20] ( 86.24, 53.38) circle (  2.13);

\path[fill=fillColor,fill opacity=0.20] ( 95.08, 58.26) circle (  2.13);

\path[fill=fillColor,fill opacity=0.20] ( 95.08, 54.19) circle (  2.13);

\path[fill=fillColor,fill opacity=0.20] ( 87.22, 64.76) circle (  2.13);

\path[fill=fillColor,fill opacity=0.20] ( 61.18, 63.95) circle (  2.13);

\path[fill=fillColor,fill opacity=0.20] ( 93.11, 58.26) circle (  2.13);

\path[fill=fillColor,fill opacity=0.20] ( 84.27, 49.32) circle (  2.13);

\path[fill=fillColor,fill opacity=0.20] ( 86.24, 41.19) circle (  2.13);

\path[fill=fillColor,fill opacity=0.20] ( 86.24, 47.69) circle (  2.13);

\path[fill=fillColor,fill opacity=0.20] ( 82.31, 55.82) circle (  2.13);

\path[fill=fillColor,fill opacity=0.20] ( 83.29, 68.01) circle (  2.13);

\path[fill=fillColor,fill opacity=0.20] ( 80.34, 76.95) circle (  2.13);

\path[fill=fillColor,fill opacity=0.20] ( 90.17, 70.45) circle (  2.13);

\path[fill=fillColor,fill opacity=0.20] ( 85.25, 80.20) circle (  2.13);

\path[fill=fillColor,fill opacity=0.20] ( 82.31, 86.70) circle (  2.13);

\path[fill=fillColor,fill opacity=0.20] ( 85.25, 83.45) circle (  2.13);

\path[fill=fillColor,fill opacity=0.20] ( 84.27, 76.14) circle (  2.13);

\path[fill=fillColor,fill opacity=0.20] ( 94.10, 70.45) circle (  2.13);

\path[fill=fillColor,fill opacity=0.20] ( 88.20, 81.01) circle (  2.13);

\path[fill=fillColor,fill opacity=0.20] ( 85.25, 87.51) circle (  2.13);

\path[fill=fillColor,fill opacity=0.20] ( 91.15, 76.95) circle (  2.13);

\path[fill=fillColor,fill opacity=0.20] ( 95.08, 77.76) circle (  2.13);

\path[fill=fillColor,fill opacity=0.20] ( 88.20, 75.32) circle (  2.13);

\path[fill=fillColor,fill opacity=0.20] ( 90.17, 69.63) circle (  2.13);

\path[fill=fillColor,fill opacity=0.20] ( 85.25, 76.95) circle (  2.13);

\path[fill=fillColor,fill opacity=0.20] ( 98.03, 54.19) circle (  2.13);

\path[fill=fillColor,fill opacity=0.20] (107.86, 46.88) circle (  2.13);

\path[fill=fillColor,fill opacity=0.20] (103.92, 43.63) circle (  2.13);

\path[fill=fillColor,fill opacity=0.20] (105.89, 54.19) circle (  2.13);

\path[fill=fillColor,fill opacity=0.20] (102.94, 60.69) circle (  2.13);

\path[fill=fillColor,fill opacity=0.20] ( 96.06, 68.01) circle (  2.13);

\path[fill=fillColor,fill opacity=0.20] ( 94.10, 55.82) circle (  2.13);

\path[fill=fillColor,fill opacity=0.20] (100.98, 45.25) circle (  2.13);

\path[fill=fillColor,fill opacity=0.20] ( 93.11, 68.82) circle (  2.13);

\path[fill=fillColor,fill opacity=0.20] ( 85.25, 76.14) circle (  2.13);

\path[fill=fillColor,fill opacity=0.20] ( 91.15, 65.57) circle (  2.13);

\path[fill=fillColor,fill opacity=0.20] ( 90.17, 70.45) circle (  2.13);

\path[fill=fillColor,fill opacity=0.20] ( 91.15, 75.32) circle (  2.13);

\path[fill=fillColor,fill opacity=0.20] ( 90.17, 78.57) circle (  2.13);

\path[fill=fillColor,fill opacity=0.20] ( 96.06, 81.01) circle (  2.13);

\path[fill=fillColor,fill opacity=0.20] ( 85.25, 72.89) circle (  2.13);

\path[fill=fillColor,fill opacity=0.20] ( 80.34, 65.57) circle (  2.13);

\path[fill=fillColor,fill opacity=0.20] ( 53.02, 65.57) circle (  2.13);

\path[fill=fillColor,fill opacity=0.20] ( 85.25, 59.07) circle (  2.13);

\path[fill=fillColor,fill opacity=0.20] ( 91.15, 49.32) circle (  2.13);

\path[fill=fillColor,fill opacity=0.20] ( 68.94, 53.38) circle (  2.13);

\path[fill=fillColor,fill opacity=0.20] ( 92.13, 64.76) circle (  2.13);

\path[fill=fillColor,fill opacity=0.20] ( 88.20, 62.32) circle (  2.13);

\path[fill=fillColor,fill opacity=0.20] ( 77.39, 41.19) circle (  2.13);

\path[fill=fillColor,fill opacity=0.20] ( 90.17, 54.19) circle (  2.13);

\path[fill=fillColor,fill opacity=0.20] ( 86.24, 49.32) circle (  2.13);

\path[fill=fillColor,fill opacity=0.20] ( 80.34, 41.19) circle (  2.13);

\path[fill=fillColor,fill opacity=0.20] ( 85.25, 40.38) circle (  2.13);

\path[fill=fillColor,fill opacity=0.20] ( 82.31, 47.69) circle (  2.13);

\path[fill=fillColor,fill opacity=0.20] ( 74.44, 50.94) circle (  2.13);

\path[fill=fillColor,fill opacity=0.20] ( 80.34, 61.51) circle (  2.13);

\path[fill=fillColor,fill opacity=0.20] ( 86.24, 59.07) circle (  2.13);

\path[fill=fillColor,fill opacity=0.20] ( 84.27, 61.51) circle (  2.13);

\path[fill=fillColor,fill opacity=0.20] ( 83.29, 66.38) circle (  2.13);

\path[fill=fillColor,fill opacity=0.20] ( 90.17, 67.20) circle (  2.13);

\path[fill=fillColor,fill opacity=0.20] ( 84.27, 68.01) circle (  2.13);

\path[fill=fillColor,fill opacity=0.20] ( 86.24, 74.51) circle (  2.13);

\path[fill=fillColor,fill opacity=0.20] ( 89.18, 73.70) circle (  2.13);

\path[fill=fillColor,fill opacity=0.20] ( 91.15, 76.95) circle (  2.13);

\path[fill=fillColor,fill opacity=0.20] ( 88.20, 79.39) circle (  2.13);

\path[fill=fillColor,fill opacity=0.20] ( 94.10, 76.14) circle (  2.13);

\path[fill=fillColor,fill opacity=0.20] ( 96.06, 77.76) circle (  2.13);

\path[fill=fillColor,fill opacity=0.20] ( 90.17, 72.07) circle (  2.13);

\path[fill=fillColor,fill opacity=0.20] ( 89.18, 64.76) circle (  2.13);

\path[fill=fillColor,fill opacity=0.20] ( 79.36, 75.32) circle (  2.13);

\path[fill=fillColor,fill opacity=0.20] (104.91, 39.56) circle (  2.13);

\path[fill=fillColor,fill opacity=0.20] ( 95.08, 39.56) circle (  2.13);

\path[fill=fillColor,fill opacity=0.20] (102.94, 41.19) circle (  2.13);

\path[fill=fillColor,fill opacity=0.20] ( 98.03, 41.19) circle (  2.13);

\path[fill=fillColor,fill opacity=0.20] (104.91, 39.56) circle (  2.13);

\path[fill=fillColor,fill opacity=0.20] (101.96, 47.69) circle (  2.13);

\path[fill=fillColor,fill opacity=0.20] ( 98.03, 52.57) circle (  2.13);

\path[fill=fillColor,fill opacity=0.20] ( 91.15, 52.57) circle (  2.13);

\path[fill=fillColor,fill opacity=0.20] ( 96.06, 54.19) circle (  2.13);

\path[fill=fillColor,fill opacity=0.20] ( 84.27, 64.76) circle (  2.13);

\path[fill=fillColor,fill opacity=0.20] ( 92.13, 76.95) circle (  2.13);

\path[fill=fillColor,fill opacity=0.20] ( 97.05, 69.63) circle (  2.13);

\path[fill=fillColor,fill opacity=0.20] ( 91.15, 63.13) circle (  2.13);

\path[fill=fillColor,fill opacity=0.20] ( 89.18, 59.88) circle (  2.13);

\path[fill=fillColor,fill opacity=0.20] ( 90.17, 57.44) circle (  2.13);

\path[fill=fillColor,fill opacity=0.20] ( 83.29, 56.63) circle (  2.13);

\path[fill=fillColor,fill opacity=0.20] ( 57.74, 51.75) circle (  2.13);

\path[fill=fillColor,fill opacity=0.20] ( 82.31, 40.38) circle (  2.13);

\path[fill=fillColor,fill opacity=0.20] ( 82.31, 41.19) circle (  2.13);

\path[fill=fillColor,fill opacity=0.20] ( 76.41, 53.38) circle (  2.13);

\path[fill=fillColor,fill opacity=0.20] ( 80.34, 50.13) circle (  2.13);

\path[fill=fillColor,fill opacity=0.20] ( 78.37, 50.94) circle (  2.13);

\path[fill=fillColor,fill opacity=0.20] ( 87.22, 47.69) circle (  2.13);

\path[fill=fillColor,fill opacity=0.20] ( 86.24, 63.13) circle (  2.13);

\path[fill=fillColor,fill opacity=0.20] ( 81.32, 74.51) circle (  2.13);

\path[fill=fillColor,fill opacity=0.20] ( 88.20, 66.38) circle (  2.13);

\path[fill=fillColor,fill opacity=0.20] ( 89.18, 62.32) circle (  2.13);

\path[fill=fillColor,fill opacity=0.20] ( 89.18, 59.88) circle (  2.13);

\path[fill=fillColor,fill opacity=0.20] ( 91.15, 59.88) circle (  2.13);

\path[fill=fillColor,fill opacity=0.20] ( 92.13, 68.01) circle (  2.13);

\path[fill=fillColor,fill opacity=0.20] ( 85.25, 61.51) circle (  2.13);

\path[fill=fillColor,fill opacity=0.20] ( 74.44, 59.88) circle (  2.13);

\path[fill=fillColor,fill opacity=0.20] ( 93.11, 39.56) circle (  2.13);

\path[fill=fillColor,fill opacity=0.20] ( 99.01, 43.63) circle (  2.13);

\path[fill=fillColor,fill opacity=0.20] ( 99.99, 41.19) circle (  2.13);

\path[fill=fillColor,fill opacity=0.20] (101.96, 38.75) circle (  2.13);

\path[fill=fillColor,fill opacity=0.20] (102.94, 57.44) circle (  2.13);

\path[fill=fillColor,fill opacity=0.20] ( 98.03, 67.20) circle (  2.13);

\path[fill=fillColor,fill opacity=0.20] ( 91.15, 58.26) circle (  2.13);

\path[fill=fillColor,fill opacity=0.20] ( 88.20, 54.19) circle (  2.13);

\path[fill=fillColor,fill opacity=0.20] ( 92.13, 56.63) circle (  2.13);

\path[fill=fillColor,fill opacity=0.20] ( 74.44, 55.01) circle (  2.13);

\path[fill=fillColor,fill opacity=0.20] ( 88.20, 54.19) circle (  2.13);

\path[fill=fillColor,fill opacity=0.20] ( 85.25, 52.57) circle (  2.13);

\path[fill=fillColor,fill opacity=0.20] ( 84.27, 50.13) circle (  2.13);

\path[fill=fillColor,fill opacity=0.20] ( 76.41, 46.88) circle (  2.13);

\path[fill=fillColor,fill opacity=0.20] ( 73.46, 40.38) circle (  2.13);

\path[fill=fillColor,fill opacity=0.20] ( 92.13, 50.13) circle (  2.13);

\path[fill=fillColor,fill opacity=0.20] ( 81.32, 67.20) circle (  2.13);

\path[fill=fillColor,fill opacity=0.20] ( 70.91, 72.07) circle (  2.13);

\path[fill=fillColor,fill opacity=0.20] ( 90.17, 46.88) circle (  2.13);

\path[fill=fillColor,fill opacity=0.20] ( 72.48, 52.57) circle (  2.13);

\path[fill=fillColor,fill opacity=0.20] ( 88.20, 45.25) circle (  2.13);

\path[fill=fillColor,fill opacity=0.20] ( 91.15, 59.88) circle (  2.13);

\path[fill=fillColor,fill opacity=0.20] ( 75.43, 50.94) circle (  2.13);

\path[fill=fillColor,fill opacity=0.20] ( 89.18, 41.19) circle (  2.13);

\path[fill=fillColor,fill opacity=0.20] ( 83.29, 49.32) circle (  2.13);

\path[fill=fillColor,fill opacity=0.20] ( 78.37, 53.38) circle (  2.13);

\path[fill=fillColor,fill opacity=0.20] ( 81.32, 49.32) circle (  2.13);

\path[fill=fillColor,fill opacity=0.20] ( 81.32, 45.25) circle (  2.13);

\path[fill=fillColor,fill opacity=0.20] ( 76.41, 48.50) circle (  2.13);

\path[fill=fillColor,fill opacity=0.20] ( 76.41, 57.44) circle (  2.13);

\path[fill=fillColor,fill opacity=0.20] ( 84.27, 50.94) circle (  2.13);

\path[fill=fillColor,fill opacity=0.20] ( 80.34,104.58) circle (  2.13);

\path[fill=fillColor,fill opacity=0.20] ( 80.34, 93.20) circle (  2.13);

\path[fill=fillColor,fill opacity=0.20] ( 78.37, 89.95) circle (  2.13);

\path[fill=fillColor,fill opacity=0.20] ( 86.24, 81.01) circle (  2.13);

\path[fill=fillColor,fill opacity=0.20] ( 83.29,103.77) circle (  2.13);

\path[fill=fillColor,fill opacity=0.20] ( 75.43, 79.39) circle (  2.13);

\path[fill=fillColor,fill opacity=0.20] ( 62.65, 75.32) circle (  2.13);

\path[fill=fillColor,fill opacity=0.20] ( 79.36, 75.32) circle (  2.13);

\path[fill=fillColor,fill opacity=0.20] ( 80.34, 81.82) circle (  2.13);

\path[fill=fillColor,fill opacity=0.20] ( 79.36, 87.51) circle (  2.13);

\path[fill=fillColor,fill opacity=0.20] ( 99.01, 85.89) circle (  2.13);

\path[fill=fillColor,fill opacity=0.20] (105.89, 92.39) circle (  2.13);

\path[fill=fillColor,fill opacity=0.20] ( 99.01, 89.14) circle (  2.13);

\path[fill=fillColor,fill opacity=0.20] (101.96, 73.70) circle (  2.13);

\path[fill=fillColor,fill opacity=0.20] (103.92, 67.20) circle (  2.13);

\path[fill=fillColor,fill opacity=0.20] ( 92.13, 71.26) circle (  2.13);

\path[fill=fillColor,fill opacity=0.20] ( 75.43, 76.95) circle (  2.13);

\path[fill=fillColor,fill opacity=0.20] ( 68.25, 86.70) circle (  2.13);

\path[fill=fillColor,fill opacity=0.20] ( 59.02, 89.95) circle (  2.13);

\path[fill=fillColor,fill opacity=0.20] ( 71.40, 79.39) circle (  2.13);

\path[fill=fillColor,fill opacity=0.20] ( 78.37, 84.26) circle (  2.13);

\path[fill=fillColor,fill opacity=0.20] ( 61.57, 94.83) circle (  2.13);

\path[fill=fillColor,fill opacity=0.20] ( 93.11, 78.57) circle (  2.13);

\path[fill=fillColor,fill opacity=0.20] ( 93.11, 70.45) circle (  2.13);

\path[fill=fillColor,fill opacity=0.20] ( 98.03, 95.64) circle (  2.13);

\path[fill=fillColor,fill opacity=0.20] ( 93.11, 77.76) circle (  2.13);

\path[fill=fillColor,fill opacity=0.20] ( 99.01, 71.26) circle (  2.13);

\path[fill=fillColor,fill opacity=0.20] ( 97.05, 74.51) circle (  2.13);

\path[fill=fillColor,fill opacity=0.20] ( 95.08, 75.32) circle (  2.13);

\path[fill=fillColor,fill opacity=0.20] ( 91.15, 64.76) circle (  2.13);

\path[fill=fillColor,fill opacity=0.20] ( 89.18, 51.75) circle (  2.13);

\path[fill=fillColor,fill opacity=0.20] ( 90.17, 48.50) circle (  2.13);

\path[fill=fillColor,fill opacity=0.20] ( 99.01, 51.75) circle (  2.13);

\path[fill=fillColor,fill opacity=0.20] (111.79, 51.75) circle (  2.13);

\path[fill=fillColor,fill opacity=0.20] ( 78.37, 89.95) circle (  2.13);

\path[fill=fillColor,fill opacity=0.20] ( 71.69, 72.07) circle (  2.13);

\path[fill=fillColor,fill opacity=0.20] ( 63.34,101.33) circle (  2.13);

\path[fill=fillColor,fill opacity=0.20] ( 65.40, 99.70) circle (  2.13);

\path[fill=fillColor,fill opacity=0.20] ( 58.52, 90.76) circle (  2.13);

\path[fill=fillColor,fill opacity=0.20] ( 80.34, 94.83) circle (  2.13);

\path[fill=fillColor,fill opacity=0.20] ( 82.31, 94.83) circle (  2.13);

\path[fill=fillColor,fill opacity=0.20] ( 81.32, 81.82) circle (  2.13);

\path[fill=fillColor,fill opacity=0.20] ( 86.24, 69.63) circle (  2.13);

\path[fill=fillColor,fill opacity=0.20] ( 96.06, 63.13) circle (  2.13);

\path[fill=fillColor,fill opacity=0.20] ( 96.06, 85.08) circle (  2.13);

\path[fill=fillColor,fill opacity=0.20] ( 92.13, 58.26) circle (  2.13);

\path[fill=fillColor,fill opacity=0.20] ( 86.24, 55.01) circle (  2.13);

\path[fill=fillColor,fill opacity=0.20] ( 88.20, 67.20) circle (  2.13);

\path[fill=fillColor,fill opacity=0.20] ( 88.20, 77.76) circle (  2.13);

\path[fill=fillColor,fill opacity=0.20] ( 93.11, 85.89) circle (  2.13);

\path[fill=fillColor,fill opacity=0.20] ( 92.13, 79.39) circle (  2.13);

\path[fill=fillColor,fill opacity=0.20] ( 91.15, 61.51) circle (  2.13);

\path[fill=fillColor,fill opacity=0.20] ( 63.34, 47.69) circle (  2.13);

\path[fill=fillColor,fill opacity=0.20] (103.92, 39.56) circle (  2.13);

\path[fill=fillColor,fill opacity=0.20] ( 80.34, 88.33) circle (  2.13);

\path[fill=fillColor,fill opacity=0.20] ( 72.48, 76.95) circle (  2.13);

\path[fill=fillColor,fill opacity=0.20] ( 68.74,105.39) circle (  2.13);

\path[fill=fillColor,fill opacity=0.20] ( 52.63,101.33) circle (  2.13);

\path[fill=fillColor,fill opacity=0.20] ( 65.89,107.02) circle (  2.13);

\path[fill=fillColor,fill opacity=0.20] ( 83.29, 87.51) circle (  2.13);

\path[fill=fillColor,fill opacity=0.20] ( 86.24, 71.26) circle (  2.13);

\path[fill=fillColor,fill opacity=0.20] ( 88.20, 79.39) circle (  2.13);

\path[fill=fillColor,fill opacity=0.20] ( 90.17, 76.14) circle (  2.13);

\path[fill=fillColor,fill opacity=0.20] ( 93.11, 54.19) circle (  2.13);

\path[fill=fillColor,fill opacity=0.20] ( 90.17, 83.45) circle (  2.13);

\path[fill=fillColor,fill opacity=0.20] ( 85.25, 62.32) circle (  2.13);

\path[fill=fillColor,fill opacity=0.20] ( 81.32, 73.70) circle (  2.13);

\path[fill=fillColor,fill opacity=0.20] ( 75.43, 92.39) circle (  2.13);

\path[fill=fillColor,fill opacity=0.20] ( 69.92,101.33) circle (  2.13);

\path[fill=fillColor,fill opacity=0.20] ( 75.43,100.52) circle (  2.13);

\path[fill=fillColor,fill opacity=0.20] ( 79.36, 91.58) circle (  2.13);

\path[fill=fillColor,fill opacity=0.20] ( 85.25, 79.39) circle (  2.13);

\path[fill=fillColor,fill opacity=0.20] ( 99.01, 72.89) circle (  2.13);

\path[fill=fillColor,fill opacity=0.20] (105.89, 70.45) circle (  2.13);

\path[fill=fillColor,fill opacity=0.20] ( 95.08, 51.75) circle (  2.13);

\path[fill=fillColor,fill opacity=0.20] (121.61, 37.94) circle (  2.13);

\path[fill=fillColor,fill opacity=0.20] ( 82.31, 94.02) circle (  2.13);

\path[fill=fillColor,fill opacity=0.20] ( 71.59, 75.32) circle (  2.13);

\path[fill=fillColor,fill opacity=0.20] ( 65.21, 97.27) circle (  2.13);

\path[fill=fillColor,fill opacity=0.20] ( 51.84, 97.27) circle (  2.13);

\path[fill=fillColor,fill opacity=0.20] ( 65.99, 96.45) circle (  2.13);

\path[fill=fillColor,fill opacity=0.20] ( 56.46,100.52) circle (  2.13);

\path[fill=fillColor,fill opacity=0.20] ( 74.44, 82.64) circle (  2.13);

\path[fill=fillColor,fill opacity=0.20] ( 87.22, 72.89) circle (  2.13);

\path[fill=fillColor,fill opacity=0.20] ( 87.22, 83.45) circle (  2.13);

\path[fill=fillColor,fill opacity=0.20] ( 87.22, 76.14) circle (  2.13);

\path[fill=fillColor,fill opacity=0.20] ( 95.08, 55.01) circle (  2.13);

\path[fill=fillColor,fill opacity=0.20] ( 84.27, 71.26) circle (  2.13);

\path[fill=fillColor,fill opacity=0.20] ( 79.36, 73.70) circle (  2.13);

\path[fill=fillColor,fill opacity=0.20] ( 69.73, 94.83) circle (  2.13);

\path[fill=fillColor,fill opacity=0.20] ( 65.89,107.83) circle (  2.13);

\path[fill=fillColor,fill opacity=0.20] ( 57.05,110.27) circle (  2.13);

\path[fill=fillColor,fill opacity=0.20] ( 73.46, 87.51) circle (  2.13);

\path[fill=fillColor,fill opacity=0.20] ( 85.25, 68.01) circle (  2.13);

\path[fill=fillColor,fill opacity=0.20] ( 98.03, 63.95) circle (  2.13);

\path[fill=fillColor,fill opacity=0.20] (109.82, 66.38) circle (  2.13);

\path[fill=fillColor,fill opacity=0.20] ( 93.11, 83.45) circle (  2.13);

\path[fill=fillColor,fill opacity=0.20] ( 79.36, 64.76) circle (  2.13);

\path[fill=fillColor,fill opacity=0.20] ( 73.46, 89.14) circle (  2.13);

\path[fill=fillColor,fill opacity=0.20] ( 70.81, 89.95) circle (  2.13);

\path[fill=fillColor,fill opacity=0.20] ( 73.46, 83.45) circle (  2.13);

\path[fill=fillColor,fill opacity=0.20] ( 74.44, 86.70) circle (  2.13);

\path[fill=fillColor,fill opacity=0.20] ( 77.39, 85.08) circle (  2.13);

\path[fill=fillColor,fill opacity=0.20] ( 84.27, 83.45) circle (  2.13);

\path[fill=fillColor,fill opacity=0.20] ( 90.17, 81.01) circle (  2.13);

\path[fill=fillColor,fill opacity=0.20] ( 74.44, 66.38) circle (  2.13);

\path[fill=fillColor,fill opacity=0.20] ( 97.05, 92.39) circle (  2.13);

\path[fill=fillColor,fill opacity=0.20] ( 86.24, 73.70) circle (  2.13);

\path[fill=fillColor,fill opacity=0.20] ( 76.41, 88.33) circle (  2.13);

\path[fill=fillColor,fill opacity=0.20] ( 67.96, 98.08) circle (  2.13);

\path[fill=fillColor,fill opacity=0.20] ( 63.93, 98.89) circle (  2.13);

\path[fill=fillColor,fill opacity=0.20] ( 53.12, 98.89) circle (  2.13);

\path[fill=fillColor,fill opacity=0.20] ( 81.32, 91.58) circle (  2.13);

\path[fill=fillColor,fill opacity=0.20] ( 95.08, 76.14) circle (  2.13);

\path[fill=fillColor,fill opacity=0.20] (101.96, 53.38) circle (  2.13);

\path[fill=fillColor,fill opacity=0.20] (114.73, 46.07) circle (  2.13);

\path[fill=fillColor,fill opacity=0.20] ( 87.22, 76.95) circle (  2.13);

\path[fill=fillColor,fill opacity=0.20] ( 86.24, 73.70) circle (  2.13);

\path[fill=fillColor,fill opacity=0.20] ( 83.29, 89.95) circle (  2.13);

\path[fill=fillColor,fill opacity=0.20] ( 66.09, 90.76) circle (  2.13);

\path[fill=fillColor,fill opacity=0.20] ( 76.41, 85.08) circle (  2.13);

\path[fill=fillColor,fill opacity=0.20] ( 77.39, 86.70) circle (  2.13);

\path[fill=fillColor,fill opacity=0.20] ( 77.39, 90.76) circle (  2.13);

\path[fill=fillColor,fill opacity=0.20] ( 77.39, 85.08) circle (  2.13);

\path[fill=fillColor,fill opacity=0.20] ( 87.22, 74.51) circle (  2.13);

\path[fill=fillColor,fill opacity=0.20] ( 89.18, 63.13) circle (  2.13);

\path[fill=fillColor,fill opacity=0.20] ( 88.20, 83.45) circle (  2.13);

\path[fill=fillColor,fill opacity=0.20] ( 79.36, 69.63) circle (  2.13);

\path[fill=fillColor,fill opacity=0.20] ( 76.41, 94.83) circle (  2.13);

\path[fill=fillColor,fill opacity=0.20] ( 69.92,104.58) circle (  2.13);

\path[fill=fillColor,fill opacity=0.20] ( 62.46, 98.89) circle (  2.13);

\path[fill=fillColor,fill opacity=0.20] ( 77.39,101.33) circle (  2.13);

\path[fill=fillColor,fill opacity=0.20] ( 79.36, 98.89) circle (  2.13);

\path[fill=fillColor,fill opacity=0.20] ( 79.36, 91.58) circle (  2.13);

\path[fill=fillColor,fill opacity=0.20] ( 99.01, 76.14) circle (  2.13);

\path[fill=fillColor,fill opacity=0.20] (110.80, 56.63) circle (  2.13);

\path[fill=fillColor,fill opacity=0.20] ( 91.15, 46.88) circle (  2.13);

\path[fill=fillColor,fill opacity=0.20] ( 87.22, 76.14) circle (  2.13);

\path[fill=fillColor,fill opacity=0.20] ( 82.31, 78.57) circle (  2.13);

\path[fill=fillColor,fill opacity=0.20] ( 87.22, 86.70) circle (  2.13);

\path[fill=fillColor,fill opacity=0.20] ( 87.22, 88.33) circle (  2.13);

\path[fill=fillColor,fill opacity=0.20] ( 83.29, 89.95) circle (  2.13);

\path[fill=fillColor,fill opacity=0.20] ( 83.29, 91.58) circle (  2.13);

\path[fill=fillColor,fill opacity=0.20] ( 70.81, 83.45) circle (  2.13);

\path[fill=fillColor,fill opacity=0.20] ( 77.39, 76.14) circle (  2.13);

\path[fill=fillColor,fill opacity=0.20] ( 81.32, 71.26) circle (  2.13);

\path[fill=fillColor,fill opacity=0.20] ( 91.15, 63.95) circle (  2.13);

\path[fill=fillColor,fill opacity=0.20] (101.96, 51.75) circle (  2.13);

\path[fill=fillColor,fill opacity=0.20] (118.66, 42.00) circle (  2.13);

\path[fill=fillColor,fill opacity=0.20] ( 87.22, 81.82) circle (  2.13);

\path[fill=fillColor,fill opacity=0.20] ( 75.43, 72.07) circle (  2.13);

\path[fill=fillColor,fill opacity=0.20] ( 73.46, 98.08) circle (  2.13);

\path[fill=fillColor,fill opacity=0.20] ( 71.69,111.89) circle (  2.13);

\path[fill=fillColor,fill opacity=0.20] ( 70.02,115.15) circle (  2.13);

\path[fill=fillColor,fill opacity=0.20] ( 79.36,111.89) circle (  2.13);

\path[fill=fillColor,fill opacity=0.20] ( 91.15, 94.83) circle (  2.13);

\path[fill=fillColor,fill opacity=0.20] ( 92.13, 77.76) circle (  2.13);

\path[fill=fillColor,fill opacity=0.20] ( 94.10, 65.57) circle (  2.13);

\path[fill=fillColor,fill opacity=0.20] ( 92.13, 76.14) circle (  2.13);

\path[fill=fillColor,fill opacity=0.20] ( 88.20, 63.95) circle (  2.13);

\path[fill=fillColor,fill opacity=0.20] ( 90.17, 71.26) circle (  2.13);

\path[fill=fillColor,fill opacity=0.20] ( 80.34, 81.01) circle (  2.13);

\path[fill=fillColor,fill opacity=0.20] ( 79.36, 93.20) circle (  2.13);

\path[fill=fillColor,fill opacity=0.20] ( 76.41, 86.70) circle (  2.13);

\path[fill=fillColor,fill opacity=0.20] ( 86.24, 73.70) circle (  2.13);

\path[fill=fillColor,fill opacity=0.20] ( 87.22, 72.07) circle (  2.13);

\path[fill=fillColor,fill opacity=0.20] ( 96.06, 68.82) circle (  2.13);

\path[fill=fillColor,fill opacity=0.20] ( 96.06, 57.44) circle (  2.13);

\path[fill=fillColor,fill opacity=0.20] ( 88.20, 82.64) circle (  2.13);

\path[fill=fillColor,fill opacity=0.20] ( 81.32, 77.76) circle (  2.13);

\path[fill=fillColor,fill opacity=0.20] ( 72.48, 93.20) circle (  2.13);

\path[fill=fillColor,fill opacity=0.20] ( 72.48,102.14) circle (  2.13);

\path[fill=fillColor,fill opacity=0.20] ( 77.39,115.15) circle (  2.13);

\path[fill=fillColor,fill opacity=0.20] ( 82.31,112.71) circle (  2.13);

\path[fill=fillColor,fill opacity=0.20] ( 91.15, 93.20) circle (  2.13);

\path[fill=fillColor,fill opacity=0.20] ( 91.15, 78.57) circle (  2.13);

\path[fill=fillColor,fill opacity=0.20] ( 92.13, 66.38) circle (  2.13);

\path[fill=fillColor,fill opacity=0.20] (112.77, 57.44) circle (  2.13);

\path[fill=fillColor,fill opacity=0.20] ( 97.05, 95.64) circle (  2.13);

\path[fill=fillColor,fill opacity=0.20] ( 90.17, 72.07) circle (  2.13);

\path[fill=fillColor,fill opacity=0.20] ( 86.24, 62.32) circle (  2.13);

\path[fill=fillColor,fill opacity=0.20] ( 83.29, 75.32) circle (  2.13);

\path[fill=fillColor,fill opacity=0.20] ( 84.27, 82.64) circle (  2.13);

\path[fill=fillColor,fill opacity=0.20] ( 83.29, 79.39) circle (  2.13);

\path[fill=fillColor,fill opacity=0.20] ( 87.22, 83.45) circle (  2.13);

\path[fill=fillColor,fill opacity=0.20] ( 83.29, 84.26) circle (  2.13);

\path[fill=fillColor,fill opacity=0.20] ( 82.31, 78.57) circle (  2.13);

\path[fill=fillColor,fill opacity=0.20] ( 90.17, 75.32) circle (  2.13);

\path[fill=fillColor,fill opacity=0.20] ( 97.05, 65.57) circle (  2.13);

\path[fill=fillColor,fill opacity=0.20] ( 72.09, 75.32) circle (  2.13);

\path[fill=fillColor,fill opacity=0.20] ( 74.44, 85.08) circle (  2.13);

\path[fill=fillColor,fill opacity=0.20] ( 71.10, 86.70) circle (  2.13);

\path[fill=fillColor,fill opacity=0.20] ( 84.27, 93.20) circle (  2.13);

\path[fill=fillColor,fill opacity=0.20] ( 89.18,100.52) circle (  2.13);

\path[fill=fillColor,fill opacity=0.20] ( 93.11, 94.02) circle (  2.13);

\path[fill=fillColor,fill opacity=0.20] ( 93.11, 84.26) circle (  2.13);

\path[fill=fillColor,fill opacity=0.20] ( 95.08, 69.63) circle (  2.13);

\path[fill=fillColor,fill opacity=0.20] (100.98, 53.38) circle (  2.13);

\path[fill=fillColor,fill opacity=0.20] ( 86.24, 89.95) circle (  2.13);

\path[fill=fillColor,fill opacity=0.20] ( 88.20, 81.82) circle (  2.13);

\path[fill=fillColor,fill opacity=0.20] ( 81.32, 80.20) circle (  2.13);

\path[fill=fillColor,fill opacity=0.20] ( 75.43, 77.76) circle (  2.13);

\path[fill=fillColor,fill opacity=0.20] ( 81.32, 65.57) circle (  2.13);

\path[fill=fillColor,fill opacity=0.20] ( 84.27, 71.26) circle (  2.13);

\path[fill=fillColor,fill opacity=0.20] ( 76.41, 83.45) circle (  2.13);

\path[fill=fillColor,fill opacity=0.20] ( 81.32, 85.08) circle (  2.13);

\path[fill=fillColor,fill opacity=0.20] ( 85.25, 74.51) circle (  2.13);

\path[fill=fillColor,fill opacity=0.20] ( 93.11, 62.32) circle (  2.13);

\path[fill=fillColor,fill opacity=0.20] (107.86, 60.69) circle (  2.13);

\path[fill=fillColor,fill opacity=0.20] ( 86.24, 70.45) circle (  2.13);

\path[fill=fillColor,fill opacity=0.20] ( 71.99, 80.20) circle (  2.13);

\path[fill=fillColor,fill opacity=0.20] ( 62.36, 77.76) circle (  2.13);

\path[fill=fillColor,fill opacity=0.20] ( 86.24, 73.70) circle (  2.13);

\path[fill=fillColor,fill opacity=0.20] ( 85.25, 85.89) circle (  2.13);

\path[fill=fillColor,fill opacity=0.20] ( 94.10, 89.14) circle (  2.13);

\path[fill=fillColor,fill opacity=0.20] (101.96, 80.20) circle (  2.13);

\path[fill=fillColor,fill opacity=0.20] (102.94, 68.82) circle (  2.13);

\path[fill=fillColor,fill opacity=0.20] (106.87, 53.38) circle (  2.13);

\path[fill=fillColor,fill opacity=0.20] (125.54, 42.00) circle (  2.13);

\path[fill=fillColor,fill opacity=0.20] ( 80.34, 88.33) circle (  2.13);

\path[fill=fillColor,fill opacity=0.20] ( 78.37, 87.51) circle (  2.13);

\path[fill=fillColor,fill opacity=0.20] ( 76.41, 85.08) circle (  2.13);

\path[fill=fillColor,fill opacity=0.20] ( 69.43, 75.32) circle (  2.13);

\path[fill=fillColor,fill opacity=0.20] ( 78.37, 67.20) circle (  2.13);

\path[fill=fillColor,fill opacity=0.20] ( 74.44, 72.07) circle (  2.13);

\path[fill=fillColor,fill opacity=0.20] ( 81.32, 77.76) circle (  2.13);

\path[fill=fillColor,fill opacity=0.20] ( 88.20, 73.70) circle (  2.13);

\path[fill=fillColor,fill opacity=0.20] (115.72, 56.63) circle (  2.13);

\path[fill=fillColor,fill opacity=0.20] (134.39, 44.44) circle (  2.13);

\path[fill=fillColor,fill opacity=0.20] ( 94.10, 75.32) circle (  2.13);

\path[fill=fillColor,fill opacity=0.20] ( 81.32, 72.89) circle (  2.13);

\path[fill=fillColor,fill opacity=0.20] ( 85.25, 81.01) circle (  2.13);

\path[fill=fillColor,fill opacity=0.20] ( 85.25, 78.57) circle (  2.13);

\path[fill=fillColor,fill opacity=0.20] ( 89.18, 80.20) circle (  2.13);

\path[fill=fillColor,fill opacity=0.20] ( 94.10, 81.01) circle (  2.13);

\path[fill=fillColor,fill opacity=0.20] ( 96.06, 72.07) circle (  2.13);

\path[fill=fillColor,fill opacity=0.20] ( 89.18, 63.13) circle (  2.13);

\path[fill=fillColor,fill opacity=0.20] ( 99.01, 53.38) circle (  2.13);

\path[fill=fillColor,fill opacity=0.20] (111.79, 41.19) circle (  2.13);

\path[fill=fillColor,fill opacity=0.20] ( 83.29, 80.20) circle (  2.13);

\path[fill=fillColor,fill opacity=0.20] ( 77.39, 81.82) circle (  2.13);

\path[fill=fillColor,fill opacity=0.20] ( 76.41, 77.76) circle (  2.13);

\path[fill=fillColor,fill opacity=0.20] ( 75.43, 81.01) circle (  2.13);

\path[fill=fillColor,fill opacity=0.20] ( 67.76, 86.70) circle (  2.13);

\path[fill=fillColor,fill opacity=0.20] ( 69.92, 85.89) circle (  2.13);

\path[fill=fillColor,fill opacity=0.20] ( 82.31, 76.95) circle (  2.13);

\path[fill=fillColor,fill opacity=0.20] ( 87.22, 65.57) circle (  2.13);

\path[fill=fillColor,fill opacity=0.20] ( 91.15, 61.51) circle (  2.13);

\path[fill=fillColor,fill opacity=0.20] ( 85.25, 81.01) circle (  2.13);

\path[fill=fillColor,fill opacity=0.20] ( 87.22, 84.26) circle (  2.13);

\path[fill=fillColor,fill opacity=0.20] ( 92.13, 77.76) circle (  2.13);

\path[fill=fillColor,fill opacity=0.20] ( 92.13, 77.76) circle (  2.13);

\path[fill=fillColor,fill opacity=0.20] ( 93.11, 72.89) circle (  2.13);

\path[fill=fillColor,fill opacity=0.20] ( 97.05, 59.07) circle (  2.13);

\path[fill=fillColor,fill opacity=0.20] (101.96, 59.88) circle (  2.13);

\path[fill=fillColor,fill opacity=0.20] (107.86, 64.76) circle (  2.13);

\path[fill=fillColor,fill opacity=0.20] ( 95.08, 61.51) circle (  2.13);

\path[fill=fillColor,fill opacity=0.20] ( 85.25, 56.63) circle (  2.13);

\path[fill=fillColor,fill opacity=0.20] ( 89.18, 81.82) circle (  2.13);

\path[fill=fillColor,fill opacity=0.20] ( 87.22, 76.14) circle (  2.13);

\path[fill=fillColor,fill opacity=0.20] ( 79.36, 79.39) circle (  2.13);

\path[fill=fillColor,fill opacity=0.20] ( 75.43, 82.64) circle (  2.13);

\path[fill=fillColor,fill opacity=0.20] ( 73.46, 85.89) circle (  2.13);

\path[fill=fillColor,fill opacity=0.20] ( 75.43, 95.64) circle (  2.13);

\path[fill=fillColor,fill opacity=0.20] ( 64.03, 91.58) circle (  2.13);

\path[fill=fillColor,fill opacity=0.20] ( 86.24, 68.82) circle (  2.13);

\path[fill=fillColor,fill opacity=0.20] ( 96.06, 59.88) circle (  2.13);

\path[fill=fillColor,fill opacity=0.20] ( 89.18, 72.07) circle (  2.13);

\path[fill=fillColor,fill opacity=0.20] ( 99.01, 71.26) circle (  2.13);

\path[fill=fillColor,fill opacity=0.20] (108.84, 46.88) circle (  2.13);

\path[fill=fillColor,fill opacity=0.20] ( 79.36, 76.14) circle (  2.13);

\path[fill=fillColor,fill opacity=0.20] ( 86.24, 73.70) circle (  2.13);

\path[fill=fillColor,fill opacity=0.20] ( 91.15, 76.14) circle (  2.13);

\path[fill=fillColor,fill opacity=0.20] ( 96.06, 78.57) circle (  2.13);

\path[fill=fillColor,fill opacity=0.20] ( 95.08, 72.89) circle (  2.13);

\path[fill=fillColor,fill opacity=0.20] (100.98, 66.38) circle (  2.13);

\path[fill=fillColor,fill opacity=0.20] ( 89.18, 68.01) circle (  2.13);

\path[fill=fillColor,fill opacity=0.20] (104.91, 74.51) circle (  2.13);

\path[fill=fillColor,fill opacity=0.20] ( 71.40, 56.63) circle (  2.13);

\path[fill=fillColor,fill opacity=0.20] ( 93.11, 51.75) circle (  2.13);

\path[fill=fillColor,fill opacity=0.20] ( 93.11, 61.51) circle (  2.13);

\path[fill=fillColor,fill opacity=0.20] ( 88.20, 63.95) circle (  2.13);

\path[fill=fillColor,fill opacity=0.20] ( 87.22, 64.76) circle (  2.13);

\path[fill=fillColor,fill opacity=0.20] ( 84.27, 81.82) circle (  2.13);

\path[fill=fillColor,fill opacity=0.20] ( 81.32, 91.58) circle (  2.13);

\path[fill=fillColor,fill opacity=0.20] ( 76.41, 91.58) circle (  2.13);

\path[fill=fillColor,fill opacity=0.20] ( 85.25, 87.51) circle (  2.13);

\path[fill=fillColor,fill opacity=0.20] ( 90.17, 77.76) circle (  2.13);

\path[fill=fillColor,fill opacity=0.20] ( 92.13, 64.76) circle (  2.13);

\path[fill=fillColor,fill opacity=0.20] ( 99.99, 69.63) circle (  2.13);

\path[fill=fillColor,fill opacity=0.20] ( 85.25, 68.01) circle (  2.13);

\path[fill=fillColor,fill opacity=0.20] ( 81.32, 68.82) circle (  2.13);

\path[fill=fillColor,fill opacity=0.20] ( 86.24, 85.08) circle (  2.13);

\path[fill=fillColor,fill opacity=0.20] ( 93.11, 81.82) circle (  2.13);

\path[fill=fillColor,fill opacity=0.20] ( 97.05, 76.95) circle (  2.13);

\path[fill=fillColor,fill opacity=0.20] (103.92, 80.20) circle (  2.13);

\path[fill=fillColor,fill opacity=0.20] (103.92, 77.76) circle (  2.13);

\path[fill=fillColor,fill opacity=0.20] ( 99.01, 65.57) circle (  2.13);

\path[fill=fillColor,fill opacity=0.20] ( 94.10, 67.20) circle (  2.13);

\path[fill=fillColor,fill opacity=0.20] (119.65, 72.89) circle (  2.13);

\path[fill=fillColor,fill opacity=0.20] ( 93.11, 54.19) circle (  2.13);

\path[fill=fillColor,fill opacity=0.20] ( 90.17, 65.57) circle (  2.13);

\path[fill=fillColor,fill opacity=0.20] ( 91.15, 72.89) circle (  2.13);

\path[fill=fillColor,fill opacity=0.20] ( 89.18, 61.51) circle (  2.13);

\path[fill=fillColor,fill opacity=0.20] ( 83.29, 56.63) circle (  2.13);

\path[fill=fillColor,fill opacity=0.20] ( 78.37, 68.82) circle (  2.13);

\path[fill=fillColor,fill opacity=0.20] ( 86.24, 78.57) circle (  2.13);

\path[fill=fillColor,fill opacity=0.20] ( 81.32, 87.51) circle (  2.13);

\path[fill=fillColor,fill opacity=0.20] ( 84.27, 92.39) circle (  2.13);

\path[fill=fillColor,fill opacity=0.20] ( 91.15, 85.89) circle (  2.13);

\path[fill=fillColor,fill opacity=0.20] (101.96, 76.14) circle (  2.13);

\path[fill=fillColor,fill opacity=0.20] ( 96.06, 72.89) circle (  2.13);

\path[fill=fillColor,fill opacity=0.20] ( 99.99, 73.70) circle (  2.13);

\path[fill=fillColor,fill opacity=0.20] ( 67.96, 76.95) circle (  2.13);

\path[fill=fillColor,fill opacity=0.20] ( 86.24, 72.07) circle (  2.13);

\path[fill=fillColor,fill opacity=0.20] ( 84.27, 79.39) circle (  2.13);

\path[fill=fillColor,fill opacity=0.20] ( 92.13, 83.45) circle (  2.13);

\path[fill=fillColor,fill opacity=0.20] ( 98.03, 88.33) circle (  2.13);

\path[fill=fillColor,fill opacity=0.20] ( 92.13, 82.64) circle (  2.13);

\path[fill=fillColor,fill opacity=0.20] ( 99.01, 65.57) circle (  2.13);

\path[fill=fillColor,fill opacity=0.20] (100.98, 69.63) circle (  2.13);

\path[fill=fillColor,fill opacity=0.20] (106.87, 76.95) circle (  2.13);

\path[fill=fillColor,fill opacity=0.20] ( 99.99, 65.57) circle (  2.13);

\path[fill=fillColor,fill opacity=0.20] (107.86, 64.76) circle (  2.13);

\path[fill=fillColor,fill opacity=0.20] (121.61, 69.63) circle (  2.13);

\path[fill=fillColor,fill opacity=0.20] ( 90.17, 63.95) circle (  2.13);

\path[fill=fillColor,fill opacity=0.20] ( 87.22, 64.76) circle (  2.13);

\path[fill=fillColor,fill opacity=0.20] ( 85.25, 72.89) circle (  2.13);

\path[fill=fillColor,fill opacity=0.20] ( 83.29, 73.70) circle (  2.13);

\path[fill=fillColor,fill opacity=0.20] ( 85.25, 68.82) circle (  2.13);

\path[fill=fillColor,fill opacity=0.20] ( 81.32, 72.89) circle (  2.13);

\path[fill=fillColor,fill opacity=0.20] ( 83.29, 81.01) circle (  2.13);

\path[fill=fillColor,fill opacity=0.20] ( 79.36, 80.20) circle (  2.13);

\path[fill=fillColor,fill opacity=0.20] ( 86.24, 79.39) circle (  2.13);

\path[fill=fillColor,fill opacity=0.20] ( 88.20, 84.26) circle (  2.13);

\path[fill=fillColor,fill opacity=0.20] ( 92.13, 82.64) circle (  2.13);

\path[fill=fillColor,fill opacity=0.20] ( 99.01, 76.14) circle (  2.13);

\path[fill=fillColor,fill opacity=0.20] (100.98, 68.82) circle (  2.13);

\path[fill=fillColor,fill opacity=0.20] (101.96, 58.26) circle (  2.13);

\path[fill=fillColor,fill opacity=0.20] ( 70.61, 75.32) circle (  2.13);

\path[fill=fillColor,fill opacity=0.20] ( 84.27, 62.32) circle (  2.13);

\path[fill=fillColor,fill opacity=0.20] ( 81.32, 76.14) circle (  2.13);

\path[fill=fillColor,fill opacity=0.20] ( 88.20, 95.64) circle (  2.13);

\path[fill=fillColor,fill opacity=0.20] ( 82.31, 79.39) circle (  2.13);

\path[fill=fillColor,fill opacity=0.20] ( 89.18, 87.51) circle (  2.13);

\path[fill=fillColor,fill opacity=0.20] ( 95.08, 82.64) circle (  2.13);

\path[fill=fillColor,fill opacity=0.20] ( 96.06, 81.82) circle (  2.13);

\path[fill=fillColor,fill opacity=0.20] ( 96.06, 75.32) circle (  2.13);

\path[fill=fillColor,fill opacity=0.20] (102.94, 67.20) circle (  2.13);

\path[fill=fillColor,fill opacity=0.20] (106.87, 61.51) circle (  2.13);

\path[fill=fillColor,fill opacity=0.20] (102.94, 61.51) circle (  2.13);

\path[fill=fillColor,fill opacity=0.20] ( 87.22, 61.51) circle (  2.13);

\path[fill=fillColor,fill opacity=0.20] ( 86.24, 59.88) circle (  2.13);

\path[fill=fillColor,fill opacity=0.20] ( 81.32, 61.51) circle (  2.13);

\path[fill=fillColor,fill opacity=0.20] ( 79.36, 71.26) circle (  2.13);

\path[fill=fillColor,fill opacity=0.20] ( 81.32, 81.01) circle (  2.13);

\path[fill=fillColor,fill opacity=0.20] ( 80.34, 81.82) circle (  2.13);

\path[fill=fillColor,fill opacity=0.20] ( 81.32, 80.20) circle (  2.13);

\path[fill=fillColor,fill opacity=0.20] ( 86.24, 75.32) circle (  2.13);

\path[fill=fillColor,fill opacity=0.20] ( 87.22, 71.26) circle (  2.13);

\path[fill=fillColor,fill opacity=0.20] ( 92.13, 70.45) circle (  2.13);

\path[fill=fillColor,fill opacity=0.20] (100.98, 72.89) circle (  2.13);

\path[fill=fillColor,fill opacity=0.20] ( 96.06, 63.95) circle (  2.13);

\path[fill=fillColor,fill opacity=0.20] ( 98.03, 56.63) circle (  2.13);

\path[fill=fillColor,fill opacity=0.20] (105.89, 54.19) circle (  2.13);

\path[fill=fillColor,fill opacity=0.20] ( 73.46, 68.01) circle (  2.13);

\path[fill=fillColor,fill opacity=0.20] ( 81.32, 74.51) circle (  2.13);

\path[fill=fillColor,fill opacity=0.20] ( 92.13, 85.08) circle (  2.13);

\path[fill=fillColor,fill opacity=0.20] ( 87.22, 89.95) circle (  2.13);

\path[fill=fillColor,fill opacity=0.20] ( 87.22, 81.01) circle (  2.13);

\path[fill=fillColor,fill opacity=0.20] ( 85.25, 71.26) circle (  2.13);

\path[fill=fillColor,fill opacity=0.20] ( 91.15, 80.20) circle (  2.13);

\path[fill=fillColor,fill opacity=0.20] ( 88.20, 82.64) circle (  2.13);

\path[fill=fillColor,fill opacity=0.20] (101.96, 79.39) circle (  2.13);

\path[fill=fillColor,fill opacity=0.20] ( 98.03, 64.76) circle (  2.13);

\path[fill=fillColor,fill opacity=0.20] ( 99.99, 55.01) circle (  2.13);

\path[fill=fillColor,fill opacity=0.20] ( 92.13, 58.26) circle (  2.13);

\path[fill=fillColor,fill opacity=0.20] ( 88.20, 55.82) circle (  2.13);

\path[fill=fillColor,fill opacity=0.20] ( 89.18, 57.44) circle (  2.13);

\path[fill=fillColor,fill opacity=0.20] ( 70.22, 62.32) circle (  2.13);

\path[fill=fillColor,fill opacity=0.20] ( 80.34, 65.57) circle (  2.13);

\path[fill=fillColor,fill opacity=0.20] ( 78.37, 77.76) circle (  2.13);

\path[fill=fillColor,fill opacity=0.20] ( 78.37, 87.51) circle (  2.13);

\path[fill=fillColor,fill opacity=0.20] ( 83.29, 81.82) circle (  2.13);

\path[fill=fillColor,fill opacity=0.20] ( 79.36, 68.01) circle (  2.13);

\path[fill=fillColor,fill opacity=0.20] ( 92.13, 53.38) circle (  2.13);

\path[fill=fillColor,fill opacity=0.20] ( 92.13, 45.25) circle (  2.13);

\path[fill=fillColor,fill opacity=0.20] (100.98, 57.44) circle (  2.13);

\path[fill=fillColor,fill opacity=0.20] (105.89, 63.13) circle (  2.13);

\path[fill=fillColor,fill opacity=0.20] (112.77, 46.07) circle (  2.13);

\path[fill=fillColor,fill opacity=0.20] (129.47, 40.38) circle (  2.13);

\path[fill=fillColor,fill opacity=0.20] ( 75.43, 67.20) circle (  2.13);

\path[fill=fillColor,fill opacity=0.20] ( 89.18, 77.76) circle (  2.13);

\path[fill=fillColor,fill opacity=0.20] ( 88.20, 89.14) circle (  2.13);

\path[fill=fillColor,fill opacity=0.20] ( 87.22, 85.08) circle (  2.13);

\path[fill=fillColor,fill opacity=0.20] ( 85.25, 81.82) circle (  2.13);

\path[fill=fillColor,fill opacity=0.20] ( 91.15, 92.39) circle (  2.13);

\path[fill=fillColor,fill opacity=0.20] ( 90.17, 87.51) circle (  2.13);

\path[fill=fillColor,fill opacity=0.20] ( 97.05, 77.76) circle (  2.13);

\path[fill=fillColor,fill opacity=0.20] ( 97.05, 65.57) circle (  2.13);

\path[fill=fillColor,fill opacity=0.20] (104.91, 62.32) circle (  2.13);

\path[fill=fillColor,fill opacity=0.20] (107.86, 69.63) circle (  2.13);

\path[fill=fillColor,fill opacity=0.20] (101.96, 69.63) circle (  2.13);

\path[fill=fillColor,fill opacity=0.20] (102.94, 56.63) circle (  2.13);

\path[fill=fillColor,fill opacity=0.20] ( 97.05, 68.82) circle (  2.13);

\path[fill=fillColor,fill opacity=0.20] ( 83.29, 58.26) circle (  2.13);

\path[fill=fillColor,fill opacity=0.20] ( 84.27, 64.76) circle (  2.13);

\path[fill=fillColor,fill opacity=0.20] ( 85.25, 76.14) circle (  2.13);

\path[fill=fillColor,fill opacity=0.20] ( 86.24, 79.39) circle (  2.13);

\path[fill=fillColor,fill opacity=0.20] ( 82.31, 78.57) circle (  2.13);

\path[fill=fillColor,fill opacity=0.20] ( 86.24, 72.89) circle (  2.13);

\path[fill=fillColor,fill opacity=0.20] ( 80.34, 74.51) circle (  2.13);

\path[fill=fillColor,fill opacity=0.20] ( 84.27, 76.95) circle (  2.13);

\path[fill=fillColor,fill opacity=0.20] ( 92.13, 62.32) circle (  2.13);

\path[fill=fillColor,fill opacity=0.20] (105.89, 48.50) circle (  2.13);

\path[fill=fillColor,fill opacity=0.20] (100.98, 47.69) circle (  2.13);

\path[fill=fillColor,fill opacity=0.20] (104.91, 50.94) circle (  2.13);

\path[fill=fillColor,fill opacity=0.20] (104.91, 55.01) circle (  2.13);

\path[fill=fillColor,fill opacity=0.20] (142.25, 37.94) circle (  2.13);

\path[fill=fillColor,fill opacity=0.20] ( 92.13, 78.57) circle (  2.13);

\path[fill=fillColor,fill opacity=0.20] ( 78.37, 79.39) circle (  2.13);

\path[fill=fillColor,fill opacity=0.20] ( 83.29, 92.39) circle (  2.13);

\path[fill=fillColor,fill opacity=0.20] ( 91.15, 97.27) circle (  2.13);

\path[fill=fillColor,fill opacity=0.20] ( 83.29, 81.01) circle (  2.13);

\path[fill=fillColor,fill opacity=0.20] ( 79.36, 63.13) circle (  2.13);

\path[fill=fillColor,fill opacity=0.20] ( 91.15, 59.88) circle (  2.13);

\path[fill=fillColor,fill opacity=0.20] ( 99.01, 61.51) circle (  2.13);

\path[fill=fillColor,fill opacity=0.20] ( 92.13, 63.95) circle (  2.13);

\path[fill=fillColor,fill opacity=0.20] ( 98.03, 63.13) circle (  2.13);

\path[fill=fillColor,fill opacity=0.20] ( 95.08, 62.32) circle (  2.13);

\path[fill=fillColor,fill opacity=0.20] (103.92, 59.88) circle (  2.13);

\path[fill=fillColor,fill opacity=0.20] (109.82, 55.01) circle (  2.13);

\path[fill=fillColor,fill opacity=0.20] ( 92.13, 79.39) circle (  2.13);

\path[fill=fillColor,fill opacity=0.20] ( 82.31, 70.45) circle (  2.13);

\path[fill=fillColor,fill opacity=0.20] ( 72.28, 65.57) circle (  2.13);

\path[fill=fillColor,fill opacity=0.20] ( 77.39, 73.70) circle (  2.13);

\path[fill=fillColor,fill opacity=0.20] ( 77.39, 75.32) circle (  2.13);

\path[fill=fillColor,fill opacity=0.20] ( 88.20, 79.39) circle (  2.13);

\path[fill=fillColor,fill opacity=0.20] ( 89.18, 81.82) circle (  2.13);

\path[fill=fillColor,fill opacity=0.20] ( 86.24, 66.38) circle (  2.13);

\path[fill=fillColor,fill opacity=0.20] ( 98.03, 59.07) circle (  2.13);

\path[fill=fillColor,fill opacity=0.20] (105.89, 59.88) circle (  2.13);

\path[fill=fillColor,fill opacity=0.20] (115.72, 50.94) circle (  2.13);

\path[fill=fillColor,fill opacity=0.20] (108.84, 50.13) circle (  2.13);

\path[fill=fillColor,fill opacity=0.20] (123.58, 62.32) circle (  2.13);

\path[fill=fillColor,fill opacity=0.20] (130.46, 63.95) circle (  2.13);

\path[fill=fillColor,fill opacity=0.20] ( 51.06, 77.76) circle (  2.13);

\path[fill=fillColor,fill opacity=0.20] ( 77.39, 83.45) circle (  2.13);

\path[fill=fillColor,fill opacity=0.20] ( 77.39, 76.14) circle (  2.13);

\path[fill=fillColor,fill opacity=0.20] ( 76.41, 70.45) circle (  2.13);

\path[fill=fillColor,fill opacity=0.20] ( 86.24, 71.26) circle (  2.13);

\path[fill=fillColor,fill opacity=0.20] ( 89.18, 65.57) circle (  2.13);

\path[fill=fillColor,fill opacity=0.20] ( 92.13, 57.44) circle (  2.13);

\path[fill=fillColor,fill opacity=0.20] ( 99.99, 61.51) circle (  2.13);

\path[fill=fillColor,fill opacity=0.20] ( 95.08, 68.82) circle (  2.13);

\path[fill=fillColor,fill opacity=0.20] (105.89, 72.07) circle (  2.13);

\path[fill=fillColor,fill opacity=0.20] (111.79, 56.63) circle (  2.13);

\path[fill=fillColor,fill opacity=0.20] (113.75, 47.69) circle (  2.13);

\path[fill=fillColor,fill opacity=0.20] (111.79, 52.57) circle (  2.13);

\path[fill=fillColor,fill opacity=0.20] ( 75.43, 68.01) circle (  2.13);

\path[fill=fillColor,fill opacity=0.20] ( 88.20, 75.32) circle (  2.13);

\path[fill=fillColor,fill opacity=0.20] ( 92.13, 79.39) circle (  2.13);

\path[fill=fillColor,fill opacity=0.20] ( 81.32, 73.70) circle (  2.13);

\path[fill=fillColor,fill opacity=0.20] ( 85.25, 68.01) circle (  2.13);

\path[fill=fillColor,fill opacity=0.20] ( 88.20, 60.69) circle (  2.13);

\path[fill=fillColor,fill opacity=0.20] ( 71.69, 48.50) circle (  2.13);

\path[fill=fillColor,fill opacity=0.20] ( 93.11, 49.32) circle (  2.13);

\path[fill=fillColor,fill opacity=0.20] ( 70.32, 68.82) circle (  2.13);

\path[fill=fillColor,fill opacity=0.20] (102.94, 77.76) circle (  2.13);

\path[fill=fillColor,fill opacity=0.20] (114.73, 71.26) circle (  2.13);

\path[fill=fillColor,fill opacity=0.20] ( 82.31, 55.82) circle (  2.13);

\path[fill=fillColor,fill opacity=0.20] ( 79.36, 76.14) circle (  2.13);

\path[fill=fillColor,fill opacity=0.20] ( 81.32, 87.51) circle (  2.13);

\path[fill=fillColor,fill opacity=0.20] ( 85.25, 84.26) circle (  2.13);

\path[fill=fillColor,fill opacity=0.20] ( 94.10, 75.32) circle (  2.13);

\path[fill=fillColor,fill opacity=0.20] ( 93.11, 68.82) circle (  2.13);

\path[fill=fillColor,fill opacity=0.20] ( 90.17, 70.45) circle (  2.13);

\path[fill=fillColor,fill opacity=0.20] ( 99.99, 72.89) circle (  2.13);

\path[fill=fillColor,fill opacity=0.20] ( 99.99, 73.70) circle (  2.13);

\path[fill=fillColor,fill opacity=0.20] (106.87, 68.01) circle (  2.13);

\path[fill=fillColor,fill opacity=0.20] (112.77, 45.25) circle (  2.13);

\path[fill=fillColor,fill opacity=0.20] ( 99.99, 53.38) circle (  2.13);

\path[fill=fillColor,fill opacity=0.20] ( 92.13, 60.69) circle (  2.13);

\path[fill=fillColor,fill opacity=0.20] ( 89.18, 66.38) circle (  2.13);

\path[fill=fillColor,fill opacity=0.20] ( 93.11, 61.51) circle (  2.13);

\path[fill=fillColor,fill opacity=0.20] ( 95.08, 64.76) circle (  2.13);

\path[fill=fillColor,fill opacity=0.20] (102.94, 64.76) circle (  2.13);

\path[fill=fillColor,fill opacity=0.20] ( 99.01, 55.82) circle (  2.13);

\path[fill=fillColor,fill opacity=0.20] ( 98.03, 50.94) circle (  2.13);

\path[fill=fillColor,fill opacity=0.20] (100.98, 48.50) circle (  2.13);

\path[fill=fillColor,fill opacity=0.20] ( 87.22, 55.82) circle (  2.13);

\path[fill=fillColor,fill opacity=0.20] (101.96, 70.45) circle (  2.13);

\path[fill=fillColor,fill opacity=0.20] (119.65, 71.26) circle (  2.13);

\path[fill=fillColor,fill opacity=0.20] ( 60.39, 95.64) circle (  2.13);

\path[fill=fillColor,fill opacity=0.20] ( 94.10, 67.20) circle (  2.13);

\path[fill=fillColor,fill opacity=0.20] ( 81.32, 73.70) circle (  2.13);

\path[fill=fillColor,fill opacity=0.20] ( 82.31, 77.76) circle (  2.13);

\path[fill=fillColor,fill opacity=0.20] ( 92.13, 80.20) circle (  2.13);

\path[fill=fillColor,fill opacity=0.20] ( 92.13, 75.32) circle (  2.13);

\path[fill=fillColor,fill opacity=0.20] ( 94.10, 68.82) circle (  2.13);

\path[fill=fillColor,fill opacity=0.20] (100.98, 70.45) circle (  2.13);

\path[fill=fillColor,fill opacity=0.20] ( 91.15, 74.51) circle (  2.13);

\path[fill=fillColor,fill opacity=0.20] ( 92.13, 74.51) circle (  2.13);

\path[fill=fillColor,fill opacity=0.20] (101.96, 75.32) circle (  2.13);

\path[fill=fillColor,fill opacity=0.20] (104.91, 71.26) circle (  2.13);

\path[fill=fillColor,fill opacity=0.20] (110.80, 50.13) circle (  2.13);

\path[fill=fillColor,fill opacity=0.20] (111.79, 37.94) circle (  2.13);

\path[fill=fillColor,fill opacity=0.20] (110.80, 45.25) circle (  2.13);

\path[fill=fillColor,fill opacity=0.20] (110.80, 59.07) circle (  2.13);

\path[fill=fillColor,fill opacity=0.20] (101.96, 76.14) circle (  2.13);

\path[fill=fillColor,fill opacity=0.20] ( 93.11, 63.13) circle (  2.13);

\path[fill=fillColor,fill opacity=0.20] ( 91.15, 66.38) circle (  2.13);

\path[fill=fillColor,fill opacity=0.20] ( 90.17, 69.63) circle (  2.13);

\path[fill=fillColor,fill opacity=0.20] ( 92.13, 53.38) circle (  2.13);

\path[fill=fillColor,fill opacity=0.20] ( 84.27, 55.01) circle (  2.13);

\path[fill=fillColor,fill opacity=0.20] ( 87.22, 61.51) circle (  2.13);

\path[fill=fillColor,fill opacity=0.20] ( 96.06, 63.13) circle (  2.13);

\path[fill=fillColor,fill opacity=0.20] (105.89, 59.88) circle (  2.13);

\path[fill=fillColor,fill opacity=0.20] ( 78.37, 50.13) circle (  2.13);

\path[fill=fillColor,fill opacity=0.20] ( 79.36, 63.95) circle (  2.13);

\path[fill=fillColor,fill opacity=0.20] ( 77.39, 67.20) circle (  2.13);

\path[fill=fillColor,fill opacity=0.20] ( 87.22, 76.14) circle (  2.13);

\path[fill=fillColor,fill opacity=0.20] ( 89.18, 72.07) circle (  2.13);

\path[fill=fillColor,fill opacity=0.20] ( 91.15, 62.32) circle (  2.13);

\path[fill=fillColor,fill opacity=0.20] ( 93.11, 62.32) circle (  2.13);

\path[fill=fillColor,fill opacity=0.20] ( 90.17, 71.26) circle (  2.13);

\path[fill=fillColor,fill opacity=0.20] ( 87.22, 72.89) circle (  2.13);

\path[fill=fillColor,fill opacity=0.20] ( 93.11, 74.51) circle (  2.13);

\path[fill=fillColor,fill opacity=0.20] (100.98, 67.20) circle (  2.13);

\path[fill=fillColor,fill opacity=0.20] (103.92, 63.13) circle (  2.13);

\path[fill=fillColor,fill opacity=0.20] (101.96, 63.13) circle (  2.13);

\path[fill=fillColor,fill opacity=0.20] (101.96, 57.44) circle (  2.13);

\path[fill=fillColor,fill opacity=0.20] ( 91.15, 51.75) circle (  2.13);

\path[fill=fillColor,fill opacity=0.20] (107.86, 51.75) circle (  2.13);

\path[fill=fillColor,fill opacity=0.20] (107.86, 51.75) circle (  2.13);

\path[fill=fillColor,fill opacity=0.20] (100.98, 55.82) circle (  2.13);

\path[fill=fillColor,fill opacity=0.20] ( 99.01, 61.51) circle (  2.13);

\path[fill=fillColor,fill opacity=0.20] (106.87, 65.57) circle (  2.13);

\path[fill=fillColor,fill opacity=0.20] (103.92, 69.63) circle (  2.13);

\path[fill=fillColor,fill opacity=0.20] (107.86, 67.20) circle (  2.13);

\path[fill=fillColor,fill opacity=0.20] (105.89, 56.63) circle (  2.13);

\path[fill=fillColor,fill opacity=0.20] (105.89, 51.75) circle (  2.13);

\path[fill=fillColor,fill opacity=0.20] (100.98, 59.07) circle (  2.13);

\path[fill=fillColor,fill opacity=0.20] (100.98, 63.95) circle (  2.13);

\path[fill=fillColor,fill opacity=0.20] ( 97.05, 61.51) circle (  2.13);

\path[fill=fillColor,fill opacity=0.20] ( 99.01, 65.57) circle (  2.13);

\path[fill=fillColor,fill opacity=0.20] (101.96, 68.01) circle (  2.13);

\path[fill=fillColor,fill opacity=0.20] ( 97.05, 63.95) circle (  2.13);

\path[fill=fillColor,fill opacity=0.20] ( 95.08, 56.63) circle (  2.13);

\path[fill=fillColor,fill opacity=0.20] ( 83.29, 47.69) circle (  2.13);

\path[fill=fillColor,fill opacity=0.20] ( 83.29, 45.25) circle (  2.13);

\path[fill=fillColor,fill opacity=0.20] ( 82.31, 61.51) circle (  2.13);

\path[fill=fillColor,fill opacity=0.20] ( 90.17, 73.70) circle (  2.13);

\path[fill=fillColor,fill opacity=0.20] ( 90.17, 59.88) circle (  2.13);

\path[fill=fillColor,fill opacity=0.20] ( 90.17, 43.63) circle (  2.13);

\path[fill=fillColor,fill opacity=0.20] ( 88.20, 46.88) circle (  2.13);

\path[fill=fillColor,fill opacity=0.20] ( 96.06, 50.94) circle (  2.13);

\path[fill=fillColor,fill opacity=0.20] ( 99.01, 53.38) circle (  2.13);

\path[fill=fillColor,fill opacity=0.20] ( 61.77, 77.76) circle (  2.13);

\path[fill=fillColor,fill opacity=0.20] ( 91.15, 64.76) circle (  2.13);

\path[fill=fillColor,fill opacity=0.20] ( 81.32, 70.45) circle (  2.13);

\path[fill=fillColor,fill opacity=0.20] ( 97.05, 69.63) circle (  2.13);

\path[fill=fillColor,fill opacity=0.20] ( 96.06, 63.13) circle (  2.13);

\path[fill=fillColor,fill opacity=0.20] ( 88.20, 63.13) circle (  2.13);

\path[fill=fillColor,fill opacity=0.20] ( 80.34, 64.76) circle (  2.13);

\path[fill=fillColor,fill opacity=0.20] ( 91.15, 60.69) circle (  2.13);

\path[fill=fillColor,fill opacity=0.20] ( 94.10, 62.32) circle (  2.13);

\path[fill=fillColor,fill opacity=0.20] ( 92.13, 75.32) circle (  2.13);

\path[fill=fillColor,fill opacity=0.20] ( 98.03, 80.20) circle (  2.13);

\path[fill=fillColor,fill opacity=0.20] ( 99.01, 72.89) circle (  2.13);

\path[fill=fillColor,fill opacity=0.20] ( 99.01, 65.57) circle (  2.13);

\path[fill=fillColor,fill opacity=0.20] ( 95.08, 57.44) circle (  2.13);

\path[fill=fillColor,fill opacity=0.20] ( 97.05, 70.45) circle (  2.13);

\path[fill=fillColor,fill opacity=0.20] (100.98, 64.76) circle (  2.13);

\path[fill=fillColor,fill opacity=0.20] (105.89, 53.38) circle (  2.13);

\path[fill=fillColor,fill opacity=0.20] (101.96, 51.75) circle (  2.13);

\path[fill=fillColor,fill opacity=0.20] (100.98, 62.32) circle (  2.13);

\path[fill=fillColor,fill opacity=0.20] ( 99.01, 70.45) circle (  2.13);

\path[fill=fillColor,fill opacity=0.20] (101.96, 63.95) circle (  2.13);

\path[fill=fillColor,fill opacity=0.20] ( 97.05, 59.88) circle (  2.13);

\path[fill=fillColor,fill opacity=0.20] ( 92.13, 68.82) circle (  2.13);

\path[fill=fillColor,fill opacity=0.20] (100.98, 68.82) circle (  2.13);

\path[fill=fillColor,fill opacity=0.20] ( 99.01, 57.44) circle (  2.13);

\path[fill=fillColor,fill opacity=0.20] ( 99.01, 58.26) circle (  2.13);

\path[fill=fillColor,fill opacity=0.20] ( 92.13, 65.57) circle (  2.13);

\path[fill=fillColor,fill opacity=0.20] ( 87.22, 63.13) circle (  2.13);

\path[fill=fillColor,fill opacity=0.20] ( 88.20, 55.01) circle (  2.13);

\path[fill=fillColor,fill opacity=0.20] ( 92.13, 60.69) circle (  2.13);

\path[fill=fillColor,fill opacity=0.20] ( 99.99, 65.57) circle (  2.13);

\path[fill=fillColor,fill opacity=0.20] ( 96.06, 51.75) circle (  2.13);

\path[fill=fillColor,fill opacity=0.20] ( 84.27, 44.44) circle (  2.13);

\path[fill=fillColor,fill opacity=0.20] ( 70.91, 54.19) circle (  2.13);

\path[fill=fillColor,fill opacity=0.20] (107.86, 71.26) circle (  2.13);

\path[fill=fillColor,fill opacity=0.20] ( 95.08, 55.01) circle (  2.13);

\path[fill=fillColor,fill opacity=0.20] ( 89.18, 50.94) circle (  2.13);

\path[fill=fillColor,fill opacity=0.20] ( 99.99, 44.44) circle (  2.13);

\path[fill=fillColor,fill opacity=0.20] ( 96.06, 53.38) circle (  2.13);

\path[fill=fillColor,fill opacity=0.20] ( 92.13, 72.89) circle (  2.13);

\path[fill=fillColor,fill opacity=0.20] ( 94.10, 74.51) circle (  2.13);

\path[fill=fillColor,fill opacity=0.20] ( 92.13, 66.38) circle (  2.13);

\path[fill=fillColor,fill opacity=0.20] ( 93.11, 68.82) circle (  2.13);

\path[fill=fillColor,fill opacity=0.20] ( 91.15, 66.38) circle (  2.13);

\path[fill=fillColor,fill opacity=0.20] ( 87.22, 67.20) circle (  2.13);

\path[fill=fillColor,fill opacity=0.20] ( 93.11, 79.39) circle (  2.13);

\path[fill=fillColor,fill opacity=0.20] ( 94.10, 70.45) circle (  2.13);

\path[fill=fillColor,fill opacity=0.20] ( 97.05, 56.63) circle (  2.13);

\path[fill=fillColor,fill opacity=0.20] ( 94.10, 66.38) circle (  2.13);

\path[fill=fillColor,fill opacity=0.20] ( 92.13, 76.14) circle (  2.13);

\path[fill=fillColor,fill opacity=0.20] ( 92.13, 72.89) circle (  2.13);

\path[fill=fillColor,fill opacity=0.20] ( 80.34, 83.45) circle (  2.13);

\path[fill=fillColor,fill opacity=0.20] ( 96.06, 78.57) circle (  2.13);

\path[fill=fillColor,fill opacity=0.20] ( 96.06, 59.07) circle (  2.13);

\path[fill=fillColor,fill opacity=0.20] ( 96.06, 55.01) circle (  2.13);

\path[fill=fillColor,fill opacity=0.20] ( 91.15, 63.95) circle (  2.13);

\path[fill=fillColor,fill opacity=0.20] ( 90.17, 65.57) circle (  2.13);

\path[fill=fillColor,fill opacity=0.20] ( 94.10, 57.44) circle (  2.13);

\path[fill=fillColor,fill opacity=0.20] ( 98.03, 53.38) circle (  2.13);

\path[fill=fillColor,fill opacity=0.20] (101.96, 58.26) circle (  2.13);

\path[fill=fillColor,fill opacity=0.20] (105.89, 64.76) circle (  2.13);

\path[fill=fillColor,fill opacity=0.20] (112.77, 69.63) circle (  2.13);

\path[fill=fillColor,fill opacity=0.20] (120.63, 74.51) circle (  2.13);

\path[fill=fillColor,fill opacity=0.20] (100.98, 55.01) circle (  2.13);

\path[fill=fillColor,fill opacity=0.20] ( 99.99, 43.63) circle (  2.13);

\path[fill=fillColor,fill opacity=0.20] (101.96, 38.75) circle (  2.13);

\path[fill=fillColor,fill opacity=0.20] ( 91.15, 46.07) circle (  2.13);

\path[fill=fillColor,fill opacity=0.20] ( 93.11, 50.94) circle (  2.13);

\path[fill=fillColor,fill opacity=0.20] (108.84, 49.32) circle (  2.13);

\path[fill=fillColor,fill opacity=0.20] ( 99.01, 54.19) circle (  2.13);

\path[fill=fillColor,fill opacity=0.20] ( 91.15, 52.57) circle (  2.13);

\path[fill=fillColor,fill opacity=0.20] ( 95.08, 46.88) circle (  2.13);

\path[fill=fillColor,fill opacity=0.20] ( 99.01, 57.44) circle (  2.13);

\path[fill=fillColor,fill opacity=0.20] ( 75.43, 60.69) circle (  2.13);

\path[fill=fillColor,fill opacity=0.20] ( 87.22, 52.57) circle (  2.13);

\path[fill=fillColor,fill opacity=0.20] ( 92.13, 58.26) circle (  2.13);

\path[fill=fillColor,fill opacity=0.20] ( 94.10, 73.70) circle (  2.13);

\path[fill=fillColor,fill opacity=0.20] ( 89.18, 81.01) circle (  2.13);

\path[fill=fillColor,fill opacity=0.20] ( 93.11, 75.32) circle (  2.13);

\path[fill=fillColor,fill opacity=0.20] ( 84.27, 72.07) circle (  2.13);

\path[fill=fillColor,fill opacity=0.20] ( 82.31, 73.70) circle (  2.13);

\path[fill=fillColor,fill opacity=0.20] ( 88.20, 74.51) circle (  2.13);

\path[fill=fillColor,fill opacity=0.20] ( 94.10, 67.20) circle (  2.13);

\path[fill=fillColor,fill opacity=0.20] ( 98.03, 57.44) circle (  2.13);

\path[fill=fillColor,fill opacity=0.20] (100.98, 53.38) circle (  2.13);

\path[fill=fillColor,fill opacity=0.20] ( 99.01, 51.75) circle (  2.13);

\path[fill=fillColor,fill opacity=0.20] ( 99.01, 51.75) circle (  2.13);

\path[fill=fillColor,fill opacity=0.20] ( 97.05, 48.50) circle (  2.13);

\path[fill=fillColor,fill opacity=0.20] (101.96, 50.13) circle (  2.13);

\path[fill=fillColor,fill opacity=0.20] (108.84, 51.75) circle (  2.13);

\path[fill=fillColor,fill opacity=0.20] (111.79, 50.13) circle (  2.13);

\path[fill=fillColor,fill opacity=0.20] ( 99.99, 47.69) circle (  2.13);

\path[fill=fillColor,fill opacity=0.20] (106.87, 46.07) circle (  2.13);

\path[fill=fillColor,fill opacity=0.20] (115.72, 43.63) circle (  2.13);

\path[fill=fillColor,fill opacity=0.20] (112.77, 49.32) circle (  2.13);

\path[fill=fillColor,fill opacity=0.20] ( 56.17, 47.69) circle (  2.13);

\path[fill=fillColor,fill opacity=0.20] ( 99.01, 42.82) circle (  2.13);

\path[fill=fillColor,fill opacity=0.20] (100.98, 61.51) circle (  2.13);

\path[fill=fillColor,fill opacity=0.20] (100.98, 59.88) circle (  2.13);

\path[fill=fillColor,fill opacity=0.20] (100.98, 55.01) circle (  2.13);

\path[fill=fillColor,fill opacity=0.20] ( 94.10, 62.32) circle (  2.13);

\path[fill=fillColor,fill opacity=0.20] ( 99.01, 60.69) circle (  2.13);

\path[fill=fillColor,fill opacity=0.20] ( 98.03, 44.44) circle (  2.13);

\path[fill=fillColor,fill opacity=0.20] (102.94, 40.38) circle (  2.13);

\path[fill=fillColor,fill opacity=0.20] (102.94, 55.82) circle (  2.13);

\path[fill=fillColor,fill opacity=0.20] ( 55.87, 61.51) circle (  2.13);

\path[fill=fillColor,fill opacity=0.20] ( 93.11, 55.82) circle (  2.13);

\path[fill=fillColor,fill opacity=0.20] (126.53, 68.01) circle (  2.13);

\path[fill=fillColor,fill opacity=0.20] (101.96, 55.82) circle (  2.13);

\path[fill=fillColor,fill opacity=0.20] (104.91, 42.82) circle (  2.13);

\path[fill=fillColor,fill opacity=0.20] (103.92, 55.82) circle (  2.13);

\path[fill=fillColor,fill opacity=0.20] (105.89, 59.07) circle (  2.13);

\path[fill=fillColor,fill opacity=0.20] (102.94, 62.32) circle (  2.13);

\path[fill=fillColor,fill opacity=0.20] ( 93.11, 70.45) circle (  2.13);

\path[fill=fillColor,fill opacity=0.20] ( 78.37, 65.57) circle (  2.13);

\path[fill=fillColor,fill opacity=0.20] (112.77, 54.19) circle (  2.13);

\path[fill=fillColor,fill opacity=0.20] (124.56, 54.19) circle (  2.13);

\path[fill=fillColor,fill opacity=0.20] (120.63, 61.51) circle (  2.13);

\path[fill=fillColor,fill opacity=0.20] (112.77, 63.13) circle (  2.13);

\path[fill=fillColor,fill opacity=0.20] ( 73.46, 56.63) circle (  2.13);

\path[fill=fillColor,fill opacity=0.20] (123.58, 59.88) circle (  2.13);

\path[fill=fillColor,fill opacity=0.20] ( 73.46, 88.33) circle (  2.13);

\path[fill=fillColor,fill opacity=0.20] ( 74.44, 86.70) circle (  2.13);

\path[fill=fillColor,fill opacity=0.20] ( 88.20, 77.76) circle (  2.13);

\path[fill=fillColor,fill opacity=0.20] ( 89.18, 74.51) circle (  2.13);

\path[fill=fillColor,fill opacity=0.20] ( 83.29, 94.83) circle (  2.13);

\path[fill=fillColor,fill opacity=0.20] ( 75.43, 93.20) circle (  2.13);

\path[fill=fillColor,fill opacity=0.20] ( 77.39, 94.83) circle (  2.13);

\path[fill=fillColor,fill opacity=0.20] ( 82.31, 90.76) circle (  2.13);

\path[fill=fillColor,fill opacity=0.20] ( 83.29, 82.64) circle (  2.13);

\path[fill=fillColor,fill opacity=0.20] ( 82.31, 81.82) circle (  2.13);

\path[fill=fillColor,fill opacity=0.20] ( 93.11, 75.32) circle (  2.13);

\path[fill=fillColor,fill opacity=0.20] ( 94.10, 63.13) circle (  2.13);

\path[fill=fillColor,fill opacity=0.20] ( 88.20, 57.44) circle (  2.13);

\path[fill=fillColor,fill opacity=0.20] (109.82, 47.69) circle (  2.13);

\path[fill=fillColor,fill opacity=0.20] ( 87.22, 87.51) circle (  2.13);

\path[fill=fillColor,fill opacity=0.20] ( 88.20, 84.26) circle (  2.13);

\path[fill=fillColor,fill opacity=0.20] ( 84.27, 92.39) circle (  2.13);

\path[fill=fillColor,fill opacity=0.20] ( 81.32, 99.70) circle (  2.13);

\path[fill=fillColor,fill opacity=0.20] ( 85.25, 94.83) circle (  2.13);

\path[fill=fillColor,fill opacity=0.20] ( 87.22, 88.33) circle (  2.13);

\path[fill=fillColor,fill opacity=0.20] ( 90.17, 81.82) circle (  2.13);

\path[fill=fillColor,fill opacity=0.20] ( 96.06, 76.14) circle (  2.13);

\path[fill=fillColor,fill opacity=0.20] ( 94.10, 65.57) circle (  2.13);

\path[fill=fillColor,fill opacity=0.20] ( 99.99, 51.75) circle (  2.13);

\path[fill=fillColor,fill opacity=0.20] ( 99.99, 42.82) circle (  2.13);

\path[fill=fillColor,fill opacity=0.20] ( 99.01, 98.89) circle (  2.13);

\path[fill=fillColor,fill opacity=0.20] ( 89.18, 84.26) circle (  2.13);

\path[fill=fillColor,fill opacity=0.20] ( 82.31, 90.76) circle (  2.13);

\path[fill=fillColor,fill opacity=0.20] ( 79.36, 97.27) circle (  2.13);

\path[fill=fillColor,fill opacity=0.20] ( 77.39,105.39) circle (  2.13);

\path[fill=fillColor,fill opacity=0.20] ( 85.25,102.14) circle (  2.13);

\path[fill=fillColor,fill opacity=0.20] ( 87.22, 97.27) circle (  2.13);

\path[fill=fillColor,fill opacity=0.20] ( 92.13, 89.14) circle (  2.13);

\path[fill=fillColor,fill opacity=0.20] ( 98.03, 81.01) circle (  2.13);

\path[fill=fillColor,fill opacity=0.20] ( 99.01, 74.51) circle (  2.13);

\path[fill=fillColor,fill opacity=0.20] (100.98, 69.63) circle (  2.13);

\path[fill=fillColor,fill opacity=0.20] ( 93.11, 65.57) circle (  2.13);

\path[fill=fillColor,fill opacity=0.20] (108.84, 56.63) circle (  2.13);

\path[fill=fillColor,fill opacity=0.20] (103.92, 42.82) circle (  2.13);

\path[fill=fillColor,fill opacity=0.20] ( 80.34, 37.94) circle (  2.13);

\path[fill=fillColor,fill opacity=0.20] (151.09, 43.63) circle (  2.13);

\path[fill=fillColor,fill opacity=0.20] ( 83.29, 95.64) circle (  2.13);

\path[fill=fillColor,fill opacity=0.20] ( 86.24,105.39) circle (  2.13);

\path[fill=fillColor,fill opacity=0.20] ( 89.18,104.58) circle (  2.13);

\path[fill=fillColor,fill opacity=0.20] ( 86.24,106.21) circle (  2.13);

\path[fill=fillColor,fill opacity=0.20] ( 88.20,109.46) circle (  2.13);

\path[fill=fillColor,fill opacity=0.20] ( 93.11,100.52) circle (  2.13);

\path[fill=fillColor,fill opacity=0.20] ( 99.99, 92.39) circle (  2.13);

\path[fill=fillColor,fill opacity=0.20] ( 99.01, 81.82) circle (  2.13);

\path[fill=fillColor,fill opacity=0.20] (102.94, 70.45) circle (  2.13);

\path[fill=fillColor,fill opacity=0.20] (109.82, 68.82) circle (  2.13);

\path[fill=fillColor,fill opacity=0.20] (105.89, 73.70) circle (  2.13);

\path[fill=fillColor,fill opacity=0.20] ( 98.03, 66.38) circle (  2.13);

\path[fill=fillColor,fill opacity=0.20] ( 96.06, 94.83) circle (  2.13);

\path[fill=fillColor,fill opacity=0.20] ( 85.25, 92.39) circle (  2.13);

\path[fill=fillColor,fill opacity=0.20] ( 90.17,104.58) circle (  2.13);

\path[fill=fillColor,fill opacity=0.20] ( 89.18, 98.08) circle (  2.13);

\path[fill=fillColor,fill opacity=0.20] ( 96.06, 99.70) circle (  2.13);

\path[fill=fillColor,fill opacity=0.20] ( 98.03,107.83) circle (  2.13);

\path[fill=fillColor,fill opacity=0.20] ( 96.06, 99.70) circle (  2.13);

\path[fill=fillColor,fill opacity=0.20] ( 99.99, 86.70) circle (  2.13);

\path[fill=fillColor,fill opacity=0.20] (101.96, 78.57) circle (  2.13);

\path[fill=fillColor,fill opacity=0.20] (104.91, 63.13) circle (  2.13);

\path[fill=fillColor,fill opacity=0.20] (124.56, 55.82) circle (  2.13);

\path[fill=fillColor,fill opacity=0.20] ( 86.24, 82.64) circle (  2.13);

\path[fill=fillColor,fill opacity=0.20] ( 89.18, 90.76) circle (  2.13);

\path[fill=fillColor,fill opacity=0.20] ( 86.24, 85.89) circle (  2.13);

\path[fill=fillColor,fill opacity=0.20] ( 87.22, 94.83) circle (  2.13);

\path[fill=fillColor,fill opacity=0.20] ( 79.36, 92.39) circle (  2.13);

\path[fill=fillColor,fill opacity=0.20] ( 97.05, 94.02) circle (  2.13);

\path[fill=fillColor,fill opacity=0.20] (101.96,100.52) circle (  2.13);

\path[fill=fillColor,fill opacity=0.20] (102.94, 96.45) circle (  2.13);

\path[fill=fillColor,fill opacity=0.20] (107.86, 81.01) circle (  2.13);

\path[fill=fillColor,fill opacity=0.20] (107.86, 71.26) circle (  2.13);

\path[fill=fillColor,fill opacity=0.20] (122.60, 69.63) circle (  2.13);

\path[fill=fillColor,fill opacity=0.20] ( 96.06, 63.95) circle (  2.13);

\path[fill=fillColor,fill opacity=0.20] ( 92.13, 62.32) circle (  2.13);

\path[fill=fillColor,fill opacity=0.20] ( 93.11, 62.32) circle (  2.13);

\path[fill=fillColor,fill opacity=0.20] ( 99.99, 63.95) circle (  2.13);

\path[fill=fillColor,fill opacity=0.20] ( 96.06,102.96) circle (  2.13);

\path[fill=fillColor,fill opacity=0.20] ( 90.17, 96.45) circle (  2.13);

\path[fill=fillColor,fill opacity=0.20] ( 87.22, 97.27) circle (  2.13);

\path[fill=fillColor,fill opacity=0.20] ( 89.18, 94.83) circle (  2.13);

\path[fill=fillColor,fill opacity=0.20] ( 97.05, 94.02) circle (  2.13);

\path[fill=fillColor,fill opacity=0.20] ( 99.99, 90.76) circle (  2.13);

\path[fill=fillColor,fill opacity=0.20] (104.91, 86.70) circle (  2.13);

\path[fill=fillColor,fill opacity=0.20] (113.75, 76.95) circle (  2.13);

\path[fill=fillColor,fill opacity=0.20] (118.66, 70.45) circle (  2.13);

\path[fill=fillColor,fill opacity=0.20] (138.32, 73.70) circle (  2.13);

\path[fill=fillColor,fill opacity=0.20] ( 94.10, 79.39) circle (  2.13);

\path[fill=fillColor,fill opacity=0.20] (102.94, 74.51) circle (  2.13);

\path[fill=fillColor,fill opacity=0.20] ( 93.11, 87.51) circle (  2.13);

\path[fill=fillColor,fill opacity=0.20] ( 87.22, 83.45) circle (  2.13);

\path[fill=fillColor,fill opacity=0.20] ( 95.08, 77.76) circle (  2.13);

\path[fill=fillColor,fill opacity=0.20] (100.98, 66.38) circle (  2.13);

\path[fill=fillColor,fill opacity=0.20] (101.96, 47.69) circle (  2.13);

\path[fill=fillColor,fill opacity=0.20] ( 95.08, 66.38) circle (  2.13);

\path[fill=fillColor,fill opacity=0.20] (105.89,111.89) circle (  2.13);

\path[fill=fillColor,fill opacity=0.20] ( 79.36,100.52) circle (  2.13);

\path[fill=fillColor,fill opacity=0.20] ( 87.22, 98.89) circle (  2.13);

\path[fill=fillColor,fill opacity=0.20] ( 95.08, 90.76) circle (  2.13);

\path[fill=fillColor,fill opacity=0.20] ( 97.05, 90.76) circle (  2.13);

\path[fill=fillColor,fill opacity=0.20] ( 92.13, 86.70) circle (  2.13);

\path[fill=fillColor,fill opacity=0.20] (107.86, 77.76) circle (  2.13);

\path[fill=fillColor,fill opacity=0.20] (114.73, 68.01) circle (  2.13);

\path[fill=fillColor,fill opacity=0.20] ( 91.15, 78.57) circle (  2.13);

\path[fill=fillColor,fill opacity=0.20] ( 93.11, 85.08) circle (  2.13);

\path[fill=fillColor,fill opacity=0.20] ( 91.15, 87.51) circle (  2.13);

\path[fill=fillColor,fill opacity=0.20] ( 93.11, 78.57) circle (  2.13);

\path[fill=fillColor,fill opacity=0.20] ( 88.20, 73.70) circle (  2.13);

\path[fill=fillColor,fill opacity=0.20] ( 93.11, 72.07) circle (  2.13);

\path[fill=fillColor,fill opacity=0.20] (104.91, 60.69) circle (  2.13);

\path[fill=fillColor,fill opacity=0.20] ( 98.03, 52.57) circle (  2.13);

\path[fill=fillColor,fill opacity=0.20] (105.89,106.21) circle (  2.13);

\path[fill=fillColor,fill opacity=0.20] ( 79.36, 81.01) circle (  2.13);

\path[fill=fillColor,fill opacity=0.20] ( 73.46, 87.51) circle (  2.13);

\path[fill=fillColor,fill opacity=0.20] ( 96.06, 80.20) circle (  2.13);

\path[fill=fillColor,fill opacity=0.20] ( 91.15, 84.26) circle (  2.13);

\path[fill=fillColor,fill opacity=0.20] ( 92.13, 85.89) circle (  2.13);

\path[fill=fillColor,fill opacity=0.20] (107.86, 71.26) circle (  2.13);

\path[fill=fillColor,fill opacity=0.20] (102.94, 63.95) circle (  2.13);

\path[fill=fillColor,fill opacity=0.20] (109.82, 59.07) circle (  2.13);

\path[fill=fillColor,fill opacity=0.20] ( 88.20, 89.95) circle (  2.13);

\path[fill=fillColor,fill opacity=0.20] ( 82.31, 77.76) circle (  2.13);

\path[fill=fillColor,fill opacity=0.20] ( 85.25, 81.82) circle (  2.13);

\path[fill=fillColor,fill opacity=0.20] ( 92.13, 77.76) circle (  2.13);

\path[fill=fillColor,fill opacity=0.20] (100.98, 71.26) circle (  2.13);

\path[fill=fillColor,fill opacity=0.20] ( 96.06, 62.32) circle (  2.13);

\path[fill=fillColor,fill opacity=0.20] ( 95.08, 62.32) circle (  2.13);

\path[fill=fillColor,fill opacity=0.20] (100.98, 61.51) circle (  2.13);

\path[fill=fillColor,fill opacity=0.20] ( 96.06, 50.13) circle (  2.13);

\path[fill=fillColor,fill opacity=0.20] (102.94,115.15) circle (  2.13);

\path[fill=fillColor,fill opacity=0.20] ( 86.24, 70.45) circle (  2.13);

\path[fill=fillColor,fill opacity=0.20] ( 87.22, 79.39) circle (  2.13);

\path[fill=fillColor,fill opacity=0.20] ( 96.06, 85.08) circle (  2.13);

\path[fill=fillColor,fill opacity=0.20] ( 98.03, 85.89) circle (  2.13);

\path[fill=fillColor,fill opacity=0.20] (100.98, 85.89) circle (  2.13);

\path[fill=fillColor,fill opacity=0.20] (107.86, 72.07) circle (  2.13);

\path[fill=fillColor,fill opacity=0.20] (106.87, 55.82) circle (  2.13);

\path[fill=fillColor,fill opacity=0.20] (107.86, 46.88) circle (  2.13);

\path[fill=fillColor,fill opacity=0.20] ( 87.22, 78.57) circle (  2.13);

\path[fill=fillColor,fill opacity=0.20] ( 79.36, 73.70) circle (  2.13);

\path[fill=fillColor,fill opacity=0.20] ( 86.24, 88.33) circle (  2.13);

\path[fill=fillColor,fill opacity=0.20] ( 90.17, 84.26) circle (  2.13);

\path[fill=fillColor,fill opacity=0.20] ( 90.17, 74.51) circle (  2.13);

\path[fill=fillColor,fill opacity=0.20] ( 96.06, 59.88) circle (  2.13);

\path[fill=fillColor,fill opacity=0.20] ( 93.11, 58.26) circle (  2.13);

\path[fill=fillColor,fill opacity=0.20] ( 99.99, 59.88) circle (  2.13);

\path[fill=fillColor,fill opacity=0.20] (104.91, 46.07) circle (  2.13);

\path[fill=fillColor,fill opacity=0.20] ( 87.22, 52.57) circle (  2.13);

\path[fill=fillColor,fill opacity=0.20] (118.66, 90.76) circle (  2.13);

\path[fill=fillColor,fill opacity=0.20] ( 98.03, 78.57) circle (  2.13);

\path[fill=fillColor,fill opacity=0.20] ( 89.18, 90.76) circle (  2.13);

\path[fill=fillColor,fill opacity=0.20] ( 90.17, 89.95) circle (  2.13);

\path[fill=fillColor,fill opacity=0.20] ( 97.05, 81.01) circle (  2.13);

\path[fill=fillColor,fill opacity=0.20] (100.98, 71.26) circle (  2.13);

\path[fill=fillColor,fill opacity=0.20] (109.82, 59.88) circle (  2.13);

\path[fill=fillColor,fill opacity=0.20] (111.79, 51.75) circle (  2.13);

\path[fill=fillColor,fill opacity=0.20] ( 81.32, 79.39) circle (  2.13);

\path[fill=fillColor,fill opacity=0.20] ( 82.31, 86.70) circle (  2.13);

\path[fill=fillColor,fill opacity=0.20] ( 86.24, 99.70) circle (  2.13);

\path[fill=fillColor,fill opacity=0.20] ( 87.22, 89.14) circle (  2.13);

\path[fill=fillColor,fill opacity=0.20] ( 86.24, 75.32) circle (  2.13);

\path[fill=fillColor,fill opacity=0.20] ( 92.13, 65.57) circle (  2.13);

\path[fill=fillColor,fill opacity=0.20] ( 87.22, 67.20) circle (  2.13);

\path[fill=fillColor,fill opacity=0.20] ( 95.08, 66.38) circle (  2.13);

\path[fill=fillColor,fill opacity=0.20] (102.94, 46.07) circle (  2.13);

\path[fill=fillColor,fill opacity=0.20] (104.91, 42.00) circle (  2.13);

\path[fill=fillColor,fill opacity=0.20] ( 99.01, 74.51) circle (  2.13);

\path[fill=fillColor,fill opacity=0.20] ( 88.20, 73.70) circle (  2.13);

\path[fill=fillColor,fill opacity=0.20] ( 88.20, 80.20) circle (  2.13);

\path[fill=fillColor,fill opacity=0.20] ( 97.05, 75.32) circle (  2.13);

\path[fill=fillColor,fill opacity=0.20] ( 99.01, 74.51) circle (  2.13);

\path[fill=fillColor,fill opacity=0.20] (100.98, 77.76) circle (  2.13);

\path[fill=fillColor,fill opacity=0.20] (109.82, 72.89) circle (  2.13);

\path[fill=fillColor,fill opacity=0.20] (113.75, 58.26) circle (  2.13);

\path[fill=fillColor,fill opacity=0.20] ( 86.24, 82.64) circle (  2.13);

\path[fill=fillColor,fill opacity=0.20] ( 90.17, 93.20) circle (  2.13);

\path[fill=fillColor,fill opacity=0.20] ( 88.20, 95.64) circle (  2.13);

\path[fill=fillColor,fill opacity=0.20] ( 91.15, 85.89) circle (  2.13);

\path[fill=fillColor,fill opacity=0.20] ( 91.15, 82.64) circle (  2.13);

\path[fill=fillColor,fill opacity=0.20] ( 94.10, 77.76) circle (  2.13);

\path[fill=fillColor,fill opacity=0.20] ( 87.22, 74.51) circle (  2.13);

\path[fill=fillColor,fill opacity=0.20] ( 87.22, 70.45) circle (  2.13);

\path[fill=fillColor,fill opacity=0.20] ( 99.99, 55.01) circle (  2.13);

\path[fill=fillColor,fill opacity=0.20] (150.11, 79.39) circle (  2.13);

\path[fill=fillColor,fill opacity=0.20] (111.79, 63.95) circle (  2.13);

\path[fill=fillColor,fill opacity=0.20] ( 99.99, 60.69) circle (  2.13);

\path[fill=fillColor,fill opacity=0.20] ( 99.01, 74.51) circle (  2.13);

\path[fill=fillColor,fill opacity=0.20] ( 98.03, 81.82) circle (  2.13);

\path[fill=fillColor,fill opacity=0.20] ( 93.11, 81.01) circle (  2.13);

\path[fill=fillColor,fill opacity=0.20] ( 99.99, 77.76) circle (  2.13);

\path[fill=fillColor,fill opacity=0.20] (107.86, 64.76) circle (  2.13);

\path[fill=fillColor,fill opacity=0.20] (114.73, 51.75) circle (  2.13);

\path[fill=fillColor,fill opacity=0.20] ( 95.08, 92.39) circle (  2.13);

\path[fill=fillColor,fill opacity=0.20] ( 95.08, 78.57) circle (  2.13);

\path[fill=fillColor,fill opacity=0.20] ( 95.08, 77.76) circle (  2.13);

\path[fill=fillColor,fill opacity=0.20] ( 91.15, 82.64) circle (  2.13);

\path[fill=fillColor,fill opacity=0.20] ( 90.17, 88.33) circle (  2.13);

\path[fill=fillColor,fill opacity=0.20] ( 81.32, 89.95) circle (  2.13);

\path[fill=fillColor,fill opacity=0.20] ( 83.29, 84.26) circle (  2.13);

\path[fill=fillColor,fill opacity=0.20] ( 85.25, 76.14) circle (  2.13);

\path[fill=fillColor,fill opacity=0.20] ( 83.29, 72.07) circle (  2.13);

\path[fill=fillColor,fill opacity=0.20] ( 95.08, 63.13) circle (  2.13);

\path[fill=fillColor,fill opacity=0.20] ( 99.01, 47.69) circle (  2.13);

\path[fill=fillColor,fill opacity=0.20] ( 90.17, 53.38) circle (  2.13);

\path[fill=fillColor,fill opacity=0.20] (146.18, 55.01) circle (  2.13);

\path[fill=fillColor,fill opacity=0.20] (112.77, 64.76) circle (  2.13);

\path[fill=fillColor,fill opacity=0.20] ( 98.03, 79.39) circle (  2.13);

\path[fill=fillColor,fill opacity=0.20] ( 90.17, 72.89) circle (  2.13);

\path[fill=fillColor,fill opacity=0.20] ( 99.01, 66.38) circle (  2.13);

\path[fill=fillColor,fill opacity=0.20] (103.92, 68.01) circle (  2.13);

\path[fill=fillColor,fill opacity=0.20] (104.91, 64.76) circle (  2.13);

\path[fill=fillColor,fill opacity=0.20] (104.91, 59.07) circle (  2.13);

\path[fill=fillColor,fill opacity=0.20] ( 93.11, 74.51) circle (  2.13);

\path[fill=fillColor,fill opacity=0.20] ( 91.15, 81.82) circle (  2.13);

\path[fill=fillColor,fill opacity=0.20] ( 87.22, 72.89) circle (  2.13);

\path[fill=fillColor,fill opacity=0.20] ( 88.20, 69.63) circle (  2.13);

\path[fill=fillColor,fill opacity=0.20] ( 87.22, 82.64) circle (  2.13);

\path[fill=fillColor,fill opacity=0.20] ( 84.27, 94.02) circle (  2.13);

\path[fill=fillColor,fill opacity=0.20] ( 79.36, 88.33) circle (  2.13);

\path[fill=fillColor,fill opacity=0.20] ( 80.34, 81.01) circle (  2.13);

\path[fill=fillColor,fill opacity=0.20] ( 84.27, 76.95) circle (  2.13);

\path[fill=fillColor,fill opacity=0.20] ( 82.31, 76.14) circle (  2.13);

\path[fill=fillColor,fill opacity=0.20] ( 88.20, 68.82) circle (  2.13);

\path[fill=fillColor,fill opacity=0.20] (100.98, 57.44) circle (  2.13);

\path[fill=fillColor,fill opacity=0.20] (132.42, 63.95) circle (  2.13);

\path[fill=fillColor,fill opacity=0.20] (139.30, 50.94) circle (  2.13);

\path[fill=fillColor,fill opacity=0.20] (103.92, 59.07) circle (  2.13);

\path[fill=fillColor,fill opacity=0.20] ( 92.13, 66.38) circle (  2.13);

\path[fill=fillColor,fill opacity=0.20] ( 99.01, 66.38) circle (  2.13);

\path[fill=fillColor,fill opacity=0.20] (100.98, 68.82) circle (  2.13);

\path[fill=fillColor,fill opacity=0.20] (100.98, 72.89) circle (  2.13);

\path[fill=fillColor,fill opacity=0.20] (108.84, 62.32) circle (  2.13);

\path[fill=fillColor,fill opacity=0.20] ( 88.20, 75.32) circle (  2.13);

\path[fill=fillColor,fill opacity=0.20] ( 84.27, 76.14) circle (  2.13);

\path[fill=fillColor,fill opacity=0.20] ( 83.29, 82.64) circle (  2.13);

\path[fill=fillColor,fill opacity=0.20] ( 79.36, 81.01) circle (  2.13);

\path[fill=fillColor,fill opacity=0.20] ( 80.34, 89.14) circle (  2.13);

\path[fill=fillColor,fill opacity=0.20] ( 75.43, 96.45) circle (  2.13);

\path[fill=fillColor,fill opacity=0.20] ( 79.36, 93.20) circle (  2.13);

\path[fill=fillColor,fill opacity=0.20] ( 81.32, 86.70) circle (  2.13);

\path[fill=fillColor,fill opacity=0.20] ( 79.36, 79.39) circle (  2.13);

\path[fill=fillColor,fill opacity=0.20] ( 81.32, 77.76) circle (  2.13);

\path[fill=fillColor,fill opacity=0.20] ( 77.39, 80.20) circle (  2.13);

\path[fill=fillColor,fill opacity=0.20] ( 88.20, 67.20) circle (  2.13);

\path[fill=fillColor,fill opacity=0.20] ( 98.03, 50.94) circle (  2.13);

\path[fill=fillColor,fill opacity=0.20] (133.40, 61.51) circle (  2.13);

\path[fill=fillColor,fill opacity=0.20] ( 86.24, 48.50) circle (  2.13);

\path[fill=fillColor,fill opacity=0.20] (103.92, 59.07) circle (  2.13);

\path[fill=fillColor,fill opacity=0.20] ( 97.05, 72.89) circle (  2.13);

\path[fill=fillColor,fill opacity=0.20] (101.96, 70.45) circle (  2.13);

\path[fill=fillColor,fill opacity=0.20] (102.94, 70.45) circle (  2.13);

\path[fill=fillColor,fill opacity=0.20] (104.91, 67.20) circle (  2.13);

\path[fill=fillColor,fill opacity=0.20] (107.86, 59.07) circle (  2.13);

\path[fill=fillColor,fill opacity=0.20] ( 91.15, 70.45) circle (  2.13);

\path[fill=fillColor,fill opacity=0.20] ( 90.17, 76.95) circle (  2.13);

\path[fill=fillColor,fill opacity=0.20] ( 86.24, 81.01) circle (  2.13);

\path[fill=fillColor,fill opacity=0.20] ( 84.27, 78.57) circle (  2.13);

\path[fill=fillColor,fill opacity=0.20] ( 82.31, 89.14) circle (  2.13);

\path[fill=fillColor,fill opacity=0.20] ( 79.36,107.02) circle (  2.13);

\path[fill=fillColor,fill opacity=0.20] ( 73.46,105.39) circle (  2.13);

\path[fill=fillColor,fill opacity=0.20] ( 78.37, 94.02) circle (  2.13);

\path[fill=fillColor,fill opacity=0.20] ( 87.22, 83.45) circle (  2.13);

\path[fill=fillColor,fill opacity=0.20] ( 82.31, 77.76) circle (  2.13);

\path[fill=fillColor,fill opacity=0.20] ( 81.32, 80.20) circle (  2.13);

\path[fill=fillColor,fill opacity=0.20] ( 78.37, 81.82) circle (  2.13);

\path[fill=fillColor,fill opacity=0.20] ( 93.11, 64.76) circle (  2.13);

\path[fill=fillColor,fill opacity=0.20] (105.89, 49.32) circle (  2.13);

\path[fill=fillColor,fill opacity=0.20] (153.06, 59.88) circle (  2.13);

\path[fill=fillColor,fill opacity=0.20] (111.79, 50.94) circle (  2.13);

\path[fill=fillColor,fill opacity=0.20] (110.80, 63.95) circle (  2.13);

\path[fill=fillColor,fill opacity=0.20] ( 93.11, 74.51) circle (  2.13);

\path[fill=fillColor,fill opacity=0.20] ( 98.03, 72.89) circle (  2.13);

\path[fill=fillColor,fill opacity=0.20] ( 97.05, 69.63) circle (  2.13);

\path[fill=fillColor,fill opacity=0.20] ( 97.05, 68.82) circle (  2.13);

\path[fill=fillColor,fill opacity=0.20] (103.92, 63.13) circle (  2.13);

\path[fill=fillColor,fill opacity=0.20] (111.79, 59.88) circle (  2.13);

\path[fill=fillColor,fill opacity=0.20] ( 99.99, 71.26) circle (  2.13);

\path[fill=fillColor,fill opacity=0.20] ( 95.08, 71.26) circle (  2.13);

\path[fill=fillColor,fill opacity=0.20] ( 84.27, 76.95) circle (  2.13);

\path[fill=fillColor,fill opacity=0.20] ( 86.24, 78.57) circle (  2.13);

\path[fill=fillColor,fill opacity=0.20] ( 82.31, 81.82) circle (  2.13);

\path[fill=fillColor,fill opacity=0.20] ( 83.29, 91.58) circle (  2.13);

\path[fill=fillColor,fill opacity=0.20] ( 79.36,100.52) circle (  2.13);

\path[fill=fillColor,fill opacity=0.20] ( 75.43,103.77) circle (  2.13);

\path[fill=fillColor,fill opacity=0.20] ( 78.37, 94.02) circle (  2.13);

\path[fill=fillColor,fill opacity=0.20] ( 85.25, 81.01) circle (  2.13);

\path[fill=fillColor,fill opacity=0.20] ( 82.31, 82.64) circle (  2.13);

\path[fill=fillColor,fill opacity=0.20] ( 75.43, 88.33) circle (  2.13);

\path[fill=fillColor,fill opacity=0.20] ( 78.37, 81.01) circle (  2.13);

\path[fill=fillColor,fill opacity=0.20] ( 92.13, 65.57) circle (  2.13);

\path[fill=fillColor,fill opacity=0.20] (104.91, 59.88) circle (  2.13);

\path[fill=fillColor,fill opacity=0.20] ( 73.46, 66.38) circle (  2.13);

\path[fill=fillColor,fill opacity=0.20] ( 94.10, 68.82) circle (  2.13);

\path[fill=fillColor,fill opacity=0.20] ( 96.06, 69.63) circle (  2.13);

\path[fill=fillColor,fill opacity=0.20] ( 93.11, 74.51) circle (  2.13);

\path[fill=fillColor,fill opacity=0.20] (104.91, 64.76) circle (  2.13);

\path[fill=fillColor,fill opacity=0.20] (107.86, 57.44) circle (  2.13);

\path[fill=fillColor,fill opacity=0.20] (110.80, 68.82) circle (  2.13);

\path[fill=fillColor,fill opacity=0.20] (100.98, 72.07) circle (  2.13);

\path[fill=fillColor,fill opacity=0.20] ( 99.01, 82.64) circle (  2.13);

\path[fill=fillColor,fill opacity=0.20] ( 82.31, 81.82) circle (  2.13);

\path[fill=fillColor,fill opacity=0.20] ( 83.29, 83.45) circle (  2.13);

\path[fill=fillColor,fill opacity=0.20] ( 83.29, 90.76) circle (  2.13);

\path[fill=fillColor,fill opacity=0.20] ( 80.34, 91.58) circle (  2.13);

\path[fill=fillColor,fill opacity=0.20] ( 83.29, 90.76) circle (  2.13);

\path[fill=fillColor,fill opacity=0.20] ( 79.36, 92.39) circle (  2.13);

\path[fill=fillColor,fill opacity=0.20] ( 80.34, 94.83) circle (  2.13);

\path[fill=fillColor,fill opacity=0.20] ( 82.31, 90.76) circle (  2.13);

\path[fill=fillColor,fill opacity=0.20] ( 80.34, 85.08) circle (  2.13);

\path[fill=fillColor,fill opacity=0.20] ( 62.65, 89.14) circle (  2.13);

\path[fill=fillColor,fill opacity=0.20] ( 74.44, 84.26) circle (  2.13);

\path[fill=fillColor,fill opacity=0.20] ( 86.24, 70.45) circle (  2.13);

\path[fill=fillColor,fill opacity=0.20] (109.82, 66.38) circle (  2.13);

\path[fill=fillColor,fill opacity=0.20] (107.86, 53.38) circle (  2.13);

\path[fill=fillColor,fill opacity=0.20] (111.79, 50.13) circle (  2.13);

\path[fill=fillColor,fill opacity=0.20] (102.94, 64.76) circle (  2.13);

\path[fill=fillColor,fill opacity=0.20] ( 95.08, 75.32) circle (  2.13);

\path[fill=fillColor,fill opacity=0.20] (100.98, 63.95) circle (  2.13);

\path[fill=fillColor,fill opacity=0.20] (103.92, 61.51) circle (  2.13);

\path[fill=fillColor,fill opacity=0.20] (105.89, 68.01) circle (  2.13);

\path[fill=fillColor,fill opacity=0.20] (104.91, 65.57) circle (  2.13);

\path[fill=fillColor,fill opacity=0.20] ( 99.99, 72.07) circle (  2.13);

\path[fill=fillColor,fill opacity=0.20] ( 94.10, 76.14) circle (  2.13);

\path[fill=fillColor,fill opacity=0.20] ( 88.20, 89.95) circle (  2.13);

\path[fill=fillColor,fill opacity=0.20] ( 81.32, 91.58) circle (  2.13);

\path[fill=fillColor,fill opacity=0.20] ( 80.34, 93.20) circle (  2.13);

\path[fill=fillColor,fill opacity=0.20] ( 80.34, 99.70) circle (  2.13);

\path[fill=fillColor,fill opacity=0.20] ( 79.36, 91.58) circle (  2.13);

\path[fill=fillColor,fill opacity=0.20] ( 79.36, 85.08) circle (  2.13);

\path[fill=fillColor,fill opacity=0.20] ( 79.36, 92.39) circle (  2.13);

\path[fill=fillColor,fill opacity=0.20] ( 56.17, 93.20) circle (  2.13);

\path[fill=fillColor,fill opacity=0.20] ( 81.32, 86.70) circle (  2.13);

\path[fill=fillColor,fill opacity=0.20] ( 80.34, 82.64) circle (  2.13);

\path[fill=fillColor,fill opacity=0.20] ( 85.25, 76.95) circle (  2.13);

\path[fill=fillColor,fill opacity=0.20] ( 85.25, 60.69) circle (  2.13);

\path[fill=fillColor,fill opacity=0.20] ( 80.34, 58.26) circle (  2.13);

\path[fill=fillColor,fill opacity=0.20] (154.04, 68.82) circle (  2.13);

\path[fill=fillColor,fill opacity=0.20] (132.42, 57.44) circle (  2.13);

\path[fill=fillColor,fill opacity=0.20] (110.80, 60.69) circle (  2.13);

\path[fill=fillColor,fill opacity=0.20] (100.98, 64.76) circle (  2.13);

\path[fill=fillColor,fill opacity=0.20] ( 97.05, 65.57) circle (  2.13);

\path[fill=fillColor,fill opacity=0.20] ( 94.10, 61.51) circle (  2.13);

\path[fill=fillColor,fill opacity=0.20] (100.98, 57.44) circle (  2.13);

\path[fill=fillColor,fill opacity=0.20] (103.92, 55.82) circle (  2.13);

\path[fill=fillColor,fill opacity=0.20] (106.87, 55.01) circle (  2.13);

\path[fill=fillColor,fill opacity=0.20] ( 91.15, 70.45) circle (  2.13);

\path[fill=fillColor,fill opacity=0.20] ( 86.24, 72.07) circle (  2.13);

\path[fill=fillColor,fill opacity=0.20] ( 83.29, 89.14) circle (  2.13);

\path[fill=fillColor,fill opacity=0.20] ( 83.29, 94.83) circle (  2.13);

\path[fill=fillColor,fill opacity=0.20] ( 78.37, 92.39) circle (  2.13);

\path[fill=fillColor,fill opacity=0.20] ( 77.39, 95.64) circle (  2.13);

\path[fill=fillColor,fill opacity=0.20] ( 71.20, 85.08) circle (  2.13);

\path[fill=fillColor,fill opacity=0.20] ( 75.43, 78.57) circle (  2.13);

\path[fill=fillColor,fill opacity=0.20] ( 76.41, 89.14) circle (  2.13);

\path[fill=fillColor,fill opacity=0.20] ( 75.43, 87.51) circle (  2.13);

\path[fill=fillColor,fill opacity=0.20] ( 81.32, 77.76) circle (  2.13);

\path[fill=fillColor,fill opacity=0.20] ( 85.25, 75.32) circle (  2.13);

\path[fill=fillColor,fill opacity=0.20] ( 95.08, 66.38) circle (  2.13);

\path[fill=fillColor,fill opacity=0.20] (131.44, 55.82) circle (  2.13);

\path[fill=fillColor,fill opacity=0.20] (143.23, 64.76) circle (  2.13);

\path[fill=fillColor,fill opacity=0.20] ( 89.18, 53.38) circle (  2.13);

\path[fill=fillColor,fill opacity=0.20] (102.94, 60.69) circle (  2.13);

\path[fill=fillColor,fill opacity=0.20] ( 98.03, 65.57) circle (  2.13);

\path[fill=fillColor,fill opacity=0.20] (103.92, 57.44) circle (  2.13);

\path[fill=fillColor,fill opacity=0.20] (106.87, 56.63) circle (  2.13);

\path[fill=fillColor,fill opacity=0.20] (115.72, 54.19) circle (  2.13);

\path[fill=fillColor,fill opacity=0.20] (116.70, 50.94) circle (  2.13);

\path[fill=fillColor,fill opacity=0.20] (103.92, 55.01) circle (  2.13);

\path[fill=fillColor,fill opacity=0.20] ( 98.03, 68.01) circle (  2.13);

\path[fill=fillColor,fill opacity=0.20] ( 89.18, 76.95) circle (  2.13);

\path[fill=fillColor,fill opacity=0.20] ( 85.25, 74.51) circle (  2.13);

\path[fill=fillColor,fill opacity=0.20] ( 85.25, 80.20) circle (  2.13);

\path[fill=fillColor,fill opacity=0.20] ( 84.27, 89.14) circle (  2.13);

\path[fill=fillColor,fill opacity=0.20] ( 82.31, 89.95) circle (  2.13);

\path[fill=fillColor,fill opacity=0.20] ( 80.34, 91.58) circle (  2.13);

\path[fill=fillColor,fill opacity=0.20] ( 78.37, 92.39) circle (  2.13);

\path[fill=fillColor,fill opacity=0.20] ( 69.24, 88.33) circle (  2.13);

\path[fill=fillColor,fill opacity=0.20] ( 49.98, 80.20) circle (  2.13);

\path[fill=fillColor,fill opacity=0.20] ( 79.36, 76.14) circle (  2.13);

\path[fill=fillColor,fill opacity=0.20] ( 83.29, 72.89) circle (  2.13);

\path[fill=fillColor,fill opacity=0.20] ( 90.17, 67.20) circle (  2.13);

\path[fill=fillColor,fill opacity=0.20] ( 99.01, 75.32) circle (  2.13);

\path[fill=fillColor,fill opacity=0.20] ( 99.01, 50.94) circle (  2.13);

\path[fill=fillColor,fill opacity=0.20] ( 99.01, 60.69) circle (  2.13);

\path[fill=fillColor,fill opacity=0.20] (104.91, 68.01) circle (  2.13);

\path[fill=fillColor,fill opacity=0.20] (107.86, 67.20) circle (  2.13);

\path[fill=fillColor,fill opacity=0.20] (106.87, 67.20) circle (  2.13);

\path[fill=fillColor,fill opacity=0.20] (109.82, 63.13) circle (  2.13);

\path[fill=fillColor,fill opacity=0.20] (110.80, 59.07) circle (  2.13);

\path[fill=fillColor,fill opacity=0.20] (106.87, 50.94) circle (  2.13);

\path[fill=fillColor,fill opacity=0.20] (104.91, 59.88) circle (  2.13);

\path[fill=fillColor,fill opacity=0.20] ( 96.06, 59.07) circle (  2.13);

\path[fill=fillColor,fill opacity=0.20] ( 92.13, 61.51) circle (  2.13);

\path[fill=fillColor,fill opacity=0.20] ( 86.24, 73.70) circle (  2.13);

\path[fill=fillColor,fill opacity=0.20] ( 80.34, 77.76) circle (  2.13);

\path[fill=fillColor,fill opacity=0.20] ( 81.32, 73.70) circle (  2.13);

\path[fill=fillColor,fill opacity=0.20] ( 85.25, 79.39) circle (  2.13);

\path[fill=fillColor,fill opacity=0.20] ( 79.36, 87.51) circle (  2.13);

\path[fill=fillColor,fill opacity=0.20] ( 78.37, 89.14) circle (  2.13);

\path[fill=fillColor,fill opacity=0.20] ( 77.39, 91.58) circle (  2.13);

\path[fill=fillColor,fill opacity=0.20] ( 82.31, 88.33) circle (  2.13);

\path[fill=fillColor,fill opacity=0.20] ( 84.27, 85.08) circle (  2.13);

\path[fill=fillColor,fill opacity=0.20] ( 83.29, 82.64) circle (  2.13);

\path[fill=fillColor,fill opacity=0.20] ( 83.29, 71.26) circle (  2.13);

\path[fill=fillColor,fill opacity=0.20] ( 95.08, 62.32) circle (  2.13);

\path[fill=fillColor,fill opacity=0.20] (142.25, 69.63) circle (  2.13);

\path[fill=fillColor,fill opacity=0.20] (134.39, 55.01) circle (  2.13);

\path[fill=fillColor,fill opacity=0.20] (116.70, 48.50) circle (  2.13);

\path[fill=fillColor,fill opacity=0.20] ( 92.13, 64.76) circle (  2.13);

\path[fill=fillColor,fill opacity=0.20] ( 96.06, 71.26) circle (  2.13);

\path[fill=fillColor,fill opacity=0.20] ( 99.01, 73.70) circle (  2.13);

\path[fill=fillColor,fill opacity=0.20] ( 93.11, 68.82) circle (  2.13);

\path[fill=fillColor,fill opacity=0.20] ( 99.99, 59.88) circle (  2.13);

\path[fill=fillColor,fill opacity=0.20] (105.89, 49.32) circle (  2.13);

\path[fill=fillColor,fill opacity=0.20] (105.89, 64.76) circle (  2.13);

\path[fill=fillColor,fill opacity=0.20] (104.91, 82.64) circle (  2.13);

\path[fill=fillColor,fill opacity=0.20] (102.94, 51.75) circle (  2.13);

\path[fill=fillColor,fill opacity=0.20] (100.98, 55.82) circle (  2.13);

\path[fill=fillColor,fill opacity=0.20] ( 93.11, 63.95) circle (  2.13);

\path[fill=fillColor,fill opacity=0.20] ( 89.18, 63.95) circle (  2.13);

\path[fill=fillColor,fill opacity=0.20] ( 81.32, 70.45) circle (  2.13);

\path[fill=fillColor,fill opacity=0.20] ( 76.41, 82.64) circle (  2.13);

\path[fill=fillColor,fill opacity=0.20] ( 76.41, 79.39) circle (  2.13);

\path[fill=fillColor,fill opacity=0.20] ( 73.46, 70.45) circle (  2.13);

\path[fill=fillColor,fill opacity=0.20] ( 83.29, 69.63) circle (  2.13);

\path[fill=fillColor,fill opacity=0.20] ( 81.32, 76.14) circle (  2.13);

\path[fill=fillColor,fill opacity=0.20] ( 86.24, 79.39) circle (  2.13);

\path[fill=fillColor,fill opacity=0.20] ( 87.22, 81.82) circle (  2.13);

\path[fill=fillColor,fill opacity=0.20] ( 92.13, 84.26) circle (  2.13);

\path[fill=fillColor,fill opacity=0.20] ( 99.01, 89.95) circle (  2.13);

\path[fill=fillColor,fill opacity=0.20] (119.65, 85.08) circle (  2.13);

\path[fill=fillColor,fill opacity=0.20] (116.70, 68.82) circle (  2.13);

\path[fill=fillColor,fill opacity=0.20] (105.89, 52.57) circle (  2.13);

\path[fill=fillColor,fill opacity=0.20] (103.92, 59.07) circle (  2.13);

\path[fill=fillColor,fill opacity=0.20] ( 92.13, 65.57) circle (  2.13);

\path[fill=fillColor,fill opacity=0.20] ( 96.06, 67.20) circle (  2.13);

\path[fill=fillColor,fill opacity=0.20] ( 98.03, 66.38) circle (  2.13);

\path[fill=fillColor,fill opacity=0.20] ( 98.03, 60.69) circle (  2.13);

\path[fill=fillColor,fill opacity=0.20] ( 97.05, 51.75) circle (  2.13);

\path[fill=fillColor,fill opacity=0.20] (108.84, 51.75) circle (  2.13);

\path[fill=fillColor,fill opacity=0.20] (106.87, 62.32) circle (  2.13);

\path[fill=fillColor,fill opacity=0.20] (107.86, 62.32) circle (  2.13);

\path[fill=fillColor,fill opacity=0.20] (111.79, 54.19) circle (  2.13);

\path[fill=fillColor,fill opacity=0.20] ( 99.99, 59.07) circle (  2.13);

\path[fill=fillColor,fill opacity=0.20] (102.94, 47.69) circle (  2.13);

\path[fill=fillColor,fill opacity=0.20] ( 97.05, 56.63) circle (  2.13);

\path[fill=fillColor,fill opacity=0.20] ( 95.08, 71.26) circle (  2.13);

\path[fill=fillColor,fill opacity=0.20] ( 88.20, 75.32) circle (  2.13);

\path[fill=fillColor,fill opacity=0.20] ( 80.34, 75.32) circle (  2.13);

\path[fill=fillColor,fill opacity=0.20] ( 83.29, 75.32) circle (  2.13);

\path[fill=fillColor,fill opacity=0.20] ( 80.34, 79.39) circle (  2.13);

\path[fill=fillColor,fill opacity=0.20] ( 75.43, 75.32) circle (  2.13);

\path[fill=fillColor,fill opacity=0.20] ( 79.36, 70.45) circle (  2.13);

\path[fill=fillColor,fill opacity=0.20] ( 85.25, 70.45) circle (  2.13);

\path[fill=fillColor,fill opacity=0.20] ( 91.15, 69.63) circle (  2.13);

\path[fill=fillColor,fill opacity=0.20] ( 99.01, 71.26) circle (  2.13);

\path[fill=fillColor,fill opacity=0.20] (104.91, 75.32) circle (  2.13);

\path[fill=fillColor,fill opacity=0.20] (113.75, 76.14) circle (  2.13);

\path[fill=fillColor,fill opacity=0.20] (102.94, 85.08) circle (  2.13);

\path[fill=fillColor,fill opacity=0.20] (106.87, 57.44) circle (  2.13);

\path[fill=fillColor,fill opacity=0.20] ( 96.06, 63.13) circle (  2.13);

\path[fill=fillColor,fill opacity=0.20] (101.96, 70.45) circle (  2.13);

\path[fill=fillColor,fill opacity=0.20] ( 92.13, 69.63) circle (  2.13);

\path[fill=fillColor,fill opacity=0.20] ( 99.99, 60.69) circle (  2.13);

\path[fill=fillColor,fill opacity=0.20] (102.94, 60.69) circle (  2.13);

\path[fill=fillColor,fill opacity=0.20] (107.86, 62.32) circle (  2.13);

\path[fill=fillColor,fill opacity=0.20] ( 97.05, 59.07) circle (  2.13);

\path[fill=fillColor,fill opacity=0.20] ( 98.03, 59.07) circle (  2.13);

\path[fill=fillColor,fill opacity=0.20] (102.94, 52.57) circle (  2.13);

\path[fill=fillColor,fill opacity=0.20] (101.96, 61.51) circle (  2.13);

\path[fill=fillColor,fill opacity=0.20] ( 97.05, 62.32) circle (  2.13);

\path[fill=fillColor,fill opacity=0.20] ( 92.13, 65.57) circle (  2.13);

\path[fill=fillColor,fill opacity=0.20] ( 83.29, 73.70) circle (  2.13);

\path[fill=fillColor,fill opacity=0.20] ( 78.37, 79.39) circle (  2.13);

\path[fill=fillColor,fill opacity=0.20] ( 83.29, 75.32) circle (  2.13);

\path[fill=fillColor,fill opacity=0.20] ( 84.27, 68.82) circle (  2.13);

\path[fill=fillColor,fill opacity=0.20] ( 81.32, 68.01) circle (  2.13);

\path[fill=fillColor,fill opacity=0.20] ( 88.20, 65.57) circle (  2.13);

\path[fill=fillColor,fill opacity=0.20] ( 95.08, 63.95) circle (  2.13);

\path[fill=fillColor,fill opacity=0.20] ( 62.85, 70.45) circle (  2.13);

\path[fill=fillColor,fill opacity=0.20] (113.75, 78.57) circle (  2.13);

\path[fill=fillColor,fill opacity=0.20] (108.84, 78.57) circle (  2.13);

\path[fill=fillColor,fill opacity=0.20] ( 91.15, 59.07) circle (  2.13);

\path[fill=fillColor,fill opacity=0.20] (107.86, 59.07) circle (  2.13);

\path[fill=fillColor,fill opacity=0.20] (103.92, 63.13) circle (  2.13);

\path[fill=fillColor,fill opacity=0.20] (100.98, 63.95) circle (  2.13);

\path[fill=fillColor,fill opacity=0.20] (105.89, 67.20) circle (  2.13);

\path[fill=fillColor,fill opacity=0.20] (106.87, 67.20) circle (  2.13);

\path[fill=fillColor,fill opacity=0.20] (107.86, 58.26) circle (  2.13);

\path[fill=fillColor,fill opacity=0.20] ( 99.99, 60.69) circle (  2.13);

\path[fill=fillColor,fill opacity=0.20] ( 98.03, 68.82) circle (  2.13);

\path[fill=fillColor,fill opacity=0.20] (104.91, 61.51) circle (  2.13);

\path[fill=fillColor,fill opacity=0.20] (106.87, 61.51) circle (  2.13);

\path[fill=fillColor,fill opacity=0.20] (105.89, 67.20) circle (  2.13);

\path[fill=fillColor,fill opacity=0.20] ( 93.11, 40.38) circle (  2.13);

\path[fill=fillColor,fill opacity=0.20] ( 99.99, 42.82) circle (  2.13);

\path[fill=fillColor,fill opacity=0.20] ( 99.99, 45.25) circle (  2.13);

\path[fill=fillColor,fill opacity=0.20] (111.79, 47.69) circle (  2.13);

\path[fill=fillColor,fill opacity=0.20] (106.87, 50.13) circle (  2.13);

\path[fill=fillColor,fill opacity=0.20] (114.73, 45.25) circle (  2.13);

\path[fill=fillColor,fill opacity=0.20] (114.73, 46.88) circle (  2.13);

\path[fill=fillColor,fill opacity=0.20] (107.86, 50.94) circle (  2.13);

\path[fill=fillColor,fill opacity=0.20] (104.91, 50.94) circle (  2.13);

\path[fill=fillColor,fill opacity=0.20] ( 94.10, 52.57) circle (  2.13);

\path[fill=fillColor,fill opacity=0.20] ( 87.22, 67.20) circle (  2.13);

\path[fill=fillColor,fill opacity=0.20] ( 88.20, 62.32) circle (  2.13);

\path[fill=fillColor,fill opacity=0.20] ( 95.08, 65.57) circle (  2.13);

\path[fill=fillColor,fill opacity=0.20] ( 93.11, 60.69) circle (  2.13);

\path[fill=fillColor,fill opacity=0.20] ( 91.15, 51.75) circle (  2.13);

\path[fill=fillColor,fill opacity=0.20] ( 61.18, 55.01) circle (  2.13);

\path[fill=fillColor,fill opacity=0.20] (104.91, 63.95) circle (  2.13);

\path[fill=fillColor,fill opacity=0.20] (114.73, 70.45) circle (  2.13);

\path[fill=fillColor,fill opacity=0.20] (117.68, 81.82) circle (  2.13);

\path[fill=fillColor,fill opacity=0.20] (111.79, 57.44) circle (  2.13);

\path[fill=fillColor,fill opacity=0.20] (115.72, 53.38) circle (  2.13);

\path[fill=fillColor,fill opacity=0.20] (102.94, 59.07) circle (  2.13);

\path[fill=fillColor,fill opacity=0.20] (105.89, 59.88) circle (  2.13);

\path[fill=fillColor,fill opacity=0.20] (101.96, 56.63) circle (  2.13);

\path[fill=fillColor,fill opacity=0.20] (100.98, 63.13) circle (  2.13);

\path[fill=fillColor,fill opacity=0.20] ( 99.01, 74.51) circle (  2.13);

\path[fill=fillColor,fill opacity=0.20] ( 97.05, 72.89) circle (  2.13);

\path[fill=fillColor,fill opacity=0.20] (104.91, 63.13) circle (  2.13);

\path[fill=fillColor,fill opacity=0.20] (103.92, 65.57) circle (  2.13);

\path[fill=fillColor,fill opacity=0.20] (110.80, 67.20) circle (  2.13);

\path[fill=fillColor,fill opacity=0.20] (107.86, 54.19) circle (  2.13);

\path[fill=fillColor,fill opacity=0.20] (101.96, 46.07) circle (  2.13);

\path[fill=fillColor,fill opacity=0.20] (102.94, 44.44) circle (  2.13);

\path[fill=fillColor,fill opacity=0.20] (106.87, 42.00) circle (  2.13);

\path[fill=fillColor,fill opacity=0.20] ( 99.99, 50.94) circle (  2.13);

\path[fill=fillColor,fill opacity=0.20] ( 83.29, 56.63) circle (  2.13);

\path[fill=fillColor,fill opacity=0.20] ( 90.17, 42.00) circle (  2.13);

\path[fill=fillColor,fill opacity=0.20] (101.96, 44.44) circle (  2.13);

\path[fill=fillColor,fill opacity=0.20] (107.86, 49.32) circle (  2.13);

\path[fill=fillColor,fill opacity=0.20] (109.82, 49.32) circle (  2.13);

\path[fill=fillColor,fill opacity=0.20] (112.77, 52.57) circle (  2.13);

\path[fill=fillColor,fill opacity=0.20] (104.91, 55.01) circle (  2.13);

\path[fill=fillColor,fill opacity=0.20] (101.96, 56.63) circle (  2.13);

\path[fill=fillColor,fill opacity=0.20] ( 83.29, 62.32) circle (  2.13);

\path[fill=fillColor,fill opacity=0.20] ( 82.31, 63.13) circle (  2.13);

\path[fill=fillColor,fill opacity=0.20] ( 93.11, 49.32) circle (  2.13);

\path[fill=fillColor,fill opacity=0.20] ( 88.20, 49.32) circle (  2.13);

\path[fill=fillColor,fill opacity=0.20] ( 87.22, 59.07) circle (  2.13);

\path[fill=fillColor,fill opacity=0.20] (132.42, 67.20) circle (  2.13);

\path[fill=fillColor,fill opacity=0.20] (112.77, 79.39) circle (  2.13);

\path[fill=fillColor,fill opacity=0.20] (128.49, 56.63) circle (  2.13);

\path[fill=fillColor,fill opacity=0.20] (104.91, 52.57) circle (  2.13);

\path[fill=fillColor,fill opacity=0.20] (108.84, 64.76) circle (  2.13);

\path[fill=fillColor,fill opacity=0.20] (107.86, 70.45) circle (  2.13);

\path[fill=fillColor,fill opacity=0.20] (114.73, 65.57) circle (  2.13);

\path[fill=fillColor,fill opacity=0.20] (117.68, 55.82) circle (  2.13);

\path[fill=fillColor,fill opacity=0.20] (110.80, 63.13) circle (  2.13);

\path[fill=fillColor,fill opacity=0.20] ( 99.01, 72.07) circle (  2.13);

\path[fill=fillColor,fill opacity=0.20] (107.86, 55.82) circle (  2.13);

\path[fill=fillColor,fill opacity=0.20] (102.94, 46.07) circle (  2.13);

\path[fill=fillColor,fill opacity=0.20] ( 93.11, 57.44) circle (  2.13);

\path[fill=fillColor,fill opacity=0.20] (105.89, 58.26) circle (  2.13);

\path[fill=fillColor,fill opacity=0.20] (114.73, 58.26) circle (  2.13);

\path[fill=fillColor,fill opacity=0.20] (105.89, 65.57) circle (  2.13);

\path[fill=fillColor,fill opacity=0.20] (106.87, 56.63) circle (  2.13);

\path[fill=fillColor,fill opacity=0.20] (109.82, 50.94) circle (  2.13);

\path[fill=fillColor,fill opacity=0.20] (106.87, 64.76) circle (  2.13);

\path[fill=fillColor,fill opacity=0.20] (103.92, 59.07) circle (  2.13);

\path[fill=fillColor,fill opacity=0.20] ( 96.06, 69.63) circle (  2.13);

\path[fill=fillColor,fill opacity=0.20] ( 93.11, 75.32) circle (  2.13);

\path[fill=fillColor,fill opacity=0.20] ( 93.11, 60.69) circle (  2.13);

\path[fill=fillColor,fill opacity=0.20] ( 87.22, 46.88) circle (  2.13);

\path[fill=fillColor,fill opacity=0.20] ( 81.32, 38.75) circle (  2.13);

\path[fill=fillColor,fill opacity=0.20] ( 96.06, 51.75) circle (  2.13);

\path[fill=fillColor,fill opacity=0.20] (113.75, 57.44) circle (  2.13);

\path[fill=fillColor,fill opacity=0.20] ( 97.05, 51.75) circle (  2.13);

\path[fill=fillColor,fill opacity=0.20] (112.77, 65.57) circle (  2.13);

\path[fill=fillColor,fill opacity=0.20] (118.66, 61.51) circle (  2.13);

\path[fill=fillColor,fill opacity=0.20] (114.73, 56.63) circle (  2.13);

\path[fill=fillColor,fill opacity=0.20] (114.73, 46.88) circle (  2.13);

\path[fill=fillColor,fill opacity=0.20] (116.70, 40.38) circle (  2.13);

\path[fill=fillColor,fill opacity=0.20] (107.86, 46.07) circle (  2.13);

\path[fill=fillColor,fill opacity=0.20] (108.84, 63.13) circle (  2.13);

\path[fill=fillColor,fill opacity=0.20] (106.87, 63.13) circle (  2.13);

\path[fill=fillColor,fill opacity=0.20] (108.84, 59.88) circle (  2.13);

\path[fill=fillColor,fill opacity=0.20] (104.91, 71.26) circle (  2.13);

\path[fill=fillColor,fill opacity=0.20] (102.94, 72.07) circle (  2.13);

\path[fill=fillColor,fill opacity=0.20] ( 99.01, 69.63) circle (  2.13);

\path[fill=fillColor,fill opacity=0.20] ( 94.10, 57.44) circle (  2.13);

\path[fill=fillColor,fill opacity=0.20] ( 78.37, 63.95) circle (  2.13);

\path[fill=fillColor,fill opacity=0.20] ( 87.22, 54.19) circle (  2.13);

\path[fill=fillColor,fill opacity=0.20] ( 84.27, 43.63) circle (  2.13);

\path[fill=fillColor,fill opacity=0.20] ( 98.03, 39.56) circle (  2.13);

\path[fill=fillColor,fill opacity=0.20] ( 90.17, 45.25) circle (  2.13);

\path[fill=fillColor,fill opacity=0.20] ( 98.03, 56.63) circle (  2.13);

\path[fill=fillColor,fill opacity=0.20] (113.75, 55.01) circle (  2.13);

\path[fill=fillColor,fill opacity=0.20] (118.66, 46.07) circle (  2.13);

\path[fill=fillColor,fill opacity=0.20] (137.34, 51.75) circle (  2.13);

\path[fill=fillColor,fill opacity=0.20] (126.53, 46.07) circle (  2.13);

\path[fill=fillColor,fill opacity=0.20] (117.68, 46.88) circle (  2.13);

\path[fill=fillColor,fill opacity=0.20] (105.89, 46.07) circle (  2.13);

\path[fill=fillColor,fill opacity=0.20] (106.87, 54.19) circle (  2.13);

\path[fill=fillColor,fill opacity=0.20] ( 93.11, 54.19) circle (  2.13);

\path[fill=fillColor,fill opacity=0.20] ( 91.15, 40.38) circle (  2.13);

\path[fill=fillColor,fill opacity=0.20] ( 99.99, 65.57) circle (  2.13);

\path[fill=fillColor,fill opacity=0.20] ( 92.13, 43.63) circle (  2.13);

\path[fill=fillColor,fill opacity=0.20] (100.98, 41.19) circle (  2.13);

\path[fill=fillColor,fill opacity=0.20] (104.91, 46.07) circle (  2.13);

\path[fill=fillColor,fill opacity=0.20] ( 87.22, 38.75) circle (  2.13);

\path[fill=fillColor,fill opacity=0.20] (122.60, 43.63) circle (  2.13);

\path[fill=fillColor,fill opacity=0.20] (136.35, 57.44) circle (  2.13);

\path[fill=fillColor,fill opacity=0.20] (137.34, 70.45) circle (  2.13);

\path[fill=fillColor,fill opacity=0.20] (131.44, 52.57) circle (  2.13);

\path[fill=fillColor,fill opacity=0.20] (102.94, 47.69) circle (  2.13);

\path[fill=fillColor,fill opacity=0.20] (104.91, 40.38) circle (  2.13);

\path[fill=fillColor,fill opacity=0.20] ( 89.18, 40.38) circle (  2.13);

\path[fill=fillColor,fill opacity=0.20] ( 94.10, 48.50) circle (  2.13);

\path[fill=fillColor,fill opacity=0.20] (133.40, 42.00) circle (  2.13);

\path[fill=fillColor,fill opacity=0.20] (130.46, 51.75) circle (  2.13);

\path[fill=fillColor,fill opacity=0.20] (127.51, 50.94) circle (  2.13);

\path[fill=fillColor,fill opacity=0.20] (121.61, 41.19) circle (  2.13);

\path[fill=fillColor,fill opacity=0.20] ( 92.13, 41.19) circle (  2.13);

\path[fill=fillColor,fill opacity=0.20] (115.72, 52.57) circle (  2.13);

\path[fill=fillColor,fill opacity=0.20] (108.84, 54.19) circle (  2.13);

\path[fill=fillColor,fill opacity=0.20] ( 98.03, 44.44) circle (  2.13);

\path[fill=fillColor,fill opacity=0.20] (150.11, 49.32) circle (  2.13);

\path[fill=fillColor,fill opacity=0.20] ( 53.91, 74.51) circle (  2.13);
\end{scope}
\begin{scope}
\path[clip] (162.27, 34.04) rectangle (277.03,119.86);
\definecolor[named]{fillColor}{rgb}{0.90,0.90,0.90}

\path[fill=fillColor] (162.27, 34.04) rectangle (277.03,119.86);
\definecolor[named]{drawColor}{rgb}{0.95,0.95,0.95}

\path[draw=drawColor,line width= 0.3pt,line join=round] (162.27, 42.41) --
	(277.03, 42.41);

\path[draw=drawColor,line width= 0.3pt,line join=round] (162.27, 62.73) --
	(277.03, 62.73);

\path[draw=drawColor,line width= 0.3pt,line join=round] (162.27, 83.04) --
	(277.03, 83.04);

\path[draw=drawColor,line width= 0.3pt,line join=round] (162.27,103.36) --
	(277.03,103.36);

\path[draw=drawColor,line width= 0.3pt,line join=round] (177.97, 34.04) --
	(177.97,119.86);

\path[draw=drawColor,line width= 0.3pt,line join=round] (202.54, 34.04) --
	(202.54,119.86);

\path[draw=drawColor,line width= 0.3pt,line join=round] (227.11, 34.04) --
	(227.11,119.86);

\path[draw=drawColor,line width= 0.3pt,line join=round] (251.67, 34.04) --
	(251.67,119.86);

\path[draw=drawColor,line width= 0.3pt,line join=round] (276.24, 34.04) --
	(276.24,119.86);
\definecolor[named]{drawColor}{rgb}{1.00,1.00,1.00}

\path[draw=drawColor,line width= 0.6pt,line join=round] (162.27, 52.57) --
	(277.03, 52.57);

\path[draw=drawColor,line width= 0.6pt,line join=round] (162.27, 72.89) --
	(277.03, 72.89);

\path[draw=drawColor,line width= 0.6pt,line join=round] (162.27, 93.20) --
	(277.03, 93.20);

\path[draw=drawColor,line width= 0.6pt,line join=round] (162.27,113.52) --
	(277.03,113.52);

\path[draw=drawColor,line width= 0.6pt,line join=round] (165.69, 34.04) --
	(165.69,119.86);

\path[draw=drawColor,line width= 0.6pt,line join=round] (190.26, 34.04) --
	(190.26,119.86);

\path[draw=drawColor,line width= 0.6pt,line join=round] (214.82, 34.04) --
	(214.82,119.86);

\path[draw=drawColor,line width= 0.6pt,line join=round] (239.39, 34.04) --
	(239.39,119.86);

\path[draw=drawColor,line width= 0.6pt,line join=round] (263.96, 34.04) --
	(263.96,119.86);
\definecolor[named]{fillColor}{rgb}{0.00,0.00,0.00}

\path[fill=fillColor,fill opacity=0.20] (219.74, 55.01) circle (  2.13);

\path[fill=fillColor,fill opacity=0.20] (214.82, 59.07) circle (  2.13);

\path[fill=fillColor,fill opacity=0.20] (222.68, 56.63) circle (  2.13);

\path[fill=fillColor,fill opacity=0.20] (214.82, 47.69) circle (  2.13);

\path[fill=fillColor,fill opacity=0.20] (224.65, 42.82) circle (  2.13);

\path[fill=fillColor,fill opacity=0.20] (236.44, 49.32) circle (  2.13);

\path[fill=fillColor,fill opacity=0.20] (217.77, 60.69) circle (  2.13);

\path[fill=fillColor,fill opacity=0.20] (207.94, 60.69) circle (  2.13);

\path[fill=fillColor,fill opacity=0.20] (203.03, 68.82) circle (  2.13);

\path[fill=fillColor,fill opacity=0.20] (203.03, 71.26) circle (  2.13);

\path[fill=fillColor,fill opacity=0.20] (208.93, 60.69) circle (  2.13);

\path[fill=fillColor,fill opacity=0.20] (205.98, 50.94) circle (  2.13);

\path[fill=fillColor,fill opacity=0.20] (206.96, 51.75) circle (  2.13);

\path[fill=fillColor,fill opacity=0.20] (217.77, 54.19) circle (  2.13);

\path[fill=fillColor,fill opacity=0.20] (229.56, 48.50) circle (  2.13);

\path[fill=fillColor,fill opacity=0.20] (239.39, 40.38) circle (  2.13);

\path[fill=fillColor,fill opacity=0.20] (211.87, 72.89) circle (  2.13);

\path[fill=fillColor,fill opacity=0.20] (203.03, 76.95) circle (  2.13);

\path[fill=fillColor,fill opacity=0.20] (198.12, 76.14) circle (  2.13);

\path[fill=fillColor,fill opacity=0.20] (198.12, 75.32) circle (  2.13);

\path[fill=fillColor,fill opacity=0.20] (205.00, 71.26) circle (  2.13);

\path[fill=fillColor,fill opacity=0.20] (208.93, 66.38) circle (  2.13);

\path[fill=fillColor,fill opacity=0.20] (211.87, 66.38) circle (  2.13);

\path[fill=fillColor,fill opacity=0.20] (210.89, 66.38) circle (  2.13);

\path[fill=fillColor,fill opacity=0.20] (206.96, 62.32) circle (  2.13);

\path[fill=fillColor,fill opacity=0.20] (218.75, 55.82) circle (  2.13);

\path[fill=fillColor,fill opacity=0.20] (223.67, 45.25) circle (  2.13);

\path[fill=fillColor,fill opacity=0.20] (208.93, 63.13) circle (  2.13);

\path[fill=fillColor,fill opacity=0.20] (200.08, 71.26) circle (  2.13);

\path[fill=fillColor,fill opacity=0.20] (193.20, 78.57) circle (  2.13);

\path[fill=fillColor,fill opacity=0.20] (193.20, 85.08) circle (  2.13);

\path[fill=fillColor,fill opacity=0.20] (198.12, 82.64) circle (  2.13);

\path[fill=fillColor,fill opacity=0.20] (202.05, 76.95) circle (  2.13);

\path[fill=fillColor,fill opacity=0.20] (203.03, 76.14) circle (  2.13);

\path[fill=fillColor,fill opacity=0.20] (207.94, 76.95) circle (  2.13);

\path[fill=fillColor,fill opacity=0.20] (206.96, 72.89) circle (  2.13);

\path[fill=fillColor,fill opacity=0.20] (200.08, 62.32) circle (  2.13);

\path[fill=fillColor,fill opacity=0.20] (202.05, 68.82) circle (  2.13);

\path[fill=fillColor,fill opacity=0.20] (211.87, 77.76) circle (  2.13);

\path[fill=fillColor,fill opacity=0.20] (218.75, 60.69) circle (  2.13);

\path[fill=fillColor,fill opacity=0.20] (207.94, 63.13) circle (  2.13);

\path[fill=fillColor,fill opacity=0.20] (196.15, 63.13) circle (  2.13);

\path[fill=fillColor,fill opacity=0.20] (193.20, 72.89) circle (  2.13);

\path[fill=fillColor,fill opacity=0.20] (196.15, 89.14) circle (  2.13);

\path[fill=fillColor,fill opacity=0.20] (198.12, 93.20) circle (  2.13);

\path[fill=fillColor,fill opacity=0.20] (200.08, 88.33) circle (  2.13);

\path[fill=fillColor,fill opacity=0.20] (203.03, 81.01) circle (  2.13);

\path[fill=fillColor,fill opacity=0.20] (204.01, 76.14) circle (  2.13);

\path[fill=fillColor,fill opacity=0.20] (209.91, 75.32) circle (  2.13);

\path[fill=fillColor,fill opacity=0.20] (203.03, 72.89) circle (  2.13);

\path[fill=fillColor,fill opacity=0.20] (199.10, 81.82) circle (  2.13);

\path[fill=fillColor,fill opacity=0.20] (205.00, 82.64) circle (  2.13);

\path[fill=fillColor,fill opacity=0.20] (208.93, 76.14) circle (  2.13);

\path[fill=fillColor,fill opacity=0.20] (213.84, 67.20) circle (  2.13);

\path[fill=fillColor,fill opacity=0.20] (217.77, 64.76) circle (  2.13);

\path[fill=fillColor,fill opacity=0.20] (215.81, 64.76) circle (  2.13);

\path[fill=fillColor,fill opacity=0.20] (210.89, 66.38) circle (  2.13);

\path[fill=fillColor,fill opacity=0.20] (202.05, 72.07) circle (  2.13);

\path[fill=fillColor,fill opacity=0.20] (199.10, 80.20) circle (  2.13);

\path[fill=fillColor,fill opacity=0.20] (199.10, 90.76) circle (  2.13);

\path[fill=fillColor,fill opacity=0.20] (201.07, 94.83) circle (  2.13);

\path[fill=fillColor,fill opacity=0.20] (203.03, 91.58) circle (  2.13);

\path[fill=fillColor,fill opacity=0.20] (205.98, 80.20) circle (  2.13);

\path[fill=fillColor,fill opacity=0.20] (208.93, 75.32) circle (  2.13);

\path[fill=fillColor,fill opacity=0.20] (221.70, 79.39) circle (  2.13);

\path[fill=fillColor,fill opacity=0.20] (199.10, 75.32) circle (  2.13);

\path[fill=fillColor,fill opacity=0.20] (195.17, 80.20) circle (  2.13);

\path[fill=fillColor,fill opacity=0.20] (197.13, 91.58) circle (  2.13);

\path[fill=fillColor,fill opacity=0.20] (198.12, 88.33) circle (  2.13);

\path[fill=fillColor,fill opacity=0.20] (198.12, 79.39) circle (  2.13);

\path[fill=fillColor,fill opacity=0.20] (205.98, 72.07) circle (  2.13);

\path[fill=fillColor,fill opacity=0.20] (205.98, 63.95) circle (  2.13);

\path[fill=fillColor,fill opacity=0.20] (209.91, 56.63) circle (  2.13);

\path[fill=fillColor,fill opacity=0.20] (214.82, 69.63) circle (  2.13);

\path[fill=fillColor,fill opacity=0.20] (212.86, 70.45) circle (  2.13);

\path[fill=fillColor,fill opacity=0.20] (203.03, 85.89) circle (  2.13);

\path[fill=fillColor,fill opacity=0.20] (201.07, 92.39) circle (  2.13);

\path[fill=fillColor,fill opacity=0.20] (200.08, 92.39) circle (  2.13);

\path[fill=fillColor,fill opacity=0.20] (201.07, 92.39) circle (  2.13);

\path[fill=fillColor,fill opacity=0.20] (205.98, 90.76) circle (  2.13);

\path[fill=fillColor,fill opacity=0.20] (208.93, 82.64) circle (  2.13);

\path[fill=fillColor,fill opacity=0.20] (213.84, 75.32) circle (  2.13);

\path[fill=fillColor,fill opacity=0.20] (202.05, 77.76) circle (  2.13);

\path[fill=fillColor,fill opacity=0.20] (195.17, 85.08) circle (  2.13);

\path[fill=fillColor,fill opacity=0.20] (198.12, 88.33) circle (  2.13);

\path[fill=fillColor,fill opacity=0.20] (200.08, 78.57) circle (  2.13);

\path[fill=fillColor,fill opacity=0.20] (202.05, 73.70) circle (  2.13);

\path[fill=fillColor,fill opacity=0.20] (203.03, 72.07) circle (  2.13);

\path[fill=fillColor,fill opacity=0.20] (203.03, 64.76) circle (  2.13);

\path[fill=fillColor,fill opacity=0.20] (212.86, 52.57) circle (  2.13);

\path[fill=fillColor,fill opacity=0.20] (210.89, 70.45) circle (  2.13);

\path[fill=fillColor,fill opacity=0.20] (201.07, 85.89) circle (  2.13);

\path[fill=fillColor,fill opacity=0.20] (201.07, 94.83) circle (  2.13);

\path[fill=fillColor,fill opacity=0.20] (204.01, 94.83) circle (  2.13);

\path[fill=fillColor,fill opacity=0.20] (202.05, 92.39) circle (  2.13);

\path[fill=fillColor,fill opacity=0.20] (205.00, 88.33) circle (  2.13);

\path[fill=fillColor,fill opacity=0.20] (207.94, 82.64) circle (  2.13);

\path[fill=fillColor,fill opacity=0.20] (213.84, 73.70) circle (  2.13);

\path[fill=fillColor,fill opacity=0.20] (204.01, 88.33) circle (  2.13);

\path[fill=fillColor,fill opacity=0.20] (198.12, 73.70) circle (  2.13);

\path[fill=fillColor,fill opacity=0.20] (200.08, 78.57) circle (  2.13);

\path[fill=fillColor,fill opacity=0.20] (200.08, 73.70) circle (  2.13);

\path[fill=fillColor,fill opacity=0.20] (201.07, 71.26) circle (  2.13);

\path[fill=fillColor,fill opacity=0.20] (205.00, 76.14) circle (  2.13);

\path[fill=fillColor,fill opacity=0.20] (206.96, 71.26) circle (  2.13);

\path[fill=fillColor,fill opacity=0.20] (212.86, 61.51) circle (  2.13);

\path[fill=fillColor,fill opacity=0.20] (216.79, 54.19) circle (  2.13);

\path[fill=fillColor,fill opacity=0.20] (221.70, 42.82) circle (  2.13);

\path[fill=fillColor,fill opacity=0.20] (209.91, 68.01) circle (  2.13);

\path[fill=fillColor,fill opacity=0.20] (203.03, 77.76) circle (  2.13);

\path[fill=fillColor,fill opacity=0.20] (203.03, 89.14) circle (  2.13);

\path[fill=fillColor,fill opacity=0.20] (205.98, 94.02) circle (  2.13);

\path[fill=fillColor,fill opacity=0.20] (204.01, 89.14) circle (  2.13);

\path[fill=fillColor,fill opacity=0.20] (205.00, 79.39) circle (  2.13);

\path[fill=fillColor,fill opacity=0.20] (208.93, 74.51) circle (  2.13);

\path[fill=fillColor,fill opacity=0.20] (211.87, 71.26) circle (  2.13);

\path[fill=fillColor,fill opacity=0.20] (218.75, 66.38) circle (  2.13);

\path[fill=fillColor,fill opacity=0.20] (200.08, 70.45) circle (  2.13);

\path[fill=fillColor,fill opacity=0.20] (197.13, 67.20) circle (  2.13);

\path[fill=fillColor,fill opacity=0.20] (203.03, 76.14) circle (  2.13);

\path[fill=fillColor,fill opacity=0.20] (204.01, 69.63) circle (  2.13);

\path[fill=fillColor,fill opacity=0.20] (205.00, 71.26) circle (  2.13);

\path[fill=fillColor,fill opacity=0.20] (207.94, 76.95) circle (  2.13);

\path[fill=fillColor,fill opacity=0.20] (209.91, 64.76) circle (  2.13);

\path[fill=fillColor,fill opacity=0.20] (215.81, 57.44) circle (  2.13);

\path[fill=fillColor,fill opacity=0.20] (220.72, 57.44) circle (  2.13);

\path[fill=fillColor,fill opacity=0.20] (224.65, 46.07) circle (  2.13);

\path[fill=fillColor,fill opacity=0.20] (210.89, 65.57) circle (  2.13);

\path[fill=fillColor,fill opacity=0.20] (204.01, 68.01) circle (  2.13);

\path[fill=fillColor,fill opacity=0.20] (201.07, 78.57) circle (  2.13);

\path[fill=fillColor,fill opacity=0.20] (200.08, 84.26) circle (  2.13);

\path[fill=fillColor,fill opacity=0.20] (202.05, 80.20) circle (  2.13);

\path[fill=fillColor,fill opacity=0.20] (205.00, 73.70) circle (  2.13);

\path[fill=fillColor,fill opacity=0.20] (210.89, 73.70) circle (  2.13);

\path[fill=fillColor,fill opacity=0.20] (214.82, 71.26) circle (  2.13);

\path[fill=fillColor,fill opacity=0.20] (216.79, 60.69) circle (  2.13);

\path[fill=fillColor,fill opacity=0.20] (227.60, 67.20) circle (  2.13);

\path[fill=fillColor,fill opacity=0.20] (206.96,111.89) circle (  2.13);

\path[fill=fillColor,fill opacity=0.20] (192.22, 63.13) circle (  2.13);

\path[fill=fillColor,fill opacity=0.20] (201.07, 68.82) circle (  2.13);

\path[fill=fillColor,fill opacity=0.20] (206.96, 76.14) circle (  2.13);

\path[fill=fillColor,fill opacity=0.20] (205.00, 72.89) circle (  2.13);

\path[fill=fillColor,fill opacity=0.20] (204.01, 74.51) circle (  2.13);

\path[fill=fillColor,fill opacity=0.20] (205.98, 70.45) circle (  2.13);

\path[fill=fillColor,fill opacity=0.20] (208.93, 59.07) circle (  2.13);

\path[fill=fillColor,fill opacity=0.20] (211.87, 56.63) circle (  2.13);

\path[fill=fillColor,fill opacity=0.20] (219.74, 55.01) circle (  2.13);

\path[fill=fillColor,fill opacity=0.20] (216.79, 42.00) circle (  2.13);

\path[fill=fillColor,fill opacity=0.20] (240.37, 39.56) circle (  2.13);

\path[fill=fillColor,fill opacity=0.20] (204.01, 58.26) circle (  2.13);

\path[fill=fillColor,fill opacity=0.20] (199.10, 67.20) circle (  2.13);

\path[fill=fillColor,fill opacity=0.20] (197.13, 68.82) circle (  2.13);

\path[fill=fillColor,fill opacity=0.20] (200.08, 68.82) circle (  2.13);

\path[fill=fillColor,fill opacity=0.20] (205.98, 76.95) circle (  2.13);

\path[fill=fillColor,fill opacity=0.20] (209.91, 79.39) circle (  2.13);

\path[fill=fillColor,fill opacity=0.20] (216.79, 72.89) circle (  2.13);

\path[fill=fillColor,fill opacity=0.20] (214.82, 65.57) circle (  2.13);

\path[fill=fillColor,fill opacity=0.20] (221.70, 63.95) circle (  2.13);

\path[fill=fillColor,fill opacity=0.20] (206.96, 82.64) circle (  2.13);

\path[fill=fillColor,fill opacity=0.20] (201.07, 71.26) circle (  2.13);

\path[fill=fillColor,fill opacity=0.20] (202.05, 72.89) circle (  2.13);

\path[fill=fillColor,fill opacity=0.20] (205.98, 69.63) circle (  2.13);

\path[fill=fillColor,fill opacity=0.20] (204.01, 72.07) circle (  2.13);

\path[fill=fillColor,fill opacity=0.20] (205.98, 74.51) circle (  2.13);

\path[fill=fillColor,fill opacity=0.20] (210.89, 68.01) circle (  2.13);

\path[fill=fillColor,fill opacity=0.20] (212.86, 62.32) circle (  2.13);

\path[fill=fillColor,fill opacity=0.20] (213.84, 59.88) circle (  2.13);

\path[fill=fillColor,fill opacity=0.20] (212.86, 50.94) circle (  2.13);

\path[fill=fillColor,fill opacity=0.20] (222.68, 38.75) circle (  2.13);

\path[fill=fillColor,fill opacity=0.20] (241.36, 42.82) circle (  2.13);

\path[fill=fillColor,fill opacity=0.20] (212.86, 51.75) circle (  2.13);

\path[fill=fillColor,fill opacity=0.20] (206.96, 55.82) circle (  2.13);

\path[fill=fillColor,fill opacity=0.20] (200.08, 59.07) circle (  2.13);

\path[fill=fillColor,fill opacity=0.20] (204.01, 66.38) circle (  2.13);

\path[fill=fillColor,fill opacity=0.20] (206.96, 77.76) circle (  2.13);

\path[fill=fillColor,fill opacity=0.20] (205.00, 76.95) circle (  2.13);

\path[fill=fillColor,fill opacity=0.20] (210.89, 73.70) circle (  2.13);

\path[fill=fillColor,fill opacity=0.20] (215.81, 72.89) circle (  2.13);

\path[fill=fillColor,fill opacity=0.20] (221.70, 70.45) circle (  2.13);

\path[fill=fillColor,fill opacity=0.20] (212.86, 95.64) circle (  2.13);

\path[fill=fillColor,fill opacity=0.20] (206.96, 68.01) circle (  2.13);

\path[fill=fillColor,fill opacity=0.20] (205.98, 72.89) circle (  2.13);

\path[fill=fillColor,fill opacity=0.20] (207.94, 68.82) circle (  2.13);

\path[fill=fillColor,fill opacity=0.20] (201.07, 65.57) circle (  2.13);

\path[fill=fillColor,fill opacity=0.20] (205.00, 70.45) circle (  2.13);

\path[fill=fillColor,fill opacity=0.20] (204.01, 72.07) circle (  2.13);

\path[fill=fillColor,fill opacity=0.20] (205.00, 71.26) circle (  2.13);

\path[fill=fillColor,fill opacity=0.20] (211.87, 70.45) circle (  2.13);

\path[fill=fillColor,fill opacity=0.20] (213.84, 61.51) circle (  2.13);

\path[fill=fillColor,fill opacity=0.20] (215.81, 47.69) circle (  2.13);

\path[fill=fillColor,fill opacity=0.20] (224.65, 44.44) circle (  2.13);

\path[fill=fillColor,fill opacity=0.20] (211.87, 52.57) circle (  2.13);

\path[fill=fillColor,fill opacity=0.20] (205.00, 52.57) circle (  2.13);

\path[fill=fillColor,fill opacity=0.20] (205.98, 63.13) circle (  2.13);

\path[fill=fillColor,fill opacity=0.20] (207.94, 70.45) circle (  2.13);

\path[fill=fillColor,fill opacity=0.20] (205.98, 72.07) circle (  2.13);

\path[fill=fillColor,fill opacity=0.20] (205.98, 75.32) circle (  2.13);

\path[fill=fillColor,fill opacity=0.20] (211.87, 76.14) circle (  2.13);

\path[fill=fillColor,fill opacity=0.20] (213.84, 72.89) circle (  2.13);

\path[fill=fillColor,fill opacity=0.20] (220.72, 69.63) circle (  2.13);

\path[fill=fillColor,fill opacity=0.20] (204.01, 75.32) circle (  2.13);

\path[fill=fillColor,fill opacity=0.20] (209.91, 70.45) circle (  2.13);

\path[fill=fillColor,fill opacity=0.20] (209.91, 71.26) circle (  2.13);

\path[fill=fillColor,fill opacity=0.20] (206.96, 66.38) circle (  2.13);

\path[fill=fillColor,fill opacity=0.20] (202.05, 67.20) circle (  2.13);

\path[fill=fillColor,fill opacity=0.20] (201.07, 72.07) circle (  2.13);

\path[fill=fillColor,fill opacity=0.20] (211.87, 72.89) circle (  2.13);

\path[fill=fillColor,fill opacity=0.20] (205.00, 72.89) circle (  2.13);

\path[fill=fillColor,fill opacity=0.20] (209.91, 70.45) circle (  2.13);

\path[fill=fillColor,fill opacity=0.20] (212.86, 55.82) circle (  2.13);

\path[fill=fillColor,fill opacity=0.20] (218.75, 43.63) circle (  2.13);

\path[fill=fillColor,fill opacity=0.20] (211.87, 50.13) circle (  2.13);

\path[fill=fillColor,fill opacity=0.20] (208.93, 55.82) circle (  2.13);

\path[fill=fillColor,fill opacity=0.20] (208.93, 65.57) circle (  2.13);

\path[fill=fillColor,fill opacity=0.20] (208.93, 74.51) circle (  2.13);

\path[fill=fillColor,fill opacity=0.20] (207.94, 78.57) circle (  2.13);

\path[fill=fillColor,fill opacity=0.20] (205.00, 73.70) circle (  2.13);

\path[fill=fillColor,fill opacity=0.20] (207.94, 70.45) circle (  2.13);

\path[fill=fillColor,fill opacity=0.20] (211.87, 68.82) circle (  2.13);

\path[fill=fillColor,fill opacity=0.20] (221.70, 63.13) circle (  2.13);

\path[fill=fillColor,fill opacity=0.20] (231.53, 66.38) circle (  2.13);

\path[fill=fillColor,fill opacity=0.20] (201.07, 75.32) circle (  2.13);

\path[fill=fillColor,fill opacity=0.20] (202.05, 72.89) circle (  2.13);

\path[fill=fillColor,fill opacity=0.20] (207.94, 75.32) circle (  2.13);

\path[fill=fillColor,fill opacity=0.20] (206.96, 76.95) circle (  2.13);

\path[fill=fillColor,fill opacity=0.20] (199.10, 70.45) circle (  2.13);

\path[fill=fillColor,fill opacity=0.20] (202.05, 66.38) circle (  2.13);

\path[fill=fillColor,fill opacity=0.20] (197.13, 71.26) circle (  2.13);

\path[fill=fillColor,fill opacity=0.20] (210.89, 72.89) circle (  2.13);

\path[fill=fillColor,fill opacity=0.20] (207.94, 68.01) circle (  2.13);

\path[fill=fillColor,fill opacity=0.20] (213.84, 60.69) circle (  2.13);

\path[fill=fillColor,fill opacity=0.20] (215.81, 49.32) circle (  2.13);

\path[fill=fillColor,fill opacity=0.20] (222.68, 42.00) circle (  2.13);

\path[fill=fillColor,fill opacity=0.20] (213.84, 55.82) circle (  2.13);

\path[fill=fillColor,fill opacity=0.20] (209.91, 64.76) circle (  2.13);

\path[fill=fillColor,fill opacity=0.20] (208.93, 72.07) circle (  2.13);

\path[fill=fillColor,fill opacity=0.20] (206.96, 75.32) circle (  2.13);

\path[fill=fillColor,fill opacity=0.20] (207.94, 71.26) circle (  2.13);

\path[fill=fillColor,fill opacity=0.20] (208.93, 71.26) circle (  2.13);

\path[fill=fillColor,fill opacity=0.20] (213.84, 74.51) circle (  2.13);

\path[fill=fillColor,fill opacity=0.20] (217.77, 69.63) circle (  2.13);

\path[fill=fillColor,fill opacity=0.20] (222.68, 63.13) circle (  2.13);

\path[fill=fillColor,fill opacity=0.20] (203.03, 71.26) circle (  2.13);

\path[fill=fillColor,fill opacity=0.20] (197.13, 74.51) circle (  2.13);

\path[fill=fillColor,fill opacity=0.20] (203.03, 76.14) circle (  2.13);

\path[fill=fillColor,fill opacity=0.20] (199.10, 76.95) circle (  2.13);

\path[fill=fillColor,fill opacity=0.20] (198.12, 80.20) circle (  2.13);

\path[fill=fillColor,fill opacity=0.20] (198.12, 74.51) circle (  2.13);

\path[fill=fillColor,fill opacity=0.20] (203.03, 66.38) circle (  2.13);

\path[fill=fillColor,fill opacity=0.20] (207.94, 67.20) circle (  2.13);

\path[fill=fillColor,fill opacity=0.20] (207.94, 66.38) circle (  2.13);

\path[fill=fillColor,fill opacity=0.20] (209.91, 57.44) circle (  2.13);

\path[fill=fillColor,fill opacity=0.20] (217.77, 50.94) circle (  2.13);

\path[fill=fillColor,fill opacity=0.20] (224.65, 47.69) circle (  2.13);

\path[fill=fillColor,fill opacity=0.20] (213.84, 65.57) circle (  2.13);

\path[fill=fillColor,fill opacity=0.20] (205.00, 62.32) circle (  2.13);

\path[fill=fillColor,fill opacity=0.20] (206.96, 64.76) circle (  2.13);

\path[fill=fillColor,fill opacity=0.20] (211.87, 68.82) circle (  2.13);

\path[fill=fillColor,fill opacity=0.20] (208.93, 71.26) circle (  2.13);

\path[fill=fillColor,fill opacity=0.20] (208.93, 74.51) circle (  2.13);

\path[fill=fillColor,fill opacity=0.20] (209.91, 72.07) circle (  2.13);

\path[fill=fillColor,fill opacity=0.20] (206.96, 67.20) circle (  2.13);

\path[fill=fillColor,fill opacity=0.20] (215.81, 72.07) circle (  2.13);

\path[fill=fillColor,fill opacity=0.20] (199.10, 74.51) circle (  2.13);

\path[fill=fillColor,fill opacity=0.20] (201.07, 76.14) circle (  2.13);

\path[fill=fillColor,fill opacity=0.20] (198.12, 75.32) circle (  2.13);

\path[fill=fillColor,fill opacity=0.20] (197.13, 73.70) circle (  2.13);

\path[fill=fillColor,fill opacity=0.20] (200.08, 71.26) circle (  2.13);

\path[fill=fillColor,fill opacity=0.20] (198.12, 72.89) circle (  2.13);

\path[fill=fillColor,fill opacity=0.20] (204.01, 72.89) circle (  2.13);

\path[fill=fillColor,fill opacity=0.20] (204.01, 67.20) circle (  2.13);

\path[fill=fillColor,fill opacity=0.20] (206.96, 59.07) circle (  2.13);

\path[fill=fillColor,fill opacity=0.20] (207.94, 47.69) circle (  2.13);

\path[fill=fillColor,fill opacity=0.20] (218.75, 42.82) circle (  2.13);

\path[fill=fillColor,fill opacity=0.20] (227.60, 46.07) circle (  2.13);

\path[fill=fillColor,fill opacity=0.20] (206.96, 58.26) circle (  2.13);

\path[fill=fillColor,fill opacity=0.20] (209.91, 63.13) circle (  2.13);

\path[fill=fillColor,fill opacity=0.20] (210.89, 61.51) circle (  2.13);

\path[fill=fillColor,fill opacity=0.20] (213.84, 61.51) circle (  2.13);

\path[fill=fillColor,fill opacity=0.20] (213.84, 65.57) circle (  2.13);

\path[fill=fillColor,fill opacity=0.20] (211.87, 63.95) circle (  2.13);

\path[fill=fillColor,fill opacity=0.20] (206.96, 62.32) circle (  2.13);

\path[fill=fillColor,fill opacity=0.20] (212.86, 68.82) circle (  2.13);

\path[fill=fillColor,fill opacity=0.20] (223.67, 73.70) circle (  2.13);

\path[fill=fillColor,fill opacity=0.20] (211.87, 66.38) circle (  2.13);

\path[fill=fillColor,fill opacity=0.20] (200.08, 75.32) circle (  2.13);

\path[fill=fillColor,fill opacity=0.20] (200.08, 76.95) circle (  2.13);

\path[fill=fillColor,fill opacity=0.20] (203.03, 71.26) circle (  2.13);

\path[fill=fillColor,fill opacity=0.20] (199.10, 70.45) circle (  2.13);

\path[fill=fillColor,fill opacity=0.20] (201.07, 65.57) circle (  2.13);

\path[fill=fillColor,fill opacity=0.20] (202.05, 62.32) circle (  2.13);

\path[fill=fillColor,fill opacity=0.20] (202.05, 71.26) circle (  2.13);

\path[fill=fillColor,fill opacity=0.20] (201.07, 77.76) circle (  2.13);

\path[fill=fillColor,fill opacity=0.20] (205.00, 68.01) circle (  2.13);

\path[fill=fillColor,fill opacity=0.20] (213.84, 52.57) circle (  2.13);

\path[fill=fillColor,fill opacity=0.20] (218.75, 42.82) circle (  2.13);

\path[fill=fillColor,fill opacity=0.20] (212.86, 39.56) circle (  2.13);

\path[fill=fillColor,fill opacity=0.20] (215.81, 57.44) circle (  2.13);

\path[fill=fillColor,fill opacity=0.20] (207.94, 51.75) circle (  2.13);

\path[fill=fillColor,fill opacity=0.20] (206.96, 51.75) circle (  2.13);

\path[fill=fillColor,fill opacity=0.20] (207.94, 57.44) circle (  2.13);

\path[fill=fillColor,fill opacity=0.20] (211.87, 63.13) circle (  2.13);

\path[fill=fillColor,fill opacity=0.20] (215.81, 61.51) circle (  2.13);

\path[fill=fillColor,fill opacity=0.20] (216.79, 55.82) circle (  2.13);

\path[fill=fillColor,fill opacity=0.20] (212.86, 63.13) circle (  2.13);

\path[fill=fillColor,fill opacity=0.20] (219.74, 71.26) circle (  2.13);

\path[fill=fillColor,fill opacity=0.20] (218.75, 69.63) circle (  2.13);

\path[fill=fillColor,fill opacity=0.20] (208.93, 50.13) circle (  2.13);

\path[fill=fillColor,fill opacity=0.20] (205.00, 67.20) circle (  2.13);

\path[fill=fillColor,fill opacity=0.20] (196.15, 76.14) circle (  2.13);

\path[fill=fillColor,fill opacity=0.20] (198.12, 71.26) circle (  2.13);

\path[fill=fillColor,fill opacity=0.20] (198.12, 70.45) circle (  2.13);

\path[fill=fillColor,fill opacity=0.20] (200.08, 72.07) circle (  2.13);

\path[fill=fillColor,fill opacity=0.20] (201.07, 67.20) circle (  2.13);

\path[fill=fillColor,fill opacity=0.20] (202.05, 65.57) circle (  2.13);

\path[fill=fillColor,fill opacity=0.20] (207.94, 72.07) circle (  2.13);

\path[fill=fillColor,fill opacity=0.20] (208.93, 70.45) circle (  2.13);

\path[fill=fillColor,fill opacity=0.20] (217.77, 56.63) circle (  2.13);

\path[fill=fillColor,fill opacity=0.20] (213.84, 45.25) circle (  2.13);

\path[fill=fillColor,fill opacity=0.20] (230.55, 46.88) circle (  2.13);

\path[fill=fillColor,fill opacity=0.20] (210.89, 48.50) circle (  2.13);

\path[fill=fillColor,fill opacity=0.20] (202.05, 53.38) circle (  2.13);

\path[fill=fillColor,fill opacity=0.20] (204.01, 64.76) circle (  2.13);

\path[fill=fillColor,fill opacity=0.20] (203.03, 62.32) circle (  2.13);

\path[fill=fillColor,fill opacity=0.20] (210.89, 59.88) circle (  2.13);

\path[fill=fillColor,fill opacity=0.20] (214.82, 64.76) circle (  2.13);

\path[fill=fillColor,fill opacity=0.20] (219.74, 63.95) circle (  2.13);

\path[fill=fillColor,fill opacity=0.20] (216.79, 63.95) circle (  2.13);

\path[fill=fillColor,fill opacity=0.20] (218.75, 65.57) circle (  2.13);

\path[fill=fillColor,fill opacity=0.20] (219.74, 56.63) circle (  2.13);

\path[fill=fillColor,fill opacity=0.20] (227.60, 50.94) circle (  2.13);

\path[fill=fillColor,fill opacity=0.20] (214.82, 52.57) circle (  2.13);

\path[fill=fillColor,fill opacity=0.20] (205.98, 58.26) circle (  2.13);

\path[fill=fillColor,fill opacity=0.20] (203.03, 68.82) circle (  2.13);

\path[fill=fillColor,fill opacity=0.20] (199.10, 73.70) circle (  2.13);

\path[fill=fillColor,fill opacity=0.20] (199.10, 71.26) circle (  2.13);

\path[fill=fillColor,fill opacity=0.20] (200.08, 70.45) circle (  2.13);

\path[fill=fillColor,fill opacity=0.20] (199.10, 76.95) circle (  2.13);

\path[fill=fillColor,fill opacity=0.20] (198.12, 79.39) circle (  2.13);

\path[fill=fillColor,fill opacity=0.20] (200.08, 72.07) circle (  2.13);

\path[fill=fillColor,fill opacity=0.20] (207.94, 65.57) circle (  2.13);

\path[fill=fillColor,fill opacity=0.20] (211.87, 64.76) circle (  2.13);

\path[fill=fillColor,fill opacity=0.20] (215.81, 55.82) circle (  2.13);

\path[fill=fillColor,fill opacity=0.20] (220.72, 42.82) circle (  2.13);

\path[fill=fillColor,fill opacity=0.20] (235.46, 44.44) circle (  2.13);

\path[fill=fillColor,fill opacity=0.20] (206.96, 66.38) circle (  2.13);

\path[fill=fillColor,fill opacity=0.20] (206.96, 59.88) circle (  2.13);

\path[fill=fillColor,fill opacity=0.20] (205.00, 54.19) circle (  2.13);

\path[fill=fillColor,fill opacity=0.20] (207.94, 63.95) circle (  2.13);

\path[fill=fillColor,fill opacity=0.20] (218.75, 63.95) circle (  2.13);

\path[fill=fillColor,fill opacity=0.20] (219.74, 66.38) circle (  2.13);

\path[fill=fillColor,fill opacity=0.20] (216.79, 65.57) circle (  2.13);

\path[fill=fillColor,fill opacity=0.20] (219.74, 59.07) circle (  2.13);

\path[fill=fillColor,fill opacity=0.20] (220.72, 59.07) circle (  2.13);

\path[fill=fillColor,fill opacity=0.20] (224.65, 57.44) circle (  2.13);

\path[fill=fillColor,fill opacity=0.20] (229.56, 50.13) circle (  2.13);

\path[fill=fillColor,fill opacity=0.20] (212.86, 54.19) circle (  2.13);

\path[fill=fillColor,fill opacity=0.20] (209.91, 56.63) circle (  2.13);

\path[fill=fillColor,fill opacity=0.20] (205.00, 62.32) circle (  2.13);

\path[fill=fillColor,fill opacity=0.20] (202.05, 73.70) circle (  2.13);

\path[fill=fillColor,fill opacity=0.20] (201.07, 81.01) circle (  2.13);

\path[fill=fillColor,fill opacity=0.20] (202.05, 75.32) circle (  2.13);

\path[fill=fillColor,fill opacity=0.20] (204.01, 69.63) circle (  2.13);

\path[fill=fillColor,fill opacity=0.20] (205.00, 72.89) circle (  2.13);

\path[fill=fillColor,fill opacity=0.20] (201.07, 76.14) circle (  2.13);

\path[fill=fillColor,fill opacity=0.20] (198.12, 73.70) circle (  2.13);

\path[fill=fillColor,fill opacity=0.20] (200.08, 66.38) circle (  2.13);

\path[fill=fillColor,fill opacity=0.20] (208.93, 56.63) circle (  2.13);

\path[fill=fillColor,fill opacity=0.20] (212.86, 51.75) circle (  2.13);

\path[fill=fillColor,fill opacity=0.20] (225.63, 49.32) circle (  2.13);

\path[fill=fillColor,fill opacity=0.20] (221.70, 50.94) circle (  2.13);

\path[fill=fillColor,fill opacity=0.20] (207.94, 48.50) circle (  2.13);

\path[fill=fillColor,fill opacity=0.20] (204.01, 54.19) circle (  2.13);

\path[fill=fillColor,fill opacity=0.20] (211.87, 59.07) circle (  2.13);

\path[fill=fillColor,fill opacity=0.20] (216.79, 62.32) circle (  2.13);

\path[fill=fillColor,fill opacity=0.20] (214.82, 64.76) circle (  2.13);

\path[fill=fillColor,fill opacity=0.20] (220.72, 64.76) circle (  2.13);

\path[fill=fillColor,fill opacity=0.20] (222.68, 63.13) circle (  2.13);

\path[fill=fillColor,fill opacity=0.20] (220.72, 59.07) circle (  2.13);

\path[fill=fillColor,fill opacity=0.20] (219.74, 51.75) circle (  2.13);

\path[fill=fillColor,fill opacity=0.20] (224.65, 49.32) circle (  2.13);

\path[fill=fillColor,fill opacity=0.20] (239.39, 52.57) circle (  2.13);

\path[fill=fillColor,fill opacity=0.20] (214.82, 59.07) circle (  2.13);

\path[fill=fillColor,fill opacity=0.20] (183.97, 55.82) circle (  2.13);

\path[fill=fillColor,fill opacity=0.20] (203.03, 62.32) circle (  2.13);

\path[fill=fillColor,fill opacity=0.20] (204.01, 72.89) circle (  2.13);

\path[fill=fillColor,fill opacity=0.20] (203.03, 72.89) circle (  2.13);

\path[fill=fillColor,fill opacity=0.20] (205.00, 70.45) circle (  2.13);

\path[fill=fillColor,fill opacity=0.20] (203.03, 72.89) circle (  2.13);

\path[fill=fillColor,fill opacity=0.20] (202.05, 70.45) circle (  2.13);

\path[fill=fillColor,fill opacity=0.20] (203.03, 64.76) circle (  2.13);

\path[fill=fillColor,fill opacity=0.20] (205.00, 63.13) circle (  2.13);

\path[fill=fillColor,fill opacity=0.20] (206.96, 62.32) circle (  2.13);

\path[fill=fillColor,fill opacity=0.20] (209.91, 61.51) circle (  2.13);

\path[fill=fillColor,fill opacity=0.20] (208.93, 59.88) circle (  2.13);

\path[fill=fillColor,fill opacity=0.20] (210.89, 54.19) circle (  2.13);

\path[fill=fillColor,fill opacity=0.20] (227.60, 51.75) circle (  2.13);

\path[fill=fillColor,fill opacity=0.20] (211.87, 50.94) circle (  2.13);

\path[fill=fillColor,fill opacity=0.20] (203.03, 50.13) circle (  2.13);

\path[fill=fillColor,fill opacity=0.20] (212.86, 54.19) circle (  2.13);

\path[fill=fillColor,fill opacity=0.20] (212.86, 53.38) circle (  2.13);

\path[fill=fillColor,fill opacity=0.20] (216.79, 55.82) circle (  2.13);

\path[fill=fillColor,fill opacity=0.20] (220.72, 59.88) circle (  2.13);

\path[fill=fillColor,fill opacity=0.20] (216.79, 61.51) circle (  2.13);

\path[fill=fillColor,fill opacity=0.20] (219.74, 66.38) circle (  2.13);

\path[fill=fillColor,fill opacity=0.20] (220.72, 68.82) circle (  2.13);

\path[fill=fillColor,fill opacity=0.20] (217.77, 66.38) circle (  2.13);

\path[fill=fillColor,fill opacity=0.20] (219.74, 59.88) circle (  2.13);

\path[fill=fillColor,fill opacity=0.20] (217.77, 63.13) circle (  2.13);

\path[fill=fillColor,fill opacity=0.20] (220.72, 65.57) circle (  2.13);

\path[fill=fillColor,fill opacity=0.20] (217.77, 60.69) circle (  2.13);

\path[fill=fillColor,fill opacity=0.20] (226.62, 57.44) circle (  2.13);

\path[fill=fillColor,fill opacity=0.20] (224.65, 61.51) circle (  2.13);

\path[fill=fillColor,fill opacity=0.20] (230.55, 63.13) circle (  2.13);

\path[fill=fillColor,fill opacity=0.20] (228.58, 59.07) circle (  2.13);

\path[fill=fillColor,fill opacity=0.20] (211.87, 59.88) circle (  2.13);

\path[fill=fillColor,fill opacity=0.20] (225.63, 62.32) circle (  2.13);

\path[fill=fillColor,fill opacity=0.20] (225.63, 60.69) circle (  2.13);

\path[fill=fillColor,fill opacity=0.20] (220.72, 58.26) circle (  2.13);

\path[fill=fillColor,fill opacity=0.20] (218.75, 60.69) circle (  2.13);

\path[fill=fillColor,fill opacity=0.20] (222.68, 59.88) circle (  2.13);

\path[fill=fillColor,fill opacity=0.20] (217.77, 52.57) circle (  2.13);

\path[fill=fillColor,fill opacity=0.20] (208.93, 51.75) circle (  2.13);

\path[fill=fillColor,fill opacity=0.20] (214.82, 57.44) circle (  2.13);

\path[fill=fillColor,fill opacity=0.20] (214.82, 55.82) circle (  2.13);

\path[fill=fillColor,fill opacity=0.20] (215.81, 51.75) circle (  2.13);

\path[fill=fillColor,fill opacity=0.20] (212.86, 54.19) circle (  2.13);

\path[fill=fillColor,fill opacity=0.20] (207.94, 56.63) circle (  2.13);

\path[fill=fillColor,fill opacity=0.20] (208.93, 59.88) circle (  2.13);

\path[fill=fillColor,fill opacity=0.20] (208.93, 63.13) circle (  2.13);

\path[fill=fillColor,fill opacity=0.20] (203.03, 62.32) circle (  2.13);

\path[fill=fillColor,fill opacity=0.20] (202.05, 62.32) circle (  2.13);

\path[fill=fillColor,fill opacity=0.20] (202.05, 63.13) circle (  2.13);

\path[fill=fillColor,fill opacity=0.20] (205.00, 60.69) circle (  2.13);

\path[fill=fillColor,fill opacity=0.20] (205.00, 57.44) circle (  2.13);

\path[fill=fillColor,fill opacity=0.20] (204.01, 56.63) circle (  2.13);

\path[fill=fillColor,fill opacity=0.20] (201.07, 56.63) circle (  2.13);

\path[fill=fillColor,fill opacity=0.20] (211.87, 55.82) circle (  2.13);

\path[fill=fillColor,fill opacity=0.20] (210.89, 52.57) circle (  2.13);

\path[fill=fillColor,fill opacity=0.20] (213.84, 50.94) circle (  2.13);

\path[fill=fillColor,fill opacity=0.20] (221.70, 55.82) circle (  2.13);

\path[fill=fillColor,fill opacity=0.20] (221.70, 63.13) circle (  2.13);

\path[fill=fillColor,fill opacity=0.20] (214.82, 53.38) circle (  2.13);

\path[fill=fillColor,fill opacity=0.20] (207.94, 49.32) circle (  2.13);

\path[fill=fillColor,fill opacity=0.20] (215.81, 49.32) circle (  2.13);

\path[fill=fillColor,fill opacity=0.20] (214.82, 51.75) circle (  2.13);

\path[fill=fillColor,fill opacity=0.20] (212.86, 55.01) circle (  2.13);

\path[fill=fillColor,fill opacity=0.20] (211.87, 63.13) circle (  2.13);

\path[fill=fillColor,fill opacity=0.20] (213.84, 69.63) circle (  2.13);

\path[fill=fillColor,fill opacity=0.20] (212.86, 70.45) circle (  2.13);

\path[fill=fillColor,fill opacity=0.20] (213.84, 69.63) circle (  2.13);

\path[fill=fillColor,fill opacity=0.20] (214.82, 70.45) circle (  2.13);

\path[fill=fillColor,fill opacity=0.20] (214.82, 72.07) circle (  2.13);

\path[fill=fillColor,fill opacity=0.20] (213.84, 64.76) circle (  2.13);

\path[fill=fillColor,fill opacity=0.20] (206.96, 57.44) circle (  2.13);

\path[fill=fillColor,fill opacity=0.20] (216.79, 58.26) circle (  2.13);

\path[fill=fillColor,fill opacity=0.20] (218.75, 59.07) circle (  2.13);

\path[fill=fillColor,fill opacity=0.20] (215.81, 55.01) circle (  2.13);

\path[fill=fillColor,fill opacity=0.20] (214.82, 55.01) circle (  2.13);

\path[fill=fillColor,fill opacity=0.20] (215.81, 61.51) circle (  2.13);

\path[fill=fillColor,fill opacity=0.20] (214.82, 63.13) circle (  2.13);

\path[fill=fillColor,fill opacity=0.20] (209.91, 56.63) circle (  2.13);

\path[fill=fillColor,fill opacity=0.20] (216.79, 49.32) circle (  2.13);

\path[fill=fillColor,fill opacity=0.20] (218.75, 50.94) circle (  2.13);

\path[fill=fillColor,fill opacity=0.20] (214.82, 56.63) circle (  2.13);

\path[fill=fillColor,fill opacity=0.20] (212.86, 56.63) circle (  2.13);

\path[fill=fillColor,fill opacity=0.20] (210.89, 55.01) circle (  2.13);

\path[fill=fillColor,fill opacity=0.20] (214.82, 55.82) circle (  2.13);

\path[fill=fillColor,fill opacity=0.20] (210.89, 51.75) circle (  2.13);

\path[fill=fillColor,fill opacity=0.20] (209.91, 50.94) circle (  2.13);

\path[fill=fillColor,fill opacity=0.20] (209.91, 56.63) circle (  2.13);

\path[fill=fillColor,fill opacity=0.20] (208.93, 55.01) circle (  2.13);

\path[fill=fillColor,fill opacity=0.20] (206.96, 52.57) circle (  2.13);

\path[fill=fillColor,fill opacity=0.20] (207.94, 56.63) circle (  2.13);

\path[fill=fillColor,fill opacity=0.20] (206.96, 58.26) circle (  2.13);

\path[fill=fillColor,fill opacity=0.20] (205.98, 54.19) circle (  2.13);

\path[fill=fillColor,fill opacity=0.20] (206.96, 51.75) circle (  2.13);

\path[fill=fillColor,fill opacity=0.20] (206.96, 53.38) circle (  2.13);

\path[fill=fillColor,fill opacity=0.20] (207.94, 57.44) circle (  2.13);

\path[fill=fillColor,fill opacity=0.20] (211.87, 58.26) circle (  2.13);

\path[fill=fillColor,fill opacity=0.20] (212.86, 59.07) circle (  2.13);

\path[fill=fillColor,fill opacity=0.20] (221.70, 59.88) circle (  2.13);

\path[fill=fillColor,fill opacity=0.20] (222.68, 57.44) circle (  2.13);

\path[fill=fillColor,fill opacity=0.20] (225.63, 52.57) circle (  2.13);

\path[fill=fillColor,fill opacity=0.20] (214.82, 46.88) circle (  2.13);

\path[fill=fillColor,fill opacity=0.20] (206.96, 47.69) circle (  2.13);

\path[fill=fillColor,fill opacity=0.20] (211.87, 51.75) circle (  2.13);

\path[fill=fillColor,fill opacity=0.20] (206.96, 55.01) circle (  2.13);

\path[fill=fillColor,fill opacity=0.20] (212.86, 55.82) circle (  2.13);

\path[fill=fillColor,fill opacity=0.20] (214.82, 53.38) circle (  2.13);

\path[fill=fillColor,fill opacity=0.20] (216.79, 55.82) circle (  2.13);

\path[fill=fillColor,fill opacity=0.20] (211.87, 62.32) circle (  2.13);

\path[fill=fillColor,fill opacity=0.20] (217.77, 64.76) circle (  2.13);

\path[fill=fillColor,fill opacity=0.20] (207.94, 68.01) circle (  2.13);

\path[fill=fillColor,fill opacity=0.20] (214.82, 68.82) circle (  2.13);

\path[fill=fillColor,fill opacity=0.20] (212.86, 65.57) circle (  2.13);

\path[fill=fillColor,fill opacity=0.20] (217.77, 61.51) circle (  2.13);

\path[fill=fillColor,fill opacity=0.20] (214.82, 59.88) circle (  2.13);

\path[fill=fillColor,fill opacity=0.20] (213.84, 63.95) circle (  2.13);

\path[fill=fillColor,fill opacity=0.20] (212.86, 68.01) circle (  2.13);

\path[fill=fillColor,fill opacity=0.20] (214.82, 63.95) circle (  2.13);

\path[fill=fillColor,fill opacity=0.20] (216.79, 59.07) circle (  2.13);

\path[fill=fillColor,fill opacity=0.20] (207.94, 57.44) circle (  2.13);

\path[fill=fillColor,fill opacity=0.20] (214.82, 55.82) circle (  2.13);

\path[fill=fillColor,fill opacity=0.20] (214.82, 59.88) circle (  2.13);

\path[fill=fillColor,fill opacity=0.20] (214.82, 62.32) circle (  2.13);

\path[fill=fillColor,fill opacity=0.20] (215.81, 58.26) circle (  2.13);

\path[fill=fillColor,fill opacity=0.20] (212.86, 52.57) circle (  2.13);

\path[fill=fillColor,fill opacity=0.20] (210.89, 49.32) circle (  2.13);

\path[fill=fillColor,fill opacity=0.20] (210.89, 50.94) circle (  2.13);

\path[fill=fillColor,fill opacity=0.20] (210.89, 55.01) circle (  2.13);

\path[fill=fillColor,fill opacity=0.20] (210.89, 56.63) circle (  2.13);

\path[fill=fillColor,fill opacity=0.20] (213.84, 56.63) circle (  2.13);

\path[fill=fillColor,fill opacity=0.20] (216.79, 59.07) circle (  2.13);

\path[fill=fillColor,fill opacity=0.20] (217.77, 62.32) circle (  2.13);

\path[fill=fillColor,fill opacity=0.20] (215.81, 63.13) circle (  2.13);

\path[fill=fillColor,fill opacity=0.20] (214.82, 67.20) circle (  2.13);

\path[fill=fillColor,fill opacity=0.20] (217.77, 72.07) circle (  2.13);

\path[fill=fillColor,fill opacity=0.20] (224.65, 72.89) circle (  2.13);

\path[fill=fillColor,fill opacity=0.20] (231.53, 71.26) circle (  2.13);

\path[fill=fillColor,fill opacity=0.20] (216.79, 45.25) circle (  2.13);

\path[fill=fillColor,fill opacity=0.20] (214.82, 46.07) circle (  2.13);

\path[fill=fillColor,fill opacity=0.20] (212.86, 50.13) circle (  2.13);

\path[fill=fillColor,fill opacity=0.20] (209.91, 46.88) circle (  2.13);

\path[fill=fillColor,fill opacity=0.20] (215.81, 38.75) circle (  2.13);

\path[fill=fillColor,fill opacity=0.20] (215.81, 38.75) circle (  2.13);

\path[fill=fillColor,fill opacity=0.20] (217.77, 45.25) circle (  2.13);

\path[fill=fillColor,fill opacity=0.20] (212.86, 50.94) circle (  2.13);

\path[fill=fillColor,fill opacity=0.20] (214.82, 63.13) circle (  2.13);

\path[fill=fillColor,fill opacity=0.20] (212.86, 67.20) circle (  2.13);

\path[fill=fillColor,fill opacity=0.20] (217.77, 60.69) circle (  2.13);

\path[fill=fillColor,fill opacity=0.20] (217.77, 55.01) circle (  2.13);

\path[fill=fillColor,fill opacity=0.20] (216.79, 55.01) circle (  2.13);

\path[fill=fillColor,fill opacity=0.20] (215.81, 58.26) circle (  2.13);

\path[fill=fillColor,fill opacity=0.20] (209.91, 59.88) circle (  2.13);

\path[fill=fillColor,fill opacity=0.20] (218.75, 53.38) circle (  2.13);

\path[fill=fillColor,fill opacity=0.20] (218.75, 50.13) circle (  2.13);

\path[fill=fillColor,fill opacity=0.20] (214.82, 53.38) circle (  2.13);

\path[fill=fillColor,fill opacity=0.20] (214.82, 55.01) circle (  2.13);

\path[fill=fillColor,fill opacity=0.20] (211.87, 55.82) circle (  2.13);

\path[fill=fillColor,fill opacity=0.20] (214.82, 56.63) circle (  2.13);

\path[fill=fillColor,fill opacity=0.20] (215.81, 54.19) circle (  2.13);

\path[fill=fillColor,fill opacity=0.20] (211.87, 52.57) circle (  2.13);

\path[fill=fillColor,fill opacity=0.20] (215.81, 56.63) circle (  2.13);

\path[fill=fillColor,fill opacity=0.20] (216.79, 60.69) circle (  2.13);

\path[fill=fillColor,fill opacity=0.20] (219.74, 66.38) circle (  2.13);

\path[fill=fillColor,fill opacity=0.20] (223.67, 55.82) circle (  2.13);

\path[fill=fillColor,fill opacity=0.20] (218.75, 46.07) circle (  2.13);

\path[fill=fillColor,fill opacity=0.20] (213.84, 45.25) circle (  2.13);

\path[fill=fillColor,fill opacity=0.20] (216.79, 46.07) circle (  2.13);

\path[fill=fillColor,fill opacity=0.20] (210.89, 49.32) circle (  2.13);

\path[fill=fillColor,fill opacity=0.20] (213.84, 55.01) circle (  2.13);

\path[fill=fillColor,fill opacity=0.20] (212.86, 55.82) circle (  2.13);

\path[fill=fillColor,fill opacity=0.20] (215.81, 51.75) circle (  2.13);

\path[fill=fillColor,fill opacity=0.20] (217.77, 49.32) circle (  2.13);

\path[fill=fillColor,fill opacity=0.20] (219.74, 47.69) circle (  2.13);

\path[fill=fillColor,fill opacity=0.20] (218.75, 48.50) circle (  2.13);

\path[fill=fillColor,fill opacity=0.20] (221.70, 49.32) circle (  2.13);

\path[fill=fillColor,fill opacity=0.20] (219.74, 46.07) circle (  2.13);

\path[fill=fillColor,fill opacity=0.20] (218.75, 43.63) circle (  2.13);

\path[fill=fillColor,fill opacity=0.20] (216.79, 47.69) circle (  2.13);

\path[fill=fillColor,fill opacity=0.20] (222.68, 51.75) circle (  2.13);

\path[fill=fillColor,fill opacity=0.20] (214.82, 55.01) circle (  2.13);

\path[fill=fillColor,fill opacity=0.20] (222.68, 62.32) circle (  2.13);

\path[fill=fillColor,fill opacity=0.20] (213.84, 68.82) circle (  2.13);

\path[fill=fillColor,fill opacity=0.20] (222.68, 55.82) circle (  2.13);

\path[fill=fillColor,fill opacity=0.20] (223.67, 50.94) circle (  2.13);

\path[fill=fillColor,fill opacity=0.20] (222.68, 50.13) circle (  2.13);

\path[fill=fillColor,fill opacity=0.20] (225.63, 47.69) circle (  2.13);

\path[fill=fillColor,fill opacity=0.20] (187.11, 98.89) circle (  2.13);

\path[fill=fillColor,fill opacity=0.20] (191.24, 93.20) circle (  2.13);

\path[fill=fillColor,fill opacity=0.20] (190.26,109.46) circle (  2.13);

\path[fill=fillColor,fill opacity=0.20] (189.67,103.77) circle (  2.13);

\path[fill=fillColor,fill opacity=0.20] (190.26,101.33) circle (  2.13);

\path[fill=fillColor,fill opacity=0.20] (191.24, 97.27) circle (  2.13);

\path[fill=fillColor,fill opacity=0.20] (189.18, 81.01) circle (  2.13);

\path[fill=fillColor,fill opacity=0.20] (196.15, 81.01) circle (  2.13);

\path[fill=fillColor,fill opacity=0.20] (186.42, 75.32) circle (  2.13);

\path[fill=fillColor,fill opacity=0.20] (191.24, 84.26) circle (  2.13);

\path[fill=fillColor,fill opacity=0.20] (189.76, 90.76) circle (  2.13);

\path[fill=fillColor,fill opacity=0.20] (185.74, 85.08) circle (  2.13);

\path[fill=fillColor,fill opacity=0.20] (185.93, 89.14) circle (  2.13);

\path[fill=fillColor,fill opacity=0.20] (183.28, 89.14) circle (  2.13);

\path[fill=fillColor,fill opacity=0.20] (182.59, 82.64) circle (  2.13);

\path[fill=fillColor,fill opacity=0.20] (192.22, 85.08) circle (  2.13);

\path[fill=fillColor,fill opacity=0.20] (205.98, 91.58) circle (  2.13);

\path[fill=fillColor,fill opacity=0.20] (177.09, 81.01) circle (  2.13);

\path[fill=fillColor,fill opacity=0.20] (181.61, 89.14) circle (  2.13);

\path[fill=fillColor,fill opacity=0.20] (182.10, 77.76) circle (  2.13);

\path[fill=fillColor,fill opacity=0.20] (185.24, 82.64) circle (  2.13);

\path[fill=fillColor,fill opacity=0.20] (184.16, 87.51) circle (  2.13);

\path[fill=fillColor,fill opacity=0.20] (177.28, 83.45) circle (  2.13);

\path[fill=fillColor,fill opacity=0.20] (185.34, 84.26) circle (  2.13);

\path[fill=fillColor,fill opacity=0.20] (206.96, 83.45) circle (  2.13);

\path[fill=fillColor,fill opacity=0.20] (187.60, 72.89) circle (  2.13);

\path[fill=fillColor,fill opacity=0.20] (174.63, 81.01) circle (  2.13);

\path[fill=fillColor,fill opacity=0.20] (173.35, 89.95) circle (  2.13);

\path[fill=fillColor,fill opacity=0.20] (169.62, 90.76) circle (  2.13);

\path[fill=fillColor,fill opacity=0.20] (171.68, 88.33) circle (  2.13);

\path[fill=fillColor,fill opacity=0.20] (169.62, 84.26) circle (  2.13);

\path[fill=fillColor,fill opacity=0.20] (186.72, 78.57) circle (  2.13);

\path[fill=fillColor,fill opacity=0.20] (197.13, 76.14) circle (  2.13);

\path[fill=fillColor,fill opacity=0.20] (188.68, 76.95) circle (  2.13);

\path[fill=fillColor,fill opacity=0.20] (177.68, 89.14) circle (  2.13);

\path[fill=fillColor,fill opacity=0.20] (182.39, 94.02) circle (  2.13);

\path[fill=fillColor,fill opacity=0.20] (184.26, 97.27) circle (  2.13);

\path[fill=fillColor,fill opacity=0.20] (180.04, 95.64) circle (  2.13);

\path[fill=fillColor,fill opacity=0.20] (179.25, 89.14) circle (  2.13);

\path[fill=fillColor,fill opacity=0.20] (180.82, 81.01) circle (  2.13);

\path[fill=fillColor,fill opacity=0.20] (197.13, 73.70) circle (  2.13);

\path[fill=fillColor,fill opacity=0.20] (211.87, 75.32) circle (  2.13);

\path[fill=fillColor,fill opacity=0.20] (194.19, 61.51) circle (  2.13);

\path[fill=fillColor,fill opacity=0.20] (195.17, 50.13) circle (  2.13);

\path[fill=fillColor,fill opacity=0.20] (203.03, 51.75) circle (  2.13);

\path[fill=fillColor,fill opacity=0.20] (211.87, 44.44) circle (  2.13);

\path[fill=fillColor,fill opacity=0.20] (215.81, 42.00) circle (  2.13);

\path[fill=fillColor,fill opacity=0.20] (213.84, 46.07) circle (  2.13);

\path[fill=fillColor,fill opacity=0.20] (214.82, 48.50) circle (  2.13);

\path[fill=fillColor,fill opacity=0.20] (200.08, 85.08) circle (  2.13);

\path[fill=fillColor,fill opacity=0.20] (189.76, 87.51) circle (  2.13);

\path[fill=fillColor,fill opacity=0.20] (188.00, 94.83) circle (  2.13);

\path[fill=fillColor,fill opacity=0.20] (188.19, 95.64) circle (  2.13);

\path[fill=fillColor,fill opacity=0.20] (187.31, 96.45) circle (  2.13);

\path[fill=fillColor,fill opacity=0.20] (183.48, 97.27) circle (  2.13);

\path[fill=fillColor,fill opacity=0.20] (181.22, 85.08) circle (  2.13);

\path[fill=fillColor,fill opacity=0.20] (185.83, 71.26) circle (  2.13);

\path[fill=fillColor,fill opacity=0.20] (206.96, 68.01) circle (  2.13);

\path[fill=fillColor,fill opacity=0.20] (248.23, 68.01) circle (  2.13);

\path[fill=fillColor,fill opacity=0.20] (198.12, 68.01) circle (  2.13);

\path[fill=fillColor,fill opacity=0.20] (195.17, 63.13) circle (  2.13);

\path[fill=fillColor,fill opacity=0.20] (193.20, 61.51) circle (  2.13);

\path[fill=fillColor,fill opacity=0.20] (202.05, 64.76) circle (  2.13);

\path[fill=fillColor,fill opacity=0.20] (209.91, 63.13) circle (  2.13);

\path[fill=fillColor,fill opacity=0.20] (204.01, 55.82) circle (  2.13);

\path[fill=fillColor,fill opacity=0.20] (205.00, 58.26) circle (  2.13);

\path[fill=fillColor,fill opacity=0.20] (225.63, 52.57) circle (  2.13);

\path[fill=fillColor,fill opacity=0.20] (203.03, 76.95) circle (  2.13);

\path[fill=fillColor,fill opacity=0.20] (192.22, 85.89) circle (  2.13);

\path[fill=fillColor,fill opacity=0.20] (179.25, 91.58) circle (  2.13);

\path[fill=fillColor,fill opacity=0.20] (183.28, 89.14) circle (  2.13);

\path[fill=fillColor,fill opacity=0.20] (185.44, 89.95) circle (  2.13);

\path[fill=fillColor,fill opacity=0.20] (184.85, 90.76) circle (  2.13);

\path[fill=fillColor,fill opacity=0.20] (185.05, 91.58) circle (  2.13);

\path[fill=fillColor,fill opacity=0.20] (197.13, 74.51) circle (  2.13);

\path[fill=fillColor,fill opacity=0.20] (208.93, 72.89) circle (  2.13);

\path[fill=fillColor,fill opacity=0.20] (201.07, 58.26) circle (  2.13);

\path[fill=fillColor,fill opacity=0.20] (197.13, 64.76) circle (  2.13);

\path[fill=fillColor,fill opacity=0.20] (197.13, 85.89) circle (  2.13);

\path[fill=fillColor,fill opacity=0.20] (200.08, 90.76) circle (  2.13);

\path[fill=fillColor,fill opacity=0.20] (200.08, 90.76) circle (  2.13);

\path[fill=fillColor,fill opacity=0.20] (205.98, 88.33) circle (  2.13);

\path[fill=fillColor,fill opacity=0.20] (207.94, 75.32) circle (  2.13);

\path[fill=fillColor,fill opacity=0.20] (205.98, 72.07) circle (  2.13);

\path[fill=fillColor,fill opacity=0.20] (210.89, 76.95) circle (  2.13);

\path[fill=fillColor,fill opacity=0.20] (194.19, 71.26) circle (  2.13);

\path[fill=fillColor,fill opacity=0.20] (196.15, 63.95) circle (  2.13);

\path[fill=fillColor,fill opacity=0.20] (186.52, 81.01) circle (  2.13);

\path[fill=fillColor,fill opacity=0.20] (169.82, 95.64) circle (  2.13);

\path[fill=fillColor,fill opacity=0.20] (167.65, 89.95) circle (  2.13);

\path[fill=fillColor,fill opacity=0.20] (181.31, 85.89) circle (  2.13);

\path[fill=fillColor,fill opacity=0.20] (174.93, 91.58) circle (  2.13);

\path[fill=fillColor,fill opacity=0.20] (183.28, 88.33) circle (  2.13);

\path[fill=fillColor,fill opacity=0.20] (187.70, 78.57) circle (  2.13);

\path[fill=fillColor,fill opacity=0.20] (209.91, 75.32) circle (  2.13);

\path[fill=fillColor,fill opacity=0.20] (207.94, 55.82) circle (  2.13);

\path[fill=fillColor,fill opacity=0.20] (200.08, 60.69) circle (  2.13);

\path[fill=fillColor,fill opacity=0.20] (188.98, 77.76) circle (  2.13);

\path[fill=fillColor,fill opacity=0.20] (195.17, 89.14) circle (  2.13);

\path[fill=fillColor,fill opacity=0.20] (202.05, 95.64) circle (  2.13);

\path[fill=fillColor,fill opacity=0.20] (205.00,102.14) circle (  2.13);

\path[fill=fillColor,fill opacity=0.20] (207.94, 96.45) circle (  2.13);

\path[fill=fillColor,fill opacity=0.20] (209.91, 81.82) circle (  2.13);

\path[fill=fillColor,fill opacity=0.20] (212.86, 76.14) circle (  2.13);

\path[fill=fillColor,fill opacity=0.20] (220.72, 74.51) circle (  2.13);

\path[fill=fillColor,fill opacity=0.20] (181.90, 83.45) circle (  2.13);

\path[fill=fillColor,fill opacity=0.20] (192.22, 85.89) circle (  2.13);

\path[fill=fillColor,fill opacity=0.20] (191.24, 86.70) circle (  2.13);

\path[fill=fillColor,fill opacity=0.20] (186.42, 99.70) circle (  2.13);

\path[fill=fillColor,fill opacity=0.20] (183.57, 94.83) circle (  2.13);

\path[fill=fillColor,fill opacity=0.20] (182.59, 86.70) circle (  2.13);

\path[fill=fillColor,fill opacity=0.20] (179.05, 90.76) circle (  2.13);

\path[fill=fillColor,fill opacity=0.20] (197.13, 74.51) circle (  2.13);

\path[fill=fillColor,fill opacity=0.20] (229.56, 68.01) circle (  2.13);

\path[fill=fillColor,fill opacity=0.20] (214.82, 53.38) circle (  2.13);

\path[fill=fillColor,fill opacity=0.20] (199.10, 64.76) circle (  2.13);

\path[fill=fillColor,fill opacity=0.20] (193.20, 86.70) circle (  2.13);

\path[fill=fillColor,fill opacity=0.20] (200.08, 88.33) circle (  2.13);

\path[fill=fillColor,fill opacity=0.20] (204.01, 88.33) circle (  2.13);

\path[fill=fillColor,fill opacity=0.20] (209.91, 96.45) circle (  2.13);

\path[fill=fillColor,fill opacity=0.20] (214.82, 90.76) circle (  2.13);

\path[fill=fillColor,fill opacity=0.20] (216.79, 77.76) circle (  2.13);

\path[fill=fillColor,fill opacity=0.20] (212.86, 73.70) circle (  2.13);

\path[fill=fillColor,fill opacity=0.20] (244.30, 70.45) circle (  2.13);

\path[fill=fillColor,fill opacity=0.20] (200.08, 85.89) circle (  2.13);

\path[fill=fillColor,fill opacity=0.20] (196.15, 79.39) circle (  2.13);

\path[fill=fillColor,fill opacity=0.20] (193.20, 90.76) circle (  2.13);

\path[fill=fillColor,fill opacity=0.20] (188.78, 99.70) circle (  2.13);

\path[fill=fillColor,fill opacity=0.20] (191.24, 97.27) circle (  2.13);

\path[fill=fillColor,fill opacity=0.20] (190.16, 92.39) circle (  2.13);

\path[fill=fillColor,fill opacity=0.20] (183.28, 89.95) circle (  2.13);

\path[fill=fillColor,fill opacity=0.20] (185.24, 90.76) circle (  2.13);

\path[fill=fillColor,fill opacity=0.20] (186.23, 96.45) circle (  2.13);

\path[fill=fillColor,fill opacity=0.20] (182.39, 92.39) circle (  2.13);

\path[fill=fillColor,fill opacity=0.20] (210.89, 77.76) circle (  2.13);

\path[fill=fillColor,fill opacity=0.20] (271.82, 69.63) circle (  2.13);

\path[fill=fillColor,fill opacity=0.20] (221.70, 56.63) circle (  2.13);

\path[fill=fillColor,fill opacity=0.20] (205.00, 61.51) circle (  2.13);

\path[fill=fillColor,fill opacity=0.20] (194.19, 84.26) circle (  2.13);

\path[fill=fillColor,fill opacity=0.20] (198.12, 91.58) circle (  2.13);

\path[fill=fillColor,fill opacity=0.20] (205.00, 87.51) circle (  2.13);

\path[fill=fillColor,fill opacity=0.20] (203.03, 89.95) circle (  2.13);

\path[fill=fillColor,fill opacity=0.20] (208.93, 86.70) circle (  2.13);

\path[fill=fillColor,fill opacity=0.20] (214.82, 74.51) circle (  2.13);

\path[fill=fillColor,fill opacity=0.20] (228.58, 76.14) circle (  2.13);

\path[fill=fillColor,fill opacity=0.20] (202.05, 83.45) circle (  2.13);

\path[fill=fillColor,fill opacity=0.20] (193.20, 86.70) circle (  2.13);

\path[fill=fillColor,fill opacity=0.20] (193.20, 88.33) circle (  2.13);

\path[fill=fillColor,fill opacity=0.20] (190.16,106.21) circle (  2.13);

\path[fill=fillColor,fill opacity=0.20] (188.98,109.46) circle (  2.13);

\path[fill=fillColor,fill opacity=0.20] (191.24, 93.20) circle (  2.13);

\path[fill=fillColor,fill opacity=0.20] (189.27, 88.33) circle (  2.13);

\path[fill=fillColor,fill opacity=0.20] (190.06, 90.76) circle (  2.13);

\path[fill=fillColor,fill opacity=0.20] (192.22, 89.95) circle (  2.13);

\path[fill=fillColor,fill opacity=0.20] (214.82, 85.89) circle (  2.13);

\path[fill=fillColor,fill opacity=0.20] (261.01, 80.20) circle (  2.13);

\path[fill=fillColor,fill opacity=0.20] (216.79, 56.63) circle (  2.13);

\path[fill=fillColor,fill opacity=0.20] (192.22, 72.07) circle (  2.13);

\path[fill=fillColor,fill opacity=0.20] (190.26, 87.51) circle (  2.13);

\path[fill=fillColor,fill opacity=0.20] (194.19, 86.70) circle (  2.13);

\path[fill=fillColor,fill opacity=0.20] (194.19, 87.51) circle (  2.13);

\path[fill=fillColor,fill opacity=0.20] (192.22, 92.39) circle (  2.13);

\path[fill=fillColor,fill opacity=0.20] (202.05, 85.08) circle (  2.13);

\path[fill=fillColor,fill opacity=0.20] (212.86, 80.20) circle (  2.13);

\path[fill=fillColor,fill opacity=0.20] (198.12, 85.89) circle (  2.13);

\path[fill=fillColor,fill opacity=0.20] (183.28, 89.95) circle (  2.13);

\path[fill=fillColor,fill opacity=0.20] (184.56, 83.45) circle (  2.13);

\path[fill=fillColor,fill opacity=0.20] (191.24, 91.58) circle (  2.13);

\path[fill=fillColor,fill opacity=0.20] (191.24,107.02) circle (  2.13);

\path[fill=fillColor,fill opacity=0.20] (189.57,111.89) circle (  2.13);

\path[fill=fillColor,fill opacity=0.20] (193.20,102.14) circle (  2.13);

\path[fill=fillColor,fill opacity=0.20] (197.13, 91.58) circle (  2.13);

\path[fill=fillColor,fill opacity=0.20] (205.00, 82.64) circle (  2.13);

\path[fill=fillColor,fill opacity=0.20] (225.63, 72.07) circle (  2.13);

\path[fill=fillColor,fill opacity=0.20] (204.01, 62.32) circle (  2.13);

\path[fill=fillColor,fill opacity=0.20] (196.15, 76.95) circle (  2.13);

\path[fill=fillColor,fill opacity=0.20] (196.15, 81.01) circle (  2.13);

\path[fill=fillColor,fill opacity=0.20] (187.70, 83.45) circle (  2.13);

\path[fill=fillColor,fill opacity=0.20] (188.49, 95.64) circle (  2.13);

\path[fill=fillColor,fill opacity=0.20] (197.13, 99.70) circle (  2.13);

\path[fill=fillColor,fill opacity=0.20] (203.03, 88.33) circle (  2.13);

\path[fill=fillColor,fill opacity=0.20] (210.89, 99.70) circle (  2.13);

\path[fill=fillColor,fill opacity=0.20] (201.07, 79.39) circle (  2.13);

\path[fill=fillColor,fill opacity=0.20] (185.34, 95.64) circle (  2.13);

\path[fill=fillColor,fill opacity=0.20] (187.11, 94.02) circle (  2.13);

\path[fill=fillColor,fill opacity=0.20] (192.22, 94.02) circle (  2.13);

\path[fill=fillColor,fill opacity=0.20] (193.20, 92.39) circle (  2.13);

\path[fill=fillColor,fill opacity=0.20] (195.17, 96.45) circle (  2.13);

\path[fill=fillColor,fill opacity=0.20] (207.94, 92.39) circle (  2.13);

\path[fill=fillColor,fill opacity=0.20] (214.82, 82.64) circle (  2.13);

\path[fill=fillColor,fill opacity=0.20] (222.68, 71.26) circle (  2.13);

\path[fill=fillColor,fill opacity=0.20] (211.87, 63.95) circle (  2.13);

\path[fill=fillColor,fill opacity=0.20] (201.07, 70.45) circle (  2.13);

\path[fill=fillColor,fill opacity=0.20] (201.07, 76.14) circle (  2.13);

\path[fill=fillColor,fill opacity=0.20] (195.17, 85.89) circle (  2.13);

\path[fill=fillColor,fill opacity=0.20] (195.17, 94.02) circle (  2.13);

\path[fill=fillColor,fill opacity=0.20] (200.08, 97.27) circle (  2.13);

\path[fill=fillColor,fill opacity=0.20] (201.07, 95.64) circle (  2.13);

\path[fill=fillColor,fill opacity=0.20] (205.98, 98.89) circle (  2.13);

\path[fill=fillColor,fill opacity=0.20] (198.12, 68.82) circle (  2.13);

\path[fill=fillColor,fill opacity=0.20] (194.19, 81.01) circle (  2.13);

\path[fill=fillColor,fill opacity=0.20] (182.20, 91.58) circle (  2.13);

\path[fill=fillColor,fill opacity=0.20] (183.18, 89.95) circle (  2.13);

\path[fill=fillColor,fill opacity=0.20] (191.24, 83.45) circle (  2.13);

\path[fill=fillColor,fill opacity=0.20] (203.03, 78.57) circle (  2.13);

\path[fill=fillColor,fill opacity=0.20] (213.84, 76.95) circle (  2.13);

\path[fill=fillColor,fill opacity=0.20] (214.82, 81.82) circle (  2.13);

\path[fill=fillColor,fill opacity=0.20] (221.70, 75.32) circle (  2.13);

\path[fill=fillColor,fill opacity=0.20] (224.65, 64.76) circle (  2.13);

\path[fill=fillColor,fill opacity=0.20] (212.86, 68.01) circle (  2.13);

\path[fill=fillColor,fill opacity=0.20] (204.01, 68.01) circle (  2.13);

\path[fill=fillColor,fill opacity=0.20] (199.10, 83.45) circle (  2.13);

\path[fill=fillColor,fill opacity=0.20] (196.15, 91.58) circle (  2.13);

\path[fill=fillColor,fill opacity=0.20] (198.12, 89.14) circle (  2.13);

\path[fill=fillColor,fill opacity=0.20] (198.12, 94.02) circle (  2.13);

\path[fill=fillColor,fill opacity=0.20] (201.07, 99.70) circle (  2.13);

\path[fill=fillColor,fill opacity=0.20] (203.03, 98.08) circle (  2.13);

\path[fill=fillColor,fill opacity=0.20] (193.20, 81.01) circle (  2.13);

\path[fill=fillColor,fill opacity=0.20] (186.03, 86.70) circle (  2.13);

\path[fill=fillColor,fill opacity=0.20] (188.09, 81.82) circle (  2.13);

\path[fill=fillColor,fill opacity=0.20] (176.50, 75.32) circle (  2.13);

\path[fill=fillColor,fill opacity=0.20] (188.19, 64.76) circle (  2.13);

\path[fill=fillColor,fill opacity=0.20] (180.04, 62.32) circle (  2.13);

\path[fill=fillColor,fill opacity=0.20] (231.53, 72.89) circle (  2.13);

\path[fill=fillColor,fill opacity=0.20] (220.72, 59.88) circle (  2.13);

\path[fill=fillColor,fill opacity=0.20] (208.93, 47.69) circle (  2.13);

\path[fill=fillColor,fill opacity=0.20] (199.10, 67.20) circle (  2.13);

\path[fill=fillColor,fill opacity=0.20] (196.15, 85.08) circle (  2.13);

\path[fill=fillColor,fill opacity=0.20] (198.12, 86.70) circle (  2.13);

\path[fill=fillColor,fill opacity=0.20] (200.08, 87.51) circle (  2.13);

\path[fill=fillColor,fill opacity=0.20] (199.10, 91.58) circle (  2.13);

\path[fill=fillColor,fill opacity=0.20] (200.08, 96.45) circle (  2.13);

\path[fill=fillColor,fill opacity=0.20] (205.00, 94.83) circle (  2.13);

\path[fill=fillColor,fill opacity=0.20] (191.24, 73.70) circle (  2.13);

\path[fill=fillColor,fill opacity=0.20] (192.22, 78.57) circle (  2.13);

\path[fill=fillColor,fill opacity=0.20] (197.13, 72.07) circle (  2.13);

\path[fill=fillColor,fill opacity=0.20] (191.24, 64.76) circle (  2.13);

\path[fill=fillColor,fill opacity=0.20] (214.82, 59.88) circle (  2.13);

\path[fill=fillColor,fill opacity=0.20] (233.49, 76.95) circle (  2.13);

\path[fill=fillColor,fill opacity=0.20] (205.00, 50.13) circle (  2.13);

\path[fill=fillColor,fill opacity=0.20] (200.08, 68.01) circle (  2.13);

\path[fill=fillColor,fill opacity=0.20] (199.10, 76.14) circle (  2.13);

\path[fill=fillColor,fill opacity=0.20] (200.08, 83.45) circle (  2.13);

\path[fill=fillColor,fill opacity=0.20] (205.00, 89.14) circle (  2.13);

\path[fill=fillColor,fill opacity=0.20] (205.98, 94.02) circle (  2.13);

\path[fill=fillColor,fill opacity=0.20] (206.96, 86.70) circle (  2.13);

\path[fill=fillColor,fill opacity=0.20] (208.93, 80.20) circle (  2.13);

\path[fill=fillColor,fill opacity=0.20] (189.67, 63.13) circle (  2.13);

\path[fill=fillColor,fill opacity=0.20] (192.22, 70.45) circle (  2.13);

\path[fill=fillColor,fill opacity=0.20] (198.12, 80.20) circle (  2.13);

\path[fill=fillColor,fill opacity=0.20] (205.98, 67.20) circle (  2.13);

\path[fill=fillColor,fill opacity=0.20] (215.81, 46.88) circle (  2.13);

\path[fill=fillColor,fill opacity=0.20] (192.22, 57.44) circle (  2.13);

\path[fill=fillColor,fill opacity=0.20] (227.60, 59.07) circle (  2.13);

\path[fill=fillColor,fill opacity=0.20] (210.89, 59.88) circle (  2.13);

\path[fill=fillColor,fill opacity=0.20] (198.12, 71.26) circle (  2.13);

\path[fill=fillColor,fill opacity=0.20] (205.98, 82.64) circle (  2.13);

\path[fill=fillColor,fill opacity=0.20] (205.00, 82.64) circle (  2.13);

\path[fill=fillColor,fill opacity=0.20] (195.17, 86.70) circle (  2.13);

\path[fill=fillColor,fill opacity=0.20] (201.07, 90.76) circle (  2.13);

\path[fill=fillColor,fill opacity=0.20] (201.07, 85.89) circle (  2.13);

\path[fill=fillColor,fill opacity=0.20] (204.01, 95.64) circle (  2.13);

\path[fill=fillColor,fill opacity=0.20] (199.10, 53.38) circle (  2.13);

\path[fill=fillColor,fill opacity=0.20] (189.76, 63.95) circle (  2.13);

\path[fill=fillColor,fill opacity=0.20] (196.15, 68.82) circle (  2.13);

\path[fill=fillColor,fill opacity=0.20] (205.00, 53.38) circle (  2.13);

\path[fill=fillColor,fill opacity=0.20] (243.32, 55.82) circle (  2.13);

\path[fill=fillColor,fill opacity=0.20] (209.91, 56.63) circle (  2.13);

\path[fill=fillColor,fill opacity=0.20] (208.93, 59.88) circle (  2.13);

\path[fill=fillColor,fill opacity=0.20] (199.10, 65.57) circle (  2.13);

\path[fill=fillColor,fill opacity=0.20] (198.12, 85.08) circle (  2.13);

\path[fill=fillColor,fill opacity=0.20] (198.12, 96.45) circle (  2.13);

\path[fill=fillColor,fill opacity=0.20] (201.07, 92.39) circle (  2.13);

\path[fill=fillColor,fill opacity=0.20] (207.94, 86.70) circle (  2.13);

\path[fill=fillColor,fill opacity=0.20] (203.03, 92.39) circle (  2.13);

\path[fill=fillColor,fill opacity=0.20] (205.98,102.14) circle (  2.13);

\path[fill=fillColor,fill opacity=0.20] (204.01, 57.44) circle (  2.13);

\path[fill=fillColor,fill opacity=0.20] (203.03, 67.20) circle (  2.13);

\path[fill=fillColor,fill opacity=0.20] (200.08, 66.38) circle (  2.13);

\path[fill=fillColor,fill opacity=0.20] (198.12, 63.95) circle (  2.13);

\path[fill=fillColor,fill opacity=0.20] (198.12, 68.01) circle (  2.13);

\path[fill=fillColor,fill opacity=0.20] (230.55, 55.82) circle (  2.13);

\path[fill=fillColor,fill opacity=0.20] (206.96, 51.75) circle (  2.13);

\path[fill=fillColor,fill opacity=0.20] (199.10, 59.88) circle (  2.13);

\path[fill=fillColor,fill opacity=0.20] (193.20, 74.51) circle (  2.13);

\path[fill=fillColor,fill opacity=0.20] (196.15, 83.45) circle (  2.13);

\path[fill=fillColor,fill opacity=0.20] (199.10, 88.33) circle (  2.13);

\path[fill=fillColor,fill opacity=0.20] (206.96, 86.70) circle (  2.13);

\path[fill=fillColor,fill opacity=0.20] (204.01, 96.45) circle (  2.13);

\path[fill=fillColor,fill opacity=0.20] (204.01,102.96) circle (  2.13);

\path[fill=fillColor,fill opacity=0.20] (210.89, 94.02) circle (  2.13);

\path[fill=fillColor,fill opacity=0.20] (216.79, 85.08) circle (  2.13);

\path[fill=fillColor,fill opacity=0.20] (219.74, 89.14) circle (  2.13);

\path[fill=fillColor,fill opacity=0.20] (215.81, 76.95) circle (  2.13);

\path[fill=fillColor,fill opacity=0.20] (211.87, 65.57) circle (  2.13);

\path[fill=fillColor,fill opacity=0.20] (207.94, 63.13) circle (  2.13);

\path[fill=fillColor,fill opacity=0.20] (205.00, 63.95) circle (  2.13);

\path[fill=fillColor,fill opacity=0.20] (205.00, 72.07) circle (  2.13);

\path[fill=fillColor,fill opacity=0.20] (199.10, 76.95) circle (  2.13);

\path[fill=fillColor,fill opacity=0.20] (197.13, 72.89) circle (  2.13);

\path[fill=fillColor,fill opacity=0.20] (200.08, 66.38) circle (  2.13);

\path[fill=fillColor,fill opacity=0.20] (216.79, 59.88) circle (  2.13);

\path[fill=fillColor,fill opacity=0.20] (215.81, 81.01) circle (  2.13);

\path[fill=fillColor,fill opacity=0.20] (231.53, 63.95) circle (  2.13);

\path[fill=fillColor,fill opacity=0.20] (211.87, 60.69) circle (  2.13);

\path[fill=fillColor,fill opacity=0.20] (199.10, 63.95) circle (  2.13);

\path[fill=fillColor,fill opacity=0.20] (200.08, 72.07) circle (  2.13);

\path[fill=fillColor,fill opacity=0.20] (202.05, 75.32) circle (  2.13);

\path[fill=fillColor,fill opacity=0.20] (204.01, 86.70) circle (  2.13);

\path[fill=fillColor,fill opacity=0.20] (205.00,102.14) circle (  2.13);

\path[fill=fillColor,fill opacity=0.20] (206.96,100.52) circle (  2.13);

\path[fill=fillColor,fill opacity=0.20] (207.94, 90.76) circle (  2.13);

\path[fill=fillColor,fill opacity=0.20] (210.89, 85.08) circle (  2.13);

\path[fill=fillColor,fill opacity=0.20] (218.75, 81.82) circle (  2.13);

\path[fill=fillColor,fill opacity=0.20] (205.00, 85.89) circle (  2.13);

\path[fill=fillColor,fill opacity=0.20] (217.77, 81.01) circle (  2.13);

\path[fill=fillColor,fill opacity=0.20] (213.84, 78.57) circle (  2.13);

\path[fill=fillColor,fill opacity=0.20] (210.89, 69.63) circle (  2.13);

\path[fill=fillColor,fill opacity=0.20] (215.81, 63.13) circle (  2.13);

\path[fill=fillColor,fill opacity=0.20] (214.82, 64.76) circle (  2.13);

\path[fill=fillColor,fill opacity=0.20] (213.84, 70.45) circle (  2.13);

\path[fill=fillColor,fill opacity=0.20] (202.05, 72.07) circle (  2.13);

\path[fill=fillColor,fill opacity=0.20] (198.12, 69.63) circle (  2.13);

\path[fill=fillColor,fill opacity=0.20] (202.05, 77.76) circle (  2.13);

\path[fill=fillColor,fill opacity=0.20] (205.98, 78.57) circle (  2.13);

\path[fill=fillColor,fill opacity=0.20] (205.00, 67.20) circle (  2.13);

\path[fill=fillColor,fill opacity=0.20] (223.67, 64.76) circle (  2.13);

\path[fill=fillColor,fill opacity=0.20] (201.07, 73.70) circle (  2.13);

\path[fill=fillColor,fill opacity=0.20] (232.51, 67.20) circle (  2.13);

\path[fill=fillColor,fill opacity=0.20] (205.00, 60.69) circle (  2.13);

\path[fill=fillColor,fill opacity=0.20] (204.01, 58.26) circle (  2.13);

\path[fill=fillColor,fill opacity=0.20] (204.01, 62.32) circle (  2.13);

\path[fill=fillColor,fill opacity=0.20] (202.05, 72.89) circle (  2.13);

\path[fill=fillColor,fill opacity=0.20] (205.98, 83.45) circle (  2.13);

\path[fill=fillColor,fill opacity=0.20] (202.05, 89.95) circle (  2.13);

\path[fill=fillColor,fill opacity=0.20] (200.08, 88.33) circle (  2.13);

\path[fill=fillColor,fill opacity=0.20] (207.94, 84.26) circle (  2.13);

\path[fill=fillColor,fill opacity=0.20] (212.86, 87.51) circle (  2.13);

\path[fill=fillColor,fill opacity=0.20] (213.84, 88.33) circle (  2.13);

\path[fill=fillColor,fill opacity=0.20] (213.84, 81.01) circle (  2.13);

\path[fill=fillColor,fill opacity=0.20] (211.87, 75.32) circle (  2.13);

\path[fill=fillColor,fill opacity=0.20] (215.81, 73.70) circle (  2.13);

\path[fill=fillColor,fill opacity=0.20] (211.87, 72.89) circle (  2.13);

\path[fill=fillColor,fill opacity=0.20] (214.82, 78.57) circle (  2.13);

\path[fill=fillColor,fill opacity=0.20] (212.86, 77.76) circle (  2.13);

\path[fill=fillColor,fill opacity=0.20] (210.89, 72.89) circle (  2.13);

\path[fill=fillColor,fill opacity=0.20] (209.91, 76.14) circle (  2.13);

\path[fill=fillColor,fill opacity=0.20] (215.81, 77.76) circle (  2.13);

\path[fill=fillColor,fill opacity=0.20] (217.77, 78.57) circle (  2.13);

\path[fill=fillColor,fill opacity=0.20] (212.86, 80.20) circle (  2.13);

\path[fill=fillColor,fill opacity=0.20] (212.86, 74.51) circle (  2.13);

\path[fill=fillColor,fill opacity=0.20] (214.82, 69.63) circle (  2.13);

\path[fill=fillColor,fill opacity=0.20] (213.84, 72.07) circle (  2.13);

\path[fill=fillColor,fill opacity=0.20] (210.89, 75.32) circle (  2.13);

\path[fill=fillColor,fill opacity=0.20] (206.96, 80.20) circle (  2.13);

\path[fill=fillColor,fill opacity=0.20] (205.98, 80.20) circle (  2.13);

\path[fill=fillColor,fill opacity=0.20] (204.01, 81.01) circle (  2.13);

\path[fill=fillColor,fill opacity=0.20] (202.05, 86.70) circle (  2.13);

\path[fill=fillColor,fill opacity=0.20] (202.05, 82.64) circle (  2.13);

\path[fill=fillColor,fill opacity=0.20] (199.10, 73.70) circle (  2.13);

\path[fill=fillColor,fill opacity=0.20] (205.98, 70.45) circle (  2.13);

\path[fill=fillColor,fill opacity=0.20] (214.82, 65.57) circle (  2.13);

\path[fill=fillColor,fill opacity=0.20] (207.94, 52.57) circle (  2.13);

\path[fill=fillColor,fill opacity=0.20] (223.67, 50.94) circle (  2.13);

\path[fill=fillColor,fill opacity=0.20] (212.86, 43.63) circle (  2.13);

\path[fill=fillColor,fill opacity=0.20] (202.05, 53.38) circle (  2.13);

\path[fill=fillColor,fill opacity=0.20] (203.03, 75.32) circle (  2.13);

\path[fill=fillColor,fill opacity=0.20] (201.07, 81.82) circle (  2.13);

\path[fill=fillColor,fill opacity=0.20] (203.03, 76.95) circle (  2.13);

\path[fill=fillColor,fill opacity=0.20] (204.01, 79.39) circle (  2.13);

\path[fill=fillColor,fill opacity=0.20] (207.94, 85.89) circle (  2.13);

\path[fill=fillColor,fill opacity=0.20] (203.03, 88.33) circle (  2.13);

\path[fill=fillColor,fill opacity=0.20] (201.07, 84.26) circle (  2.13);

\path[fill=fillColor,fill opacity=0.20] (203.03, 76.95) circle (  2.13);

\path[fill=fillColor,fill opacity=0.20] (207.94, 73.70) circle (  2.13);

\path[fill=fillColor,fill opacity=0.20] (206.96, 78.57) circle (  2.13);

\path[fill=fillColor,fill opacity=0.20] (206.96, 81.82) circle (  2.13);

\path[fill=fillColor,fill opacity=0.20] (208.93, 79.39) circle (  2.13);

\path[fill=fillColor,fill opacity=0.20] (206.96, 78.57) circle (  2.13);

\path[fill=fillColor,fill opacity=0.20] (209.91, 76.95) circle (  2.13);

\path[fill=fillColor,fill opacity=0.20] (211.87, 76.14) circle (  2.13);

\path[fill=fillColor,fill opacity=0.20] (205.00, 76.95) circle (  2.13);

\path[fill=fillColor,fill opacity=0.20] (207.94, 78.57) circle (  2.13);

\path[fill=fillColor,fill opacity=0.20] (205.00, 77.76) circle (  2.13);

\path[fill=fillColor,fill opacity=0.20] (203.03, 77.76) circle (  2.13);

\path[fill=fillColor,fill opacity=0.20] (203.03, 82.64) circle (  2.13);

\path[fill=fillColor,fill opacity=0.20] (197.13, 86.70) circle (  2.13);

\path[fill=fillColor,fill opacity=0.20] (200.08, 79.39) circle (  2.13);

\path[fill=fillColor,fill opacity=0.20] (203.03, 74.51) circle (  2.13);

\path[fill=fillColor,fill opacity=0.20] (193.20, 76.14) circle (  2.13);

\path[fill=fillColor,fill opacity=0.20] (216.79, 68.82) circle (  2.13);

\path[fill=fillColor,fill opacity=0.20] (224.65, 66.38) circle (  2.13);

\path[fill=fillColor,fill opacity=0.20] (192.22, 76.14) circle (  2.13);

\path[fill=fillColor,fill opacity=0.20] (264.94, 48.50) circle (  2.13);

\path[fill=fillColor,fill opacity=0.20] (229.56, 49.32) circle (  2.13);

\path[fill=fillColor,fill opacity=0.20] (188.49, 62.32) circle (  2.13);

\path[fill=fillColor,fill opacity=0.20] (204.01, 70.45) circle (  2.13);

\path[fill=fillColor,fill opacity=0.20] (215.81, 64.76) circle (  2.13);

\path[fill=fillColor,fill opacity=0.20] (208.93, 57.44) circle (  2.13);

\path[fill=fillColor,fill opacity=0.20] (207.94, 63.95) circle (  2.13);

\path[fill=fillColor,fill opacity=0.20] (195.17, 70.45) circle (  2.13);

\path[fill=fillColor,fill opacity=0.20] (199.10, 67.20) circle (  2.13);

\path[fill=fillColor,fill opacity=0.20] (201.07, 59.88) circle (  2.13);

\path[fill=fillColor,fill opacity=0.20] (205.98, 58.26) circle (  2.13);

\path[fill=fillColor,fill opacity=0.20] (202.05, 63.13) circle (  2.13);

\path[fill=fillColor,fill opacity=0.20] (205.00, 70.45) circle (  2.13);

\path[fill=fillColor,fill opacity=0.20] (204.01, 75.32) circle (  2.13);

\path[fill=fillColor,fill opacity=0.20] (208.93, 74.51) circle (  2.13);

\path[fill=fillColor,fill opacity=0.20] (212.86, 72.07) circle (  2.13);

\path[fill=fillColor,fill opacity=0.20] (213.84, 70.45) circle (  2.13);

\path[fill=fillColor,fill opacity=0.20] (202.05, 67.20) circle (  2.13);

\path[fill=fillColor,fill opacity=0.20] (207.94, 62.32) circle (  2.13);

\path[fill=fillColor,fill opacity=0.20] (216.79, 62.32) circle (  2.13);

\path[fill=fillColor,fill opacity=0.20] (193.20, 65.57) circle (  2.13);

\path[fill=fillColor,fill opacity=0.20] (209.91, 63.95) circle (  2.13);

\path[fill=fillColor,fill opacity=0.20] (210.89, 60.69) circle (  2.13);

\path[fill=fillColor,fill opacity=0.20] (197.13, 59.07) circle (  2.13);

\path[fill=fillColor,fill opacity=0.20] (229.56, 57.44) circle (  2.13);

\path[fill=fillColor,fill opacity=0.20] (237.42, 59.07) circle (  2.13);

\path[fill=fillColor,fill opacity=0.20] (258.06, 56.63) circle (  2.13);

\path[fill=fillColor,fill opacity=0.20] (252.16, 51.75) circle (  2.13);

\path[fill=fillColor,fill opacity=0.20] (235.46, 47.69) circle (  2.13);

\path[fill=fillColor,fill opacity=0.20] (241.36, 44.44) circle (  2.13);

\path[fill=fillColor,fill opacity=0.20] (239.39, 44.44) circle (  2.13);

\path[fill=fillColor,fill opacity=0.20] (217.77, 43.63) circle (  2.13);

\path[fill=fillColor,fill opacity=0.20] (227.60, 46.07) circle (  2.13);

\path[fill=fillColor,fill opacity=0.20] (230.55, 53.38) circle (  2.13);

\path[fill=fillColor,fill opacity=0.20] (240.37, 60.69) circle (  2.13);

\path[fill=fillColor,fill opacity=0.20] (224.65, 65.57) circle (  2.13);

\path[fill=fillColor,fill opacity=0.20] (221.70, 66.38) circle (  2.13);

\path[fill=fillColor,fill opacity=0.20] (223.67, 69.63) circle (  2.13);

\path[fill=fillColor,fill opacity=0.20] (227.60, 73.70) circle (  2.13);

\path[fill=fillColor,fill opacity=0.20] (205.00, 55.01) circle (  2.13);

\path[fill=fillColor,fill opacity=0.20] (219.74, 63.13) circle (  2.13);

\path[fill=fillColor,fill opacity=0.20] (251.18, 69.63) circle (  2.13);

\path[fill=fillColor,fill opacity=0.20] (210.89, 54.19) circle (  2.13);

\path[fill=fillColor,fill opacity=0.20] (199.10, 62.32) circle (  2.13);

\path[fill=fillColor,fill opacity=0.20] (189.67, 64.76) circle (  2.13);

\path[fill=fillColor,fill opacity=0.20] (193.20, 67.20) circle (  2.13);

\path[fill=fillColor,fill opacity=0.20] (199.10, 67.20) circle (  2.13);

\path[fill=fillColor,fill opacity=0.20] (206.96, 61.51) circle (  2.13);

\path[fill=fillColor,fill opacity=0.20] (209.91, 54.19) circle (  2.13);

\path[fill=fillColor,fill opacity=0.20] (209.91, 51.75) circle (  2.13);

\path[fill=fillColor,fill opacity=0.20] (210.89, 54.19) circle (  2.13);

\path[fill=fillColor,fill opacity=0.20] (215.81, 55.01) circle (  2.13);

\path[fill=fillColor,fill opacity=0.20] (218.75, 52.57) circle (  2.13);

\path[fill=fillColor,fill opacity=0.20] (211.87, 49.32) circle (  2.13);

\path[fill=fillColor,fill opacity=0.20] (194.19, 59.07) circle (  2.13);

\path[fill=fillColor,fill opacity=0.20] (202.05, 59.88) circle (  2.13);

\path[fill=fillColor,fill opacity=0.20] (175.42, 73.70) circle (  2.13);

\path[fill=fillColor,fill opacity=0.20] (199.10, 84.26) circle (  2.13);

\path[fill=fillColor,fill opacity=0.20] (201.07, 81.82) circle (  2.13);

\path[fill=fillColor,fill opacity=0.20] (205.98, 76.14) circle (  2.13);

\path[fill=fillColor,fill opacity=0.20] (210.89, 70.45) circle (  2.13);

\path[fill=fillColor,fill opacity=0.20] (210.89, 63.95) circle (  2.13);

\path[fill=fillColor,fill opacity=0.20] (210.89, 59.07) circle (  2.13);

\path[fill=fillColor,fill opacity=0.20] (218.75, 62.32) circle (  2.13);

\path[fill=fillColor,fill opacity=0.20] (220.72, 62.32) circle (  2.13);

\path[fill=fillColor,fill opacity=0.20] (219.74, 55.01) circle (  2.13);

\path[fill=fillColor,fill opacity=0.20] (198.12, 69.63) circle (  2.13);

\path[fill=fillColor,fill opacity=0.20] (197.13, 72.07) circle (  2.13);

\path[fill=fillColor,fill opacity=0.20] (201.07, 79.39) circle (  2.13);

\path[fill=fillColor,fill opacity=0.20] (202.05, 90.76) circle (  2.13);

\path[fill=fillColor,fill opacity=0.20] (203.03, 96.45) circle (  2.13);

\path[fill=fillColor,fill opacity=0.20] (205.00, 92.39) circle (  2.13);

\path[fill=fillColor,fill opacity=0.20] (206.96, 87.51) circle (  2.13);

\path[fill=fillColor,fill opacity=0.20] (210.89, 85.89) circle (  2.13);

\path[fill=fillColor,fill opacity=0.20] (212.86, 79.39) circle (  2.13);

\path[fill=fillColor,fill opacity=0.20] (216.79, 71.26) circle (  2.13);

\path[fill=fillColor,fill opacity=0.20] (225.63, 69.63) circle (  2.13);

\path[fill=fillColor,fill opacity=0.20] (170.60, 81.01) circle (  2.13);

\path[fill=fillColor,fill opacity=0.20] (197.13, 77.76) circle (  2.13);

\path[fill=fillColor,fill opacity=0.20] (202.05, 86.70) circle (  2.13);

\path[fill=fillColor,fill opacity=0.20] (204.01, 99.70) circle (  2.13);

\path[fill=fillColor,fill opacity=0.20] (205.00,104.58) circle (  2.13);

\path[fill=fillColor,fill opacity=0.20] (206.96,100.52) circle (  2.13);

\path[fill=fillColor,fill opacity=0.20] (207.94, 94.02) circle (  2.13);

\path[fill=fillColor,fill opacity=0.20] (210.89, 89.14) circle (  2.13);

\path[fill=fillColor,fill opacity=0.20] (212.86, 89.14) circle (  2.13);

\path[fill=fillColor,fill opacity=0.20] (216.79, 85.08) circle (  2.13);

\path[fill=fillColor,fill opacity=0.20] (224.65, 74.51) circle (  2.13);

\path[fill=fillColor,fill opacity=0.20] (239.39, 68.82) circle (  2.13);

\path[fill=fillColor,fill opacity=0.20] (208.93, 72.07) circle (  2.13);

\path[fill=fillColor,fill opacity=0.20] (205.98, 79.39) circle (  2.13);

\path[fill=fillColor,fill opacity=0.20] (190.26, 85.89) circle (  2.13);

\path[fill=fillColor,fill opacity=0.20] (211.87, 72.07) circle (  2.13);

\path[fill=fillColor,fill opacity=0.20] (200.08, 73.70) circle (  2.13);

\path[fill=fillColor,fill opacity=0.20] (203.03, 86.70) circle (  2.13);

\path[fill=fillColor,fill opacity=0.20] (206.96,105.39) circle (  2.13);

\path[fill=fillColor,fill opacity=0.20] (206.96,107.02) circle (  2.13);

\path[fill=fillColor,fill opacity=0.20] (209.91, 98.89) circle (  2.13);

\path[fill=fillColor,fill opacity=0.20] (211.87, 93.20) circle (  2.13);

\path[fill=fillColor,fill opacity=0.20] (213.84, 88.33) circle (  2.13);

\path[fill=fillColor,fill opacity=0.20] (214.82, 85.08) circle (  2.13);

\path[fill=fillColor,fill opacity=0.20] (223.67, 81.01) circle (  2.13);

\path[fill=fillColor,fill opacity=0.20] (210.89, 72.07) circle (  2.13);

\path[fill=fillColor,fill opacity=0.20] (212.86, 75.32) circle (  2.13);

\path[fill=fillColor,fill opacity=0.20] (203.03, 84.26) circle (  2.13);

\path[fill=fillColor,fill opacity=0.20] (195.17, 84.26) circle (  2.13);

\path[fill=fillColor,fill opacity=0.20] (191.24, 85.08) circle (  2.13);

\path[fill=fillColor,fill opacity=0.20] (191.24, 89.14) circle (  2.13);

\path[fill=fillColor,fill opacity=0.20] (198.12, 94.02) circle (  2.13);

\path[fill=fillColor,fill opacity=0.20] (205.98, 99.70) circle (  2.13);

\path[fill=fillColor,fill opacity=0.20] (208.93, 68.82) circle (  2.13);

\path[fill=fillColor,fill opacity=0.20] (198.12, 66.38) circle (  2.13);

\path[fill=fillColor,fill opacity=0.20] (205.00, 81.82) circle (  2.13);

\path[fill=fillColor,fill opacity=0.20] (209.91,102.96) circle (  2.13);

\path[fill=fillColor,fill opacity=0.20] (209.91,102.14) circle (  2.13);

\path[fill=fillColor,fill opacity=0.20] (215.81, 92.39) circle (  2.13);

\path[fill=fillColor,fill opacity=0.20] (215.81, 89.95) circle (  2.13);

\path[fill=fillColor,fill opacity=0.20] (213.84, 87.51) circle (  2.13);

\path[fill=fillColor,fill opacity=0.20] (215.81, 83.45) circle (  2.13);

\path[fill=fillColor,fill opacity=0.20] (207.94, 75.32) circle (  2.13);

\path[fill=fillColor,fill opacity=0.20] (201.07, 85.89) circle (  2.13);

\path[fill=fillColor,fill opacity=0.20] (200.08, 89.95) circle (  2.13);

\path[fill=fillColor,fill opacity=0.20] (203.03, 87.51) circle (  2.13);

\path[fill=fillColor,fill opacity=0.20] (191.24, 85.08) circle (  2.13);

\path[fill=fillColor,fill opacity=0.20] (191.24, 85.89) circle (  2.13);

\path[fill=fillColor,fill opacity=0.20] (196.15, 92.39) circle (  2.13);

\path[fill=fillColor,fill opacity=0.20] (200.08, 98.08) circle (  2.13);

\path[fill=fillColor,fill opacity=0.20] (200.08, 97.27) circle (  2.13);

\path[fill=fillColor,fill opacity=0.20] (168.15, 95.64) circle (  2.13);

\path[fill=fillColor,fill opacity=0.20] (211.87, 72.89) circle (  2.13);

\path[fill=fillColor,fill opacity=0.20] (200.08, 72.89) circle (  2.13);

\path[fill=fillColor,fill opacity=0.20] (207.94, 81.82) circle (  2.13);

\path[fill=fillColor,fill opacity=0.20] (210.89,100.52) circle (  2.13);

\path[fill=fillColor,fill opacity=0.20] (207.94,103.77) circle (  2.13);

\path[fill=fillColor,fill opacity=0.20] (211.87, 92.39) circle (  2.13);

\path[fill=fillColor,fill opacity=0.20] (216.79, 85.08) circle (  2.13);

\path[fill=fillColor,fill opacity=0.20] (213.84, 85.08) circle (  2.13);

\path[fill=fillColor,fill opacity=0.20] (214.82, 85.89) circle (  2.13);

\path[fill=fillColor,fill opacity=0.20] (207.94, 87.51) circle (  2.13);

\path[fill=fillColor,fill opacity=0.20] (202.05, 63.95) circle (  2.13);

\path[fill=fillColor,fill opacity=0.20] (204.01, 81.01) circle (  2.13);

\path[fill=fillColor,fill opacity=0.20] (201.07, 88.33) circle (  2.13);

\path[fill=fillColor,fill opacity=0.20] (193.20, 85.08) circle (  2.13);

\path[fill=fillColor,fill opacity=0.20] (191.24, 84.26) circle (  2.13);

\path[fill=fillColor,fill opacity=0.20] (191.24, 88.33) circle (  2.13);

\path[fill=fillColor,fill opacity=0.20] (194.19, 93.20) circle (  2.13);

\path[fill=fillColor,fill opacity=0.20] (197.13, 90.76) circle (  2.13);

\path[fill=fillColor,fill opacity=0.20] (200.08, 85.08) circle (  2.13);

\path[fill=fillColor,fill opacity=0.20] (201.07, 93.20) circle (  2.13);

\path[fill=fillColor,fill opacity=0.20] (195.17,101.33) circle (  2.13);

\path[fill=fillColor,fill opacity=0.20] (205.00, 75.32) circle (  2.13);

\path[fill=fillColor,fill opacity=0.20] (205.00, 78.57) circle (  2.13);

\path[fill=fillColor,fill opacity=0.20] (207.94, 96.45) circle (  2.13);

\path[fill=fillColor,fill opacity=0.20] (208.93,107.83) circle (  2.13);

\path[fill=fillColor,fill opacity=0.20] (205.98, 97.27) circle (  2.13);

\path[fill=fillColor,fill opacity=0.20] (210.89, 80.20) circle (  2.13);

\path[fill=fillColor,fill opacity=0.20] (216.79, 78.57) circle (  2.13);

\path[fill=fillColor,fill opacity=0.20] (218.75, 84.26) circle (  2.13);

\path[fill=fillColor,fill opacity=0.20] (219.74, 79.39) circle (  2.13);

\path[fill=fillColor,fill opacity=0.20] (224.65, 68.82) circle (  2.13);

\path[fill=fillColor,fill opacity=0.20] (204.01, 68.82) circle (  2.13);

\path[fill=fillColor,fill opacity=0.20] (204.01, 52.57) circle (  2.13);

\path[fill=fillColor,fill opacity=0.20] (201.07, 72.89) circle (  2.13);

\path[fill=fillColor,fill opacity=0.20] (201.07, 79.39) circle (  2.13);

\path[fill=fillColor,fill opacity=0.20] (197.13, 79.39) circle (  2.13);

\path[fill=fillColor,fill opacity=0.20] (197.13, 87.51) circle (  2.13);

\path[fill=fillColor,fill opacity=0.20] (194.19, 94.83) circle (  2.13);

\path[fill=fillColor,fill opacity=0.20] (191.24, 95.64) circle (  2.13);

\path[fill=fillColor,fill opacity=0.20] (192.22, 89.95) circle (  2.13);

\path[fill=fillColor,fill opacity=0.20] (195.17, 81.01) circle (  2.13);

\path[fill=fillColor,fill opacity=0.20] (190.26, 88.33) circle (  2.13);

\path[fill=fillColor,fill opacity=0.20] (203.03, 99.70) circle (  2.13);

\path[fill=fillColor,fill opacity=0.20] (193.20, 85.89) circle (  2.13);

\path[fill=fillColor,fill opacity=0.20] (204.01, 63.95) circle (  2.13);

\path[fill=fillColor,fill opacity=0.20] (200.08, 66.38) circle (  2.13);

\path[fill=fillColor,fill opacity=0.20] (205.98, 86.70) circle (  2.13);

\path[fill=fillColor,fill opacity=0.20] (209.91,101.33) circle (  2.13);

\path[fill=fillColor,fill opacity=0.20] (214.82, 92.39) circle (  2.13);

\path[fill=fillColor,fill opacity=0.20] (219.74, 77.76) circle (  2.13);

\path[fill=fillColor,fill opacity=0.20] (218.75, 76.14) circle (  2.13);

\path[fill=fillColor,fill opacity=0.20] (217.77, 77.76) circle (  2.13);

\path[fill=fillColor,fill opacity=0.20] (220.72, 72.07) circle (  2.13);

\path[fill=fillColor,fill opacity=0.20] (222.68, 64.76) circle (  2.13);

\path[fill=fillColor,fill opacity=0.20] (201.07, 67.20) circle (  2.13);

\path[fill=fillColor,fill opacity=0.20] (196.15, 59.07) circle (  2.13);

\path[fill=fillColor,fill opacity=0.20] (204.01, 75.32) circle (  2.13);

\path[fill=fillColor,fill opacity=0.20] (202.05, 76.95) circle (  2.13);

\path[fill=fillColor,fill opacity=0.20] (200.08, 82.64) circle (  2.13);

\path[fill=fillColor,fill opacity=0.20] (198.12, 94.02) circle (  2.13);

\path[fill=fillColor,fill opacity=0.20] (196.15, 96.45) circle (  2.13);

\path[fill=fillColor,fill opacity=0.20] (193.20, 93.20) circle (  2.13);

\path[fill=fillColor,fill opacity=0.20] (193.20, 93.20) circle (  2.13);

\path[fill=fillColor,fill opacity=0.20] (192.22, 87.51) circle (  2.13);

\path[fill=fillColor,fill opacity=0.20] (192.22, 85.08) circle (  2.13);

\path[fill=fillColor,fill opacity=0.20] (204.01, 91.58) circle (  2.13);

\path[fill=fillColor,fill opacity=0.20] (211.87, 90.76) circle (  2.13);

\path[fill=fillColor,fill opacity=0.20] (206.96, 57.44) circle (  2.13);

\path[fill=fillColor,fill opacity=0.20] (200.08, 58.26) circle (  2.13);

\path[fill=fillColor,fill opacity=0.20] (206.96, 76.14) circle (  2.13);

\path[fill=fillColor,fill opacity=0.20] (212.86, 89.14) circle (  2.13);

\path[fill=fillColor,fill opacity=0.20] (214.82, 83.45) circle (  2.13);

\path[fill=fillColor,fill opacity=0.20] (217.77, 77.76) circle (  2.13);

\path[fill=fillColor,fill opacity=0.20] (214.82, 80.20) circle (  2.13);

\path[fill=fillColor,fill opacity=0.20] (215.81, 79.39) circle (  2.13);

\path[fill=fillColor,fill opacity=0.20] (215.81, 72.89) circle (  2.13);

\path[fill=fillColor,fill opacity=0.20] (218.75, 63.13) circle (  2.13);

\path[fill=fillColor,fill opacity=0.20] (224.65, 59.88) circle (  2.13);

\path[fill=fillColor,fill opacity=0.20] (202.05, 76.95) circle (  2.13);

\path[fill=fillColor,fill opacity=0.20] (195.17, 69.63) circle (  2.13);

\path[fill=fillColor,fill opacity=0.20] (200.08, 80.20) circle (  2.13);

\path[fill=fillColor,fill opacity=0.20] (201.07, 83.45) circle (  2.13);

\path[fill=fillColor,fill opacity=0.20] (202.05, 94.83) circle (  2.13);

\path[fill=fillColor,fill opacity=0.20] (197.13, 96.45) circle (  2.13);

\path[fill=fillColor,fill opacity=0.20] (196.15, 90.76) circle (  2.13);

\path[fill=fillColor,fill opacity=0.20] (195.17, 94.83) circle (  2.13);

\path[fill=fillColor,fill opacity=0.20] (194.19, 98.08) circle (  2.13);

\path[fill=fillColor,fill opacity=0.20] (192.22, 89.14) circle (  2.13);

\path[fill=fillColor,fill opacity=0.20] (194.19, 81.01) circle (  2.13);

\path[fill=fillColor,fill opacity=0.20] (200.08, 82.64) circle (  2.13);

\path[fill=fillColor,fill opacity=0.20] (209.91, 88.33) circle (  2.13);

\path[fill=fillColor,fill opacity=0.20] (226.62, 65.57) circle (  2.13);

\path[fill=fillColor,fill opacity=0.20] (208.93, 59.07) circle (  2.13);

\path[fill=fillColor,fill opacity=0.20] (208.93, 66.38) circle (  2.13);

\path[fill=fillColor,fill opacity=0.20] (211.87, 80.20) circle (  2.13);

\path[fill=fillColor,fill opacity=0.20] (212.86, 82.64) circle (  2.13);

\path[fill=fillColor,fill opacity=0.20] (211.87, 79.39) circle (  2.13);

\path[fill=fillColor,fill opacity=0.20] (213.84, 81.82) circle (  2.13);

\path[fill=fillColor,fill opacity=0.20] (213.84, 85.89) circle (  2.13);

\path[fill=fillColor,fill opacity=0.20] (216.79, 82.64) circle (  2.13);

\path[fill=fillColor,fill opacity=0.20] (216.79, 68.01) circle (  2.13);

\path[fill=fillColor,fill opacity=0.20] (217.77, 57.44) circle (  2.13);

\path[fill=fillColor,fill opacity=0.20] (225.63, 62.32) circle (  2.13);

\path[fill=fillColor,fill opacity=0.20] (229.56, 74.51) circle (  2.13);

\path[fill=fillColor,fill opacity=0.20] (204.01, 75.32) circle (  2.13);

\path[fill=fillColor,fill opacity=0.20] (199.10, 81.82) circle (  2.13);

\path[fill=fillColor,fill opacity=0.20] (199.10, 87.51) circle (  2.13);

\path[fill=fillColor,fill opacity=0.20] (196.15, 98.89) circle (  2.13);

\path[fill=fillColor,fill opacity=0.20] (192.22, 94.83) circle (  2.13);

\path[fill=fillColor,fill opacity=0.20] (192.22, 85.89) circle (  2.13);

\path[fill=fillColor,fill opacity=0.20] (192.22, 98.08) circle (  2.13);

\path[fill=fillColor,fill opacity=0.20] (193.20,101.33) circle (  2.13);

\path[fill=fillColor,fill opacity=0.20] (194.19, 86.70) circle (  2.13);

\path[fill=fillColor,fill opacity=0.20] (198.12, 81.01) circle (  2.13);

\path[fill=fillColor,fill opacity=0.20] (205.00, 81.82) circle (  2.13);

\path[fill=fillColor,fill opacity=0.20] (206.96, 81.01) circle (  2.13);

\path[fill=fillColor,fill opacity=0.20] (212.86, 98.08) circle (  2.13);

\path[fill=fillColor,fill opacity=0.20] (229.56, 62.32) circle (  2.13);

\path[fill=fillColor,fill opacity=0.20] (210.89, 59.88) circle (  2.13);

\path[fill=fillColor,fill opacity=0.20] (210.89, 67.20) circle (  2.13);

\path[fill=fillColor,fill opacity=0.20] (208.93, 76.95) circle (  2.13);

\path[fill=fillColor,fill opacity=0.20] (210.89, 80.20) circle (  2.13);

\path[fill=fillColor,fill opacity=0.20] (210.89, 80.20) circle (  2.13);

\path[fill=fillColor,fill opacity=0.20] (212.86, 84.26) circle (  2.13);

\path[fill=fillColor,fill opacity=0.20] (212.86, 85.08) circle (  2.13);

\path[fill=fillColor,fill opacity=0.20] (211.87, 75.32) circle (  2.13);

\path[fill=fillColor,fill opacity=0.20] (214.82, 66.38) circle (  2.13);

\path[fill=fillColor,fill opacity=0.20] (223.67, 67.20) circle (  2.13);

\path[fill=fillColor,fill opacity=0.20] (220.72, 70.45) circle (  2.13);

\path[fill=fillColor,fill opacity=0.20] (224.65, 70.45) circle (  2.13);

\path[fill=fillColor,fill opacity=0.20] (210.89, 83.45) circle (  2.13);

\path[fill=fillColor,fill opacity=0.20] (203.03, 76.95) circle (  2.13);

\path[fill=fillColor,fill opacity=0.20] (202.05, 81.82) circle (  2.13);

\path[fill=fillColor,fill opacity=0.20] (196.15, 84.26) circle (  2.13);

\path[fill=fillColor,fill opacity=0.20] (193.20, 91.58) circle (  2.13);

\path[fill=fillColor,fill opacity=0.20] (193.20, 92.39) circle (  2.13);

\path[fill=fillColor,fill opacity=0.20] (193.20, 88.33) circle (  2.13);

\path[fill=fillColor,fill opacity=0.20] (191.24, 96.45) circle (  2.13);

\path[fill=fillColor,fill opacity=0.20] (196.15, 97.27) circle (  2.13);

\path[fill=fillColor,fill opacity=0.20] (198.12, 87.51) circle (  2.13);

\path[fill=fillColor,fill opacity=0.20] (196.15, 86.70) circle (  2.13);

\path[fill=fillColor,fill opacity=0.20] (200.08, 84.26) circle (  2.13);

\path[fill=fillColor,fill opacity=0.20] (208.93, 76.95) circle (  2.13);

\path[fill=fillColor,fill opacity=0.20] (186.62, 99.70) circle (  2.13);

\path[fill=fillColor,fill opacity=0.20] (219.74, 55.82) circle (  2.13);

\path[fill=fillColor,fill opacity=0.20] (202.05, 49.32) circle (  2.13);

\path[fill=fillColor,fill opacity=0.20] (211.87, 59.07) circle (  2.13);

\path[fill=fillColor,fill opacity=0.20] (212.86, 72.89) circle (  2.13);

\path[fill=fillColor,fill opacity=0.20] (209.91, 76.95) circle (  2.13);

\path[fill=fillColor,fill opacity=0.20] (214.82, 76.95) circle (  2.13);

\path[fill=fillColor,fill opacity=0.20] (195.17, 78.57) circle (  2.13);

\path[fill=fillColor,fill opacity=0.20] (211.87, 78.57) circle (  2.13);

\path[fill=fillColor,fill opacity=0.20] (217.77, 76.95) circle (  2.13);

\path[fill=fillColor,fill opacity=0.20] (219.74, 75.32) circle (  2.13);

\path[fill=fillColor,fill opacity=0.20] (218.75, 72.89) circle (  2.13);

\path[fill=fillColor,fill opacity=0.20] (220.72, 72.89) circle (  2.13);

\path[fill=fillColor,fill opacity=0.20] (222.68, 65.57) circle (  2.13);

\path[fill=fillColor,fill opacity=0.20] (225.63, 59.07) circle (  2.13);

\path[fill=fillColor,fill opacity=0.20] (206.96, 89.14) circle (  2.13);

\path[fill=fillColor,fill opacity=0.20] (202.05, 79.39) circle (  2.13);

\path[fill=fillColor,fill opacity=0.20] (202.05, 76.95) circle (  2.13);

\path[fill=fillColor,fill opacity=0.20] (202.05, 82.64) circle (  2.13);

\path[fill=fillColor,fill opacity=0.20] (202.05, 82.64) circle (  2.13);

\path[fill=fillColor,fill opacity=0.20] (198.12, 80.20) circle (  2.13);

\path[fill=fillColor,fill opacity=0.20] (191.24, 82.64) circle (  2.13);

\path[fill=fillColor,fill opacity=0.20] (194.19, 89.95) circle (  2.13);

\path[fill=fillColor,fill opacity=0.20] (193.20, 92.39) circle (  2.13);

\path[fill=fillColor,fill opacity=0.20] (195.17, 86.70) circle (  2.13);

\path[fill=fillColor,fill opacity=0.20] (200.08, 89.95) circle (  2.13);

\path[fill=fillColor,fill opacity=0.20] (195.17, 94.02) circle (  2.13);

\path[fill=fillColor,fill opacity=0.20] (194.19, 84.26) circle (  2.13);

\path[fill=fillColor,fill opacity=0.20] (210.89, 79.39) circle (  2.13);

\path[fill=fillColor,fill opacity=0.20] (240.37, 46.88) circle (  2.13);

\path[fill=fillColor,fill opacity=0.20] (225.63, 50.13) circle (  2.13);

\path[fill=fillColor,fill opacity=0.20] (216.79, 59.88) circle (  2.13);

\path[fill=fillColor,fill opacity=0.20] (207.94, 65.57) circle (  2.13);

\path[fill=fillColor,fill opacity=0.20] (212.86, 69.63) circle (  2.13);

\path[fill=fillColor,fill opacity=0.20] (215.81, 72.89) circle (  2.13);

\path[fill=fillColor,fill opacity=0.20] (206.96, 75.32) circle (  2.13);

\path[fill=fillColor,fill opacity=0.20] (215.81, 81.01) circle (  2.13);

\path[fill=fillColor,fill opacity=0.20] (215.81, 80.20) circle (  2.13);

\path[fill=fillColor,fill opacity=0.20] (218.75, 74.51) circle (  2.13);

\path[fill=fillColor,fill opacity=0.20] (221.70, 76.14) circle (  2.13);

\path[fill=fillColor,fill opacity=0.20] (216.79, 70.45) circle (  2.13);

\path[fill=fillColor,fill opacity=0.20] (217.77, 58.26) circle (  2.13);

\path[fill=fillColor,fill opacity=0.20] (219.74, 59.88) circle (  2.13);

\path[fill=fillColor,fill opacity=0.20] (216.79, 70.45) circle (  2.13);

\path[fill=fillColor,fill opacity=0.20] (211.87, 67.20) circle (  2.13);

\path[fill=fillColor,fill opacity=0.20] (210.89, 72.07) circle (  2.13);

\path[fill=fillColor,fill opacity=0.20] (208.93, 75.32) circle (  2.13);

\path[fill=fillColor,fill opacity=0.20] (205.00, 80.20) circle (  2.13);

\path[fill=fillColor,fill opacity=0.20] (201.07, 82.64) circle (  2.13);

\path[fill=fillColor,fill opacity=0.20] (198.12, 80.20) circle (  2.13);

\path[fill=fillColor,fill opacity=0.20] (198.12, 80.20) circle (  2.13);

\path[fill=fillColor,fill opacity=0.20] (200.08, 82.64) circle (  2.13);

\path[fill=fillColor,fill opacity=0.20] (204.01, 83.45) circle (  2.13);

\path[fill=fillColor,fill opacity=0.20] (198.12, 78.57) circle (  2.13);

\path[fill=fillColor,fill opacity=0.20] (198.12, 76.95) circle (  2.13);

\path[fill=fillColor,fill opacity=0.20] (191.24, 83.45) circle (  2.13);

\path[fill=fillColor,fill opacity=0.20] (195.17, 87.51) circle (  2.13);

\path[fill=fillColor,fill opacity=0.20] (196.15, 85.89) circle (  2.13);

\path[fill=fillColor,fill opacity=0.20] (196.15, 91.58) circle (  2.13);

\path[fill=fillColor,fill opacity=0.20] (194.19, 93.20) circle (  2.13);

\path[fill=fillColor,fill opacity=0.20] (204.01, 82.64) circle (  2.13);

\path[fill=fillColor,fill opacity=0.20] (209.91, 84.26) circle (  2.13);

\path[fill=fillColor,fill opacity=0.20] (170.31,101.33) circle (  2.13);

\path[fill=fillColor,fill opacity=0.20] (234.48, 60.69) circle (  2.13);

\path[fill=fillColor,fill opacity=0.20] (225.63, 58.26) circle (  2.13);

\path[fill=fillColor,fill opacity=0.20] (217.77, 55.82) circle (  2.13);

\path[fill=fillColor,fill opacity=0.20] (219.74, 60.69) circle (  2.13);

\path[fill=fillColor,fill opacity=0.20] (210.89, 62.32) circle (  2.13);

\path[fill=fillColor,fill opacity=0.20] (210.89, 65.57) circle (  2.13);

\path[fill=fillColor,fill opacity=0.20] (211.87, 77.76) circle (  2.13);

\path[fill=fillColor,fill opacity=0.20] (211.87, 84.26) circle (  2.13);

\path[fill=fillColor,fill opacity=0.20] (218.75, 77.76) circle (  2.13);

\path[fill=fillColor,fill opacity=0.20] (219.74, 76.14) circle (  2.13);

\path[fill=fillColor,fill opacity=0.20] (213.84, 75.32) circle (  2.13);

\path[fill=fillColor,fill opacity=0.20] (214.82, 69.63) circle (  2.13);

\path[fill=fillColor,fill opacity=0.20] (214.82, 66.38) circle (  2.13);

\path[fill=fillColor,fill opacity=0.20] (218.75, 55.82) circle (  2.13);

\path[fill=fillColor,fill opacity=0.20] (226.62, 46.88) circle (  2.13);

\path[fill=fillColor,fill opacity=0.20] (227.60, 52.57) circle (  2.13);

\path[fill=fillColor,fill opacity=0.20] (216.79, 63.13) circle (  2.13);

\path[fill=fillColor,fill opacity=0.20] (217.77, 64.76) circle (  2.13);

\path[fill=fillColor,fill opacity=0.20] (216.79, 69.63) circle (  2.13);

\path[fill=fillColor,fill opacity=0.20] (199.10, 70.45) circle (  2.13);

\path[fill=fillColor,fill opacity=0.20] (202.05, 75.32) circle (  2.13);

\path[fill=fillColor,fill opacity=0.20] (205.00, 79.39) circle (  2.13);

\path[fill=fillColor,fill opacity=0.20] (205.00, 78.57) circle (  2.13);

\path[fill=fillColor,fill opacity=0.20] (201.07, 84.26) circle (  2.13);

\path[fill=fillColor,fill opacity=0.20] (201.07, 86.70) circle (  2.13);

\path[fill=fillColor,fill opacity=0.20] (200.08, 81.82) circle (  2.13);

\path[fill=fillColor,fill opacity=0.20] (201.07, 79.39) circle (  2.13);

\path[fill=fillColor,fill opacity=0.20] (200.08, 82.64) circle (  2.13);

\path[fill=fillColor,fill opacity=0.20] (200.08, 84.26) circle (  2.13);

\path[fill=fillColor,fill opacity=0.20] (201.07, 85.08) circle (  2.13);

\path[fill=fillColor,fill opacity=0.20] (198.12, 83.45) circle (  2.13);

\path[fill=fillColor,fill opacity=0.20] (196.15, 81.01) circle (  2.13);

\path[fill=fillColor,fill opacity=0.20] (194.19, 85.08) circle (  2.13);

\path[fill=fillColor,fill opacity=0.20] (198.12, 93.20) circle (  2.13);

\path[fill=fillColor,fill opacity=0.20] (198.12, 92.39) circle (  2.13);

\path[fill=fillColor,fill opacity=0.20] (195.17, 83.45) circle (  2.13);

\path[fill=fillColor,fill opacity=0.20] (197.13, 78.57) circle (  2.13);

\path[fill=fillColor,fill opacity=0.20] (214.82, 82.64) circle (  2.13);

\path[fill=fillColor,fill opacity=0.20] (209.91, 93.20) circle (  2.13);

\path[fill=fillColor,fill opacity=0.20] (184.65, 66.38) circle (  2.13);

\path[fill=fillColor,fill opacity=0.20] (230.55, 61.51) circle (  2.13);

\path[fill=fillColor,fill opacity=0.20] (224.65, 57.44) circle (  2.13);

\path[fill=fillColor,fill opacity=0.20] (208.93, 59.07) circle (  2.13);

\path[fill=fillColor,fill opacity=0.20] (210.89, 63.95) circle (  2.13);

\path[fill=fillColor,fill opacity=0.20] (210.89, 71.26) circle (  2.13);

\path[fill=fillColor,fill opacity=0.20] (211.87, 72.89) circle (  2.13);

\path[fill=fillColor,fill opacity=0.20] (210.89, 71.26) circle (  2.13);

\path[fill=fillColor,fill opacity=0.20] (213.84, 72.07) circle (  2.13);

\path[fill=fillColor,fill opacity=0.20] (211.87, 72.07) circle (  2.13);

\path[fill=fillColor,fill opacity=0.20] (215.81, 67.20) circle (  2.13);

\path[fill=fillColor,fill opacity=0.20] (219.74, 59.07) circle (  2.13);

\path[fill=fillColor,fill opacity=0.20] (221.70, 59.88) circle (  2.13);

\path[fill=fillColor,fill opacity=0.20] (221.70, 63.95) circle (  2.13);

\path[fill=fillColor,fill opacity=0.20] (219.74, 56.63) circle (  2.13);

\path[fill=fillColor,fill opacity=0.20] (220.72, 54.19) circle (  2.13);

\path[fill=fillColor,fill opacity=0.20] (213.84, 75.32) circle (  2.13);

\path[fill=fillColor,fill opacity=0.20] (213.84, 79.39) circle (  2.13);

\path[fill=fillColor,fill opacity=0.20] (212.86, 76.95) circle (  2.13);

\path[fill=fillColor,fill opacity=0.20] (202.05, 71.26) circle (  2.13);

\path[fill=fillColor,fill opacity=0.20] (202.05, 73.70) circle (  2.13);

\path[fill=fillColor,fill opacity=0.20] (207.94, 79.39) circle (  2.13);

\path[fill=fillColor,fill opacity=0.20] (207.94, 79.39) circle (  2.13);

\path[fill=fillColor,fill opacity=0.20] (198.12, 82.64) circle (  2.13);

\path[fill=fillColor,fill opacity=0.20] (195.17, 85.08) circle (  2.13);

\path[fill=fillColor,fill opacity=0.20] (202.05, 76.95) circle (  2.13);

\path[fill=fillColor,fill opacity=0.20] (200.08, 74.51) circle (  2.13);

\path[fill=fillColor,fill opacity=0.20] (199.10, 82.64) circle (  2.13);

\path[fill=fillColor,fill opacity=0.20] (201.07, 85.89) circle (  2.13);

\path[fill=fillColor,fill opacity=0.20] (200.08, 87.51) circle (  2.13);

\path[fill=fillColor,fill opacity=0.20] (201.07, 89.14) circle (  2.13);

\path[fill=fillColor,fill opacity=0.20] (201.07, 84.26) circle (  2.13);

\path[fill=fillColor,fill opacity=0.20] (201.07, 85.08) circle (  2.13);

\path[fill=fillColor,fill opacity=0.20] (204.01, 96.45) circle (  2.13);

\path[fill=fillColor,fill opacity=0.20] (203.03, 92.39) circle (  2.13);

\path[fill=fillColor,fill opacity=0.20] (196.15, 76.95) circle (  2.13);

\path[fill=fillColor,fill opacity=0.20] (199.10, 69.63) circle (  2.13);

\path[fill=fillColor,fill opacity=0.20] (217.77, 69.63) circle (  2.13);

\path[fill=fillColor,fill opacity=0.20] (224.65, 82.64) circle (  2.13);

\path[fill=fillColor,fill opacity=0.20] (244.30, 69.63) circle (  2.13);

\path[fill=fillColor,fill opacity=0.20] (233.49, 61.51) circle (  2.13);

\path[fill=fillColor,fill opacity=0.20] (220.72, 49.32) circle (  2.13);

\path[fill=fillColor,fill opacity=0.20] (220.72, 49.32) circle (  2.13);

\path[fill=fillColor,fill opacity=0.20] (218.75, 55.82) circle (  2.13);

\path[fill=fillColor,fill opacity=0.20] (209.91, 55.82) circle (  2.13);

\path[fill=fillColor,fill opacity=0.20] (213.84, 62.32) circle (  2.13);

\path[fill=fillColor,fill opacity=0.20] (215.81, 71.26) circle (  2.13);

\path[fill=fillColor,fill opacity=0.20] (207.94, 65.57) circle (  2.13);

\path[fill=fillColor,fill opacity=0.20] (217.77, 63.13) circle (  2.13);

\path[fill=fillColor,fill opacity=0.20] (215.81, 77.76) circle (  2.13);

\path[fill=fillColor,fill opacity=0.20] (215.81, 76.14) circle (  2.13);

\path[fill=fillColor,fill opacity=0.20] (215.81, 55.82) circle (  2.13);

\path[fill=fillColor,fill opacity=0.20] (215.81, 50.94) circle (  2.13);

\path[fill=fillColor,fill opacity=0.20] (219.74, 67.20) circle (  2.13);

\path[fill=fillColor,fill opacity=0.20] (211.87, 63.95) circle (  2.13);

\path[fill=fillColor,fill opacity=0.20] (208.93, 51.75) circle (  2.13);

\path[fill=fillColor,fill opacity=0.20] (210.89, 42.82) circle (  2.13);

\path[fill=fillColor,fill opacity=0.20] (216.79, 55.01) circle (  2.13);

\path[fill=fillColor,fill opacity=0.20] (216.79, 69.63) circle (  2.13);

\path[fill=fillColor,fill opacity=0.20] (208.93, 75.32) circle (  2.13);

\path[fill=fillColor,fill opacity=0.20] (213.84, 73.70) circle (  2.13);

\path[fill=fillColor,fill opacity=0.20] (211.87, 66.38) circle (  2.13);

\path[fill=fillColor,fill opacity=0.20] (209.91, 63.95) circle (  2.13);

\path[fill=fillColor,fill opacity=0.20] (209.91, 73.70) circle (  2.13);

\path[fill=fillColor,fill opacity=0.20] (207.94, 82.64) circle (  2.13);

\path[fill=fillColor,fill opacity=0.20] (202.05, 82.64) circle (  2.13);

\path[fill=fillColor,fill opacity=0.20] (202.05, 79.39) circle (  2.13);

\path[fill=fillColor,fill opacity=0.20] (204.01, 73.70) circle (  2.13);

\path[fill=fillColor,fill opacity=0.20] (195.17, 73.70) circle (  2.13);

\path[fill=fillColor,fill opacity=0.20] (203.03, 81.01) circle (  2.13);

\path[fill=fillColor,fill opacity=0.20] (202.05, 84.26) circle (  2.13);

\path[fill=fillColor,fill opacity=0.20] (204.01, 85.89) circle (  2.13);

\path[fill=fillColor,fill opacity=0.20] (205.00, 85.89) circle (  2.13);

\path[fill=fillColor,fill opacity=0.20] (205.00, 82.64) circle (  2.13);

\path[fill=fillColor,fill opacity=0.20] (202.05, 82.64) circle (  2.13);

\path[fill=fillColor,fill opacity=0.20] (201.07, 89.95) circle (  2.13);

\path[fill=fillColor,fill opacity=0.20] (199.10, 87.51) circle (  2.13);

\path[fill=fillColor,fill opacity=0.20] (204.01, 71.26) circle (  2.13);

\path[fill=fillColor,fill opacity=0.20] (206.96, 58.26) circle (  2.13);

\path[fill=fillColor,fill opacity=0.20] (220.72, 61.51) circle (  2.13);

\path[fill=fillColor,fill opacity=0.20] (244.30, 52.57) circle (  2.13);

\path[fill=fillColor,fill opacity=0.20] (233.49, 51.75) circle (  2.13);

\path[fill=fillColor,fill opacity=0.20] (234.48, 49.32) circle (  2.13);

\path[fill=fillColor,fill opacity=0.20] (214.82, 45.25) circle (  2.13);

\path[fill=fillColor,fill opacity=0.20] (220.72, 55.01) circle (  2.13);

\path[fill=fillColor,fill opacity=0.20] (218.75, 62.32) circle (  2.13);

\path[fill=fillColor,fill opacity=0.20] (215.81, 55.82) circle (  2.13);

\path[fill=fillColor,fill opacity=0.20] (210.89, 59.88) circle (  2.13);

\path[fill=fillColor,fill opacity=0.20] (213.84, 76.95) circle (  2.13);

\path[fill=fillColor,fill opacity=0.20] (216.79, 73.70) circle (  2.13);

\path[fill=fillColor,fill opacity=0.20] (216.79, 58.26) circle (  2.13);

\path[fill=fillColor,fill opacity=0.20] (209.91, 59.88) circle (  2.13);

\path[fill=fillColor,fill opacity=0.20] (214.82, 72.07) circle (  2.13);

\path[fill=fillColor,fill opacity=0.20] (218.75, 72.07) circle (  2.13);

\path[fill=fillColor,fill opacity=0.20] (213.84, 60.69) circle (  2.13);

\path[fill=fillColor,fill opacity=0.20] (217.77, 52.57) circle (  2.13);

\path[fill=fillColor,fill opacity=0.20] (220.72, 57.44) circle (  2.13);

\path[fill=fillColor,fill opacity=0.20] (212.86, 51.75) circle (  2.13);

\path[fill=fillColor,fill opacity=0.20] (214.82, 42.00) circle (  2.13);

\path[fill=fillColor,fill opacity=0.20] (217.77, 50.13) circle (  2.13);

\path[fill=fillColor,fill opacity=0.20] (225.63, 68.82) circle (  2.13);

\path[fill=fillColor,fill opacity=0.20] (221.70, 76.95) circle (  2.13);

\path[fill=fillColor,fill opacity=0.20] (217.77, 74.51) circle (  2.13);

\path[fill=fillColor,fill opacity=0.20] (216.79, 64.76) circle (  2.13);

\path[fill=fillColor,fill opacity=0.20] (209.91, 50.94) circle (  2.13);

\path[fill=fillColor,fill opacity=0.20] (210.89, 47.69) circle (  2.13);

\path[fill=fillColor,fill opacity=0.20] (213.84, 55.01) circle (  2.13);

\path[fill=fillColor,fill opacity=0.20] (211.87, 59.07) circle (  2.13);

\path[fill=fillColor,fill opacity=0.20] (206.96, 63.95) circle (  2.13);

\path[fill=fillColor,fill opacity=0.20] (206.96, 72.89) circle (  2.13);

\path[fill=fillColor,fill opacity=0.20] (201.07, 76.14) circle (  2.13);

\path[fill=fillColor,fill opacity=0.20] (205.98, 75.32) circle (  2.13);

\path[fill=fillColor,fill opacity=0.20] (203.03, 75.32) circle (  2.13);

\path[fill=fillColor,fill opacity=0.20] (202.05, 76.14) circle (  2.13);

\path[fill=fillColor,fill opacity=0.20] (202.05, 76.95) circle (  2.13);

\path[fill=fillColor,fill opacity=0.20] (206.96, 78.57) circle (  2.13);

\path[fill=fillColor,fill opacity=0.20] (205.98, 81.01) circle (  2.13);

\path[fill=fillColor,fill opacity=0.20] (206.96, 80.20) circle (  2.13);

\path[fill=fillColor,fill opacity=0.20] (207.94, 76.95) circle (  2.13);

\path[fill=fillColor,fill opacity=0.20] (205.98, 74.51) circle (  2.13);

\path[fill=fillColor,fill opacity=0.20] (198.12, 76.95) circle (  2.13);

\path[fill=fillColor,fill opacity=0.20] (195.17, 73.70) circle (  2.13);

\path[fill=fillColor,fill opacity=0.20] (209.91, 59.88) circle (  2.13);

\path[fill=fillColor,fill opacity=0.20] (218.75, 55.01) circle (  2.13);

\path[fill=fillColor,fill opacity=0.20] (218.75, 74.51) circle (  2.13);

\path[fill=fillColor,fill opacity=0.20] (260.03, 46.07) circle (  2.13);

\path[fill=fillColor,fill opacity=0.20] (230.55, 50.94) circle (  2.13);

\path[fill=fillColor,fill opacity=0.20] (227.60, 51.75) circle (  2.13);

\path[fill=fillColor,fill opacity=0.20] (221.70, 48.50) circle (  2.13);

\path[fill=fillColor,fill opacity=0.20] (210.89, 51.75) circle (  2.13);

\path[fill=fillColor,fill opacity=0.20] (211.87, 56.63) circle (  2.13);

\path[fill=fillColor,fill opacity=0.20] (218.75, 56.63) circle (  2.13);

\path[fill=fillColor,fill opacity=0.20] (217.77, 55.82) circle (  2.13);

\path[fill=fillColor,fill opacity=0.20] (214.82, 56.63) circle (  2.13);

\path[fill=fillColor,fill opacity=0.20] (215.81, 61.51) circle (  2.13);

\path[fill=fillColor,fill opacity=0.20] (217.77, 67.20) circle (  2.13);

\path[fill=fillColor,fill opacity=0.20] (206.96, 63.13) circle (  2.13);

\path[fill=fillColor,fill opacity=0.20] (223.67, 59.88) circle (  2.13);

\path[fill=fillColor,fill opacity=0.20] (225.63, 63.13) circle (  2.13);

\path[fill=fillColor,fill opacity=0.20] (222.68, 59.88) circle (  2.13);

\path[fill=fillColor,fill opacity=0.20] (217.77, 55.01) circle (  2.13);

\path[fill=fillColor,fill opacity=0.20] (217.77, 60.69) circle (  2.13);

\path[fill=fillColor,fill opacity=0.20] (212.86, 86.70) circle (  2.13);

\path[fill=fillColor,fill opacity=0.20] (217.77, 73.70) circle (  2.13);

\path[fill=fillColor,fill opacity=0.20] (222.68, 64.76) circle (  2.13);

\path[fill=fillColor,fill opacity=0.20] (209.91, 68.01) circle (  2.13);

\path[fill=fillColor,fill opacity=0.20] (219.74, 66.38) circle (  2.13);

\path[fill=fillColor,fill opacity=0.20] (214.82, 52.57) circle (  2.13);

\path[fill=fillColor,fill opacity=0.20] (219.74, 46.07) circle (  2.13);

\path[fill=fillColor,fill opacity=0.20] (212.86, 50.94) circle (  2.13);

\path[fill=fillColor,fill opacity=0.20] (212.86, 48.50) circle (  2.13);

\path[fill=fillColor,fill opacity=0.20] (210.89, 46.07) circle (  2.13);

\path[fill=fillColor,fill opacity=0.20] (218.75, 52.57) circle (  2.13);

\path[fill=fillColor,fill opacity=0.20] (227.60, 54.19) circle (  2.13);

\path[fill=fillColor,fill opacity=0.20] (234.48, 61.51) circle (  2.13);

\path[fill=fillColor,fill opacity=0.20] (239.39, 85.89) circle (  2.13);

\path[fill=fillColor,fill opacity=0.20] (218.75, 73.70) circle (  2.13);

\path[fill=fillColor,fill opacity=0.20] (213.84, 71.26) circle (  2.13);

\path[fill=fillColor,fill opacity=0.20] (212.86, 71.26) circle (  2.13);

\path[fill=fillColor,fill opacity=0.20] (208.93, 65.57) circle (  2.13);

\path[fill=fillColor,fill opacity=0.20] (197.13, 60.69) circle (  2.13);

\path[fill=fillColor,fill opacity=0.20] (207.94, 64.76) circle (  2.13);

\path[fill=fillColor,fill opacity=0.20] (202.05, 68.82) circle (  2.13);

\path[fill=fillColor,fill opacity=0.20] (203.03, 70.45) circle (  2.13);

\path[fill=fillColor,fill opacity=0.20] (205.00, 71.26) circle (  2.13);

\path[fill=fillColor,fill opacity=0.20] (205.98, 72.07) circle (  2.13);

\path[fill=fillColor,fill opacity=0.20] (205.98, 72.07) circle (  2.13);

\path[fill=fillColor,fill opacity=0.20] (204.01, 72.89) circle (  2.13);

\path[fill=fillColor,fill opacity=0.20] (204.01, 72.07) circle (  2.13);

\path[fill=fillColor,fill opacity=0.20] (199.10, 70.45) circle (  2.13);

\path[fill=fillColor,fill opacity=0.20] (205.98, 67.20) circle (  2.13);

\path[fill=fillColor,fill opacity=0.20] (202.05, 63.95) circle (  2.13);

\path[fill=fillColor,fill opacity=0.20] (209.91, 59.88) circle (  2.13);

\path[fill=fillColor,fill opacity=0.20] (222.68, 58.26) circle (  2.13);

\path[fill=fillColor,fill opacity=0.20] (231.53, 66.38) circle (  2.13);

\path[fill=fillColor,fill opacity=0.20] (223.67, 48.50) circle (  2.13);

\path[fill=fillColor,fill opacity=0.20] (218.75, 50.13) circle (  2.13);

\path[fill=fillColor,fill opacity=0.20] (222.68, 48.50) circle (  2.13);

\path[fill=fillColor,fill opacity=0.20] (223.67, 50.94) circle (  2.13);

\path[fill=fillColor,fill opacity=0.20] (217.77, 53.38) circle (  2.13);

\path[fill=fillColor,fill opacity=0.20] (210.89, 46.07) circle (  2.13);

\path[fill=fillColor,fill opacity=0.20] (219.74, 42.00) circle (  2.13);

\path[fill=fillColor,fill opacity=0.20] (218.75, 49.32) circle (  2.13);

\path[fill=fillColor,fill opacity=0.20] (219.74, 55.82) circle (  2.13);

\path[fill=fillColor,fill opacity=0.20] (219.74, 63.95) circle (  2.13);

\path[fill=fillColor,fill opacity=0.20] (213.84, 72.07) circle (  2.13);

\path[fill=fillColor,fill opacity=0.20] (215.81, 68.82) circle (  2.13);

\path[fill=fillColor,fill opacity=0.20] (225.63, 60.69) circle (  2.13);

\path[fill=fillColor,fill opacity=0.20] (215.81, 60.69) circle (  2.13);

\path[fill=fillColor,fill opacity=0.20] (219.74, 65.57) circle (  2.13);

\path[fill=fillColor,fill opacity=0.20] (223.67, 71.26) circle (  2.13);

\path[fill=fillColor,fill opacity=0.20] (225.63, 75.32) circle (  2.13);

\path[fill=fillColor,fill opacity=0.20] (225.63, 73.70) circle (  2.13);

\path[fill=fillColor,fill opacity=0.20] (222.68, 68.82) circle (  2.13);

\path[fill=fillColor,fill opacity=0.20] (221.70, 67.20) circle (  2.13);

\path[fill=fillColor,fill opacity=0.20] (219.74, 71.26) circle (  2.13);

\path[fill=fillColor,fill opacity=0.20] (219.74, 77.76) circle (  2.13);

\path[fill=fillColor,fill opacity=0.20] (221.70, 87.51) circle (  2.13);

\path[fill=fillColor,fill opacity=0.20] (221.70, 80.20) circle (  2.13);

\path[fill=fillColor,fill opacity=0.20] (218.75, 73.70) circle (  2.13);

\path[fill=fillColor,fill opacity=0.20] (211.87, 73.70) circle (  2.13);

\path[fill=fillColor,fill opacity=0.20] (215.81, 73.70) circle (  2.13);

\path[fill=fillColor,fill opacity=0.20] (211.87, 70.45) circle (  2.13);

\path[fill=fillColor,fill opacity=0.20] (210.89, 67.20) circle (  2.13);

\path[fill=fillColor,fill opacity=0.20] (216.79, 61.51) circle (  2.13);

\path[fill=fillColor,fill opacity=0.20] (213.84, 58.26) circle (  2.13);

\path[fill=fillColor,fill opacity=0.20] (211.87, 64.76) circle (  2.13);

\path[fill=fillColor,fill opacity=0.20] (212.86, 63.95) circle (  2.13);

\path[fill=fillColor,fill opacity=0.20] (212.86, 50.94) circle (  2.13);

\path[fill=fillColor,fill opacity=0.20] (216.79, 44.44) circle (  2.13);

\path[fill=fillColor,fill opacity=0.20] (215.81, 50.13) circle (  2.13);

\path[fill=fillColor,fill opacity=0.20] (207.94, 53.38) circle (  2.13);

\path[fill=fillColor,fill opacity=0.20] (219.74, 59.07) circle (  2.13);

\path[fill=fillColor,fill opacity=0.20] (220.72, 72.07) circle (  2.13);

\path[fill=fillColor,fill opacity=0.20] (220.72, 91.58) circle (  2.13);

\path[fill=fillColor,fill opacity=0.20] (218.75, 85.08) circle (  2.13);

\path[fill=fillColor,fill opacity=0.20] (207.94, 79.39) circle (  2.13);

\path[fill=fillColor,fill opacity=0.20] (202.05, 69.63) circle (  2.13);

\path[fill=fillColor,fill opacity=0.20] (204.01, 63.95) circle (  2.13);

\path[fill=fillColor,fill opacity=0.20] (208.93, 64.76) circle (  2.13);

\path[fill=fillColor,fill opacity=0.20] (207.94, 67.20) circle (  2.13);

\path[fill=fillColor,fill opacity=0.20] (203.03, 68.82) circle (  2.13);

\path[fill=fillColor,fill opacity=0.20] (205.00, 68.01) circle (  2.13);

\path[fill=fillColor,fill opacity=0.20] (208.93, 64.76) circle (  2.13);

\path[fill=fillColor,fill opacity=0.20] (208.93, 67.20) circle (  2.13);

\path[fill=fillColor,fill opacity=0.20] (209.91, 69.63) circle (  2.13);

\path[fill=fillColor,fill opacity=0.20] (213.84, 65.57) circle (  2.13);

\path[fill=fillColor,fill opacity=0.20] (216.79, 68.82) circle (  2.13);

\path[fill=fillColor,fill opacity=0.20] (230.55, 63.95) circle (  2.13);

\path[fill=fillColor,fill opacity=0.20] (225.63, 51.75) circle (  2.13);

\path[fill=fillColor,fill opacity=0.20] (223.67, 37.94) circle (  2.13);

\path[fill=fillColor,fill opacity=0.20] (217.77, 46.07) circle (  2.13);

\path[fill=fillColor,fill opacity=0.20] (214.82, 54.19) circle (  2.13);

\path[fill=fillColor,fill opacity=0.20] (214.82, 58.26) circle (  2.13);

\path[fill=fillColor,fill opacity=0.20] (222.68, 59.88) circle (  2.13);

\path[fill=fillColor,fill opacity=0.20] (215.81, 60.69) circle (  2.13);

\path[fill=fillColor,fill opacity=0.20] (220.72, 62.32) circle (  2.13);

\path[fill=fillColor,fill opacity=0.20] (218.75, 66.38) circle (  2.13);

\path[fill=fillColor,fill opacity=0.20] (223.67, 72.89) circle (  2.13);

\path[fill=fillColor,fill opacity=0.20] (221.70, 77.76) circle (  2.13);

\path[fill=fillColor,fill opacity=0.20] (217.77, 76.14) circle (  2.13);

\path[fill=fillColor,fill opacity=0.20] (217.77, 73.70) circle (  2.13);

\path[fill=fillColor,fill opacity=0.20] (215.81, 70.45) circle (  2.13);

\path[fill=fillColor,fill opacity=0.20] (220.72, 66.38) circle (  2.13);

\path[fill=fillColor,fill opacity=0.20] (220.72, 67.20) circle (  2.13);

\path[fill=fillColor,fill opacity=0.20] (215.81, 68.82) circle (  2.13);

\path[fill=fillColor,fill opacity=0.20] (217.77, 70.45) circle (  2.13);

\path[fill=fillColor,fill opacity=0.20] (214.82, 73.70) circle (  2.13);

\path[fill=fillColor,fill opacity=0.20] (213.84, 76.95) circle (  2.13);

\path[fill=fillColor,fill opacity=0.20] (199.10, 77.76) circle (  2.13);

\path[fill=fillColor,fill opacity=0.20] (209.91, 75.32) circle (  2.13);

\path[fill=fillColor,fill opacity=0.20] (213.84, 71.26) circle (  2.13);

\path[fill=fillColor,fill opacity=0.20] (209.91, 70.45) circle (  2.13);

\path[fill=fillColor,fill opacity=0.20] (208.93, 70.45) circle (  2.13);

\path[fill=fillColor,fill opacity=0.20] (208.93, 62.32) circle (  2.13);

\path[fill=fillColor,fill opacity=0.20] (208.93, 55.82) circle (  2.13);

\path[fill=fillColor,fill opacity=0.20] (211.87, 59.07) circle (  2.13);

\path[fill=fillColor,fill opacity=0.20] (211.87, 65.57) circle (  2.13);

\path[fill=fillColor,fill opacity=0.20] (217.77, 66.38) circle (  2.13);

\path[fill=fillColor,fill opacity=0.20] (227.60, 68.01) circle (  2.13);

\path[fill=fillColor,fill opacity=0.20] (232.51, 85.89) circle (  2.13);

\path[fill=fillColor,fill opacity=0.20] (213.84, 76.95) circle (  2.13);

\path[fill=fillColor,fill opacity=0.20] (215.81, 74.51) circle (  2.13);

\path[fill=fillColor,fill opacity=0.20] (213.84, 71.26) circle (  2.13);

\path[fill=fillColor,fill opacity=0.20] (207.94, 71.26) circle (  2.13);

\path[fill=fillColor,fill opacity=0.20] (206.96, 70.45) circle (  2.13);

\path[fill=fillColor,fill opacity=0.20] (208.93, 68.01) circle (  2.13);

\path[fill=fillColor,fill opacity=0.20] (211.87, 72.89) circle (  2.13);

\path[fill=fillColor,fill opacity=0.20] (233.49, 56.63) circle (  2.13);

\path[fill=fillColor,fill opacity=0.20] (221.70, 45.25) circle (  2.13);

\path[fill=fillColor,fill opacity=0.20] (218.75, 37.94) circle (  2.13);

\path[fill=fillColor,fill opacity=0.20] (215.81, 39.56) circle (  2.13);

\path[fill=fillColor,fill opacity=0.20] (210.89, 42.00) circle (  2.13);

\path[fill=fillColor,fill opacity=0.20] (213.84, 49.32) circle (  2.13);

\path[fill=fillColor,fill opacity=0.20] (216.79, 58.26) circle (  2.13);

\path[fill=fillColor,fill opacity=0.20] (220.72, 62.32) circle (  2.13);

\path[fill=fillColor,fill opacity=0.20] (218.75, 64.76) circle (  2.13);

\path[fill=fillColor,fill opacity=0.20] (215.81, 69.63) circle (  2.13);

\path[fill=fillColor,fill opacity=0.20] (211.87, 72.89) circle (  2.13);

\path[fill=fillColor,fill opacity=0.20] (208.93, 76.95) circle (  2.13);

\path[fill=fillColor,fill opacity=0.20] (215.81, 81.01) circle (  2.13);

\path[fill=fillColor,fill opacity=0.20] (218.75, 80.20) circle (  2.13);

\path[fill=fillColor,fill opacity=0.20] (215.81, 76.95) circle (  2.13);

\path[fill=fillColor,fill opacity=0.20] (213.84, 76.14) circle (  2.13);

\path[fill=fillColor,fill opacity=0.20] (210.89, 76.14) circle (  2.13);

\path[fill=fillColor,fill opacity=0.20] (212.86, 75.32) circle (  2.13);

\path[fill=fillColor,fill opacity=0.20] (205.98, 76.95) circle (  2.13);

\path[fill=fillColor,fill opacity=0.20] (203.03, 81.82) circle (  2.13);

\path[fill=fillColor,fill opacity=0.20] (205.98, 80.20) circle (  2.13);

\path[fill=fillColor,fill opacity=0.20] (206.96, 69.63) circle (  2.13);

\path[fill=fillColor,fill opacity=0.20] (205.98, 63.13) circle (  2.13);

\path[fill=fillColor,fill opacity=0.20] (205.98, 63.95) circle (  2.13);

\path[fill=fillColor,fill opacity=0.20] (207.94, 62.32) circle (  2.13);

\path[fill=fillColor,fill opacity=0.20] (216.79, 58.26) circle (  2.13);

\path[fill=fillColor,fill opacity=0.20] (218.75, 63.13) circle (  2.13);

\path[fill=fillColor,fill opacity=0.20] (221.70, 72.89) circle (  2.13);

\path[fill=fillColor,fill opacity=0.20] (233.49, 48.50) circle (  2.13);

\path[fill=fillColor,fill opacity=0.20] (208.93, 42.82) circle (  2.13);

\path[fill=fillColor,fill opacity=0.20] (217.77, 39.56) circle (  2.13);

\path[fill=fillColor,fill opacity=0.20] (219.74, 46.88) circle (  2.13);

\path[fill=fillColor,fill opacity=0.20] (220.72, 50.94) circle (  2.13);

\path[fill=fillColor,fill opacity=0.20] (216.79, 47.69) circle (  2.13);

\path[fill=fillColor,fill opacity=0.20] (214.82, 51.75) circle (  2.13);

\path[fill=fillColor,fill opacity=0.20] (211.87, 63.13) circle (  2.13);

\path[fill=fillColor,fill opacity=0.20] (212.86, 68.82) circle (  2.13);

\path[fill=fillColor,fill opacity=0.20] (208.93, 71.26) circle (  2.13);

\path[fill=fillColor,fill opacity=0.20] (210.89, 74.51) circle (  2.13);

\path[fill=fillColor,fill opacity=0.20] (210.89, 76.14) circle (  2.13);

\path[fill=fillColor,fill opacity=0.20] (205.98, 73.70) circle (  2.13);

\path[fill=fillColor,fill opacity=0.20] (206.96, 72.89) circle (  2.13);

\path[fill=fillColor,fill opacity=0.20] (207.94, 71.26) circle (  2.13);

\path[fill=fillColor,fill opacity=0.20] (201.07, 67.20) circle (  2.13);

\path[fill=fillColor,fill opacity=0.20] (200.08, 67.20) circle (  2.13);

\path[fill=fillColor,fill opacity=0.20] (209.91, 68.82) circle (  2.13);

\path[fill=fillColor,fill opacity=0.20] (210.89, 63.13) circle (  2.13);

\path[fill=fillColor,fill opacity=0.20] (209.91, 54.19) circle (  2.13);

\path[fill=fillColor,fill opacity=0.20] (209.91, 53.38) circle (  2.13);

\path[fill=fillColor,fill opacity=0.20] (208.93, 59.07) circle (  2.13);

\path[fill=fillColor,fill opacity=0.20] (223.67, 46.88) circle (  2.13);

\path[fill=fillColor,fill opacity=0.20] (221.70, 46.88) circle (  2.13);

\path[fill=fillColor,fill opacity=0.20] (213.84, 47.69) circle (  2.13);

\path[fill=fillColor,fill opacity=0.20] (198.12, 55.82) circle (  2.13);

\path[fill=fillColor,fill opacity=0.20] (219.74, 56.63) circle (  2.13);

\path[fill=fillColor,fill opacity=0.20] (212.86, 47.69) circle (  2.13);

\path[fill=fillColor,fill opacity=0.20] (206.96, 47.69) circle (  2.13);

\path[fill=fillColor,fill opacity=0.20] (208.93, 56.63) circle (  2.13);

\path[fill=fillColor,fill opacity=0.20] (206.96, 58.26) circle (  2.13);

\path[fill=fillColor,fill opacity=0.20] (206.96, 59.07) circle (  2.13);

\path[fill=fillColor,fill opacity=0.20] (208.93, 63.13) circle (  2.13);

\path[fill=fillColor,fill opacity=0.20] (199.10, 63.13) circle (  2.13);

\path[fill=fillColor,fill opacity=0.20] (212.86, 57.44) circle (  2.13);

\path[fill=fillColor,fill opacity=0.20] (216.79, 55.82) circle (  2.13);

\path[fill=fillColor,fill opacity=0.20] (209.91, 57.44) circle (  2.13);

\path[fill=fillColor,fill opacity=0.20] (221.70, 55.82) circle (  2.13);

\path[fill=fillColor,fill opacity=0.20] (231.53, 53.38) circle (  2.13);

\path[fill=fillColor,fill opacity=0.20] (212.86, 45.25) circle (  2.13);

\path[fill=fillColor,fill opacity=0.20] (211.87, 39.56) circle (  2.13);

\path[fill=fillColor,fill opacity=0.20] (216.79, 43.63) circle (  2.13);

\path[fill=fillColor,fill opacity=0.20] (214.82, 50.94) circle (  2.13);

\path[fill=fillColor,fill opacity=0.20] (212.86, 56.63) circle (  2.13);

\path[fill=fillColor,fill opacity=0.20] (219.74, 61.51) circle (  2.13);

\path[fill=fillColor,fill opacity=0.20] (221.70, 65.57) circle (  2.13);

\path[fill=fillColor,fill opacity=0.20] (225.63, 63.13) circle (  2.13);

\path[fill=fillColor,fill opacity=0.20] (238.41, 56.63) circle (  2.13);

\path[fill=fillColor,fill opacity=0.20] (205.00, 89.14) circle (  2.13);

\path[fill=fillColor,fill opacity=0.20] (203.03, 93.20) circle (  2.13);

\path[fill=fillColor,fill opacity=0.20] (202.05,102.14) circle (  2.13);

\path[fill=fillColor,fill opacity=0.20] (203.03,107.83) circle (  2.13);

\path[fill=fillColor,fill opacity=0.20] (205.00,102.96) circle (  2.13);

\path[fill=fillColor,fill opacity=0.20] (243.32, 75.32) circle (  2.13);

\path[fill=fillColor,fill opacity=0.20] (246.27, 63.13) circle (  2.13);

\path[fill=fillColor,fill opacity=0.20] (242.34, 55.01) circle (  2.13);

\path[fill=fillColor,fill opacity=0.20] (238.41, 60.69) circle (  2.13);

\path[fill=fillColor,fill opacity=0.20] (239.39, 53.38) circle (  2.13);

\path[fill=fillColor,fill opacity=0.20] (242.34, 37.94) circle (  2.13);

\path[fill=fillColor,fill opacity=0.20] (209.91, 79.39) circle (  2.13);

\path[fill=fillColor,fill opacity=0.20] (205.00, 83.45) circle (  2.13);

\path[fill=fillColor,fill opacity=0.20] (202.05, 98.89) circle (  2.13);

\path[fill=fillColor,fill opacity=0.20] (198.12,103.77) circle (  2.13);

\path[fill=fillColor,fill opacity=0.20] (198.12,102.96) circle (  2.13);

\path[fill=fillColor,fill opacity=0.20] (201.07,105.39) circle (  2.13);

\path[fill=fillColor,fill opacity=0.20] (204.01,110.27) circle (  2.13);

\path[fill=fillColor,fill opacity=0.20] (206.96,114.33) circle (  2.13);

\path[fill=fillColor,fill opacity=0.20] (241.36, 72.07) circle (  2.13);

\path[fill=fillColor,fill opacity=0.20] (236.44, 68.82) circle (  2.13);

\path[fill=fillColor,fill opacity=0.20] (240.37, 77.76) circle (  2.13);

\path[fill=fillColor,fill opacity=0.20] (231.53, 72.89) circle (  2.13);

\path[fill=fillColor,fill opacity=0.20] (221.70, 66.38) circle (  2.13);

\path[fill=fillColor,fill opacity=0.20] (218.75, 70.45) circle (  2.13);

\path[fill=fillColor,fill opacity=0.20] (222.68, 66.38) circle (  2.13);

\path[fill=fillColor,fill opacity=0.20] (225.63, 59.07) circle (  2.13);

\path[fill=fillColor,fill opacity=0.20] (229.56, 53.38) circle (  2.13);

\path[fill=fillColor,fill opacity=0.20] (233.49, 50.94) circle (  2.13);

\path[fill=fillColor,fill opacity=0.20] (209.91, 90.76) circle (  2.13);

\path[fill=fillColor,fill opacity=0.20] (203.03, 75.32) circle (  2.13);

\path[fill=fillColor,fill opacity=0.20] (200.08,100.52) circle (  2.13);

\path[fill=fillColor,fill opacity=0.20] (196.15,102.14) circle (  2.13);

\path[fill=fillColor,fill opacity=0.20] (193.20,100.52) circle (  2.13);

\path[fill=fillColor,fill opacity=0.20] (195.17, 95.64) circle (  2.13);

\path[fill=fillColor,fill opacity=0.20] (197.13, 98.08) circle (  2.13);

\path[fill=fillColor,fill opacity=0.20] (202.05, 98.89) circle (  2.13);

\path[fill=fillColor,fill opacity=0.20] (207.94, 94.83) circle (  2.13);

\path[fill=fillColor,fill opacity=0.20] (211.87,104.58) circle (  2.13);

\path[fill=fillColor,fill opacity=0.20] (204.01, 65.57) circle (  2.13);

\path[fill=fillColor,fill opacity=0.20] (229.56, 69.63) circle (  2.13);

\path[fill=fillColor,fill opacity=0.20] (219.74, 78.57) circle (  2.13);

\path[fill=fillColor,fill opacity=0.20] (224.65, 78.57) circle (  2.13);

\path[fill=fillColor,fill opacity=0.20] (219.74, 76.95) circle (  2.13);

\path[fill=fillColor,fill opacity=0.20] (215.81, 89.14) circle (  2.13);

\path[fill=fillColor,fill opacity=0.20] (213.84, 94.02) circle (  2.13);

\path[fill=fillColor,fill opacity=0.20] (214.82, 77.76) circle (  2.13);

\path[fill=fillColor,fill opacity=0.20] (219.74, 69.63) circle (  2.13);

\path[fill=fillColor,fill opacity=0.20] (215.81, 72.07) circle (  2.13);

\path[fill=fillColor,fill opacity=0.20] (206.96, 72.89) circle (  2.13);

\path[fill=fillColor,fill opacity=0.20] (199.10, 90.76) circle (  2.13);

\path[fill=fillColor,fill opacity=0.20] (197.13,110.27) circle (  2.13);

\path[fill=fillColor,fill opacity=0.20] (193.20,106.21) circle (  2.13);

\path[fill=fillColor,fill opacity=0.20] (194.19, 98.08) circle (  2.13);

\path[fill=fillColor,fill opacity=0.20] (195.17, 89.14) circle (  2.13);

\path[fill=fillColor,fill opacity=0.20] (196.15, 86.70) circle (  2.13);

\path[fill=fillColor,fill opacity=0.20] (201.07, 80.20) circle (  2.13);

\path[fill=fillColor,fill opacity=0.20] (205.00, 73.70) circle (  2.13);

\path[fill=fillColor,fill opacity=0.20] (208.93, 85.08) circle (  2.13);

\path[fill=fillColor,fill opacity=0.20] (242.34, 63.13) circle (  2.13);

\path[fill=fillColor,fill opacity=0.20] (234.48, 52.57) circle (  2.13);

\path[fill=fillColor,fill opacity=0.20] (228.58, 81.82) circle (  2.13);

\path[fill=fillColor,fill opacity=0.20] (224.65,103.77) circle (  2.13);

\path[fill=fillColor,fill opacity=0.20] (221.70, 98.08) circle (  2.13);

\path[fill=fillColor,fill opacity=0.20] (215.81, 97.27) circle (  2.13);

\path[fill=fillColor,fill opacity=0.20] (211.87, 99.70) circle (  2.13);

\path[fill=fillColor,fill opacity=0.20] (212.86, 97.27) circle (  2.13);

\path[fill=fillColor,fill opacity=0.20] (214.82, 87.51) circle (  2.13);

\path[fill=fillColor,fill opacity=0.20] (219.74, 73.70) circle (  2.13);

\path[fill=fillColor,fill opacity=0.20] (229.56, 68.01) circle (  2.13);

\path[fill=fillColor,fill opacity=0.20] (235.46, 72.89) circle (  2.13);

\path[fill=fillColor,fill opacity=0.20] (205.98, 86.70) circle (  2.13);

\path[fill=fillColor,fill opacity=0.20] (199.10, 83.45) circle (  2.13);

\path[fill=fillColor,fill opacity=0.20] (197.13,105.39) circle (  2.13);

\path[fill=fillColor,fill opacity=0.20] (194.19,111.08) circle (  2.13);

\path[fill=fillColor,fill opacity=0.20] (194.19, 97.27) circle (  2.13);

\path[fill=fillColor,fill opacity=0.20] (193.20, 87.51) circle (  2.13);

\path[fill=fillColor,fill opacity=0.20] (195.17, 87.51) circle (  2.13);

\path[fill=fillColor,fill opacity=0.20] (199.10, 78.57) circle (  2.13);

\path[fill=fillColor,fill opacity=0.20] (201.07, 68.01) circle (  2.13);

\path[fill=fillColor,fill opacity=0.20] (204.01, 70.45) circle (  2.13);

\path[fill=fillColor,fill opacity=0.20] (208.93, 82.64) circle (  2.13);

\path[fill=fillColor,fill opacity=0.20] (257.08, 53.38) circle (  2.13);

\path[fill=fillColor,fill opacity=0.20] (224.65, 54.19) circle (  2.13);

\path[fill=fillColor,fill opacity=0.20] (216.79, 63.95) circle (  2.13);

\path[fill=fillColor,fill opacity=0.20] (220.72, 89.95) circle (  2.13);

\path[fill=fillColor,fill opacity=0.20] (218.75,105.39) circle (  2.13);

\path[fill=fillColor,fill opacity=0.20] (214.82,107.02) circle (  2.13);

\path[fill=fillColor,fill opacity=0.20] (210.89,104.58) circle (  2.13);

\path[fill=fillColor,fill opacity=0.20] (209.91, 94.83) circle (  2.13);

\path[fill=fillColor,fill opacity=0.20] (209.91, 85.08) circle (  2.13);

\path[fill=fillColor,fill opacity=0.20] (207.94, 87.51) circle (  2.13);

\path[fill=fillColor,fill opacity=0.20] (215.81, 74.51) circle (  2.13);

\path[fill=fillColor,fill opacity=0.20] (234.48, 59.88) circle (  2.13);

\path[fill=fillColor,fill opacity=0.20] (206.96, 78.57) circle (  2.13);

\path[fill=fillColor,fill opacity=0.20] (201.07, 75.32) circle (  2.13);

\path[fill=fillColor,fill opacity=0.20] (199.10, 94.83) circle (  2.13);

\path[fill=fillColor,fill opacity=0.20] (192.22,101.33) circle (  2.13);

\path[fill=fillColor,fill opacity=0.20] (192.22, 94.83) circle (  2.13);

\path[fill=fillColor,fill opacity=0.20] (195.17, 87.51) circle (  2.13);

\path[fill=fillColor,fill opacity=0.20] (195.17, 98.08) circle (  2.13);

\path[fill=fillColor,fill opacity=0.20] (196.15, 98.08) circle (  2.13);

\path[fill=fillColor,fill opacity=0.20] (198.12, 81.82) circle (  2.13);

\path[fill=fillColor,fill opacity=0.20] (203.03, 71.26) circle (  2.13);

\path[fill=fillColor,fill opacity=0.20] (207.94, 71.26) circle (  2.13);

\path[fill=fillColor,fill opacity=0.20] (211.87, 79.39) circle (  2.13);

\path[fill=fillColor,fill opacity=0.20] (241.36, 44.44) circle (  2.13);

\path[fill=fillColor,fill opacity=0.20] (221.70, 47.69) circle (  2.13);

\path[fill=fillColor,fill opacity=0.20] (216.79, 75.32) circle (  2.13);

\path[fill=fillColor,fill opacity=0.20] (209.91, 82.64) circle (  2.13);

\path[fill=fillColor,fill opacity=0.20] (207.94, 87.51) circle (  2.13);

\path[fill=fillColor,fill opacity=0.20] (204.01, 99.70) circle (  2.13);

\path[fill=fillColor,fill opacity=0.20] (200.08, 97.27) circle (  2.13);

\path[fill=fillColor,fill opacity=0.20] (204.01, 87.51) circle (  2.13);

\path[fill=fillColor,fill opacity=0.20] (205.00, 78.57) circle (  2.13);

\path[fill=fillColor,fill opacity=0.20] (204.01, 78.57) circle (  2.13);

\path[fill=fillColor,fill opacity=0.20] (211.87, 78.57) circle (  2.13);

\path[fill=fillColor,fill opacity=0.20] (206.96, 61.51) circle (  2.13);

\path[fill=fillColor,fill opacity=0.20] (202.05, 78.57) circle (  2.13);

\path[fill=fillColor,fill opacity=0.20] (201.07, 90.76) circle (  2.13);

\path[fill=fillColor,fill opacity=0.20] (193.20, 93.20) circle (  2.13);

\path[fill=fillColor,fill opacity=0.20] (193.20, 94.02) circle (  2.13);

\path[fill=fillColor,fill opacity=0.20] (197.13, 92.39) circle (  2.13);

\path[fill=fillColor,fill opacity=0.20] (197.13,102.96) circle (  2.13);

\path[fill=fillColor,fill opacity=0.20] (197.13,103.77) circle (  2.13);

\path[fill=fillColor,fill opacity=0.20] (199.10, 85.08) circle (  2.13);

\path[fill=fillColor,fill opacity=0.20] (204.01, 68.82) circle (  2.13);

\path[fill=fillColor,fill opacity=0.20] (209.91, 66.38) circle (  2.13);

\path[fill=fillColor,fill opacity=0.20] (212.86, 74.51) circle (  2.13);

\path[fill=fillColor,fill opacity=0.20] (249.22, 49.32) circle (  2.13);

\path[fill=fillColor,fill opacity=0.20] (223.67, 51.75) circle (  2.13);

\path[fill=fillColor,fill opacity=0.20] (215.81, 52.57) circle (  2.13);

\path[fill=fillColor,fill opacity=0.20] (212.86, 73.70) circle (  2.13);

\path[fill=fillColor,fill opacity=0.20] (206.96, 81.82) circle (  2.13);

\path[fill=fillColor,fill opacity=0.20] (196.15, 86.70) circle (  2.13);

\path[fill=fillColor,fill opacity=0.20] (196.15,101.33) circle (  2.13);

\path[fill=fillColor,fill opacity=0.20] (198.12, 92.39) circle (  2.13);

\path[fill=fillColor,fill opacity=0.20] (200.08, 76.14) circle (  2.13);

\path[fill=fillColor,fill opacity=0.20] (201.07, 76.14) circle (  2.13);

\path[fill=fillColor,fill opacity=0.20] (204.01, 77.76) circle (  2.13);

\path[fill=fillColor,fill opacity=0.20] (215.81, 87.51) circle (  2.13);

\path[fill=fillColor,fill opacity=0.20] (215.81, 58.26) circle (  2.13);

\path[fill=fillColor,fill opacity=0.20] (205.98, 68.82) circle (  2.13);

\path[fill=fillColor,fill opacity=0.20] (202.05, 85.08) circle (  2.13);

\path[fill=fillColor,fill opacity=0.20] (201.07, 93.20) circle (  2.13);

\path[fill=fillColor,fill opacity=0.20] (199.10, 98.89) circle (  2.13);

\path[fill=fillColor,fill opacity=0.20] (199.10, 93.20) circle (  2.13);

\path[fill=fillColor,fill opacity=0.20] (197.13, 89.95) circle (  2.13);

\path[fill=fillColor,fill opacity=0.20] (195.17, 95.64) circle (  2.13);

\path[fill=fillColor,fill opacity=0.20] (199.10, 91.58) circle (  2.13);

\path[fill=fillColor,fill opacity=0.20] (202.05, 76.95) circle (  2.13);

\path[fill=fillColor,fill opacity=0.20] (205.00, 68.01) circle (  2.13);

\path[fill=fillColor,fill opacity=0.20] (211.87, 64.76) circle (  2.13);

\path[fill=fillColor,fill opacity=0.20] (247.25, 47.69) circle (  2.13);

\path[fill=fillColor,fill opacity=0.20] (219.74, 53.38) circle (  2.13);

\path[fill=fillColor,fill opacity=0.20] (211.87, 60.69) circle (  2.13);

\path[fill=fillColor,fill opacity=0.20] (211.87, 80.20) circle (  2.13);

\path[fill=fillColor,fill opacity=0.20] (206.96, 94.83) circle (  2.13);

\path[fill=fillColor,fill opacity=0.20] (205.00,102.14) circle (  2.13);

\path[fill=fillColor,fill opacity=0.20] (202.05,105.39) circle (  2.13);

\path[fill=fillColor,fill opacity=0.20] (201.07, 98.08) circle (  2.13);

\path[fill=fillColor,fill opacity=0.20] (202.05, 76.95) circle (  2.13);

\path[fill=fillColor,fill opacity=0.20] (206.96, 72.07) circle (  2.13);

\path[fill=fillColor,fill opacity=0.20] (213.84, 80.20) circle (  2.13);

\path[fill=fillColor,fill opacity=0.20] (223.67, 66.38) circle (  2.13);

\path[fill=fillColor,fill opacity=0.20] (211.87, 67.20) circle (  2.13);

\path[fill=fillColor,fill opacity=0.20] (204.01, 81.82) circle (  2.13);

\path[fill=fillColor,fill opacity=0.20] (201.07, 85.89) circle (  2.13);

\path[fill=fillColor,fill opacity=0.20] (201.07, 93.20) circle (  2.13);

\path[fill=fillColor,fill opacity=0.20] (202.05, 97.27) circle (  2.13);

\path[fill=fillColor,fill opacity=0.20] (201.07, 91.58) circle (  2.13);

\path[fill=fillColor,fill opacity=0.20] (198.12, 87.51) circle (  2.13);

\path[fill=fillColor,fill opacity=0.20] (197.13, 87.51) circle (  2.13);

\path[fill=fillColor,fill opacity=0.20] (202.05, 84.26) circle (  2.13);

\path[fill=fillColor,fill opacity=0.20] (203.03, 81.82) circle (  2.13);

\path[fill=fillColor,fill opacity=0.20] (208.93, 74.51) circle (  2.13);

\path[fill=fillColor,fill opacity=0.20] (215.81, 63.95) circle (  2.13);

\path[fill=fillColor,fill opacity=0.20] (241.36, 44.44) circle (  2.13);

\path[fill=fillColor,fill opacity=0.20] (219.74, 47.69) circle (  2.13);

\path[fill=fillColor,fill opacity=0.20] (213.84, 64.76) circle (  2.13);

\path[fill=fillColor,fill opacity=0.20] (207.94, 85.08) circle (  2.13);

\path[fill=fillColor,fill opacity=0.20] (205.00, 94.02) circle (  2.13);

\path[fill=fillColor,fill opacity=0.20] (203.03,100.52) circle (  2.13);

\path[fill=fillColor,fill opacity=0.20] (205.00,101.33) circle (  2.13);

\path[fill=fillColor,fill opacity=0.20] (205.98, 95.64) circle (  2.13);

\path[fill=fillColor,fill opacity=0.20] (207.94, 86.70) circle (  2.13);

\path[fill=fillColor,fill opacity=0.20] (213.84, 74.51) circle (  2.13);

\path[fill=fillColor,fill opacity=0.20] (215.81, 69.63) circle (  2.13);

\path[fill=fillColor,fill opacity=0.20] (209.91, 78.57) circle (  2.13);

\path[fill=fillColor,fill opacity=0.20] (204.01, 73.70) circle (  2.13);

\path[fill=fillColor,fill opacity=0.20] (201.07, 76.14) circle (  2.13);

\path[fill=fillColor,fill opacity=0.20] (200.08, 89.14) circle (  2.13);

\path[fill=fillColor,fill opacity=0.20] (196.15, 89.95) circle (  2.13);

\path[fill=fillColor,fill opacity=0.20] (201.07, 88.33) circle (  2.13);

\path[fill=fillColor,fill opacity=0.20] (197.13, 91.58) circle (  2.13);

\path[fill=fillColor,fill opacity=0.20] (198.12, 89.14) circle (  2.13);

\path[fill=fillColor,fill opacity=0.20] (202.05, 89.14) circle (  2.13);

\path[fill=fillColor,fill opacity=0.20] (205.98, 90.76) circle (  2.13);

\path[fill=fillColor,fill opacity=0.20] (210.89, 74.51) circle (  2.13);

\path[fill=fillColor,fill opacity=0.20] (217.77, 65.57) circle (  2.13);

\path[fill=fillColor,fill opacity=0.20] (216.79, 42.82) circle (  2.13);

\path[fill=fillColor,fill opacity=0.20] (207.94, 56.63) circle (  2.13);

\path[fill=fillColor,fill opacity=0.20] (203.03, 81.01) circle (  2.13);

\path[fill=fillColor,fill opacity=0.20] (201.07, 94.02) circle (  2.13);

\path[fill=fillColor,fill opacity=0.20] (202.05, 98.08) circle (  2.13);

\path[fill=fillColor,fill opacity=0.20] (205.00, 93.20) circle (  2.13);

\path[fill=fillColor,fill opacity=0.20] (206.96, 85.89) circle (  2.13);

\path[fill=fillColor,fill opacity=0.20] (208.93, 87.51) circle (  2.13);

\path[fill=fillColor,fill opacity=0.20] (216.79, 81.82) circle (  2.13);

\path[fill=fillColor,fill opacity=0.20] (200.08, 65.57) circle (  2.13);

\path[fill=fillColor,fill opacity=0.20] (213.84, 75.32) circle (  2.13);

\path[fill=fillColor,fill opacity=0.20] (205.98, 76.95) circle (  2.13);

\path[fill=fillColor,fill opacity=0.20] (202.05, 70.45) circle (  2.13);

\path[fill=fillColor,fill opacity=0.20] (199.10, 70.45) circle (  2.13);

\path[fill=fillColor,fill opacity=0.20] (196.15, 86.70) circle (  2.13);

\path[fill=fillColor,fill opacity=0.20] (196.15, 85.89) circle (  2.13);

\path[fill=fillColor,fill opacity=0.20] (197.13, 84.26) circle (  2.13);

\path[fill=fillColor,fill opacity=0.20] (197.13, 94.83) circle (  2.13);

\path[fill=fillColor,fill opacity=0.20] (200.08, 95.64) circle (  2.13);

\path[fill=fillColor,fill opacity=0.20] (202.05, 87.51) circle (  2.13);

\path[fill=fillColor,fill opacity=0.20] (206.96, 78.57) circle (  2.13);

\path[fill=fillColor,fill opacity=0.20] (214.82, 63.95) circle (  2.13);

\path[fill=fillColor,fill opacity=0.20] (218.75, 68.82) circle (  2.13);

\path[fill=fillColor,fill opacity=0.20] (224.65, 41.19) circle (  2.13);

\path[fill=fillColor,fill opacity=0.20] (209.91, 46.88) circle (  2.13);

\path[fill=fillColor,fill opacity=0.20] (203.03, 72.07) circle (  2.13);

\path[fill=fillColor,fill opacity=0.20] (202.05, 88.33) circle (  2.13);

\path[fill=fillColor,fill opacity=0.20] (205.98, 91.58) circle (  2.13);

\path[fill=fillColor,fill opacity=0.20] (209.91, 90.76) circle (  2.13);

\path[fill=fillColor,fill opacity=0.20] (207.94, 85.89) circle (  2.13);

\path[fill=fillColor,fill opacity=0.20] (209.91, 85.08) circle (  2.13);

\path[fill=fillColor,fill opacity=0.20] (216.79, 85.89) circle (  2.13);

\path[fill=fillColor,fill opacity=0.20] (218.75, 64.76) circle (  2.13);

\path[fill=fillColor,fill opacity=0.20] (211.87, 70.45) circle (  2.13);

\path[fill=fillColor,fill opacity=0.20] (203.03, 78.57) circle (  2.13);

\path[fill=fillColor,fill opacity=0.20] (202.05, 89.95) circle (  2.13);

\path[fill=fillColor,fill opacity=0.20] (201.07, 84.26) circle (  2.13);

\path[fill=fillColor,fill opacity=0.20] (195.17, 85.08) circle (  2.13);

\path[fill=fillColor,fill opacity=0.20] (193.20, 87.51) circle (  2.13);

\path[fill=fillColor,fill opacity=0.20] (194.19, 89.14) circle (  2.13);

\path[fill=fillColor,fill opacity=0.20] (196.15, 94.02) circle (  2.13);

\path[fill=fillColor,fill opacity=0.20] (198.12, 94.83) circle (  2.13);

\path[fill=fillColor,fill opacity=0.20] (199.10, 79.39) circle (  2.13);

\path[fill=fillColor,fill opacity=0.20] (210.89, 62.32) circle (  2.13);

\path[fill=fillColor,fill opacity=0.20] (219.74, 55.82) circle (  2.13);

\path[fill=fillColor,fill opacity=0.20] (213.84, 40.38) circle (  2.13);

\path[fill=fillColor,fill opacity=0.20] (206.96, 61.51) circle (  2.13);

\path[fill=fillColor,fill opacity=0.20] (206.96, 75.32) circle (  2.13);

\path[fill=fillColor,fill opacity=0.20] (207.94, 78.57) circle (  2.13);

\path[fill=fillColor,fill opacity=0.20] (205.00, 89.95) circle (  2.13);

\path[fill=fillColor,fill opacity=0.20] (210.89, 93.20) circle (  2.13);

\path[fill=fillColor,fill opacity=0.20] (213.84, 81.01) circle (  2.13);

\path[fill=fillColor,fill opacity=0.20] (214.82, 76.14) circle (  2.13);

\path[fill=fillColor,fill opacity=0.20] (220.72, 84.26) circle (  2.13);

\path[fill=fillColor,fill opacity=0.20] (222.68, 62.32) circle (  2.13);

\path[fill=fillColor,fill opacity=0.20] (214.82, 57.44) circle (  2.13);

\path[fill=fillColor,fill opacity=0.20] (205.00, 66.38) circle (  2.13);

\path[fill=fillColor,fill opacity=0.20] (204.01, 82.64) circle (  2.13);

\path[fill=fillColor,fill opacity=0.20] (203.03, 96.45) circle (  2.13);

\path[fill=fillColor,fill opacity=0.20] (202.05, 94.83) circle (  2.13);

\path[fill=fillColor,fill opacity=0.20] (196.15, 86.70) circle (  2.13);

\path[fill=fillColor,fill opacity=0.20] (192.22, 89.95) circle (  2.13);

\path[fill=fillColor,fill opacity=0.20] (191.24, 96.45) circle (  2.13);

\path[fill=fillColor,fill opacity=0.20] (191.24, 94.02) circle (  2.13);

\path[fill=fillColor,fill opacity=0.20] (193.20, 85.08) circle (  2.13);

\path[fill=fillColor,fill opacity=0.20] (197.13, 74.51) circle (  2.13);

\path[fill=fillColor,fill opacity=0.20] (210.89, 61.51) circle (  2.13);

\path[fill=fillColor,fill opacity=0.20] (224.65, 41.19) circle (  2.13);

\path[fill=fillColor,fill opacity=0.20] (213.84, 49.32) circle (  2.13);

\path[fill=fillColor,fill opacity=0.20] (209.91, 57.44) circle (  2.13);

\path[fill=fillColor,fill opacity=0.20] (207.94, 71.26) circle (  2.13);

\path[fill=fillColor,fill opacity=0.20] (208.93, 89.95) circle (  2.13);

\path[fill=fillColor,fill opacity=0.20] (215.81, 92.39) circle (  2.13);

\path[fill=fillColor,fill opacity=0.20] (215.81, 78.57) circle (  2.13);

\path[fill=fillColor,fill opacity=0.20] (212.86, 70.45) circle (  2.13);

\path[fill=fillColor,fill opacity=0.20] (216.79, 72.07) circle (  2.13);

\path[fill=fillColor,fill opacity=0.20] (229.56, 82.64) circle (  2.13);

\path[fill=fillColor,fill opacity=0.20] (218.75, 67.20) circle (  2.13);

\path[fill=fillColor,fill opacity=0.20] (211.87, 61.51) circle (  2.13);

\path[fill=fillColor,fill opacity=0.20] (209.91, 62.32) circle (  2.13);

\path[fill=fillColor,fill opacity=0.20] (208.93, 73.70) circle (  2.13);

\path[fill=fillColor,fill opacity=0.20] (205.00, 80.20) circle (  2.13);

\path[fill=fillColor,fill opacity=0.20] (201.07, 82.64) circle (  2.13);

\path[fill=fillColor,fill opacity=0.20] (200.08, 90.76) circle (  2.13);

\path[fill=fillColor,fill opacity=0.20] (196.15, 92.39) circle (  2.13);

\path[fill=fillColor,fill opacity=0.20] (193.20, 89.95) circle (  2.13);

\path[fill=fillColor,fill opacity=0.20] (195.17, 93.20) circle (  2.13);

\path[fill=fillColor,fill opacity=0.20] (192.22, 85.89) circle (  2.13);

\path[fill=fillColor,fill opacity=0.20] (196.15, 72.89) circle (  2.13);

\path[fill=fillColor,fill opacity=0.20] (208.93, 68.82) circle (  2.13);

\path[fill=fillColor,fill opacity=0.20] (220.72, 46.07) circle (  2.13);

\path[fill=fillColor,fill opacity=0.20] (210.89, 55.01) circle (  2.13);

\path[fill=fillColor,fill opacity=0.20] (209.91, 68.82) circle (  2.13);

\path[fill=fillColor,fill opacity=0.20] (210.89, 80.20) circle (  2.13);

\path[fill=fillColor,fill opacity=0.20] (211.87, 84.26) circle (  2.13);

\path[fill=fillColor,fill opacity=0.20] (213.84, 84.26) circle (  2.13);

\path[fill=fillColor,fill opacity=0.20] (210.89, 80.20) circle (  2.13);

\path[fill=fillColor,fill opacity=0.20] (212.86, 76.95) circle (  2.13);

\path[fill=fillColor,fill opacity=0.20] (220.72, 73.70) circle (  2.13);

\path[fill=fillColor,fill opacity=0.20] (231.53, 78.57) circle (  2.13);

\path[fill=fillColor,fill opacity=0.20] (210.89, 79.39) circle (  2.13);

\path[fill=fillColor,fill opacity=0.20] (213.84, 61.51) circle (  2.13);

\path[fill=fillColor,fill opacity=0.20] (209.91, 68.82) circle (  2.13);

\path[fill=fillColor,fill opacity=0.20] (209.91, 85.08) circle (  2.13);

\path[fill=fillColor,fill opacity=0.20] (205.00, 86.70) circle (  2.13);

\path[fill=fillColor,fill opacity=0.20] (202.05, 76.95) circle (  2.13);

\path[fill=fillColor,fill opacity=0.20] (197.13, 73.70) circle (  2.13);

\path[fill=fillColor,fill opacity=0.20] (196.15, 85.89) circle (  2.13);

\path[fill=fillColor,fill opacity=0.20] (191.24, 91.58) circle (  2.13);

\path[fill=fillColor,fill opacity=0.20] (194.19, 83.45) circle (  2.13);

\path[fill=fillColor,fill opacity=0.20] (201.07, 74.51) circle (  2.13);

\path[fill=fillColor,fill opacity=0.20] (199.10, 70.45) circle (  2.13);

\path[fill=fillColor,fill opacity=0.20] (205.98, 70.45) circle (  2.13);

\path[fill=fillColor,fill opacity=0.20] (216.79, 62.32) circle (  2.13);

\path[fill=fillColor,fill opacity=0.20] (214.82, 52.57) circle (  2.13);

\path[fill=fillColor,fill opacity=0.20] (210.89, 60.69) circle (  2.13);

\path[fill=fillColor,fill opacity=0.20] (211.87, 62.32) circle (  2.13);

\path[fill=fillColor,fill opacity=0.20] (211.87, 72.89) circle (  2.13);

\path[fill=fillColor,fill opacity=0.20] (211.87, 86.70) circle (  2.13);

\path[fill=fillColor,fill opacity=0.20] (209.91, 90.76) circle (  2.13);

\path[fill=fillColor,fill opacity=0.20] (211.87, 86.70) circle (  2.13);

\path[fill=fillColor,fill opacity=0.20] (219.74, 76.95) circle (  2.13);

\path[fill=fillColor,fill opacity=0.20] (224.65, 68.01) circle (  2.13);

\path[fill=fillColor,fill opacity=0.20] (234.48, 72.89) circle (  2.13);

\path[fill=fillColor,fill opacity=0.20] (221.70, 79.39) circle (  2.13);

\path[fill=fillColor,fill opacity=0.20] (210.89, 78.57) circle (  2.13);

\path[fill=fillColor,fill opacity=0.20] (207.94, 75.32) circle (  2.13);

\path[fill=fillColor,fill opacity=0.20] (211.87, 78.57) circle (  2.13);

\path[fill=fillColor,fill opacity=0.20] (211.87, 90.76) circle (  2.13);

\path[fill=fillColor,fill opacity=0.20] (202.05, 90.76) circle (  2.13);

\path[fill=fillColor,fill opacity=0.20] (199.10, 81.82) circle (  2.13);

\path[fill=fillColor,fill opacity=0.20] (197.13, 82.64) circle (  2.13);

\path[fill=fillColor,fill opacity=0.20] (195.17, 89.95) circle (  2.13);

\path[fill=fillColor,fill opacity=0.20] (191.24, 89.95) circle (  2.13);

\path[fill=fillColor,fill opacity=0.20] (168.64, 83.45) circle (  2.13);

\path[fill=fillColor,fill opacity=0.20] (204.01, 68.01) circle (  2.13);

\path[fill=fillColor,fill opacity=0.20] (207.94, 59.88) circle (  2.13);

\path[fill=fillColor,fill opacity=0.20] (213.84, 63.13) circle (  2.13);

\path[fill=fillColor,fill opacity=0.20] (224.65, 44.44) circle (  2.13);

\path[fill=fillColor,fill opacity=0.20] (216.79, 46.88) circle (  2.13);

\path[fill=fillColor,fill opacity=0.20] (210.89, 47.69) circle (  2.13);

\path[fill=fillColor,fill opacity=0.20] (211.87, 58.26) circle (  2.13);

\path[fill=fillColor,fill opacity=0.20] (212.86, 79.39) circle (  2.13);

\path[fill=fillColor,fill opacity=0.20] (211.87, 88.33) circle (  2.13);

\path[fill=fillColor,fill opacity=0.20] (216.79, 82.64) circle (  2.13);

\path[fill=fillColor,fill opacity=0.20] (212.86, 78.57) circle (  2.13);

\path[fill=fillColor,fill opacity=0.20] (219.74, 77.76) circle (  2.13);

\path[fill=fillColor,fill opacity=0.20] (226.62, 76.95) circle (  2.13);

\path[fill=fillColor,fill opacity=0.20] (235.46, 72.89) circle (  2.13);

\path[fill=fillColor,fill opacity=0.20] (223.67, 68.01) circle (  2.13);

\path[fill=fillColor,fill opacity=0.20] (215.81, 57.44) circle (  2.13);

\path[fill=fillColor,fill opacity=0.20] (211.87, 76.95) circle (  2.13);

\path[fill=fillColor,fill opacity=0.20] (205.98, 86.70) circle (  2.13);

\path[fill=fillColor,fill opacity=0.20] (202.05, 79.39) circle (  2.13);

\path[fill=fillColor,fill opacity=0.20] (205.98, 80.20) circle (  2.13);

\path[fill=fillColor,fill opacity=0.20] (203.03, 89.14) circle (  2.13);

\path[fill=fillColor,fill opacity=0.20] (201.07, 88.33) circle (  2.13);

\path[fill=fillColor,fill opacity=0.20] (200.08, 85.08) circle (  2.13);

\path[fill=fillColor,fill opacity=0.20] (198.12, 85.08) circle (  2.13);

\path[fill=fillColor,fill opacity=0.20] (200.08, 84.26) circle (  2.13);

\path[fill=fillColor,fill opacity=0.20] (203.03, 81.01) circle (  2.13);

\path[fill=fillColor,fill opacity=0.20] (208.93, 72.89) circle (  2.13);

\path[fill=fillColor,fill opacity=0.20] (216.79, 55.01) circle (  2.13);

\path[fill=fillColor,fill opacity=0.20] (217.77, 47.69) circle (  2.13);

\path[fill=fillColor,fill opacity=0.20] (208.93, 40.38) circle (  2.13);

\path[fill=fillColor,fill opacity=0.20] (209.91, 45.25) circle (  2.13);

\path[fill=fillColor,fill opacity=0.20] (212.86, 64.76) circle (  2.13);

\path[fill=fillColor,fill opacity=0.20] (217.77, 76.14) circle (  2.13);

\path[fill=fillColor,fill opacity=0.20] (215.81, 76.14) circle (  2.13);

\path[fill=fillColor,fill opacity=0.20] (212.86, 78.57) circle (  2.13);

\path[fill=fillColor,fill opacity=0.20] (213.84, 85.89) circle (  2.13);

\path[fill=fillColor,fill opacity=0.20] (221.70, 87.51) circle (  2.13);

\path[fill=fillColor,fill opacity=0.20] (224.65, 78.57) circle (  2.13);

\path[fill=fillColor,fill opacity=0.20] (223.67, 72.07) circle (  2.13);

\path[fill=fillColor,fill opacity=0.20] (228.58, 87.51) circle (  2.13);

\path[fill=fillColor,fill opacity=0.20] (224.65, 68.82) circle (  2.13);

\path[fill=fillColor,fill opacity=0.20] (220.72, 54.19) circle (  2.13);

\path[fill=fillColor,fill opacity=0.20] (213.84, 50.94) circle (  2.13);

\path[fill=fillColor,fill opacity=0.20] (210.89, 73.70) circle (  2.13);

\path[fill=fillColor,fill opacity=0.20] (208.93, 87.51) circle (  2.13);

\path[fill=fillColor,fill opacity=0.20] (197.13, 82.64) circle (  2.13);

\path[fill=fillColor,fill opacity=0.20] (202.05, 78.57) circle (  2.13);

\path[fill=fillColor,fill opacity=0.20] (202.05, 85.89) circle (  2.13);

\path[fill=fillColor,fill opacity=0.20] (205.00, 85.08) circle (  2.13);

\path[fill=fillColor,fill opacity=0.20] (203.03, 76.95) circle (  2.13);

\path[fill=fillColor,fill opacity=0.20] (196.15, 71.26) circle (  2.13);

\path[fill=fillColor,fill opacity=0.20] (207.94, 63.13) circle (  2.13);

\path[fill=fillColor,fill opacity=0.20] (211.87, 63.13) circle (  2.13);

\path[fill=fillColor,fill opacity=0.20] (214.82, 65.57) circle (  2.13);

\path[fill=fillColor,fill opacity=0.20] (223.67, 58.26) circle (  2.13);

\path[fill=fillColor,fill opacity=0.20] (224.65, 46.88) circle (  2.13);

\path[fill=fillColor,fill opacity=0.20] (208.93, 53.38) circle (  2.13);

\path[fill=fillColor,fill opacity=0.20] (210.89, 44.44) circle (  2.13);

\path[fill=fillColor,fill opacity=0.20] (213.84, 45.25) circle (  2.13);

\path[fill=fillColor,fill opacity=0.20] (212.86, 55.82) circle (  2.13);

\path[fill=fillColor,fill opacity=0.20] (213.84, 65.57) circle (  2.13);

\path[fill=fillColor,fill opacity=0.20] (215.81, 72.07) circle (  2.13);

\path[fill=fillColor,fill opacity=0.20] (218.75, 81.82) circle (  2.13);

\path[fill=fillColor,fill opacity=0.20] (216.79, 91.58) circle (  2.13);

\path[fill=fillColor,fill opacity=0.20] (219.74, 95.64) circle (  2.13);

\path[fill=fillColor,fill opacity=0.20] (218.75, 83.45) circle (  2.13);

\path[fill=fillColor,fill opacity=0.20] (225.63, 66.38) circle (  2.13);

\path[fill=fillColor,fill opacity=0.20] (232.51, 71.26) circle (  2.13);

\path[fill=fillColor,fill opacity=0.20] (221.70, 59.88) circle (  2.13);

\path[fill=fillColor,fill opacity=0.20] (212.86, 59.07) circle (  2.13);

\path[fill=fillColor,fill opacity=0.20] (215.81, 59.07) circle (  2.13);

\path[fill=fillColor,fill opacity=0.20] (216.79, 61.51) circle (  2.13);

\path[fill=fillColor,fill opacity=0.20] (210.89, 72.89) circle (  2.13);

\path[fill=fillColor,fill opacity=0.20] (206.96, 81.01) circle (  2.13);

\path[fill=fillColor,fill opacity=0.20] (205.00, 78.57) circle (  2.13);

\path[fill=fillColor,fill opacity=0.20] (202.05, 74.51) circle (  2.13);

\path[fill=fillColor,fill opacity=0.20] (196.15, 72.89) circle (  2.13);

\path[fill=fillColor,fill opacity=0.20] (208.93, 70.45) circle (  2.13);

\path[fill=fillColor,fill opacity=0.20] (207.94, 72.89) circle (  2.13);

\path[fill=fillColor,fill opacity=0.20] (206.96, 70.45) circle (  2.13);

\path[fill=fillColor,fill opacity=0.20] (214.82, 55.82) circle (  2.13);

\path[fill=fillColor,fill opacity=0.20] (220.72, 51.75) circle (  2.13);

\path[fill=fillColor,fill opacity=0.20] (213.84, 56.63) circle (  2.13);

\path[fill=fillColor,fill opacity=0.20] (211.87, 53.38) circle (  2.13);

\path[fill=fillColor,fill opacity=0.20] (209.91, 55.01) circle (  2.13);

\path[fill=fillColor,fill opacity=0.20] (212.86, 62.32) circle (  2.13);

\path[fill=fillColor,fill opacity=0.20] (206.96, 72.07) circle (  2.13);

\path[fill=fillColor,fill opacity=0.20] (215.81, 88.33) circle (  2.13);

\path[fill=fillColor,fill opacity=0.20] (210.89,107.83) circle (  2.13);

\path[fill=fillColor,fill opacity=0.20] (209.91,102.96) circle (  2.13);

\path[fill=fillColor,fill opacity=0.20] (217.77, 72.89) circle (  2.13);

\path[fill=fillColor,fill opacity=0.20] (217.77, 59.07) circle (  2.13);

\path[fill=fillColor,fill opacity=0.20] (223.67, 80.20) circle (  2.13);

\path[fill=fillColor,fill opacity=0.20] (228.58, 68.82) circle (  2.13);

\path[fill=fillColor,fill opacity=0.20] (219.74, 68.01) circle (  2.13);

\path[fill=fillColor,fill opacity=0.20] (212.86, 78.57) circle (  2.13);

\path[fill=fillColor,fill opacity=0.20] (208.93, 77.76) circle (  2.13);

\path[fill=fillColor,fill opacity=0.20] (210.89, 73.70) circle (  2.13);

\path[fill=fillColor,fill opacity=0.20] (211.87, 69.63) circle (  2.13);

\path[fill=fillColor,fill opacity=0.20] (213.84, 68.01) circle (  2.13);

\path[fill=fillColor,fill opacity=0.20] (211.87, 63.95) circle (  2.13);

\path[fill=fillColor,fill opacity=0.20] (211.87, 56.63) circle (  2.13);

\path[fill=fillColor,fill opacity=0.20] (216.79, 56.63) circle (  2.13);

\path[fill=fillColor,fill opacity=0.20] (216.79, 63.13) circle (  2.13);

\path[fill=fillColor,fill opacity=0.20] (212.86, 58.26) circle (  2.13);

\path[fill=fillColor,fill opacity=0.20] (206.96, 48.50) circle (  2.13);

\path[fill=fillColor,fill opacity=0.20] (211.87, 53.38) circle (  2.13);

\path[fill=fillColor,fill opacity=0.20] (210.89, 76.14) circle (  2.13);

\path[fill=fillColor,fill opacity=0.20] (209.91,106.21) circle (  2.13);

\path[fill=fillColor,fill opacity=0.20] (206.96,107.83) circle (  2.13);

\path[fill=fillColor,fill opacity=0.20] (210.89, 81.82) circle (  2.13);

\path[fill=fillColor,fill opacity=0.20] (210.89, 78.57) circle (  2.13);

\path[fill=fillColor,fill opacity=0.20] (211.87, 94.02) circle (  2.13);

\path[fill=fillColor,fill opacity=0.20] (212.86, 94.02) circle (  2.13);

\path[fill=fillColor,fill opacity=0.20] (223.67, 92.39) circle (  2.13);

\path[fill=fillColor,fill opacity=0.20] (221.70, 75.32) circle (  2.13);

\path[fill=fillColor,fill opacity=0.20] (219.74, 69.63) circle (  2.13);

\path[fill=fillColor,fill opacity=0.20] (214.82, 70.45) circle (  2.13);

\path[fill=fillColor,fill opacity=0.20] (208.93, 72.07) circle (  2.13);

\path[fill=fillColor,fill opacity=0.20] (206.96, 70.45) circle (  2.13);

\path[fill=fillColor,fill opacity=0.20] (205.98, 71.26) circle (  2.13);

\path[fill=fillColor,fill opacity=0.20] (210.89, 72.07) circle (  2.13);

\path[fill=fillColor,fill opacity=0.20] (215.81, 72.89) circle (  2.13);

\path[fill=fillColor,fill opacity=0.20] (218.75, 71.26) circle (  2.13);

\path[fill=fillColor,fill opacity=0.20] (224.65, 68.82) circle (  2.13);

\path[fill=fillColor,fill opacity=0.20] (226.62, 66.38) circle (  2.13);

\path[fill=fillColor,fill opacity=0.20] (207.94, 59.07) circle (  2.13);

\path[fill=fillColor,fill opacity=0.20] (210.89, 53.38) circle (  2.13);

\path[fill=fillColor,fill opacity=0.20] (206.96, 64.76) circle (  2.13);

\path[fill=fillColor,fill opacity=0.20] (206.96, 91.58) circle (  2.13);

\path[fill=fillColor,fill opacity=0.20] (207.94, 93.20) circle (  2.13);

\path[fill=fillColor,fill opacity=0.20] (208.93, 76.95) circle (  2.13);

\path[fill=fillColor,fill opacity=0.20] (209.91, 89.14) circle (  2.13);

\path[fill=fillColor,fill opacity=0.20] (206.96,104.58) circle (  2.13);

\path[fill=fillColor,fill opacity=0.20] (216.79, 92.39) circle (  2.13);

\path[fill=fillColor,fill opacity=0.20] (216.79, 75.32) circle (  2.13);

\path[fill=fillColor,fill opacity=0.20] (214.82, 85.89) circle (  2.13);

\path[fill=fillColor,fill opacity=0.20] (213.84, 68.82) circle (  2.13);

\path[fill=fillColor,fill opacity=0.20] (213.84, 62.32) circle (  2.13);

\path[fill=fillColor,fill opacity=0.20] (213.84, 59.88) circle (  2.13);

\path[fill=fillColor,fill opacity=0.20] (207.94, 56.63) circle (  2.13);

\path[fill=fillColor,fill opacity=0.20] (207.94, 64.76) circle (  2.13);

\path[fill=fillColor,fill opacity=0.20] (215.81, 72.07) circle (  2.13);

\path[fill=fillColor,fill opacity=0.20] (210.89, 77.76) circle (  2.13);

\path[fill=fillColor,fill opacity=0.20] (215.81, 69.63) circle (  2.13);

\path[fill=fillColor,fill opacity=0.20] (210.89, 68.01) circle (  2.13);

\path[fill=fillColor,fill opacity=0.20] (209.91, 72.89) circle (  2.13);

\path[fill=fillColor,fill opacity=0.20] (210.89, 69.63) circle (  2.13);

\path[fill=fillColor,fill opacity=0.20] (209.91, 68.01) circle (  2.13);

\path[fill=fillColor,fill opacity=0.20] (206.96, 82.64) circle (  2.13);

\path[fill=fillColor,fill opacity=0.20] (206.96, 94.02) circle (  2.13);

\path[fill=fillColor,fill opacity=0.20] (210.89, 92.39) circle (  2.13);

\path[fill=fillColor,fill opacity=0.20] (214.82, 92.39) circle (  2.13);

\path[fill=fillColor,fill opacity=0.20] (211.87, 85.89) circle (  2.13);

\path[fill=fillColor,fill opacity=0.20] (219.74, 63.95) circle (  2.13);

\path[fill=fillColor,fill opacity=0.20] (222.68, 55.01) circle (  2.13);

\path[fill=fillColor,fill opacity=0.20] (224.65, 73.70) circle (  2.13);

\path[fill=fillColor,fill opacity=0.20] (222.68, 92.39) circle (  2.13);

\path[fill=fillColor,fill opacity=0.20] (219.74, 85.89) circle (  2.13);

\path[fill=fillColor,fill opacity=0.20] (213.84, 61.51) circle (  2.13);

\path[fill=fillColor,fill opacity=0.20] (213.84, 46.88) circle (  2.13);

\path[fill=fillColor,fill opacity=0.20] (212.86, 58.26) circle (  2.13);

\path[fill=fillColor,fill opacity=0.20] (214.82, 63.95) circle (  2.13);

\path[fill=fillColor,fill opacity=0.20] (215.81, 69.63) circle (  2.13);

\path[fill=fillColor,fill opacity=0.20] (214.82, 81.01) circle (  2.13);

\path[fill=fillColor,fill opacity=0.20] (216.79, 94.02) circle (  2.13);

\path[fill=fillColor,fill opacity=0.20] (217.77, 81.01) circle (  2.13);

\path[fill=fillColor,fill opacity=0.20] (214.82, 71.26) circle (  2.13);

\path[fill=fillColor,fill opacity=0.20] (210.89, 55.82) circle (  2.13);

\path[fill=fillColor,fill opacity=0.20] (210.89, 59.88) circle (  2.13);

\path[fill=fillColor,fill opacity=0.20] (207.94, 67.20) circle (  2.13);

\path[fill=fillColor,fill opacity=0.20] (211.87, 70.45) circle (  2.13);

\path[fill=fillColor,fill opacity=0.20] (208.93, 79.39) circle (  2.13);

\path[fill=fillColor,fill opacity=0.20] (210.89, 96.45) circle (  2.13);

\path[fill=fillColor,fill opacity=0.20] (212.86,101.33) circle (  2.13);

\path[fill=fillColor,fill opacity=0.20] (213.84, 85.08) circle (  2.13);

\path[fill=fillColor,fill opacity=0.20] (215.81, 74.51) circle (  2.13);

\path[fill=fillColor,fill opacity=0.20] (222.68, 80.20) circle (  2.13);

\path[fill=fillColor,fill opacity=0.20] (219.74, 79.39) circle (  2.13);

\path[fill=fillColor,fill opacity=0.20] (222.68, 72.07) circle (  2.13);

\path[fill=fillColor,fill opacity=0.20] (224.65, 77.76) circle (  2.13);

\path[fill=fillColor,fill opacity=0.20] (219.74, 91.58) circle (  2.13);

\path[fill=fillColor,fill opacity=0.20] (217.77,109.46) circle (  2.13);

\path[fill=fillColor,fill opacity=0.20] (214.82, 89.95) circle (  2.13);

\path[fill=fillColor,fill opacity=0.20] (213.84, 81.01) circle (  2.13);

\path[fill=fillColor,fill opacity=0.20] (215.81, 70.45) circle (  2.13);

\path[fill=fillColor,fill opacity=0.20] (217.77, 64.76) circle (  2.13);

\path[fill=fillColor,fill opacity=0.20] (214.82, 61.51) circle (  2.13);

\path[fill=fillColor,fill opacity=0.20] (192.22, 50.94) circle (  2.13);

\path[fill=fillColor,fill opacity=0.20] (225.63, 63.95) circle (  2.13);

\path[fill=fillColor,fill opacity=0.20] (218.75, 96.45) circle (  2.13);

\path[fill=fillColor,fill opacity=0.20] (225.63,107.02) circle (  2.13);

\path[fill=fillColor,fill opacity=0.20] (215.81, 58.26) circle (  2.13);

\path[fill=fillColor,fill opacity=0.20] (206.96, 52.57) circle (  2.13);

\path[fill=fillColor,fill opacity=0.20] (209.91, 49.32) circle (  2.13);

\path[fill=fillColor,fill opacity=0.20] (210.89, 50.13) circle (  2.13);

\path[fill=fillColor,fill opacity=0.20] (211.87, 63.13) circle (  2.13);

\path[fill=fillColor,fill opacity=0.20] (210.89, 81.82) circle (  2.13);

\path[fill=fillColor,fill opacity=0.20] (208.93, 94.02) circle (  2.13);

\path[fill=fillColor,fill opacity=0.20] (214.82, 94.83) circle (  2.13);

\path[fill=fillColor,fill opacity=0.20] (214.82, 98.89) circle (  2.13);

\path[fill=fillColor,fill opacity=0.20] (213.84, 98.89) circle (  2.13);

\path[fill=fillColor,fill opacity=0.20] (215.81, 81.82) circle (  2.13);

\path[fill=fillColor,fill opacity=0.20] (221.70, 69.63) circle (  2.13);

\path[fill=fillColor,fill opacity=0.20] (220.72, 78.57) circle (  2.13);

\path[fill=fillColor,fill opacity=0.20] (216.79, 81.01) circle (  2.13);

\path[fill=fillColor,fill opacity=0.20] (219.74, 68.82) circle (  2.13);

\path[fill=fillColor,fill opacity=0.20] (222.68, 66.38) circle (  2.13);

\path[fill=fillColor,fill opacity=0.20] (224.65, 79.39) circle (  2.13);

\path[fill=fillColor,fill opacity=0.20] (223.67, 95.64) circle (  2.13);

\path[fill=fillColor,fill opacity=0.20] (219.74,102.96) circle (  2.13);

\path[fill=fillColor,fill opacity=0.20] (217.77, 99.70) circle (  2.13);

\path[fill=fillColor,fill opacity=0.20] (215.81, 89.95) circle (  2.13);

\path[fill=fillColor,fill opacity=0.20] (217.77, 94.83) circle (  2.13);

\path[fill=fillColor,fill opacity=0.20] (222.68, 98.89) circle (  2.13);

\path[fill=fillColor,fill opacity=0.20] (216.79, 83.45) circle (  2.13);

\path[fill=fillColor,fill opacity=0.20] (217.77, 76.95) circle (  2.13);

\path[fill=fillColor,fill opacity=0.20] (215.81, 73.70) circle (  2.13);

\path[fill=fillColor,fill opacity=0.20] (213.84, 77.76) circle (  2.13);

\path[fill=fillColor,fill opacity=0.20] (212.86, 76.14) circle (  2.13);

\path[fill=fillColor,fill opacity=0.20] (216.79, 64.76) circle (  2.13);

\path[fill=fillColor,fill opacity=0.20] (214.82, 61.51) circle (  2.13);

\path[fill=fillColor,fill opacity=0.20] (215.81, 54.19) circle (  2.13);

\path[fill=fillColor,fill opacity=0.20] (220.72, 51.75) circle (  2.13);

\path[fill=fillColor,fill opacity=0.20] (222.68, 74.51) circle (  2.13);

\path[fill=fillColor,fill opacity=0.20] (209.91, 46.88) circle (  2.13);

\path[fill=fillColor,fill opacity=0.20] (207.94, 50.13) circle (  2.13);

\path[fill=fillColor,fill opacity=0.20] (210.89, 61.51) circle (  2.13);

\path[fill=fillColor,fill opacity=0.20] (214.82, 67.20) circle (  2.13);

\path[fill=fillColor,fill opacity=0.20] (216.79, 68.82) circle (  2.13);

\path[fill=fillColor,fill opacity=0.20] (215.81, 73.70) circle (  2.13);

\path[fill=fillColor,fill opacity=0.20] (211.87, 81.01) circle (  2.13);

\path[fill=fillColor,fill opacity=0.20] (210.89, 81.01) circle (  2.13);

\path[fill=fillColor,fill opacity=0.20] (214.82, 72.89) circle (  2.13);

\path[fill=fillColor,fill opacity=0.20] (217.77, 74.51) circle (  2.13);

\path[fill=fillColor,fill opacity=0.20] (217.77, 83.45) circle (  2.13);

\path[fill=fillColor,fill opacity=0.20] (215.81, 84.26) circle (  2.13);

\path[fill=fillColor,fill opacity=0.20] (216.79, 75.32) circle (  2.13);

\path[fill=fillColor,fill opacity=0.20] (221.70, 70.45) circle (  2.13);

\path[fill=fillColor,fill opacity=0.20] (218.75, 78.57) circle (  2.13);

\path[fill=fillColor,fill opacity=0.20] (216.79, 85.08) circle (  2.13);

\path[fill=fillColor,fill opacity=0.20] (214.82, 81.82) circle (  2.13);

\path[fill=fillColor,fill opacity=0.20] (217.77, 73.70) circle (  2.13);

\path[fill=fillColor,fill opacity=0.20] (217.77, 72.89) circle (  2.13);

\path[fill=fillColor,fill opacity=0.20] (215.81, 82.64) circle (  2.13);

\path[fill=fillColor,fill opacity=0.20] (211.87, 85.08) circle (  2.13);

\path[fill=fillColor,fill opacity=0.20] (215.81, 81.82) circle (  2.13);

\path[fill=fillColor,fill opacity=0.20] (216.79, 81.82) circle (  2.13);

\path[fill=fillColor,fill opacity=0.20] (215.81, 74.51) circle (  2.13);

\path[fill=fillColor,fill opacity=0.20] (215.81, 70.45) circle (  2.13);

\path[fill=fillColor,fill opacity=0.20] (214.82, 79.39) circle (  2.13);

\path[fill=fillColor,fill opacity=0.20] (209.91, 76.95) circle (  2.13);

\path[fill=fillColor,fill opacity=0.20] (217.77, 65.57) circle (  2.13);

\path[fill=fillColor,fill opacity=0.20] (212.86, 63.95) circle (  2.13);

\path[fill=fillColor,fill opacity=0.20] (214.82, 59.07) circle (  2.13);

\path[fill=fillColor,fill opacity=0.20] (202.05, 53.38) circle (  2.13);

\path[fill=fillColor,fill opacity=0.20] (213.84, 61.51) circle (  2.13);

\path[fill=fillColor,fill opacity=0.20] (215.81, 64.76) circle (  2.13);

\path[fill=fillColor,fill opacity=0.20] (218.75, 63.13) circle (  2.13);

\path[fill=fillColor,fill opacity=0.20] (218.75, 68.01) circle (  2.13);

\path[fill=fillColor,fill opacity=0.20] (221.70, 83.45) circle (  2.13);

\path[fill=fillColor,fill opacity=0.20] (216.79, 55.01) circle (  2.13);

\path[fill=fillColor,fill opacity=0.20] (214.82, 55.82) circle (  2.13);

\path[fill=fillColor,fill opacity=0.20] (211.87, 55.01) circle (  2.13);

\path[fill=fillColor,fill opacity=0.20] (209.91, 61.51) circle (  2.13);

\path[fill=fillColor,fill opacity=0.20] (210.89, 63.95) circle (  2.13);

\path[fill=fillColor,fill opacity=0.20] (213.84, 65.57) circle (  2.13);

\path[fill=fillColor,fill opacity=0.20] (215.81, 73.70) circle (  2.13);

\path[fill=fillColor,fill opacity=0.20] (216.79, 77.76) circle (  2.13);

\path[fill=fillColor,fill opacity=0.20] (214.82, 72.07) circle (  2.13);

\path[fill=fillColor,fill opacity=0.20] (212.86, 72.89) circle (  2.13);

\path[fill=fillColor,fill opacity=0.20] (216.79, 79.39) circle (  2.13);

\path[fill=fillColor,fill opacity=0.20] (212.86, 84.26) circle (  2.13);

\path[fill=fillColor,fill opacity=0.20] (208.93, 88.33) circle (  2.13);

\path[fill=fillColor,fill opacity=0.20] (212.86, 80.20) circle (  2.13);

\path[fill=fillColor,fill opacity=0.20] (216.79, 71.26) circle (  2.13);

\path[fill=fillColor,fill opacity=0.20] (215.81, 81.01) circle (  2.13);

\path[fill=fillColor,fill opacity=0.20] (213.84, 87.51) circle (  2.13);

\path[fill=fillColor,fill opacity=0.20] (213.84, 82.64) circle (  2.13);

\path[fill=fillColor,fill opacity=0.20] (215.81, 83.45) circle (  2.13);

\path[fill=fillColor,fill opacity=0.20] (214.82, 81.82) circle (  2.13);

\path[fill=fillColor,fill opacity=0.20] (213.84, 73.70) circle (  2.13);

\path[fill=fillColor,fill opacity=0.20] (211.87, 76.14) circle (  2.13);

\path[fill=fillColor,fill opacity=0.20] (210.89, 74.51) circle (  2.13);

\path[fill=fillColor,fill opacity=0.20] (209.91, 63.95) circle (  2.13);

\path[fill=fillColor,fill opacity=0.20] (210.89, 63.13) circle (  2.13);

\path[fill=fillColor,fill opacity=0.20] (211.87, 61.51) circle (  2.13);

\path[fill=fillColor,fill opacity=0.20] (216.79, 59.07) circle (  2.13);

\path[fill=fillColor,fill opacity=0.20] (218.75, 76.14) circle (  2.13);

\path[fill=fillColor,fill opacity=0.20] (208.93, 58.26) circle (  2.13);

\path[fill=fillColor,fill opacity=0.20] (209.91, 65.57) circle (  2.13);

\path[fill=fillColor,fill opacity=0.20] (210.89, 63.95) circle (  2.13);

\path[fill=fillColor,fill opacity=0.20] (214.82, 50.94) circle (  2.13);

\path[fill=fillColor,fill opacity=0.20] (218.75, 60.69) circle (  2.13);

\path[fill=fillColor,fill opacity=0.20] (215.81, 71.26) circle (  2.13);

\path[fill=fillColor,fill opacity=0.20] (212.86, 66.38) circle (  2.13);

\path[fill=fillColor,fill opacity=0.20] (213.84, 59.88) circle (  2.13);

\path[fill=fillColor,fill opacity=0.20] (214.82, 61.51) circle (  2.13);

\path[fill=fillColor,fill opacity=0.20] (210.89, 67.20) circle (  2.13);

\path[fill=fillColor,fill opacity=0.20] (210.89, 70.45) circle (  2.13);

\path[fill=fillColor,fill opacity=0.20] (210.89, 72.89) circle (  2.13);

\path[fill=fillColor,fill opacity=0.20] (211.87, 68.01) circle (  2.13);

\path[fill=fillColor,fill opacity=0.20] (213.84, 64.76) circle (  2.13);

\path[fill=fillColor,fill opacity=0.20] (212.86, 69.63) circle (  2.13);

\path[fill=fillColor,fill opacity=0.20] (212.86, 72.89) circle (  2.13);

\path[fill=fillColor,fill opacity=0.20] (212.86, 73.70) circle (  2.13);

\path[fill=fillColor,fill opacity=0.20] (210.89, 77.76) circle (  2.13);

\path[fill=fillColor,fill opacity=0.20] (210.89, 79.39) circle (  2.13);

\path[fill=fillColor,fill opacity=0.20] (210.89, 74.51) circle (  2.13);

\path[fill=fillColor,fill opacity=0.20] (209.91, 69.63) circle (  2.13);

\path[fill=fillColor,fill opacity=0.20] (208.93, 68.82) circle (  2.13);

\path[fill=fillColor,fill opacity=0.20] (210.89, 68.82) circle (  2.13);

\path[fill=fillColor,fill opacity=0.20] (213.84, 72.07) circle (  2.13);

\path[fill=fillColor,fill opacity=0.20] (221.70, 51.75) circle (  2.13);

\path[fill=fillColor,fill opacity=0.20] (216.79, 74.51) circle (  2.13);

\path[fill=fillColor,fill opacity=0.20] (213.84, 75.32) circle (  2.13);

\path[fill=fillColor,fill opacity=0.20] (216.79, 64.76) circle (  2.13);

\path[fill=fillColor,fill opacity=0.20] (216.79, 66.38) circle (  2.13);

\path[fill=fillColor,fill opacity=0.20] (213.84, 66.38) circle (  2.13);

\path[fill=fillColor,fill opacity=0.20] (212.86, 56.63) circle (  2.13);

\path[fill=fillColor,fill opacity=0.20] (210.89, 54.19) circle (  2.13);

\path[fill=fillColor,fill opacity=0.20] (210.89, 62.32) circle (  2.13);

\path[fill=fillColor,fill opacity=0.20] (212.86, 64.76) circle (  2.13);

\path[fill=fillColor,fill opacity=0.20] (213.84, 66.38) circle (  2.13);

\path[fill=fillColor,fill opacity=0.20] (215.81, 68.82) circle (  2.13);

\path[fill=fillColor,fill opacity=0.20] (213.84, 68.82) circle (  2.13);

\path[fill=fillColor,fill opacity=0.20] (210.89, 73.70) circle (  2.13);

\path[fill=fillColor,fill opacity=0.20] (211.87, 77.76) circle (  2.13);

\path[fill=fillColor,fill opacity=0.20] (216.79, 72.89) circle (  2.13);

\path[fill=fillColor,fill opacity=0.20] (216.79, 76.14) circle (  2.13);

\path[fill=fillColor,fill opacity=0.20] (218.75, 76.14) circle (  2.13);

\path[fill=fillColor,fill opacity=0.20] (217.77, 66.38) circle (  2.13);

\path[fill=fillColor,fill opacity=0.20] (212.86, 62.32) circle (  2.13);

\path[fill=fillColor,fill opacity=0.20] (211.87, 70.45) circle (  2.13);

\path[fill=fillColor,fill opacity=0.20] (205.00, 81.82) circle (  2.13);

\path[fill=fillColor,fill opacity=0.20] (214.82, 87.51) circle (  2.13);

\path[fill=fillColor,fill opacity=0.20] (216.79, 82.64) circle (  2.13);

\path[fill=fillColor,fill opacity=0.20] (204.01, 85.89) circle (  2.13);

\path[fill=fillColor,fill opacity=0.20] (202.05, 94.02) circle (  2.13);

\path[fill=fillColor,fill opacity=0.20] (218.75, 85.89) circle (  2.13);

\path[fill=fillColor,fill opacity=0.20] (223.67, 82.64) circle (  2.13);

\path[fill=fillColor,fill opacity=0.20] (209.91, 49.32) circle (  2.13);

\path[fill=fillColor,fill opacity=0.20] (213.84, 50.13) circle (  2.13);

\path[fill=fillColor,fill opacity=0.20] (213.84, 49.32) circle (  2.13);

\path[fill=fillColor,fill opacity=0.20] (215.81, 50.13) circle (  2.13);

\path[fill=fillColor,fill opacity=0.20] (211.87, 58.26) circle (  2.13);

\path[fill=fillColor,fill opacity=0.20] (206.96, 62.32) circle (  2.13);

\path[fill=fillColor,fill opacity=0.20] (205.98, 63.95) circle (  2.13);

\path[fill=fillColor,fill opacity=0.20] (210.89, 61.51) circle (  2.13);

\path[fill=fillColor,fill opacity=0.20] (212.86, 55.82) circle (  2.13);

\path[fill=fillColor,fill opacity=0.20] (215.81, 55.01) circle (  2.13);

\path[fill=fillColor,fill opacity=0.20] (222.68, 56.63) circle (  2.13);

\path[fill=fillColor,fill opacity=0.20] (229.56, 55.82) circle (  2.13);

\path[fill=fillColor,fill opacity=0.20] (207.94, 55.01) circle (  2.13);

\path[fill=fillColor,fill opacity=0.20] (208.93, 66.38) circle (  2.13);

\path[fill=fillColor,fill opacity=0.20] (200.08, 75.32) circle (  2.13);

\path[fill=fillColor,fill opacity=0.20] (205.00, 78.57) circle (  2.13);

\path[fill=fillColor,fill opacity=0.20] (205.98, 76.14) circle (  2.13);

\path[fill=fillColor,fill opacity=0.20] (209.91, 72.07) circle (  2.13);

\path[fill=fillColor,fill opacity=0.20] (212.86, 69.63) circle (  2.13);

\path[fill=fillColor,fill opacity=0.20] (215.81, 68.01) circle (  2.13);

\path[fill=fillColor,fill opacity=0.20] (221.70, 68.01) circle (  2.13);

\path[fill=fillColor,fill opacity=0.20] (240.37, 65.57) circle (  2.13);

\path[fill=fillColor,fill opacity=0.20] (248.23, 66.38) circle (  2.13);

\path[fill=fillColor,fill opacity=0.20] (213.84, 67.20) circle (  2.13);

\path[fill=fillColor,fill opacity=0.20] (208.93, 69.63) circle (  2.13);

\path[fill=fillColor,fill opacity=0.20] (203.03, 76.14) circle (  2.13);

\path[fill=fillColor,fill opacity=0.20] (205.98, 86.70) circle (  2.13);

\path[fill=fillColor,fill opacity=0.20] (205.98, 89.95) circle (  2.13);

\path[fill=fillColor,fill opacity=0.20] (207.94, 87.51) circle (  2.13);

\path[fill=fillColor,fill opacity=0.20] (213.84, 85.08) circle (  2.13);

\path[fill=fillColor,fill opacity=0.20] (212.86, 83.45) circle (  2.13);

\path[fill=fillColor,fill opacity=0.20] (218.75, 80.20) circle (  2.13);

\path[fill=fillColor,fill opacity=0.20] (224.65, 81.01) circle (  2.13);

\path[fill=fillColor,fill opacity=0.20] (245.29, 85.08) circle (  2.13);

\path[fill=fillColor,fill opacity=0.20] (227.60, 59.88) circle (  2.13);

\path[fill=fillColor,fill opacity=0.20] (212.86, 69.63) circle (  2.13);

\path[fill=fillColor,fill opacity=0.20] (210.89, 77.76) circle (  2.13);

\path[fill=fillColor,fill opacity=0.20] (210.89, 84.26) circle (  2.13);

\path[fill=fillColor,fill opacity=0.20] (209.91, 90.76) circle (  2.13);

\path[fill=fillColor,fill opacity=0.20] (209.91, 94.02) circle (  2.13);

\path[fill=fillColor,fill opacity=0.20] (211.87, 95.64) circle (  2.13);

\path[fill=fillColor,fill opacity=0.20] (215.81, 94.02) circle (  2.13);

\path[fill=fillColor,fill opacity=0.20] (222.68, 88.33) circle (  2.13);

\path[fill=fillColor,fill opacity=0.20] (221.70, 83.45) circle (  2.13);

\path[fill=fillColor,fill opacity=0.20] (236.44, 87.51) circle (  2.13);

\path[fill=fillColor,fill opacity=0.20] (214.82, 62.32) circle (  2.13);

\path[fill=fillColor,fill opacity=0.20] (213.84, 78.57) circle (  2.13);

\path[fill=fillColor,fill opacity=0.20] (213.84, 92.39) circle (  2.13);

\path[fill=fillColor,fill opacity=0.20] (212.86, 90.76) circle (  2.13);

\path[fill=fillColor,fill opacity=0.20] (210.89, 91.58) circle (  2.13);

\path[fill=fillColor,fill opacity=0.20] (212.86, 95.64) circle (  2.13);

\path[fill=fillColor,fill opacity=0.20] (216.79, 94.02) circle (  2.13);

\path[fill=fillColor,fill opacity=0.20] (226.62, 86.70) circle (  2.13);

\path[fill=fillColor,fill opacity=0.20] (232.51, 84.26) circle (  2.13);

\path[fill=fillColor,fill opacity=0.20] (245.29, 89.95) circle (  2.13);

\path[fill=fillColor,fill opacity=0.20] (214.82, 59.07) circle (  2.13);

\path[fill=fillColor,fill opacity=0.20] (214.82, 77.76) circle (  2.13);

\path[fill=fillColor,fill opacity=0.20] (211.87, 94.83) circle (  2.13);

\path[fill=fillColor,fill opacity=0.20] (209.91, 90.76) circle (  2.13);

\path[fill=fillColor,fill opacity=0.20] (214.82, 89.14) circle (  2.13);

\path[fill=fillColor,fill opacity=0.20] (216.79, 93.20) circle (  2.13);

\path[fill=fillColor,fill opacity=0.20] (217.77, 89.95) circle (  2.13);

\path[fill=fillColor,fill opacity=0.20] (227.60, 86.70) circle (  2.13);

\path[fill=fillColor,fill opacity=0.20] (229.56, 96.45) circle (  2.13);

\path[fill=fillColor,fill opacity=0.20] (209.91, 37.94) circle (  2.13);

\path[fill=fillColor,fill opacity=0.20] (208.93, 37.94) circle (  2.13);

\path[fill=fillColor,fill opacity=0.20] (208.93, 40.38) circle (  2.13);

\path[fill=fillColor,fill opacity=0.20] (211.87, 42.00) circle (  2.13);

\path[fill=fillColor,fill opacity=0.20] (211.87, 63.13) circle (  2.13);

\path[fill=fillColor,fill opacity=0.20] (209.91, 77.76) circle (  2.13);

\path[fill=fillColor,fill opacity=0.20] (211.87, 93.20) circle (  2.13);

\path[fill=fillColor,fill opacity=0.20] (210.89, 93.20) circle (  2.13);

\path[fill=fillColor,fill opacity=0.20] (210.89, 94.02) circle (  2.13);

\path[fill=fillColor,fill opacity=0.20] (219.74, 94.83) circle (  2.13);

\path[fill=fillColor,fill opacity=0.20] (220.72, 89.95) circle (  2.13);

\path[fill=fillColor,fill opacity=0.20] (226.62, 89.14) circle (  2.13);

\path[fill=fillColor,fill opacity=0.20] (205.98, 48.50) circle (  2.13);

\path[fill=fillColor,fill opacity=0.20] (210.89, 47.69) circle (  2.13);

\path[fill=fillColor,fill opacity=0.20] (213.84, 43.63) circle (  2.13);

\path[fill=fillColor,fill opacity=0.20] (213.84, 59.88) circle (  2.13);

\path[fill=fillColor,fill opacity=0.20] (205.98, 74.51) circle (  2.13);

\path[fill=fillColor,fill opacity=0.20] (207.94, 89.95) circle (  2.13);

\path[fill=fillColor,fill opacity=0.20] (209.91, 97.27) circle (  2.13);

\path[fill=fillColor,fill opacity=0.20] (215.81, 97.27) circle (  2.13);

\path[fill=fillColor,fill opacity=0.20] (221.70, 94.02) circle (  2.13);

\path[fill=fillColor,fill opacity=0.20] (221.70, 89.14) circle (  2.13);

\path[fill=fillColor,fill opacity=0.20] (222.68, 89.95) circle (  2.13);

\path[fill=fillColor,fill opacity=0.20] (215.81, 56.63) circle (  2.13);

\path[fill=fillColor,fill opacity=0.20] (206.96, 56.63) circle (  2.13);

\path[fill=fillColor,fill opacity=0.20] (210.89, 56.63) circle (  2.13);

\path[fill=fillColor,fill opacity=0.20] (216.79, 56.63) circle (  2.13);

\path[fill=fillColor,fill opacity=0.20] (213.84, 51.75) circle (  2.13);

\path[fill=fillColor,fill opacity=0.20] (210.89, 39.56) circle (  2.13);

\path[fill=fillColor,fill opacity=0.20] (224.65, 55.01) circle (  2.13);

\path[fill=fillColor,fill opacity=0.20] (205.98, 65.57) circle (  2.13);

\path[fill=fillColor,fill opacity=0.20] (210.89, 85.08) circle (  2.13);

\path[fill=fillColor,fill opacity=0.20] (209.91, 96.45) circle (  2.13);

\path[fill=fillColor,fill opacity=0.20] (213.84, 98.08) circle (  2.13);

\path[fill=fillColor,fill opacity=0.20] (216.79, 88.33) circle (  2.13);

\path[fill=fillColor,fill opacity=0.20] (220.72, 82.64) circle (  2.13);

\path[fill=fillColor,fill opacity=0.20] (220.72, 89.14) circle (  2.13);

\path[fill=fillColor,fill opacity=0.20] (221.70, 65.57) circle (  2.13);

\path[fill=fillColor,fill opacity=0.20] (218.75, 63.13) circle (  2.13);

\path[fill=fillColor,fill opacity=0.20] (219.74, 59.07) circle (  2.13);

\path[fill=fillColor,fill opacity=0.20] (217.77, 59.88) circle (  2.13);

\path[fill=fillColor,fill opacity=0.20] (220.72, 60.69) circle (  2.13);

\path[fill=fillColor,fill opacity=0.20] (209.91, 59.07) circle (  2.13);

\path[fill=fillColor,fill opacity=0.20] (217.77, 49.32) circle (  2.13);

\path[fill=fillColor,fill opacity=0.20] (216.79, 39.56) circle (  2.13);

\path[fill=fillColor,fill opacity=0.20] (214.82, 63.95) circle (  2.13);

\path[fill=fillColor,fill opacity=0.20] (213.84, 61.51) circle (  2.13);

\path[fill=fillColor,fill opacity=0.20] (210.89, 75.32) circle (  2.13);

\path[fill=fillColor,fill opacity=0.20] (213.84, 89.14) circle (  2.13);

\path[fill=fillColor,fill opacity=0.20] (215.81, 96.45) circle (  2.13);

\path[fill=fillColor,fill opacity=0.20] (216.79, 89.95) circle (  2.13);

\path[fill=fillColor,fill opacity=0.20] (216.79, 81.01) circle (  2.13);

\path[fill=fillColor,fill opacity=0.20] (215.81, 89.14) circle (  2.13);

\path[fill=fillColor,fill opacity=0.20] (224.65,103.77) circle (  2.13);

\path[fill=fillColor,fill opacity=0.20] (214.82, 78.57) circle (  2.13);

\path[fill=fillColor,fill opacity=0.20] (218.75, 73.70) circle (  2.13);

\path[fill=fillColor,fill opacity=0.20] (218.75, 67.20) circle (  2.13);

\path[fill=fillColor,fill opacity=0.20] (223.67, 62.32) circle (  2.13);

\path[fill=fillColor,fill opacity=0.20] (222.68, 60.69) circle (  2.13);

\path[fill=fillColor,fill opacity=0.20] (224.65, 61.51) circle (  2.13);

\path[fill=fillColor,fill opacity=0.20] (224.65, 50.94) circle (  2.13);

\path[fill=fillColor,fill opacity=0.20] (233.49, 37.94) circle (  2.13);

\path[fill=fillColor,fill opacity=0.20] (235.46, 59.07) circle (  2.13);

\path[fill=fillColor,fill opacity=0.20] (214.82, 64.76) circle (  2.13);

\path[fill=fillColor,fill opacity=0.20] (211.87, 80.20) circle (  2.13);

\path[fill=fillColor,fill opacity=0.20] (205.00, 94.02) circle (  2.13);

\path[fill=fillColor,fill opacity=0.20] (218.75, 94.83) circle (  2.13);

\path[fill=fillColor,fill opacity=0.20] (219.74, 85.89) circle (  2.13);

\path[fill=fillColor,fill opacity=0.20] (216.79, 85.89) circle (  2.13);

\path[fill=fillColor,fill opacity=0.20] (219.74, 93.20) circle (  2.13);

\path[fill=fillColor,fill opacity=0.20] (211.87, 81.01) circle (  2.13);

\path[fill=fillColor,fill opacity=0.20] (215.81, 76.95) circle (  2.13);

\path[fill=fillColor,fill opacity=0.20] (215.81, 70.45) circle (  2.13);

\path[fill=fillColor,fill opacity=0.20] (221.70, 63.95) circle (  2.13);

\path[fill=fillColor,fill opacity=0.20] (217.77, 61.51) circle (  2.13);

\path[fill=fillColor,fill opacity=0.20] (218.75, 63.13) circle (  2.13);

\path[fill=fillColor,fill opacity=0.20] (223.67, 54.19) circle (  2.13);

\path[fill=fillColor,fill opacity=0.20] (219.74, 38.75) circle (  2.13);

\path[fill=fillColor,fill opacity=0.20] (225.63, 39.56) circle (  2.13);

\path[fill=fillColor,fill opacity=0.20] (269.85, 50.94) circle (  2.13);

\path[fill=fillColor,fill opacity=0.20] (240.37, 57.44) circle (  2.13);

\path[fill=fillColor,fill opacity=0.20] (219.74, 72.07) circle (  2.13);

\path[fill=fillColor,fill opacity=0.20] (201.07, 79.39) circle (  2.13);

\path[fill=fillColor,fill opacity=0.20] (217.77, 87.51) circle (  2.13);

\path[fill=fillColor,fill opacity=0.20] (219.74, 88.33) circle (  2.13);

\path[fill=fillColor,fill opacity=0.20] (214.82, 80.20) circle (  2.13);

\path[fill=fillColor,fill opacity=0.20] (218.75, 81.82) circle (  2.13);

\path[fill=fillColor,fill opacity=0.20] (223.67,102.96) circle (  2.13);

\path[fill=fillColor,fill opacity=0.20] (217.77, 76.95) circle (  2.13);

\path[fill=fillColor,fill opacity=0.20] (211.87, 70.45) circle (  2.13);

\path[fill=fillColor,fill opacity=0.20] (210.89, 68.82) circle (  2.13);

\path[fill=fillColor,fill opacity=0.20] (218.75, 66.38) circle (  2.13);

\path[fill=fillColor,fill opacity=0.20] (218.75, 65.57) circle (  2.13);

\path[fill=fillColor,fill opacity=0.20] (213.84, 60.69) circle (  2.13);

\path[fill=fillColor,fill opacity=0.20] (219.74, 48.50) circle (  2.13);

\path[fill=fillColor,fill opacity=0.20] (232.51, 42.82) circle (  2.13);

\path[fill=fillColor,fill opacity=0.20] (236.44, 58.26) circle (  2.13);

\path[fill=fillColor,fill opacity=0.20] (218.75, 59.88) circle (  2.13);

\path[fill=fillColor,fill opacity=0.20] (215.81, 70.45) circle (  2.13);

\path[fill=fillColor,fill opacity=0.20] (212.86, 82.64) circle (  2.13);

\path[fill=fillColor,fill opacity=0.20] (217.77, 79.39) circle (  2.13);

\path[fill=fillColor,fill opacity=0.20] (216.79, 80.20) circle (  2.13);

\path[fill=fillColor,fill opacity=0.20] (216.79, 90.76) circle (  2.13);

\path[fill=fillColor,fill opacity=0.20] (210.89, 89.14) circle (  2.13);

\path[fill=fillColor,fill opacity=0.20] (208.93, 80.20) circle (  2.13);

\path[fill=fillColor,fill opacity=0.20] (207.94, 70.45) circle (  2.13);

\path[fill=fillColor,fill opacity=0.20] (208.93, 72.89) circle (  2.13);

\path[fill=fillColor,fill opacity=0.20] (214.82, 70.45) circle (  2.13);

\path[fill=fillColor,fill opacity=0.20] (211.87, 64.76) circle (  2.13);

\path[fill=fillColor,fill opacity=0.20] (211.87, 65.57) circle (  2.13);

\path[fill=fillColor,fill opacity=0.20] (214.82, 61.51) circle (  2.13);

\path[fill=fillColor,fill opacity=0.20] (217.77, 47.69) circle (  2.13);

\path[fill=fillColor,fill opacity=0.20] (227.60, 42.82) circle (  2.13);

\path[fill=fillColor,fill opacity=0.20] (264.94, 54.19) circle (  2.13);

\path[fill=fillColor,fill opacity=0.20] (233.49, 52.57) circle (  2.13);

\path[fill=fillColor,fill opacity=0.20] (213.84, 62.32) circle (  2.13);

\path[fill=fillColor,fill opacity=0.20] (206.96, 75.32) circle (  2.13);

\path[fill=fillColor,fill opacity=0.20] (211.87, 80.20) circle (  2.13);

\path[fill=fillColor,fill opacity=0.20] (215.81, 79.39) circle (  2.13);

\path[fill=fillColor,fill opacity=0.20] (214.82, 81.82) circle (  2.13);

\path[fill=fillColor,fill opacity=0.20] (215.81, 88.33) circle (  2.13);

\path[fill=fillColor,fill opacity=0.20] (202.05, 98.08) circle (  2.13);

\path[fill=fillColor,fill opacity=0.20] (204.01, 87.51) circle (  2.13);

\path[fill=fillColor,fill opacity=0.20] (204.01, 78.57) circle (  2.13);

\path[fill=fillColor,fill opacity=0.20] (205.98, 76.14) circle (  2.13);

\path[fill=fillColor,fill opacity=0.20] (205.98, 70.45) circle (  2.13);

\path[fill=fillColor,fill opacity=0.20] (209.91, 64.76) circle (  2.13);

\path[fill=fillColor,fill opacity=0.20] (212.86, 62.32) circle (  2.13);

\path[fill=fillColor,fill opacity=0.20] (218.75, 52.57) circle (  2.13);

\path[fill=fillColor,fill opacity=0.20] (218.75, 37.94) circle (  2.13);

\path[fill=fillColor,fill opacity=0.20] (226.62, 39.56) circle (  2.13);

\path[fill=fillColor,fill opacity=0.20] (245.29, 57.44) circle (  2.13);

\path[fill=fillColor,fill opacity=0.20] (219.74, 59.07) circle (  2.13);

\path[fill=fillColor,fill opacity=0.20] (213.84, 66.38) circle (  2.13);

\path[fill=fillColor,fill opacity=0.20] (211.87, 72.89) circle (  2.13);

\path[fill=fillColor,fill opacity=0.20] (212.86, 74.51) circle (  2.13);

\path[fill=fillColor,fill opacity=0.20] (212.86, 78.57) circle (  2.13);

\path[fill=fillColor,fill opacity=0.20] (213.84, 85.08) circle (  2.13);

\path[fill=fillColor,fill opacity=0.20] (216.79, 87.51) circle (  2.13);

\path[fill=fillColor,fill opacity=0.20] (222.68,104.58) circle (  2.13);

\path[fill=fillColor,fill opacity=0.20] (203.03, 93.20) circle (  2.13);

\path[fill=fillColor,fill opacity=0.20] (202.05, 91.58) circle (  2.13);

\path[fill=fillColor,fill opacity=0.20] (204.01, 81.82) circle (  2.13);

\path[fill=fillColor,fill opacity=0.20] (204.01, 77.76) circle (  2.13);

\path[fill=fillColor,fill opacity=0.20] (205.00, 74.51) circle (  2.13);

\path[fill=fillColor,fill opacity=0.20] (203.03, 70.45) circle (  2.13);

\path[fill=fillColor,fill opacity=0.20] (204.01, 68.82) circle (  2.13);

\path[fill=fillColor,fill opacity=0.20] (205.98, 63.95) circle (  2.13);

\path[fill=fillColor,fill opacity=0.20] (212.86, 49.32) circle (  2.13);

\path[fill=fillColor,fill opacity=0.20] (266.90, 55.82) circle (  2.13);

\path[fill=fillColor,fill opacity=0.20] (252.16, 58.26) circle (  2.13);

\path[fill=fillColor,fill opacity=0.20] (223.67, 62.32) circle (  2.13);

\path[fill=fillColor,fill opacity=0.20] (213.84, 68.82) circle (  2.13);

\path[fill=fillColor,fill opacity=0.20] (213.84, 77.76) circle (  2.13);

\path[fill=fillColor,fill opacity=0.20] (213.84, 87.51) circle (  2.13);

\path[fill=fillColor,fill opacity=0.20] (214.82, 86.70) circle (  2.13);

\path[fill=fillColor,fill opacity=0.20] (219.74, 86.70) circle (  2.13);

\path[fill=fillColor,fill opacity=0.20] (202.05, 85.89) circle (  2.13);

\path[fill=fillColor,fill opacity=0.20] (199.10, 82.64) circle (  2.13);

\path[fill=fillColor,fill opacity=0.20] (201.07, 77.76) circle (  2.13);

\path[fill=fillColor,fill opacity=0.20] (205.00, 70.45) circle (  2.13);

\path[fill=fillColor,fill opacity=0.20] (205.00, 72.07) circle (  2.13);

\path[fill=fillColor,fill opacity=0.20] (205.00, 76.14) circle (  2.13);

\path[fill=fillColor,fill opacity=0.20] (208.93, 72.07) circle (  2.13);

\path[fill=fillColor,fill opacity=0.20] (207.94, 70.45) circle (  2.13);

\path[fill=fillColor,fill opacity=0.20] (208.93, 67.20) circle (  2.13);

\path[fill=fillColor,fill opacity=0.20] (210.89, 57.44) circle (  2.13);

\path[fill=fillColor,fill opacity=0.20] (216.79, 43.63) circle (  2.13);

\path[fill=fillColor,fill opacity=0.20] (266.90, 59.88) circle (  2.13);

\path[fill=fillColor,fill opacity=0.20] (240.37, 59.07) circle (  2.13);

\path[fill=fillColor,fill opacity=0.20] (227.60, 62.32) circle (  2.13);

\path[fill=fillColor,fill opacity=0.20] (217.77, 72.07) circle (  2.13);

\path[fill=fillColor,fill opacity=0.20] (213.84, 80.20) circle (  2.13);

\path[fill=fillColor,fill opacity=0.20] (214.82, 87.51) circle (  2.13);

\path[fill=fillColor,fill opacity=0.20] (213.84, 87.51) circle (  2.13);

\path[fill=fillColor,fill opacity=0.20] (213.84, 87.51) circle (  2.13);

\path[fill=fillColor,fill opacity=0.20] (198.12, 88.33) circle (  2.13);

\path[fill=fillColor,fill opacity=0.20] (196.15, 80.20) circle (  2.13);

\path[fill=fillColor,fill opacity=0.20] (191.24, 79.39) circle (  2.13);

\path[fill=fillColor,fill opacity=0.20] (200.08, 77.76) circle (  2.13);

\path[fill=fillColor,fill opacity=0.20] (204.01, 72.89) circle (  2.13);

\path[fill=fillColor,fill opacity=0.20] (202.05, 77.76) circle (  2.13);

\path[fill=fillColor,fill opacity=0.20] (205.98, 81.82) circle (  2.13);

\path[fill=fillColor,fill opacity=0.20] (211.87, 75.32) circle (  2.13);

\path[fill=fillColor,fill opacity=0.20] (209.91, 69.63) circle (  2.13);

\path[fill=fillColor,fill opacity=0.20] (207.94, 66.38) circle (  2.13);

\path[fill=fillColor,fill opacity=0.20] (213.84, 57.44) circle (  2.13);

\path[fill=fillColor,fill opacity=0.20] (248.23, 48.50) circle (  2.13);

\path[fill=fillColor,fill opacity=0.20] (228.58, 63.13) circle (  2.13);

\path[fill=fillColor,fill opacity=0.20] (208.93, 71.26) circle (  2.13);

\path[fill=fillColor,fill opacity=0.20] (209.91, 81.82) circle (  2.13);

\path[fill=fillColor,fill opacity=0.20] (212.86, 89.95) circle (  2.13);

\path[fill=fillColor,fill opacity=0.20] (218.75, 88.33) circle (  2.13);

\path[fill=fillColor,fill opacity=0.20] (217.77, 90.76) circle (  2.13);

\path[fill=fillColor,fill opacity=0.20] (197.13, 98.89) circle (  2.13);

\path[fill=fillColor,fill opacity=0.20] (194.19, 86.70) circle (  2.13);

\path[fill=fillColor,fill opacity=0.20] (194.19, 84.26) circle (  2.13);

\path[fill=fillColor,fill opacity=0.20] (189.76, 84.26) circle (  2.13);

\path[fill=fillColor,fill opacity=0.20] (200.08, 81.01) circle (  2.13);

\path[fill=fillColor,fill opacity=0.20] (203.03, 79.39) circle (  2.13);

\path[fill=fillColor,fill opacity=0.20] (201.07, 81.01) circle (  2.13);

\path[fill=fillColor,fill opacity=0.20] (206.96, 78.57) circle (  2.13);

\path[fill=fillColor,fill opacity=0.20] (212.86, 72.89) circle (  2.13);

\path[fill=fillColor,fill opacity=0.20] (214.82, 71.26) circle (  2.13);

\path[fill=fillColor,fill opacity=0.20] (216.79, 62.32) circle (  2.13);

\path[fill=fillColor,fill opacity=0.20] (231.53, 49.32) circle (  2.13);

\path[fill=fillColor,fill opacity=0.20] (247.25, 44.44) circle (  2.13);

\path[fill=fillColor,fill opacity=0.20] (236.44, 61.51) circle (  2.13);

\path[fill=fillColor,fill opacity=0.20] (216.79, 68.82) circle (  2.13);

\path[fill=fillColor,fill opacity=0.20] (218.75, 82.64) circle (  2.13);

\path[fill=fillColor,fill opacity=0.20] (211.87, 87.51) circle (  2.13);

\path[fill=fillColor,fill opacity=0.20] (211.87, 83.45) circle (  2.13);

\path[fill=fillColor,fill opacity=0.20] (212.86, 90.76) circle (  2.13);

\path[fill=fillColor,fill opacity=0.20] (195.17,115.15) circle (  2.13);

\path[fill=fillColor,fill opacity=0.20] (199.10, 97.27) circle (  2.13);

\path[fill=fillColor,fill opacity=0.20] (197.13, 85.89) circle (  2.13);

\path[fill=fillColor,fill opacity=0.20] (193.20, 85.08) circle (  2.13);

\path[fill=fillColor,fill opacity=0.20] (194.19, 85.08) circle (  2.13);

\path[fill=fillColor,fill opacity=0.20] (198.12, 85.08) circle (  2.13);

\path[fill=fillColor,fill opacity=0.20] (203.03, 80.20) circle (  2.13);

\path[fill=fillColor,fill opacity=0.20] (201.07, 80.20) circle (  2.13);

\path[fill=fillColor,fill opacity=0.20] (203.03, 81.01) circle (  2.13);

\path[fill=fillColor,fill opacity=0.20] (208.93, 69.63) circle (  2.13);

\path[fill=fillColor,fill opacity=0.20] (221.70, 62.32) circle (  2.13);

\path[fill=fillColor,fill opacity=0.20] (223.67, 63.95) circle (  2.13);

\path[fill=fillColor,fill opacity=0.20] (256.10, 57.44) circle (  2.13);

\path[fill=fillColor,fill opacity=0.20] (241.36, 42.82) circle (  2.13);

\path[fill=fillColor,fill opacity=0.20] (241.36, 56.63) circle (  2.13);

\path[fill=fillColor,fill opacity=0.20] (223.67, 68.01) circle (  2.13);

\path[fill=fillColor,fill opacity=0.20] (213.84, 77.76) circle (  2.13);

\path[fill=fillColor,fill opacity=0.20] (212.86, 81.82) circle (  2.13);

\path[fill=fillColor,fill opacity=0.20] (208.93, 83.45) circle (  2.13);

\path[fill=fillColor,fill opacity=0.20] (205.98, 84.26) circle (  2.13);

\path[fill=fillColor,fill opacity=0.20] (212.86, 95.64) circle (  2.13);

\path[fill=fillColor,fill opacity=0.20] (193.20,103.77) circle (  2.13);

\path[fill=fillColor,fill opacity=0.20] (193.20, 96.45) circle (  2.13);

\path[fill=fillColor,fill opacity=0.20] (197.13, 83.45) circle (  2.13);

\path[fill=fillColor,fill opacity=0.20] (197.13, 81.01) circle (  2.13);

\path[fill=fillColor,fill opacity=0.20] (194.19, 85.89) circle (  2.13);

\path[fill=fillColor,fill opacity=0.20] (196.15, 81.01) circle (  2.13);

\path[fill=fillColor,fill opacity=0.20] (189.47, 78.57) circle (  2.13);

\path[fill=fillColor,fill opacity=0.20] (201.07, 75.32) circle (  2.13);

\path[fill=fillColor,fill opacity=0.20] (202.05, 74.51) circle (  2.13);

\path[fill=fillColor,fill opacity=0.20] (210.89, 74.51) circle (  2.13);

\path[fill=fillColor,fill opacity=0.20] (219.74, 59.88) circle (  2.13);

\path[fill=fillColor,fill opacity=0.20] (237.42, 48.50) circle (  2.13);

\path[fill=fillColor,fill opacity=0.20] (259.04, 51.75) circle (  2.13);

\path[fill=fillColor,fill opacity=0.20] (239.39, 50.94) circle (  2.13);

\path[fill=fillColor,fill opacity=0.20] (252.16, 55.82) circle (  2.13);

\path[fill=fillColor,fill opacity=0.20] (225.63, 61.51) circle (  2.13);

\path[fill=fillColor,fill opacity=0.20] (213.84, 70.45) circle (  2.13);

\path[fill=fillColor,fill opacity=0.20] (211.87, 77.76) circle (  2.13);

\path[fill=fillColor,fill opacity=0.20] (209.91, 79.39) circle (  2.13);

\path[fill=fillColor,fill opacity=0.20] (208.93, 81.82) circle (  2.13);

\path[fill=fillColor,fill opacity=0.20] (213.84, 92.39) circle (  2.13);

\path[fill=fillColor,fill opacity=0.20] (191.24,111.89) circle (  2.13);

\path[fill=fillColor,fill opacity=0.20] (189.37, 99.70) circle (  2.13);

\path[fill=fillColor,fill opacity=0.20] (186.23, 97.27) circle (  2.13);

\path[fill=fillColor,fill opacity=0.20] (197.13, 93.20) circle (  2.13);

\path[fill=fillColor,fill opacity=0.20] (196.15, 87.51) circle (  2.13);

\path[fill=fillColor,fill opacity=0.20] (198.12, 87.51) circle (  2.13);

\path[fill=fillColor,fill opacity=0.20] (197.13, 84.26) circle (  2.13);

\path[fill=fillColor,fill opacity=0.20] (196.15, 79.39) circle (  2.13);

\path[fill=fillColor,fill opacity=0.20] (202.05, 75.32) circle (  2.13);

\path[fill=fillColor,fill opacity=0.20] (208.93, 66.38) circle (  2.13);

\path[fill=fillColor,fill opacity=0.20] (220.72, 62.32) circle (  2.13);

\path[fill=fillColor,fill opacity=0.20] (228.58, 62.32) circle (  2.13);

\path[fill=fillColor,fill opacity=0.20] (243.32, 53.38) circle (  2.13);

\path[fill=fillColor,fill opacity=0.20] (249.22, 47.69) circle (  2.13);

\path[fill=fillColor,fill opacity=0.20] (234.48, 50.94) circle (  2.13);

\path[fill=fillColor,fill opacity=0.20] (256.10, 49.32) circle (  2.13);

\path[fill=fillColor,fill opacity=0.20] (217.77, 59.07) circle (  2.13);

\path[fill=fillColor,fill opacity=0.20] (210.89, 74.51) circle (  2.13);

\path[fill=fillColor,fill opacity=0.20] (208.93, 88.33) circle (  2.13);

\path[fill=fillColor,fill opacity=0.20] (210.89, 90.76) circle (  2.13);

\path[fill=fillColor,fill opacity=0.20] (212.86, 86.70) circle (  2.13);

\path[fill=fillColor,fill opacity=0.20] (208.93, 88.33) circle (  2.13);

\path[fill=fillColor,fill opacity=0.20] (211.87,101.33) circle (  2.13);

\path[fill=fillColor,fill opacity=0.20] (193.20,101.33) circle (  2.13);

\path[fill=fillColor,fill opacity=0.20] (191.24, 98.08) circle (  2.13);

\path[fill=fillColor,fill opacity=0.20] (179.54, 98.89) circle (  2.13);

\path[fill=fillColor,fill opacity=0.20] (196.15, 94.83) circle (  2.13);

\path[fill=fillColor,fill opacity=0.20] (197.13, 89.95) circle (  2.13);

\path[fill=fillColor,fill opacity=0.20] (193.20, 87.51) circle (  2.13);

\path[fill=fillColor,fill opacity=0.20] (194.19, 83.45) circle (  2.13);

\path[fill=fillColor,fill opacity=0.20] (197.13, 80.20) circle (  2.13);

\path[fill=fillColor,fill opacity=0.20] (205.98, 75.32) circle (  2.13);

\path[fill=fillColor,fill opacity=0.20] (221.70, 67.20) circle (  2.13);

\path[fill=fillColor,fill opacity=0.20] (233.49, 59.88) circle (  2.13);

\path[fill=fillColor,fill opacity=0.20] (259.04, 59.07) circle (  2.13);

\path[fill=fillColor,fill opacity=0.20] (244.30, 58.26) circle (  2.13);

\path[fill=fillColor,fill opacity=0.20] (235.46, 50.13) circle (  2.13);

\path[fill=fillColor,fill opacity=0.20] (223.67, 68.82) circle (  2.13);

\path[fill=fillColor,fill opacity=0.20] (217.77, 78.57) circle (  2.13);

\path[fill=fillColor,fill opacity=0.20] (212.86, 84.26) circle (  2.13);

\path[fill=fillColor,fill opacity=0.20] (210.89, 85.89) circle (  2.13);

\path[fill=fillColor,fill opacity=0.20] (210.89, 85.08) circle (  2.13);

\path[fill=fillColor,fill opacity=0.20] (208.93, 84.26) circle (  2.13);

\path[fill=fillColor,fill opacity=0.20] (213.84, 92.39) circle (  2.13);

\path[fill=fillColor,fill opacity=0.20] (217.77,105.39) circle (  2.13);

\path[fill=fillColor,fill opacity=0.20] (189.86,111.89) circle (  2.13);

\path[fill=fillColor,fill opacity=0.20] (193.20,106.21) circle (  2.13);

\path[fill=fillColor,fill opacity=0.20] (197.13, 98.89) circle (  2.13);

\path[fill=fillColor,fill opacity=0.20] (197.13, 92.39) circle (  2.13);

\path[fill=fillColor,fill opacity=0.20] (194.19, 92.39) circle (  2.13);

\path[fill=fillColor,fill opacity=0.20] (198.12, 85.08) circle (  2.13);

\path[fill=fillColor,fill opacity=0.20] (205.00, 74.51) circle (  2.13);

\path[fill=fillColor,fill opacity=0.20] (209.91, 69.63) circle (  2.13);

\path[fill=fillColor,fill opacity=0.20] (210.89, 69.63) circle (  2.13);

\path[fill=fillColor,fill opacity=0.20] (214.82, 69.63) circle (  2.13);

\path[fill=fillColor,fill opacity=0.20] (210.89, 65.57) circle (  2.13);

\path[fill=fillColor,fill opacity=0.20] (234.48, 58.26) circle (  2.13);

\path[fill=fillColor,fill opacity=0.20] (248.23, 57.44) circle (  2.13);

\path[fill=fillColor,fill opacity=0.20] (218.75, 58.26) circle (  2.13);

\path[fill=fillColor,fill opacity=0.20] (224.65, 62.32) circle (  2.13);

\path[fill=fillColor,fill opacity=0.20] (213.84, 72.89) circle (  2.13);

\path[fill=fillColor,fill opacity=0.20] (209.91, 85.89) circle (  2.13);

\path[fill=fillColor,fill opacity=0.20] (207.94, 90.76) circle (  2.13);

\path[fill=fillColor,fill opacity=0.20] (210.89, 90.76) circle (  2.13);

\path[fill=fillColor,fill opacity=0.20] (215.81, 92.39) circle (  2.13);

\path[fill=fillColor,fill opacity=0.20] (214.82, 93.20) circle (  2.13);

\path[fill=fillColor,fill opacity=0.20] (211.87, 97.27) circle (  2.13);

\path[fill=fillColor,fill opacity=0.20] (192.22,103.77) circle (  2.13);

\path[fill=fillColor,fill opacity=0.20] (189.27, 99.70) circle (  2.13);

\path[fill=fillColor,fill opacity=0.20] (195.17, 98.89) circle (  2.13);

\path[fill=fillColor,fill opacity=0.20] (203.03, 92.39) circle (  2.13);

\path[fill=fillColor,fill opacity=0.20] (208.93, 82.64) circle (  2.13);

\path[fill=fillColor,fill opacity=0.20] (198.12, 76.14) circle (  2.13);

\path[fill=fillColor,fill opacity=0.20] (212.86, 68.82) circle (  2.13);

\path[fill=fillColor,fill opacity=0.20] (227.60, 59.07) circle (  2.13);

\path[fill=fillColor,fill opacity=0.20] (244.30, 59.07) circle (  2.13);

\path[fill=fillColor,fill opacity=0.20] (250.20, 65.57) circle (  2.13);

\path[fill=fillColor,fill opacity=0.20] (231.53, 60.69) circle (  2.13);

\path[fill=fillColor,fill opacity=0.20] (260.03, 56.63) circle (  2.13);

\path[fill=fillColor,fill opacity=0.20] (269.85, 51.75) circle (  2.13);

\path[fill=fillColor,fill opacity=0.20] (239.39, 53.38) circle (  2.13);

\path[fill=fillColor,fill opacity=0.20] (225.63, 59.88) circle (  2.13);

\path[fill=fillColor,fill opacity=0.20] (212.86, 75.32) circle (  2.13);

\path[fill=fillColor,fill opacity=0.20] (208.93, 88.33) circle (  2.13);

\path[fill=fillColor,fill opacity=0.20] (215.81, 84.26) circle (  2.13);

\path[fill=fillColor,fill opacity=0.20] (215.81, 84.26) circle (  2.13);

\path[fill=fillColor,fill opacity=0.20] (215.81, 90.76) circle (  2.13);

\path[fill=fillColor,fill opacity=0.20] (213.84, 90.76) circle (  2.13);

\path[fill=fillColor,fill opacity=0.20] (211.87, 90.76) circle (  2.13);

\path[fill=fillColor,fill opacity=0.20] (208.93, 98.89) circle (  2.13);

\path[fill=fillColor,fill opacity=0.20] (198.12,115.15) circle (  2.13);

\path[fill=fillColor,fill opacity=0.20] (199.10,111.08) circle (  2.13);

\path[fill=fillColor,fill opacity=0.20] (200.08,105.39) circle (  2.13);

\path[fill=fillColor,fill opacity=0.20] (200.08, 93.20) circle (  2.13);

\path[fill=fillColor,fill opacity=0.20] (201.07, 83.45) circle (  2.13);

\path[fill=fillColor,fill opacity=0.20] (215.81, 81.82) circle (  2.13);

\path[fill=fillColor,fill opacity=0.20] (214.82, 76.14) circle (  2.13);

\path[fill=fillColor,fill opacity=0.20] (237.42, 68.82) circle (  2.13);

\path[fill=fillColor,fill opacity=0.20] (246.27, 65.57) circle (  2.13);

\path[fill=fillColor,fill opacity=0.20] (238.41, 63.95) circle (  2.13);

\path[fill=fillColor,fill opacity=0.20] (238.41, 62.32) circle (  2.13);

\path[fill=fillColor,fill opacity=0.20] (220.72, 63.13) circle (  2.13);

\path[fill=fillColor,fill opacity=0.20] (265.92, 55.01) circle (  2.13);

\path[fill=fillColor,fill opacity=0.20] (234.48, 59.07) circle (  2.13);

\path[fill=fillColor,fill opacity=0.20] (222.68, 63.95) circle (  2.13);

\path[fill=fillColor,fill opacity=0.20] (213.84, 63.13) circle (  2.13);

\path[fill=fillColor,fill opacity=0.20] (211.87, 70.45) circle (  2.13);

\path[fill=fillColor,fill opacity=0.20] (214.82, 85.89) circle (  2.13);

\path[fill=fillColor,fill opacity=0.20] (213.84, 89.95) circle (  2.13);

\path[fill=fillColor,fill opacity=0.20] (212.86, 86.70) circle (  2.13);

\path[fill=fillColor,fill opacity=0.20] (218.75, 88.33) circle (  2.13);

\path[fill=fillColor,fill opacity=0.20] (219.74, 89.14) circle (  2.13);

\path[fill=fillColor,fill opacity=0.20] (216.79, 95.64) circle (  2.13);

\path[fill=fillColor,fill opacity=0.20] (198.12,110.27) circle (  2.13);

\path[fill=fillColor,fill opacity=0.20] (201.07, 99.70) circle (  2.13);

\path[fill=fillColor,fill opacity=0.20] (205.98, 91.58) circle (  2.13);

\path[fill=fillColor,fill opacity=0.20] (207.94, 83.45) circle (  2.13);

\path[fill=fillColor,fill opacity=0.20] (209.91, 82.64) circle (  2.13);

\path[fill=fillColor,fill opacity=0.20] (220.72, 83.45) circle (  2.13);

\path[fill=fillColor,fill opacity=0.20] (225.63, 76.95) circle (  2.13);

\path[fill=fillColor,fill opacity=0.20] (228.58, 68.82) circle (  2.13);

\path[fill=fillColor,fill opacity=0.20] (236.44, 66.38) circle (  2.13);

\path[fill=fillColor,fill opacity=0.20] (243.32, 62.32) circle (  2.13);

\path[fill=fillColor,fill opacity=0.20] (232.51, 58.26) circle (  2.13);

\path[fill=fillColor,fill opacity=0.20] (244.30, 62.32) circle (  2.13);

\path[fill=fillColor,fill opacity=0.20] (264.94, 56.63) circle (  2.13);

\path[fill=fillColor,fill opacity=0.20] (258.06, 56.63) circle (  2.13);

\path[fill=fillColor,fill opacity=0.20] (256.10, 55.82) circle (  2.13);

\path[fill=fillColor,fill opacity=0.20] (210.89, 56.63) circle (  2.13);

\path[fill=fillColor,fill opacity=0.20] (225.63, 63.95) circle (  2.13);

\path[fill=fillColor,fill opacity=0.20] (215.81, 75.32) circle (  2.13);

\path[fill=fillColor,fill opacity=0.20] (211.87, 76.95) circle (  2.13);

\path[fill=fillColor,fill opacity=0.20] (214.82, 79.39) circle (  2.13);

\path[fill=fillColor,fill opacity=0.20] (217.77, 85.89) circle (  2.13);

\path[fill=fillColor,fill opacity=0.20] (219.74, 84.26) circle (  2.13);

\path[fill=fillColor,fill opacity=0.20] (219.74, 77.76) circle (  2.13);

\path[fill=fillColor,fill opacity=0.20] (220.72, 81.82) circle (  2.13);

\path[fill=fillColor,fill opacity=0.20] (224.65, 87.51) circle (  2.13);

\path[fill=fillColor,fill opacity=0.20] (228.58, 87.51) circle (  2.13);

\path[fill=fillColor,fill opacity=0.20] (217.77, 90.76) circle (  2.13);

\path[fill=fillColor,fill opacity=0.20] (209.91, 96.45) circle (  2.13);

\path[fill=fillColor,fill opacity=0.20] (224.65, 97.27) circle (  2.13);

\path[fill=fillColor,fill opacity=0.20] (214.82, 94.83) circle (  2.13);

\path[fill=fillColor,fill opacity=0.20] (217.77, 94.02) circle (  2.13);

\path[fill=fillColor,fill opacity=0.20] (217.77, 96.45) circle (  2.13);

\path[fill=fillColor,fill opacity=0.20] (218.75, 98.89) circle (  2.13);

\path[fill=fillColor,fill opacity=0.20] (222.68, 99.70) circle (  2.13);

\path[fill=fillColor,fill opacity=0.20] (205.98,102.14) circle (  2.13);

\path[fill=fillColor,fill opacity=0.20] (205.00,102.96) circle (  2.13);

\path[fill=fillColor,fill opacity=0.20] (200.08,102.14) circle (  2.13);

\path[fill=fillColor,fill opacity=0.20] (203.03,102.14) circle (  2.13);

\path[fill=fillColor,fill opacity=0.20] (203.03, 96.45) circle (  2.13);

\path[fill=fillColor,fill opacity=0.20] (200.08, 86.70) circle (  2.13);

\path[fill=fillColor,fill opacity=0.20] (201.07, 85.89) circle (  2.13);

\path[fill=fillColor,fill opacity=0.20] (196.15, 87.51) circle (  2.13);

\path[fill=fillColor,fill opacity=0.20] (214.82, 81.82) circle (  2.13);

\path[fill=fillColor,fill opacity=0.20] (218.75, 76.14) circle (  2.13);

\path[fill=fillColor,fill opacity=0.20] (232.51, 75.32) circle (  2.13);

\path[fill=fillColor,fill opacity=0.20] (234.48, 72.89) circle (  2.13);

\path[fill=fillColor,fill opacity=0.20] (224.65, 70.45) circle (  2.13);

\path[fill=fillColor,fill opacity=0.20] (271.82, 71.26) circle (  2.13);

\path[fill=fillColor,fill opacity=0.20] (248.23, 70.45) circle (  2.13);

\path[fill=fillColor,fill opacity=0.20] (234.48, 67.20) circle (  2.13);

\path[fill=fillColor,fill opacity=0.20] (232.51, 60.69) circle (  2.13);

\path[fill=fillColor,fill opacity=0.20] (245.29, 63.13) circle (  2.13);

\path[fill=fillColor,fill opacity=0.20] (252.16, 60.69) circle (  2.13);

\path[fill=fillColor,fill opacity=0.20] (238.41, 67.20) circle (  2.13);

\path[fill=fillColor,fill opacity=0.20] (218.75, 67.20) circle (  2.13);

\path[fill=fillColor,fill opacity=0.20] (219.74, 68.82) circle (  2.13);

\path[fill=fillColor,fill opacity=0.20] (219.74, 73.70) circle (  2.13);

\path[fill=fillColor,fill opacity=0.20] (220.72, 72.89) circle (  2.13);

\path[fill=fillColor,fill opacity=0.20] (218.75, 71.26) circle (  2.13);

\path[fill=fillColor,fill opacity=0.20] (211.87, 76.14) circle (  2.13);

\path[fill=fillColor,fill opacity=0.20] (219.74, 79.39) circle (  2.13);

\path[fill=fillColor,fill opacity=0.20] (221.70, 77.76) circle (  2.13);

\path[fill=fillColor,fill opacity=0.20] (219.74, 78.57) circle (  2.13);

\path[fill=fillColor,fill opacity=0.20] (216.79, 83.45) circle (  2.13);

\path[fill=fillColor,fill opacity=0.20] (225.63, 86.70) circle (  2.13);

\path[fill=fillColor,fill opacity=0.20] (220.72, 86.70) circle (  2.13);

\path[fill=fillColor,fill opacity=0.20] (214.82, 85.89) circle (  2.13);

\path[fill=fillColor,fill opacity=0.20] (216.79, 85.89) circle (  2.13);

\path[fill=fillColor,fill opacity=0.20] (210.89, 83.45) circle (  2.13);

\path[fill=fillColor,fill opacity=0.20] (211.87, 84.26) circle (  2.13);

\path[fill=fillColor,fill opacity=0.20] (214.82, 81.82) circle (  2.13);

\path[fill=fillColor,fill opacity=0.20] (215.81, 81.82) circle (  2.13);

\path[fill=fillColor,fill opacity=0.20] (205.98, 84.26) circle (  2.13);

\path[fill=fillColor,fill opacity=0.20] (205.98, 88.33) circle (  2.13);

\path[fill=fillColor,fill opacity=0.20] (204.01, 89.14) circle (  2.13);

\path[fill=fillColor,fill opacity=0.20] (203.03, 84.26) circle (  2.13);

\path[fill=fillColor,fill opacity=0.20] (203.03, 76.14) circle (  2.13);

\path[fill=fillColor,fill opacity=0.20] (210.89, 70.45) circle (  2.13);

\path[fill=fillColor,fill opacity=0.20] (216.79, 67.20) circle (  2.13);

\path[fill=fillColor,fill opacity=0.20] (219.74, 68.82) circle (  2.13);

\path[fill=fillColor,fill opacity=0.20] (238.41, 74.51) circle (  2.13);

\path[fill=fillColor,fill opacity=0.20] (245.29, 74.51) circle (  2.13);

\path[fill=fillColor,fill opacity=0.20] (248.23, 73.70) circle (  2.13);

\path[fill=fillColor,fill opacity=0.20] (262.97, 78.57) circle (  2.13);

\path[fill=fillColor,fill opacity=0.20] (250.20, 82.64) circle (  2.13);

\path[fill=fillColor,fill opacity=0.20] (244.30, 58.26) circle (  2.13);

\path[fill=fillColor,fill opacity=0.20] (247.25, 64.76) circle (  2.13);

\path[fill=fillColor,fill opacity=0.20] (247.25, 65.57) circle (  2.13);

\path[fill=fillColor,fill opacity=0.20] (263.96, 59.88) circle (  2.13);

\path[fill=fillColor,fill opacity=0.20] (234.48, 55.82) circle (  2.13);

\path[fill=fillColor,fill opacity=0.20] (236.44, 55.82) circle (  2.13);

\path[fill=fillColor,fill opacity=0.20] (222.68, 61.51) circle (  2.13);

\path[fill=fillColor,fill opacity=0.20] (219.74, 69.63) circle (  2.13);

\path[fill=fillColor,fill opacity=0.20] (216.79, 72.89) circle (  2.13);

\path[fill=fillColor,fill opacity=0.20] (214.82, 73.70) circle (  2.13);

\path[fill=fillColor,fill opacity=0.20] (210.89, 74.51) circle (  2.13);

\path[fill=fillColor,fill opacity=0.20] (216.79, 76.95) circle (  2.13);

\path[fill=fillColor,fill opacity=0.20] (215.81, 79.39) circle (  2.13);

\path[fill=fillColor,fill opacity=0.20] (209.91, 83.45) circle (  2.13);

\path[fill=fillColor,fill opacity=0.20] (209.91, 88.33) circle (  2.13);

\path[fill=fillColor,fill opacity=0.20] (210.89, 85.08) circle (  2.13);

\path[fill=fillColor,fill opacity=0.20] (208.93, 77.76) circle (  2.13);

\path[fill=fillColor,fill opacity=0.20] (209.91, 74.51) circle (  2.13);

\path[fill=fillColor,fill opacity=0.20] (217.77, 72.89) circle (  2.13);

\path[fill=fillColor,fill opacity=0.20] (213.84, 70.45) circle (  2.13);

\path[fill=fillColor,fill opacity=0.20] (213.84, 71.26) circle (  2.13);

\path[fill=fillColor,fill opacity=0.20] (221.70, 73.70) circle (  2.13);

\path[fill=fillColor,fill opacity=0.20] (217.77, 73.70) circle (  2.13);

\path[fill=fillColor,fill opacity=0.20] (220.72, 70.45) circle (  2.13);

\path[fill=fillColor,fill opacity=0.20] (231.53, 62.32) circle (  2.13);

\path[fill=fillColor,fill opacity=0.20] (246.27, 59.88) circle (  2.13);

\path[fill=fillColor,fill opacity=0.20] (247.25, 64.76) circle (  2.13);

\path[fill=fillColor,fill opacity=0.20] (242.34, 71.26) circle (  2.13);

\path[fill=fillColor,fill opacity=0.20] (244.30, 51.75) circle (  2.13);

\path[fill=fillColor,fill opacity=0.20] (244.30, 50.94) circle (  2.13);

\path[fill=fillColor,fill opacity=0.20] (259.04, 56.63) circle (  2.13);

\path[fill=fillColor,fill opacity=0.20] (225.63, 61.51) circle (  2.13);

\path[fill=fillColor,fill opacity=0.20] (228.58, 60.69) circle (  2.13);

\path[fill=fillColor,fill opacity=0.20] (221.70, 63.95) circle (  2.13);

\path[fill=fillColor,fill opacity=0.20] (218.75, 69.63) circle (  2.13);

\path[fill=fillColor,fill opacity=0.20] (216.79, 68.01) circle (  2.13);

\path[fill=fillColor,fill opacity=0.20] (213.84, 63.95) circle (  2.13);

\path[fill=fillColor,fill opacity=0.20] (208.93, 67.20) circle (  2.13);

\path[fill=fillColor,fill opacity=0.20] (209.91, 74.51) circle (  2.13);

\path[fill=fillColor,fill opacity=0.20] (218.75, 70.45) circle (  2.13);

\path[fill=fillColor,fill opacity=0.20] (222.68, 63.95) circle (  2.13);

\path[fill=fillColor,fill opacity=0.20] (226.62, 61.51) circle (  2.13);

\path[fill=fillColor,fill opacity=0.20] (242.34, 59.88) circle (  2.13);

\path[fill=fillColor,fill opacity=0.20] (233.49, 55.01) circle (  2.13);

\path[fill=fillColor,fill opacity=0.20] (261.99, 50.94) circle (  2.13);

\path[fill=fillColor,fill opacity=0.20] (268.87, 50.13) circle (  2.13);

\path[fill=fillColor,fill opacity=0.20] (244.30, 53.38) circle (  2.13);

\path[fill=fillColor,fill opacity=0.20] (237.42, 63.13) circle (  2.13);

\path[fill=fillColor,fill opacity=0.20] (239.39, 65.57) circle (  2.13);

\path[fill=fillColor,fill opacity=0.20] (237.42, 64.76) circle (  2.13);

\path[fill=fillColor,fill opacity=0.20] (249.22, 72.07) circle (  2.13);

\path[fill=fillColor,fill opacity=0.20] (245.29, 66.38) circle (  2.13);

\path[fill=fillColor,fill opacity=0.20] (254.13, 59.07) circle (  2.13);

\path[fill=fillColor,fill opacity=0.20] (253.15, 52.57) circle (  2.13);

\path[fill=fillColor,fill opacity=0.20] (260.03, 57.44) circle (  2.13);

\path[fill=fillColor,fill opacity=0.20] (237.42, 62.32) circle (  2.13);

\path[fill=fillColor,fill opacity=0.20] (209.91, 60.69) circle (  2.13);

\path[fill=fillColor,fill opacity=0.20] (235.46, 55.01) circle (  2.13);

\path[fill=fillColor,fill opacity=0.20] (232.51, 52.57) circle (  2.13);

\path[fill=fillColor,fill opacity=0.20] (220.72, 54.19) circle (  2.13);

\path[fill=fillColor,fill opacity=0.20] (265.92, 54.19) circle (  2.13);

\path[fill=fillColor,fill opacity=0.20] (253.15, 54.19) circle (  2.13);

\path[fill=fillColor,fill opacity=0.20] (270.84, 57.44) circle (  2.13);

\path[fill=fillColor,fill opacity=0.20] (245.29, 62.32) circle (  2.13);

\path[fill=fillColor,fill opacity=0.20] (245.29, 58.26) circle (  2.13);

\path[fill=fillColor,fill opacity=0.20] (244.30, 49.32) circle (  2.13);

\path[fill=fillColor,fill opacity=0.20] (231.53, 48.50) circle (  2.13);

\path[fill=fillColor,fill opacity=0.20] (210.89, 53.38) circle (  2.13);

\path[fill=fillColor,fill opacity=0.20] (229.56, 55.01) circle (  2.13);

\path[fill=fillColor,fill opacity=0.20] (237.42, 55.01) circle (  2.13);

\path[fill=fillColor,fill opacity=0.20] (239.39, 53.38) circle (  2.13);

\path[fill=fillColor,fill opacity=0.20] (236.44, 52.57) circle (  2.13);

\path[fill=fillColor,fill opacity=0.20] (233.49, 50.13) circle (  2.13);

\path[fill=fillColor,fill opacity=0.20] (254.13, 46.07) circle (  2.13);

\path[fill=fillColor,fill opacity=0.20] (250.20, 49.32) circle (  2.13);

\path[fill=fillColor,fill opacity=0.20] (243.32, 55.01) circle (  2.13);
\end{scope}
\begin{scope}
\path[clip] (  0.00,  0.00) rectangle (289.08,144.54);
\definecolor[named]{drawColor}{rgb}{0.50,0.50,0.50}

\node[text=drawColor,anchor=base east,inner sep=0pt, outer sep=0pt, scale=  0.96] at ( 37.38, 49.26) {1.25};

\node[text=drawColor,anchor=base east,inner sep=0pt, outer sep=0pt, scale=  0.96] at ( 37.38, 69.58) {1.50};

\node[text=drawColor,anchor=base east,inner sep=0pt, outer sep=0pt, scale=  0.96] at ( 37.38, 89.90) {1.75};

\node[text=drawColor,anchor=base east,inner sep=0pt, outer sep=0pt, scale=  0.96] at ( 37.38,110.21) {2.00};
\end{scope}
\begin{scope}
\path[clip] (  0.00,  0.00) rectangle (289.08,144.54);
\definecolor[named]{drawColor}{rgb}{0.50,0.50,0.50}

\path[draw=drawColor,line width= 0.6pt,line join=round] ( 40.22, 52.57) --
	( 44.49, 52.57);

\path[draw=drawColor,line width= 0.6pt,line join=round] ( 40.22, 72.89) --
	( 44.49, 72.89);

\path[draw=drawColor,line width= 0.6pt,line join=round] ( 40.22, 93.20) --
	( 44.49, 93.20);

\path[draw=drawColor,line width= 0.6pt,line join=round] ( 40.22,113.52) --
	( 44.49,113.52);
\end{scope}
\begin{scope}
\path[clip] (  0.00,  0.00) rectangle (289.08,144.54);
\definecolor[named]{drawColor}{rgb}{0.50,0.50,0.50}

\path[draw=drawColor,line width= 0.6pt,line join=round] ( 47.91, 29.77) --
	( 47.91, 34.04);

\path[draw=drawColor,line width= 0.6pt,line join=round] ( 72.48, 29.77) --
	( 72.48, 34.04);

\path[draw=drawColor,line width= 0.6pt,line join=round] ( 97.05, 29.77) --
	( 97.05, 34.04);

\path[draw=drawColor,line width= 0.6pt,line join=round] (121.61, 29.77) --
	(121.61, 34.04);

\path[draw=drawColor,line width= 0.6pt,line join=round] (146.18, 29.77) --
	(146.18, 34.04);
\end{scope}
\begin{scope}
\path[clip] (  0.00,  0.00) rectangle (289.08,144.54);
\definecolor[named]{drawColor}{rgb}{0.50,0.50,0.50}

\node[text=drawColor,anchor=base,inner sep=0pt, outer sep=0pt, scale=  0.96] at ( 47.91, 20.31) {7.5};

\node[text=drawColor,anchor=base,inner sep=0pt, outer sep=0pt, scale=  0.96] at ( 72.48, 20.31) {10.0};

\node[text=drawColor,anchor=base,inner sep=0pt, outer sep=0pt, scale=  0.96] at ( 97.05, 20.31) {12.5};

\node[text=drawColor,anchor=base,inner sep=0pt, outer sep=0pt, scale=  0.96] at (121.61, 20.31) {15.0};

\node[text=drawColor,anchor=base,inner sep=0pt, outer sep=0pt, scale=  0.96] at (146.18, 20.31) {17.5};
\end{scope}
\begin{scope}
\path[clip] (  0.00,  0.00) rectangle (289.08,144.54);
\definecolor[named]{drawColor}{rgb}{0.50,0.50,0.50}

\path[draw=drawColor,line width= 0.6pt,line join=round] (165.69, 29.77) --
	(165.69, 34.04);

\path[draw=drawColor,line width= 0.6pt,line join=round] (190.26, 29.77) --
	(190.26, 34.04);

\path[draw=drawColor,line width= 0.6pt,line join=round] (214.82, 29.77) --
	(214.82, 34.04);

\path[draw=drawColor,line width= 0.6pt,line join=round] (239.39, 29.77) --
	(239.39, 34.04);

\path[draw=drawColor,line width= 0.6pt,line join=round] (263.96, 29.77) --
	(263.96, 34.04);
\end{scope}
\begin{scope}
\path[clip] (  0.00,  0.00) rectangle (289.08,144.54);
\definecolor[named]{drawColor}{rgb}{0.50,0.50,0.50}

\node[text=drawColor,anchor=base,inner sep=0pt, outer sep=0pt, scale=  0.96] at (165.69, 20.31) {7.5};

\node[text=drawColor,anchor=base,inner sep=0pt, outer sep=0pt, scale=  0.96] at (190.26, 20.31) {10.0};

\node[text=drawColor,anchor=base,inner sep=0pt, outer sep=0pt, scale=  0.96] at (214.82, 20.31) {12.5};

\node[text=drawColor,anchor=base,inner sep=0pt, outer sep=0pt, scale=  0.96] at (239.39, 20.31) {15.0};

\node[text=drawColor,anchor=base,inner sep=0pt, outer sep=0pt, scale=  0.96] at (263.96, 20.31) {17.5};
\end{scope}
\begin{scope}
\path[clip] (  0.00,  0.00) rectangle (289.08,144.54);
\definecolor[named]{drawColor}{rgb}{0.00,0.00,0.00}

\node[text=drawColor,anchor=base,inner sep=0pt, outer sep=0pt, scale=  1.20] at (160.76,  9.03) {$a$ $[\mu m]$};
\end{scope}
\begin{scope}
\path[clip] (  0.00,  0.00) rectangle (289.08,144.54);
\definecolor[named]{drawColor}{rgb}{0.00,0.00,0.00}

\node[text=drawColor,rotate= 90.00,anchor=base,inner sep=0pt, outer sep=0pt, scale=  1.20] at ( 17.30, 76.95) {AD $[\times 10^{-9}mm^2/s]$};
\end{scope}
\end{tikzpicture}

						\end{adjustbox}\\
						\begin{adjustbox}{width={\textwidth},totalheight=\textheight,keepaspectratio}
							\strut
							% Created by tikzDevice version - on 2012-09-27 22:54:45
% !TEX encoding = UTF-8 Unicode
\begin{tikzpicture}[x=1pt,y=1pt]
\definecolor[named]{fillColor}{rgb}{1.00,1.00,1.00}
\path[use as bounding box,fill=fillColor,fill opacity=0.00] (0,0) rectangle (289.08,144.54);
\begin{scope}
\path[clip] (  0.00,  0.00) rectangle (289.08,144.54);
\definecolor[named]{fillColor}{rgb}{1.00,1.00,1.00}

\path[fill=fillColor] (  0.00,  0.00) rectangle (289.08,144.54);
\end{scope}
\begin{scope}
\path[clip] ( 39.69,119.86) rectangle (156.86,132.50);
\definecolor[named]{fillColor}{rgb}{0.80,0.80,0.80}

\path[fill=fillColor] ( 39.69,119.86) rectangle (156.86,132.50);
\definecolor[named]{drawColor}{rgb}{0.00,0.00,0.00}

\node[text=drawColor,anchor=base,inner sep=0pt, outer sep=0pt, scale=  0.96] at ( 98.27,122.87) {Scan (r=0.560)};
\end{scope}
\begin{scope}
\path[clip] (159.87,119.86) rectangle (277.04,132.50);
\definecolor[named]{fillColor}{rgb}{0.80,0.80,0.80}

\path[fill=fillColor] (159.87,119.86) rectangle (277.03,132.50);
\definecolor[named]{drawColor}{rgb}{0.00,0.00,0.00}

\node[text=drawColor,anchor=base,inner sep=0pt, outer sep=0pt, scale=  0.96] at (218.45,122.87) {Rescan (r=0.431)};
\end{scope}
\begin{scope}
\path[clip] (  0.00,  0.00) rectangle (289.08,144.54);
\definecolor[named]{drawColor}{rgb}{0.50,0.50,0.50}

\node[text=drawColor,anchor=base east,inner sep=0pt, outer sep=0pt, scale=  0.96] at ( 32.58, 45.20) {1.2};

\node[text=drawColor,anchor=base east,inner sep=0pt, outer sep=0pt, scale=  0.96] at ( 32.58, 61.45) {1.4};

\node[text=drawColor,anchor=base east,inner sep=0pt, outer sep=0pt, scale=  0.96] at ( 32.58, 77.71) {1.6};

\node[text=drawColor,anchor=base east,inner sep=0pt, outer sep=0pt, scale=  0.96] at ( 32.58, 93.96) {1.8};

\node[text=drawColor,anchor=base east,inner sep=0pt, outer sep=0pt, scale=  0.96] at ( 32.58,110.21) {2.0};
\end{scope}
\begin{scope}
\path[clip] (  0.00,  0.00) rectangle (289.08,144.54);
\definecolor[named]{drawColor}{rgb}{0.50,0.50,0.50}

\path[draw=drawColor,line width= 0.6pt,line join=round,line cap=round] ( 35.42, 48.50) -- ( 39.69, 48.50);

\path[draw=drawColor,line width= 0.6pt,line join=round,line cap=round] ( 35.42, 64.76) -- ( 39.69, 64.76);

\path[draw=drawColor,line width= 0.6pt,line join=round,line cap=round] ( 35.42, 81.01) -- ( 39.69, 81.01);

\path[draw=drawColor,line width= 0.6pt,line join=round,line cap=round] ( 35.42, 97.27) -- ( 39.69, 97.27);

\path[draw=drawColor,line width= 0.6pt,line join=round,line cap=round] ( 35.42,113.52) -- ( 39.69,113.52);
\end{scope}
\begin{scope}
\path[clip] ( 39.69, 34.04) rectangle (156.86,119.86);
\definecolor[named]{fillColor}{rgb}{0.90,0.90,0.90}

\path[fill=fillColor] ( 39.69, 34.04) rectangle (156.86,119.86);
\definecolor[named]{drawColor}{rgb}{0.95,0.95,0.95}

\path[draw=drawColor,line width= 0.3pt,line join=round,line cap=round] ( 39.69, 40.38) --
	(156.86, 40.38);

\path[draw=drawColor,line width= 0.3pt,line join=round,line cap=round] ( 39.69, 56.63) --
	(156.86, 56.63);

\path[draw=drawColor,line width= 0.3pt,line join=round,line cap=round] ( 39.69, 72.89) --
	(156.86, 72.89);

\path[draw=drawColor,line width= 0.3pt,line join=round,line cap=round] ( 39.69, 89.14) --
	(156.86, 89.14);

\path[draw=drawColor,line width= 0.3pt,line join=round,line cap=round] ( 39.69,105.39) --
	(156.86,105.39);

\path[draw=drawColor,line width= 0.3pt,line join=round,line cap=round] ( 49.61, 34.04) --
	( 49.61,119.86);

\path[draw=drawColor,line width= 0.3pt,line join=round,line cap=round] ( 71.46, 34.04) --
	( 71.46,119.86);

\path[draw=drawColor,line width= 0.3pt,line join=round,line cap=round] ( 93.31, 34.04) --
	( 93.31,119.86);

\path[draw=drawColor,line width= 0.3pt,line join=round,line cap=round] (115.16, 34.04) --
	(115.16,119.86);

\path[draw=drawColor,line width= 0.3pt,line join=round,line cap=round] (137.01, 34.04) --
	(137.01,119.86);
\definecolor[named]{drawColor}{rgb}{1.00,1.00,1.00}

\path[draw=drawColor,line width= 0.6pt,line join=round,line cap=round] ( 39.69, 48.50) --
	(156.86, 48.50);

\path[draw=drawColor,line width= 0.6pt,line join=round,line cap=round] ( 39.69, 64.76) --
	(156.86, 64.76);

\path[draw=drawColor,line width= 0.6pt,line join=round,line cap=round] ( 39.69, 81.01) --
	(156.86, 81.01);

\path[draw=drawColor,line width= 0.6pt,line join=round,line cap=round] ( 39.69, 97.27) --
	(156.86, 97.27);

\path[draw=drawColor,line width= 0.6pt,line join=round,line cap=round] ( 39.69,113.52) --
	(156.86,113.52);

\path[draw=drawColor,line width= 0.6pt,line join=round,line cap=round] ( 60.53, 34.04) --
	( 60.53,119.86);

\path[draw=drawColor,line width= 0.6pt,line join=round,line cap=round] ( 82.38, 34.04) --
	( 82.38,119.86);

\path[draw=drawColor,line width= 0.6pt,line join=round,line cap=round] (104.23, 34.04) --
	(104.23,119.86);

\path[draw=drawColor,line width= 0.6pt,line join=round,line cap=round] (126.08, 34.04) --
	(126.08,119.86);

\path[draw=drawColor,line width= 0.6pt,line join=round,line cap=round] (147.93, 34.04) --
	(147.93,119.86);
\definecolor[named]{fillColor}{rgb}{0.00,0.00,0.00}

\path[fill=fillColor,fill opacity=0.20] ( 52.23, 55.82) circle (  2.13);

\path[fill=fillColor,fill opacity=0.20] ( 64.25, 61.51) circle (  2.13);

\path[fill=fillColor,fill opacity=0.20] ( 76.26, 59.07) circle (  2.13);

\path[fill=fillColor,fill opacity=0.20] ( 86.97, 55.82) circle (  2.13);

\path[fill=fillColor,fill opacity=0.20] ( 81.07, 55.01) circle (  2.13);

\path[fill=fillColor,fill opacity=0.20] ( 75.17, 52.57) circle (  2.13);

\path[fill=fillColor,fill opacity=0.20] ( 71.02, 60.69) circle (  2.13);

\path[fill=fillColor,fill opacity=0.20] ( 59.88, 68.01) circle (  2.13);

\path[fill=fillColor,fill opacity=0.20] ( 75.17, 69.63) circle (  2.13);

\path[fill=fillColor,fill opacity=0.20] ( 85.88, 59.07) circle (  2.13);

\path[fill=fillColor,fill opacity=0.20] ( 97.68, 62.32) circle (  2.13);

\path[fill=fillColor,fill opacity=0.20] (108.60, 73.70) circle (  2.13);

\path[fill=fillColor,fill opacity=0.20] ( 96.36, 72.89) circle (  2.13);

\path[fill=fillColor,fill opacity=0.20] (103.79, 61.51) circle (  2.13);

\path[fill=fillColor,fill opacity=0.20] (100.30, 50.94) circle (  2.13);

\path[fill=fillColor,fill opacity=0.20] ( 91.78, 53.38) circle (  2.13);

\path[fill=fillColor,fill opacity=0.20] ( 69.49, 47.69) circle (  2.13);

\path[fill=fillColor,fill opacity=0.20] ( 58.13, 38.75) circle (  2.13);

\path[fill=fillColor,fill opacity=0.20] ( 69.05, 76.14) circle (  2.13);

\path[fill=fillColor,fill opacity=0.20] ( 85.22, 65.57) circle (  2.13);

\path[fill=fillColor,fill opacity=0.20] ( 90.68, 67.20) circle (  2.13);

\path[fill=fillColor,fill opacity=0.20] ( 94.84, 78.57) circle (  2.13);

\path[fill=fillColor,fill opacity=0.20] ( 98.33, 85.89) circle (  2.13);

\path[fill=fillColor,fill opacity=0.20] (102.92, 89.95) circle (  2.13);

\path[fill=fillColor,fill opacity=0.20] (100.08, 91.58) circle (  2.13);

\path[fill=fillColor,fill opacity=0.20] ( 95.27, 84.26) circle (  2.13);

\path[fill=fillColor,fill opacity=0.20] ( 93.52, 68.01) circle (  2.13);

\path[fill=fillColor,fill opacity=0.20] ( 88.06, 50.13) circle (  2.13);

\path[fill=fillColor,fill opacity=0.20] ( 76.04, 45.25) circle (  2.13);

\path[fill=fillColor,fill opacity=0.20] ( 74.95, 42.82) circle (  2.13);

\path[fill=fillColor,fill opacity=0.20] ( 71.46, 38.75) circle (  2.13);

\path[fill=fillColor,fill opacity=0.20] ( 58.56, 61.51) circle (  2.13);

\path[fill=fillColor,fill opacity=0.20] ( 81.51, 66.38) circle (  2.13);

\path[fill=fillColor,fill opacity=0.20] ( 98.11, 72.89) circle (  2.13);

\path[fill=fillColor,fill opacity=0.20] (111.66, 81.01) circle (  2.13);

\path[fill=fillColor,fill opacity=0.20] (106.42,102.96) circle (  2.13);

\path[fill=fillColor,fill opacity=0.20] ( 99.21,107.02) circle (  2.13);

\path[fill=fillColor,fill opacity=0.20] ( 92.87, 98.08) circle (  2.13);

\path[fill=fillColor,fill opacity=0.20] ( 97.68, 96.45) circle (  2.13);

\path[fill=fillColor,fill opacity=0.20] ( 99.86, 90.76) circle (  2.13);

\path[fill=fillColor,fill opacity=0.20] ( 88.94, 77.76) circle (  2.13);

\path[fill=fillColor,fill opacity=0.20] ( 79.76, 66.38) circle (  2.13);

\path[fill=fillColor,fill opacity=0.20] ( 77.14, 60.69) circle (  2.13);

\path[fill=fillColor,fill opacity=0.20] ( 77.79, 51.75) circle (  2.13);

\path[fill=fillColor,fill opacity=0.20] ( 74.30, 45.25) circle (  2.13);

\path[fill=fillColor,fill opacity=0.20] ( 90.25, 41.19) circle (  2.13);

\path[fill=fillColor,fill opacity=0.20] ( 60.97, 63.13) circle (  2.13);

\path[fill=fillColor,fill opacity=0.20] ( 76.48, 68.01) circle (  2.13);

\path[fill=fillColor,fill opacity=0.20] ( 98.77, 86.70) circle (  2.13);

\path[fill=fillColor,fill opacity=0.20] (107.95, 94.02) circle (  2.13);

\path[fill=fillColor,fill opacity=0.20] (105.10,101.33) circle (  2.13);

\path[fill=fillColor,fill opacity=0.20] (103.58,106.21) circle (  2.13);

\path[fill=fillColor,fill opacity=0.20] ( 95.27,107.02) circle (  2.13);

\path[fill=fillColor,fill opacity=0.20] ( 83.25, 99.70) circle (  2.13);

\path[fill=fillColor,fill opacity=0.20] ( 89.37, 85.89) circle (  2.13);

\path[fill=fillColor,fill opacity=0.20] ( 85.22, 76.95) circle (  2.13);

\path[fill=fillColor,fill opacity=0.20] ( 76.92, 77.76) circle (  2.13);

\path[fill=fillColor,fill opacity=0.20] ( 74.30, 76.95) circle (  2.13);

\path[fill=fillColor,fill opacity=0.20] ( 65.56, 76.95) circle (  2.13);

\path[fill=fillColor,fill opacity=0.20] ( 54.19, 85.89) circle (  2.13);

\path[fill=fillColor,fill opacity=0.20] ( 70.80, 85.89) circle (  2.13);

\path[fill=fillColor,fill opacity=0.20] ( 90.68, 65.57) circle (  2.13);

\path[fill=fillColor,fill opacity=0.20] ( 81.51, 93.20) circle (  2.13);

\path[fill=fillColor,fill opacity=0.20] ( 81.29, 94.02) circle (  2.13);

\path[fill=fillColor,fill opacity=0.20] ( 83.47, 85.08) circle (  2.13);

\path[fill=fillColor,fill opacity=0.20] ( 64.25, 72.89) circle (  2.13);

\path[fill=fillColor,fill opacity=0.20] ( 78.23, 71.26) circle (  2.13);

\path[fill=fillColor,fill opacity=0.20] ( 95.93, 92.39) circle (  2.13);

\path[fill=fillColor,fill opacity=0.20] (102.48, 98.89) circle (  2.13);

\path[fill=fillColor,fill opacity=0.20] ( 98.77, 90.76) circle (  2.13);

\path[fill=fillColor,fill opacity=0.20] (102.05, 90.76) circle (  2.13);

\path[fill=fillColor,fill opacity=0.20] ( 99.21,103.77) circle (  2.13);

\path[fill=fillColor,fill opacity=0.20] ( 87.84,104.58) circle (  2.13);

\path[fill=fillColor,fill opacity=0.20] ( 73.86, 86.70) circle (  2.13);

\path[fill=fillColor,fill opacity=0.20] ( 71.67, 76.14) circle (  2.13);

\path[fill=fillColor,fill opacity=0.20] ( 64.25, 76.95) circle (  2.13);

\path[fill=fillColor,fill opacity=0.20] ( 66.65, 94.83) circle (  2.13);

\path[fill=fillColor,fill opacity=0.20] (105.32, 72.89) circle (  2.13);

\path[fill=fillColor,fill opacity=0.20] ( 97.02, 86.70) circle (  2.13);

\path[fill=fillColor,fill opacity=0.20] ( 94.84, 86.70) circle (  2.13);

\path[fill=fillColor,fill opacity=0.20] ( 99.86, 89.14) circle (  2.13);

\path[fill=fillColor,fill opacity=0.20] ( 99.42, 89.14) circle (  2.13);

\path[fill=fillColor,fill opacity=0.20] ( 95.71, 81.82) circle (  2.13);

\path[fill=fillColor,fill opacity=0.20] ( 81.29, 89.14) circle (  2.13);

\path[fill=fillColor,fill opacity=0.20] ( 70.36, 69.63) circle (  2.13);

\path[fill=fillColor,fill opacity=0.20] ( 77.57, 72.89) circle (  2.13);

\path[fill=fillColor,fill opacity=0.20] ( 88.06, 87.51) circle (  2.13);

\path[fill=fillColor,fill opacity=0.20] ( 90.47,100.52) circle (  2.13);

\path[fill=fillColor,fill opacity=0.20] ( 87.62,101.33) circle (  2.13);

\path[fill=fillColor,fill opacity=0.20] ( 89.59, 89.14) circle (  2.13);

\path[fill=fillColor,fill opacity=0.20] ( 91.99, 90.76) circle (  2.13);

\path[fill=fillColor,fill opacity=0.20] ( 85.44,102.96) circle (  2.13);

\path[fill=fillColor,fill opacity=0.20] ( 64.46, 78.57) circle (  2.13);

\path[fill=fillColor,fill opacity=0.20] ( 77.57, 91.58) circle (  2.13);

\path[fill=fillColor,fill opacity=0.20] (111.88, 73.70) circle (  2.13);

\path[fill=fillColor,fill opacity=0.20] (110.35, 92.39) circle (  2.13);

\path[fill=fillColor,fill opacity=0.20] (116.03, 89.95) circle (  2.13);

\path[fill=fillColor,fill opacity=0.20] (126.95, 81.01) circle (  2.13);

\path[fill=fillColor,fill opacity=0.20] (146.84, 79.39) circle (  2.13);

\path[fill=fillColor,fill opacity=0.20] (113.41, 81.82) circle (  2.13);

\path[fill=fillColor,fill opacity=0.20] (105.54, 75.32) circle (  2.13);

\path[fill=fillColor,fill opacity=0.20] (101.39, 76.95) circle (  2.13);

\path[fill=fillColor,fill opacity=0.20] ( 65.34,104.58) circle (  2.13);

\path[fill=fillColor,fill opacity=0.20] ( 67.96, 62.32) circle (  2.13);

\path[fill=fillColor,fill opacity=0.20] ( 78.01, 68.82) circle (  2.13);

\path[fill=fillColor,fill opacity=0.20] ( 88.94, 91.58) circle (  2.13);

\path[fill=fillColor,fill opacity=0.20] ( 81.94, 99.70) circle (  2.13);

\path[fill=fillColor,fill opacity=0.20] ( 80.41,105.39) circle (  2.13);

\path[fill=fillColor,fill opacity=0.20] ( 83.47,100.52) circle (  2.13);

\path[fill=fillColor,fill opacity=0.20] ( 80.20, 91.58) circle (  2.13);

\path[fill=fillColor,fill opacity=0.20] ( 78.67, 94.83) circle (  2.13);

\path[fill=fillColor,fill opacity=0.20] ( 77.79, 91.58) circle (  2.13);

\path[fill=fillColor,fill opacity=0.20] ( 67.96, 78.57) circle (  2.13);

\path[fill=fillColor,fill opacity=0.20] (111.66, 75.32) circle (  2.13);

\path[fill=fillColor,fill opacity=0.20] (115.59, 91.58) circle (  2.13);

\path[fill=fillColor,fill opacity=0.20] (119.09, 94.83) circle (  2.13);

\path[fill=fillColor,fill opacity=0.20] (133.73, 80.20) circle (  2.13);

\path[fill=fillColor,fill opacity=0.20] (120.40, 80.20) circle (  2.13);

\path[fill=fillColor,fill opacity=0.20] (122.58, 86.70) circle (  2.13);

\path[fill=fillColor,fill opacity=0.20] (100.95, 83.45) circle (  2.13);

\path[fill=fillColor,fill opacity=0.20] ( 96.15, 76.14) circle (  2.13);

\path[fill=fillColor,fill opacity=0.20] (115.59, 65.57) circle (  2.13);

\path[fill=fillColor,fill opacity=0.20] ( 79.98, 75.32) circle (  2.13);

\path[fill=fillColor,fill opacity=0.20] ( 60.31, 59.88) circle (  2.13);

\path[fill=fillColor,fill opacity=0.20] ( 81.07, 67.20) circle (  2.13);

\path[fill=fillColor,fill opacity=0.20] ( 81.94, 95.64) circle (  2.13);

\path[fill=fillColor,fill opacity=0.20] ( 83.91, 91.58) circle (  2.13);

\path[fill=fillColor,fill opacity=0.20] ( 86.97, 87.51) circle (  2.13);

\path[fill=fillColor,fill opacity=0.20] ( 83.04, 99.70) circle (  2.13);

\path[fill=fillColor,fill opacity=0.20] ( 86.10, 99.70) circle (  2.13);

\path[fill=fillColor,fill opacity=0.20] ( 80.63, 88.33) circle (  2.13);

\path[fill=fillColor,fill opacity=0.20] ( 82.82, 75.32) circle (  2.13);

\path[fill=fillColor,fill opacity=0.20] ( 71.24, 72.89) circle (  2.13);

\path[fill=fillColor,fill opacity=0.20] ( 89.15, 98.89) circle (  2.13);

\path[fill=fillColor,fill opacity=0.20] (124.33, 70.45) circle (  2.13);

\path[fill=fillColor,fill opacity=0.20] (117.78, 90.76) circle (  2.13);

\path[fill=fillColor,fill opacity=0.20] (109.26, 92.39) circle (  2.13);

\path[fill=fillColor,fill opacity=0.20] (101.17, 84.26) circle (  2.13);

\path[fill=fillColor,fill opacity=0.20] (100.08, 91.58) circle (  2.13);

\path[fill=fillColor,fill opacity=0.20] (102.48, 92.39) circle (  2.13);

\path[fill=fillColor,fill opacity=0.20] ( 94.40, 85.08) circle (  2.13);

\path[fill=fillColor,fill opacity=0.20] ( 82.60, 82.64) circle (  2.13);

\path[fill=fillColor,fill opacity=0.20] ( 84.78, 65.57) circle (  2.13);

\path[fill=fillColor,fill opacity=0.20] ( 86.31, 49.32) circle (  2.13);

\path[fill=fillColor,fill opacity=0.20] ( 56.38, 63.95) circle (  2.13);

\path[fill=fillColor,fill opacity=0.20] ( 75.39, 66.38) circle (  2.13);

\path[fill=fillColor,fill opacity=0.20] ( 81.73, 85.08) circle (  2.13);

\path[fill=fillColor,fill opacity=0.20] ( 91.78, 82.64) circle (  2.13);

\path[fill=fillColor,fill opacity=0.20] ( 93.09, 75.32) circle (  2.13);

\path[fill=fillColor,fill opacity=0.20] ( 86.53, 88.33) circle (  2.13);

\path[fill=fillColor,fill opacity=0.20] ( 83.47, 95.64) circle (  2.13);

\path[fill=fillColor,fill opacity=0.20] ( 81.94, 87.51) circle (  2.13);

\path[fill=fillColor,fill opacity=0.20] ( 80.41, 73.70) circle (  2.13);

\path[fill=fillColor,fill opacity=0.20] ( 78.23, 64.76) circle (  2.13);

\path[fill=fillColor,fill opacity=0.20] (129.14, 84.26) circle (  2.13);

\path[fill=fillColor,fill opacity=0.20] (114.72, 82.64) circle (  2.13);

\path[fill=fillColor,fill opacity=0.20] (102.70, 94.83) circle (  2.13);

\path[fill=fillColor,fill opacity=0.20] (119.96, 89.95) circle (  2.13);

\path[fill=fillColor,fill opacity=0.20] (100.95, 90.76) circle (  2.13);

\path[fill=fillColor,fill opacity=0.20] ( 93.31, 94.83) circle (  2.13);

\path[fill=fillColor,fill opacity=0.20] (100.73, 91.58) circle (  2.13);

\path[fill=fillColor,fill opacity=0.20] ( 94.18, 88.33) circle (  2.13);

\path[fill=fillColor,fill opacity=0.20] ( 86.31, 85.08) circle (  2.13);

\path[fill=fillColor,fill opacity=0.20] ( 88.28, 66.38) circle (  2.13);

\path[fill=fillColor,fill opacity=0.20] ( 79.32, 54.19) circle (  2.13);

\path[fill=fillColor,fill opacity=0.20] ( 72.33, 59.88) circle (  2.13);

\path[fill=fillColor,fill opacity=0.20] ( 90.90, 68.01) circle (  2.13);

\path[fill=fillColor,fill opacity=0.20] ( 99.64, 79.39) circle (  2.13);

\path[fill=fillColor,fill opacity=0.20] ( 97.46, 84.26) circle (  2.13);

\path[fill=fillColor,fill opacity=0.20] ( 88.06, 84.26) circle (  2.13);

\path[fill=fillColor,fill opacity=0.20] ( 84.78, 81.82) circle (  2.13);

\path[fill=fillColor,fill opacity=0.20] ( 83.04, 84.26) circle (  2.13);

\path[fill=fillColor,fill opacity=0.20] ( 76.48, 87.51) circle (  2.13);

\path[fill=fillColor,fill opacity=0.20] ( 79.98, 72.07) circle (  2.13);

\path[fill=fillColor,fill opacity=0.20] (117.34, 80.20) circle (  2.13);

\path[fill=fillColor,fill opacity=0.20] (103.36,102.14) circle (  2.13);

\path[fill=fillColor,fill opacity=0.20] (109.69,102.14) circle (  2.13);

\path[fill=fillColor,fill opacity=0.20] (122.15, 89.14) circle (  2.13);

\path[fill=fillColor,fill opacity=0.20] (106.20, 89.14) circle (  2.13);

\path[fill=fillColor,fill opacity=0.20] ( 96.58, 92.39) circle (  2.13);

\path[fill=fillColor,fill opacity=0.20] (106.63, 90.76) circle (  2.13);

\path[fill=fillColor,fill opacity=0.20] ( 83.25, 91.58) circle (  2.13);

\path[fill=fillColor,fill opacity=0.20] ( 88.06, 81.01) circle (  2.13);

\path[fill=fillColor,fill opacity=0.20] ( 87.62, 59.07) circle (  2.13);

\path[fill=fillColor,fill opacity=0.20] ( 74.30, 59.88) circle (  2.13);

\path[fill=fillColor,fill opacity=0.20] ( 82.60, 51.75) circle (  2.13);

\path[fill=fillColor,fill opacity=0.20] ( 90.68, 73.70) circle (  2.13);

\path[fill=fillColor,fill opacity=0.20] ( 92.43, 88.33) circle (  2.13);

\path[fill=fillColor,fill opacity=0.20] ( 90.90, 85.89) circle (  2.13);

\path[fill=fillColor,fill opacity=0.20] ( 88.28, 76.14) circle (  2.13);

\path[fill=fillColor,fill opacity=0.20] ( 90.68, 76.95) circle (  2.13);

\path[fill=fillColor,fill opacity=0.20] ( 81.29, 93.20) circle (  2.13);

\path[fill=fillColor,fill opacity=0.20] ( 89.81, 86.70) circle (  2.13);

\path[fill=fillColor,fill opacity=0.20] ( 82.60, 70.45) circle (  2.13);

\path[fill=fillColor,fill opacity=0.20] ( 95.93, 89.95) circle (  2.13);

\path[fill=fillColor,fill opacity=0.20] ( 89.81, 91.58) circle (  2.13);

\path[fill=fillColor,fill opacity=0.20] ( 88.28,105.39) circle (  2.13);

\path[fill=fillColor,fill opacity=0.20] ( 97.89, 98.89) circle (  2.13);

\path[fill=fillColor,fill opacity=0.20] (107.73, 89.95) circle (  2.13);

\path[fill=fillColor,fill opacity=0.20] (100.52, 91.58) circle (  2.13);

\path[fill=fillColor,fill opacity=0.20] ( 98.33, 89.14) circle (  2.13);

\path[fill=fillColor,fill opacity=0.20] (101.17, 83.45) circle (  2.13);

\path[fill=fillColor,fill opacity=0.20] ( 92.87, 81.82) circle (  2.13);

\path[fill=fillColor,fill opacity=0.20] ( 85.88, 71.26) circle (  2.13);

\path[fill=fillColor,fill opacity=0.20] ( 85.66, 53.38) circle (  2.13);

\path[fill=fillColor,fill opacity=0.20] ( 66.43, 61.51) circle (  2.13);

\path[fill=fillColor,fill opacity=0.20] ( 72.11, 49.32) circle (  2.13);

\path[fill=fillColor,fill opacity=0.20] ( 80.63, 66.38) circle (  2.13);

\path[fill=fillColor,fill opacity=0.20] ( 82.38, 78.57) circle (  2.13);

\path[fill=fillColor,fill opacity=0.20] ( 91.99, 79.39) circle (  2.13);

\path[fill=fillColor,fill opacity=0.20] ( 86.75, 80.20) circle (  2.13);

\path[fill=fillColor,fill opacity=0.20] ( 87.62, 79.39) circle (  2.13);

\path[fill=fillColor,fill opacity=0.20] ( 88.50, 84.26) circle (  2.13);

\path[fill=fillColor,fill opacity=0.20] ( 79.32, 81.82) circle (  2.13);

\path[fill=fillColor,fill opacity=0.20] ( 75.83, 68.01) circle (  2.13);

\path[fill=fillColor,fill opacity=0.20] ( 64.90, 73.70) circle (  2.13);

\path[fill=fillColor,fill opacity=0.20] ( 86.31,102.14) circle (  2.13);

\path[fill=fillColor,fill opacity=0.20] (110.13, 88.33) circle (  2.13);

\path[fill=fillColor,fill opacity=0.20] ( 83.25, 99.70) circle (  2.13);

\path[fill=fillColor,fill opacity=0.20] ( 87.62, 98.08) circle (  2.13);

\path[fill=fillColor,fill opacity=0.20] ( 95.05, 91.58) circle (  2.13);

\path[fill=fillColor,fill opacity=0.20] ( 98.33, 87.51) circle (  2.13);

\path[fill=fillColor,fill opacity=0.20] (101.17, 90.76) circle (  2.13);

\path[fill=fillColor,fill opacity=0.20] ( 91.56, 85.08) circle (  2.13);

\path[fill=fillColor,fill opacity=0.20] (106.63, 74.51) circle (  2.13);

\path[fill=fillColor,fill opacity=0.20] (102.92, 73.70) circle (  2.13);

\path[fill=fillColor,fill opacity=0.20] ( 90.47, 65.57) circle (  2.13);

\path[fill=fillColor,fill opacity=0.20] ( 83.25, 51.75) circle (  2.13);

\path[fill=fillColor,fill opacity=0.20] ( 71.46, 64.76) circle (  2.13);

\path[fill=fillColor,fill opacity=0.20] ( 89.59, 65.57) circle (  2.13);

\path[fill=fillColor,fill opacity=0.20] ( 90.68, 66.38) circle (  2.13);

\path[fill=fillColor,fill opacity=0.20] ( 85.00, 77.76) circle (  2.13);

\path[fill=fillColor,fill opacity=0.20] ( 89.59, 82.64) circle (  2.13);

\path[fill=fillColor,fill opacity=0.20] ( 90.68, 74.51) circle (  2.13);

\path[fill=fillColor,fill opacity=0.20] ( 86.75, 65.57) circle (  2.13);

\path[fill=fillColor,fill opacity=0.20] ( 77.79, 60.69) circle (  2.13);

\path[fill=fillColor,fill opacity=0.20] ( 76.26, 61.51) circle (  2.13);

\path[fill=fillColor,fill opacity=0.20] ( 67.52, 72.07) circle (  2.13);

\path[fill=fillColor,fill opacity=0.20] ( 72.99,113.52) circle (  2.13);

\path[fill=fillColor,fill opacity=0.20] ( 97.24, 83.45) circle (  2.13);

\path[fill=fillColor,fill opacity=0.20] ( 96.15, 92.39) circle (  2.13);

\path[fill=fillColor,fill opacity=0.20] ( 96.80, 89.95) circle (  2.13);

\path[fill=fillColor,fill opacity=0.20] (105.32, 87.51) circle (  2.13);

\path[fill=fillColor,fill opacity=0.20] (102.26, 89.14) circle (  2.13);

\path[fill=fillColor,fill opacity=0.20] ( 99.64, 89.14) circle (  2.13);

\path[fill=fillColor,fill opacity=0.20] (100.95, 84.26) circle (  2.13);

\path[fill=fillColor,fill opacity=0.20] ( 98.99, 76.95) circle (  2.13);

\path[fill=fillColor,fill opacity=0.20] (103.36, 74.51) circle (  2.13);

\path[fill=fillColor,fill opacity=0.20] ( 95.93, 76.14) circle (  2.13);

\path[fill=fillColor,fill opacity=0.20] ( 94.84, 66.38) circle (  2.13);

\path[fill=fillColor,fill opacity=0.20] ( 78.67, 56.63) circle (  2.13);

\path[fill=fillColor,fill opacity=0.20] ( 88.06, 62.32) circle (  2.13);

\path[fill=fillColor,fill opacity=0.20] ( 99.42, 53.38) circle (  2.13);

\path[fill=fillColor,fill opacity=0.20] ( 84.35, 63.13) circle (  2.13);

\path[fill=fillColor,fill opacity=0.20] ( 89.37, 73.70) circle (  2.13);

\path[fill=fillColor,fill opacity=0.20] ( 89.81, 71.26) circle (  2.13);

\path[fill=fillColor,fill opacity=0.20] ( 85.22, 68.01) circle (  2.13);

\path[fill=fillColor,fill opacity=0.20] ( 86.97, 64.76) circle (  2.13);

\path[fill=fillColor,fill opacity=0.20] ( 78.45, 65.57) circle (  2.13);

\path[fill=fillColor,fill opacity=0.20] ( 75.17, 72.89) circle (  2.13);

\path[fill=fillColor,fill opacity=0.20] ( 60.31, 81.01) circle (  2.13);

\path[fill=fillColor,fill opacity=0.20] ( 97.46, 86.70) circle (  2.13);

\path[fill=fillColor,fill opacity=0.20] ( 94.18, 95.64) circle (  2.13);

\path[fill=fillColor,fill opacity=0.20] ( 97.46, 88.33) circle (  2.13);

\path[fill=fillColor,fill opacity=0.20] (107.29, 81.01) circle (  2.13);

\path[fill=fillColor,fill opacity=0.20] (108.38, 85.08) circle (  2.13);

\path[fill=fillColor,fill opacity=0.20] (102.70, 92.39) circle (  2.13);

\path[fill=fillColor,fill opacity=0.20] ( 98.33, 94.83) circle (  2.13);

\path[fill=fillColor,fill opacity=0.20] (107.29, 85.89) circle (  2.13);

\path[fill=fillColor,fill opacity=0.20] ( 98.33, 77.76) circle (  2.13);

\path[fill=fillColor,fill opacity=0.20] (100.52, 75.32) circle (  2.13);

\path[fill=fillColor,fill opacity=0.20] (100.52, 69.63) circle (  2.13);

\path[fill=fillColor,fill opacity=0.20] ( 86.10, 64.76) circle (  2.13);

\path[fill=fillColor,fill opacity=0.20] ( 57.69, 69.63) circle (  2.13);

\path[fill=fillColor,fill opacity=0.20] ( 55.51, 72.89) circle (  2.13);

\path[fill=fillColor,fill opacity=0.20] ( 75.61, 55.01) circle (  2.13);

\path[fill=fillColor,fill opacity=0.20] ( 88.28, 55.01) circle (  2.13);

\path[fill=fillColor,fill opacity=0.20] ( 86.53, 63.95) circle (  2.13);

\path[fill=fillColor,fill opacity=0.20] ( 81.73, 75.32) circle (  2.13);

\path[fill=fillColor,fill opacity=0.20] ( 89.37, 78.57) circle (  2.13);

\path[fill=fillColor,fill opacity=0.20] ( 86.97, 73.70) circle (  2.13);

\path[fill=fillColor,fill opacity=0.20] ( 86.10, 72.89) circle (  2.13);

\path[fill=fillColor,fill opacity=0.20] ( 77.36, 79.39) circle (  2.13);

\path[fill=fillColor,fill opacity=0.20] ( 72.77, 82.64) circle (  2.13);

\path[fill=fillColor,fill opacity=0.20] ( 65.77, 81.82) circle (  2.13);

\path[fill=fillColor,fill opacity=0.20] ( 97.68, 85.08) circle (  2.13);

\path[fill=fillColor,fill opacity=0.20] ( 94.18, 96.45) circle (  2.13);

\path[fill=fillColor,fill opacity=0.20] ( 95.05, 96.45) circle (  2.13);

\path[fill=fillColor,fill opacity=0.20] (100.95, 88.33) circle (  2.13);

\path[fill=fillColor,fill opacity=0.20] (102.05, 92.39) circle (  2.13);

\path[fill=fillColor,fill opacity=0.20] ( 99.21, 94.02) circle (  2.13);

\path[fill=fillColor,fill opacity=0.20] (100.08, 92.39) circle (  2.13);

\path[fill=fillColor,fill opacity=0.20] (103.58, 91.58) circle (  2.13);

\path[fill=fillColor,fill opacity=0.20] (100.08, 89.95) circle (  2.13);

\path[fill=fillColor,fill opacity=0.20] (104.23, 87.51) circle (  2.13);

\path[fill=fillColor,fill opacity=0.20] (102.26, 72.89) circle (  2.13);

\path[fill=fillColor,fill opacity=0.20] ( 88.06, 53.38) circle (  2.13);

\path[fill=fillColor,fill opacity=0.20] ( 55.94, 72.89) circle (  2.13);

\path[fill=fillColor,fill opacity=0.20] ( 85.00, 63.95) circle (  2.13);

\path[fill=fillColor,fill opacity=0.20] ( 93.74, 59.07) circle (  2.13);

\path[fill=fillColor,fill opacity=0.20] ( 85.22, 69.63) circle (  2.13);

\path[fill=fillColor,fill opacity=0.20] ( 91.34, 72.07) circle (  2.13);

\path[fill=fillColor,fill opacity=0.20] ( 98.11, 68.01) circle (  2.13);

\path[fill=fillColor,fill opacity=0.20] ( 86.75, 72.89) circle (  2.13);

\path[fill=fillColor,fill opacity=0.20] ( 84.78, 76.14) circle (  2.13);

\path[fill=fillColor,fill opacity=0.20] ( 83.25, 75.32) circle (  2.13);

\path[fill=fillColor,fill opacity=0.20] ( 83.47, 72.89) circle (  2.13);

\path[fill=fillColor,fill opacity=0.20] ( 68.83, 72.07) circle (  2.13);

\path[fill=fillColor,fill opacity=0.20] ( 92.21, 92.39) circle (  2.13);

\path[fill=fillColor,fill opacity=0.20] ( 93.09,100.52) circle (  2.13);

\path[fill=fillColor,fill opacity=0.20] (103.14, 96.45) circle (  2.13);

\path[fill=fillColor,fill opacity=0.20] (103.36, 87.51) circle (  2.13);

\path[fill=fillColor,fill opacity=0.20] (105.76, 88.33) circle (  2.13);

\path[fill=fillColor,fill opacity=0.20] (102.48, 94.02) circle (  2.13);

\path[fill=fillColor,fill opacity=0.20] (100.73, 96.45) circle (  2.13);

\path[fill=fillColor,fill opacity=0.20] (103.79, 89.14) circle (  2.13);

\path[fill=fillColor,fill opacity=0.20] (110.35, 85.08) circle (  2.13);

\path[fill=fillColor,fill opacity=0.20] (104.89, 88.33) circle (  2.13);

\path[fill=fillColor,fill opacity=0.20] ( 98.11, 66.38) circle (  2.13);

\path[fill=fillColor,fill opacity=0.20] ( 76.70, 46.88) circle (  2.13);

\path[fill=fillColor,fill opacity=0.20] ( 55.51, 78.57) circle (  2.13);

\path[fill=fillColor,fill opacity=0.20] ( 91.12, 59.88) circle (  2.13);

\path[fill=fillColor,fill opacity=0.20] (101.39, 55.82) circle (  2.13);

\path[fill=fillColor,fill opacity=0.20] ( 98.33, 57.44) circle (  2.13);

\path[fill=fillColor,fill opacity=0.20] (102.48, 61.51) circle (  2.13);

\path[fill=fillColor,fill opacity=0.20] ( 91.78, 70.45) circle (  2.13);

\path[fill=fillColor,fill opacity=0.20] ( 83.69, 72.89) circle (  2.13);

\path[fill=fillColor,fill opacity=0.20] ( 83.91, 76.14) circle (  2.13);

\path[fill=fillColor,fill opacity=0.20] ( 76.26, 77.76) circle (  2.13);

\path[fill=fillColor,fill opacity=0.20] ( 72.77, 74.51) circle (  2.13);

\path[fill=fillColor,fill opacity=0.20] ( 66.87, 81.82) circle (  2.13);

\path[fill=fillColor,fill opacity=0.20] ( 91.34, 95.64) circle (  2.13);

\path[fill=fillColor,fill opacity=0.20] ( 86.31,105.39) circle (  2.13);

\path[fill=fillColor,fill opacity=0.20] ( 98.11, 98.89) circle (  2.13);

\path[fill=fillColor,fill opacity=0.20] (113.41, 88.33) circle (  2.13);

\path[fill=fillColor,fill opacity=0.20] (116.69, 85.89) circle (  2.13);

\path[fill=fillColor,fill opacity=0.20] (117.34, 84.26) circle (  2.13);

\path[fill=fillColor,fill opacity=0.20] (109.04, 85.89) circle (  2.13);

\path[fill=fillColor,fill opacity=0.20] (106.63, 89.14) circle (  2.13);

\path[fill=fillColor,fill opacity=0.20] (101.61, 89.14) circle (  2.13);

\path[fill=fillColor,fill opacity=0.20] (100.08, 85.89) circle (  2.13);

\path[fill=fillColor,fill opacity=0.20] (111.88, 80.20) circle (  2.13);

\path[fill=fillColor,fill opacity=0.20] (104.67, 66.38) circle (  2.13);

\path[fill=fillColor,fill opacity=0.20] ( 74.73, 59.88) circle (  2.13);

\path[fill=fillColor,fill opacity=0.20] ( 53.98, 72.07) circle (  2.13);

\path[fill=fillColor,fill opacity=0.20] ( 98.77, 55.82) circle (  2.13);

\path[fill=fillColor,fill opacity=0.20] (102.92, 56.63) circle (  2.13);

\path[fill=fillColor,fill opacity=0.20] ( 97.02, 63.95) circle (  2.13);

\path[fill=fillColor,fill opacity=0.20] ( 97.68, 68.82) circle (  2.13);

\path[fill=fillColor,fill opacity=0.20] ( 84.78, 75.32) circle (  2.13);

\path[fill=fillColor,fill opacity=0.20] ( 82.82, 83.45) circle (  2.13);

\path[fill=fillColor,fill opacity=0.20] ( 81.51, 81.82) circle (  2.13);

\path[fill=fillColor,fill opacity=0.20] ( 72.33, 79.39) circle (  2.13);

\path[fill=fillColor,fill opacity=0.20] ( 72.33, 79.39) circle (  2.13);

\path[fill=fillColor,fill opacity=0.20] ( 68.18, 75.32) circle (  2.13);

\path[fill=fillColor,fill opacity=0.20] ( 70.36, 90.76) circle (  2.13);

\path[fill=fillColor,fill opacity=0.20] ( 97.24, 76.14) circle (  2.13);

\path[fill=fillColor,fill opacity=0.20] (102.70,101.33) circle (  2.13);

\path[fill=fillColor,fill opacity=0.20] ( 95.71,105.39) circle (  2.13);

\path[fill=fillColor,fill opacity=0.20] (112.10, 89.14) circle (  2.13);

\path[fill=fillColor,fill opacity=0.20] (112.10, 89.14) circle (  2.13);

\path[fill=fillColor,fill opacity=0.20] (116.90, 94.02) circle (  2.13);

\path[fill=fillColor,fill opacity=0.20] (114.72, 89.95) circle (  2.13);

\path[fill=fillColor,fill opacity=0.20] (110.79, 81.82) circle (  2.13);

\path[fill=fillColor,fill opacity=0.20] (109.47, 81.01) circle (  2.13);

\path[fill=fillColor,fill opacity=0.20] (100.95, 83.45) circle (  2.13);

\path[fill=fillColor,fill opacity=0.20] ( 98.33, 76.14) circle (  2.13);

\path[fill=fillColor,fill opacity=0.20] ( 97.68, 64.76) circle (  2.13);

\path[fill=fillColor,fill opacity=0.20] ( 70.14, 60.69) circle (  2.13);

\path[fill=fillColor,fill opacity=0.20] ( 91.34, 63.13) circle (  2.13);

\path[fill=fillColor,fill opacity=0.20] (100.30, 66.38) circle (  2.13);

\path[fill=fillColor,fill opacity=0.20] ( 91.34, 60.69) circle (  2.13);

\path[fill=fillColor,fill opacity=0.20] ( 87.41, 63.95) circle (  2.13);

\path[fill=fillColor,fill opacity=0.20] ( 84.78, 78.57) circle (  2.13);

\path[fill=fillColor,fill opacity=0.20] ( 82.60, 80.20) circle (  2.13);

\path[fill=fillColor,fill opacity=0.20] ( 80.41, 76.14) circle (  2.13);

\path[fill=fillColor,fill opacity=0.20] ( 72.77, 76.95) circle (  2.13);

\path[fill=fillColor,fill opacity=0.20] ( 70.80, 71.26) circle (  2.13);

\path[fill=fillColor,fill opacity=0.20] ( 81.73, 66.38) circle (  2.13);

\path[fill=fillColor,fill opacity=0.20] ( 70.58, 68.82) circle (  2.13);

\path[fill=fillColor,fill opacity=0.20] ( 58.35, 82.64) circle (  2.13);

\path[fill=fillColor,fill opacity=0.20] ( 78.01, 67.20) circle (  2.13);

\path[fill=fillColor,fill opacity=0.20] ( 87.19, 72.07) circle (  2.13);

\path[fill=fillColor,fill opacity=0.20] ( 98.33, 84.26) circle (  2.13);

\path[fill=fillColor,fill opacity=0.20] ( 92.65, 90.76) circle (  2.13);

\path[fill=fillColor,fill opacity=0.20] (102.92, 93.20) circle (  2.13);

\path[fill=fillColor,fill opacity=0.20] (105.98, 91.58) circle (  2.13);

\path[fill=fillColor,fill opacity=0.20] (106.42, 94.02) circle (  2.13);

\path[fill=fillColor,fill opacity=0.20] (110.79, 99.70) circle (  2.13);

\path[fill=fillColor,fill opacity=0.20] (111.00, 94.02) circle (  2.13);

\path[fill=fillColor,fill opacity=0.20] (111.66, 81.01) circle (  2.13);

\path[fill=fillColor,fill opacity=0.20] (106.85, 75.32) circle (  2.13);

\path[fill=fillColor,fill opacity=0.20] (100.08, 71.26) circle (  2.13);

\path[fill=fillColor,fill opacity=0.20] ( 85.88, 60.69) circle (  2.13);

\path[fill=fillColor,fill opacity=0.20] ( 71.24, 59.88) circle (  2.13);

\path[fill=fillColor,fill opacity=0.20] ( 68.62, 66.38) circle (  2.13);

\path[fill=fillColor,fill opacity=0.20] ( 85.44, 55.01) circle (  2.13);

\path[fill=fillColor,fill opacity=0.20] ( 94.40, 47.69) circle (  2.13);

\path[fill=fillColor,fill opacity=0.20] ( 92.65, 58.26) circle (  2.13);

\path[fill=fillColor,fill opacity=0.20] ( 85.88, 72.07) circle (  2.13);

\path[fill=fillColor,fill opacity=0.20] ( 84.13, 73.70) circle (  2.13);

\path[fill=fillColor,fill opacity=0.20] ( 76.70, 76.95) circle (  2.13);

\path[fill=fillColor,fill opacity=0.20] ( 70.36, 85.08) circle (  2.13);

\path[fill=fillColor,fill opacity=0.20] ( 69.27, 85.08) circle (  2.13);

\path[fill=fillColor,fill opacity=0.20] ( 73.86, 73.70) circle (  2.13);

\path[fill=fillColor,fill opacity=0.20] ( 71.89, 61.51) circle (  2.13);

\path[fill=fillColor,fill opacity=0.20] ( 72.33, 61.51) circle (  2.13);

\path[fill=fillColor,fill opacity=0.20] ( 64.90, 72.07) circle (  2.13);

\path[fill=fillColor,fill opacity=0.20] ( 67.52, 63.95) circle (  2.13);

\path[fill=fillColor,fill opacity=0.20] ( 76.04, 68.82) circle (  2.13);

\path[fill=fillColor,fill opacity=0.20] ( 85.66, 74.51) circle (  2.13);

\path[fill=fillColor,fill opacity=0.20] ( 86.75, 78.57) circle (  2.13);

\path[fill=fillColor,fill opacity=0.20] (116.90, 85.08) circle (  2.13);

\path[fill=fillColor,fill opacity=0.20] ( 93.09, 88.33) circle (  2.13);

\path[fill=fillColor,fill opacity=0.20] ( 93.96, 82.64) circle (  2.13);

\path[fill=fillColor,fill opacity=0.20] (103.58, 84.26) circle (  2.13);

\path[fill=fillColor,fill opacity=0.20] (105.10, 90.76) circle (  2.13);

\path[fill=fillColor,fill opacity=0.20] (105.54, 92.39) circle (  2.13);

\path[fill=fillColor,fill opacity=0.20] (108.38, 93.20) circle (  2.13);

\path[fill=fillColor,fill opacity=0.20] (109.47, 94.02) circle (  2.13);

\path[fill=fillColor,fill opacity=0.20] ( 99.86, 85.89) circle (  2.13);

\path[fill=fillColor,fill opacity=0.20] ( 94.62, 72.07) circle (  2.13);

\path[fill=fillColor,fill opacity=0.20] ( 80.85, 68.01) circle (  2.13);

\path[fill=fillColor,fill opacity=0.20] ( 67.96, 61.51) circle (  2.13);

\path[fill=fillColor,fill opacity=0.20] ( 78.67, 50.94) circle (  2.13);

\path[fill=fillColor,fill opacity=0.20] ( 91.34, 48.50) circle (  2.13);

\path[fill=fillColor,fill opacity=0.20] ( 93.74, 60.69) circle (  2.13);

\path[fill=fillColor,fill opacity=0.20] ( 87.84, 68.82) circle (  2.13);

\path[fill=fillColor,fill opacity=0.20] ( 89.15, 71.26) circle (  2.13);

\path[fill=fillColor,fill opacity=0.20] ( 80.41, 82.64) circle (  2.13);

\path[fill=fillColor,fill opacity=0.20] ( 77.57, 87.51) circle (  2.13);

\path[fill=fillColor,fill opacity=0.20] ( 75.61, 81.82) circle (  2.13);

\path[fill=fillColor,fill opacity=0.20] ( 75.17, 72.07) circle (  2.13);

\path[fill=fillColor,fill opacity=0.20] ( 75.39, 67.20) circle (  2.13);

\path[fill=fillColor,fill opacity=0.20] ( 77.57, 72.07) circle (  2.13);

\path[fill=fillColor,fill opacity=0.20] ( 70.80, 74.51) circle (  2.13);

\path[fill=fillColor,fill opacity=0.20] ( 70.58, 64.76) circle (  2.13);

\path[fill=fillColor,fill opacity=0.20] ( 65.99, 68.01) circle (  2.13);

\path[fill=fillColor,fill opacity=0.20] ( 66.65, 58.26) circle (  2.13);

\path[fill=fillColor,fill opacity=0.20] ( 66.21, 64.76) circle (  2.13);

\path[fill=fillColor,fill opacity=0.20] ( 90.68, 72.07) circle (  2.13);

\path[fill=fillColor,fill opacity=0.20] ( 85.66, 67.20) circle (  2.13);

\path[fill=fillColor,fill opacity=0.20] ( 86.10, 72.07) circle (  2.13);

\path[fill=fillColor,fill opacity=0.20] ( 95.71, 89.14) circle (  2.13);

\path[fill=fillColor,fill opacity=0.20] ( 89.59, 89.14) circle (  2.13);

\path[fill=fillColor,fill opacity=0.20] ( 97.68, 81.01) circle (  2.13);

\path[fill=fillColor,fill opacity=0.20] ( 90.68, 83.45) circle (  2.13);

\path[fill=fillColor,fill opacity=0.20] (101.83, 82.64) circle (  2.13);

\path[fill=fillColor,fill opacity=0.20] (107.95, 82.64) circle (  2.13);

\path[fill=fillColor,fill opacity=0.20] (119.31, 86.70) circle (  2.13);

\path[fill=fillColor,fill opacity=0.20] (105.98, 81.82) circle (  2.13);

\path[fill=fillColor,fill opacity=0.20] ( 96.80, 80.20) circle (  2.13);

\path[fill=fillColor,fill opacity=0.20] ( 92.21, 84.26) circle (  2.13);

\path[fill=fillColor,fill opacity=0.20] ( 78.67, 83.45) circle (  2.13);

\path[fill=fillColor,fill opacity=0.20] ( 75.61, 73.70) circle (  2.13);

\path[fill=fillColor,fill opacity=0.20] ( 67.30, 60.69) circle (  2.13);

\path[fill=fillColor,fill opacity=0.20] ( 80.63, 63.95) circle (  2.13);

\path[fill=fillColor,fill opacity=0.20] ( 91.34, 64.76) circle (  2.13);

\path[fill=fillColor,fill opacity=0.20] ( 97.89, 58.26) circle (  2.13);

\path[fill=fillColor,fill opacity=0.20] ( 98.11, 61.51) circle (  2.13);

\path[fill=fillColor,fill opacity=0.20] ( 87.84, 72.89) circle (  2.13);

\path[fill=fillColor,fill opacity=0.20] ( 79.10, 76.95) circle (  2.13);

\path[fill=fillColor,fill opacity=0.20] ( 78.01, 79.39) circle (  2.13);

\path[fill=fillColor,fill opacity=0.20] ( 75.83, 82.64) circle (  2.13);

\path[fill=fillColor,fill opacity=0.20] ( 76.70, 79.39) circle (  2.13);

\path[fill=fillColor,fill opacity=0.20] ( 73.20, 79.39) circle (  2.13);

\path[fill=fillColor,fill opacity=0.20] ( 71.46, 76.14) circle (  2.13);

\path[fill=fillColor,fill opacity=0.20] ( 71.67, 69.63) circle (  2.13);

\path[fill=fillColor,fill opacity=0.20] ( 72.33, 65.57) circle (  2.13);

\path[fill=fillColor,fill opacity=0.20] ( 74.95, 64.76) circle (  2.13);

\path[fill=fillColor,fill opacity=0.20] ( 71.46, 66.38) circle (  2.13);

\path[fill=fillColor,fill opacity=0.20] ( 65.99, 71.26) circle (  2.13);

\path[fill=fillColor,fill opacity=0.20] ( 76.92, 60.69) circle (  2.13);

\path[fill=fillColor,fill opacity=0.20] ( 71.24, 48.50) circle (  2.13);

\path[fill=fillColor,fill opacity=0.20] ( 67.30, 59.07) circle (  2.13);

\path[fill=fillColor,fill opacity=0.20] ( 68.62, 70.45) circle (  2.13);

\path[fill=fillColor,fill opacity=0.20] ( 74.30, 64.76) circle (  2.13);

\path[fill=fillColor,fill opacity=0.20] ( 72.11, 59.07) circle (  2.13);

\path[fill=fillColor,fill opacity=0.20] ( 71.46, 62.32) circle (  2.13);

\path[fill=fillColor,fill opacity=0.20] ( 69.71, 59.88) circle (  2.13);

\path[fill=fillColor,fill opacity=0.20] ( 77.57, 53.38) circle (  2.13);

\path[fill=fillColor,fill opacity=0.20] ( 74.08, 48.50) circle (  2.13);

\path[fill=fillColor,fill opacity=0.20] ( 76.26, 49.32) circle (  2.13);

\path[fill=fillColor,fill opacity=0.20] ( 78.23, 55.82) circle (  2.13);

\path[fill=fillColor,fill opacity=0.20] ( 74.95, 59.88) circle (  2.13);

\path[fill=fillColor,fill opacity=0.20] ( 76.70, 56.63) circle (  2.13);

\path[fill=fillColor,fill opacity=0.20] ( 76.26, 55.01) circle (  2.13);

\path[fill=fillColor,fill opacity=0.20] ( 82.82, 55.01) circle (  2.13);

\path[fill=fillColor,fill opacity=0.20] ( 80.20, 57.44) circle (  2.13);

\path[fill=fillColor,fill opacity=0.20] ( 88.94, 69.63) circle (  2.13);

\path[fill=fillColor,fill opacity=0.20] ( 89.81, 76.14) circle (  2.13);

\path[fill=fillColor,fill opacity=0.20] ( 84.13, 73.70) circle (  2.13);

\path[fill=fillColor,fill opacity=0.20] ( 88.72, 78.57) circle (  2.13);

\path[fill=fillColor,fill opacity=0.20] ( 99.42, 80.20) circle (  2.13);

\path[fill=fillColor,fill opacity=0.20] (101.39, 76.95) circle (  2.13);

\path[fill=fillColor,fill opacity=0.20] (100.30, 73.70) circle (  2.13);

\path[fill=fillColor,fill opacity=0.20] ( 97.68, 73.70) circle (  2.13);

\path[fill=fillColor,fill opacity=0.20] ( 97.02, 80.20) circle (  2.13);

\path[fill=fillColor,fill opacity=0.20] ( 88.72, 89.14) circle (  2.13);

\path[fill=fillColor,fill opacity=0.20] ( 80.20, 82.64) circle (  2.13);

\path[fill=fillColor,fill opacity=0.20] ( 89.15, 79.39) circle (  2.13);

\path[fill=fillColor,fill opacity=0.20] ( 74.73, 80.20) circle (  2.13);

\path[fill=fillColor,fill opacity=0.20] ( 73.20, 69.63) circle (  2.13);

\path[fill=fillColor,fill opacity=0.20] ( 74.30, 64.76) circle (  2.13);

\path[fill=fillColor,fill opacity=0.20] ( 82.82, 58.26) circle (  2.13);

\path[fill=fillColor,fill opacity=0.20] ( 90.90, 54.19) circle (  2.13);

\path[fill=fillColor,fill opacity=0.20] ( 91.78, 61.51) circle (  2.13);

\path[fill=fillColor,fill opacity=0.20] ( 87.19, 68.82) circle (  2.13);

\path[fill=fillColor,fill opacity=0.20] ( 82.82, 68.82) circle (  2.13);

\path[fill=fillColor,fill opacity=0.20] ( 79.76, 71.26) circle (  2.13);

\path[fill=fillColor,fill opacity=0.20] ( 74.95, 74.51) circle (  2.13);

\path[fill=fillColor,fill opacity=0.20] ( 74.51, 74.51) circle (  2.13);

\path[fill=fillColor,fill opacity=0.20] ( 74.73, 77.76) circle (  2.13);

\path[fill=fillColor,fill opacity=0.20] ( 71.02, 78.57) circle (  2.13);

\path[fill=fillColor,fill opacity=0.20] ( 75.61, 75.32) circle (  2.13);

\path[fill=fillColor,fill opacity=0.20] ( 82.82, 76.14) circle (  2.13);

\path[fill=fillColor,fill opacity=0.20] ( 73.20, 76.14) circle (  2.13);

\path[fill=fillColor,fill opacity=0.20] ( 76.48, 72.89) circle (  2.13);

\path[fill=fillColor,fill opacity=0.20] ( 74.08, 64.76) circle (  2.13);

\path[fill=fillColor,fill opacity=0.20] ( 76.26, 55.01) circle (  2.13);

\path[fill=fillColor,fill opacity=0.20] ( 75.83, 56.63) circle (  2.13);

\path[fill=fillColor,fill opacity=0.20] ( 79.32, 62.32) circle (  2.13);

\path[fill=fillColor,fill opacity=0.20] ( 72.77, 58.26) circle (  2.13);

\path[fill=fillColor,fill opacity=0.20] ( 79.32, 59.07) circle (  2.13);

\path[fill=fillColor,fill opacity=0.20] ( 77.79, 63.95) circle (  2.13);

\path[fill=fillColor,fill opacity=0.20] ( 79.76, 65.57) circle (  2.13);

\path[fill=fillColor,fill opacity=0.20] ( 85.22, 67.20) circle (  2.13);

\path[fill=fillColor,fill opacity=0.20] ( 82.60, 68.01) circle (  2.13);

\path[fill=fillColor,fill opacity=0.20] ( 83.69, 65.57) circle (  2.13);

\path[fill=fillColor,fill opacity=0.20] ( 82.38, 67.20) circle (  2.13);

\path[fill=fillColor,fill opacity=0.20] ( 79.54, 70.45) circle (  2.13);

\path[fill=fillColor,fill opacity=0.20] ( 86.97, 68.01) circle (  2.13);

\path[fill=fillColor,fill opacity=0.20] ( 89.15, 62.32) circle (  2.13);

\path[fill=fillColor,fill opacity=0.20] ( 92.21, 61.51) circle (  2.13);

\path[fill=fillColor,fill opacity=0.20] ( 95.71, 66.38) circle (  2.13);

\path[fill=fillColor,fill opacity=0.20] ( 90.68, 60.69) circle (  2.13);

\path[fill=fillColor,fill opacity=0.20] ( 96.15, 52.57) circle (  2.13);

\path[fill=fillColor,fill opacity=0.20] ( 90.68, 63.13) circle (  2.13);

\path[fill=fillColor,fill opacity=0.20] ( 90.68, 72.89) circle (  2.13);

\path[fill=fillColor,fill opacity=0.20] ( 95.27, 70.45) circle (  2.13);

\path[fill=fillColor,fill opacity=0.20] ( 87.19, 75.32) circle (  2.13);

\path[fill=fillColor,fill opacity=0.20] ( 82.16, 76.14) circle (  2.13);

\path[fill=fillColor,fill opacity=0.20] ( 81.51, 75.32) circle (  2.13);

\path[fill=fillColor,fill opacity=0.20] ( 75.61, 85.08) circle (  2.13);

\path[fill=fillColor,fill opacity=0.20] ( 67.09, 85.08) circle (  2.13);

\path[fill=fillColor,fill opacity=0.20] ( 83.25, 75.32) circle (  2.13);

\path[fill=fillColor,fill opacity=0.20] ( 61.19, 67.20) circle (  2.13);

\path[fill=fillColor,fill opacity=0.20] ( 76.92, 63.13) circle (  2.13);

\path[fill=fillColor,fill opacity=0.20] ( 79.32, 61.51) circle (  2.13);

\path[fill=fillColor,fill opacity=0.20] ( 85.00, 64.76) circle (  2.13);

\path[fill=fillColor,fill opacity=0.20] ( 90.68, 63.13) circle (  2.13);

\path[fill=fillColor,fill opacity=0.20] ( 86.75, 58.26) circle (  2.13);

\path[fill=fillColor,fill opacity=0.20] ( 85.44, 59.07) circle (  2.13);

\path[fill=fillColor,fill opacity=0.20] ( 80.41, 68.01) circle (  2.13);

\path[fill=fillColor,fill opacity=0.20] ( 76.48, 76.14) circle (  2.13);

\path[fill=fillColor,fill opacity=0.20] ( 78.45, 77.76) circle (  2.13);

\path[fill=fillColor,fill opacity=0.20] ( 76.04, 79.39) circle (  2.13);

\path[fill=fillColor,fill opacity=0.20] ( 72.55, 82.64) circle (  2.13);

\path[fill=fillColor,fill opacity=0.20] ( 75.17, 81.82) circle (  2.13);

\path[fill=fillColor,fill opacity=0.20] ( 77.36, 76.14) circle (  2.13);

\path[fill=fillColor,fill opacity=0.20] ( 73.86, 75.32) circle (  2.13);

\path[fill=fillColor,fill opacity=0.20] ( 76.92, 73.70) circle (  2.13);

\path[fill=fillColor,fill opacity=0.20] ( 78.23, 72.07) circle (  2.13);

\path[fill=fillColor,fill opacity=0.20] ( 71.67, 70.45) circle (  2.13);

\path[fill=fillColor,fill opacity=0.20] ( 74.51, 69.63) circle (  2.13);

\path[fill=fillColor,fill opacity=0.20] ( 78.01, 71.26) circle (  2.13);

\path[fill=fillColor,fill opacity=0.20] ( 77.79, 70.45) circle (  2.13);

\path[fill=fillColor,fill opacity=0.20] ( 78.01, 72.89) circle (  2.13);

\path[fill=fillColor,fill opacity=0.20] ( 74.73, 79.39) circle (  2.13);

\path[fill=fillColor,fill opacity=0.20] ( 74.73, 81.01) circle (  2.13);

\path[fill=fillColor,fill opacity=0.20] ( 77.36, 76.14) circle (  2.13);

\path[fill=fillColor,fill opacity=0.20] ( 80.85, 71.26) circle (  2.13);

\path[fill=fillColor,fill opacity=0.20] ( 83.25, 66.38) circle (  2.13);

\path[fill=fillColor,fill opacity=0.20] ( 88.94, 64.76) circle (  2.13);

\path[fill=fillColor,fill opacity=0.20] ( 88.06, 62.32) circle (  2.13);

\path[fill=fillColor,fill opacity=0.20] ( 98.77, 59.07) circle (  2.13);

\path[fill=fillColor,fill opacity=0.20] ( 96.36, 65.57) circle (  2.13);

\path[fill=fillColor,fill opacity=0.20] ( 98.33, 72.07) circle (  2.13);

\path[fill=fillColor,fill opacity=0.20] ( 81.07, 69.63) circle (  2.13);

\path[fill=fillColor,fill opacity=0.20] ( 85.00, 70.45) circle (  2.13);

\path[fill=fillColor,fill opacity=0.20] ( 76.70, 77.76) circle (  2.13);

\path[fill=fillColor,fill opacity=0.20] ( 70.58, 83.45) circle (  2.13);

\path[fill=fillColor,fill opacity=0.20] ( 54.63, 91.58) circle (  2.13);

\path[fill=fillColor,fill opacity=0.20] ( 75.17, 71.26) circle (  2.13);

\path[fill=fillColor,fill opacity=0.20] ( 81.73, 58.26) circle (  2.13);

\path[fill=fillColor,fill opacity=0.20] ( 84.57, 53.38) circle (  2.13);

\path[fill=fillColor,fill opacity=0.20] ( 90.47, 62.32) circle (  2.13);

\path[fill=fillColor,fill opacity=0.20] ( 89.37, 71.26) circle (  2.13);

\path[fill=fillColor,fill opacity=0.20] ( 87.19, 69.63) circle (  2.13);

\path[fill=fillColor,fill opacity=0.20] ( 89.37, 70.45) circle (  2.13);

\path[fill=fillColor,fill opacity=0.20] ( 88.94, 73.70) circle (  2.13);

\path[fill=fillColor,fill opacity=0.20] ( 81.73, 71.26) circle (  2.13);

\path[fill=fillColor,fill opacity=0.20] ( 79.76, 67.20) circle (  2.13);

\path[fill=fillColor,fill opacity=0.20] ( 82.60, 63.13) circle (  2.13);

\path[fill=fillColor,fill opacity=0.20] ( 82.38, 63.95) circle (  2.13);

\path[fill=fillColor,fill opacity=0.20] ( 77.79, 68.01) circle (  2.13);

\path[fill=fillColor,fill opacity=0.20] ( 79.32, 68.01) circle (  2.13);

\path[fill=fillColor,fill opacity=0.20] ( 78.67, 63.13) circle (  2.13);

\path[fill=fillColor,fill opacity=0.20] ( 78.23, 59.07) circle (  2.13);

\path[fill=fillColor,fill opacity=0.20] ( 79.32, 56.63) circle (  2.13);

\path[fill=fillColor,fill opacity=0.20] ( 83.47, 60.69) circle (  2.13);

\path[fill=fillColor,fill opacity=0.20] ( 89.37, 67.20) circle (  2.13);

\path[fill=fillColor,fill opacity=0.20] ( 83.04, 67.20) circle (  2.13);

\path[fill=fillColor,fill opacity=0.20] ( 87.62, 65.57) circle (  2.13);

\path[fill=fillColor,fill opacity=0.20] ( 88.50, 63.95) circle (  2.13);

\path[fill=fillColor,fill opacity=0.20] ( 92.87, 59.88) circle (  2.13);

\path[fill=fillColor,fill opacity=0.20] ( 89.81, 61.51) circle (  2.13);

\path[fill=fillColor,fill opacity=0.20] ( 83.69, 71.26) circle (  2.13);

\path[fill=fillColor,fill opacity=0.20] ( 71.67, 76.14) circle (  2.13);

\path[fill=fillColor,fill opacity=0.20] ( 67.52, 81.01) circle (  2.13);

\path[fill=fillColor,fill opacity=0.20] ( 52.01, 95.64) circle (  2.13);

\path[fill=fillColor,fill opacity=0.20] ( 53.54, 98.89) circle (  2.13);

\path[fill=fillColor,fill opacity=0.20] ( 67.52, 65.57) circle (  2.13);

\path[fill=fillColor,fill opacity=0.20] ( 81.07, 64.76) circle (  2.13);

\path[fill=fillColor,fill opacity=0.20] ( 88.28, 64.76) circle (  2.13);

\path[fill=fillColor,fill opacity=0.20] ( 79.32, 66.38) circle (  2.13);

\path[fill=fillColor,fill opacity=0.20] ( 93.96, 62.32) circle (  2.13);

\path[fill=fillColor,fill opacity=0.20] ( 92.65, 52.57) circle (  2.13);

\path[fill=fillColor,fill opacity=0.20] ( 97.89, 58.26) circle (  2.13);

\path[fill=fillColor,fill opacity=0.20] ( 90.68, 66.38) circle (  2.13);

\path[fill=fillColor,fill opacity=0.20] ( 93.96, 54.19) circle (  2.13);

\path[fill=fillColor,fill opacity=0.20] ( 85.22, 45.25) circle (  2.13);

\path[fill=fillColor,fill opacity=0.20] ( 85.44, 55.01) circle (  2.13);

\path[fill=fillColor,fill opacity=0.20] ( 82.82, 58.26) circle (  2.13);

\path[fill=fillColor,fill opacity=0.20] ( 85.22, 54.19) circle (  2.13);

\path[fill=fillColor,fill opacity=0.20] ( 85.88, 50.94) circle (  2.13);

\path[fill=fillColor,fill opacity=0.20] ( 87.62, 50.13) circle (  2.13);

\path[fill=fillColor,fill opacity=0.20] ( 83.91, 53.38) circle (  2.13);

\path[fill=fillColor,fill opacity=0.20] ( 89.37, 59.07) circle (  2.13);

\path[fill=fillColor,fill opacity=0.20] ( 86.97, 61.51) circle (  2.13);

\path[fill=fillColor,fill opacity=0.20] ( 75.17, 69.63) circle (  2.13);

\path[fill=fillColor,fill opacity=0.20] ( 74.30, 75.32) circle (  2.13);

\path[fill=fillColor,fill opacity=0.20] ( 76.26, 72.07) circle (  2.13);

\path[fill=fillColor,fill opacity=0.20] ( 52.01, 81.01) circle (  2.13);

\path[fill=fillColor,fill opacity=0.20] ( 57.69, 76.14) circle (  2.13);

\path[fill=fillColor,fill opacity=0.20] ( 66.21, 63.95) circle (  2.13);

\path[fill=fillColor,fill opacity=0.20] ( 80.85, 68.01) circle (  2.13);

\path[fill=fillColor,fill opacity=0.20] ( 79.76, 73.70) circle (  2.13);

\path[fill=fillColor,fill opacity=0.20] ( 77.14, 67.20) circle (  2.13);

\path[fill=fillColor,fill opacity=0.20] ( 72.55, 63.95) circle (  2.13);

\path[fill=fillColor,fill opacity=0.20] ( 69.27, 62.32) circle (  2.13);

\path[fill=fillColor,fill opacity=0.20] ( 71.67, 61.51) circle (  2.13);

\path[fill=fillColor,fill opacity=0.20] ( 71.24, 62.32) circle (  2.13);

\path[fill=fillColor,fill opacity=0.20] ( 77.36, 61.51) circle (  2.13);

\path[fill=fillColor,fill opacity=0.20] ( 97.68,101.33) circle (  2.13);

\path[fill=fillColor,fill opacity=0.20] ( 96.58, 97.27) circle (  2.13);

\path[fill=fillColor,fill opacity=0.20] ( 89.15,106.21) circle (  2.13);

\path[fill=fillColor,fill opacity=0.20] ( 94.18,102.96) circle (  2.13);

\path[fill=fillColor,fill opacity=0.20] ( 88.72, 91.58) circle (  2.13);

\path[fill=fillColor,fill opacity=0.20] ( 86.75,101.33) circle (  2.13);

\path[fill=fillColor,fill opacity=0.20] (130.89, 89.95) circle (  2.13);

\path[fill=fillColor,fill opacity=0.20] (112.10, 98.89) circle (  2.13);

\path[fill=fillColor,fill opacity=0.20] (109.91, 94.02) circle (  2.13);

\path[fill=fillColor,fill opacity=0.20] (118.21, 94.83) circle (  2.13);

\path[fill=fillColor,fill opacity=0.20] (108.82,108.64) circle (  2.13);

\path[fill=fillColor,fill opacity=0.20] (108.38,101.33) circle (  2.13);

\path[fill=fillColor,fill opacity=0.20] (109.26, 87.51) circle (  2.13);

\path[fill=fillColor,fill opacity=0.20] ( 90.25, 94.02) circle (  2.13);

\path[fill=fillColor,fill opacity=0.20] ( 62.28,115.96) circle (  2.13);

\path[fill=fillColor,fill opacity=0.20] (128.48, 98.89) circle (  2.13);

\path[fill=fillColor,fill opacity=0.20] (106.20, 89.95) circle (  2.13);

\path[fill=fillColor,fill opacity=0.20] (101.83,110.27) circle (  2.13);

\path[fill=fillColor,fill opacity=0.20] (124.99,102.14) circle (  2.13);

\path[fill=fillColor,fill opacity=0.20] (122.80,102.14) circle (  2.13);

\path[fill=fillColor,fill opacity=0.20] (114.28,109.46) circle (  2.13);

\path[fill=fillColor,fill opacity=0.20] (119.96,103.77) circle (  2.13);

\path[fill=fillColor,fill opacity=0.20] (130.67, 91.58) circle (  2.13);

\path[fill=fillColor,fill opacity=0.20] (114.94, 89.95) circle (  2.13);

\path[fill=fillColor,fill opacity=0.20] ( 75.39, 99.70) circle (  2.13);

\path[fill=fillColor,fill opacity=0.20] (124.99, 85.89) circle (  2.13);

\path[fill=fillColor,fill opacity=0.20] (106.20, 95.64) circle (  2.13);

\path[fill=fillColor,fill opacity=0.20] (116.90,105.39) circle (  2.13);

\path[fill=fillColor,fill opacity=0.20] (146.40,102.14) circle (  2.13);

\path[fill=fillColor,fill opacity=0.20] (136.13,103.77) circle (  2.13);

\path[fill=fillColor,fill opacity=0.20] (113.63, 97.27) circle (  2.13);

\path[fill=fillColor,fill opacity=0.20] (122.15, 92.39) circle (  2.13);

\path[fill=fillColor,fill opacity=0.20] (139.41, 93.20) circle (  2.13);

\path[fill=fillColor,fill opacity=0.20] (125.43, 85.08) circle (  2.13);

\path[fill=fillColor,fill opacity=0.20] ( 85.00, 84.26) circle (  2.13);

\path[fill=fillColor,fill opacity=0.20] ( 45.89, 59.88) circle (  2.13);

\path[fill=fillColor,fill opacity=0.20] ( 45.45, 56.63) circle (  2.13);

\path[fill=fillColor,fill opacity=0.20] ( 52.23, 59.88) circle (  2.13);

\path[fill=fillColor,fill opacity=0.20] (112.53, 88.33) circle (  2.13);

\path[fill=fillColor,fill opacity=0.20] (120.40, 83.45) circle (  2.13);

\path[fill=fillColor,fill opacity=0.20] (123.90, 92.39) circle (  2.13);

\path[fill=fillColor,fill opacity=0.20] (125.21,105.39) circle (  2.13);

\path[fill=fillColor,fill opacity=0.20] (126.08,108.64) circle (  2.13);

\path[fill=fillColor,fill opacity=0.20] (125.43, 94.02) circle (  2.13);

\path[fill=fillColor,fill opacity=0.20] (132.20, 85.08) circle (  2.13);

\path[fill=fillColor,fill opacity=0.20] (135.04, 88.33) circle (  2.13);

\path[fill=fillColor,fill opacity=0.20] (119.31, 83.45) circle (  2.13);

\path[fill=fillColor,fill opacity=0.20] ( 57.25,107.02) circle (  2.13);

\path[fill=fillColor,fill opacity=0.20] ( 59.00, 81.82) circle (  2.13);

\path[fill=fillColor,fill opacity=0.20] ( 71.02, 50.13) circle (  2.13);

\path[fill=fillColor,fill opacity=0.20] ( 79.98, 48.50) circle (  2.13);

\path[fill=fillColor,fill opacity=0.20] ( 92.21, 51.75) circle (  2.13);

\path[fill=fillColor,fill opacity=0.20] ( 93.09, 96.45) circle (  2.13);

\path[fill=fillColor,fill opacity=0.20] (104.89, 91.58) circle (  2.13);

\path[fill=fillColor,fill opacity=0.20] (114.06, 91.58) circle (  2.13);

\path[fill=fillColor,fill opacity=0.20] (118.87,102.96) circle (  2.13);

\path[fill=fillColor,fill opacity=0.20] (126.95,110.27) circle (  2.13);

\path[fill=fillColor,fill opacity=0.20] (134.38,104.58) circle (  2.13);

\path[fill=fillColor,fill opacity=0.20] (138.32, 94.02) circle (  2.13);

\path[fill=fillColor,fill opacity=0.20] (144.22, 83.45) circle (  2.13);

\path[fill=fillColor,fill opacity=0.20] (122.37, 81.82) circle (  2.13);

\path[fill=fillColor,fill opacity=0.20] ( 66.87,115.96) circle (  2.13);

\path[fill=fillColor,fill opacity=0.20] ( 84.57, 86.70) circle (  2.13);

\path[fill=fillColor,fill opacity=0.20] (104.23, 62.32) circle (  2.13);

\path[fill=fillColor,fill opacity=0.20] (111.44, 51.75) circle (  2.13);

\path[fill=fillColor,fill opacity=0.20] ( 97.68, 42.00) circle (  2.13);

\path[fill=fillColor,fill opacity=0.20] ( 84.35, 38.75) circle (  2.13);

\path[fill=fillColor,fill opacity=0.20] ( 78.45, 39.56) circle (  2.13);

\path[fill=fillColor,fill opacity=0.20] ( 72.11, 47.69) circle (  2.13);

\path[fill=fillColor,fill opacity=0.20] ( 73.20, 43.63) circle (  2.13);

\path[fill=fillColor,fill opacity=0.20] ( 87.41, 99.70) circle (  2.13);

\path[fill=fillColor,fill opacity=0.20] ( 92.21, 91.58) circle (  2.13);

\path[fill=fillColor,fill opacity=0.20] ( 97.02,101.33) circle (  2.13);

\path[fill=fillColor,fill opacity=0.20] (106.42, 97.27) circle (  2.13);

\path[fill=fillColor,fill opacity=0.20] (112.32, 90.76) circle (  2.13);

\path[fill=fillColor,fill opacity=0.20] (118.87, 94.02) circle (  2.13);

\path[fill=fillColor,fill opacity=0.20] (126.08,101.33) circle (  2.13);

\path[fill=fillColor,fill opacity=0.20] (135.04, 97.27) circle (  2.13);

\path[fill=fillColor,fill opacity=0.20] (142.91, 80.20) circle (  2.13);

\path[fill=fillColor,fill opacity=0.20] (104.67, 72.07) circle (  2.13);

\path[fill=fillColor,fill opacity=0.20] (112.53, 63.13) circle (  2.13);

\path[fill=fillColor,fill opacity=0.20] (111.44, 79.39) circle (  2.13);

\path[fill=fillColor,fill opacity=0.20] (106.42, 80.20) circle (  2.13);

\path[fill=fillColor,fill opacity=0.20] ( 94.62, 73.70) circle (  2.13);

\path[fill=fillColor,fill opacity=0.20] ( 89.15, 69.63) circle (  2.13);

\path[fill=fillColor,fill opacity=0.20] ( 78.01, 62.32) circle (  2.13);

\path[fill=fillColor,fill opacity=0.20] ( 82.82, 56.63) circle (  2.13);

\path[fill=fillColor,fill opacity=0.20] ( 80.85, 55.01) circle (  2.13);

\path[fill=fillColor,fill opacity=0.20] (109.26, 78.57) circle (  2.13);

\path[fill=fillColor,fill opacity=0.20] (106.42, 77.76) circle (  2.13);

\path[fill=fillColor,fill opacity=0.20] (107.95, 91.58) circle (  2.13);

\path[fill=fillColor,fill opacity=0.20] (114.28,102.96) circle (  2.13);

\path[fill=fillColor,fill opacity=0.20] (111.22, 89.95) circle (  2.13);

\path[fill=fillColor,fill opacity=0.20] (119.31, 85.08) circle (  2.13);

\path[fill=fillColor,fill opacity=0.20] (135.69, 94.83) circle (  2.13);

\path[fill=fillColor,fill opacity=0.20] (135.04, 90.76) circle (  2.13);

\path[fill=fillColor,fill opacity=0.20] (119.96, 76.95) circle (  2.13);

\path[fill=fillColor,fill opacity=0.20] ( 88.94, 66.38) circle (  2.13);

\path[fill=fillColor,fill opacity=0.20] ( 85.00, 98.08) circle (  2.13);

\path[fill=fillColor,fill opacity=0.20] (109.47, 78.57) circle (  2.13);

\path[fill=fillColor,fill opacity=0.20] (107.29, 96.45) circle (  2.13);

\path[fill=fillColor,fill opacity=0.20] (105.10,109.46) circle (  2.13);

\path[fill=fillColor,fill opacity=0.20] (104.23,100.52) circle (  2.13);

\path[fill=fillColor,fill opacity=0.20] (101.17, 96.45) circle (  2.13);

\path[fill=fillColor,fill opacity=0.20] ( 87.19, 91.58) circle (  2.13);

\path[fill=fillColor,fill opacity=0.20] ( 86.10, 72.89) circle (  2.13);

\path[fill=fillColor,fill opacity=0.20] ( 92.87, 60.69) circle (  2.13);

\path[fill=fillColor,fill opacity=0.20] ( 89.81, 57.44) circle (  2.13);

\path[fill=fillColor,fill opacity=0.20] ( 73.42, 95.64) circle (  2.13);

\path[fill=fillColor,fill opacity=0.20] ( 88.28, 76.14) circle (  2.13);

\path[fill=fillColor,fill opacity=0.20] (105.32, 85.89) circle (  2.13);

\path[fill=fillColor,fill opacity=0.20] (114.28, 81.82) circle (  2.13);

\path[fill=fillColor,fill opacity=0.20] (113.84, 92.39) circle (  2.13);

\path[fill=fillColor,fill opacity=0.20] (126.08,109.46) circle (  2.13);

\path[fill=fillColor,fill opacity=0.20] (119.31,107.83) circle (  2.13);

\path[fill=fillColor,fill opacity=0.20] (119.96,104.58) circle (  2.13);

\path[fill=fillColor,fill opacity=0.20] (137.01, 98.89) circle (  2.13);

\path[fill=fillColor,fill opacity=0.20] (131.32, 88.33) circle (  2.13);

\path[fill=fillColor,fill opacity=0.20] ( 94.40, 75.32) circle (  2.13);

\path[fill=fillColor,fill opacity=0.20] ( 95.49, 82.64) circle (  2.13);

\path[fill=fillColor,fill opacity=0.20] (117.34, 87.51) circle (  2.13);

\path[fill=fillColor,fill opacity=0.20] (138.97,112.71) circle (  2.13);

\path[fill=fillColor,fill opacity=0.20] (104.45,109.46) circle (  2.13);

\path[fill=fillColor,fill opacity=0.20] (110.35,101.33) circle (  2.13);

\path[fill=fillColor,fill opacity=0.20] ( 98.77,107.83) circle (  2.13);

\path[fill=fillColor,fill opacity=0.20] ( 91.34,100.52) circle (  2.13);

\path[fill=fillColor,fill opacity=0.20] ( 93.52, 81.82) circle (  2.13);

\path[fill=fillColor,fill opacity=0.20] ( 87.62, 70.45) circle (  2.13);

\path[fill=fillColor,fill opacity=0.20] ( 69.27,106.21) circle (  2.13);

\path[fill=fillColor,fill opacity=0.20] ( 89.15, 72.07) circle (  2.13);

\path[fill=fillColor,fill opacity=0.20] ( 95.05, 89.95) circle (  2.13);

\path[fill=fillColor,fill opacity=0.20] (127.83, 94.02) circle (  2.13);

\path[fill=fillColor,fill opacity=0.20] (118.21, 91.58) circle (  2.13);

\path[fill=fillColor,fill opacity=0.20] (118.43, 98.08) circle (  2.13);

\path[fill=fillColor,fill opacity=0.20] (137.44,101.33) circle (  2.13);

\path[fill=fillColor,fill opacity=0.20] (138.54, 94.83) circle (  2.13);

\path[fill=fillColor,fill opacity=0.20] (119.31, 81.82) circle (  2.13);

\path[fill=fillColor,fill opacity=0.20] ( 72.77,107.83) circle (  2.13);

\path[fill=fillColor,fill opacity=0.20] ( 98.55, 84.26) circle (  2.13);

\path[fill=fillColor,fill opacity=0.20] (121.49, 93.20) circle (  2.13);

\path[fill=fillColor,fill opacity=0.20] (124.77,101.33) circle (  2.13);

\path[fill=fillColor,fill opacity=0.20] (105.32, 92.39) circle (  2.13);

\path[fill=fillColor,fill opacity=0.20] (102.92, 92.39) circle (  2.13);

\path[fill=fillColor,fill opacity=0.20] ( 96.58,105.39) circle (  2.13);

\path[fill=fillColor,fill opacity=0.20] ( 83.25, 98.08) circle (  2.13);

\path[fill=fillColor,fill opacity=0.20] ( 80.41, 84.26) circle (  2.13);

\path[fill=fillColor,fill opacity=0.20] ( 76.70, 76.95) circle (  2.13);

\path[fill=fillColor,fill opacity=0.20] ( 74.73,106.21) circle (  2.13);

\path[fill=fillColor,fill opacity=0.20] ( 81.07, 75.32) circle (  2.13);

\path[fill=fillColor,fill opacity=0.20] ( 89.37, 87.51) circle (  2.13);

\path[fill=fillColor,fill opacity=0.20] (104.67, 93.20) circle (  2.13);

\path[fill=fillColor,fill opacity=0.20] (111.00, 91.58) circle (  2.13);

\path[fill=fillColor,fill opacity=0.20] (108.38, 94.02) circle (  2.13);

\path[fill=fillColor,fill opacity=0.20] (123.24, 93.20) circle (  2.13);

\path[fill=fillColor,fill opacity=0.20] (135.91, 83.45) circle (  2.13);

\path[fill=fillColor,fill opacity=0.20] (147.28, 96.45) circle (  2.13);

\path[fill=fillColor,fill opacity=0.20] (138.97, 82.64) circle (  2.13);

\path[fill=fillColor,fill opacity=0.20] ( 93.74, 63.95) circle (  2.13);

\path[fill=fillColor,fill opacity=0.20] ( 82.60, 89.95) circle (  2.13);

\path[fill=fillColor,fill opacity=0.20] (122.80, 95.64) circle (  2.13);

\path[fill=fillColor,fill opacity=0.20] (131.76, 98.89) circle (  2.13);

\path[fill=fillColor,fill opacity=0.20] (107.73, 95.64) circle (  2.13);

\path[fill=fillColor,fill opacity=0.20] (103.14, 98.08) circle (  2.13);

\path[fill=fillColor,fill opacity=0.20] ( 93.96,101.33) circle (  2.13);

\path[fill=fillColor,fill opacity=0.20] ( 80.63, 94.83) circle (  2.13);

\path[fill=fillColor,fill opacity=0.20] ( 75.17, 83.45) circle (  2.13);

\path[fill=fillColor,fill opacity=0.20] ( 72.99, 70.45) circle (  2.13);

\path[fill=fillColor,fill opacity=0.20] ( 62.93,105.39) circle (  2.13);

\path[fill=fillColor,fill opacity=0.20] ( 88.50, 77.76) circle (  2.13);

\path[fill=fillColor,fill opacity=0.20] ( 89.15, 94.83) circle (  2.13);

\path[fill=fillColor,fill opacity=0.20] ( 95.05, 98.89) circle (  2.13);

\path[fill=fillColor,fill opacity=0.20] (108.60, 92.39) circle (  2.13);

\path[fill=fillColor,fill opacity=0.20] (100.08, 95.64) circle (  2.13);

\path[fill=fillColor,fill opacity=0.20] (103.58, 98.89) circle (  2.13);

\path[fill=fillColor,fill opacity=0.20] (121.27, 94.02) circle (  2.13);

\path[fill=fillColor,fill opacity=0.20] (123.90, 83.45) circle (  2.13);

\path[fill=fillColor,fill opacity=0.20] (125.64, 90.76) circle (  2.13);

\path[fill=fillColor,fill opacity=0.20] ( 99.42, 80.20) circle (  2.13);

\path[fill=fillColor,fill opacity=0.20] ( 74.30, 89.95) circle (  2.13);

\path[fill=fillColor,fill opacity=0.20] (107.73, 86.70) circle (  2.13);

\path[fill=fillColor,fill opacity=0.20] (135.91,100.52) circle (  2.13);

\path[fill=fillColor,fill opacity=0.20] (123.02,112.71) circle (  2.13);

\path[fill=fillColor,fill opacity=0.20] (102.92,111.08) circle (  2.13);

\path[fill=fillColor,fill opacity=0.20] (100.30,103.77) circle (  2.13);

\path[fill=fillColor,fill opacity=0.20] ( 88.50, 98.89) circle (  2.13);

\path[fill=fillColor,fill opacity=0.20] ( 89.37, 86.70) circle (  2.13);

\path[fill=fillColor,fill opacity=0.20] ( 80.41, 68.82) circle (  2.13);

\path[fill=fillColor,fill opacity=0.20] ( 61.40, 78.57) circle (  2.13);

\path[fill=fillColor,fill opacity=0.20] ( 63.15, 94.83) circle (  2.13);

\path[fill=fillColor,fill opacity=0.20] ( 78.88, 78.57) circle (  2.13);

\path[fill=fillColor,fill opacity=0.20] ( 88.72, 89.14) circle (  2.13);

\path[fill=fillColor,fill opacity=0.20] ( 99.64, 91.58) circle (  2.13);

\path[fill=fillColor,fill opacity=0.20] (112.75, 97.27) circle (  2.13);

\path[fill=fillColor,fill opacity=0.20] (118.65, 98.08) circle (  2.13);

\path[fill=fillColor,fill opacity=0.20] ( 99.42, 98.08) circle (  2.13);

\path[fill=fillColor,fill opacity=0.20] (102.92, 98.89) circle (  2.13);

\path[fill=fillColor,fill opacity=0.20] (120.84, 96.45) circle (  2.13);

\path[fill=fillColor,fill opacity=0.20] (114.06, 93.20) circle (  2.13);

\path[fill=fillColor,fill opacity=0.20] (107.73, 91.58) circle (  2.13);

\path[fill=fillColor,fill opacity=0.20] ( 97.24, 82.64) circle (  2.13);

\path[fill=fillColor,fill opacity=0.20] ( 81.73, 82.64) circle (  2.13);

\path[fill=fillColor,fill opacity=0.20] (108.82, 94.02) circle (  2.13);

\path[fill=fillColor,fill opacity=0.20] (116.90,106.21) circle (  2.13);

\path[fill=fillColor,fill opacity=0.20] (114.50,105.39) circle (  2.13);

\path[fill=fillColor,fill opacity=0.20] (109.26,106.21) circle (  2.13);

\path[fill=fillColor,fill opacity=0.20] (108.82,104.58) circle (  2.13);

\path[fill=fillColor,fill opacity=0.20] (102.92, 90.76) circle (  2.13);

\path[fill=fillColor,fill opacity=0.20] ( 88.50, 81.82) circle (  2.13);

\path[fill=fillColor,fill opacity=0.20] ( 73.42, 83.45) circle (  2.13);

\path[fill=fillColor,fill opacity=0.20] ( 63.37, 88.33) circle (  2.13);

\path[fill=fillColor,fill opacity=0.20] ( 82.38, 72.07) circle (  2.13);

\path[fill=fillColor,fill opacity=0.20] ( 94.40, 89.95) circle (  2.13);

\path[fill=fillColor,fill opacity=0.20] ( 95.27, 88.33) circle (  2.13);

\path[fill=fillColor,fill opacity=0.20] (102.70, 81.01) circle (  2.13);

\path[fill=fillColor,fill opacity=0.20] (116.47, 89.14) circle (  2.13);

\path[fill=fillColor,fill opacity=0.20] (116.90, 89.14) circle (  2.13);

\path[fill=fillColor,fill opacity=0.20] (109.04, 82.64) circle (  2.13);

\path[fill=fillColor,fill opacity=0.20] (112.97, 81.82) circle (  2.13);

\path[fill=fillColor,fill opacity=0.20] (105.54, 85.08) circle (  2.13);

\path[fill=fillColor,fill opacity=0.20] ( 94.62, 85.89) circle (  2.13);

\path[fill=fillColor,fill opacity=0.20] ( 81.29, 81.01) circle (  2.13);

\path[fill=fillColor,fill opacity=0.20] ( 66.21, 84.26) circle (  2.13);

\path[fill=fillColor,fill opacity=0.20] ( 91.34, 94.83) circle (  2.13);

\path[fill=fillColor,fill opacity=0.20] (113.41, 94.83) circle (  2.13);

\path[fill=fillColor,fill opacity=0.20] (111.22, 90.76) circle (  2.13);

\path[fill=fillColor,fill opacity=0.20] (111.66,103.77) circle (  2.13);

\path[fill=fillColor,fill opacity=0.20] (108.82,102.96) circle (  2.13);

\path[fill=fillColor,fill opacity=0.20] (104.01, 86.70) circle (  2.13);

\path[fill=fillColor,fill opacity=0.20] ( 94.84, 85.08) circle (  2.13);

\path[fill=fillColor,fill opacity=0.20] ( 80.85, 86.70) circle (  2.13);

\path[fill=fillColor,fill opacity=0.20] ( 69.05, 82.64) circle (  2.13);

\path[fill=fillColor,fill opacity=0.20] ( 50.48,102.14) circle (  2.13);

\path[fill=fillColor,fill opacity=0.20] ( 52.01,100.52) circle (  2.13);

\path[fill=fillColor,fill opacity=0.20] ( 70.80, 81.01) circle (  2.13);

\path[fill=fillColor,fill opacity=0.20] ( 78.23, 75.32) circle (  2.13);

\path[fill=fillColor,fill opacity=0.20] ( 87.62, 86.70) circle (  2.13);

\path[fill=fillColor,fill opacity=0.20] ( 99.42, 90.76) circle (  2.13);

\path[fill=fillColor,fill opacity=0.20] (108.38, 85.89) circle (  2.13);

\path[fill=fillColor,fill opacity=0.20] (114.28, 75.32) circle (  2.13);

\path[fill=fillColor,fill opacity=0.20] (110.13, 68.01) circle (  2.13);

\path[fill=fillColor,fill opacity=0.20] (104.45, 74.51) circle (  2.13);

\path[fill=fillColor,fill opacity=0.20] ( 77.57, 77.76) circle (  2.13);

\path[fill=fillColor,fill opacity=0.20] ( 77.14, 82.64) circle (  2.13);

\path[fill=fillColor,fill opacity=0.20] ( 83.47, 85.89) circle (  2.13);

\path[fill=fillColor,fill opacity=0.20] ( 80.85, 92.39) circle (  2.13);

\path[fill=fillColor,fill opacity=0.20] (124.77, 74.51) circle (  2.13);

\path[fill=fillColor,fill opacity=0.20] (112.75, 74.51) circle (  2.13);

\path[fill=fillColor,fill opacity=0.20] (112.10, 98.89) circle (  2.13);

\path[fill=fillColor,fill opacity=0.20] (107.51,103.77) circle (  2.13);

\path[fill=fillColor,fill opacity=0.20] ( 97.02, 88.33) circle (  2.13);

\path[fill=fillColor,fill opacity=0.20] ( 96.36, 86.70) circle (  2.13);

\path[fill=fillColor,fill opacity=0.20] ( 88.72, 91.58) circle (  2.13);

\path[fill=fillColor,fill opacity=0.20] ( 83.47, 82.64) circle (  2.13);

\path[fill=fillColor,fill opacity=0.20] ( 79.98, 76.95) circle (  2.13);

\path[fill=fillColor,fill opacity=0.20] ( 57.91, 95.64) circle (  2.13);

\path[fill=fillColor,fill opacity=0.20] ( 77.36, 85.08) circle (  2.13);

\path[fill=fillColor,fill opacity=0.20] ( 74.73, 71.26) circle (  2.13);

\path[fill=fillColor,fill opacity=0.20] ( 93.96, 80.20) circle (  2.13);

\path[fill=fillColor,fill opacity=0.20] ( 95.93, 91.58) circle (  2.13);

\path[fill=fillColor,fill opacity=0.20] (103.14, 81.01) circle (  2.13);

\path[fill=fillColor,fill opacity=0.20] (110.79, 70.45) circle (  2.13);

\path[fill=fillColor,fill opacity=0.20] (117.78, 64.76) circle (  2.13);

\path[fill=fillColor,fill opacity=0.20] (102.05, 55.82) circle (  2.13);

\path[fill=fillColor,fill opacity=0.20] (102.26, 63.95) circle (  2.13);

\path[fill=fillColor,fill opacity=0.20] (134.82, 90.76) circle (  2.13);

\path[fill=fillColor,fill opacity=0.20] (103.79, 94.02) circle (  2.13);

\path[fill=fillColor,fill opacity=0.20] ( 96.58, 93.20) circle (  2.13);

\path[fill=fillColor,fill opacity=0.20] ( 92.65, 89.14) circle (  2.13);

\path[fill=fillColor,fill opacity=0.20] ( 95.05, 84.26) circle (  2.13);

\path[fill=fillColor,fill opacity=0.20] ( 84.35, 76.14) circle (  2.13);

\path[fill=fillColor,fill opacity=0.20] ( 75.39, 69.63) circle (  2.13);

\path[fill=fillColor,fill opacity=0.20] ( 63.37, 89.95) circle (  2.13);

\path[fill=fillColor,fill opacity=0.20] ( 53.98, 77.76) circle (  2.13);

\path[fill=fillColor,fill opacity=0.20] ( 64.90, 76.95) circle (  2.13);

\path[fill=fillColor,fill opacity=0.20] ( 78.88, 66.38) circle (  2.13);

\path[fill=fillColor,fill opacity=0.20] ( 95.05, 66.38) circle (  2.13);

\path[fill=fillColor,fill opacity=0.20] ( 95.93, 81.82) circle (  2.13);

\path[fill=fillColor,fill opacity=0.20] (106.42, 85.89) circle (  2.13);

\path[fill=fillColor,fill opacity=0.20] (115.37, 74.51) circle (  2.13);

\path[fill=fillColor,fill opacity=0.20] (127.83, 66.38) circle (  2.13);

\path[fill=fillColor,fill opacity=0.20] ( 83.25, 59.07) circle (  2.13);

\path[fill=fillColor,fill opacity=0.20] ( 87.84, 79.39) circle (  2.13);

\path[fill=fillColor,fill opacity=0.20] (102.05, 81.01) circle (  2.13);

\path[fill=fillColor,fill opacity=0.20] (107.51, 84.26) circle (  2.13);

\path[fill=fillColor,fill opacity=0.20] (100.73, 91.58) circle (  2.13);

\path[fill=fillColor,fill opacity=0.20] ( 97.46, 92.39) circle (  2.13);

\path[fill=fillColor,fill opacity=0.20] ( 88.28, 92.39) circle (  2.13);

\path[fill=fillColor,fill opacity=0.20] ( 87.41, 91.58) circle (  2.13);

\path[fill=fillColor,fill opacity=0.20] ( 82.82, 76.95) circle (  2.13);

\path[fill=fillColor,fill opacity=0.20] ( 79.98, 70.45) circle (  2.13);

\path[fill=fillColor,fill opacity=0.20] ( 64.68, 85.08) circle (  2.13);

\path[fill=fillColor,fill opacity=0.20] ( 64.90, 55.01) circle (  2.13);

\path[fill=fillColor,fill opacity=0.20] ( 65.12, 57.44) circle (  2.13);

\path[fill=fillColor,fill opacity=0.20] ( 67.52, 61.51) circle (  2.13);

\path[fill=fillColor,fill opacity=0.20] ( 80.20, 56.63) circle (  2.13);

\path[fill=fillColor,fill opacity=0.20] ( 89.81, 57.44) circle (  2.13);

\path[fill=fillColor,fill opacity=0.20] ( 95.27, 63.13) circle (  2.13);

\path[fill=fillColor,fill opacity=0.20] (106.63, 71.26) circle (  2.13);

\path[fill=fillColor,fill opacity=0.20] ( 97.24, 70.45) circle (  2.13);

\path[fill=fillColor,fill opacity=0.20] ( 91.34, 68.82) circle (  2.13);

\path[fill=fillColor,fill opacity=0.20] ( 96.36, 73.70) circle (  2.13);

\path[fill=fillColor,fill opacity=0.20] (101.17, 82.64) circle (  2.13);

\path[fill=fillColor,fill opacity=0.20] ( 93.96, 89.95) circle (  2.13);

\path[fill=fillColor,fill opacity=0.20] ( 89.15, 94.02) circle (  2.13);

\path[fill=fillColor,fill opacity=0.20] ( 87.62, 87.51) circle (  2.13);

\path[fill=fillColor,fill opacity=0.20] (124.11, 69.63) circle (  2.13);

\path[fill=fillColor,fill opacity=0.20] ( 83.91, 65.57) circle (  2.13);

\path[fill=fillColor,fill opacity=0.20] ( 78.67, 76.95) circle (  2.13);

\path[fill=fillColor,fill opacity=0.20] ( 63.15, 87.51) circle (  2.13);

\path[fill=fillColor,fill opacity=0.20] ( 55.72, 89.95) circle (  2.13);

\path[fill=fillColor,fill opacity=0.20] ( 46.98, 97.27) circle (  2.13);

\path[fill=fillColor,fill opacity=0.20] ( 48.95, 90.76) circle (  2.13);

\path[fill=fillColor,fill opacity=0.20] ( 64.46, 59.88) circle (  2.13);

\path[fill=fillColor,fill opacity=0.20] ( 78.01, 49.32) circle (  2.13);

\path[fill=fillColor,fill opacity=0.20] ( 78.67, 63.95) circle (  2.13);

\path[fill=fillColor,fill opacity=0.20] ( 85.44, 63.13) circle (  2.13);

\path[fill=fillColor,fill opacity=0.20] ( 95.93, 61.51) circle (  2.13);

\path[fill=fillColor,fill opacity=0.20] (103.58, 72.89) circle (  2.13);

\path[fill=fillColor,fill opacity=0.20] (115.37, 74.51) circle (  2.13);

\path[fill=fillColor,fill opacity=0.20] ( 81.94, 62.32) circle (  2.13);

\path[fill=fillColor,fill opacity=0.20] ( 79.10, 68.82) circle (  2.13);

\path[fill=fillColor,fill opacity=0.20] ( 91.78, 62.32) circle (  2.13);

\path[fill=fillColor,fill opacity=0.20] ( 93.96, 66.38) circle (  2.13);

\path[fill=fillColor,fill opacity=0.20] ( 92.65, 82.64) circle (  2.13);

\path[fill=fillColor,fill opacity=0.20] ( 92.43, 83.45) circle (  2.13);

\path[fill=fillColor,fill opacity=0.20] ( 88.72, 69.63) circle (  2.13);

\path[fill=fillColor,fill opacity=0.20] ( 83.47, 70.45) circle (  2.13);

\path[fill=fillColor,fill opacity=0.20] ( 86.31, 77.76) circle (  2.13);

\path[fill=fillColor,fill opacity=0.20] ( 81.51, 77.76) circle (  2.13);

\path[fill=fillColor,fill opacity=0.20] ( 71.89, 76.95) circle (  2.13);

\path[fill=fillColor,fill opacity=0.20] ( 70.58, 72.89) circle (  2.13);

\path[fill=fillColor,fill opacity=0.20] ( 63.37, 80.20) circle (  2.13);

\path[fill=fillColor,fill opacity=0.20] ( 55.94,102.14) circle (  2.13);

\path[fill=fillColor,fill opacity=0.20] ( 49.61,113.52) circle (  2.13);

\path[fill=fillColor,fill opacity=0.20] ( 57.03, 89.95) circle (  2.13);

\path[fill=fillColor,fill opacity=0.20] ( 65.77, 79.39) circle (  2.13);

\path[fill=fillColor,fill opacity=0.20] ( 68.40, 82.64) circle (  2.13);

\path[fill=fillColor,fill opacity=0.20] ( 74.51, 68.82) circle (  2.13);

\path[fill=fillColor,fill opacity=0.20] ( 84.78, 51.75) circle (  2.13);

\path[fill=fillColor,fill opacity=0.20] ( 97.89, 61.51) circle (  2.13);

\path[fill=fillColor,fill opacity=0.20] ( 99.86, 69.63) circle (  2.13);

\path[fill=fillColor,fill opacity=0.20] (108.60, 68.01) circle (  2.13);

\path[fill=fillColor,fill opacity=0.20] (118.21, 72.07) circle (  2.13);

\path[fill=fillColor,fill opacity=0.20] (102.92, 67.20) circle (  2.13);

\path[fill=fillColor,fill opacity=0.20] ( 97.46, 63.95) circle (  2.13);

\path[fill=fillColor,fill opacity=0.20] ( 98.33, 74.51) circle (  2.13);

\path[fill=fillColor,fill opacity=0.20] ( 90.68, 80.20) circle (  2.13);

\path[fill=fillColor,fill opacity=0.20] ( 87.62, 80.20) circle (  2.13);

\path[fill=fillColor,fill opacity=0.20] ( 87.62, 83.45) circle (  2.13);

\path[fill=fillColor,fill opacity=0.20] ( 78.23, 81.82) circle (  2.13);

\path[fill=fillColor,fill opacity=0.20] ( 75.83, 76.14) circle (  2.13);

\path[fill=fillColor,fill opacity=0.20] ( 78.45, 75.32) circle (  2.13);

\path[fill=fillColor,fill opacity=0.20] ( 79.10, 72.89) circle (  2.13);

\path[fill=fillColor,fill opacity=0.20] ( 70.58, 68.82) circle (  2.13);

\path[fill=fillColor,fill opacity=0.20] ( 73.86, 73.70) circle (  2.13);

\path[fill=fillColor,fill opacity=0.20] ( 70.80, 86.70) circle (  2.13);

\path[fill=fillColor,fill opacity=0.20] ( 63.37,100.52) circle (  2.13);

\path[fill=fillColor,fill opacity=0.20] ( 61.84,110.27) circle (  2.13);

\path[fill=fillColor,fill opacity=0.20] ( 64.25,112.71) circle (  2.13);

\path[fill=fillColor,fill opacity=0.20] ( 97.02, 99.70) circle (  2.13);

\path[fill=fillColor,fill opacity=0.20] ( 52.01,102.14) circle (  2.13);

\path[fill=fillColor,fill opacity=0.20] ( 47.64,114.33) circle (  2.13);

\path[fill=fillColor,fill opacity=0.20] ( 55.51,109.46) circle (  2.13);

\path[fill=fillColor,fill opacity=0.20] ( 61.62, 89.95) circle (  2.13);

\path[fill=fillColor,fill opacity=0.20] ( 62.72, 78.57) circle (  2.13);

\path[fill=fillColor,fill opacity=0.20] ( 65.77, 84.26) circle (  2.13);

\path[fill=fillColor,fill opacity=0.20] ( 69.27, 86.70) circle (  2.13);

\path[fill=fillColor,fill opacity=0.20] ( 75.17, 65.57) circle (  2.13);

\path[fill=fillColor,fill opacity=0.20] ( 76.70, 41.19) circle (  2.13);

\path[fill=fillColor,fill opacity=0.20] ( 74.95, 45.25) circle (  2.13);

\path[fill=fillColor,fill opacity=0.20] ( 82.38, 54.19) circle (  2.13);

\path[fill=fillColor,fill opacity=0.20] ( 84.78, 55.01) circle (  2.13);

\path[fill=fillColor,fill opacity=0.20] ( 83.47, 68.01) circle (  2.13);

\path[fill=fillColor,fill opacity=0.20] ( 90.68, 77.76) circle (  2.13);

\path[fill=fillColor,fill opacity=0.20] (104.23, 68.01) circle (  2.13);

\path[fill=fillColor,fill opacity=0.20] (108.60, 57.44) circle (  2.13);

\path[fill=fillColor,fill opacity=0.20] ( 89.59, 58.26) circle (  2.13);

\path[fill=fillColor,fill opacity=0.20] ( 90.68, 69.63) circle (  2.13);

\path[fill=fillColor,fill opacity=0.20] ( 94.84, 76.14) circle (  2.13);

\path[fill=fillColor,fill opacity=0.20] ( 98.55, 72.89) circle (  2.13);

\path[fill=fillColor,fill opacity=0.20] ( 93.52, 80.20) circle (  2.13);

\path[fill=fillColor,fill opacity=0.20] ( 78.67, 85.89) circle (  2.13);

\path[fill=fillColor,fill opacity=0.20] ( 81.07, 79.39) circle (  2.13);

\path[fill=fillColor,fill opacity=0.20] ( 77.14, 81.82) circle (  2.13);

\path[fill=fillColor,fill opacity=0.20] ( 76.26, 79.39) circle (  2.13);

\path[fill=fillColor,fill opacity=0.20] ( 71.02, 69.63) circle (  2.13);

\path[fill=fillColor,fill opacity=0.20] ( 81.51, 72.07) circle (  2.13);

\path[fill=fillColor,fill opacity=0.20] ( 90.25, 74.51) circle (  2.13);

\path[fill=fillColor,fill opacity=0.20] ( 85.88, 71.26) circle (  2.13);

\path[fill=fillColor,fill opacity=0.20] ( 83.69, 70.45) circle (  2.13);

\path[fill=fillColor,fill opacity=0.20] ( 85.88, 73.70) circle (  2.13);

\path[fill=fillColor,fill opacity=0.20] ( 77.57, 75.32) circle (  2.13);

\path[fill=fillColor,fill opacity=0.20] ( 76.70, 74.51) circle (  2.13);

\path[fill=fillColor,fill opacity=0.20] ( 70.14, 67.20) circle (  2.13);

\path[fill=fillColor,fill opacity=0.20] ( 70.80, 64.76) circle (  2.13);

\path[fill=fillColor,fill opacity=0.20] ( 77.57, 68.82) circle (  2.13);

\path[fill=fillColor,fill opacity=0.20] ( 74.95, 70.45) circle (  2.13);

\path[fill=fillColor,fill opacity=0.20] ( 75.83, 68.01) circle (  2.13);

\path[fill=fillColor,fill opacity=0.20] ( 76.26, 68.01) circle (  2.13);

\path[fill=fillColor,fill opacity=0.20] ( 81.29, 67.20) circle (  2.13);

\path[fill=fillColor,fill opacity=0.20] ( 75.83, 62.32) circle (  2.13);

\path[fill=fillColor,fill opacity=0.20] ( 72.77, 57.44) circle (  2.13);

\path[fill=fillColor,fill opacity=0.20] ( 79.10, 62.32) circle (  2.13);

\path[fill=fillColor,fill opacity=0.20] ( 80.85, 68.82) circle (  2.13);

\path[fill=fillColor,fill opacity=0.20] ( 75.83, 66.38) circle (  2.13);

\path[fill=fillColor,fill opacity=0.20] ( 75.61, 65.57) circle (  2.13);

\path[fill=fillColor,fill opacity=0.20] ( 77.36, 66.38) circle (  2.13);

\path[fill=fillColor,fill opacity=0.20] ( 88.06, 65.57) circle (  2.13);

\path[fill=fillColor,fill opacity=0.20] ( 91.99, 75.32) circle (  2.13);

\path[fill=fillColor,fill opacity=0.20] ( 92.21, 83.45) circle (  2.13);

\path[fill=fillColor,fill opacity=0.20] ( 92.21, 66.38) circle (  2.13);

\path[fill=fillColor,fill opacity=0.20] ( 94.62, 49.32) circle (  2.13);

\path[fill=fillColor,fill opacity=0.20] ( 85.66, 59.88) circle (  2.13);

\path[fill=fillColor,fill opacity=0.20] ( 93.09, 58.26) circle (  2.13);

\path[fill=fillColor,fill opacity=0.20] (102.48, 68.82) circle (  2.13);

\path[fill=fillColor,fill opacity=0.20] ( 98.77, 76.14) circle (  2.13);

\path[fill=fillColor,fill opacity=0.20] ( 86.75, 82.64) circle (  2.13);

\path[fill=fillColor,fill opacity=0.20] ( 87.62, 86.70) circle (  2.13);

\path[fill=fillColor,fill opacity=0.20] ( 88.28, 80.20) circle (  2.13);

\path[fill=fillColor,fill opacity=0.20] ( 82.16, 76.95) circle (  2.13);

\path[fill=fillColor,fill opacity=0.20] ( 77.14, 85.08) circle (  2.13);

\path[fill=fillColor,fill opacity=0.20] ( 77.57, 84.26) circle (  2.13);

\path[fill=fillColor,fill opacity=0.20] ( 74.30, 80.20) circle (  2.13);

\path[fill=fillColor,fill opacity=0.20] ( 84.13, 76.95) circle (  2.13);

\path[fill=fillColor,fill opacity=0.20] ( 88.94, 72.07) circle (  2.13);

\path[fill=fillColor,fill opacity=0.20] ( 78.88, 79.39) circle (  2.13);

\path[fill=fillColor,fill opacity=0.20] ( 75.17, 88.33) circle (  2.13);

\path[fill=fillColor,fill opacity=0.20] ( 77.36, 79.39) circle (  2.13);

\path[fill=fillColor,fill opacity=0.20] ( 79.54, 68.82) circle (  2.13);

\path[fill=fillColor,fill opacity=0.20] ( 79.10, 69.63) circle (  2.13);

\path[fill=fillColor,fill opacity=0.20] ( 79.54, 67.20) circle (  2.13);

\path[fill=fillColor,fill opacity=0.20] ( 83.25, 60.69) circle (  2.13);

\path[fill=fillColor,fill opacity=0.20] ( 83.69, 64.76) circle (  2.13);

\path[fill=fillColor,fill opacity=0.20] ( 82.60, 69.63) circle (  2.13);

\path[fill=fillColor,fill opacity=0.20] ( 87.19, 70.45) circle (  2.13);

\path[fill=fillColor,fill opacity=0.20] ( 76.04, 76.14) circle (  2.13);

\path[fill=fillColor,fill opacity=0.20] ( 74.51, 85.89) circle (  2.13);

\path[fill=fillColor,fill opacity=0.20] ( 76.92, 88.33) circle (  2.13);

\path[fill=fillColor,fill opacity=0.20] ( 80.41, 88.33) circle (  2.13);

\path[fill=fillColor,fill opacity=0.20] ( 78.67, 88.33) circle (  2.13);

\path[fill=fillColor,fill opacity=0.20] ( 87.19, 79.39) circle (  2.13);

\path[fill=fillColor,fill opacity=0.20] ( 94.18, 71.26) circle (  2.13);

\path[fill=fillColor,fill opacity=0.20] ( 91.99, 76.95) circle (  2.13);

\path[fill=fillColor,fill opacity=0.20] ( 81.51, 71.26) circle (  2.13);

\path[fill=fillColor,fill opacity=0.20] ( 92.65, 71.26) circle (  2.13);

\path[fill=fillColor,fill opacity=0.20] ( 91.56, 74.51) circle (  2.13);

\path[fill=fillColor,fill opacity=0.20] ( 91.99, 71.26) circle (  2.13);

\path[fill=fillColor,fill opacity=0.20] ( 91.56, 72.07) circle (  2.13);

\path[fill=fillColor,fill opacity=0.20] ( 90.90, 76.14) circle (  2.13);

\path[fill=fillColor,fill opacity=0.20] ( 83.25, 81.01) circle (  2.13);

\path[fill=fillColor,fill opacity=0.20] ( 82.60, 81.01) circle (  2.13);

\path[fill=fillColor,fill opacity=0.20] ( 88.94, 76.14) circle (  2.13);

\path[fill=fillColor,fill opacity=0.20] ( 88.28, 70.45) circle (  2.13);

\path[fill=fillColor,fill opacity=0.20] ( 84.57, 73.70) circle (  2.13);

\path[fill=fillColor,fill opacity=0.20] ( 84.35, 83.45) circle (  2.13);

\path[fill=fillColor,fill opacity=0.20] ( 90.03, 81.82) circle (  2.13);

\path[fill=fillColor,fill opacity=0.20] ( 90.03, 76.95) circle (  2.13);

\path[fill=fillColor,fill opacity=0.20] ( 88.72, 74.51) circle (  2.13);

\path[fill=fillColor,fill opacity=0.20] ( 81.73, 69.63) circle (  2.13);

\path[fill=fillColor,fill opacity=0.20] ( 90.47, 65.57) circle (  2.13);

\path[fill=fillColor,fill opacity=0.20] (103.58, 73.70) circle (  2.13);

\path[fill=fillColor,fill opacity=0.20] ( 81.73, 81.82) circle (  2.13);

\path[fill=fillColor,fill opacity=0.20] ( 80.20, 78.57) circle (  2.13);

\path[fill=fillColor,fill opacity=0.20] ( 88.72, 76.95) circle (  2.13);

\path[fill=fillColor,fill opacity=0.20] ( 87.84, 81.01) circle (  2.13);

\path[fill=fillColor,fill opacity=0.20] ( 85.00, 81.82) circle (  2.13);

\path[fill=fillColor,fill opacity=0.20] ( 91.34, 79.39) circle (  2.13);

\path[fill=fillColor,fill opacity=0.20] ( 91.34, 73.70) circle (  2.13);

\path[fill=fillColor,fill opacity=0.20] ( 83.69, 65.57) circle (  2.13);

\path[fill=fillColor,fill opacity=0.20] ( 85.44, 59.07) circle (  2.13);

\path[fill=fillColor,fill opacity=0.20] ( 74.30, 62.32) circle (  2.13);

\path[fill=fillColor,fill opacity=0.20] ( 81.73, 64.76) circle (  2.13);

\path[fill=fillColor,fill opacity=0.20] ( 83.04, 66.38) circle (  2.13);

\path[fill=fillColor,fill opacity=0.20] ( 92.65, 62.32) circle (  2.13);

\path[fill=fillColor,fill opacity=0.20] ( 91.78, 60.69) circle (  2.13);

\path[fill=fillColor,fill opacity=0.20] ( 94.40, 56.63) circle (  2.13);

\path[fill=fillColor,fill opacity=0.20] (105.98, 48.50) circle (  2.13);

\path[fill=fillColor,fill opacity=0.20] ( 95.93, 50.13) circle (  2.13);

\path[fill=fillColor,fill opacity=0.20] ( 94.40, 58.26) circle (  2.13);

\path[fill=fillColor,fill opacity=0.20] ( 86.75, 63.13) circle (  2.13);

\path[fill=fillColor,fill opacity=0.20] ( 84.78, 60.69) circle (  2.13);

\path[fill=fillColor,fill opacity=0.20] ( 78.23, 60.69) circle (  2.13);

\path[fill=fillColor,fill opacity=0.20] ( 74.95, 63.95) circle (  2.13);

\path[fill=fillColor,fill opacity=0.20] ( 83.04, 65.57) circle (  2.13);

\path[fill=fillColor,fill opacity=0.20] ( 78.67, 66.38) circle (  2.13);

\path[fill=fillColor,fill opacity=0.20] ( 79.32, 68.01) circle (  2.13);

\path[fill=fillColor,fill opacity=0.20] ( 86.10, 65.57) circle (  2.13);

\path[fill=fillColor,fill opacity=0.20] ( 79.98, 59.07) circle (  2.13);

\path[fill=fillColor,fill opacity=0.20] ( 78.23, 57.44) circle (  2.13);

\path[fill=fillColor,fill opacity=0.20] ( 47.64, 99.70) circle (  2.13);

\path[fill=fillColor,fill opacity=0.20] ( 45.02, 98.08) circle (  2.13);

\path[fill=fillColor,fill opacity=0.20] ( 46.77, 98.08) circle (  2.13);

\path[fill=fillColor,fill opacity=0.20] ( 51.79,115.15) circle (  2.13);

\path[fill=fillColor,fill opacity=0.20] ( 52.23,108.64) circle (  2.13);

\path[fill=fillColor,fill opacity=0.20] ( 54.19,105.39) circle (  2.13);

\path[fill=fillColor,fill opacity=0.20] ( 86.97,106.21) circle (  2.13);

\path[fill=fillColor,fill opacity=0.20] ( 59.66, 97.27) circle (  2.13);

\path[fill=fillColor,fill opacity=0.20] ( 58.35, 86.70) circle (  2.13);

\path[fill=fillColor,fill opacity=0.20] ( 72.55, 89.95) circle (  2.13);

\path[fill=fillColor,fill opacity=0.20] ( 55.29, 89.95) circle (  2.13);

\path[fill=fillColor,fill opacity=0.20] ( 55.51, 82.64) circle (  2.13);

\path[fill=fillColor,fill opacity=0.20] ( 51.79, 81.82) circle (  2.13);

\path[fill=fillColor,fill opacity=0.20] ( 45.67,103.77) circle (  2.13);

\path[fill=fillColor,fill opacity=0.20] ( 54.63,111.89) circle (  2.13);

\path[fill=fillColor,fill opacity=0.20] ( 73.42,103.77) circle (  2.13);

\path[fill=fillColor,fill opacity=0.20] ( 68.62, 94.83) circle (  2.13);

\path[fill=fillColor,fill opacity=0.20] ( 73.20, 85.89) circle (  2.13);

\path[fill=fillColor,fill opacity=0.20] ( 74.08, 76.95) circle (  2.13);

\path[fill=fillColor,fill opacity=0.20] (140.72, 78.57) circle (  2.13);

\path[fill=fillColor,fill opacity=0.20] ( 89.37,105.39) circle (  2.13);

\path[fill=fillColor,fill opacity=0.20] ( 95.71,115.15) circle (  2.13);

\path[fill=fillColor,fill opacity=0.20] (103.36,102.96) circle (  2.13);

\path[fill=fillColor,fill opacity=0.20] ( 88.06, 91.58) circle (  2.13);

\path[fill=fillColor,fill opacity=0.20] ( 84.57, 72.89) circle (  2.13);

\path[fill=fillColor,fill opacity=0.20] ( 83.04, 61.51) circle (  2.13);

\path[fill=fillColor,fill opacity=0.20] ( 81.73, 65.57) circle (  2.13);

\path[fill=fillColor,fill opacity=0.20] ( 78.01, 61.51) circle (  2.13);

\path[fill=fillColor,fill opacity=0.20] ( 57.25, 96.45) circle (  2.13);

\path[fill=fillColor,fill opacity=0.20] ( 79.76, 99.70) circle (  2.13);

\path[fill=fillColor,fill opacity=0.20] ( 93.31,113.52) circle (  2.13);

\path[fill=fillColor,fill opacity=0.20] (126.52,109.46) circle (  2.13);

\path[fill=fillColor,fill opacity=0.20] (134.82,103.77) circle (  2.13);

\path[fill=fillColor,fill opacity=0.20] (103.14, 94.83) circle (  2.13);

\path[fill=fillColor,fill opacity=0.20] ( 93.09, 96.45) circle (  2.13);

\path[fill=fillColor,fill opacity=0.20] ( 85.00, 97.27) circle (  2.13);

\path[fill=fillColor,fill opacity=0.20] ( 86.31, 74.51) circle (  2.13);

\path[fill=fillColor,fill opacity=0.20] (102.05, 55.01) circle (  2.13);

\path[fill=fillColor,fill opacity=0.20] ( 83.91, 59.07) circle (  2.13);

\path[fill=fillColor,fill opacity=0.20] ( 74.73, 63.95) circle (  2.13);

\path[fill=fillColor,fill opacity=0.20] ( 87.19, 48.50) circle (  2.13);

\path[fill=fillColor,fill opacity=0.20] ( 67.09, 84.26) circle (  2.13);

\path[fill=fillColor,fill opacity=0.20] (124.99, 86.70) circle (  2.13);

\path[fill=fillColor,fill opacity=0.20] (112.53,102.14) circle (  2.13);

\path[fill=fillColor,fill opacity=0.20] ( 93.31,109.46) circle (  2.13);

\path[fill=fillColor,fill opacity=0.20] ( 91.78,108.64) circle (  2.13);

\path[fill=fillColor,fill opacity=0.20] ( 93.09, 96.45) circle (  2.13);

\path[fill=fillColor,fill opacity=0.20] ( 94.40, 89.95) circle (  2.13);

\path[fill=fillColor,fill opacity=0.20] ( 83.91, 95.64) circle (  2.13);

\path[fill=fillColor,fill opacity=0.20] ( 81.51, 81.82) circle (  2.13);

\path[fill=fillColor,fill opacity=0.20] ( 85.44, 57.44) circle (  2.13);

\path[fill=fillColor,fill opacity=0.20] ( 79.98, 71.26) circle (  2.13);

\path[fill=fillColor,fill opacity=0.20] ( 91.34, 72.89) circle (  2.13);

\path[fill=fillColor,fill opacity=0.20] (102.92, 81.01) circle (  2.13);

\path[fill=fillColor,fill opacity=0.20] (105.76, 75.32) circle (  2.13);

\path[fill=fillColor,fill opacity=0.20] (148.80, 56.63) circle (  2.13);

\path[fill=fillColor,fill opacity=0.20] ( 86.53, 76.14) circle (  2.13);

\path[fill=fillColor,fill opacity=0.20] (103.36, 74.51) circle (  2.13);

\path[fill=fillColor,fill opacity=0.20] (119.31, 94.83) circle (  2.13);

\path[fill=fillColor,fill opacity=0.20] ( 99.64,106.21) circle (  2.13);

\path[fill=fillColor,fill opacity=0.20] ( 93.31,108.64) circle (  2.13);

\path[fill=fillColor,fill opacity=0.20] ( 84.78, 96.45) circle (  2.13);

\path[fill=fillColor,fill opacity=0.20] ( 95.05, 85.08) circle (  2.13);

\path[fill=fillColor,fill opacity=0.20] ( 78.45, 85.89) circle (  2.13);

\path[fill=fillColor,fill opacity=0.20] ( 79.98, 68.82) circle (  2.13);

\path[fill=fillColor,fill opacity=0.20] ( 91.34, 61.51) circle (  2.13);

\path[fill=fillColor,fill opacity=0.20] (101.39, 62.32) circle (  2.13);

\path[fill=fillColor,fill opacity=0.20] ( 98.33, 77.76) circle (  2.13);

\path[fill=fillColor,fill opacity=0.20] (123.24, 89.14) circle (  2.13);

\path[fill=fillColor,fill opacity=0.20] (123.24, 99.70) circle (  2.13);

\path[fill=fillColor,fill opacity=0.20] (128.27,101.33) circle (  2.13);

\path[fill=fillColor,fill opacity=0.20] (138.32, 91.58) circle (  2.13);

\path[fill=fillColor,fill opacity=0.20] ( 88.50, 73.70) circle (  2.13);

\path[fill=fillColor,fill opacity=0.20] (107.07, 65.57) circle (  2.13);

\path[fill=fillColor,fill opacity=0.20] ( 95.71, 85.89) circle (  2.13);

\path[fill=fillColor,fill opacity=0.20] (100.52,102.96) circle (  2.13);

\path[fill=fillColor,fill opacity=0.20] ( 95.27, 94.83) circle (  2.13);

\path[fill=fillColor,fill opacity=0.20] ( 96.80, 85.89) circle (  2.13);

\path[fill=fillColor,fill opacity=0.20] ( 92.43, 81.82) circle (  2.13);

\path[fill=fillColor,fill opacity=0.20] ( 78.23, 77.76) circle (  2.13);

\path[fill=fillColor,fill opacity=0.20] ( 63.81, 68.82) circle (  2.13);

\path[fill=fillColor,fill opacity=0.20] ( 86.97, 68.01) circle (  2.13);

\path[fill=fillColor,fill opacity=0.20] ( 94.62, 83.45) circle (  2.13);

\path[fill=fillColor,fill opacity=0.20] (102.92, 91.58) circle (  2.13);

\path[fill=fillColor,fill opacity=0.20] (120.18, 94.83) circle (  2.13);

\path[fill=fillColor,fill opacity=0.20] (131.54,107.02) circle (  2.13);

\path[fill=fillColor,fill opacity=0.20] (121.27,108.64) circle (  2.13);

\path[fill=fillColor,fill opacity=0.20] (110.79, 72.89) circle (  2.13);

\path[fill=fillColor,fill opacity=0.20] ( 98.55, 71.26) circle (  2.13);

\path[fill=fillColor,fill opacity=0.20] (131.76, 55.82) circle (  2.13);

\path[fill=fillColor,fill opacity=0.20] ( 85.88, 80.20) circle (  2.13);

\path[fill=fillColor,fill opacity=0.20] ( 87.84, 97.27) circle (  2.13);

\path[fill=fillColor,fill opacity=0.20] ( 92.43, 82.64) circle (  2.13);

\path[fill=fillColor,fill opacity=0.20] ( 91.34, 76.14) circle (  2.13);

\path[fill=fillColor,fill opacity=0.20] ( 84.13, 85.08) circle (  2.13);

\path[fill=fillColor,fill opacity=0.20] (112.10, 83.45) circle (  2.13);

\path[fill=fillColor,fill opacity=0.20] ( 77.14, 74.51) circle (  2.13);

\path[fill=fillColor,fill opacity=0.20] ( 97.46, 72.89) circle (  2.13);

\path[fill=fillColor,fill opacity=0.20] ( 99.86, 81.82) circle (  2.13);

\path[fill=fillColor,fill opacity=0.20] ( 96.36, 87.51) circle (  2.13);

\path[fill=fillColor,fill opacity=0.20] (105.98, 94.02) circle (  2.13);

\path[fill=fillColor,fill opacity=0.20] (123.02,108.64) circle (  2.13);

\path[fill=fillColor,fill opacity=0.20] (125.43,115.15) circle (  2.13);

\path[fill=fillColor,fill opacity=0.20] (118.43, 98.89) circle (  2.13);

\path[fill=fillColor,fill opacity=0.20] (126.30, 76.14) circle (  2.13);

\path[fill=fillColor,fill opacity=0.20] ( 94.84, 55.01) circle (  2.13);

\path[fill=fillColor,fill opacity=0.20] (139.63, 69.63) circle (  2.13);

\path[fill=fillColor,fill opacity=0.20] (112.75, 49.32) circle (  2.13);

\path[fill=fillColor,fill opacity=0.20] ( 91.99, 73.70) circle (  2.13);

\path[fill=fillColor,fill opacity=0.20] ( 84.35, 89.14) circle (  2.13);

\path[fill=fillColor,fill opacity=0.20] ( 85.44, 76.14) circle (  2.13);

\path[fill=fillColor,fill opacity=0.20] ( 82.16, 72.07) circle (  2.13);

\path[fill=fillColor,fill opacity=0.20] ( 80.85, 81.01) circle (  2.13);

\path[fill=fillColor,fill opacity=0.20] ( 78.01, 81.01) circle (  2.13);

\path[fill=fillColor,fill opacity=0.20] ( 74.73, 67.20) circle (  2.13);

\path[fill=fillColor,fill opacity=0.20] ( 73.64, 59.07) circle (  2.13);

\path[fill=fillColor,fill opacity=0.20] ( 69.93, 84.26) circle (  2.13);

\path[fill=fillColor,fill opacity=0.20] (116.03, 60.69) circle (  2.13);

\path[fill=fillColor,fill opacity=0.20] (102.70, 78.57) circle (  2.13);

\path[fill=fillColor,fill opacity=0.20] (100.73, 81.01) circle (  2.13);

\path[fill=fillColor,fill opacity=0.20] (103.58, 85.89) circle (  2.13);

\path[fill=fillColor,fill opacity=0.20] (109.47,105.39) circle (  2.13);

\path[fill=fillColor,fill opacity=0.20] (139.63,105.39) circle (  2.13);

\path[fill=fillColor,fill opacity=0.20] (147.49, 93.20) circle (  2.13);

\path[fill=fillColor,fill opacity=0.20] (145.53,101.33) circle (  2.13);

\path[fill=fillColor,fill opacity=0.20] ( 93.96, 63.13) circle (  2.13);

\path[fill=fillColor,fill opacity=0.20] (102.26, 55.01) circle (  2.13);

\path[fill=fillColor,fill opacity=0.20] ( 97.02, 67.20) circle (  2.13);

\path[fill=fillColor,fill opacity=0.20] ( 91.12, 81.82) circle (  2.13);

\path[fill=fillColor,fill opacity=0.20] ( 86.53, 81.82) circle (  2.13);

\path[fill=fillColor,fill opacity=0.20] ( 92.87, 78.57) circle (  2.13);

\path[fill=fillColor,fill opacity=0.20] ( 84.57, 76.14) circle (  2.13);

\path[fill=fillColor,fill opacity=0.20] ( 85.00, 73.70) circle (  2.13);

\path[fill=fillColor,fill opacity=0.20] ( 79.98, 65.57) circle (  2.13);

\path[fill=fillColor,fill opacity=0.20] ( 88.06, 64.76) circle (  2.13);

\path[fill=fillColor,fill opacity=0.20] ( 69.71, 71.26) circle (  2.13);

\path[fill=fillColor,fill opacity=0.20] ( 55.07, 89.95) circle (  2.13);

\path[fill=fillColor,fill opacity=0.20] ( 75.61, 74.51) circle (  2.13);

\path[fill=fillColor,fill opacity=0.20] (110.79, 59.07) circle (  2.13);

\path[fill=fillColor,fill opacity=0.20] (101.83, 81.01) circle (  2.13);

\path[fill=fillColor,fill opacity=0.20] (110.57, 84.26) circle (  2.13);

\path[fill=fillColor,fill opacity=0.20] (111.22, 89.14) circle (  2.13);

\path[fill=fillColor,fill opacity=0.20] (105.32,101.33) circle (  2.13);

\path[fill=fillColor,fill opacity=0.20] (115.16,102.96) circle (  2.13);

\path[fill=fillColor,fill opacity=0.20] (112.97, 95.64) circle (  2.13);

\path[fill=fillColor,fill opacity=0.20] (117.78, 98.89) circle (  2.13);

\path[fill=fillColor,fill opacity=0.20] (125.64,105.39) circle (  2.13);

\path[fill=fillColor,fill opacity=0.20] (141.38,107.02) circle (  2.13);

\path[fill=fillColor,fill opacity=0.20] (129.58,109.46) circle (  2.13);

\path[fill=fillColor,fill opacity=0.20] (123.46, 93.20) circle (  2.13);

\path[fill=fillColor,fill opacity=0.20] ( 94.84, 54.19) circle (  2.13);

\path[fill=fillColor,fill opacity=0.20] ( 98.33, 62.32) circle (  2.13);

\path[fill=fillColor,fill opacity=0.20] (102.48, 55.82) circle (  2.13);

\path[fill=fillColor,fill opacity=0.20] ( 96.58, 68.01) circle (  2.13);

\path[fill=fillColor,fill opacity=0.20] ( 92.21, 82.64) circle (  2.13);

\path[fill=fillColor,fill opacity=0.20] ( 79.76, 85.89) circle (  2.13);

\path[fill=fillColor,fill opacity=0.20] ( 81.07, 80.20) circle (  2.13);

\path[fill=fillColor,fill opacity=0.20] ( 86.97, 68.82) circle (  2.13);

\path[fill=fillColor,fill opacity=0.20] ( 83.25, 72.07) circle (  2.13);

\path[fill=fillColor,fill opacity=0.20] ( 77.36, 83.45) circle (  2.13);

\path[fill=fillColor,fill opacity=0.20] ( 71.24, 82.64) circle (  2.13);

\path[fill=fillColor,fill opacity=0.20] ( 61.40, 79.39) circle (  2.13);

\path[fill=fillColor,fill opacity=0.20] ( 53.76, 92.39) circle (  2.13);

\path[fill=fillColor,fill opacity=0.20] ( 97.68, 73.70) circle (  2.13);

\path[fill=fillColor,fill opacity=0.20] (109.04, 75.32) circle (  2.13);

\path[fill=fillColor,fill opacity=0.20] (107.95, 81.01) circle (  2.13);

\path[fill=fillColor,fill opacity=0.20] (102.26, 94.83) circle (  2.13);

\path[fill=fillColor,fill opacity=0.20] (106.20, 96.45) circle (  2.13);

\path[fill=fillColor,fill opacity=0.20] (114.28, 90.76) circle (  2.13);

\path[fill=fillColor,fill opacity=0.20] (116.69, 98.08) circle (  2.13);

\path[fill=fillColor,fill opacity=0.20] (110.57,111.08) circle (  2.13);

\path[fill=fillColor,fill opacity=0.20] (117.78,107.02) circle (  2.13);

\path[fill=fillColor,fill opacity=0.20] (136.57,100.52) circle (  2.13);

\path[fill=fillColor,fill opacity=0.20] (125.43,111.89) circle (  2.13);

\path[fill=fillColor,fill opacity=0.20] (109.04,111.89) circle (  2.13);

\path[fill=fillColor,fill opacity=0.20] (125.64, 84.26) circle (  2.13);

\path[fill=fillColor,fill opacity=0.20] (135.26, 55.01) circle (  2.13);

\path[fill=fillColor,fill opacity=0.20] (101.83, 49.32) circle (  2.13);

\path[fill=fillColor,fill opacity=0.20] ( 96.58, 45.25) circle (  2.13);

\path[fill=fillColor,fill opacity=0.20] ( 88.06, 59.07) circle (  2.13);

\path[fill=fillColor,fill opacity=0.20] ( 77.79, 74.51) circle (  2.13);

\path[fill=fillColor,fill opacity=0.20] ( 77.57, 72.07) circle (  2.13);

\path[fill=fillColor,fill opacity=0.20] ( 81.73, 68.82) circle (  2.13);

\path[fill=fillColor,fill opacity=0.20] ( 76.26, 81.01) circle (  2.13);

\path[fill=fillColor,fill opacity=0.20] ( 74.95, 92.39) circle (  2.13);

\path[fill=fillColor,fill opacity=0.20] ( 82.60, 88.33) circle (  2.13);

\path[fill=fillColor,fill opacity=0.20] ( 74.30, 84.26) circle (  2.13);

\path[fill=fillColor,fill opacity=0.20] ( 64.46, 83.45) circle (  2.13);

\path[fill=fillColor,fill opacity=0.20] ( 56.82, 88.33) circle (  2.13);

\path[fill=fillColor,fill opacity=0.20] ( 64.90,102.14) circle (  2.13);

\path[fill=fillColor,fill opacity=0.20] ( 91.12, 90.76) circle (  2.13);

\path[fill=fillColor,fill opacity=0.20] (114.28, 85.08) circle (  2.13);

\path[fill=fillColor,fill opacity=0.20] (104.23, 85.08) circle (  2.13);

\path[fill=fillColor,fill opacity=0.20] (103.58, 92.39) circle (  2.13);

\path[fill=fillColor,fill opacity=0.20] (103.58, 93.20) circle (  2.13);

\path[fill=fillColor,fill opacity=0.20] (127.39, 99.70) circle (  2.13);

\path[fill=fillColor,fill opacity=0.20] (129.80,115.15) circle (  2.13);

\path[fill=fillColor,fill opacity=0.20] (109.47,113.52) circle (  2.13);

\path[fill=fillColor,fill opacity=0.20] (116.90, 98.89) circle (  2.13);

\path[fill=fillColor,fill opacity=0.20] (135.26, 99.70) circle (  2.13);

\path[fill=fillColor,fill opacity=0.20] (120.40,115.96) circle (  2.13);

\path[fill=fillColor,fill opacity=0.20] (100.08,105.39) circle (  2.13);

\path[fill=fillColor,fill opacity=0.20] (116.90, 43.63) circle (  2.13);

\path[fill=fillColor,fill opacity=0.20] ( 85.00, 48.50) circle (  2.13);

\path[fill=fillColor,fill opacity=0.20] ( 83.69, 52.57) circle (  2.13);

\path[fill=fillColor,fill opacity=0.20] ( 84.57, 61.51) circle (  2.13);

\path[fill=fillColor,fill opacity=0.20] ( 72.11, 76.95) circle (  2.13);

\path[fill=fillColor,fill opacity=0.20] ( 75.83, 81.01) circle (  2.13);

\path[fill=fillColor,fill opacity=0.20] ( 96.80, 72.07) circle (  2.13);

\path[fill=fillColor,fill opacity=0.20] ( 78.45, 76.95) circle (  2.13);

\path[fill=fillColor,fill opacity=0.20] ( 72.99, 81.82) circle (  2.13);

\path[fill=fillColor,fill opacity=0.20] ( 63.81, 76.14) circle (  2.13);

\path[fill=fillColor,fill opacity=0.20] ( 61.84, 85.89) circle (  2.13);

\path[fill=fillColor,fill opacity=0.20] ( 70.36,102.96) circle (  2.13);

\path[fill=fillColor,fill opacity=0.20] ( 83.04, 89.95) circle (  2.13);

\path[fill=fillColor,fill opacity=0.20] ( 93.31, 98.89) circle (  2.13);

\path[fill=fillColor,fill opacity=0.20] (104.23, 97.27) circle (  2.13);

\path[fill=fillColor,fill opacity=0.20] (113.41, 87.51) circle (  2.13);

\path[fill=fillColor,fill opacity=0.20] (116.25, 92.39) circle (  2.13);

\path[fill=fillColor,fill opacity=0.20] (115.37,107.02) circle (  2.13);

\path[fill=fillColor,fill opacity=0.20] (109.47,114.33) circle (  2.13);

\path[fill=fillColor,fill opacity=0.20] (116.03,106.21) circle (  2.13);

\path[fill=fillColor,fill opacity=0.20] (118.43,101.33) circle (  2.13);

\path[fill=fillColor,fill opacity=0.20] (119.74,105.39) circle (  2.13);

\path[fill=fillColor,fill opacity=0.20] (122.80,108.64) circle (  2.13);

\path[fill=fillColor,fill opacity=0.20] (116.25,111.08) circle (  2.13);

\path[fill=fillColor,fill opacity=0.20] (107.73, 41.19) circle (  2.13);

\path[fill=fillColor,fill opacity=0.20] ( 90.03, 40.38) circle (  2.13);

\path[fill=fillColor,fill opacity=0.20] ( 81.51, 45.25) circle (  2.13);

\path[fill=fillColor,fill opacity=0.20] ( 84.13, 53.38) circle (  2.13);

\path[fill=fillColor,fill opacity=0.20] ( 82.60, 56.63) circle (  2.13);

\path[fill=fillColor,fill opacity=0.20] ( 86.75, 59.07) circle (  2.13);

\path[fill=fillColor,fill opacity=0.20] ( 88.72, 59.07) circle (  2.13);

\path[fill=fillColor,fill opacity=0.20] ( 90.25, 66.38) circle (  2.13);

\path[fill=fillColor,fill opacity=0.20] ( 87.62, 71.26) circle (  2.13);

\path[fill=fillColor,fill opacity=0.20] ( 77.57, 69.63) circle (  2.13);

\path[fill=fillColor,fill opacity=0.20] ( 73.64, 72.07) circle (  2.13);

\path[fill=fillColor,fill opacity=0.20] ( 67.52, 74.51) circle (  2.13);

\path[fill=fillColor,fill opacity=0.20] ( 60.75, 71.26) circle (  2.13);

\path[fill=fillColor,fill opacity=0.20] ( 54.41, 80.20) circle (  2.13);

\path[fill=fillColor,fill opacity=0.20] ( 53.76,108.64) circle (  2.13);

\path[fill=fillColor,fill opacity=0.20] ( 56.82,103.77) circle (  2.13);

\path[fill=fillColor,fill opacity=0.20] ( 64.90, 97.27) circle (  2.13);

\path[fill=fillColor,fill opacity=0.20] ( 56.16,104.58) circle (  2.13);

\path[fill=fillColor,fill opacity=0.20] ( 72.55, 90.76) circle (  2.13);

\path[fill=fillColor,fill opacity=0.20] (103.58, 77.76) circle (  2.13);

\path[fill=fillColor,fill opacity=0.20] ( 93.09, 79.39) circle (  2.13);

\path[fill=fillColor,fill opacity=0.20] ( 91.78, 93.20) circle (  2.13);

\path[fill=fillColor,fill opacity=0.20] (104.89, 92.39) circle (  2.13);

\path[fill=fillColor,fill opacity=0.20] (108.82, 81.01) circle (  2.13);

\path[fill=fillColor,fill opacity=0.20] (114.72, 83.45) circle (  2.13);

\path[fill=fillColor,fill opacity=0.20] (109.47, 98.08) circle (  2.13);

\path[fill=fillColor,fill opacity=0.20] (116.03, 98.89) circle (  2.13);

\path[fill=fillColor,fill opacity=0.20] (118.65, 91.58) circle (  2.13);

\path[fill=fillColor,fill opacity=0.20] (112.75,104.58) circle (  2.13);

\path[fill=fillColor,fill opacity=0.20] (115.16,110.27) circle (  2.13);

\path[fill=fillColor,fill opacity=0.20] (113.63, 90.76) circle (  2.13);

\path[fill=fillColor,fill opacity=0.20] (106.42, 96.45) circle (  2.13);

\path[fill=fillColor,fill opacity=0.20] ( 84.35, 49.32) circle (  2.13);

\path[fill=fillColor,fill opacity=0.20] ( 91.34, 50.94) circle (  2.13);

\path[fill=fillColor,fill opacity=0.20] ( 97.24, 49.32) circle (  2.13);

\path[fill=fillColor,fill opacity=0.20] ( 85.88, 54.19) circle (  2.13);

\path[fill=fillColor,fill opacity=0.20] ( 85.44, 59.88) circle (  2.13);

\path[fill=fillColor,fill opacity=0.20] ( 78.23, 56.63) circle (  2.13);

\path[fill=fillColor,fill opacity=0.20] ( 83.25, 54.19) circle (  2.13);

\path[fill=fillColor,fill opacity=0.20] ( 78.88, 57.44) circle (  2.13);

\path[fill=fillColor,fill opacity=0.20] ( 72.33, 64.76) circle (  2.13);

\path[fill=fillColor,fill opacity=0.20] ( 69.27, 60.69) circle (  2.13);

\path[fill=fillColor,fill opacity=0.20] ( 71.24, 64.76) circle (  2.13);

\path[fill=fillColor,fill opacity=0.20] ( 68.40, 72.89) circle (  2.13);

\path[fill=fillColor,fill opacity=0.20] ( 75.17, 73.70) circle (  2.13);

\path[fill=fillColor,fill opacity=0.20] ( 60.31, 75.32) circle (  2.13);

\path[fill=fillColor,fill opacity=0.20] ( 55.51, 84.26) circle (  2.13);

\path[fill=fillColor,fill opacity=0.20] ( 83.91, 83.45) circle (  2.13);

\path[fill=fillColor,fill opacity=0.20] ( 66.21, 81.82) circle (  2.13);

\path[fill=fillColor,fill opacity=0.20] ( 74.51, 72.89) circle (  2.13);

\path[fill=fillColor,fill opacity=0.20] ( 74.73, 79.39) circle (  2.13);

\path[fill=fillColor,fill opacity=0.20] ( 74.08, 89.95) circle (  2.13);

\path[fill=fillColor,fill opacity=0.20] ( 78.23, 81.82) circle (  2.13);

\path[fill=fillColor,fill opacity=0.20] ( 83.91, 76.95) circle (  2.13);

\path[fill=fillColor,fill opacity=0.20] ( 95.71, 76.14) circle (  2.13);

\path[fill=fillColor,fill opacity=0.20] ( 92.21, 72.07) circle (  2.13);

\path[fill=fillColor,fill opacity=0.20] ( 99.42, 79.39) circle (  2.13);

\path[fill=fillColor,fill opacity=0.20] ( 91.12, 77.76) circle (  2.13);

\path[fill=fillColor,fill opacity=0.20] (104.23, 78.57) circle (  2.13);

\path[fill=fillColor,fill opacity=0.20] (104.67, 89.14) circle (  2.13);

\path[fill=fillColor,fill opacity=0.20] (109.04, 94.83) circle (  2.13);

\path[fill=fillColor,fill opacity=0.20] (110.57, 85.89) circle (  2.13);

\path[fill=fillColor,fill opacity=0.20] (119.09, 89.14) circle (  2.13);

\path[fill=fillColor,fill opacity=0.20] (115.81, 96.45) circle (  2.13);

\path[fill=fillColor,fill opacity=0.20] (115.59, 81.82) circle (  2.13);

\path[fill=fillColor,fill opacity=0.20] (115.37, 82.64) circle (  2.13);

\path[fill=fillColor,fill opacity=0.20] (105.32,115.96) circle (  2.13);

\path[fill=fillColor,fill opacity=0.20] ( 85.88, 52.57) circle (  2.13);

\path[fill=fillColor,fill opacity=0.20] ( 79.54, 50.94) circle (  2.13);

\path[fill=fillColor,fill opacity=0.20] ( 76.70, 46.88) circle (  2.13);

\path[fill=fillColor,fill opacity=0.20] ( 79.32, 46.07) circle (  2.13);

\path[fill=fillColor,fill opacity=0.20] ( 79.76, 55.01) circle (  2.13);

\path[fill=fillColor,fill opacity=0.20] ( 78.67, 63.13) circle (  2.13);

\path[fill=fillColor,fill opacity=0.20] ( 81.51, 61.51) circle (  2.13);

\path[fill=fillColor,fill opacity=0.20] ( 77.79, 58.26) circle (  2.13);

\path[fill=fillColor,fill opacity=0.20] ( 65.34, 63.95) circle (  2.13);

\path[fill=fillColor,fill opacity=0.20] ( 77.14, 68.01) circle (  2.13);

\path[fill=fillColor,fill opacity=0.20] ( 64.90, 69.63) circle (  2.13);

\path[fill=fillColor,fill opacity=0.20] ( 57.25, 76.14) circle (  2.13);

\path[fill=fillColor,fill opacity=0.20] ( 52.01, 79.39) circle (  2.13);

\path[fill=fillColor,fill opacity=0.20] ( 66.43, 79.39) circle (  2.13);

\path[fill=fillColor,fill opacity=0.20] ( 76.04, 79.39) circle (  2.13);

\path[fill=fillColor,fill opacity=0.20] ( 78.67, 66.38) circle (  2.13);

\path[fill=fillColor,fill opacity=0.20] ( 89.59, 67.20) circle (  2.13);

\path[fill=fillColor,fill opacity=0.20] ( 94.84, 89.14) circle (  2.13);

\path[fill=fillColor,fill opacity=0.20] ( 94.40, 94.02) circle (  2.13);

\path[fill=fillColor,fill opacity=0.20] ( 84.57, 82.64) circle (  2.13);

\path[fill=fillColor,fill opacity=0.20] ( 86.10, 80.20) circle (  2.13);

\path[fill=fillColor,fill opacity=0.20] (106.42, 85.89) circle (  2.13);

\path[fill=fillColor,fill opacity=0.20] ( 98.77, 73.70) circle (  2.13);

\path[fill=fillColor,fill opacity=0.20] ( 90.47, 65.57) circle (  2.13);

\path[fill=fillColor,fill opacity=0.20] ( 95.49, 76.14) circle (  2.13);

\path[fill=fillColor,fill opacity=0.20] ( 92.21, 87.51) circle (  2.13);

\path[fill=fillColor,fill opacity=0.20] ( 96.58, 88.33) circle (  2.13);

\path[fill=fillColor,fill opacity=0.20] (103.58, 85.08) circle (  2.13);

\path[fill=fillColor,fill opacity=0.20] (100.95, 89.95) circle (  2.13);

\path[fill=fillColor,fill opacity=0.20] (100.52, 89.95) circle (  2.13);

\path[fill=fillColor,fill opacity=0.20] (129.36, 80.20) circle (  2.13);

\path[fill=fillColor,fill opacity=0.20] (114.50, 83.45) circle (  2.13);

\path[fill=fillColor,fill opacity=0.20] (121.06, 85.89) circle (  2.13);

\path[fill=fillColor,fill opacity=0.20] (128.70, 80.20) circle (  2.13);

\path[fill=fillColor,fill opacity=0.20] (104.89,111.89) circle (  2.13);

\path[fill=fillColor,fill opacity=0.20] ( 86.10, 53.38) circle (  2.13);

\path[fill=fillColor,fill opacity=0.20] ( 77.79, 49.32) circle (  2.13);

\path[fill=fillColor,fill opacity=0.20] ( 79.10, 44.44) circle (  2.13);

\path[fill=fillColor,fill opacity=0.20] ( 82.16, 46.88) circle (  2.13);

\path[fill=fillColor,fill opacity=0.20] ( 77.36, 46.07) circle (  2.13);

\path[fill=fillColor,fill opacity=0.20] ( 76.70, 43.63) circle (  2.13);

\path[fill=fillColor,fill opacity=0.20] ( 77.79, 46.07) circle (  2.13);

\path[fill=fillColor,fill opacity=0.20] ( 79.32, 57.44) circle (  2.13);

\path[fill=fillColor,fill opacity=0.20] ( 75.61, 57.44) circle (  2.13);

\path[fill=fillColor,fill opacity=0.20] ( 74.51, 61.51) circle (  2.13);

\path[fill=fillColor,fill opacity=0.20] ( 70.80, 68.82) circle (  2.13);

\path[fill=fillColor,fill opacity=0.20] ( 68.62, 68.82) circle (  2.13);

\path[fill=fillColor,fill opacity=0.20] ( 62.50, 65.57) circle (  2.13);

\path[fill=fillColor,fill opacity=0.20] ( 59.66, 74.51) circle (  2.13);

\path[fill=fillColor,fill opacity=0.20] ( 52.66, 89.14) circle (  2.13);

\path[fill=fillColor,fill opacity=0.20] ( 69.71, 72.07) circle (  2.13);

\path[fill=fillColor,fill opacity=0.20] ( 73.86, 63.95) circle (  2.13);

\path[fill=fillColor,fill opacity=0.20] ( 76.04, 78.57) circle (  2.13);

\path[fill=fillColor,fill opacity=0.20] ( 85.66, 81.01) circle (  2.13);

\path[fill=fillColor,fill opacity=0.20] ( 77.36, 68.82) circle (  2.13);

\path[fill=fillColor,fill opacity=0.20] ( 85.88, 71.26) circle (  2.13);

\path[fill=fillColor,fill opacity=0.20] ( 87.84, 89.14) circle (  2.13);

\path[fill=fillColor,fill opacity=0.20] (101.17, 90.76) circle (  2.13);

\path[fill=fillColor,fill opacity=0.20] ( 85.66, 85.08) circle (  2.13);

\path[fill=fillColor,fill opacity=0.20] ( 90.90, 76.14) circle (  2.13);

\path[fill=fillColor,fill opacity=0.20] ( 90.68, 69.63) circle (  2.13);

\path[fill=fillColor,fill opacity=0.20] ( 91.99, 74.51) circle (  2.13);

\path[fill=fillColor,fill opacity=0.20] ( 89.81, 78.57) circle (  2.13);

\path[fill=fillColor,fill opacity=0.20] ( 86.97, 85.08) circle (  2.13);

\path[fill=fillColor,fill opacity=0.20] ( 98.99, 94.83) circle (  2.13);

\path[fill=fillColor,fill opacity=0.20] ( 99.86, 92.39) circle (  2.13);

\path[fill=fillColor,fill opacity=0.20] ( 99.64, 81.82) circle (  2.13);

\path[fill=fillColor,fill opacity=0.20] (100.08, 85.08) circle (  2.13);

\path[fill=fillColor,fill opacity=0.20] (101.39, 85.89) circle (  2.13);

\path[fill=fillColor,fill opacity=0.20] (108.16, 82.64) circle (  2.13);

\path[fill=fillColor,fill opacity=0.20] ( 97.02, 85.89) circle (  2.13);

\path[fill=fillColor,fill opacity=0.20] (117.34, 83.45) circle (  2.13);

\path[fill=fillColor,fill opacity=0.20] (136.79, 81.01) circle (  2.13);

\path[fill=fillColor,fill opacity=0.20] (103.36,113.52) circle (  2.13);

\path[fill=fillColor,fill opacity=0.20] ( 73.42, 40.38) circle (  2.13);

\path[fill=fillColor,fill opacity=0.20] ( 80.41, 40.38) circle (  2.13);

\path[fill=fillColor,fill opacity=0.20] ( 79.98, 44.44) circle (  2.13);

\path[fill=fillColor,fill opacity=0.20] ( 78.88, 48.50) circle (  2.13);

\path[fill=fillColor,fill opacity=0.20] ( 76.04, 53.38) circle (  2.13);

\path[fill=fillColor,fill opacity=0.20] ( 76.92, 55.01) circle (  2.13);

\path[fill=fillColor,fill opacity=0.20] ( 71.89, 59.07) circle (  2.13);

\path[fill=fillColor,fill opacity=0.20] ( 68.62, 60.69) circle (  2.13);

\path[fill=fillColor,fill opacity=0.20] ( 71.46, 59.07) circle (  2.13);

\path[fill=fillColor,fill opacity=0.20] ( 68.18, 56.63) circle (  2.13);

\path[fill=fillColor,fill opacity=0.20] ( 65.56, 59.07) circle (  2.13);

\path[fill=fillColor,fill opacity=0.20] ( 59.44, 75.32) circle (  2.13);

\path[fill=fillColor,fill opacity=0.20] ( 53.54, 77.76) circle (  2.13);

\path[fill=fillColor,fill opacity=0.20] ( 58.35, 66.38) circle (  2.13);

\path[fill=fillColor,fill opacity=0.20] ( 64.03, 63.95) circle (  2.13);

\path[fill=fillColor,fill opacity=0.20] ( 72.11, 67.20) circle (  2.13);

\path[fill=fillColor,fill opacity=0.20] ( 64.68, 74.51) circle (  2.13);

\path[fill=fillColor,fill opacity=0.20] ( 70.14, 59.07) circle (  2.13);

\path[fill=fillColor,fill opacity=0.20] ( 69.49, 59.88) circle (  2.13);

\path[fill=fillColor,fill opacity=0.20] ( 70.58, 59.07) circle (  2.13);

\path[fill=fillColor,fill opacity=0.20] ( 71.46, 59.88) circle (  2.13);

\path[fill=fillColor,fill opacity=0.20] ( 76.04, 61.51) circle (  2.13);

\path[fill=fillColor,fill opacity=0.20] ( 80.20, 66.38) circle (  2.13);

\path[fill=fillColor,fill opacity=0.20] ( 79.76, 71.26) circle (  2.13);

\path[fill=fillColor,fill opacity=0.20] ( 81.94, 67.20) circle (  2.13);

\path[fill=fillColor,fill opacity=0.20] ( 86.53, 67.20) circle (  2.13);

\path[fill=fillColor,fill opacity=0.20] ( 83.25, 74.51) circle (  2.13);

\path[fill=fillColor,fill opacity=0.20] ( 84.13, 81.01) circle (  2.13);

\path[fill=fillColor,fill opacity=0.20] ( 91.34, 82.64) circle (  2.13);

\path[fill=fillColor,fill opacity=0.20] ( 98.55, 75.32) circle (  2.13);

\path[fill=fillColor,fill opacity=0.20] ( 93.09, 68.01) circle (  2.13);

\path[fill=fillColor,fill opacity=0.20] ( 84.57, 79.39) circle (  2.13);

\path[fill=fillColor,fill opacity=0.20] ( 87.62, 89.95) circle (  2.13);

\path[fill=fillColor,fill opacity=0.20] ( 95.71, 90.76) circle (  2.13);

\path[fill=fillColor,fill opacity=0.20] (100.95, 88.33) circle (  2.13);

\path[fill=fillColor,fill opacity=0.20] (100.73, 77.76) circle (  2.13);

\path[fill=fillColor,fill opacity=0.20] ( 92.87, 80.20) circle (  2.13);

\path[fill=fillColor,fill opacity=0.20] ( 99.64, 85.08) circle (  2.13);

\path[fill=fillColor,fill opacity=0.20] ( 98.99, 77.76) circle (  2.13);

\path[fill=fillColor,fill opacity=0.20] ( 95.93, 81.01) circle (  2.13);

\path[fill=fillColor,fill opacity=0.20] ( 91.78, 81.82) circle (  2.13);

\path[fill=fillColor,fill opacity=0.20] (115.37, 74.51) circle (  2.13);

\path[fill=fillColor,fill opacity=0.20] (123.24, 81.01) circle (  2.13);

\path[fill=fillColor,fill opacity=0.20] ( 82.60, 41.19) circle (  2.13);

\path[fill=fillColor,fill opacity=0.20] ( 92.65, 42.82) circle (  2.13);

\path[fill=fillColor,fill opacity=0.20] ( 77.36, 44.44) circle (  2.13);

\path[fill=fillColor,fill opacity=0.20] ( 72.99, 50.13) circle (  2.13);

\path[fill=fillColor,fill opacity=0.20] ( 69.71, 50.94) circle (  2.13);

\path[fill=fillColor,fill opacity=0.20] ( 76.48, 55.01) circle (  2.13);

\path[fill=fillColor,fill opacity=0.20] ( 79.32, 57.44) circle (  2.13);

\path[fill=fillColor,fill opacity=0.20] ( 76.26, 56.63) circle (  2.13);

\path[fill=fillColor,fill opacity=0.20] ( 71.24, 63.95) circle (  2.13);

\path[fill=fillColor,fill opacity=0.20] ( 55.29, 79.39) circle (  2.13);

\path[fill=fillColor,fill opacity=0.20] ( 56.82, 66.38) circle (  2.13);

\path[fill=fillColor,fill opacity=0.20] ( 47.20, 97.27) circle (  2.13);

\path[fill=fillColor,fill opacity=0.20] ( 69.27, 91.58) circle (  2.13);

\path[fill=fillColor,fill opacity=0.20] ( 54.85, 80.20) circle (  2.13);

\path[fill=fillColor,fill opacity=0.20] ( 63.81, 76.14) circle (  2.13);

\path[fill=fillColor,fill opacity=0.20] ( 65.77, 72.07) circle (  2.13);

\path[fill=fillColor,fill opacity=0.20] ( 62.72, 46.07) circle (  2.13);

\path[fill=fillColor,fill opacity=0.20] ( 81.29, 53.38) circle (  2.13);

\path[fill=fillColor,fill opacity=0.20] ( 75.17, 58.26) circle (  2.13);

\path[fill=fillColor,fill opacity=0.20] ( 80.85, 54.19) circle (  2.13);

\path[fill=fillColor,fill opacity=0.20] ( 78.23, 64.76) circle (  2.13);

\path[fill=fillColor,fill opacity=0.20] (109.69, 63.95) circle (  2.13);

\path[fill=fillColor,fill opacity=0.20] ( 72.77, 58.26) circle (  2.13);

\path[fill=fillColor,fill opacity=0.20] ( 82.82, 49.32) circle (  2.13);

\path[fill=fillColor,fill opacity=0.20] ( 93.96, 41.19) circle (  2.13);

\path[fill=fillColor,fill opacity=0.20] ( 92.43, 47.69) circle (  2.13);

\path[fill=fillColor,fill opacity=0.20] ( 93.52, 55.82) circle (  2.13);

\path[fill=fillColor,fill opacity=0.20] ( 97.02, 68.01) circle (  2.13);

\path[fill=fillColor,fill opacity=0.20] ( 95.49, 76.95) circle (  2.13);

\path[fill=fillColor,fill opacity=0.20] ( 89.81, 70.45) circle (  2.13);

\path[fill=fillColor,fill opacity=0.20] ( 91.34, 80.20) circle (  2.13);

\path[fill=fillColor,fill opacity=0.20] ( 94.62, 86.70) circle (  2.13);

\path[fill=fillColor,fill opacity=0.20] ( 93.31, 83.45) circle (  2.13);

\path[fill=fillColor,fill opacity=0.20] ( 95.05, 76.14) circle (  2.13);

\path[fill=fillColor,fill opacity=0.20] ( 86.10, 70.45) circle (  2.13);

\path[fill=fillColor,fill opacity=0.20] ( 93.52, 81.01) circle (  2.13);

\path[fill=fillColor,fill opacity=0.20] ( 93.31, 87.51) circle (  2.13);

\path[fill=fillColor,fill opacity=0.20] ( 90.03, 76.95) circle (  2.13);

\path[fill=fillColor,fill opacity=0.20] ( 85.00, 77.76) circle (  2.13);

\path[fill=fillColor,fill opacity=0.20] ( 92.87, 75.32) circle (  2.13);

\path[fill=fillColor,fill opacity=0.20] ( 93.09, 69.63) circle (  2.13);

\path[fill=fillColor,fill opacity=0.20] ( 99.64, 76.95) circle (  2.13);

\path[fill=fillColor,fill opacity=0.20] ( 77.36, 54.19) circle (  2.13);

\path[fill=fillColor,fill opacity=0.20] ( 71.67, 46.88) circle (  2.13);

\path[fill=fillColor,fill opacity=0.20] ( 74.08, 43.63) circle (  2.13);

\path[fill=fillColor,fill opacity=0.20] ( 73.42, 54.19) circle (  2.13);

\path[fill=fillColor,fill opacity=0.20] ( 70.58, 60.69) circle (  2.13);

\path[fill=fillColor,fill opacity=0.20] ( 69.93, 68.01) circle (  2.13);

\path[fill=fillColor,fill opacity=0.20] ( 75.39, 55.82) circle (  2.13);

\path[fill=fillColor,fill opacity=0.20] ( 69.49, 45.25) circle (  2.13);

\path[fill=fillColor,fill opacity=0.20] ( 68.62, 68.82) circle (  2.13);

\path[fill=fillColor,fill opacity=0.20] ( 68.83, 76.14) circle (  2.13);

\path[fill=fillColor,fill opacity=0.20] ( 63.81, 65.57) circle (  2.13);

\path[fill=fillColor,fill opacity=0.20] ( 64.25, 70.45) circle (  2.13);

\path[fill=fillColor,fill opacity=0.20] ( 64.03, 75.32) circle (  2.13);

\path[fill=fillColor,fill opacity=0.20] ( 71.24, 78.57) circle (  2.13);

\path[fill=fillColor,fill opacity=0.20] ( 64.25, 81.01) circle (  2.13);

\path[fill=fillColor,fill opacity=0.20] ( 68.18, 72.89) circle (  2.13);

\path[fill=fillColor,fill opacity=0.20] ( 75.61, 65.57) circle (  2.13);

\path[fill=fillColor,fill opacity=0.20] (115.81, 65.57) circle (  2.13);

\path[fill=fillColor,fill opacity=0.20] ( 78.01, 59.07) circle (  2.13);

\path[fill=fillColor,fill opacity=0.20] ( 71.67, 49.32) circle (  2.13);

\path[fill=fillColor,fill opacity=0.20] (101.83, 53.38) circle (  2.13);

\path[fill=fillColor,fill opacity=0.20] ( 82.16, 64.76) circle (  2.13);

\path[fill=fillColor,fill opacity=0.20] ( 75.61, 62.32) circle (  2.13);

\path[fill=fillColor,fill opacity=0.20] ( 82.16, 41.19) circle (  2.13);

\path[fill=fillColor,fill opacity=0.20] ( 81.73, 54.19) circle (  2.13);

\path[fill=fillColor,fill opacity=0.20] ( 80.63, 49.32) circle (  2.13);

\path[fill=fillColor,fill opacity=0.20] ( 87.19, 41.19) circle (  2.13);

\path[fill=fillColor,fill opacity=0.20] ( 85.22, 40.38) circle (  2.13);

\path[fill=fillColor,fill opacity=0.20] ( 92.87, 47.69) circle (  2.13);

\path[fill=fillColor,fill opacity=0.20] (104.45, 50.94) circle (  2.13);

\path[fill=fillColor,fill opacity=0.20] ( 94.62, 61.51) circle (  2.13);

\path[fill=fillColor,fill opacity=0.20] ( 84.57, 59.07) circle (  2.13);

\path[fill=fillColor,fill opacity=0.20] ( 89.15, 61.51) circle (  2.13);

\path[fill=fillColor,fill opacity=0.20] ( 93.52, 66.38) circle (  2.13);

\path[fill=fillColor,fill opacity=0.20] ( 90.03, 67.20) circle (  2.13);

\path[fill=fillColor,fill opacity=0.20] ( 92.65, 68.01) circle (  2.13);

\path[fill=fillColor,fill opacity=0.20] ( 94.40, 74.51) circle (  2.13);

\path[fill=fillColor,fill opacity=0.20] ( 92.21, 73.70) circle (  2.13);

\path[fill=fillColor,fill opacity=0.20] ( 87.41, 76.95) circle (  2.13);

\path[fill=fillColor,fill opacity=0.20] ( 89.37, 79.39) circle (  2.13);

\path[fill=fillColor,fill opacity=0.20] ( 85.22, 76.14) circle (  2.13);

\path[fill=fillColor,fill opacity=0.20] ( 78.88, 77.76) circle (  2.13);

\path[fill=fillColor,fill opacity=0.20] ( 90.68, 72.07) circle (  2.13);

\path[fill=fillColor,fill opacity=0.20] ( 94.40, 64.76) circle (  2.13);

\path[fill=fillColor,fill opacity=0.20] ( 97.24, 75.32) circle (  2.13);

\path[fill=fillColor,fill opacity=0.20] ( 71.46, 39.56) circle (  2.13);

\path[fill=fillColor,fill opacity=0.20] ( 81.73, 39.56) circle (  2.13);

\path[fill=fillColor,fill opacity=0.20] ( 77.79, 41.19) circle (  2.13);

\path[fill=fillColor,fill opacity=0.20] ( 80.85, 41.19) circle (  2.13);

\path[fill=fillColor,fill opacity=0.20] ( 74.73, 39.56) circle (  2.13);

\path[fill=fillColor,fill opacity=0.20] ( 75.39, 47.69) circle (  2.13);

\path[fill=fillColor,fill opacity=0.20] ( 76.04, 52.57) circle (  2.13);

\path[fill=fillColor,fill opacity=0.20] ( 84.78, 52.57) circle (  2.13);

\path[fill=fillColor,fill opacity=0.20] ( 78.45, 54.19) circle (  2.13);

\path[fill=fillColor,fill opacity=0.20] ( 81.73, 64.76) circle (  2.13);

\path[fill=fillColor,fill opacity=0.20] ( 76.04, 76.95) circle (  2.13);

\path[fill=fillColor,fill opacity=0.20] ( 74.30, 69.63) circle (  2.13);

\path[fill=fillColor,fill opacity=0.20] ( 77.79, 63.13) circle (  2.13);

\path[fill=fillColor,fill opacity=0.20] ( 79.76, 59.88) circle (  2.13);

\path[fill=fillColor,fill opacity=0.20] ( 81.29, 57.44) circle (  2.13);

\path[fill=fillColor,fill opacity=0.20] ( 83.91, 56.63) circle (  2.13);

\path[fill=fillColor,fill opacity=0.20] (123.02, 51.75) circle (  2.13);

\path[fill=fillColor,fill opacity=0.20] ( 91.78, 40.38) circle (  2.13);

\path[fill=fillColor,fill opacity=0.20] ( 88.06, 41.19) circle (  2.13);

\path[fill=fillColor,fill opacity=0.20] ( 86.10, 53.38) circle (  2.13);

\path[fill=fillColor,fill opacity=0.20] ( 94.18, 50.13) circle (  2.13);

\path[fill=fillColor,fill opacity=0.20] ( 98.99, 50.94) circle (  2.13);

\path[fill=fillColor,fill opacity=0.20] ( 93.96, 47.69) circle (  2.13);

\path[fill=fillColor,fill opacity=0.20] ( 90.03, 63.13) circle (  2.13);

\path[fill=fillColor,fill opacity=0.20] ( 96.80, 74.51) circle (  2.13);

\path[fill=fillColor,fill opacity=0.20] ( 94.18, 66.38) circle (  2.13);

\path[fill=fillColor,fill opacity=0.20] ( 87.84, 62.32) circle (  2.13);

\path[fill=fillColor,fill opacity=0.20] ( 91.99, 59.88) circle (  2.13);

\path[fill=fillColor,fill opacity=0.20] ( 86.75, 59.88) circle (  2.13);

\path[fill=fillColor,fill opacity=0.20] ( 86.75, 68.01) circle (  2.13);

\path[fill=fillColor,fill opacity=0.20] ( 93.52, 61.51) circle (  2.13);

\path[fill=fillColor,fill opacity=0.20] (102.48, 59.88) circle (  2.13);

\path[fill=fillColor,fill opacity=0.20] ( 84.35, 39.56) circle (  2.13);

\path[fill=fillColor,fill opacity=0.20] ( 73.42, 43.63) circle (  2.13);

\path[fill=fillColor,fill opacity=0.20] ( 75.61, 41.19) circle (  2.13);

\path[fill=fillColor,fill opacity=0.20] ( 79.98, 38.75) circle (  2.13);

\path[fill=fillColor,fill opacity=0.20] ( 75.39, 57.44) circle (  2.13);

\path[fill=fillColor,fill opacity=0.20] ( 76.70, 67.20) circle (  2.13);

\path[fill=fillColor,fill opacity=0.20] ( 83.91, 58.26) circle (  2.13);

\path[fill=fillColor,fill opacity=0.20] ( 82.82, 54.19) circle (  2.13);

\path[fill=fillColor,fill opacity=0.20] ( 82.82, 56.63) circle (  2.13);

\path[fill=fillColor,fill opacity=0.20] (100.30, 55.01) circle (  2.13);

\path[fill=fillColor,fill opacity=0.20] ( 81.29, 54.19) circle (  2.13);

\path[fill=fillColor,fill opacity=0.20] ( 84.35, 52.57) circle (  2.13);

\path[fill=fillColor,fill opacity=0.20] ( 85.00, 50.13) circle (  2.13);

\path[fill=fillColor,fill opacity=0.20] ( 92.65, 46.88) circle (  2.13);

\path[fill=fillColor,fill opacity=0.20] (100.52, 40.38) circle (  2.13);

\path[fill=fillColor,fill opacity=0.20] ( 84.57, 50.13) circle (  2.13);

\path[fill=fillColor,fill opacity=0.20] ( 94.18, 67.20) circle (  2.13);

\path[fill=fillColor,fill opacity=0.20] (105.54, 72.07) circle (  2.13);

\path[fill=fillColor,fill opacity=0.20] ( 88.50, 46.88) circle (  2.13);

\path[fill=fillColor,fill opacity=0.20] (108.16, 52.57) circle (  2.13);

\path[fill=fillColor,fill opacity=0.20] ( 86.75, 45.25) circle (  2.13);

\path[fill=fillColor,fill opacity=0.20] ( 82.82, 59.88) circle (  2.13);

\path[fill=fillColor,fill opacity=0.20] ( 97.02, 50.94) circle (  2.13);

\path[fill=fillColor,fill opacity=0.20] ( 87.41, 41.19) circle (  2.13);

\path[fill=fillColor,fill opacity=0.20] ( 90.68, 49.32) circle (  2.13);

\path[fill=fillColor,fill opacity=0.20] ( 94.18, 53.38) circle (  2.13);

\path[fill=fillColor,fill opacity=0.20] ( 89.59, 49.32) circle (  2.13);

\path[fill=fillColor,fill opacity=0.20] ( 88.72, 45.25) circle (  2.13);

\path[fill=fillColor,fill opacity=0.20] ( 97.46, 48.50) circle (  2.13);

\path[fill=fillColor,fill opacity=0.20] ( 95.93, 57.44) circle (  2.13);

\path[fill=fillColor,fill opacity=0.20] ( 88.94, 50.94) circle (  2.13);

\path[fill=fillColor,fill opacity=0.20] (100.08,104.58) circle (  2.13);

\path[fill=fillColor,fill opacity=0.20] (101.83, 93.20) circle (  2.13);

\path[fill=fillColor,fill opacity=0.20] (100.95, 89.95) circle (  2.13);

\path[fill=fillColor,fill opacity=0.20] ( 90.47, 81.01) circle (  2.13);

\path[fill=fillColor,fill opacity=0.20] ( 93.52,103.77) circle (  2.13);

\path[fill=fillColor,fill opacity=0.20] (108.60, 79.39) circle (  2.13);

\path[fill=fillColor,fill opacity=0.20] (130.23, 75.32) circle (  2.13);

\path[fill=fillColor,fill opacity=0.20] (102.92, 75.32) circle (  2.13);

\path[fill=fillColor,fill opacity=0.20] ( 93.52, 81.82) circle (  2.13);

\path[fill=fillColor,fill opacity=0.20] ( 89.15, 87.51) circle (  2.13);

\path[fill=fillColor,fill opacity=0.20] ( 79.10, 85.89) circle (  2.13);

\path[fill=fillColor,fill opacity=0.20] ( 75.83, 92.39) circle (  2.13);

\path[fill=fillColor,fill opacity=0.20] ( 82.82, 89.14) circle (  2.13);

\path[fill=fillColor,fill opacity=0.20] ( 78.67, 73.70) circle (  2.13);

\path[fill=fillColor,fill opacity=0.20] ( 76.70, 67.20) circle (  2.13);

\path[fill=fillColor,fill opacity=0.20] ( 85.44, 71.26) circle (  2.13);

\path[fill=fillColor,fill opacity=0.20] (100.95, 76.95) circle (  2.13);

\path[fill=fillColor,fill opacity=0.20] (110.13, 86.70) circle (  2.13);

\path[fill=fillColor,fill opacity=0.20] (136.57, 89.95) circle (  2.13);

\path[fill=fillColor,fill opacity=0.20] (110.57, 79.39) circle (  2.13);

\path[fill=fillColor,fill opacity=0.20] ( 97.02, 84.26) circle (  2.13);

\path[fill=fillColor,fill opacity=0.20] (119.53, 94.83) circle (  2.13);

\path[fill=fillColor,fill opacity=0.20] ( 85.88, 78.57) circle (  2.13);

\path[fill=fillColor,fill opacity=0.20] ( 72.77, 70.45) circle (  2.13);

\path[fill=fillColor,fill opacity=0.20] ( 86.75, 95.64) circle (  2.13);

\path[fill=fillColor,fill opacity=0.20] ( 92.21, 77.76) circle (  2.13);

\path[fill=fillColor,fill opacity=0.20] ( 84.57, 71.26) circle (  2.13);

\path[fill=fillColor,fill opacity=0.20] ( 85.66, 74.51) circle (  2.13);

\path[fill=fillColor,fill opacity=0.20] ( 87.19, 75.32) circle (  2.13);

\path[fill=fillColor,fill opacity=0.20] ( 90.25, 64.76) circle (  2.13);

\path[fill=fillColor,fill opacity=0.20] ( 95.27, 51.75) circle (  2.13);

\path[fill=fillColor,fill opacity=0.20] ( 94.84, 48.50) circle (  2.13);

\path[fill=fillColor,fill opacity=0.20] ( 81.29, 51.75) circle (  2.13);

\path[fill=fillColor,fill opacity=0.20] ( 67.52, 51.75) circle (  2.13);

\path[fill=fillColor,fill opacity=0.20] (105.10, 89.95) circle (  2.13);

\path[fill=fillColor,fill opacity=0.20] (107.73, 72.07) circle (  2.13);

\path[fill=fillColor,fill opacity=0.20] (122.58,101.33) circle (  2.13);

\path[fill=fillColor,fill opacity=0.20] (123.90, 99.70) circle (  2.13);

\path[fill=fillColor,fill opacity=0.20] (136.35, 90.76) circle (  2.13);

\path[fill=fillColor,fill opacity=0.20] (100.08, 94.83) circle (  2.13);

\path[fill=fillColor,fill opacity=0.20] ( 98.33, 94.83) circle (  2.13);

\path[fill=fillColor,fill opacity=0.20] ( 98.55, 81.82) circle (  2.13);

\path[fill=fillColor,fill opacity=0.20] ( 98.11, 69.63) circle (  2.13);

\path[fill=fillColor,fill opacity=0.20] ( 77.36, 63.13) circle (  2.13);

\path[fill=fillColor,fill opacity=0.20] ( 88.50, 85.08) circle (  2.13);

\path[fill=fillColor,fill opacity=0.20] ( 92.87, 58.26) circle (  2.13);

\path[fill=fillColor,fill opacity=0.20] (100.95, 55.01) circle (  2.13);

\path[fill=fillColor,fill opacity=0.20] ( 96.36, 67.20) circle (  2.13);

\path[fill=fillColor,fill opacity=0.20] ( 96.36, 77.76) circle (  2.13);

\path[fill=fillColor,fill opacity=0.20] ( 87.84, 85.89) circle (  2.13);

\path[fill=fillColor,fill opacity=0.20] ( 90.90, 79.39) circle (  2.13);

\path[fill=fillColor,fill opacity=0.20] ( 91.34, 61.51) circle (  2.13);

\path[fill=fillColor,fill opacity=0.20] (126.52, 47.69) circle (  2.13);

\path[fill=fillColor,fill opacity=0.20] ( 74.51, 39.56) circle (  2.13);

\path[fill=fillColor,fill opacity=0.20] (106.85, 88.33) circle (  2.13);

\path[fill=fillColor,fill opacity=0.20] (113.84, 76.95) circle (  2.13);

\path[fill=fillColor,fill opacity=0.20] (115.37,105.39) circle (  2.13);

\path[fill=fillColor,fill opacity=0.20] (151.43,101.33) circle (  2.13);

\path[fill=fillColor,fill opacity=0.20] (123.02,107.02) circle (  2.13);

\path[fill=fillColor,fill opacity=0.20] (101.17, 87.51) circle (  2.13);

\path[fill=fillColor,fill opacity=0.20] ( 91.56, 71.26) circle (  2.13);

\path[fill=fillColor,fill opacity=0.20] ( 89.37, 79.39) circle (  2.13);

\path[fill=fillColor,fill opacity=0.20] ( 85.44, 76.14) circle (  2.13);

\path[fill=fillColor,fill opacity=0.20] ( 76.48, 54.19) circle (  2.13);

\path[fill=fillColor,fill opacity=0.20] ( 94.84, 83.45) circle (  2.13);

\path[fill=fillColor,fill opacity=0.20] (102.05, 62.32) circle (  2.13);

\path[fill=fillColor,fill opacity=0.20] (106.85, 73.70) circle (  2.13);

\path[fill=fillColor,fill opacity=0.20] (113.41, 92.39) circle (  2.13);

\path[fill=fillColor,fill opacity=0.20] (123.24,101.33) circle (  2.13);

\path[fill=fillColor,fill opacity=0.20] (114.50,100.52) circle (  2.13);

\path[fill=fillColor,fill opacity=0.20] (106.63, 91.58) circle (  2.13);

\path[fill=fillColor,fill opacity=0.20] ( 97.46, 79.39) circle (  2.13);

\path[fill=fillColor,fill opacity=0.20] ( 81.51, 72.89) circle (  2.13);

\path[fill=fillColor,fill opacity=0.20] ( 73.64, 70.45) circle (  2.13);

\path[fill=fillColor,fill opacity=0.20] ( 76.48, 51.75) circle (  2.13);

\path[fill=fillColor,fill opacity=0.20] ( 52.01, 37.94) circle (  2.13);

\path[fill=fillColor,fill opacity=0.20] (102.70, 94.02) circle (  2.13);

\path[fill=fillColor,fill opacity=0.20] (112.53, 75.32) circle (  2.13);

\path[fill=fillColor,fill opacity=0.20] (124.55, 97.27) circle (  2.13);

\path[fill=fillColor,fill opacity=0.20] (124.11, 96.45) circle (  2.13);

\path[fill=fillColor,fill opacity=0.20] (145.53,100.52) circle (  2.13);

\path[fill=fillColor,fill opacity=0.20] (117.78, 82.64) circle (  2.13);

\path[fill=fillColor,fill opacity=0.20] ( 94.18, 72.89) circle (  2.13);

\path[fill=fillColor,fill opacity=0.20] ( 88.50, 83.45) circle (  2.13);

\path[fill=fillColor,fill opacity=0.20] ( 92.21, 76.14) circle (  2.13);

\path[fill=fillColor,fill opacity=0.20] ( 80.20, 55.01) circle (  2.13);

\path[fill=fillColor,fill opacity=0.20] (103.14, 71.26) circle (  2.13);

\path[fill=fillColor,fill opacity=0.20] (107.73, 73.70) circle (  2.13);

\path[fill=fillColor,fill opacity=0.20] (125.43, 94.83) circle (  2.13);

\path[fill=fillColor,fill opacity=0.20] (131.11,107.83) circle (  2.13);

\path[fill=fillColor,fill opacity=0.20] (148.80,110.27) circle (  2.13);

\path[fill=fillColor,fill opacity=0.20] (109.47, 87.51) circle (  2.13);

\path[fill=fillColor,fill opacity=0.20] ( 95.93, 68.01) circle (  2.13);

\path[fill=fillColor,fill opacity=0.20] ( 83.69, 63.95) circle (  2.13);

\path[fill=fillColor,fill opacity=0.20] ( 66.21, 66.38) circle (  2.13);

\path[fill=fillColor,fill opacity=0.20] ( 83.69, 83.45) circle (  2.13);

\path[fill=fillColor,fill opacity=0.20] ( 96.15, 64.76) circle (  2.13);

\path[fill=fillColor,fill opacity=0.20] (110.35, 89.14) circle (  2.13);

\path[fill=fillColor,fill opacity=0.20] (124.11, 89.95) circle (  2.13);

\path[fill=fillColor,fill opacity=0.20] (114.28, 83.45) circle (  2.13);

\path[fill=fillColor,fill opacity=0.20] (111.22, 86.70) circle (  2.13);

\path[fill=fillColor,fill opacity=0.20] (112.32, 85.08) circle (  2.13);

\path[fill=fillColor,fill opacity=0.20] ( 99.64, 83.45) circle (  2.13);

\path[fill=fillColor,fill opacity=0.20] ( 89.59, 81.01) circle (  2.13);

\path[fill=fillColor,fill opacity=0.20] (114.28, 66.38) circle (  2.13);

\path[fill=fillColor,fill opacity=0.20] ( 87.62, 92.39) circle (  2.13);

\path[fill=fillColor,fill opacity=0.20] ( 99.42, 73.70) circle (  2.13);

\path[fill=fillColor,fill opacity=0.20] (113.19, 88.33) circle (  2.13);

\path[fill=fillColor,fill opacity=0.20] (128.27, 98.08) circle (  2.13);

\path[fill=fillColor,fill opacity=0.20] (132.85, 98.89) circle (  2.13);

\path[fill=fillColor,fill opacity=0.20] ( 95.49, 91.58) circle (  2.13);

\path[fill=fillColor,fill opacity=0.20] ( 83.91, 76.14) circle (  2.13);

\path[fill=fillColor,fill opacity=0.20] ( 77.79, 53.38) circle (  2.13);

\path[fill=fillColor,fill opacity=0.20] ( 63.15, 46.07) circle (  2.13);

\path[fill=fillColor,fill opacity=0.20] ( 91.78, 76.95) circle (  2.13);

\path[fill=fillColor,fill opacity=0.20] ( 90.03, 73.70) circle (  2.13);

\path[fill=fillColor,fill opacity=0.20] ( 94.62, 89.95) circle (  2.13);

\path[fill=fillColor,fill opacity=0.20] (121.49, 90.76) circle (  2.13);

\path[fill=fillColor,fill opacity=0.20] (106.20, 85.08) circle (  2.13);

\path[fill=fillColor,fill opacity=0.20] (107.29, 86.70) circle (  2.13);

\path[fill=fillColor,fill opacity=0.20] (111.22, 90.76) circle (  2.13);

\path[fill=fillColor,fill opacity=0.20] (110.13, 85.08) circle (  2.13);

\path[fill=fillColor,fill opacity=0.20] ( 95.27, 74.51) circle (  2.13);

\path[fill=fillColor,fill opacity=0.20] ( 91.34, 63.13) circle (  2.13);

\path[fill=fillColor,fill opacity=0.20] ( 98.77, 83.45) circle (  2.13);

\path[fill=fillColor,fill opacity=0.20] (109.26, 69.63) circle (  2.13);

\path[fill=fillColor,fill opacity=0.20] (112.97, 94.83) circle (  2.13);

\path[fill=fillColor,fill opacity=0.20] (120.84,104.58) circle (  2.13);

\path[fill=fillColor,fill opacity=0.20] (132.42, 98.89) circle (  2.13);

\path[fill=fillColor,fill opacity=0.20] (105.76,101.33) circle (  2.13);

\path[fill=fillColor,fill opacity=0.20] (105.32, 98.89) circle (  2.13);

\path[fill=fillColor,fill opacity=0.20] (101.61, 91.58) circle (  2.13);

\path[fill=fillColor,fill opacity=0.20] ( 78.01, 76.14) circle (  2.13);

\path[fill=fillColor,fill opacity=0.20] ( 68.18, 56.63) circle (  2.13);

\path[fill=fillColor,fill opacity=0.20] ( 76.26, 46.88) circle (  2.13);

\path[fill=fillColor,fill opacity=0.20] ( 90.25, 76.14) circle (  2.13);

\path[fill=fillColor,fill opacity=0.20] ( 99.21, 78.57) circle (  2.13);

\path[fill=fillColor,fill opacity=0.20] ( 92.65, 86.70) circle (  2.13);

\path[fill=fillColor,fill opacity=0.20] ( 88.06, 88.33) circle (  2.13);

\path[fill=fillColor,fill opacity=0.20] ( 94.84, 89.95) circle (  2.13);

\path[fill=fillColor,fill opacity=0.20] ( 99.64, 91.58) circle (  2.13);

\path[fill=fillColor,fill opacity=0.20] (117.56, 83.45) circle (  2.13);

\path[fill=fillColor,fill opacity=0.20] (109.47, 76.14) circle (  2.13);

\path[fill=fillColor,fill opacity=0.20] ( 98.77, 71.26) circle (  2.13);

\path[fill=fillColor,fill opacity=0.20] ( 82.38, 63.95) circle (  2.13);

\path[fill=fillColor,fill opacity=0.20] ( 65.77, 51.75) circle (  2.13);

\path[fill=fillColor,fill opacity=0.20] ( 59.88, 42.00) circle (  2.13);

\path[fill=fillColor,fill opacity=0.20] ( 97.02, 81.82) circle (  2.13);

\path[fill=fillColor,fill opacity=0.20] (114.06, 72.07) circle (  2.13);

\path[fill=fillColor,fill opacity=0.20] (118.65, 98.08) circle (  2.13);

\path[fill=fillColor,fill opacity=0.20] (119.96,111.89) circle (  2.13);

\path[fill=fillColor,fill opacity=0.20] (114.94,115.15) circle (  2.13);

\path[fill=fillColor,fill opacity=0.20] (100.52,111.89) circle (  2.13);

\path[fill=fillColor,fill opacity=0.20] ( 87.19, 94.83) circle (  2.13);

\path[fill=fillColor,fill opacity=0.20] ( 86.97, 77.76) circle (  2.13);

\path[fill=fillColor,fill opacity=0.20] ( 80.85, 65.57) circle (  2.13);

\path[fill=fillColor,fill opacity=0.20] ( 82.38, 76.14) circle (  2.13);

\path[fill=fillColor,fill opacity=0.20] ( 88.50, 63.95) circle (  2.13);

\path[fill=fillColor,fill opacity=0.20] ( 89.37, 71.26) circle (  2.13);

\path[fill=fillColor,fill opacity=0.20] (102.26, 81.01) circle (  2.13);

\path[fill=fillColor,fill opacity=0.20] ( 96.58, 93.20) circle (  2.13);

\path[fill=fillColor,fill opacity=0.20] (102.05, 86.70) circle (  2.13);

\path[fill=fillColor,fill opacity=0.20] ( 93.96, 73.70) circle (  2.13);

\path[fill=fillColor,fill opacity=0.20] ( 91.99, 72.07) circle (  2.13);

\path[fill=fillColor,fill opacity=0.20] ( 84.78, 68.82) circle (  2.13);

\path[fill=fillColor,fill opacity=0.20] ( 73.86, 57.44) circle (  2.13);

\path[fill=fillColor,fill opacity=0.20] ( 94.84, 82.64) circle (  2.13);

\path[fill=fillColor,fill opacity=0.20] (104.89, 77.76) circle (  2.13);

\path[fill=fillColor,fill opacity=0.20] (119.96, 93.20) circle (  2.13);

\path[fill=fillColor,fill opacity=0.20] (121.49,102.14) circle (  2.13);

\path[fill=fillColor,fill opacity=0.20] (107.95,115.15) circle (  2.13);

\path[fill=fillColor,fill opacity=0.20] ( 96.15,112.71) circle (  2.13);

\path[fill=fillColor,fill opacity=0.20] ( 83.25, 93.20) circle (  2.13);

\path[fill=fillColor,fill opacity=0.20] ( 82.38, 78.57) circle (  2.13);

\path[fill=fillColor,fill opacity=0.20] ( 79.76, 66.38) circle (  2.13);

\path[fill=fillColor,fill opacity=0.20] ( 62.50, 57.44) circle (  2.13);

\path[fill=fillColor,fill opacity=0.20] ( 77.79, 95.64) circle (  2.13);

\path[fill=fillColor,fill opacity=0.20] ( 82.82, 72.07) circle (  2.13);

\path[fill=fillColor,fill opacity=0.20] ( 85.66, 62.32) circle (  2.13);

\path[fill=fillColor,fill opacity=0.20] ( 94.18, 75.32) circle (  2.13);

\path[fill=fillColor,fill opacity=0.20] ( 98.11, 82.64) circle (  2.13);

\path[fill=fillColor,fill opacity=0.20] ( 98.77, 79.39) circle (  2.13);

\path[fill=fillColor,fill opacity=0.20] ( 92.87, 83.45) circle (  2.13);

\path[fill=fillColor,fill opacity=0.20] ( 93.09, 84.26) circle (  2.13);

\path[fill=fillColor,fill opacity=0.20] ( 93.31, 78.57) circle (  2.13);

\path[fill=fillColor,fill opacity=0.20] ( 88.94, 75.32) circle (  2.13);

\path[fill=fillColor,fill opacity=0.20] ( 83.69, 65.57) circle (  2.13);

\path[fill=fillColor,fill opacity=0.20] (120.62, 75.32) circle (  2.13);

\path[fill=fillColor,fill opacity=0.20] (114.50, 85.08) circle (  2.13);

\path[fill=fillColor,fill opacity=0.20] (124.77, 86.70) circle (  2.13);

\path[fill=fillColor,fill opacity=0.20] (101.61, 93.20) circle (  2.13);

\path[fill=fillColor,fill opacity=0.20] ( 92.87,100.52) circle (  2.13);

\path[fill=fillColor,fill opacity=0.20] ( 83.47, 94.02) circle (  2.13);

\path[fill=fillColor,fill opacity=0.20] ( 77.14, 84.26) circle (  2.13);

\path[fill=fillColor,fill opacity=0.20] ( 78.67, 69.63) circle (  2.13);

\path[fill=fillColor,fill opacity=0.20] ( 72.11, 53.38) circle (  2.13);

\path[fill=fillColor,fill opacity=0.20] ( 84.13, 89.95) circle (  2.13);

\path[fill=fillColor,fill opacity=0.20] ( 83.25, 81.82) circle (  2.13);

\path[fill=fillColor,fill opacity=0.20] ( 92.21, 80.20) circle (  2.13);

\path[fill=fillColor,fill opacity=0.20] (108.16, 77.76) circle (  2.13);

\path[fill=fillColor,fill opacity=0.20] (101.61, 65.57) circle (  2.13);

\path[fill=fillColor,fill opacity=0.20] ( 94.40, 71.26) circle (  2.13);

\path[fill=fillColor,fill opacity=0.20] (101.83, 83.45) circle (  2.13);

\path[fill=fillColor,fill opacity=0.20] ( 95.93, 85.08) circle (  2.13);

\path[fill=fillColor,fill opacity=0.20] ( 95.27, 74.51) circle (  2.13);

\path[fill=fillColor,fill opacity=0.20] ( 88.06, 62.32) circle (  2.13);

\path[fill=fillColor,fill opacity=0.20] ( 68.40, 60.69) circle (  2.13);

\path[fill=fillColor,fill opacity=0.20] ( 99.42, 70.45) circle (  2.13);

\path[fill=fillColor,fill opacity=0.20] (120.84, 80.20) circle (  2.13);

\path[fill=fillColor,fill opacity=0.20] (138.97, 77.76) circle (  2.13);

\path[fill=fillColor,fill opacity=0.20] ( 99.21, 73.70) circle (  2.13);

\path[fill=fillColor,fill opacity=0.20] ( 99.86, 85.89) circle (  2.13);

\path[fill=fillColor,fill opacity=0.20] ( 88.94, 89.14) circle (  2.13);

\path[fill=fillColor,fill opacity=0.20] ( 75.17, 80.20) circle (  2.13);

\path[fill=fillColor,fill opacity=0.20] ( 74.95, 68.82) circle (  2.13);

\path[fill=fillColor,fill opacity=0.20] ( 73.42, 53.38) circle (  2.13);

\path[fill=fillColor,fill opacity=0.20] ( 57.03, 42.00) circle (  2.13);

\path[fill=fillColor,fill opacity=0.20] ( 94.18, 88.33) circle (  2.13);

\path[fill=fillColor,fill opacity=0.20] (105.32, 87.51) circle (  2.13);

\path[fill=fillColor,fill opacity=0.20] (108.60, 85.08) circle (  2.13);

\path[fill=fillColor,fill opacity=0.20] (118.00, 75.32) circle (  2.13);

\path[fill=fillColor,fill opacity=0.20] (101.17, 67.20) circle (  2.13);

\path[fill=fillColor,fill opacity=0.20] (109.26, 72.07) circle (  2.13);

\path[fill=fillColor,fill opacity=0.20] ( 96.80, 77.76) circle (  2.13);

\path[fill=fillColor,fill opacity=0.20] ( 94.18, 73.70) circle (  2.13);

\path[fill=fillColor,fill opacity=0.20] ( 64.25, 56.63) circle (  2.13);

\path[fill=fillColor,fill opacity=0.20] ( 57.47, 44.44) circle (  2.13);

\path[fill=fillColor,fill opacity=0.20] ( 83.69, 75.32) circle (  2.13);

\path[fill=fillColor,fill opacity=0.20] (103.14, 72.89) circle (  2.13);

\path[fill=fillColor,fill opacity=0.20] ( 99.86, 81.01) circle (  2.13);

\path[fill=fillColor,fill opacity=0.20] ( 99.21, 78.57) circle (  2.13);

\path[fill=fillColor,fill opacity=0.20] ( 95.49, 80.20) circle (  2.13);

\path[fill=fillColor,fill opacity=0.20] ( 90.25, 81.01) circle (  2.13);

\path[fill=fillColor,fill opacity=0.20] ( 86.75, 72.07) circle (  2.13);

\path[fill=fillColor,fill opacity=0.20] ( 86.75, 63.13) circle (  2.13);

\path[fill=fillColor,fill opacity=0.20] ( 74.08, 53.38) circle (  2.13);

\path[fill=fillColor,fill opacity=0.20] ( 65.12, 41.19) circle (  2.13);

\path[fill=fillColor,fill opacity=0.20] ( 88.94, 80.20) circle (  2.13);

\path[fill=fillColor,fill opacity=0.20] (102.26, 81.82) circle (  2.13);

\path[fill=fillColor,fill opacity=0.20] (111.44, 77.76) circle (  2.13);

\path[fill=fillColor,fill opacity=0.20] (108.60, 81.01) circle (  2.13);

\path[fill=fillColor,fill opacity=0.20] (115.59, 86.70) circle (  2.13);

\path[fill=fillColor,fill opacity=0.20] (113.41, 85.89) circle (  2.13);

\path[fill=fillColor,fill opacity=0.20] ( 98.77, 76.95) circle (  2.13);

\path[fill=fillColor,fill opacity=0.20] ( 89.15, 65.57) circle (  2.13);

\path[fill=fillColor,fill opacity=0.20] ( 88.28, 61.51) circle (  2.13);

\path[fill=fillColor,fill opacity=0.20] ( 99.86, 81.01) circle (  2.13);

\path[fill=fillColor,fill opacity=0.20] ( 95.93, 84.26) circle (  2.13);

\path[fill=fillColor,fill opacity=0.20] ( 88.28, 77.76) circle (  2.13);

\path[fill=fillColor,fill opacity=0.20] ( 88.94, 77.76) circle (  2.13);

\path[fill=fillColor,fill opacity=0.20] ( 87.62, 72.89) circle (  2.13);

\path[fill=fillColor,fill opacity=0.20] ( 78.45, 59.07) circle (  2.13);

\path[fill=fillColor,fill opacity=0.20] ( 71.02, 59.88) circle (  2.13);

\path[fill=fillColor,fill opacity=0.20] ( 68.62, 64.76) circle (  2.13);

\path[fill=fillColor,fill opacity=0.20] ( 74.30, 61.51) circle (  2.13);

\path[fill=fillColor,fill opacity=0.20] ( 85.22, 56.63) circle (  2.13);

\path[fill=fillColor,fill opacity=0.20] ( 86.31, 81.82) circle (  2.13);

\path[fill=fillColor,fill opacity=0.20] ( 90.47, 76.14) circle (  2.13);

\path[fill=fillColor,fill opacity=0.20] ( 97.46, 79.39) circle (  2.13);

\path[fill=fillColor,fill opacity=0.20] (105.32, 82.64) circle (  2.13);

\path[fill=fillColor,fill opacity=0.20] (107.73, 85.89) circle (  2.13);

\path[fill=fillColor,fill opacity=0.20] (104.67, 95.64) circle (  2.13);

\path[fill=fillColor,fill opacity=0.20] (121.27, 91.58) circle (  2.13);

\path[fill=fillColor,fill opacity=0.20] ( 88.06, 68.82) circle (  2.13);

\path[fill=fillColor,fill opacity=0.20] ( 77.79, 59.88) circle (  2.13);

\path[fill=fillColor,fill opacity=0.20] ( 83.25, 72.07) circle (  2.13);

\path[fill=fillColor,fill opacity=0.20] ( 74.30, 71.26) circle (  2.13);

\path[fill=fillColor,fill opacity=0.20] ( 64.03, 46.88) circle (  2.13);

\path[fill=fillColor,fill opacity=0.20] (106.85, 76.14) circle (  2.13);

\path[fill=fillColor,fill opacity=0.20] ( 97.02, 73.70) circle (  2.13);

\path[fill=fillColor,fill opacity=0.20] ( 81.29, 76.14) circle (  2.13);

\path[fill=fillColor,fill opacity=0.20] ( 79.76, 78.57) circle (  2.13);

\path[fill=fillColor,fill opacity=0.20] ( 80.63, 72.89) circle (  2.13);

\path[fill=fillColor,fill opacity=0.20] ( 76.04, 66.38) circle (  2.13);

\path[fill=fillColor,fill opacity=0.20] ( 85.22, 68.01) circle (  2.13);

\path[fill=fillColor,fill opacity=0.20] ( 70.36, 74.51) circle (  2.13);

\path[fill=fillColor,fill opacity=0.20] ( 99.21, 56.63) circle (  2.13);

\path[fill=fillColor,fill opacity=0.20] ( 81.07, 51.75) circle (  2.13);

\path[fill=fillColor,fill opacity=0.20] ( 85.88, 61.51) circle (  2.13);

\path[fill=fillColor,fill opacity=0.20] ( 90.47, 63.95) circle (  2.13);

\path[fill=fillColor,fill opacity=0.20] ( 91.34, 64.76) circle (  2.13);

\path[fill=fillColor,fill opacity=0.20] ( 92.65, 81.82) circle (  2.13);

\path[fill=fillColor,fill opacity=0.20] ( 91.78, 91.58) circle (  2.13);

\path[fill=fillColor,fill opacity=0.20] ( 98.33, 91.58) circle (  2.13);

\path[fill=fillColor,fill opacity=0.20] ( 92.65, 87.51) circle (  2.13);

\path[fill=fillColor,fill opacity=0.20] ( 83.91, 77.76) circle (  2.13);

\path[fill=fillColor,fill opacity=0.20] ( 83.25, 64.76) circle (  2.13);

\path[fill=fillColor,fill opacity=0.20] ( 74.08, 69.63) circle (  2.13);

\path[fill=fillColor,fill opacity=0.20] ( 90.47, 68.01) circle (  2.13);

\path[fill=fillColor,fill opacity=0.20] (102.92, 68.82) circle (  2.13);

\path[fill=fillColor,fill opacity=0.20] ( 92.21, 85.08) circle (  2.13);

\path[fill=fillColor,fill opacity=0.20] ( 78.23, 81.82) circle (  2.13);

\path[fill=fillColor,fill opacity=0.20] ( 77.14, 76.95) circle (  2.13);

\path[fill=fillColor,fill opacity=0.20] ( 77.36, 80.20) circle (  2.13);

\path[fill=fillColor,fill opacity=0.20] ( 78.01, 77.76) circle (  2.13);

\path[fill=fillColor,fill opacity=0.20] ( 71.89, 65.57) circle (  2.13);

\path[fill=fillColor,fill opacity=0.20] ( 76.04, 67.20) circle (  2.13);

\path[fill=fillColor,fill opacity=0.20] ( 60.75, 72.89) circle (  2.13);

\path[fill=fillColor,fill opacity=0.20] ( 84.13, 54.19) circle (  2.13);

\path[fill=fillColor,fill opacity=0.20] ( 89.37, 65.57) circle (  2.13);

\path[fill=fillColor,fill opacity=0.20] ( 90.68, 72.89) circle (  2.13);

\path[fill=fillColor,fill opacity=0.20] ( 87.41, 61.51) circle (  2.13);

\path[fill=fillColor,fill opacity=0.20] ( 91.12, 56.63) circle (  2.13);

\path[fill=fillColor,fill opacity=0.20] (104.45, 68.82) circle (  2.13);

\path[fill=fillColor,fill opacity=0.20] ( 92.43, 78.57) circle (  2.13);

\path[fill=fillColor,fill opacity=0.20] ( 91.78, 87.51) circle (  2.13);

\path[fill=fillColor,fill opacity=0.20] ( 88.50, 92.39) circle (  2.13);

\path[fill=fillColor,fill opacity=0.20] ( 86.97, 85.89) circle (  2.13);

\path[fill=fillColor,fill opacity=0.20] ( 75.39, 76.14) circle (  2.13);

\path[fill=fillColor,fill opacity=0.20] ( 83.47, 72.89) circle (  2.13);

\path[fill=fillColor,fill opacity=0.20] ( 82.82, 73.70) circle (  2.13);

\path[fill=fillColor,fill opacity=0.20] ( 84.57, 76.95) circle (  2.13);

\path[fill=fillColor,fill opacity=0.20] ( 89.37, 72.07) circle (  2.13);

\path[fill=fillColor,fill opacity=0.20] ( 99.64, 79.39) circle (  2.13);

\path[fill=fillColor,fill opacity=0.20] ( 82.82, 83.45) circle (  2.13);

\path[fill=fillColor,fill opacity=0.20] ( 80.20, 88.33) circle (  2.13);

\path[fill=fillColor,fill opacity=0.20] ( 89.37, 82.64) circle (  2.13);

\path[fill=fillColor,fill opacity=0.20] ( 80.41, 65.57) circle (  2.13);

\path[fill=fillColor,fill opacity=0.20] ( 74.30, 69.63) circle (  2.13);

\path[fill=fillColor,fill opacity=0.20] ( 73.20, 76.95) circle (  2.13);

\path[fill=fillColor,fill opacity=0.20] ( 75.39, 65.57) circle (  2.13);

\path[fill=fillColor,fill opacity=0.20] ( 67.52, 64.76) circle (  2.13);

\path[fill=fillColor,fill opacity=0.20] ( 60.09, 69.63) circle (  2.13);

\path[fill=fillColor,fill opacity=0.20] ( 85.66, 63.95) circle (  2.13);

\path[fill=fillColor,fill opacity=0.20] ( 89.81, 64.76) circle (  2.13);

\path[fill=fillColor,fill opacity=0.20] ( 91.12, 72.89) circle (  2.13);

\path[fill=fillColor,fill opacity=0.20] ( 92.43, 73.70) circle (  2.13);

\path[fill=fillColor,fill opacity=0.20] ( 89.15, 68.82) circle (  2.13);

\path[fill=fillColor,fill opacity=0.20] ( 94.18, 72.89) circle (  2.13);

\path[fill=fillColor,fill opacity=0.20] ( 96.80, 81.01) circle (  2.13);

\path[fill=fillColor,fill opacity=0.20] (102.26, 80.20) circle (  2.13);

\path[fill=fillColor,fill opacity=0.20] ( 86.10, 79.39) circle (  2.13);

\path[fill=fillColor,fill opacity=0.20] ( 91.12, 84.26) circle (  2.13);

\path[fill=fillColor,fill opacity=0.20] ( 85.22, 82.64) circle (  2.13);

\path[fill=fillColor,fill opacity=0.20] ( 81.94, 76.14) circle (  2.13);

\path[fill=fillColor,fill opacity=0.20] ( 81.94, 68.82) circle (  2.13);

\path[fill=fillColor,fill opacity=0.20] ( 75.83, 58.26) circle (  2.13);

\path[fill=fillColor,fill opacity=0.20] ( 79.98, 75.32) circle (  2.13);

\path[fill=fillColor,fill opacity=0.20] ( 91.78, 62.32) circle (  2.13);

\path[fill=fillColor,fill opacity=0.20] (103.79, 76.14) circle (  2.13);

\path[fill=fillColor,fill opacity=0.20] ( 91.34, 95.64) circle (  2.13);

\path[fill=fillColor,fill opacity=0.20] ( 93.09, 79.39) circle (  2.13);

\path[fill=fillColor,fill opacity=0.20] ( 87.19, 87.51) circle (  2.13);

\path[fill=fillColor,fill opacity=0.20] ( 82.16, 82.64) circle (  2.13);

\path[fill=fillColor,fill opacity=0.20] ( 79.10, 81.82) circle (  2.13);

\path[fill=fillColor,fill opacity=0.20] ( 78.23, 75.32) circle (  2.13);

\path[fill=fillColor,fill opacity=0.20] ( 75.39, 67.20) circle (  2.13);

\path[fill=fillColor,fill opacity=0.20] ( 71.02, 61.51) circle (  2.13);

\path[fill=fillColor,fill opacity=0.20] ( 72.11, 61.51) circle (  2.13);

\path[fill=fillColor,fill opacity=0.20] ( 86.97, 61.51) circle (  2.13);

\path[fill=fillColor,fill opacity=0.20] ( 90.47, 59.88) circle (  2.13);

\path[fill=fillColor,fill opacity=0.20] ( 94.18, 61.51) circle (  2.13);

\path[fill=fillColor,fill opacity=0.20] ( 93.74, 71.26) circle (  2.13);

\path[fill=fillColor,fill opacity=0.20] ( 94.62, 81.01) circle (  2.13);

\path[fill=fillColor,fill opacity=0.20] ( 97.24, 81.82) circle (  2.13);

\path[fill=fillColor,fill opacity=0.20] ( 97.68, 80.20) circle (  2.13);

\path[fill=fillColor,fill opacity=0.20] ( 93.31, 75.32) circle (  2.13);

\path[fill=fillColor,fill opacity=0.20] ( 91.56, 71.26) circle (  2.13);

\path[fill=fillColor,fill opacity=0.20] ( 82.38, 70.45) circle (  2.13);

\path[fill=fillColor,fill opacity=0.20] ( 77.79, 72.89) circle (  2.13);

\path[fill=fillColor,fill opacity=0.20] ( 79.10, 63.95) circle (  2.13);

\path[fill=fillColor,fill opacity=0.20] ( 75.17, 56.63) circle (  2.13);

\path[fill=fillColor,fill opacity=0.20] ( 70.36, 54.19) circle (  2.13);

\path[fill=fillColor,fill opacity=0.20] ( 90.47, 68.01) circle (  2.13);

\path[fill=fillColor,fill opacity=0.20] ( 97.46, 74.51) circle (  2.13);

\path[fill=fillColor,fill opacity=0.20] ( 82.38, 85.08) circle (  2.13);

\path[fill=fillColor,fill opacity=0.20] ( 90.03, 89.95) circle (  2.13);

\path[fill=fillColor,fill opacity=0.20] ( 89.59, 81.01) circle (  2.13);

\path[fill=fillColor,fill opacity=0.20] ( 90.03, 71.26) circle (  2.13);

\path[fill=fillColor,fill opacity=0.20] ( 86.31, 80.20) circle (  2.13);

\path[fill=fillColor,fill opacity=0.20] ( 91.99, 82.64) circle (  2.13);

\path[fill=fillColor,fill opacity=0.20] ( 79.76, 79.39) circle (  2.13);

\path[fill=fillColor,fill opacity=0.20] ( 76.92, 64.76) circle (  2.13);

\path[fill=fillColor,fill opacity=0.20] ( 73.64, 55.01) circle (  2.13);

\path[fill=fillColor,fill opacity=0.20] ( 77.57, 58.26) circle (  2.13);

\path[fill=fillColor,fill opacity=0.20] ( 82.60, 55.82) circle (  2.13);

\path[fill=fillColor,fill opacity=0.20] ( 83.69, 57.44) circle (  2.13);

\path[fill=fillColor,fill opacity=0.20] (104.23, 62.32) circle (  2.13);

\path[fill=fillColor,fill opacity=0.20] ( 99.86, 65.57) circle (  2.13);

\path[fill=fillColor,fill opacity=0.20] (100.73, 77.76) circle (  2.13);

\path[fill=fillColor,fill opacity=0.20] (101.61, 87.51) circle (  2.13);

\path[fill=fillColor,fill opacity=0.20] ( 97.24, 81.82) circle (  2.13);

\path[fill=fillColor,fill opacity=0.20] (102.05, 68.01) circle (  2.13);

\path[fill=fillColor,fill opacity=0.20] ( 86.75, 53.38) circle (  2.13);

\path[fill=fillColor,fill opacity=0.20] ( 83.69, 45.25) circle (  2.13);

\path[fill=fillColor,fill opacity=0.20] ( 71.89, 57.44) circle (  2.13);

\path[fill=fillColor,fill opacity=0.20] ( 69.93, 63.13) circle (  2.13);

\path[fill=fillColor,fill opacity=0.20] ( 66.21, 46.07) circle (  2.13);

\path[fill=fillColor,fill opacity=0.20] ( 55.72, 40.38) circle (  2.13);

\path[fill=fillColor,fill opacity=0.20] ( 64.03, 67.20) circle (  2.13);

\path[fill=fillColor,fill opacity=0.20] ( 81.29, 77.76) circle (  2.13);

\path[fill=fillColor,fill opacity=0.20] ( 95.49, 89.14) circle (  2.13);

\path[fill=fillColor,fill opacity=0.20] ( 95.49, 85.08) circle (  2.13);

\path[fill=fillColor,fill opacity=0.20] ( 94.40, 81.82) circle (  2.13);

\path[fill=fillColor,fill opacity=0.20] ( 87.84, 92.39) circle (  2.13);

\path[fill=fillColor,fill opacity=0.20] ( 92.21, 87.51) circle (  2.13);

\path[fill=fillColor,fill opacity=0.20] ( 85.66, 77.76) circle (  2.13);

\path[fill=fillColor,fill opacity=0.20] ( 82.60, 65.57) circle (  2.13);

\path[fill=fillColor,fill opacity=0.20] ( 76.04, 62.32) circle (  2.13);

\path[fill=fillColor,fill opacity=0.20] ( 74.51, 69.63) circle (  2.13);

\path[fill=fillColor,fill opacity=0.20] ( 71.02, 69.63) circle (  2.13);

\path[fill=fillColor,fill opacity=0.20] ( 69.27, 56.63) circle (  2.13);

\path[fill=fillColor,fill opacity=0.20] ( 69.27, 68.82) circle (  2.13);

\path[fill=fillColor,fill opacity=0.20] ( 83.25, 58.26) circle (  2.13);

\path[fill=fillColor,fill opacity=0.20] ( 85.00, 64.76) circle (  2.13);

\path[fill=fillColor,fill opacity=0.20] ( 89.59, 76.14) circle (  2.13);

\path[fill=fillColor,fill opacity=0.20] ( 92.87, 79.39) circle (  2.13);

\path[fill=fillColor,fill opacity=0.20] (100.30, 78.57) circle (  2.13);

\path[fill=fillColor,fill opacity=0.20] ( 97.02, 72.89) circle (  2.13);

\path[fill=fillColor,fill opacity=0.20] ( 97.24, 74.51) circle (  2.13);

\path[fill=fillColor,fill opacity=0.20] ( 90.68, 76.95) circle (  2.13);

\path[fill=fillColor,fill opacity=0.20] ( 83.47, 62.32) circle (  2.13);

\path[fill=fillColor,fill opacity=0.20] ( 69.05, 48.50) circle (  2.13);

\path[fill=fillColor,fill opacity=0.20] ( 68.40, 47.69) circle (  2.13);

\path[fill=fillColor,fill opacity=0.20] ( 65.77, 50.94) circle (  2.13);

\path[fill=fillColor,fill opacity=0.20] ( 67.09, 55.01) circle (  2.13);

\path[fill=fillColor,fill opacity=0.20] ( 53.76, 37.94) circle (  2.13);

\path[fill=fillColor,fill opacity=0.20] ( 73.20, 78.57) circle (  2.13);

\path[fill=fillColor,fill opacity=0.20] (106.20, 79.39) circle (  2.13);

\path[fill=fillColor,fill opacity=0.20] ( 99.42, 92.39) circle (  2.13);

\path[fill=fillColor,fill opacity=0.20] ( 87.62, 97.27) circle (  2.13);

\path[fill=fillColor,fill opacity=0.20] ( 95.49, 81.01) circle (  2.13);

\path[fill=fillColor,fill opacity=0.20] ( 99.64, 63.13) circle (  2.13);

\path[fill=fillColor,fill opacity=0.20] ( 90.47, 59.88) circle (  2.13);

\path[fill=fillColor,fill opacity=0.20] ( 81.73, 61.51) circle (  2.13);

\path[fill=fillColor,fill opacity=0.20] ( 87.84, 63.95) circle (  2.13);

\path[fill=fillColor,fill opacity=0.20] ( 76.48, 63.13) circle (  2.13);

\path[fill=fillColor,fill opacity=0.20] ( 75.39, 62.32) circle (  2.13);

\path[fill=fillColor,fill opacity=0.20] ( 72.77, 59.88) circle (  2.13);

\path[fill=fillColor,fill opacity=0.20] ( 70.80, 55.01) circle (  2.13);

\path[fill=fillColor,fill opacity=0.20] ( 78.23, 79.39) circle (  2.13);

\path[fill=fillColor,fill opacity=0.20] ( 88.72, 70.45) circle (  2.13);

\path[fill=fillColor,fill opacity=0.20] (110.35, 65.57) circle (  2.13);

\path[fill=fillColor,fill opacity=0.20] (114.72, 73.70) circle (  2.13);

\path[fill=fillColor,fill opacity=0.20] (111.88, 75.32) circle (  2.13);

\path[fill=fillColor,fill opacity=0.20] ( 89.15, 79.39) circle (  2.13);

\path[fill=fillColor,fill opacity=0.20] ( 81.94, 81.82) circle (  2.13);

\path[fill=fillColor,fill opacity=0.20] ( 78.67, 66.38) circle (  2.13);

\path[fill=fillColor,fill opacity=0.20] ( 68.18, 59.07) circle (  2.13);

\path[fill=fillColor,fill opacity=0.20] ( 68.62, 59.88) circle (  2.13);

\path[fill=fillColor,fill opacity=0.20] ( 61.62, 50.94) circle (  2.13);

\path[fill=fillColor,fill opacity=0.20] ( 59.22, 50.13) circle (  2.13);

\path[fill=fillColor,fill opacity=0.20] ( 53.10, 62.32) circle (  2.13);

\path[fill=fillColor,fill opacity=0.20] ( 55.29, 63.95) circle (  2.13);

\path[fill=fillColor,fill opacity=0.20] (142.25, 79.39) circle (  2.13);

\path[fill=fillColor,fill opacity=0.20] (145.09, 77.76) circle (  2.13);

\path[fill=fillColor,fill opacity=0.20] (107.29, 83.45) circle (  2.13);

\path[fill=fillColor,fill opacity=0.20] (100.08, 76.14) circle (  2.13);

\path[fill=fillColor,fill opacity=0.20] (102.05, 70.45) circle (  2.13);

\path[fill=fillColor,fill opacity=0.20] ( 92.87, 71.26) circle (  2.13);

\path[fill=fillColor,fill opacity=0.20] ( 85.00, 65.57) circle (  2.13);

\path[fill=fillColor,fill opacity=0.20] ( 83.25, 57.44) circle (  2.13);

\path[fill=fillColor,fill opacity=0.20] ( 79.54, 61.51) circle (  2.13);

\path[fill=fillColor,fill opacity=0.20] ( 80.20, 68.82) circle (  2.13);

\path[fill=fillColor,fill opacity=0.20] ( 72.11, 72.07) circle (  2.13);

\path[fill=fillColor,fill opacity=0.20] ( 69.49, 56.63) circle (  2.13);

\path[fill=fillColor,fill opacity=0.20] ( 67.74, 47.69) circle (  2.13);

\path[fill=fillColor,fill opacity=0.20] ( 65.99, 52.57) circle (  2.13);

\path[fill=fillColor,fill opacity=0.20] ( 91.56, 68.01) circle (  2.13);

\path[fill=fillColor,fill opacity=0.20] ( 84.13, 75.32) circle (  2.13);

\path[fill=fillColor,fill opacity=0.20] ( 86.75, 79.39) circle (  2.13);

\path[fill=fillColor,fill opacity=0.20] ( 97.02, 73.70) circle (  2.13);

\path[fill=fillColor,fill opacity=0.20] ( 95.71, 68.01) circle (  2.13);

\path[fill=fillColor,fill opacity=0.20] ( 87.62, 60.69) circle (  2.13);

\path[fill=fillColor,fill opacity=0.20] ( 88.50, 48.50) circle (  2.13);

\path[fill=fillColor,fill opacity=0.20] ( 59.66, 49.32) circle (  2.13);

\path[fill=fillColor,fill opacity=0.20] ( 62.50, 68.82) circle (  2.13);

\path[fill=fillColor,fill opacity=0.20] ( 47.42, 77.76) circle (  2.13);

\path[fill=fillColor,fill opacity=0.20] ( 54.41, 71.26) circle (  2.13);

\path[fill=fillColor,fill opacity=0.20] ( 89.37, 55.82) circle (  2.13);

\path[fill=fillColor,fill opacity=0.20] (103.79, 76.14) circle (  2.13);

\path[fill=fillColor,fill opacity=0.20] (101.61, 87.51) circle (  2.13);

\path[fill=fillColor,fill opacity=0.20] ( 93.74, 84.26) circle (  2.13);

\path[fill=fillColor,fill opacity=0.20] ( 83.69, 75.32) circle (  2.13);

\path[fill=fillColor,fill opacity=0.20] ( 85.66, 68.82) circle (  2.13);

\path[fill=fillColor,fill opacity=0.20] ( 86.10, 70.45) circle (  2.13);

\path[fill=fillColor,fill opacity=0.20] ( 76.48, 72.89) circle (  2.13);

\path[fill=fillColor,fill opacity=0.20] ( 76.26, 73.70) circle (  2.13);

\path[fill=fillColor,fill opacity=0.20] ( 70.36, 68.01) circle (  2.13);

\path[fill=fillColor,fill opacity=0.20] ( 70.80, 45.25) circle (  2.13);

\path[fill=fillColor,fill opacity=0.20] ( 79.98, 53.38) circle (  2.13);

\path[fill=fillColor,fill opacity=0.20] ( 82.16, 60.69) circle (  2.13);

\path[fill=fillColor,fill opacity=0.20] ( 86.97, 66.38) circle (  2.13);

\path[fill=fillColor,fill opacity=0.20] ( 83.25, 61.51) circle (  2.13);

\path[fill=fillColor,fill opacity=0.20] ( 83.47, 64.76) circle (  2.13);

\path[fill=fillColor,fill opacity=0.20] ( 74.51, 64.76) circle (  2.13);

\path[fill=fillColor,fill opacity=0.20] ( 71.89, 55.82) circle (  2.13);

\path[fill=fillColor,fill opacity=0.20] ( 67.74, 50.94) circle (  2.13);

\path[fill=fillColor,fill opacity=0.20] ( 58.13, 48.50) circle (  2.13);

\path[fill=fillColor,fill opacity=0.20] ( 51.79, 55.82) circle (  2.13);

\path[fill=fillColor,fill opacity=0.20] ( 66.65, 95.64) circle (  2.13);

\path[fill=fillColor,fill opacity=0.20] ( 70.58, 67.20) circle (  2.13);

\path[fill=fillColor,fill opacity=0.20] ( 97.24, 73.70) circle (  2.13);

\path[fill=fillColor,fill opacity=0.20] (100.30, 77.76) circle (  2.13);

\path[fill=fillColor,fill opacity=0.20] ( 91.34, 80.20) circle (  2.13);

\path[fill=fillColor,fill opacity=0.20] ( 91.56, 75.32) circle (  2.13);

\path[fill=fillColor,fill opacity=0.20] ( 87.84, 68.82) circle (  2.13);

\path[fill=fillColor,fill opacity=0.20] ( 79.54, 70.45) circle (  2.13);

\path[fill=fillColor,fill opacity=0.20] ( 84.13, 74.51) circle (  2.13);

\path[fill=fillColor,fill opacity=0.20] ( 83.69, 74.51) circle (  2.13);

\path[fill=fillColor,fill opacity=0.20] ( 78.45, 75.32) circle (  2.13);

\path[fill=fillColor,fill opacity=0.20] ( 77.36, 71.26) circle (  2.13);

\path[fill=fillColor,fill opacity=0.20] ( 67.74, 50.13) circle (  2.13);

\path[fill=fillColor,fill opacity=0.20] ( 68.83, 37.94) circle (  2.13);

\path[fill=fillColor,fill opacity=0.20] ( 69.71, 45.25) circle (  2.13);

\path[fill=fillColor,fill opacity=0.20] ( 70.14, 59.07) circle (  2.13);

\path[fill=fillColor,fill opacity=0.20] ( 75.39, 76.14) circle (  2.13);

\path[fill=fillColor,fill opacity=0.20] ( 81.07, 63.13) circle (  2.13);

\path[fill=fillColor,fill opacity=0.20] ( 77.57, 66.38) circle (  2.13);

\path[fill=fillColor,fill opacity=0.20] ( 87.41, 69.63) circle (  2.13);

\path[fill=fillColor,fill opacity=0.20] ( 89.37, 53.38) circle (  2.13);

\path[fill=fillColor,fill opacity=0.20] ( 94.40, 55.01) circle (  2.13);

\path[fill=fillColor,fill opacity=0.20] ( 93.31, 61.51) circle (  2.13);

\path[fill=fillColor,fill opacity=0.20] ( 82.82, 63.13) circle (  2.13);

\path[fill=fillColor,fill opacity=0.20] ( 64.90, 59.88) circle (  2.13);

\path[fill=fillColor,fill opacity=0.20] ( 63.81, 50.13) circle (  2.13);

\path[fill=fillColor,fill opacity=0.20] ( 70.14, 63.95) circle (  2.13);

\path[fill=fillColor,fill opacity=0.20] ( 94.18, 67.20) circle (  2.13);

\path[fill=fillColor,fill opacity=0.20] ( 91.12, 76.14) circle (  2.13);

\path[fill=fillColor,fill opacity=0.20] ( 89.37, 72.07) circle (  2.13);

\path[fill=fillColor,fill opacity=0.20] ( 91.34, 62.32) circle (  2.13);

\path[fill=fillColor,fill opacity=0.20] ( 90.25, 62.32) circle (  2.13);

\path[fill=fillColor,fill opacity=0.20] ( 94.62, 71.26) circle (  2.13);

\path[fill=fillColor,fill opacity=0.20] ( 95.93, 72.89) circle (  2.13);

\path[fill=fillColor,fill opacity=0.20] ( 88.72, 74.51) circle (  2.13);

\path[fill=fillColor,fill opacity=0.20] ( 81.51, 67.20) circle (  2.13);

\path[fill=fillColor,fill opacity=0.20] ( 75.83, 63.13) circle (  2.13);

\path[fill=fillColor,fill opacity=0.20] ( 74.73, 63.13) circle (  2.13);

\path[fill=fillColor,fill opacity=0.20] ( 74.51, 57.44) circle (  2.13);

\path[fill=fillColor,fill opacity=0.20] ( 81.29, 51.75) circle (  2.13);

\path[fill=fillColor,fill opacity=0.20] ( 70.80, 51.75) circle (  2.13);

\path[fill=fillColor,fill opacity=0.20] ( 71.02, 51.75) circle (  2.13);

\path[fill=fillColor,fill opacity=0.20] ( 71.24, 55.82) circle (  2.13);

\path[fill=fillColor,fill opacity=0.20] ( 72.55, 61.51) circle (  2.13);

\path[fill=fillColor,fill opacity=0.20] ( 75.17, 65.57) circle (  2.13);

\path[fill=fillColor,fill opacity=0.20] ( 78.88, 69.63) circle (  2.13);

\path[fill=fillColor,fill opacity=0.20] ( 72.77, 67.20) circle (  2.13);

\path[fill=fillColor,fill opacity=0.20] ( 72.99, 56.63) circle (  2.13);

\path[fill=fillColor,fill opacity=0.20] ( 77.14, 51.75) circle (  2.13);

\path[fill=fillColor,fill opacity=0.20] ( 79.98, 59.07) circle (  2.13);

\path[fill=fillColor,fill opacity=0.20] ( 77.36, 63.95) circle (  2.13);

\path[fill=fillColor,fill opacity=0.20] ( 79.10, 61.51) circle (  2.13);

\path[fill=fillColor,fill opacity=0.20] ( 80.20, 65.57) circle (  2.13);

\path[fill=fillColor,fill opacity=0.20] ( 78.01, 68.01) circle (  2.13);

\path[fill=fillColor,fill opacity=0.20] ( 78.23, 63.95) circle (  2.13);

\path[fill=fillColor,fill opacity=0.20] ( 83.25, 56.63) circle (  2.13);

\path[fill=fillColor,fill opacity=0.20] ( 92.65, 47.69) circle (  2.13);

\path[fill=fillColor,fill opacity=0.20] ( 90.25, 45.25) circle (  2.13);

\path[fill=fillColor,fill opacity=0.20] ( 93.52, 61.51) circle (  2.13);

\path[fill=fillColor,fill opacity=0.20] ( 91.34, 73.70) circle (  2.13);

\path[fill=fillColor,fill opacity=0.20] ( 93.52, 59.88) circle (  2.13);

\path[fill=fillColor,fill opacity=0.20] ( 90.68, 43.63) circle (  2.13);

\path[fill=fillColor,fill opacity=0.20] ( 82.82, 46.88) circle (  2.13);

\path[fill=fillColor,fill opacity=0.20] ( 61.19, 50.94) circle (  2.13);

\path[fill=fillColor,fill opacity=0.20] ( 46.55, 53.38) circle (  2.13);

\path[fill=fillColor,fill opacity=0.20] ( 67.09, 77.76) circle (  2.13);

\path[fill=fillColor,fill opacity=0.20] ( 61.19, 64.76) circle (  2.13);

\path[fill=fillColor,fill opacity=0.20] ( 72.99, 70.45) circle (  2.13);

\path[fill=fillColor,fill opacity=0.20] ( 71.24, 69.63) circle (  2.13);

\path[fill=fillColor,fill opacity=0.20] ( 82.82, 63.13) circle (  2.13);

\path[fill=fillColor,fill opacity=0.20] ( 98.11, 63.13) circle (  2.13);

\path[fill=fillColor,fill opacity=0.20] (109.69, 64.76) circle (  2.13);

\path[fill=fillColor,fill opacity=0.20] ( 93.74, 60.69) circle (  2.13);

\path[fill=fillColor,fill opacity=0.20] ( 87.62, 62.32) circle (  2.13);

\path[fill=fillColor,fill opacity=0.20] ( 90.25, 75.32) circle (  2.13);

\path[fill=fillColor,fill opacity=0.20] ( 85.66, 80.20) circle (  2.13);

\path[fill=fillColor,fill opacity=0.20] ( 80.85, 72.89) circle (  2.13);

\path[fill=fillColor,fill opacity=0.20] ( 78.67, 65.57) circle (  2.13);

\path[fill=fillColor,fill opacity=0.20] ( 76.26, 57.44) circle (  2.13);

\path[fill=fillColor,fill opacity=0.20] ( 76.48, 70.45) circle (  2.13);

\path[fill=fillColor,fill opacity=0.20] ( 79.10, 64.76) circle (  2.13);

\path[fill=fillColor,fill opacity=0.20] ( 73.42, 53.38) circle (  2.13);

\path[fill=fillColor,fill opacity=0.20] ( 69.93, 51.75) circle (  2.13);

\path[fill=fillColor,fill opacity=0.20] ( 71.89, 62.32) circle (  2.13);

\path[fill=fillColor,fill opacity=0.20] ( 77.79, 70.45) circle (  2.13);

\path[fill=fillColor,fill opacity=0.20] ( 74.73, 63.95) circle (  2.13);

\path[fill=fillColor,fill opacity=0.20] ( 75.17, 59.88) circle (  2.13);

\path[fill=fillColor,fill opacity=0.20] ( 80.41, 68.82) circle (  2.13);

\path[fill=fillColor,fill opacity=0.20] ( 78.45, 68.82) circle (  2.13);

\path[fill=fillColor,fill opacity=0.20] ( 80.63, 57.44) circle (  2.13);

\path[fill=fillColor,fill opacity=0.20] ( 80.41, 58.26) circle (  2.13);

\path[fill=fillColor,fill opacity=0.20] ( 89.81, 65.57) circle (  2.13);

\path[fill=fillColor,fill opacity=0.20] ( 95.71, 63.13) circle (  2.13);

\path[fill=fillColor,fill opacity=0.20] ( 90.90, 55.01) circle (  2.13);

\path[fill=fillColor,fill opacity=0.20] ( 78.01, 60.69) circle (  2.13);

\path[fill=fillColor,fill opacity=0.20] ( 68.40, 65.57) circle (  2.13);

\path[fill=fillColor,fill opacity=0.20] ( 66.43, 51.75) circle (  2.13);

\path[fill=fillColor,fill opacity=0.20] ( 67.96, 44.44) circle (  2.13);

\path[fill=fillColor,fill opacity=0.20] ( 69.71, 54.19) circle (  2.13);

\path[fill=fillColor,fill opacity=0.20] ( 78.88, 57.44) circle (  2.13);

\path[fill=fillColor,fill opacity=0.20] ( 55.94, 71.26) circle (  2.13);

\path[fill=fillColor,fill opacity=0.20] ( 74.51, 55.01) circle (  2.13);

\path[fill=fillColor,fill opacity=0.20] ( 90.47, 50.94) circle (  2.13);

\path[fill=fillColor,fill opacity=0.20] ( 79.76, 44.44) circle (  2.13);

\path[fill=fillColor,fill opacity=0.20] ( 82.38, 53.38) circle (  2.13);

\path[fill=fillColor,fill opacity=0.20] ( 88.28, 72.89) circle (  2.13);

\path[fill=fillColor,fill opacity=0.20] ( 88.72, 74.51) circle (  2.13);

\path[fill=fillColor,fill opacity=0.20] ( 92.87, 66.38) circle (  2.13);

\path[fill=fillColor,fill opacity=0.20] ( 91.56, 68.82) circle (  2.13);

\path[fill=fillColor,fill opacity=0.20] ( 94.40, 66.38) circle (  2.13);

\path[fill=fillColor,fill opacity=0.20] ( 96.80, 67.20) circle (  2.13);

\path[fill=fillColor,fill opacity=0.20] ( 90.90, 79.39) circle (  2.13);

\path[fill=fillColor,fill opacity=0.20] ( 88.06, 70.45) circle (  2.13);

\path[fill=fillColor,fill opacity=0.20] ( 79.32, 56.63) circle (  2.13);

\path[fill=fillColor,fill opacity=0.20] ( 79.32, 66.38) circle (  2.13);

\path[fill=fillColor,fill opacity=0.20] ( 83.69, 76.14) circle (  2.13);

\path[fill=fillColor,fill opacity=0.20] ( 85.44, 72.89) circle (  2.13);

\path[fill=fillColor,fill opacity=0.20] (103.14, 83.45) circle (  2.13);

\path[fill=fillColor,fill opacity=0.20] ( 86.75, 78.57) circle (  2.13);

\path[fill=fillColor,fill opacity=0.20] ( 86.31, 59.07) circle (  2.13);

\path[fill=fillColor,fill opacity=0.20] ( 85.00, 55.01) circle (  2.13);

\path[fill=fillColor,fill opacity=0.20] ( 90.03, 63.95) circle (  2.13);

\path[fill=fillColor,fill opacity=0.20] ( 91.99, 65.57) circle (  2.13);

\path[fill=fillColor,fill opacity=0.20] ( 78.45, 57.44) circle (  2.13);

\path[fill=fillColor,fill opacity=0.20] ( 60.53, 53.38) circle (  2.13);

\path[fill=fillColor,fill opacity=0.20] ( 52.01, 58.26) circle (  2.13);

\path[fill=fillColor,fill opacity=0.20] ( 48.51, 64.76) circle (  2.13);

\path[fill=fillColor,fill opacity=0.20] ( 45.02, 69.63) circle (  2.13);

\path[fill=fillColor,fill opacity=0.20] ( 45.24, 74.51) circle (  2.13);

\path[fill=fillColor,fill opacity=0.20] ( 55.29, 55.01) circle (  2.13);

\path[fill=fillColor,fill opacity=0.20] ( 59.44, 43.63) circle (  2.13);

\path[fill=fillColor,fill opacity=0.20] ( 61.19, 38.75) circle (  2.13);

\path[fill=fillColor,fill opacity=0.20] ( 69.49, 46.07) circle (  2.13);

\path[fill=fillColor,fill opacity=0.20] ( 74.73, 50.94) circle (  2.13);

\path[fill=fillColor,fill opacity=0.20] ( 71.02, 49.32) circle (  2.13);

\path[fill=fillColor,fill opacity=0.20] ( 80.41, 54.19) circle (  2.13);

\path[fill=fillColor,fill opacity=0.20] ( 91.99, 52.57) circle (  2.13);

\path[fill=fillColor,fill opacity=0.20] ( 84.35, 46.88) circle (  2.13);

\path[fill=fillColor,fill opacity=0.20] ( 83.69, 57.44) circle (  2.13);

\path[fill=fillColor,fill opacity=0.20] (118.87, 60.69) circle (  2.13);

\path[fill=fillColor,fill opacity=0.20] ( 99.21, 52.57) circle (  2.13);

\path[fill=fillColor,fill opacity=0.20] ( 91.34, 58.26) circle (  2.13);

\path[fill=fillColor,fill opacity=0.20] ( 90.25, 73.70) circle (  2.13);

\path[fill=fillColor,fill opacity=0.20] ( 95.93, 81.01) circle (  2.13);

\path[fill=fillColor,fill opacity=0.20] ( 88.50, 75.32) circle (  2.13);

\path[fill=fillColor,fill opacity=0.20] (101.17, 72.07) circle (  2.13);

\path[fill=fillColor,fill opacity=0.20] (105.98, 73.70) circle (  2.13);

\path[fill=fillColor,fill opacity=0.20] ( 96.36, 74.51) circle (  2.13);

\path[fill=fillColor,fill opacity=0.20] ( 85.22, 67.20) circle (  2.13);

\path[fill=fillColor,fill opacity=0.20] ( 80.20, 57.44) circle (  2.13);

\path[fill=fillColor,fill opacity=0.20] ( 75.17, 53.38) circle (  2.13);

\path[fill=fillColor,fill opacity=0.20] ( 73.64, 51.75) circle (  2.13);

\path[fill=fillColor,fill opacity=0.20] ( 69.05, 51.75) circle (  2.13);

\path[fill=fillColor,fill opacity=0.20] ( 62.50, 48.50) circle (  2.13);

\path[fill=fillColor,fill opacity=0.20] ( 52.23, 50.13) circle (  2.13);

\path[fill=fillColor,fill opacity=0.20] ( 50.48, 51.75) circle (  2.13);

\path[fill=fillColor,fill opacity=0.20] ( 50.92, 50.13) circle (  2.13);

\path[fill=fillColor,fill opacity=0.20] ( 57.25, 47.69) circle (  2.13);

\path[fill=fillColor,fill opacity=0.20] ( 54.85, 46.07) circle (  2.13);

\path[fill=fillColor,fill opacity=0.20] ( 49.17, 43.63) circle (  2.13);

\path[fill=fillColor,fill opacity=0.20] ( 51.79, 49.32) circle (  2.13);

\path[fill=fillColor,fill opacity=0.20] (122.15, 47.69) circle (  2.13);

\path[fill=fillColor,fill opacity=0.20] ( 78.67, 42.82) circle (  2.13);

\path[fill=fillColor,fill opacity=0.20] ( 78.45, 61.51) circle (  2.13);

\path[fill=fillColor,fill opacity=0.20] ( 74.95, 59.88) circle (  2.13);

\path[fill=fillColor,fill opacity=0.20] ( 71.67, 55.01) circle (  2.13);

\path[fill=fillColor,fill opacity=0.20] ( 78.01, 62.32) circle (  2.13);

\path[fill=fillColor,fill opacity=0.20] ( 74.30, 60.69) circle (  2.13);

\path[fill=fillColor,fill opacity=0.20] ( 70.14, 44.44) circle (  2.13);

\path[fill=fillColor,fill opacity=0.20] ( 63.15, 40.38) circle (  2.13);

\path[fill=fillColor,fill opacity=0.20] ( 65.12, 55.82) circle (  2.13);

\path[fill=fillColor,fill opacity=0.20] (105.10, 61.51) circle (  2.13);

\path[fill=fillColor,fill opacity=0.20] ( 66.43, 55.82) circle (  2.13);

\path[fill=fillColor,fill opacity=0.20] ( 58.78, 55.82) circle (  2.13);

\path[fill=fillColor,fill opacity=0.20] ( 59.22, 42.82) circle (  2.13);

\path[fill=fillColor,fill opacity=0.20] ( 56.82, 55.82) circle (  2.13);

\path[fill=fillColor,fill opacity=0.20] ( 54.85, 59.07) circle (  2.13);

\path[fill=fillColor,fill opacity=0.20] ( 54.85, 62.32) circle (  2.13);

\path[fill=fillColor,fill opacity=0.20] ( 56.38, 70.45) circle (  2.13);

\path[fill=fillColor,fill opacity=0.20] ( 59.22, 65.57) circle (  2.13);

\path[fill=fillColor,fill opacity=0.20] ( 46.98, 54.19) circle (  2.13);

\path[fill=fillColor,fill opacity=0.20] ( 48.08, 54.19) circle (  2.13);

\path[fill=fillColor,fill opacity=0.20] ( 50.70, 61.51) circle (  2.13);

\path[fill=fillColor,fill opacity=0.20] ( 54.41, 63.13) circle (  2.13);

\path[fill=fillColor,fill opacity=0.20] ( 81.51, 56.63) circle (  2.13);

\path[fill=fillColor,fill opacity=0.20] ( 53.98, 59.88) circle (  2.13);

\path[fill=fillColor,fill opacity=0.20] ( 85.66, 88.33) circle (  2.13);

\path[fill=fillColor,fill opacity=0.20] (133.07, 93.20) circle (  2.13);

\path[fill=fillColor,fill opacity=0.20] ( 95.93, 86.70) circle (  2.13);

\path[fill=fillColor,fill opacity=0.20] ( 81.51, 77.76) circle (  2.13);

\path[fill=fillColor,fill opacity=0.20] ( 78.23, 74.51) circle (  2.13);

\path[fill=fillColor,fill opacity=0.20] ( 73.42, 94.83) circle (  2.13);

\path[fill=fillColor,fill opacity=0.20] ( 93.52, 93.20) circle (  2.13);

\path[fill=fillColor,fill opacity=0.20] (101.17, 94.83) circle (  2.13);

\path[fill=fillColor,fill opacity=0.20] ( 97.24, 90.76) circle (  2.13);

\path[fill=fillColor,fill opacity=0.20] ( 98.11, 82.64) circle (  2.13);

\path[fill=fillColor,fill opacity=0.20] ( 94.84, 81.82) circle (  2.13);

\path[fill=fillColor,fill opacity=0.20] ( 81.73, 75.32) circle (  2.13);

\path[fill=fillColor,fill opacity=0.20] ( 79.76, 63.13) circle (  2.13);

\path[fill=fillColor,fill opacity=0.20] ( 85.66, 57.44) circle (  2.13);

\path[fill=fillColor,fill opacity=0.20] ( 70.14, 47.69) circle (  2.13);

\path[fill=fillColor,fill opacity=0.20] ( 72.77, 87.51) circle (  2.13);

\path[fill=fillColor,fill opacity=0.20] ( 86.31, 84.26) circle (  2.13);

\path[fill=fillColor,fill opacity=0.20] ( 92.87, 92.39) circle (  2.13);

\path[fill=fillColor,fill opacity=0.20] ( 99.64, 99.70) circle (  2.13);

\path[fill=fillColor,fill opacity=0.20] ( 95.93, 94.83) circle (  2.13);

\path[fill=fillColor,fill opacity=0.20] ( 92.21, 88.33) circle (  2.13);

\path[fill=fillColor,fill opacity=0.20] ( 89.15, 81.82) circle (  2.13);

\path[fill=fillColor,fill opacity=0.20] ( 80.20, 76.14) circle (  2.13);

\path[fill=fillColor,fill opacity=0.20] ( 81.29, 65.57) circle (  2.13);

\path[fill=fillColor,fill opacity=0.20] ( 77.14, 51.75) circle (  2.13);

\path[fill=fillColor,fill opacity=0.20] ( 66.43, 42.82) circle (  2.13);

\path[fill=fillColor,fill opacity=0.20] ( 60.31, 98.89) circle (  2.13);

\path[fill=fillColor,fill opacity=0.20] ( 84.35, 84.26) circle (  2.13);

\path[fill=fillColor,fill opacity=0.20] ( 95.05, 90.76) circle (  2.13);

\path[fill=fillColor,fill opacity=0.20] ( 96.58, 97.27) circle (  2.13);

\path[fill=fillColor,fill opacity=0.20] ( 96.15,105.39) circle (  2.13);

\path[fill=fillColor,fill opacity=0.20] ( 91.12,102.14) circle (  2.13);

\path[fill=fillColor,fill opacity=0.20] ( 87.62, 97.27) circle (  2.13);

\path[fill=fillColor,fill opacity=0.20] ( 85.00, 89.14) circle (  2.13);

\path[fill=fillColor,fill opacity=0.20] ( 82.38, 81.01) circle (  2.13);

\path[fill=fillColor,fill opacity=0.20] ( 77.57, 74.51) circle (  2.13);

\path[fill=fillColor,fill opacity=0.20] ( 73.64, 69.63) circle (  2.13);

\path[fill=fillColor,fill opacity=0.20] ( 74.51, 65.57) circle (  2.13);

\path[fill=fillColor,fill opacity=0.20] ( 62.93, 56.63) circle (  2.13);

\path[fill=fillColor,fill opacity=0.20] ( 62.28, 42.82) circle (  2.13);

\path[fill=fillColor,fill opacity=0.20] ( 71.02, 37.94) circle (  2.13);

\path[fill=fillColor,fill opacity=0.20] ( 95.93, 95.64) circle (  2.13);

\path[fill=fillColor,fill opacity=0.20] ( 94.84,105.39) circle (  2.13);

\path[fill=fillColor,fill opacity=0.20] ( 90.90,104.58) circle (  2.13);

\path[fill=fillColor,fill opacity=0.20] ( 87.19,106.21) circle (  2.13);

\path[fill=fillColor,fill opacity=0.20] ( 85.44,109.46) circle (  2.13);

\path[fill=fillColor,fill opacity=0.20] ( 81.73,100.52) circle (  2.13);

\path[fill=fillColor,fill opacity=0.20] ( 76.26, 92.39) circle (  2.13);

\path[fill=fillColor,fill opacity=0.20] ( 78.67, 81.82) circle (  2.13);

\path[fill=fillColor,fill opacity=0.20] ( 76.70, 70.45) circle (  2.13);

\path[fill=fillColor,fill opacity=0.20] ( 64.90, 68.82) circle (  2.13);

\path[fill=fillColor,fill opacity=0.20] ( 60.53, 73.70) circle (  2.13);

\path[fill=fillColor,fill opacity=0.20] ( 61.84, 66.38) circle (  2.13);

\path[fill=fillColor,fill opacity=0.20] ( 74.08, 94.83) circle (  2.13);

\path[fill=fillColor,fill opacity=0.20] ( 96.36, 92.39) circle (  2.13);

\path[fill=fillColor,fill opacity=0.20] ( 92.43,104.58) circle (  2.13);

\path[fill=fillColor,fill opacity=0.20] ( 95.49, 98.08) circle (  2.13);

\path[fill=fillColor,fill opacity=0.20] ( 83.69, 99.70) circle (  2.13);

\path[fill=fillColor,fill opacity=0.20] ( 76.70,107.83) circle (  2.13);

\path[fill=fillColor,fill opacity=0.20] ( 79.32, 99.70) circle (  2.13);

\path[fill=fillColor,fill opacity=0.20] ( 76.92, 86.70) circle (  2.13);

\path[fill=fillColor,fill opacity=0.20] ( 73.42, 78.57) circle (  2.13);

\path[fill=fillColor,fill opacity=0.20] ( 67.74, 63.13) circle (  2.13);

\path[fill=fillColor,fill opacity=0.20] ( 53.98, 55.82) circle (  2.13);

\path[fill=fillColor,fill opacity=0.20] ( 68.62, 82.64) circle (  2.13);

\path[fill=fillColor,fill opacity=0.20] ( 80.20, 90.76) circle (  2.13);

\path[fill=fillColor,fill opacity=0.20] ( 91.78, 85.89) circle (  2.13);

\path[fill=fillColor,fill opacity=0.20] ( 91.12, 94.83) circle (  2.13);

\path[fill=fillColor,fill opacity=0.20] (104.01, 92.39) circle (  2.13);

\path[fill=fillColor,fill opacity=0.20] ( 84.13, 94.02) circle (  2.13);

\path[fill=fillColor,fill opacity=0.20] ( 77.14,100.52) circle (  2.13);

\path[fill=fillColor,fill opacity=0.20] ( 74.08, 96.45) circle (  2.13);

\path[fill=fillColor,fill opacity=0.20] ( 71.89, 81.01) circle (  2.13);

\path[fill=fillColor,fill opacity=0.20] ( 67.96, 71.26) circle (  2.13);

\path[fill=fillColor,fill opacity=0.20] ( 56.16, 69.63) circle (  2.13);

\path[fill=fillColor,fill opacity=0.20] ( 72.77, 63.95) circle (  2.13);

\path[fill=fillColor,fill opacity=0.20] ( 79.10, 62.32) circle (  2.13);

\path[fill=fillColor,fill opacity=0.20] ( 77.57, 62.32) circle (  2.13);

\path[fill=fillColor,fill opacity=0.20] ( 72.33, 63.95) circle (  2.13);

\path[fill=fillColor,fill opacity=0.20] ( 76.04,102.96) circle (  2.13);

\path[fill=fillColor,fill opacity=0.20] ( 93.31, 96.45) circle (  2.13);

\path[fill=fillColor,fill opacity=0.20] ( 89.59, 97.27) circle (  2.13);

\path[fill=fillColor,fill opacity=0.20] ( 85.44, 94.83) circle (  2.13);

\path[fill=fillColor,fill opacity=0.20] ( 78.45, 94.02) circle (  2.13);

\path[fill=fillColor,fill opacity=0.20] ( 78.45, 90.76) circle (  2.13);

\path[fill=fillColor,fill opacity=0.20] ( 73.86, 86.70) circle (  2.13);

\path[fill=fillColor,fill opacity=0.20] ( 68.62, 76.95) circle (  2.13);

\path[fill=fillColor,fill opacity=0.20] ( 61.62, 70.45) circle (  2.13);

\path[fill=fillColor,fill opacity=0.20] ( 50.04, 73.70) circle (  2.13);

\path[fill=fillColor,fill opacity=0.20] ( 76.92, 79.39) circle (  2.13);

\path[fill=fillColor,fill opacity=0.20] ( 74.30, 74.51) circle (  2.13);

\path[fill=fillColor,fill opacity=0.20] ( 79.98, 87.51) circle (  2.13);

\path[fill=fillColor,fill opacity=0.20] ( 88.06, 83.45) circle (  2.13);

\path[fill=fillColor,fill opacity=0.20] ( 79.76, 77.76) circle (  2.13);

\path[fill=fillColor,fill opacity=0.20] ( 78.67, 66.38) circle (  2.13);

\path[fill=fillColor,fill opacity=0.20] ( 77.14, 47.69) circle (  2.13);

\path[fill=fillColor,fill opacity=0.20] ( 66.87, 66.38) circle (  2.13);

\path[fill=fillColor,fill opacity=0.20] ( 64.68,111.89) circle (  2.13);

\path[fill=fillColor,fill opacity=0.20] (112.10,100.52) circle (  2.13);

\path[fill=fillColor,fill opacity=0.20] ( 94.84, 98.89) circle (  2.13);

\path[fill=fillColor,fill opacity=0.20] ( 84.13, 90.76) circle (  2.13);

\path[fill=fillColor,fill opacity=0.20] ( 82.60, 90.76) circle (  2.13);

\path[fill=fillColor,fill opacity=0.20] ( 83.25, 86.70) circle (  2.13);

\path[fill=fillColor,fill opacity=0.20] ( 72.77, 77.76) circle (  2.13);

\path[fill=fillColor,fill opacity=0.20] ( 66.21, 68.01) circle (  2.13);

\path[fill=fillColor,fill opacity=0.20] ( 86.97, 78.57) circle (  2.13);

\path[fill=fillColor,fill opacity=0.20] ( 84.13, 85.08) circle (  2.13);

\path[fill=fillColor,fill opacity=0.20] ( 88.72, 87.51) circle (  2.13);

\path[fill=fillColor,fill opacity=0.20] ( 86.97, 78.57) circle (  2.13);

\path[fill=fillColor,fill opacity=0.20] ( 86.10, 73.70) circle (  2.13);

\path[fill=fillColor,fill opacity=0.20] ( 81.94, 72.07) circle (  2.13);

\path[fill=fillColor,fill opacity=0.20] ( 75.83, 60.69) circle (  2.13);

\path[fill=fillColor,fill opacity=0.20] ( 78.88, 52.57) circle (  2.13);

\path[fill=fillColor,fill opacity=0.20] ( 59.88,106.21) circle (  2.13);

\path[fill=fillColor,fill opacity=0.20] (110.57, 81.01) circle (  2.13);

\path[fill=fillColor,fill opacity=0.20] (114.28, 87.51) circle (  2.13);

\path[fill=fillColor,fill opacity=0.20] ( 88.06, 80.20) circle (  2.13);

\path[fill=fillColor,fill opacity=0.20] ( 91.99, 84.26) circle (  2.13);

\path[fill=fillColor,fill opacity=0.20] ( 89.15, 85.89) circle (  2.13);

\path[fill=fillColor,fill opacity=0.20] ( 72.33, 71.26) circle (  2.13);

\path[fill=fillColor,fill opacity=0.20] ( 75.83, 63.95) circle (  2.13);

\path[fill=fillColor,fill opacity=0.20] ( 68.40, 59.07) circle (  2.13);

\path[fill=fillColor,fill opacity=0.20] ( 86.53, 89.95) circle (  2.13);

\path[fill=fillColor,fill opacity=0.20] ( 94.84, 77.76) circle (  2.13);

\path[fill=fillColor,fill opacity=0.20] ( 94.84, 81.82) circle (  2.13);

\path[fill=fillColor,fill opacity=0.20] ( 90.90, 77.76) circle (  2.13);

\path[fill=fillColor,fill opacity=0.20] ( 81.51, 71.26) circle (  2.13);

\path[fill=fillColor,fill opacity=0.20] ( 85.44, 62.32) circle (  2.13);

\path[fill=fillColor,fill opacity=0.20] ( 86.53, 62.32) circle (  2.13);

\path[fill=fillColor,fill opacity=0.20] ( 77.36, 61.51) circle (  2.13);

\path[fill=fillColor,fill opacity=0.20] ( 84.78, 50.13) circle (  2.13);

\path[fill=fillColor,fill opacity=0.20] ( 59.66,115.15) circle (  2.13);

\path[fill=fillColor,fill opacity=0.20] ( 97.68, 70.45) circle (  2.13);

\path[fill=fillColor,fill opacity=0.20] ( 98.77, 79.39) circle (  2.13);

\path[fill=fillColor,fill opacity=0.20] ( 84.78, 85.08) circle (  2.13);

\path[fill=fillColor,fill opacity=0.20] ( 85.66, 85.89) circle (  2.13);

\path[fill=fillColor,fill opacity=0.20] ( 82.16, 85.89) circle (  2.13);

\path[fill=fillColor,fill opacity=0.20] ( 74.95, 72.07) circle (  2.13);

\path[fill=fillColor,fill opacity=0.20] ( 74.51, 55.82) circle (  2.13);

\path[fill=fillColor,fill opacity=0.20] ( 68.62, 46.88) circle (  2.13);

\path[fill=fillColor,fill opacity=0.20] ( 94.62, 78.57) circle (  2.13);

\path[fill=fillColor,fill opacity=0.20] (102.26, 73.70) circle (  2.13);

\path[fill=fillColor,fill opacity=0.20] ( 94.62, 88.33) circle (  2.13);

\path[fill=fillColor,fill opacity=0.20] ( 92.21, 84.26) circle (  2.13);

\path[fill=fillColor,fill opacity=0.20] ( 92.43, 74.51) circle (  2.13);

\path[fill=fillColor,fill opacity=0.20] ( 87.41, 59.88) circle (  2.13);

\path[fill=fillColor,fill opacity=0.20] ( 91.12, 58.26) circle (  2.13);

\path[fill=fillColor,fill opacity=0.20] ( 81.94, 59.88) circle (  2.13);

\path[fill=fillColor,fill opacity=0.20] ( 78.67, 46.07) circle (  2.13);

\path[fill=fillColor,fill opacity=0.20] ( 91.56, 52.57) circle (  2.13);

\path[fill=fillColor,fill opacity=0.20] ( 65.56, 90.76) circle (  2.13);

\path[fill=fillColor,fill opacity=0.20] ( 82.60, 78.57) circle (  2.13);

\path[fill=fillColor,fill opacity=0.20] ( 89.81, 90.76) circle (  2.13);

\path[fill=fillColor,fill opacity=0.20] ( 85.88, 89.95) circle (  2.13);

\path[fill=fillColor,fill opacity=0.20] ( 82.60, 81.01) circle (  2.13);

\path[fill=fillColor,fill opacity=0.20] ( 81.29, 71.26) circle (  2.13);

\path[fill=fillColor,fill opacity=0.20] ( 71.89, 59.88) circle (  2.13);

\path[fill=fillColor,fill opacity=0.20] ( 66.65, 51.75) circle (  2.13);

\path[fill=fillColor,fill opacity=0.20] (104.67, 79.39) circle (  2.13);

\path[fill=fillColor,fill opacity=0.20] ( 98.11, 86.70) circle (  2.13);

\path[fill=fillColor,fill opacity=0.20] ( 91.99, 99.70) circle (  2.13);

\path[fill=fillColor,fill opacity=0.20] ( 96.58, 89.14) circle (  2.13);

\path[fill=fillColor,fill opacity=0.20] ( 99.21, 75.32) circle (  2.13);

\path[fill=fillColor,fill opacity=0.20] ( 92.87, 65.57) circle (  2.13);

\path[fill=fillColor,fill opacity=0.20] ( 99.21, 67.20) circle (  2.13);

\path[fill=fillColor,fill opacity=0.20] ( 88.06, 66.38) circle (  2.13);

\path[fill=fillColor,fill opacity=0.20] ( 80.41, 46.07) circle (  2.13);

\path[fill=fillColor,fill opacity=0.20] ( 78.23, 42.00) circle (  2.13);

\path[fill=fillColor,fill opacity=0.20] ( 76.04, 74.51) circle (  2.13);

\path[fill=fillColor,fill opacity=0.20] ( 87.84, 73.70) circle (  2.13);

\path[fill=fillColor,fill opacity=0.20] ( 90.47, 80.20) circle (  2.13);

\path[fill=fillColor,fill opacity=0.20] ( 77.79, 75.32) circle (  2.13);

\path[fill=fillColor,fill opacity=0.20] ( 79.32, 74.51) circle (  2.13);

\path[fill=fillColor,fill opacity=0.20] ( 75.17, 77.76) circle (  2.13);

\path[fill=fillColor,fill opacity=0.20] ( 65.34, 72.89) circle (  2.13);

\path[fill=fillColor,fill opacity=0.20] ( 67.74, 58.26) circle (  2.13);

\path[fill=fillColor,fill opacity=0.20] ( 96.36, 82.64) circle (  2.13);

\path[fill=fillColor,fill opacity=0.20] ( 87.41, 93.20) circle (  2.13);

\path[fill=fillColor,fill opacity=0.20] ( 89.59, 95.64) circle (  2.13);

\path[fill=fillColor,fill opacity=0.20] ( 92.43, 85.89) circle (  2.13);

\path[fill=fillColor,fill opacity=0.20] ( 93.52, 82.64) circle (  2.13);

\path[fill=fillColor,fill opacity=0.20] ( 87.41, 77.76) circle (  2.13);

\path[fill=fillColor,fill opacity=0.20] ( 93.31, 74.51) circle (  2.13);

\path[fill=fillColor,fill opacity=0.20] ( 92.65, 70.45) circle (  2.13);

\path[fill=fillColor,fill opacity=0.20] ( 80.41, 55.01) circle (  2.13);

\path[fill=fillColor,fill opacity=0.20] ( 46.55, 79.39) circle (  2.13);

\path[fill=fillColor,fill opacity=0.20] ( 66.43, 63.95) circle (  2.13);

\path[fill=fillColor,fill opacity=0.20] ( 83.04, 60.69) circle (  2.13);

\path[fill=fillColor,fill opacity=0.20] ( 82.16, 74.51) circle (  2.13);

\path[fill=fillColor,fill opacity=0.20] ( 76.70, 81.82) circle (  2.13);

\path[fill=fillColor,fill opacity=0.20] ( 78.67, 81.01) circle (  2.13);

\path[fill=fillColor,fill opacity=0.20] ( 72.77, 77.76) circle (  2.13);

\path[fill=fillColor,fill opacity=0.20] ( 69.71, 64.76) circle (  2.13);

\path[fill=fillColor,fill opacity=0.20] ( 66.87, 51.75) circle (  2.13);

\path[fill=fillColor,fill opacity=0.20] ( 78.88, 92.39) circle (  2.13);

\path[fill=fillColor,fill opacity=0.20] ( 84.78, 78.57) circle (  2.13);

\path[fill=fillColor,fill opacity=0.20] ( 83.91, 77.76) circle (  2.13);

\path[fill=fillColor,fill opacity=0.20] ( 91.12, 82.64) circle (  2.13);

\path[fill=fillColor,fill opacity=0.20] ( 93.31, 88.33) circle (  2.13);

\path[fill=fillColor,fill opacity=0.20] (100.08, 89.95) circle (  2.13);

\path[fill=fillColor,fill opacity=0.20] ( 98.55, 84.26) circle (  2.13);

\path[fill=fillColor,fill opacity=0.20] ( 95.49, 76.14) circle (  2.13);

\path[fill=fillColor,fill opacity=0.20] ( 91.12, 72.07) circle (  2.13);

\path[fill=fillColor,fill opacity=0.20] ( 84.13, 63.13) circle (  2.13);

\path[fill=fillColor,fill opacity=0.20] ( 82.60, 47.69) circle (  2.13);

\path[fill=fillColor,fill opacity=0.20] ( 74.51, 53.38) circle (  2.13);

\path[fill=fillColor,fill opacity=0.20] ( 52.45, 55.01) circle (  2.13);

\path[fill=fillColor,fill opacity=0.20] ( 71.89, 64.76) circle (  2.13);

\path[fill=fillColor,fill opacity=0.20] ( 80.63, 79.39) circle (  2.13);

\path[fill=fillColor,fill opacity=0.20] ( 86.10, 72.89) circle (  2.13);

\path[fill=fillColor,fill opacity=0.20] ( 78.01, 66.38) circle (  2.13);

\path[fill=fillColor,fill opacity=0.20] ( 72.33, 68.01) circle (  2.13);

\path[fill=fillColor,fill opacity=0.20] ( 68.62, 64.76) circle (  2.13);

\path[fill=fillColor,fill opacity=0.20] ( 70.80, 59.07) circle (  2.13);

\path[fill=fillColor,fill opacity=0.20] ( 76.26, 74.51) circle (  2.13);

\path[fill=fillColor,fill opacity=0.20] ( 81.07, 81.82) circle (  2.13);

\path[fill=fillColor,fill opacity=0.20] ( 90.90, 72.89) circle (  2.13);

\path[fill=fillColor,fill opacity=0.20] ( 95.05, 69.63) circle (  2.13);

\path[fill=fillColor,fill opacity=0.20] ( 94.18, 82.64) circle (  2.13);

\path[fill=fillColor,fill opacity=0.20] ( 97.89, 94.02) circle (  2.13);

\path[fill=fillColor,fill opacity=0.20] (104.45, 88.33) circle (  2.13);

\path[fill=fillColor,fill opacity=0.20] (107.07, 81.01) circle (  2.13);

\path[fill=fillColor,fill opacity=0.20] ( 99.42, 76.95) circle (  2.13);

\path[fill=fillColor,fill opacity=0.20] ( 93.96, 76.14) circle (  2.13);

\path[fill=fillColor,fill opacity=0.20] ( 91.12, 68.82) circle (  2.13);

\path[fill=fillColor,fill opacity=0.20] ( 81.07, 57.44) circle (  2.13);

\path[fill=fillColor,fill opacity=0.20] ( 52.88, 63.95) circle (  2.13);

\path[fill=fillColor,fill opacity=0.20] ( 54.19, 50.94) circle (  2.13);

\path[fill=fillColor,fill opacity=0.20] ( 77.36, 59.07) circle (  2.13);

\path[fill=fillColor,fill opacity=0.20] ( 90.90, 66.38) circle (  2.13);

\path[fill=fillColor,fill opacity=0.20] ( 79.98, 66.38) circle (  2.13);

\path[fill=fillColor,fill opacity=0.20] ( 75.61, 68.82) circle (  2.13);

\path[fill=fillColor,fill opacity=0.20] ( 73.42, 72.89) circle (  2.13);

\path[fill=fillColor,fill opacity=0.20] ( 69.49, 62.32) circle (  2.13);

\path[fill=fillColor,fill opacity=0.20] ( 82.82, 75.32) circle (  2.13);

\path[fill=fillColor,fill opacity=0.20] ( 89.15, 76.14) circle (  2.13);

\path[fill=fillColor,fill opacity=0.20] ( 86.75, 82.64) circle (  2.13);

\path[fill=fillColor,fill opacity=0.20] ( 97.02, 81.01) circle (  2.13);

\path[fill=fillColor,fill opacity=0.20] (101.83, 89.14) circle (  2.13);

\path[fill=fillColor,fill opacity=0.20] (109.91, 96.45) circle (  2.13);

\path[fill=fillColor,fill opacity=0.20] (105.10, 93.20) circle (  2.13);

\path[fill=fillColor,fill opacity=0.20] (104.45, 86.70) circle (  2.13);

\path[fill=fillColor,fill opacity=0.20] (100.52, 79.39) circle (  2.13);

\path[fill=fillColor,fill opacity=0.20] (100.08, 77.76) circle (  2.13);

\path[fill=fillColor,fill opacity=0.20] (101.39, 80.20) circle (  2.13);

\path[fill=fillColor,fill opacity=0.20] ( 94.18, 67.20) circle (  2.13);

\path[fill=fillColor,fill opacity=0.20] ( 83.04, 50.94) circle (  2.13);

\path[fill=fillColor,fill opacity=0.20] ( 53.32, 61.51) circle (  2.13);

\path[fill=fillColor,fill opacity=0.20] ( 80.85, 48.50) circle (  2.13);

\path[fill=fillColor,fill opacity=0.20] ( 76.26, 59.07) circle (  2.13);

\path[fill=fillColor,fill opacity=0.20] ( 85.66, 72.89) circle (  2.13);

\path[fill=fillColor,fill opacity=0.20] ( 79.54, 70.45) circle (  2.13);

\path[fill=fillColor,fill opacity=0.20] ( 76.70, 70.45) circle (  2.13);

\path[fill=fillColor,fill opacity=0.20] ( 74.51, 67.20) circle (  2.13);

\path[fill=fillColor,fill opacity=0.20] ( 72.99, 59.07) circle (  2.13);

\path[fill=fillColor,fill opacity=0.20] ( 80.85, 70.45) circle (  2.13);

\path[fill=fillColor,fill opacity=0.20] ( 88.50, 76.95) circle (  2.13);

\path[fill=fillColor,fill opacity=0.20] ( 90.68, 81.01) circle (  2.13);

\path[fill=fillColor,fill opacity=0.20] ( 92.65, 78.57) circle (  2.13);

\path[fill=fillColor,fill opacity=0.20] ( 97.46, 89.14) circle (  2.13);

\path[fill=fillColor,fill opacity=0.20] ( 96.15,107.02) circle (  2.13);

\path[fill=fillColor,fill opacity=0.20] (106.85,105.39) circle (  2.13);

\path[fill=fillColor,fill opacity=0.20] (105.76, 94.02) circle (  2.13);

\path[fill=fillColor,fill opacity=0.20] ( 97.02, 83.45) circle (  2.13);

\path[fill=fillColor,fill opacity=0.20] ( 99.42, 77.76) circle (  2.13);

\path[fill=fillColor,fill opacity=0.20] ( 99.86, 80.20) circle (  2.13);

\path[fill=fillColor,fill opacity=0.20] (102.48, 81.82) circle (  2.13);

\path[fill=fillColor,fill opacity=0.20] ( 89.37, 64.76) circle (  2.13);

\path[fill=fillColor,fill opacity=0.20] ( 74.30, 49.32) circle (  2.13);

\path[fill=fillColor,fill opacity=0.20] ( 48.08, 59.88) circle (  2.13);

\path[fill=fillColor,fill opacity=0.20] ( 62.72, 50.94) circle (  2.13);

\path[fill=fillColor,fill opacity=0.20] ( 71.89, 63.95) circle (  2.13);

\path[fill=fillColor,fill opacity=0.20] ( 87.19, 74.51) circle (  2.13);

\path[fill=fillColor,fill opacity=0.20] ( 81.07, 72.89) circle (  2.13);

\path[fill=fillColor,fill opacity=0.20] ( 78.01, 69.63) circle (  2.13);

\path[fill=fillColor,fill opacity=0.20] ( 82.60, 68.82) circle (  2.13);

\path[fill=fillColor,fill opacity=0.20] ( 73.20, 63.13) circle (  2.13);

\path[fill=fillColor,fill opacity=0.20] ( 64.90, 59.88) circle (  2.13);

\path[fill=fillColor,fill opacity=0.20] ( 73.20, 71.26) circle (  2.13);

\path[fill=fillColor,fill opacity=0.20] ( 86.97, 71.26) circle (  2.13);

\path[fill=fillColor,fill opacity=0.20] ( 97.24, 76.95) circle (  2.13);

\path[fill=fillColor,fill opacity=0.20] ( 91.34, 78.57) circle (  2.13);

\path[fill=fillColor,fill opacity=0.20] (101.17, 81.82) circle (  2.13);

\path[fill=fillColor,fill opacity=0.20] (100.73, 91.58) circle (  2.13);

\path[fill=fillColor,fill opacity=0.20] (102.05,100.52) circle (  2.13);

\path[fill=fillColor,fill opacity=0.20] (104.45,103.77) circle (  2.13);

\path[fill=fillColor,fill opacity=0.20] (105.32, 94.02) circle (  2.13);

\path[fill=fillColor,fill opacity=0.20] (100.08, 81.01) circle (  2.13);

\path[fill=fillColor,fill opacity=0.20] (103.79, 82.64) circle (  2.13);

\path[fill=fillColor,fill opacity=0.20] (105.76, 88.33) circle (  2.13);

\path[fill=fillColor,fill opacity=0.20] (102.48, 81.01) circle (  2.13);

\path[fill=fillColor,fill opacity=0.20] ( 86.53, 65.57) circle (  2.13);

\path[fill=fillColor,fill opacity=0.20] ( 69.49, 59.88) circle (  2.13);

\path[fill=fillColor,fill opacity=0.20] (104.23, 66.38) circle (  2.13);

\path[fill=fillColor,fill opacity=0.20] ( 86.75, 68.82) circle (  2.13);

\path[fill=fillColor,fill opacity=0.20] ( 84.13, 69.63) circle (  2.13);

\path[fill=fillColor,fill opacity=0.20] ( 81.29, 74.51) circle (  2.13);

\path[fill=fillColor,fill opacity=0.20] ( 72.11, 64.76) circle (  2.13);

\path[fill=fillColor,fill opacity=0.20] ( 73.20, 57.44) circle (  2.13);

\path[fill=fillColor,fill opacity=0.20] ( 63.37, 68.82) circle (  2.13);

\path[fill=fillColor,fill opacity=0.20] ( 72.77, 72.07) circle (  2.13);

\path[fill=fillColor,fill opacity=0.20] ( 75.83, 82.64) circle (  2.13);

\path[fill=fillColor,fill opacity=0.20] ( 95.71, 81.82) circle (  2.13);

\path[fill=fillColor,fill opacity=0.20] ( 98.77, 83.45) circle (  2.13);

\path[fill=fillColor,fill opacity=0.20] ( 90.68, 90.76) circle (  2.13);

\path[fill=fillColor,fill opacity=0.20] ( 96.80, 91.58) circle (  2.13);

\path[fill=fillColor,fill opacity=0.20] ( 99.86, 90.76) circle (  2.13);

\path[fill=fillColor,fill opacity=0.20] (107.29, 92.39) circle (  2.13);

\path[fill=fillColor,fill opacity=0.20] (101.83, 94.83) circle (  2.13);

\path[fill=fillColor,fill opacity=0.20] ( 98.77, 90.76) circle (  2.13);

\path[fill=fillColor,fill opacity=0.20] (105.32, 85.08) circle (  2.13);

\path[fill=fillColor,fill opacity=0.20] (130.45, 89.14) circle (  2.13);

\path[fill=fillColor,fill opacity=0.20] (100.30, 84.26) circle (  2.13);

\path[fill=fillColor,fill opacity=0.20] ( 88.50, 70.45) circle (  2.13);

\path[fill=fillColor,fill opacity=0.20] ( 68.62, 66.38) circle (  2.13);

\path[fill=fillColor,fill opacity=0.20] ( 68.62, 53.38) circle (  2.13);

\path[fill=fillColor,fill opacity=0.20] ( 72.33, 50.13) circle (  2.13);

\path[fill=fillColor,fill opacity=0.20] ( 78.88, 64.76) circle (  2.13);

\path[fill=fillColor,fill opacity=0.20] ( 76.26, 75.32) circle (  2.13);

\path[fill=fillColor,fill opacity=0.20] ( 74.08, 63.95) circle (  2.13);

\path[fill=fillColor,fill opacity=0.20] ( 73.42, 61.51) circle (  2.13);

\path[fill=fillColor,fill opacity=0.20] ( 69.49, 68.01) circle (  2.13);

\path[fill=fillColor,fill opacity=0.20] ( 69.27, 65.57) circle (  2.13);

\path[fill=fillColor,fill opacity=0.20] ( 77.14, 72.07) circle (  2.13);

\path[fill=fillColor,fill opacity=0.20] ( 80.20, 76.14) circle (  2.13);

\path[fill=fillColor,fill opacity=0.20] ( 87.19, 89.95) circle (  2.13);

\path[fill=fillColor,fill opacity=0.20] ( 96.58, 91.58) circle (  2.13);

\path[fill=fillColor,fill opacity=0.20] (101.17, 93.20) circle (  2.13);

\path[fill=fillColor,fill opacity=0.20] ( 98.99, 99.70) circle (  2.13);

\path[fill=fillColor,fill opacity=0.20] ( 97.24, 91.58) circle (  2.13);

\path[fill=fillColor,fill opacity=0.20] ( 99.86, 85.08) circle (  2.13);

\path[fill=fillColor,fill opacity=0.20] (101.83, 92.39) circle (  2.13);

\path[fill=fillColor,fill opacity=0.20] (138.75, 93.20) circle (  2.13);

\path[fill=fillColor,fill opacity=0.20] (101.17, 86.70) circle (  2.13);

\path[fill=fillColor,fill opacity=0.20] ( 93.96, 82.64) circle (  2.13);

\path[fill=fillColor,fill opacity=0.20] ( 83.69, 76.95) circle (  2.13);

\path[fill=fillColor,fill opacity=0.20] ( 83.47, 60.69) circle (  2.13);

\path[fill=fillColor,fill opacity=0.20] ( 88.28, 58.26) circle (  2.13);

\path[fill=fillColor,fill opacity=0.20] ( 55.51, 57.44) circle (  2.13);

\path[fill=fillColor,fill opacity=0.20] ( 68.83, 60.69) circle (  2.13);

\path[fill=fillColor,fill opacity=0.20] ( 77.36, 64.76) circle (  2.13);

\path[fill=fillColor,fill opacity=0.20] ( 77.79, 65.57) circle (  2.13);

\path[fill=fillColor,fill opacity=0.20] ( 81.94, 61.51) circle (  2.13);

\path[fill=fillColor,fill opacity=0.20] ( 78.88, 57.44) circle (  2.13);

\path[fill=fillColor,fill opacity=0.20] ( 75.61, 55.82) circle (  2.13);

\path[fill=fillColor,fill opacity=0.20] ( 71.02, 55.01) circle (  2.13);

\path[fill=fillColor,fill opacity=0.20] ( 85.22, 70.45) circle (  2.13);

\path[fill=fillColor,fill opacity=0.20] ( 91.78, 72.07) circle (  2.13);

\path[fill=fillColor,fill opacity=0.20] ( 97.68, 89.14) circle (  2.13);

\path[fill=fillColor,fill opacity=0.20] ( 98.11, 94.83) circle (  2.13);

\path[fill=fillColor,fill opacity=0.20] (102.92, 92.39) circle (  2.13);

\path[fill=fillColor,fill opacity=0.20] (104.23, 95.64) circle (  2.13);

\path[fill=fillColor,fill opacity=0.20] (111.66, 85.08) circle (  2.13);

\path[fill=fillColor,fill opacity=0.20] (105.32, 78.57) circle (  2.13);

\path[fill=fillColor,fill opacity=0.20] ( 98.77, 89.14) circle (  2.13);

\path[fill=fillColor,fill opacity=0.20] ( 99.86, 87.51) circle (  2.13);

\path[fill=fillColor,fill opacity=0.20] ( 94.40, 77.76) circle (  2.13);

\path[fill=fillColor,fill opacity=0.20] ( 81.07, 75.32) circle (  2.13);

\path[fill=fillColor,fill opacity=0.20] ( 73.42, 66.38) circle (  2.13);

\path[fill=fillColor,fill opacity=0.20] ( 57.91, 55.82) circle (  2.13);

\path[fill=fillColor,fill opacity=0.20] ( 51.57, 64.76) circle (  2.13);

\path[fill=fillColor,fill opacity=0.20] ( 93.09, 53.38) circle (  2.13);

\path[fill=fillColor,fill opacity=0.20] ( 80.20, 60.69) circle (  2.13);

\path[fill=fillColor,fill opacity=0.20] ( 81.73, 65.57) circle (  2.13);

\path[fill=fillColor,fill opacity=0.20] ( 75.83, 57.44) circle (  2.13);

\path[fill=fillColor,fill opacity=0.20] ( 74.51, 56.63) circle (  2.13);

\path[fill=fillColor,fill opacity=0.20] ( 69.05, 54.19) circle (  2.13);

\path[fill=fillColor,fill opacity=0.20] ( 68.18, 50.94) circle (  2.13);

\path[fill=fillColor,fill opacity=0.20] ( 74.73, 55.01) circle (  2.13);

\path[fill=fillColor,fill opacity=0.20] ( 74.51, 68.01) circle (  2.13);

\path[fill=fillColor,fill opacity=0.20] ( 83.47, 76.95) circle (  2.13);

\path[fill=fillColor,fill opacity=0.20] ( 93.96, 74.51) circle (  2.13);

\path[fill=fillColor,fill opacity=0.20] ( 97.24, 80.20) circle (  2.13);

\path[fill=fillColor,fill opacity=0.20] ( 95.49, 89.14) circle (  2.13);

\path[fill=fillColor,fill opacity=0.20] (100.08, 89.95) circle (  2.13);

\path[fill=fillColor,fill opacity=0.20] ( 98.33, 91.58) circle (  2.13);

\path[fill=fillColor,fill opacity=0.20] ( 99.21, 92.39) circle (  2.13);

\path[fill=fillColor,fill opacity=0.20] (111.22, 88.33) circle (  2.13);

\path[fill=fillColor,fill opacity=0.20] ( 92.87, 76.14) circle (  2.13);

\path[fill=fillColor,fill opacity=0.20] ( 86.10, 72.89) circle (  2.13);

\path[fill=fillColor,fill opacity=0.20] ( 77.14, 67.20) circle (  2.13);

\path[fill=fillColor,fill opacity=0.20] ( 70.36, 75.32) circle (  2.13);

\path[fill=fillColor,fill opacity=0.20] ( 46.98, 57.44) circle (  2.13);

\path[fill=fillColor,fill opacity=0.20] ( 82.82, 50.94) circle (  2.13);

\path[fill=fillColor,fill opacity=0.20] ( 82.60, 60.69) circle (  2.13);

\path[fill=fillColor,fill opacity=0.20] ( 76.70, 68.01) circle (  2.13);

\path[fill=fillColor,fill opacity=0.20] ( 74.73, 67.20) circle (  2.13);

\path[fill=fillColor,fill opacity=0.20] ( 71.89, 67.20) circle (  2.13);

\path[fill=fillColor,fill opacity=0.20] ( 72.11, 63.13) circle (  2.13);

\path[fill=fillColor,fill opacity=0.20] ( 71.67, 59.07) circle (  2.13);

\path[fill=fillColor,fill opacity=0.20] ( 69.49, 50.94) circle (  2.13);

\path[fill=fillColor,fill opacity=0.20] ( 70.14, 59.88) circle (  2.13);

\path[fill=fillColor,fill opacity=0.20] ( 79.32, 59.07) circle (  2.13);

\path[fill=fillColor,fill opacity=0.20] ( 85.22, 61.51) circle (  2.13);

\path[fill=fillColor,fill opacity=0.20] ( 88.94, 73.70) circle (  2.13);

\path[fill=fillColor,fill opacity=0.20] ( 96.80, 77.76) circle (  2.13);

\path[fill=fillColor,fill opacity=0.20] ( 98.55, 73.70) circle (  2.13);

\path[fill=fillColor,fill opacity=0.20] ( 96.80, 79.39) circle (  2.13);

\path[fill=fillColor,fill opacity=0.20] (100.52, 87.51) circle (  2.13);

\path[fill=fillColor,fill opacity=0.20] (102.48, 89.14) circle (  2.13);

\path[fill=fillColor,fill opacity=0.20] (104.67, 91.58) circle (  2.13);

\path[fill=fillColor,fill opacity=0.20] ( 95.49, 88.33) circle (  2.13);

\path[fill=fillColor,fill opacity=0.20] ( 88.94, 85.08) circle (  2.13);

\path[fill=fillColor,fill opacity=0.20] ( 89.59, 82.64) circle (  2.13);

\path[fill=fillColor,fill opacity=0.20] ( 87.62, 71.26) circle (  2.13);

\path[fill=fillColor,fill opacity=0.20] ( 74.95, 62.32) circle (  2.13);

\path[fill=fillColor,fill opacity=0.20] ( 52.01, 69.63) circle (  2.13);

\path[fill=fillColor,fill opacity=0.20] ( 52.01, 55.01) circle (  2.13);

\path[fill=fillColor,fill opacity=0.20] ( 69.05, 48.50) circle (  2.13);

\path[fill=fillColor,fill opacity=0.20] ( 90.25, 64.76) circle (  2.13);

\path[fill=fillColor,fill opacity=0.20] ( 78.23, 71.26) circle (  2.13);

\path[fill=fillColor,fill opacity=0.20] ( 74.51, 73.70) circle (  2.13);

\path[fill=fillColor,fill opacity=0.20] ( 82.16, 68.82) circle (  2.13);

\path[fill=fillColor,fill opacity=0.20] ( 79.98, 59.88) circle (  2.13);

\path[fill=fillColor,fill opacity=0.20] ( 70.14, 49.32) circle (  2.13);

\path[fill=fillColor,fill opacity=0.20] ( 62.50, 64.76) circle (  2.13);

\path[fill=fillColor,fill opacity=0.20] ( 58.56, 82.64) circle (  2.13);

\path[fill=fillColor,fill opacity=0.20] ( 71.89, 51.75) circle (  2.13);

\path[fill=fillColor,fill opacity=0.20] ( 77.14, 55.82) circle (  2.13);

\path[fill=fillColor,fill opacity=0.20] ( 78.88, 63.95) circle (  2.13);

\path[fill=fillColor,fill opacity=0.20] ( 89.59, 63.95) circle (  2.13);

\path[fill=fillColor,fill opacity=0.20] ( 98.99, 70.45) circle (  2.13);

\path[fill=fillColor,fill opacity=0.20] (101.61, 82.64) circle (  2.13);

\path[fill=fillColor,fill opacity=0.20] (105.10, 79.39) circle (  2.13);

\path[fill=fillColor,fill opacity=0.20] (115.16, 70.45) circle (  2.13);

\path[fill=fillColor,fill opacity=0.20] ( 93.09, 69.63) circle (  2.13);

\path[fill=fillColor,fill opacity=0.20] ( 91.12, 76.14) circle (  2.13);

\path[fill=fillColor,fill opacity=0.20] ( 83.91, 79.39) circle (  2.13);

\path[fill=fillColor,fill opacity=0.20] ( 83.04, 81.82) circle (  2.13);

\path[fill=fillColor,fill opacity=0.20] ( 78.67, 84.26) circle (  2.13);

\path[fill=fillColor,fill opacity=0.20] ( 70.36, 89.95) circle (  2.13);

\path[fill=fillColor,fill opacity=0.20] ( 61.62, 85.08) circle (  2.13);

\path[fill=fillColor,fill opacity=0.20] ( 59.88, 68.82) circle (  2.13);

\path[fill=fillColor,fill opacity=0.20] ( 45.67, 51.75) circle (  2.13);

\path[fill=fillColor,fill opacity=0.20] ( 75.17, 52.57) circle (  2.13);

\path[fill=fillColor,fill opacity=0.20] ( 75.83, 59.07) circle (  2.13);

\path[fill=fillColor,fill opacity=0.20] ( 77.57, 65.57) circle (  2.13);

\path[fill=fillColor,fill opacity=0.20] ( 77.57, 67.20) circle (  2.13);

\path[fill=fillColor,fill opacity=0.20] ( 76.26, 66.38) circle (  2.13);

\path[fill=fillColor,fill opacity=0.20] ( 80.41, 60.69) circle (  2.13);

\path[fill=fillColor,fill opacity=0.20] ( 80.63, 51.75) circle (  2.13);

\path[fill=fillColor,fill opacity=0.20] ( 70.58, 51.75) circle (  2.13);

\path[fill=fillColor,fill opacity=0.20] ( 66.43, 62.32) circle (  2.13);

\path[fill=fillColor,fill opacity=0.20] ( 68.40, 62.32) circle (  2.13);

\path[fill=fillColor,fill opacity=0.20] ( 66.87, 54.19) circle (  2.13);

\path[fill=fillColor,fill opacity=0.20] ( 71.89, 59.07) circle (  2.13);

\path[fill=fillColor,fill opacity=0.20] ( 72.11, 47.69) circle (  2.13);

\path[fill=fillColor,fill opacity=0.20] ( 74.95, 56.63) circle (  2.13);

\path[fill=fillColor,fill opacity=0.20] ( 78.01, 71.26) circle (  2.13);

\path[fill=fillColor,fill opacity=0.20] ( 87.41, 75.32) circle (  2.13);

\path[fill=fillColor,fill opacity=0.20] ( 97.24, 75.32) circle (  2.13);

\path[fill=fillColor,fill opacity=0.20] ( 93.52, 75.32) circle (  2.13);

\path[fill=fillColor,fill opacity=0.20] ( 91.78, 79.39) circle (  2.13);

\path[fill=fillColor,fill opacity=0.20] ( 92.21, 75.32) circle (  2.13);

\path[fill=fillColor,fill opacity=0.20] ( 92.87, 70.45) circle (  2.13);

\path[fill=fillColor,fill opacity=0.20] ( 87.84, 70.45) circle (  2.13);

\path[fill=fillColor,fill opacity=0.20] ( 78.45, 69.63) circle (  2.13);

\path[fill=fillColor,fill opacity=0.20] ( 68.18, 71.26) circle (  2.13);

\path[fill=fillColor,fill opacity=0.20] ( 65.12, 75.32) circle (  2.13);

\path[fill=fillColor,fill opacity=0.20] ( 59.22, 76.14) circle (  2.13);

\path[fill=fillColor,fill opacity=0.20] ( 59.66, 85.08) circle (  2.13);

\path[fill=fillColor,fill opacity=0.20] ( 74.08, 57.44) circle (  2.13);

\path[fill=fillColor,fill opacity=0.20] ( 76.26, 63.13) circle (  2.13);

\path[fill=fillColor,fill opacity=0.20] ( 70.36, 70.45) circle (  2.13);

\path[fill=fillColor,fill opacity=0.20] ( 79.98, 69.63) circle (  2.13);

\path[fill=fillColor,fill opacity=0.20] ( 83.04, 60.69) circle (  2.13);

\path[fill=fillColor,fill opacity=0.20] ( 79.54, 60.69) circle (  2.13);

\path[fill=fillColor,fill opacity=0.20] ( 75.39, 62.32) circle (  2.13);

\path[fill=fillColor,fill opacity=0.20] ( 75.39, 59.07) circle (  2.13);

\path[fill=fillColor,fill opacity=0.20] ( 77.79, 59.07) circle (  2.13);

\path[fill=fillColor,fill opacity=0.20] ( 74.73, 52.57) circle (  2.13);

\path[fill=fillColor,fill opacity=0.20] ( 66.43, 61.51) circle (  2.13);

\path[fill=fillColor,fill opacity=0.20] ( 77.36, 62.32) circle (  2.13);

\path[fill=fillColor,fill opacity=0.20] ( 79.76, 65.57) circle (  2.13);

\path[fill=fillColor,fill opacity=0.20] ( 88.94, 73.70) circle (  2.13);

\path[fill=fillColor,fill opacity=0.20] ( 98.11, 79.39) circle (  2.13);

\path[fill=fillColor,fill opacity=0.20] ( 89.59, 75.32) circle (  2.13);

\path[fill=fillColor,fill opacity=0.20] ( 89.81, 68.82) circle (  2.13);

\path[fill=fillColor,fill opacity=0.20] ( 91.12, 68.01) circle (  2.13);

\path[fill=fillColor,fill opacity=0.20] ( 75.83, 65.57) circle (  2.13);

\path[fill=fillColor,fill opacity=0.20] ( 67.96, 63.95) circle (  2.13);

\path[fill=fillColor,fill opacity=0.20] (103.36, 70.45) circle (  2.13);

\path[fill=fillColor,fill opacity=0.20] ( 60.31, 78.57) circle (  2.13);

\path[fill=fillColor,fill opacity=0.20] ( 58.13, 78.57) circle (  2.13);

\path[fill=fillColor,fill opacity=0.20] ( 83.04, 59.07) circle (  2.13);

\path[fill=fillColor,fill opacity=0.20] ( 69.71, 59.07) circle (  2.13);

\path[fill=fillColor,fill opacity=0.20] ( 71.02, 63.13) circle (  2.13);

\path[fill=fillColor,fill opacity=0.20] ( 78.01, 63.95) circle (  2.13);

\path[fill=fillColor,fill opacity=0.20] ( 75.39, 67.20) circle (  2.13);

\path[fill=fillColor,fill opacity=0.20] ( 74.95, 67.20) circle (  2.13);

\path[fill=fillColor,fill opacity=0.20] ( 73.20, 58.26) circle (  2.13);

\path[fill=fillColor,fill opacity=0.20] ( 75.61, 60.69) circle (  2.13);

\path[fill=fillColor,fill opacity=0.20] ( 76.92, 68.82) circle (  2.13);

\path[fill=fillColor,fill opacity=0.20] ( 77.36, 61.51) circle (  2.13);

\path[fill=fillColor,fill opacity=0.20] ( 72.77, 61.51) circle (  2.13);

\path[fill=fillColor,fill opacity=0.20] ( 68.62, 67.20) circle (  2.13);

\path[fill=fillColor,fill opacity=0.20] ( 78.88, 40.38) circle (  2.13);

\path[fill=fillColor,fill opacity=0.20] ( 72.77, 42.82) circle (  2.13);

\path[fill=fillColor,fill opacity=0.20] ( 71.89, 45.25) circle (  2.13);

\path[fill=fillColor,fill opacity=0.20] ( 65.56, 47.69) circle (  2.13);

\path[fill=fillColor,fill opacity=0.20] ( 68.40, 50.13) circle (  2.13);

\path[fill=fillColor,fill opacity=0.20] ( 67.30, 45.25) circle (  2.13);

\path[fill=fillColor,fill opacity=0.20] ( 62.06, 46.88) circle (  2.13);

\path[fill=fillColor,fill opacity=0.20] ( 63.59, 50.94) circle (  2.13);

\path[fill=fillColor,fill opacity=0.20] ( 68.83, 50.94) circle (  2.13);

\path[fill=fillColor,fill opacity=0.20] ( 78.67, 52.57) circle (  2.13);

\path[fill=fillColor,fill opacity=0.20] ( 80.85, 67.20) circle (  2.13);

\path[fill=fillColor,fill opacity=0.20] ( 78.45, 62.32) circle (  2.13);

\path[fill=fillColor,fill opacity=0.20] ( 74.51, 65.57) circle (  2.13);

\path[fill=fillColor,fill opacity=0.20] ( 72.77, 60.69) circle (  2.13);

\path[fill=fillColor,fill opacity=0.20] ( 70.14, 51.75) circle (  2.13);

\path[fill=fillColor,fill opacity=0.20] (106.85, 55.01) circle (  2.13);

\path[fill=fillColor,fill opacity=0.20] ( 64.68, 63.95) circle (  2.13);

\path[fill=fillColor,fill opacity=0.20] ( 57.03, 70.45) circle (  2.13);

\path[fill=fillColor,fill opacity=0.20] ( 51.35, 81.82) circle (  2.13);

\path[fill=fillColor,fill opacity=0.20] ( 66.65, 57.44) circle (  2.13);

\path[fill=fillColor,fill opacity=0.20] ( 61.84, 53.38) circle (  2.13);

\path[fill=fillColor,fill opacity=0.20] ( 67.52, 59.07) circle (  2.13);

\path[fill=fillColor,fill opacity=0.20] ( 69.49, 59.88) circle (  2.13);

\path[fill=fillColor,fill opacity=0.20] ( 71.46, 56.63) circle (  2.13);

\path[fill=fillColor,fill opacity=0.20] ( 74.08, 63.13) circle (  2.13);

\path[fill=fillColor,fill opacity=0.20] ( 77.57, 74.51) circle (  2.13);

\path[fill=fillColor,fill opacity=0.20] ( 78.01, 72.89) circle (  2.13);

\path[fill=fillColor,fill opacity=0.20] ( 76.70, 63.13) circle (  2.13);

\path[fill=fillColor,fill opacity=0.20] ( 79.32, 65.57) circle (  2.13);

\path[fill=fillColor,fill opacity=0.20] ( 71.46, 67.20) circle (  2.13);

\path[fill=fillColor,fill opacity=0.20] ( 72.99, 54.19) circle (  2.13);

\path[fill=fillColor,fill opacity=0.20] ( 81.07, 46.07) circle (  2.13);

\path[fill=fillColor,fill opacity=0.20] ( 78.01, 44.44) circle (  2.13);

\path[fill=fillColor,fill opacity=0.20] ( 75.61, 42.00) circle (  2.13);

\path[fill=fillColor,fill opacity=0.20] ( 76.92, 50.94) circle (  2.13);

\path[fill=fillColor,fill opacity=0.20] ( 88.28, 56.63) circle (  2.13);

\path[fill=fillColor,fill opacity=0.20] ( 86.75, 42.00) circle (  2.13);

\path[fill=fillColor,fill opacity=0.20] ( 80.41, 44.44) circle (  2.13);

\path[fill=fillColor,fill opacity=0.20] ( 74.08, 49.32) circle (  2.13);

\path[fill=fillColor,fill opacity=0.20] ( 71.89, 49.32) circle (  2.13);

\path[fill=fillColor,fill opacity=0.20] ( 71.24, 52.57) circle (  2.13);

\path[fill=fillColor,fill opacity=0.20] ( 70.14, 55.01) circle (  2.13);

\path[fill=fillColor,fill opacity=0.20] ( 72.55, 56.63) circle (  2.13);

\path[fill=fillColor,fill opacity=0.20] ( 91.56, 62.32) circle (  2.13);

\path[fill=fillColor,fill opacity=0.20] ( 85.00, 63.13) circle (  2.13);

\path[fill=fillColor,fill opacity=0.20] ( 66.43, 49.32) circle (  2.13);

\path[fill=fillColor,fill opacity=0.20] ( 69.05, 49.32) circle (  2.13);

\path[fill=fillColor,fill opacity=0.20] ( 69.05, 59.07) circle (  2.13);

\path[fill=fillColor,fill opacity=0.20] ( 49.17, 67.20) circle (  2.13);

\path[fill=fillColor,fill opacity=0.20] ( 55.72, 79.39) circle (  2.13);

\path[fill=fillColor,fill opacity=0.20] ( 54.63, 56.63) circle (  2.13);

\path[fill=fillColor,fill opacity=0.20] ( 67.52, 52.57) circle (  2.13);

\path[fill=fillColor,fill opacity=0.20] ( 65.56, 64.76) circle (  2.13);

\path[fill=fillColor,fill opacity=0.20] ( 69.27, 70.45) circle (  2.13);

\path[fill=fillColor,fill opacity=0.20] ( 68.83, 65.57) circle (  2.13);

\path[fill=fillColor,fill opacity=0.20] ( 67.09, 55.82) circle (  2.13);

\path[fill=fillColor,fill opacity=0.20] ( 73.20, 63.13) circle (  2.13);

\path[fill=fillColor,fill opacity=0.20] ( 83.04, 72.07) circle (  2.13);

\path[fill=fillColor,fill opacity=0.20] ( 75.17, 55.82) circle (  2.13);

\path[fill=fillColor,fill opacity=0.20] ( 79.54, 46.07) circle (  2.13);

\path[fill=fillColor,fill opacity=0.20] ( 87.41, 57.44) circle (  2.13);

\path[fill=fillColor,fill opacity=0.20] ( 74.95, 58.26) circle (  2.13);

\path[fill=fillColor,fill opacity=0.20] ( 70.58, 58.26) circle (  2.13);

\path[fill=fillColor,fill opacity=0.20] ( 76.48, 65.57) circle (  2.13);

\path[fill=fillColor,fill opacity=0.20] ( 76.48, 56.63) circle (  2.13);

\path[fill=fillColor,fill opacity=0.20] ( 74.30, 50.94) circle (  2.13);

\path[fill=fillColor,fill opacity=0.20] ( 72.99, 64.76) circle (  2.13);

\path[fill=fillColor,fill opacity=0.20] ( 75.61, 59.07) circle (  2.13);

\path[fill=fillColor,fill opacity=0.20] ( 78.01, 69.63) circle (  2.13);

\path[fill=fillColor,fill opacity=0.20] ( 78.67, 75.32) circle (  2.13);

\path[fill=fillColor,fill opacity=0.20] ( 76.04, 60.69) circle (  2.13);

\path[fill=fillColor,fill opacity=0.20] ( 78.23, 46.88) circle (  2.13);

\path[fill=fillColor,fill opacity=0.20] ( 80.85, 38.75) circle (  2.13);

\path[fill=fillColor,fill opacity=0.20] ( 66.43, 51.75) circle (  2.13);

\path[fill=fillColor,fill opacity=0.20] ( 55.51, 57.44) circle (  2.13);

\path[fill=fillColor,fill opacity=0.20] ( 62.93, 51.75) circle (  2.13);

\path[fill=fillColor,fill opacity=0.20] ( 59.22, 65.57) circle (  2.13);

\path[fill=fillColor,fill opacity=0.20] ( 60.75, 61.51) circle (  2.13);

\path[fill=fillColor,fill opacity=0.20] ( 62.50, 56.63) circle (  2.13);

\path[fill=fillColor,fill opacity=0.20] ( 64.90, 46.88) circle (  2.13);

\path[fill=fillColor,fill opacity=0.20] ( 62.93, 40.38) circle (  2.13);

\path[fill=fillColor,fill opacity=0.20] ( 67.74, 46.07) circle (  2.13);

\path[fill=fillColor,fill opacity=0.20] ( 70.80, 63.13) circle (  2.13);

\path[fill=fillColor,fill opacity=0.20] ( 73.20, 63.13) circle (  2.13);

\path[fill=fillColor,fill opacity=0.20] ( 71.46, 59.88) circle (  2.13);

\path[fill=fillColor,fill opacity=0.20] ( 75.17, 71.26) circle (  2.13);

\path[fill=fillColor,fill opacity=0.20] ( 75.39, 72.07) circle (  2.13);

\path[fill=fillColor,fill opacity=0.20] ( 80.41, 69.63) circle (  2.13);

\path[fill=fillColor,fill opacity=0.20] ( 77.36, 57.44) circle (  2.13);

\path[fill=fillColor,fill opacity=0.20] ( 85.88, 63.95) circle (  2.13);

\path[fill=fillColor,fill opacity=0.20] ( 84.57, 54.19) circle (  2.13);

\path[fill=fillColor,fill opacity=0.20] ( 88.50, 43.63) circle (  2.13);

\path[fill=fillColor,fill opacity=0.20] ( 81.51, 39.56) circle (  2.13);

\path[fill=fillColor,fill opacity=0.20] ( 88.72, 45.25) circle (  2.13);

\path[fill=fillColor,fill opacity=0.20] ( 74.51, 56.63) circle (  2.13);

\path[fill=fillColor,fill opacity=0.20] ( 58.13, 55.01) circle (  2.13);

\path[fill=fillColor,fill opacity=0.20] ( 53.10, 46.07) circle (  2.13);

\path[fill=fillColor,fill opacity=0.20] ( 53.76, 51.75) circle (  2.13);

\path[fill=fillColor,fill opacity=0.20] ( 59.00, 46.07) circle (  2.13);

\path[fill=fillColor,fill opacity=0.20] ( 61.62, 46.88) circle (  2.13);

\path[fill=fillColor,fill opacity=0.20] ( 69.05, 46.07) circle (  2.13);

\path[fill=fillColor,fill opacity=0.20] ( 67.74, 54.19) circle (  2.13);

\path[fill=fillColor,fill opacity=0.20] ( 82.38, 54.19) circle (  2.13);

\path[fill=fillColor,fill opacity=0.20] ( 80.85, 40.38) circle (  2.13);

\path[fill=fillColor,fill opacity=0.20] ( 76.70, 65.57) circle (  2.13);

\path[fill=fillColor,fill opacity=0.20] ( 81.29, 43.63) circle (  2.13);

\path[fill=fillColor,fill opacity=0.20] ( 70.36, 41.19) circle (  2.13);

\path[fill=fillColor,fill opacity=0.20] ( 64.68, 46.07) circle (  2.13);

\path[fill=fillColor,fill opacity=0.20] ( 79.54, 38.75) circle (  2.13);

\path[fill=fillColor,fill opacity=0.20] ( 60.31, 43.63) circle (  2.13);

\path[fill=fillColor,fill opacity=0.20] ( 55.29, 57.44) circle (  2.13);

\path[fill=fillColor,fill opacity=0.20] ( 53.98, 70.45) circle (  2.13);

\path[fill=fillColor,fill opacity=0.20] ( 55.29, 52.57) circle (  2.13);

\path[fill=fillColor,fill opacity=0.20] ( 69.27, 47.69) circle (  2.13);

\path[fill=fillColor,fill opacity=0.20] ( 68.18, 40.38) circle (  2.13);

\path[fill=fillColor,fill opacity=0.20] ( 86.97, 40.38) circle (  2.13);

\path[fill=fillColor,fill opacity=0.20] ( 83.91, 48.50) circle (  2.13);

\path[fill=fillColor,fill opacity=0.20] ( 49.61, 42.00) circle (  2.13);

\path[fill=fillColor,fill opacity=0.20] ( 46.77, 51.75) circle (  2.13);

\path[fill=fillColor,fill opacity=0.20] ( 58.35, 50.94) circle (  2.13);

\path[fill=fillColor,fill opacity=0.20] ( 62.50, 41.19) circle (  2.13);

\path[fill=fillColor,fill opacity=0.20] ( 78.01, 41.19) circle (  2.13);

\path[fill=fillColor,fill opacity=0.20] ( 65.99, 52.57) circle (  2.13);

\path[fill=fillColor,fill opacity=0.20] ( 69.49, 54.19) circle (  2.13);

\path[fill=fillColor,fill opacity=0.20] ( 78.45, 44.44) circle (  2.13);

\path[fill=fillColor,fill opacity=0.20] ( 48.95, 49.32) circle (  2.13);

\path[fill=fillColor,fill opacity=0.20] (104.45, 74.51) circle (  2.13);
\end{scope}
\begin{scope}
\path[clip] (159.87, 34.04) rectangle (277.04,119.86);
\definecolor[named]{fillColor}{rgb}{0.90,0.90,0.90}

\path[fill=fillColor] (159.87, 34.04) rectangle (277.03,119.86);
\definecolor[named]{drawColor}{rgb}{0.95,0.95,0.95}

\path[draw=drawColor,line width= 0.3pt,line join=round,line cap=round] (159.87, 40.38) --
	(277.04, 40.38);

\path[draw=drawColor,line width= 0.3pt,line join=round,line cap=round] (159.87, 56.63) --
	(277.04, 56.63);

\path[draw=drawColor,line width= 0.3pt,line join=round,line cap=round] (159.87, 72.89) --
	(277.04, 72.89);

\path[draw=drawColor,line width= 0.3pt,line join=round,line cap=round] (159.87, 89.14) --
	(277.04, 89.14);

\path[draw=drawColor,line width= 0.3pt,line join=round,line cap=round] (159.87,105.39) --
	(277.04,105.39);

\path[draw=drawColor,line width= 0.3pt,line join=round,line cap=round] (169.78, 34.04) --
	(169.78,119.86);

\path[draw=drawColor,line width= 0.3pt,line join=round,line cap=round] (191.63, 34.04) --
	(191.63,119.86);

\path[draw=drawColor,line width= 0.3pt,line join=round,line cap=round] (213.48, 34.04) --
	(213.48,119.86);

\path[draw=drawColor,line width= 0.3pt,line join=round,line cap=round] (235.33, 34.04) --
	(235.33,119.86);

\path[draw=drawColor,line width= 0.3pt,line join=round,line cap=round] (257.18, 34.04) --
	(257.18,119.86);
\definecolor[named]{drawColor}{rgb}{1.00,1.00,1.00}

\path[draw=drawColor,line width= 0.6pt,line join=round,line cap=round] (159.87, 48.50) --
	(277.04, 48.50);

\path[draw=drawColor,line width= 0.6pt,line join=round,line cap=round] (159.87, 64.76) --
	(277.04, 64.76);

\path[draw=drawColor,line width= 0.6pt,line join=round,line cap=round] (159.87, 81.01) --
	(277.04, 81.01);

\path[draw=drawColor,line width= 0.6pt,line join=round,line cap=round] (159.87, 97.27) --
	(277.04, 97.27);

\path[draw=drawColor,line width= 0.6pt,line join=round,line cap=round] (159.87,113.52) --
	(277.04,113.52);

\path[draw=drawColor,line width= 0.6pt,line join=round,line cap=round] (180.71, 34.04) --
	(180.71,119.86);

\path[draw=drawColor,line width= 0.6pt,line join=round,line cap=round] (202.56, 34.04) --
	(202.56,119.86);

\path[draw=drawColor,line width= 0.6pt,line join=round,line cap=round] (224.41, 34.04) --
	(224.41,119.86);

\path[draw=drawColor,line width= 0.6pt,line join=round,line cap=round] (246.26, 34.04) --
	(246.26,119.86);

\path[draw=drawColor,line width= 0.6pt,line join=round,line cap=round] (268.11, 34.04) --
	(268.11,119.86);
\definecolor[named]{fillColor}{rgb}{0.00,0.00,0.00}

\path[fill=fillColor,fill opacity=0.20] (185.95, 55.01) circle (  2.13);

\path[fill=fillColor,fill opacity=0.20] (193.16, 59.07) circle (  2.13);

\path[fill=fillColor,fill opacity=0.20] (191.41, 56.63) circle (  2.13);

\path[fill=fillColor,fill opacity=0.20] (196.88, 47.69) circle (  2.13);

\path[fill=fillColor,fill opacity=0.20] (186.61, 42.82) circle (  2.13);

\path[fill=fillColor,fill opacity=0.20] (177.65, 49.32) circle (  2.13);

\path[fill=fillColor,fill opacity=0.20] (187.48, 60.69) circle (  2.13);

\path[fill=fillColor,fill opacity=0.20] (202.56, 60.69) circle (  2.13);

\path[fill=fillColor,fill opacity=0.20] (213.05, 68.82) circle (  2.13);

\path[fill=fillColor,fill opacity=0.20] (214.14, 71.26) circle (  2.13);

\path[fill=fillColor,fill opacity=0.20] (207.58, 60.69) circle (  2.13);

\path[fill=fillColor,fill opacity=0.20] (209.99, 50.94) circle (  2.13);

\path[fill=fillColor,fill opacity=0.20] (210.42, 51.75) circle (  2.13);

\path[fill=fillColor,fill opacity=0.20] (200.15, 54.19) circle (  2.13);

\path[fill=fillColor,fill opacity=0.20] (186.83, 48.50) circle (  2.13);

\path[fill=fillColor,fill opacity=0.20] (177.21, 40.38) circle (  2.13);

\path[fill=fillColor,fill opacity=0.20] (194.69, 72.89) circle (  2.13);

\path[fill=fillColor,fill opacity=0.20] (211.30, 76.95) circle (  2.13);

\path[fill=fillColor,fill opacity=0.20] (222.00, 76.14) circle (  2.13);

\path[fill=fillColor,fill opacity=0.20] (219.38, 75.32) circle (  2.13);

\path[fill=fillColor,fill opacity=0.20] (211.52, 71.26) circle (  2.13);

\path[fill=fillColor,fill opacity=0.20] (206.93, 66.38) circle (  2.13);

\path[fill=fillColor,fill opacity=0.20] (202.12, 66.38) circle (  2.13);

\path[fill=fillColor,fill opacity=0.20] (202.56, 66.38) circle (  2.13);

\path[fill=fillColor,fill opacity=0.20] (201.47, 62.32) circle (  2.13);

\path[fill=fillColor,fill opacity=0.20] (194.91, 55.82) circle (  2.13);

\path[fill=fillColor,fill opacity=0.20] (188.14, 45.25) circle (  2.13);

\path[fill=fillColor,fill opacity=0.20] (189.67, 63.13) circle (  2.13);

\path[fill=fillColor,fill opacity=0.20] (206.27, 71.26) circle (  2.13);

\path[fill=fillColor,fill opacity=0.20] (220.69, 78.57) circle (  2.13);

\path[fill=fillColor,fill opacity=0.20] (221.13, 85.08) circle (  2.13);

\path[fill=fillColor,fill opacity=0.20] (216.54, 82.64) circle (  2.13);

\path[fill=fillColor,fill opacity=0.20] (216.54, 76.95) circle (  2.13);

\path[fill=fillColor,fill opacity=0.20] (213.70, 76.14) circle (  2.13);

\path[fill=fillColor,fill opacity=0.20] (205.62, 76.95) circle (  2.13);

\path[fill=fillColor,fill opacity=0.20] (203.00, 72.89) circle (  2.13);

\path[fill=fillColor,fill opacity=0.20] (214.79, 62.32) circle (  2.13);

\path[fill=fillColor,fill opacity=0.20] (214.79, 68.82) circle (  2.13);

\path[fill=fillColor,fill opacity=0.20] (200.81, 77.76) circle (  2.13);

\path[fill=fillColor,fill opacity=0.20] (182.24, 60.69) circle (  2.13);

\path[fill=fillColor,fill opacity=0.20] (195.57, 63.13) circle (  2.13);

\path[fill=fillColor,fill opacity=0.20] (215.67, 63.13) circle (  2.13);

\path[fill=fillColor,fill opacity=0.20] (222.22, 72.89) circle (  2.13);

\path[fill=fillColor,fill opacity=0.20] (215.67, 89.14) circle (  2.13);

\path[fill=fillColor,fill opacity=0.20] (218.07, 93.20) circle (  2.13);

\path[fill=fillColor,fill opacity=0.20] (219.60, 88.33) circle (  2.13);

\path[fill=fillColor,fill opacity=0.20] (209.33, 81.01) circle (  2.13);

\path[fill=fillColor,fill opacity=0.20] (204.96, 76.14) circle (  2.13);

\path[fill=fillColor,fill opacity=0.20] (199.50, 75.32) circle (  2.13);

\path[fill=fillColor,fill opacity=0.20] (215.01, 72.89) circle (  2.13);

\path[fill=fillColor,fill opacity=0.20] (215.23, 81.82) circle (  2.13);

\path[fill=fillColor,fill opacity=0.20] (208.46, 82.64) circle (  2.13);

\path[fill=fillColor,fill opacity=0.20] (204.96, 76.14) circle (  2.13);

\path[fill=fillColor,fill opacity=0.20] (201.90, 67.20) circle (  2.13);

\path[fill=fillColor,fill opacity=0.20] (195.57, 64.76) circle (  2.13);

\path[fill=fillColor,fill opacity=0.20] (184.64, 64.76) circle (  2.13);

\path[fill=fillColor,fill opacity=0.20] (197.97, 66.38) circle (  2.13);

\path[fill=fillColor,fill opacity=0.20] (215.89, 72.07) circle (  2.13);

\path[fill=fillColor,fill opacity=0.20] (218.73, 80.20) circle (  2.13);

\path[fill=fillColor,fill opacity=0.20] (213.92, 90.76) circle (  2.13);

\path[fill=fillColor,fill opacity=0.20] (211.74, 94.83) circle (  2.13);

\path[fill=fillColor,fill opacity=0.20] (210.86, 91.58) circle (  2.13);

\path[fill=fillColor,fill opacity=0.20] (206.71, 80.20) circle (  2.13);

\path[fill=fillColor,fill opacity=0.20] (203.65, 75.32) circle (  2.13);

\path[fill=fillColor,fill opacity=0.20] (190.32, 79.39) circle (  2.13);

\path[fill=fillColor,fill opacity=0.20] (209.77, 75.32) circle (  2.13);

\path[fill=fillColor,fill opacity=0.20] (221.13, 80.20) circle (  2.13);

\path[fill=fillColor,fill opacity=0.20] (216.54, 91.58) circle (  2.13);

\path[fill=fillColor,fill opacity=0.20] (215.45, 88.33) circle (  2.13);

\path[fill=fillColor,fill opacity=0.20] (219.60, 79.39) circle (  2.13);

\path[fill=fillColor,fill opacity=0.20] (213.05, 72.07) circle (  2.13);

\path[fill=fillColor,fill opacity=0.20] (210.64, 63.95) circle (  2.13);

\path[fill=fillColor,fill opacity=0.20] (208.68, 56.63) circle (  2.13);

\path[fill=fillColor,fill opacity=0.20] (185.52, 69.63) circle (  2.13);

\path[fill=fillColor,fill opacity=0.20] (199.50, 70.45) circle (  2.13);

\path[fill=fillColor,fill opacity=0.20] (215.45, 85.89) circle (  2.13);

\path[fill=fillColor,fill opacity=0.20] (215.89, 92.39) circle (  2.13);

\path[fill=fillColor,fill opacity=0.20] (213.48, 92.39) circle (  2.13);

\path[fill=fillColor,fill opacity=0.20] (211.74, 92.39) circle (  2.13);

\path[fill=fillColor,fill opacity=0.20] (206.71, 90.76) circle (  2.13);

\path[fill=fillColor,fill opacity=0.20] (206.93, 82.64) circle (  2.13);

\path[fill=fillColor,fill opacity=0.20] (201.68, 75.32) circle (  2.13);

\path[fill=fillColor,fill opacity=0.20] (214.79, 77.76) circle (  2.13);

\path[fill=fillColor,fill opacity=0.20] (220.26, 85.08) circle (  2.13);

\path[fill=fillColor,fill opacity=0.20] (218.95, 88.33) circle (  2.13);

\path[fill=fillColor,fill opacity=0.20] (222.00, 78.57) circle (  2.13);

\path[fill=fillColor,fill opacity=0.20] (218.95, 73.70) circle (  2.13);

\path[fill=fillColor,fill opacity=0.20] (217.42, 72.07) circle (  2.13);

\path[fill=fillColor,fill opacity=0.20] (218.51, 64.76) circle (  2.13);

\path[fill=fillColor,fill opacity=0.20] (206.93, 52.57) circle (  2.13);

\path[fill=fillColor,fill opacity=0.20] (199.72, 70.45) circle (  2.13);

\path[fill=fillColor,fill opacity=0.20] (213.26, 85.89) circle (  2.13);

\path[fill=fillColor,fill opacity=0.20] (213.70, 94.83) circle (  2.13);

\path[fill=fillColor,fill opacity=0.20] (211.74, 94.83) circle (  2.13);

\path[fill=fillColor,fill opacity=0.20] (215.23, 92.39) circle (  2.13);

\path[fill=fillColor,fill opacity=0.20] (210.64, 88.33) circle (  2.13);

\path[fill=fillColor,fill opacity=0.20] (208.46, 82.64) circle (  2.13);

\path[fill=fillColor,fill opacity=0.20] (201.25, 73.70) circle (  2.13);

\path[fill=fillColor,fill opacity=0.20] (206.05, 88.33) circle (  2.13);

\path[fill=fillColor,fill opacity=0.20] (227.03, 73.70) circle (  2.13);

\path[fill=fillColor,fill opacity=0.20] (219.38, 78.57) circle (  2.13);

\path[fill=fillColor,fill opacity=0.20] (224.19, 73.70) circle (  2.13);

\path[fill=fillColor,fill opacity=0.20] (223.10, 71.26) circle (  2.13);

\path[fill=fillColor,fill opacity=0.20] (213.05, 76.14) circle (  2.13);

\path[fill=fillColor,fill opacity=0.20] (212.61, 71.26) circle (  2.13);

\path[fill=fillColor,fill opacity=0.20] (206.71, 61.51) circle (  2.13);

\path[fill=fillColor,fill opacity=0.20] (203.21, 54.19) circle (  2.13);

\path[fill=fillColor,fill opacity=0.20] (196.88, 42.82) circle (  2.13);

\path[fill=fillColor,fill opacity=0.20] (199.94, 68.01) circle (  2.13);

\path[fill=fillColor,fill opacity=0.20] (209.33, 77.76) circle (  2.13);

\path[fill=fillColor,fill opacity=0.20] (212.39, 89.14) circle (  2.13);

\path[fill=fillColor,fill opacity=0.20] (211.30, 94.02) circle (  2.13);

\path[fill=fillColor,fill opacity=0.20] (215.23, 89.14) circle (  2.13);

\path[fill=fillColor,fill opacity=0.20] (214.14, 79.39) circle (  2.13);

\path[fill=fillColor,fill opacity=0.20] (206.49, 74.51) circle (  2.13);

\path[fill=fillColor,fill opacity=0.20] (201.68, 71.26) circle (  2.13);

\path[fill=fillColor,fill opacity=0.20] (189.89, 66.38) circle (  2.13);

\path[fill=fillColor,fill opacity=0.20] (224.85, 70.45) circle (  2.13);

\path[fill=fillColor,fill opacity=0.20] (232.27, 67.20) circle (  2.13);

\path[fill=fillColor,fill opacity=0.20] (218.51, 76.14) circle (  2.13);

\path[fill=fillColor,fill opacity=0.20] (219.82, 69.63) circle (  2.13);

\path[fill=fillColor,fill opacity=0.20] (215.01, 71.26) circle (  2.13);

\path[fill=fillColor,fill opacity=0.20] (207.37, 76.95) circle (  2.13);

\path[fill=fillColor,fill opacity=0.20] (211.30, 64.76) circle (  2.13);

\path[fill=fillColor,fill opacity=0.20] (204.52, 57.44) circle (  2.13);

\path[fill=fillColor,fill opacity=0.20] (199.94, 57.44) circle (  2.13);

\path[fill=fillColor,fill opacity=0.20] (194.04, 46.07) circle (  2.13);

\path[fill=fillColor,fill opacity=0.20] (195.78, 65.57) circle (  2.13);

\path[fill=fillColor,fill opacity=0.20] (210.42, 68.01) circle (  2.13);

\path[fill=fillColor,fill opacity=0.20] (217.85, 78.57) circle (  2.13);

\path[fill=fillColor,fill opacity=0.20] (215.89, 84.26) circle (  2.13);

\path[fill=fillColor,fill opacity=0.20] (216.98, 80.20) circle (  2.13);

\path[fill=fillColor,fill opacity=0.20] (215.23, 73.70) circle (  2.13);

\path[fill=fillColor,fill opacity=0.20] (205.62, 73.70) circle (  2.13);

\path[fill=fillColor,fill opacity=0.20] (199.06, 71.26) circle (  2.13);

\path[fill=fillColor,fill opacity=0.20] (195.57, 60.69) circle (  2.13);

\path[fill=fillColor,fill opacity=0.20] (176.99, 67.20) circle (  2.13);

\path[fill=fillColor,fill opacity=0.20] (185.73,111.89) circle (  2.13);

\path[fill=fillColor,fill opacity=0.20] (240.58, 63.13) circle (  2.13);

\path[fill=fillColor,fill opacity=0.20] (222.00, 68.82) circle (  2.13);

\path[fill=fillColor,fill opacity=0.20] (214.36, 76.14) circle (  2.13);

\path[fill=fillColor,fill opacity=0.20] (217.20, 72.89) circle (  2.13);

\path[fill=fillColor,fill opacity=0.20] (216.76, 74.51) circle (  2.13);

\path[fill=fillColor,fill opacity=0.20] (215.01, 70.45) circle (  2.13);

\path[fill=fillColor,fill opacity=0.20] (214.58, 59.07) circle (  2.13);

\path[fill=fillColor,fill opacity=0.20] (209.33, 56.63) circle (  2.13);

\path[fill=fillColor,fill opacity=0.20] (201.03, 55.01) circle (  2.13);

\path[fill=fillColor,fill opacity=0.20] (204.09, 42.00) circle (  2.13);

\path[fill=fillColor,fill opacity=0.20] (183.11, 39.56) circle (  2.13);

\path[fill=fillColor,fill opacity=0.20] (210.86, 58.26) circle (  2.13);

\path[fill=fillColor,fill opacity=0.20] (221.35, 67.20) circle (  2.13);

\path[fill=fillColor,fill opacity=0.20] (221.79, 68.82) circle (  2.13);

\path[fill=fillColor,fill opacity=0.20] (216.76, 68.82) circle (  2.13);

\path[fill=fillColor,fill opacity=0.20] (213.26, 76.95) circle (  2.13);

\path[fill=fillColor,fill opacity=0.20] (208.89, 79.39) circle (  2.13);

\path[fill=fillColor,fill opacity=0.20] (200.59, 72.89) circle (  2.13);

\path[fill=fillColor,fill opacity=0.20] (201.03, 65.57) circle (  2.13);

\path[fill=fillColor,fill opacity=0.20] (190.76, 63.95) circle (  2.13);

\path[fill=fillColor,fill opacity=0.20] (204.96, 82.64) circle (  2.13);

\path[fill=fillColor,fill opacity=0.20] (221.79, 71.26) circle (  2.13);

\path[fill=fillColor,fill opacity=0.20] (213.48, 72.89) circle (  2.13);

\path[fill=fillColor,fill opacity=0.20] (214.79, 69.63) circle (  2.13);

\path[fill=fillColor,fill opacity=0.20] (218.07, 72.07) circle (  2.13);

\path[fill=fillColor,fill opacity=0.20] (214.36, 74.51) circle (  2.13);

\path[fill=fillColor,fill opacity=0.20] (211.30, 68.01) circle (  2.13);

\path[fill=fillColor,fill opacity=0.20] (209.77, 62.32) circle (  2.13);

\path[fill=fillColor,fill opacity=0.20] (207.15, 59.88) circle (  2.13);

\path[fill=fillColor,fill opacity=0.20] (208.68, 50.94) circle (  2.13);

\path[fill=fillColor,fill opacity=0.20] (198.19, 38.75) circle (  2.13);

\path[fill=fillColor,fill opacity=0.20] (180.71, 42.82) circle (  2.13);

\path[fill=fillColor,fill opacity=0.20] (198.63, 51.75) circle (  2.13);

\path[fill=fillColor,fill opacity=0.20] (208.89, 55.82) circle (  2.13);

\path[fill=fillColor,fill opacity=0.20] (218.95, 59.07) circle (  2.13);

\path[fill=fillColor,fill opacity=0.20] (214.14, 66.38) circle (  2.13);

\path[fill=fillColor,fill opacity=0.20] (210.64, 77.76) circle (  2.13);

\path[fill=fillColor,fill opacity=0.20] (215.89, 76.95) circle (  2.13);

\path[fill=fillColor,fill opacity=0.20] (209.33, 73.70) circle (  2.13);

\path[fill=fillColor,fill opacity=0.20] (204.09, 72.89) circle (  2.13);

\path[fill=fillColor,fill opacity=0.20] (197.31, 70.45) circle (  2.13);

\path[fill=fillColor,fill opacity=0.20] (186.83, 95.64) circle (  2.13);

\path[fill=fillColor,fill opacity=0.20] (211.52, 68.01) circle (  2.13);

\path[fill=fillColor,fill opacity=0.20] (208.89, 72.89) circle (  2.13);

\path[fill=fillColor,fill opacity=0.20] (209.33, 68.82) circle (  2.13);

\path[fill=fillColor,fill opacity=0.20] (221.35, 65.57) circle (  2.13);

\path[fill=fillColor,fill opacity=0.20] (214.79, 70.45) circle (  2.13);

\path[fill=fillColor,fill opacity=0.20] (218.07, 72.07) circle (  2.13);

\path[fill=fillColor,fill opacity=0.20] (218.73, 71.26) circle (  2.13);

\path[fill=fillColor,fill opacity=0.20] (210.21, 70.45) circle (  2.13);

\path[fill=fillColor,fill opacity=0.20] (206.71, 61.51) circle (  2.13);

\path[fill=fillColor,fill opacity=0.20] (205.84, 47.69) circle (  2.13);

\path[fill=fillColor,fill opacity=0.20] (194.91, 44.44) circle (  2.13);

\path[fill=fillColor,fill opacity=0.20] (198.41, 52.57) circle (  2.13);

\path[fill=fillColor,fill opacity=0.20] (213.05, 52.57) circle (  2.13);

\path[fill=fillColor,fill opacity=0.20] (212.83, 63.13) circle (  2.13);

\path[fill=fillColor,fill opacity=0.20] (209.33, 70.45) circle (  2.13);

\path[fill=fillColor,fill opacity=0.20] (214.58, 72.07) circle (  2.13);

\path[fill=fillColor,fill opacity=0.20] (216.76, 75.32) circle (  2.13);

\path[fill=fillColor,fill opacity=0.20] (206.93, 76.14) circle (  2.13);

\path[fill=fillColor,fill opacity=0.20] (201.25, 72.89) circle (  2.13);

\path[fill=fillColor,fill opacity=0.20] (187.48, 69.63) circle (  2.13);

\path[fill=fillColor,fill opacity=0.20] (208.46, 75.32) circle (  2.13);

\path[fill=fillColor,fill opacity=0.20] (206.49, 70.45) circle (  2.13);

\path[fill=fillColor,fill opacity=0.20] (209.11, 71.26) circle (  2.13);

\path[fill=fillColor,fill opacity=0.20] (213.48, 66.38) circle (  2.13);

\path[fill=fillColor,fill opacity=0.20] (217.42, 67.20) circle (  2.13);

\path[fill=fillColor,fill opacity=0.20] (215.01, 72.07) circle (  2.13);

\path[fill=fillColor,fill opacity=0.20] (207.15, 72.89) circle (  2.13);

\path[fill=fillColor,fill opacity=0.20] (215.45, 72.89) circle (  2.13);

\path[fill=fillColor,fill opacity=0.20] (210.86, 70.45) circle (  2.13);

\path[fill=fillColor,fill opacity=0.20] (208.24, 55.82) circle (  2.13);

\path[fill=fillColor,fill opacity=0.20] (199.94, 43.63) circle (  2.13);

\path[fill=fillColor,fill opacity=0.20] (203.00, 50.13) circle (  2.13);

\path[fill=fillColor,fill opacity=0.20] (209.33, 55.82) circle (  2.13);

\path[fill=fillColor,fill opacity=0.20] (208.02, 65.57) circle (  2.13);

\path[fill=fillColor,fill opacity=0.20] (208.68, 74.51) circle (  2.13);

\path[fill=fillColor,fill opacity=0.20] (211.95, 78.57) circle (  2.13);

\path[fill=fillColor,fill opacity=0.20] (209.11, 73.70) circle (  2.13);

\path[fill=fillColor,fill opacity=0.20] (204.74, 70.45) circle (  2.13);

\path[fill=fillColor,fill opacity=0.20] (197.97, 68.82) circle (  2.13);

\path[fill=fillColor,fill opacity=0.20] (189.01, 63.13) circle (  2.13);

\path[fill=fillColor,fill opacity=0.20] (176.12, 66.38) circle (  2.13);

\path[fill=fillColor,fill opacity=0.20] (210.64, 75.32) circle (  2.13);

\path[fill=fillColor,fill opacity=0.20] (212.83, 72.89) circle (  2.13);

\path[fill=fillColor,fill opacity=0.20] (208.89, 75.32) circle (  2.13);

\path[fill=fillColor,fill opacity=0.20] (213.48, 76.95) circle (  2.13);

\path[fill=fillColor,fill opacity=0.20] (222.22, 70.45) circle (  2.13);

\path[fill=fillColor,fill opacity=0.20] (217.85, 66.38) circle (  2.13);

\path[fill=fillColor,fill opacity=0.20] (219.82, 71.26) circle (  2.13);

\path[fill=fillColor,fill opacity=0.20] (205.40, 72.89) circle (  2.13);

\path[fill=fillColor,fill opacity=0.20] (208.02, 68.01) circle (  2.13);

\path[fill=fillColor,fill opacity=0.20] (205.84, 60.69) circle (  2.13);

\path[fill=fillColor,fill opacity=0.20] (204.74, 49.32) circle (  2.13);

\path[fill=fillColor,fill opacity=0.20] (196.88, 42.00) circle (  2.13);

\path[fill=fillColor,fill opacity=0.20] (200.15, 55.82) circle (  2.13);

\path[fill=fillColor,fill opacity=0.20] (208.24, 64.76) circle (  2.13);

\path[fill=fillColor,fill opacity=0.20] (209.55, 72.07) circle (  2.13);

\path[fill=fillColor,fill opacity=0.20] (210.42, 75.32) circle (  2.13);

\path[fill=fillColor,fill opacity=0.20] (207.58, 71.26) circle (  2.13);

\path[fill=fillColor,fill opacity=0.20] (207.58, 71.26) circle (  2.13);

\path[fill=fillColor,fill opacity=0.20] (202.56, 74.51) circle (  2.13);

\path[fill=fillColor,fill opacity=0.20] (200.59, 69.63) circle (  2.13);

\path[fill=fillColor,fill opacity=0.20] (190.98, 63.13) circle (  2.13);

\path[fill=fillColor,fill opacity=0.20] (208.89, 71.26) circle (  2.13);

\path[fill=fillColor,fill opacity=0.20] (217.63, 74.51) circle (  2.13);

\path[fill=fillColor,fill opacity=0.20] (210.86, 76.14) circle (  2.13);

\path[fill=fillColor,fill opacity=0.20] (216.32, 76.95) circle (  2.13);

\path[fill=fillColor,fill opacity=0.20] (217.63, 80.20) circle (  2.13);

\path[fill=fillColor,fill opacity=0.20] (222.00, 74.51) circle (  2.13);

\path[fill=fillColor,fill opacity=0.20] (215.45, 66.38) circle (  2.13);

\path[fill=fillColor,fill opacity=0.20] (210.42, 67.20) circle (  2.13);

\path[fill=fillColor,fill opacity=0.20] (211.08, 66.38) circle (  2.13);

\path[fill=fillColor,fill opacity=0.20] (210.42, 57.44) circle (  2.13);

\path[fill=fillColor,fill opacity=0.20] (202.78, 50.94) circle (  2.13);

\path[fill=fillColor,fill opacity=0.20] (196.22, 47.69) circle (  2.13);

\path[fill=fillColor,fill opacity=0.20] (201.03, 65.57) circle (  2.13);

\path[fill=fillColor,fill opacity=0.20] (217.42, 62.32) circle (  2.13);

\path[fill=fillColor,fill opacity=0.20] (211.52, 64.76) circle (  2.13);

\path[fill=fillColor,fill opacity=0.20] (204.09, 68.82) circle (  2.13);

\path[fill=fillColor,fill opacity=0.20] (211.52, 71.26) circle (  2.13);

\path[fill=fillColor,fill opacity=0.20] (209.77, 74.51) circle (  2.13);

\path[fill=fillColor,fill opacity=0.20] (206.93, 72.07) circle (  2.13);

\path[fill=fillColor,fill opacity=0.20] (204.96, 67.20) circle (  2.13);

\path[fill=fillColor,fill opacity=0.20] (192.94, 72.07) circle (  2.13);

\path[fill=fillColor,fill opacity=0.20] (214.14, 74.51) circle (  2.13);

\path[fill=fillColor,fill opacity=0.20] (211.74, 76.14) circle (  2.13);

\path[fill=fillColor,fill opacity=0.20] (219.38, 75.32) circle (  2.13);

\path[fill=fillColor,fill opacity=0.20] (221.79, 73.70) circle (  2.13);

\path[fill=fillColor,fill opacity=0.20] (218.07, 71.26) circle (  2.13);

\path[fill=fillColor,fill opacity=0.20] (218.29, 72.89) circle (  2.13);

\path[fill=fillColor,fill opacity=0.20] (215.23, 72.89) circle (  2.13);

\path[fill=fillColor,fill opacity=0.20] (218.51, 67.20) circle (  2.13);

\path[fill=fillColor,fill opacity=0.20] (213.92, 59.07) circle (  2.13);

\path[fill=fillColor,fill opacity=0.20] (213.26, 47.69) circle (  2.13);

\path[fill=fillColor,fill opacity=0.20] (202.12, 42.82) circle (  2.13);

\path[fill=fillColor,fill opacity=0.20] (190.54, 46.07) circle (  2.13);

\path[fill=fillColor,fill opacity=0.20] (215.45, 58.26) circle (  2.13);

\path[fill=fillColor,fill opacity=0.20] (209.99, 63.13) circle (  2.13);

\path[fill=fillColor,fill opacity=0.20] (208.02, 61.51) circle (  2.13);

\path[fill=fillColor,fill opacity=0.20] (206.05, 61.51) circle (  2.13);

\path[fill=fillColor,fill opacity=0.20] (203.65, 65.57) circle (  2.13);

\path[fill=fillColor,fill opacity=0.20] (203.87, 63.95) circle (  2.13);

\path[fill=fillColor,fill opacity=0.20] (207.15, 62.32) circle (  2.13);

\path[fill=fillColor,fill opacity=0.20] (200.15, 68.82) circle (  2.13);

\path[fill=fillColor,fill opacity=0.20] (182.24, 73.70) circle (  2.13);

\path[fill=fillColor,fill opacity=0.20] (202.56, 66.38) circle (  2.13);

\path[fill=fillColor,fill opacity=0.20] (212.17, 75.32) circle (  2.13);

\path[fill=fillColor,fill opacity=0.20] (215.01, 76.95) circle (  2.13);

\path[fill=fillColor,fill opacity=0.20] (215.89, 71.26) circle (  2.13);

\path[fill=fillColor,fill opacity=0.20] (222.22, 70.45) circle (  2.13);

\path[fill=fillColor,fill opacity=0.20] (219.60, 65.57) circle (  2.13);

\path[fill=fillColor,fill opacity=0.20] (220.48, 62.32) circle (  2.13);

\path[fill=fillColor,fill opacity=0.20] (216.98, 71.26) circle (  2.13);

\path[fill=fillColor,fill opacity=0.20] (215.89, 77.76) circle (  2.13);

\path[fill=fillColor,fill opacity=0.20] (212.61, 68.01) circle (  2.13);

\path[fill=fillColor,fill opacity=0.20] (205.62, 52.57) circle (  2.13);

\path[fill=fillColor,fill opacity=0.20] (201.90, 42.82) circle (  2.13);

\path[fill=fillColor,fill opacity=0.20] (202.78, 39.56) circle (  2.13);

\path[fill=fillColor,fill opacity=0.20] (201.25, 57.44) circle (  2.13);

\path[fill=fillColor,fill opacity=0.20] (212.39, 51.75) circle (  2.13);

\path[fill=fillColor,fill opacity=0.20] (213.48, 51.75) circle (  2.13);

\path[fill=fillColor,fill opacity=0.20] (214.36, 57.44) circle (  2.13);

\path[fill=fillColor,fill opacity=0.20] (207.58, 63.13) circle (  2.13);

\path[fill=fillColor,fill opacity=0.20] (203.87, 61.51) circle (  2.13);

\path[fill=fillColor,fill opacity=0.20] (200.59, 55.82) circle (  2.13);

\path[fill=fillColor,fill opacity=0.20] (199.94, 63.13) circle (  2.13);

\path[fill=fillColor,fill opacity=0.20] (193.60, 71.26) circle (  2.13);

\path[fill=fillColor,fill opacity=0.20] (172.84, 69.63) circle (  2.13);

\path[fill=fillColor,fill opacity=0.20] (205.62, 50.13) circle (  2.13);

\path[fill=fillColor,fill opacity=0.20] (206.49, 67.20) circle (  2.13);

\path[fill=fillColor,fill opacity=0.20] (216.98, 76.14) circle (  2.13);

\path[fill=fillColor,fill opacity=0.20] (221.79, 71.26) circle (  2.13);

\path[fill=fillColor,fill opacity=0.20] (225.50, 70.45) circle (  2.13);

\path[fill=fillColor,fill opacity=0.20] (223.32, 72.07) circle (  2.13);

\path[fill=fillColor,fill opacity=0.20] (222.66, 67.20) circle (  2.13);

\path[fill=fillColor,fill opacity=0.20] (219.16, 65.57) circle (  2.13);

\path[fill=fillColor,fill opacity=0.20] (209.77, 72.07) circle (  2.13);

\path[fill=fillColor,fill opacity=0.20] (205.18, 70.45) circle (  2.13);

\path[fill=fillColor,fill opacity=0.20] (202.12, 56.63) circle (  2.13);

\path[fill=fillColor,fill opacity=0.20] (204.74, 45.25) circle (  2.13);

\path[fill=fillColor,fill opacity=0.20] (187.92, 46.88) circle (  2.13);

\path[fill=fillColor,fill opacity=0.20] (206.71, 48.50) circle (  2.13);

\path[fill=fillColor,fill opacity=0.20] (220.91, 53.38) circle (  2.13);

\path[fill=fillColor,fill opacity=0.20] (218.51, 64.76) circle (  2.13);

\path[fill=fillColor,fill opacity=0.20] (218.51, 62.32) circle (  2.13);

\path[fill=fillColor,fill opacity=0.20] (208.24, 59.88) circle (  2.13);

\path[fill=fillColor,fill opacity=0.20] (201.90, 64.76) circle (  2.13);

\path[fill=fillColor,fill opacity=0.20] (197.75, 63.95) circle (  2.13);

\path[fill=fillColor,fill opacity=0.20] (198.19, 63.95) circle (  2.13);

\path[fill=fillColor,fill opacity=0.20] (198.41, 65.57) circle (  2.13);

\path[fill=fillColor,fill opacity=0.20] (193.16, 56.63) circle (  2.13);

\path[fill=fillColor,fill opacity=0.20] (180.05, 50.94) circle (  2.13);

\path[fill=fillColor,fill opacity=0.20] (193.60, 52.57) circle (  2.13);

\path[fill=fillColor,fill opacity=0.20] (206.93, 58.26) circle (  2.13);

\path[fill=fillColor,fill opacity=0.20] (208.46, 68.82) circle (  2.13);

\path[fill=fillColor,fill opacity=0.20] (210.86, 73.70) circle (  2.13);

\path[fill=fillColor,fill opacity=0.20] (217.20, 71.26) circle (  2.13);

\path[fill=fillColor,fill opacity=0.20] (218.07, 70.45) circle (  2.13);

\path[fill=fillColor,fill opacity=0.20] (220.48, 76.95) circle (  2.13);

\path[fill=fillColor,fill opacity=0.20] (221.35, 79.39) circle (  2.13);

\path[fill=fillColor,fill opacity=0.20] (222.00, 72.07) circle (  2.13);

\path[fill=fillColor,fill opacity=0.20] (209.55, 65.57) circle (  2.13);

\path[fill=fillColor,fill opacity=0.20] (205.84, 64.76) circle (  2.13);

\path[fill=fillColor,fill opacity=0.20] (201.90, 55.82) circle (  2.13);

\path[fill=fillColor,fill opacity=0.20] (200.15, 42.82) circle (  2.13);

\path[fill=fillColor,fill opacity=0.20] (185.30, 44.44) circle (  2.13);

\path[fill=fillColor,fill opacity=0.20] (207.37, 66.38) circle (  2.13);

\path[fill=fillColor,fill opacity=0.20] (212.83, 59.88) circle (  2.13);

\path[fill=fillColor,fill opacity=0.20] (211.08, 54.19) circle (  2.13);

\path[fill=fillColor,fill opacity=0.20] (203.65, 63.95) circle (  2.13);

\path[fill=fillColor,fill opacity=0.20] (197.75, 63.95) circle (  2.13);

\path[fill=fillColor,fill opacity=0.20] (199.06, 66.38) circle (  2.13);

\path[fill=fillColor,fill opacity=0.20] (200.81, 65.57) circle (  2.13);

\path[fill=fillColor,fill opacity=0.20] (196.88, 59.07) circle (  2.13);

\path[fill=fillColor,fill opacity=0.20] (190.98, 59.07) circle (  2.13);

\path[fill=fillColor,fill opacity=0.20] (183.99, 57.44) circle (  2.13);

\path[fill=fillColor,fill opacity=0.20] (178.30, 50.13) circle (  2.13);

\path[fill=fillColor,fill opacity=0.20] (193.60, 54.19) circle (  2.13);

\path[fill=fillColor,fill opacity=0.20] (203.87, 56.63) circle (  2.13);

\path[fill=fillColor,fill opacity=0.20] (207.58, 62.32) circle (  2.13);

\path[fill=fillColor,fill opacity=0.20] (206.71, 73.70) circle (  2.13);

\path[fill=fillColor,fill opacity=0.20] (207.37, 81.01) circle (  2.13);

\path[fill=fillColor,fill opacity=0.20] (211.74, 75.32) circle (  2.13);

\path[fill=fillColor,fill opacity=0.20] (213.48, 69.63) circle (  2.13);

\path[fill=fillColor,fill opacity=0.20] (213.05, 72.89) circle (  2.13);

\path[fill=fillColor,fill opacity=0.20] (215.89, 76.14) circle (  2.13);

\path[fill=fillColor,fill opacity=0.20] (219.16, 73.70) circle (  2.13);

\path[fill=fillColor,fill opacity=0.20] (222.22, 66.38) circle (  2.13);

\path[fill=fillColor,fill opacity=0.20] (210.86, 56.63) circle (  2.13);

\path[fill=fillColor,fill opacity=0.20] (206.49, 51.75) circle (  2.13);

\path[fill=fillColor,fill opacity=0.20] (194.04, 49.32) circle (  2.13);

\path[fill=fillColor,fill opacity=0.20] (190.54, 50.94) circle (  2.13);

\path[fill=fillColor,fill opacity=0.20] (208.68, 48.50) circle (  2.13);

\path[fill=fillColor,fill opacity=0.20] (211.74, 54.19) circle (  2.13);

\path[fill=fillColor,fill opacity=0.20] (204.52, 59.07) circle (  2.13);

\path[fill=fillColor,fill opacity=0.20] (198.19, 62.32) circle (  2.13);

\path[fill=fillColor,fill opacity=0.20] (201.03, 64.76) circle (  2.13);

\path[fill=fillColor,fill opacity=0.20] (194.47, 64.76) circle (  2.13);

\path[fill=fillColor,fill opacity=0.20] (191.85, 63.13) circle (  2.13);

\path[fill=fillColor,fill opacity=0.20] (194.47, 59.07) circle (  2.13);

\path[fill=fillColor,fill opacity=0.20] (196.66, 51.75) circle (  2.13);

\path[fill=fillColor,fill opacity=0.20] (189.67, 49.32) circle (  2.13);

\path[fill=fillColor,fill opacity=0.20] (171.97, 52.57) circle (  2.13);

\path[fill=fillColor,fill opacity=0.20] (193.16, 59.07) circle (  2.13);

\path[fill=fillColor,fill opacity=0.20] (224.63, 55.82) circle (  2.13);

\path[fill=fillColor,fill opacity=0.20] (207.37, 62.32) circle (  2.13);

\path[fill=fillColor,fill opacity=0.20] (206.05, 72.89) circle (  2.13);

\path[fill=fillColor,fill opacity=0.20] (205.18, 72.89) circle (  2.13);

\path[fill=fillColor,fill opacity=0.20] (207.15, 70.45) circle (  2.13);

\path[fill=fillColor,fill opacity=0.20] (211.74, 72.89) circle (  2.13);

\path[fill=fillColor,fill opacity=0.20] (213.05, 70.45) circle (  2.13);

\path[fill=fillColor,fill opacity=0.20] (216.98, 64.76) circle (  2.13);

\path[fill=fillColor,fill opacity=0.20] (217.42, 63.13) circle (  2.13);

\path[fill=fillColor,fill opacity=0.20] (211.30, 62.32) circle (  2.13);

\path[fill=fillColor,fill opacity=0.20] (209.55, 61.51) circle (  2.13);

\path[fill=fillColor,fill opacity=0.20] (209.11, 59.88) circle (  2.13);

\path[fill=fillColor,fill opacity=0.20] (204.31, 54.19) circle (  2.13);

\path[fill=fillColor,fill opacity=0.20] (189.89, 51.75) circle (  2.13);

\path[fill=fillColor,fill opacity=0.20] (206.27, 50.94) circle (  2.13);

\path[fill=fillColor,fill opacity=0.20] (217.42, 50.13) circle (  2.13);

\path[fill=fillColor,fill opacity=0.20] (203.65, 54.19) circle (  2.13);

\path[fill=fillColor,fill opacity=0.20] (203.65, 53.38) circle (  2.13);

\path[fill=fillColor,fill opacity=0.20] (199.50, 55.82) circle (  2.13);

\path[fill=fillColor,fill opacity=0.20] (196.44, 59.88) circle (  2.13);

\path[fill=fillColor,fill opacity=0.20] (197.97, 61.51) circle (  2.13);

\path[fill=fillColor,fill opacity=0.20] (194.47, 66.38) circle (  2.13);

\path[fill=fillColor,fill opacity=0.20] (194.04, 68.82) circle (  2.13);

\path[fill=fillColor,fill opacity=0.20] (198.19, 66.38) circle (  2.13);

\path[fill=fillColor,fill opacity=0.20] (190.98, 59.88) circle (  2.13);

\path[fill=fillColor,fill opacity=0.20] (190.98, 63.13) circle (  2.13);

\path[fill=fillColor,fill opacity=0.20] (184.86, 65.57) circle (  2.13);

\path[fill=fillColor,fill opacity=0.20] (180.49, 60.69) circle (  2.13);

\path[fill=fillColor,fill opacity=0.20] (179.83, 57.44) circle (  2.13);

\path[fill=fillColor,fill opacity=0.20] (181.15, 61.51) circle (  2.13);

\path[fill=fillColor,fill opacity=0.20] (176.12, 63.13) circle (  2.13);

\path[fill=fillColor,fill opacity=0.20] (176.34, 59.07) circle (  2.13);

\path[fill=fillColor,fill opacity=0.20] (185.52, 59.88) circle (  2.13);

\path[fill=fillColor,fill opacity=0.20] (179.40, 62.32) circle (  2.13);

\path[fill=fillColor,fill opacity=0.20] (180.05, 60.69) circle (  2.13);

\path[fill=fillColor,fill opacity=0.20] (180.27, 58.26) circle (  2.13);

\path[fill=fillColor,fill opacity=0.20] (180.93, 60.69) circle (  2.13);

\path[fill=fillColor,fill opacity=0.20] (180.05, 59.88) circle (  2.13);

\path[fill=fillColor,fill opacity=0.20] (186.61, 52.57) circle (  2.13);

\path[fill=fillColor,fill opacity=0.20] (196.00, 51.75) circle (  2.13);

\path[fill=fillColor,fill opacity=0.20] (187.70, 57.44) circle (  2.13);

\path[fill=fillColor,fill opacity=0.20] (189.23, 55.82) circle (  2.13);

\path[fill=fillColor,fill opacity=0.20] (191.85, 51.75) circle (  2.13);

\path[fill=fillColor,fill opacity=0.20] (194.47, 54.19) circle (  2.13);

\path[fill=fillColor,fill opacity=0.20] (200.15, 56.63) circle (  2.13);

\path[fill=fillColor,fill opacity=0.20] (198.19, 59.88) circle (  2.13);

\path[fill=fillColor,fill opacity=0.20] (197.53, 63.13) circle (  2.13);

\path[fill=fillColor,fill opacity=0.20] (206.93, 62.32) circle (  2.13);

\path[fill=fillColor,fill opacity=0.20] (208.46, 62.32) circle (  2.13);

\path[fill=fillColor,fill opacity=0.20] (211.74, 63.13) circle (  2.13);

\path[fill=fillColor,fill opacity=0.20] (215.67, 60.69) circle (  2.13);

\path[fill=fillColor,fill opacity=0.20] (215.45, 57.44) circle (  2.13);

\path[fill=fillColor,fill opacity=0.20] (213.70, 56.63) circle (  2.13);

\path[fill=fillColor,fill opacity=0.20] (217.20, 56.63) circle (  2.13);

\path[fill=fillColor,fill opacity=0.20] (206.93, 55.82) circle (  2.13);

\path[fill=fillColor,fill opacity=0.20] (208.46, 52.57) circle (  2.13);

\path[fill=fillColor,fill opacity=0.20] (204.09, 50.94) circle (  2.13);

\path[fill=fillColor,fill opacity=0.20] (196.00, 55.82) circle (  2.13);

\path[fill=fillColor,fill opacity=0.20] (190.76, 63.13) circle (  2.13);

\path[fill=fillColor,fill opacity=0.20] (202.12, 53.38) circle (  2.13);

\path[fill=fillColor,fill opacity=0.20] (212.39, 49.32) circle (  2.13);

\path[fill=fillColor,fill opacity=0.20] (205.18, 49.32) circle (  2.13);

\path[fill=fillColor,fill opacity=0.20] (204.74, 51.75) circle (  2.13);

\path[fill=fillColor,fill opacity=0.20] (203.65, 55.01) circle (  2.13);

\path[fill=fillColor,fill opacity=0.20] (200.59, 63.13) circle (  2.13);

\path[fill=fillColor,fill opacity=0.20] (195.78, 69.63) circle (  2.13);

\path[fill=fillColor,fill opacity=0.20] (199.50, 70.45) circle (  2.13);

\path[fill=fillColor,fill opacity=0.20] (199.50, 69.63) circle (  2.13);

\path[fill=fillColor,fill opacity=0.20] (197.31, 70.45) circle (  2.13);

\path[fill=fillColor,fill opacity=0.20] (196.66, 72.07) circle (  2.13);

\path[fill=fillColor,fill opacity=0.20] (196.66, 64.76) circle (  2.13);

\path[fill=fillColor,fill opacity=0.20] (205.40, 57.44) circle (  2.13);

\path[fill=fillColor,fill opacity=0.20] (198.84, 58.26) circle (  2.13);

\path[fill=fillColor,fill opacity=0.20] (194.26, 59.07) circle (  2.13);

\path[fill=fillColor,fill opacity=0.20] (198.41, 55.01) circle (  2.13);

\path[fill=fillColor,fill opacity=0.20] (200.59, 55.01) circle (  2.13);

\path[fill=fillColor,fill opacity=0.20] (201.25, 61.51) circle (  2.13);

\path[fill=fillColor,fill opacity=0.20] (198.84, 63.13) circle (  2.13);

\path[fill=fillColor,fill opacity=0.20] (200.81, 56.63) circle (  2.13);

\path[fill=fillColor,fill opacity=0.20] (199.72, 49.32) circle (  2.13);

\path[fill=fillColor,fill opacity=0.20] (196.44, 50.94) circle (  2.13);

\path[fill=fillColor,fill opacity=0.20] (197.97, 56.63) circle (  2.13);

\path[fill=fillColor,fill opacity=0.20] (199.28, 56.63) circle (  2.13);

\path[fill=fillColor,fill opacity=0.20] (201.90, 55.01) circle (  2.13);

\path[fill=fillColor,fill opacity=0.20] (200.37, 55.82) circle (  2.13);

\path[fill=fillColor,fill opacity=0.20] (202.78, 51.75) circle (  2.13);

\path[fill=fillColor,fill opacity=0.20] (203.43, 50.94) circle (  2.13);

\path[fill=fillColor,fill opacity=0.20] (206.71, 56.63) circle (  2.13);

\path[fill=fillColor,fill opacity=0.20] (206.93, 55.01) circle (  2.13);

\path[fill=fillColor,fill opacity=0.20] (209.11, 52.57) circle (  2.13);

\path[fill=fillColor,fill opacity=0.20] (208.02, 56.63) circle (  2.13);

\path[fill=fillColor,fill opacity=0.20] (209.99, 58.26) circle (  2.13);

\path[fill=fillColor,fill opacity=0.20] (212.83, 54.19) circle (  2.13);

\path[fill=fillColor,fill opacity=0.20] (214.14, 51.75) circle (  2.13);

\path[fill=fillColor,fill opacity=0.20] (215.67, 53.38) circle (  2.13);

\path[fill=fillColor,fill opacity=0.20] (209.33, 57.44) circle (  2.13);

\path[fill=fillColor,fill opacity=0.20] (204.09, 58.26) circle (  2.13);

\path[fill=fillColor,fill opacity=0.20] (201.47, 59.07) circle (  2.13);

\path[fill=fillColor,fill opacity=0.20] (189.67, 59.88) circle (  2.13);

\path[fill=fillColor,fill opacity=0.20] (187.70, 57.44) circle (  2.13);

\path[fill=fillColor,fill opacity=0.20] (182.46, 52.57) circle (  2.13);

\path[fill=fillColor,fill opacity=0.20] (205.40, 46.88) circle (  2.13);

\path[fill=fillColor,fill opacity=0.20] (212.83, 47.69) circle (  2.13);

\path[fill=fillColor,fill opacity=0.20] (204.31, 51.75) circle (  2.13);

\path[fill=fillColor,fill opacity=0.20] (205.62, 55.01) circle (  2.13);

\path[fill=fillColor,fill opacity=0.20] (204.31, 55.82) circle (  2.13);

\path[fill=fillColor,fill opacity=0.20] (203.43, 53.38) circle (  2.13);

\path[fill=fillColor,fill opacity=0.20] (199.06, 55.82) circle (  2.13);

\path[fill=fillColor,fill opacity=0.20] (198.84, 62.32) circle (  2.13);

\path[fill=fillColor,fill opacity=0.20] (197.31, 64.76) circle (  2.13);

\path[fill=fillColor,fill opacity=0.20] (204.96, 68.01) circle (  2.13);

\path[fill=fillColor,fill opacity=0.20] (199.06, 68.82) circle (  2.13);

\path[fill=fillColor,fill opacity=0.20] (201.25, 65.57) circle (  2.13);

\path[fill=fillColor,fill opacity=0.20] (197.97, 61.51) circle (  2.13);

\path[fill=fillColor,fill opacity=0.20] (201.25, 59.88) circle (  2.13);

\path[fill=fillColor,fill opacity=0.20] (203.00, 63.95) circle (  2.13);

\path[fill=fillColor,fill opacity=0.20] (200.59, 68.01) circle (  2.13);

\path[fill=fillColor,fill opacity=0.20] (199.72, 63.95) circle (  2.13);

\path[fill=fillColor,fill opacity=0.20] (199.50, 59.07) circle (  2.13);

\path[fill=fillColor,fill opacity=0.20] (206.49, 57.44) circle (  2.13);

\path[fill=fillColor,fill opacity=0.20] (199.50, 55.82) circle (  2.13);

\path[fill=fillColor,fill opacity=0.20] (201.03, 59.88) circle (  2.13);

\path[fill=fillColor,fill opacity=0.20] (201.47, 62.32) circle (  2.13);

\path[fill=fillColor,fill opacity=0.20] (203.87, 58.26) circle (  2.13);

\path[fill=fillColor,fill opacity=0.20] (208.02, 52.57) circle (  2.13);

\path[fill=fillColor,fill opacity=0.20] (208.02, 49.32) circle (  2.13);

\path[fill=fillColor,fill opacity=0.20] (208.68, 50.94) circle (  2.13);

\path[fill=fillColor,fill opacity=0.20] (208.89, 55.01) circle (  2.13);

\path[fill=fillColor,fill opacity=0.20] (208.46, 56.63) circle (  2.13);

\path[fill=fillColor,fill opacity=0.20] (205.18, 56.63) circle (  2.13);

\path[fill=fillColor,fill opacity=0.20] (200.81, 59.07) circle (  2.13);

\path[fill=fillColor,fill opacity=0.20] (199.28, 62.32) circle (  2.13);

\path[fill=fillColor,fill opacity=0.20] (199.72, 63.13) circle (  2.13);

\path[fill=fillColor,fill opacity=0.20] (195.78, 67.20) circle (  2.13);

\path[fill=fillColor,fill opacity=0.20] (187.70, 72.07) circle (  2.13);

\path[fill=fillColor,fill opacity=0.20] (179.83, 72.89) circle (  2.13);

\path[fill=fillColor,fill opacity=0.20] (178.52, 71.26) circle (  2.13);

\path[fill=fillColor,fill opacity=0.20] (198.19, 45.25) circle (  2.13);

\path[fill=fillColor,fill opacity=0.20] (201.47, 46.07) circle (  2.13);

\path[fill=fillColor,fill opacity=0.20] (202.78, 50.13) circle (  2.13);

\path[fill=fillColor,fill opacity=0.20] (206.27, 46.88) circle (  2.13);

\path[fill=fillColor,fill opacity=0.20] (200.37, 38.75) circle (  2.13);

\path[fill=fillColor,fill opacity=0.20] (201.68, 38.75) circle (  2.13);

\path[fill=fillColor,fill opacity=0.20] (201.90, 45.25) circle (  2.13);

\path[fill=fillColor,fill opacity=0.20] (204.96, 50.94) circle (  2.13);

\path[fill=fillColor,fill opacity=0.20] (203.00, 63.13) circle (  2.13);

\path[fill=fillColor,fill opacity=0.20] (201.68, 67.20) circle (  2.13);

\path[fill=fillColor,fill opacity=0.20] (202.34, 60.69) circle (  2.13);

\path[fill=fillColor,fill opacity=0.20] (202.12, 55.01) circle (  2.13);

\path[fill=fillColor,fill opacity=0.20] (202.34, 55.01) circle (  2.13);

\path[fill=fillColor,fill opacity=0.20] (200.81, 58.26) circle (  2.13);

\path[fill=fillColor,fill opacity=0.20] (204.74, 59.88) circle (  2.13);

\path[fill=fillColor,fill opacity=0.20] (198.41, 53.38) circle (  2.13);

\path[fill=fillColor,fill opacity=0.20] (199.72, 50.13) circle (  2.13);

\path[fill=fillColor,fill opacity=0.20] (204.74, 53.38) circle (  2.13);

\path[fill=fillColor,fill opacity=0.20] (202.34, 55.01) circle (  2.13);

\path[fill=fillColor,fill opacity=0.20] (204.74, 55.82) circle (  2.13);

\path[fill=fillColor,fill opacity=0.20] (203.65, 56.63) circle (  2.13);

\path[fill=fillColor,fill opacity=0.20] (204.31, 54.19) circle (  2.13);

\path[fill=fillColor,fill opacity=0.20] (208.46, 52.57) circle (  2.13);

\path[fill=fillColor,fill opacity=0.20] (201.90, 56.63) circle (  2.13);

\path[fill=fillColor,fill opacity=0.20] (198.84, 60.69) circle (  2.13);

\path[fill=fillColor,fill opacity=0.20] (193.60, 66.38) circle (  2.13);

\path[fill=fillColor,fill opacity=0.20] (187.92, 55.82) circle (  2.13);

\path[fill=fillColor,fill opacity=0.20] (194.04, 46.07) circle (  2.13);

\path[fill=fillColor,fill opacity=0.20] (200.37, 45.25) circle (  2.13);

\path[fill=fillColor,fill opacity=0.20] (202.12, 46.07) circle (  2.13);

\path[fill=fillColor,fill opacity=0.20] (209.11, 49.32) circle (  2.13);

\path[fill=fillColor,fill opacity=0.20] (205.18, 55.01) circle (  2.13);

\path[fill=fillColor,fill opacity=0.20] (204.74, 55.82) circle (  2.13);

\path[fill=fillColor,fill opacity=0.20] (204.52, 51.75) circle (  2.13);

\path[fill=fillColor,fill opacity=0.20] (203.43, 49.32) circle (  2.13);

\path[fill=fillColor,fill opacity=0.20] (200.37, 47.69) circle (  2.13);

\path[fill=fillColor,fill opacity=0.20] (200.59, 48.50) circle (  2.13);

\path[fill=fillColor,fill opacity=0.20] (197.10, 49.32) circle (  2.13);

\path[fill=fillColor,fill opacity=0.20] (198.84, 46.07) circle (  2.13);

\path[fill=fillColor,fill opacity=0.20] (199.50, 43.63) circle (  2.13);

\path[fill=fillColor,fill opacity=0.20] (199.94, 47.69) circle (  2.13);

\path[fill=fillColor,fill opacity=0.20] (197.31, 51.75) circle (  2.13);

\path[fill=fillColor,fill opacity=0.20] (198.84, 55.01) circle (  2.13);

\path[fill=fillColor,fill opacity=0.20] (191.20, 62.32) circle (  2.13);

\path[fill=fillColor,fill opacity=0.20] (191.63, 68.82) circle (  2.13);

\path[fill=fillColor,fill opacity=0.20] (191.41, 55.82) circle (  2.13);

\path[fill=fillColor,fill opacity=0.20] (192.94, 50.94) circle (  2.13);

\path[fill=fillColor,fill opacity=0.20] (191.20, 50.13) circle (  2.13);

\path[fill=fillColor,fill opacity=0.20] (190.98, 47.69) circle (  2.13);

\path[fill=fillColor,fill opacity=0.20] (250.19, 98.89) circle (  2.13);

\path[fill=fillColor,fill opacity=0.20] (237.52, 93.20) circle (  2.13);

\path[fill=fillColor,fill opacity=0.20] (240.36,109.46) circle (  2.13);

\path[fill=fillColor,fill opacity=0.20] (240.14,103.77) circle (  2.13);

\path[fill=fillColor,fill opacity=0.20] (238.17,101.33) circle (  2.13);

\path[fill=fillColor,fill opacity=0.20] (239.70, 97.27) circle (  2.13);

\path[fill=fillColor,fill opacity=0.20] (239.92, 81.01) circle (  2.13);

\path[fill=fillColor,fill opacity=0.20] (214.14, 81.01) circle (  2.13);

\path[fill=fillColor,fill opacity=0.20] (235.55, 75.32) circle (  2.13);

\path[fill=fillColor,fill opacity=0.20] (234.24, 84.26) circle (  2.13);

\path[fill=fillColor,fill opacity=0.20] (243.20, 90.76) circle (  2.13);

\path[fill=fillColor,fill opacity=0.20] (247.79, 85.08) circle (  2.13);

\path[fill=fillColor,fill opacity=0.20] (250.85, 89.14) circle (  2.13);

\path[fill=fillColor,fill opacity=0.20] (258.28, 89.14) circle (  2.13);

\path[fill=fillColor,fill opacity=0.20] (260.24, 82.64) circle (  2.13);

\path[fill=fillColor,fill opacity=0.20] (230.96, 85.08) circle (  2.13);

\path[fill=fillColor,fill opacity=0.20] (196.44, 91.58) circle (  2.13);

\path[fill=fillColor,fill opacity=0.20] (266.14, 81.01) circle (  2.13);

\path[fill=fillColor,fill opacity=0.20] (253.03, 89.14) circle (  2.13);

\path[fill=fillColor,fill opacity=0.20] (256.75, 77.76) circle (  2.13);

\path[fill=fillColor,fill opacity=0.20] (251.72, 82.64) circle (  2.13);

\path[fill=fillColor,fill opacity=0.20] (256.96, 87.51) circle (  2.13);

\path[fill=fillColor,fill opacity=0.20] (252.81, 84.26) circle (  2.13);

\path[fill=fillColor,fill opacity=0.20] (200.81, 83.45) circle (  2.13);

\path[fill=fillColor,fill opacity=0.20] (242.33, 72.89) circle (  2.13);

\path[fill=fillColor,fill opacity=0.20] (269.86, 81.01) circle (  2.13);

\path[fill=fillColor,fill opacity=0.20] (243.85, 78.57) circle (  2.13);

\path[fill=fillColor,fill opacity=0.20] (217.63, 76.14) circle (  2.13);

\path[fill=fillColor,fill opacity=0.20] (231.40, 76.95) circle (  2.13);

\path[fill=fillColor,fill opacity=0.20] (261.12, 89.14) circle (  2.13);

\path[fill=fillColor,fill opacity=0.20] (253.25, 94.02) circle (  2.13);

\path[fill=fillColor,fill opacity=0.20] (247.57, 97.27) circle (  2.13);

\path[fill=fillColor,fill opacity=0.20] (256.09, 95.64) circle (  2.13);

\path[fill=fillColor,fill opacity=0.20] (264.83, 89.14) circle (  2.13);

\path[fill=fillColor,fill opacity=0.20] (259.15, 81.01) circle (  2.13);

\path[fill=fillColor,fill opacity=0.20] (222.88, 73.70) circle (  2.13);

\path[fill=fillColor,fill opacity=0.20] (194.47, 75.32) circle (  2.13);

\path[fill=fillColor,fill opacity=0.20] (232.06, 61.51) circle (  2.13);

\path[fill=fillColor,fill opacity=0.20] (229.87, 50.13) circle (  2.13);

\path[fill=fillColor,fill opacity=0.20] (217.85, 51.75) circle (  2.13);

\path[fill=fillColor,fill opacity=0.20] (207.80, 44.44) circle (  2.13);

\path[fill=fillColor,fill opacity=0.20] (203.65, 42.00) circle (  2.13);

\path[fill=fillColor,fill opacity=0.20] (203.43, 46.07) circle (  2.13);

\path[fill=fillColor,fill opacity=0.20] (196.44, 48.50) circle (  2.13);

\path[fill=fillColor,fill opacity=0.20] (216.54, 85.08) circle (  2.13);

\path[fill=fillColor,fill opacity=0.20] (232.06, 87.51) circle (  2.13);

\path[fill=fillColor,fill opacity=0.20] (235.77, 94.83) circle (  2.13);

\path[fill=fillColor,fill opacity=0.20] (240.36, 95.64) circle (  2.13);

\path[fill=fillColor,fill opacity=0.20] (244.07, 96.45) circle (  2.13);

\path[fill=fillColor,fill opacity=0.20] (251.94, 97.27) circle (  2.13);

\path[fill=fillColor,fill opacity=0.20] (256.31, 85.08) circle (  2.13);

\path[fill=fillColor,fill opacity=0.20] (246.48, 71.26) circle (  2.13);

\path[fill=fillColor,fill opacity=0.20] (207.58, 68.01) circle (  2.13);

\path[fill=fillColor,fill opacity=0.20] (175.03, 68.01) circle (  2.13);

\path[fill=fillColor,fill opacity=0.20] (225.06, 68.01) circle (  2.13);

\path[fill=fillColor,fill opacity=0.20] (228.12, 63.13) circle (  2.13);

\path[fill=fillColor,fill opacity=0.20] (228.78, 61.51) circle (  2.13);

\path[fill=fillColor,fill opacity=0.20] (214.58, 64.76) circle (  2.13);

\path[fill=fillColor,fill opacity=0.20] (207.58, 63.13) circle (  2.13);

\path[fill=fillColor,fill opacity=0.20] (215.89, 55.82) circle (  2.13);

\path[fill=fillColor,fill opacity=0.20] (208.89, 58.26) circle (  2.13);

\path[fill=fillColor,fill opacity=0.20] (180.05, 52.57) circle (  2.13);

\path[fill=fillColor,fill opacity=0.20] (217.85, 76.95) circle (  2.13);

\path[fill=fillColor,fill opacity=0.20] (233.80, 85.89) circle (  2.13);

\path[fill=fillColor,fill opacity=0.20] (260.46, 91.58) circle (  2.13);

\path[fill=fillColor,fill opacity=0.20] (252.16, 89.14) circle (  2.13);

\path[fill=fillColor,fill opacity=0.20] (247.79, 89.95) circle (  2.13);

\path[fill=fillColor,fill opacity=0.20] (253.69, 90.76) circle (  2.13);

\path[fill=fillColor,fill opacity=0.20] (255.65, 91.58) circle (  2.13);

\path[fill=fillColor,fill opacity=0.20] (228.12, 74.51) circle (  2.13);

\path[fill=fillColor,fill opacity=0.20] (198.84, 72.89) circle (  2.13);

\path[fill=fillColor,fill opacity=0.20] (220.69, 58.26) circle (  2.13);

\path[fill=fillColor,fill opacity=0.20] (226.59, 64.76) circle (  2.13);

\path[fill=fillColor,fill opacity=0.20] (229.00, 85.89) circle (  2.13);

\path[fill=fillColor,fill opacity=0.20] (222.88, 90.76) circle (  2.13);

\path[fill=fillColor,fill opacity=0.20] (218.29, 90.76) circle (  2.13);

\path[fill=fillColor,fill opacity=0.20] (216.11, 88.33) circle (  2.13);

\path[fill=fillColor,fill opacity=0.20] (213.48, 75.32) circle (  2.13);

\path[fill=fillColor,fill opacity=0.20] (210.86, 72.07) circle (  2.13);

\path[fill=fillColor,fill opacity=0.20] (201.25, 76.95) circle (  2.13);

\path[fill=fillColor,fill opacity=0.20] (217.63, 71.26) circle (  2.13);

\path[fill=fillColor,fill opacity=0.20] (226.16, 63.95) circle (  2.13);

\path[fill=fillColor,fill opacity=0.20] (242.33, 81.01) circle (  2.13);

\path[fill=fillColor,fill opacity=0.20] (258.06, 85.89) circle (  2.13);

\path[fill=fillColor,fill opacity=0.20] (271.39, 91.58) circle (  2.13);

\path[fill=fillColor,fill opacity=0.20] (258.28, 88.33) circle (  2.13);

\path[fill=fillColor,fill opacity=0.20] (251.28, 78.57) circle (  2.13);

\path[fill=fillColor,fill opacity=0.20] (209.77, 75.32) circle (  2.13);

\path[fill=fillColor,fill opacity=0.20] (212.17, 55.82) circle (  2.13);

\path[fill=fillColor,fill opacity=0.20] (221.57, 60.69) circle (  2.13);

\path[fill=fillColor,fill opacity=0.20] (242.33, 77.76) circle (  2.13);

\path[fill=fillColor,fill opacity=0.20] (233.80, 89.14) circle (  2.13);

\path[fill=fillColor,fill opacity=0.20] (222.88, 95.64) circle (  2.13);

\path[fill=fillColor,fill opacity=0.20] (216.98,102.14) circle (  2.13);

\path[fill=fillColor,fill opacity=0.20] (213.26, 96.45) circle (  2.13);

\path[fill=fillColor,fill opacity=0.20] (211.95, 81.82) circle (  2.13);

\path[fill=fillColor,fill opacity=0.20] (207.15, 76.14) circle (  2.13);

\path[fill=fillColor,fill opacity=0.20] (186.39, 74.51) circle (  2.13);

\path[fill=fillColor,fill opacity=0.20] (235.33, 83.45) circle (  2.13);

\path[fill=fillColor,fill opacity=0.20] (228.12, 85.89) circle (  2.13);

\path[fill=fillColor,fill opacity=0.20] (233.59, 86.70) circle (  2.13);

\path[fill=fillColor,fill opacity=0.20] (242.33, 99.70) circle (  2.13);

\path[fill=fillColor,fill opacity=0.20] (252.16, 94.83) circle (  2.13);

\path[fill=fillColor,fill opacity=0.20] (249.75, 86.70) circle (  2.13);

\path[fill=fillColor,fill opacity=0.20] (260.24, 90.76) circle (  2.13);

\path[fill=fillColor,fill opacity=0.20] (221.79, 74.51) circle (  2.13);

\path[fill=fillColor,fill opacity=0.20] (185.08, 68.01) circle (  2.13);

\path[fill=fillColor,fill opacity=0.20] (204.74, 53.38) circle (  2.13);

\path[fill=fillColor,fill opacity=0.20] (227.90, 64.76) circle (  2.13);

\path[fill=fillColor,fill opacity=0.20] (235.11, 86.70) circle (  2.13);

\path[fill=fillColor,fill opacity=0.20] (224.19, 88.33) circle (  2.13);

\path[fill=fillColor,fill opacity=0.20] (219.38, 88.33) circle (  2.13);

\path[fill=fillColor,fill opacity=0.20] (209.77, 96.45) circle (  2.13);

\path[fill=fillColor,fill opacity=0.20] (204.09, 90.76) circle (  2.13);

\path[fill=fillColor,fill opacity=0.20] (204.09, 77.76) circle (  2.13);

\path[fill=fillColor,fill opacity=0.20] (201.25, 73.70) circle (  2.13);

\path[fill=fillColor,fill opacity=0.20] (168.47, 70.45) circle (  2.13);

\path[fill=fillColor,fill opacity=0.20] (214.36, 85.89) circle (  2.13);

\path[fill=fillColor,fill opacity=0.20] (219.38, 79.39) circle (  2.13);

\path[fill=fillColor,fill opacity=0.20] (223.32, 90.76) circle (  2.13);

\path[fill=fillColor,fill opacity=0.20] (230.09, 99.70) circle (  2.13);

\path[fill=fillColor,fill opacity=0.20] (233.59, 97.27) circle (  2.13);

\path[fill=fillColor,fill opacity=0.20] (240.80, 92.39) circle (  2.13);

\path[fill=fillColor,fill opacity=0.20] (249.10, 89.95) circle (  2.13);

\path[fill=fillColor,fill opacity=0.20] (240.36, 90.76) circle (  2.13);

\path[fill=fillColor,fill opacity=0.20] (242.54, 96.45) circle (  2.13);

\path[fill=fillColor,fill opacity=0.20] (246.48, 92.39) circle (  2.13);

\path[fill=fillColor,fill opacity=0.20] (196.00, 77.76) circle (  2.13);

\path[fill=fillColor,fill opacity=0.20] (197.10, 56.63) circle (  2.13);

\path[fill=fillColor,fill opacity=0.20] (218.51, 61.51) circle (  2.13);

\path[fill=fillColor,fill opacity=0.20] (235.11, 84.26) circle (  2.13);

\path[fill=fillColor,fill opacity=0.20] (227.69, 91.58) circle (  2.13);

\path[fill=fillColor,fill opacity=0.20] (218.95, 87.51) circle (  2.13);

\path[fill=fillColor,fill opacity=0.20] (218.95, 89.95) circle (  2.13);

\path[fill=fillColor,fill opacity=0.20] (206.93, 86.70) circle (  2.13);

\path[fill=fillColor,fill opacity=0.20] (201.68, 74.51) circle (  2.13);

\path[fill=fillColor,fill opacity=0.20] (191.41, 76.14) circle (  2.13);

\path[fill=fillColor,fill opacity=0.20] (217.42, 83.45) circle (  2.13);

\path[fill=fillColor,fill opacity=0.20] (230.09, 86.70) circle (  2.13);

\path[fill=fillColor,fill opacity=0.20] (225.28, 88.33) circle (  2.13);

\path[fill=fillColor,fill opacity=0.20] (223.97,106.21) circle (  2.13);

\path[fill=fillColor,fill opacity=0.20] (235.11,109.46) circle (  2.13);

\path[fill=fillColor,fill opacity=0.20] (237.74, 93.20) circle (  2.13);

\path[fill=fillColor,fill opacity=0.20] (235.55, 88.33) circle (  2.13);

\path[fill=fillColor,fill opacity=0.20] (233.37, 90.76) circle (  2.13);

\path[fill=fillColor,fill opacity=0.20] (234.90, 89.95) circle (  2.13);

\path[fill=fillColor,fill opacity=0.20] (197.53, 85.89) circle (  2.13);

\path[fill=fillColor,fill opacity=0.20] (169.78, 80.20) circle (  2.13);

\path[fill=fillColor,fill opacity=0.20] (202.12, 56.63) circle (  2.13);

\path[fill=fillColor,fill opacity=0.20] (237.96, 72.07) circle (  2.13);

\path[fill=fillColor,fill opacity=0.20] (241.89, 87.51) circle (  2.13);

\path[fill=fillColor,fill opacity=0.20] (234.24, 86.70) circle (  2.13);

\path[fill=fillColor,fill opacity=0.20] (232.71, 87.51) circle (  2.13);

\path[fill=fillColor,fill opacity=0.20] (231.40, 92.39) circle (  2.13);

\path[fill=fillColor,fill opacity=0.20] (214.36, 85.08) circle (  2.13);

\path[fill=fillColor,fill opacity=0.20] (203.65, 80.20) circle (  2.13);

\path[fill=fillColor,fill opacity=0.20] (216.76, 85.89) circle (  2.13);

\path[fill=fillColor,fill opacity=0.20] (245.60, 89.95) circle (  2.13);

\path[fill=fillColor,fill opacity=0.20] (245.17, 83.45) circle (  2.13);

\path[fill=fillColor,fill opacity=0.20] (233.15, 91.58) circle (  2.13);

\path[fill=fillColor,fill opacity=0.20] (230.31,107.02) circle (  2.13);

\path[fill=fillColor,fill opacity=0.20] (234.46,111.89) circle (  2.13);

\path[fill=fillColor,fill opacity=0.20] (234.90,102.14) circle (  2.13);

\path[fill=fillColor,fill opacity=0.20] (227.90, 91.58) circle (  2.13);

\path[fill=fillColor,fill opacity=0.20] (216.98, 82.64) circle (  2.13);

\path[fill=fillColor,fill opacity=0.20] (191.85, 72.07) circle (  2.13);

\path[fill=fillColor,fill opacity=0.20] (167.38, 74.51) circle (  2.13);

\path[fill=fillColor,fill opacity=0.20] (212.61, 62.32) circle (  2.13);

\path[fill=fillColor,fill opacity=0.20] (227.03, 76.95) circle (  2.13);

\path[fill=fillColor,fill opacity=0.20] (231.62, 81.01) circle (  2.13);

\path[fill=fillColor,fill opacity=0.20] (244.73, 83.45) circle (  2.13);

\path[fill=fillColor,fill opacity=0.20] (240.58, 95.64) circle (  2.13);

\path[fill=fillColor,fill opacity=0.20] (223.10, 99.70) circle (  2.13);

\path[fill=fillColor,fill opacity=0.20] (216.76, 88.33) circle (  2.13);

\path[fill=fillColor,fill opacity=0.20] (201.47, 99.70) circle (  2.13);

\path[fill=fillColor,fill opacity=0.20] (212.83, 79.39) circle (  2.13);

\path[fill=fillColor,fill opacity=0.20] (245.38, 95.64) circle (  2.13);

\path[fill=fillColor,fill opacity=0.20] (248.01, 94.02) circle (  2.13);

\path[fill=fillColor,fill opacity=0.20] (237.52, 94.02) circle (  2.13);

\path[fill=fillColor,fill opacity=0.20] (231.62, 92.39) circle (  2.13);

\path[fill=fillColor,fill opacity=0.20] (221.57, 96.45) circle (  2.13);

\path[fill=fillColor,fill opacity=0.20] (211.08, 92.39) circle (  2.13);

\path[fill=fillColor,fill opacity=0.20] (204.52, 82.64) circle (  2.13);

\path[fill=fillColor,fill opacity=0.20] (189.23, 71.26) circle (  2.13);

\path[fill=fillColor,fill opacity=0.20] (198.84, 63.95) circle (  2.13);

\path[fill=fillColor,fill opacity=0.20] (213.05, 70.45) circle (  2.13);

\path[fill=fillColor,fill opacity=0.20] (224.19, 76.14) circle (  2.13);

\path[fill=fillColor,fill opacity=0.20] (231.62, 85.89) circle (  2.13);

\path[fill=fillColor,fill opacity=0.20] (226.81, 94.02) circle (  2.13);

\path[fill=fillColor,fill opacity=0.20] (221.57, 97.27) circle (  2.13);

\path[fill=fillColor,fill opacity=0.20] (216.98, 95.64) circle (  2.13);

\path[fill=fillColor,fill opacity=0.20] (206.27, 98.89) circle (  2.13);

\path[fill=fillColor,fill opacity=0.20] (220.04, 68.82) circle (  2.13);

\path[fill=fillColor,fill opacity=0.20] (232.93, 81.01) circle (  2.13);

\path[fill=fillColor,fill opacity=0.20] (265.49, 91.58) circle (  2.13);

\path[fill=fillColor,fill opacity=0.20] (260.46, 89.95) circle (  2.13);

\path[fill=fillColor,fill opacity=0.20] (228.78, 83.45) circle (  2.13);

\path[fill=fillColor,fill opacity=0.20] (203.00, 78.57) circle (  2.13);

\path[fill=fillColor,fill opacity=0.20] (185.73, 76.95) circle (  2.13);

\path[fill=fillColor,fill opacity=0.20] (183.77, 81.82) circle (  2.13);

\path[fill=fillColor,fill opacity=0.20] (185.08, 75.32) circle (  2.13);

\path[fill=fillColor,fill opacity=0.20] (181.80, 64.76) circle (  2.13);

\path[fill=fillColor,fill opacity=0.20] (206.05, 68.01) circle (  2.13);

\path[fill=fillColor,fill opacity=0.20] (220.04, 68.01) circle (  2.13);

\path[fill=fillColor,fill opacity=0.20] (223.10, 83.45) circle (  2.13);

\path[fill=fillColor,fill opacity=0.20] (224.85, 91.58) circle (  2.13);

\path[fill=fillColor,fill opacity=0.20] (223.97, 89.14) circle (  2.13);

\path[fill=fillColor,fill opacity=0.20] (220.04, 94.02) circle (  2.13);

\path[fill=fillColor,fill opacity=0.20] (211.95, 99.70) circle (  2.13);

\path[fill=fillColor,fill opacity=0.20] (210.64, 98.08) circle (  2.13);

\path[fill=fillColor,fill opacity=0.20] (230.96, 81.01) circle (  2.13);

\path[fill=fillColor,fill opacity=0.20] (254.56, 86.70) circle (  2.13);

\path[fill=fillColor,fill opacity=0.20] (249.10, 81.82) circle (  2.13);

\path[fill=fillColor,fill opacity=0.20] (251.50, 75.32) circle (  2.13);

\path[fill=fillColor,fill opacity=0.20] (203.43, 64.76) circle (  2.13);

\path[fill=fillColor,fill opacity=0.20] (201.25, 62.32) circle (  2.13);

\path[fill=fillColor,fill opacity=0.20] (168.04, 72.89) circle (  2.13);

\path[fill=fillColor,fill opacity=0.20] (192.51, 59.88) circle (  2.13);

\path[fill=fillColor,fill opacity=0.20] (210.21, 47.69) circle (  2.13);

\path[fill=fillColor,fill opacity=0.20] (217.42, 67.20) circle (  2.13);

\path[fill=fillColor,fill opacity=0.20] (228.56, 85.08) circle (  2.13);

\path[fill=fillColor,fill opacity=0.20] (223.75, 86.70) circle (  2.13);

\path[fill=fillColor,fill opacity=0.20] (218.29, 87.51) circle (  2.13);

\path[fill=fillColor,fill opacity=0.20] (218.07, 91.58) circle (  2.13);

\path[fill=fillColor,fill opacity=0.20] (215.01, 96.45) circle (  2.13);

\path[fill=fillColor,fill opacity=0.20] (210.64, 94.83) circle (  2.13);

\path[fill=fillColor,fill opacity=0.20] (232.71, 73.70) circle (  2.13);

\path[fill=fillColor,fill opacity=0.20] (239.70, 78.57) circle (  2.13);

\path[fill=fillColor,fill opacity=0.20] (224.63, 72.07) circle (  2.13);

\path[fill=fillColor,fill opacity=0.20] (211.95, 64.76) circle (  2.13);

\path[fill=fillColor,fill opacity=0.20] (177.65, 59.88) circle (  2.13);

\path[fill=fillColor,fill opacity=0.20] (199.72, 50.13) circle (  2.13);

\path[fill=fillColor,fill opacity=0.20] (217.63, 68.01) circle (  2.13);

\path[fill=fillColor,fill opacity=0.20] (224.85, 76.14) circle (  2.13);

\path[fill=fillColor,fill opacity=0.20] (222.88, 83.45) circle (  2.13);

\path[fill=fillColor,fill opacity=0.20] (213.92, 89.14) circle (  2.13);

\path[fill=fillColor,fill opacity=0.20] (211.52, 94.02) circle (  2.13);

\path[fill=fillColor,fill opacity=0.20] (210.21, 86.70) circle (  2.13);

\path[fill=fillColor,fill opacity=0.20] (208.02, 80.20) circle (  2.13);

\path[fill=fillColor,fill opacity=0.20] (231.62, 63.13) circle (  2.13);

\path[fill=fillColor,fill opacity=0.20] (233.59, 70.45) circle (  2.13);

\path[fill=fillColor,fill opacity=0.20] (204.31, 80.20) circle (  2.13);

\path[fill=fillColor,fill opacity=0.20] (185.95, 67.20) circle (  2.13);

\path[fill=fillColor,fill opacity=0.20] (170.66, 46.88) circle (  2.13);

\path[fill=fillColor,fill opacity=0.20] (170.66, 57.44) circle (  2.13);

\path[fill=fillColor,fill opacity=0.20] (182.89, 59.07) circle (  2.13);

\path[fill=fillColor,fill opacity=0.20] (202.56, 59.88) circle (  2.13);

\path[fill=fillColor,fill opacity=0.20] (224.85, 71.26) circle (  2.13);

\path[fill=fillColor,fill opacity=0.20] (216.11, 82.64) circle (  2.13);

\path[fill=fillColor,fill opacity=0.20] (216.98, 82.64) circle (  2.13);

\path[fill=fillColor,fill opacity=0.20] (224.63, 86.70) circle (  2.13);

\path[fill=fillColor,fill opacity=0.20] (216.98, 90.76) circle (  2.13);

\path[fill=fillColor,fill opacity=0.20] (217.85, 85.89) circle (  2.13);

\path[fill=fillColor,fill opacity=0.20] (218.07, 95.64) circle (  2.13);

\path[fill=fillColor,fill opacity=0.20] (217.63, 53.38) circle (  2.13);

\path[fill=fillColor,fill opacity=0.20] (245.82, 63.95) circle (  2.13);

\path[fill=fillColor,fill opacity=0.20] (229.22, 68.82) circle (  2.13);

\path[fill=fillColor,fill opacity=0.20] (182.89, 53.38) circle (  2.13);

\path[fill=fillColor,fill opacity=0.20] (208.89, 54.19) circle (  2.13);

\path[fill=fillColor,fill opacity=0.20] (174.37, 55.82) circle (  2.13);

\path[fill=fillColor,fill opacity=0.20] (194.26, 56.63) circle (  2.13);

\path[fill=fillColor,fill opacity=0.20] (209.99, 59.88) circle (  2.13);

\path[fill=fillColor,fill opacity=0.20] (224.85, 65.57) circle (  2.13);

\path[fill=fillColor,fill opacity=0.20] (225.50, 85.08) circle (  2.13);

\path[fill=fillColor,fill opacity=0.20] (223.10, 96.45) circle (  2.13);

\path[fill=fillColor,fill opacity=0.20] (218.07, 92.39) circle (  2.13);

\path[fill=fillColor,fill opacity=0.20] (209.33, 86.70) circle (  2.13);

\path[fill=fillColor,fill opacity=0.20] (213.70, 92.39) circle (  2.13);

\path[fill=fillColor,fill opacity=0.20] (209.55,102.14) circle (  2.13);

\path[fill=fillColor,fill opacity=0.20] (211.52, 57.44) circle (  2.13);

\path[fill=fillColor,fill opacity=0.20] (216.98, 67.20) circle (  2.13);

\path[fill=fillColor,fill opacity=0.20] (223.53, 66.38) circle (  2.13);

\path[fill=fillColor,fill opacity=0.20] (230.53, 63.95) circle (  2.13);

\path[fill=fillColor,fill opacity=0.20] (215.01, 68.01) circle (  2.13);

\path[fill=fillColor,fill opacity=0.20] (211.52, 56.63) circle (  2.13);

\path[fill=fillColor,fill opacity=0.20] (177.65, 55.82) circle (  2.13);

\path[fill=fillColor,fill opacity=0.20] (188.57, 51.75) circle (  2.13);

\path[fill=fillColor,fill opacity=0.20] (208.89, 59.88) circle (  2.13);

\path[fill=fillColor,fill opacity=0.20] (229.43, 74.51) circle (  2.13);

\path[fill=fillColor,fill opacity=0.20] (229.00, 83.45) circle (  2.13);

\path[fill=fillColor,fill opacity=0.20] (222.00, 88.33) circle (  2.13);

\path[fill=fillColor,fill opacity=0.20] (212.39, 86.70) circle (  2.13);

\path[fill=fillColor,fill opacity=0.20] (209.33, 96.45) circle (  2.13);

\path[fill=fillColor,fill opacity=0.20] (207.58,102.96) circle (  2.13);

\path[fill=fillColor,fill opacity=0.20] (199.94, 94.02) circle (  2.13);

\path[fill=fillColor,fill opacity=0.20] (197.97, 85.08) circle (  2.13);

\path[fill=fillColor,fill opacity=0.20] (198.63, 89.14) circle (  2.13);

\path[fill=fillColor,fill opacity=0.20] (203.00, 76.95) circle (  2.13);

\path[fill=fillColor,fill opacity=0.20] (206.93, 65.57) circle (  2.13);

\path[fill=fillColor,fill opacity=0.20] (208.24, 63.13) circle (  2.13);

\path[fill=fillColor,fill opacity=0.20] (211.74, 63.95) circle (  2.13);

\path[fill=fillColor,fill opacity=0.20] (216.32, 72.07) circle (  2.13);

\path[fill=fillColor,fill opacity=0.20] (229.00, 76.95) circle (  2.13);

\path[fill=fillColor,fill opacity=0.20] (223.53, 72.89) circle (  2.13);

\path[fill=fillColor,fill opacity=0.20] (197.31, 66.38) circle (  2.13);

\path[fill=fillColor,fill opacity=0.20] (166.51, 59.88) circle (  2.13);

\path[fill=fillColor,fill opacity=0.20] (174.15, 63.95) circle (  2.13);

\path[fill=fillColor,fill opacity=0.20] (189.67, 60.69) circle (  2.13);

\path[fill=fillColor,fill opacity=0.20] (213.26, 63.95) circle (  2.13);

\path[fill=fillColor,fill opacity=0.20] (221.35, 72.07) circle (  2.13);

\path[fill=fillColor,fill opacity=0.20] (221.13, 75.32) circle (  2.13);

\path[fill=fillColor,fill opacity=0.20] (217.63, 86.70) circle (  2.13);

\path[fill=fillColor,fill opacity=0.20] (212.17,102.14) circle (  2.13);

\path[fill=fillColor,fill opacity=0.20] (206.27,100.52) circle (  2.13);

\path[fill=fillColor,fill opacity=0.20] (205.84, 90.76) circle (  2.13);

\path[fill=fillColor,fill opacity=0.20] (201.25, 85.08) circle (  2.13);

\path[fill=fillColor,fill opacity=0.20] (195.57, 81.82) circle (  2.13);

\path[fill=fillColor,fill opacity=0.20] (210.64, 85.89) circle (  2.13);

\path[fill=fillColor,fill opacity=0.20] (199.72, 81.01) circle (  2.13);

\path[fill=fillColor,fill opacity=0.20] (199.94, 78.57) circle (  2.13);

\path[fill=fillColor,fill opacity=0.20] (201.68, 69.63) circle (  2.13);

\path[fill=fillColor,fill opacity=0.20] (199.06, 63.13) circle (  2.13);

\path[fill=fillColor,fill opacity=0.20] (200.37, 64.76) circle (  2.13);

\path[fill=fillColor,fill opacity=0.20] (203.43, 70.45) circle (  2.13);

\path[fill=fillColor,fill opacity=0.20] (220.69, 72.07) circle (  2.13);

\path[fill=fillColor,fill opacity=0.20] (224.85, 69.63) circle (  2.13);

\path[fill=fillColor,fill opacity=0.20] (221.35, 77.76) circle (  2.13);

\path[fill=fillColor,fill opacity=0.20] (213.70, 78.57) circle (  2.13);

\path[fill=fillColor,fill opacity=0.20] (197.31, 67.20) circle (  2.13);

\path[fill=fillColor,fill opacity=0.20] (171.09, 64.76) circle (  2.13);

\path[fill=fillColor,fill opacity=0.20] (167.38, 73.70) circle (  2.13);

\path[fill=fillColor,fill opacity=0.20] (175.68, 67.20) circle (  2.13);

\path[fill=fillColor,fill opacity=0.20] (202.56, 60.69) circle (  2.13);

\path[fill=fillColor,fill opacity=0.20] (211.52, 58.26) circle (  2.13);

\path[fill=fillColor,fill opacity=0.20] (215.89, 62.32) circle (  2.13);

\path[fill=fillColor,fill opacity=0.20] (221.57, 72.89) circle (  2.13);

\path[fill=fillColor,fill opacity=0.20] (217.42, 83.45) circle (  2.13);

\path[fill=fillColor,fill opacity=0.20] (219.60, 89.95) circle (  2.13);

\path[fill=fillColor,fill opacity=0.20] (217.85, 88.33) circle (  2.13);

\path[fill=fillColor,fill opacity=0.20] (208.68, 84.26) circle (  2.13);

\path[fill=fillColor,fill opacity=0.20] (206.05, 87.51) circle (  2.13);

\path[fill=fillColor,fill opacity=0.20] (205.40, 88.33) circle (  2.13);

\path[fill=fillColor,fill opacity=0.20] (201.47, 81.01) circle (  2.13);

\path[fill=fillColor,fill opacity=0.20] (201.68, 75.32) circle (  2.13);

\path[fill=fillColor,fill opacity=0.20] (201.25, 73.70) circle (  2.13);

\path[fill=fillColor,fill opacity=0.20] (203.00, 72.89) circle (  2.13);

\path[fill=fillColor,fill opacity=0.20] (198.63, 78.57) circle (  2.13);

\path[fill=fillColor,fill opacity=0.20] (202.56, 77.76) circle (  2.13);

\path[fill=fillColor,fill opacity=0.20] (203.65, 72.89) circle (  2.13);

\path[fill=fillColor,fill opacity=0.20] (203.00, 76.14) circle (  2.13);

\path[fill=fillColor,fill opacity=0.20] (196.22, 77.76) circle (  2.13);

\path[fill=fillColor,fill opacity=0.20] (199.50, 78.57) circle (  2.13);

\path[fill=fillColor,fill opacity=0.20] (201.25, 80.20) circle (  2.13);

\path[fill=fillColor,fill opacity=0.20] (198.41, 74.51) circle (  2.13);

\path[fill=fillColor,fill opacity=0.20] (200.81, 69.63) circle (  2.13);

\path[fill=fillColor,fill opacity=0.20] (203.21, 72.07) circle (  2.13);

\path[fill=fillColor,fill opacity=0.20] (203.87, 75.32) circle (  2.13);

\path[fill=fillColor,fill opacity=0.20] (208.68, 80.20) circle (  2.13);

\path[fill=fillColor,fill opacity=0.20] (214.36, 80.20) circle (  2.13);

\path[fill=fillColor,fill opacity=0.20] (216.98, 81.01) circle (  2.13);

\path[fill=fillColor,fill opacity=0.20] (219.38, 86.70) circle (  2.13);

\path[fill=fillColor,fill opacity=0.20] (217.63, 82.64) circle (  2.13);

\path[fill=fillColor,fill opacity=0.20] (210.42, 73.70) circle (  2.13);

\path[fill=fillColor,fill opacity=0.20] (192.73, 70.45) circle (  2.13);

\path[fill=fillColor,fill opacity=0.20] (175.46, 65.57) circle (  2.13);

\path[fill=fillColor,fill opacity=0.20] (215.89, 68.82) circle (  2.13);

\path[fill=fillColor,fill opacity=0.20] (190.76, 52.57) circle (  2.13);

\path[fill=fillColor,fill opacity=0.20] (186.39, 50.94) circle (  2.13);

\path[fill=fillColor,fill opacity=0.20] (196.00, 43.63) circle (  2.13);

\path[fill=fillColor,fill opacity=0.20] (211.52, 53.38) circle (  2.13);

\path[fill=fillColor,fill opacity=0.20] (215.45, 75.32) circle (  2.13);

\path[fill=fillColor,fill opacity=0.20] (217.20, 81.82) circle (  2.13);

\path[fill=fillColor,fill opacity=0.20] (219.38, 76.95) circle (  2.13);

\path[fill=fillColor,fill opacity=0.20] (218.95, 79.39) circle (  2.13);

\path[fill=fillColor,fill opacity=0.20] (215.01, 85.89) circle (  2.13);

\path[fill=fillColor,fill opacity=0.20] (220.04, 88.33) circle (  2.13);

\path[fill=fillColor,fill opacity=0.20] (218.95, 84.26) circle (  2.13);

\path[fill=fillColor,fill opacity=0.20] (215.45, 76.95) circle (  2.13);

\path[fill=fillColor,fill opacity=0.20] (211.08, 73.70) circle (  2.13);

\path[fill=fillColor,fill opacity=0.20] (214.14, 78.57) circle (  2.13);

\path[fill=fillColor,fill opacity=0.20] (213.05, 81.82) circle (  2.13);

\path[fill=fillColor,fill opacity=0.20] (211.95, 79.39) circle (  2.13);

\path[fill=fillColor,fill opacity=0.20] (212.39, 78.57) circle (  2.13);

\path[fill=fillColor,fill opacity=0.20] (208.24, 76.95) circle (  2.13);

\path[fill=fillColor,fill opacity=0.20] (208.68, 76.14) circle (  2.13);

\path[fill=fillColor,fill opacity=0.20] (216.32, 76.95) circle (  2.13);

\path[fill=fillColor,fill opacity=0.20] (213.70, 78.57) circle (  2.13);

\path[fill=fillColor,fill opacity=0.20] (217.42, 77.76) circle (  2.13);

\path[fill=fillColor,fill opacity=0.20] (218.29, 77.76) circle (  2.13);

\path[fill=fillColor,fill opacity=0.20] (221.57, 82.64) circle (  2.13);

\path[fill=fillColor,fill opacity=0.20] (231.40, 86.70) circle (  2.13);

\path[fill=fillColor,fill opacity=0.20] (225.06, 79.39) circle (  2.13);

\path[fill=fillColor,fill opacity=0.20] (212.61, 74.51) circle (  2.13);

\path[fill=fillColor,fill opacity=0.20] (211.74, 76.14) circle (  2.13);

\path[fill=fillColor,fill opacity=0.20] (179.83, 68.82) circle (  2.13);

\path[fill=fillColor,fill opacity=0.20] (169.56, 66.38) circle (  2.13);

\path[fill=fillColor,fill opacity=0.20] (179.40, 76.14) circle (  2.13);

\path[fill=fillColor,fill opacity=0.20] (166.07, 48.50) circle (  2.13);

\path[fill=fillColor,fill opacity=0.20] (175.68, 49.32) circle (  2.13);

\path[fill=fillColor,fill opacity=0.20] (208.46, 62.32) circle (  2.13);

\path[fill=fillColor,fill opacity=0.20] (196.88, 70.45) circle (  2.13);

\path[fill=fillColor,fill opacity=0.20] (192.29, 64.76) circle (  2.13);

\path[fill=fillColor,fill opacity=0.20] (198.84, 57.44) circle (  2.13);

\path[fill=fillColor,fill opacity=0.20] (205.84, 63.95) circle (  2.13);

\path[fill=fillColor,fill opacity=0.20] (224.85, 70.45) circle (  2.13);

\path[fill=fillColor,fill opacity=0.20] (217.85, 67.20) circle (  2.13);

\path[fill=fillColor,fill opacity=0.20] (216.54, 59.88) circle (  2.13);

\path[fill=fillColor,fill opacity=0.20] (215.23, 58.26) circle (  2.13);

\path[fill=fillColor,fill opacity=0.20] (221.57, 63.13) circle (  2.13);

\path[fill=fillColor,fill opacity=0.20] (213.48, 70.45) circle (  2.13);

\path[fill=fillColor,fill opacity=0.20] (214.14, 75.32) circle (  2.13);

\path[fill=fillColor,fill opacity=0.20] (206.27, 74.51) circle (  2.13);

\path[fill=fillColor,fill opacity=0.20] (203.43, 72.07) circle (  2.13);

\path[fill=fillColor,fill opacity=0.20] (192.73, 70.45) circle (  2.13);

\path[fill=fillColor,fill opacity=0.20] (198.41, 67.20) circle (  2.13);

\path[fill=fillColor,fill opacity=0.20] (202.34, 62.32) circle (  2.13);

\path[fill=fillColor,fill opacity=0.20] (197.31, 62.32) circle (  2.13);

\path[fill=fillColor,fill opacity=0.20] (212.83, 65.57) circle (  2.13);

\path[fill=fillColor,fill opacity=0.20] (193.38, 63.95) circle (  2.13);

\path[fill=fillColor,fill opacity=0.20] (192.07, 60.69) circle (  2.13);

\path[fill=fillColor,fill opacity=0.20] (192.94, 59.07) circle (  2.13);

\path[fill=fillColor,fill opacity=0.20] (174.81, 57.44) circle (  2.13);

\path[fill=fillColor,fill opacity=0.20] (168.47, 59.07) circle (  2.13);

\path[fill=fillColor,fill opacity=0.20] (165.19, 56.63) circle (  2.13);

\path[fill=fillColor,fill opacity=0.20] (166.94, 51.75) circle (  2.13);

\path[fill=fillColor,fill opacity=0.20] (172.41, 47.69) circle (  2.13);

\path[fill=fillColor,fill opacity=0.20] (169.35, 44.44) circle (  2.13);

\path[fill=fillColor,fill opacity=0.20] (169.35, 44.44) circle (  2.13);

\path[fill=fillColor,fill opacity=0.20] (177.21, 43.63) circle (  2.13);

\path[fill=fillColor,fill opacity=0.20] (172.41, 46.07) circle (  2.13);

\path[fill=fillColor,fill opacity=0.20] (173.50, 53.38) circle (  2.13);

\path[fill=fillColor,fill opacity=0.20] (169.78, 60.69) circle (  2.13);

\path[fill=fillColor,fill opacity=0.20] (175.25, 65.57) circle (  2.13);

\path[fill=fillColor,fill opacity=0.20] (177.65, 66.38) circle (  2.13);

\path[fill=fillColor,fill opacity=0.20] (170.00, 69.63) circle (  2.13);

\path[fill=fillColor,fill opacity=0.20] (165.19, 73.70) circle (  2.13);

\path[fill=fillColor,fill opacity=0.20] (233.59, 64.76) circle (  2.13);

\path[fill=fillColor,fill opacity=0.20] (183.77, 55.01) circle (  2.13);

\path[fill=fillColor,fill opacity=0.20] (172.62, 63.13) circle (  2.13);

\path[fill=fillColor,fill opacity=0.20] (204.09, 54.19) circle (  2.13);

\path[fill=fillColor,fill opacity=0.20] (215.01, 62.32) circle (  2.13);

\path[fill=fillColor,fill opacity=0.20] (230.74, 64.76) circle (  2.13);

\path[fill=fillColor,fill opacity=0.20] (224.85, 67.20) circle (  2.13);

\path[fill=fillColor,fill opacity=0.20] (215.89, 67.20) circle (  2.13);

\path[fill=fillColor,fill opacity=0.20] (206.05, 61.51) circle (  2.13);

\path[fill=fillColor,fill opacity=0.20] (202.56, 54.19) circle (  2.13);

\path[fill=fillColor,fill opacity=0.20] (201.90, 51.75) circle (  2.13);

\path[fill=fillColor,fill opacity=0.20] (202.78, 54.19) circle (  2.13);

\path[fill=fillColor,fill opacity=0.20] (196.66, 55.01) circle (  2.13);

\path[fill=fillColor,fill opacity=0.20] (191.85, 52.57) circle (  2.13);

\path[fill=fillColor,fill opacity=0.20] (191.63, 49.32) circle (  2.13);

\path[fill=fillColor,fill opacity=0.20] (223.10, 59.07) circle (  2.13);

\path[fill=fillColor,fill opacity=0.20] (212.39, 59.88) circle (  2.13);

\path[fill=fillColor,fill opacity=0.20] (258.71, 73.70) circle (  2.13);

\path[fill=fillColor,fill opacity=0.20] (216.98, 84.26) circle (  2.13);

\path[fill=fillColor,fill opacity=0.20] (211.74, 81.82) circle (  2.13);

\path[fill=fillColor,fill opacity=0.20] (204.52, 76.14) circle (  2.13);

\path[fill=fillColor,fill opacity=0.20] (199.06, 70.45) circle (  2.13);

\path[fill=fillColor,fill opacity=0.20] (198.41, 63.95) circle (  2.13);

\path[fill=fillColor,fill opacity=0.20] (196.22, 59.07) circle (  2.13);

\path[fill=fillColor,fill opacity=0.20] (189.23, 62.32) circle (  2.13);

\path[fill=fillColor,fill opacity=0.20] (180.71, 62.32) circle (  2.13);

\path[fill=fillColor,fill opacity=0.20] (174.37, 55.01) circle (  2.13);

\path[fill=fillColor,fill opacity=0.20] (214.58, 69.63) circle (  2.13);

\path[fill=fillColor,fill opacity=0.20] (218.29, 72.07) circle (  2.13);

\path[fill=fillColor,fill opacity=0.20] (209.33, 79.39) circle (  2.13);

\path[fill=fillColor,fill opacity=0.20] (211.08, 90.76) circle (  2.13);

\path[fill=fillColor,fill opacity=0.20] (211.08, 96.45) circle (  2.13);

\path[fill=fillColor,fill opacity=0.20] (207.58, 92.39) circle (  2.13);

\path[fill=fillColor,fill opacity=0.20] (204.52, 87.51) circle (  2.13);

\path[fill=fillColor,fill opacity=0.20] (198.19, 85.89) circle (  2.13);

\path[fill=fillColor,fill opacity=0.20] (197.75, 79.39) circle (  2.13);

\path[fill=fillColor,fill opacity=0.20] (192.73, 71.26) circle (  2.13);

\path[fill=fillColor,fill opacity=0.20] (180.05, 69.63) circle (  2.13);

\path[fill=fillColor,fill opacity=0.20] (213.70, 81.01) circle (  2.13);

\path[fill=fillColor,fill opacity=0.20] (252.16, 85.08) circle (  2.13);

\path[fill=fillColor,fill opacity=0.20] (214.79, 77.76) circle (  2.13);

\path[fill=fillColor,fill opacity=0.20] (213.26, 86.70) circle (  2.13);

\path[fill=fillColor,fill opacity=0.20] (208.68, 99.70) circle (  2.13);

\path[fill=fillColor,fill opacity=0.20] (205.84,104.58) circle (  2.13);

\path[fill=fillColor,fill opacity=0.20] (205.18,100.52) circle (  2.13);

\path[fill=fillColor,fill opacity=0.20] (203.65, 94.02) circle (  2.13);

\path[fill=fillColor,fill opacity=0.20] (201.47, 89.14) circle (  2.13);

\path[fill=fillColor,fill opacity=0.20] (197.53, 89.14) circle (  2.13);

\path[fill=fillColor,fill opacity=0.20] (191.20, 85.08) circle (  2.13);

\path[fill=fillColor,fill opacity=0.20] (183.11, 74.51) circle (  2.13);

\path[fill=fillColor,fill opacity=0.20] (168.25, 68.82) circle (  2.13);

\path[fill=fillColor,fill opacity=0.20] (199.06, 72.07) circle (  2.13);

\path[fill=fillColor,fill opacity=0.20] (203.65, 79.39) circle (  2.13);

\path[fill=fillColor,fill opacity=0.20] (212.39, 85.89) circle (  2.13);

\path[fill=fillColor,fill opacity=0.20] (193.60, 72.07) circle (  2.13);

\path[fill=fillColor,fill opacity=0.20] (215.23, 73.70) circle (  2.13);

\path[fill=fillColor,fill opacity=0.20] (215.67, 86.70) circle (  2.13);

\path[fill=fillColor,fill opacity=0.20] (206.71,105.39) circle (  2.13);

\path[fill=fillColor,fill opacity=0.20] (204.96,107.02) circle (  2.13);

\path[fill=fillColor,fill opacity=0.20] (202.34, 98.89) circle (  2.13);

\path[fill=fillColor,fill opacity=0.20] (200.15, 93.20) circle (  2.13);

\path[fill=fillColor,fill opacity=0.20] (199.94, 88.33) circle (  2.13);

\path[fill=fillColor,fill opacity=0.20] (197.10, 85.08) circle (  2.13);

\path[fill=fillColor,fill opacity=0.20] (187.26, 81.01) circle (  2.13);

\path[fill=fillColor,fill opacity=0.20] (205.62, 72.07) circle (  2.13);

\path[fill=fillColor,fill opacity=0.20] (205.40, 75.32) circle (  2.13);

\path[fill=fillColor,fill opacity=0.20] (214.36, 84.26) circle (  2.13);

\path[fill=fillColor,fill opacity=0.20] (224.19, 84.26) circle (  2.13);

\path[fill=fillColor,fill opacity=0.20] (225.06, 85.08) circle (  2.13);

\path[fill=fillColor,fill opacity=0.20] (219.60, 89.14) circle (  2.13);

\path[fill=fillColor,fill opacity=0.20] (213.05, 94.02) circle (  2.13);

\path[fill=fillColor,fill opacity=0.20] (207.37, 99.70) circle (  2.13);

\path[fill=fillColor,fill opacity=0.20] (193.82, 68.82) circle (  2.13);

\path[fill=fillColor,fill opacity=0.20] (218.07, 66.38) circle (  2.13);

\path[fill=fillColor,fill opacity=0.20] (212.83, 81.82) circle (  2.13);

\path[fill=fillColor,fill opacity=0.20] (206.27,102.96) circle (  2.13);

\path[fill=fillColor,fill opacity=0.20] (205.18,102.14) circle (  2.13);

\path[fill=fillColor,fill opacity=0.20] (200.15, 92.39) circle (  2.13);

\path[fill=fillColor,fill opacity=0.20] (197.10, 89.95) circle (  2.13);

\path[fill=fillColor,fill opacity=0.20] (197.97, 87.51) circle (  2.13);

\path[fill=fillColor,fill opacity=0.20] (194.91, 83.45) circle (  2.13);

\path[fill=fillColor,fill opacity=0.20] (208.89, 75.32) circle (  2.13);

\path[fill=fillColor,fill opacity=0.20] (218.29, 85.89) circle (  2.13);

\path[fill=fillColor,fill opacity=0.20] (222.22, 89.95) circle (  2.13);

\path[fill=fillColor,fill opacity=0.20] (219.38, 87.51) circle (  2.13);

\path[fill=fillColor,fill opacity=0.20] (234.46, 85.08) circle (  2.13);

\path[fill=fillColor,fill opacity=0.20] (229.87, 85.89) circle (  2.13);

\path[fill=fillColor,fill opacity=0.20] (219.60, 92.39) circle (  2.13);

\path[fill=fillColor,fill opacity=0.20] (215.89, 98.08) circle (  2.13);

\path[fill=fillColor,fill opacity=0.20] (210.64, 97.27) circle (  2.13);

\path[fill=fillColor,fill opacity=0.20] (227.90, 95.64) circle (  2.13);

\path[fill=fillColor,fill opacity=0.20] (191.41, 72.89) circle (  2.13);

\path[fill=fillColor,fill opacity=0.20] (217.85, 72.89) circle (  2.13);

\path[fill=fillColor,fill opacity=0.20] (207.80, 81.82) circle (  2.13);

\path[fill=fillColor,fill opacity=0.20] (202.12,100.52) circle (  2.13);

\path[fill=fillColor,fill opacity=0.20] (203.00,103.77) circle (  2.13);

\path[fill=fillColor,fill opacity=0.20] (200.81, 92.39) circle (  2.13);

\path[fill=fillColor,fill opacity=0.20] (198.63, 85.08) circle (  2.13);

\path[fill=fillColor,fill opacity=0.20] (197.75, 85.08) circle (  2.13);

\path[fill=fillColor,fill opacity=0.20] (195.13, 85.89) circle (  2.13);

\path[fill=fillColor,fill opacity=0.20] (206.71, 87.51) circle (  2.13);

\path[fill=fillColor,fill opacity=0.20] (211.30, 63.95) circle (  2.13);

\path[fill=fillColor,fill opacity=0.20] (211.95, 81.01) circle (  2.13);

\path[fill=fillColor,fill opacity=0.20] (219.60, 88.33) circle (  2.13);

\path[fill=fillColor,fill opacity=0.20] (234.68, 85.08) circle (  2.13);

\path[fill=fillColor,fill opacity=0.20] (238.61, 84.26) circle (  2.13);

\path[fill=fillColor,fill opacity=0.20] (237.52, 88.33) circle (  2.13);

\path[fill=fillColor,fill opacity=0.20] (230.09, 93.20) circle (  2.13);

\path[fill=fillColor,fill opacity=0.20] (224.41, 90.76) circle (  2.13);

\path[fill=fillColor,fill opacity=0.20] (213.05, 85.08) circle (  2.13);

\path[fill=fillColor,fill opacity=0.20] (201.68, 93.20) circle (  2.13);

\path[fill=fillColor,fill opacity=0.20] (196.22,101.33) circle (  2.13);

\path[fill=fillColor,fill opacity=0.20] (212.83, 75.32) circle (  2.13);

\path[fill=fillColor,fill opacity=0.20] (215.23, 78.57) circle (  2.13);

\path[fill=fillColor,fill opacity=0.20] (203.21, 96.45) circle (  2.13);

\path[fill=fillColor,fill opacity=0.20] (203.43,107.83) circle (  2.13);

\path[fill=fillColor,fill opacity=0.20] (210.42, 97.27) circle (  2.13);

\path[fill=fillColor,fill opacity=0.20] (203.65, 80.20) circle (  2.13);

\path[fill=fillColor,fill opacity=0.20] (196.00, 78.57) circle (  2.13);

\path[fill=fillColor,fill opacity=0.20] (194.04, 84.26) circle (  2.13);

\path[fill=fillColor,fill opacity=0.20] (194.91, 79.39) circle (  2.13);

\path[fill=fillColor,fill opacity=0.20] (188.57, 68.82) circle (  2.13);

\path[fill=fillColor,fill opacity=0.20] (219.16, 68.82) circle (  2.13);

\path[fill=fillColor,fill opacity=0.20] (215.45, 52.57) circle (  2.13);

\path[fill=fillColor,fill opacity=0.20] (217.42, 72.89) circle (  2.13);

\path[fill=fillColor,fill opacity=0.20] (221.35, 79.39) circle (  2.13);

\path[fill=fillColor,fill opacity=0.20] (227.03, 79.39) circle (  2.13);

\path[fill=fillColor,fill opacity=0.20] (229.22, 87.51) circle (  2.13);

\path[fill=fillColor,fill opacity=0.20] (234.02, 94.83) circle (  2.13);

\path[fill=fillColor,fill opacity=0.20] (239.27, 95.64) circle (  2.13);

\path[fill=fillColor,fill opacity=0.20] (237.96, 89.95) circle (  2.13);

\path[fill=fillColor,fill opacity=0.20] (226.16, 81.01) circle (  2.13);

\path[fill=fillColor,fill opacity=0.20] (218.07, 88.33) circle (  2.13);

\path[fill=fillColor,fill opacity=0.20] (196.00, 99.70) circle (  2.13);

\path[fill=fillColor,fill opacity=0.20] (196.00, 85.89) circle (  2.13);

\path[fill=fillColor,fill opacity=0.20] (208.46, 63.95) circle (  2.13);

\path[fill=fillColor,fill opacity=0.20] (222.44, 66.38) circle (  2.13);

\path[fill=fillColor,fill opacity=0.20] (211.74, 86.70) circle (  2.13);

\path[fill=fillColor,fill opacity=0.20] (207.15,101.33) circle (  2.13);

\path[fill=fillColor,fill opacity=0.20] (205.40, 92.39) circle (  2.13);

\path[fill=fillColor,fill opacity=0.20] (198.84, 77.76) circle (  2.13);

\path[fill=fillColor,fill opacity=0.20] (192.94, 76.14) circle (  2.13);

\path[fill=fillColor,fill opacity=0.20] (194.91, 77.76) circle (  2.13);

\path[fill=fillColor,fill opacity=0.20] (196.44, 72.07) circle (  2.13);

\path[fill=fillColor,fill opacity=0.20] (193.82, 64.76) circle (  2.13);

\path[fill=fillColor,fill opacity=0.20] (225.28, 67.20) circle (  2.13);

\path[fill=fillColor,fill opacity=0.20] (228.56, 59.07) circle (  2.13);

\path[fill=fillColor,fill opacity=0.20] (218.07, 75.32) circle (  2.13);

\path[fill=fillColor,fill opacity=0.20] (222.88, 76.95) circle (  2.13);

\path[fill=fillColor,fill opacity=0.20] (225.72, 82.64) circle (  2.13);

\path[fill=fillColor,fill opacity=0.20] (226.16, 94.02) circle (  2.13);

\path[fill=fillColor,fill opacity=0.20] (230.09, 96.45) circle (  2.13);

\path[fill=fillColor,fill opacity=0.20] (235.99, 93.20) circle (  2.13);

\path[fill=fillColor,fill opacity=0.20] (238.39, 93.20) circle (  2.13);

\path[fill=fillColor,fill opacity=0.20] (234.68, 87.51) circle (  2.13);

\path[fill=fillColor,fill opacity=0.20] (224.41, 85.08) circle (  2.13);

\path[fill=fillColor,fill opacity=0.20] (204.74, 91.58) circle (  2.13);

\path[fill=fillColor,fill opacity=0.20] (196.00, 90.76) circle (  2.13);

\path[fill=fillColor,fill opacity=0.20] (196.00, 57.44) circle (  2.13);

\path[fill=fillColor,fill opacity=0.20] (219.16, 58.26) circle (  2.13);

\path[fill=fillColor,fill opacity=0.20] (214.14, 76.14) circle (  2.13);

\path[fill=fillColor,fill opacity=0.20] (205.84, 89.14) circle (  2.13);

\path[fill=fillColor,fill opacity=0.20] (204.31, 83.45) circle (  2.13);

\path[fill=fillColor,fill opacity=0.20] (200.15, 77.76) circle (  2.13);

\path[fill=fillColor,fill opacity=0.20] (199.28, 80.20) circle (  2.13);

\path[fill=fillColor,fill opacity=0.20] (197.53, 79.39) circle (  2.13);

\path[fill=fillColor,fill opacity=0.20] (200.15, 72.89) circle (  2.13);

\path[fill=fillColor,fill opacity=0.20] (198.84, 63.13) circle (  2.13);

\path[fill=fillColor,fill opacity=0.20] (194.04, 59.88) circle (  2.13);

\path[fill=fillColor,fill opacity=0.20] (222.88, 76.95) circle (  2.13);

\path[fill=fillColor,fill opacity=0.20] (231.62, 69.63) circle (  2.13);

\path[fill=fillColor,fill opacity=0.20] (226.16, 80.20) circle (  2.13);

\path[fill=fillColor,fill opacity=0.20] (224.41, 83.45) circle (  2.13);

\path[fill=fillColor,fill opacity=0.20] (220.04, 94.83) circle (  2.13);

\path[fill=fillColor,fill opacity=0.20] (226.16, 96.45) circle (  2.13);

\path[fill=fillColor,fill opacity=0.20] (232.06, 90.76) circle (  2.13);

\path[fill=fillColor,fill opacity=0.20] (234.02, 94.83) circle (  2.13);

\path[fill=fillColor,fill opacity=0.20] (234.02, 98.08) circle (  2.13);

\path[fill=fillColor,fill opacity=0.20] (238.17, 89.14) circle (  2.13);

\path[fill=fillColor,fill opacity=0.20] (231.40, 81.01) circle (  2.13);

\path[fill=fillColor,fill opacity=0.20] (215.23, 82.64) circle (  2.13);

\path[fill=fillColor,fill opacity=0.20] (196.00, 88.33) circle (  2.13);

\path[fill=fillColor,fill opacity=0.20] (177.87, 65.57) circle (  2.13);

\path[fill=fillColor,fill opacity=0.20] (203.65, 59.07) circle (  2.13);

\path[fill=fillColor,fill opacity=0.20] (209.99, 66.38) circle (  2.13);

\path[fill=fillColor,fill opacity=0.20] (206.71, 80.20) circle (  2.13);

\path[fill=fillColor,fill opacity=0.20] (205.40, 82.64) circle (  2.13);

\path[fill=fillColor,fill opacity=0.20] (207.15, 79.39) circle (  2.13);

\path[fill=fillColor,fill opacity=0.20] (204.52, 81.82) circle (  2.13);

\path[fill=fillColor,fill opacity=0.20] (203.43, 85.89) circle (  2.13);

\path[fill=fillColor,fill opacity=0.20] (199.28, 82.64) circle (  2.13);

\path[fill=fillColor,fill opacity=0.20] (201.90, 68.01) circle (  2.13);

\path[fill=fillColor,fill opacity=0.20] (198.84, 57.44) circle (  2.13);

\path[fill=fillColor,fill opacity=0.20] (192.73, 62.32) circle (  2.13);

\path[fill=fillColor,fill opacity=0.20] (186.83, 74.51) circle (  2.13);

\path[fill=fillColor,fill opacity=0.20] (219.16, 75.32) circle (  2.13);

\path[fill=fillColor,fill opacity=0.20] (227.03, 81.82) circle (  2.13);

\path[fill=fillColor,fill opacity=0.20] (225.50, 87.51) circle (  2.13);

\path[fill=fillColor,fill opacity=0.20] (223.75, 98.89) circle (  2.13);

\path[fill=fillColor,fill opacity=0.20] (232.71, 94.83) circle (  2.13);

\path[fill=fillColor,fill opacity=0.20] (236.21, 85.89) circle (  2.13);

\path[fill=fillColor,fill opacity=0.20] (237.08, 98.08) circle (  2.13);

\path[fill=fillColor,fill opacity=0.20] (235.33,101.33) circle (  2.13);

\path[fill=fillColor,fill opacity=0.20] (236.21, 86.70) circle (  2.13);

\path[fill=fillColor,fill opacity=0.20] (228.56, 81.01) circle (  2.13);

\path[fill=fillColor,fill opacity=0.20] (215.23, 81.82) circle (  2.13);

\path[fill=fillColor,fill opacity=0.20] (199.50, 81.01) circle (  2.13);

\path[fill=fillColor,fill opacity=0.20] (181.58, 98.08) circle (  2.13);

\path[fill=fillColor,fill opacity=0.20] (179.18, 62.32) circle (  2.13);

\path[fill=fillColor,fill opacity=0.20] (203.43, 59.88) circle (  2.13);

\path[fill=fillColor,fill opacity=0.20] (208.89, 67.20) circle (  2.13);

\path[fill=fillColor,fill opacity=0.20] (209.77, 76.95) circle (  2.13);

\path[fill=fillColor,fill opacity=0.20] (204.52, 80.20) circle (  2.13);

\path[fill=fillColor,fill opacity=0.20] (206.05, 80.20) circle (  2.13);

\path[fill=fillColor,fill opacity=0.20] (205.18, 84.26) circle (  2.13);

\path[fill=fillColor,fill opacity=0.20] (203.87, 85.08) circle (  2.13);

\path[fill=fillColor,fill opacity=0.20] (206.71, 75.32) circle (  2.13);

\path[fill=fillColor,fill opacity=0.20] (202.34, 66.38) circle (  2.13);

\path[fill=fillColor,fill opacity=0.20] (194.47, 67.20) circle (  2.13);

\path[fill=fillColor,fill opacity=0.20] (194.69, 70.45) circle (  2.13);

\path[fill=fillColor,fill opacity=0.20] (190.54, 70.45) circle (  2.13);

\path[fill=fillColor,fill opacity=0.20] (204.52, 83.45) circle (  2.13);

\path[fill=fillColor,fill opacity=0.20] (217.20, 76.95) circle (  2.13);

\path[fill=fillColor,fill opacity=0.20] (222.22, 81.82) circle (  2.13);

\path[fill=fillColor,fill opacity=0.20] (228.12, 84.26) circle (  2.13);

\path[fill=fillColor,fill opacity=0.20] (229.00, 91.58) circle (  2.13);

\path[fill=fillColor,fill opacity=0.20] (233.15, 92.39) circle (  2.13);

\path[fill=fillColor,fill opacity=0.20] (231.40, 88.33) circle (  2.13);

\path[fill=fillColor,fill opacity=0.20] (234.02, 96.45) circle (  2.13);

\path[fill=fillColor,fill opacity=0.20] (232.49, 97.27) circle (  2.13);

\path[fill=fillColor,fill opacity=0.20] (230.74, 87.51) circle (  2.13);

\path[fill=fillColor,fill opacity=0.20] (232.27, 86.70) circle (  2.13);

\path[fill=fillColor,fill opacity=0.20] (222.88, 84.26) circle (  2.13);

\path[fill=fillColor,fill opacity=0.20] (199.28, 76.95) circle (  2.13);

\path[fill=fillColor,fill opacity=0.20] (197.31, 99.70) circle (  2.13);

\path[fill=fillColor,fill opacity=0.20] (183.99, 55.82) circle (  2.13);

\path[fill=fillColor,fill opacity=0.20] (203.21, 49.32) circle (  2.13);

\path[fill=fillColor,fill opacity=0.20] (204.74, 59.07) circle (  2.13);

\path[fill=fillColor,fill opacity=0.20] (205.84, 72.89) circle (  2.13);

\path[fill=fillColor,fill opacity=0.20] (208.68, 76.95) circle (  2.13);

\path[fill=fillColor,fill opacity=0.20] (205.18, 76.95) circle (  2.13);

\path[fill=fillColor,fill opacity=0.20] (224.41, 78.57) circle (  2.13);

\path[fill=fillColor,fill opacity=0.20] (203.87, 78.57) circle (  2.13);

\path[fill=fillColor,fill opacity=0.20] (200.37, 76.95) circle (  2.13);

\path[fill=fillColor,fill opacity=0.20] (198.63, 75.32) circle (  2.13);

\path[fill=fillColor,fill opacity=0.20] (198.63, 72.89) circle (  2.13);

\path[fill=fillColor,fill opacity=0.20] (196.44, 72.89) circle (  2.13);

\path[fill=fillColor,fill opacity=0.20] (194.26, 65.57) circle (  2.13);

\path[fill=fillColor,fill opacity=0.20] (191.41, 59.07) circle (  2.13);

\path[fill=fillColor,fill opacity=0.20] (196.88, 89.14) circle (  2.13);

\path[fill=fillColor,fill opacity=0.20] (211.95, 79.39) circle (  2.13);

\path[fill=fillColor,fill opacity=0.20] (217.63, 76.95) circle (  2.13);

\path[fill=fillColor,fill opacity=0.20] (219.82, 82.64) circle (  2.13);

\path[fill=fillColor,fill opacity=0.20] (220.48, 82.64) circle (  2.13);

\path[fill=fillColor,fill opacity=0.20] (226.59, 80.20) circle (  2.13);

\path[fill=fillColor,fill opacity=0.20] (239.92, 82.64) circle (  2.13);

\path[fill=fillColor,fill opacity=0.20] (228.12, 89.95) circle (  2.13);

\path[fill=fillColor,fill opacity=0.20] (231.84, 92.39) circle (  2.13);

\path[fill=fillColor,fill opacity=0.20] (233.59, 86.70) circle (  2.13);

\path[fill=fillColor,fill opacity=0.20] (224.19, 89.95) circle (  2.13);

\path[fill=fillColor,fill opacity=0.20] (232.49, 94.02) circle (  2.13);

\path[fill=fillColor,fill opacity=0.20] (236.64, 84.26) circle (  2.13);

\path[fill=fillColor,fill opacity=0.20] (200.15, 79.39) circle (  2.13);

\path[fill=fillColor,fill opacity=0.20] (218.73,101.33) circle (  2.13);

\path[fill=fillColor,fill opacity=0.20] (173.72, 46.88) circle (  2.13);

\path[fill=fillColor,fill opacity=0.20] (186.61, 50.13) circle (  2.13);

\path[fill=fillColor,fill opacity=0.20] (199.28, 59.88) circle (  2.13);

\path[fill=fillColor,fill opacity=0.20] (205.62, 65.57) circle (  2.13);

\path[fill=fillColor,fill opacity=0.20] (205.62, 69.63) circle (  2.13);

\path[fill=fillColor,fill opacity=0.20] (203.00, 72.89) circle (  2.13);

\path[fill=fillColor,fill opacity=0.20] (208.89, 75.32) circle (  2.13);

\path[fill=fillColor,fill opacity=0.20] (199.50, 81.01) circle (  2.13);

\path[fill=fillColor,fill opacity=0.20] (198.84, 80.20) circle (  2.13);

\path[fill=fillColor,fill opacity=0.20] (197.75, 74.51) circle (  2.13);

\path[fill=fillColor,fill opacity=0.20] (195.35, 76.14) circle (  2.13);

\path[fill=fillColor,fill opacity=0.20] (195.13, 70.45) circle (  2.13);

\path[fill=fillColor,fill opacity=0.20] (196.00, 58.26) circle (  2.13);

\path[fill=fillColor,fill opacity=0.20] (193.60, 59.88) circle (  2.13);

\path[fill=fillColor,fill opacity=0.20] (189.01, 70.45) circle (  2.13);

\path[fill=fillColor,fill opacity=0.20] (193.16, 67.20) circle (  2.13);

\path[fill=fillColor,fill opacity=0.20] (192.51, 72.07) circle (  2.13);

\path[fill=fillColor,fill opacity=0.20] (197.53, 75.32) circle (  2.13);

\path[fill=fillColor,fill opacity=0.20] (205.84, 80.20) circle (  2.13);

\path[fill=fillColor,fill opacity=0.20] (214.79, 82.64) circle (  2.13);

\path[fill=fillColor,fill opacity=0.20] (217.42, 80.20) circle (  2.13);

\path[fill=fillColor,fill opacity=0.20] (216.11, 80.20) circle (  2.13);

\path[fill=fillColor,fill opacity=0.20] (214.58, 82.64) circle (  2.13);

\path[fill=fillColor,fill opacity=0.20] (217.42, 83.45) circle (  2.13);

\path[fill=fillColor,fill opacity=0.20] (228.78, 78.57) circle (  2.13);

\path[fill=fillColor,fill opacity=0.20] (226.37, 76.95) circle (  2.13);

\path[fill=fillColor,fill opacity=0.20] (232.93, 83.45) circle (  2.13);

\path[fill=fillColor,fill opacity=0.20] (230.96, 87.51) circle (  2.13);

\path[fill=fillColor,fill opacity=0.20] (229.65, 85.89) circle (  2.13);

\path[fill=fillColor,fill opacity=0.20] (223.53, 91.58) circle (  2.13);

\path[fill=fillColor,fill opacity=0.20] (233.15, 93.20) circle (  2.13);

\path[fill=fillColor,fill opacity=0.20] (221.13, 82.64) circle (  2.13);

\path[fill=fillColor,fill opacity=0.20] (199.50, 84.26) circle (  2.13);

\path[fill=fillColor,fill opacity=0.20] (204.31,101.33) circle (  2.13);

\path[fill=fillColor,fill opacity=0.20] (173.50, 60.69) circle (  2.13);

\path[fill=fillColor,fill opacity=0.20] (177.43, 58.26) circle (  2.13);

\path[fill=fillColor,fill opacity=0.20] (183.55, 55.82) circle (  2.13);

\path[fill=fillColor,fill opacity=0.20] (190.54, 60.69) circle (  2.13);

\path[fill=fillColor,fill opacity=0.20] (204.31, 62.32) circle (  2.13);

\path[fill=fillColor,fill opacity=0.20] (202.56, 65.57) circle (  2.13);

\path[fill=fillColor,fill opacity=0.20] (202.78, 77.76) circle (  2.13);

\path[fill=fillColor,fill opacity=0.20] (200.81, 84.26) circle (  2.13);

\path[fill=fillColor,fill opacity=0.20] (198.84, 77.76) circle (  2.13);

\path[fill=fillColor,fill opacity=0.20] (195.35, 76.14) circle (  2.13);

\path[fill=fillColor,fill opacity=0.20] (197.97, 75.32) circle (  2.13);

\path[fill=fillColor,fill opacity=0.20] (202.12, 69.63) circle (  2.13);

\path[fill=fillColor,fill opacity=0.20] (198.84, 66.38) circle (  2.13);

\path[fill=fillColor,fill opacity=0.20] (196.88, 55.82) circle (  2.13);

\path[fill=fillColor,fill opacity=0.20] (192.94, 46.88) circle (  2.13);

\path[fill=fillColor,fill opacity=0.20] (190.98, 52.57) circle (  2.13);

\path[fill=fillColor,fill opacity=0.20] (187.26, 63.13) circle (  2.13);

\path[fill=fillColor,fill opacity=0.20] (192.73, 64.76) circle (  2.13);

\path[fill=fillColor,fill opacity=0.20] (197.97, 69.63) circle (  2.13);

\path[fill=fillColor,fill opacity=0.20] (213.26, 70.45) circle (  2.13);

\path[fill=fillColor,fill opacity=0.20] (209.33, 75.32) circle (  2.13);

\path[fill=fillColor,fill opacity=0.20] (206.71, 79.39) circle (  2.13);

\path[fill=fillColor,fill opacity=0.20] (204.31, 78.57) circle (  2.13);

\path[fill=fillColor,fill opacity=0.20] (213.70, 84.26) circle (  2.13);

\path[fill=fillColor,fill opacity=0.20] (218.51, 86.70) circle (  2.13);

\path[fill=fillColor,fill opacity=0.20] (214.79, 81.82) circle (  2.13);

\path[fill=fillColor,fill opacity=0.20] (210.64, 79.39) circle (  2.13);

\path[fill=fillColor,fill opacity=0.20] (213.70, 82.64) circle (  2.13);

\path[fill=fillColor,fill opacity=0.20] (216.98, 84.26) circle (  2.13);

\path[fill=fillColor,fill opacity=0.20] (220.69, 85.08) circle (  2.13);

\path[fill=fillColor,fill opacity=0.20] (223.53, 83.45) circle (  2.13);

\path[fill=fillColor,fill opacity=0.20] (226.16, 81.01) circle (  2.13);

\path[fill=fillColor,fill opacity=0.20] (230.31, 85.08) circle (  2.13);

\path[fill=fillColor,fill opacity=0.20] (224.85, 93.20) circle (  2.13);

\path[fill=fillColor,fill opacity=0.20] (223.10, 92.39) circle (  2.13);

\path[fill=fillColor,fill opacity=0.20] (232.27, 83.45) circle (  2.13);

\path[fill=fillColor,fill opacity=0.20] (230.53, 78.57) circle (  2.13);

\path[fill=fillColor,fill opacity=0.20] (194.47, 82.64) circle (  2.13);

\path[fill=fillColor,fill opacity=0.20] (171.53, 93.20) circle (  2.13);

\path[fill=fillColor,fill opacity=0.20] (196.88, 66.38) circle (  2.13);

\path[fill=fillColor,fill opacity=0.20] (174.37, 61.51) circle (  2.13);

\path[fill=fillColor,fill opacity=0.20] (184.42, 57.44) circle (  2.13);

\path[fill=fillColor,fill opacity=0.20] (198.84, 59.07) circle (  2.13);

\path[fill=fillColor,fill opacity=0.20] (205.40, 63.95) circle (  2.13);

\path[fill=fillColor,fill opacity=0.20] (205.40, 71.26) circle (  2.13);

\path[fill=fillColor,fill opacity=0.20] (205.62, 72.89) circle (  2.13);

\path[fill=fillColor,fill opacity=0.20] (204.96, 71.26) circle (  2.13);

\path[fill=fillColor,fill opacity=0.20] (204.96, 72.07) circle (  2.13);

\path[fill=fillColor,fill opacity=0.20] (205.62, 72.07) circle (  2.13);

\path[fill=fillColor,fill opacity=0.20] (201.03, 67.20) circle (  2.13);

\path[fill=fillColor,fill opacity=0.20] (196.44, 59.07) circle (  2.13);

\path[fill=fillColor,fill opacity=0.20] (194.91, 59.88) circle (  2.13);

\path[fill=fillColor,fill opacity=0.20] (194.91, 63.95) circle (  2.13);

\path[fill=fillColor,fill opacity=0.20] (198.84, 56.63) circle (  2.13);

\path[fill=fillColor,fill opacity=0.20] (193.60, 54.19) circle (  2.13);

\path[fill=fillColor,fill opacity=0.20] (194.26, 75.32) circle (  2.13);

\path[fill=fillColor,fill opacity=0.20] (196.44, 79.39) circle (  2.13);

\path[fill=fillColor,fill opacity=0.20] (201.03, 76.95) circle (  2.13);

\path[fill=fillColor,fill opacity=0.20] (209.55, 71.26) circle (  2.13);

\path[fill=fillColor,fill opacity=0.20] (210.86, 73.70) circle (  2.13);

\path[fill=fillColor,fill opacity=0.20] (206.27, 79.39) circle (  2.13);

\path[fill=fillColor,fill opacity=0.20] (205.40, 79.39) circle (  2.13);

\path[fill=fillColor,fill opacity=0.20] (216.98, 82.64) circle (  2.13);

\path[fill=fillColor,fill opacity=0.20] (226.81, 85.08) circle (  2.13);

\path[fill=fillColor,fill opacity=0.20] (216.54, 76.95) circle (  2.13);

\path[fill=fillColor,fill opacity=0.20] (218.51, 74.51) circle (  2.13);

\path[fill=fillColor,fill opacity=0.20] (221.35, 82.64) circle (  2.13);

\path[fill=fillColor,fill opacity=0.20] (217.42, 85.89) circle (  2.13);

\path[fill=fillColor,fill opacity=0.20] (216.98, 87.51) circle (  2.13);

\path[fill=fillColor,fill opacity=0.20] (218.51, 89.14) circle (  2.13);

\path[fill=fillColor,fill opacity=0.20] (221.35, 84.26) circle (  2.13);

\path[fill=fillColor,fill opacity=0.20] (222.44, 85.08) circle (  2.13);

\path[fill=fillColor,fill opacity=0.20] (217.63, 96.45) circle (  2.13);

\path[fill=fillColor,fill opacity=0.20] (220.69, 92.39) circle (  2.13);

\path[fill=fillColor,fill opacity=0.20] (230.74, 76.95) circle (  2.13);

\path[fill=fillColor,fill opacity=0.20] (222.44, 69.63) circle (  2.13);

\path[fill=fillColor,fill opacity=0.20] (190.76, 69.63) circle (  2.13);

\path[fill=fillColor,fill opacity=0.20] (168.04, 82.64) circle (  2.13);

\path[fill=fillColor,fill opacity=0.20] (172.62, 69.63) circle (  2.13);

\path[fill=fillColor,fill opacity=0.20] (176.99, 61.51) circle (  2.13);

\path[fill=fillColor,fill opacity=0.20] (189.01, 49.32) circle (  2.13);

\path[fill=fillColor,fill opacity=0.20] (194.69, 49.32) circle (  2.13);

\path[fill=fillColor,fill opacity=0.20] (197.53, 55.82) circle (  2.13);

\path[fill=fillColor,fill opacity=0.20] (204.96, 55.82) circle (  2.13);

\path[fill=fillColor,fill opacity=0.20] (204.31, 62.32) circle (  2.13);

\path[fill=fillColor,fill opacity=0.20] (203.43, 71.26) circle (  2.13);

\path[fill=fillColor,fill opacity=0.20] (211.52, 65.57) circle (  2.13);

\path[fill=fillColor,fill opacity=0.20] (201.68, 63.13) circle (  2.13);

\path[fill=fillColor,fill opacity=0.20] (199.06, 77.76) circle (  2.13);

\path[fill=fillColor,fill opacity=0.20] (197.75, 76.14) circle (  2.13);

\path[fill=fillColor,fill opacity=0.20] (198.19, 55.82) circle (  2.13);

\path[fill=fillColor,fill opacity=0.20] (196.66, 50.94) circle (  2.13);

\path[fill=fillColor,fill opacity=0.20] (192.94, 67.20) circle (  2.13);

\path[fill=fillColor,fill opacity=0.20] (200.15, 63.95) circle (  2.13);

\path[fill=fillColor,fill opacity=0.20] (208.68, 51.75) circle (  2.13);

\path[fill=fillColor,fill opacity=0.20] (205.84, 42.82) circle (  2.13);

\path[fill=fillColor,fill opacity=0.20] (193.82, 55.01) circle (  2.13);

\path[fill=fillColor,fill opacity=0.20] (193.38, 69.63) circle (  2.13);

\path[fill=fillColor,fill opacity=0.20] (201.03, 75.32) circle (  2.13);

\path[fill=fillColor,fill opacity=0.20] (201.25, 73.70) circle (  2.13);

\path[fill=fillColor,fill opacity=0.20] (203.21, 66.38) circle (  2.13);

\path[fill=fillColor,fill opacity=0.20] (204.09, 63.95) circle (  2.13);

\path[fill=fillColor,fill opacity=0.20] (206.71, 73.70) circle (  2.13);

\path[fill=fillColor,fill opacity=0.20] (207.80, 82.64) circle (  2.13);

\path[fill=fillColor,fill opacity=0.20] (214.58, 82.64) circle (  2.13);

\path[fill=fillColor,fill opacity=0.20] (217.20, 79.39) circle (  2.13);

\path[fill=fillColor,fill opacity=0.20] (217.42, 73.70) circle (  2.13);

\path[fill=fillColor,fill opacity=0.20] (231.18, 73.70) circle (  2.13);

\path[fill=fillColor,fill opacity=0.20] (217.42, 81.01) circle (  2.13);

\path[fill=fillColor,fill opacity=0.20] (214.36, 84.26) circle (  2.13);

\path[fill=fillColor,fill opacity=0.20] (212.39, 85.89) circle (  2.13);

\path[fill=fillColor,fill opacity=0.20] (215.01, 85.89) circle (  2.13);

\path[fill=fillColor,fill opacity=0.20] (217.42, 82.64) circle (  2.13);

\path[fill=fillColor,fill opacity=0.20] (217.42, 82.64) circle (  2.13);

\path[fill=fillColor,fill opacity=0.20] (218.51, 89.95) circle (  2.13);

\path[fill=fillColor,fill opacity=0.20] (225.50, 87.51) circle (  2.13);

\path[fill=fillColor,fill opacity=0.20] (221.57, 71.26) circle (  2.13);

\path[fill=fillColor,fill opacity=0.20] (213.05, 58.26) circle (  2.13);

\path[fill=fillColor,fill opacity=0.20] (185.08, 61.51) circle (  2.13);

\path[fill=fillColor,fill opacity=0.20] (175.25, 52.57) circle (  2.13);

\path[fill=fillColor,fill opacity=0.20] (179.62, 51.75) circle (  2.13);

\path[fill=fillColor,fill opacity=0.20] (181.58, 49.32) circle (  2.13);

\path[fill=fillColor,fill opacity=0.20] (193.16, 45.25) circle (  2.13);

\path[fill=fillColor,fill opacity=0.20] (194.26, 55.01) circle (  2.13);

\path[fill=fillColor,fill opacity=0.20] (199.06, 62.32) circle (  2.13);

\path[fill=fillColor,fill opacity=0.20] (204.31, 55.82) circle (  2.13);

\path[fill=fillColor,fill opacity=0.20] (209.33, 59.88) circle (  2.13);

\path[fill=fillColor,fill opacity=0.20] (201.68, 76.95) circle (  2.13);

\path[fill=fillColor,fill opacity=0.20] (200.15, 73.70) circle (  2.13);

\path[fill=fillColor,fill opacity=0.20] (200.37, 58.26) circle (  2.13);

\path[fill=fillColor,fill opacity=0.20] (204.96, 59.88) circle (  2.13);

\path[fill=fillColor,fill opacity=0.20] (193.82, 72.07) circle (  2.13);

\path[fill=fillColor,fill opacity=0.20] (192.29, 72.07) circle (  2.13);

\path[fill=fillColor,fill opacity=0.20] (199.72, 60.69) circle (  2.13);

\path[fill=fillColor,fill opacity=0.20] (199.50, 52.57) circle (  2.13);

\path[fill=fillColor,fill opacity=0.20] (198.19, 57.44) circle (  2.13);

\path[fill=fillColor,fill opacity=0.20] (205.40, 51.75) circle (  2.13);

\path[fill=fillColor,fill opacity=0.20] (204.09, 42.00) circle (  2.13);

\path[fill=fillColor,fill opacity=0.20] (196.00, 50.13) circle (  2.13);

\path[fill=fillColor,fill opacity=0.20] (189.23, 68.82) circle (  2.13);

\path[fill=fillColor,fill opacity=0.20] (195.35, 76.95) circle (  2.13);

\path[fill=fillColor,fill opacity=0.20] (202.12, 74.51) circle (  2.13);

\path[fill=fillColor,fill opacity=0.20] (202.78, 64.76) circle (  2.13);

\path[fill=fillColor,fill opacity=0.20] (206.05, 50.94) circle (  2.13);

\path[fill=fillColor,fill opacity=0.20] (209.99, 47.69) circle (  2.13);

\path[fill=fillColor,fill opacity=0.20] (207.58, 55.01) circle (  2.13);

\path[fill=fillColor,fill opacity=0.20] (207.58, 59.07) circle (  2.13);

\path[fill=fillColor,fill opacity=0.20] (207.15, 63.95) circle (  2.13);

\path[fill=fillColor,fill opacity=0.20] (210.21, 72.89) circle (  2.13);

\path[fill=fillColor,fill opacity=0.20] (217.20, 76.14) circle (  2.13);

\path[fill=fillColor,fill opacity=0.20] (211.30, 75.32) circle (  2.13);

\path[fill=fillColor,fill opacity=0.20] (216.32, 75.32) circle (  2.13);

\path[fill=fillColor,fill opacity=0.20] (218.29, 76.14) circle (  2.13);

\path[fill=fillColor,fill opacity=0.20] (215.45, 76.95) circle (  2.13);

\path[fill=fillColor,fill opacity=0.20] (209.11, 78.57) circle (  2.13);

\path[fill=fillColor,fill opacity=0.20] (211.08, 81.01) circle (  2.13);

\path[fill=fillColor,fill opacity=0.20] (210.64, 80.20) circle (  2.13);

\path[fill=fillColor,fill opacity=0.20] (209.33, 76.95) circle (  2.13);

\path[fill=fillColor,fill opacity=0.20] (210.64, 74.51) circle (  2.13);

\path[fill=fillColor,fill opacity=0.20] (217.42, 76.95) circle (  2.13);

\path[fill=fillColor,fill opacity=0.20] (226.16, 73.70) circle (  2.13);

\path[fill=fillColor,fill opacity=0.20] (212.17, 59.88) circle (  2.13);

\path[fill=fillColor,fill opacity=0.20] (200.37, 55.01) circle (  2.13);

\path[fill=fillColor,fill opacity=0.20] (179.40, 74.51) circle (  2.13);

\path[fill=fillColor,fill opacity=0.20] (172.41, 46.07) circle (  2.13);

\path[fill=fillColor,fill opacity=0.20] (181.80, 50.94) circle (  2.13);

\path[fill=fillColor,fill opacity=0.20] (184.42, 51.75) circle (  2.13);

\path[fill=fillColor,fill opacity=0.20] (191.85, 48.50) circle (  2.13);

\path[fill=fillColor,fill opacity=0.20] (200.37, 51.75) circle (  2.13);

\path[fill=fillColor,fill opacity=0.20] (202.12, 56.63) circle (  2.13);

\path[fill=fillColor,fill opacity=0.20] (199.94, 56.63) circle (  2.13);

\path[fill=fillColor,fill opacity=0.20] (201.25, 55.82) circle (  2.13);

\path[fill=fillColor,fill opacity=0.20] (206.27, 56.63) circle (  2.13);

\path[fill=fillColor,fill opacity=0.20] (201.47, 61.51) circle (  2.13);

\path[fill=fillColor,fill opacity=0.20] (197.75, 67.20) circle (  2.13);

\path[fill=fillColor,fill opacity=0.20] (206.93, 63.13) circle (  2.13);

\path[fill=fillColor,fill opacity=0.20] (194.69, 59.88) circle (  2.13);

\path[fill=fillColor,fill opacity=0.20] (192.94, 63.13) circle (  2.13);

\path[fill=fillColor,fill opacity=0.20] (196.66, 59.88) circle (  2.13);

\path[fill=fillColor,fill opacity=0.20] (200.37, 55.01) circle (  2.13);

\path[fill=fillColor,fill opacity=0.20] (197.10, 60.69) circle (  2.13);

\path[fill=fillColor,fill opacity=0.20] (190.10, 86.70) circle (  2.13);

\path[fill=fillColor,fill opacity=0.20] (190.76, 73.70) circle (  2.13);

\path[fill=fillColor,fill opacity=0.20] (189.01, 64.76) circle (  2.13);

\path[fill=fillColor,fill opacity=0.20] (198.84, 68.01) circle (  2.13);

\path[fill=fillColor,fill opacity=0.20] (194.26, 66.38) circle (  2.13);

\path[fill=fillColor,fill opacity=0.20] (200.81, 52.57) circle (  2.13);

\path[fill=fillColor,fill opacity=0.20] (196.22, 46.07) circle (  2.13);

\path[fill=fillColor,fill opacity=0.20] (204.09, 50.94) circle (  2.13);

\path[fill=fillColor,fill opacity=0.20] (204.52, 48.50) circle (  2.13);

\path[fill=fillColor,fill opacity=0.20] (204.52, 46.07) circle (  2.13);

\path[fill=fillColor,fill opacity=0.20] (198.19, 52.57) circle (  2.13);

\path[fill=fillColor,fill opacity=0.20] (188.57, 54.19) circle (  2.13);

\path[fill=fillColor,fill opacity=0.20] (176.34, 61.51) circle (  2.13);

\path[fill=fillColor,fill opacity=0.20] (166.29, 85.89) circle (  2.13);

\path[fill=fillColor,fill opacity=0.20] (200.59, 73.70) circle (  2.13);

\path[fill=fillColor,fill opacity=0.20] (208.68, 71.26) circle (  2.13);

\path[fill=fillColor,fill opacity=0.20] (208.68, 71.26) circle (  2.13);

\path[fill=fillColor,fill opacity=0.20] (210.64, 65.57) circle (  2.13);

\path[fill=fillColor,fill opacity=0.20] (225.50, 60.69) circle (  2.13);

\path[fill=fillColor,fill opacity=0.20] (210.64, 64.76) circle (  2.13);

\path[fill=fillColor,fill opacity=0.20] (215.01, 68.82) circle (  2.13);

\path[fill=fillColor,fill opacity=0.20] (213.70, 70.45) circle (  2.13);

\path[fill=fillColor,fill opacity=0.20] (212.17, 71.26) circle (  2.13);

\path[fill=fillColor,fill opacity=0.20] (212.61, 72.07) circle (  2.13);

\path[fill=fillColor,fill opacity=0.20] (213.05, 72.07) circle (  2.13);

\path[fill=fillColor,fill opacity=0.20] (211.95, 72.89) circle (  2.13);

\path[fill=fillColor,fill opacity=0.20] (211.95, 72.07) circle (  2.13);

\path[fill=fillColor,fill opacity=0.20] (216.32, 70.45) circle (  2.13);

\path[fill=fillColor,fill opacity=0.20] (209.11, 67.20) circle (  2.13);

\path[fill=fillColor,fill opacity=0.20] (216.54, 63.95) circle (  2.13);

\path[fill=fillColor,fill opacity=0.20] (211.52, 59.88) circle (  2.13);

\path[fill=fillColor,fill opacity=0.20] (198.19, 58.26) circle (  2.13);

\path[fill=fillColor,fill opacity=0.20] (182.24, 66.38) circle (  2.13);

\path[fill=fillColor,fill opacity=0.20] (182.24, 48.50) circle (  2.13);

\path[fill=fillColor,fill opacity=0.20] (184.86, 50.13) circle (  2.13);

\path[fill=fillColor,fill opacity=0.20] (183.11, 48.50) circle (  2.13);

\path[fill=fillColor,fill opacity=0.20] (187.04, 50.94) circle (  2.13);

\path[fill=fillColor,fill opacity=0.20] (194.26, 53.38) circle (  2.13);

\path[fill=fillColor,fill opacity=0.20] (206.27, 46.07) circle (  2.13);

\path[fill=fillColor,fill opacity=0.20] (200.15, 42.00) circle (  2.13);

\path[fill=fillColor,fill opacity=0.20] (201.25, 49.32) circle (  2.13);

\path[fill=fillColor,fill opacity=0.20] (200.81, 55.82) circle (  2.13);

\path[fill=fillColor,fill opacity=0.20] (200.15, 63.95) circle (  2.13);

\path[fill=fillColor,fill opacity=0.20] (199.28, 72.07) circle (  2.13);

\path[fill=fillColor,fill opacity=0.20] (199.06, 68.82) circle (  2.13);

\path[fill=fillColor,fill opacity=0.20] (192.29, 60.69) circle (  2.13);

\path[fill=fillColor,fill opacity=0.20] (201.25, 60.69) circle (  2.13);

\path[fill=fillColor,fill opacity=0.20] (197.53, 65.57) circle (  2.13);

\path[fill=fillColor,fill opacity=0.20] (193.82, 71.26) circle (  2.13);

\path[fill=fillColor,fill opacity=0.20] (191.85, 75.32) circle (  2.13);

\path[fill=fillColor,fill opacity=0.20] (191.41, 73.70) circle (  2.13);

\path[fill=fillColor,fill opacity=0.20] (193.38, 68.82) circle (  2.13);

\path[fill=fillColor,fill opacity=0.20] (198.84, 67.20) circle (  2.13);

\path[fill=fillColor,fill opacity=0.20] (196.66, 71.26) circle (  2.13);

\path[fill=fillColor,fill opacity=0.20] (193.82, 77.76) circle (  2.13);

\path[fill=fillColor,fill opacity=0.20] (191.41, 87.51) circle (  2.13);

\path[fill=fillColor,fill opacity=0.20] (192.94, 80.20) circle (  2.13);

\path[fill=fillColor,fill opacity=0.20] (197.10, 73.70) circle (  2.13);

\path[fill=fillColor,fill opacity=0.20] (200.59, 73.70) circle (  2.13);

\path[fill=fillColor,fill opacity=0.20] (201.25, 73.70) circle (  2.13);

\path[fill=fillColor,fill opacity=0.20] (204.31, 70.45) circle (  2.13);

\path[fill=fillColor,fill opacity=0.20] (205.40, 67.20) circle (  2.13);

\path[fill=fillColor,fill opacity=0.20] (203.00, 61.51) circle (  2.13);

\path[fill=fillColor,fill opacity=0.20] (204.09, 58.26) circle (  2.13);

\path[fill=fillColor,fill opacity=0.20] (206.05, 64.76) circle (  2.13);

\path[fill=fillColor,fill opacity=0.20] (205.62, 63.95) circle (  2.13);

\path[fill=fillColor,fill opacity=0.20] (204.74, 50.94) circle (  2.13);

\path[fill=fillColor,fill opacity=0.20] (201.47, 44.44) circle (  2.13);

\path[fill=fillColor,fill opacity=0.20] (200.59, 50.13) circle (  2.13);

\path[fill=fillColor,fill opacity=0.20] (203.00, 53.38) circle (  2.13);

\path[fill=fillColor,fill opacity=0.20] (184.86, 59.07) circle (  2.13);

\path[fill=fillColor,fill opacity=0.20] (175.25, 72.07) circle (  2.13);

\path[fill=fillColor,fill opacity=0.20] (190.32, 91.58) circle (  2.13);

\path[fill=fillColor,fill opacity=0.20] (196.00, 85.08) circle (  2.13);

\path[fill=fillColor,fill opacity=0.20] (207.58, 79.39) circle (  2.13);

\path[fill=fillColor,fill opacity=0.20] (216.32, 69.63) circle (  2.13);

\path[fill=fillColor,fill opacity=0.20] (212.17, 63.95) circle (  2.13);

\path[fill=fillColor,fill opacity=0.20] (207.37, 64.76) circle (  2.13);

\path[fill=fillColor,fill opacity=0.20] (209.11, 67.20) circle (  2.13);

\path[fill=fillColor,fill opacity=0.20] (216.98, 68.82) circle (  2.13);

\path[fill=fillColor,fill opacity=0.20] (214.14, 68.01) circle (  2.13);

\path[fill=fillColor,fill opacity=0.20] (207.58, 64.76) circle (  2.13);

\path[fill=fillColor,fill opacity=0.20] (209.77, 67.20) circle (  2.13);

\path[fill=fillColor,fill opacity=0.20] (209.55, 69.63) circle (  2.13);

\path[fill=fillColor,fill opacity=0.20] (203.87, 65.57) circle (  2.13);

\path[fill=fillColor,fill opacity=0.20] (198.84, 68.82) circle (  2.13);

\path[fill=fillColor,fill opacity=0.20] (178.74, 63.95) circle (  2.13);

\path[fill=fillColor,fill opacity=0.20] (181.36, 51.75) circle (  2.13);

\path[fill=fillColor,fill opacity=0.20] (192.51, 37.94) circle (  2.13);

\path[fill=fillColor,fill opacity=0.20] (202.34, 46.07) circle (  2.13);

\path[fill=fillColor,fill opacity=0.20] (206.71, 54.19) circle (  2.13);

\path[fill=fillColor,fill opacity=0.20] (204.96, 58.26) circle (  2.13);

\path[fill=fillColor,fill opacity=0.20] (196.44, 59.88) circle (  2.13);

\path[fill=fillColor,fill opacity=0.20] (201.25, 60.69) circle (  2.13);

\path[fill=fillColor,fill opacity=0.20] (198.84, 62.32) circle (  2.13);

\path[fill=fillColor,fill opacity=0.20] (200.59, 66.38) circle (  2.13);

\path[fill=fillColor,fill opacity=0.20] (196.66, 72.89) circle (  2.13);

\path[fill=fillColor,fill opacity=0.20] (196.44, 77.76) circle (  2.13);

\path[fill=fillColor,fill opacity=0.20] (198.84, 76.14) circle (  2.13);

\path[fill=fillColor,fill opacity=0.20] (197.97, 73.70) circle (  2.13);

\path[fill=fillColor,fill opacity=0.20] (201.90, 70.45) circle (  2.13);

\path[fill=fillColor,fill opacity=0.20] (199.94, 66.38) circle (  2.13);

\path[fill=fillColor,fill opacity=0.20] (198.41, 67.20) circle (  2.13);

\path[fill=fillColor,fill opacity=0.20] (203.43, 68.82) circle (  2.13);

\path[fill=fillColor,fill opacity=0.20] (200.37, 70.45) circle (  2.13);

\path[fill=fillColor,fill opacity=0.20] (199.94, 73.70) circle (  2.13);

\path[fill=fillColor,fill opacity=0.20] (201.03, 76.95) circle (  2.13);

\path[fill=fillColor,fill opacity=0.20] (217.20, 77.76) circle (  2.13);

\path[fill=fillColor,fill opacity=0.20] (209.33, 75.32) circle (  2.13);

\path[fill=fillColor,fill opacity=0.20] (206.05, 71.26) circle (  2.13);

\path[fill=fillColor,fill opacity=0.20] (207.58, 70.45) circle (  2.13);

\path[fill=fillColor,fill opacity=0.20] (211.08, 70.45) circle (  2.13);

\path[fill=fillColor,fill opacity=0.20] (213.05, 62.32) circle (  2.13);

\path[fill=fillColor,fill opacity=0.20] (209.11, 55.82) circle (  2.13);

\path[fill=fillColor,fill opacity=0.20] (202.56, 59.07) circle (  2.13);

\path[fill=fillColor,fill opacity=0.20] (196.88, 65.57) circle (  2.13);

\path[fill=fillColor,fill opacity=0.20] (187.92, 66.38) circle (  2.13);

\path[fill=fillColor,fill opacity=0.20] (178.52, 68.01) circle (  2.13);

\path[fill=fillColor,fill opacity=0.20] (167.38, 85.89) circle (  2.13);

\path[fill=fillColor,fill opacity=0.20] (199.28, 76.95) circle (  2.13);

\path[fill=fillColor,fill opacity=0.20] (195.57, 74.51) circle (  2.13);

\path[fill=fillColor,fill opacity=0.20] (198.41, 71.26) circle (  2.13);

\path[fill=fillColor,fill opacity=0.20] (205.40, 71.26) circle (  2.13);

\path[fill=fillColor,fill opacity=0.20] (206.05, 70.45) circle (  2.13);

\path[fill=fillColor,fill opacity=0.20] (200.37, 68.01) circle (  2.13);

\path[fill=fillColor,fill opacity=0.20] (199.94, 72.89) circle (  2.13);

\path[fill=fillColor,fill opacity=0.20] (176.12, 56.63) circle (  2.13);

\path[fill=fillColor,fill opacity=0.20] (185.95, 45.25) circle (  2.13);

\path[fill=fillColor,fill opacity=0.20] (193.82, 37.94) circle (  2.13);

\path[fill=fillColor,fill opacity=0.20] (199.28, 39.56) circle (  2.13);

\path[fill=fillColor,fill opacity=0.20] (209.11, 42.00) circle (  2.13);

\path[fill=fillColor,fill opacity=0.20] (206.05, 49.32) circle (  2.13);

\path[fill=fillColor,fill opacity=0.20] (202.34, 58.26) circle (  2.13);

\path[fill=fillColor,fill opacity=0.20] (199.28, 62.32) circle (  2.13);

\path[fill=fillColor,fill opacity=0.20] (201.68, 64.76) circle (  2.13);

\path[fill=fillColor,fill opacity=0.20] (203.43, 69.63) circle (  2.13);

\path[fill=fillColor,fill opacity=0.20] (202.34, 72.89) circle (  2.13);

\path[fill=fillColor,fill opacity=0.20] (205.62, 76.95) circle (  2.13);

\path[fill=fillColor,fill opacity=0.20] (202.12, 81.01) circle (  2.13);

\path[fill=fillColor,fill opacity=0.20] (201.03, 80.20) circle (  2.13);

\path[fill=fillColor,fill opacity=0.20] (200.59, 76.95) circle (  2.13);

\path[fill=fillColor,fill opacity=0.20] (204.74, 76.14) circle (  2.13);

\path[fill=fillColor,fill opacity=0.20] (205.40, 76.14) circle (  2.13);

\path[fill=fillColor,fill opacity=0.20] (203.65, 75.32) circle (  2.13);

\path[fill=fillColor,fill opacity=0.20] (210.42, 76.95) circle (  2.13);

\path[fill=fillColor,fill opacity=0.20] (212.17, 81.82) circle (  2.13);

\path[fill=fillColor,fill opacity=0.20] (211.95, 80.20) circle (  2.13);

\path[fill=fillColor,fill opacity=0.20] (213.48, 69.63) circle (  2.13);

\path[fill=fillColor,fill opacity=0.20] (212.17, 63.13) circle (  2.13);

\path[fill=fillColor,fill opacity=0.20] (211.08, 63.95) circle (  2.13);

\path[fill=fillColor,fill opacity=0.20] (204.52, 62.32) circle (  2.13);

\path[fill=fillColor,fill opacity=0.20] (191.41, 58.26) circle (  2.13);

\path[fill=fillColor,fill opacity=0.20] (181.36, 63.13) circle (  2.13);

\path[fill=fillColor,fill opacity=0.20] (175.68, 72.89) circle (  2.13);

\path[fill=fillColor,fill opacity=0.20] (180.93, 48.50) circle (  2.13);

\path[fill=fillColor,fill opacity=0.20] (203.00, 42.82) circle (  2.13);

\path[fill=fillColor,fill opacity=0.20] (194.69, 39.56) circle (  2.13);

\path[fill=fillColor,fill opacity=0.20] (194.47, 46.88) circle (  2.13);

\path[fill=fillColor,fill opacity=0.20] (196.66, 50.94) circle (  2.13);

\path[fill=fillColor,fill opacity=0.20] (200.59, 47.69) circle (  2.13);

\path[fill=fillColor,fill opacity=0.20] (203.43, 51.75) circle (  2.13);

\path[fill=fillColor,fill opacity=0.20] (205.84, 63.13) circle (  2.13);

\path[fill=fillColor,fill opacity=0.20] (206.71, 68.82) circle (  2.13);

\path[fill=fillColor,fill opacity=0.20] (209.11, 71.26) circle (  2.13);

\path[fill=fillColor,fill opacity=0.20] (208.68, 74.51) circle (  2.13);

\path[fill=fillColor,fill opacity=0.20] (207.80, 76.14) circle (  2.13);

\path[fill=fillColor,fill opacity=0.20] (212.17, 73.70) circle (  2.13);

\path[fill=fillColor,fill opacity=0.20] (213.48, 72.89) circle (  2.13);

\path[fill=fillColor,fill opacity=0.20] (211.30, 71.26) circle (  2.13);

\path[fill=fillColor,fill opacity=0.20] (218.29, 67.20) circle (  2.13);

\path[fill=fillColor,fill opacity=0.20] (219.16, 67.20) circle (  2.13);

\path[fill=fillColor,fill opacity=0.20] (208.89, 68.82) circle (  2.13);

\path[fill=fillColor,fill opacity=0.20] (206.27, 63.13) circle (  2.13);

\path[fill=fillColor,fill opacity=0.20] (202.34, 54.19) circle (  2.13);

\path[fill=fillColor,fill opacity=0.20] (197.75, 53.38) circle (  2.13);

\path[fill=fillColor,fill opacity=0.20] (188.57, 59.07) circle (  2.13);

\path[fill=fillColor,fill opacity=0.20] (184.86, 46.88) circle (  2.13);

\path[fill=fillColor,fill opacity=0.20] (188.79, 46.88) circle (  2.13);

\path[fill=fillColor,fill opacity=0.20] (197.10, 47.69) circle (  2.13);

\path[fill=fillColor,fill opacity=0.20] (219.60, 55.82) circle (  2.13);

\path[fill=fillColor,fill opacity=0.20] (199.06, 56.63) circle (  2.13);

\path[fill=fillColor,fill opacity=0.20] (206.93, 47.69) circle (  2.13);

\path[fill=fillColor,fill opacity=0.20] (212.17, 47.69) circle (  2.13);

\path[fill=fillColor,fill opacity=0.20] (211.95, 56.63) circle (  2.13);

\path[fill=fillColor,fill opacity=0.20] (215.23, 58.26) circle (  2.13);

\path[fill=fillColor,fill opacity=0.20] (213.05, 59.07) circle (  2.13);

\path[fill=fillColor,fill opacity=0.20] (208.89, 63.13) circle (  2.13);

\path[fill=fillColor,fill opacity=0.20] (214.58, 63.13) circle (  2.13);

\path[fill=fillColor,fill opacity=0.20] (199.94, 57.44) circle (  2.13);

\path[fill=fillColor,fill opacity=0.20] (194.47, 55.82) circle (  2.13);

\path[fill=fillColor,fill opacity=0.20] (195.13, 57.44) circle (  2.13);

\path[fill=fillColor,fill opacity=0.20] (183.99, 55.82) circle (  2.13);

\path[fill=fillColor,fill opacity=0.20] (179.62, 53.38) circle (  2.13);

\path[fill=fillColor,fill opacity=0.20] (196.66, 45.25) circle (  2.13);

\path[fill=fillColor,fill opacity=0.20] (200.59, 39.56) circle (  2.13);

\path[fill=fillColor,fill opacity=0.20] (195.57, 43.63) circle (  2.13);

\path[fill=fillColor,fill opacity=0.20] (196.22, 50.94) circle (  2.13);

\path[fill=fillColor,fill opacity=0.20] (196.44, 56.63) circle (  2.13);

\path[fill=fillColor,fill opacity=0.20] (189.89, 61.51) circle (  2.13);

\path[fill=fillColor,fill opacity=0.20] (185.30, 65.57) circle (  2.13);

\path[fill=fillColor,fill opacity=0.20] (182.46, 63.13) circle (  2.13);

\path[fill=fillColor,fill opacity=0.20] (177.65, 56.63) circle (  2.13);

\path[fill=fillColor,fill opacity=0.20] (204.74, 89.14) circle (  2.13);

\path[fill=fillColor,fill opacity=0.20] (209.99, 93.20) circle (  2.13);

\path[fill=fillColor,fill opacity=0.20] (210.21,102.14) circle (  2.13);

\path[fill=fillColor,fill opacity=0.20] (209.55,107.83) circle (  2.13);

\path[fill=fillColor,fill opacity=0.20] (205.62,102.96) circle (  2.13);

\path[fill=fillColor,fill opacity=0.20] (180.05, 75.32) circle (  2.13);

\path[fill=fillColor,fill opacity=0.20] (177.21, 63.13) circle (  2.13);

\path[fill=fillColor,fill opacity=0.20] (182.24, 55.01) circle (  2.13);

\path[fill=fillColor,fill opacity=0.20] (183.99, 60.69) circle (  2.13);

\path[fill=fillColor,fill opacity=0.20] (183.11, 53.38) circle (  2.13);

\path[fill=fillColor,fill opacity=0.20] (180.49, 37.94) circle (  2.13);

\path[fill=fillColor,fill opacity=0.20] (201.90, 79.39) circle (  2.13);

\path[fill=fillColor,fill opacity=0.20] (215.67, 83.45) circle (  2.13);

\path[fill=fillColor,fill opacity=0.20] (217.20, 98.89) circle (  2.13);

\path[fill=fillColor,fill opacity=0.20] (223.32,103.77) circle (  2.13);

\path[fill=fillColor,fill opacity=0.20] (225.28,102.96) circle (  2.13);

\path[fill=fillColor,fill opacity=0.20] (220.04,105.39) circle (  2.13);

\path[fill=fillColor,fill opacity=0.20] (212.61,110.27) circle (  2.13);

\path[fill=fillColor,fill opacity=0.20] (202.34,114.33) circle (  2.13);

\path[fill=fillColor,fill opacity=0.20] (180.49, 72.07) circle (  2.13);

\path[fill=fillColor,fill opacity=0.20] (181.15, 68.82) circle (  2.13);

\path[fill=fillColor,fill opacity=0.20] (182.46, 77.76) circle (  2.13);

\path[fill=fillColor,fill opacity=0.20] (190.54, 72.89) circle (  2.13);

\path[fill=fillColor,fill opacity=0.20] (199.06, 66.38) circle (  2.13);

\path[fill=fillColor,fill opacity=0.20] (201.90, 70.45) circle (  2.13);

\path[fill=fillColor,fill opacity=0.20] (198.63, 66.38) circle (  2.13);

\path[fill=fillColor,fill opacity=0.20] (194.47, 59.07) circle (  2.13);

\path[fill=fillColor,fill opacity=0.20] (190.10, 53.38) circle (  2.13);

\path[fill=fillColor,fill opacity=0.20] (185.52, 50.94) circle (  2.13);

\path[fill=fillColor,fill opacity=0.20] (185.73, 90.76) circle (  2.13);

\path[fill=fillColor,fill opacity=0.20] (219.38, 75.32) circle (  2.13);

\path[fill=fillColor,fill opacity=0.20] (220.91,100.52) circle (  2.13);

\path[fill=fillColor,fill opacity=0.20] (224.19,102.14) circle (  2.13);

\path[fill=fillColor,fill opacity=0.20] (232.93,100.52) circle (  2.13);

\path[fill=fillColor,fill opacity=0.20] (234.02, 95.64) circle (  2.13);

\path[fill=fillColor,fill opacity=0.20] (229.00, 98.08) circle (  2.13);

\path[fill=fillColor,fill opacity=0.20] (217.42, 98.89) circle (  2.13);

\path[fill=fillColor,fill opacity=0.20] (206.93, 94.83) circle (  2.13);

\path[fill=fillColor,fill opacity=0.20] (197.75,104.58) circle (  2.13);

\path[fill=fillColor,fill opacity=0.20] (206.71, 65.57) circle (  2.13);

\path[fill=fillColor,fill opacity=0.20] (190.10, 69.63) circle (  2.13);

\path[fill=fillColor,fill opacity=0.20] (197.53, 78.57) circle (  2.13);

\path[fill=fillColor,fill opacity=0.20] (196.66, 78.57) circle (  2.13);

\path[fill=fillColor,fill opacity=0.20] (201.68, 76.95) circle (  2.13);

\path[fill=fillColor,fill opacity=0.20] (204.96, 89.14) circle (  2.13);

\path[fill=fillColor,fill opacity=0.20] (206.71, 94.02) circle (  2.13);

\path[fill=fillColor,fill opacity=0.20] (205.40, 77.76) circle (  2.13);

\path[fill=fillColor,fill opacity=0.20] (197.97, 69.63) circle (  2.13);

\path[fill=fillColor,fill opacity=0.20] (195.57, 72.07) circle (  2.13);

\path[fill=fillColor,fill opacity=0.20] (198.84, 72.89) circle (  2.13);

\path[fill=fillColor,fill opacity=0.20] (227.25, 90.76) circle (  2.13);

\path[fill=fillColor,fill opacity=0.20] (225.72,110.27) circle (  2.13);

\path[fill=fillColor,fill opacity=0.20] (233.59,106.21) circle (  2.13);

\path[fill=fillColor,fill opacity=0.20] (234.68, 98.08) circle (  2.13);

\path[fill=fillColor,fill opacity=0.20] (233.15, 89.14) circle (  2.13);

\path[fill=fillColor,fill opacity=0.20] (232.27, 86.70) circle (  2.13);

\path[fill=fillColor,fill opacity=0.20] (222.44, 80.20) circle (  2.13);

\path[fill=fillColor,fill opacity=0.20] (217.20, 73.70) circle (  2.13);

\path[fill=fillColor,fill opacity=0.20] (206.27, 85.08) circle (  2.13);

\path[fill=fillColor,fill opacity=0.20] (179.18, 63.13) circle (  2.13);

\path[fill=fillColor,fill opacity=0.20] (188.14, 52.57) circle (  2.13);

\path[fill=fillColor,fill opacity=0.20] (192.29, 81.82) circle (  2.13);

\path[fill=fillColor,fill opacity=0.20] (196.44,103.77) circle (  2.13);

\path[fill=fillColor,fill opacity=0.20] (199.28, 98.08) circle (  2.13);

\path[fill=fillColor,fill opacity=0.20] (204.74, 97.27) circle (  2.13);

\path[fill=fillColor,fill opacity=0.20] (208.02, 99.70) circle (  2.13);

\path[fill=fillColor,fill opacity=0.20] (206.27, 97.27) circle (  2.13);

\path[fill=fillColor,fill opacity=0.20] (205.18, 87.51) circle (  2.13);

\path[fill=fillColor,fill opacity=0.20] (194.69, 73.70) circle (  2.13);

\path[fill=fillColor,fill opacity=0.20] (185.73, 68.01) circle (  2.13);

\path[fill=fillColor,fill opacity=0.20] (177.21, 72.89) circle (  2.13);

\path[fill=fillColor,fill opacity=0.20] (197.97, 86.70) circle (  2.13);

\path[fill=fillColor,fill opacity=0.20] (225.06, 83.45) circle (  2.13);

\path[fill=fillColor,fill opacity=0.20] (225.50,105.39) circle (  2.13);

\path[fill=fillColor,fill opacity=0.20] (234.24,111.08) circle (  2.13);

\path[fill=fillColor,fill opacity=0.20] (235.33, 97.27) circle (  2.13);

\path[fill=fillColor,fill opacity=0.20] (235.11, 87.51) circle (  2.13);

\path[fill=fillColor,fill opacity=0.20] (233.80, 87.51) circle (  2.13);

\path[fill=fillColor,fill opacity=0.20] (226.37, 78.57) circle (  2.13);

\path[fill=fillColor,fill opacity=0.20] (223.75, 68.01) circle (  2.13);

\path[fill=fillColor,fill opacity=0.20] (216.11, 70.45) circle (  2.13);

\path[fill=fillColor,fill opacity=0.20] (205.62, 82.64) circle (  2.13);

\path[fill=fillColor,fill opacity=0.20] (175.25, 53.38) circle (  2.13);

\path[fill=fillColor,fill opacity=0.20] (192.94, 54.19) circle (  2.13);

\path[fill=fillColor,fill opacity=0.20] (201.68, 63.95) circle (  2.13);

\path[fill=fillColor,fill opacity=0.20] (199.94, 89.95) circle (  2.13);

\path[fill=fillColor,fill opacity=0.20] (201.25,105.39) circle (  2.13);

\path[fill=fillColor,fill opacity=0.20] (204.31,107.02) circle (  2.13);

\path[fill=fillColor,fill opacity=0.20] (207.80,104.58) circle (  2.13);

\path[fill=fillColor,fill opacity=0.20] (209.33, 94.83) circle (  2.13);

\path[fill=fillColor,fill opacity=0.20] (206.93, 85.08) circle (  2.13);

\path[fill=fillColor,fill opacity=0.20] (204.31, 87.51) circle (  2.13);

\path[fill=fillColor,fill opacity=0.20] (196.22, 74.51) circle (  2.13);

\path[fill=fillColor,fill opacity=0.20] (173.28, 59.88) circle (  2.13);

\path[fill=fillColor,fill opacity=0.20] (197.10, 78.57) circle (  2.13);

\path[fill=fillColor,fill opacity=0.20] (220.91, 75.32) circle (  2.13);

\path[fill=fillColor,fill opacity=0.20] (223.53, 94.83) circle (  2.13);

\path[fill=fillColor,fill opacity=0.20] (234.46,101.33) circle (  2.13);

\path[fill=fillColor,fill opacity=0.20] (238.17, 94.83) circle (  2.13);

\path[fill=fillColor,fill opacity=0.20] (233.15, 87.51) circle (  2.13);

\path[fill=fillColor,fill opacity=0.20] (230.31, 98.08) circle (  2.13);

\path[fill=fillColor,fill opacity=0.20] (229.22, 98.08) circle (  2.13);

\path[fill=fillColor,fill opacity=0.20] (226.37, 81.82) circle (  2.13);

\path[fill=fillColor,fill opacity=0.20] (216.54, 71.26) circle (  2.13);

\path[fill=fillColor,fill opacity=0.20] (208.02, 71.26) circle (  2.13);

\path[fill=fillColor,fill opacity=0.20] (194.26, 79.39) circle (  2.13);

\path[fill=fillColor,fill opacity=0.20] (183.11, 44.44) circle (  2.13);

\path[fill=fillColor,fill opacity=0.20] (198.63, 47.69) circle (  2.13);

\path[fill=fillColor,fill opacity=0.20] (202.56, 75.32) circle (  2.13);

\path[fill=fillColor,fill opacity=0.20] (208.46, 82.64) circle (  2.13);

\path[fill=fillColor,fill opacity=0.20] (210.42, 87.51) circle (  2.13);

\path[fill=fillColor,fill opacity=0.20] (213.05, 99.70) circle (  2.13);

\path[fill=fillColor,fill opacity=0.20] (215.45, 97.27) circle (  2.13);

\path[fill=fillColor,fill opacity=0.20] (212.17, 87.51) circle (  2.13);

\path[fill=fillColor,fill opacity=0.20] (208.89, 78.57) circle (  2.13);

\path[fill=fillColor,fill opacity=0.20] (203.43, 78.57) circle (  2.13);

\path[fill=fillColor,fill opacity=0.20] (189.67, 78.57) circle (  2.13);

\path[fill=fillColor,fill opacity=0.20] (202.34, 61.51) circle (  2.13);

\path[fill=fillColor,fill opacity=0.20] (219.38, 78.57) circle (  2.13);

\path[fill=fillColor,fill opacity=0.20] (220.91, 90.76) circle (  2.13);

\path[fill=fillColor,fill opacity=0.20] (234.68, 93.20) circle (  2.13);

\path[fill=fillColor,fill opacity=0.20] (237.30, 94.02) circle (  2.13);

\path[fill=fillColor,fill opacity=0.20] (231.18, 92.39) circle (  2.13);

\path[fill=fillColor,fill opacity=0.20] (227.90,102.96) circle (  2.13);

\path[fill=fillColor,fill opacity=0.20] (225.28,103.77) circle (  2.13);

\path[fill=fillColor,fill opacity=0.20] (222.44, 85.08) circle (  2.13);

\path[fill=fillColor,fill opacity=0.20] (216.32, 68.82) circle (  2.13);

\path[fill=fillColor,fill opacity=0.20] (204.52, 66.38) circle (  2.13);

\path[fill=fillColor,fill opacity=0.20] (191.63, 74.51) circle (  2.13);

\path[fill=fillColor,fill opacity=0.20] (177.43, 49.32) circle (  2.13);

\path[fill=fillColor,fill opacity=0.20] (194.04, 51.75) circle (  2.13);

\path[fill=fillColor,fill opacity=0.20] (204.31, 52.57) circle (  2.13);

\path[fill=fillColor,fill opacity=0.20] (208.46, 73.70) circle (  2.13);

\path[fill=fillColor,fill opacity=0.20] (213.48, 81.82) circle (  2.13);

\path[fill=fillColor,fill opacity=0.20] (223.53, 86.70) circle (  2.13);

\path[fill=fillColor,fill opacity=0.20] (220.04,101.33) circle (  2.13);

\path[fill=fillColor,fill opacity=0.20] (217.42, 92.39) circle (  2.13);

\path[fill=fillColor,fill opacity=0.20] (214.14, 76.14) circle (  2.13);

\path[fill=fillColor,fill opacity=0.20] (209.55, 76.14) circle (  2.13);

\path[fill=fillColor,fill opacity=0.20] (196.22, 77.76) circle (  2.13);

\path[fill=fillColor,fill opacity=0.20] (174.81, 87.51) circle (  2.13);

\path[fill=fillColor,fill opacity=0.20] (187.92, 58.26) circle (  2.13);

\path[fill=fillColor,fill opacity=0.20] (213.05, 68.82) circle (  2.13);

\path[fill=fillColor,fill opacity=0.20] (217.85, 85.08) circle (  2.13);

\path[fill=fillColor,fill opacity=0.20] (216.98, 93.20) circle (  2.13);

\path[fill=fillColor,fill opacity=0.20] (222.88, 98.89) circle (  2.13);

\path[fill=fillColor,fill opacity=0.20] (227.25, 93.20) circle (  2.13);

\path[fill=fillColor,fill opacity=0.20] (231.18, 89.95) circle (  2.13);

\path[fill=fillColor,fill opacity=0.20] (232.27, 95.64) circle (  2.13);

\path[fill=fillColor,fill opacity=0.20] (224.63, 91.58) circle (  2.13);

\path[fill=fillColor,fill opacity=0.20] (220.04, 76.95) circle (  2.13);

\path[fill=fillColor,fill opacity=0.20] (216.76, 68.01) circle (  2.13);

\path[fill=fillColor,fill opacity=0.20] (204.74, 64.76) circle (  2.13);

\path[fill=fillColor,fill opacity=0.20] (177.43, 47.69) circle (  2.13);

\path[fill=fillColor,fill opacity=0.20] (196.66, 53.38) circle (  2.13);

\path[fill=fillColor,fill opacity=0.20] (206.49, 60.69) circle (  2.13);

\path[fill=fillColor,fill opacity=0.20] (209.77, 80.20) circle (  2.13);

\path[fill=fillColor,fill opacity=0.20] (215.23, 94.83) circle (  2.13);

\path[fill=fillColor,fill opacity=0.20] (215.45,102.14) circle (  2.13);

\path[fill=fillColor,fill opacity=0.20] (215.23,105.39) circle (  2.13);

\path[fill=fillColor,fill opacity=0.20] (215.01, 98.08) circle (  2.13);

\path[fill=fillColor,fill opacity=0.20] (211.95, 76.95) circle (  2.13);

\path[fill=fillColor,fill opacity=0.20] (202.78, 72.07) circle (  2.13);

\path[fill=fillColor,fill opacity=0.20] (184.20, 80.20) circle (  2.13);

\path[fill=fillColor,fill opacity=0.20] (178.96, 66.38) circle (  2.13);

\path[fill=fillColor,fill opacity=0.20] (203.00, 67.20) circle (  2.13);

\path[fill=fillColor,fill opacity=0.20] (213.70, 81.82) circle (  2.13);

\path[fill=fillColor,fill opacity=0.20] (216.32, 85.89) circle (  2.13);

\path[fill=fillColor,fill opacity=0.20] (213.92, 93.20) circle (  2.13);

\path[fill=fillColor,fill opacity=0.20] (218.95, 97.27) circle (  2.13);

\path[fill=fillColor,fill opacity=0.20] (223.53, 91.58) circle (  2.13);

\path[fill=fillColor,fill opacity=0.20] (228.78, 87.51) circle (  2.13);

\path[fill=fillColor,fill opacity=0.20] (227.47, 87.51) circle (  2.13);

\path[fill=fillColor,fill opacity=0.20] (219.60, 84.26) circle (  2.13);

\path[fill=fillColor,fill opacity=0.20] (219.38, 81.82) circle (  2.13);

\path[fill=fillColor,fill opacity=0.20] (211.74, 74.51) circle (  2.13);

\path[fill=fillColor,fill opacity=0.20] (198.84, 63.95) circle (  2.13);

\path[fill=fillColor,fill opacity=0.20] (179.62, 44.44) circle (  2.13);

\path[fill=fillColor,fill opacity=0.20] (194.26, 47.69) circle (  2.13);

\path[fill=fillColor,fill opacity=0.20] (204.52, 64.76) circle (  2.13);

\path[fill=fillColor,fill opacity=0.20] (212.61, 85.08) circle (  2.13);

\path[fill=fillColor,fill opacity=0.20] (217.85, 94.02) circle (  2.13);

\path[fill=fillColor,fill opacity=0.20] (218.73,100.52) circle (  2.13);

\path[fill=fillColor,fill opacity=0.20] (215.23,101.33) circle (  2.13);

\path[fill=fillColor,fill opacity=0.20] (211.95, 95.64) circle (  2.13);

\path[fill=fillColor,fill opacity=0.20] (207.15, 86.70) circle (  2.13);

\path[fill=fillColor,fill opacity=0.20] (194.91, 74.51) circle (  2.13);

\path[fill=fillColor,fill opacity=0.20] (191.41, 69.63) circle (  2.13);

\path[fill=fillColor,fill opacity=0.20] (202.12, 78.57) circle (  2.13);

\path[fill=fillColor,fill opacity=0.20] (209.11, 73.70) circle (  2.13);

\path[fill=fillColor,fill opacity=0.20] (213.92, 76.14) circle (  2.13);

\path[fill=fillColor,fill opacity=0.20] (218.07, 89.14) circle (  2.13);

\path[fill=fillColor,fill opacity=0.20] (226.16, 89.95) circle (  2.13);

\path[fill=fillColor,fill opacity=0.20] (220.26, 88.33) circle (  2.13);

\path[fill=fillColor,fill opacity=0.20] (224.41, 91.58) circle (  2.13);

\path[fill=fillColor,fill opacity=0.20] (223.10, 89.14) circle (  2.13);

\path[fill=fillColor,fill opacity=0.20] (217.85, 89.14) circle (  2.13);

\path[fill=fillColor,fill opacity=0.20] (213.26, 90.76) circle (  2.13);

\path[fill=fillColor,fill opacity=0.20] (207.37, 74.51) circle (  2.13);

\path[fill=fillColor,fill opacity=0.20] (193.60, 65.57) circle (  2.13);

\path[fill=fillColor,fill opacity=0.20] (194.26, 42.82) circle (  2.13);

\path[fill=fillColor,fill opacity=0.20] (210.86, 56.63) circle (  2.13);

\path[fill=fillColor,fill opacity=0.20] (215.67, 81.01) circle (  2.13);

\path[fill=fillColor,fill opacity=0.20] (219.16, 94.02) circle (  2.13);

\path[fill=fillColor,fill opacity=0.20] (217.85, 98.08) circle (  2.13);

\path[fill=fillColor,fill opacity=0.20] (212.39, 93.20) circle (  2.13);

\path[fill=fillColor,fill opacity=0.20] (208.46, 85.89) circle (  2.13);

\path[fill=fillColor,fill opacity=0.20] (204.52, 87.51) circle (  2.13);

\path[fill=fillColor,fill opacity=0.20] (191.63, 81.82) circle (  2.13);

\path[fill=fillColor,fill opacity=0.20] (194.47, 65.57) circle (  2.13);

\path[fill=fillColor,fill opacity=0.20] (196.00, 75.32) circle (  2.13);

\path[fill=fillColor,fill opacity=0.20] (203.65, 76.95) circle (  2.13);

\path[fill=fillColor,fill opacity=0.20] (208.46, 70.45) circle (  2.13);

\path[fill=fillColor,fill opacity=0.20] (215.67, 70.45) circle (  2.13);

\path[fill=fillColor,fill opacity=0.20] (223.10, 86.70) circle (  2.13);

\path[fill=fillColor,fill opacity=0.20] (226.37, 85.89) circle (  2.13);

\path[fill=fillColor,fill opacity=0.20] (223.10, 84.26) circle (  2.13);

\path[fill=fillColor,fill opacity=0.20] (220.48, 94.83) circle (  2.13);

\path[fill=fillColor,fill opacity=0.20] (218.73, 95.64) circle (  2.13);

\path[fill=fillColor,fill opacity=0.20] (216.54, 87.51) circle (  2.13);

\path[fill=fillColor,fill opacity=0.20] (209.11, 78.57) circle (  2.13);

\path[fill=fillColor,fill opacity=0.20] (198.84, 63.95) circle (  2.13);

\path[fill=fillColor,fill opacity=0.20] (184.86, 68.82) circle (  2.13);

\path[fill=fillColor,fill opacity=0.20] (191.20, 41.19) circle (  2.13);

\path[fill=fillColor,fill opacity=0.20] (208.46, 46.88) circle (  2.13);

\path[fill=fillColor,fill opacity=0.20] (214.58, 72.07) circle (  2.13);

\path[fill=fillColor,fill opacity=0.20] (214.58, 88.33) circle (  2.13);

\path[fill=fillColor,fill opacity=0.20] (209.55, 91.58) circle (  2.13);

\path[fill=fillColor,fill opacity=0.20] (204.31, 90.76) circle (  2.13);

\path[fill=fillColor,fill opacity=0.20] (205.62, 85.89) circle (  2.13);

\path[fill=fillColor,fill opacity=0.20] (203.87, 85.08) circle (  2.13);

\path[fill=fillColor,fill opacity=0.20] (194.26, 85.89) circle (  2.13);

\path[fill=fillColor,fill opacity=0.20] (187.92, 64.76) circle (  2.13);

\path[fill=fillColor,fill opacity=0.20] (199.06, 70.45) circle (  2.13);

\path[fill=fillColor,fill opacity=0.20] (209.11, 78.57) circle (  2.13);

\path[fill=fillColor,fill opacity=0.20] (211.74, 89.95) circle (  2.13);

\path[fill=fillColor,fill opacity=0.20] (213.26, 84.26) circle (  2.13);

\path[fill=fillColor,fill opacity=0.20] (223.53, 85.08) circle (  2.13);

\path[fill=fillColor,fill opacity=0.20] (228.12, 87.51) circle (  2.13);

\path[fill=fillColor,fill opacity=0.20] (225.28, 89.14) circle (  2.13);

\path[fill=fillColor,fill opacity=0.20] (222.22, 94.02) circle (  2.13);

\path[fill=fillColor,fill opacity=0.20] (220.69, 94.83) circle (  2.13);

\path[fill=fillColor,fill opacity=0.20] (222.22, 79.39) circle (  2.13);

\path[fill=fillColor,fill opacity=0.20] (206.49, 62.32) circle (  2.13);

\path[fill=fillColor,fill opacity=0.20] (190.32, 55.82) circle (  2.13);

\path[fill=fillColor,fill opacity=0.20] (201.03, 40.38) circle (  2.13);

\path[fill=fillColor,fill opacity=0.20] (210.86, 61.51) circle (  2.13);

\path[fill=fillColor,fill opacity=0.20] (210.42, 75.32) circle (  2.13);

\path[fill=fillColor,fill opacity=0.20] (209.11, 78.57) circle (  2.13);

\path[fill=fillColor,fill opacity=0.20] (209.11, 89.95) circle (  2.13);

\path[fill=fillColor,fill opacity=0.20] (203.43, 93.20) circle (  2.13);

\path[fill=fillColor,fill opacity=0.20] (201.90, 81.01) circle (  2.13);

\path[fill=fillColor,fill opacity=0.20] (199.06, 76.14) circle (  2.13);

\path[fill=fillColor,fill opacity=0.20] (183.55, 84.26) circle (  2.13);

\path[fill=fillColor,fill opacity=0.20] (185.08, 62.32) circle (  2.13);

\path[fill=fillColor,fill opacity=0.20] (197.31, 57.44) circle (  2.13);

\path[fill=fillColor,fill opacity=0.20] (210.64, 66.38) circle (  2.13);

\path[fill=fillColor,fill opacity=0.20] (214.58, 82.64) circle (  2.13);

\path[fill=fillColor,fill opacity=0.20] (213.92, 96.45) circle (  2.13);

\path[fill=fillColor,fill opacity=0.20] (213.70, 94.83) circle (  2.13);

\path[fill=fillColor,fill opacity=0.20] (223.97, 86.70) circle (  2.13);

\path[fill=fillColor,fill opacity=0.20] (231.84, 89.95) circle (  2.13);

\path[fill=fillColor,fill opacity=0.20] (231.40, 96.45) circle (  2.13);

\path[fill=fillColor,fill opacity=0.20] (231.84, 94.02) circle (  2.13);

\path[fill=fillColor,fill opacity=0.20] (229.22, 85.08) circle (  2.13);

\path[fill=fillColor,fill opacity=0.20] (220.69, 74.51) circle (  2.13);

\path[fill=fillColor,fill opacity=0.20] (201.68, 61.51) circle (  2.13);

\path[fill=fillColor,fill opacity=0.20] (191.63, 41.19) circle (  2.13);

\path[fill=fillColor,fill opacity=0.20] (204.09, 49.32) circle (  2.13);

\path[fill=fillColor,fill opacity=0.20] (208.89, 57.44) circle (  2.13);

\path[fill=fillColor,fill opacity=0.20] (209.99, 71.26) circle (  2.13);

\path[fill=fillColor,fill opacity=0.20] (207.58, 89.95) circle (  2.13);

\path[fill=fillColor,fill opacity=0.20] (201.90, 92.39) circle (  2.13);

\path[fill=fillColor,fill opacity=0.20] (201.90, 78.57) circle (  2.13);

\path[fill=fillColor,fill opacity=0.20] (204.96, 70.45) circle (  2.13);

\path[fill=fillColor,fill opacity=0.20] (198.41, 72.07) circle (  2.13);

\path[fill=fillColor,fill opacity=0.20] (177.87, 82.64) circle (  2.13);

\path[fill=fillColor,fill opacity=0.20] (185.52, 67.20) circle (  2.13);

\path[fill=fillColor,fill opacity=0.20] (197.10, 61.51) circle (  2.13);

\path[fill=fillColor,fill opacity=0.20] (203.21, 62.32) circle (  2.13);

\path[fill=fillColor,fill opacity=0.20] (209.99, 73.70) circle (  2.13);

\path[fill=fillColor,fill opacity=0.20] (215.45, 80.20) circle (  2.13);

\path[fill=fillColor,fill opacity=0.20] (217.42, 82.64) circle (  2.13);

\path[fill=fillColor,fill opacity=0.20] (218.51, 90.76) circle (  2.13);

\path[fill=fillColor,fill opacity=0.20] (224.63, 92.39) circle (  2.13);

\path[fill=fillColor,fill opacity=0.20] (230.53, 89.95) circle (  2.13);

\path[fill=fillColor,fill opacity=0.20] (227.90, 93.20) circle (  2.13);

\path[fill=fillColor,fill opacity=0.20] (232.49, 85.89) circle (  2.13);

\path[fill=fillColor,fill opacity=0.20] (222.88, 72.89) circle (  2.13);

\path[fill=fillColor,fill opacity=0.20] (203.87, 68.82) circle (  2.13);

\path[fill=fillColor,fill opacity=0.20] (198.63, 46.07) circle (  2.13);

\path[fill=fillColor,fill opacity=0.20] (207.58, 55.01) circle (  2.13);

\path[fill=fillColor,fill opacity=0.20] (208.68, 68.82) circle (  2.13);

\path[fill=fillColor,fill opacity=0.20] (207.37, 80.20) circle (  2.13);

\path[fill=fillColor,fill opacity=0.20] (207.37, 84.26) circle (  2.13);

\path[fill=fillColor,fill opacity=0.20] (203.65, 84.26) circle (  2.13);

\path[fill=fillColor,fill opacity=0.20] (205.40, 80.20) circle (  2.13);

\path[fill=fillColor,fill opacity=0.20] (202.12, 76.95) circle (  2.13);

\path[fill=fillColor,fill opacity=0.20] (190.10, 73.70) circle (  2.13);

\path[fill=fillColor,fill opacity=0.20] (176.56, 78.57) circle (  2.13);

\path[fill=fillColor,fill opacity=0.20] (186.61, 79.39) circle (  2.13);

\path[fill=fillColor,fill opacity=0.20] (193.16, 61.51) circle (  2.13);

\path[fill=fillColor,fill opacity=0.20] (201.90, 68.82) circle (  2.13);

\path[fill=fillColor,fill opacity=0.20] (202.56, 85.08) circle (  2.13);

\path[fill=fillColor,fill opacity=0.20] (208.68, 86.70) circle (  2.13);

\path[fill=fillColor,fill opacity=0.20] (215.89, 76.95) circle (  2.13);

\path[fill=fillColor,fill opacity=0.20] (225.72, 73.70) circle (  2.13);

\path[fill=fillColor,fill opacity=0.20] (223.75, 85.89) circle (  2.13);

\path[fill=fillColor,fill opacity=0.20] (227.90, 91.58) circle (  2.13);

\path[fill=fillColor,fill opacity=0.20] (222.66, 83.45) circle (  2.13);

\path[fill=fillColor,fill opacity=0.20] (219.16, 74.51) circle (  2.13);

\path[fill=fillColor,fill opacity=0.20] (224.41, 70.45) circle (  2.13);

\path[fill=fillColor,fill opacity=0.20] (210.64, 70.45) circle (  2.13);

\path[fill=fillColor,fill opacity=0.20] (195.35, 62.32) circle (  2.13);

\path[fill=fillColor,fill opacity=0.20] (203.21, 52.57) circle (  2.13);

\path[fill=fillColor,fill opacity=0.20] (208.02, 60.69) circle (  2.13);

\path[fill=fillColor,fill opacity=0.20] (208.68, 62.32) circle (  2.13);

\path[fill=fillColor,fill opacity=0.20] (209.11, 72.89) circle (  2.13);

\path[fill=fillColor,fill opacity=0.20] (207.80, 86.70) circle (  2.13);

\path[fill=fillColor,fill opacity=0.20] (206.27, 90.76) circle (  2.13);

\path[fill=fillColor,fill opacity=0.20] (202.34, 86.70) circle (  2.13);

\path[fill=fillColor,fill opacity=0.20] (195.78, 76.95) circle (  2.13);

\path[fill=fillColor,fill opacity=0.20] (191.85, 68.01) circle (  2.13);

\path[fill=fillColor,fill opacity=0.20] (180.49, 72.89) circle (  2.13);

\path[fill=fillColor,fill opacity=0.20] (182.02, 79.39) circle (  2.13);

\path[fill=fillColor,fill opacity=0.20] (194.91, 78.57) circle (  2.13);

\path[fill=fillColor,fill opacity=0.20] (203.00, 75.32) circle (  2.13);

\path[fill=fillColor,fill opacity=0.20] (201.47, 78.57) circle (  2.13);

\path[fill=fillColor,fill opacity=0.20] (200.15, 90.76) circle (  2.13);

\path[fill=fillColor,fill opacity=0.20] (210.21, 90.76) circle (  2.13);

\path[fill=fillColor,fill opacity=0.20] (218.29, 81.82) circle (  2.13);

\path[fill=fillColor,fill opacity=0.20] (225.06, 82.64) circle (  2.13);

\path[fill=fillColor,fill opacity=0.20] (225.28, 89.95) circle (  2.13);

\path[fill=fillColor,fill opacity=0.20] (225.06, 89.95) circle (  2.13);

\path[fill=fillColor,fill opacity=0.20] (270.51, 83.45) circle (  2.13);

\path[fill=fillColor,fill opacity=0.20] (212.61, 68.01) circle (  2.13);

\path[fill=fillColor,fill opacity=0.20] (209.33, 59.88) circle (  2.13);

\path[fill=fillColor,fill opacity=0.20] (200.37, 63.13) circle (  2.13);

\path[fill=fillColor,fill opacity=0.20] (190.32, 44.44) circle (  2.13);

\path[fill=fillColor,fill opacity=0.20] (201.03, 46.88) circle (  2.13);

\path[fill=fillColor,fill opacity=0.20] (209.99, 47.69) circle (  2.13);

\path[fill=fillColor,fill opacity=0.20] (208.02, 58.26) circle (  2.13);

\path[fill=fillColor,fill opacity=0.20] (206.27, 79.39) circle (  2.13);

\path[fill=fillColor,fill opacity=0.20] (206.27, 88.33) circle (  2.13);

\path[fill=fillColor,fill opacity=0.20] (201.47, 82.64) circle (  2.13);

\path[fill=fillColor,fill opacity=0.20] (204.52, 78.57) circle (  2.13);

\path[fill=fillColor,fill opacity=0.20] (199.28, 77.76) circle (  2.13);

\path[fill=fillColor,fill opacity=0.20] (193.38, 76.95) circle (  2.13);

\path[fill=fillColor,fill opacity=0.20] (181.36, 72.89) circle (  2.13);

\path[fill=fillColor,fill opacity=0.20] (181.36, 68.01) circle (  2.13);

\path[fill=fillColor,fill opacity=0.20] (195.57, 57.44) circle (  2.13);

\path[fill=fillColor,fill opacity=0.20] (204.31, 76.95) circle (  2.13);

\path[fill=fillColor,fill opacity=0.20] (208.68, 86.70) circle (  2.13);

\path[fill=fillColor,fill opacity=0.20] (214.14, 79.39) circle (  2.13);

\path[fill=fillColor,fill opacity=0.20] (208.46, 80.20) circle (  2.13);

\path[fill=fillColor,fill opacity=0.20] (213.70, 89.14) circle (  2.13);

\path[fill=fillColor,fill opacity=0.20] (216.76, 88.33) circle (  2.13);

\path[fill=fillColor,fill opacity=0.20] (218.29, 85.08) circle (  2.13);

\path[fill=fillColor,fill opacity=0.20] (219.82, 85.08) circle (  2.13);

\path[fill=fillColor,fill opacity=0.20] (214.79, 84.26) circle (  2.13);

\path[fill=fillColor,fill opacity=0.20] (211.30, 81.01) circle (  2.13);

\path[fill=fillColor,fill opacity=0.20] (208.89, 72.89) circle (  2.13);

\path[fill=fillColor,fill opacity=0.20] (198.63, 55.01) circle (  2.13);

\path[fill=fillColor,fill opacity=0.20] (194.69, 47.69) circle (  2.13);

\path[fill=fillColor,fill opacity=0.20] (208.02, 40.38) circle (  2.13);

\path[fill=fillColor,fill opacity=0.20] (209.55, 45.25) circle (  2.13);

\path[fill=fillColor,fill opacity=0.20] (206.71, 64.76) circle (  2.13);

\path[fill=fillColor,fill opacity=0.20] (200.81, 76.14) circle (  2.13);

\path[fill=fillColor,fill opacity=0.20] (202.78, 76.14) circle (  2.13);

\path[fill=fillColor,fill opacity=0.20] (205.18, 78.57) circle (  2.13);

\path[fill=fillColor,fill opacity=0.20] (201.25, 85.89) circle (  2.13);

\path[fill=fillColor,fill opacity=0.20] (193.60, 87.51) circle (  2.13);

\path[fill=fillColor,fill opacity=0.20] (191.20, 78.57) circle (  2.13);

\path[fill=fillColor,fill opacity=0.20] (188.57, 72.07) circle (  2.13);

\path[fill=fillColor,fill opacity=0.20] (177.65, 87.51) circle (  2.13);

\path[fill=fillColor,fill opacity=0.20] (180.27, 68.82) circle (  2.13);

\path[fill=fillColor,fill opacity=0.20] (191.41, 54.19) circle (  2.13);

\path[fill=fillColor,fill opacity=0.20] (201.90, 50.94) circle (  2.13);

\path[fill=fillColor,fill opacity=0.20] (204.52, 73.70) circle (  2.13);

\path[fill=fillColor,fill opacity=0.20] (205.84, 87.51) circle (  2.13);

\path[fill=fillColor,fill opacity=0.20] (217.63, 82.64) circle (  2.13);

\path[fill=fillColor,fill opacity=0.20] (215.45, 78.57) circle (  2.13);

\path[fill=fillColor,fill opacity=0.20] (216.32, 85.89) circle (  2.13);

\path[fill=fillColor,fill opacity=0.20] (212.17, 85.08) circle (  2.13);

\path[fill=fillColor,fill opacity=0.20] (211.52, 76.95) circle (  2.13);

\path[fill=fillColor,fill opacity=0.20] (220.91, 71.26) circle (  2.13);

\path[fill=fillColor,fill opacity=0.20] (210.86, 63.13) circle (  2.13);

\path[fill=fillColor,fill opacity=0.20] (202.56, 63.13) circle (  2.13);

\path[fill=fillColor,fill opacity=0.20] (200.37, 65.57) circle (  2.13);

\path[fill=fillColor,fill opacity=0.20] (186.17, 58.26) circle (  2.13);

\path[fill=fillColor,fill opacity=0.20] (185.73, 46.88) circle (  2.13);

\path[fill=fillColor,fill opacity=0.20] (198.63, 53.38) circle (  2.13);

\path[fill=fillColor,fill opacity=0.20] (204.09, 44.44) circle (  2.13);

\path[fill=fillColor,fill opacity=0.20] (204.09, 45.25) circle (  2.13);

\path[fill=fillColor,fill opacity=0.20] (205.84, 55.82) circle (  2.13);

\path[fill=fillColor,fill opacity=0.20] (204.96, 65.57) circle (  2.13);

\path[fill=fillColor,fill opacity=0.20] (202.56, 72.07) circle (  2.13);

\path[fill=fillColor,fill opacity=0.20] (196.88, 81.82) circle (  2.13);

\path[fill=fillColor,fill opacity=0.20] (198.41, 91.58) circle (  2.13);

\path[fill=fillColor,fill opacity=0.20] (195.35, 95.64) circle (  2.13);

\path[fill=fillColor,fill opacity=0.20] (197.53, 83.45) circle (  2.13);

\path[fill=fillColor,fill opacity=0.20] (191.85, 66.38) circle (  2.13);

\path[fill=fillColor,fill opacity=0.20] (180.93, 71.26) circle (  2.13);

\path[fill=fillColor,fill opacity=0.20] (183.55, 59.88) circle (  2.13);

\path[fill=fillColor,fill opacity=0.20] (197.75, 59.07) circle (  2.13);

\path[fill=fillColor,fill opacity=0.20] (199.28, 59.07) circle (  2.13);

\path[fill=fillColor,fill opacity=0.20] (195.78, 61.51) circle (  2.13);

\path[fill=fillColor,fill opacity=0.20] (203.87, 72.89) circle (  2.13);

\path[fill=fillColor,fill opacity=0.20] (209.33, 81.01) circle (  2.13);

\path[fill=fillColor,fill opacity=0.20] (212.83, 78.57) circle (  2.13);

\path[fill=fillColor,fill opacity=0.20] (216.76, 74.51) circle (  2.13);

\path[fill=fillColor,fill opacity=0.20] (224.19, 72.89) circle (  2.13);

\path[fill=fillColor,fill opacity=0.20] (207.58, 70.45) circle (  2.13);

\path[fill=fillColor,fill opacity=0.20] (207.80, 72.89) circle (  2.13);

\path[fill=fillColor,fill opacity=0.20] (208.02, 70.45) circle (  2.13);

\path[fill=fillColor,fill opacity=0.20] (201.03, 55.82) circle (  2.13);

\path[fill=fillColor,fill opacity=0.20] (192.29, 51.75) circle (  2.13);

\path[fill=fillColor,fill opacity=0.20] (191.20, 56.63) circle (  2.13);

\path[fill=fillColor,fill opacity=0.20] (199.94, 53.38) circle (  2.13);

\path[fill=fillColor,fill opacity=0.20] (208.89, 55.01) circle (  2.13);

\path[fill=fillColor,fill opacity=0.20] (205.18, 62.32) circle (  2.13);

\path[fill=fillColor,fill opacity=0.20] (211.52, 72.07) circle (  2.13);

\path[fill=fillColor,fill opacity=0.20] (203.00, 88.33) circle (  2.13);

\path[fill=fillColor,fill opacity=0.20] (203.87,107.83) circle (  2.13);

\path[fill=fillColor,fill opacity=0.20] (204.52,102.96) circle (  2.13);

\path[fill=fillColor,fill opacity=0.20] (201.90, 72.89) circle (  2.13);

\path[fill=fillColor,fill opacity=0.20] (202.34, 59.07) circle (  2.13);

\path[fill=fillColor,fill opacity=0.20] (187.70, 80.20) circle (  2.13);

\path[fill=fillColor,fill opacity=0.20] (182.46, 68.82) circle (  2.13);

\path[fill=fillColor,fill opacity=0.20] (192.94, 68.01) circle (  2.13);

\path[fill=fillColor,fill opacity=0.20] (201.25, 78.57) circle (  2.13);

\path[fill=fillColor,fill opacity=0.20] (203.21, 77.76) circle (  2.13);

\path[fill=fillColor,fill opacity=0.20] (203.21, 73.70) circle (  2.13);

\path[fill=fillColor,fill opacity=0.20] (205.40, 69.63) circle (  2.13);

\path[fill=fillColor,fill opacity=0.20] (204.74, 68.01) circle (  2.13);

\path[fill=fillColor,fill opacity=0.20] (205.40, 63.95) circle (  2.13);

\path[fill=fillColor,fill opacity=0.20] (204.96, 56.63) circle (  2.13);

\path[fill=fillColor,fill opacity=0.20] (198.41, 56.63) circle (  2.13);

\path[fill=fillColor,fill opacity=0.20] (197.53, 63.13) circle (  2.13);

\path[fill=fillColor,fill opacity=0.20] (197.31, 58.26) circle (  2.13);

\path[fill=fillColor,fill opacity=0.20] (211.74, 48.50) circle (  2.13);

\path[fill=fillColor,fill opacity=0.20] (208.24, 53.38) circle (  2.13);

\path[fill=fillColor,fill opacity=0.20] (207.37, 76.14) circle (  2.13);

\path[fill=fillColor,fill opacity=0.20] (201.47,106.21) circle (  2.13);

\path[fill=fillColor,fill opacity=0.20] (204.31,107.83) circle (  2.13);

\path[fill=fillColor,fill opacity=0.20] (208.46, 81.82) circle (  2.13);

\path[fill=fillColor,fill opacity=0.20] (209.99, 78.57) circle (  2.13);

\path[fill=fillColor,fill opacity=0.20] (203.00, 94.02) circle (  2.13);

\path[fill=fillColor,fill opacity=0.20] (192.94, 94.02) circle (  2.13);

\path[fill=fillColor,fill opacity=0.20] (182.02, 92.39) circle (  2.13);

\path[fill=fillColor,fill opacity=0.20] (185.08, 75.32) circle (  2.13);

\path[fill=fillColor,fill opacity=0.20] (192.51, 69.63) circle (  2.13);

\path[fill=fillColor,fill opacity=0.20] (199.94, 70.45) circle (  2.13);

\path[fill=fillColor,fill opacity=0.20] (208.46, 72.07) circle (  2.13);

\path[fill=fillColor,fill opacity=0.20] (209.99, 70.45) circle (  2.13);

\path[fill=fillColor,fill opacity=0.20] (207.80, 71.26) circle (  2.13);

\path[fill=fillColor,fill opacity=0.20] (201.47, 72.07) circle (  2.13);

\path[fill=fillColor,fill opacity=0.20] (197.75, 72.89) circle (  2.13);

\path[fill=fillColor,fill opacity=0.20] (194.26, 71.26) circle (  2.13);

\path[fill=fillColor,fill opacity=0.20] (190.10, 68.82) circle (  2.13);

\path[fill=fillColor,fill opacity=0.20] (185.95, 66.38) circle (  2.13);

\path[fill=fillColor,fill opacity=0.20] (197.75, 59.07) circle (  2.13);

\path[fill=fillColor,fill opacity=0.20] (207.37, 53.38) circle (  2.13);

\path[fill=fillColor,fill opacity=0.20] (211.95, 64.76) circle (  2.13);

\path[fill=fillColor,fill opacity=0.20] (208.46, 91.58) circle (  2.13);

\path[fill=fillColor,fill opacity=0.20] (206.93, 93.20) circle (  2.13);

\path[fill=fillColor,fill opacity=0.20] (209.33, 76.95) circle (  2.13);

\path[fill=fillColor,fill opacity=0.20] (208.02, 89.14) circle (  2.13);

\path[fill=fillColor,fill opacity=0.20] (205.18,104.58) circle (  2.13);

\path[fill=fillColor,fill opacity=0.20] (195.35, 92.39) circle (  2.13);

\path[fill=fillColor,fill opacity=0.20] (197.31, 75.32) circle (  2.13);

\path[fill=fillColor,fill opacity=0.20] (186.83, 85.89) circle (  2.13);

\path[fill=fillColor,fill opacity=0.20] (194.04, 68.82) circle (  2.13);

\path[fill=fillColor,fill opacity=0.20] (200.59, 62.32) circle (  2.13);

\path[fill=fillColor,fill opacity=0.20] (201.03, 59.88) circle (  2.13);

\path[fill=fillColor,fill opacity=0.20] (208.24, 56.63) circle (  2.13);

\path[fill=fillColor,fill opacity=0.20] (204.09, 64.76) circle (  2.13);

\path[fill=fillColor,fill opacity=0.20] (192.73, 72.07) circle (  2.13);

\path[fill=fillColor,fill opacity=0.20] (188.79, 77.76) circle (  2.13);

\path[fill=fillColor,fill opacity=0.20] (185.08, 69.63) circle (  2.13);

\path[fill=fillColor,fill opacity=0.20] (199.72, 68.01) circle (  2.13);

\path[fill=fillColor,fill opacity=0.20] (209.11, 72.89) circle (  2.13);

\path[fill=fillColor,fill opacity=0.20] (208.46, 69.63) circle (  2.13);

\path[fill=fillColor,fill opacity=0.20] (207.58, 68.01) circle (  2.13);

\path[fill=fillColor,fill opacity=0.20] (207.80, 82.64) circle (  2.13);

\path[fill=fillColor,fill opacity=0.20] (208.68, 94.02) circle (  2.13);

\path[fill=fillColor,fill opacity=0.20] (203.00, 92.39) circle (  2.13);

\path[fill=fillColor,fill opacity=0.20] (201.25, 92.39) circle (  2.13);

\path[fill=fillColor,fill opacity=0.20] (202.78, 85.89) circle (  2.13);

\path[fill=fillColor,fill opacity=0.20] (196.88, 63.95) circle (  2.13);

\path[fill=fillColor,fill opacity=0.20] (192.94, 55.01) circle (  2.13);

\path[fill=fillColor,fill opacity=0.20] (187.04, 73.70) circle (  2.13);

\path[fill=fillColor,fill opacity=0.20] (181.36, 92.39) circle (  2.13);

\path[fill=fillColor,fill opacity=0.20] (183.11, 85.89) circle (  2.13);

\path[fill=fillColor,fill opacity=0.20] (191.20, 61.51) circle (  2.13);

\path[fill=fillColor,fill opacity=0.20] (194.26, 46.88) circle (  2.13);

\path[fill=fillColor,fill opacity=0.20] (200.37, 58.26) circle (  2.13);

\path[fill=fillColor,fill opacity=0.20] (196.44, 63.95) circle (  2.13);

\path[fill=fillColor,fill opacity=0.20] (189.45, 69.63) circle (  2.13);

\path[fill=fillColor,fill opacity=0.20] (183.33, 81.01) circle (  2.13);

\path[fill=fillColor,fill opacity=0.20] (175.90, 94.02) circle (  2.13);

\path[fill=fillColor,fill opacity=0.20] (179.62, 81.01) circle (  2.13);

\path[fill=fillColor,fill opacity=0.20] (192.07, 71.26) circle (  2.13);

\path[fill=fillColor,fill opacity=0.20] (203.21, 55.82) circle (  2.13);

\path[fill=fillColor,fill opacity=0.20] (206.49, 59.88) circle (  2.13);

\path[fill=fillColor,fill opacity=0.20] (209.77, 67.20) circle (  2.13);

\path[fill=fillColor,fill opacity=0.20] (207.15, 70.45) circle (  2.13);

\path[fill=fillColor,fill opacity=0.20] (206.71, 79.39) circle (  2.13);

\path[fill=fillColor,fill opacity=0.20] (203.00, 96.45) circle (  2.13);

\path[fill=fillColor,fill opacity=0.20] (200.37,101.33) circle (  2.13);

\path[fill=fillColor,fill opacity=0.20] (199.94, 85.08) circle (  2.13);

\path[fill=fillColor,fill opacity=0.20] (198.19, 74.51) circle (  2.13);

\path[fill=fillColor,fill opacity=0.20] (192.94, 80.20) circle (  2.13);

\path[fill=fillColor,fill opacity=0.20] (193.38, 79.39) circle (  2.13);

\path[fill=fillColor,fill opacity=0.20] (190.10, 72.07) circle (  2.13);

\path[fill=fillColor,fill opacity=0.20] (186.83, 77.76) circle (  2.13);

\path[fill=fillColor,fill opacity=0.20] (182.02, 91.58) circle (  2.13);

\path[fill=fillColor,fill opacity=0.20] (172.41,109.46) circle (  2.13);

\path[fill=fillColor,fill opacity=0.20] (186.39, 89.95) circle (  2.13);

\path[fill=fillColor,fill opacity=0.20] (192.07, 81.01) circle (  2.13);

\path[fill=fillColor,fill opacity=0.20] (194.91, 70.45) circle (  2.13);

\path[fill=fillColor,fill opacity=0.20] (194.26, 64.76) circle (  2.13);

\path[fill=fillColor,fill opacity=0.20] (198.63, 61.51) circle (  2.13);

\path[fill=fillColor,fill opacity=0.20] (220.48, 50.94) circle (  2.13);

\path[fill=fillColor,fill opacity=0.20] (182.24, 63.95) circle (  2.13);

\path[fill=fillColor,fill opacity=0.20] (175.68, 96.45) circle (  2.13);

\path[fill=fillColor,fill opacity=0.20] (167.60,107.02) circle (  2.13);

\path[fill=fillColor,fill opacity=0.20] (184.86, 58.26) circle (  2.13);

\path[fill=fillColor,fill opacity=0.20] (203.21, 52.57) circle (  2.13);

\path[fill=fillColor,fill opacity=0.20] (209.99, 49.32) circle (  2.13);

\path[fill=fillColor,fill opacity=0.20] (206.05, 50.13) circle (  2.13);

\path[fill=fillColor,fill opacity=0.20] (205.18, 63.13) circle (  2.13);

\path[fill=fillColor,fill opacity=0.20] (206.27, 81.82) circle (  2.13);

\path[fill=fillColor,fill opacity=0.20] (205.40, 94.02) circle (  2.13);

\path[fill=fillColor,fill opacity=0.20] (199.94, 94.83) circle (  2.13);

\path[fill=fillColor,fill opacity=0.20] (199.94, 98.89) circle (  2.13);

\path[fill=fillColor,fill opacity=0.20] (197.97, 98.89) circle (  2.13);

\path[fill=fillColor,fill opacity=0.20] (197.31, 81.82) circle (  2.13);

\path[fill=fillColor,fill opacity=0.20] (196.00, 69.63) circle (  2.13);

\path[fill=fillColor,fill opacity=0.20] (196.22, 78.57) circle (  2.13);

\path[fill=fillColor,fill opacity=0.20] (193.38, 81.01) circle (  2.13);

\path[fill=fillColor,fill opacity=0.20] (190.10, 68.82) circle (  2.13);

\path[fill=fillColor,fill opacity=0.20] (187.70, 66.38) circle (  2.13);

\path[fill=fillColor,fill opacity=0.20] (183.33, 79.39) circle (  2.13);

\path[fill=fillColor,fill opacity=0.20] (178.74, 95.64) circle (  2.13);

\path[fill=fillColor,fill opacity=0.20] (178.09,102.96) circle (  2.13);

\path[fill=fillColor,fill opacity=0.20] (178.74, 99.70) circle (  2.13);

\path[fill=fillColor,fill opacity=0.20] (181.36, 89.95) circle (  2.13);

\path[fill=fillColor,fill opacity=0.20] (180.93, 94.83) circle (  2.13);

\path[fill=fillColor,fill opacity=0.20] (179.18, 98.89) circle (  2.13);

\path[fill=fillColor,fill opacity=0.20] (188.79, 83.45) circle (  2.13);

\path[fill=fillColor,fill opacity=0.20] (189.45, 76.95) circle (  2.13);

\path[fill=fillColor,fill opacity=0.20] (192.73, 73.70) circle (  2.13);

\path[fill=fillColor,fill opacity=0.20] (197.31, 77.76) circle (  2.13);

\path[fill=fillColor,fill opacity=0.20] (199.94, 76.14) circle (  2.13);

\path[fill=fillColor,fill opacity=0.20] (198.63, 64.76) circle (  2.13);

\path[fill=fillColor,fill opacity=0.20] (199.72, 61.51) circle (  2.13);

\path[fill=fillColor,fill opacity=0.20] (198.84, 54.19) circle (  2.13);

\path[fill=fillColor,fill opacity=0.20] (192.07, 51.75) circle (  2.13);

\path[fill=fillColor,fill opacity=0.20] (182.02, 74.51) circle (  2.13);

\path[fill=fillColor,fill opacity=0.20] (195.57, 46.88) circle (  2.13);

\path[fill=fillColor,fill opacity=0.20] (202.12, 50.13) circle (  2.13);

\path[fill=fillColor,fill opacity=0.20] (202.56, 61.51) circle (  2.13);

\path[fill=fillColor,fill opacity=0.20] (201.25, 67.20) circle (  2.13);

\path[fill=fillColor,fill opacity=0.20] (201.68, 68.82) circle (  2.13);

\path[fill=fillColor,fill opacity=0.20] (204.09, 73.70) circle (  2.13);

\path[fill=fillColor,fill opacity=0.20] (208.02, 81.01) circle (  2.13);

\path[fill=fillColor,fill opacity=0.20] (206.71, 81.01) circle (  2.13);

\path[fill=fillColor,fill opacity=0.20] (200.59, 72.89) circle (  2.13);

\path[fill=fillColor,fill opacity=0.20] (199.50, 74.51) circle (  2.13);

\path[fill=fillColor,fill opacity=0.20] (198.19, 83.45) circle (  2.13);

\path[fill=fillColor,fill opacity=0.20] (195.78, 84.26) circle (  2.13);

\path[fill=fillColor,fill opacity=0.20] (195.35, 75.32) circle (  2.13);

\path[fill=fillColor,fill opacity=0.20] (194.91, 70.45) circle (  2.13);

\path[fill=fillColor,fill opacity=0.20] (194.47, 78.57) circle (  2.13);

\path[fill=fillColor,fill opacity=0.20] (195.35, 85.08) circle (  2.13);

\path[fill=fillColor,fill opacity=0.20] (195.78, 81.82) circle (  2.13);

\path[fill=fillColor,fill opacity=0.20] (194.91, 73.70) circle (  2.13);

\path[fill=fillColor,fill opacity=0.20] (196.22, 72.89) circle (  2.13);

\path[fill=fillColor,fill opacity=0.20] (195.13, 82.64) circle (  2.13);

\path[fill=fillColor,fill opacity=0.20] (197.31, 85.08) circle (  2.13);

\path[fill=fillColor,fill opacity=0.20] (195.13, 81.82) circle (  2.13);

\path[fill=fillColor,fill opacity=0.20] (195.35, 81.82) circle (  2.13);

\path[fill=fillColor,fill opacity=0.20] (195.78, 74.51) circle (  2.13);

\path[fill=fillColor,fill opacity=0.20] (195.78, 70.45) circle (  2.13);

\path[fill=fillColor,fill opacity=0.20] (195.35, 79.39) circle (  2.13);

\path[fill=fillColor,fill opacity=0.20] (200.81, 76.95) circle (  2.13);

\path[fill=fillColor,fill opacity=0.20] (196.88, 65.57) circle (  2.13);

\path[fill=fillColor,fill opacity=0.20] (200.59, 63.95) circle (  2.13);

\path[fill=fillColor,fill opacity=0.20] (200.59, 59.07) circle (  2.13);

\path[fill=fillColor,fill opacity=0.20] (214.14, 53.38) circle (  2.13);

\path[fill=fillColor,fill opacity=0.20] (201.68, 61.51) circle (  2.13);

\path[fill=fillColor,fill opacity=0.20] (197.97, 64.76) circle (  2.13);

\path[fill=fillColor,fill opacity=0.20] (194.04, 63.13) circle (  2.13);

\path[fill=fillColor,fill opacity=0.20] (186.17, 68.01) circle (  2.13);

\path[fill=fillColor,fill opacity=0.20] (175.46, 83.45) circle (  2.13);

\path[fill=fillColor,fill opacity=0.20] (193.60, 55.01) circle (  2.13);

\path[fill=fillColor,fill opacity=0.20] (198.41, 55.82) circle (  2.13);

\path[fill=fillColor,fill opacity=0.20] (206.49, 55.01) circle (  2.13);

\path[fill=fillColor,fill opacity=0.20] (208.24, 61.51) circle (  2.13);

\path[fill=fillColor,fill opacity=0.20] (208.24, 63.95) circle (  2.13);

\path[fill=fillColor,fill opacity=0.20] (205.62, 65.57) circle (  2.13);

\path[fill=fillColor,fill opacity=0.20] (203.00, 73.70) circle (  2.13);

\path[fill=fillColor,fill opacity=0.20] (198.41, 77.76) circle (  2.13);

\path[fill=fillColor,fill opacity=0.20] (201.47, 72.07) circle (  2.13);

\path[fill=fillColor,fill opacity=0.20] (203.43, 72.89) circle (  2.13);

\path[fill=fillColor,fill opacity=0.20] (199.50, 79.39) circle (  2.13);

\path[fill=fillColor,fill opacity=0.20] (201.25, 84.26) circle (  2.13);

\path[fill=fillColor,fill opacity=0.20] (203.21, 88.33) circle (  2.13);

\path[fill=fillColor,fill opacity=0.20] (198.41, 80.20) circle (  2.13);

\path[fill=fillColor,fill opacity=0.20] (196.22, 71.26) circle (  2.13);

\path[fill=fillColor,fill opacity=0.20] (198.19, 81.01) circle (  2.13);

\path[fill=fillColor,fill opacity=0.20] (199.28, 87.51) circle (  2.13);

\path[fill=fillColor,fill opacity=0.20] (201.68, 82.64) circle (  2.13);

\path[fill=fillColor,fill opacity=0.20] (201.47, 83.45) circle (  2.13);

\path[fill=fillColor,fill opacity=0.20] (199.06, 81.82) circle (  2.13);

\path[fill=fillColor,fill opacity=0.20] (200.37, 73.70) circle (  2.13);

\path[fill=fillColor,fill opacity=0.20] (201.68, 76.14) circle (  2.13);

\path[fill=fillColor,fill opacity=0.20] (203.43, 74.51) circle (  2.13);

\path[fill=fillColor,fill opacity=0.20] (204.31, 63.95) circle (  2.13);

\path[fill=fillColor,fill opacity=0.20] (206.71, 63.13) circle (  2.13);

\path[fill=fillColor,fill opacity=0.20] (204.74, 61.51) circle (  2.13);

\path[fill=fillColor,fill opacity=0.20] (195.57, 59.07) circle (  2.13);

\path[fill=fillColor,fill opacity=0.20] (184.20, 76.14) circle (  2.13);

\path[fill=fillColor,fill opacity=0.20] (190.76, 58.26) circle (  2.13);

\path[fill=fillColor,fill opacity=0.20] (195.13, 65.57) circle (  2.13);

\path[fill=fillColor,fill opacity=0.20] (198.41, 63.95) circle (  2.13);

\path[fill=fillColor,fill opacity=0.20] (199.72, 50.94) circle (  2.13);

\path[fill=fillColor,fill opacity=0.20] (197.53, 60.69) circle (  2.13);

\path[fill=fillColor,fill opacity=0.20] (197.75, 71.26) circle (  2.13);

\path[fill=fillColor,fill opacity=0.20] (201.90, 66.38) circle (  2.13);

\path[fill=fillColor,fill opacity=0.20] (200.59, 59.88) circle (  2.13);

\path[fill=fillColor,fill opacity=0.20] (202.34, 61.51) circle (  2.13);

\path[fill=fillColor,fill opacity=0.20] (206.71, 67.20) circle (  2.13);

\path[fill=fillColor,fill opacity=0.20] (207.15, 70.45) circle (  2.13);

\path[fill=fillColor,fill opacity=0.20] (206.05, 72.89) circle (  2.13);

\path[fill=fillColor,fill opacity=0.20] (204.52, 68.01) circle (  2.13);

\path[fill=fillColor,fill opacity=0.20] (203.65, 64.76) circle (  2.13);

\path[fill=fillColor,fill opacity=0.20] (205.40, 69.63) circle (  2.13);

\path[fill=fillColor,fill opacity=0.20] (207.37, 72.89) circle (  2.13);

\path[fill=fillColor,fill opacity=0.20] (206.93, 73.70) circle (  2.13);

\path[fill=fillColor,fill opacity=0.20] (205.18, 77.76) circle (  2.13);

\path[fill=fillColor,fill opacity=0.20] (203.00, 79.39) circle (  2.13);

\path[fill=fillColor,fill opacity=0.20] (205.18, 74.51) circle (  2.13);

\path[fill=fillColor,fill opacity=0.20] (206.49, 69.63) circle (  2.13);

\path[fill=fillColor,fill opacity=0.20] (205.84, 68.82) circle (  2.13);

\path[fill=fillColor,fill opacity=0.20] (201.90, 68.82) circle (  2.13);

\path[fill=fillColor,fill opacity=0.20] (194.04, 72.07) circle (  2.13);

\path[fill=fillColor,fill opacity=0.20] (184.86, 51.75) circle (  2.13);

\path[fill=fillColor,fill opacity=0.20] (186.17, 74.51) circle (  2.13);

\path[fill=fillColor,fill opacity=0.20] (188.79, 75.32) circle (  2.13);

\path[fill=fillColor,fill opacity=0.20] (187.26, 64.76) circle (  2.13);

\path[fill=fillColor,fill opacity=0.20] (187.92, 66.38) circle (  2.13);

\path[fill=fillColor,fill opacity=0.20] (196.44, 66.38) circle (  2.13);

\path[fill=fillColor,fill opacity=0.20] (203.87, 56.63) circle (  2.13);

\path[fill=fillColor,fill opacity=0.20] (205.18, 54.19) circle (  2.13);

\path[fill=fillColor,fill opacity=0.20] (204.96, 62.32) circle (  2.13);

\path[fill=fillColor,fill opacity=0.20] (203.43, 64.76) circle (  2.13);

\path[fill=fillColor,fill opacity=0.20] (202.56, 66.38) circle (  2.13);

\path[fill=fillColor,fill opacity=0.20] (200.59, 68.82) circle (  2.13);

\path[fill=fillColor,fill opacity=0.20] (201.90, 68.82) circle (  2.13);

\path[fill=fillColor,fill opacity=0.20] (201.25, 73.70) circle (  2.13);

\path[fill=fillColor,fill opacity=0.20] (198.41, 77.76) circle (  2.13);

\path[fill=fillColor,fill opacity=0.20] (194.91, 72.89) circle (  2.13);

\path[fill=fillColor,fill opacity=0.20] (189.01, 76.14) circle (  2.13);

\path[fill=fillColor,fill opacity=0.20] (182.89, 76.14) circle (  2.13);

\path[fill=fillColor,fill opacity=0.20] (186.61, 66.38) circle (  2.13);

\path[fill=fillColor,fill opacity=0.20] (191.41, 62.32) circle (  2.13);

\path[fill=fillColor,fill opacity=0.20] (191.41, 70.45) circle (  2.13);

\path[fill=fillColor,fill opacity=0.20] (196.66, 81.82) circle (  2.13);

\path[fill=fillColor,fill opacity=0.20] (187.70, 87.51) circle (  2.13);

\path[fill=fillColor,fill opacity=0.20] (183.77, 82.64) circle (  2.13);

\path[fill=fillColor,fill opacity=0.20] (190.98, 85.89) circle (  2.13);

\path[fill=fillColor,fill opacity=0.20] (189.67, 94.02) circle (  2.13);

\path[fill=fillColor,fill opacity=0.20] (180.27, 85.89) circle (  2.13);

\path[fill=fillColor,fill opacity=0.20] (178.52, 82.64) circle (  2.13);

\path[fill=fillColor,fill opacity=0.20] (204.31, 49.32) circle (  2.13);

\path[fill=fillColor,fill opacity=0.20] (203.21, 50.13) circle (  2.13);

\path[fill=fillColor,fill opacity=0.20] (203.43, 49.32) circle (  2.13);

\path[fill=fillColor,fill opacity=0.20] (198.84, 50.13) circle (  2.13);

\path[fill=fillColor,fill opacity=0.20] (202.34, 58.26) circle (  2.13);

\path[fill=fillColor,fill opacity=0.20] (207.37, 62.32) circle (  2.13);

\path[fill=fillColor,fill opacity=0.20] (207.58, 63.95) circle (  2.13);

\path[fill=fillColor,fill opacity=0.20] (206.05, 61.51) circle (  2.13);

\path[fill=fillColor,fill opacity=0.20] (205.18, 55.82) circle (  2.13);

\path[fill=fillColor,fill opacity=0.20] (200.15, 55.01) circle (  2.13);

\path[fill=fillColor,fill opacity=0.20] (192.73, 56.63) circle (  2.13);

\path[fill=fillColor,fill opacity=0.20] (181.58, 55.82) circle (  2.13);

\path[fill=fillColor,fill opacity=0.20] (207.37, 55.01) circle (  2.13);

\path[fill=fillColor,fill opacity=0.20] (206.49, 66.38) circle (  2.13);

\path[fill=fillColor,fill opacity=0.20] (213.92, 75.32) circle (  2.13);

\path[fill=fillColor,fill opacity=0.20] (210.86, 78.57) circle (  2.13);

\path[fill=fillColor,fill opacity=0.20] (210.86, 76.14) circle (  2.13);

\path[fill=fillColor,fill opacity=0.20] (206.71, 72.07) circle (  2.13);

\path[fill=fillColor,fill opacity=0.20] (203.87, 69.63) circle (  2.13);

\path[fill=fillColor,fill opacity=0.20] (201.03, 68.01) circle (  2.13);

\path[fill=fillColor,fill opacity=0.20] (190.10, 68.01) circle (  2.13);

\path[fill=fillColor,fill opacity=0.20] (171.97, 65.57) circle (  2.13);

\path[fill=fillColor,fill opacity=0.20] (200.59, 67.20) circle (  2.13);

\path[fill=fillColor,fill opacity=0.20] (207.58, 69.63) circle (  2.13);

\path[fill=fillColor,fill opacity=0.20] (212.39, 76.14) circle (  2.13);

\path[fill=fillColor,fill opacity=0.20] (215.01, 86.70) circle (  2.13);

\path[fill=fillColor,fill opacity=0.20] (211.74, 89.95) circle (  2.13);

\path[fill=fillColor,fill opacity=0.20] (209.11, 87.51) circle (  2.13);

\path[fill=fillColor,fill opacity=0.20] (204.31, 85.08) circle (  2.13);

\path[fill=fillColor,fill opacity=0.20] (200.37, 83.45) circle (  2.13);

\path[fill=fillColor,fill opacity=0.20] (193.82, 80.20) circle (  2.13);

\path[fill=fillColor,fill opacity=0.20] (181.58, 81.01) circle (  2.13);

\path[fill=fillColor,fill opacity=0.20] (166.94, 85.08) circle (  2.13);

\path[fill=fillColor,fill opacity=0.20] (180.05, 59.88) circle (  2.13);

\path[fill=fillColor,fill opacity=0.20] (205.18, 69.63) circle (  2.13);

\path[fill=fillColor,fill opacity=0.20] (208.24, 77.76) circle (  2.13);

\path[fill=fillColor,fill opacity=0.20] (208.68, 84.26) circle (  2.13);

\path[fill=fillColor,fill opacity=0.20] (207.37, 90.76) circle (  2.13);

\path[fill=fillColor,fill opacity=0.20] (207.80, 94.02) circle (  2.13);

\path[fill=fillColor,fill opacity=0.20] (204.96, 95.64) circle (  2.13);

\path[fill=fillColor,fill opacity=0.20] (199.06, 94.02) circle (  2.13);

\path[fill=fillColor,fill opacity=0.20] (192.29, 88.33) circle (  2.13);

\path[fill=fillColor,fill opacity=0.20] (183.77, 83.45) circle (  2.13);

\path[fill=fillColor,fill opacity=0.20] (168.69, 87.51) circle (  2.13);

\path[fill=fillColor,fill opacity=0.20] (204.31, 62.32) circle (  2.13);

\path[fill=fillColor,fill opacity=0.20] (205.40, 78.57) circle (  2.13);

\path[fill=fillColor,fill opacity=0.20] (206.71, 92.39) circle (  2.13);

\path[fill=fillColor,fill opacity=0.20] (205.18, 90.76) circle (  2.13);

\path[fill=fillColor,fill opacity=0.20] (205.40, 91.58) circle (  2.13);

\path[fill=fillColor,fill opacity=0.20] (201.03, 95.64) circle (  2.13);

\path[fill=fillColor,fill opacity=0.20] (198.84, 94.02) circle (  2.13);

\path[fill=fillColor,fill opacity=0.20] (187.48, 86.70) circle (  2.13);

\path[fill=fillColor,fill opacity=0.20] (172.41, 84.26) circle (  2.13);

\path[fill=fillColor,fill opacity=0.20] (203.87, 59.07) circle (  2.13);

\path[fill=fillColor,fill opacity=0.20] (205.40, 77.76) circle (  2.13);

\path[fill=fillColor,fill opacity=0.20] (207.37, 94.83) circle (  2.13);

\path[fill=fillColor,fill opacity=0.20] (209.77, 90.76) circle (  2.13);

\path[fill=fillColor,fill opacity=0.20] (201.03, 89.14) circle (  2.13);

\path[fill=fillColor,fill opacity=0.20] (200.81, 93.20) circle (  2.13);

\path[fill=fillColor,fill opacity=0.20] (199.06, 89.95) circle (  2.13);

\path[fill=fillColor,fill opacity=0.20] (186.17, 86.70) circle (  2.13);

\path[fill=fillColor,fill opacity=0.20] (170.00, 96.45) circle (  2.13);

\path[fill=fillColor,fill opacity=0.20] (209.55, 37.94) circle (  2.13);

\path[fill=fillColor,fill opacity=0.20] (210.86, 37.94) circle (  2.13);

\path[fill=fillColor,fill opacity=0.20] (210.64, 40.38) circle (  2.13);

\path[fill=fillColor,fill opacity=0.20] (207.37, 42.00) circle (  2.13);

\path[fill=fillColor,fill opacity=0.20] (205.18, 63.13) circle (  2.13);

\path[fill=fillColor,fill opacity=0.20] (209.11, 77.76) circle (  2.13);

\path[fill=fillColor,fill opacity=0.20] (207.37, 93.20) circle (  2.13);

\path[fill=fillColor,fill opacity=0.20] (208.02, 93.20) circle (  2.13);

\path[fill=fillColor,fill opacity=0.20] (204.09, 94.02) circle (  2.13);

\path[fill=fillColor,fill opacity=0.20] (200.15, 94.83) circle (  2.13);

\path[fill=fillColor,fill opacity=0.20] (199.72, 89.95) circle (  2.13);

\path[fill=fillColor,fill opacity=0.20] (188.57, 89.14) circle (  2.13);

\path[fill=fillColor,fill opacity=0.20] (212.17, 48.50) circle (  2.13);

\path[fill=fillColor,fill opacity=0.20] (207.37, 47.69) circle (  2.13);

\path[fill=fillColor,fill opacity=0.20] (206.49, 43.63) circle (  2.13);

\path[fill=fillColor,fill opacity=0.20] (196.44, 59.88) circle (  2.13);

\path[fill=fillColor,fill opacity=0.20] (211.74, 74.51) circle (  2.13);

\path[fill=fillColor,fill opacity=0.20] (208.02, 89.95) circle (  2.13);

\path[fill=fillColor,fill opacity=0.20] (208.46, 97.27) circle (  2.13);

\path[fill=fillColor,fill opacity=0.20] (202.78, 97.27) circle (  2.13);

\path[fill=fillColor,fill opacity=0.20] (196.00, 94.02) circle (  2.13);

\path[fill=fillColor,fill opacity=0.20] (197.10, 89.14) circle (  2.13);

\path[fill=fillColor,fill opacity=0.20] (192.29, 89.95) circle (  2.13);

\path[fill=fillColor,fill opacity=0.20] (196.88, 56.63) circle (  2.13);

\path[fill=fillColor,fill opacity=0.20] (208.46, 56.63) circle (  2.13);

\path[fill=fillColor,fill opacity=0.20] (206.05, 56.63) circle (  2.13);

\path[fill=fillColor,fill opacity=0.20] (202.34, 56.63) circle (  2.13);

\path[fill=fillColor,fill opacity=0.20] (201.03, 51.75) circle (  2.13);

\path[fill=fillColor,fill opacity=0.20] (209.11, 39.56) circle (  2.13);

\path[fill=fillColor,fill opacity=0.20] (182.67, 55.01) circle (  2.13);

\path[fill=fillColor,fill opacity=0.20] (206.93, 65.57) circle (  2.13);

\path[fill=fillColor,fill opacity=0.20] (206.71, 85.08) circle (  2.13);

\path[fill=fillColor,fill opacity=0.20] (205.40, 96.45) circle (  2.13);

\path[fill=fillColor,fill opacity=0.20] (203.00, 98.08) circle (  2.13);

\path[fill=fillColor,fill opacity=0.20] (197.31, 88.33) circle (  2.13);

\path[fill=fillColor,fill opacity=0.20] (197.53, 82.64) circle (  2.13);

\path[fill=fillColor,fill opacity=0.20] (197.53, 89.14) circle (  2.13);

\path[fill=fillColor,fill opacity=0.20] (188.14, 65.57) circle (  2.13);

\path[fill=fillColor,fill opacity=0.20] (192.73, 63.13) circle (  2.13);

\path[fill=fillColor,fill opacity=0.20] (195.35, 59.07) circle (  2.13);

\path[fill=fillColor,fill opacity=0.20] (199.28, 59.88) circle (  2.13);

\path[fill=fillColor,fill opacity=0.20] (197.53, 60.69) circle (  2.13);

\path[fill=fillColor,fill opacity=0.20] (204.31, 59.07) circle (  2.13);

\path[fill=fillColor,fill opacity=0.20] (197.10, 49.32) circle (  2.13);

\path[fill=fillColor,fill opacity=0.20] (200.37, 39.56) circle (  2.13);

\path[fill=fillColor,fill opacity=0.20] (185.73, 63.95) circle (  2.13);

\path[fill=fillColor,fill opacity=0.20] (199.94, 61.51) circle (  2.13);

\path[fill=fillColor,fill opacity=0.20] (208.02, 75.32) circle (  2.13);

\path[fill=fillColor,fill opacity=0.20] (202.78, 89.14) circle (  2.13);

\path[fill=fillColor,fill opacity=0.20] (200.15, 96.45) circle (  2.13);

\path[fill=fillColor,fill opacity=0.20] (200.59, 89.95) circle (  2.13);

\path[fill=fillColor,fill opacity=0.20] (199.28, 81.01) circle (  2.13);

\path[fill=fillColor,fill opacity=0.20] (201.90, 89.14) circle (  2.13);

\path[fill=fillColor,fill opacity=0.20] (191.41,103.77) circle (  2.13);

\path[fill=fillColor,fill opacity=0.20] (192.29, 78.57) circle (  2.13);

\path[fill=fillColor,fill opacity=0.20] (196.88, 73.70) circle (  2.13);

\path[fill=fillColor,fill opacity=0.20] (201.03, 67.20) circle (  2.13);

\path[fill=fillColor,fill opacity=0.20] (194.26, 62.32) circle (  2.13);

\path[fill=fillColor,fill opacity=0.20] (194.69, 60.69) circle (  2.13);

\path[fill=fillColor,fill opacity=0.20] (194.69, 61.51) circle (  2.13);

\path[fill=fillColor,fill opacity=0.20] (194.91, 50.94) circle (  2.13);

\path[fill=fillColor,fill opacity=0.20] (186.17, 37.94) circle (  2.13);

\path[fill=fillColor,fill opacity=0.20] (181.36, 59.07) circle (  2.13);

\path[fill=fillColor,fill opacity=0.20] (199.06, 64.76) circle (  2.13);

\path[fill=fillColor,fill opacity=0.20] (204.31, 80.20) circle (  2.13);

\path[fill=fillColor,fill opacity=0.20] (216.76, 94.02) circle (  2.13);

\path[fill=fillColor,fill opacity=0.20] (201.68, 94.83) circle (  2.13);

\path[fill=fillColor,fill opacity=0.20] (199.28, 85.89) circle (  2.13);

\path[fill=fillColor,fill opacity=0.20] (203.21, 85.89) circle (  2.13);

\path[fill=fillColor,fill opacity=0.20] (198.63, 93.20) circle (  2.13);

\path[fill=fillColor,fill opacity=0.20] (200.15, 81.01) circle (  2.13);

\path[fill=fillColor,fill opacity=0.20] (200.37, 76.95) circle (  2.13);

\path[fill=fillColor,fill opacity=0.20] (201.47, 70.45) circle (  2.13);

\path[fill=fillColor,fill opacity=0.20] (197.31, 63.95) circle (  2.13);

\path[fill=fillColor,fill opacity=0.20] (200.59, 61.51) circle (  2.13);

\path[fill=fillColor,fill opacity=0.20] (199.28, 63.13) circle (  2.13);

\path[fill=fillColor,fill opacity=0.20] (194.47, 54.19) circle (  2.13);

\path[fill=fillColor,fill opacity=0.20] (199.72, 38.75) circle (  2.13);

\path[fill=fillColor,fill opacity=0.20] (191.20, 39.56) circle (  2.13);

\path[fill=fillColor,fill opacity=0.20] (167.82, 50.94) circle (  2.13);

\path[fill=fillColor,fill opacity=0.20] (175.46, 57.44) circle (  2.13);

\path[fill=fillColor,fill opacity=0.20] (194.47, 72.07) circle (  2.13);

\path[fill=fillColor,fill opacity=0.20] (214.79, 79.39) circle (  2.13);

\path[fill=fillColor,fill opacity=0.20] (202.34, 87.51) circle (  2.13);

\path[fill=fillColor,fill opacity=0.20] (200.59, 88.33) circle (  2.13);

\path[fill=fillColor,fill opacity=0.20] (204.09, 80.20) circle (  2.13);

\path[fill=fillColor,fill opacity=0.20] (200.59, 81.82) circle (  2.13);

\path[fill=fillColor,fill opacity=0.20] (195.13,102.96) circle (  2.13);

\path[fill=fillColor,fill opacity=0.20] (196.88, 76.95) circle (  2.13);

\path[fill=fillColor,fill opacity=0.20] (201.03, 70.45) circle (  2.13);

\path[fill=fillColor,fill opacity=0.20] (206.27, 68.82) circle (  2.13);

\path[fill=fillColor,fill opacity=0.20] (198.63, 66.38) circle (  2.13);

\path[fill=fillColor,fill opacity=0.20] (197.10, 65.57) circle (  2.13);

\path[fill=fillColor,fill opacity=0.20] (204.52, 60.69) circle (  2.13);

\path[fill=fillColor,fill opacity=0.20] (198.63, 48.50) circle (  2.13);

\path[fill=fillColor,fill opacity=0.20] (185.73, 42.82) circle (  2.13);

\path[fill=fillColor,fill opacity=0.20] (175.68, 58.26) circle (  2.13);

\path[fill=fillColor,fill opacity=0.20] (193.16, 59.88) circle (  2.13);

\path[fill=fillColor,fill opacity=0.20] (202.56, 70.45) circle (  2.13);

\path[fill=fillColor,fill opacity=0.20] (206.71, 82.64) circle (  2.13);

\path[fill=fillColor,fill opacity=0.20] (201.03, 79.39) circle (  2.13);

\path[fill=fillColor,fill opacity=0.20] (200.37, 80.20) circle (  2.13);

\path[fill=fillColor,fill opacity=0.20] (199.94, 90.76) circle (  2.13);

\path[fill=fillColor,fill opacity=0.20] (197.97, 89.14) circle (  2.13);

\path[fill=fillColor,fill opacity=0.20] (205.18, 80.20) circle (  2.13);

\path[fill=fillColor,fill opacity=0.20] (210.21, 70.45) circle (  2.13);

\path[fill=fillColor,fill opacity=0.20] (209.55, 72.89) circle (  2.13);

\path[fill=fillColor,fill opacity=0.20] (200.59, 70.45) circle (  2.13);

\path[fill=fillColor,fill opacity=0.20] (199.72, 64.76) circle (  2.13);

\path[fill=fillColor,fill opacity=0.20] (204.31, 65.57) circle (  2.13);

\path[fill=fillColor,fill opacity=0.20] (201.90, 61.51) circle (  2.13);

\path[fill=fillColor,fill opacity=0.20] (199.06, 47.69) circle (  2.13);

\path[fill=fillColor,fill opacity=0.20] (191.63, 42.82) circle (  2.13);

\path[fill=fillColor,fill opacity=0.20] (171.97, 54.19) circle (  2.13);

\path[fill=fillColor,fill opacity=0.20] (167.38, 53.38) circle (  2.13);

\path[fill=fillColor,fill opacity=0.20] (177.43, 52.57) circle (  2.13);

\path[fill=fillColor,fill opacity=0.20] (192.73, 62.32) circle (  2.13);

\path[fill=fillColor,fill opacity=0.20] (205.62, 75.32) circle (  2.13);

\path[fill=fillColor,fill opacity=0.20] (203.43, 80.20) circle (  2.13);

\path[fill=fillColor,fill opacity=0.20] (199.94, 79.39) circle (  2.13);

\path[fill=fillColor,fill opacity=0.20] (200.15, 81.82) circle (  2.13);

\path[fill=fillColor,fill opacity=0.20] (196.22, 88.33) circle (  2.13);

\path[fill=fillColor,fill opacity=0.20] (211.95, 98.08) circle (  2.13);

\path[fill=fillColor,fill opacity=0.20] (211.08, 87.51) circle (  2.13);

\path[fill=fillColor,fill opacity=0.20] (210.64, 78.57) circle (  2.13);

\path[fill=fillColor,fill opacity=0.20] (211.74, 76.14) circle (  2.13);

\path[fill=fillColor,fill opacity=0.20] (207.80, 70.45) circle (  2.13);

\path[fill=fillColor,fill opacity=0.20] (207.58, 64.76) circle (  2.13);

\path[fill=fillColor,fill opacity=0.20] (202.12, 62.32) circle (  2.13);

\path[fill=fillColor,fill opacity=0.20] (199.06, 52.57) circle (  2.13);

\path[fill=fillColor,fill opacity=0.20] (199.50, 37.94) circle (  2.13);

\path[fill=fillColor,fill opacity=0.20] (189.23, 39.56) circle (  2.13);

\path[fill=fillColor,fill opacity=0.20] (178.96, 57.44) circle (  2.13);

\path[fill=fillColor,fill opacity=0.20] (183.77, 59.07) circle (  2.13);

\path[fill=fillColor,fill opacity=0.20] (186.83, 66.38) circle (  2.13);

\path[fill=fillColor,fill opacity=0.20] (200.59, 72.89) circle (  2.13);

\path[fill=fillColor,fill opacity=0.20] (204.74, 74.51) circle (  2.13);

\path[fill=fillColor,fill opacity=0.20] (202.56, 78.57) circle (  2.13);

\path[fill=fillColor,fill opacity=0.20] (198.19, 85.08) circle (  2.13);

\path[fill=fillColor,fill opacity=0.20] (194.91, 87.51) circle (  2.13);

\path[fill=fillColor,fill opacity=0.20] (190.54,104.58) circle (  2.13);

\path[fill=fillColor,fill opacity=0.20] (210.21, 93.20) circle (  2.13);

\path[fill=fillColor,fill opacity=0.20] (217.20, 91.58) circle (  2.13);

\path[fill=fillColor,fill opacity=0.20] (213.48, 81.82) circle (  2.13);

\path[fill=fillColor,fill opacity=0.20] (209.99, 77.76) circle (  2.13);

\path[fill=fillColor,fill opacity=0.20] (209.33, 74.51) circle (  2.13);

\path[fill=fillColor,fill opacity=0.20] (213.70, 70.45) circle (  2.13);

\path[fill=fillColor,fill opacity=0.20] (212.61, 68.82) circle (  2.13);

\path[fill=fillColor,fill opacity=0.20] (208.68, 63.95) circle (  2.13);

\path[fill=fillColor,fill opacity=0.20] (203.00, 49.32) circle (  2.13);

\path[fill=fillColor,fill opacity=0.20] (170.88, 55.82) circle (  2.13);

\path[fill=fillColor,fill opacity=0.20] (169.56, 58.26) circle (  2.13);

\path[fill=fillColor,fill opacity=0.20] (180.27, 62.32) circle (  2.13);

\path[fill=fillColor,fill opacity=0.20] (198.19, 68.82) circle (  2.13);

\path[fill=fillColor,fill opacity=0.20] (203.21, 77.76) circle (  2.13);

\path[fill=fillColor,fill opacity=0.20] (201.25, 87.51) circle (  2.13);

\path[fill=fillColor,fill opacity=0.20] (201.03, 86.70) circle (  2.13);

\path[fill=fillColor,fill opacity=0.20] (194.04, 86.70) circle (  2.13);

\path[fill=fillColor,fill opacity=0.20] (210.21, 85.89) circle (  2.13);

\path[fill=fillColor,fill opacity=0.20] (220.04, 82.64) circle (  2.13);

\path[fill=fillColor,fill opacity=0.20] (219.38, 77.76) circle (  2.13);

\path[fill=fillColor,fill opacity=0.20] (212.83, 70.45) circle (  2.13);

\path[fill=fillColor,fill opacity=0.20] (211.08, 72.07) circle (  2.13);

\path[fill=fillColor,fill opacity=0.20] (206.93, 76.14) circle (  2.13);

\path[fill=fillColor,fill opacity=0.20] (205.84, 72.07) circle (  2.13);

\path[fill=fillColor,fill opacity=0.20] (206.71, 70.45) circle (  2.13);

\path[fill=fillColor,fill opacity=0.20] (204.31, 67.20) circle (  2.13);

\path[fill=fillColor,fill opacity=0.20] (200.15, 57.44) circle (  2.13);

\path[fill=fillColor,fill opacity=0.20] (188.57, 43.63) circle (  2.13);

\path[fill=fillColor,fill opacity=0.20] (170.88, 59.88) circle (  2.13);

\path[fill=fillColor,fill opacity=0.20] (175.90, 59.07) circle (  2.13);

\path[fill=fillColor,fill opacity=0.20] (181.15, 62.32) circle (  2.13);

\path[fill=fillColor,fill opacity=0.20] (194.69, 72.07) circle (  2.13);

\path[fill=fillColor,fill opacity=0.20] (203.87, 80.20) circle (  2.13);

\path[fill=fillColor,fill opacity=0.20] (203.87, 87.51) circle (  2.13);

\path[fill=fillColor,fill opacity=0.20] (202.34, 87.51) circle (  2.13);

\path[fill=fillColor,fill opacity=0.20] (198.41, 87.51) circle (  2.13);

\path[fill=fillColor,fill opacity=0.20] (214.14, 88.33) circle (  2.13);

\path[fill=fillColor,fill opacity=0.20] (218.29, 80.20) circle (  2.13);

\path[fill=fillColor,fill opacity=0.20] (227.69, 79.39) circle (  2.13);

\path[fill=fillColor,fill opacity=0.20] (218.07, 77.76) circle (  2.13);

\path[fill=fillColor,fill opacity=0.20] (213.05, 72.89) circle (  2.13);

\path[fill=fillColor,fill opacity=0.20] (215.23, 77.76) circle (  2.13);

\path[fill=fillColor,fill opacity=0.20] (208.68, 81.82) circle (  2.13);

\path[fill=fillColor,fill opacity=0.20] (202.78, 75.32) circle (  2.13);

\path[fill=fillColor,fill opacity=0.20] (202.78, 69.63) circle (  2.13);

\path[fill=fillColor,fill opacity=0.20] (201.90, 66.38) circle (  2.13);

\path[fill=fillColor,fill opacity=0.20] (196.22, 57.44) circle (  2.13);

\path[fill=fillColor,fill opacity=0.20] (177.65, 48.50) circle (  2.13);

\path[fill=fillColor,fill opacity=0.20] (166.07, 59.88) circle (  2.13);

\path[fill=fillColor,fill opacity=0.20] (182.02, 63.13) circle (  2.13);

\path[fill=fillColor,fill opacity=0.20] (201.90, 71.26) circle (  2.13);

\path[fill=fillColor,fill opacity=0.20] (208.24, 81.82) circle (  2.13);

\path[fill=fillColor,fill opacity=0.20] (206.05, 89.95) circle (  2.13);

\path[fill=fillColor,fill opacity=0.20] (201.03, 88.33) circle (  2.13);

\path[fill=fillColor,fill opacity=0.20] (200.15, 90.76) circle (  2.13);

\path[fill=fillColor,fill opacity=0.20] (217.42, 98.89) circle (  2.13);

\path[fill=fillColor,fill opacity=0.20] (222.44, 86.70) circle (  2.13);

\path[fill=fillColor,fill opacity=0.20] (220.69, 84.26) circle (  2.13);

\path[fill=fillColor,fill opacity=0.20] (223.10, 84.26) circle (  2.13);

\path[fill=fillColor,fill opacity=0.20] (215.45, 81.01) circle (  2.13);

\path[fill=fillColor,fill opacity=0.20] (213.92, 79.39) circle (  2.13);

\path[fill=fillColor,fill opacity=0.20] (214.14, 81.01) circle (  2.13);

\path[fill=fillColor,fill opacity=0.20] (212.17, 78.57) circle (  2.13);

\path[fill=fillColor,fill opacity=0.20] (207.37, 72.89) circle (  2.13);

\path[fill=fillColor,fill opacity=0.20] (196.88, 71.26) circle (  2.13);

\path[fill=fillColor,fill opacity=0.20] (190.98, 62.32) circle (  2.13);

\path[fill=fillColor,fill opacity=0.20] (184.20, 49.32) circle (  2.13);

\path[fill=fillColor,fill opacity=0.20] (180.71, 44.44) circle (  2.13);

\path[fill=fillColor,fill opacity=0.20] (165.41, 58.26) circle (  2.13);

\path[fill=fillColor,fill opacity=0.20] (178.74, 61.51) circle (  2.13);

\path[fill=fillColor,fill opacity=0.20] (193.82, 68.82) circle (  2.13);

\path[fill=fillColor,fill opacity=0.20] (198.63, 82.64) circle (  2.13);

\path[fill=fillColor,fill opacity=0.20] (208.46, 87.51) circle (  2.13);

\path[fill=fillColor,fill opacity=0.20] (207.37, 83.45) circle (  2.13);

\path[fill=fillColor,fill opacity=0.20] (204.52, 90.76) circle (  2.13);

\path[fill=fillColor,fill opacity=0.20] (219.82,115.15) circle (  2.13);

\path[fill=fillColor,fill opacity=0.20] (214.36, 97.27) circle (  2.13);

\path[fill=fillColor,fill opacity=0.20] (219.82, 85.89) circle (  2.13);

\path[fill=fillColor,fill opacity=0.20] (223.97, 85.08) circle (  2.13);

\path[fill=fillColor,fill opacity=0.20] (219.82, 85.08) circle (  2.13);

\path[fill=fillColor,fill opacity=0.20] (212.39, 85.08) circle (  2.13);

\path[fill=fillColor,fill opacity=0.20] (212.61, 80.20) circle (  2.13);

\path[fill=fillColor,fill opacity=0.20] (217.20, 80.20) circle (  2.13);

\path[fill=fillColor,fill opacity=0.20] (208.89, 81.01) circle (  2.13);

\path[fill=fillColor,fill opacity=0.20] (203.43, 69.63) circle (  2.13);

\path[fill=fillColor,fill opacity=0.20] (192.94, 62.32) circle (  2.13);

\path[fill=fillColor,fill opacity=0.20] (185.08, 63.95) circle (  2.13);

\path[fill=fillColor,fill opacity=0.20] (175.68, 57.44) circle (  2.13);

\path[fill=fillColor,fill opacity=0.20] (183.99, 42.82) circle (  2.13);

\path[fill=fillColor,fill opacity=0.20] (166.94, 53.38) circle (  2.13);

\path[fill=fillColor,fill opacity=0.20] (175.46, 56.63) circle (  2.13);

\path[fill=fillColor,fill opacity=0.20] (187.48, 68.01) circle (  2.13);

\path[fill=fillColor,fill opacity=0.20] (201.90, 77.76) circle (  2.13);

\path[fill=fillColor,fill opacity=0.20] (206.71, 81.82) circle (  2.13);

\path[fill=fillColor,fill opacity=0.20] (204.09, 83.45) circle (  2.13);

\path[fill=fillColor,fill opacity=0.20] (204.52, 84.26) circle (  2.13);

\path[fill=fillColor,fill opacity=0.20] (198.41, 95.64) circle (  2.13);

\path[fill=fillColor,fill opacity=0.20] (221.57,103.77) circle (  2.13);

\path[fill=fillColor,fill opacity=0.20] (225.72, 96.45) circle (  2.13);

\path[fill=fillColor,fill opacity=0.20] (219.60, 83.45) circle (  2.13);

\path[fill=fillColor,fill opacity=0.20] (220.48, 81.01) circle (  2.13);

\path[fill=fillColor,fill opacity=0.20] (227.03, 85.89) circle (  2.13);

\path[fill=fillColor,fill opacity=0.20] (219.38, 81.01) circle (  2.13);

\path[fill=fillColor,fill opacity=0.20] (232.06, 78.57) circle (  2.13);

\path[fill=fillColor,fill opacity=0.20] (219.38, 75.32) circle (  2.13);

\path[fill=fillColor,fill opacity=0.20] (208.46, 74.51) circle (  2.13);

\path[fill=fillColor,fill opacity=0.20] (198.41, 74.51) circle (  2.13);

\path[fill=fillColor,fill opacity=0.20] (192.29, 59.88) circle (  2.13);

\path[fill=fillColor,fill opacity=0.20] (178.52, 48.50) circle (  2.13);

\path[fill=fillColor,fill opacity=0.20] (174.15, 51.75) circle (  2.13);

\path[fill=fillColor,fill opacity=0.20] (185.30, 50.94) circle (  2.13);

\path[fill=fillColor,fill opacity=0.20] (169.56, 55.82) circle (  2.13);

\path[fill=fillColor,fill opacity=0.20] (183.99, 61.51) circle (  2.13);

\path[fill=fillColor,fill opacity=0.20] (197.75, 70.45) circle (  2.13);

\path[fill=fillColor,fill opacity=0.20] (207.15, 77.76) circle (  2.13);

\path[fill=fillColor,fill opacity=0.20] (205.18, 79.39) circle (  2.13);

\path[fill=fillColor,fill opacity=0.20] (203.87, 81.82) circle (  2.13);

\path[fill=fillColor,fill opacity=0.20] (197.97, 92.39) circle (  2.13);

\path[fill=fillColor,fill opacity=0.20] (225.72,111.89) circle (  2.13);

\path[fill=fillColor,fill opacity=0.20] (228.56, 99.70) circle (  2.13);

\path[fill=fillColor,fill opacity=0.20] (234.02, 97.27) circle (  2.13);

\path[fill=fillColor,fill opacity=0.20] (222.88, 93.20) circle (  2.13);

\path[fill=fillColor,fill opacity=0.20] (223.97, 87.51) circle (  2.13);

\path[fill=fillColor,fill opacity=0.20] (216.76, 87.51) circle (  2.13);

\path[fill=fillColor,fill opacity=0.20] (221.35, 84.26) circle (  2.13);

\path[fill=fillColor,fill opacity=0.20] (219.38, 79.39) circle (  2.13);

\path[fill=fillColor,fill opacity=0.20] (209.77, 75.32) circle (  2.13);

\path[fill=fillColor,fill opacity=0.20] (202.12, 66.38) circle (  2.13);

\path[fill=fillColor,fill opacity=0.20] (187.48, 62.32) circle (  2.13);

\path[fill=fillColor,fill opacity=0.20] (183.11, 62.32) circle (  2.13);

\path[fill=fillColor,fill opacity=0.20] (182.67, 53.38) circle (  2.13);

\path[fill=fillColor,fill opacity=0.20] (179.83, 47.69) circle (  2.13);

\path[fill=fillColor,fill opacity=0.20] (188.79, 50.94) circle (  2.13);

\path[fill=fillColor,fill opacity=0.20] (166.94, 49.32) circle (  2.13);

\path[fill=fillColor,fill opacity=0.20] (185.30, 59.07) circle (  2.13);

\path[fill=fillColor,fill opacity=0.20] (197.31, 74.51) circle (  2.13);

\path[fill=fillColor,fill opacity=0.20] (205.40, 88.33) circle (  2.13);

\path[fill=fillColor,fill opacity=0.20] (206.27, 90.76) circle (  2.13);

\path[fill=fillColor,fill opacity=0.20] (200.37, 86.70) circle (  2.13);

\path[fill=fillColor,fill opacity=0.20] (200.59, 88.33) circle (  2.13);

\path[fill=fillColor,fill opacity=0.20] (198.41,101.33) circle (  2.13);

\path[fill=fillColor,fill opacity=0.20] (225.50,101.33) circle (  2.13);

\path[fill=fillColor,fill opacity=0.20] (230.74, 98.08) circle (  2.13);

\path[fill=fillColor,fill opacity=0.20] (247.35, 98.89) circle (  2.13);

\path[fill=fillColor,fill opacity=0.20] (219.38, 94.83) circle (  2.13);

\path[fill=fillColor,fill opacity=0.20] (219.60, 89.95) circle (  2.13);

\path[fill=fillColor,fill opacity=0.20] (221.13, 87.51) circle (  2.13);

\path[fill=fillColor,fill opacity=0.20] (214.36, 83.45) circle (  2.13);

\path[fill=fillColor,fill opacity=0.20] (206.49, 80.20) circle (  2.13);

\path[fill=fillColor,fill opacity=0.20] (196.66, 75.32) circle (  2.13);

\path[fill=fillColor,fill opacity=0.20] (182.24, 67.20) circle (  2.13);

\path[fill=fillColor,fill opacity=0.20] (182.02, 59.88) circle (  2.13);

\path[fill=fillColor,fill opacity=0.20] (174.81, 59.07) circle (  2.13);

\path[fill=fillColor,fill opacity=0.20] (182.67, 58.26) circle (  2.13);

\path[fill=fillColor,fill opacity=0.20] (188.14, 50.13) circle (  2.13);

\path[fill=fillColor,fill opacity=0.20] (180.27, 68.82) circle (  2.13);

\path[fill=fillColor,fill opacity=0.20] (189.89, 78.57) circle (  2.13);

\path[fill=fillColor,fill opacity=0.20] (198.41, 84.26) circle (  2.13);

\path[fill=fillColor,fill opacity=0.20] (204.96, 85.89) circle (  2.13);

\path[fill=fillColor,fill opacity=0.20] (201.25, 85.08) circle (  2.13);

\path[fill=fillColor,fill opacity=0.20] (200.81, 84.26) circle (  2.13);

\path[fill=fillColor,fill opacity=0.20] (201.03, 92.39) circle (  2.13);

\path[fill=fillColor,fill opacity=0.20] (196.88,105.39) circle (  2.13);

\path[fill=fillColor,fill opacity=0.20] (230.09,111.89) circle (  2.13);

\path[fill=fillColor,fill opacity=0.20] (225.28,106.21) circle (  2.13);

\path[fill=fillColor,fill opacity=0.20] (220.26, 98.89) circle (  2.13);

\path[fill=fillColor,fill opacity=0.20] (219.82, 92.39) circle (  2.13);

\path[fill=fillColor,fill opacity=0.20] (217.63, 92.39) circle (  2.13);

\path[fill=fillColor,fill opacity=0.20] (208.89, 85.08) circle (  2.13);

\path[fill=fillColor,fill opacity=0.20] (200.37, 74.51) circle (  2.13);

\path[fill=fillColor,fill opacity=0.20] (191.41, 69.63) circle (  2.13);

\path[fill=fillColor,fill opacity=0.20] (190.32, 69.63) circle (  2.13);

\path[fill=fillColor,fill opacity=0.20] (183.99, 69.63) circle (  2.13);

\path[fill=fillColor,fill opacity=0.20] (191.41, 65.57) circle (  2.13);

\path[fill=fillColor,fill opacity=0.20] (184.64, 58.26) circle (  2.13);

\path[fill=fillColor,fill opacity=0.20] (179.62, 57.44) circle (  2.13);

\path[fill=fillColor,fill opacity=0.20] (168.04, 60.69) circle (  2.13);

\path[fill=fillColor,fill opacity=0.20] (183.77, 58.26) circle (  2.13);

\path[fill=fillColor,fill opacity=0.20] (183.55, 62.32) circle (  2.13);

\path[fill=fillColor,fill opacity=0.20] (198.41, 72.89) circle (  2.13);

\path[fill=fillColor,fill opacity=0.20] (201.25, 85.89) circle (  2.13);

\path[fill=fillColor,fill opacity=0.20] (203.87, 90.76) circle (  2.13);

\path[fill=fillColor,fill opacity=0.20] (205.84, 90.76) circle (  2.13);

\path[fill=fillColor,fill opacity=0.20] (200.81, 92.39) circle (  2.13);

\path[fill=fillColor,fill opacity=0.20] (197.10, 93.20) circle (  2.13);

\path[fill=fillColor,fill opacity=0.20] (199.06, 97.27) circle (  2.13);

\path[fill=fillColor,fill opacity=0.20] (224.41,103.77) circle (  2.13);

\path[fill=fillColor,fill opacity=0.20] (227.25, 99.70) circle (  2.13);

\path[fill=fillColor,fill opacity=0.20] (212.61, 98.89) circle (  2.13);

\path[fill=fillColor,fill opacity=0.20] (205.84, 92.39) circle (  2.13);

\path[fill=fillColor,fill opacity=0.20] (200.59, 82.64) circle (  2.13);

\path[fill=fillColor,fill opacity=0.20] (208.46, 76.14) circle (  2.13);

\path[fill=fillColor,fill opacity=0.20] (189.23, 68.82) circle (  2.13);

\path[fill=fillColor,fill opacity=0.20] (185.73, 59.07) circle (  2.13);

\path[fill=fillColor,fill opacity=0.20] (182.02, 59.07) circle (  2.13);

\path[fill=fillColor,fill opacity=0.20] (165.19, 65.57) circle (  2.13);

\path[fill=fillColor,fill opacity=0.20] (178.96, 65.57) circle (  2.13);

\path[fill=fillColor,fill opacity=0.20] (190.98, 60.69) circle (  2.13);

\path[fill=fillColor,fill opacity=0.20] (173.72, 56.63) circle (  2.13);

\path[fill=fillColor,fill opacity=0.20] (170.00, 51.75) circle (  2.13);

\path[fill=fillColor,fill opacity=0.20] (175.68, 53.38) circle (  2.13);

\path[fill=fillColor,fill opacity=0.20] (180.05, 59.88) circle (  2.13);

\path[fill=fillColor,fill opacity=0.20] (193.60, 75.32) circle (  2.13);

\path[fill=fillColor,fill opacity=0.20] (200.59, 88.33) circle (  2.13);

\path[fill=fillColor,fill opacity=0.20] (198.63, 84.26) circle (  2.13);

\path[fill=fillColor,fill opacity=0.20] (202.34, 84.26) circle (  2.13);

\path[fill=fillColor,fill opacity=0.20] (198.63, 90.76) circle (  2.13);

\path[fill=fillColor,fill opacity=0.20] (198.84, 90.76) circle (  2.13);

\path[fill=fillColor,fill opacity=0.20] (200.37, 90.76) circle (  2.13);

\path[fill=fillColor,fill opacity=0.20] (200.59, 98.89) circle (  2.13);

\path[fill=fillColor,fill opacity=0.20] (209.77,115.15) circle (  2.13);

\path[fill=fillColor,fill opacity=0.20] (209.11,111.08) circle (  2.13);

\path[fill=fillColor,fill opacity=0.20] (205.84,105.39) circle (  2.13);

\path[fill=fillColor,fill opacity=0.20] (201.90, 93.20) circle (  2.13);

\path[fill=fillColor,fill opacity=0.20] (198.84, 83.45) circle (  2.13);

\path[fill=fillColor,fill opacity=0.20] (186.17, 81.82) circle (  2.13);

\path[fill=fillColor,fill opacity=0.20] (192.07, 76.14) circle (  2.13);

\path[fill=fillColor,fill opacity=0.20] (183.55, 68.82) circle (  2.13);

\path[fill=fillColor,fill opacity=0.20] (180.27, 65.57) circle (  2.13);

\path[fill=fillColor,fill opacity=0.20] (185.73, 63.95) circle (  2.13);

\path[fill=fillColor,fill opacity=0.20] (186.39, 62.32) circle (  2.13);

\path[fill=fillColor,fill opacity=0.20] (201.03, 63.13) circle (  2.13);

\path[fill=fillColor,fill opacity=0.20] (171.31, 55.01) circle (  2.13);

\path[fill=fillColor,fill opacity=0.20] (176.34, 59.07) circle (  2.13);

\path[fill=fillColor,fill opacity=0.20] (184.86, 63.95) circle (  2.13);

\path[fill=fillColor,fill opacity=0.20] (194.04, 63.13) circle (  2.13);

\path[fill=fillColor,fill opacity=0.20] (197.75, 70.45) circle (  2.13);

\path[fill=fillColor,fill opacity=0.20] (195.78, 85.89) circle (  2.13);

\path[fill=fillColor,fill opacity=0.20] (201.03, 89.95) circle (  2.13);

\path[fill=fillColor,fill opacity=0.20] (203.87, 86.70) circle (  2.13);

\path[fill=fillColor,fill opacity=0.20] (199.06, 88.33) circle (  2.13);

\path[fill=fillColor,fill opacity=0.20] (193.82, 89.14) circle (  2.13);

\path[fill=fillColor,fill opacity=0.20] (196.44, 95.64) circle (  2.13);

\path[fill=fillColor,fill opacity=0.20] (209.33,110.27) circle (  2.13);

\path[fill=fillColor,fill opacity=0.20] (202.12, 99.70) circle (  2.13);

\path[fill=fillColor,fill opacity=0.20] (192.51, 91.58) circle (  2.13);

\path[fill=fillColor,fill opacity=0.20] (185.95, 83.45) circle (  2.13);

\path[fill=fillColor,fill opacity=0.20] (185.30, 82.64) circle (  2.13);

\path[fill=fillColor,fill opacity=0.20] (179.18, 83.45) circle (  2.13);

\path[fill=fillColor,fill opacity=0.20] (179.40, 76.95) circle (  2.13);

\path[fill=fillColor,fill opacity=0.20] (180.27, 68.82) circle (  2.13);

\path[fill=fillColor,fill opacity=0.20] (185.08, 66.38) circle (  2.13);

\path[fill=fillColor,fill opacity=0.20] (183.11, 62.32) circle (  2.13);

\path[fill=fillColor,fill opacity=0.20] (190.32, 58.26) circle (  2.13);

\path[fill=fillColor,fill opacity=0.20] (182.24, 62.32) circle (  2.13);

\path[fill=fillColor,fill opacity=0.20] (172.19, 56.63) circle (  2.13);

\path[fill=fillColor,fill opacity=0.20] (174.59, 56.63) circle (  2.13);

\path[fill=fillColor,fill opacity=0.20] (175.46, 55.82) circle (  2.13);

\path[fill=fillColor,fill opacity=0.20] (192.07, 56.63) circle (  2.13);

\path[fill=fillColor,fill opacity=0.20] (182.67, 63.95) circle (  2.13);

\path[fill=fillColor,fill opacity=0.20] (187.70, 75.32) circle (  2.13);

\path[fill=fillColor,fill opacity=0.20] (196.22, 76.95) circle (  2.13);

\path[fill=fillColor,fill opacity=0.20] (200.37, 79.39) circle (  2.13);

\path[fill=fillColor,fill opacity=0.20] (199.28, 85.89) circle (  2.13);

\path[fill=fillColor,fill opacity=0.20] (194.91, 84.26) circle (  2.13);

\path[fill=fillColor,fill opacity=0.20] (193.60, 77.76) circle (  2.13);

\path[fill=fillColor,fill opacity=0.20] (195.35, 81.82) circle (  2.13);

\path[fill=fillColor,fill opacity=0.20] (192.94, 87.51) circle (  2.13);

\path[fill=fillColor,fill opacity=0.20] (190.54, 87.51) circle (  2.13);

\path[fill=fillColor,fill opacity=0.20] (196.88, 90.76) circle (  2.13);

\path[fill=fillColor,fill opacity=0.20] (204.09, 96.45) circle (  2.13);

\path[fill=fillColor,fill opacity=0.20] (190.54, 97.27) circle (  2.13);

\path[fill=fillColor,fill opacity=0.20] (199.06, 94.83) circle (  2.13);

\path[fill=fillColor,fill opacity=0.20] (198.41, 94.02) circle (  2.13);

\path[fill=fillColor,fill opacity=0.20] (199.50, 96.45) circle (  2.13);

\path[fill=fillColor,fill opacity=0.20] (198.41, 98.89) circle (  2.13);

\path[fill=fillColor,fill opacity=0.20] (195.78, 99.70) circle (  2.13);

\path[fill=fillColor,fill opacity=0.20] (206.93,102.14) circle (  2.13);

\path[fill=fillColor,fill opacity=0.20] (209.33,102.96) circle (  2.13);

\path[fill=fillColor,fill opacity=0.20] (215.67,102.14) circle (  2.13);

\path[fill=fillColor,fill opacity=0.20] (212.61,102.14) circle (  2.13);

\path[fill=fillColor,fill opacity=0.20] (209.55, 96.45) circle (  2.13);

\path[fill=fillColor,fill opacity=0.20] (209.55, 86.70) circle (  2.13);

\path[fill=fillColor,fill opacity=0.20] (201.90, 85.89) circle (  2.13);

\path[fill=fillColor,fill opacity=0.20] (199.50, 87.51) circle (  2.13);

\path[fill=fillColor,fill opacity=0.20] (185.08, 81.82) circle (  2.13);

\path[fill=fillColor,fill opacity=0.20] (182.02, 76.14) circle (  2.13);

\path[fill=fillColor,fill opacity=0.20] (175.03, 75.32) circle (  2.13);

\path[fill=fillColor,fill opacity=0.20] (173.28, 72.89) circle (  2.13);

\path[fill=fillColor,fill opacity=0.20] (182.89, 70.45) circle (  2.13);

\path[fill=fillColor,fill opacity=0.20] (168.91, 71.26) circle (  2.13);

\path[fill=fillColor,fill opacity=0.20] (179.83, 70.45) circle (  2.13);

\path[fill=fillColor,fill opacity=0.20] (188.57, 67.20) circle (  2.13);

\path[fill=fillColor,fill opacity=0.20] (190.10, 60.69) circle (  2.13);

\path[fill=fillColor,fill opacity=0.20] (181.58, 63.13) circle (  2.13);

\path[fill=fillColor,fill opacity=0.20] (177.87, 60.69) circle (  2.13);

\path[fill=fillColor,fill opacity=0.20] (169.13, 61.51) circle (  2.13);

\path[fill=fillColor,fill opacity=0.20] (175.25, 67.20) circle (  2.13);

\path[fill=fillColor,fill opacity=0.20] (185.52, 67.20) circle (  2.13);

\path[fill=fillColor,fill opacity=0.20] (188.79, 68.82) circle (  2.13);

\path[fill=fillColor,fill opacity=0.20] (189.01, 73.70) circle (  2.13);

\path[fill=fillColor,fill opacity=0.20] (188.14, 72.89) circle (  2.13);

\path[fill=fillColor,fill opacity=0.20] (190.54, 71.26) circle (  2.13);

\path[fill=fillColor,fill opacity=0.20] (199.94, 76.14) circle (  2.13);

\path[fill=fillColor,fill opacity=0.20] (196.44, 79.39) circle (  2.13);

\path[fill=fillColor,fill opacity=0.20] (196.66, 77.76) circle (  2.13);

\path[fill=fillColor,fill opacity=0.20] (198.41, 78.57) circle (  2.13);

\path[fill=fillColor,fill opacity=0.20] (197.97, 83.45) circle (  2.13);

\path[fill=fillColor,fill opacity=0.20] (192.07, 86.70) circle (  2.13);

\path[fill=fillColor,fill opacity=0.20] (196.00, 86.70) circle (  2.13);

\path[fill=fillColor,fill opacity=0.20] (201.47, 85.89) circle (  2.13);

\path[fill=fillColor,fill opacity=0.20] (202.34, 85.89) circle (  2.13);

\path[fill=fillColor,fill opacity=0.20] (208.24, 83.45) circle (  2.13);

\path[fill=fillColor,fill opacity=0.20] (202.12, 84.26) circle (  2.13);

\path[fill=fillColor,fill opacity=0.20] (202.34, 81.82) circle (  2.13);

\path[fill=fillColor,fill opacity=0.20] (199.94, 81.82) circle (  2.13);

\path[fill=fillColor,fill opacity=0.20] (206.27, 84.26) circle (  2.13);

\path[fill=fillColor,fill opacity=0.20] (206.27, 88.33) circle (  2.13);

\path[fill=fillColor,fill opacity=0.20] (206.93, 89.14) circle (  2.13);

\path[fill=fillColor,fill opacity=0.20] (205.18, 84.26) circle (  2.13);

\path[fill=fillColor,fill opacity=0.20] (201.68, 76.14) circle (  2.13);

\path[fill=fillColor,fill opacity=0.20] (192.29, 70.45) circle (  2.13);

\path[fill=fillColor,fill opacity=0.20] (182.02, 67.20) circle (  2.13);

\path[fill=fillColor,fill opacity=0.20] (175.68, 68.82) circle (  2.13);

\path[fill=fillColor,fill opacity=0.20] (168.04, 74.51) circle (  2.13);

\path[fill=fillColor,fill opacity=0.20] (175.46, 74.51) circle (  2.13);

\path[fill=fillColor,fill opacity=0.20] (178.74, 73.70) circle (  2.13);

\path[fill=fillColor,fill opacity=0.20] (172.84, 78.57) circle (  2.13);

\path[fill=fillColor,fill opacity=0.20] (179.40, 82.64) circle (  2.13);

\path[fill=fillColor,fill opacity=0.20] (182.46, 58.26) circle (  2.13);

\path[fill=fillColor,fill opacity=0.20] (180.05, 64.76) circle (  2.13);

\path[fill=fillColor,fill opacity=0.20] (179.83, 65.57) circle (  2.13);

\path[fill=fillColor,fill opacity=0.20] (172.19, 59.88) circle (  2.13);

\path[fill=fillColor,fill opacity=0.20] (178.09, 55.82) circle (  2.13);

\path[fill=fillColor,fill opacity=0.20] (175.46, 55.82) circle (  2.13);

\path[fill=fillColor,fill opacity=0.20] (182.02, 61.51) circle (  2.13);

\path[fill=fillColor,fill opacity=0.20] (189.01, 69.63) circle (  2.13);

\path[fill=fillColor,fill opacity=0.20] (192.94, 72.89) circle (  2.13);

\path[fill=fillColor,fill opacity=0.20] (194.91, 73.70) circle (  2.13);

\path[fill=fillColor,fill opacity=0.20] (199.94, 74.51) circle (  2.13);

\path[fill=fillColor,fill opacity=0.20] (197.31, 76.95) circle (  2.13);

\path[fill=fillColor,fill opacity=0.20] (200.15, 79.39) circle (  2.13);

\path[fill=fillColor,fill opacity=0.20] (204.74, 83.45) circle (  2.13);

\path[fill=fillColor,fill opacity=0.20] (208.02, 88.33) circle (  2.13);

\path[fill=fillColor,fill opacity=0.20] (203.43, 85.08) circle (  2.13);

\path[fill=fillColor,fill opacity=0.20] (203.65, 77.76) circle (  2.13);

\path[fill=fillColor,fill opacity=0.20] (196.66, 74.51) circle (  2.13);

\path[fill=fillColor,fill opacity=0.20] (188.14, 72.89) circle (  2.13);

\path[fill=fillColor,fill opacity=0.20] (190.54, 70.45) circle (  2.13);

\path[fill=fillColor,fill opacity=0.20] (189.45, 71.26) circle (  2.13);

\path[fill=fillColor,fill opacity=0.20] (183.55, 73.70) circle (  2.13);

\path[fill=fillColor,fill opacity=0.20] (185.30, 73.70) circle (  2.13);

\path[fill=fillColor,fill opacity=0.20] (183.99, 70.45) circle (  2.13);

\path[fill=fillColor,fill opacity=0.20] (178.96, 62.32) circle (  2.13);

\path[fill=fillColor,fill opacity=0.20] (174.59, 59.88) circle (  2.13);

\path[fill=fillColor,fill opacity=0.20] (173.50, 64.76) circle (  2.13);

\path[fill=fillColor,fill opacity=0.20] (183.55, 71.26) circle (  2.13);

\path[fill=fillColor,fill opacity=0.20] (182.46, 51.75) circle (  2.13);

\path[fill=fillColor,fill opacity=0.20] (182.24, 50.94) circle (  2.13);

\path[fill=fillColor,fill opacity=0.20] (174.15, 56.63) circle (  2.13);

\path[fill=fillColor,fill opacity=0.20] (167.82, 61.51) circle (  2.13);

\path[fill=fillColor,fill opacity=0.20] (180.27, 61.51) circle (  2.13);

\path[fill=fillColor,fill opacity=0.20] (179.40, 60.69) circle (  2.13);

\path[fill=fillColor,fill opacity=0.20] (184.20, 63.95) circle (  2.13);

\path[fill=fillColor,fill opacity=0.20] (188.14, 69.63) circle (  2.13);

\path[fill=fillColor,fill opacity=0.20] (193.38, 68.01) circle (  2.13);

\path[fill=fillColor,fill opacity=0.20] (195.35, 63.95) circle (  2.13);

\path[fill=fillColor,fill opacity=0.20] (199.50, 67.20) circle (  2.13);

\path[fill=fillColor,fill opacity=0.20] (196.88, 74.51) circle (  2.13);

\path[fill=fillColor,fill opacity=0.20] (188.57, 70.45) circle (  2.13);

\path[fill=fillColor,fill opacity=0.20] (183.55, 63.95) circle (  2.13);

\path[fill=fillColor,fill opacity=0.20] (178.09, 61.51) circle (  2.13);

\path[fill=fillColor,fill opacity=0.20] (174.15, 59.88) circle (  2.13);

\path[fill=fillColor,fill opacity=0.20] (176.78, 55.01) circle (  2.13);

\path[fill=fillColor,fill opacity=0.20] (171.09, 50.94) circle (  2.13);

\path[fill=fillColor,fill opacity=0.20] (170.66, 50.13) circle (  2.13);

\path[fill=fillColor,fill opacity=0.20] (181.36, 53.38) circle (  2.13);

\path[fill=fillColor,fill opacity=0.20] (185.73, 63.13) circle (  2.13);

\path[fill=fillColor,fill opacity=0.20] (185.08, 65.57) circle (  2.13);

\path[fill=fillColor,fill opacity=0.20] (186.61, 64.76) circle (  2.13);

\path[fill=fillColor,fill opacity=0.20] (179.40, 72.07) circle (  2.13);

\path[fill=fillColor,fill opacity=0.20] (181.80, 66.38) circle (  2.13);

\path[fill=fillColor,fill opacity=0.20] (176.56, 59.07) circle (  2.13);

\path[fill=fillColor,fill opacity=0.20] (177.21, 52.57) circle (  2.13);

\path[fill=fillColor,fill opacity=0.20] (171.97, 57.44) circle (  2.13);

\path[fill=fillColor,fill opacity=0.20] (180.49, 62.32) circle (  2.13);

\path[fill=fillColor,fill opacity=0.20] (193.38, 60.69) circle (  2.13);

\path[fill=fillColor,fill opacity=0.20] (181.36, 55.01) circle (  2.13);

\path[fill=fillColor,fill opacity=0.20] (178.96, 52.57) circle (  2.13);

\path[fill=fillColor,fill opacity=0.20] (180.05, 54.19) circle (  2.13);

\path[fill=fillColor,fill opacity=0.20] (170.66, 54.19) circle (  2.13);

\path[fill=fillColor,fill opacity=0.20] (177.43, 54.19) circle (  2.13);

\path[fill=fillColor,fill opacity=0.20] (169.56, 57.44) circle (  2.13);

\path[fill=fillColor,fill opacity=0.20] (181.36, 62.32) circle (  2.13);

\path[fill=fillColor,fill opacity=0.20] (181.80, 58.26) circle (  2.13);

\path[fill=fillColor,fill opacity=0.20] (182.24, 49.32) circle (  2.13);

\path[fill=fillColor,fill opacity=0.20] (190.76, 48.50) circle (  2.13);

\path[fill=fillColor,fill opacity=0.20] (210.64, 53.38) circle (  2.13);

\path[fill=fillColor,fill opacity=0.20] (192.07, 55.01) circle (  2.13);

\path[fill=fillColor,fill opacity=0.20] (186.83, 55.01) circle (  2.13);

\path[fill=fillColor,fill opacity=0.20] (185.30, 53.38) circle (  2.13);

\path[fill=fillColor,fill opacity=0.20] (187.26, 52.57) circle (  2.13);

\path[fill=fillColor,fill opacity=0.20] (189.67, 50.13) circle (  2.13);

\path[fill=fillColor,fill opacity=0.20] (176.56, 46.07) circle (  2.13);

\path[fill=fillColor,fill opacity=0.20] (178.74, 49.32) circle (  2.13);

\path[fill=fillColor,fill opacity=0.20] (182.89, 55.01) circle (  2.13);
\end{scope}
\begin{scope}
\path[clip] (  0.00,  0.00) rectangle (289.08,144.54);
\definecolor[named]{drawColor}{rgb}{0.50,0.50,0.50}

\node[text=drawColor,anchor=base,inner sep=0pt, outer sep=0pt, scale=  0.96] at ( 60.53, 20.31) {0.02};

\node[text=drawColor,anchor=base,inner sep=0pt, outer sep=0pt, scale=  0.96] at ( 82.38, 20.31) {0.03};

\node[text=drawColor,anchor=base,inner sep=0pt, outer sep=0pt, scale=  0.96] at (104.23, 20.31) {0.04};

\node[text=drawColor,anchor=base,inner sep=0pt, outer sep=0pt, scale=  0.96] at (126.08, 20.31) {0.05};

\node[text=drawColor,anchor=base,inner sep=0pt, outer sep=0pt, scale=  0.96] at (147.93, 20.31) {0.06};
\end{scope}
\begin{scope}
\path[clip] (  0.00,  0.00) rectangle (289.08,144.54);
\definecolor[named]{drawColor}{rgb}{0.50,0.50,0.50}

\path[draw=drawColor,line width= 0.6pt,line join=round,line cap=round] ( 60.53, 29.77) -- ( 60.53, 34.04);

\path[draw=drawColor,line width= 0.6pt,line join=round,line cap=round] ( 82.38, 29.77) -- ( 82.38, 34.04);

\path[draw=drawColor,line width= 0.6pt,line join=round,line cap=round] (104.23, 29.77) -- (104.23, 34.04);

\path[draw=drawColor,line width= 0.6pt,line join=round,line cap=round] (126.08, 29.77) -- (126.08, 34.04);

\path[draw=drawColor,line width= 0.6pt,line join=round,line cap=round] (147.93, 29.77) -- (147.93, 34.04);
\end{scope}
\begin{scope}
\path[clip] (  0.00,  0.00) rectangle (289.08,144.54);
\definecolor[named]{drawColor}{rgb}{0.50,0.50,0.50}

\node[text=drawColor,anchor=base,inner sep=0pt, outer sep=0pt, scale=  0.96] at (180.71, 20.31) {0.02};

\node[text=drawColor,anchor=base,inner sep=0pt, outer sep=0pt, scale=  0.96] at (202.56, 20.31) {0.03};

\node[text=drawColor,anchor=base,inner sep=0pt, outer sep=0pt, scale=  0.96] at (224.41, 20.31) {0.04};

\node[text=drawColor,anchor=base,inner sep=0pt, outer sep=0pt, scale=  0.96] at (246.26, 20.31) {0.05};

\node[text=drawColor,anchor=base,inner sep=0pt, outer sep=0pt, scale=  0.96] at (268.11, 20.31) {0.06};
\end{scope}
\begin{scope}
\path[clip] (  0.00,  0.00) rectangle (289.08,144.54);
\definecolor[named]{drawColor}{rgb}{0.50,0.50,0.50}

\path[draw=drawColor,line width= 0.6pt,line join=round,line cap=round] (180.71, 29.77) -- (180.71, 34.04);

\path[draw=drawColor,line width= 0.6pt,line join=round,line cap=round] (202.56, 29.77) -- (202.56, 34.04);

\path[draw=drawColor,line width= 0.6pt,line join=round,line cap=round] (224.41, 29.77) -- (224.41, 34.04);

\path[draw=drawColor,line width= 0.6pt,line join=round,line cap=round] (246.26, 29.77) -- (246.26, 34.04);

\path[draw=drawColor,line width= 0.6pt,line join=round,line cap=round] (268.11, 29.77) -- (268.11, 34.04);
\end{scope}
\begin{scope}
\path[clip] (  0.00,  0.00) rectangle (289.08,144.54);
\definecolor[named]{drawColor}{rgb}{0.00,0.00,0.00}

\node[text=drawColor,anchor=base,inner sep=0pt, outer sep=0pt, scale=  1.20] at (158.36,  9.03) {$\rho$ $[\mu m^{-2}]$};
\end{scope}
\begin{scope}
\path[clip] (  0.00,  0.00) rectangle (289.08,144.54);
\definecolor[named]{drawColor}{rgb}{0.00,0.00,0.00}

\node[text=drawColor,rotate= 90.00,anchor=base,inner sep=0pt, outer sep=0pt, scale=  1.20] at ( 17.30, 76.95) {AD $[\times 10^{-9}mm^2/s]$};
\end{scope}
\end{tikzpicture}

						\end{adjustbox}
						\end{minipage}
					}
				\subfloat[RD]{
					\begin{minipage}{0.5\textwidth}
					\begin{adjustbox}{width={\textwidth},totalheight=\textheight,keepaspectratio}
						\strut
						% Created by tikzDevice version - on 2012-09-27 22:36:56
% !TEX encoding = UTF-8 Unicode
\begin{tikzpicture}[x=1pt,y=1pt]
\definecolor[named]{fillColor}{rgb}{1.00,1.00,1.00}
\path[use as bounding box,fill=fillColor,fill opacity=0.00] (0,0) rectangle (289.08,144.54);
\begin{scope}
\path[clip] (  0.00,  0.00) rectangle (289.08,144.54);
\definecolor[named]{fillColor}{rgb}{1.00,1.00,1.00}

\path[fill=fillColor] (  0.00,  0.00) rectangle (289.08,144.54);
\end{scope}
\begin{scope}
\path[clip] ( 39.69,119.86) rectangle (156.86,132.50);
\definecolor[named]{fillColor}{rgb}{0.80,0.80,0.80}

\path[fill=fillColor] ( 39.69,119.86) rectangle (156.86,132.50);
\definecolor[named]{drawColor}{rgb}{0.00,0.00,0.00}

\node[text=drawColor,anchor=base,inner sep=0pt, outer sep=0pt, scale=  0.96] at ( 98.27,122.87) {Scan (r=0.511)};
\end{scope}
\begin{scope}
\path[clip] (159.87,119.86) rectangle (277.04,132.50);
\definecolor[named]{fillColor}{rgb}{0.80,0.80,0.80}

\path[fill=fillColor] (159.87,119.86) rectangle (277.03,132.50);
\definecolor[named]{drawColor}{rgb}{0.00,0.00,0.00}

\node[text=drawColor,anchor=base,inner sep=0pt, outer sep=0pt, scale=  0.96] at (218.45,122.87) {Rescan (r=0.431)};
\end{scope}
\begin{scope}
\path[clip] (  0.00,  0.00) rectangle (289.08,144.54);
\definecolor[named]{drawColor}{rgb}{0.50,0.50,0.50}

\node[text=drawColor,anchor=base east,inner sep=0pt, outer sep=0pt, scale=  0.96] at ( 32.58, 39.45) {0.2};

\node[text=drawColor,anchor=base east,inner sep=0pt, outer sep=0pt, scale=  0.96] at ( 32.58, 56.68) {0.4};

\node[text=drawColor,anchor=base east,inner sep=0pt, outer sep=0pt, scale=  0.96] at ( 32.58, 73.90) {0.6};

\node[text=drawColor,anchor=base east,inner sep=0pt, outer sep=0pt, scale=  0.96] at ( 32.58, 91.12) {0.8};

\node[text=drawColor,anchor=base east,inner sep=0pt, outer sep=0pt, scale=  0.96] at ( 32.58,108.35) {1.0};
\end{scope}
\begin{scope}
\path[clip] (  0.00,  0.00) rectangle (289.08,144.54);
\definecolor[named]{drawColor}{rgb}{0.50,0.50,0.50}

\path[draw=drawColor,line width= 0.6pt,line join=round,line cap=round] ( 35.42, 42.76) -- ( 39.69, 42.76);

\path[draw=drawColor,line width= 0.6pt,line join=round,line cap=round] ( 35.42, 59.98) -- ( 39.69, 59.98);

\path[draw=drawColor,line width= 0.6pt,line join=round,line cap=round] ( 35.42, 77.21) -- ( 39.69, 77.21);

\path[draw=drawColor,line width= 0.6pt,line join=round,line cap=round] ( 35.42, 94.43) -- ( 39.69, 94.43);

\path[draw=drawColor,line width= 0.6pt,line join=round,line cap=round] ( 35.42,111.65) -- ( 39.69,111.65);
\end{scope}
\begin{scope}
\path[clip] ( 39.69, 34.04) rectangle (156.86,119.86);
\definecolor[named]{fillColor}{rgb}{0.90,0.90,0.90}

\path[fill=fillColor] ( 39.69, 34.04) rectangle (156.86,119.86);
\definecolor[named]{drawColor}{rgb}{0.95,0.95,0.95}

\path[draw=drawColor,line width= 0.3pt,line join=round,line cap=round] ( 39.69, 34.15) --
	(156.86, 34.15);

\path[draw=drawColor,line width= 0.3pt,line join=round,line cap=round] ( 39.69, 51.37) --
	(156.86, 51.37);

\path[draw=drawColor,line width= 0.3pt,line join=round,line cap=round] ( 39.69, 68.60) --
	(156.86, 68.60);

\path[draw=drawColor,line width= 0.3pt,line join=round,line cap=round] ( 39.69, 85.82) --
	(156.86, 85.82);

\path[draw=drawColor,line width= 0.3pt,line join=round,line cap=round] ( 39.69,103.04) --
	(156.86,103.04);

\path[draw=drawColor,line width= 0.3pt,line join=round,line cap=round] ( 58.23, 34.04) --
	( 58.23,119.86);

\path[draw=drawColor,line width= 0.3pt,line join=round,line cap=round] ( 78.30, 34.04) --
	( 78.30,119.86);

\path[draw=drawColor,line width= 0.3pt,line join=round,line cap=round] ( 98.36, 34.04) --
	( 98.36,119.86);

\path[draw=drawColor,line width= 0.3pt,line join=round,line cap=round] (118.43, 34.04) --
	(118.43,119.86);

\path[draw=drawColor,line width= 0.3pt,line join=round,line cap=round] (138.49, 34.04) --
	(138.49,119.86);
\definecolor[named]{drawColor}{rgb}{1.00,1.00,1.00}

\path[draw=drawColor,line width= 0.6pt,line join=round,line cap=round] ( 39.69, 42.76) --
	(156.86, 42.76);

\path[draw=drawColor,line width= 0.6pt,line join=round,line cap=round] ( 39.69, 59.98) --
	(156.86, 59.98);

\path[draw=drawColor,line width= 0.6pt,line join=round,line cap=round] ( 39.69, 77.21) --
	(156.86, 77.21);

\path[draw=drawColor,line width= 0.6pt,line join=round,line cap=round] ( 39.69, 94.43) --
	(156.86, 94.43);

\path[draw=drawColor,line width= 0.6pt,line join=round,line cap=round] ( 39.69,111.65) --
	(156.86,111.65);

\path[draw=drawColor,line width= 0.6pt,line join=round,line cap=round] ( 48.20, 34.04) --
	( 48.20,119.86);

\path[draw=drawColor,line width= 0.6pt,line join=round,line cap=round] ( 68.26, 34.04) --
	( 68.26,119.86);

\path[draw=drawColor,line width= 0.6pt,line join=round,line cap=round] ( 88.33, 34.04) --
	( 88.33,119.86);

\path[draw=drawColor,line width= 0.6pt,line join=round,line cap=round] (108.39, 34.04) --
	(108.39,119.86);

\path[draw=drawColor,line width= 0.6pt,line join=round,line cap=round] (128.46, 34.04) --
	(128.46,119.86);

\path[draw=drawColor,line width= 0.6pt,line join=round,line cap=round] (148.52, 34.04) --
	(148.52,119.86);
\definecolor[named]{fillColor}{rgb}{0.00,0.00,0.00}

\path[fill=fillColor,fill opacity=0.20] (141.50, 87.63) circle (  2.13);

\path[fill=fillColor,fill opacity=0.20] ( 99.36, 94.86) circle (  2.13);

\path[fill=fillColor,fill opacity=0.20] ( 94.35, 81.34) circle (  2.13);

\path[fill=fillColor,fill opacity=0.20] ( 89.33, 67.65) circle (  2.13);

\path[fill=fillColor,fill opacity=0.20] ( 91.34, 64.72) circle (  2.13);

\path[fill=fillColor,fill opacity=0.20] (102.37, 65.84) circle (  2.13);

\path[fill=fillColor,fill opacity=0.20] (109.40, 81.51) circle (  2.13);

\path[fill=fillColor,fill opacity=0.20] (107.39,107.09) circle (  2.13);

\path[fill=fillColor,fill opacity=0.20] ( 86.32, 94.69) circle (  2.13);

\path[fill=fillColor,fill opacity=0.20] ( 88.33, 68.60) circle (  2.13);

\path[fill=fillColor,fill opacity=0.20] ( 83.31, 57.83) circle (  2.13);

\path[fill=fillColor,fill opacity=0.20] ( 77.29, 59.90) circle (  2.13);

\path[fill=fillColor,fill opacity=0.20] ( 83.31, 60.33) circle (  2.13);

\path[fill=fillColor,fill opacity=0.20] ( 74.28, 57.23) circle (  2.13);

\path[fill=fillColor,fill opacity=0.20] ( 82.31, 55.85) circle (  2.13);

\path[fill=fillColor,fill opacity=0.20] ( 87.33, 69.28) circle (  2.13);

\path[fill=fillColor,fill opacity=0.20] (105.38, 76.09) circle (  2.13);

\path[fill=fillColor,fill opacity=0.20] (127.45, 77.81) circle (  2.13);

\path[fill=fillColor,fill opacity=0.20] ( 81.31,113.37) circle (  2.13);

\path[fill=fillColor,fill opacity=0.20] ( 82.31, 83.58) circle (  2.13);

\path[fill=fillColor,fill opacity=0.20] ( 83.31, 66.18) circle (  2.13);

\path[fill=fillColor,fill opacity=0.20] ( 82.31, 60.93) circle (  2.13);

\path[fill=fillColor,fill opacity=0.20] ( 83.31, 55.51) circle (  2.13);

\path[fill=fillColor,fill opacity=0.20] ( 80.30, 53.18) circle (  2.13);

\path[fill=fillColor,fill opacity=0.20] ( 81.31, 58.86) circle (  2.13);

\path[fill=fillColor,fill opacity=0.20] ( 83.31, 64.12) circle (  2.13);

\path[fill=fillColor,fill opacity=0.20] ( 84.32, 59.81) circle (  2.13);

\path[fill=fillColor,fill opacity=0.20] ( 86.32, 53.09) circle (  2.13);

\path[fill=fillColor,fill opacity=0.20] (100.37, 61.88) circle (  2.13);

\path[fill=fillColor,fill opacity=0.20] ( 98.36, 74.28) circle (  2.13);

\path[fill=fillColor,fill opacity=0.20] (105.38, 81.00) circle (  2.13);

\path[fill=fillColor,fill opacity=0.20] ( 99.36,101.66) circle (  2.13);

\path[fill=fillColor,fill opacity=0.20] ( 81.31, 86.68) circle (  2.13);

\path[fill=fillColor,fill opacity=0.20] ( 81.31, 68.08) circle (  2.13);

\path[fill=fillColor,fill opacity=0.20] ( 73.28, 53.78) circle (  2.13);

\path[fill=fillColor,fill opacity=0.20] ( 75.29, 62.83) circle (  2.13);

\path[fill=fillColor,fill opacity=0.20] ( 80.30, 60.07) circle (  2.13);

\path[fill=fillColor,fill opacity=0.20] ( 86.32, 49.22) circle (  2.13);

\path[fill=fillColor,fill opacity=0.20] ( 83.31, 55.16) circle (  2.13);

\path[fill=fillColor,fill opacity=0.20] ( 81.31, 62.65) circle (  2.13);

\path[fill=fillColor,fill opacity=0.20] ( 86.32, 60.41) circle (  2.13);

\path[fill=fillColor,fill opacity=0.20] ( 93.34, 59.55) circle (  2.13);

\path[fill=fillColor,fill opacity=0.20] ( 97.36, 67.82) circle (  2.13);

\path[fill=fillColor,fill opacity=0.20] ( 94.35, 73.68) circle (  2.13);

\path[fill=fillColor,fill opacity=0.20] ( 99.36, 80.13) circle (  2.13);

\path[fill=fillColor,fill opacity=0.20] ( 86.32, 37.94) circle (  2.13);

\path[fill=fillColor,fill opacity=0.20] (101.37, 93.48) circle (  2.13);

\path[fill=fillColor,fill opacity=0.20] ( 94.35, 70.58) circle (  2.13);

\path[fill=fillColor,fill opacity=0.20] ( 81.31, 61.45) circle (  2.13);

\path[fill=fillColor,fill opacity=0.20] ( 76.29, 49.74) circle (  2.13);

\path[fill=fillColor,fill opacity=0.20] ( 77.29, 50.17) circle (  2.13);

\path[fill=fillColor,fill opacity=0.20] ( 77.29, 55.08) circle (  2.13);

\path[fill=fillColor,fill opacity=0.20] ( 81.31, 58.35) circle (  2.13);

\path[fill=fillColor,fill opacity=0.20] ( 85.32, 58.78) circle (  2.13);

\path[fill=fillColor,fill opacity=0.20] ( 85.32, 55.25) circle (  2.13);

\path[fill=fillColor,fill opacity=0.20] ( 92.34, 56.37) circle (  2.13);

\path[fill=fillColor,fill opacity=0.20] (100.37, 64.98) circle (  2.13);

\path[fill=fillColor,fill opacity=0.20] ( 95.35, 75.66) circle (  2.13);

\path[fill=fillColor,fill opacity=0.20] ( 98.36, 88.40) circle (  2.13);

\path[fill=fillColor,fill opacity=0.20] (102.37,114.24) circle (  2.13);

\path[fill=fillColor,fill opacity=0.20] ( 96.35, 75.83) circle (  2.13);

\path[fill=fillColor,fill opacity=0.20] ( 85.32, 48.62) circle (  2.13);

\path[fill=fillColor,fill opacity=0.20] ( 84.32, 73.16) circle (  2.13);

\path[fill=fillColor,fill opacity=0.20] ( 92.34, 72.82) circle (  2.13);

\path[fill=fillColor,fill opacity=0.20] ( 92.34, 62.65) circle (  2.13);

\path[fill=fillColor,fill opacity=0.20] (102.37, 96.15) circle (  2.13);

\path[fill=fillColor,fill opacity=0.20] ( 96.35, 61.96) circle (  2.13);

\path[fill=fillColor,fill opacity=0.20] ( 80.30, 52.84) circle (  2.13);

\path[fill=fillColor,fill opacity=0.20] ( 78.30, 46.89) circle (  2.13);

\path[fill=fillColor,fill opacity=0.20] ( 79.30, 40.01) circle (  2.13);

\path[fill=fillColor,fill opacity=0.20] ( 80.30, 59.04) circle (  2.13);

\path[fill=fillColor,fill opacity=0.20] ( 78.30, 66.53) circle (  2.13);

\path[fill=fillColor,fill opacity=0.20] ( 93.34, 59.64) circle (  2.13);

\path[fill=fillColor,fill opacity=0.20] (103.38, 58.18) circle (  2.13);

\path[fill=fillColor,fill opacity=0.20] (114.41, 64.63) circle (  2.13);

\path[fill=fillColor,fill opacity=0.20] ( 99.36, 85.47) circle (  2.13);

\path[fill=fillColor,fill opacity=0.20] ( 77.29, 53.87) circle (  2.13);

\path[fill=fillColor,fill opacity=0.20] ( 79.30, 61.19) circle (  2.13);

\path[fill=fillColor,fill opacity=0.20] ( 82.31, 56.45) circle (  2.13);

\path[fill=fillColor,fill opacity=0.20] ( 80.30, 56.11) circle (  2.13);

\path[fill=fillColor,fill opacity=0.20] ( 82.31, 57.40) circle (  2.13);

\path[fill=fillColor,fill opacity=0.20] ( 86.32, 51.89) circle (  2.13);

\path[fill=fillColor,fill opacity=0.20] ( 97.36, 60.76) circle (  2.13);

\path[fill=fillColor,fill opacity=0.20] (101.37,104.07) circle (  2.13);

\path[fill=fillColor,fill opacity=0.20] ( 95.35, 89.52) circle (  2.13);

\path[fill=fillColor,fill opacity=0.20] (100.37, 58.86) circle (  2.13);

\path[fill=fillColor,fill opacity=0.20] ( 91.34, 44.48) circle (  2.13);

\path[fill=fillColor,fill opacity=0.20] ( 88.33, 49.56) circle (  2.13);

\path[fill=fillColor,fill opacity=0.20] ( 87.33, 52.66) circle (  2.13);

\path[fill=fillColor,fill opacity=0.20] ( 88.33, 43.45) circle (  2.13);

\path[fill=fillColor,fill opacity=0.20] ( 86.32, 50.08) circle (  2.13);

\path[fill=fillColor,fill opacity=0.20] ( 82.31, 70.92) circle (  2.13);

\path[fill=fillColor,fill opacity=0.20] (106.39, 65.93) circle (  2.13);

\path[fill=fillColor,fill opacity=0.20] ( 87.33, 77.29) circle (  2.13);

\path[fill=fillColor,fill opacity=0.20] ( 72.28, 51.03) circle (  2.13);

\path[fill=fillColor,fill opacity=0.20] ( 69.27, 62.22) circle (  2.13);

\path[fill=fillColor,fill opacity=0.20] ( 70.27, 55.25) circle (  2.13);

\path[fill=fillColor,fill opacity=0.20] ( 67.36, 43.02) circle (  2.13);

\path[fill=fillColor,fill opacity=0.20] ( 60.04, 40.78) circle (  2.13);

\path[fill=fillColor,fill opacity=0.20] ( 74.28, 46.29) circle (  2.13);

\path[fill=fillColor,fill opacity=0.20] ( 79.30, 42.16) circle (  2.13);

\path[fill=fillColor,fill opacity=0.20] ( 82.31, 47.76) circle (  2.13);

\path[fill=fillColor,fill opacity=0.20] (109.40, 84.27) circle (  2.13);

\path[fill=fillColor,fill opacity=0.20] (106.39, 83.67) circle (  2.13);

\path[fill=fillColor,fill opacity=0.20] (100.37, 59.64) circle (  2.13);

\path[fill=fillColor,fill opacity=0.20] ( 89.33, 54.30) circle (  2.13);

\path[fill=fillColor,fill opacity=0.20] ( 93.34, 53.35) circle (  2.13);

\path[fill=fillColor,fill opacity=0.20] ( 92.34, 61.10) circle (  2.13);

\path[fill=fillColor,fill opacity=0.20] ( 90.33, 60.16) circle (  2.13);

\path[fill=fillColor,fill opacity=0.20] ( 96.35, 56.02) circle (  2.13);

\path[fill=fillColor,fill opacity=0.20] ( 92.34, 68.60) circle (  2.13);

\path[fill=fillColor,fill opacity=0.20] ( 93.34, 76.43) circle (  2.13);

\path[fill=fillColor,fill opacity=0.20] (104.38, 72.90) circle (  2.13);

\path[fill=fillColor,fill opacity=0.20] ( 85.32,107.95) circle (  2.13);

\path[fill=fillColor,fill opacity=0.20] ( 75.29, 48.96) circle (  2.13);

\path[fill=fillColor,fill opacity=0.20] ( 69.27, 59.81) circle (  2.13);

\path[fill=fillColor,fill opacity=0.20] ( 66.06, 58.61) circle (  2.13);

\path[fill=fillColor,fill opacity=0.20] ( 64.55, 39.75) circle (  2.13);

\path[fill=fillColor,fill opacity=0.20] ( 70.27, 38.71) circle (  2.13);

\path[fill=fillColor,fill opacity=0.20] ( 68.26, 46.12) circle (  2.13);

\path[fill=fillColor,fill opacity=0.20] ( 81.31, 47.33) circle (  2.13);

\path[fill=fillColor,fill opacity=0.20] ( 84.32, 43.62) circle (  2.13);

\path[fill=fillColor,fill opacity=0.20] (100.37, 56.80) circle (  2.13);

\path[fill=fillColor,fill opacity=0.20] (115.42, 87.45) circle (  2.13);

\path[fill=fillColor,fill opacity=0.20] ( 97.36, 68.94) circle (  2.13);

\path[fill=fillColor,fill opacity=0.20] ( 93.34, 69.37) circle (  2.13);

\path[fill=fillColor,fill opacity=0.20] ( 89.33, 51.29) circle (  2.13);

\path[fill=fillColor,fill opacity=0.20] ( 90.33, 46.46) circle (  2.13);

\path[fill=fillColor,fill opacity=0.20] ( 92.34, 61.79) circle (  2.13);

\path[fill=fillColor,fill opacity=0.20] ( 87.33, 68.60) circle (  2.13);

\path[fill=fillColor,fill opacity=0.20] ( 93.34, 64.46) circle (  2.13);

\path[fill=fillColor,fill opacity=0.20] ( 90.33, 62.40) circle (  2.13);

\path[fill=fillColor,fill opacity=0.20] (102.37, 69.97) circle (  2.13);

\path[fill=fillColor,fill opacity=0.20] (133.47,110.10) circle (  2.13);

\path[fill=fillColor,fill opacity=0.20] ( 77.29, 71.70) circle (  2.13);

\path[fill=fillColor,fill opacity=0.20] ( 69.27, 52.15) circle (  2.13);

\path[fill=fillColor,fill opacity=0.20] ( 72.28, 51.03) circle (  2.13);

\path[fill=fillColor,fill opacity=0.20] ( 81.31, 42.85) circle (  2.13);

\path[fill=fillColor,fill opacity=0.20] ( 78.30, 50.94) circle (  2.13);

\path[fill=fillColor,fill opacity=0.20] ( 77.29, 53.96) circle (  2.13);

\path[fill=fillColor,fill opacity=0.20] ( 86.32, 50.68) circle (  2.13);

\path[fill=fillColor,fill opacity=0.20] ( 92.34, 51.63) circle (  2.13);

\path[fill=fillColor,fill opacity=0.20] ( 90.33, 39.32) circle (  2.13);

\path[fill=fillColor,fill opacity=0.20] (109.40, 98.39) circle (  2.13);

\path[fill=fillColor,fill opacity=0.20] (101.37, 77.90) circle (  2.13);

\path[fill=fillColor,fill opacity=0.20] ( 92.34, 70.92) circle (  2.13);

\path[fill=fillColor,fill opacity=0.20] ( 84.32, 48.27) circle (  2.13);

\path[fill=fillColor,fill opacity=0.20] ( 91.34, 51.46) circle (  2.13);

\path[fill=fillColor,fill opacity=0.20] ( 93.34, 64.38) circle (  2.13);

\path[fill=fillColor,fill opacity=0.20] ( 95.35, 64.12) circle (  2.13);

\path[fill=fillColor,fill opacity=0.20] ( 96.35, 58.52) circle (  2.13);

\path[fill=fillColor,fill opacity=0.20] ( 97.36, 59.04) circle (  2.13);

\path[fill=fillColor,fill opacity=0.20] ( 66.66, 48.79) circle (  2.13);

\path[fill=fillColor,fill opacity=0.20] ( 72.28, 44.14) circle (  2.13);

\path[fill=fillColor,fill opacity=0.20] ( 76.29, 53.09) circle (  2.13);

\path[fill=fillColor,fill opacity=0.20] ( 67.96, 46.72) circle (  2.13);

\path[fill=fillColor,fill opacity=0.20] ( 79.30, 48.79) circle (  2.13);

\path[fill=fillColor,fill opacity=0.20] ( 81.31, 54.64) circle (  2.13);

\path[fill=fillColor,fill opacity=0.20] ( 78.30, 53.78) circle (  2.13);

\path[fill=fillColor,fill opacity=0.20] ( 82.31, 55.16) circle (  2.13);

\path[fill=fillColor,fill opacity=0.20] ( 88.33, 57.14) circle (  2.13);

\path[fill=fillColor,fill opacity=0.20] ( 91.34, 45.09) circle (  2.13);

\path[fill=fillColor,fill opacity=0.20] (100.37, 44.83) circle (  2.13);

\path[fill=fillColor,fill opacity=0.20] (103.38, 82.72) circle (  2.13);

\path[fill=fillColor,fill opacity=0.20] ( 85.32, 66.27) circle (  2.13);

\path[fill=fillColor,fill opacity=0.20] ( 79.30, 54.99) circle (  2.13);

\path[fill=fillColor,fill opacity=0.20] ( 81.31, 49.39) circle (  2.13);

\path[fill=fillColor,fill opacity=0.20] ( 91.34, 48.53) circle (  2.13);

\path[fill=fillColor,fill opacity=0.20] ( 94.35, 49.99) circle (  2.13);

\path[fill=fillColor,fill opacity=0.20] ( 94.35, 57.40) circle (  2.13);

\path[fill=fillColor,fill opacity=0.20] ( 98.36, 69.11) circle (  2.13);

\path[fill=fillColor,fill opacity=0.20] ( 95.35, 62.14) circle (  2.13);

\path[fill=fillColor,fill opacity=0.20] ( 87.33, 95.29) circle (  2.13);

\path[fill=fillColor,fill opacity=0.20] ( 69.27, 42.33) circle (  2.13);

\path[fill=fillColor,fill opacity=0.20] ( 72.28, 63.00) circle (  2.13);

\path[fill=fillColor,fill opacity=0.20] ( 74.28, 61.53) circle (  2.13);

\path[fill=fillColor,fill opacity=0.20] ( 69.27, 47.33) circle (  2.13);

\path[fill=fillColor,fill opacity=0.20] ( 77.29, 48.96) circle (  2.13);

\path[fill=fillColor,fill opacity=0.20] ( 83.31, 54.04) circle (  2.13);

\path[fill=fillColor,fill opacity=0.20] ( 76.29, 55.08) circle (  2.13);

\path[fill=fillColor,fill opacity=0.20] ( 91.34, 60.41) circle (  2.13);

\path[fill=fillColor,fill opacity=0.20] ( 88.33, 56.80) circle (  2.13);

\path[fill=fillColor,fill opacity=0.20] ( 93.34, 45.34) circle (  2.13);

\path[fill=fillColor,fill opacity=0.20] (106.39, 61.10) circle (  2.13);

\path[fill=fillColor,fill opacity=0.20] ( 95.35, 63.86) circle (  2.13);

\path[fill=fillColor,fill opacity=0.20] ( 86.32, 63.69) circle (  2.13);

\path[fill=fillColor,fill opacity=0.20] ( 84.32, 62.91) circle (  2.13);

\path[fill=fillColor,fill opacity=0.20] ( 88.33, 55.33) circle (  2.13);

\path[fill=fillColor,fill opacity=0.20] ( 91.34, 45.26) circle (  2.13);

\path[fill=fillColor,fill opacity=0.20] ( 89.33, 49.39) circle (  2.13);

\path[fill=fillColor,fill opacity=0.20] ( 93.34, 71.87) circle (  2.13);

\path[fill=fillColor,fill opacity=0.20] ( 82.31, 73.85) circle (  2.13);

\path[fill=fillColor,fill opacity=0.20] ( 93.34, 64.98) circle (  2.13);

\path[fill=fillColor,fill opacity=0.20] ( 80.30, 56.54) circle (  2.13);

\path[fill=fillColor,fill opacity=0.20] ( 79.30, 55.85) circle (  2.13);

\path[fill=fillColor,fill opacity=0.20] ( 80.30, 68.16) circle (  2.13);

\path[fill=fillColor,fill opacity=0.20] ( 81.31, 60.33) circle (  2.13);

\path[fill=fillColor,fill opacity=0.20] ( 77.29, 50.51) circle (  2.13);

\path[fill=fillColor,fill opacity=0.20] ( 79.30, 54.13) circle (  2.13);

\path[fill=fillColor,fill opacity=0.20] ( 80.30, 54.21) circle (  2.13);

\path[fill=fillColor,fill opacity=0.20] ( 79.30, 49.74) circle (  2.13);

\path[fill=fillColor,fill opacity=0.20] ( 82.31, 54.21) circle (  2.13);

\path[fill=fillColor,fill opacity=0.20] ( 93.34, 53.35) circle (  2.13);

\path[fill=fillColor,fill opacity=0.20] ( 94.35, 49.05) circle (  2.13);

\path[fill=fillColor,fill opacity=0.20] (110.40, 75.40) circle (  2.13);

\path[fill=fillColor,fill opacity=0.20] (100.37, 75.48) circle (  2.13);

\path[fill=fillColor,fill opacity=0.20] ( 95.35, 70.66) circle (  2.13);

\path[fill=fillColor,fill opacity=0.20] ( 95.35, 65.24) circle (  2.13);

\path[fill=fillColor,fill opacity=0.20] ( 87.33, 57.06) circle (  2.13);

\path[fill=fillColor,fill opacity=0.20] ( 91.34, 53.53) circle (  2.13);

\path[fill=fillColor,fill opacity=0.20] ( 90.33, 53.61) circle (  2.13);

\path[fill=fillColor,fill opacity=0.20] ( 87.33, 62.40) circle (  2.13);

\path[fill=fillColor,fill opacity=0.20] ( 96.35, 67.65) circle (  2.13);

\path[fill=fillColor,fill opacity=0.20] (101.37, 60.67) circle (  2.13);

\path[fill=fillColor,fill opacity=0.20] (114.41, 74.02) circle (  2.13);

\path[fill=fillColor,fill opacity=0.20] ( 80.30, 70.75) circle (  2.13);

\path[fill=fillColor,fill opacity=0.20] ( 69.27, 53.61) circle (  2.13);

\path[fill=fillColor,fill opacity=0.20] ( 86.32, 63.77) circle (  2.13);

\path[fill=fillColor,fill opacity=0.20] ( 84.32, 62.31) circle (  2.13);

\path[fill=fillColor,fill opacity=0.20] ( 80.30, 53.70) circle (  2.13);

\path[fill=fillColor,fill opacity=0.20] ( 80.30, 50.08) circle (  2.13);

\path[fill=fillColor,fill opacity=0.20] ( 75.29, 54.39) circle (  2.13);

\path[fill=fillColor,fill opacity=0.20] ( 83.31, 51.11) circle (  2.13);

\path[fill=fillColor,fill opacity=0.20] ( 77.29, 43.62) circle (  2.13);

\path[fill=fillColor,fill opacity=0.20] ( 77.29, 50.68) circle (  2.13);

\path[fill=fillColor,fill opacity=0.20] ( 89.33, 54.99) circle (  2.13);

\path[fill=fillColor,fill opacity=0.20] ( 95.35, 58.52) circle (  2.13);

\path[fill=fillColor,fill opacity=0.20] ( 98.36, 84.10) circle (  2.13);

\path[fill=fillColor,fill opacity=0.20] ( 86.32, 65.32) circle (  2.13);

\path[fill=fillColor,fill opacity=0.20] ( 87.33, 53.96) circle (  2.13);

\path[fill=fillColor,fill opacity=0.20] ( 89.33, 57.92) circle (  2.13);

\path[fill=fillColor,fill opacity=0.20] ( 87.33, 57.92) circle (  2.13);

\path[fill=fillColor,fill opacity=0.20] ( 88.33, 50.34) circle (  2.13);

\path[fill=fillColor,fill opacity=0.20] ( 90.33, 46.38) circle (  2.13);

\path[fill=fillColor,fill opacity=0.20] ( 97.36, 48.44) circle (  2.13);

\path[fill=fillColor,fill opacity=0.20] ( 97.36, 56.37) circle (  2.13);

\path[fill=fillColor,fill opacity=0.20] (105.38, 74.80) circle (  2.13);

\path[fill=fillColor,fill opacity=0.20] ( 84.32, 86.08) circle (  2.13);

\path[fill=fillColor,fill opacity=0.20] ( 76.29, 49.48) circle (  2.13);

\path[fill=fillColor,fill opacity=0.20] ( 77.29, 57.23) circle (  2.13);

\path[fill=fillColor,fill opacity=0.20] ( 82.31, 54.21) circle (  2.13);

\path[fill=fillColor,fill opacity=0.20] ( 75.29, 51.72) circle (  2.13);

\path[fill=fillColor,fill opacity=0.20] ( 75.29, 53.18) circle (  2.13);

\path[fill=fillColor,fill opacity=0.20] ( 75.29, 52.58) circle (  2.13);

\path[fill=fillColor,fill opacity=0.20] ( 74.28, 49.31) circle (  2.13);

\path[fill=fillColor,fill opacity=0.20] ( 78.30, 45.52) circle (  2.13);

\path[fill=fillColor,fill opacity=0.20] ( 79.30, 48.19) circle (  2.13);

\path[fill=fillColor,fill opacity=0.20] ( 84.32, 59.38) circle (  2.13);

\path[fill=fillColor,fill opacity=0.20] ( 81.31, 65.41) circle (  2.13);

\path[fill=fillColor,fill opacity=0.20] ( 90.33, 74.45) circle (  2.13);

\path[fill=fillColor,fill opacity=0.20] ( 83.31, 78.84) circle (  2.13);

\path[fill=fillColor,fill opacity=0.20] ( 77.29, 54.64) circle (  2.13);

\path[fill=fillColor,fill opacity=0.20] ( 90.33, 52.92) circle (  2.13);

\path[fill=fillColor,fill opacity=0.20] ( 86.32, 54.56) circle (  2.13);

\path[fill=fillColor,fill opacity=0.20] ( 89.33, 48.27) circle (  2.13);

\path[fill=fillColor,fill opacity=0.20] ( 93.34, 46.38) circle (  2.13);

\path[fill=fillColor,fill opacity=0.20] ( 87.33, 48.96) circle (  2.13);

\path[fill=fillColor,fill opacity=0.20] ( 92.34, 55.76) circle (  2.13);

\path[fill=fillColor,fill opacity=0.20] ( 96.35, 68.51) circle (  2.13);

\path[fill=fillColor,fill opacity=0.20] (111.40, 82.55) circle (  2.13);

\path[fill=fillColor,fill opacity=0.20] ( 82.31, 96.58) circle (  2.13);

\path[fill=fillColor,fill opacity=0.20] ( 78.30, 56.71) circle (  2.13);

\path[fill=fillColor,fill opacity=0.20] ( 76.29, 62.74) circle (  2.13);

\path[fill=fillColor,fill opacity=0.20] ( 80.30, 53.78) circle (  2.13);

\path[fill=fillColor,fill opacity=0.20] ( 76.29, 45.95) circle (  2.13);

\path[fill=fillColor,fill opacity=0.20] ( 75.29, 49.99) circle (  2.13);

\path[fill=fillColor,fill opacity=0.20] ( 75.29, 58.18) circle (  2.13);

\path[fill=fillColor,fill opacity=0.20] ( 75.29, 61.19) circle (  2.13);

\path[fill=fillColor,fill opacity=0.20] ( 72.28, 54.56) circle (  2.13);

\path[fill=fillColor,fill opacity=0.20] ( 79.30, 50.34) circle (  2.13);

\path[fill=fillColor,fill opacity=0.20] ( 78.30, 56.11) circle (  2.13);

\path[fill=fillColor,fill opacity=0.20] ( 80.30, 62.83) circle (  2.13);

\path[fill=fillColor,fill opacity=0.20] ( 83.31, 76.17) circle (  2.13);

\path[fill=fillColor,fill opacity=0.20] (100.37,100.80) circle (  2.13);

\path[fill=fillColor,fill opacity=0.20] (112.41,106.49) circle (  2.13);

\path[fill=fillColor,fill opacity=0.20] ( 94.35, 72.64) circle (  2.13);

\path[fill=fillColor,fill opacity=0.20] ( 88.33, 57.83) circle (  2.13);

\path[fill=fillColor,fill opacity=0.20] ( 90.33, 54.04) circle (  2.13);

\path[fill=fillColor,fill opacity=0.20] ( 93.34, 57.66) circle (  2.13);

\path[fill=fillColor,fill opacity=0.20] ( 86.32, 58.43) circle (  2.13);

\path[fill=fillColor,fill opacity=0.20] ( 90.33, 55.68) circle (  2.13);

\path[fill=fillColor,fill opacity=0.20] ( 89.33, 59.21) circle (  2.13);

\path[fill=fillColor,fill opacity=0.20] ( 95.35, 70.32) circle (  2.13);

\path[fill=fillColor,fill opacity=0.20] ( 96.35, 77.64) circle (  2.13);

\path[fill=fillColor,fill opacity=0.20] (105.38, 84.18) circle (  2.13);

\path[fill=fillColor,fill opacity=0.20] ( 89.33, 99.51) circle (  2.13);

\path[fill=fillColor,fill opacity=0.20] ( 80.30, 54.56) circle (  2.13);

\path[fill=fillColor,fill opacity=0.20] ( 77.29, 66.53) circle (  2.13);

\path[fill=fillColor,fill opacity=0.20] ( 80.30, 62.74) circle (  2.13);

\path[fill=fillColor,fill opacity=0.20] ( 78.30, 53.78) circle (  2.13);

\path[fill=fillColor,fill opacity=0.20] ( 73.28, 56.80) circle (  2.13);

\path[fill=fillColor,fill opacity=0.20] ( 79.30, 58.18) circle (  2.13);

\path[fill=fillColor,fill opacity=0.20] ( 79.30, 57.66) circle (  2.13);

\path[fill=fillColor,fill opacity=0.20] ( 75.29, 59.21) circle (  2.13);

\path[fill=fillColor,fill opacity=0.20] ( 76.29, 61.62) circle (  2.13);

\path[fill=fillColor,fill opacity=0.20] ( 74.28, 66.10) circle (  2.13);

\path[fill=fillColor,fill opacity=0.20] ( 77.29, 62.14) circle (  2.13);

\path[fill=fillColor,fill opacity=0.20] ( 90.33, 58.26) circle (  2.13);

\path[fill=fillColor,fill opacity=0.20] (111.40,106.14) circle (  2.13);

\path[fill=fillColor,fill opacity=0.20] ( 86.32, 81.86) circle (  2.13);

\path[fill=fillColor,fill opacity=0.20] ( 85.32, 61.79) circle (  2.13);

\path[fill=fillColor,fill opacity=0.20] ( 91.34, 61.10) circle (  2.13);

\path[fill=fillColor,fill opacity=0.20] ( 86.32, 55.42) circle (  2.13);

\path[fill=fillColor,fill opacity=0.20] ( 83.31, 47.76) circle (  2.13);

\path[fill=fillColor,fill opacity=0.20] ( 90.33, 54.90) circle (  2.13);

\path[fill=fillColor,fill opacity=0.20] ( 91.34, 60.93) circle (  2.13);

\path[fill=fillColor,fill opacity=0.20] ( 91.34, 63.95) circle (  2.13);

\path[fill=fillColor,fill opacity=0.20] ( 93.34, 66.61) circle (  2.13);

\path[fill=fillColor,fill opacity=0.20] (103.38, 72.13) circle (  2.13);

\path[fill=fillColor,fill opacity=0.20] ( 95.35,106.49) circle (  2.13);

\path[fill=fillColor,fill opacity=0.20] ( 84.32, 59.04) circle (  2.13);

\path[fill=fillColor,fill opacity=0.20] ( 79.30, 67.05) circle (  2.13);

\path[fill=fillColor,fill opacity=0.20] ( 74.28, 62.74) circle (  2.13);

\path[fill=fillColor,fill opacity=0.20] ( 77.29, 50.34) circle (  2.13);

\path[fill=fillColor,fill opacity=0.20] ( 72.28, 49.39) circle (  2.13);

\path[fill=fillColor,fill opacity=0.20] ( 73.28, 55.94) circle (  2.13);

\path[fill=fillColor,fill opacity=0.20] ( 76.29, 59.47) circle (  2.13);

\path[fill=fillColor,fill opacity=0.20] ( 75.29, 54.90) circle (  2.13);

\path[fill=fillColor,fill opacity=0.20] ( 71.27, 53.70) circle (  2.13);

\path[fill=fillColor,fill opacity=0.20] ( 72.28, 62.74) circle (  2.13);

\path[fill=fillColor,fill opacity=0.20] ( 45.69, 69.54) circle (  2.13);

\path[fill=fillColor,fill opacity=0.20] ( 78.30, 67.05) circle (  2.13);

\path[fill=fillColor,fill opacity=0.20] ( 98.36, 65.06) circle (  2.13);

\path[fill=fillColor,fill opacity=0.20] (112.41,111.65) circle (  2.13);

\path[fill=fillColor,fill opacity=0.20] ( 85.32, 77.81) circle (  2.13);

\path[fill=fillColor,fill opacity=0.20] ( 81.31, 59.47) circle (  2.13);

\path[fill=fillColor,fill opacity=0.20] ( 82.31, 48.19) circle (  2.13);

\path[fill=fillColor,fill opacity=0.20] ( 76.29, 43.79) circle (  2.13);

\path[fill=fillColor,fill opacity=0.20] ( 85.32, 51.80) circle (  2.13);

\path[fill=fillColor,fill opacity=0.20] ( 89.33, 55.42) circle (  2.13);

\path[fill=fillColor,fill opacity=0.20] ( 89.33, 60.59) circle (  2.13);

\path[fill=fillColor,fill opacity=0.20] ( 97.36, 66.44) circle (  2.13);

\path[fill=fillColor,fill opacity=0.20] (102.37, 66.96) circle (  2.13);

\path[fill=fillColor,fill opacity=0.20] (107.39, 78.15) circle (  2.13);

\path[fill=fillColor,fill opacity=0.20] ( 93.34,100.03) circle (  2.13);

\path[fill=fillColor,fill opacity=0.20] ( 81.31, 65.84) circle (  2.13);

\path[fill=fillColor,fill opacity=0.20] ( 83.31, 71.27) circle (  2.13);

\path[fill=fillColor,fill opacity=0.20] ( 77.29, 63.77) circle (  2.13);

\path[fill=fillColor,fill opacity=0.20] ( 72.28, 52.15) circle (  2.13);

\path[fill=fillColor,fill opacity=0.20] ( 69.27, 45.69) circle (  2.13);

\path[fill=fillColor,fill opacity=0.20] ( 66.66, 43.54) circle (  2.13);

\path[fill=fillColor,fill opacity=0.20] ( 72.28, 45.43) circle (  2.13);

\path[fill=fillColor,fill opacity=0.20] ( 73.28, 52.06) circle (  2.13);

\path[fill=fillColor,fill opacity=0.20] ( 71.27, 56.97) circle (  2.13);

\path[fill=fillColor,fill opacity=0.20] ( 73.28, 58.69) circle (  2.13);

\path[fill=fillColor,fill opacity=0.20] ( 67.96, 60.24) circle (  2.13);

\path[fill=fillColor,fill opacity=0.20] ( 75.29, 62.22) circle (  2.13);

\path[fill=fillColor,fill opacity=0.20] ( 95.35, 74.28) circle (  2.13);

\path[fill=fillColor,fill opacity=0.20] (110.40,107.69) circle (  2.13);

\path[fill=fillColor,fill opacity=0.20] ( 74.28, 75.31) circle (  2.13);

\path[fill=fillColor,fill opacity=0.20] ( 79.30, 61.53) circle (  2.13);

\path[fill=fillColor,fill opacity=0.20] ( 78.30, 58.52) circle (  2.13);

\path[fill=fillColor,fill opacity=0.20] ( 80.30, 57.14) circle (  2.13);

\path[fill=fillColor,fill opacity=0.20] ( 89.33, 61.28) circle (  2.13);

\path[fill=fillColor,fill opacity=0.20] ( 87.33, 68.08) circle (  2.13);

\path[fill=fillColor,fill opacity=0.20] ( 91.34, 68.42) circle (  2.13);

\path[fill=fillColor,fill opacity=0.20] (102.37, 68.85) circle (  2.13);

\path[fill=fillColor,fill opacity=0.20] ( 99.36, 70.92) circle (  2.13);

\path[fill=fillColor,fill opacity=0.20] (104.38, 69.37) circle (  2.13);

\path[fill=fillColor,fill opacity=0.20] ( 99.36, 78.67) circle (  2.13);

\path[fill=fillColor,fill opacity=0.20] ( 80.30, 54.73) circle (  2.13);

\path[fill=fillColor,fill opacity=0.20] ( 66.26, 73.07) circle (  2.13);

\path[fill=fillColor,fill opacity=0.20] ( 75.29, 72.64) circle (  2.13);

\path[fill=fillColor,fill opacity=0.20] ( 71.27, 54.47) circle (  2.13);

\path[fill=fillColor,fill opacity=0.20] ( 69.27, 53.53) circle (  2.13);

\path[fill=fillColor,fill opacity=0.20] ( 66.66, 56.88) circle (  2.13);

\path[fill=fillColor,fill opacity=0.20] ( 68.26, 51.37) circle (  2.13);

\path[fill=fillColor,fill opacity=0.20] ( 72.28, 45.43) circle (  2.13);

\path[fill=fillColor,fill opacity=0.20] ( 72.28, 48.10) circle (  2.13);

\path[fill=fillColor,fill opacity=0.20] ( 73.28, 56.11) circle (  2.13);

\path[fill=fillColor,fill opacity=0.20] ( 76.29, 56.80) circle (  2.13);

\path[fill=fillColor,fill opacity=0.20] ( 75.29, 57.31) circle (  2.13);

\path[fill=fillColor,fill opacity=0.20] ( 96.35, 69.80) circle (  2.13);

\path[fill=fillColor,fill opacity=0.20] ( 75.29, 86.42) circle (  2.13);

\path[fill=fillColor,fill opacity=0.20] ( 76.29, 77.21) circle (  2.13);

\path[fill=fillColor,fill opacity=0.20] ( 86.32, 62.05) circle (  2.13);

\path[fill=fillColor,fill opacity=0.20] ( 89.33, 58.61) circle (  2.13);

\path[fill=fillColor,fill opacity=0.20] ( 85.32, 68.60) circle (  2.13);

\path[fill=fillColor,fill opacity=0.20] ( 89.33, 67.13) circle (  2.13);

\path[fill=fillColor,fill opacity=0.20] ( 96.35, 63.34) circle (  2.13);

\path[fill=fillColor,fill opacity=0.20] ( 99.36, 64.55) circle (  2.13);

\path[fill=fillColor,fill opacity=0.20] (105.38, 60.41) circle (  2.13);

\path[fill=fillColor,fill opacity=0.20] ( 96.35, 59.12) circle (  2.13);

\path[fill=fillColor,fill opacity=0.20] (105.38, 66.10) circle (  2.13);

\path[fill=fillColor,fill opacity=0.20] (112.41, 86.42) circle (  2.13);

\path[fill=fillColor,fill opacity=0.20] ( 93.34, 56.97) circle (  2.13);

\path[fill=fillColor,fill opacity=0.20] ( 85.32, 55.85) circle (  2.13);

\path[fill=fillColor,fill opacity=0.20] ( 72.28, 62.22) circle (  2.13);

\path[fill=fillColor,fill opacity=0.20] ( 77.29, 63.17) circle (  2.13);

\path[fill=fillColor,fill opacity=0.20] ( 74.28, 62.22) circle (  2.13);

\path[fill=fillColor,fill opacity=0.20] ( 73.28, 59.81) circle (  2.13);

\path[fill=fillColor,fill opacity=0.20] ( 70.27, 62.31) circle (  2.13);

\path[fill=fillColor,fill opacity=0.20] ( 67.66, 67.65) circle (  2.13);

\path[fill=fillColor,fill opacity=0.20] ( 67.66, 62.74) circle (  2.13);

\path[fill=fillColor,fill opacity=0.20] ( 69.27, 52.06) circle (  2.13);

\path[fill=fillColor,fill opacity=0.20] ( 73.28, 50.25) circle (  2.13);

\path[fill=fillColor,fill opacity=0.20] ( 76.29, 52.84) circle (  2.13);

\path[fill=fillColor,fill opacity=0.20] ( 90.33, 52.58) circle (  2.13);

\path[fill=fillColor,fill opacity=0.20] ( 95.35, 66.70) circle (  2.13);

\path[fill=fillColor,fill opacity=0.20] ( 93.34, 95.12) circle (  2.13);

\path[fill=fillColor,fill opacity=0.20] ( 88.33, 72.13) circle (  2.13);

\path[fill=fillColor,fill opacity=0.20] ( 86.32, 54.04) circle (  2.13);

\path[fill=fillColor,fill opacity=0.20] ( 83.31, 57.57) circle (  2.13);

\path[fill=fillColor,fill opacity=0.20] ( 86.32, 64.29) circle (  2.13);

\path[fill=fillColor,fill opacity=0.20] ( 93.34, 61.79) circle (  2.13);

\path[fill=fillColor,fill opacity=0.20] (101.37, 63.00) circle (  2.13);

\path[fill=fillColor,fill opacity=0.20] (105.38, 71.87) circle (  2.13);

\path[fill=fillColor,fill opacity=0.20] (104.38, 74.11) circle (  2.13);

\path[fill=fillColor,fill opacity=0.20] (103.38, 63.69) circle (  2.13);

\path[fill=fillColor,fill opacity=0.20] (107.39, 55.08) circle (  2.13);

\path[fill=fillColor,fill opacity=0.20] (108.39, 59.55) circle (  2.13);

\path[fill=fillColor,fill opacity=0.20] (115.42, 76.52) circle (  2.13);

\path[fill=fillColor,fill opacity=0.20] (106.39, 74.37) circle (  2.13);

\path[fill=fillColor,fill opacity=0.20] ( 97.36, 69.97) circle (  2.13);

\path[fill=fillColor,fill opacity=0.20] ( 86.32, 66.70) circle (  2.13);

\path[fill=fillColor,fill opacity=0.20] ( 81.31, 65.84) circle (  2.13);

\path[fill=fillColor,fill opacity=0.20] ( 57.13, 67.48) circle (  2.13);

\path[fill=fillColor,fill opacity=0.20] ( 76.29, 65.75) circle (  2.13);

\path[fill=fillColor,fill opacity=0.20] ( 82.31, 55.59) circle (  2.13);

\path[fill=fillColor,fill opacity=0.20] ( 73.28, 54.90) circle (  2.13);

\path[fill=fillColor,fill opacity=0.20] ( 71.27, 63.08) circle (  2.13);

\path[fill=fillColor,fill opacity=0.20] ( 71.27, 66.36) circle (  2.13);

\path[fill=fillColor,fill opacity=0.20] ( 69.27, 67.22) circle (  2.13);

\path[fill=fillColor,fill opacity=0.20] ( 66.46, 70.06) circle (  2.13);

\path[fill=fillColor,fill opacity=0.20] ( 74.28, 66.10) circle (  2.13);

\path[fill=fillColor,fill opacity=0.20] ( 78.30, 58.00) circle (  2.13);

\path[fill=fillColor,fill opacity=0.20] ( 92.34, 63.43) circle (  2.13);

\path[fill=fillColor,fill opacity=0.20] (102.37, 95.29) circle (  2.13);

\path[fill=fillColor,fill opacity=0.20] ( 90.33, 72.99) circle (  2.13);

\path[fill=fillColor,fill opacity=0.20] ( 83.31, 59.98) circle (  2.13);

\path[fill=fillColor,fill opacity=0.20] ( 85.32, 62.74) circle (  2.13);

\path[fill=fillColor,fill opacity=0.20] ( 91.34, 62.57) circle (  2.13);

\path[fill=fillColor,fill opacity=0.20] ( 90.33, 59.12) circle (  2.13);

\path[fill=fillColor,fill opacity=0.20] ( 93.34, 68.51) circle (  2.13);

\path[fill=fillColor,fill opacity=0.20] ( 95.35, 73.25) circle (  2.13);

\path[fill=fillColor,fill opacity=0.20] (100.37, 65.58) circle (  2.13);

\path[fill=fillColor,fill opacity=0.20] (101.37, 57.49) circle (  2.13);

\path[fill=fillColor,fill opacity=0.20] (100.37, 56.88) circle (  2.13);

\path[fill=fillColor,fill opacity=0.20] ( 99.36, 65.93) circle (  2.13);

\path[fill=fillColor,fill opacity=0.20] (105.38, 73.93) circle (  2.13);

\path[fill=fillColor,fill opacity=0.20] (109.40, 68.60) circle (  2.13);

\path[fill=fillColor,fill opacity=0.20] (111.40, 75.57) circle (  2.13);

\path[fill=fillColor,fill opacity=0.20] (107.39, 79.53) circle (  2.13);

\path[fill=fillColor,fill opacity=0.20] (106.39, 79.70) circle (  2.13);

\path[fill=fillColor,fill opacity=0.20] ( 78.30, 80.74) circle (  2.13);

\path[fill=fillColor,fill opacity=0.20] ( 87.33, 69.28) circle (  2.13);

\path[fill=fillColor,fill opacity=0.20] ( 83.31, 67.22) circle (  2.13);

\path[fill=fillColor,fill opacity=0.20] ( 73.28, 78.07) circle (  2.13);

\path[fill=fillColor,fill opacity=0.20] ( 77.29, 74.62) circle (  2.13);

\path[fill=fillColor,fill opacity=0.20] ( 76.29, 63.69) circle (  2.13);

\path[fill=fillColor,fill opacity=0.20] ( 84.32, 63.17) circle (  2.13);

\path[fill=fillColor,fill opacity=0.20] ( 76.29, 58.86) circle (  2.13);

\path[fill=fillColor,fill opacity=0.20] ( 69.27, 58.26) circle (  2.13);

\path[fill=fillColor,fill opacity=0.20] ( 61.24, 64.72) circle (  2.13);

\path[fill=fillColor,fill opacity=0.20] ( 73.28, 63.34) circle (  2.13);

\path[fill=fillColor,fill opacity=0.20] ( 77.29, 64.12) circle (  2.13);

\path[fill=fillColor,fill opacity=0.20] ( 75.29, 74.11) circle (  2.13);

\path[fill=fillColor,fill opacity=0.20] ( 86.32, 80.39) circle (  2.13);

\path[fill=fillColor,fill opacity=0.20] ( 89.33, 78.33) circle (  2.13);

\path[fill=fillColor,fill opacity=0.20] (100.37, 85.73) circle (  2.13);

\path[fill=fillColor,fill opacity=0.20] ( 89.33, 79.53) circle (  2.13);

\path[fill=fillColor,fill opacity=0.20] ( 87.33, 69.11) circle (  2.13);

\path[fill=fillColor,fill opacity=0.20] ( 82.31, 53.87) circle (  2.13);

\path[fill=fillColor,fill opacity=0.20] ( 82.31, 50.86) circle (  2.13);

\path[fill=fillColor,fill opacity=0.20] ( 90.33, 59.21) circle (  2.13);

\path[fill=fillColor,fill opacity=0.20] ( 97.36, 59.81) circle (  2.13);

\path[fill=fillColor,fill opacity=0.20] ( 96.35, 62.31) circle (  2.13);

\path[fill=fillColor,fill opacity=0.20] (101.37, 67.82) circle (  2.13);

\path[fill=fillColor,fill opacity=0.20] (101.37, 68.68) circle (  2.13);

\path[fill=fillColor,fill opacity=0.20] (103.38, 70.83) circle (  2.13);

\path[fill=fillColor,fill opacity=0.20] (107.39, 71.09) circle (  2.13);

\path[fill=fillColor,fill opacity=0.20] (106.39, 66.87) circle (  2.13);

\path[fill=fillColor,fill opacity=0.20] (103.38, 65.50) circle (  2.13);

\path[fill=fillColor,fill opacity=0.20] (102.37, 67.05) circle (  2.13);

\path[fill=fillColor,fill opacity=0.20] (102.37, 69.63) circle (  2.13);

\path[fill=fillColor,fill opacity=0.20] (104.38, 76.69) circle (  2.13);

\path[fill=fillColor,fill opacity=0.20] ( 98.36, 67.65) circle (  2.13);

\path[fill=fillColor,fill opacity=0.20] (108.39, 58.09) circle (  2.13);

\path[fill=fillColor,fill opacity=0.20] (109.40, 71.44) circle (  2.13);

\path[fill=fillColor,fill opacity=0.20] ( 99.36, 84.53) circle (  2.13);

\path[fill=fillColor,fill opacity=0.20] ( 98.36, 78.93) circle (  2.13);

\path[fill=fillColor,fill opacity=0.20] (101.37, 75.57) circle (  2.13);

\path[fill=fillColor,fill opacity=0.20] (102.37, 80.57) circle (  2.13);

\path[fill=fillColor,fill opacity=0.20] (103.38, 80.48) circle (  2.13);

\path[fill=fillColor,fill opacity=0.20] ( 96.35, 72.04) circle (  2.13);

\path[fill=fillColor,fill opacity=0.20] (100.37, 67.82) circle (  2.13);

\path[fill=fillColor,fill opacity=0.20] (101.37, 70.58) circle (  2.13);

\path[fill=fillColor,fill opacity=0.20] ( 99.36, 79.88) circle (  2.13);

\path[fill=fillColor,fill opacity=0.20] ( 98.36, 83.49) circle (  2.13);

\path[fill=fillColor,fill opacity=0.20] (100.37, 77.21) circle (  2.13);

\path[fill=fillColor,fill opacity=0.20] (100.37, 72.73) circle (  2.13);

\path[fill=fillColor,fill opacity=0.20] ( 94.35, 69.37) circle (  2.13);

\path[fill=fillColor,fill opacity=0.20] ( 88.33, 67.65) circle (  2.13);

\path[fill=fillColor,fill opacity=0.20] ( 81.31, 74.71) circle (  2.13);

\path[fill=fillColor,fill opacity=0.20] ( 79.30, 75.23) circle (  2.13);

\path[fill=fillColor,fill opacity=0.20] ( 85.32, 68.25) circle (  2.13);

\path[fill=fillColor,fill opacity=0.20] ( 81.31, 69.03) circle (  2.13);

\path[fill=fillColor,fill opacity=0.20] ( 76.29, 66.61) circle (  2.13);

\path[fill=fillColor,fill opacity=0.20] ( 79.30, 60.93) circle (  2.13);

\path[fill=fillColor,fill opacity=0.20] ( 80.30, 58.00) circle (  2.13);

\path[fill=fillColor,fill opacity=0.20] ( 81.31, 57.49) circle (  2.13);

\path[fill=fillColor,fill opacity=0.20] ( 78.30, 64.46) circle (  2.13);

\path[fill=fillColor,fill opacity=0.20] ( 82.31, 75.83) circle (  2.13);

\path[fill=fillColor,fill opacity=0.20] ( 90.33, 72.64) circle (  2.13);

\path[fill=fillColor,fill opacity=0.20] ( 76.29, 75.31) circle (  2.13);

\path[fill=fillColor,fill opacity=0.20] ( 88.33, 81.94) circle (  2.13);

\path[fill=fillColor,fill opacity=0.20] ( 95.35, 77.81) circle (  2.13);

\path[fill=fillColor,fill opacity=0.20] ( 89.33, 86.59) circle (  2.13);

\path[fill=fillColor,fill opacity=0.20] ( 91.34, 71.18) circle (  2.13);

\path[fill=fillColor,fill opacity=0.20] ( 88.33, 58.69) circle (  2.13);

\path[fill=fillColor,fill opacity=0.20] ( 87.33, 58.26) circle (  2.13);

\path[fill=fillColor,fill opacity=0.20] ( 89.33, 61.45) circle (  2.13);

\path[fill=fillColor,fill opacity=0.20] ( 92.34, 57.75) circle (  2.13);

\path[fill=fillColor,fill opacity=0.20] ( 97.36, 60.07) circle (  2.13);

\path[fill=fillColor,fill opacity=0.20] (102.37, 63.26) circle (  2.13);

\path[fill=fillColor,fill opacity=0.20] (102.37, 63.43) circle (  2.13);

\path[fill=fillColor,fill opacity=0.20] (101.37, 67.39) circle (  2.13);

\path[fill=fillColor,fill opacity=0.20] (103.38, 70.15) circle (  2.13);

\path[fill=fillColor,fill opacity=0.20] ( 97.36, 68.25) circle (  2.13);

\path[fill=fillColor,fill opacity=0.20] ( 88.33, 70.06) circle (  2.13);

\path[fill=fillColor,fill opacity=0.20] (100.37, 71.78) circle (  2.13);

\path[fill=fillColor,fill opacity=0.20] ( 99.36, 69.46) circle (  2.13);

\path[fill=fillColor,fill opacity=0.20] (102.37, 63.34) circle (  2.13);

\path[fill=fillColor,fill opacity=0.20] ( 99.36, 55.94) circle (  2.13);

\path[fill=fillColor,fill opacity=0.20] (102.37, 59.21) circle (  2.13);

\path[fill=fillColor,fill opacity=0.20] ( 98.36, 64.98) circle (  2.13);

\path[fill=fillColor,fill opacity=0.20] (103.38, 61.28) circle (  2.13);

\path[fill=fillColor,fill opacity=0.20] ( 96.35, 61.10) circle (  2.13);

\path[fill=fillColor,fill opacity=0.20] ( 93.34, 66.96) circle (  2.13);

\path[fill=fillColor,fill opacity=0.20] ( 91.34, 67.65) circle (  2.13);

\path[fill=fillColor,fill opacity=0.20] ( 89.33, 67.82) circle (  2.13);

\path[fill=fillColor,fill opacity=0.20] ( 92.34, 69.46) circle (  2.13);

\path[fill=fillColor,fill opacity=0.20] ( 92.34, 69.97) circle (  2.13);

\path[fill=fillColor,fill opacity=0.20] ( 90.33, 74.62) circle (  2.13);

\path[fill=fillColor,fill opacity=0.20] ( 90.33, 78.58) circle (  2.13);

\path[fill=fillColor,fill opacity=0.20] ( 86.32, 74.71) circle (  2.13);

\path[fill=fillColor,fill opacity=0.20] ( 86.32, 68.34) circle (  2.13);

\path[fill=fillColor,fill opacity=0.20] ( 82.31, 66.44) circle (  2.13);

\path[fill=fillColor,fill opacity=0.20] ( 81.31, 68.08) circle (  2.13);

\path[fill=fillColor,fill opacity=0.20] ( 87.33, 58.00) circle (  2.13);

\path[fill=fillColor,fill opacity=0.20] ( 83.31, 45.52) circle (  2.13);

\path[fill=fillColor,fill opacity=0.20] ( 87.33, 54.13) circle (  2.13);

\path[fill=fillColor,fill opacity=0.20] ( 84.32, 61.79) circle (  2.13);

\path[fill=fillColor,fill opacity=0.20] ( 83.31, 58.35) circle (  2.13);

\path[fill=fillColor,fill opacity=0.20] ( 86.32, 64.89) circle (  2.13);

\path[fill=fillColor,fill opacity=0.20] ( 88.33, 69.37) circle (  2.13);

\path[fill=fillColor,fill opacity=0.20] ( 84.32, 72.04) circle (  2.13);

\path[fill=fillColor,fill opacity=0.20] ( 86.32, 86.34) circle (  2.13);

\path[fill=fillColor,fill opacity=0.20] ( 98.36, 90.04) circle (  2.13);

\path[fill=fillColor,fill opacity=0.20] ( 73.28, 85.73) circle (  2.13);

\path[fill=fillColor,fill opacity=0.20] (111.40, 99.17) circle (  2.13);

\path[fill=fillColor,fill opacity=0.20] ( 89.33, 87.54) circle (  2.13);

\path[fill=fillColor,fill opacity=0.20] ( 88.33, 76.86) circle (  2.13);

\path[fill=fillColor,fill opacity=0.20] ( 91.34, 71.87) circle (  2.13);

\path[fill=fillColor,fill opacity=0.20] ( 87.33, 64.20) circle (  2.13);

\path[fill=fillColor,fill opacity=0.20] ( 91.34, 54.39) circle (  2.13);

\path[fill=fillColor,fill opacity=0.20] ( 92.34, 52.84) circle (  2.13);

\path[fill=fillColor,fill opacity=0.20] ( 96.35, 59.98) circle (  2.13);

\path[fill=fillColor,fill opacity=0.20] ( 97.36, 67.73) circle (  2.13);

\path[fill=fillColor,fill opacity=0.20] ( 95.35, 68.08) circle (  2.13);

\path[fill=fillColor,fill opacity=0.20] ( 97.36, 69.28) circle (  2.13);

\path[fill=fillColor,fill opacity=0.20] ( 99.36, 73.68) circle (  2.13);

\path[fill=fillColor,fill opacity=0.20] ( 96.35, 72.82) circle (  2.13);

\path[fill=fillColor,fill opacity=0.20] ( 95.35, 68.25) circle (  2.13);

\path[fill=fillColor,fill opacity=0.20] ( 97.36, 68.08) circle (  2.13);

\path[fill=fillColor,fill opacity=0.20] ( 96.35, 68.94) circle (  2.13);

\path[fill=fillColor,fill opacity=0.20] ( 96.35, 67.39) circle (  2.13);

\path[fill=fillColor,fill opacity=0.20] (102.37, 65.75) circle (  2.13);

\path[fill=fillColor,fill opacity=0.20] ( 99.36, 64.20) circle (  2.13);

\path[fill=fillColor,fill opacity=0.20] ( 89.33, 64.72) circle (  2.13);

\path[fill=fillColor,fill opacity=0.20] ( 95.35, 65.15) circle (  2.13);

\path[fill=fillColor,fill opacity=0.20] (100.37, 66.18) circle (  2.13);

\path[fill=fillColor,fill opacity=0.20] (100.37, 71.78) circle (  2.13);

\path[fill=fillColor,fill opacity=0.20] ( 95.35, 74.02) circle (  2.13);

\path[fill=fillColor,fill opacity=0.20] ( 97.36, 71.27) circle (  2.13);

\path[fill=fillColor,fill opacity=0.20] ( 93.34, 68.85) circle (  2.13);

\path[fill=fillColor,fill opacity=0.20] ( 92.34, 66.01) circle (  2.13);

\path[fill=fillColor,fill opacity=0.20] ( 89.33, 64.81) circle (  2.13);

\path[fill=fillColor,fill opacity=0.20] ( 91.34, 61.96) circle (  2.13);

\path[fill=fillColor,fill opacity=0.20] ( 82.31, 59.12) circle (  2.13);

\path[fill=fillColor,fill opacity=0.20] ( 81.31, 64.89) circle (  2.13);

\path[fill=fillColor,fill opacity=0.20] ( 73.28, 70.66) circle (  2.13);

\path[fill=fillColor,fill opacity=0.20] ( 89.33, 67.73) circle (  2.13);

\path[fill=fillColor,fill opacity=0.20] ( 82.31, 68.42) circle (  2.13);

\path[fill=fillColor,fill opacity=0.20] ( 87.33, 76.35) circle (  2.13);

\path[fill=fillColor,fill opacity=0.20] ( 87.33, 84.96) circle (  2.13);

\path[fill=fillColor,fill opacity=0.20] (103.38,108.38) circle (  2.13);

\path[fill=fillColor,fill opacity=0.20] ( 92.34, 89.18) circle (  2.13);

\path[fill=fillColor,fill opacity=0.20] ( 90.33, 70.32) circle (  2.13);

\path[fill=fillColor,fill opacity=0.20] ( 90.33, 60.33) circle (  2.13);

\path[fill=fillColor,fill opacity=0.20] ( 84.32, 66.27) circle (  2.13);

\path[fill=fillColor,fill opacity=0.20] ( 86.32, 71.61) circle (  2.13);

\path[fill=fillColor,fill opacity=0.20] ( 89.33, 67.48) circle (  2.13);

\path[fill=fillColor,fill opacity=0.20] ( 87.33, 65.32) circle (  2.13);

\path[fill=fillColor,fill opacity=0.20] ( 85.32, 66.87) circle (  2.13);

\path[fill=fillColor,fill opacity=0.20] ( 90.33, 64.38) circle (  2.13);

\path[fill=fillColor,fill opacity=0.20] ( 93.34, 60.50) circle (  2.13);

\path[fill=fillColor,fill opacity=0.20] ( 94.35, 56.37) circle (  2.13);

\path[fill=fillColor,fill opacity=0.20] ( 95.35, 58.09) circle (  2.13);

\path[fill=fillColor,fill opacity=0.20] (100.37, 63.86) circle (  2.13);

\path[fill=fillColor,fill opacity=0.20] ( 97.36, 64.46) circle (  2.13);

\path[fill=fillColor,fill opacity=0.20] ( 98.36, 59.21) circle (  2.13);

\path[fill=fillColor,fill opacity=0.20] ( 99.36, 55.08) circle (  2.13);

\path[fill=fillColor,fill opacity=0.20] ( 99.36, 53.01) circle (  2.13);

\path[fill=fillColor,fill opacity=0.20] ( 94.35, 57.40) circle (  2.13);

\path[fill=fillColor,fill opacity=0.20] ( 86.32, 64.55) circle (  2.13);

\path[fill=fillColor,fill opacity=0.20] ( 91.34, 65.24) circle (  2.13);

\path[fill=fillColor,fill opacity=0.20] ( 86.32, 63.95) circle (  2.13);

\path[fill=fillColor,fill opacity=0.20] ( 89.33, 64.81) circle (  2.13);

\path[fill=fillColor,fill opacity=0.20] ( 84.32, 62.91) circle (  2.13);

\path[fill=fillColor,fill opacity=0.20] ( 84.32, 66.27) circle (  2.13);

\path[fill=fillColor,fill opacity=0.20] ( 85.32, 76.95) circle (  2.13);

\path[fill=fillColor,fill opacity=0.20] ( 96.35, 83.41) circle (  2.13);

\path[fill=fillColor,fill opacity=0.20] ( 88.33, 91.24) circle (  2.13);

\path[fill=fillColor,fill opacity=0.20] ( 98.36,112.51) circle (  2.13);

\path[fill=fillColor,fill opacity=0.20] ( 96.35, 92.02) circle (  2.13);

\path[fill=fillColor,fill opacity=0.20] ( 81.31, 86.42) circle (  2.13);

\path[fill=fillColor,fill opacity=0.20] ( 84.32, 80.74) circle (  2.13);

\path[fill=fillColor,fill opacity=0.20] ( 92.34, 78.41) circle (  2.13);

\path[fill=fillColor,fill opacity=0.20] ( 82.31, 69.97) circle (  2.13);

\path[fill=fillColor,fill opacity=0.20] ( 87.33, 57.49) circle (  2.13);

\path[fill=fillColor,fill opacity=0.20] ( 78.30, 61.62) circle (  2.13);

\path[fill=fillColor,fill opacity=0.20] ( 83.31, 69.97) circle (  2.13);

\path[fill=fillColor,fill opacity=0.20] ( 85.32, 58.43) circle (  2.13);

\path[fill=fillColor,fill opacity=0.20] ( 94.35, 50.17) circle (  2.13);

\path[fill=fillColor,fill opacity=0.20] ( 93.34, 61.79) circle (  2.13);

\path[fill=fillColor,fill opacity=0.20] ( 95.35, 67.99) circle (  2.13);

\path[fill=fillColor,fill opacity=0.20] ( 92.34, 64.29) circle (  2.13);

\path[fill=fillColor,fill opacity=0.20] ( 93.34, 62.05) circle (  2.13);

\path[fill=fillColor,fill opacity=0.20] ( 90.33, 62.14) circle (  2.13);

\path[fill=fillColor,fill opacity=0.20] ( 91.34, 66.79) circle (  2.13);

\path[fill=fillColor,fill opacity=0.20] ( 83.31, 72.64) circle (  2.13);

\path[fill=fillColor,fill opacity=0.20] ( 82.31, 76.17) circle (  2.13);

\path[fill=fillColor,fill opacity=0.20] ( 90.33, 85.65) circle (  2.13);

\path[fill=fillColor,fill opacity=0.20] ( 86.32, 92.54) circle (  2.13);

\path[fill=fillColor,fill opacity=0.20] ( 80.30, 90.81) circle (  2.13);

\path[fill=fillColor,fill opacity=0.20] (115.42,110.71) circle (  2.13);

\path[fill=fillColor,fill opacity=0.20] (105.38,102.35) circle (  2.13);

\path[fill=fillColor,fill opacity=0.20] (100.37, 86.42) circle (  2.13);

\path[fill=fillColor,fill opacity=0.20] ( 81.31, 88.75) circle (  2.13);

\path[fill=fillColor,fill opacity=0.20] ( 79.30, 93.83) circle (  2.13);

\path[fill=fillColor,fill opacity=0.20] ( 89.33, 88.23) circle (  2.13);

\path[fill=fillColor,fill opacity=0.20] ( 96.35, 86.16) circle (  2.13);

\path[fill=fillColor,fill opacity=0.20] ( 99.36, 86.25) circle (  2.13);

\path[fill=fillColor,fill opacity=0.20] ( 95.35, 87.28) circle (  2.13);

\path[fill=fillColor,fill opacity=0.20] ( 96.35, 90.12) circle (  2.13);

\path[fill=fillColor,fill opacity=0.20] ( 87.33, 91.33) circle (  2.13);

\path[fill=fillColor,fill opacity=0.20] ( 53.52,111.65) circle (  2.13);

\path[fill=fillColor,fill opacity=0.20] ( 63.75, 88.06) circle (  2.13);

\path[fill=fillColor,fill opacity=0.20] ( 63.85, 82.46) circle (  2.13);

\path[fill=fillColor,fill opacity=0.20] ( 52.01, 99.77) circle (  2.13);

\path[fill=fillColor,fill opacity=0.20] ( 58.03, 83.23) circle (  2.13);

\path[fill=fillColor,fill opacity=0.20] ( 72.28, 58.78) circle (  2.13);

\path[fill=fillColor,fill opacity=0.20] ( 75.29, 49.56) circle (  2.13);

\path[fill=fillColor,fill opacity=0.20] ( 75.29, 59.21) circle (  2.13);

\path[fill=fillColor,fill opacity=0.20] ( 65.96, 61.10) circle (  2.13);

\path[fill=fillColor,fill opacity=0.20] ( 70.27, 54.82) circle (  2.13);

\path[fill=fillColor,fill opacity=0.20] ( 58.83, 69.20) circle (  2.13);

\path[fill=fillColor,fill opacity=0.20] ( 54.92,109.84) circle (  2.13);

\path[fill=fillColor,fill opacity=0.20] ( 61.94, 52.84) circle (  2.13);

\path[fill=fillColor,fill opacity=0.20] ( 65.86, 56.63) circle (  2.13);

\path[fill=fillColor,fill opacity=0.20] ( 71.27, 46.89) circle (  2.13);

\path[fill=fillColor,fill opacity=0.20] ( 69.27, 44.48) circle (  2.13);

\path[fill=fillColor,fill opacity=0.20] ( 70.27, 58.95) circle (  2.13);

\path[fill=fillColor,fill opacity=0.20] ( 68.26, 56.45) circle (  2.13);

\path[fill=fillColor,fill opacity=0.20] ( 68.26, 46.64) circle (  2.13);

\path[fill=fillColor,fill opacity=0.20] ( 72.28, 58.09) circle (  2.13);

\path[fill=fillColor,fill opacity=0.20] ( 83.31, 88.75) circle (  2.13);

\path[fill=fillColor,fill opacity=0.20] ( 60.84, 62.74) circle (  2.13);

\path[fill=fillColor,fill opacity=0.20] ( 74.28, 48.70) circle (  2.13);

\path[fill=fillColor,fill opacity=0.20] ( 68.26, 66.96) circle (  2.13);

\path[fill=fillColor,fill opacity=0.20] ( 60.44, 54.56) circle (  2.13);

\path[fill=fillColor,fill opacity=0.20] ( 62.95, 52.15) circle (  2.13);

\path[fill=fillColor,fill opacity=0.20] ( 68.26, 59.04) circle (  2.13);

\path[fill=fillColor,fill opacity=0.20] ( 65.76, 57.66) circle (  2.13);

\path[fill=fillColor,fill opacity=0.20] ( 62.75, 51.37) circle (  2.13);

\path[fill=fillColor,fill opacity=0.20] ( 67.76, 55.25) circle (  2.13);

\path[fill=fillColor,fill opacity=0.20] ( 85.32, 73.07) circle (  2.13);

\path[fill=fillColor,fill opacity=0.20] ( 67.96, 46.89) circle (  2.13);

\path[fill=fillColor,fill opacity=0.20] ( 75.29, 52.49) circle (  2.13);

\path[fill=fillColor,fill opacity=0.20] ( 64.85, 60.07) circle (  2.13);

\path[fill=fillColor,fill opacity=0.20] ( 53.42, 52.41) circle (  2.13);

\path[fill=fillColor,fill opacity=0.20] ( 56.12, 50.86) circle (  2.13);

\path[fill=fillColor,fill opacity=0.20] ( 73.28, 44.31) circle (  2.13);

\path[fill=fillColor,fill opacity=0.20] ( 69.27, 45.09) circle (  2.13);

\path[fill=fillColor,fill opacity=0.20] ( 58.93, 53.78) circle (  2.13);

\path[fill=fillColor,fill opacity=0.20] ( 66.36, 51.11) circle (  2.13);

\path[fill=fillColor,fill opacity=0.20] ( 90.33, 59.81) circle (  2.13);

\path[fill=fillColor,fill opacity=0.20] (142.50,106.06) circle (  2.13);

\path[fill=fillColor,fill opacity=0.20] ( 91.34,114.24) circle (  2.13);

\path[fill=fillColor,fill opacity=0.20] ( 75.29, 51.63) circle (  2.13);

\path[fill=fillColor,fill opacity=0.20] ( 69.27, 40.44) circle (  2.13);

\path[fill=fillColor,fill opacity=0.20] ( 65.86, 44.57) circle (  2.13);

\path[fill=fillColor,fill opacity=0.20] ( 64.95, 52.58) circle (  2.13);

\path[fill=fillColor,fill opacity=0.20] ( 64.85, 52.92) circle (  2.13);

\path[fill=fillColor,fill opacity=0.20] ( 68.06, 38.37) circle (  2.13);

\path[fill=fillColor,fill opacity=0.20] ( 62.95, 48.62) circle (  2.13);

\path[fill=fillColor,fill opacity=0.20] ( 68.26, 51.29) circle (  2.13);

\path[fill=fillColor,fill opacity=0.20] ( 89.33,113.37) circle (  2.13);

\path[fill=fillColor,fill opacity=0.20] ( 93.34, 76.95) circle (  2.13);

\path[fill=fillColor,fill opacity=0.20] ( 94.35, 76.60) circle (  2.13);

\path[fill=fillColor,fill opacity=0.20] ( 88.33, 83.75) circle (  2.13);

\path[fill=fillColor,fill opacity=0.20] ( 87.33, 71.52) circle (  2.13);

\path[fill=fillColor,fill opacity=0.20] ( 83.31, 68.16) circle (  2.13);

\path[fill=fillColor,fill opacity=0.20] (104.38, 82.55) circle (  2.13);

\path[fill=fillColor,fill opacity=0.20] ( 99.36, 79.79) circle (  2.13);

\path[fill=fillColor,fill opacity=0.20] (103.38, 80.31) circle (  2.13);

\path[fill=fillColor,fill opacity=0.20] ( 91.34,107.00) circle (  2.13);

\path[fill=fillColor,fill opacity=0.20] ( 81.31, 63.60) circle (  2.13);

\path[fill=fillColor,fill opacity=0.20] ( 74.28, 50.86) circle (  2.13);

\path[fill=fillColor,fill opacity=0.20] ( 70.27, 44.91) circle (  2.13);

\path[fill=fillColor,fill opacity=0.20] ( 70.27, 51.03) circle (  2.13);

\path[fill=fillColor,fill opacity=0.20] ( 66.56, 52.66) circle (  2.13);

\path[fill=fillColor,fill opacity=0.20] ( 61.64, 46.89) circle (  2.13);

\path[fill=fillColor,fill opacity=0.20] ( 61.74, 45.00) circle (  2.13);

\path[fill=fillColor,fill opacity=0.20] ( 58.83, 44.66) circle (  2.13);

\path[fill=fillColor,fill opacity=0.20] ( 64.65, 51.54) circle (  2.13);

\path[fill=fillColor,fill opacity=0.20] ( 72.28,102.18) circle (  2.13);

\path[fill=fillColor,fill opacity=0.20] ( 69.27, 68.94) circle (  2.13);

\path[fill=fillColor,fill opacity=0.20] ( 70.27, 56.63) circle (  2.13);

\path[fill=fillColor,fill opacity=0.20] ( 82.31, 49.22) circle (  2.13);

\path[fill=fillColor,fill opacity=0.20] ( 94.35, 51.29) circle (  2.13);

\path[fill=fillColor,fill opacity=0.20] ( 96.35, 57.92) circle (  2.13);

\path[fill=fillColor,fill opacity=0.20] ( 90.33, 71.44) circle (  2.13);

\path[fill=fillColor,fill opacity=0.20] ( 94.35, 87.89) circle (  2.13);

\path[fill=fillColor,fill opacity=0.20] ( 96.35, 90.30) circle (  2.13);

\path[fill=fillColor,fill opacity=0.20] ( 81.31, 76.09) circle (  2.13);

\path[fill=fillColor,fill opacity=0.20] ( 78.30, 58.35) circle (  2.13);

\path[fill=fillColor,fill opacity=0.20] ( 74.28, 59.98) circle (  2.13);

\path[fill=fillColor,fill opacity=0.20] ( 74.28, 48.96) circle (  2.13);

\path[fill=fillColor,fill opacity=0.20] ( 65.45, 47.07) circle (  2.13);

\path[fill=fillColor,fill opacity=0.20] ( 59.23, 51.80) circle (  2.13);

\path[fill=fillColor,fill opacity=0.20] ( 58.73, 45.09) circle (  2.13);

\path[fill=fillColor,fill opacity=0.20] ( 75.29, 47.07) circle (  2.13);

\path[fill=fillColor,fill opacity=0.20] ( 72.28, 45.52) circle (  2.13);

\path[fill=fillColor,fill opacity=0.20] ( 74.28, 57.23) circle (  2.13);

\path[fill=fillColor,fill opacity=0.20] ( 77.29, 59.90) circle (  2.13);

\path[fill=fillColor,fill opacity=0.20] ( 79.30, 59.55) circle (  2.13);

\path[fill=fillColor,fill opacity=0.20] ( 82.31, 63.34) circle (  2.13);

\path[fill=fillColor,fill opacity=0.20] ( 96.35, 65.58) circle (  2.13);

\path[fill=fillColor,fill opacity=0.20] ( 90.33, 71.18) circle (  2.13);

\path[fill=fillColor,fill opacity=0.20] ( 89.33, 79.45) circle (  2.13);

\path[fill=fillColor,fill opacity=0.20] ( 72.28, 49.65) circle (  2.13);

\path[fill=fillColor,fill opacity=0.20] ( 73.28, 40.01) circle (  2.13);

\path[fill=fillColor,fill opacity=0.20] ( 69.27, 47.67) circle (  2.13);

\path[fill=fillColor,fill opacity=0.20] ( 67.96, 53.01) circle (  2.13);

\path[fill=fillColor,fill opacity=0.20] ( 61.24, 44.05) circle (  2.13);

\path[fill=fillColor,fill opacity=0.20] ( 61.44, 49.31) circle (  2.13);

\path[fill=fillColor,fill opacity=0.20] ( 69.27, 46.46) circle (  2.13);

\path[fill=fillColor,fill opacity=0.20] ( 88.33, 49.74) circle (  2.13);

\path[fill=fillColor,fill opacity=0.20] ( 72.28, 82.12) circle (  2.13);

\path[fill=fillColor,fill opacity=0.20] ( 72.28, 40.78) circle (  2.13);

\path[fill=fillColor,fill opacity=0.20] ( 72.28, 51.03) circle (  2.13);

\path[fill=fillColor,fill opacity=0.20] ( 72.28, 60.93) circle (  2.13);

\path[fill=fillColor,fill opacity=0.20] ( 77.29, 53.78) circle (  2.13);

\path[fill=fillColor,fill opacity=0.20] ( 77.29, 57.92) circle (  2.13);

\path[fill=fillColor,fill opacity=0.20] ( 91.34, 62.57) circle (  2.13);

\path[fill=fillColor,fill opacity=0.20] ( 91.34, 55.51) circle (  2.13);

\path[fill=fillColor,fill opacity=0.20] ( 86.32, 54.47) circle (  2.13);

\path[fill=fillColor,fill opacity=0.20] ( 90.33, 63.60) circle (  2.13);

\path[fill=fillColor,fill opacity=0.20] ( 87.33, 86.08) circle (  2.13);

\path[fill=fillColor,fill opacity=0.20] ( 78.30, 53.35) circle (  2.13);

\path[fill=fillColor,fill opacity=0.20] ( 66.86, 54.64) circle (  2.13);

\path[fill=fillColor,fill opacity=0.20] ( 70.27, 42.24) circle (  2.13);

\path[fill=fillColor,fill opacity=0.20] ( 70.27, 48.10) circle (  2.13);

\path[fill=fillColor,fill opacity=0.20] ( 60.24, 59.55) circle (  2.13);

\path[fill=fillColor,fill opacity=0.20] ( 63.95, 54.90) circle (  2.13);

\path[fill=fillColor,fill opacity=0.20] ( 63.75, 51.46) circle (  2.13);

\path[fill=fillColor,fill opacity=0.20] ( 59.84, 52.92) circle (  2.13);

\path[fill=fillColor,fill opacity=0.20] ( 62.55, 52.58) circle (  2.13);

\path[fill=fillColor,fill opacity=0.20] ( 84.32, 50.34) circle (  2.13);

\path[fill=fillColor,fill opacity=0.20] ( 79.30, 51.11) circle (  2.13);

\path[fill=fillColor,fill opacity=0.20] ( 56.02, 52.06) circle (  2.13);

\path[fill=fillColor,fill opacity=0.20] ( 77.29, 48.19) circle (  2.13);

\path[fill=fillColor,fill opacity=0.20] ( 75.29, 40.78) circle (  2.13);

\path[fill=fillColor,fill opacity=0.20] ( 82.31, 55.33) circle (  2.13);

\path[fill=fillColor,fill opacity=0.20] ( 88.33, 59.98) circle (  2.13);

\path[fill=fillColor,fill opacity=0.20] ( 86.32, 54.64) circle (  2.13);

\path[fill=fillColor,fill opacity=0.20] ( 91.34, 57.83) circle (  2.13);

\path[fill=fillColor,fill opacity=0.20] ( 86.32,102.87) circle (  2.13);

\path[fill=fillColor,fill opacity=0.20] ( 83.31, 57.66) circle (  2.13);

\path[fill=fillColor,fill opacity=0.20] ( 73.28, 66.79) circle (  2.13);

\path[fill=fillColor,fill opacity=0.20] ( 55.42, 62.65) circle (  2.13);

\path[fill=fillColor,fill opacity=0.20] ( 66.36, 52.66) circle (  2.13);

\path[fill=fillColor,fill opacity=0.20] ( 68.26, 53.44) circle (  2.13);

\path[fill=fillColor,fill opacity=0.20] ( 59.13, 51.98) circle (  2.13);

\path[fill=fillColor,fill opacity=0.20] ( 49.10, 55.68) circle (  2.13);

\path[fill=fillColor,fill opacity=0.20] ( 59.74, 52.58) circle (  2.13);

\path[fill=fillColor,fill opacity=0.20] ( 68.26, 50.51) circle (  2.13);

\path[fill=fillColor,fill opacity=0.20] ( 68.26,112.51) circle (  2.13);

\path[fill=fillColor,fill opacity=0.20] ( 74.28, 54.73) circle (  2.13);

\path[fill=fillColor,fill opacity=0.20] ( 68.26, 38.71) circle (  2.13);

\path[fill=fillColor,fill opacity=0.20] ( 65.96, 40.01) circle (  2.13);

\path[fill=fillColor,fill opacity=0.20] ( 84.32, 51.63) circle (  2.13);

\path[fill=fillColor,fill opacity=0.20] ( 93.34, 57.31) circle (  2.13);

\path[fill=fillColor,fill opacity=0.20] ( 97.36, 58.26) circle (  2.13);

\path[fill=fillColor,fill opacity=0.20] (100.37, 66.79) circle (  2.13);

\path[fill=fillColor,fill opacity=0.20] ( 78.30,106.31) circle (  2.13);

\path[fill=fillColor,fill opacity=0.20] ( 88.33, 63.95) circle (  2.13);

\path[fill=fillColor,fill opacity=0.20] ( 81.31, 68.42) circle (  2.13);

\path[fill=fillColor,fill opacity=0.20] ( 71.27, 65.50) circle (  2.13);

\path[fill=fillColor,fill opacity=0.20] ( 68.26, 56.63) circle (  2.13);

\path[fill=fillColor,fill opacity=0.20] ( 69.27, 53.87) circle (  2.13);

\path[fill=fillColor,fill opacity=0.20] ( 64.05, 47.67) circle (  2.13);

\path[fill=fillColor,fill opacity=0.20] ( 57.03, 52.06) circle (  2.13);

\path[fill=fillColor,fill opacity=0.20] ( 60.54, 45.34) circle (  2.13);

\path[fill=fillColor,fill opacity=0.20] ( 86.32, 39.92) circle (  2.13);

\path[fill=fillColor,fill opacity=0.20] ( 89.33, 70.66) circle (  2.13);

\path[fill=fillColor,fill opacity=0.20] ( 67.66, 50.08) circle (  2.13);

\path[fill=fillColor,fill opacity=0.20] ( 62.75, 40.87) circle (  2.13);

\path[fill=fillColor,fill opacity=0.20] ( 85.32, 47.33) circle (  2.13);

\path[fill=fillColor,fill opacity=0.20] ( 94.35, 53.27) circle (  2.13);

\path[fill=fillColor,fill opacity=0.20] ( 98.36, 57.66) circle (  2.13);

\path[fill=fillColor,fill opacity=0.20] (101.37, 60.07) circle (  2.13);

\path[fill=fillColor,fill opacity=0.20] ( 92.34,107.09) circle (  2.13);

\path[fill=fillColor,fill opacity=0.20] ( 82.31, 65.41) circle (  2.13);

\path[fill=fillColor,fill opacity=0.20] ( 80.30, 74.45) circle (  2.13);

\path[fill=fillColor,fill opacity=0.20] ( 77.29, 71.95) circle (  2.13);

\path[fill=fillColor,fill opacity=0.20] ( 74.28, 56.80) circle (  2.13);

\path[fill=fillColor,fill opacity=0.20] ( 72.28, 53.44) circle (  2.13);

\path[fill=fillColor,fill opacity=0.20] ( 70.27, 54.47) circle (  2.13);

\path[fill=fillColor,fill opacity=0.20] ( 64.95, 48.19) circle (  2.13);

\path[fill=fillColor,fill opacity=0.20] ( 66.56, 39.75) circle (  2.13);

\path[fill=fillColor,fill opacity=0.20] ( 66.16, 53.35) circle (  2.13);

\path[fill=fillColor,fill opacity=0.20] ( 77.29, 54.04) circle (  2.13);

\path[fill=fillColor,fill opacity=0.20] (101.37, 86.42) circle (  2.13);

\path[fill=fillColor,fill opacity=0.20] ( 78.30, 54.30) circle (  2.13);

\path[fill=fillColor,fill opacity=0.20] ( 61.14, 48.44) circle (  2.13);

\path[fill=fillColor,fill opacity=0.20] ( 63.85, 52.58) circle (  2.13);

\path[fill=fillColor,fill opacity=0.20] ( 79.30, 51.03) circle (  2.13);

\path[fill=fillColor,fill opacity=0.20] ( 79.30, 49.05) circle (  2.13);

\path[fill=fillColor,fill opacity=0.20] ( 87.33, 55.59) circle (  2.13);

\path[fill=fillColor,fill opacity=0.20] ( 86.32, 57.92) circle (  2.13);

\path[fill=fillColor,fill opacity=0.20] ( 94.35, 54.21) circle (  2.13);

\path[fill=fillColor,fill opacity=0.20] (113.41, 77.12) circle (  2.13);

\path[fill=fillColor,fill opacity=0.20] ( 96.35,100.03) circle (  2.13);

\path[fill=fillColor,fill opacity=0.20] ( 89.33, 68.68) circle (  2.13);

\path[fill=fillColor,fill opacity=0.20] ( 81.31, 69.03) circle (  2.13);

\path[fill=fillColor,fill opacity=0.20] ( 74.28, 63.95) circle (  2.13);

\path[fill=fillColor,fill opacity=0.20] ( 69.27, 63.08) circle (  2.13);

\path[fill=fillColor,fill opacity=0.20] ( 67.36, 57.40) circle (  2.13);

\path[fill=fillColor,fill opacity=0.20] ( 76.29, 51.89) circle (  2.13);

\path[fill=fillColor,fill opacity=0.20] ( 72.28, 52.32) circle (  2.13);

\path[fill=fillColor,fill opacity=0.20] ( 66.06, 51.89) circle (  2.13);

\path[fill=fillColor,fill opacity=0.20] ( 70.27, 52.49) circle (  2.13);

\path[fill=fillColor,fill opacity=0.20] ( 73.28, 55.59) circle (  2.13);

\path[fill=fillColor,fill opacity=0.20] ( 78.30, 56.37) circle (  2.13);

\path[fill=fillColor,fill opacity=0.20] ( 97.36, 66.61) circle (  2.13);

\path[fill=fillColor,fill opacity=0.20] ( 72.28, 53.70) circle (  2.13);

\path[fill=fillColor,fill opacity=0.20] ( 70.27, 52.23) circle (  2.13);

\path[fill=fillColor,fill opacity=0.20] ( 73.28, 45.86) circle (  2.13);

\path[fill=fillColor,fill opacity=0.20] ( 73.28, 51.20) circle (  2.13);

\path[fill=fillColor,fill opacity=0.20] ( 73.28, 59.90) circle (  2.13);

\path[fill=fillColor,fill opacity=0.20] ( 78.30, 57.06) circle (  2.13);

\path[fill=fillColor,fill opacity=0.20] ( 87.33, 60.50) circle (  2.13);

\path[fill=fillColor,fill opacity=0.20] ( 98.36, 75.40) circle (  2.13);

\path[fill=fillColor,fill opacity=0.20] (102.37, 97.27) circle (  2.13);

\path[fill=fillColor,fill opacity=0.20] ( 91.34, 64.89) circle (  2.13);

\path[fill=fillColor,fill opacity=0.20] ( 75.29, 73.33) circle (  2.13);

\path[fill=fillColor,fill opacity=0.20] ( 77.29, 64.12) circle (  2.13);

\path[fill=fillColor,fill opacity=0.20] ( 78.30, 50.25) circle (  2.13);

\path[fill=fillColor,fill opacity=0.20] ( 68.26, 54.39) circle (  2.13);

\path[fill=fillColor,fill opacity=0.20] ( 67.26, 48.88) circle (  2.13);

\path[fill=fillColor,fill opacity=0.20] ( 73.28, 40.09) circle (  2.13);

\path[fill=fillColor,fill opacity=0.20] ( 69.27, 42.93) circle (  2.13);

\path[fill=fillColor,fill opacity=0.20] ( 75.29, 49.91) circle (  2.13);

\path[fill=fillColor,fill opacity=0.20] ( 84.32, 57.66) circle (  2.13);

\path[fill=fillColor,fill opacity=0.20] ( 91.34, 61.19) circle (  2.13);

\path[fill=fillColor,fill opacity=0.20] (113.41, 88.57) circle (  2.13);

\path[fill=fillColor,fill opacity=0.20] ( 83.31, 74.80) circle (  2.13);

\path[fill=fillColor,fill opacity=0.20] ( 67.66, 52.84) circle (  2.13);

\path[fill=fillColor,fill opacity=0.20] ( 74.28, 38.80) circle (  2.13);

\path[fill=fillColor,fill opacity=0.20] ( 72.28, 51.54) circle (  2.13);

\path[fill=fillColor,fill opacity=0.20] ( 72.28, 56.97) circle (  2.13);

\path[fill=fillColor,fill opacity=0.20] ( 77.29, 49.91) circle (  2.13);

\path[fill=fillColor,fill opacity=0.20] ( 82.31, 58.18) circle (  2.13);

\path[fill=fillColor,fill opacity=0.20] ( 89.33, 70.66) circle (  2.13);

\path[fill=fillColor,fill opacity=0.20] (104.38, 79.27) circle (  2.13);

\path[fill=fillColor,fill opacity=0.20] (117.42,115.10) circle (  2.13);

\path[fill=fillColor,fill opacity=0.20] ( 94.35, 92.02) circle (  2.13);

\path[fill=fillColor,fill opacity=0.20] ( 89.33, 72.90) circle (  2.13);

\path[fill=fillColor,fill opacity=0.20] ( 82.31, 74.88) circle (  2.13);

\path[fill=fillColor,fill opacity=0.20] ( 75.29, 70.66) circle (  2.13);

\path[fill=fillColor,fill opacity=0.20] ( 71.27, 60.76) circle (  2.13);

\path[fill=fillColor,fill opacity=0.20] ( 70.27, 47.41) circle (  2.13);

\path[fill=fillColor,fill opacity=0.20] ( 71.27, 40.09) circle (  2.13);

\path[fill=fillColor,fill opacity=0.20] ( 70.27, 46.46) circle (  2.13);

\path[fill=fillColor,fill opacity=0.20] ( 88.33, 51.20) circle (  2.13);

\path[fill=fillColor,fill opacity=0.20] ( 89.33, 61.53) circle (  2.13);

\path[fill=fillColor,fill opacity=0.20] ( 83.31, 72.13) circle (  2.13);

\path[fill=fillColor,fill opacity=0.20] ( 91.34, 93.22) circle (  2.13);

\path[fill=fillColor,fill opacity=0.20] ( 67.66, 52.23) circle (  2.13);

\path[fill=fillColor,fill opacity=0.20] ( 67.66, 55.33) circle (  2.13);

\path[fill=fillColor,fill opacity=0.20] ( 71.27, 58.52) circle (  2.13);

\path[fill=fillColor,fill opacity=0.20] ( 83.31, 48.27) circle (  2.13);

\path[fill=fillColor,fill opacity=0.20] ( 83.31, 54.99) circle (  2.13);

\path[fill=fillColor,fill opacity=0.20] ( 84.32, 68.51) circle (  2.13);

\path[fill=fillColor,fill opacity=0.20] ( 91.34, 68.77) circle (  2.13);

\path[fill=fillColor,fill opacity=0.20] ( 91.34, 74.80) circle (  2.13);

\path[fill=fillColor,fill opacity=0.20] (101.37,107.17) circle (  2.13);

\path[fill=fillColor,fill opacity=0.20] ( 78.30,105.45) circle (  2.13);

\path[fill=fillColor,fill opacity=0.20] ( 92.34, 79.70) circle (  2.13);

\path[fill=fillColor,fill opacity=0.20] ( 77.29, 80.65) circle (  2.13);

\path[fill=fillColor,fill opacity=0.20] ( 75.29, 82.98) circle (  2.13);

\path[fill=fillColor,fill opacity=0.20] ( 73.28, 65.75) circle (  2.13);

\path[fill=fillColor,fill opacity=0.20] ( 72.28, 50.68) circle (  2.13);

\path[fill=fillColor,fill opacity=0.20] ( 67.46, 43.11) circle (  2.13);

\path[fill=fillColor,fill opacity=0.20] ( 79.30, 55.16) circle (  2.13);

\path[fill=fillColor,fill opacity=0.20] ( 81.31, 45.78) circle (  2.13);

\path[fill=fillColor,fill opacity=0.20] ( 55.12, 60.67) circle (  2.13);

\path[fill=fillColor,fill opacity=0.20] ( 75.29, 53.53) circle (  2.13);

\path[fill=fillColor,fill opacity=0.20] ( 81.31, 57.06) circle (  2.13);

\path[fill=fillColor,fill opacity=0.20] ( 86.32, 59.98) circle (  2.13);

\path[fill=fillColor,fill opacity=0.20] ( 79.30, 62.14) circle (  2.13);

\path[fill=fillColor,fill opacity=0.20] ( 82.31, 63.34) circle (  2.13);

\path[fill=fillColor,fill opacity=0.20] ( 93.34, 66.27) circle (  2.13);

\path[fill=fillColor,fill opacity=0.20] ( 94.35, 98.22) circle (  2.13);

\path[fill=fillColor,fill opacity=0.20] ( 93.34,106.40) circle (  2.13);

\path[fill=fillColor,fill opacity=0.20] ( 88.33, 85.73) circle (  2.13);

\path[fill=fillColor,fill opacity=0.20] ( 76.29, 75.31) circle (  2.13);

\path[fill=fillColor,fill opacity=0.20] ( 75.29, 82.72) circle (  2.13);

\path[fill=fillColor,fill opacity=0.20] ( 71.27, 80.13) circle (  2.13);

\path[fill=fillColor,fill opacity=0.20] ( 66.76, 62.57) circle (  2.13);

\path[fill=fillColor,fill opacity=0.20] ( 58.53, 51.03) circle (  2.13);

\path[fill=fillColor,fill opacity=0.20] ( 84.32, 43.28) circle (  2.13);

\path[fill=fillColor,fill opacity=0.20] ( 84.32, 70.23) circle (  2.13);

\path[fill=fillColor,fill opacity=0.20] ( 77.29, 57.40) circle (  2.13);

\path[fill=fillColor,fill opacity=0.20] ( 74.28, 53.09) circle (  2.13);

\path[fill=fillColor,fill opacity=0.20] ( 77.29, 58.43) circle (  2.13);

\path[fill=fillColor,fill opacity=0.20] ( 80.30, 61.71) circle (  2.13);

\path[fill=fillColor,fill opacity=0.20] ( 84.32, 66.96) circle (  2.13);

\path[fill=fillColor,fill opacity=0.20] ( 79.30, 72.90) circle (  2.13);

\path[fill=fillColor,fill opacity=0.20] ( 86.32, 66.70) circle (  2.13);

\path[fill=fillColor,fill opacity=0.20] ( 90.33, 67.99) circle (  2.13);

\path[fill=fillColor,fill opacity=0.20] (100.37, 91.59) circle (  2.13);

\path[fill=fillColor,fill opacity=0.20] ( 86.32,100.63) circle (  2.13);

\path[fill=fillColor,fill opacity=0.20] ( 89.33,101.23) circle (  2.13);

\path[fill=fillColor,fill opacity=0.20] ( 91.34, 96.32) circle (  2.13);

\path[fill=fillColor,fill opacity=0.20] ( 81.31, 83.06) circle (  2.13);

\path[fill=fillColor,fill opacity=0.20] ( 78.30, 75.66) circle (  2.13);

\path[fill=fillColor,fill opacity=0.20] ( 75.29, 73.68) circle (  2.13);

\path[fill=fillColor,fill opacity=0.20] ( 69.27, 73.93) circle (  2.13);

\path[fill=fillColor,fill opacity=0.20] ( 74.28, 66.18) circle (  2.13);

\path[fill=fillColor,fill opacity=0.20] ( 84.32, 57.14) circle (  2.13);

\path[fill=fillColor,fill opacity=0.20] ( 78.30, 54.39) circle (  2.13);

\path[fill=fillColor,fill opacity=0.20] ( 72.28, 59.98) circle (  2.13);

\path[fill=fillColor,fill opacity=0.20] ( 79.30, 66.27) circle (  2.13);

\path[fill=fillColor,fill opacity=0.20] ( 83.31, 73.16) circle (  2.13);

\path[fill=fillColor,fill opacity=0.20] ( 85.32, 71.70) circle (  2.13);

\path[fill=fillColor,fill opacity=0.20] ( 58.93, 58.43) circle (  2.13);

\path[fill=fillColor,fill opacity=0.20] ( 91.34, 60.16) circle (  2.13);

\path[fill=fillColor,fill opacity=0.20] ( 96.35, 77.98) circle (  2.13);

\path[fill=fillColor,fill opacity=0.20] (104.38, 95.12) circle (  2.13);

\path[fill=fillColor,fill opacity=0.20] (106.39,102.87) circle (  2.13);

\path[fill=fillColor,fill opacity=0.20] (121.43,114.24) circle (  2.13);

\path[fill=fillColor,fill opacity=0.20] ( 93.34,104.76) circle (  2.13);

\path[fill=fillColor,fill opacity=0.20] ( 89.33, 85.56) circle (  2.13);

\path[fill=fillColor,fill opacity=0.20] ( 85.32, 92.88) circle (  2.13);

\path[fill=fillColor,fill opacity=0.20] ( 84.32, 84.27) circle (  2.13);

\path[fill=fillColor,fill opacity=0.20] ( 77.29, 74.02) circle (  2.13);

\path[fill=fillColor,fill opacity=0.20] ( 71.27, 79.88) circle (  2.13);

\path[fill=fillColor,fill opacity=0.20] ( 65.15, 75.92) circle (  2.13);

\path[fill=fillColor,fill opacity=0.20] ( 82.31, 57.49) circle (  2.13);

\path[fill=fillColor,fill opacity=0.20] ( 90.33, 70.06) circle (  2.13);

\path[fill=fillColor,fill opacity=0.20] ( 82.31, 52.32) circle (  2.13);

\path[fill=fillColor,fill opacity=0.20] ( 83.31, 48.10) circle (  2.13);

\path[fill=fillColor,fill opacity=0.20] ( 85.32, 61.19) circle (  2.13);

\path[fill=fillColor,fill opacity=0.20] ( 85.32, 63.60) circle (  2.13);

\path[fill=fillColor,fill opacity=0.20] ( 90.33, 52.41) circle (  2.13);

\path[fill=fillColor,fill opacity=0.20] ( 94.35, 56.80) circle (  2.13);

\path[fill=fillColor,fill opacity=0.20] ( 86.32, 68.60) circle (  2.13);

\path[fill=fillColor,fill opacity=0.20] ( 91.34, 72.04) circle (  2.13);

\path[fill=fillColor,fill opacity=0.20] ( 99.36, 76.26) circle (  2.13);

\path[fill=fillColor,fill opacity=0.20] ( 98.36, 77.90) circle (  2.13);

\path[fill=fillColor,fill opacity=0.20] (107.39, 89.26) circle (  2.13);

\path[fill=fillColor,fill opacity=0.20] (101.37,114.24) circle (  2.13);

\path[fill=fillColor,fill opacity=0.20] ( 90.33,100.72) circle (  2.13);

\path[fill=fillColor,fill opacity=0.20] ( 88.33, 76.00) circle (  2.13);

\path[fill=fillColor,fill opacity=0.20] ( 79.30, 79.45) circle (  2.13);

\path[fill=fillColor,fill opacity=0.20] ( 81.31, 81.77) circle (  2.13);

\path[fill=fillColor,fill opacity=0.20] ( 71.27, 73.25) circle (  2.13);

\path[fill=fillColor,fill opacity=0.20] ( 65.25, 71.09) circle (  2.13);

\path[fill=fillColor,fill opacity=0.20] ( 72.28, 61.45) circle (  2.13);

\path[fill=fillColor,fill opacity=0.20] ( 81.31, 48.96) circle (  2.13);

\path[fill=fillColor,fill opacity=0.20] ( 79.30, 56.63) circle (  2.13);

\path[fill=fillColor,fill opacity=0.20] ( 82.31, 62.83) circle (  2.13);

\path[fill=fillColor,fill opacity=0.20] ( 87.33, 64.03) circle (  2.13);

\path[fill=fillColor,fill opacity=0.20] ( 82.31, 69.03) circle (  2.13);

\path[fill=fillColor,fill opacity=0.20] ( 91.34, 68.60) circle (  2.13);

\path[fill=fillColor,fill opacity=0.20] ( 97.36, 65.15) circle (  2.13);

\path[fill=fillColor,fill opacity=0.20] ( 95.35, 68.25) circle (  2.13);

\path[fill=fillColor,fill opacity=0.20] ( 96.35, 69.03) circle (  2.13);

\path[fill=fillColor,fill opacity=0.20] (104.38, 66.87) circle (  2.13);

\path[fill=fillColor,fill opacity=0.20] ( 98.36, 73.68) circle (  2.13);

\path[fill=fillColor,fill opacity=0.20] ( 96.35, 90.81) circle (  2.13);

\path[fill=fillColor,fill opacity=0.20] ( 94.35,108.81) circle (  2.13);

\path[fill=fillColor,fill opacity=0.20] ( 49.00,114.24) circle (  2.13);

\path[fill=fillColor,fill opacity=0.20] ( 92.34,111.65) circle (  2.13);

\path[fill=fillColor,fill opacity=0.20] ( 99.36, 99.94) circle (  2.13);

\path[fill=fillColor,fill opacity=0.20] ( 94.35,106.92) circle (  2.13);

\path[fill=fillColor,fill opacity=0.20] ( 86.32,108.72) circle (  2.13);

\path[fill=fillColor,fill opacity=0.20] ( 92.34, 86.08) circle (  2.13);

\path[fill=fillColor,fill opacity=0.20] ( 99.36, 59.12) circle (  2.13);

\path[fill=fillColor,fill opacity=0.20] (101.37, 62.14) circle (  2.13);

\path[fill=fillColor,fill opacity=0.20] ( 91.34, 68.94) circle (  2.13);

\path[fill=fillColor,fill opacity=0.20] ( 89.33, 64.89) circle (  2.13);

\path[fill=fillColor,fill opacity=0.20] ( 91.34, 72.82) circle (  2.13);

\path[fill=fillColor,fill opacity=0.20] ( 87.33, 79.88) circle (  2.13);

\path[fill=fillColor,fill opacity=0.20] ( 79.30, 64.38) circle (  2.13);

\path[fill=fillColor,fill opacity=0.20] ( 74.28, 48.88) circle (  2.13);

\path[fill=fillColor,fill opacity=0.20] ( 81.31, 46.89) circle (  2.13);

\path[fill=fillColor,fill opacity=0.20] ( 81.31, 63.43) circle (  2.13);

\path[fill=fillColor,fill opacity=0.20] ( 82.31, 66.01) circle (  2.13);

\path[fill=fillColor,fill opacity=0.20] ( 81.31, 59.38) circle (  2.13);

\path[fill=fillColor,fill opacity=0.20] ( 83.31, 63.26) circle (  2.13);

\path[fill=fillColor,fill opacity=0.20] ( 93.34, 66.36) circle (  2.13);

\path[fill=fillColor,fill opacity=0.20] ( 92.34, 60.59) circle (  2.13);

\path[fill=fillColor,fill opacity=0.20] ( 95.35, 64.72) circle (  2.13);

\path[fill=fillColor,fill opacity=0.20] (101.37, 63.77) circle (  2.13);

\path[fill=fillColor,fill opacity=0.20] (109.40, 53.96) circle (  2.13);

\path[fill=fillColor,fill opacity=0.20] ( 96.35, 57.14) circle (  2.13);

\path[fill=fillColor,fill opacity=0.20] ( 85.32, 63.00) circle (  2.13);

\path[fill=fillColor,fill opacity=0.20] ( 90.33, 62.65) circle (  2.13);

\path[fill=fillColor,fill opacity=0.20] ( 87.33, 65.75) circle (  2.13);

\path[fill=fillColor,fill opacity=0.20] ( 78.30, 71.52) circle (  2.13);

\path[fill=fillColor,fill opacity=0.20] ( 89.33, 74.62) circle (  2.13);

\path[fill=fillColor,fill opacity=0.20] ( 96.35, 73.07) circle (  2.13);

\path[fill=fillColor,fill opacity=0.20] (108.39, 67.65) circle (  2.13);

\path[fill=fillColor,fill opacity=0.20] (101.37, 66.27) circle (  2.13);

\path[fill=fillColor,fill opacity=0.20] ( 83.31, 73.42) circle (  2.13);

\path[fill=fillColor,fill opacity=0.20] ( 90.33, 76.35) circle (  2.13);

\path[fill=fillColor,fill opacity=0.20] ( 96.35, 73.25) circle (  2.13);

\path[fill=fillColor,fill opacity=0.20] ( 96.35, 73.68) circle (  2.13);

\path[fill=fillColor,fill opacity=0.20] ( 91.34, 73.85) circle (  2.13);

\path[fill=fillColor,fill opacity=0.20] ( 97.36, 68.42) circle (  2.13);

\path[fill=fillColor,fill opacity=0.20] (100.37, 62.31) circle (  2.13);

\path[fill=fillColor,fill opacity=0.20] ( 90.33, 65.15) circle (  2.13);

\path[fill=fillColor,fill opacity=0.20] ( 89.33, 67.99) circle (  2.13);

\path[fill=fillColor,fill opacity=0.20] (102.37, 63.34) circle (  2.13);

\path[fill=fillColor,fill opacity=0.20] (101.37, 61.36) circle (  2.13);

\path[fill=fillColor,fill opacity=0.20] (101.37, 60.50) circle (  2.13);

\path[fill=fillColor,fill opacity=0.20] ( 91.34, 59.47) circle (  2.13);

\path[fill=fillColor,fill opacity=0.20] ( 87.33, 68.51) circle (  2.13);

\path[fill=fillColor,fill opacity=0.20] ( 82.31, 73.68) circle (  2.13);

\path[fill=fillColor,fill opacity=0.20] ( 85.32, 55.68) circle (  2.13);

\path[fill=fillColor,fill opacity=0.20] ( 82.31, 37.94) circle (  2.13);

\path[fill=fillColor,fill opacity=0.20] ( 92.34, 65.67) circle (  2.13);

\path[fill=fillColor,fill opacity=0.20] ( 87.33, 56.11) circle (  2.13);

\path[fill=fillColor,fill opacity=0.20] ( 81.31, 58.43) circle (  2.13);

\path[fill=fillColor,fill opacity=0.20] ( 83.31, 60.33) circle (  2.13);

\path[fill=fillColor,fill opacity=0.20] ( 89.33, 63.77) circle (  2.13);

\path[fill=fillColor,fill opacity=0.20] ( 84.32, 66.01) circle (  2.13);

\path[fill=fillColor,fill opacity=0.20] ( 87.33, 57.57) circle (  2.13);

\path[fill=fillColor,fill opacity=0.20] ( 94.35, 53.78) circle (  2.13);

\path[fill=fillColor,fill opacity=0.20] ( 97.36, 63.86) circle (  2.13);

\path[fill=fillColor,fill opacity=0.20] ( 97.36, 66.44) circle (  2.13);

\path[fill=fillColor,fill opacity=0.20] (103.38, 64.46) circle (  2.13);

\path[fill=fillColor,fill opacity=0.20] ( 91.34, 63.34) circle (  2.13);

\path[fill=fillColor,fill opacity=0.20] ( 82.31, 61.02) circle (  2.13);

\path[fill=fillColor,fill opacity=0.20] ( 87.33, 70.23) circle (  2.13);

\path[fill=fillColor,fill opacity=0.20] ( 94.35, 79.79) circle (  2.13);

\path[fill=fillColor,fill opacity=0.20] ( 97.36, 70.66) circle (  2.13);

\path[fill=fillColor,fill opacity=0.20] ( 95.35, 61.96) circle (  2.13);

\path[fill=fillColor,fill opacity=0.20] ( 93.34, 66.10) circle (  2.13);

\path[fill=fillColor,fill opacity=0.20] ( 99.36, 63.17) circle (  2.13);

\path[fill=fillColor,fill opacity=0.20] ( 95.35, 56.02) circle (  2.13);

\path[fill=fillColor,fill opacity=0.20] ( 86.32, 60.93) circle (  2.13);

\path[fill=fillColor,fill opacity=0.20] ( 85.32, 65.93) circle (  2.13);

\path[fill=fillColor,fill opacity=0.20] ( 84.32, 64.55) circle (  2.13);

\path[fill=fillColor,fill opacity=0.20] ( 97.36, 68.68) circle (  2.13);

\path[fill=fillColor,fill opacity=0.20] ( 97.36, 75.83) circle (  2.13);

\path[fill=fillColor,fill opacity=0.20] ( 95.35, 76.60) circle (  2.13);

\path[fill=fillColor,fill opacity=0.20] ( 94.35, 74.45) circle (  2.13);

\path[fill=fillColor,fill opacity=0.20] ( 94.35, 73.33) circle (  2.13);

\path[fill=fillColor,fill opacity=0.20] ( 91.34, 63.86) circle (  2.13);

\path[fill=fillColor,fill opacity=0.20] ( 87.33, 56.54) circle (  2.13);

\path[fill=fillColor,fill opacity=0.20] ( 84.32, 62.91) circle (  2.13);

\path[fill=fillColor,fill opacity=0.20] ( 90.33, 59.38) circle (  2.13);

\path[fill=fillColor,fill opacity=0.20] ( 80.30, 63.26) circle (  2.13);

\path[fill=fillColor,fill opacity=0.20] ( 84.32, 61.10) circle (  2.13);

\path[fill=fillColor,fill opacity=0.20] ( 87.33, 54.13) circle (  2.13);

\path[fill=fillColor,fill opacity=0.20] ( 87.33, 52.15) circle (  2.13);

\path[fill=fillColor,fill opacity=0.20] ( 87.33, 57.75) circle (  2.13);

\path[fill=fillColor,fill opacity=0.20] ( 93.34, 64.46) circle (  2.13);

\path[fill=fillColor,fill opacity=0.20] ( 95.35, 66.36) circle (  2.13);

\path[fill=fillColor,fill opacity=0.20] ( 88.33, 63.34) circle (  2.13);

\path[fill=fillColor,fill opacity=0.20] ( 90.33, 57.49) circle (  2.13);

\path[fill=fillColor,fill opacity=0.20] ( 91.34, 63.00) circle (  2.13);

\path[fill=fillColor,fill opacity=0.20] ( 84.32, 73.59) circle (  2.13);

\path[fill=fillColor,fill opacity=0.20] ( 82.31, 71.44) circle (  2.13);

\path[fill=fillColor,fill opacity=0.20] ( 87.33, 67.56) circle (  2.13);

\path[fill=fillColor,fill opacity=0.20] ( 89.33, 66.36) circle (  2.13);

\path[fill=fillColor,fill opacity=0.20] ( 95.35, 60.50) circle (  2.13);

\path[fill=fillColor,fill opacity=0.20] ( 86.32, 55.59) circle (  2.13);

\path[fill=fillColor,fill opacity=0.20] ( 70.27, 62.57) circle (  2.13);

\path[fill=fillColor,fill opacity=0.20] ( 88.33, 69.89) circle (  2.13);

\path[fill=fillColor,fill opacity=0.20] ( 91.34, 65.93) circle (  2.13);

\path[fill=fillColor,fill opacity=0.20] ( 89.33, 61.88) circle (  2.13);

\path[fill=fillColor,fill opacity=0.20] ( 87.33, 64.81) circle (  2.13);

\path[fill=fillColor,fill opacity=0.20] ( 89.33, 65.32) circle (  2.13);

\path[fill=fillColor,fill opacity=0.20] ( 86.32, 63.26) circle (  2.13);

\path[fill=fillColor,fill opacity=0.20] ( 82.31, 58.52) circle (  2.13);

\path[fill=fillColor,fill opacity=0.20] ( 91.34, 53.53) circle (  2.13);

\path[fill=fillColor,fill opacity=0.20] ( 93.34, 58.52) circle (  2.13);

\path[fill=fillColor,fill opacity=0.20] (106.39, 57.75) circle (  2.13);

\path[fill=fillColor,fill opacity=0.20] ( 88.33, 59.38) circle (  2.13);

\path[fill=fillColor,fill opacity=0.20] ( 93.34, 60.07) circle (  2.13);

\path[fill=fillColor,fill opacity=0.20] ( 87.33, 54.73) circle (  2.13);

\path[fill=fillColor,fill opacity=0.20] ( 88.33, 53.18) circle (  2.13);

\path[fill=fillColor,fill opacity=0.20] ( 86.32, 49.82) circle (  2.13);

\path[fill=fillColor,fill opacity=0.20] ( 79.30, 41.38) circle (  2.13);

\path[fill=fillColor,fill opacity=0.20] ( 84.32, 41.73) circle (  2.13);

\path[fill=fillColor,fill opacity=0.20] ( 84.32, 49.56) circle (  2.13);

\path[fill=fillColor,fill opacity=0.20] ( 93.34, 54.73) circle (  2.13);

\path[fill=fillColor,fill opacity=0.20] ( 94.35, 54.21) circle (  2.13);

\path[fill=fillColor,fill opacity=0.20] ( 97.36, 52.41) circle (  2.13);

\path[fill=fillColor,fill opacity=0.20] ( 97.36, 54.82) circle (  2.13);

\path[fill=fillColor,fill opacity=0.20] ( 84.32, 54.04) circle (  2.13);

\path[fill=fillColor,fill opacity=0.20] ( 97.36, 54.04) circle (  2.13);

\path[fill=fillColor,fill opacity=0.20] ( 99.36, 56.37) circle (  2.13);

\path[fill=fillColor,fill opacity=0.20] ( 89.33, 52.32) circle (  2.13);

\path[fill=fillColor,fill opacity=0.20] ( 92.34, 45.34) circle (  2.13);

\path[fill=fillColor,fill opacity=0.20] ( 99.36, 45.43) circle (  2.13);

\path[fill=fillColor,fill opacity=0.20] (119.43,115.10) circle (  2.13);

\path[fill=fillColor,fill opacity=0.20] (127.45,105.28) circle (  2.13);

\path[fill=fillColor,fill opacity=0.20] (118.43,101.15) circle (  2.13);

\path[fill=fillColor,fill opacity=0.20] ( 71.27,103.39) circle (  2.13);

\path[fill=fillColor,fill opacity=0.20] (107.39, 98.82) circle (  2.13);

\path[fill=fillColor,fill opacity=0.20] (110.40, 93.31) circle (  2.13);

\path[fill=fillColor,fill opacity=0.20] ( 87.33,102.52) circle (  2.13);

\path[fill=fillColor,fill opacity=0.20] (123.44,108.12) circle (  2.13);

\path[fill=fillColor,fill opacity=0.20] (110.40,107.17) circle (  2.13);

\path[fill=fillColor,fill opacity=0.20] (126.45,115.10) circle (  2.13);

\path[fill=fillColor,fill opacity=0.20] (151.53,115.96) circle (  2.13);

\path[fill=fillColor,fill opacity=0.20] (107.39,106.31) circle (  2.13);

\path[fill=fillColor,fill opacity=0.20] ( 68.16,100.46) circle (  2.13);

\path[fill=fillColor,fill opacity=0.20] ( 82.31,100.72) circle (  2.13);

\path[fill=fillColor,fill opacity=0.20] ( 82.31, 92.71) circle (  2.13);

\path[fill=fillColor,fill opacity=0.20] ( 87.33, 78.76) circle (  2.13);

\path[fill=fillColor,fill opacity=0.20] ( 92.34, 72.99) circle (  2.13);

\path[fill=fillColor,fill opacity=0.20] ( 86.32, 72.64) circle (  2.13);

\path[fill=fillColor,fill opacity=0.20] ( 90.33, 72.64) circle (  2.13);

\path[fill=fillColor,fill opacity=0.20] ( 65.05,103.90) circle (  2.13);

\path[fill=fillColor,fill opacity=0.20] ( 74.28, 94.52) circle (  2.13);

\path[fill=fillColor,fill opacity=0.20] ( 68.16, 81.60) circle (  2.13);

\path[fill=fillColor,fill opacity=0.20] ( 65.25, 75.92) circle (  2.13);

\path[fill=fillColor,fill opacity=0.20] ( 62.75, 69.03) circle (  2.13);

\path[fill=fillColor,fill opacity=0.20] ( 77.29, 62.91) circle (  2.13);

\path[fill=fillColor,fill opacity=0.20] ( 81.31, 55.68) circle (  2.13);

\path[fill=fillColor,fill opacity=0.20] ( 85.32, 54.30) circle (  2.13);

\path[fill=fillColor,fill opacity=0.20] ( 83.31, 68.77) circle (  2.13);

\path[fill=fillColor,fill opacity=0.20] ( 83.31, 73.68) circle (  2.13);

\path[fill=fillColor,fill opacity=0.20] ( 96.35,110.88) circle (  2.13);

\path[fill=fillColor,fill opacity=0.20] ( 71.27, 87.97) circle (  2.13);

\path[fill=fillColor,fill opacity=0.20] ( 67.06, 79.02) circle (  2.13);

\path[fill=fillColor,fill opacity=0.20] ( 50.91, 65.24) circle (  2.13);

\path[fill=fillColor,fill opacity=0.20] ( 51.81, 59.98) circle (  2.13);

\path[fill=fillColor,fill opacity=0.20] ( 73.28, 57.49) circle (  2.13);

\path[fill=fillColor,fill opacity=0.20] ( 75.29, 63.60) circle (  2.13);

\path[fill=fillColor,fill opacity=0.20] ( 79.30, 71.61) circle (  2.13);

\path[fill=fillColor,fill opacity=0.20] ( 85.32, 61.45) circle (  2.13);

\path[fill=fillColor,fill opacity=0.20] ( 77.29, 53.96) circle (  2.13);

\path[fill=fillColor,fill opacity=0.20] ( 90.33, 69.80) circle (  2.13);

\path[fill=fillColor,fill opacity=0.20] ( 97.36, 47.93) circle (  2.13);

\path[fill=fillColor,fill opacity=0.20] ( 83.31, 94.00) circle (  2.13);

\path[fill=fillColor,fill opacity=0.20] ( 52.11, 69.80) circle (  2.13);

\path[fill=fillColor,fill opacity=0.20] ( 64.45, 64.63) circle (  2.13);

\path[fill=fillColor,fill opacity=0.20] ( 80.30, 64.46) circle (  2.13);

\path[fill=fillColor,fill opacity=0.20] ( 80.30, 67.13) circle (  2.13);

\path[fill=fillColor,fill opacity=0.20] ( 80.30, 62.31) circle (  2.13);

\path[fill=fillColor,fill opacity=0.20] ( 80.30, 62.65) circle (  2.13);

\path[fill=fillColor,fill opacity=0.20] ( 84.32, 76.52) circle (  2.13);

\path[fill=fillColor,fill opacity=0.20] ( 87.33, 74.80) circle (  2.13);

\path[fill=fillColor,fill opacity=0.20] ( 91.34, 62.65) circle (  2.13);

\path[fill=fillColor,fill opacity=0.20] ( 91.34, 60.67) circle (  2.13);

\path[fill=fillColor,fill opacity=0.20] ( 82.31, 52.06) circle (  2.13);

\path[fill=fillColor,fill opacity=0.20] ( 71.27, 51.98) circle (  2.13);

\path[fill=fillColor,fill opacity=0.20] ( 70.27, 87.80) circle (  2.13);

\path[fill=fillColor,fill opacity=0.20] ( 75.29, 59.47) circle (  2.13);

\path[fill=fillColor,fill opacity=0.20] ( 63.65, 59.64) circle (  2.13);

\path[fill=fillColor,fill opacity=0.20] ( 77.29, 64.29) circle (  2.13);

\path[fill=fillColor,fill opacity=0.20] ( 80.30, 69.28) circle (  2.13);

\path[fill=fillColor,fill opacity=0.20] ( 86.32, 64.12) circle (  2.13);

\path[fill=fillColor,fill opacity=0.20] ( 82.31, 60.50) circle (  2.13);

\path[fill=fillColor,fill opacity=0.20] ( 85.32, 82.03) circle (  2.13);

\path[fill=fillColor,fill opacity=0.20] ( 89.33, 75.23) circle (  2.13);

\path[fill=fillColor,fill opacity=0.20] ( 75.29, 47.58) circle (  2.13);

\path[fill=fillColor,fill opacity=0.20] ( 80.30, 40.44) circle (  2.13);

\path[fill=fillColor,fill opacity=0.20] ( 79.30, 50.08) circle (  2.13);

\path[fill=fillColor,fill opacity=0.20] ( 59.84, 53.09) circle (  2.13);

\path[fill=fillColor,fill opacity=0.20] ( 58.83, 57.23) circle (  2.13);

\path[fill=fillColor,fill opacity=0.20] ( 53.12, 53.53) circle (  2.13);

\path[fill=fillColor,fill opacity=0.20] ( 56.63, 40.26) circle (  2.13);

\path[fill=fillColor,fill opacity=0.20] ( 79.30, 89.87) circle (  2.13);

\path[fill=fillColor,fill opacity=0.20] ( 76.29, 54.47) circle (  2.13);

\path[fill=fillColor,fill opacity=0.20] ( 79.30, 55.25) circle (  2.13);

\path[fill=fillColor,fill opacity=0.20] ( 76.29, 65.41) circle (  2.13);

\path[fill=fillColor,fill opacity=0.20] ( 83.31, 58.95) circle (  2.13);

\path[fill=fillColor,fill opacity=0.20] ( 76.29, 55.68) circle (  2.13);

\path[fill=fillColor,fill opacity=0.20] ( 83.31, 58.86) circle (  2.13);

\path[fill=fillColor,fill opacity=0.20] ( 91.34, 82.55) circle (  2.13);

\path[fill=fillColor,fill opacity=0.20] (110.40, 87.63) circle (  2.13);

\path[fill=fillColor,fill opacity=0.20] ( 87.33, 47.24) circle (  2.13);

\path[fill=fillColor,fill opacity=0.20] ( 73.28, 54.73) circle (  2.13);

\path[fill=fillColor,fill opacity=0.20] ( 70.27, 55.42) circle (  2.13);

\path[fill=fillColor,fill opacity=0.20] ( 67.56, 49.39) circle (  2.13);

\path[fill=fillColor,fill opacity=0.20] ( 59.94, 56.11) circle (  2.13);

\path[fill=fillColor,fill opacity=0.20] ( 45.09, 64.63) circle (  2.13);

\path[fill=fillColor,fill opacity=0.20] ( 61.64, 51.89) circle (  2.13);

\path[fill=fillColor,fill opacity=0.20] ( 69.27, 93.91) circle (  2.13);

\path[fill=fillColor,fill opacity=0.20] ( 59.54, 50.60) circle (  2.13);

\path[fill=fillColor,fill opacity=0.20] ( 85.32, 54.21) circle (  2.13);

\path[fill=fillColor,fill opacity=0.20] ( 86.32, 63.08) circle (  2.13);

\path[fill=fillColor,fill opacity=0.20] ( 87.33, 49.31) circle (  2.13);

\path[fill=fillColor,fill opacity=0.20] ( 85.32, 48.79) circle (  2.13);

\path[fill=fillColor,fill opacity=0.20] ( 91.34, 63.86) circle (  2.13);

\path[fill=fillColor,fill opacity=0.20] ( 67.36, 68.08) circle (  2.13);

\path[fill=fillColor,fill opacity=0.20] ( 99.36, 67.05) circle (  2.13);

\path[fill=fillColor,fill opacity=0.20] ( 79.30, 52.06) circle (  2.13);

\path[fill=fillColor,fill opacity=0.20] ( 70.27, 51.98) circle (  2.13);

\path[fill=fillColor,fill opacity=0.20] ( 77.29, 46.38) circle (  2.13);

\path[fill=fillColor,fill opacity=0.20] ( 76.29, 43.11) circle (  2.13);

\path[fill=fillColor,fill opacity=0.20] ( 68.16, 52.23) circle (  2.13);

\path[fill=fillColor,fill opacity=0.20] ( 65.25, 56.54) circle (  2.13);

\path[fill=fillColor,fill opacity=0.20] ( 60.74, 38.80) circle (  2.13);

\path[fill=fillColor,fill opacity=0.20] ( 67.96, 53.35) circle (  2.13);

\path[fill=fillColor,fill opacity=0.20] ( 83.31, 56.02) circle (  2.13);

\path[fill=fillColor,fill opacity=0.20] ( 91.34, 60.67) circle (  2.13);

\path[fill=fillColor,fill opacity=0.20] ( 93.34, 46.89) circle (  2.13);

\path[fill=fillColor,fill opacity=0.20] ( 96.35, 46.81) circle (  2.13);

\path[fill=fillColor,fill opacity=0.20] ( 96.35, 61.45) circle (  2.13);

\path[fill=fillColor,fill opacity=0.20] ( 99.36, 66.53) circle (  2.13);

\path[fill=fillColor,fill opacity=0.20] (100.37, 56.97) circle (  2.13);

\path[fill=fillColor,fill opacity=0.20] (101.37, 55.16) circle (  2.13);

\path[fill=fillColor,fill opacity=0.20] ( 94.35, 77.29) circle (  2.13);

\path[fill=fillColor,fill opacity=0.20] ( 72.28, 40.35) circle (  2.13);

\path[fill=fillColor,fill opacity=0.20] ( 75.29, 49.65) circle (  2.13);

\path[fill=fillColor,fill opacity=0.20] ( 80.30, 41.56) circle (  2.13);

\path[fill=fillColor,fill opacity=0.20] ( 74.28, 47.33) circle (  2.13);

\path[fill=fillColor,fill opacity=0.20] ( 70.27, 56.80) circle (  2.13);

\path[fill=fillColor,fill opacity=0.20] ( 59.34, 45.26) circle (  2.13);

\path[fill=fillColor,fill opacity=0.20] ( 46.29, 41.21) circle (  2.13);

\path[fill=fillColor,fill opacity=0.20] ( 51.91, 42.33) circle (  2.13);

\path[fill=fillColor,fill opacity=0.20] ( 72.28, 72.47) circle (  2.13);

\path[fill=fillColor,fill opacity=0.20] ( 82.31, 62.14) circle (  2.13);

\path[fill=fillColor,fill opacity=0.20] ( 86.32, 63.26) circle (  2.13);

\path[fill=fillColor,fill opacity=0.20] ( 88.33, 57.92) circle (  2.13);

\path[fill=fillColor,fill opacity=0.20] ( 85.32, 56.20) circle (  2.13);

\path[fill=fillColor,fill opacity=0.20] ( 93.34, 58.52) circle (  2.13);

\path[fill=fillColor,fill opacity=0.20] ( 92.34, 58.18) circle (  2.13);

\path[fill=fillColor,fill opacity=0.20] ( 94.35, 53.01) circle (  2.13);

\path[fill=fillColor,fill opacity=0.20] ( 84.32, 57.75) circle (  2.13);

\path[fill=fillColor,fill opacity=0.20] (104.38, 70.15) circle (  2.13);

\path[fill=fillColor,fill opacity=0.20] (107.39, 96.67) circle (  2.13);

\path[fill=fillColor,fill opacity=0.20] ( 95.35, 63.08) circle (  2.13);

\path[fill=fillColor,fill opacity=0.20] ( 78.30, 52.23) circle (  2.13);

\path[fill=fillColor,fill opacity=0.20] ( 73.28, 46.46) circle (  2.13);

\path[fill=fillColor,fill opacity=0.20] ( 73.28, 41.21) circle (  2.13);

\path[fill=fillColor,fill opacity=0.20] ( 77.29, 46.64) circle (  2.13);

\path[fill=fillColor,fill opacity=0.20] ( 71.27, 44.05) circle (  2.13);

\path[fill=fillColor,fill opacity=0.20] ( 66.66, 40.69) circle (  2.13);

\path[fill=fillColor,fill opacity=0.20] ( 56.63, 43.28) circle (  2.13);

\path[fill=fillColor,fill opacity=0.20] ( 60.24, 48.53) circle (  2.13);

\path[fill=fillColor,fill opacity=0.20] ( 72.28, 96.75) circle (  2.13);

\path[fill=fillColor,fill opacity=0.20] ( 73.28, 69.89) circle (  2.13);

\path[fill=fillColor,fill opacity=0.20] ( 82.31, 63.95) circle (  2.13);

\path[fill=fillColor,fill opacity=0.20] ( 81.31, 69.11) circle (  2.13);

\path[fill=fillColor,fill opacity=0.20] ( 91.34, 70.66) circle (  2.13);

\path[fill=fillColor,fill opacity=0.20] ( 95.35, 64.81) circle (  2.13);

\path[fill=fillColor,fill opacity=0.20] ( 91.34, 53.27) circle (  2.13);

\path[fill=fillColor,fill opacity=0.20] ( 95.35, 56.11) circle (  2.13);

\path[fill=fillColor,fill opacity=0.20] (100.37, 72.56) circle (  2.13);

\path[fill=fillColor,fill opacity=0.20] ( 99.36, 77.03) circle (  2.13);

\path[fill=fillColor,fill opacity=0.20] (106.39, 78.84) circle (  2.13);

\path[fill=fillColor,fill opacity=0.20] (107.39, 98.74) circle (  2.13);

\path[fill=fillColor,fill opacity=0.20] ( 84.32, 50.86) circle (  2.13);

\path[fill=fillColor,fill opacity=0.20] ( 73.28, 44.57) circle (  2.13);

\path[fill=fillColor,fill opacity=0.20] ( 73.28, 43.11) circle (  2.13);

\path[fill=fillColor,fill opacity=0.20] ( 78.30, 48.96) circle (  2.13);

\path[fill=fillColor,fill opacity=0.20] ( 77.29, 44.66) circle (  2.13);

\path[fill=fillColor,fill opacity=0.20] ( 71.27, 41.64) circle (  2.13);

\path[fill=fillColor,fill opacity=0.20] ( 70.27, 51.63) circle (  2.13);

\path[fill=fillColor,fill opacity=0.20] ( 71.27, 45.17) circle (  2.13);

\path[fill=fillColor,fill opacity=0.20] ( 64.05, 37.94) circle (  2.13);

\path[fill=fillColor,fill opacity=0.20] ( 64.05, 51.98) circle (  2.13);

\path[fill=fillColor,fill opacity=0.20] ( 68.26, 53.53) circle (  2.13);

\path[fill=fillColor,fill opacity=0.20] ( 71.27, 83.67) circle (  2.13);

\path[fill=fillColor,fill opacity=0.20] ( 80.30, 62.48) circle (  2.13);

\path[fill=fillColor,fill opacity=0.20] ( 88.33, 61.88) circle (  2.13);

\path[fill=fillColor,fill opacity=0.20] ( 94.35, 69.20) circle (  2.13);

\path[fill=fillColor,fill opacity=0.20] ( 99.36, 62.14) circle (  2.13);

\path[fill=fillColor,fill opacity=0.20] ( 97.36, 53.61) circle (  2.13);

\path[fill=fillColor,fill opacity=0.20] ( 99.36, 64.81) circle (  2.13);

\path[fill=fillColor,fill opacity=0.20] ( 97.36, 76.78) circle (  2.13);

\path[fill=fillColor,fill opacity=0.20] ( 91.34, 76.95) circle (  2.13);

\path[fill=fillColor,fill opacity=0.20] ( 93.34, 76.86) circle (  2.13);

\path[fill=fillColor,fill opacity=0.20] (101.37, 80.91) circle (  2.13);

\path[fill=fillColor,fill opacity=0.20] (103.38, 91.59) circle (  2.13);

\path[fill=fillColor,fill opacity=0.20] ( 73.28,115.10) circle (  2.13);

\path[fill=fillColor,fill opacity=0.20] ( 88.33, 68.42) circle (  2.13);

\path[fill=fillColor,fill opacity=0.20] ( 70.27, 55.25) circle (  2.13);

\path[fill=fillColor,fill opacity=0.20] ( 76.29, 47.84) circle (  2.13);

\path[fill=fillColor,fill opacity=0.20] ( 76.29, 47.24) circle (  2.13);

\path[fill=fillColor,fill opacity=0.20] ( 76.29, 43.11) circle (  2.13);

\path[fill=fillColor,fill opacity=0.20] ( 65.86, 47.15) circle (  2.13);

\path[fill=fillColor,fill opacity=0.20] ( 60.64, 60.85) circle (  2.13);

\path[fill=fillColor,fill opacity=0.20] ( 72.28, 57.40) circle (  2.13);

\path[fill=fillColor,fill opacity=0.20] ( 72.28, 38.46) circle (  2.13);

\path[fill=fillColor,fill opacity=0.20] ( 64.65, 38.89) circle (  2.13);

\path[fill=fillColor,fill opacity=0.20] ( 68.26, 58.00) circle (  2.13);

\path[fill=fillColor,fill opacity=0.20] ( 64.25, 60.41) circle (  2.13);

\path[fill=fillColor,fill opacity=0.20] ( 77.29, 46.12) circle (  2.13);

\path[fill=fillColor,fill opacity=0.20] ( 58.43, 80.13) circle (  2.13);

\path[fill=fillColor,fill opacity=0.20] ( 90.33, 58.78) circle (  2.13);

\path[fill=fillColor,fill opacity=0.20] ( 92.34, 59.73) circle (  2.13);

\path[fill=fillColor,fill opacity=0.20] ( 90.33, 54.47) circle (  2.13);

\path[fill=fillColor,fill opacity=0.20] ( 90.33, 57.14) circle (  2.13);

\path[fill=fillColor,fill opacity=0.20] ( 97.36, 66.70) circle (  2.13);

\path[fill=fillColor,fill opacity=0.20] ( 92.34, 67.91) circle (  2.13);

\path[fill=fillColor,fill opacity=0.20] ( 76.29, 59.12) circle (  2.13);

\path[fill=fillColor,fill opacity=0.20] ( 91.34, 65.75) circle (  2.13);

\path[fill=fillColor,fill opacity=0.20] ( 98.36, 72.47) circle (  2.13);

\path[fill=fillColor,fill opacity=0.20] (103.38, 72.04) circle (  2.13);

\path[fill=fillColor,fill opacity=0.20] (102.37, 87.80) circle (  2.13);

\path[fill=fillColor,fill opacity=0.20] ( 82.31, 91.85) circle (  2.13);

\path[fill=fillColor,fill opacity=0.20] ( 85.32, 69.54) circle (  2.13);

\path[fill=fillColor,fill opacity=0.20] ( 78.30, 73.25) circle (  2.13);

\path[fill=fillColor,fill opacity=0.20] ( 76.29, 63.86) circle (  2.13);

\path[fill=fillColor,fill opacity=0.20] ( 72.28, 46.55) circle (  2.13);

\path[fill=fillColor,fill opacity=0.20] ( 69.27, 45.17) circle (  2.13);

\path[fill=fillColor,fill opacity=0.20] ( 66.66, 58.95) circle (  2.13);

\path[fill=fillColor,fill opacity=0.20] ( 69.27, 65.84) circle (  2.13);

\path[fill=fillColor,fill opacity=0.20] ( 71.27, 54.21) circle (  2.13);

\path[fill=fillColor,fill opacity=0.20] ( 70.27, 45.09) circle (  2.13);

\path[fill=fillColor,fill opacity=0.20] ( 68.16, 50.25) circle (  2.13);

\path[fill=fillColor,fill opacity=0.20] ( 67.36, 54.39) circle (  2.13);

\path[fill=fillColor,fill opacity=0.20] ( 71.27, 58.43) circle (  2.13);

\path[fill=fillColor,fill opacity=0.20] ( 67.86, 66.01) circle (  2.13);

\path[fill=fillColor,fill opacity=0.20] ( 62.24, 82.72) circle (  2.13);

\path[fill=fillColor,fill opacity=0.20] ( 86.32, 70.75) circle (  2.13);

\path[fill=fillColor,fill opacity=0.20] ( 89.33, 66.87) circle (  2.13);

\path[fill=fillColor,fill opacity=0.20] ( 89.33, 65.84) circle (  2.13);

\path[fill=fillColor,fill opacity=0.20] ( 92.34, 62.05) circle (  2.13);

\path[fill=fillColor,fill opacity=0.20] ( 86.32, 58.35) circle (  2.13);

\path[fill=fillColor,fill opacity=0.20] ( 82.31, 55.25) circle (  2.13);

\path[fill=fillColor,fill opacity=0.20] ( 81.31, 59.12) circle (  2.13);

\path[fill=fillColor,fill opacity=0.20] ( 83.31, 63.34) circle (  2.13);

\path[fill=fillColor,fill opacity=0.20] ( 97.36, 61.71) circle (  2.13);

\path[fill=fillColor,fill opacity=0.20] ( 97.36, 66.10) circle (  2.13);

\path[fill=fillColor,fill opacity=0.20] ( 99.36, 74.54) circle (  2.13);

\path[fill=fillColor,fill opacity=0.20] (109.40, 78.58) circle (  2.13);

\path[fill=fillColor,fill opacity=0.20] (117.42, 93.83) circle (  2.13);

\path[fill=fillColor,fill opacity=0.20] ( 81.31,107.17) circle (  2.13);

\path[fill=fillColor,fill opacity=0.20] ( 63.15, 96.58) circle (  2.13);

\path[fill=fillColor,fill opacity=0.20] ( 86.32,101.15) circle (  2.13);

\path[fill=fillColor,fill opacity=0.20] ( 77.29, 83.06) circle (  2.13);

\path[fill=fillColor,fill opacity=0.20] ( 61.34, 63.26) circle (  2.13);

\path[fill=fillColor,fill opacity=0.20] ( 73.28, 59.12) circle (  2.13);

\path[fill=fillColor,fill opacity=0.20] ( 77.29, 68.77) circle (  2.13);

\path[fill=fillColor,fill opacity=0.20] ( 75.29, 64.12) circle (  2.13);

\path[fill=fillColor,fill opacity=0.20] ( 77.29, 45.86) circle (  2.13);

\path[fill=fillColor,fill opacity=0.20] ( 73.28, 44.40) circle (  2.13);

\path[fill=fillColor,fill opacity=0.20] ( 73.28, 57.31) circle (  2.13);

\path[fill=fillColor,fill opacity=0.20] ( 69.27, 56.02) circle (  2.13);

\path[fill=fillColor,fill opacity=0.20] ( 70.27, 45.00) circle (  2.13);

\path[fill=fillColor,fill opacity=0.20] ( 71.27, 54.82) circle (  2.13);

\path[fill=fillColor,fill opacity=0.20] ( 66.76, 61.19) circle (  2.13);

\path[fill=fillColor,fill opacity=0.20] ( 70.27, 42.33) circle (  2.13);

\path[fill=fillColor,fill opacity=0.20] ( 77.29, 47.76) circle (  2.13);

\path[fill=fillColor,fill opacity=0.20] ( 64.15, 78.67) circle (  2.13);

\path[fill=fillColor,fill opacity=0.20] ( 82.31, 88.49) circle (  2.13);

\path[fill=fillColor,fill opacity=0.20] ( 82.31, 83.06) circle (  2.13);

\path[fill=fillColor,fill opacity=0.20] ( 83.31, 73.42) circle (  2.13);

\path[fill=fillColor,fill opacity=0.20] ( 89.33, 73.85) circle (  2.13);

\path[fill=fillColor,fill opacity=0.20] ( 86.32, 74.45) circle (  2.13);

\path[fill=fillColor,fill opacity=0.20] ( 99.36, 66.79) circle (  2.13);

\path[fill=fillColor,fill opacity=0.20] ( 92.34, 58.09) circle (  2.13);

\path[fill=fillColor,fill opacity=0.20] ( 95.35, 56.11) circle (  2.13);

\path[fill=fillColor,fill opacity=0.20] (100.37, 59.21) circle (  2.13);

\path[fill=fillColor,fill opacity=0.20] (105.38, 56.97) circle (  2.13);

\path[fill=fillColor,fill opacity=0.20] ( 97.36, 63.17) circle (  2.13);

\path[fill=fillColor,fill opacity=0.20] ( 96.35, 74.37) circle (  2.13);

\path[fill=fillColor,fill opacity=0.20] ( 74.28, 78.24) circle (  2.13);

\path[fill=fillColor,fill opacity=0.20] ( 94.35, 84.87) circle (  2.13);

\path[fill=fillColor,fill opacity=0.20] ( 97.36, 93.05) circle (  2.13);

\path[fill=fillColor,fill opacity=0.20] ( 58.13, 87.97) circle (  2.13);

\path[fill=fillColor,fill opacity=0.20] ( 87.33, 83.32) circle (  2.13);

\path[fill=fillColor,fill opacity=0.20] ( 83.31, 67.56) circle (  2.13);

\path[fill=fillColor,fill opacity=0.20] ( 86.32, 68.68) circle (  2.13);

\path[fill=fillColor,fill opacity=0.20] ( 81.31, 79.70) circle (  2.13);

\path[fill=fillColor,fill opacity=0.20] ( 80.30, 69.63) circle (  2.13);

\path[fill=fillColor,fill opacity=0.20] ( 80.30, 62.14) circle (  2.13);

\path[fill=fillColor,fill opacity=0.20] ( 75.29, 53.78) circle (  2.13);

\path[fill=fillColor,fill opacity=0.20] ( 79.30, 46.64) circle (  2.13);

\path[fill=fillColor,fill opacity=0.20] ( 77.29, 52.06) circle (  2.13);

\path[fill=fillColor,fill opacity=0.20] ( 88.33, 48.88) circle (  2.13);

\path[fill=fillColor,fill opacity=0.20] ( 79.30, 47.50) circle (  2.13);

\path[fill=fillColor,fill opacity=0.20] ( 78.30, 55.85) circle (  2.13);

\path[fill=fillColor,fill opacity=0.20] ( 73.28, 56.54) circle (  2.13);

\path[fill=fillColor,fill opacity=0.20] ( 72.28, 44.57) circle (  2.13);

\path[fill=fillColor,fill opacity=0.20] ( 70.27, 45.52) circle (  2.13);

\path[fill=fillColor,fill opacity=0.20] ( 69.27, 52.92) circle (  2.13);

\path[fill=fillColor,fill opacity=0.20] ( 69.27, 39.14) circle (  2.13);

\path[fill=fillColor,fill opacity=0.20] ( 67.06, 74.80) circle (  2.13);

\path[fill=fillColor,fill opacity=0.20] ( 87.33, 80.65) circle (  2.13);

\path[fill=fillColor,fill opacity=0.20] ( 96.35, 73.85) circle (  2.13);

\path[fill=fillColor,fill opacity=0.20] ( 97.36, 63.26) circle (  2.13);

\path[fill=fillColor,fill opacity=0.20] ( 97.36, 56.11) circle (  2.13);

\path[fill=fillColor,fill opacity=0.20] ( 96.35, 61.36) circle (  2.13);

\path[fill=fillColor,fill opacity=0.20] ( 92.34, 68.08) circle (  2.13);

\path[fill=fillColor,fill opacity=0.20] ( 87.33, 64.46) circle (  2.13);

\path[fill=fillColor,fill opacity=0.20] ( 92.34, 58.43) circle (  2.13);

\path[fill=fillColor,fill opacity=0.20] ( 99.36, 64.55) circle (  2.13);

\path[fill=fillColor,fill opacity=0.20] ( 78.30, 70.58) circle (  2.13);

\path[fill=fillColor,fill opacity=0.20] ( 95.35, 77.64) circle (  2.13);

\path[fill=fillColor,fill opacity=0.20] ( 97.36, 89.87) circle (  2.13);

\path[fill=fillColor,fill opacity=0.20] (117.42,103.13) circle (  2.13);

\path[fill=fillColor,fill opacity=0.20] ( 97.36, 79.36) circle (  2.13);

\path[fill=fillColor,fill opacity=0.20] ( 89.33, 74.54) circle (  2.13);

\path[fill=fillColor,fill opacity=0.20] ( 91.34, 56.88) circle (  2.13);

\path[fill=fillColor,fill opacity=0.20] ( 81.31, 52.15) circle (  2.13);

\path[fill=fillColor,fill opacity=0.20] ( 80.30, 71.78) circle (  2.13);

\path[fill=fillColor,fill opacity=0.20] ( 77.29, 77.47) circle (  2.13);

\path[fill=fillColor,fill opacity=0.20] ( 85.32, 65.75) circle (  2.13);

\path[fill=fillColor,fill opacity=0.20] ( 83.31, 63.00) circle (  2.13);

\path[fill=fillColor,fill opacity=0.20] ( 66.26, 67.73) circle (  2.13);

\path[fill=fillColor,fill opacity=0.20] ( 79.30, 52.75) circle (  2.13);

\path[fill=fillColor,fill opacity=0.20] ( 83.31, 41.47) circle (  2.13);

\path[fill=fillColor,fill opacity=0.20] ( 73.28, 49.91) circle (  2.13);

\path[fill=fillColor,fill opacity=0.20] ( 82.31, 59.47) circle (  2.13);

\path[fill=fillColor,fill opacity=0.20] ( 84.32, 60.85) circle (  2.13);

\path[fill=fillColor,fill opacity=0.20] ( 78.30, 55.33) circle (  2.13);

\path[fill=fillColor,fill opacity=0.20] ( 75.29, 55.16) circle (  2.13);

\path[fill=fillColor,fill opacity=0.20] ( 75.29, 51.46) circle (  2.13);

\path[fill=fillColor,fill opacity=0.20] ( 64.25, 40.87) circle (  2.13);

\path[fill=fillColor,fill opacity=0.20] ( 72.28, 43.71) circle (  2.13);

\path[fill=fillColor,fill opacity=0.20] ( 67.96, 47.24) circle (  2.13);

\path[fill=fillColor,fill opacity=0.20] ( 65.35, 40.52) circle (  2.13);

\path[fill=fillColor,fill opacity=0.20] ( 66.16, 74.80) circle (  2.13);

\path[fill=fillColor,fill opacity=0.20] ( 82.31, 91.93) circle (  2.13);

\path[fill=fillColor,fill opacity=0.20] ( 89.33, 84.87) circle (  2.13);

\path[fill=fillColor,fill opacity=0.20] ( 90.33, 74.71) circle (  2.13);

\path[fill=fillColor,fill opacity=0.20] ( 94.35, 72.73) circle (  2.13);

\path[fill=fillColor,fill opacity=0.20] ( 95.35, 70.40) circle (  2.13);

\path[fill=fillColor,fill opacity=0.20] ( 98.36, 63.86) circle (  2.13);

\path[fill=fillColor,fill opacity=0.20] ( 98.36, 60.59) circle (  2.13);

\path[fill=fillColor,fill opacity=0.20] ( 92.34, 66.79) circle (  2.13);

\path[fill=fillColor,fill opacity=0.20] ( 97.36, 63.77) circle (  2.13);

\path[fill=fillColor,fill opacity=0.20] ( 96.35, 67.22) circle (  2.13);

\path[fill=fillColor,fill opacity=0.20] ( 95.35, 75.48) circle (  2.13);

\path[fill=fillColor,fill opacity=0.20] ( 90.33, 75.92) circle (  2.13);

\path[fill=fillColor,fill opacity=0.20] ( 97.36, 74.02) circle (  2.13);

\path[fill=fillColor,fill opacity=0.20] ( 96.35, 86.25) circle (  2.13);

\path[fill=fillColor,fill opacity=0.20] (100.37,104.59) circle (  2.13);

\path[fill=fillColor,fill opacity=0.20] ( 87.33, 91.59) circle (  2.13);

\path[fill=fillColor,fill opacity=0.20] ( 92.34, 75.31) circle (  2.13);

\path[fill=fillColor,fill opacity=0.20] ( 94.35, 77.47) circle (  2.13);

\path[fill=fillColor,fill opacity=0.20] ( 86.32, 76.17) circle (  2.13);

\path[fill=fillColor,fill opacity=0.20] ( 98.36, 57.66) circle (  2.13);

\path[fill=fillColor,fill opacity=0.20] ( 92.34, 54.04) circle (  2.13);

\path[fill=fillColor,fill opacity=0.20] ( 88.33, 70.83) circle (  2.13);

\path[fill=fillColor,fill opacity=0.20] ( 71.27, 73.93) circle (  2.13);

\path[fill=fillColor,fill opacity=0.20] ( 87.33, 66.96) circle (  2.13);

\path[fill=fillColor,fill opacity=0.20] ( 85.32, 57.06) circle (  2.13);

\path[fill=fillColor,fill opacity=0.20] ( 84.32, 50.60) circle (  2.13);

\path[fill=fillColor,fill opacity=0.20] ( 83.31, 53.96) circle (  2.13);

\path[fill=fillColor,fill opacity=0.20] ( 83.31, 56.37) circle (  2.13);

\path[fill=fillColor,fill opacity=0.20] ( 78.30, 58.52) circle (  2.13);

\path[fill=fillColor,fill opacity=0.20] ( 72.28, 67.30) circle (  2.13);

\path[fill=fillColor,fill opacity=0.20] ( 79.30, 64.29) circle (  2.13);

\path[fill=fillColor,fill opacity=0.20] ( 80.30, 51.20) circle (  2.13);

\path[fill=fillColor,fill opacity=0.20] ( 74.28, 50.77) circle (  2.13);

\path[fill=fillColor,fill opacity=0.20] ( 74.28, 51.03) circle (  2.13);

\path[fill=fillColor,fill opacity=0.20] ( 75.29, 47.76) circle (  2.13);

\path[fill=fillColor,fill opacity=0.20] ( 79.30, 51.20) circle (  2.13);

\path[fill=fillColor,fill opacity=0.20] ( 69.27, 50.60) circle (  2.13);

\path[fill=fillColor,fill opacity=0.20] ( 63.45, 48.88) circle (  2.13);

\path[fill=fillColor,fill opacity=0.20] ( 68.16, 83.15) circle (  2.13);

\path[fill=fillColor,fill opacity=0.20] ( 94.35, 75.48) circle (  2.13);

\path[fill=fillColor,fill opacity=0.20] ( 96.35, 79.45) circle (  2.13);

\path[fill=fillColor,fill opacity=0.20] ( 96.35, 74.02) circle (  2.13);

\path[fill=fillColor,fill opacity=0.20] ( 94.35, 71.78) circle (  2.13);

\path[fill=fillColor,fill opacity=0.20] ( 98.36, 70.75) circle (  2.13);

\path[fill=fillColor,fill opacity=0.20] (100.37, 71.01) circle (  2.13);

\path[fill=fillColor,fill opacity=0.20] ( 97.36, 69.71) circle (  2.13);

\path[fill=fillColor,fill opacity=0.20] ( 97.36, 69.11) circle (  2.13);

\path[fill=fillColor,fill opacity=0.20] (103.38, 68.51) circle (  2.13);

\path[fill=fillColor,fill opacity=0.20] (100.37, 65.32) circle (  2.13);

\path[fill=fillColor,fill opacity=0.20] ( 97.36, 64.63) circle (  2.13);

\path[fill=fillColor,fill opacity=0.20] ( 96.35, 68.60) circle (  2.13);

\path[fill=fillColor,fill opacity=0.20] ( 96.35, 87.37) circle (  2.13);

\path[fill=fillColor,fill opacity=0.20] ( 95.35,105.28) circle (  2.13);

\path[fill=fillColor,fill opacity=0.20] ( 91.34, 90.12) circle (  2.13);

\path[fill=fillColor,fill opacity=0.20] ( 86.32, 83.92) circle (  2.13);

\path[fill=fillColor,fill opacity=0.20] ( 83.31, 88.32) circle (  2.13);

\path[fill=fillColor,fill opacity=0.20] ( 90.33, 96.15) circle (  2.13);

\path[fill=fillColor,fill opacity=0.20] ( 89.33, 82.20) circle (  2.13);

\path[fill=fillColor,fill opacity=0.20] ( 97.36, 83.06) circle (  2.13);

\path[fill=fillColor,fill opacity=0.20] ( 94.35, 87.02) circle (  2.13);

\path[fill=fillColor,fill opacity=0.20] ( 92.34, 81.86) circle (  2.13);

\path[fill=fillColor,fill opacity=0.20] ( 92.34, 77.03) circle (  2.13);

\path[fill=fillColor,fill opacity=0.20] ( 89.33, 76.69) circle (  2.13);

\path[fill=fillColor,fill opacity=0.20] ( 88.33, 78.67) circle (  2.13);

\path[fill=fillColor,fill opacity=0.20] ( 92.34, 67.30) circle (  2.13);

\path[fill=fillColor,fill opacity=0.20] ( 88.33, 60.07) circle (  2.13);

\path[fill=fillColor,fill opacity=0.20] ( 90.33, 64.81) circle (  2.13);

\path[fill=fillColor,fill opacity=0.20] ( 88.33, 69.89) circle (  2.13);

\path[fill=fillColor,fill opacity=0.20] ( 81.31, 69.71) circle (  2.13);

\path[fill=fillColor,fill opacity=0.20] ( 80.30, 59.21) circle (  2.13);

\path[fill=fillColor,fill opacity=0.20] ( 83.31, 49.91) circle (  2.13);

\path[fill=fillColor,fill opacity=0.20] ( 87.33, 59.21) circle (  2.13);

\path[fill=fillColor,fill opacity=0.20] ( 82.31, 67.05) circle (  2.13);

\path[fill=fillColor,fill opacity=0.20] ( 77.29, 64.89) circle (  2.13);

\path[fill=fillColor,fill opacity=0.20] ( 73.28, 60.16) circle (  2.13);

\path[fill=fillColor,fill opacity=0.20] ( 78.30, 49.65) circle (  2.13);

\path[fill=fillColor,fill opacity=0.20] ( 86.32, 51.54) circle (  2.13);

\path[fill=fillColor,fill opacity=0.20] ( 79.30, 54.99) circle (  2.13);

\path[fill=fillColor,fill opacity=0.20] ( 80.30, 48.01) circle (  2.13);

\path[fill=fillColor,fill opacity=0.20] ( 83.31, 52.41) circle (  2.13);

\path[fill=fillColor,fill opacity=0.20] ( 84.32, 54.47) circle (  2.13);

\path[fill=fillColor,fill opacity=0.20] ( 68.26, 48.44) circle (  2.13);

\path[fill=fillColor,fill opacity=0.20] ( 68.26, 57.06) circle (  2.13);

\path[fill=fillColor,fill opacity=0.20] ( 83.31, 85.47) circle (  2.13);

\path[fill=fillColor,fill opacity=0.20] ( 72.28, 84.44) circle (  2.13);

\path[fill=fillColor,fill opacity=0.20] ( 94.35, 74.19) circle (  2.13);

\path[fill=fillColor,fill opacity=0.20] ( 94.35, 64.81) circle (  2.13);

\path[fill=fillColor,fill opacity=0.20] ( 99.36, 71.35) circle (  2.13);

\path[fill=fillColor,fill opacity=0.20] (105.38, 73.33) circle (  2.13);

\path[fill=fillColor,fill opacity=0.20] (107.39, 67.73) circle (  2.13);

\path[fill=fillColor,fill opacity=0.20] ( 98.36, 70.75) circle (  2.13);

\path[fill=fillColor,fill opacity=0.20] ( 90.33, 69.80) circle (  2.13);

\path[fill=fillColor,fill opacity=0.20] ( 88.33, 66.53) circle (  2.13);

\path[fill=fillColor,fill opacity=0.20] ( 90.33, 72.04) circle (  2.13);

\path[fill=fillColor,fill opacity=0.20] ( 95.35, 96.15) circle (  2.13);

\path[fill=fillColor,fill opacity=0.20] ( 91.34, 82.72) circle (  2.13);

\path[fill=fillColor,fill opacity=0.20] (105.38,112.51) circle (  2.13);

\path[fill=fillColor,fill opacity=0.20] ( 65.25,106.31) circle (  2.13);

\path[fill=fillColor,fill opacity=0.20] ( 92.34, 94.34) circle (  2.13);

\path[fill=fillColor,fill opacity=0.20] ( 81.31, 90.55) circle (  2.13);

\path[fill=fillColor,fill opacity=0.20] ( 88.33, 91.50) circle (  2.13);

\path[fill=fillColor,fill opacity=0.20] ( 95.35, 65.32) circle (  2.13);

\path[fill=fillColor,fill opacity=0.20] ( 85.32, 53.44) circle (  2.13);

\path[fill=fillColor,fill opacity=0.20] ( 82.31, 71.78) circle (  2.13);

\path[fill=fillColor,fill opacity=0.20] ( 91.34, 78.15) circle (  2.13);

\path[fill=fillColor,fill opacity=0.20] ( 91.34, 78.07) circle (  2.13);

\path[fill=fillColor,fill opacity=0.20] ( 83.31, 94.17) circle (  2.13);

\path[fill=fillColor,fill opacity=0.20] ( 56.73, 97.27) circle (  2.13);

\path[fill=fillColor,fill opacity=0.20] ( 89.33, 94.60) circle (  2.13);

\path[fill=fillColor,fill opacity=0.20] ( 80.30, 82.20) circle (  2.13);

\path[fill=fillColor,fill opacity=0.20] ( 82.31, 63.95) circle (  2.13);

\path[fill=fillColor,fill opacity=0.20] ( 82.31, 62.74) circle (  2.13);

\path[fill=fillColor,fill opacity=0.20] ( 82.31, 60.67) circle (  2.13);

\path[fill=fillColor,fill opacity=0.20] ( 78.30, 64.20) circle (  2.13);

\path[fill=fillColor,fill opacity=0.20] ( 79.30, 73.25) circle (  2.13);

\path[fill=fillColor,fill opacity=0.20] ( 76.29, 78.15) circle (  2.13);

\path[fill=fillColor,fill opacity=0.20] ( 86.32, 59.55) circle (  2.13);

\path[fill=fillColor,fill opacity=0.20] ( 81.31, 64.55) circle (  2.13);

\path[fill=fillColor,fill opacity=0.20] ( 78.30, 67.56) circle (  2.13);

\path[fill=fillColor,fill opacity=0.20] ( 81.31, 61.02) circle (  2.13);

\path[fill=fillColor,fill opacity=0.20] ( 80.30, 51.37) circle (  2.13);

\path[fill=fillColor,fill opacity=0.20] ( 90.33, 46.12) circle (  2.13);

\path[fill=fillColor,fill opacity=0.20] ( 84.32, 58.35) circle (  2.13);

\path[fill=fillColor,fill opacity=0.20] ( 81.31, 65.15) circle (  2.13);

\path[fill=fillColor,fill opacity=0.20] ( 87.33, 55.33) circle (  2.13);

\path[fill=fillColor,fill opacity=0.20] ( 91.34, 57.06) circle (  2.13);

\path[fill=fillColor,fill opacity=0.20] ( 84.32, 57.23) circle (  2.13);

\path[fill=fillColor,fill opacity=0.20] ( 86.32, 53.61) circle (  2.13);

\path[fill=fillColor,fill opacity=0.20] ( 81.31, 64.63) circle (  2.13);

\path[fill=fillColor,fill opacity=0.20] ( 90.33, 73.42) circle (  2.13);

\path[fill=fillColor,fill opacity=0.20] ( 98.36, 78.15) circle (  2.13);

\path[fill=fillColor,fill opacity=0.20] ( 94.35, 77.64) circle (  2.13);

\path[fill=fillColor,fill opacity=0.20] ( 94.35, 89.00) circle (  2.13);

\path[fill=fillColor,fill opacity=0.20] (104.38, 76.86) circle (  2.13);

\path[fill=fillColor,fill opacity=0.20] (100.37, 61.45) circle (  2.13);

\path[fill=fillColor,fill opacity=0.20] (100.37, 65.15) circle (  2.13);

\path[fill=fillColor,fill opacity=0.20] (102.37, 69.71) circle (  2.13);

\path[fill=fillColor,fill opacity=0.20] ( 99.36, 74.97) circle (  2.13);

\path[fill=fillColor,fill opacity=0.20] ( 92.34, 81.34) circle (  2.13);

\path[fill=fillColor,fill opacity=0.20] ( 90.33, 67.48) circle (  2.13);

\path[fill=fillColor,fill opacity=0.20] ( 97.36, 53.78) circle (  2.13);

\path[fill=fillColor,fill opacity=0.20] ( 89.33, 76.17) circle (  2.13);

\path[fill=fillColor,fill opacity=0.20] ( 81.31, 81.68) circle (  2.13);

\path[fill=fillColor,fill opacity=0.20] ( 87.33, 66.61) circle (  2.13);

\path[fill=fillColor,fill opacity=0.20] ( 86.32, 69.11) circle (  2.13);

\path[fill=fillColor,fill opacity=0.20] ( 87.33, 73.33) circle (  2.13);

\path[fill=fillColor,fill opacity=0.20] ( 86.32, 76.86) circle (  2.13);

\path[fill=fillColor,fill opacity=0.20] ( 92.34, 81.43) circle (  2.13);

\path[fill=fillColor,fill opacity=0.20] ( 81.31, 73.42) circle (  2.13);

\path[fill=fillColor,fill opacity=0.20] ( 76.29, 68.25) circle (  2.13);

\path[fill=fillColor,fill opacity=0.20] ( 48.40, 71.52) circle (  2.13);

\path[fill=fillColor,fill opacity=0.20] ( 81.31, 67.22) circle (  2.13);

\path[fill=fillColor,fill opacity=0.20] ( 87.33, 59.90) circle (  2.13);

\path[fill=fillColor,fill opacity=0.20] ( 64.65, 67.39) circle (  2.13);

\path[fill=fillColor,fill opacity=0.20] ( 88.33, 82.89) circle (  2.13);

\path[fill=fillColor,fill opacity=0.20] ( 84.32, 84.27) circle (  2.13);

\path[fill=fillColor,fill opacity=0.20] ( 73.28, 66.53) circle (  2.13);

\path[fill=fillColor,fill opacity=0.20] ( 83.31, 59.81) circle (  2.13);

\path[fill=fillColor,fill opacity=0.20] ( 86.32, 82.72) circle (  2.13);

\path[fill=fillColor,fill opacity=0.20] ( 82.31, 85.39) circle (  2.13);

\path[fill=fillColor,fill opacity=0.20] ( 76.29, 84.44) circle (  2.13);

\path[fill=fillColor,fill opacity=0.20] ( 81.31, 75.40) circle (  2.13);

\path[fill=fillColor,fill opacity=0.20] ( 78.30, 77.47) circle (  2.13);

\path[fill=fillColor,fill opacity=0.20] ( 70.27, 76.60) circle (  2.13);

\path[fill=fillColor,fill opacity=0.20] ( 76.29, 82.12) circle (  2.13);

\path[fill=fillColor,fill opacity=0.20] ( 82.31, 73.76) circle (  2.13);

\path[fill=fillColor,fill opacity=0.20] ( 80.30, 68.68) circle (  2.13);

\path[fill=fillColor,fill opacity=0.20] ( 79.30, 67.22) circle (  2.13);

\path[fill=fillColor,fill opacity=0.20] ( 86.32, 62.65) circle (  2.13);

\path[fill=fillColor,fill opacity=0.20] ( 80.30, 57.49) circle (  2.13);

\path[fill=fillColor,fill opacity=0.20] ( 82.31, 60.33) circle (  2.13);

\path[fill=fillColor,fill opacity=0.20] ( 85.32, 59.47) circle (  2.13);

\path[fill=fillColor,fill opacity=0.20] ( 87.33, 64.46) circle (  2.13);

\path[fill=fillColor,fill opacity=0.20] ( 84.32, 67.73) circle (  2.13);

\path[fill=fillColor,fill opacity=0.20] ( 90.33, 64.72) circle (  2.13);

\path[fill=fillColor,fill opacity=0.20] ( 92.34, 68.08) circle (  2.13);

\path[fill=fillColor,fill opacity=0.20] ( 86.32, 65.75) circle (  2.13);

\path[fill=fillColor,fill opacity=0.20] ( 85.32, 61.53) circle (  2.13);

\path[fill=fillColor,fill opacity=0.20] ( 75.29, 76.17) circle (  2.13);

\path[fill=fillColor,fill opacity=0.20] ( 89.33, 78.50) circle (  2.13);

\path[fill=fillColor,fill opacity=0.20] ( 88.33, 78.76) circle (  2.13);

\path[fill=fillColor,fill opacity=0.20] (101.37, 80.22) circle (  2.13);

\path[fill=fillColor,fill opacity=0.20] ( 91.34, 75.31) circle (  2.13);

\path[fill=fillColor,fill opacity=0.20] ( 99.36, 72.04) circle (  2.13);

\path[fill=fillColor,fill opacity=0.20] ( 94.35, 67.56) circle (  2.13);

\path[fill=fillColor,fill opacity=0.20] (101.37, 61.28) circle (  2.13);

\path[fill=fillColor,fill opacity=0.20] ( 98.36, 64.03) circle (  2.13);

\path[fill=fillColor,fill opacity=0.20] ( 94.35, 64.72) circle (  2.13);

\path[fill=fillColor,fill opacity=0.20] ( 87.33, 60.07) circle (  2.13);

\path[fill=fillColor,fill opacity=0.20] ( 92.34, 57.83) circle (  2.13);

\path[fill=fillColor,fill opacity=0.20] ( 80.30, 63.34) circle (  2.13);

\path[fill=fillColor,fill opacity=0.20] ( 88.33, 71.78) circle (  2.13);

\path[fill=fillColor,fill opacity=0.20] ( 93.34, 66.61) circle (  2.13);

\path[fill=fillColor,fill opacity=0.20] ( 87.33, 63.26) circle (  2.13);

\path[fill=fillColor,fill opacity=0.20] ( 85.32, 65.06) circle (  2.13);

\path[fill=fillColor,fill opacity=0.20] ( 86.32, 68.34) circle (  2.13);

\path[fill=fillColor,fill opacity=0.20] ( 79.30, 72.82) circle (  2.13);

\path[fill=fillColor,fill opacity=0.20] ( 53.22, 71.61) circle (  2.13);

\path[fill=fillColor,fill opacity=0.20] ( 78.30, 63.69) circle (  2.13);

\path[fill=fillColor,fill opacity=0.20] ( 78.30, 69.54) circle (  2.13);

\path[fill=fillColor,fill opacity=0.20] ( 72.28, 84.96) circle (  2.13);

\path[fill=fillColor,fill opacity=0.20] ( 76.29, 86.42) circle (  2.13);

\path[fill=fillColor,fill opacity=0.20] ( 74.28, 72.99) circle (  2.13);

\path[fill=fillColor,fill opacity=0.20] ( 83.31, 62.83) circle (  2.13);

\path[fill=fillColor,fill opacity=0.20] ( 82.31, 70.83) circle (  2.13);

\path[fill=fillColor,fill opacity=0.20] ( 77.29, 76.86) circle (  2.13);

\path[fill=fillColor,fill opacity=0.20] ( 84.32, 66.61) circle (  2.13);

\path[fill=fillColor,fill opacity=0.20] ( 85.32, 62.91) circle (  2.13);

\path[fill=fillColor,fill opacity=0.20] ( 85.32, 61.36) circle (  2.13);

\path[fill=fillColor,fill opacity=0.20] ( 87.33, 61.10) circle (  2.13);

\path[fill=fillColor,fill opacity=0.20] ( 88.33, 72.38) circle (  2.13);

\path[fill=fillColor,fill opacity=0.20] ( 81.31, 69.89) circle (  2.13);

\path[fill=fillColor,fill opacity=0.20] ( 70.27, 71.35) circle (  2.13);

\path[fill=fillColor,fill opacity=0.20] ( 94.35, 76.09) circle (  2.13);

\path[fill=fillColor,fill opacity=0.20] ( 88.33, 74.19) circle (  2.13);

\path[fill=fillColor,fill opacity=0.20] ( 88.33, 76.52) circle (  2.13);

\path[fill=fillColor,fill opacity=0.20] ( 89.33, 79.10) circle (  2.13);

\path[fill=fillColor,fill opacity=0.20] ( 95.35, 80.57) circle (  2.13);

\path[fill=fillColor,fill opacity=0.20] ( 96.35, 73.85) circle (  2.13);

\path[fill=fillColor,fill opacity=0.20] ( 99.36, 65.41) circle (  2.13);

\path[fill=fillColor,fill opacity=0.20] (102.37, 57.23) circle (  2.13);

\path[fill=fillColor,fill opacity=0.20] ( 98.36, 55.51) circle (  2.13);

\path[fill=fillColor,fill opacity=0.20] ( 99.36, 69.89) circle (  2.13);

\path[fill=fillColor,fill opacity=0.20] ( 94.35, 77.81) circle (  2.13);

\path[fill=fillColor,fill opacity=0.20] ( 87.33, 65.41) circle (  2.13);

\path[fill=fillColor,fill opacity=0.20] ( 84.32, 60.16) circle (  2.13);

\path[fill=fillColor,fill opacity=0.20] ( 88.33, 64.98) circle (  2.13);

\path[fill=fillColor,fill opacity=0.20] ( 70.27, 67.65) circle (  2.13);

\path[fill=fillColor,fill opacity=0.20] ( 84.32, 72.21) circle (  2.13);

\path[fill=fillColor,fill opacity=0.20] ( 81.31, 76.52) circle (  2.13);

\path[fill=fillColor,fill opacity=0.20] ( 80.30, 79.53) circle (  2.13);

\path[fill=fillColor,fill opacity=0.20] ( 72.28, 79.53) circle (  2.13);

\path[fill=fillColor,fill opacity=0.20] ( 69.27, 77.64) circle (  2.13);

\path[fill=fillColor,fill opacity=0.20] ( 88.33, 84.18) circle (  2.13);

\path[fill=fillColor,fill opacity=0.20] ( 77.29, 94.69) circle (  2.13);

\path[fill=fillColor,fill opacity=0.20] ( 66.66, 91.85) circle (  2.13);

\path[fill=fillColor,fill opacity=0.20] ( 86.32, 62.40) circle (  2.13);

\path[fill=fillColor,fill opacity=0.20] ( 68.26, 69.89) circle (  2.13);

\path[fill=fillColor,fill opacity=0.20] ( 88.33, 78.93) circle (  2.13);

\path[fill=fillColor,fill opacity=0.20] ( 82.31, 69.80) circle (  2.13);

\path[fill=fillColor,fill opacity=0.20] ( 91.34, 64.81) circle (  2.13);

\path[fill=fillColor,fill opacity=0.20] ( 84.32, 75.31) circle (  2.13);

\path[fill=fillColor,fill opacity=0.20] ( 87.33, 87.37) circle (  2.13);

\path[fill=fillColor,fill opacity=0.20] ( 71.27, 76.26) circle (  2.13);

\path[fill=fillColor,fill opacity=0.20] ( 85.32, 63.51) circle (  2.13);

\path[fill=fillColor,fill opacity=0.20] ( 79.30, 73.85) circle (  2.13);

\path[fill=fillColor,fill opacity=0.20] ( 74.28, 81.43) circle (  2.13);

\path[fill=fillColor,fill opacity=0.20] ( 77.29, 80.65) circle (  2.13);

\path[fill=fillColor,fill opacity=0.20] ( 77.29, 81.77) circle (  2.13);

\path[fill=fillColor,fill opacity=0.20] ( 80.30, 78.24) circle (  2.13);

\path[fill=fillColor,fill opacity=0.20] ( 72.28, 88.23) circle (  2.13);

\path[fill=fillColor,fill opacity=0.20] ( 72.28, 95.64) circle (  2.13);

\path[fill=fillColor,fill opacity=0.20] ( 80.30, 87.63) circle (  2.13);

\path[fill=fillColor,fill opacity=0.20] ( 76.29, 95.12) circle (  2.13);

\path[fill=fillColor,fill opacity=0.20] ( 76.29, 79.96) circle (  2.13);

\path[fill=fillColor,fill opacity=0.20] ( 74.28, 77.03) circle (  2.13);

\path[fill=fillColor,fill opacity=0.20] ( 82.31, 69.71) circle (  2.13);

\path[fill=fillColor,fill opacity=0.20] ( 79.30, 89.61) circle (  2.13);

\path[fill=fillColor,fill opacity=0.20] ( 71.27, 59.64) circle (  2.13);

\path[fill=fillColor,fill opacity=0.20] ( 58.23, 50.77) circle (  2.13);

\path[fill=fillColor,fill opacity=0.20] ( 75.29, 52.06) circle (  2.13);

\path[fill=fillColor,fill opacity=0.20] ( 76.29, 61.53) circle (  2.13);

\path[fill=fillColor,fill opacity=0.20] ( 75.29, 73.85) circle (  2.13);

\path[fill=fillColor,fill opacity=0.20] ( 95.35,115.10) circle (  2.13);

\path[fill=fillColor,fill opacity=0.20] ( 95.35,112.51) circle (  2.13);

\path[fill=fillColor,fill opacity=0.20] ( 98.36, 99.17) circle (  2.13);

\path[fill=fillColor,fill opacity=0.20] (100.37, 96.41) circle (  2.13);

\path[fill=fillColor,fill opacity=0.20] ( 88.33,107.35) circle (  2.13);

\path[fill=fillColor,fill opacity=0.20] ( 82.31,109.67) circle (  2.13);

\path[fill=fillColor,fill opacity=0.20] ( 71.27, 55.25) circle (  2.13);

\path[fill=fillColor,fill opacity=0.20] ( 63.95, 58.69) circle (  2.13);

\path[fill=fillColor,fill opacity=0.20] ( 54.52, 59.30) circle (  2.13);

\path[fill=fillColor,fill opacity=0.20] ( 67.16, 48.96) circle (  2.13);

\path[fill=fillColor,fill opacity=0.20] ( 74.28, 58.61) circle (  2.13);

\path[fill=fillColor,fill opacity=0.20] ( 57.13, 74.45) circle (  2.13);

\path[fill=fillColor,fill opacity=0.20] ( 89.33, 74.62) circle (  2.13);

\path[fill=fillColor,fill opacity=0.20] ( 89.33, 81.00) circle (  2.13);

\path[fill=fillColor,fill opacity=0.20] ( 89.33, 91.16) circle (  2.13);

\path[fill=fillColor,fill opacity=0.20] ( 95.35, 76.43) circle (  2.13);

\path[fill=fillColor,fill opacity=0.20] ( 93.34, 74.11) circle (  2.13);

\path[fill=fillColor,fill opacity=0.20] ( 91.34, 74.80) circle (  2.13);

\path[fill=fillColor,fill opacity=0.20] ( 87.33, 69.71) circle (  2.13);

\path[fill=fillColor,fill opacity=0.20] ( 85.32, 64.46) circle (  2.13);

\path[fill=fillColor,fill opacity=0.20] ( 86.32, 72.47) circle (  2.13);

\path[fill=fillColor,fill opacity=0.20] ( 95.35, 87.37) circle (  2.13);

\path[fill=fillColor,fill opacity=0.20] (108.39, 98.22) circle (  2.13);

\path[fill=fillColor,fill opacity=0.20] ( 74.28, 76.69) circle (  2.13);

\path[fill=fillColor,fill opacity=0.20] ( 67.46, 46.55) circle (  2.13);

\path[fill=fillColor,fill opacity=0.20] ( 58.93, 70.66) circle (  2.13);

\path[fill=fillColor,fill opacity=0.20] ( 61.04, 65.67) circle (  2.13);

\path[fill=fillColor,fill opacity=0.20] ( 54.02, 57.57) circle (  2.13);

\path[fill=fillColor,fill opacity=0.20] ( 76.29, 66.18) circle (  2.13);

\path[fill=fillColor,fill opacity=0.20] ( 78.30, 69.11) circle (  2.13);

\path[fill=fillColor,fill opacity=0.20] ( 77.29, 63.60) circle (  2.13);

\path[fill=fillColor,fill opacity=0.20] ( 82.31, 59.90) circle (  2.13);

\path[fill=fillColor,fill opacity=0.20] ( 92.34, 65.75) circle (  2.13);

\path[fill=fillColor,fill opacity=0.20] ( 92.34,101.49) circle (  2.13);

\path[fill=fillColor,fill opacity=0.20] ( 88.33, 54.90) circle (  2.13);

\path[fill=fillColor,fill opacity=0.20] ( 84.32, 41.90) circle (  2.13);

\path[fill=fillColor,fill opacity=0.20] ( 84.32, 49.82) circle (  2.13);

\path[fill=fillColor,fill opacity=0.20] ( 89.33, 60.85) circle (  2.13);

\path[fill=fillColor,fill opacity=0.20] ( 88.33, 64.46) circle (  2.13);

\path[fill=fillColor,fill opacity=0.20] ( 87.33, 59.98) circle (  2.13);

\path[fill=fillColor,fill opacity=0.20] ( 58.93, 59.73) circle (  2.13);

\path[fill=fillColor,fill opacity=0.20] (100.37, 65.58) circle (  2.13);

\path[fill=fillColor,fill opacity=0.20] (115.42, 71.95) circle (  2.13);

\path[fill=fillColor,fill opacity=0.20] ( 76.29, 74.37) circle (  2.13);

\path[fill=fillColor,fill opacity=0.20] ( 68.26, 52.32) circle (  2.13);

\path[fill=fillColor,fill opacity=0.20] ( 64.45, 73.25) circle (  2.13);

\path[fill=fillColor,fill opacity=0.20] ( 48.00, 65.84) circle (  2.13);

\path[fill=fillColor,fill opacity=0.20] ( 61.54, 73.68) circle (  2.13);

\path[fill=fillColor,fill opacity=0.20] ( 79.30, 58.78) circle (  2.13);

\path[fill=fillColor,fill opacity=0.20] ( 82.31, 48.27) circle (  2.13);

\path[fill=fillColor,fill opacity=0.20] ( 84.32, 65.75) circle (  2.13);

\path[fill=fillColor,fill opacity=0.20] ( 86.32, 72.99) circle (  2.13);

\path[fill=fillColor,fill opacity=0.20] ( 89.33, 70.58) circle (  2.13);

\path[fill=fillColor,fill opacity=0.20] ( 86.32, 95.72) circle (  2.13);

\path[fill=fillColor,fill opacity=0.20] ( 81.31, 48.96) circle (  2.13);

\path[fill=fillColor,fill opacity=0.20] ( 77.29, 39.23) circle (  2.13);

\path[fill=fillColor,fill opacity=0.20] ( 71.27, 45.78) circle (  2.13);

\path[fill=fillColor,fill opacity=0.20] ( 65.66, 48.88) circle (  2.13);

\path[fill=fillColor,fill opacity=0.20] ( 71.27, 50.77) circle (  2.13);

\path[fill=fillColor,fill opacity=0.20] ( 75.29, 50.77) circle (  2.13);

\path[fill=fillColor,fill opacity=0.20] ( 81.31, 49.74) circle (  2.13);

\path[fill=fillColor,fill opacity=0.20] ( 95.35, 60.24) circle (  2.13);

\path[fill=fillColor,fill opacity=0.20] (102.37, 72.21) circle (  2.13);

\path[fill=fillColor,fill opacity=0.20] ( 91.34, 66.61) circle (  2.13);

\path[fill=fillColor,fill opacity=0.20] (118.43, 74.45) circle (  2.13);

\path[fill=fillColor,fill opacity=0.20] ( 78.30, 80.82) circle (  2.13);

\path[fill=fillColor,fill opacity=0.20] ( 67.36, 49.91) circle (  2.13);

\path[fill=fillColor,fill opacity=0.20] ( 60.84, 66.01) circle (  2.13);

\path[fill=fillColor,fill opacity=0.20] ( 47.20, 58.61) circle (  2.13);

\path[fill=fillColor,fill opacity=0.20] ( 61.64, 56.71) circle (  2.13);

\path[fill=fillColor,fill opacity=0.20] ( 51.91, 63.51) circle (  2.13);

\path[fill=fillColor,fill opacity=0.20] ( 70.27, 49.05) circle (  2.13);

\path[fill=fillColor,fill opacity=0.20] ( 83.31, 42.68) circle (  2.13);

\path[fill=fillColor,fill opacity=0.20] ( 83.31, 64.63) circle (  2.13);

\path[fill=fillColor,fill opacity=0.20] ( 83.31, 69.28) circle (  2.13);

\path[fill=fillColor,fill opacity=0.20] ( 91.34, 66.79) circle (  2.13);

\path[fill=fillColor,fill opacity=0.20] ( 80.30, 56.97) circle (  2.13);

\path[fill=fillColor,fill opacity=0.20] ( 65.45, 40.95) circle (  2.13);

\path[fill=fillColor,fill opacity=0.20] ( 61.54, 46.64) circle (  2.13);

\path[fill=fillColor,fill opacity=0.20] ( 52.51, 49.91) circle (  2.13);

\path[fill=fillColor,fill opacity=0.20] ( 69.27, 43.97) circle (  2.13);

\path[fill=fillColor,fill opacity=0.20] ( 81.31, 38.37) circle (  2.13);

\path[fill=fillColor,fill opacity=0.20] ( 94.35, 51.54) circle (  2.13);

\path[fill=fillColor,fill opacity=0.20] (106.39, 68.94) circle (  2.13);

\path[fill=fillColor,fill opacity=0.20] ( 89.33, 70.92) circle (  2.13);

\path[fill=fillColor,fill opacity=0.20] ( 75.29, 41.81) circle (  2.13);

\path[fill=fillColor,fill opacity=0.20] ( 69.27, 57.06) circle (  2.13);

\path[fill=fillColor,fill opacity=0.20] ( 66.56, 52.92) circle (  2.13);

\path[fill=fillColor,fill opacity=0.20] ( 69.27, 43.54) circle (  2.13);

\path[fill=fillColor,fill opacity=0.20] ( 70.27, 47.67) circle (  2.13);

\path[fill=fillColor,fill opacity=0.20] ( 73.28, 48.19) circle (  2.13);

\path[fill=fillColor,fill opacity=0.20] ( 80.30, 51.37) circle (  2.13);

\path[fill=fillColor,fill opacity=0.20] ( 86.32, 56.45) circle (  2.13);

\path[fill=fillColor,fill opacity=0.20] ( 70.27, 54.64) circle (  2.13);

\path[fill=fillColor,fill opacity=0.20] ( 93.34, 93.91) circle (  2.13);

\path[fill=fillColor,fill opacity=0.20] ( 82.31, 41.56) circle (  2.13);

\path[fill=fillColor,fill opacity=0.20] ( 63.65, 38.54) circle (  2.13);

\path[fill=fillColor,fill opacity=0.20] ( 48.50, 48.62) circle (  2.13);

\path[fill=fillColor,fill opacity=0.20] ( 77.29, 55.33) circle (  2.13);

\path[fill=fillColor,fill opacity=0.20] ( 91.34, 56.71) circle (  2.13);

\path[fill=fillColor,fill opacity=0.20] ( 98.36, 51.46) circle (  2.13);

\path[fill=fillColor,fill opacity=0.20] (111.40, 57.75) circle (  2.13);

\path[fill=fillColor,fill opacity=0.20] ( 83.31, 65.24) circle (  2.13);

\path[fill=fillColor,fill opacity=0.20] ( 82.31, 52.49) circle (  2.13);

\path[fill=fillColor,fill opacity=0.20] ( 79.30, 60.41) circle (  2.13);

\path[fill=fillColor,fill opacity=0.20] ( 61.74, 56.88) circle (  2.13);

\path[fill=fillColor,fill opacity=0.20] ( 72.28, 47.24) circle (  2.13);

\path[fill=fillColor,fill opacity=0.20] ( 73.28, 49.99) circle (  2.13);

\path[fill=fillColor,fill opacity=0.20] ( 73.28, 56.11) circle (  2.13);

\path[fill=fillColor,fill opacity=0.20] ( 73.28, 52.84) circle (  2.13);

\path[fill=fillColor,fill opacity=0.20] ( 83.31, 50.25) circle (  2.13);

\path[fill=fillColor,fill opacity=0.20] ( 85.32, 53.18) circle (  2.13);

\path[fill=fillColor,fill opacity=0.20] ( 84.32, 74.62) circle (  2.13);

\path[fill=fillColor,fill opacity=0.20] ( 72.28, 40.69) circle (  2.13);

\path[fill=fillColor,fill opacity=0.20] ( 65.66, 46.38) circle (  2.13);

\path[fill=fillColor,fill opacity=0.20] ( 58.03, 39.83) circle (  2.13);

\path[fill=fillColor,fill opacity=0.20] ( 73.28, 45.52) circle (  2.13);

\path[fill=fillColor,fill opacity=0.20] ( 75.29, 55.16) circle (  2.13);

\path[fill=fillColor,fill opacity=0.20] ( 75.29, 63.43) circle (  2.13);

\path[fill=fillColor,fill opacity=0.20] ( 95.35, 67.48) circle (  2.13);

\path[fill=fillColor,fill opacity=0.20] (107.39, 64.03) circle (  2.13);

\path[fill=fillColor,fill opacity=0.20] ( 87.33, 69.37) circle (  2.13);

\path[fill=fillColor,fill opacity=0.20] ( 83.31, 62.48) circle (  2.13);

\path[fill=fillColor,fill opacity=0.20] ( 78.30, 57.57) circle (  2.13);

\path[fill=fillColor,fill opacity=0.20] ( 83.31, 58.26) circle (  2.13);

\path[fill=fillColor,fill opacity=0.20] ( 83.31, 56.37) circle (  2.13);

\path[fill=fillColor,fill opacity=0.20] ( 79.30, 56.20) circle (  2.13);

\path[fill=fillColor,fill opacity=0.20] ( 79.30, 57.06) circle (  2.13);

\path[fill=fillColor,fill opacity=0.20] ( 66.56, 52.32) circle (  2.13);

\path[fill=fillColor,fill opacity=0.20] ( 73.28, 47.84) circle (  2.13);

\path[fill=fillColor,fill opacity=0.20] ( 77.29, 52.06) circle (  2.13);

\path[fill=fillColor,fill opacity=0.20] ( 87.33, 57.92) circle (  2.13);

\path[fill=fillColor,fill opacity=0.20] ( 98.36, 65.58) circle (  2.13);

\path[fill=fillColor,fill opacity=0.20] (115.42, 76.60) circle (  2.13);

\path[fill=fillColor,fill opacity=0.20] ( 83.31, 69.63) circle (  2.13);

\path[fill=fillColor,fill opacity=0.20] ( 69.27, 43.54) circle (  2.13);

\path[fill=fillColor,fill opacity=0.20] ( 67.46, 54.82) circle (  2.13);

\path[fill=fillColor,fill opacity=0.20] ( 65.76, 58.09) circle (  2.13);

\path[fill=fillColor,fill opacity=0.20] ( 75.29, 60.07) circle (  2.13);

\path[fill=fillColor,fill opacity=0.20] ( 87.33, 55.85) circle (  2.13);

\path[fill=fillColor,fill opacity=0.20] ( 88.33, 54.56) circle (  2.13);

\path[fill=fillColor,fill opacity=0.20] ( 90.33, 61.71) circle (  2.13);

\path[fill=fillColor,fill opacity=0.20] ( 88.33, 70.32) circle (  2.13);

\path[fill=fillColor,fill opacity=0.20] ( 84.32, 48.53) circle (  2.13);

\path[fill=fillColor,fill opacity=0.20] ( 86.32, 49.65) circle (  2.13);

\path[fill=fillColor,fill opacity=0.20] ( 76.29, 53.61) circle (  2.13);

\path[fill=fillColor,fill opacity=0.20] ( 75.29, 61.96) circle (  2.13);

\path[fill=fillColor,fill opacity=0.20] ( 72.28, 56.71) circle (  2.13);

\path[fill=fillColor,fill opacity=0.20] ( 82.31, 45.60) circle (  2.13);

\path[fill=fillColor,fill opacity=0.20] ( 83.31, 48.10) circle (  2.13);

\path[fill=fillColor,fill opacity=0.20] ( 92.34, 51.98) circle (  2.13);

\path[fill=fillColor,fill opacity=0.20] ( 92.34, 55.08) circle (  2.13);

\path[fill=fillColor,fill opacity=0.20] ( 84.32, 78.93) circle (  2.13);

\path[fill=fillColor,fill opacity=0.20] ( 77.29, 45.60) circle (  2.13);

\path[fill=fillColor,fill opacity=0.20] ( 68.26, 45.00) circle (  2.13);

\path[fill=fillColor,fill opacity=0.20] ( 68.26, 49.99) circle (  2.13);

\path[fill=fillColor,fill opacity=0.20] ( 73.28, 64.55) circle (  2.13);

\path[fill=fillColor,fill opacity=0.20] ( 78.30, 68.85) circle (  2.13);

\path[fill=fillColor,fill opacity=0.20] ( 87.33, 58.43) circle (  2.13);

\path[fill=fillColor,fill opacity=0.20] ( 87.33, 59.38) circle (  2.13);

\path[fill=fillColor,fill opacity=0.20] ( 88.33, 65.50) circle (  2.13);

\path[fill=fillColor,fill opacity=0.20] (109.40, 76.09) circle (  2.13);

\path[fill=fillColor,fill opacity=0.20] ( 93.34,102.27) circle (  2.13);

\path[fill=fillColor,fill opacity=0.20] ( 86.32, 66.53) circle (  2.13);

\path[fill=fillColor,fill opacity=0.20] ( 82.31, 49.65) circle (  2.13);

\path[fill=fillColor,fill opacity=0.20] ( 79.30, 55.85) circle (  2.13);

\path[fill=fillColor,fill opacity=0.20] ( 80.30, 59.38) circle (  2.13);

\path[fill=fillColor,fill opacity=0.20] ( 79.30, 53.70) circle (  2.13);

\path[fill=fillColor,fill opacity=0.20] ( 83.31, 57.57) circle (  2.13);

\path[fill=fillColor,fill opacity=0.20] ( 79.30, 57.57) circle (  2.13);

\path[fill=fillColor,fill opacity=0.20] ( 78.30, 53.78) circle (  2.13);

\path[fill=fillColor,fill opacity=0.20] ( 86.32, 53.44) circle (  2.13);

\path[fill=fillColor,fill opacity=0.20] ( 93.34, 49.74) circle (  2.13);

\path[fill=fillColor,fill opacity=0.20] ( 67.86, 61.45) circle (  2.13);

\path[fill=fillColor,fill opacity=0.20] ( 70.27, 51.29) circle (  2.13);

\path[fill=fillColor,fill opacity=0.20] ( 66.86, 43.19) circle (  2.13);

\path[fill=fillColor,fill opacity=0.20] ( 80.30, 51.46) circle (  2.13);

\path[fill=fillColor,fill opacity=0.20] ( 85.32, 62.91) circle (  2.13);

\path[fill=fillColor,fill opacity=0.20] ( 89.33, 65.06) circle (  2.13);

\path[fill=fillColor,fill opacity=0.20] ( 89.33, 68.34) circle (  2.13);

\path[fill=fillColor,fill opacity=0.20] ( 91.34, 69.89) circle (  2.13);

\path[fill=fillColor,fill opacity=0.20] ( 97.36, 71.87) circle (  2.13);

\path[fill=fillColor,fill opacity=0.20] ( 82.31, 91.93) circle (  2.13);

\path[fill=fillColor,fill opacity=0.20] ( 84.32, 75.48) circle (  2.13);

\path[fill=fillColor,fill opacity=0.20] ( 77.29, 66.79) circle (  2.13);

\path[fill=fillColor,fill opacity=0.20] ( 71.27, 53.61) circle (  2.13);

\path[fill=fillColor,fill opacity=0.20] ( 77.29, 41.47) circle (  2.13);

\path[fill=fillColor,fill opacity=0.20] ( 80.30, 46.98) circle (  2.13);

\path[fill=fillColor,fill opacity=0.20] ( 72.28, 59.98) circle (  2.13);

\path[fill=fillColor,fill opacity=0.20] ( 77.29, 63.17) circle (  2.13);

\path[fill=fillColor,fill opacity=0.20] ( 81.31, 54.04) circle (  2.13);

\path[fill=fillColor,fill opacity=0.20] ( 89.33, 50.43) circle (  2.13);

\path[fill=fillColor,fill opacity=0.20] (104.38, 66.36) circle (  2.13);

\path[fill=fillColor,fill opacity=0.20] ( 82.31, 73.93) circle (  2.13);

\path[fill=fillColor,fill opacity=0.20] ( 67.76, 60.07) circle (  2.13);

\path[fill=fillColor,fill opacity=0.20] ( 57.93, 44.83) circle (  2.13);

\path[fill=fillColor,fill opacity=0.20] ( 81.31, 51.29) circle (  2.13);

\path[fill=fillColor,fill opacity=0.20] ( 90.33, 60.93) circle (  2.13);

\path[fill=fillColor,fill opacity=0.20] ( 98.36, 63.17) circle (  2.13);

\path[fill=fillColor,fill opacity=0.20] ( 99.36, 67.99) circle (  2.13);

\path[fill=fillColor,fill opacity=0.20] (103.38, 69.80) circle (  2.13);

\path[fill=fillColor,fill opacity=0.20] (122.44, 73.42) circle (  2.13);

\path[fill=fillColor,fill opacity=0.20] ( 76.29, 76.17) circle (  2.13);

\path[fill=fillColor,fill opacity=0.20] ( 74.28, 69.97) circle (  2.13);

\path[fill=fillColor,fill opacity=0.20] ( 72.28, 62.74) circle (  2.13);

\path[fill=fillColor,fill opacity=0.20] ( 65.15, 49.22) circle (  2.13);

\path[fill=fillColor,fill opacity=0.20] ( 74.28, 43.19) circle (  2.13);

\path[fill=fillColor,fill opacity=0.20] ( 70.27, 49.65) circle (  2.13);

\path[fill=fillColor,fill opacity=0.20] ( 77.29, 55.68) circle (  2.13);

\path[fill=fillColor,fill opacity=0.20] ( 84.32, 55.16) circle (  2.13);

\path[fill=fillColor,fill opacity=0.20] (112.41, 73.42) circle (  2.13);

\path[fill=fillColor,fill opacity=0.20] (131.47, 78.15) circle (  2.13);

\path[fill=fillColor,fill opacity=0.20] ( 90.33, 88.57) circle (  2.13);

\path[fill=fillColor,fill opacity=0.20] ( 77.29, 63.60) circle (  2.13);

\path[fill=fillColor,fill opacity=0.20] ( 81.31, 54.82) circle (  2.13);

\path[fill=fillColor,fill opacity=0.20] ( 81.31, 48.19) circle (  2.13);

\path[fill=fillColor,fill opacity=0.20] ( 85.32, 49.82) circle (  2.13);

\path[fill=fillColor,fill opacity=0.20] ( 90.33, 54.82) circle (  2.13);

\path[fill=fillColor,fill opacity=0.20] ( 92.34, 54.47) circle (  2.13);

\path[fill=fillColor,fill opacity=0.20] ( 85.32, 60.24) circle (  2.13);

\path[fill=fillColor,fill opacity=0.20] ( 95.35, 65.24) circle (  2.13);

\path[fill=fillColor,fill opacity=0.20] (108.39, 66.44) circle (  2.13);

\path[fill=fillColor,fill opacity=0.20] ( 79.30, 74.80) circle (  2.13);

\path[fill=fillColor,fill opacity=0.20] ( 73.28, 67.05) circle (  2.13);

\path[fill=fillColor,fill opacity=0.20] ( 72.28, 56.63) circle (  2.13);

\path[fill=fillColor,fill opacity=0.20] ( 71.27, 54.30) circle (  2.13);

\path[fill=fillColor,fill opacity=0.20] ( 63.45, 61.62) circle (  2.13);

\path[fill=fillColor,fill opacity=0.20] ( 65.66, 63.86) circle (  2.13);

\path[fill=fillColor,fill opacity=0.20] ( 78.30, 56.37) circle (  2.13);

\path[fill=fillColor,fill opacity=0.20] ( 83.31, 46.72) circle (  2.13);

\path[fill=fillColor,fill opacity=0.20] ( 87.33, 46.38) circle (  2.13);

\path[fill=fillColor,fill opacity=0.20] ( 81.31, 64.63) circle (  2.13);

\path[fill=fillColor,fill opacity=0.20] ( 83.31, 58.35) circle (  2.13);

\path[fill=fillColor,fill opacity=0.20] ( 88.33, 50.51) circle (  2.13);

\path[fill=fillColor,fill opacity=0.20] ( 88.33, 53.18) circle (  2.13);

\path[fill=fillColor,fill opacity=0.20] ( 89.33, 53.35) circle (  2.13);

\path[fill=fillColor,fill opacity=0.20] ( 93.34, 50.08) circle (  2.13);

\path[fill=fillColor,fill opacity=0.20] ( 98.36, 65.41) circle (  2.13);

\path[fill=fillColor,fill opacity=0.20] (104.38, 81.94) circle (  2.13);

\path[fill=fillColor,fill opacity=0.20] ( 91.34, 90.12) circle (  2.13);

\path[fill=fillColor,fill opacity=0.20] ( 81.31, 71.78) circle (  2.13);

\path[fill=fillColor,fill opacity=0.20] ( 85.32, 80.57) circle (  2.13);

\path[fill=fillColor,fill opacity=0.20] ( 83.31, 69.28) circle (  2.13);

\path[fill=fillColor,fill opacity=0.20] ( 75.29, 63.34) circle (  2.13);

\path[fill=fillColor,fill opacity=0.20] ( 71.27, 59.90) circle (  2.13);

\path[fill=fillColor,fill opacity=0.20] ( 69.27, 59.73) circle (  2.13);

\path[fill=fillColor,fill opacity=0.20] ( 71.27, 70.23) circle (  2.13);

\path[fill=fillColor,fill opacity=0.20] ( 59.64, 68.85) circle (  2.13);

\path[fill=fillColor,fill opacity=0.20] ( 82.31, 50.68) circle (  2.13);

\path[fill=fillColor,fill opacity=0.20] ( 92.34, 44.23) circle (  2.13);

\path[fill=fillColor,fill opacity=0.20] ( 85.32, 64.72) circle (  2.13);

\path[fill=fillColor,fill opacity=0.20] ( 95.35, 74.97) circle (  2.13);

\path[fill=fillColor,fill opacity=0.20] (105.38, 64.38) circle (  2.13);

\path[fill=fillColor,fill opacity=0.20] ( 75.29, 71.35) circle (  2.13);

\path[fill=fillColor,fill opacity=0.20] ( 82.31, 56.54) circle (  2.13);

\path[fill=fillColor,fill opacity=0.20] ( 87.33, 52.58) circle (  2.13);

\path[fill=fillColor,fill opacity=0.20] ( 92.34, 53.27) circle (  2.13);

\path[fill=fillColor,fill opacity=0.20] ( 91.34, 51.63) circle (  2.13);

\path[fill=fillColor,fill opacity=0.20] ( 97.36, 53.18) circle (  2.13);

\path[fill=fillColor,fill opacity=0.20] ( 85.32, 66.53) circle (  2.13);

\path[fill=fillColor,fill opacity=0.20] (101.37, 82.89) circle (  2.13);

\path[fill=fillColor,fill opacity=0.20] ( 67.16, 86.25) circle (  2.13);

\path[fill=fillColor,fill opacity=0.20] ( 89.33, 71.09) circle (  2.13);

\path[fill=fillColor,fill opacity=0.20] ( 89.33, 70.66) circle (  2.13);

\path[fill=fillColor,fill opacity=0.20] ( 84.32, 59.04) circle (  2.13);

\path[fill=fillColor,fill opacity=0.20] ( 83.31, 55.59) circle (  2.13);

\path[fill=fillColor,fill opacity=0.20] ( 80.30, 66.96) circle (  2.13);

\path[fill=fillColor,fill opacity=0.20] ( 77.29, 70.83) circle (  2.13);

\path[fill=fillColor,fill opacity=0.20] ( 72.28, 67.48) circle (  2.13);

\path[fill=fillColor,fill opacity=0.20] ( 81.31, 64.29) circle (  2.13);

\path[fill=fillColor,fill opacity=0.20] ( 86.32, 56.97) circle (  2.13);

\path[fill=fillColor,fill opacity=0.20] ( 88.33, 48.36) circle (  2.13);

\path[fill=fillColor,fill opacity=0.20] ( 96.35, 58.18) circle (  2.13);

\path[fill=fillColor,fill opacity=0.20] (115.42, 58.95) circle (  2.13);

\path[fill=fillColor,fill opacity=0.20] ( 81.31, 75.31) circle (  2.13);

\path[fill=fillColor,fill opacity=0.20] ( 77.29, 61.19) circle (  2.13);

\path[fill=fillColor,fill opacity=0.20] ( 82.31, 67.39) circle (  2.13);

\path[fill=fillColor,fill opacity=0.20] ( 89.33, 59.21) circle (  2.13);

\path[fill=fillColor,fill opacity=0.20] ( 93.34, 54.47) circle (  2.13);

\path[fill=fillColor,fill opacity=0.20] (100.37, 64.63) circle (  2.13);

\path[fill=fillColor,fill opacity=0.20] (100.37, 70.23) circle (  2.13);

\path[fill=fillColor,fill opacity=0.20] ( 95.35, 64.29) circle (  2.13);

\path[fill=fillColor,fill opacity=0.20] ( 90.33, 73.76) circle (  2.13);

\path[fill=fillColor,fill opacity=0.20] (116.42, 89.09) circle (  2.13);

\path[fill=fillColor,fill opacity=0.20] ( 89.33, 71.70) circle (  2.13);

\path[fill=fillColor,fill opacity=0.20] ( 86.32, 73.16) circle (  2.13);

\path[fill=fillColor,fill opacity=0.20] ( 87.33, 71.52) circle (  2.13);

\path[fill=fillColor,fill opacity=0.20] ( 85.32, 53.96) circle (  2.13);

\path[fill=fillColor,fill opacity=0.20] ( 79.30, 44.40) circle (  2.13);

\path[fill=fillColor,fill opacity=0.20] ( 74.28, 54.30) circle (  2.13);

\path[fill=fillColor,fill opacity=0.20] ( 82.31, 61.88) circle (  2.13);

\path[fill=fillColor,fill opacity=0.20] ( 77.29, 66.10) circle (  2.13);

\path[fill=fillColor,fill opacity=0.20] ( 80.30, 69.71) circle (  2.13);

\path[fill=fillColor,fill opacity=0.20] ( 87.33, 65.15) circle (  2.13);

\path[fill=fillColor,fill opacity=0.20] ( 98.36, 60.24) circle (  2.13);

\path[fill=fillColor,fill opacity=0.20] ( 92.34, 62.48) circle (  2.13);

\path[fill=fillColor,fill opacity=0.20] ( 96.35, 69.71) circle (  2.13);

\path[fill=fillColor,fill opacity=0.20] ( 63.65, 97.62) circle (  2.13);

\path[fill=fillColor,fill opacity=0.20] ( 82.31, 76.52) circle (  2.13);

\path[fill=fillColor,fill opacity=0.20] ( 80.30, 71.01) circle (  2.13);

\path[fill=fillColor,fill opacity=0.20] ( 88.33, 65.75) circle (  2.13);

\path[fill=fillColor,fill opacity=0.20] ( 94.35, 67.13) circle (  2.13);

\path[fill=fillColor,fill opacity=0.20] ( 88.33, 64.63) circle (  2.13);

\path[fill=fillColor,fill opacity=0.20] ( 95.35, 49.31) circle (  2.13);

\path[fill=fillColor,fill opacity=0.20] ( 97.36, 58.00) circle (  2.13);

\path[fill=fillColor,fill opacity=0.20] (103.38, 73.25) circle (  2.13);

\path[fill=fillColor,fill opacity=0.20] ( 96.35, 69.37) circle (  2.13);

\path[fill=fillColor,fill opacity=0.20] (104.38, 76.69) circle (  2.13);

\path[fill=fillColor,fill opacity=0.20] (118.43, 88.66) circle (  2.13);

\path[fill=fillColor,fill opacity=0.20] ( 86.32, 76.09) circle (  2.13);

\path[fill=fillColor,fill opacity=0.20] ( 83.31, 68.77) circle (  2.13);

\path[fill=fillColor,fill opacity=0.20] ( 81.31, 68.68) circle (  2.13);

\path[fill=fillColor,fill opacity=0.20] ( 79.30, 63.86) circle (  2.13);

\path[fill=fillColor,fill opacity=0.20] ( 81.31, 51.46) circle (  2.13);

\path[fill=fillColor,fill opacity=0.20] ( 77.29, 53.35) circle (  2.13);

\path[fill=fillColor,fill opacity=0.20] ( 79.30, 63.08) circle (  2.13);

\path[fill=fillColor,fill opacity=0.20] ( 75.29, 59.64) circle (  2.13);

\path[fill=fillColor,fill opacity=0.20] ( 82.31, 56.88) circle (  2.13);

\path[fill=fillColor,fill opacity=0.20] ( 84.32, 65.75) circle (  2.13);

\path[fill=fillColor,fill opacity=0.20] ( 88.33, 68.85) circle (  2.13);

\path[fill=fillColor,fill opacity=0.20] ( 95.35, 70.23) circle (  2.13);

\path[fill=fillColor,fill opacity=0.20] ( 97.36, 70.32) circle (  2.13);

\path[fill=fillColor,fill opacity=0.20] ( 98.36, 68.85) circle (  2.13);

\path[fill=fillColor,fill opacity=0.20] ( 66.36, 92.36) circle (  2.13);

\path[fill=fillColor,fill opacity=0.20] ( 80.30, 63.17) circle (  2.13);

\path[fill=fillColor,fill opacity=0.20] ( 77.29, 65.24) circle (  2.13);

\path[fill=fillColor,fill opacity=0.20] ( 84.32, 80.05) circle (  2.13);

\path[fill=fillColor,fill opacity=0.20] ( 78.30, 59.81) circle (  2.13);

\path[fill=fillColor,fill opacity=0.20] ( 85.32, 67.13) circle (  2.13);

\path[fill=fillColor,fill opacity=0.20] ( 91.34, 63.86) circle (  2.13);

\path[fill=fillColor,fill opacity=0.20] ( 92.34, 67.30) circle (  2.13);

\path[fill=fillColor,fill opacity=0.20] ( 92.34, 68.77) circle (  2.13);

\path[fill=fillColor,fill opacity=0.20] ( 99.36, 67.91) circle (  2.13);

\path[fill=fillColor,fill opacity=0.20] (103.38, 68.60) circle (  2.13);

\path[fill=fillColor,fill opacity=0.20] ( 99.36, 75.83) circle (  2.13);

\path[fill=fillColor,fill opacity=0.20] ( 83.31, 73.85) circle (  2.13);

\path[fill=fillColor,fill opacity=0.20] ( 82.31, 63.17) circle (  2.13);

\path[fill=fillColor,fill opacity=0.20] ( 77.29, 57.57) circle (  2.13);

\path[fill=fillColor,fill opacity=0.20] ( 75.29, 59.64) circle (  2.13);

\path[fill=fillColor,fill opacity=0.20] ( 77.29, 64.98) circle (  2.13);

\path[fill=fillColor,fill opacity=0.20] ( 76.29, 64.55) circle (  2.13);

\path[fill=fillColor,fill opacity=0.20] ( 77.29, 62.91) circle (  2.13);

\path[fill=fillColor,fill opacity=0.20] ( 82.31, 60.24) circle (  2.13);

\path[fill=fillColor,fill opacity=0.20] ( 83.31, 55.16) circle (  2.13);

\path[fill=fillColor,fill opacity=0.20] ( 88.33, 59.38) circle (  2.13);

\path[fill=fillColor,fill opacity=0.20] ( 97.36, 68.60) circle (  2.13);

\path[fill=fillColor,fill opacity=0.20] ( 92.34, 64.89) circle (  2.13);

\path[fill=fillColor,fill opacity=0.20] ( 94.35, 66.27) circle (  2.13);

\path[fill=fillColor,fill opacity=0.20] (102.37, 75.14) circle (  2.13);

\path[fill=fillColor,fill opacity=0.20] ( 69.27, 73.50) circle (  2.13);

\path[fill=fillColor,fill opacity=0.20] ( 77.29, 66.53) circle (  2.13);

\path[fill=fillColor,fill opacity=0.20] ( 88.33, 68.42) circle (  2.13);

\path[fill=fillColor,fill opacity=0.20] ( 83.31, 67.73) circle (  2.13);

\path[fill=fillColor,fill opacity=0.20] ( 83.31, 58.00) circle (  2.13);

\path[fill=fillColor,fill opacity=0.20] ( 81.31, 51.11) circle (  2.13);

\path[fill=fillColor,fill opacity=0.20] ( 87.33, 66.87) circle (  2.13);

\path[fill=fillColor,fill opacity=0.20] ( 84.32, 75.14) circle (  2.13);

\path[fill=fillColor,fill opacity=0.20] ( 98.36, 77.29) circle (  2.13);

\path[fill=fillColor,fill opacity=0.20] ( 94.35, 71.35) circle (  2.13);

\path[fill=fillColor,fill opacity=0.20] ( 96.35, 71.87) circle (  2.13);

\path[fill=fillColor,fill opacity=0.20] ( 88.33, 81.43) circle (  2.13);

\path[fill=fillColor,fill opacity=0.20] ( 84.32, 68.08) circle (  2.13);

\path[fill=fillColor,fill opacity=0.20] ( 85.32, 62.22) circle (  2.13);

\path[fill=fillColor,fill opacity=0.20] ( 65.96, 59.98) circle (  2.13);

\path[fill=fillColor,fill opacity=0.20] ( 76.29, 59.90) circle (  2.13);

\path[fill=fillColor,fill opacity=0.20] ( 74.28, 68.25) circle (  2.13);

\path[fill=fillColor,fill opacity=0.20] ( 74.28, 77.55) circle (  2.13);

\path[fill=fillColor,fill opacity=0.20] ( 79.30, 74.02) circle (  2.13);

\path[fill=fillColor,fill opacity=0.20] ( 75.29, 62.83) circle (  2.13);

\path[fill=fillColor,fill opacity=0.20] ( 88.33, 51.63) circle (  2.13);

\path[fill=fillColor,fill opacity=0.20] ( 88.33, 48.10) circle (  2.13);

\path[fill=fillColor,fill opacity=0.20] ( 97.36, 66.36) circle (  2.13);

\path[fill=fillColor,fill opacity=0.20] (102.37, 79.36) circle (  2.13);

\path[fill=fillColor,fill opacity=0.20] (109.40, 67.22) circle (  2.13);

\path[fill=fillColor,fill opacity=0.20] (126.45, 69.63) circle (  2.13);

\path[fill=fillColor,fill opacity=0.20] ( 71.27, 73.68) circle (  2.13);

\path[fill=fillColor,fill opacity=0.20] ( 85.32, 70.83) circle (  2.13);

\path[fill=fillColor,fill opacity=0.20] ( 84.32, 74.02) circle (  2.13);

\path[fill=fillColor,fill opacity=0.20] ( 83.31, 63.77) circle (  2.13);

\path[fill=fillColor,fill opacity=0.20] ( 81.31, 61.45) circle (  2.13);

\path[fill=fillColor,fill opacity=0.20] ( 87.33, 73.85) circle (  2.13);

\path[fill=fillColor,fill opacity=0.20] ( 86.32, 74.71) circle (  2.13);

\path[fill=fillColor,fill opacity=0.20] ( 93.34, 67.91) circle (  2.13);

\path[fill=fillColor,fill opacity=0.20] ( 93.34, 62.14) circle (  2.13);

\path[fill=fillColor,fill opacity=0.20] (101.37, 67.82) circle (  2.13);

\path[fill=fillColor,fill opacity=0.20] (104.38, 84.78) circle (  2.13);

\path[fill=fillColor,fill opacity=0.20] ( 98.36, 91.50) circle (  2.13);

\path[fill=fillColor,fill opacity=0.20] ( 99.36, 86.16) circle (  2.13);

\path[fill=fillColor,fill opacity=0.20] ( 93.34, 99.08) circle (  2.13);

\path[fill=fillColor,fill opacity=0.20] ( 79.30, 79.02) circle (  2.13);

\path[fill=fillColor,fill opacity=0.20] ( 80.30, 75.83) circle (  2.13);

\path[fill=fillColor,fill opacity=0.20] ( 81.31, 79.27) circle (  2.13);

\path[fill=fillColor,fill opacity=0.20] ( 82.31, 76.52) circle (  2.13);

\path[fill=fillColor,fill opacity=0.20] ( 78.30, 72.73) circle (  2.13);

\path[fill=fillColor,fill opacity=0.20] ( 82.31, 68.85) circle (  2.13);

\path[fill=fillColor,fill opacity=0.20] ( 76.29, 71.18) circle (  2.13);

\path[fill=fillColor,fill opacity=0.20] ( 80.30, 77.98) circle (  2.13);

\path[fill=fillColor,fill opacity=0.20] ( 88.33, 69.37) circle (  2.13);

\path[fill=fillColor,fill opacity=0.20] (102.37, 61.28) circle (  2.13);

\path[fill=fillColor,fill opacity=0.20] ( 97.36, 67.30) circle (  2.13);

\path[fill=fillColor,fill opacity=0.20] (101.37, 75.31) circle (  2.13);

\path[fill=fillColor,fill opacity=0.20] (101.37, 84.70) circle (  2.13);

\path[fill=fillColor,fill opacity=0.20] (139.49, 77.55) circle (  2.13);

\path[fill=fillColor,fill opacity=0.20] ( 88.33, 72.56) circle (  2.13);

\path[fill=fillColor,fill opacity=0.20] ( 74.28, 63.60) circle (  2.13);

\path[fill=fillColor,fill opacity=0.20] ( 79.30, 74.19) circle (  2.13);

\path[fill=fillColor,fill opacity=0.20] ( 87.33, 77.64) circle (  2.13);

\path[fill=fillColor,fill opacity=0.20] ( 79.30, 62.48) circle (  2.13);

\path[fill=fillColor,fill opacity=0.20] ( 75.29, 45.43) circle (  2.13);

\path[fill=fillColor,fill opacity=0.20] ( 87.33, 47.15) circle (  2.13);

\path[fill=fillColor,fill opacity=0.20] ( 95.35, 58.61) circle (  2.13);

\path[fill=fillColor,fill opacity=0.20] ( 88.33, 67.82) circle (  2.13);

\path[fill=fillColor,fill opacity=0.20] ( 94.35, 73.59) circle (  2.13);

\path[fill=fillColor,fill opacity=0.20] ( 91.34, 80.74) circle (  2.13);

\path[fill=fillColor,fill opacity=0.20] (100.37, 85.90) circle (  2.13);

\path[fill=fillColor,fill opacity=0.20] (106.39, 87.11) circle (  2.13);

\path[fill=fillColor,fill opacity=0.20] ( 88.33,101.66) circle (  2.13);

\path[fill=fillColor,fill opacity=0.20] ( 78.30, 85.82) circle (  2.13);

\path[fill=fillColor,fill opacity=0.20] ( 68.06, 75.66) circle (  2.13);

\path[fill=fillColor,fill opacity=0.20] ( 73.28, 76.86) circle (  2.13);

\path[fill=fillColor,fill opacity=0.20] ( 73.28, 71.78) circle (  2.13);

\path[fill=fillColor,fill opacity=0.20] ( 84.32, 71.95) circle (  2.13);

\path[fill=fillColor,fill opacity=0.20] ( 85.32, 77.03) circle (  2.13);

\path[fill=fillColor,fill opacity=0.20] ( 82.31, 67.30) circle (  2.13);

\path[fill=fillColor,fill opacity=0.20] ( 94.35, 64.03) circle (  2.13);

\path[fill=fillColor,fill opacity=0.20] (102.37, 72.38) circle (  2.13);

\path[fill=fillColor,fill opacity=0.20] (112.41, 70.32) circle (  2.13);

\path[fill=fillColor,fill opacity=0.20] (105.38, 75.57) circle (  2.13);

\path[fill=fillColor,fill opacity=0.20] (120.43, 95.20) circle (  2.13);

\path[fill=fillColor,fill opacity=0.20] (127.45,103.30) circle (  2.13);

\path[fill=fillColor,fill opacity=0.20] ( 46.39, 64.89) circle (  2.13);

\path[fill=fillColor,fill opacity=0.20] ( 73.28, 66.01) circle (  2.13);

\path[fill=fillColor,fill opacity=0.20] ( 73.28, 56.71) circle (  2.13);

\path[fill=fillColor,fill opacity=0.20] ( 72.28, 51.37) circle (  2.13);

\path[fill=fillColor,fill opacity=0.20] ( 82.31, 54.56) circle (  2.13);

\path[fill=fillColor,fill opacity=0.20] ( 85.32, 53.87) circle (  2.13);

\path[fill=fillColor,fill opacity=0.20] ( 88.33, 52.92) circle (  2.13);

\path[fill=fillColor,fill opacity=0.20] ( 96.35, 61.10) circle (  2.13);

\path[fill=fillColor,fill opacity=0.20] ( 91.34, 76.35) circle (  2.13);

\path[fill=fillColor,fill opacity=0.20] (102.37, 84.87) circle (  2.13);

\path[fill=fillColor,fill opacity=0.20] (108.39, 74.97) circle (  2.13);

\path[fill=fillColor,fill opacity=0.20] (110.40, 71.01) circle (  2.13);

\path[fill=fillColor,fill opacity=0.20] (108.39, 83.32) circle (  2.13);

\path[fill=fillColor,fill opacity=0.20] ( 71.27, 92.36) circle (  2.13);

\path[fill=fillColor,fill opacity=0.20] ( 84.32, 95.55) circle (  2.13);

\path[fill=fillColor,fill opacity=0.20] ( 88.33, 94.00) circle (  2.13);

\path[fill=fillColor,fill opacity=0.20] ( 77.29, 84.78) circle (  2.13);

\path[fill=fillColor,fill opacity=0.20] ( 81.31, 76.00) circle (  2.13);

\path[fill=fillColor,fill opacity=0.20] ( 84.32, 66.87) circle (  2.13);

\path[fill=fillColor,fill opacity=0.20] ( 67.46, 50.68) circle (  2.13);

\path[fill=fillColor,fill opacity=0.20] ( 89.33, 50.94) circle (  2.13);

\path[fill=fillColor,fill opacity=0.20] ( 66.06, 77.03) circle (  2.13);

\path[fill=fillColor,fill opacity=0.20] ( 99.36, 93.65) circle (  2.13);

\path[fill=fillColor,fill opacity=0.20] (111.40, 92.62) circle (  2.13);

\path[fill=fillColor,fill opacity=0.20] ( 78.30, 44.31) circle (  2.13);

\path[fill=fillColor,fill opacity=0.20] ( 75.29, 59.90) circle (  2.13);

\path[fill=fillColor,fill opacity=0.20] ( 77.29, 70.23) circle (  2.13);

\path[fill=fillColor,fill opacity=0.20] ( 81.31, 66.61) circle (  2.13);

\path[fill=fillColor,fill opacity=0.20] ( 90.33, 60.07) circle (  2.13);

\path[fill=fillColor,fill opacity=0.20] ( 89.33, 56.54) circle (  2.13);

\path[fill=fillColor,fill opacity=0.20] ( 86.32, 62.05) circle (  2.13);

\path[fill=fillColor,fill opacity=0.20] ( 96.35, 71.27) circle (  2.13);

\path[fill=fillColor,fill opacity=0.20] ( 96.35, 76.26) circle (  2.13);

\path[fill=fillColor,fill opacity=0.20] (103.38, 75.92) circle (  2.13);

\path[fill=fillColor,fill opacity=0.20] (109.40, 72.56) circle (  2.13);

\path[fill=fillColor,fill opacity=0.20] ( 96.35, 81.60) circle (  2.13);

\path[fill=fillColor,fill opacity=0.20] ( 88.33, 87.02) circle (  2.13);

\path[fill=fillColor,fill opacity=0.20] ( 85.32, 89.18) circle (  2.13);

\path[fill=fillColor,fill opacity=0.20] ( 89.33, 79.53) circle (  2.13);

\path[fill=fillColor,fill opacity=0.20] ( 91.34, 79.70) circle (  2.13);

\path[fill=fillColor,fill opacity=0.20] ( 99.36, 78.76) circle (  2.13);

\path[fill=fillColor,fill opacity=0.20] ( 95.35, 68.16) circle (  2.13);

\path[fill=fillColor,fill opacity=0.20] ( 94.35, 66.27) circle (  2.13);

\path[fill=fillColor,fill opacity=0.20] ( 97.36, 66.36) circle (  2.13);

\path[fill=fillColor,fill opacity=0.20] ( 83.31, 76.78) circle (  2.13);

\path[fill=fillColor,fill opacity=0.20] ( 98.36, 91.33) circle (  2.13);

\path[fill=fillColor,fill opacity=0.20] (116.42, 95.29) circle (  2.13);

\path[fill=fillColor,fill opacity=0.20] ( 55.92,101.58) circle (  2.13);

\path[fill=fillColor,fill opacity=0.20] ( 90.33, 63.86) circle (  2.13);

\path[fill=fillColor,fill opacity=0.20] ( 77.29, 62.83) circle (  2.13);

\path[fill=fillColor,fill opacity=0.20] ( 78.30, 64.29) circle (  2.13);

\path[fill=fillColor,fill opacity=0.20] ( 88.33, 67.30) circle (  2.13);

\path[fill=fillColor,fill opacity=0.20] ( 88.33, 63.00) circle (  2.13);

\path[fill=fillColor,fill opacity=0.20] ( 90.33, 58.26) circle (  2.13);

\path[fill=fillColor,fill opacity=0.20] ( 97.36, 64.89) circle (  2.13);

\path[fill=fillColor,fill opacity=0.20] ( 87.33, 72.21) circle (  2.13);

\path[fill=fillColor,fill opacity=0.20] ( 88.33, 76.26) circle (  2.13);

\path[fill=fillColor,fill opacity=0.20] ( 98.36, 81.17) circle (  2.13);

\path[fill=fillColor,fill opacity=0.20] (101.37, 81.60) circle (  2.13);

\path[fill=fillColor,fill opacity=0.20] (107.39, 68.42) circle (  2.13);

\path[fill=fillColor,fill opacity=0.20] (108.39, 59.21) circle (  2.13);

\path[fill=fillColor,fill opacity=0.20] (107.39, 71.70) circle (  2.13);

\path[fill=fillColor,fill opacity=0.20] (107.39, 90.21) circle (  2.13);

\path[fill=fillColor,fill opacity=0.20] ( 98.36, 99.25) circle (  2.13);

\path[fill=fillColor,fill opacity=0.20] ( 89.33, 85.99) circle (  2.13);

\path[fill=fillColor,fill opacity=0.20] ( 87.33, 88.06) circle (  2.13);

\path[fill=fillColor,fill opacity=0.20] ( 86.32, 86.68) circle (  2.13);

\path[fill=fillColor,fill opacity=0.20] ( 88.33, 66.96) circle (  2.13);

\path[fill=fillColor,fill opacity=0.20] ( 80.30, 67.22) circle (  2.13);

\path[fill=fillColor,fill opacity=0.20] ( 83.31, 74.37) circle (  2.13);

\path[fill=fillColor,fill opacity=0.20] ( 92.34, 75.83) circle (  2.13);

\path[fill=fillColor,fill opacity=0.20] (102.37, 75.48) circle (  2.13);

\path[fill=fillColor,fill opacity=0.20] ( 74.28, 69.37) circle (  2.13);

\path[fill=fillColor,fill opacity=0.20] ( 75.29, 70.15) circle (  2.13);

\path[fill=fillColor,fill opacity=0.20] ( 73.28, 65.06) circle (  2.13);

\path[fill=fillColor,fill opacity=0.20] ( 83.31, 71.01) circle (  2.13);

\path[fill=fillColor,fill opacity=0.20] ( 85.32, 66.61) circle (  2.13);

\path[fill=fillColor,fill opacity=0.20] ( 87.33, 55.76) circle (  2.13);

\path[fill=fillColor,fill opacity=0.20] ( 89.33, 58.43) circle (  2.13);

\path[fill=fillColor,fill opacity=0.20] ( 86.32, 69.11) circle (  2.13);

\path[fill=fillColor,fill opacity=0.20] ( 83.31, 73.85) circle (  2.13);

\path[fill=fillColor,fill opacity=0.20] ( 89.33, 76.60) circle (  2.13);

\path[fill=fillColor,fill opacity=0.20] ( 97.36, 72.30) circle (  2.13);

\path[fill=fillColor,fill opacity=0.20] (100.37, 72.13) circle (  2.13);

\path[fill=fillColor,fill opacity=0.20] ( 98.36, 74.11) circle (  2.13);

\path[fill=fillColor,fill opacity=0.20] ( 98.36, 70.32) circle (  2.13);

\path[fill=fillColor,fill opacity=0.20] ( 87.33, 67.05) circle (  2.13);

\path[fill=fillColor,fill opacity=0.20] (104.38, 68.85) circle (  2.13);

\path[fill=fillColor,fill opacity=0.20] (104.38, 69.11) circle (  2.13);

\path[fill=fillColor,fill opacity=0.20] ( 97.36, 72.56) circle (  2.13);

\path[fill=fillColor,fill opacity=0.20] ( 95.35, 81.86) circle (  2.13);

\path[fill=fillColor,fill opacity=0.20] (103.38, 88.23) circle (  2.13);

\path[fill=fillColor,fill opacity=0.20] (100.37, 92.45) circle (  2.13);

\path[fill=fillColor,fill opacity=0.20] (104.38, 90.47) circle (  2.13);

\path[fill=fillColor,fill opacity=0.20] (102.37, 78.33) circle (  2.13);

\path[fill=fillColor,fill opacity=0.20] (102.37, 73.25) circle (  2.13);

\path[fill=fillColor,fill opacity=0.20] ( 97.36, 81.08) circle (  2.13);

\path[fill=fillColor,fill opacity=0.20] ( 97.36, 83.06) circle (  2.13);

\path[fill=fillColor,fill opacity=0.20] ( 93.34, 79.70) circle (  2.13);

\path[fill=fillColor,fill opacity=0.20] ( 95.35, 80.48) circle (  2.13);

\path[fill=fillColor,fill opacity=0.20] ( 98.36, 80.91) circle (  2.13);

\path[fill=fillColor,fill opacity=0.20] ( 93.34, 75.31) circle (  2.13);

\path[fill=fillColor,fill opacity=0.20] ( 91.34, 68.16) circle (  2.13);

\path[fill=fillColor,fill opacity=0.20] ( 79.30, 57.31) circle (  2.13);

\path[fill=fillColor,fill opacity=0.20] ( 79.30, 53.70) circle (  2.13);

\path[fill=fillColor,fill opacity=0.20] ( 78.30, 69.63) circle (  2.13);

\path[fill=fillColor,fill opacity=0.20] ( 86.32, 82.12) circle (  2.13);

\path[fill=fillColor,fill opacity=0.20] ( 86.32, 68.94) circle (  2.13);

\path[fill=fillColor,fill opacity=0.20] ( 86.32, 52.41) circle (  2.13);

\path[fill=fillColor,fill opacity=0.20] ( 84.32, 59.30) circle (  2.13);

\path[fill=fillColor,fill opacity=0.20] ( 92.34, 68.51) circle (  2.13);

\path[fill=fillColor,fill opacity=0.20] ( 95.35, 77.72) circle (  2.13);

\path[fill=fillColor,fill opacity=0.20] ( 57.33, 99.68) circle (  2.13);

\path[fill=fillColor,fill opacity=0.20] ( 87.33, 78.76) circle (  2.13);

\path[fill=fillColor,fill opacity=0.20] ( 77.29, 81.00) circle (  2.13);

\path[fill=fillColor,fill opacity=0.20] ( 93.34, 77.64) circle (  2.13);

\path[fill=fillColor,fill opacity=0.20] ( 92.34, 67.56) circle (  2.13);

\path[fill=fillColor,fill opacity=0.20] ( 84.32, 65.58) circle (  2.13);

\path[fill=fillColor,fill opacity=0.20] ( 76.29, 67.22) circle (  2.13);

\path[fill=fillColor,fill opacity=0.20] ( 87.33, 62.65) circle (  2.13);

\path[fill=fillColor,fill opacity=0.20] ( 90.33, 65.41) circle (  2.13);

\path[fill=fillColor,fill opacity=0.20] ( 88.33, 79.88) circle (  2.13);

\path[fill=fillColor,fill opacity=0.20] ( 94.35, 86.16) circle (  2.13);

\path[fill=fillColor,fill opacity=0.20] ( 95.35, 78.24) circle (  2.13);

\path[fill=fillColor,fill opacity=0.20] ( 95.35, 73.07) circle (  2.13);

\path[fill=fillColor,fill opacity=0.20] ( 91.34, 66.44) circle (  2.13);

\path[fill=fillColor,fill opacity=0.20] ( 93.34, 79.45) circle (  2.13);

\path[fill=fillColor,fill opacity=0.20] ( 97.36, 75.83) circle (  2.13);

\path[fill=fillColor,fill opacity=0.20] (102.37, 65.32) circle (  2.13);

\path[fill=fillColor,fill opacity=0.20] ( 98.36, 62.48) circle (  2.13);

\path[fill=fillColor,fill opacity=0.20] ( 97.36, 72.56) circle (  2.13);

\path[fill=fillColor,fill opacity=0.20] ( 95.35, 81.34) circle (  2.13);

\path[fill=fillColor,fill opacity=0.20] ( 98.36, 74.54) circle (  2.13);

\path[fill=fillColor,fill opacity=0.20] ( 93.34, 68.25) circle (  2.13);

\path[fill=fillColor,fill opacity=0.20] ( 88.33, 75.23) circle (  2.13);

\path[fill=fillColor,fill opacity=0.20] ( 97.36, 72.90) circle (  2.13);

\path[fill=fillColor,fill opacity=0.20] ( 95.35, 60.41) circle (  2.13);

\path[fill=fillColor,fill opacity=0.20] ( 95.35, 60.93) circle (  2.13);

\path[fill=fillColor,fill opacity=0.20] ( 88.33, 70.06) circle (  2.13);

\path[fill=fillColor,fill opacity=0.20] ( 83.31, 68.42) circle (  2.13);

\path[fill=fillColor,fill opacity=0.20] ( 84.32, 62.14) circle (  2.13);

\path[fill=fillColor,fill opacity=0.20] ( 88.33, 68.51) circle (  2.13);

\path[fill=fillColor,fill opacity=0.20] ( 96.35, 76.00) circle (  2.13);

\path[fill=fillColor,fill opacity=0.20] ( 92.34, 66.61) circle (  2.13);

\path[fill=fillColor,fill opacity=0.20] ( 80.30, 63.26) circle (  2.13);

\path[fill=fillColor,fill opacity=0.20] ( 66.66, 78.15) circle (  2.13);

\path[fill=fillColor,fill opacity=0.20] (104.38, 95.64) circle (  2.13);

\path[fill=fillColor,fill opacity=0.20] ( 91.34, 73.59) circle (  2.13);

\path[fill=fillColor,fill opacity=0.20] ( 85.32, 62.91) circle (  2.13);

\path[fill=fillColor,fill opacity=0.20] ( 96.35, 52.92) circle (  2.13);

\path[fill=fillColor,fill opacity=0.20] ( 92.34, 59.12) circle (  2.13);

\path[fill=fillColor,fill opacity=0.20] ( 88.33, 77.98) circle (  2.13);

\path[fill=fillColor,fill opacity=0.20] ( 90.33, 79.62) circle (  2.13);

\path[fill=fillColor,fill opacity=0.20] ( 88.33, 71.95) circle (  2.13);

\path[fill=fillColor,fill opacity=0.20] ( 89.33, 75.92) circle (  2.13);

\path[fill=fillColor,fill opacity=0.20] ( 87.33, 72.13) circle (  2.13);

\path[fill=fillColor,fill opacity=0.20] ( 83.31, 72.47) circle (  2.13);

\path[fill=fillColor,fill opacity=0.20] ( 89.33, 83.58) circle (  2.13);

\path[fill=fillColor,fill opacity=0.20] ( 90.33, 74.11) circle (  2.13);

\path[fill=fillColor,fill opacity=0.20] ( 93.34, 60.67) circle (  2.13);

\path[fill=fillColor,fill opacity=0.20] ( 90.33, 68.77) circle (  2.13);

\path[fill=fillColor,fill opacity=0.20] ( 88.33, 77.29) circle (  2.13);

\path[fill=fillColor,fill opacity=0.20] ( 88.33, 73.25) circle (  2.13);

\path[fill=fillColor,fill opacity=0.20] ( 76.29, 82.80) circle (  2.13);

\path[fill=fillColor,fill opacity=0.20] ( 92.34, 76.17) circle (  2.13);

\path[fill=fillColor,fill opacity=0.20] ( 92.34, 55.68) circle (  2.13);

\path[fill=fillColor,fill opacity=0.20] ( 92.34, 54.99) circle (  2.13);

\path[fill=fillColor,fill opacity=0.20] ( 87.33, 67.30) circle (  2.13);

\path[fill=fillColor,fill opacity=0.20] ( 86.32, 71.95) circle (  2.13);

\path[fill=fillColor,fill opacity=0.20] ( 90.33, 69.80) circle (  2.13);

\path[fill=fillColor,fill opacity=0.20] ( 94.35, 71.52) circle (  2.13);

\path[fill=fillColor,fill opacity=0.20] ( 98.36, 82.29) circle (  2.13);

\path[fill=fillColor,fill opacity=0.20] (102.37, 94.09) circle (  2.13);

\path[fill=fillColor,fill opacity=0.20] (109.40,105.37) circle (  2.13);

\path[fill=fillColor,fill opacity=0.20] (117.42,115.96) circle (  2.13);

\path[fill=fillColor,fill opacity=0.20] ( 97.36, 86.77) circle (  2.13);

\path[fill=fillColor,fill opacity=0.20] ( 96.35, 70.06) circle (  2.13);

\path[fill=fillColor,fill opacity=0.20] ( 98.36, 58.78) circle (  2.13);

\path[fill=fillColor,fill opacity=0.20] ( 87.33, 63.26) circle (  2.13);

\path[fill=fillColor,fill opacity=0.20] ( 89.33, 64.72) circle (  2.13);

\path[fill=fillColor,fill opacity=0.20] (105.38, 62.14) circle (  2.13);

\path[fill=fillColor,fill opacity=0.20] ( 95.35, 68.34) circle (  2.13);

\path[fill=fillColor,fill opacity=0.20] ( 87.33, 63.43) circle (  2.13);

\path[fill=fillColor,fill opacity=0.20] ( 91.34, 54.56) circle (  2.13);

\path[fill=fillColor,fill opacity=0.20] ( 95.35, 61.79) circle (  2.13);

\path[fill=fillColor,fill opacity=0.20] ( 71.27, 64.72) circle (  2.13);

\path[fill=fillColor,fill opacity=0.20] ( 83.31, 54.82) circle (  2.13);

\path[fill=fillColor,fill opacity=0.20] ( 88.33, 58.26) circle (  2.13);

\path[fill=fillColor,fill opacity=0.20] ( 90.33, 73.07) circle (  2.13);

\path[fill=fillColor,fill opacity=0.20] ( 85.32, 78.58) circle (  2.13);

\path[fill=fillColor,fill opacity=0.20] ( 89.33, 72.82) circle (  2.13);

\path[fill=fillColor,fill opacity=0.20] ( 80.30, 69.37) circle (  2.13);

\path[fill=fillColor,fill opacity=0.20] ( 78.30, 71.01) circle (  2.13);

\path[fill=fillColor,fill opacity=0.20] ( 84.32, 71.61) circle (  2.13);

\path[fill=fillColor,fill opacity=0.20] ( 90.33, 63.60) circle (  2.13);

\path[fill=fillColor,fill opacity=0.20] ( 94.35, 55.42) circle (  2.13);

\path[fill=fillColor,fill opacity=0.20] ( 97.36, 55.68) circle (  2.13);

\path[fill=fillColor,fill opacity=0.20] ( 95.35, 61.36) circle (  2.13);

\path[fill=fillColor,fill opacity=0.20] ( 95.35, 68.51) circle (  2.13);

\path[fill=fillColor,fill opacity=0.20] ( 93.34, 73.50) circle (  2.13);

\path[fill=fillColor,fill opacity=0.20] ( 98.36, 86.77) circle (  2.13);

\path[fill=fillColor,fill opacity=0.20] (105.38, 84.70) circle (  2.13);

\path[fill=fillColor,fill opacity=0.20] (108.39, 77.64) circle (  2.13);

\path[fill=fillColor,fill opacity=0.20] ( 96.35, 74.97) circle (  2.13);

\path[fill=fillColor,fill opacity=0.20] (103.38, 69.54) circle (  2.13);

\path[fill=fillColor,fill opacity=0.20] (112.41, 62.40) circle (  2.13);

\path[fill=fillColor,fill opacity=0.20] (109.40, 63.00) circle (  2.13);

\path[fill=fillColor,fill opacity=0.20] ( 51.61, 59.55) circle (  2.13);

\path[fill=fillColor,fill opacity=0.20] ( 90.33, 44.14) circle (  2.13);

\path[fill=fillColor,fill opacity=0.20] ( 95.35, 49.22) circle (  2.13);

\path[fill=fillColor,fill opacity=0.20] ( 97.36, 66.18) circle (  2.13);

\path[fill=fillColor,fill opacity=0.20] ( 97.36, 62.91) circle (  2.13);

\path[fill=fillColor,fill opacity=0.20] ( 97.36, 56.11) circle (  2.13);

\path[fill=fillColor,fill opacity=0.20] ( 90.33, 63.69) circle (  2.13);

\path[fill=fillColor,fill opacity=0.20] ( 95.35, 62.91) circle (  2.13);

\path[fill=fillColor,fill opacity=0.20] ( 94.35, 47.58) circle (  2.13);

\path[fill=fillColor,fill opacity=0.20] ( 99.36, 46.21) circle (  2.13);

\path[fill=fillColor,fill opacity=0.20] ( 99.36, 65.32) circle (  2.13);

\path[fill=fillColor,fill opacity=0.20] ( 51.31, 77.98) circle (  2.13);

\path[fill=fillColor,fill opacity=0.20] ( 89.33, 80.31) circle (  2.13);

\path[fill=fillColor,fill opacity=0.20] (123.44,106.06) circle (  2.13);

\path[fill=fillColor,fill opacity=0.20] ( 98.36, 83.41) circle (  2.13);

\path[fill=fillColor,fill opacity=0.20] (101.37, 64.38) circle (  2.13);

\path[fill=fillColor,fill opacity=0.20] (100.37, 75.40) circle (  2.13);

\path[fill=fillColor,fill opacity=0.20] (102.37, 75.31) circle (  2.13);

\path[fill=fillColor,fill opacity=0.20] ( 99.36, 76.00) circle (  2.13);

\path[fill=fillColor,fill opacity=0.20] ( 89.33, 85.39) circle (  2.13);

\path[fill=fillColor,fill opacity=0.20] ( 74.28, 82.20) circle (  2.13);

\path[fill=fillColor,fill opacity=0.20] (109.40, 73.85) circle (  2.13);

\path[fill=fillColor,fill opacity=0.20] (121.43, 78.58) circle (  2.13);

\path[fill=fillColor,fill opacity=0.20] (117.42, 90.73) circle (  2.13);

\path[fill=fillColor,fill opacity=0.20] (109.40, 98.65) circle (  2.13);

\path[fill=fillColor,fill opacity=0.20] ( 69.27, 99.25) circle (  2.13);

\path[fill=fillColor,fill opacity=0.20] (120.43,101.23) circle (  2.13);

\path[fill=fillColor,fill opacity=0.20] ( 69.27, 91.33) circle (  2.13);

\path[fill=fillColor,fill opacity=0.20] ( 70.27, 85.73) circle (  2.13);

\path[fill=fillColor,fill opacity=0.20] ( 84.32, 81.08) circle (  2.13);

\path[fill=fillColor,fill opacity=0.20] ( 85.32, 84.01) circle (  2.13);

\path[fill=fillColor,fill opacity=0.20] ( 79.30,108.38) circle (  2.13);

\path[fill=fillColor,fill opacity=0.20] ( 71.27, 91.16) circle (  2.13);

\path[fill=fillColor,fill opacity=0.20] ( 73.28, 82.72) circle (  2.13);

\path[fill=fillColor,fill opacity=0.20] ( 78.30, 72.04) circle (  2.13);

\path[fill=fillColor,fill opacity=0.20] ( 79.30, 64.29) circle (  2.13);

\path[fill=fillColor,fill opacity=0.20] ( 78.30, 68.34) circle (  2.13);

\path[fill=fillColor,fill opacity=0.20] ( 89.33, 69.89) circle (  2.13);

\path[fill=fillColor,fill opacity=0.20] ( 90.33, 70.32) circle (  2.13);

\path[fill=fillColor,fill opacity=0.20] ( 84.32, 80.05) circle (  2.13);

\path[fill=fillColor,fill opacity=0.20] (106.39, 83.32) circle (  2.13);

\path[fill=fillColor,fill opacity=0.20] (110.40, 77.72) circle (  2.13);

\path[fill=fillColor,fill opacity=0.20] ( 83.31,102.52) circle (  2.13);

\path[fill=fillColor,fill opacity=0.20] ( 84.32, 77.90) circle (  2.13);

\path[fill=fillColor,fill opacity=0.20] ( 80.30, 69.89) circle (  2.13);

\path[fill=fillColor,fill opacity=0.20] ( 77.29, 67.99) circle (  2.13);

\path[fill=fillColor,fill opacity=0.20] ( 81.31, 59.30) circle (  2.13);

\path[fill=fillColor,fill opacity=0.20] ( 83.31, 55.76) circle (  2.13);

\path[fill=fillColor,fill opacity=0.20] ( 86.32, 56.71) circle (  2.13);

\path[fill=fillColor,fill opacity=0.20] ( 92.34, 60.41) circle (  2.13);

\path[fill=fillColor,fill opacity=0.20] ( 90.33, 62.91) circle (  2.13);

\path[fill=fillColor,fill opacity=0.20] ( 96.35, 62.48) circle (  2.13);

\path[fill=fillColor,fill opacity=0.20] ( 90.33, 62.05) circle (  2.13);

\path[fill=fillColor,fill opacity=0.20] ( 88.33, 73.42) circle (  2.13);

\path[fill=fillColor,fill opacity=0.20] ( 96.35, 93.40) circle (  2.13);

\path[fill=fillColor,fill opacity=0.20] ( 95.35,112.51) circle (  2.13);

\path[fill=fillColor,fill opacity=0.20] ( 85.32, 75.74) circle (  2.13);

\path[fill=fillColor,fill opacity=0.20] ( 78.30, 61.19) circle (  2.13);

\path[fill=fillColor,fill opacity=0.20] ( 75.29, 56.71) circle (  2.13);

\path[fill=fillColor,fill opacity=0.20] ( 73.28, 59.12) circle (  2.13);

\path[fill=fillColor,fill opacity=0.20] ( 81.31, 56.45) circle (  2.13);

\path[fill=fillColor,fill opacity=0.20] ( 83.31, 57.31) circle (  2.13);

\path[fill=fillColor,fill opacity=0.20] ( 88.33, 55.85) circle (  2.13);

\path[fill=fillColor,fill opacity=0.20] ( 94.35, 56.63) circle (  2.13);

\path[fill=fillColor,fill opacity=0.20] ( 95.35, 62.14) circle (  2.13);

\path[fill=fillColor,fill opacity=0.20] ( 97.36, 70.75) circle (  2.13);

\path[fill=fillColor,fill opacity=0.20] ( 89.33, 83.15) circle (  2.13);

\path[fill=fillColor,fill opacity=0.20] (105.38, 88.83) circle (  2.13);

\path[fill=fillColor,fill opacity=0.20] (100.37, 87.54) circle (  2.13);

\path[fill=fillColor,fill opacity=0.20] ( 76.29, 94.60) circle (  2.13);

\path[fill=fillColor,fill opacity=0.20] (148.52,104.59) circle (  2.13);

\path[fill=fillColor,fill opacity=0.20] ( 79.30, 67.91) circle (  2.13);

\path[fill=fillColor,fill opacity=0.20] ( 82.31, 65.67) circle (  2.13);

\path[fill=fillColor,fill opacity=0.20] ( 85.32, 57.49) circle (  2.13);

\path[fill=fillColor,fill opacity=0.20] ( 82.31, 60.24) circle (  2.13);

\path[fill=fillColor,fill opacity=0.20] ( 84.32, 64.03) circle (  2.13);

\path[fill=fillColor,fill opacity=0.20] ( 89.33, 61.96) circle (  2.13);

\path[fill=fillColor,fill opacity=0.20] ( 96.35, 61.88) circle (  2.13);

\path[fill=fillColor,fill opacity=0.20] ( 95.35, 60.76) circle (  2.13);

\path[fill=fillColor,fill opacity=0.20] ( 99.36, 57.49) circle (  2.13);

\path[fill=fillColor,fill opacity=0.20] (106.39, 65.41) circle (  2.13);

\path[fill=fillColor,fill opacity=0.20] (102.37, 83.49) circle (  2.13);

\path[fill=fillColor,fill opacity=0.20] ( 94.35, 90.47) circle (  2.13);

\path[fill=fillColor,fill opacity=0.20] ( 92.34, 82.37) circle (  2.13);

\path[fill=fillColor,fill opacity=0.20] ( 81.31, 60.41) circle (  2.13);

\path[fill=fillColor,fill opacity=0.20] ( 86.32, 64.46) circle (  2.13);

\path[fill=fillColor,fill opacity=0.20] ( 85.32, 53.70) circle (  2.13);

\path[fill=fillColor,fill opacity=0.20] ( 92.34, 58.26) circle (  2.13);

\path[fill=fillColor,fill opacity=0.20] ( 94.35, 71.78) circle (  2.13);

\path[fill=fillColor,fill opacity=0.20] ( 92.34, 69.54) circle (  2.13);

\path[fill=fillColor,fill opacity=0.20] ( 96.35, 64.72) circle (  2.13);

\path[fill=fillColor,fill opacity=0.20] ( 98.36, 65.93) circle (  2.13);

\path[fill=fillColor,fill opacity=0.20] (101.37, 59.55) circle (  2.13);

\path[fill=fillColor,fill opacity=0.20] (121.43, 60.85) circle (  2.13);

\path[fill=fillColor,fill opacity=0.20] ( 82.31, 94.43) circle (  2.13);

\path[fill=fillColor,fill opacity=0.20] ( 85.32, 71.52) circle (  2.13);

\path[fill=fillColor,fill opacity=0.20] ( 82.31, 52.66) circle (  2.13);

\path[fill=fillColor,fill opacity=0.20] ( 83.31, 55.16) circle (  2.13);

\path[fill=fillColor,fill opacity=0.20] ( 75.29, 51.63) circle (  2.13);

\path[fill=fillColor,fill opacity=0.20] ( 93.34, 57.06) circle (  2.13);

\path[fill=fillColor,fill opacity=0.20] ( 98.36, 68.25) circle (  2.13);

\path[fill=fillColor,fill opacity=0.20] ( 99.36, 72.64) circle (  2.13);

\path[fill=fillColor,fill opacity=0.20] (104.38, 65.84) circle (  2.13);

\path[fill=fillColor,fill opacity=0.20] (104.38, 68.60) circle (  2.13);

\path[fill=fillColor,fill opacity=0.20] (119.43, 77.98) circle (  2.13);

\path[fill=fillColor,fill opacity=0.20] ( 92.34, 62.83) circle (  2.13);

\path[fill=fillColor,fill opacity=0.20] ( 88.33, 61.36) circle (  2.13);

\path[fill=fillColor,fill opacity=0.20] ( 89.33, 64.38) circle (  2.13);

\path[fill=fillColor,fill opacity=0.20] ( 96.35, 72.04) circle (  2.13);

\path[fill=fillColor,fill opacity=0.20] ( 92.34, 82.46) circle (  2.13);

\path[fill=fillColor,fill opacity=0.20] ( 86.32, 63.08) circle (  2.13);

\path[fill=fillColor,fill opacity=0.20] ( 83.31, 61.19) circle (  2.13);

\path[fill=fillColor,fill opacity=0.20] ( 85.32, 57.49) circle (  2.13);

\path[fill=fillColor,fill opacity=0.20] ( 93.34, 59.64) circle (  2.13);

\path[fill=fillColor,fill opacity=0.20] ( 96.35, 62.22) circle (  2.13);

\path[fill=fillColor,fill opacity=0.20] (101.37, 67.22) circle (  2.13);

\path[fill=fillColor,fill opacity=0.20] (110.40, 70.49) circle (  2.13);

\path[fill=fillColor,fill opacity=0.20] (115.42, 76.78) circle (  2.13);

\path[fill=fillColor,fill opacity=0.20] (135.48, 92.19) circle (  2.13);

\path[fill=fillColor,fill opacity=0.20] ( 90.33, 71.35) circle (  2.13);

\path[fill=fillColor,fill opacity=0.20] ( 99.36, 63.00) circle (  2.13);

\path[fill=fillColor,fill opacity=0.20] ( 89.33, 76.78) circle (  2.13);

\path[fill=fillColor,fill opacity=0.20] ( 83.31, 75.57) circle (  2.13);

\path[fill=fillColor,fill opacity=0.20] ( 91.34, 73.33) circle (  2.13);

\path[fill=fillColor,fill opacity=0.20] ( 97.36, 69.20) circle (  2.13);

\path[fill=fillColor,fill opacity=0.20] ( 98.36, 58.35) circle (  2.13);

\path[fill=fillColor,fill opacity=0.20] ( 91.34, 93.22) circle (  2.13);

\path[fill=fillColor,fill opacity=0.20] (102.37, 92.19) circle (  2.13);

\path[fill=fillColor,fill opacity=0.20] ( 75.29, 68.51) circle (  2.13);

\path[fill=fillColor,fill opacity=0.20] ( 83.31, 63.77) circle (  2.13);

\path[fill=fillColor,fill opacity=0.20] ( 91.34, 54.73) circle (  2.13);

\path[fill=fillColor,fill opacity=0.20] ( 93.34, 58.26) circle (  2.13);

\path[fill=fillColor,fill opacity=0.20] ( 88.33, 61.19) circle (  2.13);

\path[fill=fillColor,fill opacity=0.20] (104.38, 62.65) circle (  2.13);

\path[fill=fillColor,fill opacity=0.20] (111.40, 83.58) circle (  2.13);

\path[fill=fillColor,fill opacity=0.20] ( 87.33, 62.91) circle (  2.13);

\path[fill=fillColor,fill opacity=0.20] ( 89.33, 66.10) circle (  2.13);

\path[fill=fillColor,fill opacity=0.20] ( 87.33, 70.23) circle (  2.13);

\path[fill=fillColor,fill opacity=0.20] ( 89.33, 64.98) circle (  2.13);

\path[fill=fillColor,fill opacity=0.20] ( 84.32, 65.67) circle (  2.13);

\path[fill=fillColor,fill opacity=0.20] ( 89.33, 70.58) circle (  2.13);

\path[fill=fillColor,fill opacity=0.20] (101.37, 68.34) circle (  2.13);

\path[fill=fillColor,fill opacity=0.20] ( 94.35, 75.05) circle (  2.13);

\path[fill=fillColor,fill opacity=0.20] (102.37, 91.85) circle (  2.13);

\path[fill=fillColor,fill opacity=0.20] ( 75.29, 51.29) circle (  2.13);

\path[fill=fillColor,fill opacity=0.20] ( 69.27, 54.64) circle (  2.13);

\path[fill=fillColor,fill opacity=0.20] ( 92.34, 48.01) circle (  2.13);

\path[fill=fillColor,fill opacity=0.20] ( 87.33, 55.25) circle (  2.13);

\path[fill=fillColor,fill opacity=0.20] ( 88.33, 64.29) circle (  2.13);

\path[fill=fillColor,fill opacity=0.20] (104.38, 59.64) circle (  2.13);

\path[fill=fillColor,fill opacity=0.20] ( 99.36, 68.51) circle (  2.13);

\path[fill=fillColor,fill opacity=0.20] (106.39, 80.13) circle (  2.13);

\path[fill=fillColor,fill opacity=0.20] ( 84.32, 74.45) circle (  2.13);

\path[fill=fillColor,fill opacity=0.20] ( 78.30, 56.63) circle (  2.13);

\path[fill=fillColor,fill opacity=0.20] ( 81.31, 58.95) circle (  2.13);

\path[fill=fillColor,fill opacity=0.20] ( 88.33, 55.33) circle (  2.13);

\path[fill=fillColor,fill opacity=0.20] ( 97.36, 53.96) circle (  2.13);

\path[fill=fillColor,fill opacity=0.20] ( 92.34, 50.68) circle (  2.13);

\path[fill=fillColor,fill opacity=0.20] ( 91.34, 57.31) circle (  2.13);

\path[fill=fillColor,fill opacity=0.20] ( 97.36, 66.27) circle (  2.13);

\path[fill=fillColor,fill opacity=0.20] ( 92.34, 68.42) circle (  2.13);

\path[fill=fillColor,fill opacity=0.20] ( 99.36,114.24) circle (  2.13);

\path[fill=fillColor,fill opacity=0.20] ( 82.31, 49.13) circle (  2.13);

\path[fill=fillColor,fill opacity=0.20] ( 83.31, 51.89) circle (  2.13);

\path[fill=fillColor,fill opacity=0.20] ( 92.34, 56.28) circle (  2.13);

\path[fill=fillColor,fill opacity=0.20] ( 94.35, 59.81) circle (  2.13);

\path[fill=fillColor,fill opacity=0.20] ( 97.36, 65.50) circle (  2.13);

\path[fill=fillColor,fill opacity=0.20] (104.38, 59.90) circle (  2.13);

\path[fill=fillColor,fill opacity=0.20] (103.38, 58.26) circle (  2.13);

\path[fill=fillColor,fill opacity=0.20] (104.38, 65.75) circle (  2.13);

\path[fill=fillColor,fill opacity=0.20] ( 83.31, 58.26) circle (  2.13);

\path[fill=fillColor,fill opacity=0.20] ( 75.29, 49.74) circle (  2.13);

\path[fill=fillColor,fill opacity=0.20] ( 82.31, 62.65) circle (  2.13);

\path[fill=fillColor,fill opacity=0.20] ( 86.32, 57.92) circle (  2.13);

\path[fill=fillColor,fill opacity=0.20] ( 86.32, 52.41) circle (  2.13);

\path[fill=fillColor,fill opacity=0.20] ( 92.34, 44.05) circle (  2.13);

\path[fill=fillColor,fill opacity=0.20] ( 89.33, 49.91) circle (  2.13);

\path[fill=fillColor,fill opacity=0.20] ( 96.35, 60.16) circle (  2.13);

\path[fill=fillColor,fill opacity=0.20] (101.37, 59.04) circle (  2.13);

\path[fill=fillColor,fill opacity=0.20] ( 83.31, 82.63) circle (  2.13);

\path[fill=fillColor,fill opacity=0.20] (115.42, 86.68) circle (  2.13);

\path[fill=fillColor,fill opacity=0.20] ( 94.35, 60.33) circle (  2.13);

\path[fill=fillColor,fill opacity=0.20] ( 85.32, 66.01) circle (  2.13);

\path[fill=fillColor,fill opacity=0.20] ( 86.32, 66.61) circle (  2.13);

\path[fill=fillColor,fill opacity=0.20] ( 93.34, 61.53) circle (  2.13);

\path[fill=fillColor,fill opacity=0.20] ( 97.36, 59.12) circle (  2.13);

\path[fill=fillColor,fill opacity=0.20] (106.39, 58.95) circle (  2.13);

\path[fill=fillColor,fill opacity=0.20] (108.39, 64.46) circle (  2.13);

\path[fill=fillColor,fill opacity=0.20] ( 77.29, 56.88) circle (  2.13);

\path[fill=fillColor,fill opacity=0.20] ( 78.30, 59.21) circle (  2.13);

\path[fill=fillColor,fill opacity=0.20] ( 82.31, 70.58) circle (  2.13);

\path[fill=fillColor,fill opacity=0.20] ( 83.31, 59.21) circle (  2.13);

\path[fill=fillColor,fill opacity=0.20] ( 82.31, 49.74) circle (  2.13);

\path[fill=fillColor,fill opacity=0.20] ( 88.33, 45.60) circle (  2.13);

\path[fill=fillColor,fill opacity=0.20] ( 83.31, 54.13) circle (  2.13);

\path[fill=fillColor,fill opacity=0.20] ( 91.34, 62.40) circle (  2.13);

\path[fill=fillColor,fill opacity=0.20] ( 99.36, 55.16) circle (  2.13);

\path[fill=fillColor,fill opacity=0.20] (101.37, 66.70) circle (  2.13);

\path[fill=fillColor,fill opacity=0.20] ( 95.35, 69.89) circle (  2.13);

\path[fill=fillColor,fill opacity=0.20] ( 84.32, 57.57) circle (  2.13);

\path[fill=fillColor,fill opacity=0.20] ( 84.32, 59.55) circle (  2.13);

\path[fill=fillColor,fill opacity=0.20] ( 93.34, 57.66) circle (  2.13);

\path[fill=fillColor,fill opacity=0.20] ( 95.35, 62.83) circle (  2.13);

\path[fill=fillColor,fill opacity=0.20] ( 97.36, 74.80) circle (  2.13);

\path[fill=fillColor,fill opacity=0.20] (106.39, 81.00) circle (  2.13);

\path[fill=fillColor,fill opacity=0.20] (110.40, 78.67) circle (  2.13);

\path[fill=fillColor,fill opacity=0.20] ( 82.31, 58.00) circle (  2.13);

\path[fill=fillColor,fill opacity=0.20] ( 86.32, 63.26) circle (  2.13);

\path[fill=fillColor,fill opacity=0.20] ( 84.32, 63.17) circle (  2.13);

\path[fill=fillColor,fill opacity=0.20] ( 87.33, 53.53) circle (  2.13);

\path[fill=fillColor,fill opacity=0.20] ( 87.33, 54.73) circle (  2.13);

\path[fill=fillColor,fill opacity=0.20] ( 90.33, 55.94) circle (  2.13);

\path[fill=fillColor,fill opacity=0.20] ( 83.31, 58.09) circle (  2.13);

\path[fill=fillColor,fill opacity=0.20] ( 83.31, 62.31) circle (  2.13);

\path[fill=fillColor,fill opacity=0.20] ( 96.35, 57.57) circle (  2.13);

\path[fill=fillColor,fill opacity=0.20] ( 90.33, 54.13) circle (  2.13);

\path[fill=fillColor,fill opacity=0.20] (147.52, 94.77) circle (  2.13);

\path[fill=fillColor,fill opacity=0.20] (108.39, 63.95) circle (  2.13);

\path[fill=fillColor,fill opacity=0.20] ( 96.35, 49.82) circle (  2.13);

\path[fill=fillColor,fill opacity=0.20] ( 95.35, 62.57) circle (  2.13);

\path[fill=fillColor,fill opacity=0.20] ( 94.35, 72.82) circle (  2.13);

\path[fill=fillColor,fill opacity=0.20] ( 89.33, 77.90) circle (  2.13);

\path[fill=fillColor,fill opacity=0.20] ( 96.35, 82.29) circle (  2.13);

\path[fill=fillColor,fill opacity=0.20] (104.38, 79.79) circle (  2.13);

\path[fill=fillColor,fill opacity=0.20] (111.40, 78.67) circle (  2.13);

\path[fill=fillColor,fill opacity=0.20] ( 91.34, 75.57) circle (  2.13);

\path[fill=fillColor,fill opacity=0.20] ( 91.34, 53.96) circle (  2.13);

\path[fill=fillColor,fill opacity=0.20] ( 91.34, 47.84) circle (  2.13);

\path[fill=fillColor,fill opacity=0.20] ( 87.33, 49.82) circle (  2.13);

\path[fill=fillColor,fill opacity=0.20] ( 86.32, 57.57) circle (  2.13);

\path[fill=fillColor,fill opacity=0.20] ( 77.29, 62.74) circle (  2.13);

\path[fill=fillColor,fill opacity=0.20] ( 79.30, 61.45) circle (  2.13);

\path[fill=fillColor,fill opacity=0.20] ( 81.31, 57.75) circle (  2.13);

\path[fill=fillColor,fill opacity=0.20] ( 79.30, 60.50) circle (  2.13);

\path[fill=fillColor,fill opacity=0.20] ( 91.34, 62.91) circle (  2.13);

\path[fill=fillColor,fill opacity=0.20] ( 95.35, 63.43) circle (  2.13);

\path[fill=fillColor,fill opacity=0.20] ( 86.32, 84.96) circle (  2.13);

\path[fill=fillColor,fill opacity=0.20] (143.51, 60.41) circle (  2.13);

\path[fill=fillColor,fill opacity=0.20] (109.40, 62.91) circle (  2.13);

\path[fill=fillColor,fill opacity=0.20] ( 94.35, 74.28) circle (  2.13);

\path[fill=fillColor,fill opacity=0.20] ( 86.32, 68.85) circle (  2.13);

\path[fill=fillColor,fill opacity=0.20] ( 95.35, 67.82) circle (  2.13);

\path[fill=fillColor,fill opacity=0.20] (100.37, 78.76) circle (  2.13);

\path[fill=fillColor,fill opacity=0.20] (101.37, 84.10) circle (  2.13);

\path[fill=fillColor,fill opacity=0.20] (101.37, 88.83) circle (  2.13);

\path[fill=fillColor,fill opacity=0.20] ( 89.33, 65.32) circle (  2.13);

\path[fill=fillColor,fill opacity=0.20] ( 87.33, 63.26) circle (  2.13);

\path[fill=fillColor,fill opacity=0.20] ( 83.31, 49.39) circle (  2.13);

\path[fill=fillColor,fill opacity=0.20] ( 84.32, 40.18) circle (  2.13);

\path[fill=fillColor,fill opacity=0.20] ( 83.31, 51.11) circle (  2.13);

\path[fill=fillColor,fill opacity=0.20] ( 80.30, 61.53) circle (  2.13);

\path[fill=fillColor,fill opacity=0.20] ( 75.29, 60.07) circle (  2.13);

\path[fill=fillColor,fill opacity=0.20] ( 76.29, 55.25) circle (  2.13);

\path[fill=fillColor,fill opacity=0.20] ( 80.30, 55.59) circle (  2.13);

\path[fill=fillColor,fill opacity=0.20] ( 78.30, 62.57) circle (  2.13);

\path[fill=fillColor,fill opacity=0.20] ( 84.32, 66.10) circle (  2.13);

\path[fill=fillColor,fill opacity=0.20] ( 97.36, 71.52) circle (  2.13);

\path[fill=fillColor,fill opacity=0.20] (129.46, 94.09) circle (  2.13);

\path[fill=fillColor,fill opacity=0.20] (136.48, 63.77) circle (  2.13);

\path[fill=fillColor,fill opacity=0.20] (100.37, 63.34) circle (  2.13);

\path[fill=fillColor,fill opacity=0.20] ( 88.33, 66.18) circle (  2.13);

\path[fill=fillColor,fill opacity=0.20] ( 95.35, 66.87) circle (  2.13);

\path[fill=fillColor,fill opacity=0.20] ( 97.36, 74.45) circle (  2.13);

\path[fill=fillColor,fill opacity=0.20] ( 97.36, 84.96) circle (  2.13);

\path[fill=fillColor,fill opacity=0.20] (105.38, 80.13) circle (  2.13);

\path[fill=fillColor,fill opacity=0.20] ( 84.32, 72.64) circle (  2.13);

\path[fill=fillColor,fill opacity=0.20] ( 80.30, 62.65) circle (  2.13);

\path[fill=fillColor,fill opacity=0.20] ( 79.30, 61.36) circle (  2.13);

\path[fill=fillColor,fill opacity=0.20] ( 75.29, 54.39) circle (  2.13);

\path[fill=fillColor,fill opacity=0.20] ( 76.29, 59.81) circle (  2.13);

\path[fill=fillColor,fill opacity=0.20] ( 71.27, 63.34) circle (  2.13);

\path[fill=fillColor,fill opacity=0.20] ( 75.29, 58.00) circle (  2.13);

\path[fill=fillColor,fill opacity=0.20] ( 77.29, 53.09) circle (  2.13);

\path[fill=fillColor,fill opacity=0.20] ( 75.29, 48.62) circle (  2.13);

\path[fill=fillColor,fill opacity=0.20] ( 77.29, 50.77) circle (  2.13);

\path[fill=fillColor,fill opacity=0.20] ( 73.28, 62.57) circle (  2.13);

\path[fill=fillColor,fill opacity=0.20] ( 84.32, 61.88) circle (  2.13);

\path[fill=fillColor,fill opacity=0.20] ( 94.35, 63.51) circle (  2.13);

\path[fill=fillColor,fill opacity=0.20] (130.46, 92.54) circle (  2.13);

\path[fill=fillColor,fill opacity=0.20] ( 82.31, 66.87) circle (  2.13);

\path[fill=fillColor,fill opacity=0.20] (100.37, 66.87) circle (  2.13);

\path[fill=fillColor,fill opacity=0.20] ( 93.34, 73.59) circle (  2.13);

\path[fill=fillColor,fill opacity=0.20] ( 98.36, 72.13) circle (  2.13);

\path[fill=fillColor,fill opacity=0.20] ( 99.36, 74.62) circle (  2.13);

\path[fill=fillColor,fill opacity=0.20] (101.37, 74.88) circle (  2.13);

\path[fill=fillColor,fill opacity=0.20] (104.38, 75.66) circle (  2.13);

\path[fill=fillColor,fill opacity=0.20] ( 87.33, 71.09) circle (  2.13);

\path[fill=fillColor,fill opacity=0.20] ( 86.32, 67.39) circle (  2.13);

\path[fill=fillColor,fill opacity=0.20] ( 82.31, 63.17) circle (  2.13);

\path[fill=fillColor,fill opacity=0.20] ( 80.30, 52.49) circle (  2.13);

\path[fill=fillColor,fill opacity=0.20] ( 78.30, 60.24) circle (  2.13);

\path[fill=fillColor,fill opacity=0.20] ( 75.29, 76.17) circle (  2.13);

\path[fill=fillColor,fill opacity=0.20] ( 69.27, 71.61) circle (  2.13);

\path[fill=fillColor,fill opacity=0.20] ( 74.28, 55.94) circle (  2.13);

\path[fill=fillColor,fill opacity=0.20] ( 83.31, 47.15) circle (  2.13);

\path[fill=fillColor,fill opacity=0.20] ( 78.30, 43.28) circle (  2.13);

\path[fill=fillColor,fill opacity=0.20] ( 77.29, 52.32) circle (  2.13);

\path[fill=fillColor,fill opacity=0.20] ( 74.28, 64.29) circle (  2.13);

\path[fill=fillColor,fill opacity=0.20] ( 89.33, 60.33) circle (  2.13);

\path[fill=fillColor,fill opacity=0.20] (102.37, 64.72) circle (  2.13);

\path[fill=fillColor,fill opacity=0.20] (150.53, 94.00) circle (  2.13);

\path[fill=fillColor,fill opacity=0.20] (108.39, 72.30) circle (  2.13);

\path[fill=fillColor,fill opacity=0.20] (107.39, 71.27) circle (  2.13);

\path[fill=fillColor,fill opacity=0.20] ( 89.33, 76.60) circle (  2.13);

\path[fill=fillColor,fill opacity=0.20] ( 94.35, 72.13) circle (  2.13);

\path[fill=fillColor,fill opacity=0.20] ( 93.34, 71.52) circle (  2.13);

\path[fill=fillColor,fill opacity=0.20] ( 93.34, 77.29) circle (  2.13);

\path[fill=fillColor,fill opacity=0.20] (100.37, 80.31) circle (  2.13);

\path[fill=fillColor,fill opacity=0.20] (108.39, 86.59) circle (  2.13);

\path[fill=fillColor,fill opacity=0.20] ( 96.35, 74.02) circle (  2.13);

\path[fill=fillColor,fill opacity=0.20] ( 91.34, 63.86) circle (  2.13);

\path[fill=fillColor,fill opacity=0.20] ( 80.30, 61.28) circle (  2.13);

\path[fill=fillColor,fill opacity=0.20] ( 82.31, 54.82) circle (  2.13);

\path[fill=fillColor,fill opacity=0.20] ( 78.30, 52.49) circle (  2.13);

\path[fill=fillColor,fill opacity=0.20] ( 79.30, 60.41) circle (  2.13);

\path[fill=fillColor,fill opacity=0.20] ( 75.29, 67.99) circle (  2.13);

\path[fill=fillColor,fill opacity=0.20] ( 71.27, 66.87) circle (  2.13);

\path[fill=fillColor,fill opacity=0.20] ( 74.28, 55.94) circle (  2.13);

\path[fill=fillColor,fill opacity=0.20] ( 81.31, 43.71) circle (  2.13);

\path[fill=fillColor,fill opacity=0.20] ( 78.30, 49.65) circle (  2.13);

\path[fill=fillColor,fill opacity=0.20] ( 71.27, 63.95) circle (  2.13);

\path[fill=fillColor,fill opacity=0.20] ( 74.28, 66.61) circle (  2.13);

\path[fill=fillColor,fill opacity=0.20] ( 88.33, 66.53) circle (  2.13);

\path[fill=fillColor,fill opacity=0.20] (101.37, 80.65) circle (  2.13);

\path[fill=fillColor,fill opacity=0.20] ( 69.27, 76.17) circle (  2.13);

\path[fill=fillColor,fill opacity=0.20] ( 90.33, 69.11) circle (  2.13);

\path[fill=fillColor,fill opacity=0.20] ( 92.34, 67.22) circle (  2.13);

\path[fill=fillColor,fill opacity=0.20] ( 89.33, 75.92) circle (  2.13);

\path[fill=fillColor,fill opacity=0.20] (101.37, 74.62) circle (  2.13);

\path[fill=fillColor,fill opacity=0.20] (104.38, 73.59) circle (  2.13);

\path[fill=fillColor,fill opacity=0.20] (107.39, 95.64) circle (  2.13);

\path[fill=fillColor,fill opacity=0.20] ( 97.36, 76.52) circle (  2.13);

\path[fill=fillColor,fill opacity=0.20] ( 95.35, 75.74) circle (  2.13);

\path[fill=fillColor,fill opacity=0.20] ( 78.30, 65.41) circle (  2.13);

\path[fill=fillColor,fill opacity=0.20] ( 79.30, 59.55) circle (  2.13);

\path[fill=fillColor,fill opacity=0.20] ( 79.30, 60.16) circle (  2.13);

\path[fill=fillColor,fill opacity=0.20] ( 76.29, 56.54) circle (  2.13);

\path[fill=fillColor,fill opacity=0.20] ( 79.30, 54.82) circle (  2.13);

\path[fill=fillColor,fill opacity=0.20] ( 75.29, 56.11) circle (  2.13);

\path[fill=fillColor,fill opacity=0.20] ( 76.29, 57.75) circle (  2.13);

\path[fill=fillColor,fill opacity=0.20] ( 78.30, 52.66) circle (  2.13);

\path[fill=fillColor,fill opacity=0.20] ( 76.29, 49.82) circle (  2.13);

\path[fill=fillColor,fill opacity=0.20] ( 58.23, 61.19) circle (  2.13);

\path[fill=fillColor,fill opacity=0.20] ( 70.27, 65.32) circle (  2.13);

\path[fill=fillColor,fill opacity=0.20] ( 82.31, 63.86) circle (  2.13);

\path[fill=fillColor,fill opacity=0.20] (106.39, 77.55) circle (  2.13);

\path[fill=fillColor,fill opacity=0.20] (104.38, 77.29) circle (  2.13);

\path[fill=fillColor,fill opacity=0.20] (108.39, 59.21) circle (  2.13);

\path[fill=fillColor,fill opacity=0.20] ( 99.36, 63.95) circle (  2.13);

\path[fill=fillColor,fill opacity=0.20] ( 91.34, 74.02) circle (  2.13);

\path[fill=fillColor,fill opacity=0.20] ( 97.36, 65.15) circle (  2.13);

\path[fill=fillColor,fill opacity=0.20] (100.37, 68.25) circle (  2.13);

\path[fill=fillColor,fill opacity=0.20] (102.37, 82.63) circle (  2.13);

\path[fill=fillColor,fill opacity=0.20] (101.37, 88.83) circle (  2.13);

\path[fill=fillColor,fill opacity=0.20] ( 96.35, 74.54) circle (  2.13);

\path[fill=fillColor,fill opacity=0.20] ( 90.33, 69.03) circle (  2.13);

\path[fill=fillColor,fill opacity=0.20] ( 84.32, 74.37) circle (  2.13);

\path[fill=fillColor,fill opacity=0.20] ( 77.29, 67.82) circle (  2.13);

\path[fill=fillColor,fill opacity=0.20] ( 76.29, 62.83) circle (  2.13);

\path[fill=fillColor,fill opacity=0.20] ( 76.29, 64.46) circle (  2.13);

\path[fill=fillColor,fill opacity=0.20] ( 75.29, 53.53) circle (  2.13);

\path[fill=fillColor,fill opacity=0.20] ( 75.29, 47.33) circle (  2.13);

\path[fill=fillColor,fill opacity=0.20] ( 75.29, 57.40) circle (  2.13);

\path[fill=fillColor,fill opacity=0.20] ( 51.61, 58.43) circle (  2.13);

\path[fill=fillColor,fill opacity=0.20] ( 77.29, 51.72) circle (  2.13);

\path[fill=fillColor,fill opacity=0.20] ( 76.29, 54.64) circle (  2.13);

\path[fill=fillColor,fill opacity=0.20] ( 81.31, 58.35) circle (  2.13);

\path[fill=fillColor,fill opacity=0.20] ( 81.31, 56.71) circle (  2.13);

\path[fill=fillColor,fill opacity=0.20] ( 76.29, 68.51) circle (  2.13);

\path[fill=fillColor,fill opacity=0.20] (151.53,109.33) circle (  2.13);

\path[fill=fillColor,fill opacity=0.20] (129.46, 81.08) circle (  2.13);

\path[fill=fillColor,fill opacity=0.20] (107.39, 69.20) circle (  2.13);

\path[fill=fillColor,fill opacity=0.20] ( 97.36, 64.81) circle (  2.13);

\path[fill=fillColor,fill opacity=0.20] ( 93.34, 61.28) circle (  2.13);

\path[fill=fillColor,fill opacity=0.20] ( 90.33, 59.73) circle (  2.13);

\path[fill=fillColor,fill opacity=0.20] ( 97.36, 61.19) circle (  2.13);

\path[fill=fillColor,fill opacity=0.20] (100.37, 67.56) circle (  2.13);

\path[fill=fillColor,fill opacity=0.20] (103.38, 76.26) circle (  2.13);

\path[fill=fillColor,fill opacity=0.20] ( 87.33, 65.06) circle (  2.13);

\path[fill=fillColor,fill opacity=0.20] ( 82.31, 57.83) circle (  2.13);

\path[fill=fillColor,fill opacity=0.20] ( 79.30, 67.30) circle (  2.13);

\path[fill=fillColor,fill opacity=0.20] ( 79.30, 67.13) circle (  2.13);

\path[fill=fillColor,fill opacity=0.20] ( 74.28, 60.33) circle (  2.13);

\path[fill=fillColor,fill opacity=0.20] ( 73.28, 59.30) circle (  2.13);

\path[fill=fillColor,fill opacity=0.20] ( 66.96, 48.88) circle (  2.13);

\path[fill=fillColor,fill opacity=0.20] ( 71.27, 43.79) circle (  2.13);

\path[fill=fillColor,fill opacity=0.20] ( 72.28, 57.57) circle (  2.13);

\path[fill=fillColor,fill opacity=0.20] ( 71.27, 59.04) circle (  2.13);

\path[fill=fillColor,fill opacity=0.20] ( 77.29, 53.78) circle (  2.13);

\path[fill=fillColor,fill opacity=0.20] ( 81.31, 61.36) circle (  2.13);

\path[fill=fillColor,fill opacity=0.20] ( 91.34, 64.20) circle (  2.13);

\path[fill=fillColor,fill opacity=0.20] (128.46, 68.60) circle (  2.13);

\path[fill=fillColor,fill opacity=0.20] (140.50, 91.85) circle (  2.13);

\path[fill=fillColor,fill opacity=0.20] ( 85.32, 64.20) circle (  2.13);

\path[fill=fillColor,fill opacity=0.20] ( 99.36, 61.45) circle (  2.13);

\path[fill=fillColor,fill opacity=0.20] ( 94.35, 62.22) circle (  2.13);

\path[fill=fillColor,fill opacity=0.20] (100.37, 57.57) circle (  2.13);

\path[fill=fillColor,fill opacity=0.20] (103.38, 62.57) circle (  2.13);

\path[fill=fillColor,fill opacity=0.20] (112.41, 68.34) circle (  2.13);

\path[fill=fillColor,fill opacity=0.20] (113.41, 71.87) circle (  2.13);

\path[fill=fillColor,fill opacity=0.20] (100.37, 70.58) circle (  2.13);

\path[fill=fillColor,fill opacity=0.20] ( 94.35, 73.93) circle (  2.13);

\path[fill=fillColor,fill opacity=0.20] ( 85.32, 73.59) circle (  2.13);

\path[fill=fillColor,fill opacity=0.20] ( 81.31, 63.08) circle (  2.13);

\path[fill=fillColor,fill opacity=0.20] ( 81.31, 60.50) circle (  2.13);

\path[fill=fillColor,fill opacity=0.20] ( 80.30, 63.60) circle (  2.13);

\path[fill=fillColor,fill opacity=0.20] ( 78.30, 60.33) circle (  2.13);

\path[fill=fillColor,fill opacity=0.20] ( 76.29, 58.35) circle (  2.13);

\path[fill=fillColor,fill opacity=0.20] ( 74.28, 59.55) circle (  2.13);

\path[fill=fillColor,fill opacity=0.20] ( 64.95, 55.33) circle (  2.13);

\path[fill=fillColor,fill opacity=0.20] ( 45.29, 50.86) circle (  2.13);

\path[fill=fillColor,fill opacity=0.20] ( 75.29, 52.32) circle (  2.13);

\path[fill=fillColor,fill opacity=0.20] ( 79.30, 55.25) circle (  2.13);

\path[fill=fillColor,fill opacity=0.20] ( 86.32, 59.90) circle (  2.13);

\path[fill=fillColor,fill opacity=0.20] ( 95.35, 80.39) circle (  2.13);

\path[fill=fillColor,fill opacity=0.20] ( 95.35, 65.84) circle (  2.13);

\path[fill=fillColor,fill opacity=0.20] ( 95.35, 66.53) circle (  2.13);

\path[fill=fillColor,fill opacity=0.20] (101.37, 72.13) circle (  2.13);

\path[fill=fillColor,fill opacity=0.20] (104.38, 71.70) circle (  2.13);

\path[fill=fillColor,fill opacity=0.20] (103.38, 76.43) circle (  2.13);

\path[fill=fillColor,fill opacity=0.20] (106.39, 76.43) circle (  2.13);

\path[fill=fillColor,fill opacity=0.20] (107.39, 76.69) circle (  2.13);

\path[fill=fillColor,fill opacity=0.20] (103.38, 76.52) circle (  2.13);

\path[fill=fillColor,fill opacity=0.20] (101.37, 78.58) circle (  2.13);

\path[fill=fillColor,fill opacity=0.20] ( 92.34, 70.83) circle (  2.13);

\path[fill=fillColor,fill opacity=0.20] ( 88.33, 63.69) circle (  2.13);

\path[fill=fillColor,fill opacity=0.20] ( 82.31, 67.91) circle (  2.13);

\path[fill=fillColor,fill opacity=0.20] ( 76.29, 65.67) circle (  2.13);

\path[fill=fillColor,fill opacity=0.20] ( 77.29, 56.54) circle (  2.13);

\path[fill=fillColor,fill opacity=0.20] ( 81.31, 57.14) circle (  2.13);

\path[fill=fillColor,fill opacity=0.20] ( 75.29, 60.59) circle (  2.13);

\path[fill=fillColor,fill opacity=0.20] ( 74.28, 59.81) circle (  2.13);

\path[fill=fillColor,fill opacity=0.20] ( 73.28, 62.05) circle (  2.13);

\path[fill=fillColor,fill opacity=0.20] ( 78.30, 60.24) circle (  2.13);

\path[fill=fillColor,fill opacity=0.20] ( 80.30, 59.81) circle (  2.13);

\path[fill=fillColor,fill opacity=0.20] ( 79.30, 63.08) circle (  2.13);

\path[fill=fillColor,fill opacity=0.20] ( 79.30, 61.62) circle (  2.13);

\path[fill=fillColor,fill opacity=0.20] ( 91.34, 63.26) circle (  2.13);

\path[fill=fillColor,fill opacity=0.20] (139.49, 84.10) circle (  2.13);

\path[fill=fillColor,fill opacity=0.20] (131.47, 84.70) circle (  2.13);

\path[fill=fillColor,fill opacity=0.20] (113.41, 68.16) circle (  2.13);

\path[fill=fillColor,fill opacity=0.20] ( 88.33, 75.83) circle (  2.13);

\path[fill=fillColor,fill opacity=0.20] ( 92.34, 78.24) circle (  2.13);

\path[fill=fillColor,fill opacity=0.20] ( 95.35, 80.57) circle (  2.13);

\path[fill=fillColor,fill opacity=0.20] ( 89.33, 76.86) circle (  2.13);

\path[fill=fillColor,fill opacity=0.20] ( 96.35, 71.95) circle (  2.13);

\path[fill=fillColor,fill opacity=0.20] (102.37, 69.89) circle (  2.13);

\path[fill=fillColor,fill opacity=0.20] (102.37, 89.26) circle (  2.13);

\path[fill=fillColor,fill opacity=0.20] (101.37,109.07) circle (  2.13);

\path[fill=fillColor,fill opacity=0.20] ( 99.36, 69.28) circle (  2.13);

\path[fill=fillColor,fill opacity=0.20] ( 97.36, 66.61) circle (  2.13);

\path[fill=fillColor,fill opacity=0.20] ( 89.33, 69.89) circle (  2.13);

\path[fill=fillColor,fill opacity=0.20] ( 85.32, 64.20) circle (  2.13);

\path[fill=fillColor,fill opacity=0.20] ( 77.29, 65.41) circle (  2.13);

\path[fill=fillColor,fill opacity=0.20] ( 72.28, 71.61) circle (  2.13);

\path[fill=fillColor,fill opacity=0.20] ( 72.28, 64.29) circle (  2.13);

\path[fill=fillColor,fill opacity=0.20] ( 69.27, 50.86) circle (  2.13);

\path[fill=fillColor,fill opacity=0.20] ( 79.30, 48.79) circle (  2.13);

\path[fill=fillColor,fill opacity=0.20] ( 77.29, 53.44) circle (  2.13);

\path[fill=fillColor,fill opacity=0.20] ( 82.31, 58.00) circle (  2.13);

\path[fill=fillColor,fill opacity=0.20] ( 83.31, 62.31) circle (  2.13);

\path[fill=fillColor,fill opacity=0.20] ( 88.33, 69.20) circle (  2.13);

\path[fill=fillColor,fill opacity=0.20] ( 95.35, 80.74) circle (  2.13);

\path[fill=fillColor,fill opacity=0.20] (116.42, 84.53) circle (  2.13);

\path[fill=fillColor,fill opacity=0.20] (113.41, 78.76) circle (  2.13);

\path[fill=fillColor,fill opacity=0.20] (102.37, 73.25) circle (  2.13);

\path[fill=fillColor,fill opacity=0.20] (100.37, 70.92) circle (  2.13);

\path[fill=fillColor,fill opacity=0.20] ( 88.33, 71.52) circle (  2.13);

\path[fill=fillColor,fill opacity=0.20] ( 92.34, 73.68) circle (  2.13);

\path[fill=fillColor,fill opacity=0.20] ( 94.35, 73.93) circle (  2.13);

\path[fill=fillColor,fill opacity=0.20] ( 94.35, 70.23) circle (  2.13);

\path[fill=fillColor,fill opacity=0.20] ( 93.34, 63.86) circle (  2.13);

\path[fill=fillColor,fill opacity=0.20] (105.38, 66.87) circle (  2.13);

\path[fill=fillColor,fill opacity=0.20] (103.38, 79.70) circle (  2.13);

\path[fill=fillColor,fill opacity=0.20] (104.38, 81.17) circle (  2.13);

\path[fill=fillColor,fill opacity=0.20] (108.39, 75.31) circle (  2.13);

\path[fill=fillColor,fill opacity=0.20] ( 96.35, 86.51) circle (  2.13);

\path[fill=fillColor,fill opacity=0.20] ( 99.36, 68.85) circle (  2.13);

\path[fill=fillColor,fill opacity=0.20] ( 93.34, 70.32) circle (  2.13);

\path[fill=fillColor,fill opacity=0.20] ( 91.34, 79.36) circle (  2.13);

\path[fill=fillColor,fill opacity=0.20] ( 84.32, 78.41) circle (  2.13);

\path[fill=fillColor,fill opacity=0.20] ( 76.29, 73.07) circle (  2.13);

\path[fill=fillColor,fill opacity=0.20] ( 79.30, 70.66) circle (  2.13);

\path[fill=fillColor,fill opacity=0.20] ( 76.29, 70.75) circle (  2.13);

\path[fill=fillColor,fill opacity=0.20] ( 71.27, 64.29) circle (  2.13);

\path[fill=fillColor,fill opacity=0.20] ( 75.29, 60.16) circle (  2.13);

\path[fill=fillColor,fill opacity=0.20] ( 81.31, 60.76) circle (  2.13);

\path[fill=fillColor,fill opacity=0.20] ( 87.33, 60.76) circle (  2.13);

\path[fill=fillColor,fill opacity=0.20] ( 95.35, 64.81) circle (  2.13);

\path[fill=fillColor,fill opacity=0.20] (101.37, 72.82) circle (  2.13);

\path[fill=fillColor,fill opacity=0.20] (110.40, 78.24) circle (  2.13);

\path[fill=fillColor,fill opacity=0.20] ( 99.36, 93.31) circle (  2.13);

\path[fill=fillColor,fill opacity=0.20] (103.38, 70.75) circle (  2.13);

\path[fill=fillColor,fill opacity=0.20] ( 92.34, 72.38) circle (  2.13);

\path[fill=fillColor,fill opacity=0.20] ( 98.36, 77.21) circle (  2.13);

\path[fill=fillColor,fill opacity=0.20] ( 88.33, 74.80) circle (  2.13);

\path[fill=fillColor,fill opacity=0.20] ( 96.35, 66.61) circle (  2.13);

\path[fill=fillColor,fill opacity=0.20] ( 99.36, 68.34) circle (  2.13);

\path[fill=fillColor,fill opacity=0.20] (104.38, 72.47) circle (  2.13);

\path[fill=fillColor,fill opacity=0.20] ( 93.34, 71.87) circle (  2.13);

\path[fill=fillColor,fill opacity=0.20] ( 94.35, 74.45) circle (  2.13);

\path[fill=fillColor,fill opacity=0.20] ( 99.36, 70.40) circle (  2.13);

\path[fill=fillColor,fill opacity=0.20] ( 98.36, 78.84) circle (  2.13);

\path[fill=fillColor,fill opacity=0.20] ( 93.34, 80.39) circle (  2.13);

\path[fill=fillColor,fill opacity=0.20] ( 88.33, 78.24) circle (  2.13);

\path[fill=fillColor,fill opacity=0.20] ( 79.30, 80.31) circle (  2.13);

\path[fill=fillColor,fill opacity=0.20] ( 74.28, 81.34) circle (  2.13);

\path[fill=fillColor,fill opacity=0.20] ( 79.30, 73.42) circle (  2.13);

\path[fill=fillColor,fill opacity=0.20] ( 80.30, 65.50) circle (  2.13);

\path[fill=fillColor,fill opacity=0.20] ( 77.29, 64.12) circle (  2.13);

\path[fill=fillColor,fill opacity=0.20] ( 84.32, 61.88) circle (  2.13);

\path[fill=fillColor,fill opacity=0.20] ( 91.34, 63.34) circle (  2.13);

\path[fill=fillColor,fill opacity=0.20] ( 58.43, 75.23) circle (  2.13);

\path[fill=fillColor,fill opacity=0.20] (110.40, 88.49) circle (  2.13);

\path[fill=fillColor,fill opacity=0.20] (105.38, 93.14) circle (  2.13);

\path[fill=fillColor,fill opacity=0.20] ( 87.33, 80.57) circle (  2.13);

\path[fill=fillColor,fill opacity=0.20] (104.38, 73.59) circle (  2.13);

\path[fill=fillColor,fill opacity=0.20] (100.37, 71.52) circle (  2.13);

\path[fill=fillColor,fill opacity=0.20] ( 97.36, 71.09) circle (  2.13);

\path[fill=fillColor,fill opacity=0.20] (102.37, 75.14) circle (  2.13);

\path[fill=fillColor,fill opacity=0.20] (103.38, 75.48) circle (  2.13);

\path[fill=fillColor,fill opacity=0.20] (104.38, 65.84) circle (  2.13);

\path[fill=fillColor,fill opacity=0.20] ( 96.35, 70.32) circle (  2.13);

\path[fill=fillColor,fill opacity=0.20] ( 94.35, 80.48) circle (  2.13);

\path[fill=fillColor,fill opacity=0.20] (101.37, 76.09) circle (  2.13);

\path[fill=fillColor,fill opacity=0.20] (103.38, 76.60) circle (  2.13);

\path[fill=fillColor,fill opacity=0.20] (102.37, 84.01) circle (  2.13);

\path[fill=fillColor,fill opacity=0.20] ( 89.33, 77.90) circle (  2.13);

\path[fill=fillColor,fill opacity=0.20] ( 96.35, 80.22) circle (  2.13);

\path[fill=fillColor,fill opacity=0.20] ( 96.35, 82.72) circle (  2.13);

\path[fill=fillColor,fill opacity=0.20] (108.39, 83.67) circle (  2.13);

\path[fill=fillColor,fill opacity=0.20] (103.38, 85.73) circle (  2.13);

\path[fill=fillColor,fill opacity=0.20] (111.40, 79.45) circle (  2.13);

\path[fill=fillColor,fill opacity=0.20] (111.40, 78.76) circle (  2.13);

\path[fill=fillColor,fill opacity=0.20] (104.38, 77.90) circle (  2.13);

\path[fill=fillColor,fill opacity=0.20] (101.37, 76.43) circle (  2.13);

\path[fill=fillColor,fill opacity=0.20] ( 90.33, 73.50) circle (  2.13);

\path[fill=fillColor,fill opacity=0.20] ( 83.31, 79.79) circle (  2.13);

\path[fill=fillColor,fill opacity=0.20] ( 84.32, 71.18) circle (  2.13);

\path[fill=fillColor,fill opacity=0.20] ( 91.34, 71.61) circle (  2.13);

\path[fill=fillColor,fill opacity=0.20] ( 89.33, 64.20) circle (  2.13);

\path[fill=fillColor,fill opacity=0.20] ( 87.33, 53.87) circle (  2.13);

\path[fill=fillColor,fill opacity=0.20] ( 56.73, 60.16) circle (  2.13);

\path[fill=fillColor,fill opacity=0.20] (101.37, 71.52) circle (  2.13);

\path[fill=fillColor,fill opacity=0.20] (111.40, 82.55) circle (  2.13);

\path[fill=fillColor,fill opacity=0.20] (114.41,101.84) circle (  2.13);

\path[fill=fillColor,fill opacity=0.20] (108.39, 84.78) circle (  2.13);

\path[fill=fillColor,fill opacity=0.20] (112.41, 70.49) circle (  2.13);

\path[fill=fillColor,fill opacity=0.20] ( 99.36, 73.25) circle (  2.13);

\path[fill=fillColor,fill opacity=0.20] (102.37, 72.73) circle (  2.13);

\path[fill=fillColor,fill opacity=0.20] ( 98.36, 64.29) circle (  2.13);

\path[fill=fillColor,fill opacity=0.20] ( 97.36, 70.66) circle (  2.13);

\path[fill=fillColor,fill opacity=0.20] ( 95.35, 82.80) circle (  2.13);

\path[fill=fillColor,fill opacity=0.20] ( 93.34, 81.34) circle (  2.13);

\path[fill=fillColor,fill opacity=0.20] (101.37, 72.47) circle (  2.13);

\path[fill=fillColor,fill opacity=0.20] (100.37, 76.26) circle (  2.13);

\path[fill=fillColor,fill opacity=0.20] (107.39, 79.53) circle (  2.13);

\path[fill=fillColor,fill opacity=0.20] (104.38, 70.92) circle (  2.13);

\path[fill=fillColor,fill opacity=0.20] ( 98.36, 65.24) circle (  2.13);

\path[fill=fillColor,fill opacity=0.20] ( 99.36, 68.42) circle (  2.13);

\path[fill=fillColor,fill opacity=0.20] (103.38, 69.37) circle (  2.13);

\path[fill=fillColor,fill opacity=0.20] ( 96.35, 81.00) circle (  2.13);

\path[fill=fillColor,fill opacity=0.20] ( 79.30, 87.45) circle (  2.13);

\path[fill=fillColor,fill opacity=0.20] ( 86.32, 72.21) circle (  2.13);

\path[fill=fillColor,fill opacity=0.20] ( 89.33, 64.03) circle (  2.13);

\path[fill=fillColor,fill opacity=0.20] ( 98.36, 75.74) circle (  2.13);

\path[fill=fillColor,fill opacity=0.20] (104.38, 78.07) circle (  2.13);

\path[fill=fillColor,fill opacity=0.20] (106.39, 76.35) circle (  2.13);

\path[fill=fillColor,fill opacity=0.20] (109.40, 79.19) circle (  2.13);

\path[fill=fillColor,fill opacity=0.20] (101.37, 76.35) circle (  2.13);

\path[fill=fillColor,fill opacity=0.20] ( 98.36, 75.66) circle (  2.13);

\path[fill=fillColor,fill opacity=0.20] ( 79.30, 79.45) circle (  2.13);

\path[fill=fillColor,fill opacity=0.20] ( 78.30, 73.93) circle (  2.13);

\path[fill=fillColor,fill opacity=0.20] ( 89.33, 59.98) circle (  2.13);

\path[fill=fillColor,fill opacity=0.20] ( 84.32, 60.85) circle (  2.13);

\path[fill=fillColor,fill opacity=0.20] ( 83.31, 71.87) circle (  2.13);

\path[fill=fillColor,fill opacity=0.20] (129.46, 84.87) circle (  2.13);

\path[fill=fillColor,fill opacity=0.20] (109.40,101.84) circle (  2.13);

\path[fill=fillColor,fill opacity=0.20] (125.45, 79.53) circle (  2.13);

\path[fill=fillColor,fill opacity=0.20] (101.37, 70.66) circle (  2.13);

\path[fill=fillColor,fill opacity=0.20] (105.38, 79.27) circle (  2.13);

\path[fill=fillColor,fill opacity=0.20] (104.38, 83.06) circle (  2.13);

\path[fill=fillColor,fill opacity=0.20] (111.40, 74.97) circle (  2.13);

\path[fill=fillColor,fill opacity=0.20] (114.41, 63.69) circle (  2.13);

\path[fill=fillColor,fill opacity=0.20] (107.39, 71.18) circle (  2.13);

\path[fill=fillColor,fill opacity=0.20] ( 95.35, 81.77) circle (  2.13);

\path[fill=fillColor,fill opacity=0.20] (104.38, 66.44) circle (  2.13);

\path[fill=fillColor,fill opacity=0.20] ( 99.36, 59.12) circle (  2.13);

\path[fill=fillColor,fill opacity=0.20] ( 89.33, 74.62) circle (  2.13);

\path[fill=fillColor,fill opacity=0.20] (102.37, 77.55) circle (  2.13);

\path[fill=fillColor,fill opacity=0.20] (111.40, 78.84) circle (  2.13);

\path[fill=fillColor,fill opacity=0.20] (102.37, 85.30) circle (  2.13);

\path[fill=fillColor,fill opacity=0.20] (103.38, 74.97) circle (  2.13);

\path[fill=fillColor,fill opacity=0.20] (106.39, 70.23) circle (  2.13);

\path[fill=fillColor,fill opacity=0.20] (103.38, 84.01) circle (  2.13);

\path[fill=fillColor,fill opacity=0.20] (100.37, 77.21) circle (  2.13);

\path[fill=fillColor,fill opacity=0.20] ( 92.34, 86.59) circle (  2.13);

\path[fill=fillColor,fill opacity=0.20] ( 89.33, 93.48) circle (  2.13);

\path[fill=fillColor,fill opacity=0.20] ( 89.33, 75.23) circle (  2.13);

\path[fill=fillColor,fill opacity=0.20] ( 83.31, 63.34) circle (  2.13);

\path[fill=fillColor,fill opacity=0.20] ( 77.29, 55.59) circle (  2.13);

\path[fill=fillColor,fill opacity=0.20] ( 92.34, 69.71) circle (  2.13);

\path[fill=fillColor,fill opacity=0.20] (110.40, 75.57) circle (  2.13);

\path[fill=fillColor,fill opacity=0.20] ( 93.34, 69.80) circle (  2.13);

\path[fill=fillColor,fill opacity=0.20] (109.40, 99.60) circle (  2.13);

\path[fill=fillColor,fill opacity=0.20] (115.42, 90.55) circle (  2.13);

\path[fill=fillColor,fill opacity=0.20] (111.40, 79.79) circle (  2.13);

\path[fill=fillColor,fill opacity=0.20] (111.40, 66.10) circle (  2.13);

\path[fill=fillColor,fill opacity=0.20] (113.41, 55.51) circle (  2.13);

\path[fill=fillColor,fill opacity=0.20] (104.38, 58.69) circle (  2.13);

\path[fill=fillColor,fill opacity=0.20] (105.38, 74.62) circle (  2.13);

\path[fill=fillColor,fill opacity=0.20] (103.38, 74.45) circle (  2.13);

\path[fill=fillColor,fill opacity=0.20] (105.38, 71.01) circle (  2.13);

\path[fill=fillColor,fill opacity=0.20] (101.37, 83.49) circle (  2.13);

\path[fill=fillColor,fill opacity=0.20] ( 99.36, 84.01) circle (  2.13);

\path[fill=fillColor,fill opacity=0.20] ( 95.35, 82.29) circle (  2.13);

\path[fill=fillColor,fill opacity=0.20] ( 90.33, 72.13) circle (  2.13);

\path[fill=fillColor,fill opacity=0.20] ( 74.28, 81.60) circle (  2.13);

\path[fill=fillColor,fill opacity=0.20] ( 83.31, 72.82) circle (  2.13);

\path[fill=fillColor,fill opacity=0.20] ( 80.30, 62.22) circle (  2.13);

\path[fill=fillColor,fill opacity=0.20] ( 94.35, 59.55) circle (  2.13);

\path[fill=fillColor,fill opacity=0.20] ( 86.32, 66.87) circle (  2.13);

\path[fill=fillColor,fill opacity=0.20] ( 94.35, 79.10) circle (  2.13);

\path[fill=fillColor,fill opacity=0.20] (110.40, 80.31) circle (  2.13);

\path[fill=fillColor,fill opacity=0.20] (115.42, 72.82) circle (  2.13);

\path[fill=fillColor,fill opacity=0.20] (134.48, 84.96) circle (  2.13);

\path[fill=fillColor,fill opacity=0.20] (123.44, 75.92) circle (  2.13);

\path[fill=fillColor,fill opacity=0.20] (114.41, 72.99) circle (  2.13);

\path[fill=fillColor,fill opacity=0.20] (102.37, 66.87) circle (  2.13);

\path[fill=fillColor,fill opacity=0.20] (103.38, 72.56) circle (  2.13);

\path[fill=fillColor,fill opacity=0.20] ( 89.33, 66.36) circle (  2.13);

\path[fill=fillColor,fill opacity=0.20] ( 87.33, 50.68) circle (  2.13);

\path[fill=fillColor,fill opacity=0.20] ( 96.35, 79.79) circle (  2.13);

\path[fill=fillColor,fill opacity=0.20] ( 88.33, 57.92) circle (  2.13);

\path[fill=fillColor,fill opacity=0.20] ( 96.35, 40.61) circle (  2.13);

\path[fill=fillColor,fill opacity=0.20] (103.38, 47.24) circle (  2.13);

\path[fill=fillColor,fill opacity=0.20] ( 97.36, 62.57) circle (  2.13);

\path[fill=fillColor,fill opacity=0.20] (101.37, 72.47) circle (  2.13);

\path[fill=fillColor,fill opacity=0.20] ( 83.31, 67.48) circle (  2.13);

\path[fill=fillColor,fill opacity=0.20] ( 98.36, 63.26) circle (  2.13);

\path[fill=fillColor,fill opacity=0.20] (119.43, 75.14) circle (  2.13);

\path[fill=fillColor,fill opacity=0.20] (133.47, 90.12) circle (  2.13);

\path[fill=fillColor,fill opacity=0.20] (134.48,103.56) circle (  2.13);

\path[fill=fillColor,fill opacity=0.20] (128.46, 86.85) circle (  2.13);

\path[fill=fillColor,fill opacity=0.20] ( 99.36, 76.95) circle (  2.13);

\path[fill=fillColor,fill opacity=0.20] (101.37, 62.91) circle (  2.13);

\path[fill=fillColor,fill opacity=0.20] ( 92.34, 52.41) circle (  2.13);

\path[fill=fillColor,fill opacity=0.20] ( 89.33, 52.75) circle (  2.13);

\path[fill=fillColor,fill opacity=0.20] ( 85.32, 57.06) circle (  2.13);

\path[fill=fillColor,fill opacity=0.20] ( 90.33, 68.68) circle (  2.13);

\path[fill=fillColor,fill opacity=0.20] (113.41, 61.02) circle (  2.13);

\path[fill=fillColor,fill opacity=0.20] (130.46, 71.27) circle (  2.13);

\path[fill=fillColor,fill opacity=0.20] (127.45, 82.72) circle (  2.13);

\path[fill=fillColor,fill opacity=0.20] (124.44, 89.52) circle (  2.13);

\path[fill=fillColor,fill opacity=0.20] (118.43, 77.38) circle (  2.13);

\path[fill=fillColor,fill opacity=0.20] ( 88.33, 76.86) circle (  2.13);

\path[fill=fillColor,fill opacity=0.20] (112.41, 85.65) circle (  2.13);

\path[fill=fillColor,fill opacity=0.20] (105.38, 85.90) circle (  2.13);

\path[fill=fillColor,fill opacity=0.20] ( 94.35, 76.86) circle (  2.13);

\path[fill=fillColor,fill opacity=0.20] (147.52, 86.59) circle (  2.13);
\end{scope}
\begin{scope}
\path[clip] (159.87, 34.04) rectangle (277.04,119.86);
\definecolor[named]{fillColor}{rgb}{0.90,0.90,0.90}

\path[fill=fillColor] (159.87, 34.04) rectangle (277.03,119.86);
\definecolor[named]{drawColor}{rgb}{0.95,0.95,0.95}

\path[draw=drawColor,line width= 0.3pt,line join=round,line cap=round] (159.87, 34.15) --
	(277.04, 34.15);

\path[draw=drawColor,line width= 0.3pt,line join=round,line cap=round] (159.87, 51.37) --
	(277.04, 51.37);

\path[draw=drawColor,line width= 0.3pt,line join=round,line cap=round] (159.87, 68.60) --
	(277.04, 68.60);

\path[draw=drawColor,line width= 0.3pt,line join=round,line cap=round] (159.87, 85.82) --
	(277.04, 85.82);

\path[draw=drawColor,line width= 0.3pt,line join=round,line cap=round] (159.87,103.04) --
	(277.04,103.04);

\path[draw=drawColor,line width= 0.3pt,line join=round,line cap=round] (178.41, 34.04) --
	(178.41,119.86);

\path[draw=drawColor,line width= 0.3pt,line join=round,line cap=round] (198.47, 34.04) --
	(198.47,119.86);

\path[draw=drawColor,line width= 0.3pt,line join=round,line cap=round] (218.54, 34.04) --
	(218.54,119.86);

\path[draw=drawColor,line width= 0.3pt,line join=round,line cap=round] (238.60, 34.04) --
	(238.60,119.86);

\path[draw=drawColor,line width= 0.3pt,line join=round,line cap=round] (258.67, 34.04) --
	(258.67,119.86);
\definecolor[named]{drawColor}{rgb}{1.00,1.00,1.00}

\path[draw=drawColor,line width= 0.6pt,line join=round,line cap=round] (159.87, 42.76) --
	(277.04, 42.76);

\path[draw=drawColor,line width= 0.6pt,line join=round,line cap=round] (159.87, 59.98) --
	(277.04, 59.98);

\path[draw=drawColor,line width= 0.6pt,line join=round,line cap=round] (159.87, 77.21) --
	(277.04, 77.21);

\path[draw=drawColor,line width= 0.6pt,line join=round,line cap=round] (159.87, 94.43) --
	(277.04, 94.43);

\path[draw=drawColor,line width= 0.6pt,line join=round,line cap=round] (159.87,111.65) --
	(277.04,111.65);

\path[draw=drawColor,line width= 0.6pt,line join=round,line cap=round] (168.38, 34.04) --
	(168.38,119.86);

\path[draw=drawColor,line width= 0.6pt,line join=round,line cap=round] (188.44, 34.04) --
	(188.44,119.86);

\path[draw=drawColor,line width= 0.6pt,line join=round,line cap=round] (208.51, 34.04) --
	(208.51,119.86);

\path[draw=drawColor,line width= 0.6pt,line join=round,line cap=round] (228.57, 34.04) --
	(228.57,119.86);

\path[draw=drawColor,line width= 0.6pt,line join=round,line cap=round] (248.64, 34.04) --
	(248.64,119.86);

\path[draw=drawColor,line width= 0.6pt,line join=round,line cap=round] (268.70, 34.04) --
	(268.70,119.86);
\definecolor[named]{fillColor}{rgb}{0.00,0.00,0.00}

\path[fill=fillColor,fill opacity=0.20] (218.54, 85.65) circle (  2.13);

\path[fill=fillColor,fill opacity=0.20] (213.52, 85.73) circle (  2.13);

\path[fill=fillColor,fill opacity=0.20] (221.55, 82.46) circle (  2.13);

\path[fill=fillColor,fill opacity=0.20] (213.52, 75.31) circle (  2.13);

\path[fill=fillColor,fill opacity=0.20] (223.55, 74.28) circle (  2.13);

\path[fill=fillColor,fill opacity=0.20] (235.59, 87.20) circle (  2.13);

\path[fill=fillColor,fill opacity=0.20] (216.53, 85.73) circle (  2.13);

\path[fill=fillColor,fill opacity=0.20] (206.50, 76.09) circle (  2.13);

\path[fill=fillColor,fill opacity=0.20] (201.48, 77.98) circle (  2.13);

\path[fill=fillColor,fill opacity=0.20] (201.48, 77.90) circle (  2.13);

\path[fill=fillColor,fill opacity=0.20] (207.50, 69.63) circle (  2.13);

\path[fill=fillColor,fill opacity=0.20] (204.49, 64.72) circle (  2.13);

\path[fill=fillColor,fill opacity=0.20] (205.50, 72.90) circle (  2.13);

\path[fill=fillColor,fill opacity=0.20] (216.53, 83.84) circle (  2.13);

\path[fill=fillColor,fill opacity=0.20] (228.57, 87.37) circle (  2.13);

\path[fill=fillColor,fill opacity=0.20] (238.60, 87.28) circle (  2.13);

\path[fill=fillColor,fill opacity=0.20] (263.68, 90.38) circle (  2.13);

\path[fill=fillColor,fill opacity=0.20] (210.51, 90.99) circle (  2.13);

\path[fill=fillColor,fill opacity=0.20] (201.48, 82.12) circle (  2.13);

\path[fill=fillColor,fill opacity=0.20] (196.47, 74.11) circle (  2.13);

\path[fill=fillColor,fill opacity=0.20] (196.47, 69.37) circle (  2.13);

\path[fill=fillColor,fill opacity=0.20] (203.49, 63.86) circle (  2.13);

\path[fill=fillColor,fill opacity=0.20] (207.50, 62.91) circle (  2.13);

\path[fill=fillColor,fill opacity=0.20] (210.51, 69.63) circle (  2.13);

\path[fill=fillColor,fill opacity=0.20] (209.51, 77.81) circle (  2.13);

\path[fill=fillColor,fill opacity=0.20] (205.50, 83.58) circle (  2.13);

\path[fill=fillColor,fill opacity=0.20] (217.53, 88.32) circle (  2.13);

\path[fill=fillColor,fill opacity=0.20] (222.55, 88.57) circle (  2.13);

\path[fill=fillColor,fill opacity=0.20] (224.56, 90.81) circle (  2.13);

\path[fill=fillColor,fill opacity=0.20] (207.50, 83.67) circle (  2.13);

\path[fill=fillColor,fill opacity=0.20] (198.47, 74.11) circle (  2.13);

\path[fill=fillColor,fill opacity=0.20] (191.45, 71.44) circle (  2.13);

\path[fill=fillColor,fill opacity=0.20] (191.45, 73.85) circle (  2.13);

\path[fill=fillColor,fill opacity=0.20] (196.47, 69.37) circle (  2.13);

\path[fill=fillColor,fill opacity=0.20] (200.48, 63.60) circle (  2.13);

\path[fill=fillColor,fill opacity=0.20] (201.48, 66.87) circle (  2.13);

\path[fill=fillColor,fill opacity=0.20] (206.50, 74.37) circle (  2.13);

\path[fill=fillColor,fill opacity=0.20] (205.50, 77.29) circle (  2.13);

\path[fill=fillColor,fill opacity=0.20] (198.47, 73.85) circle (  2.13);

\path[fill=fillColor,fill opacity=0.20] (200.48, 79.96) circle (  2.13);

\path[fill=fillColor,fill opacity=0.20] (210.51, 90.81) circle (  2.13);

\path[fill=fillColor,fill opacity=0.20] (217.53, 96.75) circle (  2.13);

\path[fill=fillColor,fill opacity=0.20] (206.50, 74.28) circle (  2.13);

\path[fill=fillColor,fill opacity=0.20] (194.46, 55.33) circle (  2.13);

\path[fill=fillColor,fill opacity=0.20] (191.45, 57.06) circle (  2.13);

\path[fill=fillColor,fill opacity=0.20] (194.46, 71.78) circle (  2.13);

\path[fill=fillColor,fill opacity=0.20] (196.47, 75.05) circle (  2.13);

\path[fill=fillColor,fill opacity=0.20] (198.47, 71.09) circle (  2.13);

\path[fill=fillColor,fill opacity=0.20] (201.48, 68.85) circle (  2.13);

\path[fill=fillColor,fill opacity=0.20] (202.49, 71.70) circle (  2.13);

\path[fill=fillColor,fill opacity=0.20] (208.51, 78.15) circle (  2.13);

\path[fill=fillColor,fill opacity=0.20] (201.48, 75.57) circle (  2.13);

\path[fill=fillColor,fill opacity=0.20] (197.47, 80.48) circle (  2.13);

\path[fill=fillColor,fill opacity=0.20] (203.49, 82.37) circle (  2.13);

\path[fill=fillColor,fill opacity=0.20] (207.50, 78.24) circle (  2.13);

\path[fill=fillColor,fill opacity=0.20] (212.52, 73.85) circle (  2.13);

\path[fill=fillColor,fill opacity=0.20] (216.53, 80.57) circle (  2.13);

\path[fill=fillColor,fill opacity=0.20] (214.53, 97.70) circle (  2.13);

\path[fill=fillColor,fill opacity=0.20] (209.51, 73.42) circle (  2.13);

\path[fill=fillColor,fill opacity=0.20] (200.48, 59.04) circle (  2.13);

\path[fill=fillColor,fill opacity=0.20] (197.47, 59.38) circle (  2.13);

\path[fill=fillColor,fill opacity=0.20] (197.47, 68.51) circle (  2.13);

\path[fill=fillColor,fill opacity=0.20] (199.48, 73.33) circle (  2.13);

\path[fill=fillColor,fill opacity=0.20] (201.48, 71.61) circle (  2.13);

\path[fill=fillColor,fill opacity=0.20] (204.49, 66.44) circle (  2.13);

\path[fill=fillColor,fill opacity=0.20] (207.50, 69.54) circle (  2.13);

\path[fill=fillColor,fill opacity=0.20] (220.54, 83.67) circle (  2.13);

\path[fill=fillColor,fill opacity=0.20] (197.47, 81.68) circle (  2.13);

\path[fill=fillColor,fill opacity=0.20] (193.46, 76.78) circle (  2.13);

\path[fill=fillColor,fill opacity=0.20] (195.46, 82.89) circle (  2.13);

\path[fill=fillColor,fill opacity=0.20] (196.47, 77.55) circle (  2.13);

\path[fill=fillColor,fill opacity=0.20] (196.47, 70.06) circle (  2.13);

\path[fill=fillColor,fill opacity=0.20] (204.49, 67.05) circle (  2.13);

\path[fill=fillColor,fill opacity=0.20] (204.49, 65.06) circle (  2.13);

\path[fill=fillColor,fill opacity=0.20] (208.51, 68.25) circle (  2.13);

\path[fill=fillColor,fill opacity=0.20] (213.52,104.33) circle (  2.13);

\path[fill=fillColor,fill opacity=0.20] (211.52, 79.96) circle (  2.13);

\path[fill=fillColor,fill opacity=0.20] (201.48, 73.33) circle (  2.13);

\path[fill=fillColor,fill opacity=0.20] (199.48, 70.32) circle (  2.13);

\path[fill=fillColor,fill opacity=0.20] (198.47, 68.42) circle (  2.13);

\path[fill=fillColor,fill opacity=0.20] (199.48, 68.42) circle (  2.13);

\path[fill=fillColor,fill opacity=0.20] (204.49, 69.11) circle (  2.13);

\path[fill=fillColor,fill opacity=0.20] (207.50, 67.22) circle (  2.13);

\path[fill=fillColor,fill opacity=0.20] (212.52, 68.60) circle (  2.13);

\path[fill=fillColor,fill opacity=0.20] (200.48, 70.58) circle (  2.13);

\path[fill=fillColor,fill opacity=0.20] (193.46, 71.35) circle (  2.13);

\path[fill=fillColor,fill opacity=0.20] (196.47, 71.09) circle (  2.13);

\path[fill=fillColor,fill opacity=0.20] (198.47, 61.36) circle (  2.13);

\path[fill=fillColor,fill opacity=0.20] (200.48, 58.69) circle (  2.13);

\path[fill=fillColor,fill opacity=0.20] (201.48, 62.48) circle (  2.13);

\path[fill=fillColor,fill opacity=0.20] (201.48, 61.79) circle (  2.13);

\path[fill=fillColor,fill opacity=0.20] (211.52, 60.59) circle (  2.13);

\path[fill=fillColor,fill opacity=0.20] (209.51, 87.02) circle (  2.13);

\path[fill=fillColor,fill opacity=0.20] (199.48, 78.67) circle (  2.13);

\path[fill=fillColor,fill opacity=0.20] (199.48, 74.88) circle (  2.13);

\path[fill=fillColor,fill opacity=0.20] (202.49, 72.04) circle (  2.13);

\path[fill=fillColor,fill opacity=0.20] (200.48, 68.85) circle (  2.13);

\path[fill=fillColor,fill opacity=0.20] (203.49, 66.87) circle (  2.13);

\path[fill=fillColor,fill opacity=0.20] (206.50, 66.44) circle (  2.13);

\path[fill=fillColor,fill opacity=0.20] (212.52, 66.70) circle (  2.13);

\path[fill=fillColor,fill opacity=0.20] (202.49, 79.53) circle (  2.13);

\path[fill=fillColor,fill opacity=0.20] (196.47, 56.37) circle (  2.13);

\path[fill=fillColor,fill opacity=0.20] (198.47, 56.80) circle (  2.13);

\path[fill=fillColor,fill opacity=0.20] (198.47, 50.68) circle (  2.13);

\path[fill=fillColor,fill opacity=0.20] (199.48, 49.31) circle (  2.13);

\path[fill=fillColor,fill opacity=0.20] (203.49, 59.04) circle (  2.13);

\path[fill=fillColor,fill opacity=0.20] (205.50, 58.95) circle (  2.13);

\path[fill=fillColor,fill opacity=0.20] (211.52, 55.59) circle (  2.13);

\path[fill=fillColor,fill opacity=0.20] (215.53, 59.64) circle (  2.13);

\path[fill=fillColor,fill opacity=0.20] (220.54, 62.91) circle (  2.13);

\path[fill=fillColor,fill opacity=0.20] (208.51, 94.26) circle (  2.13);

\path[fill=fillColor,fill opacity=0.20] (201.48, 79.36) circle (  2.13);

\path[fill=fillColor,fill opacity=0.20] (201.48, 73.76) circle (  2.13);

\path[fill=fillColor,fill opacity=0.20] (204.49, 72.73) circle (  2.13);

\path[fill=fillColor,fill opacity=0.20] (202.49, 66.79) circle (  2.13);

\path[fill=fillColor,fill opacity=0.20] (203.49, 57.31) circle (  2.13);

\path[fill=fillColor,fill opacity=0.20] (207.50, 57.83) circle (  2.13);

\path[fill=fillColor,fill opacity=0.20] (210.51, 63.17) circle (  2.13);

\path[fill=fillColor,fill opacity=0.20] (217.53, 69.54) circle (  2.13);

\path[fill=fillColor,fill opacity=0.20] (198.47, 50.77) circle (  2.13);

\path[fill=fillColor,fill opacity=0.20] (195.46, 42.59) circle (  2.13);

\path[fill=fillColor,fill opacity=0.20] (201.48, 49.65) circle (  2.13);

\path[fill=fillColor,fill opacity=0.20] (202.49, 43.28) circle (  2.13);

\path[fill=fillColor,fill opacity=0.20] (203.49, 48.19) circle (  2.13);

\path[fill=fillColor,fill opacity=0.20] (206.50, 58.00) circle (  2.13);

\path[fill=fillColor,fill opacity=0.20] (208.51, 50.43) circle (  2.13);

\path[fill=fillColor,fill opacity=0.20] (214.53, 48.96) circle (  2.13);

\path[fill=fillColor,fill opacity=0.20] (219.54, 61.53) circle (  2.13);

\path[fill=fillColor,fill opacity=0.20] (223.55, 65.75) circle (  2.13);

\path[fill=fillColor,fill opacity=0.20] (209.51,101.23) circle (  2.13);

\path[fill=fillColor,fill opacity=0.20] (202.49, 79.19) circle (  2.13);

\path[fill=fillColor,fill opacity=0.20] (199.48, 70.06) circle (  2.13);

\path[fill=fillColor,fill opacity=0.20] (198.47, 65.32) circle (  2.13);

\path[fill=fillColor,fill opacity=0.20] (200.48, 56.71) circle (  2.13);

\path[fill=fillColor,fill opacity=0.20] (203.49, 49.74) circle (  2.13);

\path[fill=fillColor,fill opacity=0.20] (209.51, 53.87) circle (  2.13);

\path[fill=fillColor,fill opacity=0.20] (213.52, 59.38) circle (  2.13);

\path[fill=fillColor,fill opacity=0.20] (215.53, 59.30) circle (  2.13);

\path[fill=fillColor,fill opacity=0.20] (226.56, 75.57) circle (  2.13);

\path[fill=fillColor,fill opacity=0.20] (205.50, 94.60) circle (  2.13);

\path[fill=fillColor,fill opacity=0.20] (190.45, 38.63) circle (  2.13);

\path[fill=fillColor,fill opacity=0.20] (199.48, 42.50) circle (  2.13);

\path[fill=fillColor,fill opacity=0.20] (205.50, 48.44) circle (  2.13);

\path[fill=fillColor,fill opacity=0.20] (203.49, 46.38) circle (  2.13);

\path[fill=fillColor,fill opacity=0.20] (202.49, 51.20) circle (  2.13);

\path[fill=fillColor,fill opacity=0.20] (204.49, 50.51) circle (  2.13);

\path[fill=fillColor,fill opacity=0.20] (207.50, 42.42) circle (  2.13);

\path[fill=fillColor,fill opacity=0.20] (210.51, 46.72) circle (  2.13);

\path[fill=fillColor,fill opacity=0.20] (218.54, 58.43) circle (  2.13);

\path[fill=fillColor,fill opacity=0.20] (215.53, 62.48) circle (  2.13);

\path[fill=fillColor,fill opacity=0.20] (239.61, 77.55) circle (  2.13);

\path[fill=fillColor,fill opacity=0.20] (202.49, 79.79) circle (  2.13);

\path[fill=fillColor,fill opacity=0.20] (197.47, 66.36) circle (  2.13);

\path[fill=fillColor,fill opacity=0.20] (195.46, 53.61) circle (  2.13);

\path[fill=fillColor,fill opacity=0.20] (198.47, 46.64) circle (  2.13);

\path[fill=fillColor,fill opacity=0.20] (204.49, 52.49) circle (  2.13);

\path[fill=fillColor,fill opacity=0.20] (208.51, 56.28) circle (  2.13);

\path[fill=fillColor,fill opacity=0.20] (215.53, 56.02) circle (  2.13);

\path[fill=fillColor,fill opacity=0.20] (213.52, 56.97) circle (  2.13);

\path[fill=fillColor,fill opacity=0.20] (220.54, 64.63) circle (  2.13);

\path[fill=fillColor,fill opacity=0.20] (205.50, 61.28) circle (  2.13);

\path[fill=fillColor,fill opacity=0.20] (199.48, 46.64) circle (  2.13);

\path[fill=fillColor,fill opacity=0.20] (200.48, 47.24) circle (  2.13);

\path[fill=fillColor,fill opacity=0.20] (204.49, 42.85) circle (  2.13);

\path[fill=fillColor,fill opacity=0.20] (202.49, 46.46) circle (  2.13);

\path[fill=fillColor,fill opacity=0.20] (204.49, 52.66) circle (  2.13);

\path[fill=fillColor,fill opacity=0.20] (209.51, 48.88) circle (  2.13);

\path[fill=fillColor,fill opacity=0.20] (211.52, 46.81) circle (  2.13);

\path[fill=fillColor,fill opacity=0.20] (212.52, 51.89) circle (  2.13);

\path[fill=fillColor,fill opacity=0.20] (211.52, 56.28) circle (  2.13);

\path[fill=fillColor,fill opacity=0.20] (221.55, 62.22) circle (  2.13);

\path[fill=fillColor,fill opacity=0.20] (240.61, 84.44) circle (  2.13);

\path[fill=fillColor,fill opacity=0.20] (211.52, 85.90) circle (  2.13);

\path[fill=fillColor,fill opacity=0.20] (205.50, 68.25) circle (  2.13);

\path[fill=fillColor,fill opacity=0.20] (198.47, 53.96) circle (  2.13);

\path[fill=fillColor,fill opacity=0.20] (202.49, 51.11) circle (  2.13);

\path[fill=fillColor,fill opacity=0.20] (205.50, 56.54) circle (  2.13);

\path[fill=fillColor,fill opacity=0.20] (203.49, 54.21) circle (  2.13);

\path[fill=fillColor,fill opacity=0.20] (209.51, 53.87) circle (  2.13);

\path[fill=fillColor,fill opacity=0.20] (214.53, 60.67) circle (  2.13);

\path[fill=fillColor,fill opacity=0.20] (220.54, 66.18) circle (  2.13);

\path[fill=fillColor,fill opacity=0.20] (211.52, 82.98) circle (  2.13);

\path[fill=fillColor,fill opacity=0.20] (205.50, 48.44) circle (  2.13);

\path[fill=fillColor,fill opacity=0.20] (204.49, 51.72) circle (  2.13);

\path[fill=fillColor,fill opacity=0.20] (206.50, 46.72) circle (  2.13);

\path[fill=fillColor,fill opacity=0.20] (199.48, 41.64) circle (  2.13);

\path[fill=fillColor,fill opacity=0.20] (203.49, 47.93) circle (  2.13);

\path[fill=fillColor,fill opacity=0.20] (202.49, 52.75) circle (  2.13);

\path[fill=fillColor,fill opacity=0.20] (203.49, 54.90) circle (  2.13);

\path[fill=fillColor,fill opacity=0.20] (210.51, 58.35) circle (  2.13);

\path[fill=fillColor,fill opacity=0.20] (212.52, 57.92) circle (  2.13);

\path[fill=fillColor,fill opacity=0.20] (214.53, 58.95) circle (  2.13);

\path[fill=fillColor,fill opacity=0.20] (223.55, 73.76) circle (  2.13);

\path[fill=fillColor,fill opacity=0.20] (210.51, 80.82) circle (  2.13);

\path[fill=fillColor,fill opacity=0.20] (203.49, 62.83) circle (  2.13);

\path[fill=fillColor,fill opacity=0.20] (204.49, 60.59) circle (  2.13);

\path[fill=fillColor,fill opacity=0.20] (206.50, 57.92) circle (  2.13);

\path[fill=fillColor,fill opacity=0.20] (204.49, 53.61) circle (  2.13);

\path[fill=fillColor,fill opacity=0.20] (204.49, 57.06) circle (  2.13);

\path[fill=fillColor,fill opacity=0.20] (210.51, 63.08) circle (  2.13);

\path[fill=fillColor,fill opacity=0.20] (212.52, 66.79) circle (  2.13);

\path[fill=fillColor,fill opacity=0.20] (219.54, 71.44) circle (  2.13);

\path[fill=fillColor,fill opacity=0.20] (202.49, 61.10) circle (  2.13);

\path[fill=fillColor,fill opacity=0.20] (208.51, 52.84) circle (  2.13);

\path[fill=fillColor,fill opacity=0.20] (208.51, 53.27) circle (  2.13);

\path[fill=fillColor,fill opacity=0.20] (205.50, 47.58) circle (  2.13);

\path[fill=fillColor,fill opacity=0.20] (200.48, 47.58) circle (  2.13);

\path[fill=fillColor,fill opacity=0.20] (199.48, 54.30) circle (  2.13);

\path[fill=fillColor,fill opacity=0.20] (210.51, 56.97) circle (  2.13);

\path[fill=fillColor,fill opacity=0.20] (203.49, 60.85) circle (  2.13);

\path[fill=fillColor,fill opacity=0.20] (208.51, 63.86) circle (  2.13);

\path[fill=fillColor,fill opacity=0.20] (211.52, 59.12) circle (  2.13);

\path[fill=fillColor,fill opacity=0.20] (217.53, 62.05) circle (  2.13);

\path[fill=fillColor,fill opacity=0.20] (210.51, 76.86) circle (  2.13);

\path[fill=fillColor,fill opacity=0.20] (207.50, 68.42) circle (  2.13);

\path[fill=fillColor,fill opacity=0.20] (207.50, 64.72) circle (  2.13);

\path[fill=fillColor,fill opacity=0.20] (207.50, 63.34) circle (  2.13);

\path[fill=fillColor,fill opacity=0.20] (206.50, 64.29) circle (  2.13);

\path[fill=fillColor,fill opacity=0.20] (203.49, 61.02) circle (  2.13);

\path[fill=fillColor,fill opacity=0.20] (206.50, 61.96) circle (  2.13);

\path[fill=fillColor,fill opacity=0.20] (210.51, 66.01) circle (  2.13);

\path[fill=fillColor,fill opacity=0.20] (220.54, 66.79) circle (  2.13);

\path[fill=fillColor,fill opacity=0.20] (230.58, 77.90) circle (  2.13);

\path[fill=fillColor,fill opacity=0.20] (199.48, 64.55) circle (  2.13);

\path[fill=fillColor,fill opacity=0.20] (200.48, 57.31) circle (  2.13);

\path[fill=fillColor,fill opacity=0.20] (206.50, 57.92) circle (  2.13);

\path[fill=fillColor,fill opacity=0.20] (205.50, 60.24) circle (  2.13);

\path[fill=fillColor,fill opacity=0.20] (197.47, 53.87) circle (  2.13);

\path[fill=fillColor,fill opacity=0.20] (200.48, 49.48) circle (  2.13);

\path[fill=fillColor,fill opacity=0.20] (195.46, 56.37) circle (  2.13);

\path[fill=fillColor,fill opacity=0.20] (209.51, 61.19) circle (  2.13);

\path[fill=fillColor,fill opacity=0.20] (206.50, 60.50) circle (  2.13);

\path[fill=fillColor,fill opacity=0.20] (212.52, 60.93) circle (  2.13);

\path[fill=fillColor,fill opacity=0.20] (214.53, 61.53) circle (  2.13);

\path[fill=fillColor,fill opacity=0.20] (221.55, 69.54) circle (  2.13);

\path[fill=fillColor,fill opacity=0.20] (212.52, 84.53) circle (  2.13);

\path[fill=fillColor,fill opacity=0.20] (208.51, 77.47) circle (  2.13);

\path[fill=fillColor,fill opacity=0.20] (207.50, 70.32) circle (  2.13);

\path[fill=fillColor,fill opacity=0.20] (205.50, 65.06) circle (  2.13);

\path[fill=fillColor,fill opacity=0.20] (206.50, 58.61) circle (  2.13);

\path[fill=fillColor,fill opacity=0.20] (207.50, 60.16) circle (  2.13);

\path[fill=fillColor,fill opacity=0.20] (212.52, 65.93) circle (  2.13);

\path[fill=fillColor,fill opacity=0.20] (216.53, 64.63) circle (  2.13);

\path[fill=fillColor,fill opacity=0.20] (221.55, 64.03) circle (  2.13);

\path[fill=fillColor,fill opacity=0.20] (201.48, 63.86) circle (  2.13);

\path[fill=fillColor,fill opacity=0.20] (195.46, 61.45) circle (  2.13);

\path[fill=fillColor,fill opacity=0.20] (201.48, 59.98) circle (  2.13);

\path[fill=fillColor,fill opacity=0.20] (197.47, 60.33) circle (  2.13);

\path[fill=fillColor,fill opacity=0.20] (196.47, 63.95) circle (  2.13);

\path[fill=fillColor,fill opacity=0.20] (196.47, 58.78) circle (  2.13);

\path[fill=fillColor,fill opacity=0.20] (201.48, 51.37) circle (  2.13);

\path[fill=fillColor,fill opacity=0.20] (206.50, 54.56) circle (  2.13);

\path[fill=fillColor,fill opacity=0.20] (206.50, 58.35) circle (  2.13);

\path[fill=fillColor,fill opacity=0.20] (208.51, 55.33) circle (  2.13);

\path[fill=fillColor,fill opacity=0.20] (216.53, 58.95) circle (  2.13);

\path[fill=fillColor,fill opacity=0.20] (223.55, 70.58) circle (  2.13);

\path[fill=fillColor,fill opacity=0.20] (212.52, 93.40) circle (  2.13);

\path[fill=fillColor,fill opacity=0.20] (203.49, 72.21) circle (  2.13);

\path[fill=fillColor,fill opacity=0.20] (205.50, 60.93) circle (  2.13);

\path[fill=fillColor,fill opacity=0.20] (210.51, 57.57) circle (  2.13);

\path[fill=fillColor,fill opacity=0.20] (207.50, 58.52) circle (  2.13);

\path[fill=fillColor,fill opacity=0.20] (207.50, 61.71) circle (  2.13);

\path[fill=fillColor,fill opacity=0.20] (208.51, 60.93) circle (  2.13);

\path[fill=fillColor,fill opacity=0.20] (205.50, 60.24) circle (  2.13);

\path[fill=fillColor,fill opacity=0.20] (214.53, 68.68) circle (  2.13);

\path[fill=fillColor,fill opacity=0.20] (197.47, 64.63) circle (  2.13);

\path[fill=fillColor,fill opacity=0.20] (199.48, 62.22) circle (  2.13);

\path[fill=fillColor,fill opacity=0.20] (196.47, 59.04) circle (  2.13);

\path[fill=fillColor,fill opacity=0.20] (195.46, 56.97) circle (  2.13);

\path[fill=fillColor,fill opacity=0.20] (198.47, 56.37) circle (  2.13);

\path[fill=fillColor,fill opacity=0.20] (196.47, 59.90) circle (  2.13);

\path[fill=fillColor,fill opacity=0.20] (202.49, 62.22) circle (  2.13);

\path[fill=fillColor,fill opacity=0.20] (202.49, 59.73) circle (  2.13);

\path[fill=fillColor,fill opacity=0.20] (205.50, 56.11) circle (  2.13);

\path[fill=fillColor,fill opacity=0.20] (206.50, 53.78) circle (  2.13);

\path[fill=fillColor,fill opacity=0.20] (217.53, 61.53) circle (  2.13);

\path[fill=fillColor,fill opacity=0.20] (226.56, 80.82) circle (  2.13);

\path[fill=fillColor,fill opacity=0.20] (205.50, 66.79) circle (  2.13);

\path[fill=fillColor,fill opacity=0.20] (208.51, 59.98) circle (  2.13);

\path[fill=fillColor,fill opacity=0.20] (209.51, 53.35) circle (  2.13);

\path[fill=fillColor,fill opacity=0.20] (212.52, 51.63) circle (  2.13);

\path[fill=fillColor,fill opacity=0.20] (212.52, 55.51) circle (  2.13);

\path[fill=fillColor,fill opacity=0.20] (210.51, 55.68) circle (  2.13);

\path[fill=fillColor,fill opacity=0.20] (205.50, 60.07) circle (  2.13);

\path[fill=fillColor,fill opacity=0.20] (211.52, 72.64) circle (  2.13);

\path[fill=fillColor,fill opacity=0.20] (222.55, 83.92) circle (  2.13);

\path[fill=fillColor,fill opacity=0.20] (210.51, 65.67) circle (  2.13);

\path[fill=fillColor,fill opacity=0.20] (198.47, 68.25) circle (  2.13);

\path[fill=fillColor,fill opacity=0.20] (198.47, 64.29) circle (  2.13);

\path[fill=fillColor,fill opacity=0.20] (201.48, 55.94) circle (  2.13);

\path[fill=fillColor,fill opacity=0.20] (197.47, 53.44) circle (  2.13);

\path[fill=fillColor,fill opacity=0.20] (199.48, 49.82) circle (  2.13);

\path[fill=fillColor,fill opacity=0.20] (200.48, 49.31) circle (  2.13);

\path[fill=fillColor,fill opacity=0.20] (200.48, 61.45) circle (  2.13);

\path[fill=fillColor,fill opacity=0.20] (199.48, 71.35) circle (  2.13);

\path[fill=fillColor,fill opacity=0.20] (203.49, 66.79) circle (  2.13);

\path[fill=fillColor,fill opacity=0.20] (212.52, 58.35) circle (  2.13);

\path[fill=fillColor,fill opacity=0.20] (217.53, 59.64) circle (  2.13);

\path[fill=fillColor,fill opacity=0.20] (211.52, 70.49) circle (  2.13);

\path[fill=fillColor,fill opacity=0.20] (214.53, 82.80) circle (  2.13);

\path[fill=fillColor,fill opacity=0.20] (206.50, 64.20) circle (  2.13);

\path[fill=fillColor,fill opacity=0.20] (205.50, 56.20) circle (  2.13);

\path[fill=fillColor,fill opacity=0.20] (206.50, 56.54) circle (  2.13);

\path[fill=fillColor,fill opacity=0.20] (210.51, 59.47) circle (  2.13);

\path[fill=fillColor,fill opacity=0.20] (214.53, 56.63) circle (  2.13);

\path[fill=fillColor,fill opacity=0.20] (215.53, 53.61) circle (  2.13);

\path[fill=fillColor,fill opacity=0.20] (211.52, 64.20) circle (  2.13);

\path[fill=fillColor,fill opacity=0.20] (218.54, 77.55) circle (  2.13);

\path[fill=fillColor,fill opacity=0.20] (217.53, 85.30) circle (  2.13);

\path[fill=fillColor,fill opacity=0.20] (207.50, 53.53) circle (  2.13);

\path[fill=fillColor,fill opacity=0.20] (203.49, 63.43) circle (  2.13);

\path[fill=fillColor,fill opacity=0.20] (194.46, 66.61) circle (  2.13);

\path[fill=fillColor,fill opacity=0.20] (196.47, 56.71) circle (  2.13);

\path[fill=fillColor,fill opacity=0.20] (196.47, 53.96) circle (  2.13);

\path[fill=fillColor,fill opacity=0.20] (198.47, 55.08) circle (  2.13);

\path[fill=fillColor,fill opacity=0.20] (199.48, 51.46) circle (  2.13);

\path[fill=fillColor,fill opacity=0.20] (200.48, 53.61) circle (  2.13);

\path[fill=fillColor,fill opacity=0.20] (206.50, 64.29) circle (  2.13);

\path[fill=fillColor,fill opacity=0.20] (207.50, 68.34) circle (  2.13);

\path[fill=fillColor,fill opacity=0.20] (216.53, 61.62) circle (  2.13);

\path[fill=fillColor,fill opacity=0.20] (212.52, 60.16) circle (  2.13);

\path[fill=fillColor,fill opacity=0.20] (229.57, 74.71) circle (  2.13);

\path[fill=fillColor,fill opacity=0.20] (209.51, 78.24) circle (  2.13);

\path[fill=fillColor,fill opacity=0.20] (200.48, 72.90) circle (  2.13);

\path[fill=fillColor,fill opacity=0.20] (202.49, 76.60) circle (  2.13);

\path[fill=fillColor,fill opacity=0.20] (201.48, 66.96) circle (  2.13);

\path[fill=fillColor,fill opacity=0.20] (209.51, 60.16) circle (  2.13);

\path[fill=fillColor,fill opacity=0.20] (213.52, 63.51) circle (  2.13);

\path[fill=fillColor,fill opacity=0.20] (218.54, 62.48) circle (  2.13);

\path[fill=fillColor,fill opacity=0.20] (215.53, 64.89) circle (  2.13);

\path[fill=fillColor,fill opacity=0.20] (217.53, 69.37) circle (  2.13);

\path[fill=fillColor,fill opacity=0.20] (218.54, 63.86) circle (  2.13);

\path[fill=fillColor,fill opacity=0.20] (226.56, 64.38) circle (  2.13);

\path[fill=fillColor,fill opacity=0.20] (213.52, 67.99) circle (  2.13);

\path[fill=fillColor,fill opacity=0.20] (204.49, 65.75) circle (  2.13);

\path[fill=fillColor,fill opacity=0.20] (201.48, 69.63) circle (  2.13);

\path[fill=fillColor,fill opacity=0.20] (197.47, 67.13) circle (  2.13);

\path[fill=fillColor,fill opacity=0.20] (197.47, 59.12) circle (  2.13);

\path[fill=fillColor,fill opacity=0.20] (198.47, 54.39) circle (  2.13);

\path[fill=fillColor,fill opacity=0.20] (197.47, 59.21) circle (  2.13);

\path[fill=fillColor,fill opacity=0.20] (196.47, 62.57) circle (  2.13);

\path[fill=fillColor,fill opacity=0.20] (198.47, 57.83) circle (  2.13);

\path[fill=fillColor,fill opacity=0.20] (206.50, 56.37) circle (  2.13);

\path[fill=fillColor,fill opacity=0.20] (210.51, 61.19) circle (  2.13);

\path[fill=fillColor,fill opacity=0.20] (214.53, 60.16) circle (  2.13);

\path[fill=fillColor,fill opacity=0.20] (219.54, 57.57) circle (  2.13);

\path[fill=fillColor,fill opacity=0.20] (234.59, 71.52) circle (  2.13);

\path[fill=fillColor,fill opacity=0.20] (205.50, 92.45) circle (  2.13);

\path[fill=fillColor,fill opacity=0.20] (205.50, 75.92) circle (  2.13);

\path[fill=fillColor,fill opacity=0.20] (203.49, 62.31) circle (  2.13);

\path[fill=fillColor,fill opacity=0.20] (206.50, 67.13) circle (  2.13);

\path[fill=fillColor,fill opacity=0.20] (217.53, 63.26) circle (  2.13);

\path[fill=fillColor,fill opacity=0.20] (218.54, 67.48) circle (  2.13);

\path[fill=fillColor,fill opacity=0.20] (215.53, 69.89) circle (  2.13);

\path[fill=fillColor,fill opacity=0.20] (218.54, 68.25) circle (  2.13);

\path[fill=fillColor,fill opacity=0.20] (219.54, 73.50) circle (  2.13);

\path[fill=fillColor,fill opacity=0.20] (223.55, 77.47) circle (  2.13);

\path[fill=fillColor,fill opacity=0.20] (228.57, 75.83) circle (  2.13);

\path[fill=fillColor,fill opacity=0.20] (211.52, 74.37) circle (  2.13);

\path[fill=fillColor,fill opacity=0.20] (208.51, 70.15) circle (  2.13);

\path[fill=fillColor,fill opacity=0.20] (203.49, 70.58) circle (  2.13);

\path[fill=fillColor,fill opacity=0.20] (200.48, 76.52) circle (  2.13);

\path[fill=fillColor,fill opacity=0.20] (199.48, 77.55) circle (  2.13);

\path[fill=fillColor,fill opacity=0.20] (200.48, 65.75) circle (  2.13);

\path[fill=fillColor,fill opacity=0.20] (202.49, 56.11) circle (  2.13);

\path[fill=fillColor,fill opacity=0.20] (203.49, 57.92) circle (  2.13);

\path[fill=fillColor,fill opacity=0.20] (199.48, 62.31) circle (  2.13);

\path[fill=fillColor,fill opacity=0.20] (196.47, 62.31) circle (  2.13);

\path[fill=fillColor,fill opacity=0.20] (198.47, 59.12) circle (  2.13);

\path[fill=fillColor,fill opacity=0.20] (207.50, 56.20) circle (  2.13);

\path[fill=fillColor,fill opacity=0.20] (211.52, 59.04) circle (  2.13);

\path[fill=fillColor,fill opacity=0.20] (224.56, 67.13) circle (  2.13);

\path[fill=fillColor,fill opacity=0.20] (220.54, 80.05) circle (  2.13);

\path[fill=fillColor,fill opacity=0.20] (206.50, 62.83) circle (  2.13);

\path[fill=fillColor,fill opacity=0.20] (202.49, 62.57) circle (  2.13);

\path[fill=fillColor,fill opacity=0.20] (210.51, 65.06) circle (  2.13);

\path[fill=fillColor,fill opacity=0.20] (215.53, 67.05) circle (  2.13);

\path[fill=fillColor,fill opacity=0.20] (213.52, 70.06) circle (  2.13);

\path[fill=fillColor,fill opacity=0.20] (219.54, 72.64) circle (  2.13);

\path[fill=fillColor,fill opacity=0.20] (221.55, 74.11) circle (  2.13);

\path[fill=fillColor,fill opacity=0.20] (219.54, 74.02) circle (  2.13);

\path[fill=fillColor,fill opacity=0.20] (218.54, 71.35) circle (  2.13);

\path[fill=fillColor,fill opacity=0.20] (223.55, 70.75) circle (  2.13);

\path[fill=fillColor,fill opacity=0.20] (238.60, 79.79) circle (  2.13);

\path[fill=fillColor,fill opacity=0.20] (213.52, 82.46) circle (  2.13);

\path[fill=fillColor,fill opacity=0.20] (182.02, 74.28) circle (  2.13);

\path[fill=fillColor,fill opacity=0.20] (201.48, 74.28) circle (  2.13);

\path[fill=fillColor,fill opacity=0.20] (202.49, 78.93) circle (  2.13);

\path[fill=fillColor,fill opacity=0.20] (201.48, 74.62) circle (  2.13);

\path[fill=fillColor,fill opacity=0.20] (203.49, 68.42) circle (  2.13);

\path[fill=fillColor,fill opacity=0.20] (201.48, 66.61) circle (  2.13);

\path[fill=fillColor,fill opacity=0.20] (200.48, 61.10) circle (  2.13);

\path[fill=fillColor,fill opacity=0.20] (201.48, 54.90) circle (  2.13);

\path[fill=fillColor,fill opacity=0.20] (203.49, 55.68) circle (  2.13);

\path[fill=fillColor,fill opacity=0.20] (205.50, 58.95) circle (  2.13);

\path[fill=fillColor,fill opacity=0.20] (208.51, 63.17) circle (  2.13);

\path[fill=fillColor,fill opacity=0.20] (207.50, 68.34) circle (  2.13);

\path[fill=fillColor,fill opacity=0.20] (209.51, 70.92) circle (  2.13);

\path[fill=fillColor,fill opacity=0.20] (226.56, 77.38) circle (  2.13);

\path[fill=fillColor,fill opacity=0.20] (210.51, 82.46) circle (  2.13);

\path[fill=fillColor,fill opacity=0.20] (201.48, 73.50) circle (  2.13);

\path[fill=fillColor,fill opacity=0.20] (211.52, 72.13) circle (  2.13);

\path[fill=fillColor,fill opacity=0.20] (211.52, 66.79) circle (  2.13);

\path[fill=fillColor,fill opacity=0.20] (215.53, 65.06) circle (  2.13);

\path[fill=fillColor,fill opacity=0.20] (219.54, 67.91) circle (  2.13);

\path[fill=fillColor,fill opacity=0.20] (215.53, 70.32) circle (  2.13);

\path[fill=fillColor,fill opacity=0.20] (218.54, 77.12) circle (  2.13);

\path[fill=fillColor,fill opacity=0.20] (219.54, 81.68) circle (  2.13);

\path[fill=fillColor,fill opacity=0.20] (216.53, 79.70) circle (  2.13);

\path[fill=fillColor,fill opacity=0.20] (218.54, 75.57) circle (  2.13);

\path[fill=fillColor,fill opacity=0.20] (216.53, 79.27) circle (  2.13);

\path[fill=fillColor,fill opacity=0.20] (219.54, 80.57) circle (  2.13);

\path[fill=fillColor,fill opacity=0.20] (216.53, 74.62) circle (  2.13);

\path[fill=fillColor,fill opacity=0.20] (225.56, 72.47) circle (  2.13);

\path[fill=fillColor,fill opacity=0.20] (223.55, 78.76) circle (  2.13);

\path[fill=fillColor,fill opacity=0.20] (229.57, 82.12) circle (  2.13);

\path[fill=fillColor,fill opacity=0.20] (227.57, 80.22) circle (  2.13);

\path[fill=fillColor,fill opacity=0.20] (210.51, 82.63) circle (  2.13);

\path[fill=fillColor,fill opacity=0.20] (224.56, 88.57) circle (  2.13);

\path[fill=fillColor,fill opacity=0.20] (224.56, 88.57) circle (  2.13);

\path[fill=fillColor,fill opacity=0.20] (219.54, 87.45) circle (  2.13);

\path[fill=fillColor,fill opacity=0.20] (217.53, 90.99) circle (  2.13);

\path[fill=fillColor,fill opacity=0.20] (221.55, 90.38) circle (  2.13);

\path[fill=fillColor,fill opacity=0.20] (216.53, 81.86) circle (  2.13);

\path[fill=fillColor,fill opacity=0.20] (207.50, 79.88) circle (  2.13);

\path[fill=fillColor,fill opacity=0.20] (213.52, 85.22) circle (  2.13);

\path[fill=fillColor,fill opacity=0.20] (213.52, 83.06) circle (  2.13);

\path[fill=fillColor,fill opacity=0.20] (214.53, 78.15) circle (  2.13);

\path[fill=fillColor,fill opacity=0.20] (211.52, 78.58) circle (  2.13);

\path[fill=fillColor,fill opacity=0.20] (206.50, 78.15) circle (  2.13);

\path[fill=fillColor,fill opacity=0.20] (207.50, 77.72) circle (  2.13);

\path[fill=fillColor,fill opacity=0.20] (207.50, 78.76) circle (  2.13);

\path[fill=fillColor,fill opacity=0.20] (201.48, 74.02) circle (  2.13);

\path[fill=fillColor,fill opacity=0.20] (200.48, 68.60) circle (  2.13);

\path[fill=fillColor,fill opacity=0.20] (200.48, 66.01) circle (  2.13);

\path[fill=fillColor,fill opacity=0.20] (203.49, 60.67) circle (  2.13);

\path[fill=fillColor,fill opacity=0.20] (203.49, 56.63) circle (  2.13);

\path[fill=fillColor,fill opacity=0.20] (202.49, 55.59) circle (  2.13);

\path[fill=fillColor,fill opacity=0.20] (199.48, 55.25) circle (  2.13);

\path[fill=fillColor,fill opacity=0.20] (210.51, 57.23) circle (  2.13);

\path[fill=fillColor,fill opacity=0.20] (209.51, 59.90) circle (  2.13);

\path[fill=fillColor,fill opacity=0.20] (212.52, 63.69) circle (  2.13);

\path[fill=fillColor,fill opacity=0.20] (220.54, 75.40) circle (  2.13);

\path[fill=fillColor,fill opacity=0.20] (220.54, 89.87) circle (  2.13);

\path[fill=fillColor,fill opacity=0.20] (213.52, 84.96) circle (  2.13);

\path[fill=fillColor,fill opacity=0.20] (206.50, 73.25) circle (  2.13);

\path[fill=fillColor,fill opacity=0.20] (214.53, 65.93) circle (  2.13);

\path[fill=fillColor,fill opacity=0.20] (213.52, 63.95) circle (  2.13);

\path[fill=fillColor,fill opacity=0.20] (211.52, 65.93) circle (  2.13);

\path[fill=fillColor,fill opacity=0.20] (210.51, 72.90) circle (  2.13);

\path[fill=fillColor,fill opacity=0.20] (212.52, 79.36) circle (  2.13);

\path[fill=fillColor,fill opacity=0.20] (211.52, 78.67) circle (  2.13);

\path[fill=fillColor,fill opacity=0.20] (212.52, 77.38) circle (  2.13);

\path[fill=fillColor,fill opacity=0.20] (213.52, 79.62) circle (  2.13);

\path[fill=fillColor,fill opacity=0.20] (213.52, 81.00) circle (  2.13);

\path[fill=fillColor,fill opacity=0.20] (212.52, 73.33) circle (  2.13);

\path[fill=fillColor,fill opacity=0.20] (205.50, 65.06) circle (  2.13);

\path[fill=fillColor,fill opacity=0.20] (215.53, 66.01) circle (  2.13);

\path[fill=fillColor,fill opacity=0.20] (217.53, 67.48) circle (  2.13);

\path[fill=fillColor,fill opacity=0.20] (214.53, 64.72) circle (  2.13);

\path[fill=fillColor,fill opacity=0.20] (213.52, 66.27) circle (  2.13);

\path[fill=fillColor,fill opacity=0.20] (214.53, 73.68) circle (  2.13);

\path[fill=fillColor,fill opacity=0.20] (213.52, 76.52) circle (  2.13);

\path[fill=fillColor,fill opacity=0.20] (208.51, 69.97) circle (  2.13);

\path[fill=fillColor,fill opacity=0.20] (215.53, 62.57) circle (  2.13);

\path[fill=fillColor,fill opacity=0.20] (217.53, 64.98) circle (  2.13);

\path[fill=fillColor,fill opacity=0.20] (213.52, 71.35) circle (  2.13);

\path[fill=fillColor,fill opacity=0.20] (211.52, 70.15) circle (  2.13);

\path[fill=fillColor,fill opacity=0.20] (209.51, 68.42) circle (  2.13);

\path[fill=fillColor,fill opacity=0.20] (213.52, 69.20) circle (  2.13);

\path[fill=fillColor,fill opacity=0.20] (209.51, 65.06) circle (  2.13);

\path[fill=fillColor,fill opacity=0.20] (208.51, 65.32) circle (  2.13);

\path[fill=fillColor,fill opacity=0.20] (208.51, 70.75) circle (  2.13);

\path[fill=fillColor,fill opacity=0.20] (207.50, 67.39) circle (  2.13);

\path[fill=fillColor,fill opacity=0.20] (205.50, 62.83) circle (  2.13);

\path[fill=fillColor,fill opacity=0.20] (206.50, 66.87) circle (  2.13);

\path[fill=fillColor,fill opacity=0.20] (205.50, 67.30) circle (  2.13);

\path[fill=fillColor,fill opacity=0.20] (204.49, 60.85) circle (  2.13);

\path[fill=fillColor,fill opacity=0.20] (205.50, 57.23) circle (  2.13);

\path[fill=fillColor,fill opacity=0.20] (205.50, 61.02) circle (  2.13);

\path[fill=fillColor,fill opacity=0.20] (206.50, 67.73) circle (  2.13);

\path[fill=fillColor,fill opacity=0.20] (210.51, 71.44) circle (  2.13);

\path[fill=fillColor,fill opacity=0.20] (211.52, 74.37) circle (  2.13);

\path[fill=fillColor,fill opacity=0.20] (220.54, 79.53) circle (  2.13);

\path[fill=fillColor,fill opacity=0.20] (221.55, 83.23) circle (  2.13);

\path[fill=fillColor,fill opacity=0.20] (224.56, 84.44) circle (  2.13);

\path[fill=fillColor,fill opacity=0.20] (213.52, 69.63) circle (  2.13);

\path[fill=fillColor,fill opacity=0.20] (205.50, 66.96) circle (  2.13);

\path[fill=fillColor,fill opacity=0.20] (210.51, 67.73) circle (  2.13);

\path[fill=fillColor,fill opacity=0.20] (205.50, 68.85) circle (  2.13);

\path[fill=fillColor,fill opacity=0.20] (211.52, 67.65) circle (  2.13);

\path[fill=fillColor,fill opacity=0.20] (213.52, 63.26) circle (  2.13);

\path[fill=fillColor,fill opacity=0.20] (215.53, 64.98) circle (  2.13);

\path[fill=fillColor,fill opacity=0.20] (210.51, 70.32) circle (  2.13);

\path[fill=fillColor,fill opacity=0.20] (216.53, 72.30) circle (  2.13);

\path[fill=fillColor,fill opacity=0.20] (206.50, 74.37) circle (  2.13);

\path[fill=fillColor,fill opacity=0.20] (213.52, 74.62) circle (  2.13);

\path[fill=fillColor,fill opacity=0.20] (211.52, 70.92) circle (  2.13);

\path[fill=fillColor,fill opacity=0.20] (216.53, 66.10) circle (  2.13);

\path[fill=fillColor,fill opacity=0.20] (213.52, 64.98) circle (  2.13);

\path[fill=fillColor,fill opacity=0.20] (212.52, 70.58) circle (  2.13);

\path[fill=fillColor,fill opacity=0.20] (211.52, 75.05) circle (  2.13);

\path[fill=fillColor,fill opacity=0.20] (213.52, 70.92) circle (  2.13);

\path[fill=fillColor,fill opacity=0.20] (215.53, 66.53) circle (  2.13);

\path[fill=fillColor,fill opacity=0.20] (206.50, 64.20) circle (  2.13);

\path[fill=fillColor,fill opacity=0.20] (213.52, 64.20) circle (  2.13);

\path[fill=fillColor,fill opacity=0.20] (213.52, 68.42) circle (  2.13);

\path[fill=fillColor,fill opacity=0.20] (213.52, 69.71) circle (  2.13);

\path[fill=fillColor,fill opacity=0.20] (214.53, 64.98) circle (  2.13);

\path[fill=fillColor,fill opacity=0.20] (211.52, 59.73) circle (  2.13);

\path[fill=fillColor,fill opacity=0.20] (209.51, 57.57) circle (  2.13);

\path[fill=fillColor,fill opacity=0.20] (209.51, 60.33) circle (  2.13);

\path[fill=fillColor,fill opacity=0.20] (209.51, 65.58) circle (  2.13);

\path[fill=fillColor,fill opacity=0.20] (209.51, 67.13) circle (  2.13);

\path[fill=fillColor,fill opacity=0.20] (212.52, 67.99) circle (  2.13);

\path[fill=fillColor,fill opacity=0.20] (215.53, 73.25) circle (  2.13);

\path[fill=fillColor,fill opacity=0.20] (216.53, 77.90) circle (  2.13);

\path[fill=fillColor,fill opacity=0.20] (214.53, 79.88) circle (  2.13);

\path[fill=fillColor,fill opacity=0.20] (213.52, 85.47) circle (  2.13);

\path[fill=fillColor,fill opacity=0.20] (216.53, 94.86) circle (  2.13);

\path[fill=fillColor,fill opacity=0.20] (223.55,100.20) circle (  2.13);

\path[fill=fillColor,fill opacity=0.20] (230.58,101.49) circle (  2.13);

\path[fill=fillColor,fill opacity=0.20] (215.53, 78.41) circle (  2.13);

\path[fill=fillColor,fill opacity=0.20] (213.52, 74.71) circle (  2.13);

\path[fill=fillColor,fill opacity=0.20] (211.52, 75.31) circle (  2.13);

\path[fill=fillColor,fill opacity=0.20] (208.51, 69.80) circle (  2.13);

\path[fill=fillColor,fill opacity=0.20] (214.53, 57.66) circle (  2.13);

\path[fill=fillColor,fill opacity=0.20] (214.53, 54.47) circle (  2.13);

\path[fill=fillColor,fill opacity=0.20] (216.53, 58.86) circle (  2.13);

\path[fill=fillColor,fill opacity=0.20] (211.52, 63.77) circle (  2.13);

\path[fill=fillColor,fill opacity=0.20] (213.52, 74.37) circle (  2.13);

\path[fill=fillColor,fill opacity=0.20] (211.52, 77.38) circle (  2.13);

\path[fill=fillColor,fill opacity=0.20] (216.53, 69.11) circle (  2.13);

\path[fill=fillColor,fill opacity=0.20] (216.53, 62.40) circle (  2.13);

\path[fill=fillColor,fill opacity=0.20] (215.53, 63.08) circle (  2.13);

\path[fill=fillColor,fill opacity=0.20] (214.53, 67.99) circle (  2.13);

\path[fill=fillColor,fill opacity=0.20] (208.51, 70.15) circle (  2.13);

\path[fill=fillColor,fill opacity=0.20] (217.53, 64.03) circle (  2.13);

\path[fill=fillColor,fill opacity=0.20] (217.53, 61.62) circle (  2.13);

\path[fill=fillColor,fill opacity=0.20] (213.52, 65.84) circle (  2.13);

\path[fill=fillColor,fill opacity=0.20] (213.52, 68.77) circle (  2.13);

\path[fill=fillColor,fill opacity=0.20] (210.51, 70.49) circle (  2.13);

\path[fill=fillColor,fill opacity=0.20] (213.52, 70.49) circle (  2.13);

\path[fill=fillColor,fill opacity=0.20] (214.53, 67.65) circle (  2.13);

\path[fill=fillColor,fill opacity=0.20] (210.51, 66.44) circle (  2.13);

\path[fill=fillColor,fill opacity=0.20] (214.53, 71.52) circle (  2.13);

\path[fill=fillColor,fill opacity=0.20] (215.53, 78.07) circle (  2.13);

\path[fill=fillColor,fill opacity=0.20] (218.54, 85.04) circle (  2.13);

\path[fill=fillColor,fill opacity=0.20] (222.55, 91.42) circle (  2.13);

\path[fill=fillColor,fill opacity=0.20] (217.53, 77.72) circle (  2.13);

\path[fill=fillColor,fill opacity=0.20] (212.52, 72.73) circle (  2.13);

\path[fill=fillColor,fill opacity=0.20] (215.53, 71.09) circle (  2.13);

\path[fill=fillColor,fill opacity=0.20] (209.51, 72.21) circle (  2.13);

\path[fill=fillColor,fill opacity=0.20] (212.52, 77.29) circle (  2.13);

\path[fill=fillColor,fill opacity=0.20] (211.52, 76.09) circle (  2.13);

\path[fill=fillColor,fill opacity=0.20] (214.53, 70.40) circle (  2.13);

\path[fill=fillColor,fill opacity=0.20] (216.53, 67.82) circle (  2.13);

\path[fill=fillColor,fill opacity=0.20] (218.54, 67.48) circle (  2.13);

\path[fill=fillColor,fill opacity=0.20] (217.53, 69.80) circle (  2.13);

\path[fill=fillColor,fill opacity=0.20] (220.54, 72.38) circle (  2.13);

\path[fill=fillColor,fill opacity=0.20] (218.54, 69.46) circle (  2.13);

\path[fill=fillColor,fill opacity=0.20] (217.53, 68.51) circle (  2.13);

\path[fill=fillColor,fill opacity=0.20] (215.53, 74.45) circle (  2.13);

\path[fill=fillColor,fill opacity=0.20] (221.55, 80.82) circle (  2.13);

\path[fill=fillColor,fill opacity=0.20] (213.52, 85.47) circle (  2.13);

\path[fill=fillColor,fill opacity=0.20] (221.55, 92.79) circle (  2.13);

\path[fill=fillColor,fill opacity=0.20] (212.52,100.80) circle (  2.13);

\path[fill=fillColor,fill opacity=0.20] (221.55, 92.10) circle (  2.13);

\path[fill=fillColor,fill opacity=0.20] (222.55, 86.25) circle (  2.13);

\path[fill=fillColor,fill opacity=0.20] (221.55, 89.61) circle (  2.13);

\path[fill=fillColor,fill opacity=0.20] (224.56, 88.75) circle (  2.13);

\path[fill=fillColor,fill opacity=0.20] (178.11, 48.44) circle (  2.13);

\path[fill=fillColor,fill opacity=0.20] (185.23, 76.09) circle (  2.13);

\path[fill=fillColor,fill opacity=0.20] (189.44, 60.50) circle (  2.13);

\path[fill=fillColor,fill opacity=0.20] (188.44, 73.50) circle (  2.13);

\path[fill=fillColor,fill opacity=0.20] (187.84, 65.06) circle (  2.13);

\path[fill=fillColor,fill opacity=0.20] (188.44, 63.00) circle (  2.13);

\path[fill=fillColor,fill opacity=0.20] (189.44, 62.40) circle (  2.13);

\path[fill=fillColor,fill opacity=0.20] (187.34, 51.29) circle (  2.13);

\path[fill=fillColor,fill opacity=0.20] (194.46, 57.06) circle (  2.13);

\path[fill=fillColor,fill opacity=0.20] (184.53, 42.33) circle (  2.13);

\path[fill=fillColor,fill opacity=0.20] (189.44, 45.95) circle (  2.13);

\path[fill=fillColor,fill opacity=0.20] (187.94, 51.03) circle (  2.13);

\path[fill=fillColor,fill opacity=0.20] (183.83, 43.54) circle (  2.13);

\path[fill=fillColor,fill opacity=0.20] (184.03, 47.76) circle (  2.13);

\path[fill=fillColor,fill opacity=0.20] (181.32, 52.15) circle (  2.13);

\path[fill=fillColor,fill opacity=0.20] (180.62, 51.29) circle (  2.13);

\path[fill=fillColor,fill opacity=0.20] (190.45, 60.24) circle (  2.13);

\path[fill=fillColor,fill opacity=0.20] (204.49, 77.21) circle (  2.13);

\path[fill=fillColor,fill opacity=0.20] (175.00, 54.64) circle (  2.13);

\path[fill=fillColor,fill opacity=0.20] (179.61, 50.51) circle (  2.13);

\path[fill=fillColor,fill opacity=0.20] (183.32, 38.80) circle (  2.13);

\path[fill=fillColor,fill opacity=0.20] (182.22, 43.19) circle (  2.13);

\path[fill=fillColor,fill opacity=0.20] (175.20, 43.88) circle (  2.13);

\path[fill=fillColor,fill opacity=0.20] (183.43, 57.92) circle (  2.13);

\path[fill=fillColor,fill opacity=0.20] (205.50, 67.82) circle (  2.13);

\path[fill=fillColor,fill opacity=0.20] (185.73, 45.69) circle (  2.13);

\path[fill=fillColor,fill opacity=0.20] (172.49, 46.12) circle (  2.13);

\path[fill=fillColor,fill opacity=0.20] (171.19, 43.36) circle (  2.13);

\path[fill=fillColor,fill opacity=0.20] (167.37, 44.66) circle (  2.13);

\path[fill=fillColor,fill opacity=0.20] (169.48, 47.33) circle (  2.13);

\path[fill=fillColor,fill opacity=0.20] (167.37, 51.11) circle (  2.13);

\path[fill=fillColor,fill opacity=0.20] (184.83, 52.06) circle (  2.13);

\path[fill=fillColor,fill opacity=0.20] (195.46, 50.17) circle (  2.13);

\path[fill=fillColor,fill opacity=0.20] (186.84, 40.95) circle (  2.13);

\path[fill=fillColor,fill opacity=0.20] (175.60, 47.58) circle (  2.13);

\path[fill=fillColor,fill opacity=0.20] (180.42, 48.70) circle (  2.13);

\path[fill=fillColor,fill opacity=0.20] (182.32, 49.82) circle (  2.13);

\path[fill=fillColor,fill opacity=0.20] (178.01, 49.05) circle (  2.13);

\path[fill=fillColor,fill opacity=0.20] (177.21, 48.96) circle (  2.13);

\path[fill=fillColor,fill opacity=0.20] (178.81, 48.53) circle (  2.13);

\path[fill=fillColor,fill opacity=0.20] (195.46, 49.31) circle (  2.13);

\path[fill=fillColor,fill opacity=0.20] (210.51, 63.95) circle (  2.13);

\path[fill=fillColor,fill opacity=0.20] (192.45, 90.99) circle (  2.13);

\path[fill=fillColor,fill opacity=0.20] (193.46, 74.80) circle (  2.13);

\path[fill=fillColor,fill opacity=0.20] (201.48, 75.31) circle (  2.13);

\path[fill=fillColor,fill opacity=0.20] (210.51, 68.51) circle (  2.13);

\path[fill=fillColor,fill opacity=0.20] (214.53, 66.96) circle (  2.13);

\path[fill=fillColor,fill opacity=0.20] (212.52, 74.97) circle (  2.13);

\path[fill=fillColor,fill opacity=0.20] (213.52, 82.03) circle (  2.13);

\path[fill=fillColor,fill opacity=0.20] (233.59, 71.01) circle (  2.13);

\path[fill=fillColor,fill opacity=0.20] (198.47, 59.90) circle (  2.13);

\path[fill=fillColor,fill opacity=0.20] (187.94, 52.84) circle (  2.13);

\path[fill=fillColor,fill opacity=0.20] (186.13, 54.39) circle (  2.13);

\path[fill=fillColor,fill opacity=0.20] (186.33, 52.15) circle (  2.13);

\path[fill=fillColor,fill opacity=0.20] (185.43, 49.99) circle (  2.13);

\path[fill=fillColor,fill opacity=0.20] (181.52, 51.98) circle (  2.13);

\path[fill=fillColor,fill opacity=0.20] (179.21, 46.98) circle (  2.13);

\path[fill=fillColor,fill opacity=0.20] (183.93, 42.42) circle (  2.13);

\path[fill=fillColor,fill opacity=0.20] (205.50, 49.48) circle (  2.13);

\path[fill=fillColor,fill opacity=0.20] (247.63, 63.51) circle (  2.13);

\path[fill=fillColor,fill opacity=0.20] (196.47, 83.92) circle (  2.13);

\path[fill=fillColor,fill opacity=0.20] (193.46, 67.05) circle (  2.13);

\path[fill=fillColor,fill opacity=0.20] (191.45, 60.93) circle (  2.13);

\path[fill=fillColor,fill opacity=0.20] (200.48, 64.89) circle (  2.13);

\path[fill=fillColor,fill opacity=0.20] (208.51, 64.81) circle (  2.13);

\path[fill=fillColor,fill opacity=0.20] (202.49, 60.50) circle (  2.13);

\path[fill=fillColor,fill opacity=0.20] (203.49, 68.08) circle (  2.13);

\path[fill=fillColor,fill opacity=0.20] (224.56, 76.95) circle (  2.13);

\path[fill=fillColor,fill opacity=0.20] (201.48, 57.57) circle (  2.13);

\path[fill=fillColor,fill opacity=0.20] (190.45, 58.95) circle (  2.13);

\path[fill=fillColor,fill opacity=0.20] (177.21, 56.20) circle (  2.13);

\path[fill=fillColor,fill opacity=0.20] (181.32, 48.53) circle (  2.13);

\path[fill=fillColor,fill opacity=0.20] (183.53, 46.03) circle (  2.13);

\path[fill=fillColor,fill opacity=0.20] (182.92, 44.74) circle (  2.13);

\path[fill=fillColor,fill opacity=0.20] (183.12, 47.15) circle (  2.13);

\path[fill=fillColor,fill opacity=0.20] (195.46, 49.39) circle (  2.13);

\path[fill=fillColor,fill opacity=0.20] (207.50, 59.98) circle (  2.13);

\path[fill=fillColor,fill opacity=0.20] (199.48, 71.09) circle (  2.13);

\path[fill=fillColor,fill opacity=0.20] (195.46, 57.49) circle (  2.13);

\path[fill=fillColor,fill opacity=0.20] (195.46, 66.87) circle (  2.13);

\path[fill=fillColor,fill opacity=0.20] (198.47, 66.53) circle (  2.13);

\path[fill=fillColor,fill opacity=0.20] (198.47, 67.56) circle (  2.13);

\path[fill=fillColor,fill opacity=0.20] (204.49, 68.16) circle (  2.13);

\path[fill=fillColor,fill opacity=0.20] (206.50, 60.41) circle (  2.13);

\path[fill=fillColor,fill opacity=0.20] (204.49, 63.95) circle (  2.13);

\path[fill=fillColor,fill opacity=0.20] (209.51, 76.86) circle (  2.13);

\path[fill=fillColor,fill opacity=0.20] (192.45, 56.54) circle (  2.13);

\path[fill=fillColor,fill opacity=0.20] (194.46, 40.61) circle (  2.13);

\path[fill=fillColor,fill opacity=0.20] (184.63, 51.20) circle (  2.13);

\path[fill=fillColor,fill opacity=0.20] (167.57, 58.61) circle (  2.13);

\path[fill=fillColor,fill opacity=0.20] (165.37, 47.93) circle (  2.13);

\path[fill=fillColor,fill opacity=0.20] (179.31, 41.56) circle (  2.13);

\path[fill=fillColor,fill opacity=0.20] (172.79, 45.86) circle (  2.13);

\path[fill=fillColor,fill opacity=0.20] (181.32, 46.21) circle (  2.13);

\path[fill=fillColor,fill opacity=0.20] (185.83, 43.79) circle (  2.13);

\path[fill=fillColor,fill opacity=0.20] (208.51, 52.32) circle (  2.13);

\path[fill=fillColor,fill opacity=0.20] (206.50, 83.15) circle (  2.13);

\path[fill=fillColor,fill opacity=0.20] (198.47, 60.59) circle (  2.13);

\path[fill=fillColor,fill opacity=0.20] (187.14, 55.08) circle (  2.13);

\path[fill=fillColor,fill opacity=0.20] (193.46, 55.08) circle (  2.13);

\path[fill=fillColor,fill opacity=0.20] (200.48, 58.78) circle (  2.13);

\path[fill=fillColor,fill opacity=0.20] (203.49, 67.48) circle (  2.13);

\path[fill=fillColor,fill opacity=0.20] (206.50, 66.79) circle (  2.13);

\path[fill=fillColor,fill opacity=0.20] (208.51, 59.12) circle (  2.13);

\path[fill=fillColor,fill opacity=0.20] (211.52, 61.88) circle (  2.13);

\path[fill=fillColor,fill opacity=0.20] (219.54, 69.37) circle (  2.13);

\path[fill=fillColor,fill opacity=0.20] (179.91, 64.63) circle (  2.13);

\path[fill=fillColor,fill opacity=0.20] (190.45, 60.85) circle (  2.13);

\path[fill=fillColor,fill opacity=0.20] (189.44, 54.82) circle (  2.13);

\path[fill=fillColor,fill opacity=0.20] (184.53, 61.62) circle (  2.13);

\path[fill=fillColor,fill opacity=0.20] (181.62, 52.92) circle (  2.13);

\path[fill=fillColor,fill opacity=0.20] (180.62, 43.71) circle (  2.13);

\path[fill=fillColor,fill opacity=0.20] (177.00, 52.32) circle (  2.13);

\path[fill=fillColor,fill opacity=0.20] (195.46, 44.31) circle (  2.13);

\path[fill=fillColor,fill opacity=0.20] (228.57, 51.20) circle (  2.13);

\path[fill=fillColor,fill opacity=0.20] (213.52, 78.67) circle (  2.13);

\path[fill=fillColor,fill opacity=0.20] (197.47, 60.50) circle (  2.13);

\path[fill=fillColor,fill opacity=0.20] (191.45, 58.78) circle (  2.13);

\path[fill=fillColor,fill opacity=0.20] (198.47, 49.31) circle (  2.13);

\path[fill=fillColor,fill opacity=0.20] (202.49, 46.89) circle (  2.13);

\path[fill=fillColor,fill opacity=0.20] (208.51, 57.92) circle (  2.13);

\path[fill=fillColor,fill opacity=0.20] (213.52, 57.57) circle (  2.13);

\path[fill=fillColor,fill opacity=0.20] (215.53, 53.09) circle (  2.13);

\path[fill=fillColor,fill opacity=0.20] (211.52, 58.35) circle (  2.13);

\path[fill=fillColor,fill opacity=0.20] (243.62, 65.24) circle (  2.13);

\path[fill=fillColor,fill opacity=0.20] (198.47, 73.33) circle (  2.13);

\path[fill=fillColor,fill opacity=0.20] (194.46, 60.50) circle (  2.13);

\path[fill=fillColor,fill opacity=0.20] (191.45, 66.96) circle (  2.13);

\path[fill=fillColor,fill opacity=0.20] (186.94, 70.23) circle (  2.13);

\path[fill=fillColor,fill opacity=0.20] (189.44, 61.28) circle (  2.13);

\path[fill=fillColor,fill opacity=0.20] (188.34, 51.03) circle (  2.13);

\path[fill=fillColor,fill opacity=0.20] (181.32, 47.33) circle (  2.13);

\path[fill=fillColor,fill opacity=0.20] (183.32, 49.48) circle (  2.13);

\path[fill=fillColor,fill opacity=0.20] (184.33, 57.49) circle (  2.13);

\path[fill=fillColor,fill opacity=0.20] (180.42, 59.30) circle (  2.13);

\path[fill=fillColor,fill opacity=0.20] (209.51, 56.11) circle (  2.13);

\path[fill=fillColor,fill opacity=0.20] (271.71, 65.32) circle (  2.13);

\path[fill=fillColor,fill opacity=0.20] (220.54, 84.78) circle (  2.13);

\path[fill=fillColor,fill opacity=0.20] (203.49, 60.50) circle (  2.13);

\path[fill=fillColor,fill opacity=0.20] (192.45, 59.47) circle (  2.13);

\path[fill=fillColor,fill opacity=0.20] (196.47, 55.85) circle (  2.13);

\path[fill=fillColor,fill opacity=0.20] (203.49, 47.15) circle (  2.13);

\path[fill=fillColor,fill opacity=0.20] (201.48, 50.51) circle (  2.13);

\path[fill=fillColor,fill opacity=0.20] (207.50, 51.03) circle (  2.13);

\path[fill=fillColor,fill opacity=0.20] (213.52, 46.64) circle (  2.13);

\path[fill=fillColor,fill opacity=0.20] (227.57, 56.97) circle (  2.13);

\path[fill=fillColor,fill opacity=0.20] (200.48, 63.43) circle (  2.13);

\path[fill=fillColor,fill opacity=0.20] (191.45, 61.10) circle (  2.13);

\path[fill=fillColor,fill opacity=0.20] (191.45, 57.06) circle (  2.13);

\path[fill=fillColor,fill opacity=0.20] (188.34, 70.32) circle (  2.13);

\path[fill=fillColor,fill opacity=0.20] (187.14, 68.77) circle (  2.13);

\path[fill=fillColor,fill opacity=0.20] (189.44, 50.17) circle (  2.13);

\path[fill=fillColor,fill opacity=0.20] (187.44, 47.33) circle (  2.13);

\path[fill=fillColor,fill opacity=0.20] (188.24, 52.75) circle (  2.13);

\path[fill=fillColor,fill opacity=0.20] (190.45, 55.33) circle (  2.13);

\path[fill=fillColor,fill opacity=0.20] (213.52, 60.93) circle (  2.13);

\path[fill=fillColor,fill opacity=0.20] (260.67, 70.23) circle (  2.13);

\path[fill=fillColor,fill opacity=0.20] (215.53, 64.72) circle (  2.13);

\path[fill=fillColor,fill opacity=0.20] (190.45, 55.85) circle (  2.13);

\path[fill=fillColor,fill opacity=0.20] (188.44, 56.63) circle (  2.13);

\path[fill=fillColor,fill opacity=0.20] (192.45, 49.13) circle (  2.13);

\path[fill=fillColor,fill opacity=0.20] (192.45, 47.84) circle (  2.13);

\path[fill=fillColor,fill opacity=0.20] (190.45, 55.85) circle (  2.13);

\path[fill=fillColor,fill opacity=0.20] (200.48, 55.16) circle (  2.13);

\path[fill=fillColor,fill opacity=0.20] (211.52, 56.37) circle (  2.13);

\path[fill=fillColor,fill opacity=0.20] (196.47, 67.13) circle (  2.13);

\path[fill=fillColor,fill opacity=0.20] (181.32, 63.86) circle (  2.13);

\path[fill=fillColor,fill opacity=0.20] (182.62, 52.06) circle (  2.13);

\path[fill=fillColor,fill opacity=0.20] (189.44, 55.85) circle (  2.13);

\path[fill=fillColor,fill opacity=0.20] (189.44, 67.65) circle (  2.13);

\path[fill=fillColor,fill opacity=0.20] (187.74, 71.35) circle (  2.13);

\path[fill=fillColor,fill opacity=0.20] (191.45, 61.96) circle (  2.13);

\path[fill=fillColor,fill opacity=0.20] (195.46, 55.76) circle (  2.13);

\path[fill=fillColor,fill opacity=0.20] (203.49, 50.68) circle (  2.13);

\path[fill=fillColor,fill opacity=0.20] (224.56, 47.50) circle (  2.13);

\path[fill=fillColor,fill opacity=0.20] (202.49, 58.35) circle (  2.13);

\path[fill=fillColor,fill opacity=0.20] (194.46, 54.21) circle (  2.13);

\path[fill=fillColor,fill opacity=0.20] (194.46, 47.93) circle (  2.13);

\path[fill=fillColor,fill opacity=0.20] (185.83, 45.95) circle (  2.13);

\path[fill=fillColor,fill opacity=0.20] (186.64, 59.04) circle (  2.13);

\path[fill=fillColor,fill opacity=0.20] (195.46, 68.16) circle (  2.13);

\path[fill=fillColor,fill opacity=0.20] (201.48, 61.02) circle (  2.13);

\path[fill=fillColor,fill opacity=0.20] (209.51, 79.45) circle (  2.13);

\path[fill=fillColor,fill opacity=0.20] (199.48, 65.06) circle (  2.13);

\path[fill=fillColor,fill opacity=0.20] (183.43, 67.82) circle (  2.13);

\path[fill=fillColor,fill opacity=0.20] (185.23, 62.57) circle (  2.13);

\path[fill=fillColor,fill opacity=0.20] (190.45, 58.61) circle (  2.13);

\path[fill=fillColor,fill opacity=0.20] (191.45, 55.08) circle (  2.13);

\path[fill=fillColor,fill opacity=0.20] (193.46, 61.19) circle (  2.13);

\path[fill=fillColor,fill opacity=0.20] (206.50, 61.79) circle (  2.13);

\path[fill=fillColor,fill opacity=0.20] (213.52, 57.92) circle (  2.13);

\path[fill=fillColor,fill opacity=0.20] (221.55, 54.64) circle (  2.13);

\path[fill=fillColor,fill opacity=0.20] (210.51, 75.57) circle (  2.13);

\path[fill=fillColor,fill opacity=0.20] (199.48, 60.85) circle (  2.13);

\path[fill=fillColor,fill opacity=0.20] (199.48, 50.86) circle (  2.13);

\path[fill=fillColor,fill opacity=0.20] (193.46, 52.06) circle (  2.13);

\path[fill=fillColor,fill opacity=0.20] (193.46, 58.00) circle (  2.13);

\path[fill=fillColor,fill opacity=0.20] (198.47, 63.34) circle (  2.13);

\path[fill=fillColor,fill opacity=0.20] (199.48, 65.15) circle (  2.13);

\path[fill=fillColor,fill opacity=0.20] (204.49, 73.76) circle (  2.13);

\path[fill=fillColor,fill opacity=0.20] (196.47, 61.62) circle (  2.13);

\path[fill=fillColor,fill opacity=0.20] (192.45, 65.93) circle (  2.13);

\path[fill=fillColor,fill opacity=0.20] (180.21, 70.66) circle (  2.13);

\path[fill=fillColor,fill opacity=0.20] (181.22, 66.10) circle (  2.13);

\path[fill=fillColor,fill opacity=0.20] (189.44, 57.75) circle (  2.13);

\path[fill=fillColor,fill opacity=0.20] (201.48, 51.80) circle (  2.13);

\path[fill=fillColor,fill opacity=0.20] (212.52, 51.63) circle (  2.13);

\path[fill=fillColor,fill opacity=0.20] (213.52, 61.79) circle (  2.13);

\path[fill=fillColor,fill opacity=0.20] (220.54, 62.83) circle (  2.13);

\path[fill=fillColor,fill opacity=0.20] (223.55, 59.47) circle (  2.13);

\path[fill=fillColor,fill opacity=0.20] (211.52, 74.80) circle (  2.13);

\path[fill=fillColor,fill opacity=0.20] (202.49, 54.99) circle (  2.13);

\path[fill=fillColor,fill opacity=0.20] (197.47, 55.94) circle (  2.13);

\path[fill=fillColor,fill opacity=0.20] (194.46, 56.11) circle (  2.13);

\path[fill=fillColor,fill opacity=0.20] (196.47, 52.92) circle (  2.13);

\path[fill=fillColor,fill opacity=0.20] (196.47, 58.69) circle (  2.13);

\path[fill=fillColor,fill opacity=0.20] (199.48, 68.85) circle (  2.13);

\path[fill=fillColor,fill opacity=0.20] (201.48, 73.85) circle (  2.13);

\path[fill=fillColor,fill opacity=0.20] (191.45, 81.51) circle (  2.13);

\path[fill=fillColor,fill opacity=0.20] (184.13, 79.10) circle (  2.13);

\path[fill=fillColor,fill opacity=0.20] (186.23, 69.03) circle (  2.13);

\path[fill=fillColor,fill opacity=0.20] (174.40, 59.90) circle (  2.13);

\path[fill=fillColor,fill opacity=0.20] (186.33, 49.56) circle (  2.13);

\path[fill=fillColor,fill opacity=0.20] (178.01, 49.39) circle (  2.13);

\path[fill=fillColor,fill opacity=0.20] (230.58, 64.03) circle (  2.13);

\path[fill=fillColor,fill opacity=0.20] (219.54, 85.22) circle (  2.13);

\path[fill=fillColor,fill opacity=0.20] (207.50, 53.44) circle (  2.13);

\path[fill=fillColor,fill opacity=0.20] (197.47, 55.25) circle (  2.13);

\path[fill=fillColor,fill opacity=0.20] (194.46, 58.86) circle (  2.13);

\path[fill=fillColor,fill opacity=0.20] (196.47, 54.21) circle (  2.13);

\path[fill=fillColor,fill opacity=0.20] (198.47, 52.49) circle (  2.13);

\path[fill=fillColor,fill opacity=0.20] (197.47, 58.43) circle (  2.13);

\path[fill=fillColor,fill opacity=0.20] (198.47, 68.34) circle (  2.13);

\path[fill=fillColor,fill opacity=0.20] (203.49, 70.83) circle (  2.13);

\path[fill=fillColor,fill opacity=0.20] (189.44, 87.63) circle (  2.13);

\path[fill=fillColor,fill opacity=0.20] (190.45, 85.47) circle (  2.13);

\path[fill=fillColor,fill opacity=0.20] (195.46, 71.18) circle (  2.13);

\path[fill=fillColor,fill opacity=0.20] (189.44, 58.09) circle (  2.13);

\path[fill=fillColor,fill opacity=0.20] (213.52, 51.46) circle (  2.13);

\path[fill=fillColor,fill opacity=0.20] (232.58, 76.60) circle (  2.13);

\path[fill=fillColor,fill opacity=0.20] (218.54, 59.30) circle (  2.13);

\path[fill=fillColor,fill opacity=0.20] (203.49, 61.62) circle (  2.13);

\path[fill=fillColor,fill opacity=0.20] (198.47, 62.14) circle (  2.13);

\path[fill=fillColor,fill opacity=0.20] (197.47, 56.88) circle (  2.13);

\path[fill=fillColor,fill opacity=0.20] (198.47, 56.54) circle (  2.13);

\path[fill=fillColor,fill opacity=0.20] (203.49, 58.26) circle (  2.13);

\path[fill=fillColor,fill opacity=0.20] (204.49, 63.69) circle (  2.13);

\path[fill=fillColor,fill opacity=0.20] (205.50, 57.66) circle (  2.13);

\path[fill=fillColor,fill opacity=0.20] (207.50, 53.78) circle (  2.13);

\path[fill=fillColor,fill opacity=0.20] (187.84, 86.85) circle (  2.13);

\path[fill=fillColor,fill opacity=0.20] (190.45, 86.51) circle (  2.13);

\path[fill=fillColor,fill opacity=0.20] (196.47, 90.30) circle (  2.13);

\path[fill=fillColor,fill opacity=0.20] (204.49, 72.73) circle (  2.13);

\path[fill=fillColor,fill opacity=0.20] (214.53, 48.19) circle (  2.13);

\path[fill=fillColor,fill opacity=0.20] (190.45, 57.23) circle (  2.13);

\path[fill=fillColor,fill opacity=0.20] (226.56, 75.66) circle (  2.13);

\path[fill=fillColor,fill opacity=0.20] (209.51, 60.24) circle (  2.13);

\path[fill=fillColor,fill opacity=0.20] (196.47, 58.00) circle (  2.13);

\path[fill=fillColor,fill opacity=0.20] (204.49, 58.78) circle (  2.13);

\path[fill=fillColor,fill opacity=0.20] (203.49, 53.18) circle (  2.13);

\path[fill=fillColor,fill opacity=0.20] (193.46, 54.99) circle (  2.13);

\path[fill=fillColor,fill opacity=0.20] (199.48, 60.24) circle (  2.13);

\path[fill=fillColor,fill opacity=0.20] (199.48, 58.35) circle (  2.13);

\path[fill=fillColor,fill opacity=0.20] (202.49, 73.50) circle (  2.13);

\path[fill=fillColor,fill opacity=0.20] (197.47, 71.52) circle (  2.13);

\path[fill=fillColor,fill opacity=0.20] (187.94, 78.76) circle (  2.13);

\path[fill=fillColor,fill opacity=0.20] (194.46, 81.34) circle (  2.13);

\path[fill=fillColor,fill opacity=0.20] (203.49, 60.59) circle (  2.13);

\path[fill=fillColor,fill opacity=0.20] (242.62, 76.60) circle (  2.13);

\path[fill=fillColor,fill opacity=0.20] (208.51, 62.14) circle (  2.13);

\path[fill=fillColor,fill opacity=0.20] (207.50, 50.86) circle (  2.13);

\path[fill=fillColor,fill opacity=0.20] (197.47, 45.09) circle (  2.13);

\path[fill=fillColor,fill opacity=0.20] (196.47, 58.09) circle (  2.13);

\path[fill=fillColor,fill opacity=0.20] (196.47, 66.96) circle (  2.13);

\path[fill=fillColor,fill opacity=0.20] (199.48, 63.43) circle (  2.13);

\path[fill=fillColor,fill opacity=0.20] (206.50, 59.81) circle (  2.13);

\path[fill=fillColor,fill opacity=0.20] (201.48, 68.85) circle (  2.13);

\path[fill=fillColor,fill opacity=0.20] (204.49, 82.80) circle (  2.13);

\path[fill=fillColor,fill opacity=0.20] (202.49, 77.03) circle (  2.13);

\path[fill=fillColor,fill opacity=0.20] (201.48, 81.51) circle (  2.13);

\path[fill=fillColor,fill opacity=0.20] (198.47, 74.19) circle (  2.13);

\path[fill=fillColor,fill opacity=0.20] (196.47, 66.61) circle (  2.13);

\path[fill=fillColor,fill opacity=0.20] (196.47, 70.23) circle (  2.13);

\path[fill=fillColor,fill opacity=0.20] (229.57, 83.06) circle (  2.13);

\path[fill=fillColor,fill opacity=0.20] (205.50, 64.81) circle (  2.13);

\path[fill=fillColor,fill opacity=0.20] (197.47, 58.95) circle (  2.13);

\path[fill=fillColor,fill opacity=0.20] (191.45, 61.36) circle (  2.13);

\path[fill=fillColor,fill opacity=0.20] (194.46, 61.79) circle (  2.13);

\path[fill=fillColor,fill opacity=0.20] (197.47, 62.14) circle (  2.13);

\path[fill=fillColor,fill opacity=0.20] (205.50, 58.35) circle (  2.13);

\path[fill=fillColor,fill opacity=0.20] (202.49, 68.60) circle (  2.13);

\path[fill=fillColor,fill opacity=0.20] (202.49, 77.64) circle (  2.13);

\path[fill=fillColor,fill opacity=0.20] (209.51, 70.15) circle (  2.13);

\path[fill=fillColor,fill opacity=0.20] (215.53, 63.17) circle (  2.13);

\path[fill=fillColor,fill opacity=0.20] (218.54, 71.44) circle (  2.13);

\path[fill=fillColor,fill opacity=0.20] (214.53, 88.66) circle (  2.13);

\path[fill=fillColor,fill opacity=0.20] (210.51, 74.37) circle (  2.13);

\path[fill=fillColor,fill opacity=0.20] (206.50, 67.13) circle (  2.13);

\path[fill=fillColor,fill opacity=0.20] (203.49, 64.46) circle (  2.13);

\path[fill=fillColor,fill opacity=0.20] (203.49, 70.92) circle (  2.13);

\path[fill=fillColor,fill opacity=0.20] (197.47, 72.99) circle (  2.13);

\path[fill=fillColor,fill opacity=0.20] (195.46, 65.84) circle (  2.13);

\path[fill=fillColor,fill opacity=0.20] (198.47, 59.30) circle (  2.13);

\path[fill=fillColor,fill opacity=0.20] (215.53, 56.71) circle (  2.13);

\path[fill=fillColor,fill opacity=0.20] (214.53, 85.65) circle (  2.13);

\path[fill=fillColor,fill opacity=0.20] (230.58, 85.82) circle (  2.13);

\path[fill=fillColor,fill opacity=0.20] (210.51, 69.37) circle (  2.13);

\path[fill=fillColor,fill opacity=0.20] (197.47, 59.81) circle (  2.13);

\path[fill=fillColor,fill opacity=0.20] (198.47, 58.18) circle (  2.13);

\path[fill=fillColor,fill opacity=0.20] (200.48, 54.04) circle (  2.13);

\path[fill=fillColor,fill opacity=0.20] (202.49, 62.48) circle (  2.13);

\path[fill=fillColor,fill opacity=0.20] (203.49, 76.86) circle (  2.13);

\path[fill=fillColor,fill opacity=0.20] (205.50, 74.37) circle (  2.13);

\path[fill=fillColor,fill opacity=0.20] (206.50, 65.06) circle (  2.13);

\path[fill=fillColor,fill opacity=0.20] (209.51, 62.40) circle (  2.13);

\path[fill=fillColor,fill opacity=0.20] (217.53, 62.31) circle (  2.13);

\path[fill=fillColor,fill opacity=0.20] (203.49, 70.32) circle (  2.13);

\path[fill=fillColor,fill opacity=0.20] (216.53, 81.77) circle (  2.13);

\path[fill=fillColor,fill opacity=0.20] (212.52, 77.29) circle (  2.13);

\path[fill=fillColor,fill opacity=0.20] (209.51, 64.98) circle (  2.13);

\path[fill=fillColor,fill opacity=0.20] (214.53, 54.99) circle (  2.13);

\path[fill=fillColor,fill opacity=0.20] (213.52, 54.39) circle (  2.13);

\path[fill=fillColor,fill opacity=0.20] (212.52, 60.07) circle (  2.13);

\path[fill=fillColor,fill opacity=0.20] (200.48, 60.16) circle (  2.13);

\path[fill=fillColor,fill opacity=0.20] (196.47, 55.33) circle (  2.13);

\path[fill=fillColor,fill opacity=0.20] (200.48, 62.22) circle (  2.13);

\path[fill=fillColor,fill opacity=0.20] (204.49, 63.34) circle (  2.13);

\path[fill=fillColor,fill opacity=0.20] (203.49, 52.66) circle (  2.13);

\path[fill=fillColor,fill opacity=0.20] (222.55, 52.92) circle (  2.13);

\path[fill=fillColor,fill opacity=0.20] (199.48, 68.34) circle (  2.13);

\path[fill=fillColor,fill opacity=0.20] (231.58, 86.34) circle (  2.13);

\path[fill=fillColor,fill opacity=0.20] (203.49, 69.11) circle (  2.13);

\path[fill=fillColor,fill opacity=0.20] (202.49, 56.54) circle (  2.13);

\path[fill=fillColor,fill opacity=0.20] (202.49, 52.58) circle (  2.13);

\path[fill=fillColor,fill opacity=0.20] (200.48, 57.23) circle (  2.13);

\path[fill=fillColor,fill opacity=0.20] (204.49, 63.08) circle (  2.13);

\path[fill=fillColor,fill opacity=0.20] (200.48, 66.36) circle (  2.13);

\path[fill=fillColor,fill opacity=0.20] (198.47, 64.20) circle (  2.13);

\path[fill=fillColor,fill opacity=0.20] (206.50, 61.45) circle (  2.13);

\path[fill=fillColor,fill opacity=0.20] (211.52, 66.96) circle (  2.13);

\path[fill=fillColor,fill opacity=0.20] (212.52, 69.20) circle (  2.13);

\path[fill=fillColor,fill opacity=0.20] (212.52, 62.65) circle (  2.13);

\path[fill=fillColor,fill opacity=0.20] (210.51, 57.92) circle (  2.13);

\path[fill=fillColor,fill opacity=0.20] (214.53, 56.71) circle (  2.13);

\path[fill=fillColor,fill opacity=0.20] (210.51, 56.02) circle (  2.13);

\path[fill=fillColor,fill opacity=0.20] (213.52, 61.02) circle (  2.13);

\path[fill=fillColor,fill opacity=0.20] (211.52, 58.95) circle (  2.13);

\path[fill=fillColor,fill opacity=0.20] (209.51, 53.44) circle (  2.13);

\path[fill=fillColor,fill opacity=0.20] (208.51, 58.09) circle (  2.13);

\path[fill=fillColor,fill opacity=0.20] (214.53, 63.08) circle (  2.13);

\path[fill=fillColor,fill opacity=0.20] (216.53, 67.30) circle (  2.13);

\path[fill=fillColor,fill opacity=0.20] (211.52, 69.71) circle (  2.13);

\path[fill=fillColor,fill opacity=0.20] (211.52, 63.26) circle (  2.13);

\path[fill=fillColor,fill opacity=0.20] (213.52, 58.35) circle (  2.13);

\path[fill=fillColor,fill opacity=0.20] (212.52, 61.19) circle (  2.13);

\path[fill=fillColor,fill opacity=0.20] (209.51, 63.17) circle (  2.13);

\path[fill=fillColor,fill opacity=0.20] (205.50, 65.41) circle (  2.13);

\path[fill=fillColor,fill opacity=0.20] (204.49, 61.71) circle (  2.13);

\path[fill=fillColor,fill opacity=0.20] (202.49, 60.93) circle (  2.13);

\path[fill=fillColor,fill opacity=0.20] (200.48, 66.96) circle (  2.13);

\path[fill=fillColor,fill opacity=0.20] (200.48, 63.17) circle (  2.13);

\path[fill=fillColor,fill opacity=0.20] (197.47, 53.35) circle (  2.13);

\path[fill=fillColor,fill opacity=0.20] (204.49, 52.15) circle (  2.13);

\path[fill=fillColor,fill opacity=0.20] (213.52, 51.37) circle (  2.13);

\path[fill=fillColor,fill opacity=0.20] (206.50, 75.31) circle (  2.13);

\path[fill=fillColor,fill opacity=0.20] (222.55, 64.46) circle (  2.13);

\path[fill=fillColor,fill opacity=0.20] (211.52, 48.44) circle (  2.13);

\path[fill=fillColor,fill opacity=0.20] (200.48, 49.91) circle (  2.13);

\path[fill=fillColor,fill opacity=0.20] (201.48, 65.93) circle (  2.13);

\path[fill=fillColor,fill opacity=0.20] (199.48, 69.28) circle (  2.13);

\path[fill=fillColor,fill opacity=0.20] (201.48, 61.36) circle (  2.13);

\path[fill=fillColor,fill opacity=0.20] (202.49, 62.74) circle (  2.13);

\path[fill=fillColor,fill opacity=0.20] (206.50, 69.97) circle (  2.13);

\path[fill=fillColor,fill opacity=0.20] (201.48, 72.30) circle (  2.13);

\path[fill=fillColor,fill opacity=0.20] (199.48, 67.91) circle (  2.13);

\path[fill=fillColor,fill opacity=0.20] (201.48, 59.12) circle (  2.13);

\path[fill=fillColor,fill opacity=0.20] (206.50, 55.85) circle (  2.13);

\path[fill=fillColor,fill opacity=0.20] (205.50, 59.64) circle (  2.13);

\path[fill=fillColor,fill opacity=0.20] (205.50, 60.50) circle (  2.13);

\path[fill=fillColor,fill opacity=0.20] (207.50, 58.09) circle (  2.13);

\path[fill=fillColor,fill opacity=0.20] (205.50, 57.66) circle (  2.13);

\path[fill=fillColor,fill opacity=0.20] (208.51, 58.35) circle (  2.13);

\path[fill=fillColor,fill opacity=0.20] (210.51, 60.67) circle (  2.13);

\path[fill=fillColor,fill opacity=0.20] (203.49, 62.22) circle (  2.13);

\path[fill=fillColor,fill opacity=0.20] (206.50, 63.60) circle (  2.13);

\path[fill=fillColor,fill opacity=0.20] (203.49, 63.69) circle (  2.13);

\path[fill=fillColor,fill opacity=0.20] (201.48, 63.51) circle (  2.13);

\path[fill=fillColor,fill opacity=0.20] (201.48, 67.82) circle (  2.13);

\path[fill=fillColor,fill opacity=0.20] (195.46, 69.89) circle (  2.13);

\path[fill=fillColor,fill opacity=0.20] (198.47, 60.33) circle (  2.13);

\path[fill=fillColor,fill opacity=0.20] (201.48, 55.51) circle (  2.13);

\path[fill=fillColor,fill opacity=0.20] (191.45, 60.07) circle (  2.13);

\path[fill=fillColor,fill opacity=0.20] (215.53, 55.25) circle (  2.13);

\path[fill=fillColor,fill opacity=0.20] (223.55, 55.42) circle (  2.13);

\path[fill=fillColor,fill opacity=0.20] (190.45, 71.61) circle (  2.13);

\path[fill=fillColor,fill opacity=0.20] (264.69, 76.00) circle (  2.13);

\path[fill=fillColor,fill opacity=0.20] (228.57, 69.46) circle (  2.13);

\path[fill=fillColor,fill opacity=0.20] (186.64, 75.66) circle (  2.13);

\path[fill=fillColor,fill opacity=0.20] (202.49, 78.76) circle (  2.13);

\path[fill=fillColor,fill opacity=0.20] (214.53, 67.56) circle (  2.13);

\path[fill=fillColor,fill opacity=0.20] (207.50, 56.80) circle (  2.13);

\path[fill=fillColor,fill opacity=0.20] (206.50, 61.45) circle (  2.13);

\path[fill=fillColor,fill opacity=0.20] (193.46, 66.96) circle (  2.13);

\path[fill=fillColor,fill opacity=0.20] (197.47, 61.71) circle (  2.13);

\path[fill=fillColor,fill opacity=0.20] (199.48, 52.92) circle (  2.13);

\path[fill=fillColor,fill opacity=0.20] (204.49, 50.08) circle (  2.13);

\path[fill=fillColor,fill opacity=0.20] (200.48, 52.92) circle (  2.13);

\path[fill=fillColor,fill opacity=0.20] (203.49, 58.43) circle (  2.13);

\path[fill=fillColor,fill opacity=0.20] (202.49, 62.05) circle (  2.13);

\path[fill=fillColor,fill opacity=0.20] (207.50, 61.88) circle (  2.13);

\path[fill=fillColor,fill opacity=0.20] (211.52, 61.19) circle (  2.13);

\path[fill=fillColor,fill opacity=0.20] (212.52, 61.10) circle (  2.13);

\path[fill=fillColor,fill opacity=0.20] (200.48, 56.63) circle (  2.13);

\path[fill=fillColor,fill opacity=0.20] (206.50, 51.11) circle (  2.13);

\path[fill=fillColor,fill opacity=0.20] (215.53, 51.63) circle (  2.13);

\path[fill=fillColor,fill opacity=0.20] (191.45, 55.59) circle (  2.13);

\path[fill=fillColor,fill opacity=0.20] (208.51, 53.61) circle (  2.13);

\path[fill=fillColor,fill opacity=0.20] (209.51, 48.88) circle (  2.13);

\path[fill=fillColor,fill opacity=0.20] (195.46, 48.53) circle (  2.13);

\path[fill=fillColor,fill opacity=0.20] (228.57, 49.65) circle (  2.13);

\path[fill=fillColor,fill opacity=0.20] (236.60, 56.28) circle (  2.13);

\path[fill=fillColor,fill opacity=0.20] (257.66, 76.26) circle (  2.13);

\path[fill=fillColor,fill opacity=0.20] (251.64, 67.91) circle (  2.13);

\path[fill=fillColor,fill opacity=0.20] (234.59, 59.73) circle (  2.13);

\path[fill=fillColor,fill opacity=0.20] (240.61, 53.44) circle (  2.13);

\path[fill=fillColor,fill opacity=0.20] (238.60, 52.32) circle (  2.13);

\path[fill=fillColor,fill opacity=0.20] (216.53, 51.29) circle (  2.13);

\path[fill=fillColor,fill opacity=0.20] (226.56, 51.37) circle (  2.13);

\path[fill=fillColor,fill opacity=0.20] (229.57, 56.97) circle (  2.13);

\path[fill=fillColor,fill opacity=0.20] (239.61, 63.34) circle (  2.13);

\path[fill=fillColor,fill opacity=0.20] (223.55, 68.51) circle (  2.13);

\path[fill=fillColor,fill opacity=0.20] (220.54, 71.44) circle (  2.13);

\path[fill=fillColor,fill opacity=0.20] (222.55, 76.09) circle (  2.13);

\path[fill=fillColor,fill opacity=0.20] (226.56, 79.45) circle (  2.13);

\path[fill=fillColor,fill opacity=0.20] (203.49, 59.90) circle (  2.13);

\path[fill=fillColor,fill opacity=0.20] (218.54, 68.77) circle (  2.13);

\path[fill=fillColor,fill opacity=0.20] (250.64, 76.00) circle (  2.13);

\path[fill=fillColor,fill opacity=0.20] (209.51, 86.25) circle (  2.13);

\path[fill=fillColor,fill opacity=0.20] (197.47, 93.31) circle (  2.13);

\path[fill=fillColor,fill opacity=0.20] (187.84, 87.63) circle (  2.13);

\path[fill=fillColor,fill opacity=0.20] (191.45, 86.94) circle (  2.13);

\path[fill=fillColor,fill opacity=0.20] (197.47, 86.59) circle (  2.13);

\path[fill=fillColor,fill opacity=0.20] (205.50, 80.48) circle (  2.13);

\path[fill=fillColor,fill opacity=0.20] (208.51, 72.82) circle (  2.13);

\path[fill=fillColor,fill opacity=0.20] (208.51, 72.30) circle (  2.13);

\path[fill=fillColor,fill opacity=0.20] (209.51, 78.50) circle (  2.13);

\path[fill=fillColor,fill opacity=0.20] (214.53, 86.08) circle (  2.13);

\path[fill=fillColor,fill opacity=0.20] (217.53, 90.30) circle (  2.13);

\path[fill=fillColor,fill opacity=0.20] (210.51, 93.14) circle (  2.13);

\path[fill=fillColor,fill opacity=0.20] (192.45, 86.16) circle (  2.13);

\path[fill=fillColor,fill opacity=0.20] (200.48, 71.01) circle (  2.13);

\path[fill=fillColor,fill opacity=0.20] (173.29, 74.88) circle (  2.13);

\path[fill=fillColor,fill opacity=0.20] (197.47, 82.29) circle (  2.13);

\path[fill=fillColor,fill opacity=0.20] (199.48, 80.82) circle (  2.13);

\path[fill=fillColor,fill opacity=0.20] (204.49, 74.80) circle (  2.13);

\path[fill=fillColor,fill opacity=0.20] (209.51, 70.58) circle (  2.13);

\path[fill=fillColor,fill opacity=0.20] (209.51, 67.99) circle (  2.13);

\path[fill=fillColor,fill opacity=0.20] (209.51, 69.71) circle (  2.13);

\path[fill=fillColor,fill opacity=0.20] (217.53, 81.77) circle (  2.13);

\path[fill=fillColor,fill opacity=0.20] (219.54, 90.81) circle (  2.13);

\path[fill=fillColor,fill opacity=0.20] (218.54, 91.33) circle (  2.13);

\path[fill=fillColor,fill opacity=0.20] (196.47, 98.74) circle (  2.13);

\path[fill=fillColor,fill opacity=0.20] (195.46, 80.05) circle (  2.13);

\path[fill=fillColor,fill opacity=0.20] (199.48, 66.44) circle (  2.13);

\path[fill=fillColor,fill opacity=0.20] (200.48, 66.70) circle (  2.13);

\path[fill=fillColor,fill opacity=0.20] (201.48, 71.18) circle (  2.13);

\path[fill=fillColor,fill opacity=0.20] (203.49, 69.80) circle (  2.13);

\path[fill=fillColor,fill opacity=0.20] (205.50, 68.42) circle (  2.13);

\path[fill=fillColor,fill opacity=0.20] (209.51, 71.61) circle (  2.13);

\path[fill=fillColor,fill opacity=0.20] (211.52, 72.56) circle (  2.13);

\path[fill=fillColor,fill opacity=0.20] (215.53, 73.68) circle (  2.13);

\path[fill=fillColor,fill opacity=0.20] (224.56, 82.72) circle (  2.13);

\path[fill=fillColor,fill opacity=0.20] (168.38, 78.41) circle (  2.13);

\path[fill=fillColor,fill opacity=0.20] (195.46, 96.24) circle (  2.13);

\path[fill=fillColor,fill opacity=0.20] (200.48, 79.62) circle (  2.13);

\path[fill=fillColor,fill opacity=0.20] (202.49, 69.89) circle (  2.13);

\path[fill=fillColor,fill opacity=0.20] (203.49, 64.63) circle (  2.13);

\path[fill=fillColor,fill opacity=0.20] (205.50, 63.43) circle (  2.13);

\path[fill=fillColor,fill opacity=0.20] (206.50, 63.17) circle (  2.13);

\path[fill=fillColor,fill opacity=0.20] (209.51, 64.38) circle (  2.13);

\path[fill=fillColor,fill opacity=0.20] (211.52, 71.78) circle (  2.13);

\path[fill=fillColor,fill opacity=0.20] (215.53, 77.81) circle (  2.13);

\path[fill=fillColor,fill opacity=0.20] (223.55, 78.93) circle (  2.13);

\path[fill=fillColor,fill opacity=0.20] (238.60, 84.96) circle (  2.13);

\path[fill=fillColor,fill opacity=0.20] (207.50, 67.05) circle (  2.13);

\path[fill=fillColor,fill opacity=0.20] (204.49, 66.79) circle (  2.13);

\path[fill=fillColor,fill opacity=0.20] (188.44, 63.43) circle (  2.13);

\path[fill=fillColor,fill opacity=0.20] (210.51,106.14) circle (  2.13);

\path[fill=fillColor,fill opacity=0.20] (198.47, 86.77) circle (  2.13);

\path[fill=fillColor,fill opacity=0.20] (201.48, 73.25) circle (  2.13);

\path[fill=fillColor,fill opacity=0.20] (205.50, 69.89) circle (  2.13);

\path[fill=fillColor,fill opacity=0.20] (205.50, 64.98) circle (  2.13);

\path[fill=fillColor,fill opacity=0.20] (208.51, 62.31) circle (  2.13);

\path[fill=fillColor,fill opacity=0.20] (210.51, 65.50) circle (  2.13);

\path[fill=fillColor,fill opacity=0.20] (212.52, 67.56) circle (  2.13);

\path[fill=fillColor,fill opacity=0.20] (213.52, 72.38) circle (  2.13);

\path[fill=fillColor,fill opacity=0.20] (222.55, 79.62) circle (  2.13);

\path[fill=fillColor,fill opacity=0.20] (209.51, 66.10) circle (  2.13);

\path[fill=fillColor,fill opacity=0.20] (211.52, 61.79) circle (  2.13);

\path[fill=fillColor,fill opacity=0.20] (201.48, 63.69) circle (  2.13);

\path[fill=fillColor,fill opacity=0.20] (193.46, 56.63) circle (  2.13);

\path[fill=fillColor,fill opacity=0.20] (189.44, 53.87) circle (  2.13);

\path[fill=fillColor,fill opacity=0.20] (189.44, 55.33) circle (  2.13);

\path[fill=fillColor,fill opacity=0.20] (196.47, 61.02) circle (  2.13);

\path[fill=fillColor,fill opacity=0.20] (204.49, 69.28) circle (  2.13);

\path[fill=fillColor,fill opacity=0.20] (207.50,101.15) circle (  2.13);

\path[fill=fillColor,fill opacity=0.20] (196.47, 77.29) circle (  2.13);

\path[fill=fillColor,fill opacity=0.20] (203.49, 66.27) circle (  2.13);

\path[fill=fillColor,fill opacity=0.20] (208.51, 68.08) circle (  2.13);

\path[fill=fillColor,fill opacity=0.20] (208.51, 63.60) circle (  2.13);

\path[fill=fillColor,fill opacity=0.20] (214.53, 60.67) circle (  2.13);

\path[fill=fillColor,fill opacity=0.20] (214.53, 66.53) circle (  2.13);

\path[fill=fillColor,fill opacity=0.20] (212.52, 71.35) circle (  2.13);

\path[fill=fillColor,fill opacity=0.20] (214.53, 75.48) circle (  2.13);

\path[fill=fillColor,fill opacity=0.20] (206.50, 63.60) circle (  2.13);

\path[fill=fillColor,fill opacity=0.20] (199.48, 66.36) circle (  2.13);

\path[fill=fillColor,fill opacity=0.20] (198.47, 63.43) circle (  2.13);

\path[fill=fillColor,fill opacity=0.20] (201.48, 54.82) circle (  2.13);

\path[fill=fillColor,fill opacity=0.20] (189.44, 48.62) circle (  2.13);

\path[fill=fillColor,fill opacity=0.20] (189.44, 46.98) circle (  2.13);

\path[fill=fillColor,fill opacity=0.20] (194.46, 53.09) circle (  2.13);

\path[fill=fillColor,fill opacity=0.20] (198.47, 60.24) circle (  2.13);

\path[fill=fillColor,fill opacity=0.20] (198.47, 63.51) circle (  2.13);

\path[fill=fillColor,fill opacity=0.20] (165.87, 67.48) circle (  2.13);

\path[fill=fillColor,fill opacity=0.20] (210.51,104.76) circle (  2.13);

\path[fill=fillColor,fill opacity=0.20] (198.47, 83.15) circle (  2.13);

\path[fill=fillColor,fill opacity=0.20] (206.50, 66.10) circle (  2.13);

\path[fill=fillColor,fill opacity=0.20] (209.51, 66.96) circle (  2.13);

\path[fill=fillColor,fill opacity=0.20] (206.50, 67.56) circle (  2.13);

\path[fill=fillColor,fill opacity=0.20] (210.51, 62.74) circle (  2.13);

\path[fill=fillColor,fill opacity=0.20] (215.53, 62.65) circle (  2.13);

\path[fill=fillColor,fill opacity=0.20] (212.52, 69.97) circle (  2.13);

\path[fill=fillColor,fill opacity=0.20] (213.52, 80.39) circle (  2.13);

\path[fill=fillColor,fill opacity=0.20] (206.50, 84.78) circle (  2.13);

\path[fill=fillColor,fill opacity=0.20] (200.48, 49.56) circle (  2.13);

\path[fill=fillColor,fill opacity=0.20] (202.49, 58.35) circle (  2.13);

\path[fill=fillColor,fill opacity=0.20] (199.48, 58.52) circle (  2.13);

\path[fill=fillColor,fill opacity=0.20] (191.45, 48.96) circle (  2.13);

\path[fill=fillColor,fill opacity=0.20] (189.44, 44.40) circle (  2.13);

\path[fill=fillColor,fill opacity=0.20] (189.44, 46.55) circle (  2.13);

\path[fill=fillColor,fill opacity=0.20] (192.45, 51.11) circle (  2.13);

\path[fill=fillColor,fill opacity=0.20] (195.46, 48.88) circle (  2.13);

\path[fill=fillColor,fill opacity=0.20] (198.47, 46.81) circle (  2.13);

\path[fill=fillColor,fill opacity=0.20] (199.48, 60.67) circle (  2.13);

\path[fill=fillColor,fill opacity=0.20] (193.46, 77.64) circle (  2.13);

\path[fill=fillColor,fill opacity=0.20] (203.49, 88.14) circle (  2.13);

\path[fill=fillColor,fill opacity=0.20] (203.49, 64.98) circle (  2.13);

\path[fill=fillColor,fill opacity=0.20] (206.50, 64.98) circle (  2.13);

\path[fill=fillColor,fill opacity=0.20] (207.50, 73.59) circle (  2.13);

\path[fill=fillColor,fill opacity=0.20] (204.49, 67.65) circle (  2.13);

\path[fill=fillColor,fill opacity=0.20] (209.51, 57.49) circle (  2.13);

\path[fill=fillColor,fill opacity=0.20] (215.53, 62.91) circle (  2.13);

\path[fill=fillColor,fill opacity=0.20] (217.53, 77.81) circle (  2.13);

\path[fill=fillColor,fill opacity=0.20] (218.54, 83.75) circle (  2.13);

\path[fill=fillColor,fill opacity=0.20] (223.55, 84.18) circle (  2.13);

\path[fill=fillColor,fill opacity=0.20] (202.49, 64.98) circle (  2.13);

\path[fill=fillColor,fill opacity=0.20] (199.48, 49.39) circle (  2.13);

\path[fill=fillColor,fill opacity=0.20] (199.48, 47.33) circle (  2.13);

\path[fill=fillColor,fill opacity=0.20] (195.46, 39.32) circle (  2.13);

\path[fill=fillColor,fill opacity=0.20] (195.46, 44.31) circle (  2.13);

\path[fill=fillColor,fill opacity=0.20] (192.45, 49.91) circle (  2.13);

\path[fill=fillColor,fill opacity=0.20] (189.44, 49.91) circle (  2.13);

\path[fill=fillColor,fill opacity=0.20] (190.45, 44.83) circle (  2.13);

\path[fill=fillColor,fill opacity=0.20] (193.46, 38.20) circle (  2.13);

\path[fill=fillColor,fill opacity=0.20] (188.44, 51.11) circle (  2.13);

\path[fill=fillColor,fill opacity=0.20] (201.48, 70.23) circle (  2.13);

\path[fill=fillColor,fill opacity=0.20] (191.45, 67.48) circle (  2.13);

\path[fill=fillColor,fill opacity=0.20] (202.49, 81.68) circle (  2.13);

\path[fill=fillColor,fill opacity=0.20] (198.47, 60.16) circle (  2.13);

\path[fill=fillColor,fill opacity=0.20] (204.49, 61.45) circle (  2.13);

\path[fill=fillColor,fill opacity=0.20] (208.51, 69.80) circle (  2.13);

\path[fill=fillColor,fill opacity=0.20] (213.52, 63.08) circle (  2.13);

\path[fill=fillColor,fill opacity=0.20] (218.54, 54.04) circle (  2.13);

\path[fill=fillColor,fill opacity=0.20] (217.53, 59.04) circle (  2.13);

\path[fill=fillColor,fill opacity=0.20] (216.53, 68.68) circle (  2.13);

\path[fill=fillColor,fill opacity=0.20] (219.54, 72.30) circle (  2.13);

\path[fill=fillColor,fill opacity=0.20] (221.55, 72.64) circle (  2.13);

\path[fill=fillColor,fill opacity=0.20] (199.48, 63.08) circle (  2.13);

\path[fill=fillColor,fill opacity=0.20] (194.46, 43.62) circle (  2.13);

\path[fill=fillColor,fill opacity=0.20] (202.49, 51.20) circle (  2.13);

\path[fill=fillColor,fill opacity=0.20] (200.48, 42.42) circle (  2.13);

\path[fill=fillColor,fill opacity=0.20] (198.47, 40.78) circle (  2.13);

\path[fill=fillColor,fill opacity=0.20] (196.47, 48.88) circle (  2.13);

\path[fill=fillColor,fill opacity=0.20] (194.46, 49.31) circle (  2.13);

\path[fill=fillColor,fill opacity=0.20] (191.45, 45.26) circle (  2.13);

\path[fill=fillColor,fill opacity=0.20] (191.45, 44.48) circle (  2.13);

\path[fill=fillColor,fill opacity=0.20] (190.45, 40.18) circle (  2.13);

\path[fill=fillColor,fill opacity=0.20] (190.45, 42.59) circle (  2.13);

\path[fill=fillColor,fill opacity=0.20] (202.49, 56.11) circle (  2.13);

\path[fill=fillColor,fill opacity=0.20] (210.51, 64.55) circle (  2.13);

\path[fill=fillColor,fill opacity=0.20] (205.50, 84.10) circle (  2.13);

\path[fill=fillColor,fill opacity=0.20] (198.47, 62.57) circle (  2.13);

\path[fill=fillColor,fill opacity=0.20] (205.50, 60.33) circle (  2.13);

\path[fill=fillColor,fill opacity=0.20] (211.52, 63.08) circle (  2.13);

\path[fill=fillColor,fill opacity=0.20] (213.52, 56.28) circle (  2.13);

\path[fill=fillColor,fill opacity=0.20] (216.53, 54.13) circle (  2.13);

\path[fill=fillColor,fill opacity=0.20] (213.52, 61.62) circle (  2.13);

\path[fill=fillColor,fill opacity=0.20] (214.53, 66.96) circle (  2.13);

\path[fill=fillColor,fill opacity=0.20] (214.53, 66.70) circle (  2.13);

\path[fill=fillColor,fill opacity=0.20] (217.53, 63.08) circle (  2.13);

\path[fill=fillColor,fill opacity=0.20] (223.55, 66.27) circle (  2.13);

\path[fill=fillColor,fill opacity=0.20] (200.48, 71.44) circle (  2.13);

\path[fill=fillColor,fill opacity=0.20] (193.46, 53.44) circle (  2.13);

\path[fill=fillColor,fill opacity=0.20] (198.47, 55.59) circle (  2.13);

\path[fill=fillColor,fill opacity=0.20] (199.48, 49.48) circle (  2.13);

\path[fill=fillColor,fill opacity=0.20] (200.48, 53.61) circle (  2.13);

\path[fill=fillColor,fill opacity=0.20] (195.46, 51.63) circle (  2.13);

\path[fill=fillColor,fill opacity=0.20] (194.46, 43.45) circle (  2.13);

\path[fill=fillColor,fill opacity=0.20] (193.46, 46.03) circle (  2.13);

\path[fill=fillColor,fill opacity=0.20] (192.45, 47.50) circle (  2.13);

\path[fill=fillColor,fill opacity=0.20] (190.45, 40.09) circle (  2.13);

\path[fill=fillColor,fill opacity=0.20] (198.47, 43.79) circle (  2.13);

\path[fill=fillColor,fill opacity=0.20] (208.51, 57.31) circle (  2.13);

\path[fill=fillColor,fill opacity=0.20] (225.56,101.32) circle (  2.13);

\path[fill=fillColor,fill opacity=0.20] (207.50, 74.45) circle (  2.13);

\path[fill=fillColor,fill opacity=0.20] (207.50, 63.43) circle (  2.13);

\path[fill=fillColor,fill opacity=0.20] (210.51, 63.95) circle (  2.13);

\path[fill=fillColor,fill opacity=0.20] (211.52, 60.33) circle (  2.13);

\path[fill=fillColor,fill opacity=0.20] (210.51, 56.37) circle (  2.13);

\path[fill=fillColor,fill opacity=0.20] (212.52, 61.19) circle (  2.13);

\path[fill=fillColor,fill opacity=0.20] (212.52, 70.06) circle (  2.13);

\path[fill=fillColor,fill opacity=0.20] (215.53, 71.44) circle (  2.13);

\path[fill=fillColor,fill opacity=0.20] (215.53, 60.93) circle (  2.13);

\path[fill=fillColor,fill opacity=0.20] (216.53, 54.82) circle (  2.13);

\path[fill=fillColor,fill opacity=0.20] (224.56, 66.96) circle (  2.13);

\path[fill=fillColor,fill opacity=0.20] (228.57, 88.23) circle (  2.13);

\path[fill=fillColor,fill opacity=0.20] (202.49, 58.43) circle (  2.13);

\path[fill=fillColor,fill opacity=0.20] (197.47, 57.23) circle (  2.13);

\path[fill=fillColor,fill opacity=0.20] (197.47, 54.30) circle (  2.13);

\path[fill=fillColor,fill opacity=0.20] (194.46, 60.76) circle (  2.13);

\path[fill=fillColor,fill opacity=0.20] (190.45, 53.18) circle (  2.13);

\path[fill=fillColor,fill opacity=0.20] (190.45, 41.13) circle (  2.13);

\path[fill=fillColor,fill opacity=0.20] (190.45, 51.72) circle (  2.13);

\path[fill=fillColor,fill opacity=0.20] (191.45, 53.61) circle (  2.13);

\path[fill=fillColor,fill opacity=0.20] (192.45, 39.40) circle (  2.13);

\path[fill=fillColor,fill opacity=0.20] (203.49, 43.36) circle (  2.13);

\path[fill=fillColor,fill opacity=0.20] (205.50, 49.39) circle (  2.13);

\path[fill=fillColor,fill opacity=0.20] (211.52, 76.00) circle (  2.13);

\path[fill=fillColor,fill opacity=0.20] (228.57, 89.69) circle (  2.13);

\path[fill=fillColor,fill opacity=0.20] (209.51, 69.89) circle (  2.13);

\path[fill=fillColor,fill opacity=0.20] (209.51, 63.08) circle (  2.13);

\path[fill=fillColor,fill opacity=0.20] (207.50, 61.88) circle (  2.13);

\path[fill=fillColor,fill opacity=0.20] (209.51, 59.30) circle (  2.13);

\path[fill=fillColor,fill opacity=0.20] (209.51, 58.52) circle (  2.13);

\path[fill=fillColor,fill opacity=0.20] (211.52, 65.06) circle (  2.13);

\path[fill=fillColor,fill opacity=0.20] (211.52, 69.28) circle (  2.13);

\path[fill=fillColor,fill opacity=0.20] (210.51, 62.57) circle (  2.13);

\path[fill=fillColor,fill opacity=0.20] (213.52, 57.49) circle (  2.13);

\path[fill=fillColor,fill opacity=0.20] (222.55, 65.84) circle (  2.13);

\path[fill=fillColor,fill opacity=0.20] (219.54, 76.00) circle (  2.13);

\path[fill=fillColor,fill opacity=0.20] (223.55, 83.92) circle (  2.13);

\path[fill=fillColor,fill opacity=0.20] (209.51, 73.50) circle (  2.13);

\path[fill=fillColor,fill opacity=0.20] (201.48, 59.73) circle (  2.13);

\path[fill=fillColor,fill opacity=0.20] (200.48, 57.92) circle (  2.13);

\path[fill=fillColor,fill opacity=0.20] (194.46, 53.61) circle (  2.13);

\path[fill=fillColor,fill opacity=0.20] (191.45, 55.94) circle (  2.13);

\path[fill=fillColor,fill opacity=0.20] (191.45, 54.04) circle (  2.13);

\path[fill=fillColor,fill opacity=0.20] (191.45, 47.58) circle (  2.13);

\path[fill=fillColor,fill opacity=0.20] (189.44, 53.18) circle (  2.13);

\path[fill=fillColor,fill opacity=0.20] (194.46, 52.41) circle (  2.13);

\path[fill=fillColor,fill opacity=0.20] (196.47, 43.02) circle (  2.13);

\path[fill=fillColor,fill opacity=0.20] (194.46, 46.55) circle (  2.13);

\path[fill=fillColor,fill opacity=0.20] (198.47, 48.10) circle (  2.13);

\path[fill=fillColor,fill opacity=0.20] (207.50, 45.52) circle (  2.13);

\path[fill=fillColor,fill opacity=0.20] (184.73, 77.03) circle (  2.13);

\path[fill=fillColor,fill opacity=0.20] (218.54, 80.82) circle (  2.13);

\path[fill=fillColor,fill opacity=0.20] (200.48, 59.30) circle (  2.13);

\path[fill=fillColor,fill opacity=0.20] (210.51, 54.90) circle (  2.13);

\path[fill=fillColor,fill opacity=0.20] (211.52, 59.21) circle (  2.13);

\path[fill=fillColor,fill opacity=0.20] (208.51, 59.04) circle (  2.13);

\path[fill=fillColor,fill opacity=0.20] (213.52, 58.00) circle (  2.13);

\path[fill=fillColor,fill opacity=0.20] (193.46, 60.50) circle (  2.13);

\path[fill=fillColor,fill opacity=0.20] (210.51, 61.79) circle (  2.13);

\path[fill=fillColor,fill opacity=0.20] (216.53, 64.03) circle (  2.13);

\path[fill=fillColor,fill opacity=0.20] (218.54, 68.60) circle (  2.13);

\path[fill=fillColor,fill opacity=0.20] (217.53, 72.82) circle (  2.13);

\path[fill=fillColor,fill opacity=0.20] (219.54, 78.33) circle (  2.13);

\path[fill=fillColor,fill opacity=0.20] (221.55, 77.98) circle (  2.13);

\path[fill=fillColor,fill opacity=0.20] (224.56, 79.88) circle (  2.13);

\path[fill=fillColor,fill opacity=0.20] (205.50, 84.61) circle (  2.13);

\path[fill=fillColor,fill opacity=0.20] (200.48, 66.96) circle (  2.13);

\path[fill=fillColor,fill opacity=0.20] (200.48, 58.61) circle (  2.13);

\path[fill=fillColor,fill opacity=0.20] (200.48, 58.95) circle (  2.13);

\path[fill=fillColor,fill opacity=0.20] (200.48, 53.53) circle (  2.13);

\path[fill=fillColor,fill opacity=0.20] (196.47, 46.89) circle (  2.13);

\path[fill=fillColor,fill opacity=0.20] (189.44, 46.98) circle (  2.13);

\path[fill=fillColor,fill opacity=0.20] (192.45, 51.20) circle (  2.13);

\path[fill=fillColor,fill opacity=0.20] (191.45, 51.63) circle (  2.13);

\path[fill=fillColor,fill opacity=0.20] (193.46, 44.83) circle (  2.13);

\path[fill=fillColor,fill opacity=0.20] (198.47, 48.19) circle (  2.13);

\path[fill=fillColor,fill opacity=0.20] (193.46, 56.71) circle (  2.13);

\path[fill=fillColor,fill opacity=0.20] (192.45, 49.31) circle (  2.13);

\path[fill=fillColor,fill opacity=0.20] (209.51, 48.19) circle (  2.13);

\path[fill=fillColor,fill opacity=0.20] (239.61, 74.71) circle (  2.13);

\path[fill=fillColor,fill opacity=0.20] (224.56, 62.74) circle (  2.13);

\path[fill=fillColor,fill opacity=0.20] (215.53, 59.73) circle (  2.13);

\path[fill=fillColor,fill opacity=0.20] (206.50, 57.83) circle (  2.13);

\path[fill=fillColor,fill opacity=0.20] (211.52, 57.14) circle (  2.13);

\path[fill=fillColor,fill opacity=0.20] (214.53, 56.71) circle (  2.13);

\path[fill=fillColor,fill opacity=0.20] (205.50, 57.75) circle (  2.13);

\path[fill=fillColor,fill opacity=0.20] (214.53, 65.75) circle (  2.13);

\path[fill=fillColor,fill opacity=0.20] (214.53, 70.15) circle (  2.13);

\path[fill=fillColor,fill opacity=0.20] (217.53, 69.89) circle (  2.13);

\path[fill=fillColor,fill opacity=0.20] (220.54, 76.52) circle (  2.13);

\path[fill=fillColor,fill opacity=0.20] (215.53, 75.66) circle (  2.13);

\path[fill=fillColor,fill opacity=0.20] (216.53, 70.49) circle (  2.13);

\path[fill=fillColor,fill opacity=0.20] (218.54, 78.58) circle (  2.13);

\path[fill=fillColor,fill opacity=0.20] (215.53, 89.35) circle (  2.13);

\path[fill=fillColor,fill opacity=0.20] (210.51, 79.62) circle (  2.13);

\path[fill=fillColor,fill opacity=0.20] (209.51, 78.84) circle (  2.13);

\path[fill=fillColor,fill opacity=0.20] (207.50, 76.43) circle (  2.13);

\path[fill=fillColor,fill opacity=0.20] (203.49, 76.86) circle (  2.13);

\path[fill=fillColor,fill opacity=0.20] (199.48, 72.30) circle (  2.13);

\path[fill=fillColor,fill opacity=0.20] (196.47, 64.03) circle (  2.13);

\path[fill=fillColor,fill opacity=0.20] (196.47, 59.12) circle (  2.13);

\path[fill=fillColor,fill opacity=0.20] (198.47, 58.09) circle (  2.13);

\path[fill=fillColor,fill opacity=0.20] (202.49, 54.82) circle (  2.13);

\path[fill=fillColor,fill opacity=0.20] (196.47, 47.07) circle (  2.13);

\path[fill=fillColor,fill opacity=0.20] (196.47, 42.33) circle (  2.13);

\path[fill=fillColor,fill opacity=0.20] (189.44, 47.41) circle (  2.13);

\path[fill=fillColor,fill opacity=0.20] (193.46, 50.25) circle (  2.13);

\path[fill=fillColor,fill opacity=0.20] (194.46, 47.24) circle (  2.13);

\path[fill=fillColor,fill opacity=0.20] (194.46, 52.66) circle (  2.13);

\path[fill=fillColor,fill opacity=0.20] (192.45, 57.23) circle (  2.13);

\path[fill=fillColor,fill opacity=0.20] (202.49, 49.82) circle (  2.13);

\path[fill=fillColor,fill opacity=0.20] (208.51, 55.85) circle (  2.13);

\path[fill=fillColor,fill opacity=0.20] (168.08, 81.68) circle (  2.13);

\path[fill=fillColor,fill opacity=0.20] (233.59, 92.71) circle (  2.13);

\path[fill=fillColor,fill opacity=0.20] (224.56, 76.60) circle (  2.13);

\path[fill=fillColor,fill opacity=0.20] (216.53, 63.86) circle (  2.13);

\path[fill=fillColor,fill opacity=0.20] (218.54, 60.24) circle (  2.13);

\path[fill=fillColor,fill opacity=0.20] (209.51, 55.94) circle (  2.13);

\path[fill=fillColor,fill opacity=0.20] (209.51, 54.21) circle (  2.13);

\path[fill=fillColor,fill opacity=0.20] (210.51, 65.24) circle (  2.13);

\path[fill=fillColor,fill opacity=0.20] (210.51, 74.19) circle (  2.13);

\path[fill=fillColor,fill opacity=0.20] (217.53, 70.92) circle (  2.13);

\path[fill=fillColor,fill opacity=0.20] (218.54, 72.73) circle (  2.13);

\path[fill=fillColor,fill opacity=0.20] (212.52, 75.31) circle (  2.13);

\path[fill=fillColor,fill opacity=0.20] (213.52, 74.71) circle (  2.13);

\path[fill=fillColor,fill opacity=0.20] (213.52, 75.92) circle (  2.13);

\path[fill=fillColor,fill opacity=0.20] (217.53, 70.15) circle (  2.13);

\path[fill=fillColor,fill opacity=0.20] (225.56, 64.81) circle (  2.13);

\path[fill=fillColor,fill opacity=0.20] (226.56, 75.57) circle (  2.13);

\path[fill=fillColor,fill opacity=0.20] (215.53, 87.11) circle (  2.13);

\path[fill=fillColor,fill opacity=0.20] (216.53, 83.58) circle (  2.13);

\path[fill=fillColor,fill opacity=0.20] (215.53, 82.20) circle (  2.13);

\path[fill=fillColor,fill opacity=0.20] (197.47, 77.55) circle (  2.13);

\path[fill=fillColor,fill opacity=0.20] (200.48, 77.47) circle (  2.13);

\path[fill=fillColor,fill opacity=0.20] (203.49, 76.78) circle (  2.13);

\path[fill=fillColor,fill opacity=0.20] (203.49, 70.92) circle (  2.13);

\path[fill=fillColor,fill opacity=0.20] (199.48, 72.21) circle (  2.13);

\path[fill=fillColor,fill opacity=0.20] (199.48, 70.49) circle (  2.13);

\path[fill=fillColor,fill opacity=0.20] (198.47, 62.05) circle (  2.13);

\path[fill=fillColor,fill opacity=0.20] (199.48, 56.37) circle (  2.13);

\path[fill=fillColor,fill opacity=0.20] (198.47, 57.06) circle (  2.13);

\path[fill=fillColor,fill opacity=0.20] (198.47, 56.37) circle (  2.13);

\path[fill=fillColor,fill opacity=0.20] (199.48, 55.85) circle (  2.13);

\path[fill=fillColor,fill opacity=0.20] (196.47, 52.58) circle (  2.13);

\path[fill=fillColor,fill opacity=0.20] (194.46, 47.84) circle (  2.13);

\path[fill=fillColor,fill opacity=0.20] (192.45, 49.99) circle (  2.13);

\path[fill=fillColor,fill opacity=0.20] (196.47, 57.23) circle (  2.13);

\path[fill=fillColor,fill opacity=0.20] (196.47, 56.11) circle (  2.13);

\path[fill=fillColor,fill opacity=0.20] (193.46, 50.08) circle (  2.13);

\path[fill=fillColor,fill opacity=0.20] (195.46, 49.22) circle (  2.13);

\path[fill=fillColor,fill opacity=0.20] (213.52, 58.69) circle (  2.13);

\path[fill=fillColor,fill opacity=0.20] (208.51, 79.27) circle (  2.13);

\path[fill=fillColor,fill opacity=0.20] (182.72, 94.00) circle (  2.13);

\path[fill=fillColor,fill opacity=0.20] (229.57, 80.05) circle (  2.13);

\path[fill=fillColor,fill opacity=0.20] (223.55, 69.11) circle (  2.13);

\path[fill=fillColor,fill opacity=0.20] (207.50, 63.26) circle (  2.13);

\path[fill=fillColor,fill opacity=0.20] (209.51, 63.08) circle (  2.13);

\path[fill=fillColor,fill opacity=0.20] (209.51, 69.03) circle (  2.13);

\path[fill=fillColor,fill opacity=0.20] (210.51, 69.20) circle (  2.13);

\path[fill=fillColor,fill opacity=0.20] (209.51, 66.79) circle (  2.13);

\path[fill=fillColor,fill opacity=0.20] (212.52, 67.91) circle (  2.13);

\path[fill=fillColor,fill opacity=0.20] (210.51, 70.06) circle (  2.13);

\path[fill=fillColor,fill opacity=0.20] (214.53, 66.87) circle (  2.13);

\path[fill=fillColor,fill opacity=0.20] (218.54, 60.93) circle (  2.13);

\path[fill=fillColor,fill opacity=0.20] (220.54, 65.67) circle (  2.13);

\path[fill=fillColor,fill opacity=0.20] (220.54, 73.33) circle (  2.13);

\path[fill=fillColor,fill opacity=0.20] (218.54, 70.23) circle (  2.13);

\path[fill=fillColor,fill opacity=0.20] (219.54, 73.42) circle (  2.13);

\path[fill=fillColor,fill opacity=0.20] (212.52, 87.11) circle (  2.13);

\path[fill=fillColor,fill opacity=0.20] (212.52, 85.99) circle (  2.13);

\path[fill=fillColor,fill opacity=0.20] (211.52, 77.03) circle (  2.13);

\path[fill=fillColor,fill opacity=0.20] (200.48, 65.93) circle (  2.13);

\path[fill=fillColor,fill opacity=0.20] (200.48, 64.89) circle (  2.13);

\path[fill=fillColor,fill opacity=0.20] (206.50, 66.87) circle (  2.13);

\path[fill=fillColor,fill opacity=0.20] (206.50, 61.79) circle (  2.13);

\path[fill=fillColor,fill opacity=0.20] (196.47, 62.14) circle (  2.13);

\path[fill=fillColor,fill opacity=0.20] (193.46, 62.40) circle (  2.13);

\path[fill=fillColor,fill opacity=0.20] (200.48, 52.84) circle (  2.13);

\path[fill=fillColor,fill opacity=0.20] (198.47, 49.22) circle (  2.13);

\path[fill=fillColor,fill opacity=0.20] (197.47, 56.37) circle (  2.13);

\path[fill=fillColor,fill opacity=0.20] (199.48, 58.61) circle (  2.13);

\path[fill=fillColor,fill opacity=0.20] (198.47, 60.59) circle (  2.13);

\path[fill=fillColor,fill opacity=0.20] (199.48, 61.45) circle (  2.13);

\path[fill=fillColor,fill opacity=0.20] (199.48, 53.44) circle (  2.13);

\path[fill=fillColor,fill opacity=0.20] (199.48, 52.41) circle (  2.13);

\path[fill=fillColor,fill opacity=0.20] (202.49, 63.34) circle (  2.13);

\path[fill=fillColor,fill opacity=0.20] (201.48, 60.41) circle (  2.13);

\path[fill=fillColor,fill opacity=0.20] (194.46, 47.84) circle (  2.13);

\path[fill=fillColor,fill opacity=0.20] (197.47, 45.00) circle (  2.13);

\path[fill=fillColor,fill opacity=0.20] (216.53, 51.98) circle (  2.13);

\path[fill=fillColor,fill opacity=0.20] (223.55, 77.03) circle (  2.13);

\path[fill=fillColor,fill opacity=0.20] (243.62,102.44) circle (  2.13);

\path[fill=fillColor,fill opacity=0.20] (232.58, 87.28) circle (  2.13);

\path[fill=fillColor,fill opacity=0.20] (219.54, 69.28) circle (  2.13);

\path[fill=fillColor,fill opacity=0.20] (219.54, 64.55) circle (  2.13);

\path[fill=fillColor,fill opacity=0.20] (217.53, 66.61) circle (  2.13);

\path[fill=fillColor,fill opacity=0.20] (208.51, 60.50) circle (  2.13);

\path[fill=fillColor,fill opacity=0.20] (212.52, 62.48) circle (  2.13);

\path[fill=fillColor,fill opacity=0.20] (214.53, 68.94) circle (  2.13);

\path[fill=fillColor,fill opacity=0.20] (206.50, 60.50) circle (  2.13);

\path[fill=fillColor,fill opacity=0.20] (216.53, 57.40) circle (  2.13);

\path[fill=fillColor,fill opacity=0.20] (214.53, 72.90) circle (  2.13);

\path[fill=fillColor,fill opacity=0.20] (214.53, 74.62) circle (  2.13);

\path[fill=fillColor,fill opacity=0.20] (214.53, 56.63) circle (  2.13);

\path[fill=fillColor,fill opacity=0.20] (214.53, 56.97) circle (  2.13);

\path[fill=fillColor,fill opacity=0.20] (218.54, 79.62) circle (  2.13);

\path[fill=fillColor,fill opacity=0.20] (210.51, 94.00) circle (  2.13);

\path[fill=fillColor,fill opacity=0.20] (207.50, 73.42) circle (  2.13);

\path[fill=fillColor,fill opacity=0.20] (209.51, 55.85) circle (  2.13);

\path[fill=fillColor,fill opacity=0.20] (215.53, 62.83) circle (  2.13);

\path[fill=fillColor,fill opacity=0.20] (215.53, 73.50) circle (  2.13);

\path[fill=fillColor,fill opacity=0.20] (207.50, 74.19) circle (  2.13);

\path[fill=fillColor,fill opacity=0.20] (212.52, 66.36) circle (  2.13);

\path[fill=fillColor,fill opacity=0.20] (210.51, 53.09) circle (  2.13);

\path[fill=fillColor,fill opacity=0.20] (208.51, 47.50) circle (  2.13);

\path[fill=fillColor,fill opacity=0.20] (208.51, 54.39) circle (  2.13);

\path[fill=fillColor,fill opacity=0.20] (206.50, 59.47) circle (  2.13);

\path[fill=fillColor,fill opacity=0.20] (200.48, 57.49) circle (  2.13);

\path[fill=fillColor,fill opacity=0.20] (200.48, 52.23) circle (  2.13);

\path[fill=fillColor,fill opacity=0.20] (202.49, 46.29) circle (  2.13);

\path[fill=fillColor,fill opacity=0.20] (193.46, 46.89) circle (  2.13);

\path[fill=fillColor,fill opacity=0.20] (201.48, 54.04) circle (  2.13);

\path[fill=fillColor,fill opacity=0.20] (200.48, 56.88) circle (  2.13);

\path[fill=fillColor,fill opacity=0.20] (202.49, 59.12) circle (  2.13);

\path[fill=fillColor,fill opacity=0.20] (203.49, 60.59) circle (  2.13);

\path[fill=fillColor,fill opacity=0.20] (203.49, 55.59) circle (  2.13);

\path[fill=fillColor,fill opacity=0.20] (200.48, 54.04) circle (  2.13);

\path[fill=fillColor,fill opacity=0.20] (199.48, 62.05) circle (  2.13);

\path[fill=fillColor,fill opacity=0.20] (197.47, 62.65) circle (  2.13);

\path[fill=fillColor,fill opacity=0.20] (202.49, 50.77) circle (  2.13);

\path[fill=fillColor,fill opacity=0.20] (205.50, 42.50) circle (  2.13);

\path[fill=fillColor,fill opacity=0.20] (219.54, 54.73) circle (  2.13);

\path[fill=fillColor,fill opacity=0.20] (243.62, 91.76) circle (  2.13);

\path[fill=fillColor,fill opacity=0.20] (232.58, 87.02) circle (  2.13);

\path[fill=fillColor,fill opacity=0.20] (233.59, 79.45) circle (  2.13);

\path[fill=fillColor,fill opacity=0.20] (213.52, 68.51) circle (  2.13);

\path[fill=fillColor,fill opacity=0.20] (219.54, 71.70) circle (  2.13);

\path[fill=fillColor,fill opacity=0.20] (217.53, 73.59) circle (  2.13);

\path[fill=fillColor,fill opacity=0.20] (214.53, 60.67) circle (  2.13);

\path[fill=fillColor,fill opacity=0.20] (209.51, 59.12) circle (  2.13);

\path[fill=fillColor,fill opacity=0.20] (212.52, 72.73) circle (  2.13);

\path[fill=fillColor,fill opacity=0.20] (215.53, 68.42) circle (  2.13);

\path[fill=fillColor,fill opacity=0.20] (215.53, 53.61) circle (  2.13);

\path[fill=fillColor,fill opacity=0.20] (208.51, 59.04) circle (  2.13);

\path[fill=fillColor,fill opacity=0.20] (213.52, 75.83) circle (  2.13);

\path[fill=fillColor,fill opacity=0.20] (217.53, 79.19) circle (  2.13);

\path[fill=fillColor,fill opacity=0.20] (212.52, 71.44) circle (  2.13);

\path[fill=fillColor,fill opacity=0.20] (216.53, 68.34) circle (  2.13);

\path[fill=fillColor,fill opacity=0.20] (219.54, 76.95) circle (  2.13);

\path[fill=fillColor,fill opacity=0.20] (211.52, 80.48) circle (  2.13);

\path[fill=fillColor,fill opacity=0.20] (213.52, 68.34) circle (  2.13);

\path[fill=fillColor,fill opacity=0.20] (216.53, 73.76) circle (  2.13);

\path[fill=fillColor,fill opacity=0.20] (224.56, 87.54) circle (  2.13);

\path[fill=fillColor,fill opacity=0.20] (220.54, 90.38) circle (  2.13);

\path[fill=fillColor,fill opacity=0.20] (216.53, 82.29) circle (  2.13);

\path[fill=fillColor,fill opacity=0.20] (215.53, 68.34) circle (  2.13);

\path[fill=fillColor,fill opacity=0.20] (208.51, 50.34) circle (  2.13);

\path[fill=fillColor,fill opacity=0.20] (209.51, 42.50) circle (  2.13);

\path[fill=fillColor,fill opacity=0.20] (212.52, 45.78) circle (  2.13);

\path[fill=fillColor,fill opacity=0.20] (210.51, 47.24) circle (  2.13);

\path[fill=fillColor,fill opacity=0.20] (205.50, 48.88) circle (  2.13);

\path[fill=fillColor,fill opacity=0.20] (205.50, 54.47) circle (  2.13);

\path[fill=fillColor,fill opacity=0.20] (199.48, 55.25) circle (  2.13);

\path[fill=fillColor,fill opacity=0.20] (204.49, 51.98) circle (  2.13);

\path[fill=fillColor,fill opacity=0.20] (201.48, 50.60) circle (  2.13);

\path[fill=fillColor,fill opacity=0.20] (200.48, 51.98) circle (  2.13);

\path[fill=fillColor,fill opacity=0.20] (200.48, 52.15) circle (  2.13);

\path[fill=fillColor,fill opacity=0.20] (205.50, 52.58) circle (  2.13);

\path[fill=fillColor,fill opacity=0.20] (204.49, 56.45) circle (  2.13);

\path[fill=fillColor,fill opacity=0.20] (205.50, 58.43) circle (  2.13);

\path[fill=fillColor,fill opacity=0.20] (206.50, 55.59) circle (  2.13);

\path[fill=fillColor,fill opacity=0.20] (204.49, 53.27) circle (  2.13);

\path[fill=fillColor,fill opacity=0.20] (196.47, 56.88) circle (  2.13);

\path[fill=fillColor,fill opacity=0.20] (193.46, 58.09) circle (  2.13);

\path[fill=fillColor,fill opacity=0.20] (208.51, 50.17) circle (  2.13);

\path[fill=fillColor,fill opacity=0.20] (217.53, 53.18) circle (  2.13);

\path[fill=fillColor,fill opacity=0.20] (217.53, 82.37) circle (  2.13);

\path[fill=fillColor,fill opacity=0.20] (259.67, 88.32) circle (  2.13);

\path[fill=fillColor,fill opacity=0.20] (229.57, 87.97) circle (  2.13);

\path[fill=fillColor,fill opacity=0.20] (226.56, 83.06) circle (  2.13);

\path[fill=fillColor,fill opacity=0.20] (220.54, 73.16) circle (  2.13);

\path[fill=fillColor,fill opacity=0.20] (209.51, 69.46) circle (  2.13);

\path[fill=fillColor,fill opacity=0.20] (210.51, 66.96) circle (  2.13);

\path[fill=fillColor,fill opacity=0.20] (217.53, 60.93) circle (  2.13);

\path[fill=fillColor,fill opacity=0.20] (216.53, 57.06) circle (  2.13);

\path[fill=fillColor,fill opacity=0.20] (213.52, 56.71) circle (  2.13);

\path[fill=fillColor,fill opacity=0.20] (214.53, 62.14) circle (  2.13);

\path[fill=fillColor,fill opacity=0.20] (216.53, 68.08) circle (  2.13);

\path[fill=fillColor,fill opacity=0.20] (205.50, 64.81) circle (  2.13);

\path[fill=fillColor,fill opacity=0.20] (222.55, 64.03) circle (  2.13);

\path[fill=fillColor,fill opacity=0.20] (224.56, 70.06) circle (  2.13);

\path[fill=fillColor,fill opacity=0.20] (221.55, 67.99) circle (  2.13);

\path[fill=fillColor,fill opacity=0.20] (216.53, 63.08) circle (  2.13);

\path[fill=fillColor,fill opacity=0.20] (216.53, 70.15) circle (  2.13);

\path[fill=fillColor,fill opacity=0.20] (211.52, 97.44) circle (  2.13);

\path[fill=fillColor,fill opacity=0.20] (216.53, 85.13) circle (  2.13);

\path[fill=fillColor,fill opacity=0.20] (221.55, 76.43) circle (  2.13);

\path[fill=fillColor,fill opacity=0.20] (208.51, 81.08) circle (  2.13);

\path[fill=fillColor,fill opacity=0.20] (218.54, 80.48) circle (  2.13);

\path[fill=fillColor,fill opacity=0.20] (213.52, 67.82) circle (  2.13);

\path[fill=fillColor,fill opacity=0.20] (218.54, 64.20) circle (  2.13);

\path[fill=fillColor,fill opacity=0.20] (211.52, 71.35) circle (  2.13);

\path[fill=fillColor,fill opacity=0.20] (211.52, 71.52) circle (  2.13);

\path[fill=fillColor,fill opacity=0.20] (209.51, 70.49) circle (  2.13);

\path[fill=fillColor,fill opacity=0.20] (217.53, 77.90) circle (  2.13);

\path[fill=fillColor,fill opacity=0.20] (226.56, 80.74) circle (  2.13);

\path[fill=fillColor,fill opacity=0.20] (233.59, 88.92) circle (  2.13);

\path[fill=fillColor,fill opacity=0.20] (238.60,113.37) circle (  2.13);

\path[fill=fillColor,fill opacity=0.20] (217.53, 84.70) circle (  2.13);

\path[fill=fillColor,fill opacity=0.20] (212.52, 78.58) circle (  2.13);

\path[fill=fillColor,fill opacity=0.20] (211.52, 75.31) circle (  2.13);

\path[fill=fillColor,fill opacity=0.20] (207.50, 65.75) circle (  2.13);

\path[fill=fillColor,fill opacity=0.20] (195.46, 56.71) circle (  2.13);

\path[fill=fillColor,fill opacity=0.20] (206.50, 56.97) circle (  2.13);

\path[fill=fillColor,fill opacity=0.20] (200.48, 57.40) circle (  2.13);

\path[fill=fillColor,fill opacity=0.20] (201.48, 56.71) circle (  2.13);

\path[fill=fillColor,fill opacity=0.20] (203.49, 57.23) circle (  2.13);

\path[fill=fillColor,fill opacity=0.20] (204.49, 55.51) circle (  2.13);

\path[fill=fillColor,fill opacity=0.20] (204.49, 53.78) circle (  2.13);

\path[fill=fillColor,fill opacity=0.20] (202.49, 55.68) circle (  2.13);

\path[fill=fillColor,fill opacity=0.20] (202.49, 58.00) circle (  2.13);

\path[fill=fillColor,fill opacity=0.20] (197.47, 58.78) circle (  2.13);

\path[fill=fillColor,fill opacity=0.20] (204.49, 56.63) circle (  2.13);

\path[fill=fillColor,fill opacity=0.20] (200.48, 55.08) circle (  2.13);

\path[fill=fillColor,fill opacity=0.20] (208.51, 55.85) circle (  2.13);

\path[fill=fillColor,fill opacity=0.20] (221.55, 62.22) circle (  2.13);

\path[fill=fillColor,fill opacity=0.20] (230.58, 80.57) circle (  2.13);

\path[fill=fillColor,fill opacity=0.20] (222.55, 92.36) circle (  2.13);

\path[fill=fillColor,fill opacity=0.20] (217.53, 87.11) circle (  2.13);

\path[fill=fillColor,fill opacity=0.20] (221.55, 78.76) circle (  2.13);

\path[fill=fillColor,fill opacity=0.20] (222.55, 74.45) circle (  2.13);

\path[fill=fillColor,fill opacity=0.20] (216.53, 71.18) circle (  2.13);

\path[fill=fillColor,fill opacity=0.20] (209.51, 58.69) circle (  2.13);

\path[fill=fillColor,fill opacity=0.20] (218.54, 50.34) circle (  2.13);

\path[fill=fillColor,fill opacity=0.20] (217.53, 54.39) circle (  2.13);

\path[fill=fillColor,fill opacity=0.20] (218.54, 58.18) circle (  2.13);

\path[fill=fillColor,fill opacity=0.20] (218.54, 65.58) circle (  2.13);

\path[fill=fillColor,fill opacity=0.20] (212.52, 73.42) circle (  2.13);

\path[fill=fillColor,fill opacity=0.20] (214.53, 68.85) circle (  2.13);

\path[fill=fillColor,fill opacity=0.20] (224.56, 58.78) circle (  2.13);

\path[fill=fillColor,fill opacity=0.20] (214.53, 58.95) circle (  2.13);

\path[fill=fillColor,fill opacity=0.20] (218.54, 65.06) circle (  2.13);

\path[fill=fillColor,fill opacity=0.20] (222.55, 72.47) circle (  2.13);

\path[fill=fillColor,fill opacity=0.20] (224.56, 77.64) circle (  2.13);

\path[fill=fillColor,fill opacity=0.20] (224.56, 75.57) circle (  2.13);

\path[fill=fillColor,fill opacity=0.20] (221.55, 70.92) circle (  2.13);

\path[fill=fillColor,fill opacity=0.20] (220.54, 69.89) circle (  2.13);

\path[fill=fillColor,fill opacity=0.20] (218.54, 74.11) circle (  2.13);

\path[fill=fillColor,fill opacity=0.20] (218.54, 81.94) circle (  2.13);

\path[fill=fillColor,fill opacity=0.20] (220.54, 89.09) circle (  2.13);

\path[fill=fillColor,fill opacity=0.20] (220.54, 81.43) circle (  2.13);

\path[fill=fillColor,fill opacity=0.20] (217.53, 73.42) circle (  2.13);

\path[fill=fillColor,fill opacity=0.20] (210.51, 70.23) circle (  2.13);

\path[fill=fillColor,fill opacity=0.20] (214.53, 67.56) circle (  2.13);

\path[fill=fillColor,fill opacity=0.20] (210.51, 65.24) circle (  2.13);

\path[fill=fillColor,fill opacity=0.20] (209.51, 64.63) circle (  2.13);

\path[fill=fillColor,fill opacity=0.20] (215.53, 60.67) circle (  2.13);

\path[fill=fillColor,fill opacity=0.20] (212.52, 60.24) circle (  2.13);

\path[fill=fillColor,fill opacity=0.20] (210.51, 69.97) circle (  2.13);

\path[fill=fillColor,fill opacity=0.20] (211.52, 73.85) circle (  2.13);

\path[fill=fillColor,fill opacity=0.20] (211.52, 65.67) circle (  2.13);

\path[fill=fillColor,fill opacity=0.20] (215.53, 63.34) circle (  2.13);

\path[fill=fillColor,fill opacity=0.20] (214.53, 72.13) circle (  2.13);

\path[fill=fillColor,fill opacity=0.20] (206.50, 78.93) circle (  2.13);

\path[fill=fillColor,fill opacity=0.20] (218.54, 88.49) circle (  2.13);

\path[fill=fillColor,fill opacity=0.20] (219.54,104.16) circle (  2.13);

\path[fill=fillColor,fill opacity=0.20] (219.54,113.37) circle (  2.13);

\path[fill=fillColor,fill opacity=0.20] (217.53,102.52) circle (  2.13);

\path[fill=fillColor,fill opacity=0.20] (206.50, 92.10) circle (  2.13);

\path[fill=fillColor,fill opacity=0.20] (200.48, 78.15) circle (  2.13);

\path[fill=fillColor,fill opacity=0.20] (202.49, 70.23) circle (  2.13);

\path[fill=fillColor,fill opacity=0.20] (207.50, 68.51) circle (  2.13);

\path[fill=fillColor,fill opacity=0.20] (206.50, 67.05) circle (  2.13);

\path[fill=fillColor,fill opacity=0.20] (201.48, 66.10) circle (  2.13);

\path[fill=fillColor,fill opacity=0.20] (203.49, 65.15) circle (  2.13);

\path[fill=fillColor,fill opacity=0.20] (207.50, 64.63) circle (  2.13);

\path[fill=fillColor,fill opacity=0.20] (207.50, 70.15) circle (  2.13);

\path[fill=fillColor,fill opacity=0.20] (208.51, 73.16) circle (  2.13);

\path[fill=fillColor,fill opacity=0.20] (212.52, 70.49) circle (  2.13);

\path[fill=fillColor,fill opacity=0.20] (215.53, 79.27) circle (  2.13);

\path[fill=fillColor,fill opacity=0.20] (229.57,103.39) circle (  2.13);

\path[fill=fillColor,fill opacity=0.20] (224.56, 85.13) circle (  2.13);

\path[fill=fillColor,fill opacity=0.20] (230.58, 65.24) circle (  2.13);

\path[fill=fillColor,fill opacity=0.20] (222.55, 60.24) circle (  2.13);

\path[fill=fillColor,fill opacity=0.20] (216.53, 63.26) circle (  2.13);

\path[fill=fillColor,fill opacity=0.20] (213.52, 66.36) circle (  2.13);

\path[fill=fillColor,fill opacity=0.20] (213.52, 66.79) circle (  2.13);

\path[fill=fillColor,fill opacity=0.20] (221.55, 64.29) circle (  2.13);

\path[fill=fillColor,fill opacity=0.20] (214.53, 61.71) circle (  2.13);

\path[fill=fillColor,fill opacity=0.20] (219.54, 59.98) circle (  2.13);

\path[fill=fillColor,fill opacity=0.20] (217.53, 61.02) circle (  2.13);

\path[fill=fillColor,fill opacity=0.20] (222.55, 67.48) circle (  2.13);

\path[fill=fillColor,fill opacity=0.20] (220.54, 71.52) circle (  2.13);

\path[fill=fillColor,fill opacity=0.20] (216.53, 68.25) circle (  2.13);

\path[fill=fillColor,fill opacity=0.20] (216.53, 65.15) circle (  2.13);

\path[fill=fillColor,fill opacity=0.20] (214.53, 62.05) circle (  2.13);

\path[fill=fillColor,fill opacity=0.20] (219.54, 58.26) circle (  2.13);

\path[fill=fillColor,fill opacity=0.20] (219.54, 59.81) circle (  2.13);

\path[fill=fillColor,fill opacity=0.20] (214.53, 61.62) circle (  2.13);

\path[fill=fillColor,fill opacity=0.20] (216.53, 62.05) circle (  2.13);

\path[fill=fillColor,fill opacity=0.20] (213.52, 66.61) circle (  2.13);

\path[fill=fillColor,fill opacity=0.20] (212.52, 70.83) circle (  2.13);

\path[fill=fillColor,fill opacity=0.20] (197.47, 70.92) circle (  2.13);

\path[fill=fillColor,fill opacity=0.20] (208.51, 65.67) circle (  2.13);

\path[fill=fillColor,fill opacity=0.20] (212.52, 60.07) circle (  2.13);

\path[fill=fillColor,fill opacity=0.20] (208.51, 61.71) circle (  2.13);

\path[fill=fillColor,fill opacity=0.20] (207.50, 65.67) circle (  2.13);

\path[fill=fillColor,fill opacity=0.20] (207.50, 61.10) circle (  2.13);

\path[fill=fillColor,fill opacity=0.20] (207.50, 57.06) circle (  2.13);

\path[fill=fillColor,fill opacity=0.20] (210.51, 66.10) circle (  2.13);

\path[fill=fillColor,fill opacity=0.20] (210.51, 79.10) circle (  2.13);

\path[fill=fillColor,fill opacity=0.20] (216.53, 87.20) circle (  2.13);

\path[fill=fillColor,fill opacity=0.20] (226.56, 94.52) circle (  2.13);

\path[fill=fillColor,fill opacity=0.20] (212.52,103.73) circle (  2.13);

\path[fill=fillColor,fill opacity=0.20] (214.53, 98.13) circle (  2.13);

\path[fill=fillColor,fill opacity=0.20] (212.52, 90.64) circle (  2.13);

\path[fill=fillColor,fill opacity=0.20] (206.50, 87.45) circle (  2.13);

\path[fill=fillColor,fill opacity=0.20] (205.50, 86.08) circle (  2.13);

\path[fill=fillColor,fill opacity=0.20] (207.50, 85.39) circle (  2.13);

\path[fill=fillColor,fill opacity=0.20] (210.51, 92.28) circle (  2.13);

\path[fill=fillColor,fill opacity=0.20] (232.58,100.46) circle (  2.13);

\path[fill=fillColor,fill opacity=0.20] (220.54, 82.98) circle (  2.13);

\path[fill=fillColor,fill opacity=0.20] (217.53, 69.71) circle (  2.13);

\path[fill=fillColor,fill opacity=0.20] (214.53, 64.98) circle (  2.13);

\path[fill=fillColor,fill opacity=0.20] (209.51, 62.14) circle (  2.13);

\path[fill=fillColor,fill opacity=0.20] (212.52, 63.69) circle (  2.13);

\path[fill=fillColor,fill opacity=0.20] (215.53, 67.56) circle (  2.13);

\path[fill=fillColor,fill opacity=0.20] (219.54, 66.01) circle (  2.13);

\path[fill=fillColor,fill opacity=0.20] (217.53, 64.03) circle (  2.13);

\path[fill=fillColor,fill opacity=0.20] (214.53, 65.41) circle (  2.13);

\path[fill=fillColor,fill opacity=0.20] (210.51, 64.81) circle (  2.13);

\path[fill=fillColor,fill opacity=0.20] (207.50, 66.10) circle (  2.13);

\path[fill=fillColor,fill opacity=0.20] (214.53, 69.03) circle (  2.13);

\path[fill=fillColor,fill opacity=0.20] (217.53, 66.87) circle (  2.13);

\path[fill=fillColor,fill opacity=0.20] (214.53, 63.60) circle (  2.13);

\path[fill=fillColor,fill opacity=0.20] (212.52, 62.65) circle (  2.13);

\path[fill=fillColor,fill opacity=0.20] (209.51, 61.79) circle (  2.13);

\path[fill=fillColor,fill opacity=0.20] (211.52, 62.48) circle (  2.13);

\path[fill=fillColor,fill opacity=0.20] (204.49, 66.44) circle (  2.13);

\path[fill=fillColor,fill opacity=0.20] (201.48, 72.30) circle (  2.13);

\path[fill=fillColor,fill opacity=0.20] (204.49, 70.83) circle (  2.13);

\path[fill=fillColor,fill opacity=0.20] (205.50, 61.53) circle (  2.13);

\path[fill=fillColor,fill opacity=0.20] (204.49, 58.69) circle (  2.13);

\path[fill=fillColor,fill opacity=0.20] (204.49, 65.24) circle (  2.13);

\path[fill=fillColor,fill opacity=0.20] (206.50, 68.60) circle (  2.13);

\path[fill=fillColor,fill opacity=0.20] (215.53, 69.63) circle (  2.13);

\path[fill=fillColor,fill opacity=0.20] (217.53, 80.57) circle (  2.13);

\path[fill=fillColor,fill opacity=0.20] (220.54, 98.22) circle (  2.13);

\path[fill=fillColor,fill opacity=0.20] (232.58, 92.71) circle (  2.13);

\path[fill=fillColor,fill opacity=0.20] (207.50, 81.51) circle (  2.13);

\path[fill=fillColor,fill opacity=0.20] (216.53, 71.87) circle (  2.13);

\path[fill=fillColor,fill opacity=0.20] (218.54, 73.33) circle (  2.13);

\path[fill=fillColor,fill opacity=0.20] (219.54, 71.61) circle (  2.13);

\path[fill=fillColor,fill opacity=0.20] (215.53, 62.40) circle (  2.13);

\path[fill=fillColor,fill opacity=0.20] (213.52, 60.67) circle (  2.13);

\path[fill=fillColor,fill opacity=0.20] (210.51, 66.79) circle (  2.13);

\path[fill=fillColor,fill opacity=0.20] (211.52, 68.42) circle (  2.13);

\path[fill=fillColor,fill opacity=0.20] (207.50, 66.96) circle (  2.13);

\path[fill=fillColor,fill opacity=0.20] (209.51, 67.48) circle (  2.13);

\path[fill=fillColor,fill opacity=0.20] (209.51, 66.87) circle (  2.13);

\path[fill=fillColor,fill opacity=0.20] (204.49, 63.08) circle (  2.13);

\path[fill=fillColor,fill opacity=0.20] (205.50, 60.67) circle (  2.13);

\path[fill=fillColor,fill opacity=0.20] (206.50, 60.41) circle (  2.13);

\path[fill=fillColor,fill opacity=0.20] (199.48, 58.52) circle (  2.13);

\path[fill=fillColor,fill opacity=0.20] (198.47, 60.76) circle (  2.13);

\path[fill=fillColor,fill opacity=0.20] (208.51, 65.41) circle (  2.13);

\path[fill=fillColor,fill opacity=0.20] (209.51, 63.60) circle (  2.13);

\path[fill=fillColor,fill opacity=0.20] (208.51, 60.67) circle (  2.13);

\path[fill=fillColor,fill opacity=0.20] (208.51, 66.96) circle (  2.13);

\path[fill=fillColor,fill opacity=0.20] (207.50, 79.10) circle (  2.13);

\path[fill=fillColor,fill opacity=0.20] (222.55, 83.67) circle (  2.13);

\path[fill=fillColor,fill opacity=0.20] (220.54, 79.02) circle (  2.13);

\path[fill=fillColor,fill opacity=0.20] (212.52, 75.23) circle (  2.13);

\path[fill=fillColor,fill opacity=0.20] (196.47, 78.24) circle (  2.13);

\path[fill=fillColor,fill opacity=0.20] (218.54, 73.93) circle (  2.13);

\path[fill=fillColor,fill opacity=0.20] (211.52, 60.16) circle (  2.13);

\path[fill=fillColor,fill opacity=0.20] (205.50, 55.85) circle (  2.13);

\path[fill=fillColor,fill opacity=0.20] (207.50, 62.14) circle (  2.13);

\path[fill=fillColor,fill opacity=0.20] (205.50, 61.45) circle (  2.13);

\path[fill=fillColor,fill opacity=0.20] (205.50, 59.81) circle (  2.13);

\path[fill=fillColor,fill opacity=0.20] (207.50, 65.15) circle (  2.13);

\path[fill=fillColor,fill opacity=0.20] (197.47, 66.96) circle (  2.13);

\path[fill=fillColor,fill opacity=0.20] (211.52, 63.34) circle (  2.13);

\path[fill=fillColor,fill opacity=0.20] (215.53, 65.06) circle (  2.13);

\path[fill=fillColor,fill opacity=0.20] (208.51, 71.61) circle (  2.13);

\path[fill=fillColor,fill opacity=0.20] (220.54, 77.21) circle (  2.13);

\path[fill=fillColor,fill opacity=0.20] (230.58, 82.46) circle (  2.13);

\path[fill=fillColor,fill opacity=0.20] (211.52, 75.40) circle (  2.13);

\path[fill=fillColor,fill opacity=0.20] (210.51, 65.67) circle (  2.13);

\path[fill=fillColor,fill opacity=0.20] (215.53, 67.13) circle (  2.13);

\path[fill=fillColor,fill opacity=0.20] (213.52, 72.90) circle (  2.13);

\path[fill=fillColor,fill opacity=0.20] (211.52, 77.21) circle (  2.13);

\path[fill=fillColor,fill opacity=0.20] (218.54, 82.72) circle (  2.13);

\path[fill=fillColor,fill opacity=0.20] (220.54, 88.75) circle (  2.13);

\path[fill=fillColor,fill opacity=0.20] (224.56, 88.23) circle (  2.13);

\path[fill=fillColor,fill opacity=0.20] (237.60, 83.84) circle (  2.13);

\path[fill=fillColor,fill opacity=0.20] (187.54, 97.44) circle (  2.13);

\path[fill=fillColor,fill opacity=0.20] (186.33,105.28) circle (  2.13);

\path[fill=fillColor,fill opacity=0.20] (204.49,103.47) circle (  2.13);

\path[fill=fillColor,fill opacity=0.20] (203.49, 59.90) circle (  2.13);

\path[fill=fillColor,fill opacity=0.20] (201.48, 56.11) circle (  2.13);

\path[fill=fillColor,fill opacity=0.20] (200.48, 62.91) circle (  2.13);

\path[fill=fillColor,fill opacity=0.20] (201.48, 72.21) circle (  2.13);

\path[fill=fillColor,fill opacity=0.20] (203.49, 71.52) circle (  2.13);

\path[fill=fillColor,fill opacity=0.20] (242.62, 99.60) circle (  2.13);

\path[fill=fillColor,fill opacity=0.20] (245.63, 80.48) circle (  2.13);

\path[fill=fillColor,fill opacity=0.20] (241.61, 71.52) circle (  2.13);

\path[fill=fillColor,fill opacity=0.20] (237.60, 81.17) circle (  2.13);

\path[fill=fillColor,fill opacity=0.20] (238.60, 78.24) circle (  2.13);

\path[fill=fillColor,fill opacity=0.20] (241.61, 70.49) circle (  2.13);

\path[fill=fillColor,fill opacity=0.20] (208.51, 51.89) circle (  2.13);

\path[fill=fillColor,fill opacity=0.20] (203.49, 48.01) circle (  2.13);

\path[fill=fillColor,fill opacity=0.20] (200.48, 57.92) circle (  2.13);

\path[fill=fillColor,fill opacity=0.20] (196.47, 59.81) circle (  2.13);

\path[fill=fillColor,fill opacity=0.20] (196.47, 60.85) circle (  2.13);

\path[fill=fillColor,fill opacity=0.20] (199.48, 66.61) circle (  2.13);

\path[fill=fillColor,fill opacity=0.20] (202.49, 75.31) circle (  2.13);

\path[fill=fillColor,fill opacity=0.20] (205.50, 85.90) circle (  2.13);

\path[fill=fillColor,fill opacity=0.20] (240.61,103.21) circle (  2.13);

\path[fill=fillColor,fill opacity=0.20] (235.59, 88.83) circle (  2.13);

\path[fill=fillColor,fill opacity=0.20] (239.61, 87.63) circle (  2.13);

\path[fill=fillColor,fill opacity=0.20] (230.58, 73.25) circle (  2.13);

\path[fill=fillColor,fill opacity=0.20] (220.54, 61.96) circle (  2.13);

\path[fill=fillColor,fill opacity=0.20] (217.53, 67.39) circle (  2.13);

\path[fill=fillColor,fill opacity=0.20] (221.55, 66.53) circle (  2.13);

\path[fill=fillColor,fill opacity=0.20] (224.56, 64.03) circle (  2.13);

\path[fill=fillColor,fill opacity=0.20] (228.57, 66.36) circle (  2.13);

\path[fill=fillColor,fill opacity=0.20] (232.58, 73.50) circle (  2.13);

\path[fill=fillColor,fill opacity=0.20] (208.51, 72.21) circle (  2.13);

\path[fill=fillColor,fill opacity=0.20] (201.48, 45.60) circle (  2.13);

\path[fill=fillColor,fill opacity=0.20] (198.47, 62.91) circle (  2.13);

\path[fill=fillColor,fill opacity=0.20] (194.46, 58.00) circle (  2.13);

\path[fill=fillColor,fill opacity=0.20] (191.45, 52.58) circle (  2.13);

\path[fill=fillColor,fill opacity=0.20] (193.46, 48.88) circle (  2.13);

\path[fill=fillColor,fill opacity=0.20] (195.46, 53.53) circle (  2.13);

\path[fill=fillColor,fill opacity=0.20] (200.48, 59.90) circle (  2.13);

\path[fill=fillColor,fill opacity=0.20] (206.50, 61.88) circle (  2.13);

\path[fill=fillColor,fill opacity=0.20] (210.51, 78.15) circle (  2.13);

\path[fill=fillColor,fill opacity=0.20] (202.49, 93.57) circle (  2.13);

\path[fill=fillColor,fill opacity=0.20] (228.57, 81.86) circle (  2.13);

\path[fill=fillColor,fill opacity=0.20] (218.54, 74.80) circle (  2.13);

\path[fill=fillColor,fill opacity=0.20] (223.55, 60.33) circle (  2.13);

\path[fill=fillColor,fill opacity=0.20] (218.54, 51.37) circle (  2.13);

\path[fill=fillColor,fill opacity=0.20] (214.53, 61.19) circle (  2.13);

\path[fill=fillColor,fill opacity=0.20] (212.52, 70.66) circle (  2.13);

\path[fill=fillColor,fill opacity=0.20] (213.52, 61.71) circle (  2.13);

\path[fill=fillColor,fill opacity=0.20] (218.54, 58.78) circle (  2.13);

\path[fill=fillColor,fill opacity=0.20] (214.53, 70.15) circle (  2.13);

\path[fill=fillColor,fill opacity=0.20] (205.50, 50.60) circle (  2.13);

\path[fill=fillColor,fill opacity=0.20] (197.47, 57.49) circle (  2.13);

\path[fill=fillColor,fill opacity=0.20] (195.46, 71.27) circle (  2.13);

\path[fill=fillColor,fill opacity=0.20] (191.45, 59.21) circle (  2.13);

\path[fill=fillColor,fill opacity=0.20] (192.45, 45.69) circle (  2.13);

\path[fill=fillColor,fill opacity=0.20] (193.46, 38.28) circle (  2.13);

\path[fill=fillColor,fill opacity=0.20] (194.46, 39.23) circle (  2.13);

\path[fill=fillColor,fill opacity=0.20] (199.48, 38.54) circle (  2.13);

\path[fill=fillColor,fill opacity=0.20] (203.49, 38.37) circle (  2.13);

\path[fill=fillColor,fill opacity=0.20] (207.50, 56.20) circle (  2.13);

\path[fill=fillColor,fill opacity=0.20] (241.61, 93.22) circle (  2.13);

\path[fill=fillColor,fill opacity=0.20] (233.59, 61.10) circle (  2.13);

\path[fill=fillColor,fill opacity=0.20] (227.57, 71.70) circle (  2.13);

\path[fill=fillColor,fill opacity=0.20] (223.55, 78.07) circle (  2.13);

\path[fill=fillColor,fill opacity=0.20] (220.54, 61.19) circle (  2.13);

\path[fill=fillColor,fill opacity=0.20] (214.53, 54.99) circle (  2.13);

\path[fill=fillColor,fill opacity=0.20] (210.51, 58.78) circle (  2.13);

\path[fill=fillColor,fill opacity=0.20] (211.52, 63.51) circle (  2.13);

\path[fill=fillColor,fill opacity=0.20] (213.52, 63.43) circle (  2.13);

\path[fill=fillColor,fill opacity=0.20] (218.54, 56.88) circle (  2.13);

\path[fill=fillColor,fill opacity=0.20] (228.57, 59.81) circle (  2.13);

\path[fill=fillColor,fill opacity=0.20] (234.59, 72.99) circle (  2.13);

\path[fill=fillColor,fill opacity=0.20] (204.49, 65.15) circle (  2.13);

\path[fill=fillColor,fill opacity=0.20] (197.47, 51.80) circle (  2.13);

\path[fill=fillColor,fill opacity=0.20] (195.46, 64.89) circle (  2.13);

\path[fill=fillColor,fill opacity=0.20] (192.45, 63.51) circle (  2.13);

\path[fill=fillColor,fill opacity=0.20] (192.45, 44.74) circle (  2.13);

\path[fill=fillColor,fill opacity=0.20] (193.46, 40.01) circle (  2.13);

\path[fill=fillColor,fill opacity=0.20] (202.49, 42.16) circle (  2.13);

\path[fill=fillColor,fill opacity=0.20] (207.50, 66.87) circle (  2.13);

\path[fill=fillColor,fill opacity=0.20] (256.66, 89.44) circle (  2.13);

\path[fill=fillColor,fill opacity=0.20] (223.55, 69.54) circle (  2.13);

\path[fill=fillColor,fill opacity=0.20] (215.53, 55.16) circle (  2.13);

\path[fill=fillColor,fill opacity=0.20] (219.54, 61.62) circle (  2.13);

\path[fill=fillColor,fill opacity=0.20] (217.53, 65.50) circle (  2.13);

\path[fill=fillColor,fill opacity=0.20] (213.52, 59.98) circle (  2.13);

\path[fill=fillColor,fill opacity=0.20] (209.51, 58.69) circle (  2.13);

\path[fill=fillColor,fill opacity=0.20] (208.51, 54.04) circle (  2.13);

\path[fill=fillColor,fill opacity=0.20] (208.51, 54.82) circle (  2.13);

\path[fill=fillColor,fill opacity=0.20] (206.50, 67.48) circle (  2.13);

\path[fill=fillColor,fill opacity=0.20] (214.53, 64.12) circle (  2.13);

\path[fill=fillColor,fill opacity=0.20] (233.59, 65.24) circle (  2.13);

\path[fill=fillColor,fill opacity=0.20] (205.50, 58.09) circle (  2.13);

\path[fill=fillColor,fill opacity=0.20] (199.48, 43.36) circle (  2.13);

\path[fill=fillColor,fill opacity=0.20] (197.47, 53.87) circle (  2.13);

\path[fill=fillColor,fill opacity=0.20] (190.45, 54.64) circle (  2.13);

\path[fill=fillColor,fill opacity=0.20] (190.45, 43.19) circle (  2.13);

\path[fill=fillColor,fill opacity=0.20] (193.46, 50.17) circle (  2.13);

\path[fill=fillColor,fill opacity=0.20] (194.46, 56.80) circle (  2.13);

\path[fill=fillColor,fill opacity=0.20] (196.47, 47.41) circle (  2.13);

\path[fill=fillColor,fill opacity=0.20] (201.48, 43.11) circle (  2.13);

\path[fill=fillColor,fill opacity=0.20] (206.50, 56.63) circle (  2.13);

\path[fill=fillColor,fill opacity=0.20] (210.51, 88.57) circle (  2.13);

\path[fill=fillColor,fill opacity=0.20] (240.61, 72.21) circle (  2.13);

\path[fill=fillColor,fill opacity=0.20] (220.54, 52.15) circle (  2.13);

\path[fill=fillColor,fill opacity=0.20] (215.53, 53.35) circle (  2.13);

\path[fill=fillColor,fill opacity=0.20] (208.51, 40.95) circle (  2.13);

\path[fill=fillColor,fill opacity=0.20] (206.50, 41.38) circle (  2.13);

\path[fill=fillColor,fill opacity=0.20] (202.49, 53.87) circle (  2.13);

\path[fill=fillColor,fill opacity=0.20] (198.47, 56.88) circle (  2.13);

\path[fill=fillColor,fill opacity=0.20] (202.49, 52.32) circle (  2.13);

\path[fill=fillColor,fill opacity=0.20] (203.49, 54.82) circle (  2.13);

\path[fill=fillColor,fill opacity=0.20] (202.49, 64.63) circle (  2.13);

\path[fill=fillColor,fill opacity=0.20] (210.51, 74.80) circle (  2.13);

\path[fill=fillColor,fill opacity=0.20] (205.50, 40.09) circle (  2.13);

\path[fill=fillColor,fill opacity=0.20] (200.48, 46.12) circle (  2.13);

\path[fill=fillColor,fill opacity=0.20] (199.48, 50.43) circle (  2.13);

\path[fill=fillColor,fill opacity=0.20] (191.45, 48.19) circle (  2.13);

\path[fill=fillColor,fill opacity=0.20] (191.45, 45.17) circle (  2.13);

\path[fill=fillColor,fill opacity=0.20] (195.46, 40.61) circle (  2.13);

\path[fill=fillColor,fill opacity=0.20] (195.46, 53.18) circle (  2.13);

\path[fill=fillColor,fill opacity=0.20] (195.46, 64.12) circle (  2.13);

\path[fill=fillColor,fill opacity=0.20] (197.47, 51.46) circle (  2.13);

\path[fill=fillColor,fill opacity=0.20] (202.49, 42.33) circle (  2.13);

\path[fill=fillColor,fill opacity=0.20] (208.51, 53.53) circle (  2.13);

\path[fill=fillColor,fill opacity=0.20] (211.52, 84.27) circle (  2.13);

\path[fill=fillColor,fill opacity=0.20] (248.64, 91.76) circle (  2.13);

\path[fill=fillColor,fill opacity=0.20] (222.55, 76.78) circle (  2.13);

\path[fill=fillColor,fill opacity=0.20] (214.53, 50.86) circle (  2.13);

\path[fill=fillColor,fill opacity=0.20] (211.52, 43.71) circle (  2.13);

\path[fill=fillColor,fill opacity=0.20] (194.46, 56.63) circle (  2.13);

\path[fill=fillColor,fill opacity=0.20] (196.47, 54.73) circle (  2.13);

\path[fill=fillColor,fill opacity=0.20] (198.47, 46.21) circle (  2.13);

\path[fill=fillColor,fill opacity=0.20] (199.48, 56.71) circle (  2.13);

\path[fill=fillColor,fill opacity=0.20] (202.49, 68.25) circle (  2.13);

\path[fill=fillColor,fill opacity=0.20] (214.53, 90.90) circle (  2.13);

\path[fill=fillColor,fill opacity=0.20] (214.53, 50.08) circle (  2.13);

\path[fill=fillColor,fill opacity=0.20] (204.49, 48.10) circle (  2.13);

\path[fill=fillColor,fill opacity=0.20] (200.48, 56.28) circle (  2.13);

\path[fill=fillColor,fill opacity=0.20] (199.48, 55.42) circle (  2.13);

\path[fill=fillColor,fill opacity=0.20] (197.47, 56.54) circle (  2.13);

\path[fill=fillColor,fill opacity=0.20] (197.47, 47.41) circle (  2.13);

\path[fill=fillColor,fill opacity=0.20] (195.46, 41.13) circle (  2.13);

\path[fill=fillColor,fill opacity=0.20] (193.46, 48.79) circle (  2.13);

\path[fill=fillColor,fill opacity=0.20] (197.47, 52.66) circle (  2.13);

\path[fill=fillColor,fill opacity=0.20] (200.48, 44.83) circle (  2.13);

\path[fill=fillColor,fill opacity=0.20] (203.49, 45.95) circle (  2.13);

\path[fill=fillColor,fill opacity=0.20] (210.51, 57.14) circle (  2.13);

\path[fill=fillColor,fill opacity=0.20] (246.63, 88.06) circle (  2.13);

\path[fill=fillColor,fill opacity=0.20] (218.54, 77.90) circle (  2.13);

\path[fill=fillColor,fill opacity=0.20] (210.51, 58.78) circle (  2.13);

\path[fill=fillColor,fill opacity=0.20] (210.51, 48.10) circle (  2.13);

\path[fill=fillColor,fill opacity=0.20] (205.50, 45.52) circle (  2.13);

\path[fill=fillColor,fill opacity=0.20] (203.49, 53.87) circle (  2.13);

\path[fill=fillColor,fill opacity=0.20] (200.48, 64.89) circle (  2.13);

\path[fill=fillColor,fill opacity=0.20] (199.48, 63.77) circle (  2.13);

\path[fill=fillColor,fill opacity=0.20] (200.48, 50.60) circle (  2.13);

\path[fill=fillColor,fill opacity=0.20] (205.50, 56.54) circle (  2.13);

\path[fill=fillColor,fill opacity=0.20] (212.52, 76.52) circle (  2.13);

\path[fill=fillColor,fill opacity=0.20] (222.55, 72.56) circle (  2.13);

\path[fill=fillColor,fill opacity=0.20] (210.51, 59.12) circle (  2.13);

\path[fill=fillColor,fill opacity=0.20] (202.49, 60.76) circle (  2.13);

\path[fill=fillColor,fill opacity=0.20] (199.48, 56.63) circle (  2.13);

\path[fill=fillColor,fill opacity=0.20] (199.48, 58.43) circle (  2.13);

\path[fill=fillColor,fill opacity=0.20] (200.48, 57.23) circle (  2.13);

\path[fill=fillColor,fill opacity=0.20] (199.48, 48.70) circle (  2.13);

\path[fill=fillColor,fill opacity=0.20] (196.47, 42.33) circle (  2.13);

\path[fill=fillColor,fill opacity=0.20] (195.46, 44.57) circle (  2.13);

\path[fill=fillColor,fill opacity=0.20] (200.48, 46.64) circle (  2.13);

\path[fill=fillColor,fill opacity=0.20] (201.48, 54.21) circle (  2.13);

\path[fill=fillColor,fill opacity=0.20] (207.50, 59.30) circle (  2.13);

\path[fill=fillColor,fill opacity=0.20] (214.53, 66.18) circle (  2.13);

\path[fill=fillColor,fill opacity=0.20] (240.61, 83.41) circle (  2.13);

\path[fill=fillColor,fill opacity=0.20] (218.54, 72.21) circle (  2.13);

\path[fill=fillColor,fill opacity=0.20] (212.52, 64.98) circle (  2.13);

\path[fill=fillColor,fill opacity=0.20] (206.50, 55.33) circle (  2.13);

\path[fill=fillColor,fill opacity=0.20] (203.49, 49.74) circle (  2.13);

\path[fill=fillColor,fill opacity=0.20] (201.48, 56.71) circle (  2.13);

\path[fill=fillColor,fill opacity=0.20] (203.49, 63.00) circle (  2.13);

\path[fill=fillColor,fill opacity=0.20] (204.49, 65.24) circle (  2.13);

\path[fill=fillColor,fill opacity=0.20] (206.50, 63.69) circle (  2.13);

\path[fill=fillColor,fill opacity=0.20] (212.52, 61.45) circle (  2.13);

\path[fill=fillColor,fill opacity=0.20] (214.53, 71.35) circle (  2.13);

\path[fill=fillColor,fill opacity=0.20] (208.51, 69.28) circle (  2.13);

\path[fill=fillColor,fill opacity=0.20] (202.49, 51.98) circle (  2.13);

\path[fill=fillColor,fill opacity=0.20] (199.48, 46.55) circle (  2.13);

\path[fill=fillColor,fill opacity=0.20] (198.47, 53.53) circle (  2.13);

\path[fill=fillColor,fill opacity=0.20] (194.46, 50.08) circle (  2.13);

\path[fill=fillColor,fill opacity=0.20] (199.48, 46.03) circle (  2.13);

\path[fill=fillColor,fill opacity=0.20] (195.46, 48.36) circle (  2.13);

\path[fill=fillColor,fill opacity=0.20] (196.47, 48.62) circle (  2.13);

\path[fill=fillColor,fill opacity=0.20] (200.48, 56.02) circle (  2.13);

\path[fill=fillColor,fill opacity=0.20] (204.49, 69.46) circle (  2.13);

\path[fill=fillColor,fill opacity=0.20] (209.51, 68.68) circle (  2.13);

\path[fill=fillColor,fill opacity=0.20] (216.53, 80.48) circle (  2.13);

\path[fill=fillColor,fill opacity=0.20] (215.53, 71.35) circle (  2.13);

\path[fill=fillColor,fill opacity=0.20] (206.50, 62.31) circle (  2.13);

\path[fill=fillColor,fill opacity=0.20] (201.48, 60.33) circle (  2.13);

\path[fill=fillColor,fill opacity=0.20] (199.48, 57.40) circle (  2.13);

\path[fill=fillColor,fill opacity=0.20] (200.48, 59.38) circle (  2.13);

\path[fill=fillColor,fill opacity=0.20] (203.49, 58.09) circle (  2.13);

\path[fill=fillColor,fill opacity=0.20] (205.50, 56.63) circle (  2.13);

\path[fill=fillColor,fill opacity=0.20] (207.50, 67.22) circle (  2.13);

\path[fill=fillColor,fill opacity=0.20] (215.53, 71.52) circle (  2.13);

\path[fill=fillColor,fill opacity=0.20] (198.47, 72.47) circle (  2.13);

\path[fill=fillColor,fill opacity=0.20] (212.52, 69.80) circle (  2.13);

\path[fill=fillColor,fill opacity=0.20] (204.49, 62.74) circle (  2.13);

\path[fill=fillColor,fill opacity=0.20] (200.48, 45.09) circle (  2.13);

\path[fill=fillColor,fill opacity=0.20] (194.46, 48.53) circle (  2.13);

\path[fill=fillColor,fill opacity=0.20] (194.46, 45.69) circle (  2.13);

\path[fill=fillColor,fill opacity=0.20] (195.46, 43.28) circle (  2.13);

\path[fill=fillColor,fill opacity=0.20] (195.46, 54.99) circle (  2.13);

\path[fill=fillColor,fill opacity=0.20] (198.47, 59.90) circle (  2.13);

\path[fill=fillColor,fill opacity=0.20] (200.48, 60.50) circle (  2.13);

\path[fill=fillColor,fill opacity=0.20] (205.50, 64.12) circle (  2.13);

\path[fill=fillColor,fill opacity=0.20] (213.52, 69.71) circle (  2.13);

\path[fill=fillColor,fill opacity=0.20] (217.53, 97.19) circle (  2.13);

\path[fill=fillColor,fill opacity=0.20] (223.55, 76.60) circle (  2.13);

\path[fill=fillColor,fill opacity=0.20] (208.51, 61.10) circle (  2.13);

\path[fill=fillColor,fill opacity=0.20] (201.48, 64.03) circle (  2.13);

\path[fill=fillColor,fill opacity=0.20] (200.48, 62.22) circle (  2.13);

\path[fill=fillColor,fill opacity=0.20] (204.49, 56.80) circle (  2.13);

\path[fill=fillColor,fill opacity=0.20] (208.51, 56.28) circle (  2.13);

\path[fill=fillColor,fill opacity=0.20] (206.50, 57.57) circle (  2.13);

\path[fill=fillColor,fill opacity=0.20] (208.51, 64.20) circle (  2.13);

\path[fill=fillColor,fill opacity=0.20] (215.53, 75.40) circle (  2.13);

\path[fill=fillColor,fill opacity=0.20] (217.53, 66.36) circle (  2.13);

\path[fill=fillColor,fill opacity=0.20] (210.51, 61.53) circle (  2.13);

\path[fill=fillColor,fill opacity=0.20] (201.48, 62.31) circle (  2.13);

\path[fill=fillColor,fill opacity=0.20] (200.48, 63.51) circle (  2.13);

\path[fill=fillColor,fill opacity=0.20] (199.48, 54.21) circle (  2.13);

\path[fill=fillColor,fill opacity=0.20] (193.46, 49.13) circle (  2.13);

\path[fill=fillColor,fill opacity=0.20] (191.45, 48.70) circle (  2.13);

\path[fill=fillColor,fill opacity=0.20] (192.45, 50.77) circle (  2.13);

\path[fill=fillColor,fill opacity=0.20] (194.46, 58.86) circle (  2.13);

\path[fill=fillColor,fill opacity=0.20] (196.47, 65.67) circle (  2.13);

\path[fill=fillColor,fill opacity=0.20] (197.47, 58.86) circle (  2.13);

\path[fill=fillColor,fill opacity=0.20] (209.51, 55.08) circle (  2.13);

\path[fill=fillColor,fill opacity=0.20] (218.54, 72.13) circle (  2.13);

\path[fill=fillColor,fill opacity=0.20] (234.59, 76.52) circle (  2.13);

\path[fill=fillColor,fill opacity=0.20] (212.52, 65.15) circle (  2.13);

\path[fill=fillColor,fill opacity=0.20] (205.50, 66.27) circle (  2.13);

\path[fill=fillColor,fill opacity=0.20] (205.50, 59.55) circle (  2.13);

\path[fill=fillColor,fill opacity=0.20] (206.50, 48.88) circle (  2.13);

\path[fill=fillColor,fill opacity=0.20] (203.49, 57.14) circle (  2.13);

\path[fill=fillColor,fill opacity=0.20] (209.51, 64.03) circle (  2.13);

\path[fill=fillColor,fill opacity=0.20] (212.52, 59.47) circle (  2.13);

\path[fill=fillColor,fill opacity=0.20] (213.52, 60.93) circle (  2.13);

\path[fill=fillColor,fill opacity=0.20] (219.54, 77.55) circle (  2.13);

\path[fill=fillColor,fill opacity=0.20] (221.55, 68.77) circle (  2.13);

\path[fill=fillColor,fill opacity=0.20] (213.52, 53.70) circle (  2.13);

\path[fill=fillColor,fill opacity=0.20] (203.49, 55.08) circle (  2.13);

\path[fill=fillColor,fill opacity=0.20] (202.49, 64.03) circle (  2.13);

\path[fill=fillColor,fill opacity=0.20] (201.48, 70.66) circle (  2.13);

\path[fill=fillColor,fill opacity=0.20] (200.48, 64.72) circle (  2.13);

\path[fill=fillColor,fill opacity=0.20] (194.46, 53.53) circle (  2.13);

\path[fill=fillColor,fill opacity=0.20] (190.45, 54.56) circle (  2.13);

\path[fill=fillColor,fill opacity=0.20] (189.44, 63.08) circle (  2.13);

\path[fill=fillColor,fill opacity=0.20] (189.44, 64.12) circle (  2.13);

\path[fill=fillColor,fill opacity=0.20] (191.45, 60.93) circle (  2.13);

\path[fill=fillColor,fill opacity=0.20] (195.46, 60.50) circle (  2.13);

\path[fill=fillColor,fill opacity=0.20] (209.51, 64.38) circle (  2.13);

\path[fill=fillColor,fill opacity=0.20] (223.55, 76.78) circle (  2.13);

\path[fill=fillColor,fill opacity=0.20] (212.52, 67.65) circle (  2.13);

\path[fill=fillColor,fill opacity=0.20] (208.51, 55.94) circle (  2.13);

\path[fill=fillColor,fill opacity=0.20] (206.50, 50.86) circle (  2.13);

\path[fill=fillColor,fill opacity=0.20] (207.50, 59.47) circle (  2.13);

\path[fill=fillColor,fill opacity=0.20] (214.53, 60.93) circle (  2.13);

\path[fill=fillColor,fill opacity=0.20] (214.53, 51.54) circle (  2.13);

\path[fill=fillColor,fill opacity=0.20] (211.52, 49.82) circle (  2.13);

\path[fill=fillColor,fill opacity=0.20] (215.53, 57.49) circle (  2.13);

\path[fill=fillColor,fill opacity=0.20] (228.57, 78.07) circle (  2.13);

\path[fill=fillColor,fill opacity=0.20] (217.53, 76.78) circle (  2.13);

\path[fill=fillColor,fill opacity=0.20] (210.51, 60.67) circle (  2.13);

\path[fill=fillColor,fill opacity=0.20] (208.51, 54.47) circle (  2.13);

\path[fill=fillColor,fill opacity=0.20] (207.50, 59.21) circle (  2.13);

\path[fill=fillColor,fill opacity=0.20] (203.49, 57.83) circle (  2.13);

\path[fill=fillColor,fill opacity=0.20] (199.48, 54.64) circle (  2.13);

\path[fill=fillColor,fill opacity=0.20] (198.47, 59.04) circle (  2.13);

\path[fill=fillColor,fill opacity=0.20] (194.46, 56.80) circle (  2.13);

\path[fill=fillColor,fill opacity=0.20] (191.45, 55.68) circle (  2.13);

\path[fill=fillColor,fill opacity=0.20] (193.46, 62.57) circle (  2.13);

\path[fill=fillColor,fill opacity=0.20] (190.45, 61.71) circle (  2.13);

\path[fill=fillColor,fill opacity=0.20] (194.46, 56.71) circle (  2.13);

\path[fill=fillColor,fill opacity=0.20] (207.50, 65.24) circle (  2.13);

\path[fill=fillColor,fill opacity=0.20] (219.54, 74.71) circle (  2.13);

\path[fill=fillColor,fill opacity=0.20] (209.51, 65.06) circle (  2.13);

\path[fill=fillColor,fill opacity=0.20] (208.51, 60.59) circle (  2.13);

\path[fill=fillColor,fill opacity=0.20] (209.51, 56.37) circle (  2.13);

\path[fill=fillColor,fill opacity=0.20] (210.51, 52.92) circle (  2.13);

\path[fill=fillColor,fill opacity=0.20] (212.52, 52.58) circle (  2.13);

\path[fill=fillColor,fill opacity=0.20] (209.51, 54.56) circle (  2.13);

\path[fill=fillColor,fill opacity=0.20] (211.52, 56.97) circle (  2.13);

\path[fill=fillColor,fill opacity=0.20] (219.54, 58.52) circle (  2.13);

\path[fill=fillColor,fill opacity=0.20] (230.58, 72.99) circle (  2.13);

\path[fill=fillColor,fill opacity=0.20] (209.51, 88.83) circle (  2.13);

\path[fill=fillColor,fill opacity=0.20] (212.52, 61.45) circle (  2.13);

\path[fill=fillColor,fill opacity=0.20] (208.51, 60.16) circle (  2.13);

\path[fill=fillColor,fill opacity=0.20] (208.51, 69.37) circle (  2.13);

\path[fill=fillColor,fill opacity=0.20] (203.49, 63.95) circle (  2.13);

\path[fill=fillColor,fill opacity=0.20] (200.48, 50.94) circle (  2.13);

\path[fill=fillColor,fill opacity=0.20] (195.46, 43.11) circle (  2.13);

\path[fill=fillColor,fill opacity=0.20] (194.46, 51.80) circle (  2.13);

\path[fill=fillColor,fill opacity=0.20] (189.44, 57.83) circle (  2.13);

\path[fill=fillColor,fill opacity=0.20] (192.45, 53.09) circle (  2.13);

\path[fill=fillColor,fill opacity=0.20] (199.48, 49.31) circle (  2.13);

\path[fill=fillColor,fill opacity=0.20] (197.47, 53.09) circle (  2.13);

\path[fill=fillColor,fill opacity=0.20] (204.49, 64.29) circle (  2.13);

\path[fill=fillColor,fill opacity=0.20] (215.53, 70.58) circle (  2.13);

\path[fill=fillColor,fill opacity=0.20] (213.52, 75.57) circle (  2.13);

\path[fill=fillColor,fill opacity=0.20] (209.51, 63.95) circle (  2.13);

\path[fill=fillColor,fill opacity=0.20] (210.51, 48.27) circle (  2.13);

\path[fill=fillColor,fill opacity=0.20] (210.51, 45.52) circle (  2.13);

\path[fill=fillColor,fill opacity=0.20] (210.51, 54.99) circle (  2.13);

\path[fill=fillColor,fill opacity=0.20] (208.51, 61.02) circle (  2.13);

\path[fill=fillColor,fill opacity=0.20] (210.51, 59.73) circle (  2.13);

\path[fill=fillColor,fill opacity=0.20] (218.54, 55.42) circle (  2.13);

\path[fill=fillColor,fill opacity=0.20] (223.55, 54.56) circle (  2.13);

\path[fill=fillColor,fill opacity=0.20] (233.59, 65.32) circle (  2.13);

\path[fill=fillColor,fill opacity=0.20] (220.54, 89.26) circle (  2.13);

\path[fill=fillColor,fill opacity=0.20] (209.51, 79.62) circle (  2.13);

\path[fill=fillColor,fill opacity=0.20] (206.50, 67.39) circle (  2.13);

\path[fill=fillColor,fill opacity=0.20] (210.51, 63.60) circle (  2.13);

\path[fill=fillColor,fill opacity=0.20] (210.51, 70.06) circle (  2.13);

\path[fill=fillColor,fill opacity=0.20] (200.48, 64.55) circle (  2.13);

\path[fill=fillColor,fill opacity=0.20] (197.47, 52.66) circle (  2.13);

\path[fill=fillColor,fill opacity=0.20] (195.46, 52.92) circle (  2.13);

\path[fill=fillColor,fill opacity=0.20] (193.46, 60.76) circle (  2.13);

\path[fill=fillColor,fill opacity=0.20] (189.44, 60.67) circle (  2.13);

\path[fill=fillColor,fill opacity=0.20] (166.37, 57.92) circle (  2.13);

\path[fill=fillColor,fill opacity=0.20] (202.49, 50.51) circle (  2.13);

\path[fill=fillColor,fill opacity=0.20] (206.50, 51.72) circle (  2.13);

\path[fill=fillColor,fill opacity=0.20] (212.52, 69.20) circle (  2.13);

\path[fill=fillColor,fill opacity=0.20] (223.55, 80.57) circle (  2.13);

\path[fill=fillColor,fill opacity=0.20] (215.53, 67.48) circle (  2.13);

\path[fill=fillColor,fill opacity=0.20] (209.51, 50.94) circle (  2.13);

\path[fill=fillColor,fill opacity=0.20] (210.51, 47.41) circle (  2.13);

\path[fill=fillColor,fill opacity=0.20] (211.52, 58.18) circle (  2.13);

\path[fill=fillColor,fill opacity=0.20] (210.51, 61.71) circle (  2.13);

\path[fill=fillColor,fill opacity=0.20] (215.53, 57.31) circle (  2.13);

\path[fill=fillColor,fill opacity=0.20] (211.52, 55.68) circle (  2.13);

\path[fill=fillColor,fill opacity=0.20] (218.54, 61.19) circle (  2.13);

\path[fill=fillColor,fill opacity=0.20] (225.56, 63.60) circle (  2.13);

\path[fill=fillColor,fill opacity=0.20] (234.59, 64.20) circle (  2.13);

\path[fill=fillColor,fill opacity=0.20] (222.55, 81.43) circle (  2.13);

\path[fill=fillColor,fill opacity=0.20] (214.53, 60.85) circle (  2.13);

\path[fill=fillColor,fill opacity=0.20] (210.51, 71.95) circle (  2.13);

\path[fill=fillColor,fill opacity=0.20] (204.49, 73.93) circle (  2.13);

\path[fill=fillColor,fill opacity=0.20] (200.48, 59.30) circle (  2.13);

\path[fill=fillColor,fill opacity=0.20] (204.49, 56.37) circle (  2.13);

\path[fill=fillColor,fill opacity=0.20] (201.48, 63.95) circle (  2.13);

\path[fill=fillColor,fill opacity=0.20] (199.48, 61.45) circle (  2.13);

\path[fill=fillColor,fill opacity=0.20] (198.47, 60.41) circle (  2.13);

\path[fill=fillColor,fill opacity=0.20] (196.47, 63.43) circle (  2.13);

\path[fill=fillColor,fill opacity=0.20] (198.47, 65.24) circle (  2.13);

\path[fill=fillColor,fill opacity=0.20] (201.48, 68.25) circle (  2.13);

\path[fill=fillColor,fill opacity=0.20] (207.50, 68.16) circle (  2.13);

\path[fill=fillColor,fill opacity=0.20] (215.53, 61.53) circle (  2.13);

\path[fill=fillColor,fill opacity=0.20] (216.53, 85.99) circle (  2.13);

\path[fill=fillColor,fill opacity=0.20] (207.50, 62.65) circle (  2.13);

\path[fill=fillColor,fill opacity=0.20] (208.51, 53.35) circle (  2.13);

\path[fill=fillColor,fill opacity=0.20] (211.52, 59.38) circle (  2.13);

\path[fill=fillColor,fill opacity=0.20] (216.53, 61.28) circle (  2.13);

\path[fill=fillColor,fill opacity=0.20] (214.53, 55.25) circle (  2.13);

\path[fill=fillColor,fill opacity=0.20] (211.52, 57.57) circle (  2.13);

\path[fill=fillColor,fill opacity=0.20] (212.52, 65.93) circle (  2.13);

\path[fill=fillColor,fill opacity=0.20] (220.54, 69.54) circle (  2.13);

\path[fill=fillColor,fill opacity=0.20] (223.55, 62.74) circle (  2.13);

\path[fill=fillColor,fill opacity=0.20] (222.55, 61.10) circle (  2.13);

\path[fill=fillColor,fill opacity=0.20] (227.57, 86.59) circle (  2.13);

\path[fill=fillColor,fill opacity=0.20] (223.55, 79.79) circle (  2.13);

\path[fill=fillColor,fill opacity=0.20] (219.54, 57.23) circle (  2.13);

\path[fill=fillColor,fill opacity=0.20] (212.52, 45.34) circle (  2.13);

\path[fill=fillColor,fill opacity=0.20] (209.51, 61.88) circle (  2.13);

\path[fill=fillColor,fill opacity=0.20] (207.50, 70.58) circle (  2.13);

\path[fill=fillColor,fill opacity=0.20] (195.46, 60.76) circle (  2.13);

\path[fill=fillColor,fill opacity=0.20] (200.48, 54.99) circle (  2.13);

\path[fill=fillColor,fill opacity=0.20] (200.48, 65.50) circle (  2.13);

\path[fill=fillColor,fill opacity=0.20] (203.49, 66.70) circle (  2.13);

\path[fill=fillColor,fill opacity=0.20] (201.48, 62.91) circle (  2.13);

\path[fill=fillColor,fill opacity=0.20] (194.46, 61.79) circle (  2.13);

\path[fill=fillColor,fill opacity=0.20] (206.50, 58.61) circle (  2.13);

\path[fill=fillColor,fill opacity=0.20] (210.51, 66.96) circle (  2.13);

\path[fill=fillColor,fill opacity=0.20] (213.52, 78.84) circle (  2.13);

\path[fill=fillColor,fill opacity=0.20] (222.55, 81.34) circle (  2.13);

\path[fill=fillColor,fill opacity=0.20] (223.55, 95.29) circle (  2.13);

\path[fill=fillColor,fill opacity=0.20] (207.50, 92.36) circle (  2.13);

\path[fill=fillColor,fill opacity=0.20] (209.51, 70.40) circle (  2.13);

\path[fill=fillColor,fill opacity=0.20] (212.52, 55.25) circle (  2.13);

\path[fill=fillColor,fill opacity=0.20] (211.52, 52.84) circle (  2.13);

\path[fill=fillColor,fill opacity=0.20] (212.52, 53.09) circle (  2.13);

\path[fill=fillColor,fill opacity=0.20] (214.53, 53.27) circle (  2.13);

\path[fill=fillColor,fill opacity=0.20] (217.53, 60.59) circle (  2.13);

\path[fill=fillColor,fill opacity=0.20] (215.53, 69.20) circle (  2.13);

\path[fill=fillColor,fill opacity=0.20] (218.54, 74.62) circle (  2.13);

\path[fill=fillColor,fill opacity=0.20] (217.53, 66.44) circle (  2.13);

\path[fill=fillColor,fill opacity=0.20] (224.56, 53.53) circle (  2.13);

\path[fill=fillColor,fill opacity=0.20] (231.58, 64.03) circle (  2.13);

\path[fill=fillColor,fill opacity=0.20] (220.54, 69.63) circle (  2.13);

\path[fill=fillColor,fill opacity=0.20] (211.52, 59.04) circle (  2.13);

\path[fill=fillColor,fill opacity=0.20] (214.53, 53.61) circle (  2.13);

\path[fill=fillColor,fill opacity=0.20] (215.53, 51.89) circle (  2.13);

\path[fill=fillColor,fill opacity=0.20] (209.51, 58.43) circle (  2.13);

\path[fill=fillColor,fill opacity=0.20] (205.50, 63.77) circle (  2.13);

\path[fill=fillColor,fill opacity=0.20] (203.49, 60.93) circle (  2.13);

\path[fill=fillColor,fill opacity=0.20] (200.48, 59.81) circle (  2.13);

\path[fill=fillColor,fill opacity=0.20] (194.46, 62.05) circle (  2.13);

\path[fill=fillColor,fill opacity=0.20] (207.50, 64.29) circle (  2.13);

\path[fill=fillColor,fill opacity=0.20] (206.50, 72.13) circle (  2.13);

\path[fill=fillColor,fill opacity=0.20] (205.50, 77.29) circle (  2.13);

\path[fill=fillColor,fill opacity=0.20] (213.52, 69.46) circle (  2.13);

\path[fill=fillColor,fill opacity=0.20] (219.54, 74.45) circle (  2.13);

\path[fill=fillColor,fill opacity=0.20] (212.52, 85.99) circle (  2.13);

\path[fill=fillColor,fill opacity=0.20] (210.51, 68.51) circle (  2.13);

\path[fill=fillColor,fill opacity=0.20] (208.51, 58.18) circle (  2.13);

\path[fill=fillColor,fill opacity=0.20] (211.52, 53.09) circle (  2.13);

\path[fill=fillColor,fill opacity=0.20] (205.50, 53.70) circle (  2.13);

\path[fill=fillColor,fill opacity=0.20] (214.53, 64.12) circle (  2.13);

\path[fill=fillColor,fill opacity=0.20] (209.51, 83.06) circle (  2.13);

\path[fill=fillColor,fill opacity=0.20] (208.51, 79.45) circle (  2.13);

\path[fill=fillColor,fill opacity=0.20] (216.53, 50.25) circle (  2.13);

\path[fill=fillColor,fill opacity=0.20] (222.55, 64.38) circle (  2.13);

\path[fill=fillColor,fill opacity=0.20] (227.57, 70.32) circle (  2.13);

\path[fill=fillColor,fill opacity=0.20] (218.54, 64.12) circle (  2.13);

\path[fill=fillColor,fill opacity=0.20] (211.52, 68.68) circle (  2.13);

\path[fill=fillColor,fill opacity=0.20] (207.50, 67.91) circle (  2.13);

\path[fill=fillColor,fill opacity=0.20] (209.51, 62.83) circle (  2.13);

\path[fill=fillColor,fill opacity=0.20] (210.51, 61.02) circle (  2.13);

\path[fill=fillColor,fill opacity=0.20] (212.52, 62.31) circle (  2.13);

\path[fill=fillColor,fill opacity=0.20] (210.51, 61.71) circle (  2.13);

\path[fill=fillColor,fill opacity=0.20] (210.51, 59.90) circle (  2.13);

\path[fill=fillColor,fill opacity=0.20] (215.53, 66.10) circle (  2.13);

\path[fill=fillColor,fill opacity=0.20] (215.53, 78.67) circle (  2.13);

\path[fill=fillColor,fill opacity=0.20] (211.52, 78.15) circle (  2.13);

\path[fill=fillColor,fill opacity=0.20] (205.50, 53.44) circle (  2.13);

\path[fill=fillColor,fill opacity=0.20] (210.51, 44.83) circle (  2.13);

\path[fill=fillColor,fill opacity=0.20] (209.51, 55.85) circle (  2.13);

\path[fill=fillColor,fill opacity=0.20] (208.51, 80.57) circle (  2.13);

\path[fill=fillColor,fill opacity=0.20] (205.50, 78.41) circle (  2.13);

\path[fill=fillColor,fill opacity=0.20] (209.51, 50.86) circle (  2.13);

\path[fill=fillColor,fill opacity=0.20] (209.51, 48.53) circle (  2.13);

\path[fill=fillColor,fill opacity=0.20] (210.51, 69.63) circle (  2.13);

\path[fill=fillColor,fill opacity=0.20] (211.52, 74.19) circle (  2.13);

\path[fill=fillColor,fill opacity=0.20] (222.55, 78.50) circle (  2.13);

\path[fill=fillColor,fill opacity=0.20] (220.54, 76.09) circle (  2.13);

\path[fill=fillColor,fill opacity=0.20] (218.54, 66.44) circle (  2.13);

\path[fill=fillColor,fill opacity=0.20] (213.52, 63.69) circle (  2.13);

\path[fill=fillColor,fill opacity=0.20] (207.50, 63.60) circle (  2.13);

\path[fill=fillColor,fill opacity=0.20] (205.50, 65.84) circle (  2.13);

\path[fill=fillColor,fill opacity=0.20] (204.49, 68.85) circle (  2.13);

\path[fill=fillColor,fill opacity=0.20] (209.51, 73.59) circle (  2.13);

\path[fill=fillColor,fill opacity=0.20] (214.53, 78.24) circle (  2.13);

\path[fill=fillColor,fill opacity=0.20] (217.53, 82.46) circle (  2.13);

\path[fill=fillColor,fill opacity=0.20] (223.55, 84.01) circle (  2.13);

\path[fill=fillColor,fill opacity=0.20] (225.56, 88.06) circle (  2.13);

\path[fill=fillColor,fill opacity=0.20] (206.50, 83.06) circle (  2.13);

\path[fill=fillColor,fill opacity=0.20] (209.51, 60.07) circle (  2.13);

\path[fill=fillColor,fill opacity=0.20] (205.50, 57.83) circle (  2.13);

\path[fill=fillColor,fill opacity=0.20] (205.50, 73.85) circle (  2.13);

\path[fill=fillColor,fill opacity=0.20] (206.50, 67.39) circle (  2.13);

\path[fill=fillColor,fill opacity=0.20] (207.50, 46.38) circle (  2.13);

\path[fill=fillColor,fill opacity=0.20] (208.51, 56.11) circle (  2.13);

\path[fill=fillColor,fill opacity=0.20] (205.50, 75.57) circle (  2.13);

\path[fill=fillColor,fill opacity=0.20] (215.53, 66.18) circle (  2.13);

\path[fill=fillColor,fill opacity=0.20] (215.53, 53.35) circle (  2.13);

\path[fill=fillColor,fill opacity=0.20] (213.52, 93.65) circle (  2.13);

\path[fill=fillColor,fill opacity=0.20] (212.52, 71.87) circle (  2.13);

\path[fill=fillColor,fill opacity=0.20] (212.52, 62.65) circle (  2.13);

\path[fill=fillColor,fill opacity=0.20] (212.52, 58.18) circle (  2.13);

\path[fill=fillColor,fill opacity=0.20] (206.50, 54.56) circle (  2.13);

\path[fill=fillColor,fill opacity=0.20] (206.50, 65.41) circle (  2.13);

\path[fill=fillColor,fill opacity=0.20] (214.53, 80.31) circle (  2.13);

\path[fill=fillColor,fill opacity=0.20] (209.51, 93.74) circle (  2.13);

\path[fill=fillColor,fill opacity=0.20] (214.53, 97.96) circle (  2.13);

\path[fill=fillColor,fill opacity=0.20] (209.51, 81.34) circle (  2.13);

\path[fill=fillColor,fill opacity=0.20] (208.51, 72.73) circle (  2.13);

\path[fill=fillColor,fill opacity=0.20] (209.51, 57.66) circle (  2.13);

\path[fill=fillColor,fill opacity=0.20] (208.51, 45.78) circle (  2.13);

\path[fill=fillColor,fill opacity=0.20] (205.50, 53.35) circle (  2.13);

\path[fill=fillColor,fill opacity=0.20] (205.50, 63.08) circle (  2.13);

\path[fill=fillColor,fill opacity=0.20] (209.51, 63.86) circle (  2.13);

\path[fill=fillColor,fill opacity=0.20] (213.52, 67.22) circle (  2.13);

\path[fill=fillColor,fill opacity=0.20] (210.51, 64.46) circle (  2.13);

\path[fill=fillColor,fill opacity=0.20] (218.54, 45.52) circle (  2.13);

\path[fill=fillColor,fill opacity=0.20] (221.55, 40.26) circle (  2.13);

\path[fill=fillColor,fill opacity=0.20] (223.55, 63.86) circle (  2.13);

\path[fill=fillColor,fill opacity=0.20] (221.55, 85.22) circle (  2.13);

\path[fill=fillColor,fill opacity=0.20] (218.54, 94.17) circle (  2.13);

\path[fill=fillColor,fill opacity=0.20] (212.52, 68.08) circle (  2.13);

\path[fill=fillColor,fill opacity=0.20] (212.52, 51.98) circle (  2.13);

\path[fill=fillColor,fill opacity=0.20] (211.52, 63.43) circle (  2.13);

\path[fill=fillColor,fill opacity=0.20] (213.52, 69.63) circle (  2.13);

\path[fill=fillColor,fill opacity=0.20] (214.53, 76.09) circle (  2.13);

\path[fill=fillColor,fill opacity=0.20] (213.52, 92.36) circle (  2.13);

\path[fill=fillColor,fill opacity=0.20] (215.53,110.53) circle (  2.13);

\path[fill=fillColor,fill opacity=0.20] (216.53,115.96) circle (  2.13);

\path[fill=fillColor,fill opacity=0.20] (213.52, 91.93) circle (  2.13);

\path[fill=fillColor,fill opacity=0.20] (209.51, 64.55) circle (  2.13);

\path[fill=fillColor,fill opacity=0.20] (209.51, 55.16) circle (  2.13);

\path[fill=fillColor,fill opacity=0.20] (206.50, 52.49) circle (  2.13);

\path[fill=fillColor,fill opacity=0.20] (210.51, 49.22) circle (  2.13);

\path[fill=fillColor,fill opacity=0.20] (207.50, 55.33) circle (  2.13);

\path[fill=fillColor,fill opacity=0.20] (209.51, 72.30) circle (  2.13);

\path[fill=fillColor,fill opacity=0.20] (211.52, 78.67) circle (  2.13);

\path[fill=fillColor,fill opacity=0.20] (212.52, 65.06) circle (  2.13);

\path[fill=fillColor,fill opacity=0.20] (214.53, 56.37) circle (  2.13);

\path[fill=fillColor,fill opacity=0.20] (221.55, 64.89) circle (  2.13);

\path[fill=fillColor,fill opacity=0.20] (218.54, 64.98) circle (  2.13);

\path[fill=fillColor,fill opacity=0.20] (221.55, 59.90) circle (  2.13);

\path[fill=fillColor,fill opacity=0.20] (223.55, 70.83) circle (  2.13);

\path[fill=fillColor,fill opacity=0.20] (218.54, 92.45) circle (  2.13);

\path[fill=fillColor,fill opacity=0.20] (213.52, 91.50) circle (  2.13);

\path[fill=fillColor,fill opacity=0.20] (212.52, 81.77) circle (  2.13);

\path[fill=fillColor,fill opacity=0.20] (214.53, 71.01) circle (  2.13);

\path[fill=fillColor,fill opacity=0.20] (216.53, 66.61) circle (  2.13);

\path[fill=fillColor,fill opacity=0.20] (213.52, 65.75) circle (  2.13);

\path[fill=fillColor,fill opacity=0.20] (190.45, 58.35) circle (  2.13);

\path[fill=fillColor,fill opacity=0.20] (224.56, 76.60) circle (  2.13);

\path[fill=fillColor,fill opacity=0.20] (217.53,111.57) circle (  2.13);

\path[fill=fillColor,fill opacity=0.20] (214.53, 88.23) circle (  2.13);

\path[fill=fillColor,fill opacity=0.20] (205.50, 71.01) circle (  2.13);

\path[fill=fillColor,fill opacity=0.20] (208.51, 57.75) circle (  2.13);

\path[fill=fillColor,fill opacity=0.20] (209.51, 48.44) circle (  2.13);

\path[fill=fillColor,fill opacity=0.20] (210.51, 53.78) circle (  2.13);

\path[fill=fillColor,fill opacity=0.20] (209.51, 68.60) circle (  2.13);

\path[fill=fillColor,fill opacity=0.20] (207.50, 77.81) circle (  2.13);

\path[fill=fillColor,fill opacity=0.20] (213.52, 76.95) circle (  2.13);

\path[fill=fillColor,fill opacity=0.20] (213.52, 81.77) circle (  2.13);

\path[fill=fillColor,fill opacity=0.20] (212.52, 80.48) circle (  2.13);

\path[fill=fillColor,fill opacity=0.20] (214.53, 61.79) circle (  2.13);

\path[fill=fillColor,fill opacity=0.20] (220.54, 52.32) circle (  2.13);

\path[fill=fillColor,fill opacity=0.20] (219.54, 64.20) circle (  2.13);

\path[fill=fillColor,fill opacity=0.20] (215.53, 70.23) circle (  2.13);

\path[fill=fillColor,fill opacity=0.20] (218.54, 61.96) circle (  2.13);

\path[fill=fillColor,fill opacity=0.20] (221.55, 62.22) circle (  2.13);

\path[fill=fillColor,fill opacity=0.20] (223.55, 77.47) circle (  2.13);

\path[fill=fillColor,fill opacity=0.20] (222.55, 95.55) circle (  2.13);

\path[fill=fillColor,fill opacity=0.20] (218.54,101.41) circle (  2.13);

\path[fill=fillColor,fill opacity=0.20] (216.53, 99.94) circle (  2.13);

\path[fill=fillColor,fill opacity=0.20] (214.53, 90.64) circle (  2.13);

\path[fill=fillColor,fill opacity=0.20] (216.53, 95.29) circle (  2.13);

\path[fill=fillColor,fill opacity=0.20] (221.55, 97.44) circle (  2.13);

\path[fill=fillColor,fill opacity=0.20] (215.53, 80.48) circle (  2.13);

\path[fill=fillColor,fill opacity=0.20] (216.53, 73.93) circle (  2.13);

\path[fill=fillColor,fill opacity=0.20] (214.53, 68.77) circle (  2.13);

\path[fill=fillColor,fill opacity=0.20] (212.52, 74.54) circle (  2.13);

\path[fill=fillColor,fill opacity=0.20] (211.52, 73.33) circle (  2.13);

\path[fill=fillColor,fill opacity=0.20] (215.53, 60.76) circle (  2.13);

\path[fill=fillColor,fill opacity=0.20] (213.52, 60.33) circle (  2.13);

\path[fill=fillColor,fill opacity=0.20] (214.53, 55.08) circle (  2.13);

\path[fill=fillColor,fill opacity=0.20] (219.54, 55.51) circle (  2.13);

\path[fill=fillColor,fill opacity=0.20] (221.55, 83.75) circle (  2.13);

\path[fill=fillColor,fill opacity=0.20] (208.51, 74.80) circle (  2.13);

\path[fill=fillColor,fill opacity=0.20] (206.50, 68.94) circle (  2.13);

\path[fill=fillColor,fill opacity=0.20] (209.51, 72.73) circle (  2.13);

\path[fill=fillColor,fill opacity=0.20] (213.52, 71.78) circle (  2.13);

\path[fill=fillColor,fill opacity=0.20] (215.53, 67.56) circle (  2.13);

\path[fill=fillColor,fill opacity=0.20] (214.53, 68.08) circle (  2.13);

\path[fill=fillColor,fill opacity=0.20] (210.51, 72.64) circle (  2.13);

\path[fill=fillColor,fill opacity=0.20] (209.51, 67.82) circle (  2.13);

\path[fill=fillColor,fill opacity=0.20] (213.52, 57.66) circle (  2.13);

\path[fill=fillColor,fill opacity=0.20] (216.53, 56.20) circle (  2.13);

\path[fill=fillColor,fill opacity=0.20] (216.53, 65.84) circle (  2.13);

\path[fill=fillColor,fill opacity=0.20] (214.53, 69.03) circle (  2.13);

\path[fill=fillColor,fill opacity=0.20] (215.53, 60.41) circle (  2.13);

\path[fill=fillColor,fill opacity=0.20] (220.54, 57.31) circle (  2.13);

\path[fill=fillColor,fill opacity=0.20] (217.53, 67.13) circle (  2.13);

\path[fill=fillColor,fill opacity=0.20] (215.53, 73.25) circle (  2.13);

\path[fill=fillColor,fill opacity=0.20] (213.52, 71.70) circle (  2.13);

\path[fill=fillColor,fill opacity=0.20] (216.53, 61.02) circle (  2.13);

\path[fill=fillColor,fill opacity=0.20] (216.53, 59.98) circle (  2.13);

\path[fill=fillColor,fill opacity=0.20] (214.53, 69.63) circle (  2.13);

\path[fill=fillColor,fill opacity=0.20] (210.51, 69.11) circle (  2.13);

\path[fill=fillColor,fill opacity=0.20] (214.53, 66.53) circle (  2.13);

\path[fill=fillColor,fill opacity=0.20] (215.53, 68.85) circle (  2.13);

\path[fill=fillColor,fill opacity=0.20] (214.53, 61.79) circle (  2.13);

\path[fill=fillColor,fill opacity=0.20] (214.53, 57.49) circle (  2.13);

\path[fill=fillColor,fill opacity=0.20] (213.52, 66.61) circle (  2.13);

\path[fill=fillColor,fill opacity=0.20] (208.51, 63.43) circle (  2.13);

\path[fill=fillColor,fill opacity=0.20] (216.53, 53.96) circle (  2.13);

\path[fill=fillColor,fill opacity=0.20] (211.52, 55.16) circle (  2.13);

\path[fill=fillColor,fill opacity=0.20] (213.52, 50.34) circle (  2.13);

\path[fill=fillColor,fill opacity=0.20] (200.48, 47.93) circle (  2.13);

\path[fill=fillColor,fill opacity=0.20] (212.52, 58.52) circle (  2.13);

\path[fill=fillColor,fill opacity=0.20] (214.53, 65.58) circle (  2.13);

\path[fill=fillColor,fill opacity=0.20] (217.53, 70.58) circle (  2.13);

\path[fill=fillColor,fill opacity=0.20] (217.53, 80.13) circle (  2.13);

\path[fill=fillColor,fill opacity=0.20] (220.54,102.78) circle (  2.13);

\path[fill=fillColor,fill opacity=0.20] (215.53, 75.05) circle (  2.13);

\path[fill=fillColor,fill opacity=0.20] (213.52, 72.56) circle (  2.13);

\path[fill=fillColor,fill opacity=0.20] (210.51, 65.75) circle (  2.13);

\path[fill=fillColor,fill opacity=0.20] (208.51, 65.75) circle (  2.13);

\path[fill=fillColor,fill opacity=0.20] (209.51, 64.63) circle (  2.13);

\path[fill=fillColor,fill opacity=0.20] (212.52, 62.31) circle (  2.13);

\path[fill=fillColor,fill opacity=0.20] (214.53, 67.13) circle (  2.13);

\path[fill=fillColor,fill opacity=0.20] (215.53, 67.82) circle (  2.13);

\path[fill=fillColor,fill opacity=0.20] (213.52, 60.59) circle (  2.13);

\path[fill=fillColor,fill opacity=0.20] (211.52, 61.19) circle (  2.13);

\path[fill=fillColor,fill opacity=0.20] (215.53, 67.30) circle (  2.13);

\path[fill=fillColor,fill opacity=0.20] (211.52, 69.71) circle (  2.13);

\path[fill=fillColor,fill opacity=0.20] (207.50, 72.73) circle (  2.13);

\path[fill=fillColor,fill opacity=0.20] (211.52, 63.08) circle (  2.13);

\path[fill=fillColor,fill opacity=0.20] (215.53, 52.58) circle (  2.13);

\path[fill=fillColor,fill opacity=0.20] (214.53, 59.04) circle (  2.13);

\path[fill=fillColor,fill opacity=0.20] (212.52, 64.63) circle (  2.13);

\path[fill=fillColor,fill opacity=0.20] (212.52, 58.78) circle (  2.13);

\path[fill=fillColor,fill opacity=0.20] (214.53, 61.62) circle (  2.13);

\path[fill=fillColor,fill opacity=0.20] (213.52, 62.74) circle (  2.13);

\path[fill=fillColor,fill opacity=0.20] (212.52, 56.02) circle (  2.13);

\path[fill=fillColor,fill opacity=0.20] (210.51, 56.97) circle (  2.13);

\path[fill=fillColor,fill opacity=0.20] (209.51, 56.20) circle (  2.13);

\path[fill=fillColor,fill opacity=0.20] (208.51, 50.25) circle (  2.13);

\path[fill=fillColor,fill opacity=0.20] (209.51, 55.16) circle (  2.13);

\path[fill=fillColor,fill opacity=0.20] (210.51, 57.66) circle (  2.13);

\path[fill=fillColor,fill opacity=0.20] (215.53, 62.91) circle (  2.13);

\path[fill=fillColor,fill opacity=0.20] (217.53, 82.98) circle (  2.13);

\path[fill=fillColor,fill opacity=0.20] (207.50, 91.42) circle (  2.13);

\path[fill=fillColor,fill opacity=0.20] (208.51, 93.31) circle (  2.13);

\path[fill=fillColor,fill opacity=0.20] (209.51, 87.37) circle (  2.13);

\path[fill=fillColor,fill opacity=0.20] (213.52, 69.89) circle (  2.13);

\path[fill=fillColor,fill opacity=0.20] (217.53, 74.54) circle (  2.13);

\path[fill=fillColor,fill opacity=0.20] (214.53, 80.74) circle (  2.13);

\path[fill=fillColor,fill opacity=0.20] (211.52, 71.61) circle (  2.13);

\path[fill=fillColor,fill opacity=0.20] (212.52, 61.79) circle (  2.13);

\path[fill=fillColor,fill opacity=0.20] (213.52, 59.98) circle (  2.13);

\path[fill=fillColor,fill opacity=0.20] (209.51, 59.30) circle (  2.13);

\path[fill=fillColor,fill opacity=0.20] (209.51, 59.55) circle (  2.13);

\path[fill=fillColor,fill opacity=0.20] (209.51, 60.59) circle (  2.13);

\path[fill=fillColor,fill opacity=0.20] (210.51, 52.49) circle (  2.13);

\path[fill=fillColor,fill opacity=0.20] (212.52, 46.03) circle (  2.13);

\path[fill=fillColor,fill opacity=0.20] (211.52, 47.07) circle (  2.13);

\path[fill=fillColor,fill opacity=0.20] (211.52, 50.34) circle (  2.13);

\path[fill=fillColor,fill opacity=0.20] (211.52, 53.61) circle (  2.13);

\path[fill=fillColor,fill opacity=0.20] (209.51, 60.85) circle (  2.13);

\path[fill=fillColor,fill opacity=0.20] (209.51, 67.05) circle (  2.13);

\path[fill=fillColor,fill opacity=0.20] (209.51, 63.00) circle (  2.13);

\path[fill=fillColor,fill opacity=0.20] (208.51, 61.19) circle (  2.13);

\path[fill=fillColor,fill opacity=0.20] (207.50, 66.27) circle (  2.13);

\path[fill=fillColor,fill opacity=0.20] (209.51, 72.56) circle (  2.13);

\path[fill=fillColor,fill opacity=0.20] (212.52, 83.67) circle (  2.13);

\path[fill=fillColor,fill opacity=0.20] (220.54, 87.37) circle (  2.13);

\path[fill=fillColor,fill opacity=0.20] (215.53,107.09) circle (  2.13);

\path[fill=fillColor,fill opacity=0.20] (212.52,102.61) circle (  2.13);

\path[fill=fillColor,fill opacity=0.20] (215.53, 88.14) circle (  2.13);

\path[fill=fillColor,fill opacity=0.20] (215.53, 82.72) circle (  2.13);

\path[fill=fillColor,fill opacity=0.20] (212.52, 75.57) circle (  2.13);

\path[fill=fillColor,fill opacity=0.20] (211.52, 59.81) circle (  2.13);

\path[fill=fillColor,fill opacity=0.20] (209.51, 55.33) circle (  2.13);

\path[fill=fillColor,fill opacity=0.20] (209.51, 59.73) circle (  2.13);

\path[fill=fillColor,fill opacity=0.20] (211.52, 58.18) circle (  2.13);

\path[fill=fillColor,fill opacity=0.20] (212.52, 56.88) circle (  2.13);

\path[fill=fillColor,fill opacity=0.20] (214.53, 59.81) circle (  2.13);

\path[fill=fillColor,fill opacity=0.20] (212.52, 63.08) circle (  2.13);

\path[fill=fillColor,fill opacity=0.20] (209.51, 75.83) circle (  2.13);

\path[fill=fillColor,fill opacity=0.20] (210.51, 84.27) circle (  2.13);

\path[fill=fillColor,fill opacity=0.20] (215.53, 82.98) circle (  2.13);

\path[fill=fillColor,fill opacity=0.20] (215.53, 94.43) circle (  2.13);

\path[fill=fillColor,fill opacity=0.20] (217.53,102.52) circle (  2.13);

\path[fill=fillColor,fill opacity=0.20] (216.53, 92.19) circle (  2.13);

\path[fill=fillColor,fill opacity=0.20] (211.52, 83.92) circle (  2.13);

\path[fill=fillColor,fill opacity=0.20] (210.51, 88.57) circle (  2.13);

\path[fill=fillColor,fill opacity=0.20] (203.49, 98.48) circle (  2.13);

\path[fill=fillColor,fill opacity=0.20] (213.52,101.41) circle (  2.13);

\path[fill=fillColor,fill opacity=0.20] (215.53, 96.67) circle (  2.13);

\path[fill=fillColor,fill opacity=0.20] (202.49,101.58) circle (  2.13);

\path[fill=fillColor,fill opacity=0.20] (200.48,115.96) circle (  2.13);

\path[fill=fillColor,fill opacity=0.20] (217.53,115.10) circle (  2.13);

\path[fill=fillColor,fill opacity=0.20] (222.55,112.51) circle (  2.13);

\path[fill=fillColor,fill opacity=0.20] (208.51, 82.03) circle (  2.13);

\path[fill=fillColor,fill opacity=0.20] (212.52, 81.77) circle (  2.13);

\path[fill=fillColor,fill opacity=0.20] (212.52, 80.57) circle (  2.13);

\path[fill=fillColor,fill opacity=0.20] (214.53, 87.37) circle (  2.13);

\path[fill=fillColor,fill opacity=0.20] (210.51, 85.65) circle (  2.13);

\path[fill=fillColor,fill opacity=0.20] (205.50, 82.46) circle (  2.13);

\path[fill=fillColor,fill opacity=0.20] (204.49, 79.10) circle (  2.13);

\path[fill=fillColor,fill opacity=0.20] (209.51, 73.93) circle (  2.13);

\path[fill=fillColor,fill opacity=0.20] (211.52, 69.46) circle (  2.13);

\path[fill=fillColor,fill opacity=0.20] (214.53, 70.49) circle (  2.13);

\path[fill=fillColor,fill opacity=0.20] (221.55, 76.09) circle (  2.13);

\path[fill=fillColor,fill opacity=0.20] (228.57, 79.53) circle (  2.13);

\path[fill=fillColor,fill opacity=0.20] (206.50, 89.09) circle (  2.13);

\path[fill=fillColor,fill opacity=0.20] (207.50, 84.10) circle (  2.13);

\path[fill=fillColor,fill opacity=0.20] (198.47, 81.08) circle (  2.13);

\path[fill=fillColor,fill opacity=0.20] (203.49, 76.69) circle (  2.13);

\path[fill=fillColor,fill opacity=0.20] (204.49, 69.80) circle (  2.13);

\path[fill=fillColor,fill opacity=0.20] (208.51, 65.84) circle (  2.13);

\path[fill=fillColor,fill opacity=0.20] (211.52, 66.61) circle (  2.13);

\path[fill=fillColor,fill opacity=0.20] (214.53, 69.54) circle (  2.13);

\path[fill=fillColor,fill opacity=0.20] (220.54, 75.14) circle (  2.13);

\path[fill=fillColor,fill opacity=0.20] (239.61, 80.13) circle (  2.13);

\path[fill=fillColor,fill opacity=0.20] (247.63, 89.18) circle (  2.13);

\path[fill=fillColor,fill opacity=0.20] (212.52,100.20) circle (  2.13);

\path[fill=fillColor,fill opacity=0.20] (207.50, 84.44) circle (  2.13);

\path[fill=fillColor,fill opacity=0.20] (201.48, 71.52) circle (  2.13);

\path[fill=fillColor,fill opacity=0.20] (204.49, 68.68) circle (  2.13);

\path[fill=fillColor,fill opacity=0.20] (204.49, 65.41) circle (  2.13);

\path[fill=fillColor,fill opacity=0.20] (206.50, 61.53) circle (  2.13);

\path[fill=fillColor,fill opacity=0.20] (212.52, 62.05) circle (  2.13);

\path[fill=fillColor,fill opacity=0.20] (211.52, 65.50) circle (  2.13);

\path[fill=fillColor,fill opacity=0.20] (217.53, 69.37) circle (  2.13);

\path[fill=fillColor,fill opacity=0.20] (223.55, 78.33) circle (  2.13);

\path[fill=fillColor,fill opacity=0.20] (244.62, 90.81) circle (  2.13);

\path[fill=fillColor,fill opacity=0.20] (226.56,100.54) circle (  2.13);

\path[fill=fillColor,fill opacity=0.20] (211.52, 91.07) circle (  2.13);

\path[fill=fillColor,fill opacity=0.20] (209.51, 74.97) circle (  2.13);

\path[fill=fillColor,fill opacity=0.20] (209.51, 59.81) circle (  2.13);

\path[fill=fillColor,fill opacity=0.20] (208.51, 53.70) circle (  2.13);

\path[fill=fillColor,fill opacity=0.20] (208.51, 55.16) circle (  2.13);

\path[fill=fillColor,fill opacity=0.20] (210.51, 59.47) circle (  2.13);

\path[fill=fillColor,fill opacity=0.20] (214.53, 63.60) circle (  2.13);

\path[fill=fillColor,fill opacity=0.20] (221.55, 63.34) circle (  2.13);

\path[fill=fillColor,fill opacity=0.20] (220.54, 65.24) circle (  2.13);

\path[fill=fillColor,fill opacity=0.20] (235.59, 77.98) circle (  2.13);

\path[fill=fillColor,fill opacity=0.20] (213.52, 74.80) circle (  2.13);

\path[fill=fillColor,fill opacity=0.20] (212.52, 65.50) circle (  2.13);

\path[fill=fillColor,fill opacity=0.20] (212.52, 57.14) circle (  2.13);

\path[fill=fillColor,fill opacity=0.20] (211.52, 47.24) circle (  2.13);

\path[fill=fillColor,fill opacity=0.20] (209.51, 48.36) circle (  2.13);

\path[fill=fillColor,fill opacity=0.20] (211.52, 57.66) circle (  2.13);

\path[fill=fillColor,fill opacity=0.20] (215.53, 62.22) circle (  2.13);

\path[fill=fillColor,fill opacity=0.20] (225.56, 62.48) circle (  2.13);

\path[fill=fillColor,fill opacity=0.20] (231.58, 67.05) circle (  2.13);

\path[fill=fillColor,fill opacity=0.20] (244.62, 82.03) circle (  2.13);

\path[fill=fillColor,fill opacity=0.20] (210.51, 81.25) circle (  2.13);

\path[fill=fillColor,fill opacity=0.20] (209.51, 77.81) circle (  2.13);

\path[fill=fillColor,fill opacity=0.20] (213.52, 66.61) circle (  2.13);

\path[fill=fillColor,fill opacity=0.20] (213.52, 59.64) circle (  2.13);

\path[fill=fillColor,fill opacity=0.20] (210.51, 56.54) circle (  2.13);

\path[fill=fillColor,fill opacity=0.20] (208.51, 45.95) circle (  2.13);

\path[fill=fillColor,fill opacity=0.20] (213.52, 47.41) circle (  2.13);

\path[fill=fillColor,fill opacity=0.20] (215.53, 56.37) circle (  2.13);

\path[fill=fillColor,fill opacity=0.20] (216.53, 60.07) circle (  2.13);

\path[fill=fillColor,fill opacity=0.20] (226.56, 65.50) circle (  2.13);

\path[fill=fillColor,fill opacity=0.20] (228.57, 84.78) circle (  2.13);

\path[fill=fillColor,fill opacity=0.20] (208.51, 74.28) circle (  2.13);

\path[fill=fillColor,fill opacity=0.20] (207.50, 71.52) circle (  2.13);

\path[fill=fillColor,fill opacity=0.20] (207.50, 73.59) circle (  2.13);

\path[fill=fillColor,fill opacity=0.20] (210.51, 77.21) circle (  2.13);

\path[fill=fillColor,fill opacity=0.20] (210.51, 68.85) circle (  2.13);

\path[fill=fillColor,fill opacity=0.20] (208.51, 59.90) circle (  2.13);

\path[fill=fillColor,fill opacity=0.20] (210.51, 55.08) circle (  2.13);

\path[fill=fillColor,fill opacity=0.20] (209.51, 48.62) circle (  2.13);

\path[fill=fillColor,fill opacity=0.20] (209.51, 52.06) circle (  2.13);

\path[fill=fillColor,fill opacity=0.20] (218.54, 59.55) circle (  2.13);

\path[fill=fillColor,fill opacity=0.20] (219.54, 61.10) circle (  2.13);

\path[fill=fillColor,fill opacity=0.20] (225.56, 70.75) circle (  2.13);

\path[fill=fillColor,fill opacity=0.20] (204.49, 68.94) circle (  2.13);

\path[fill=fillColor,fill opacity=0.20] (209.51, 68.16) circle (  2.13);

\path[fill=fillColor,fill opacity=0.20] (212.52, 67.48) circle (  2.13);

\path[fill=fillColor,fill opacity=0.20] (206.50, 66.79) circle (  2.13);

\path[fill=fillColor,fill opacity=0.20] (212.52, 68.77) circle (  2.13);

\path[fill=fillColor,fill opacity=0.20] (204.49, 60.41) circle (  2.13);

\path[fill=fillColor,fill opacity=0.20] (206.50, 56.02) circle (  2.13);

\path[fill=fillColor,fill opacity=0.20] (208.51, 55.33) circle (  2.13);

\path[fill=fillColor,fill opacity=0.20] (214.53, 57.49) circle (  2.13);

\path[fill=fillColor,fill opacity=0.20] (220.54, 59.04) circle (  2.13);

\path[fill=fillColor,fill opacity=0.20] (220.54, 61.02) circle (  2.13);

\path[fill=fillColor,fill opacity=0.20] (221.55, 72.30) circle (  2.13);

\path[fill=fillColor,fill opacity=0.20] (214.53, 71.52) circle (  2.13);

\path[fill=fillColor,fill opacity=0.20] (205.50, 68.16) circle (  2.13);

\path[fill=fillColor,fill opacity=0.20] (209.51, 64.81) circle (  2.13);

\path[fill=fillColor,fill opacity=0.20] (215.53, 66.27) circle (  2.13);

\path[fill=fillColor,fill opacity=0.20] (212.52, 66.44) circle (  2.13);

\path[fill=fillColor,fill opacity=0.20] (209.51, 64.81) circle (  2.13);

\path[fill=fillColor,fill opacity=0.20] (210.51, 71.95) circle (  2.13);

\path[fill=fillColor,fill opacity=0.20] (223.55, 70.49) circle (  2.13);

\path[fill=fillColor,fill opacity=0.20] (204.49, 60.16) circle (  2.13);

\path[fill=fillColor,fill opacity=0.20] (209.51, 58.35) circle (  2.13);

\path[fill=fillColor,fill opacity=0.20] (208.51, 59.64) circle (  2.13);

\path[fill=fillColor,fill opacity=0.20] (212.52, 59.73) circle (  2.13);

\path[fill=fillColor,fill opacity=0.20] (215.53, 52.84) circle (  2.13);

\path[fill=fillColor,fill opacity=0.20] (219.54, 53.44) circle (  2.13);

\path[fill=fillColor,fill opacity=0.20] (219.54, 69.71) circle (  2.13);

\path[fill=fillColor,fill opacity=0.20] (220.54, 78.93) circle (  2.13);

\path[fill=fillColor,fill opacity=0.20] (217.53, 71.09) circle (  2.13);

\path[fill=fillColor,fill opacity=0.20] (218.54, 63.43) circle (  2.13);

\path[fill=fillColor,fill opacity=0.20] (216.53, 62.14) circle (  2.13);

\path[fill=fillColor,fill opacity=0.20] (219.54, 65.06) circle (  2.13);

\path[fill=fillColor,fill opacity=0.20] (208.51, 68.08) circle (  2.13);

\path[fill=fillColor,fill opacity=0.20] (216.53, 68.68) circle (  2.13);

\path[fill=fillColor,fill opacity=0.20] (215.53, 72.47) circle (  2.13);

\path[fill=fillColor,fill opacity=0.20] (229.57, 82.98) circle (  2.13);

\path[fill=fillColor,fill opacity=0.20] (213.52, 87.37) circle (  2.13);

\path[fill=fillColor,fill opacity=0.20] (212.52, 65.15) circle (  2.13);

\path[fill=fillColor,fill opacity=0.20] (209.51, 58.61) circle (  2.13);

\path[fill=fillColor,fill opacity=0.20] (212.52, 58.69) circle (  2.13);

\path[fill=fillColor,fill opacity=0.20] (214.53, 62.05) circle (  2.13);

\path[fill=fillColor,fill opacity=0.20] (215.53, 55.42) circle (  2.13);

\path[fill=fillColor,fill opacity=0.20] (215.53, 51.37) circle (  2.13);

\path[fill=fillColor,fill opacity=0.20] (214.53, 68.16) circle (  2.13);

\path[fill=fillColor,fill opacity=0.20] (223.55, 96.75) circle (  2.13);

\path[fill=fillColor,fill opacity=0.20] (213.52, 85.39) circle (  2.13);

\path[fill=fillColor,fill opacity=0.20] (217.53, 76.09) circle (  2.13);

\path[fill=fillColor,fill opacity=0.20] (217.53, 66.96) circle (  2.13);

\path[fill=fillColor,fill opacity=0.20] (222.55, 60.76) circle (  2.13);

\path[fill=fillColor,fill opacity=0.20] (221.55, 59.81) circle (  2.13);

\path[fill=fillColor,fill opacity=0.20] (223.55, 65.84) circle (  2.13);

\path[fill=fillColor,fill opacity=0.20] (223.55, 66.79) circle (  2.13);

\path[fill=fillColor,fill opacity=0.20] (215.53, 66.61) circle (  2.13);

\path[fill=fillColor,fill opacity=0.20] (232.58, 79.79) circle (  2.13);

\path[fill=fillColor,fill opacity=0.20] (234.59, 72.04) circle (  2.13);

\path[fill=fillColor,fill opacity=0.20] (213.52, 59.64) circle (  2.13);

\path[fill=fillColor,fill opacity=0.20] (210.51, 58.35) circle (  2.13);

\path[fill=fillColor,fill opacity=0.20] (203.49, 63.43) circle (  2.13);

\path[fill=fillColor,fill opacity=0.20] (217.53, 64.63) circle (  2.13);

\path[fill=fillColor,fill opacity=0.20] (218.54, 57.83) circle (  2.13);

\path[fill=fillColor,fill opacity=0.20] (215.53, 65.24) circle (  2.13);

\path[fill=fillColor,fill opacity=0.20] (218.54, 82.89) circle (  2.13);

\path[fill=fillColor,fill opacity=0.20] (210.51, 82.46) circle (  2.13);

\path[fill=fillColor,fill opacity=0.20] (214.53, 74.37) circle (  2.13);

\path[fill=fillColor,fill opacity=0.20] (214.53, 65.32) circle (  2.13);

\path[fill=fillColor,fill opacity=0.20] (220.54, 58.35) circle (  2.13);

\path[fill=fillColor,fill opacity=0.20] (216.53, 56.28) circle (  2.13);

\path[fill=fillColor,fill opacity=0.20] (217.53, 63.00) circle (  2.13);

\path[fill=fillColor,fill opacity=0.20] (222.55, 65.58) circle (  2.13);

\path[fill=fillColor,fill opacity=0.20] (218.54, 65.32) circle (  2.13);

\path[fill=fillColor,fill opacity=0.20] (224.56, 77.98) circle (  2.13);

\path[fill=fillColor,fill opacity=0.20] (269.70, 75.31) circle (  2.13);

\path[fill=fillColor,fill opacity=0.20] (239.61, 64.89) circle (  2.13);

\path[fill=fillColor,fill opacity=0.20] (218.54, 62.40) circle (  2.13);

\path[fill=fillColor,fill opacity=0.20] (199.48, 58.61) circle (  2.13);

\path[fill=fillColor,fill opacity=0.20] (216.53, 62.05) circle (  2.13);

\path[fill=fillColor,fill opacity=0.20] (218.54, 61.88) circle (  2.13);

\path[fill=fillColor,fill opacity=0.20] (213.52, 58.18) circle (  2.13);

\path[fill=fillColor,fill opacity=0.20] (217.53, 66.96) circle (  2.13);

\path[fill=fillColor,fill opacity=0.20] (222.55, 97.79) circle (  2.13);

\path[fill=fillColor,fill opacity=0.20] (216.53, 73.25) circle (  2.13);

\path[fill=fillColor,fill opacity=0.20] (210.51, 63.34) circle (  2.13);

\path[fill=fillColor,fill opacity=0.20] (209.51, 59.47) circle (  2.13);

\path[fill=fillColor,fill opacity=0.20] (217.53, 58.00) circle (  2.13);

\path[fill=fillColor,fill opacity=0.20] (217.53, 63.60) circle (  2.13);

\path[fill=fillColor,fill opacity=0.20] (212.52, 69.71) circle (  2.13);

\path[fill=fillColor,fill opacity=0.20] (218.54, 71.87) circle (  2.13);

\path[fill=fillColor,fill opacity=0.20] (231.58, 79.10) circle (  2.13);

\path[fill=fillColor,fill opacity=0.20] (235.59, 64.20) circle (  2.13);

\path[fill=fillColor,fill opacity=0.20] (217.53, 50.77) circle (  2.13);

\path[fill=fillColor,fill opacity=0.20] (214.53, 53.35) circle (  2.13);

\path[fill=fillColor,fill opacity=0.20] (211.52, 58.95) circle (  2.13);

\path[fill=fillColor,fill opacity=0.20] (216.53, 56.71) circle (  2.13);

\path[fill=fillColor,fill opacity=0.20] (215.53, 61.28) circle (  2.13);

\path[fill=fillColor,fill opacity=0.20] (215.53, 79.53) circle (  2.13);

\path[fill=fillColor,fill opacity=0.20] (209.51, 88.14) circle (  2.13);

\path[fill=fillColor,fill opacity=0.20] (207.50, 74.11) circle (  2.13);

\path[fill=fillColor,fill opacity=0.20] (206.50, 61.53) circle (  2.13);

\path[fill=fillColor,fill opacity=0.20] (207.50, 61.88) circle (  2.13);

\path[fill=fillColor,fill opacity=0.20] (213.52, 60.24) circle (  2.13);

\path[fill=fillColor,fill opacity=0.20] (210.51, 57.31) circle (  2.13);

\path[fill=fillColor,fill opacity=0.20] (210.51, 63.43) circle (  2.13);

\path[fill=fillColor,fill opacity=0.20] (213.52, 69.46) circle (  2.13);

\path[fill=fillColor,fill opacity=0.20] (216.53, 70.92) circle (  2.13);

\path[fill=fillColor,fill opacity=0.20] (226.56, 78.33) circle (  2.13);

\path[fill=fillColor,fill opacity=0.20] (264.69, 90.30) circle (  2.13);

\path[fill=fillColor,fill opacity=0.20] (232.58, 56.28) circle (  2.13);

\path[fill=fillColor,fill opacity=0.20] (212.52, 54.56) circle (  2.13);

\path[fill=fillColor,fill opacity=0.20] (205.50, 59.04) circle (  2.13);

\path[fill=fillColor,fill opacity=0.20] (210.51, 58.95) circle (  2.13);

\path[fill=fillColor,fill opacity=0.20] (214.53, 60.41) circle (  2.13);

\path[fill=fillColor,fill opacity=0.20] (213.52, 65.32) circle (  2.13);

\path[fill=fillColor,fill opacity=0.20] (214.53, 77.47) circle (  2.13);

\path[fill=fillColor,fill opacity=0.20] (200.48, 95.03) circle (  2.13);

\path[fill=fillColor,fill opacity=0.20] (202.49, 79.62) circle (  2.13);

\path[fill=fillColor,fill opacity=0.20] (202.49, 68.85) circle (  2.13);

\path[fill=fillColor,fill opacity=0.20] (204.49, 65.50) circle (  2.13);

\path[fill=fillColor,fill opacity=0.20] (204.49, 59.64) circle (  2.13);

\path[fill=fillColor,fill opacity=0.20] (208.51, 55.94) circle (  2.13);

\path[fill=fillColor,fill opacity=0.20] (211.52, 59.38) circle (  2.13);

\path[fill=fillColor,fill opacity=0.20] (217.53, 61.19) circle (  2.13);

\path[fill=fillColor,fill opacity=0.20] (217.53, 61.02) circle (  2.13);

\path[fill=fillColor,fill opacity=0.20] (225.56, 74.11) circle (  2.13);

\path[fill=fillColor,fill opacity=0.20] (244.62, 77.21) circle (  2.13);

\path[fill=fillColor,fill opacity=0.20] (218.54, 65.84) circle (  2.13);

\path[fill=fillColor,fill opacity=0.20] (212.52, 60.93) circle (  2.13);

\path[fill=fillColor,fill opacity=0.20] (210.51, 58.18) circle (  2.13);

\path[fill=fillColor,fill opacity=0.20] (211.52, 56.37) circle (  2.13);

\path[fill=fillColor,fill opacity=0.20] (211.52, 59.30) circle (  2.13);

\path[fill=fillColor,fill opacity=0.20] (212.52, 68.85) circle (  2.13);

\path[fill=fillColor,fill opacity=0.20] (215.53, 76.17) circle (  2.13);

\path[fill=fillColor,fill opacity=0.20] (221.55,102.52) circle (  2.13);

\path[fill=fillColor,fill opacity=0.20] (201.48, 91.76) circle (  2.13);

\path[fill=fillColor,fill opacity=0.20] (200.48, 83.84) circle (  2.13);

\path[fill=fillColor,fill opacity=0.20] (202.49, 70.83) circle (  2.13);

\path[fill=fillColor,fill opacity=0.20] (202.49, 65.32) circle (  2.13);

\path[fill=fillColor,fill opacity=0.20] (203.49, 62.74) circle (  2.13);

\path[fill=fillColor,fill opacity=0.20] (201.48, 59.64) circle (  2.13);

\path[fill=fillColor,fill opacity=0.20] (202.49, 58.78) circle (  2.13);

\path[fill=fillColor,fill opacity=0.20] (204.49, 58.95) circle (  2.13);

\path[fill=fillColor,fill opacity=0.20] (211.52, 58.09) circle (  2.13);

\path[fill=fillColor,fill opacity=0.20] (219.54, 58.95) circle (  2.13);

\path[fill=fillColor,fill opacity=0.20] (241.61, 69.80) circle (  2.13);

\path[fill=fillColor,fill opacity=0.20] (266.69, 80.57) circle (  2.13);

\path[fill=fillColor,fill opacity=0.20] (251.64, 68.16) circle (  2.13);

\path[fill=fillColor,fill opacity=0.20] (222.55, 59.98) circle (  2.13);

\path[fill=fillColor,fill opacity=0.20] (212.52, 57.83) circle (  2.13);

\path[fill=fillColor,fill opacity=0.20] (212.52, 61.71) circle (  2.13);

\path[fill=fillColor,fill opacity=0.20] (212.52, 68.94) circle (  2.13);

\path[fill=fillColor,fill opacity=0.20] (213.52, 70.66) circle (  2.13);

\path[fill=fillColor,fill opacity=0.20] (218.54, 75.05) circle (  2.13);

\path[fill=fillColor,fill opacity=0.20] (200.48, 85.39) circle (  2.13);

\path[fill=fillColor,fill opacity=0.20] (197.47, 75.05) circle (  2.13);

\path[fill=fillColor,fill opacity=0.20] (199.48, 66.53) circle (  2.13);

\path[fill=fillColor,fill opacity=0.20] (203.49, 56.37) circle (  2.13);

\path[fill=fillColor,fill opacity=0.20] (203.49, 57.40) circle (  2.13);

\path[fill=fillColor,fill opacity=0.20] (203.49, 62.31) circle (  2.13);

\path[fill=fillColor,fill opacity=0.20] (207.50, 59.81) circle (  2.13);

\path[fill=fillColor,fill opacity=0.20] (206.50, 59.04) circle (  2.13);

\path[fill=fillColor,fill opacity=0.20] (207.50, 63.08) circle (  2.13);

\path[fill=fillColor,fill opacity=0.20] (209.51, 66.96) circle (  2.13);

\path[fill=fillColor,fill opacity=0.20] (215.53, 68.94) circle (  2.13);

\path[fill=fillColor,fill opacity=0.20] (251.64, 73.50) circle (  2.13);

\path[fill=fillColor,fill opacity=0.20] (266.69, 87.80) circle (  2.13);

\path[fill=fillColor,fill opacity=0.20] (239.61, 73.68) circle (  2.13);

\path[fill=fillColor,fill opacity=0.20] (226.56, 64.38) circle (  2.13);

\path[fill=fillColor,fill opacity=0.20] (216.53, 63.86) circle (  2.13);

\path[fill=fillColor,fill opacity=0.20] (212.52, 65.06) circle (  2.13);

\path[fill=fillColor,fill opacity=0.20] (213.52, 69.37) circle (  2.13);

\path[fill=fillColor,fill opacity=0.20] (212.52, 70.15) circle (  2.13);

\path[fill=fillColor,fill opacity=0.20] (212.52, 75.74) circle (  2.13);

\path[fill=fillColor,fill opacity=0.20] (196.47, 83.92) circle (  2.13);

\path[fill=fillColor,fill opacity=0.20] (194.46, 72.21) circle (  2.13);

\path[fill=fillColor,fill opacity=0.20] (189.44, 67.30) circle (  2.13);

\path[fill=fillColor,fill opacity=0.20] (198.47, 62.31) circle (  2.13);

\path[fill=fillColor,fill opacity=0.20] (202.49, 55.51) circle (  2.13);

\path[fill=fillColor,fill opacity=0.20] (200.48, 60.33) circle (  2.13);

\path[fill=fillColor,fill opacity=0.20] (204.49, 66.79) circle (  2.13);

\path[fill=fillColor,fill opacity=0.20] (210.51, 62.05) circle (  2.13);

\path[fill=fillColor,fill opacity=0.20] (208.51, 60.24) circle (  2.13);

\path[fill=fillColor,fill opacity=0.20] (206.50, 65.84) circle (  2.13);

\path[fill=fillColor,fill opacity=0.20] (212.52, 71.35) circle (  2.13);

\path[fill=fillColor,fill opacity=0.20] (247.63, 78.76) circle (  2.13);

\path[fill=fillColor,fill opacity=0.20] (227.57, 67.82) circle (  2.13);

\path[fill=fillColor,fill opacity=0.20] (207.50, 64.20) circle (  2.13);

\path[fill=fillColor,fill opacity=0.20] (208.51, 66.70) circle (  2.13);

\path[fill=fillColor,fill opacity=0.20] (211.52, 72.21) circle (  2.13);

\path[fill=fillColor,fill opacity=0.20] (217.53, 71.52) circle (  2.13);

\path[fill=fillColor,fill opacity=0.20] (216.53, 78.67) circle (  2.13);

\path[fill=fillColor,fill opacity=0.20] (195.46, 87.80) circle (  2.13);

\path[fill=fillColor,fill opacity=0.20] (192.45, 72.90) circle (  2.13);

\path[fill=fillColor,fill opacity=0.20] (192.45, 69.80) circle (  2.13);

\path[fill=fillColor,fill opacity=0.20] (187.94, 68.16) circle (  2.13);

\path[fill=fillColor,fill opacity=0.20] (198.47, 62.83) circle (  2.13);

\path[fill=fillColor,fill opacity=0.20] (201.48, 60.50) circle (  2.13);

\path[fill=fillColor,fill opacity=0.20] (199.48, 64.03) circle (  2.13);

\path[fill=fillColor,fill opacity=0.20] (205.50, 64.12) circle (  2.13);

\path[fill=fillColor,fill opacity=0.20] (211.52, 62.57) circle (  2.13);

\path[fill=fillColor,fill opacity=0.20] (213.52, 66.96) circle (  2.13);

\path[fill=fillColor,fill opacity=0.20] (215.53, 69.03) circle (  2.13);

\path[fill=fillColor,fill opacity=0.20] (230.58, 70.06) circle (  2.13);

\path[fill=fillColor,fill opacity=0.20] (246.63, 80.31) circle (  2.13);

\path[fill=fillColor,fill opacity=0.20] (235.59, 67.73) circle (  2.13);

\path[fill=fillColor,fill opacity=0.20] (215.53, 63.08) circle (  2.13);

\path[fill=fillColor,fill opacity=0.20] (217.53, 68.94) circle (  2.13);

\path[fill=fillColor,fill opacity=0.20] (210.51, 70.92) circle (  2.13);

\path[fill=fillColor,fill opacity=0.20] (210.51, 66.44) circle (  2.13);

\path[fill=fillColor,fill opacity=0.20] (211.52, 77.29) circle (  2.13);

\path[fill=fillColor,fill opacity=0.20] (193.46,103.73) circle (  2.13);

\path[fill=fillColor,fill opacity=0.20] (197.47, 80.48) circle (  2.13);

\path[fill=fillColor,fill opacity=0.20] (195.46, 66.27) circle (  2.13);

\path[fill=fillColor,fill opacity=0.20] (191.45, 63.60) circle (  2.13);

\path[fill=fillColor,fill opacity=0.20] (192.45, 64.12) circle (  2.13);

\path[fill=fillColor,fill opacity=0.20] (196.47, 63.00) circle (  2.13);

\path[fill=fillColor,fill opacity=0.20] (201.48, 58.00) circle (  2.13);

\path[fill=fillColor,fill opacity=0.20] (199.48, 60.67) circle (  2.13);

\path[fill=fillColor,fill opacity=0.20] (201.48, 64.89) circle (  2.13);

\path[fill=fillColor,fill opacity=0.20] (207.50, 58.09) circle (  2.13);

\path[fill=fillColor,fill opacity=0.20] (220.54, 58.09) circle (  2.13);

\path[fill=fillColor,fill opacity=0.20] (222.55, 69.89) circle (  2.13);

\path[fill=fillColor,fill opacity=0.20] (255.66, 74.97) circle (  2.13);

\path[fill=fillColor,fill opacity=0.20] (240.61, 74.28) circle (  2.13);

\path[fill=fillColor,fill opacity=0.20] (240.61, 65.15) circle (  2.13);

\path[fill=fillColor,fill opacity=0.20] (222.55, 65.24) circle (  2.13);

\path[fill=fillColor,fill opacity=0.20] (212.52, 66.10) circle (  2.13);

\path[fill=fillColor,fill opacity=0.20] (211.52, 64.63) circle (  2.13);

\path[fill=fillColor,fill opacity=0.20] (207.50, 64.98) circle (  2.13);

\path[fill=fillColor,fill opacity=0.20] (204.49, 69.20) circle (  2.13);

\path[fill=fillColor,fill opacity=0.20] (211.52, 87.97) circle (  2.13);

\path[fill=fillColor,fill opacity=0.20] (193.46, 99.17) circle (  2.13);

\path[fill=fillColor,fill opacity=0.20] (191.45, 82.37) circle (  2.13);

\path[fill=fillColor,fill opacity=0.20] (191.45, 73.76) circle (  2.13);

\path[fill=fillColor,fill opacity=0.20] (195.46, 58.61) circle (  2.13);

\path[fill=fillColor,fill opacity=0.20] (195.46, 53.96) circle (  2.13);

\path[fill=fillColor,fill opacity=0.20] (192.45, 59.55) circle (  2.13);

\path[fill=fillColor,fill opacity=0.20] (194.46, 57.06) circle (  2.13);

\path[fill=fillColor,fill opacity=0.20] (187.64, 56.11) circle (  2.13);

\path[fill=fillColor,fill opacity=0.20] (199.48, 55.68) circle (  2.13);

\path[fill=fillColor,fill opacity=0.20] (200.48, 59.47) circle (  2.13);

\path[fill=fillColor,fill opacity=0.20] (209.51, 66.10) circle (  2.13);

\path[fill=fillColor,fill opacity=0.20] (218.54, 59.98) circle (  2.13);

\path[fill=fillColor,fill opacity=0.20] (236.60, 57.23) circle (  2.13);

\path[fill=fillColor,fill opacity=0.20] (258.67, 71.44) circle (  2.13);

\path[fill=fillColor,fill opacity=0.20] (238.60, 83.15) circle (  2.13);

\path[fill=fillColor,fill opacity=0.20] (251.64, 66.10) circle (  2.13);

\path[fill=fillColor,fill opacity=0.20] (224.56, 58.69) circle (  2.13);

\path[fill=fillColor,fill opacity=0.20] (212.52, 57.92) circle (  2.13);

\path[fill=fillColor,fill opacity=0.20] (210.51, 60.76) circle (  2.13);

\path[fill=fillColor,fill opacity=0.20] (208.51, 60.85) circle (  2.13);

\path[fill=fillColor,fill opacity=0.20] (207.50, 65.67) circle (  2.13);

\path[fill=fillColor,fill opacity=0.20] (212.52, 81.51) circle (  2.13);

\path[fill=fillColor,fill opacity=0.20] (189.44, 82.12) circle (  2.13);

\path[fill=fillColor,fill opacity=0.20] (187.54, 69.20) circle (  2.13);

\path[fill=fillColor,fill opacity=0.20] (184.33, 65.67) circle (  2.13);

\path[fill=fillColor,fill opacity=0.20] (195.46, 63.60) circle (  2.13);

\path[fill=fillColor,fill opacity=0.20] (194.46, 58.52) circle (  2.13);

\path[fill=fillColor,fill opacity=0.20] (196.47, 58.61) circle (  2.13);

\path[fill=fillColor,fill opacity=0.20] (195.46, 59.64) circle (  2.13);

\path[fill=fillColor,fill opacity=0.20] (194.46, 60.59) circle (  2.13);

\path[fill=fillColor,fill opacity=0.20] (200.48, 62.74) circle (  2.13);

\path[fill=fillColor,fill opacity=0.20] (207.50, 58.86) circle (  2.13);

\path[fill=fillColor,fill opacity=0.20] (219.54, 61.53) circle (  2.13);

\path[fill=fillColor,fill opacity=0.20] (227.57, 69.71) circle (  2.13);

\path[fill=fillColor,fill opacity=0.20] (242.62, 69.97) circle (  2.13);

\path[fill=fillColor,fill opacity=0.20] (248.64, 71.70) circle (  2.13);

\path[fill=fillColor,fill opacity=0.20] (233.59, 85.13) circle (  2.13);

\path[fill=fillColor,fill opacity=0.20] (255.66, 60.41) circle (  2.13);

\path[fill=fillColor,fill opacity=0.20] (216.53, 57.14) circle (  2.13);

\path[fill=fillColor,fill opacity=0.20] (209.51, 63.60) circle (  2.13);

\path[fill=fillColor,fill opacity=0.20] (207.50, 71.70) circle (  2.13);

\path[fill=fillColor,fill opacity=0.20] (209.51, 72.13) circle (  2.13);

\path[fill=fillColor,fill opacity=0.20] (211.52, 69.46) circle (  2.13);

\path[fill=fillColor,fill opacity=0.20] (207.50, 73.25) circle (  2.13);

\path[fill=fillColor,fill opacity=0.20] (210.51, 91.33) circle (  2.13);

\path[fill=fillColor,fill opacity=0.20] (190.45, 87.63) circle (  2.13);

\path[fill=fillColor,fill opacity=0.20] (191.45, 66.61) circle (  2.13);

\path[fill=fillColor,fill opacity=0.20] (189.44, 61.10) circle (  2.13);

\path[fill=fillColor,fill opacity=0.20] (177.51, 62.65) circle (  2.13);

\path[fill=fillColor,fill opacity=0.20] (194.46, 60.67) circle (  2.13);

\path[fill=fillColor,fill opacity=0.20] (195.46, 59.04) circle (  2.13);

\path[fill=fillColor,fill opacity=0.20] (191.45, 60.33) circle (  2.13);

\path[fill=fillColor,fill opacity=0.20] (192.45, 60.85) circle (  2.13);

\path[fill=fillColor,fill opacity=0.20] (195.46, 65.50) circle (  2.13);

\path[fill=fillColor,fill opacity=0.20] (204.49, 69.54) circle (  2.13);

\path[fill=fillColor,fill opacity=0.20] (220.54, 68.85) circle (  2.13);

\path[fill=fillColor,fill opacity=0.20] (232.58, 68.85) circle (  2.13);

\path[fill=fillColor,fill opacity=0.20] (258.67, 76.95) circle (  2.13);

\path[fill=fillColor,fill opacity=0.20] (243.62, 83.58) circle (  2.13);

\path[fill=fillColor,fill opacity=0.20] (234.59, 82.55) circle (  2.13);

\path[fill=fillColor,fill opacity=0.20] (222.55, 70.32) circle (  2.13);

\path[fill=fillColor,fill opacity=0.20] (216.53, 70.83) circle (  2.13);

\path[fill=fillColor,fill opacity=0.20] (211.52, 68.51) circle (  2.13);

\path[fill=fillColor,fill opacity=0.20] (209.51, 67.05) circle (  2.13);

\path[fill=fillColor,fill opacity=0.20] (209.51, 65.15) circle (  2.13);

\path[fill=fillColor,fill opacity=0.20] (207.50, 65.15) circle (  2.13);

\path[fill=fillColor,fill opacity=0.20] (212.52, 76.78) circle (  2.13);

\path[fill=fillColor,fill opacity=0.20] (216.53, 95.12) circle (  2.13);

\path[fill=fillColor,fill opacity=0.20] (190.45, 89.18) circle (  2.13);

\path[fill=fillColor,fill opacity=0.20] (188.04, 75.92) circle (  2.13);

\path[fill=fillColor,fill opacity=0.20] (191.45, 69.54) circle (  2.13);

\path[fill=fillColor,fill opacity=0.20] (195.46, 61.96) circle (  2.13);

\path[fill=fillColor,fill opacity=0.20] (195.46, 58.18) circle (  2.13);

\path[fill=fillColor,fill opacity=0.20] (192.45, 60.50) circle (  2.13);

\path[fill=fillColor,fill opacity=0.20] (196.47, 58.61) circle (  2.13);

\path[fill=fillColor,fill opacity=0.20] (203.49, 53.53) circle (  2.13);

\path[fill=fillColor,fill opacity=0.20] (208.51, 55.76) circle (  2.13);

\path[fill=fillColor,fill opacity=0.20] (209.51, 63.60) circle (  2.13);

\path[fill=fillColor,fill opacity=0.20] (213.52, 71.70) circle (  2.13);

\path[fill=fillColor,fill opacity=0.20] (209.51, 76.26) circle (  2.13);

\path[fill=fillColor,fill opacity=0.20] (233.59, 77.55) circle (  2.13);

\path[fill=fillColor,fill opacity=0.20] (247.63, 83.67) circle (  2.13);

\path[fill=fillColor,fill opacity=0.20] (217.53, 63.51) circle (  2.13);

\path[fill=fillColor,fill opacity=0.20] (223.55, 56.45) circle (  2.13);

\path[fill=fillColor,fill opacity=0.20] (212.52, 59.30) circle (  2.13);

\path[fill=fillColor,fill opacity=0.20] (208.51, 68.08) circle (  2.13);

\path[fill=fillColor,fill opacity=0.20] (206.50, 71.27) circle (  2.13);

\path[fill=fillColor,fill opacity=0.20] (209.51, 71.44) circle (  2.13);

\path[fill=fillColor,fill opacity=0.20] (214.53, 74.37) circle (  2.13);

\path[fill=fillColor,fill opacity=0.20] (213.52, 79.45) circle (  2.13);

\path[fill=fillColor,fill opacity=0.20] (210.51, 88.83) circle (  2.13);

\path[fill=fillColor,fill opacity=0.20] (183.22,106.83) circle (  2.13);

\path[fill=fillColor,fill opacity=0.20] (193.46, 82.80) circle (  2.13);

\path[fill=fillColor,fill opacity=0.20] (190.45, 68.25) circle (  2.13);

\path[fill=fillColor,fill opacity=0.20] (187.44, 66.53) circle (  2.13);

\path[fill=fillColor,fill opacity=0.20] (193.46, 69.37) circle (  2.13);

\path[fill=fillColor,fill opacity=0.20] (201.48, 67.48) circle (  2.13);

\path[fill=fillColor,fill opacity=0.20] (207.50, 61.36) circle (  2.13);

\path[fill=fillColor,fill opacity=0.20] (196.47, 59.73) circle (  2.13);

\path[fill=fillColor,fill opacity=0.20] (211.52, 58.35) circle (  2.13);

\path[fill=fillColor,fill opacity=0.20] (226.56, 57.23) circle (  2.13);

\path[fill=fillColor,fill opacity=0.20] (243.62, 64.98) circle (  2.13);

\path[fill=fillColor,fill opacity=0.20] (249.64, 86.08) circle (  2.13);

\path[fill=fillColor,fill opacity=0.20] (230.58, 89.35) circle (  2.13);

\path[fill=fillColor,fill opacity=0.20] (259.67, 92.45) circle (  2.13);

\path[fill=fillColor,fill opacity=0.20] (269.70, 74.54) circle (  2.13);

\path[fill=fillColor,fill opacity=0.20] (238.60, 62.83) circle (  2.13);

\path[fill=fillColor,fill opacity=0.20] (224.56, 58.61) circle (  2.13);

\path[fill=fillColor,fill opacity=0.20] (211.52, 66.61) circle (  2.13);

\path[fill=fillColor,fill opacity=0.20] (207.50, 73.93) circle (  2.13);

\path[fill=fillColor,fill opacity=0.20] (214.53, 65.84) circle (  2.13);

\path[fill=fillColor,fill opacity=0.20] (214.53, 65.58) circle (  2.13);

\path[fill=fillColor,fill opacity=0.20] (214.53, 73.76) circle (  2.13);

\path[fill=fillColor,fill opacity=0.20] (212.52, 77.38) circle (  2.13);

\path[fill=fillColor,fill opacity=0.20] (210.51, 80.22) circle (  2.13);

\path[fill=fillColor,fill opacity=0.20] (207.50, 90.64) circle (  2.13);

\path[fill=fillColor,fill opacity=0.20] (197.47, 99.17) circle (  2.13);

\path[fill=fillColor,fill opacity=0.20] (196.47, 83.67) circle (  2.13);

\path[fill=fillColor,fill opacity=0.20] (197.47, 81.60) circle (  2.13);

\path[fill=fillColor,fill opacity=0.20] (198.47, 77.38) circle (  2.13);

\path[fill=fillColor,fill opacity=0.20] (198.47, 68.94) circle (  2.13);

\path[fill=fillColor,fill opacity=0.20] (199.48, 65.24) circle (  2.13);

\path[fill=fillColor,fill opacity=0.20] (214.53, 68.16) circle (  2.13);

\path[fill=fillColor,fill opacity=0.20] (213.52, 69.20) circle (  2.13);

\path[fill=fillColor,fill opacity=0.20] (236.60, 66.87) circle (  2.13);

\path[fill=fillColor,fill opacity=0.20] (245.63, 68.51) circle (  2.13);

\path[fill=fillColor,fill opacity=0.20] (237.60, 74.28) circle (  2.13);

\path[fill=fillColor,fill opacity=0.20] (237.60, 79.88) circle (  2.13);

\path[fill=fillColor,fill opacity=0.20] (219.54, 87.28) circle (  2.13);

\path[fill=fillColor,fill opacity=0.20] (265.69, 84.35) circle (  2.13);

\path[fill=fillColor,fill opacity=0.20] (233.59, 64.46) circle (  2.13);

\path[fill=fillColor,fill opacity=0.20] (221.55, 61.96) circle (  2.13);

\path[fill=fillColor,fill opacity=0.20] (212.52, 54.13) circle (  2.13);

\path[fill=fillColor,fill opacity=0.20] (210.51, 56.80) circle (  2.13);

\path[fill=fillColor,fill opacity=0.20] (213.52, 69.63) circle (  2.13);

\path[fill=fillColor,fill opacity=0.20] (212.52, 74.37) circle (  2.13);

\path[fill=fillColor,fill opacity=0.20] (211.52, 72.64) circle (  2.13);

\path[fill=fillColor,fill opacity=0.20] (217.53, 74.62) circle (  2.13);

\path[fill=fillColor,fill opacity=0.20] (218.54, 77.21) circle (  2.13);

\path[fill=fillColor,fill opacity=0.20] (215.53, 85.99) circle (  2.13);

\path[fill=fillColor,fill opacity=0.20] (194.46, 93.40) circle (  2.13);

\path[fill=fillColor,fill opacity=0.20] (196.47, 83.84) circle (  2.13);

\path[fill=fillColor,fill opacity=0.20] (199.48, 73.68) circle (  2.13);

\path[fill=fillColor,fill opacity=0.20] (204.49, 67.82) circle (  2.13);

\path[fill=fillColor,fill opacity=0.20] (206.50, 64.72) circle (  2.13);

\path[fill=fillColor,fill opacity=0.20] (208.51, 68.94) circle (  2.13);

\path[fill=fillColor,fill opacity=0.20] (219.54, 74.97) circle (  2.13);

\path[fill=fillColor,fill opacity=0.20] (224.56, 73.07) circle (  2.13);

\path[fill=fillColor,fill opacity=0.20] (227.57, 70.83) circle (  2.13);

\path[fill=fillColor,fill opacity=0.20] (235.59, 74.54) circle (  2.13);

\path[fill=fillColor,fill opacity=0.20] (242.62, 75.83) circle (  2.13);

\path[fill=fillColor,fill opacity=0.20] (231.58, 75.92) circle (  2.13);

\path[fill=fillColor,fill opacity=0.20] (243.62, 84.61) circle (  2.13);

\path[fill=fillColor,fill opacity=0.20] (264.69, 90.04) circle (  2.13);

\path[fill=fillColor,fill opacity=0.20] (257.66, 79.36) circle (  2.13);

\path[fill=fillColor,fill opacity=0.20] (255.66, 69.37) circle (  2.13);

\path[fill=fillColor,fill opacity=0.20] (209.51, 61.96) circle (  2.13);

\path[fill=fillColor,fill opacity=0.20] (224.56, 62.22) circle (  2.13);

\path[fill=fillColor,fill opacity=0.20] (214.53, 66.79) circle (  2.13);

\path[fill=fillColor,fill opacity=0.20] (210.51, 65.24) circle (  2.13);

\path[fill=fillColor,fill opacity=0.20] (213.52, 65.93) circle (  2.13);

\path[fill=fillColor,fill opacity=0.20] (216.53, 72.04) circle (  2.13);

\path[fill=fillColor,fill opacity=0.20] (218.54, 70.58) circle (  2.13);

\path[fill=fillColor,fill opacity=0.20] (218.54, 65.24) circle (  2.13);

\path[fill=fillColor,fill opacity=0.20] (219.54, 70.58) circle (  2.13);

\path[fill=fillColor,fill opacity=0.20] (223.55, 78.24) circle (  2.13);

\path[fill=fillColor,fill opacity=0.20] (227.57, 79.36) circle (  2.13);

\path[fill=fillColor,fill opacity=0.20] (216.53, 82.98) circle (  2.13);

\path[fill=fillColor,fill opacity=0.20] (208.51, 89.00) circle (  2.13);

\path[fill=fillColor,fill opacity=0.20] (223.55, 89.26) circle (  2.13);

\path[fill=fillColor,fill opacity=0.20] (213.52, 86.68) circle (  2.13);

\path[fill=fillColor,fill opacity=0.20] (216.53, 87.28) circle (  2.13);

\path[fill=fillColor,fill opacity=0.20] (216.53, 89.95) circle (  2.13);

\path[fill=fillColor,fill opacity=0.20] (217.53, 92.54) circle (  2.13);

\path[fill=fillColor,fill opacity=0.20] (221.55, 91.85) circle (  2.13);

\path[fill=fillColor,fill opacity=0.20] (204.49, 86.42) circle (  2.13);

\path[fill=fillColor,fill opacity=0.20] (203.49, 83.67) circle (  2.13);

\path[fill=fillColor,fill opacity=0.20] (198.47, 81.43) circle (  2.13);

\path[fill=fillColor,fill opacity=0.20] (201.48, 80.05) circle (  2.13);

\path[fill=fillColor,fill opacity=0.20] (201.48, 73.76) circle (  2.13);

\path[fill=fillColor,fill opacity=0.20] (198.47, 65.24) circle (  2.13);

\path[fill=fillColor,fill opacity=0.20] (199.48, 66.27) circle (  2.13);

\path[fill=fillColor,fill opacity=0.20] (194.46, 71.95) circle (  2.13);

\path[fill=fillColor,fill opacity=0.20] (213.52, 70.32) circle (  2.13);

\path[fill=fillColor,fill opacity=0.20] (217.53, 68.25) circle (  2.13);

\path[fill=fillColor,fill opacity=0.20] (231.58, 72.99) circle (  2.13);

\path[fill=fillColor,fill opacity=0.20] (233.59, 76.60) circle (  2.13);

\path[fill=fillColor,fill opacity=0.20] (223.55, 80.39) circle (  2.13);

\path[fill=fillColor,fill opacity=0.20] (271.71, 84.61) circle (  2.13);

\path[fill=fillColor,fill opacity=0.20] (247.63, 88.66) circle (  2.13);

\path[fill=fillColor,fill opacity=0.20] (233.59, 90.64) circle (  2.13);

\path[fill=fillColor,fill opacity=0.20] (231.58, 89.18) circle (  2.13);

\path[fill=fillColor,fill opacity=0.20] (244.62, 94.95) circle (  2.13);

\path[fill=fillColor,fill opacity=0.20] (251.64, 85.39) circle (  2.13);

\path[fill=fillColor,fill opacity=0.20] (237.60, 75.48) circle (  2.13);

\path[fill=fillColor,fill opacity=0.20] (217.53, 69.20) circle (  2.13);

\path[fill=fillColor,fill opacity=0.20] (218.54, 65.15) circle (  2.13);

\path[fill=fillColor,fill opacity=0.20] (218.54, 66.79) circle (  2.13);

\path[fill=fillColor,fill opacity=0.20] (219.54, 62.65) circle (  2.13);

\path[fill=fillColor,fill opacity=0.20] (217.53, 58.61) circle (  2.13);

\path[fill=fillColor,fill opacity=0.20] (210.51, 63.69) circle (  2.13);

\path[fill=fillColor,fill opacity=0.20] (218.54, 66.61) circle (  2.13);

\path[fill=fillColor,fill opacity=0.20] (220.54, 66.01) circle (  2.13);

\path[fill=fillColor,fill opacity=0.20] (218.54, 66.70) circle (  2.13);

\path[fill=fillColor,fill opacity=0.20] (215.53, 70.40) circle (  2.13);

\path[fill=fillColor,fill opacity=0.20] (224.56, 72.64) circle (  2.13);

\path[fill=fillColor,fill opacity=0.20] (219.54, 72.13) circle (  2.13);

\path[fill=fillColor,fill opacity=0.20] (213.52, 72.13) circle (  2.13);

\path[fill=fillColor,fill opacity=0.20] (215.53, 72.04) circle (  2.13);

\path[fill=fillColor,fill opacity=0.20] (209.51, 70.75) circle (  2.13);

\path[fill=fillColor,fill opacity=0.20] (210.51, 69.80) circle (  2.13);

\path[fill=fillColor,fill opacity=0.20] (213.52, 67.99) circle (  2.13);

\path[fill=fillColor,fill opacity=0.20] (214.53, 67.39) circle (  2.13);

\path[fill=fillColor,fill opacity=0.20] (204.49, 71.18) circle (  2.13);

\path[fill=fillColor,fill opacity=0.20] (204.49, 74.54) circle (  2.13);

\path[fill=fillColor,fill opacity=0.20] (202.49, 75.31) circle (  2.13);

\path[fill=fillColor,fill opacity=0.20] (201.48, 70.23) circle (  2.13);

\path[fill=fillColor,fill opacity=0.20] (201.48, 62.65) circle (  2.13);

\path[fill=fillColor,fill opacity=0.20] (209.51, 59.21) circle (  2.13);

\path[fill=fillColor,fill opacity=0.20] (215.53, 58.95) circle (  2.13);

\path[fill=fillColor,fill opacity=0.20] (218.54, 65.41) circle (  2.13);

\path[fill=fillColor,fill opacity=0.20] (237.60, 75.31) circle (  2.13);

\path[fill=fillColor,fill opacity=0.20] (244.62, 79.45) circle (  2.13);

\path[fill=fillColor,fill opacity=0.20] (247.63, 84.10) circle (  2.13);

\path[fill=fillColor,fill opacity=0.20] (262.68, 95.98) circle (  2.13);

\path[fill=fillColor,fill opacity=0.20] (249.64,106.57) circle (  2.13);

\path[fill=fillColor,fill opacity=0.20] (243.62, 92.62) circle (  2.13);

\path[fill=fillColor,fill opacity=0.20] (246.63, 91.33) circle (  2.13);

\path[fill=fillColor,fill opacity=0.20] (246.63, 85.13) circle (  2.13);

\path[fill=fillColor,fill opacity=0.20] (263.68, 73.42) circle (  2.13);

\path[fill=fillColor,fill opacity=0.20] (233.59, 63.26) circle (  2.13);

\path[fill=fillColor,fill opacity=0.20] (235.59, 56.71) circle (  2.13);

\path[fill=fillColor,fill opacity=0.20] (221.55, 57.06) circle (  2.13);

\path[fill=fillColor,fill opacity=0.20] (218.54, 62.74) circle (  2.13);

\path[fill=fillColor,fill opacity=0.20] (215.53, 63.51) circle (  2.13);

\path[fill=fillColor,fill opacity=0.20] (213.52, 61.02) circle (  2.13);

\path[fill=fillColor,fill opacity=0.20] (209.51, 61.19) circle (  2.13);

\path[fill=fillColor,fill opacity=0.20] (215.53, 61.45) circle (  2.13);

\path[fill=fillColor,fill opacity=0.20] (214.53, 63.08) circle (  2.13);

\path[fill=fillColor,fill opacity=0.20] (208.51, 67.56) circle (  2.13);

\path[fill=fillColor,fill opacity=0.20] (208.51, 72.13) circle (  2.13);

\path[fill=fillColor,fill opacity=0.20] (209.51, 71.01) circle (  2.13);

\path[fill=fillColor,fill opacity=0.20] (207.50, 65.93) circle (  2.13);

\path[fill=fillColor,fill opacity=0.20] (208.51, 64.29) circle (  2.13);

\path[fill=fillColor,fill opacity=0.20] (216.53, 63.69) circle (  2.13);

\path[fill=fillColor,fill opacity=0.20] (212.52, 63.95) circle (  2.13);

\path[fill=fillColor,fill opacity=0.20] (212.52, 68.42) circle (  2.13);

\path[fill=fillColor,fill opacity=0.20] (220.54, 72.38) circle (  2.13);

\path[fill=fillColor,fill opacity=0.20] (216.53, 72.90) circle (  2.13);

\path[fill=fillColor,fill opacity=0.20] (219.54, 69.54) circle (  2.13);

\path[fill=fillColor,fill opacity=0.20] (230.58, 63.00) circle (  2.13);

\path[fill=fillColor,fill opacity=0.20] (245.63, 62.74) circle (  2.13);

\path[fill=fillColor,fill opacity=0.20] (246.63, 72.13) circle (  2.13);

\path[fill=fillColor,fill opacity=0.20] (241.61, 89.87) circle (  2.13);

\path[fill=fillColor,fill opacity=0.20] (243.62, 84.87) circle (  2.13);

\path[fill=fillColor,fill opacity=0.20] (243.62, 76.78) circle (  2.13);

\path[fill=fillColor,fill opacity=0.20] (258.67, 75.48) circle (  2.13);

\path[fill=fillColor,fill opacity=0.20] (224.56, 70.06) circle (  2.13);

\path[fill=fillColor,fill opacity=0.20] (227.57, 63.51) circle (  2.13);

\path[fill=fillColor,fill opacity=0.20] (220.54, 63.00) circle (  2.13);

\path[fill=fillColor,fill opacity=0.20] (217.53, 64.89) circle (  2.13);

\path[fill=fillColor,fill opacity=0.20] (215.53, 61.62) circle (  2.13);

\path[fill=fillColor,fill opacity=0.20] (212.52, 55.42) circle (  2.13);

\path[fill=fillColor,fill opacity=0.20] (207.50, 58.61) circle (  2.13);

\path[fill=fillColor,fill opacity=0.20] (208.51, 67.48) circle (  2.13);

\path[fill=fillColor,fill opacity=0.20] (217.53, 67.13) circle (  2.13);

\path[fill=fillColor,fill opacity=0.20] (221.55, 63.34) circle (  2.13);

\path[fill=fillColor,fill opacity=0.20] (225.56, 63.08) circle (  2.13);

\path[fill=fillColor,fill opacity=0.20] (241.61, 65.06) circle (  2.13);

\path[fill=fillColor,fill opacity=0.20] (232.58, 63.86) circle (  2.13);

\path[fill=fillColor,fill opacity=0.20] (261.68, 63.00) circle (  2.13);

\path[fill=fillColor,fill opacity=0.20] (268.70, 62.83) circle (  2.13);

\path[fill=fillColor,fill opacity=0.20] (243.62, 67.39) circle (  2.13);

\path[fill=fillColor,fill opacity=0.20] (236.60, 77.64) circle (  2.13);

\path[fill=fillColor,fill opacity=0.20] (238.60, 80.39) circle (  2.13);

\path[fill=fillColor,fill opacity=0.20] (236.60, 81.86) circle (  2.13);

\path[fill=fillColor,fill opacity=0.20] (248.64, 93.57) circle (  2.13);

\path[fill=fillColor,fill opacity=0.20] (244.62, 99.08) circle (  2.13);

\path[fill=fillColor,fill opacity=0.20] (253.65, 87.37) circle (  2.13);

\path[fill=fillColor,fill opacity=0.20] (252.65, 75.48) circle (  2.13);

\path[fill=fillColor,fill opacity=0.20] (259.67, 75.83) circle (  2.13);

\path[fill=fillColor,fill opacity=0.20] (236.60, 77.21) circle (  2.13);

\path[fill=fillColor,fill opacity=0.20] (208.51, 72.56) circle (  2.13);

\path[fill=fillColor,fill opacity=0.20] (234.59, 65.67) circle (  2.13);

\path[fill=fillColor,fill opacity=0.20] (231.58, 61.53) circle (  2.13);

\path[fill=fillColor,fill opacity=0.20] (219.54, 64.03) circle (  2.13);

\path[fill=fillColor,fill opacity=0.20] (265.69, 67.05) circle (  2.13);

\path[fill=fillColor,fill opacity=0.20] (252.65, 70.32) circle (  2.13);

\path[fill=fillColor,fill opacity=0.20] (270.71, 76.86) circle (  2.13);

\path[fill=fillColor,fill opacity=0.20] (244.62, 83.58) circle (  2.13);

\path[fill=fillColor,fill opacity=0.20] (244.62, 82.29) circle (  2.13);

\path[fill=fillColor,fill opacity=0.20] (243.62, 76.35) circle (  2.13);

\path[fill=fillColor,fill opacity=0.20] (230.58, 76.52) circle (  2.13);

\path[fill=fillColor,fill opacity=0.20] (209.51, 82.03) circle (  2.13);

\path[fill=fillColor,fill opacity=0.20] (228.57, 90.99) circle (  2.13);

\path[fill=fillColor,fill opacity=0.20] (236.60, 88.75) circle (  2.13);

\path[fill=fillColor,fill opacity=0.20] (238.60, 84.35) circle (  2.13);

\path[fill=fillColor,fill opacity=0.20] (235.59, 82.72) circle (  2.13);

\path[fill=fillColor,fill opacity=0.20] (232.58, 77.29) circle (  2.13);

\path[fill=fillColor,fill opacity=0.20] (253.65, 74.19) circle (  2.13);

\path[fill=fillColor,fill opacity=0.20] (249.64, 80.13) circle (  2.13);

\path[fill=fillColor,fill opacity=0.20] (242.62, 88.23) circle (  2.13);
\end{scope}
\begin{scope}
\path[clip] (  0.00,  0.00) rectangle (289.08,144.54);
\definecolor[named]{drawColor}{rgb}{0.50,0.50,0.50}

\node[text=drawColor,anchor=base,inner sep=0pt, outer sep=0pt, scale=  0.96] at ( 48.20, 20.31) {8};

\node[text=drawColor,anchor=base,inner sep=0pt, outer sep=0pt, scale=  0.96] at ( 68.26, 20.31) {10};

\node[text=drawColor,anchor=base,inner sep=0pt, outer sep=0pt, scale=  0.96] at ( 88.33, 20.31) {12};

\node[text=drawColor,anchor=base,inner sep=0pt, outer sep=0pt, scale=  0.96] at (108.39, 20.31) {14};

\node[text=drawColor,anchor=base,inner sep=0pt, outer sep=0pt, scale=  0.96] at (128.46, 20.31) {16};

\node[text=drawColor,anchor=base,inner sep=0pt, outer sep=0pt, scale=  0.96] at (148.52, 20.31) {18};
\end{scope}
\begin{scope}
\path[clip] (  0.00,  0.00) rectangle (289.08,144.54);
\definecolor[named]{drawColor}{rgb}{0.50,0.50,0.50}

\path[draw=drawColor,line width= 0.6pt,line join=round,line cap=round] ( 48.20, 29.77) -- ( 48.20, 34.04);

\path[draw=drawColor,line width= 0.6pt,line join=round,line cap=round] ( 68.26, 29.77) -- ( 68.26, 34.04);

\path[draw=drawColor,line width= 0.6pt,line join=round,line cap=round] ( 88.33, 29.77) -- ( 88.33, 34.04);

\path[draw=drawColor,line width= 0.6pt,line join=round,line cap=round] (108.39, 29.77) -- (108.39, 34.04);

\path[draw=drawColor,line width= 0.6pt,line join=round,line cap=round] (128.46, 29.77) -- (128.46, 34.04);

\path[draw=drawColor,line width= 0.6pt,line join=round,line cap=round] (148.52, 29.77) -- (148.52, 34.04);
\end{scope}
\begin{scope}
\path[clip] (  0.00,  0.00) rectangle (289.08,144.54);
\definecolor[named]{drawColor}{rgb}{0.50,0.50,0.50}

\node[text=drawColor,anchor=base,inner sep=0pt, outer sep=0pt, scale=  0.96] at (168.38, 20.31) {8};

\node[text=drawColor,anchor=base,inner sep=0pt, outer sep=0pt, scale=  0.96] at (188.44, 20.31) {10};

\node[text=drawColor,anchor=base,inner sep=0pt, outer sep=0pt, scale=  0.96] at (208.51, 20.31) {12};

\node[text=drawColor,anchor=base,inner sep=0pt, outer sep=0pt, scale=  0.96] at (228.57, 20.31) {14};

\node[text=drawColor,anchor=base,inner sep=0pt, outer sep=0pt, scale=  0.96] at (248.64, 20.31) {16};

\node[text=drawColor,anchor=base,inner sep=0pt, outer sep=0pt, scale=  0.96] at (268.70, 20.31) {18};
\end{scope}
\begin{scope}
\path[clip] (  0.00,  0.00) rectangle (289.08,144.54);
\definecolor[named]{drawColor}{rgb}{0.50,0.50,0.50}

\path[draw=drawColor,line width= 0.6pt,line join=round,line cap=round] (168.38, 29.77) -- (168.38, 34.04);

\path[draw=drawColor,line width= 0.6pt,line join=round,line cap=round] (188.44, 29.77) -- (188.44, 34.04);

\path[draw=drawColor,line width= 0.6pt,line join=round,line cap=round] (208.51, 29.77) -- (208.51, 34.04);

\path[draw=drawColor,line width= 0.6pt,line join=round,line cap=round] (228.57, 29.77) -- (228.57, 34.04);

\path[draw=drawColor,line width= 0.6pt,line join=round,line cap=round] (248.64, 29.77) -- (248.64, 34.04);

\path[draw=drawColor,line width= 0.6pt,line join=round,line cap=round] (268.70, 29.77) -- (268.70, 34.04);
\end{scope}
\begin{scope}
\path[clip] (  0.00,  0.00) rectangle (289.08,144.54);
\definecolor[named]{drawColor}{rgb}{0.00,0.00,0.00}

\node[text=drawColor,anchor=base,inner sep=0pt, outer sep=0pt, scale=  1.20] at (158.36,  9.03) {$a$ $[\mu m]$};
\end{scope}
\begin{scope}
\path[clip] (  0.00,  0.00) rectangle (289.08,144.54);
\definecolor[named]{drawColor}{rgb}{0.00,0.00,0.00}

\node[text=drawColor,rotate= 90.00,anchor=base,inner sep=0pt, outer sep=0pt, scale=  1.20] at ( 17.30, 76.95) {RD $[\times 10^{-9}mm^2/s]$};
\end{scope}
\end{tikzpicture}

					\end{adjustbox}\\
					\begin{adjustbox}{width={\textwidth},totalheight=\textheight,keepaspectratio}
						\strut
						% Created by tikzDevice version - on 2012-09-27 22:37:13
% !TEX encoding = UTF-8 Unicode
\begin{tikzpicture}[x=1pt,y=1pt]
\definecolor[named]{fillColor}{rgb}{1.00,1.00,1.00}
\path[use as bounding box,fill=fillColor,fill opacity=0.00] (0,0) rectangle (289.08,144.54);
\begin{scope}
\path[clip] (  0.00,  0.00) rectangle (289.08,144.54);
\definecolor[named]{fillColor}{rgb}{1.00,1.00,1.00}

\path[fill=fillColor] (  0.00,  0.00) rectangle (289.08,144.54);
\end{scope}
\begin{scope}
\path[clip] ( 39.69,119.86) rectangle (156.86,132.50);
\definecolor[named]{fillColor}{rgb}{0.80,0.80,0.80}

\path[fill=fillColor] ( 39.69,119.86) rectangle (156.86,132.50);
\definecolor[named]{drawColor}{rgb}{0.00,0.00,0.00}

\node[text=drawColor,anchor=base,inner sep=0pt, outer sep=0pt, scale=  0.96] at ( 98.27,122.87) {Scan (r=-0.633)};
\end{scope}
\begin{scope}
\path[clip] (159.87,119.86) rectangle (277.04,132.50);
\definecolor[named]{fillColor}{rgb}{0.80,0.80,0.80}

\path[fill=fillColor] (159.87,119.86) rectangle (277.03,132.50);
\definecolor[named]{drawColor}{rgb}{0.00,0.00,0.00}

\node[text=drawColor,anchor=base,inner sep=0pt, outer sep=0pt, scale=  0.96] at (218.45,122.87) {Rescan (r=-0.567)};
\end{scope}
\begin{scope}
\path[clip] (  0.00,  0.00) rectangle (289.08,144.54);
\definecolor[named]{drawColor}{rgb}{0.50,0.50,0.50}

\node[text=drawColor,anchor=base east,inner sep=0pt, outer sep=0pt, scale=  0.96] at ( 32.58, 39.45) {0.2};

\node[text=drawColor,anchor=base east,inner sep=0pt, outer sep=0pt, scale=  0.96] at ( 32.58, 56.68) {0.4};

\node[text=drawColor,anchor=base east,inner sep=0pt, outer sep=0pt, scale=  0.96] at ( 32.58, 73.90) {0.6};

\node[text=drawColor,anchor=base east,inner sep=0pt, outer sep=0pt, scale=  0.96] at ( 32.58, 91.12) {0.8};

\node[text=drawColor,anchor=base east,inner sep=0pt, outer sep=0pt, scale=  0.96] at ( 32.58,108.35) {1.0};
\end{scope}
\begin{scope}
\path[clip] (  0.00,  0.00) rectangle (289.08,144.54);
\definecolor[named]{drawColor}{rgb}{0.50,0.50,0.50}

\path[draw=drawColor,line width= 0.6pt,line join=round,line cap=round] ( 35.42, 42.76) -- ( 39.69, 42.76);

\path[draw=drawColor,line width= 0.6pt,line join=round,line cap=round] ( 35.42, 59.98) -- ( 39.69, 59.98);

\path[draw=drawColor,line width= 0.6pt,line join=round,line cap=round] ( 35.42, 77.21) -- ( 39.69, 77.21);

\path[draw=drawColor,line width= 0.6pt,line join=round,line cap=round] ( 35.42, 94.43) -- ( 39.69, 94.43);

\path[draw=drawColor,line width= 0.6pt,line join=round,line cap=round] ( 35.42,111.65) -- ( 39.69,111.65);
\end{scope}
\begin{scope}
\path[clip] ( 39.69, 34.04) rectangle (156.86,119.86);
\definecolor[named]{fillColor}{rgb}{0.90,0.90,0.90}

\path[fill=fillColor] ( 39.69, 34.04) rectangle (156.86,119.86);
\definecolor[named]{drawColor}{rgb}{0.95,0.95,0.95}

\path[draw=drawColor,line width= 0.3pt,line join=round,line cap=round] ( 39.69, 34.15) --
	(156.86, 34.15);

\path[draw=drawColor,line width= 0.3pt,line join=round,line cap=round] ( 39.69, 51.37) --
	(156.86, 51.37);

\path[draw=drawColor,line width= 0.3pt,line join=round,line cap=round] ( 39.69, 68.60) --
	(156.86, 68.60);

\path[draw=drawColor,line width= 0.3pt,line join=round,line cap=round] ( 39.69, 85.82) --
	(156.86, 85.82);

\path[draw=drawColor,line width= 0.3pt,line join=round,line cap=round] ( 39.69,103.04) --
	(156.86,103.04);

\path[draw=drawColor,line width= 0.3pt,line join=round,line cap=round] ( 49.61, 34.04) --
	( 49.61,119.86);

\path[draw=drawColor,line width= 0.3pt,line join=round,line cap=round] ( 71.46, 34.04) --
	( 71.46,119.86);

\path[draw=drawColor,line width= 0.3pt,line join=round,line cap=round] ( 93.31, 34.04) --
	( 93.31,119.86);

\path[draw=drawColor,line width= 0.3pt,line join=round,line cap=round] (115.16, 34.04) --
	(115.16,119.86);

\path[draw=drawColor,line width= 0.3pt,line join=round,line cap=round] (137.01, 34.04) --
	(137.01,119.86);
\definecolor[named]{drawColor}{rgb}{1.00,1.00,1.00}

\path[draw=drawColor,line width= 0.6pt,line join=round,line cap=round] ( 39.69, 42.76) --
	(156.86, 42.76);

\path[draw=drawColor,line width= 0.6pt,line join=round,line cap=round] ( 39.69, 59.98) --
	(156.86, 59.98);

\path[draw=drawColor,line width= 0.6pt,line join=round,line cap=round] ( 39.69, 77.21) --
	(156.86, 77.21);

\path[draw=drawColor,line width= 0.6pt,line join=round,line cap=round] ( 39.69, 94.43) --
	(156.86, 94.43);

\path[draw=drawColor,line width= 0.6pt,line join=round,line cap=round] ( 39.69,111.65) --
	(156.86,111.65);

\path[draw=drawColor,line width= 0.6pt,line join=round,line cap=round] ( 60.53, 34.04) --
	( 60.53,119.86);

\path[draw=drawColor,line width= 0.6pt,line join=round,line cap=round] ( 82.38, 34.04) --
	( 82.38,119.86);

\path[draw=drawColor,line width= 0.6pt,line join=round,line cap=round] (104.23, 34.04) --
	(104.23,119.86);

\path[draw=drawColor,line width= 0.6pt,line join=round,line cap=round] (126.08, 34.04) --
	(126.08,119.86);

\path[draw=drawColor,line width= 0.6pt,line join=round,line cap=round] (147.93, 34.04) --
	(147.93,119.86);
\definecolor[named]{fillColor}{rgb}{0.00,0.00,0.00}

\path[fill=fillColor,fill opacity=0.20] ( 52.23, 87.63) circle (  2.13);

\path[fill=fillColor,fill opacity=0.20] ( 64.25, 94.86) circle (  2.13);

\path[fill=fillColor,fill opacity=0.20] ( 76.26, 81.34) circle (  2.13);

\path[fill=fillColor,fill opacity=0.20] ( 86.97, 67.65) circle (  2.13);

\path[fill=fillColor,fill opacity=0.20] ( 81.07, 64.72) circle (  2.13);

\path[fill=fillColor,fill opacity=0.20] ( 75.17, 65.84) circle (  2.13);

\path[fill=fillColor,fill opacity=0.20] ( 71.02, 81.51) circle (  2.13);

\path[fill=fillColor,fill opacity=0.20] ( 59.88,107.09) circle (  2.13);

\path[fill=fillColor,fill opacity=0.20] ( 75.17, 94.69) circle (  2.13);

\path[fill=fillColor,fill opacity=0.20] ( 85.88, 68.60) circle (  2.13);

\path[fill=fillColor,fill opacity=0.20] ( 97.68, 57.83) circle (  2.13);

\path[fill=fillColor,fill opacity=0.20] (108.60, 59.90) circle (  2.13);

\path[fill=fillColor,fill opacity=0.20] ( 96.36, 60.33) circle (  2.13);

\path[fill=fillColor,fill opacity=0.20] (103.79, 57.23) circle (  2.13);

\path[fill=fillColor,fill opacity=0.20] (100.30, 55.85) circle (  2.13);

\path[fill=fillColor,fill opacity=0.20] ( 91.78, 69.28) circle (  2.13);

\path[fill=fillColor,fill opacity=0.20] ( 69.49, 76.09) circle (  2.13);

\path[fill=fillColor,fill opacity=0.20] ( 58.13, 77.81) circle (  2.13);

\path[fill=fillColor,fill opacity=0.20] ( 69.05,113.37) circle (  2.13);

\path[fill=fillColor,fill opacity=0.20] ( 85.22, 83.58) circle (  2.13);

\path[fill=fillColor,fill opacity=0.20] ( 90.68, 66.18) circle (  2.13);

\path[fill=fillColor,fill opacity=0.20] ( 94.84, 60.93) circle (  2.13);

\path[fill=fillColor,fill opacity=0.20] ( 98.33, 55.51) circle (  2.13);

\path[fill=fillColor,fill opacity=0.20] (102.92, 53.18) circle (  2.13);

\path[fill=fillColor,fill opacity=0.20] (100.08, 58.86) circle (  2.13);

\path[fill=fillColor,fill opacity=0.20] ( 95.27, 64.12) circle (  2.13);

\path[fill=fillColor,fill opacity=0.20] ( 93.52, 59.81) circle (  2.13);

\path[fill=fillColor,fill opacity=0.20] ( 88.06, 53.09) circle (  2.13);

\path[fill=fillColor,fill opacity=0.20] ( 76.04, 61.88) circle (  2.13);

\path[fill=fillColor,fill opacity=0.20] ( 74.95, 74.28) circle (  2.13);

\path[fill=fillColor,fill opacity=0.20] ( 71.46, 81.00) circle (  2.13);

\path[fill=fillColor,fill opacity=0.20] ( 58.56,101.66) circle (  2.13);

\path[fill=fillColor,fill opacity=0.20] ( 81.51, 86.68) circle (  2.13);

\path[fill=fillColor,fill opacity=0.20] ( 98.11, 68.08) circle (  2.13);

\path[fill=fillColor,fill opacity=0.20] (111.66, 53.78) circle (  2.13);

\path[fill=fillColor,fill opacity=0.20] (106.42, 62.83) circle (  2.13);

\path[fill=fillColor,fill opacity=0.20] ( 99.21, 60.07) circle (  2.13);

\path[fill=fillColor,fill opacity=0.20] ( 92.87, 49.22) circle (  2.13);

\path[fill=fillColor,fill opacity=0.20] ( 97.68, 55.16) circle (  2.13);

\path[fill=fillColor,fill opacity=0.20] ( 99.86, 62.65) circle (  2.13);

\path[fill=fillColor,fill opacity=0.20] ( 88.94, 60.41) circle (  2.13);

\path[fill=fillColor,fill opacity=0.20] ( 79.76, 59.55) circle (  2.13);

\path[fill=fillColor,fill opacity=0.20] ( 77.14, 67.82) circle (  2.13);

\path[fill=fillColor,fill opacity=0.20] ( 77.79, 73.68) circle (  2.13);

\path[fill=fillColor,fill opacity=0.20] ( 74.30, 80.13) circle (  2.13);

\path[fill=fillColor,fill opacity=0.20] ( 90.25, 37.94) circle (  2.13);

\path[fill=fillColor,fill opacity=0.20] ( 60.97, 93.48) circle (  2.13);

\path[fill=fillColor,fill opacity=0.20] ( 76.48, 70.58) circle (  2.13);

\path[fill=fillColor,fill opacity=0.20] ( 98.77, 61.45) circle (  2.13);

\path[fill=fillColor,fill opacity=0.20] (107.95, 49.74) circle (  2.13);

\path[fill=fillColor,fill opacity=0.20] (105.10, 50.17) circle (  2.13);

\path[fill=fillColor,fill opacity=0.20] (103.58, 55.08) circle (  2.13);

\path[fill=fillColor,fill opacity=0.20] ( 95.27, 58.35) circle (  2.13);

\path[fill=fillColor,fill opacity=0.20] ( 83.25, 58.78) circle (  2.13);

\path[fill=fillColor,fill opacity=0.20] ( 89.37, 55.25) circle (  2.13);

\path[fill=fillColor,fill opacity=0.20] ( 85.22, 56.37) circle (  2.13);

\path[fill=fillColor,fill opacity=0.20] ( 76.92, 64.98) circle (  2.13);

\path[fill=fillColor,fill opacity=0.20] ( 74.30, 75.66) circle (  2.13);

\path[fill=fillColor,fill opacity=0.20] ( 65.56, 88.40) circle (  2.13);

\path[fill=fillColor,fill opacity=0.20] ( 54.19,114.24) circle (  2.13);

\path[fill=fillColor,fill opacity=0.20] ( 70.80, 75.83) circle (  2.13);

\path[fill=fillColor,fill opacity=0.20] ( 90.68, 48.62) circle (  2.13);

\path[fill=fillColor,fill opacity=0.20] ( 81.51, 73.16) circle (  2.13);

\path[fill=fillColor,fill opacity=0.20] ( 81.29, 72.82) circle (  2.13);

\path[fill=fillColor,fill opacity=0.20] ( 83.47, 62.65) circle (  2.13);

\path[fill=fillColor,fill opacity=0.20] ( 64.25, 96.15) circle (  2.13);

\path[fill=fillColor,fill opacity=0.20] ( 78.23, 61.96) circle (  2.13);

\path[fill=fillColor,fill opacity=0.20] ( 95.93, 52.84) circle (  2.13);

\path[fill=fillColor,fill opacity=0.20] (102.48, 46.89) circle (  2.13);

\path[fill=fillColor,fill opacity=0.20] (102.05, 40.01) circle (  2.13);

\path[fill=fillColor,fill opacity=0.20] ( 99.21, 59.04) circle (  2.13);

\path[fill=fillColor,fill opacity=0.20] ( 87.84, 66.53) circle (  2.13);

\path[fill=fillColor,fill opacity=0.20] ( 73.86, 59.64) circle (  2.13);

\path[fill=fillColor,fill opacity=0.20] ( 71.67, 58.18) circle (  2.13);

\path[fill=fillColor,fill opacity=0.20] ( 64.25, 64.63) circle (  2.13);

\path[fill=fillColor,fill opacity=0.20] ( 66.65, 85.47) circle (  2.13);

\path[fill=fillColor,fill opacity=0.20] (105.32, 53.87) circle (  2.13);

\path[fill=fillColor,fill opacity=0.20] ( 97.02, 61.19) circle (  2.13);

\path[fill=fillColor,fill opacity=0.20] ( 94.84, 56.45) circle (  2.13);

\path[fill=fillColor,fill opacity=0.20] ( 99.86, 56.11) circle (  2.13);

\path[fill=fillColor,fill opacity=0.20] ( 99.42, 57.40) circle (  2.13);

\path[fill=fillColor,fill opacity=0.20] ( 95.71, 51.89) circle (  2.13);

\path[fill=fillColor,fill opacity=0.20] ( 81.29, 60.76) circle (  2.13);

\path[fill=fillColor,fill opacity=0.20] ( 55.94,104.07) circle (  2.13);

\path[fill=fillColor,fill opacity=0.20] ( 70.36, 89.52) circle (  2.13);

\path[fill=fillColor,fill opacity=0.20] ( 77.57, 58.86) circle (  2.13);

\path[fill=fillColor,fill opacity=0.20] ( 88.06, 44.48) circle (  2.13);

\path[fill=fillColor,fill opacity=0.20] ( 90.47, 49.56) circle (  2.13);

\path[fill=fillColor,fill opacity=0.20] ( 87.62, 52.66) circle (  2.13);

\path[fill=fillColor,fill opacity=0.20] ( 89.59, 43.45) circle (  2.13);

\path[fill=fillColor,fill opacity=0.20] ( 91.99, 50.08) circle (  2.13);

\path[fill=fillColor,fill opacity=0.20] ( 85.44, 70.92) circle (  2.13);

\path[fill=fillColor,fill opacity=0.20] ( 64.46, 65.93) circle (  2.13);

\path[fill=fillColor,fill opacity=0.20] ( 77.57, 77.29) circle (  2.13);

\path[fill=fillColor,fill opacity=0.20] (111.88, 51.03) circle (  2.13);

\path[fill=fillColor,fill opacity=0.20] (110.35, 62.22) circle (  2.13);

\path[fill=fillColor,fill opacity=0.20] (116.03, 55.25) circle (  2.13);

\path[fill=fillColor,fill opacity=0.20] (126.95, 43.02) circle (  2.13);

\path[fill=fillColor,fill opacity=0.20] (146.84, 40.78) circle (  2.13);

\path[fill=fillColor,fill opacity=0.20] (113.41, 46.29) circle (  2.13);

\path[fill=fillColor,fill opacity=0.20] (105.54, 42.16) circle (  2.13);

\path[fill=fillColor,fill opacity=0.20] (101.39, 47.76) circle (  2.13);

\path[fill=fillColor,fill opacity=0.20] ( 65.34, 84.27) circle (  2.13);

\path[fill=fillColor,fill opacity=0.20] ( 67.96, 83.67) circle (  2.13);

\path[fill=fillColor,fill opacity=0.20] ( 78.01, 59.64) circle (  2.13);

\path[fill=fillColor,fill opacity=0.20] ( 88.94, 54.30) circle (  2.13);

\path[fill=fillColor,fill opacity=0.20] ( 81.94, 53.35) circle (  2.13);

\path[fill=fillColor,fill opacity=0.20] ( 80.41, 61.10) circle (  2.13);

\path[fill=fillColor,fill opacity=0.20] ( 83.47, 60.16) circle (  2.13);

\path[fill=fillColor,fill opacity=0.20] ( 80.20, 56.02) circle (  2.13);

\path[fill=fillColor,fill opacity=0.20] ( 78.67, 68.60) circle (  2.13);

\path[fill=fillColor,fill opacity=0.20] ( 77.79, 76.43) circle (  2.13);

\path[fill=fillColor,fill opacity=0.20] ( 67.96, 72.90) circle (  2.13);

\path[fill=fillColor,fill opacity=0.20] ( 59.88,107.95) circle (  2.13);

\path[fill=fillColor,fill opacity=0.20] (111.66, 48.96) circle (  2.13);

\path[fill=fillColor,fill opacity=0.20] (115.59, 59.81) circle (  2.13);

\path[fill=fillColor,fill opacity=0.20] (119.09, 58.61) circle (  2.13);

\path[fill=fillColor,fill opacity=0.20] (133.73, 39.75) circle (  2.13);

\path[fill=fillColor,fill opacity=0.20] (120.40, 38.71) circle (  2.13);

\path[fill=fillColor,fill opacity=0.20] (122.58, 46.12) circle (  2.13);

\path[fill=fillColor,fill opacity=0.20] (100.95, 47.33) circle (  2.13);

\path[fill=fillColor,fill opacity=0.20] ( 96.15, 43.62) circle (  2.13);

\path[fill=fillColor,fill opacity=0.20] ( 79.98, 56.80) circle (  2.13);

\path[fill=fillColor,fill opacity=0.20] ( 60.31, 87.45) circle (  2.13);

\path[fill=fillColor,fill opacity=0.20] ( 81.07, 68.94) circle (  2.13);

\path[fill=fillColor,fill opacity=0.20] ( 81.94, 69.37) circle (  2.13);

\path[fill=fillColor,fill opacity=0.20] ( 83.91, 51.29) circle (  2.13);

\path[fill=fillColor,fill opacity=0.20] ( 86.97, 46.46) circle (  2.13);

\path[fill=fillColor,fill opacity=0.20] ( 83.04, 61.79) circle (  2.13);

\path[fill=fillColor,fill opacity=0.20] ( 86.10, 68.60) circle (  2.13);

\path[fill=fillColor,fill opacity=0.20] ( 80.63, 64.46) circle (  2.13);

\path[fill=fillColor,fill opacity=0.20] ( 82.82, 62.40) circle (  2.13);

\path[fill=fillColor,fill opacity=0.20] ( 71.24, 69.97) circle (  2.13);

\path[fill=fillColor,fill opacity=0.20] ( 89.15, 71.70) circle (  2.13);

\path[fill=fillColor,fill opacity=0.20] (117.78, 52.15) circle (  2.13);

\path[fill=fillColor,fill opacity=0.20] (109.26, 51.03) circle (  2.13);

\path[fill=fillColor,fill opacity=0.20] (101.17, 42.85) circle (  2.13);

\path[fill=fillColor,fill opacity=0.20] (100.08, 50.94) circle (  2.13);

\path[fill=fillColor,fill opacity=0.20] (102.48, 53.96) circle (  2.13);

\path[fill=fillColor,fill opacity=0.20] ( 94.40, 50.68) circle (  2.13);

\path[fill=fillColor,fill opacity=0.20] ( 82.60, 51.63) circle (  2.13);

\path[fill=fillColor,fill opacity=0.20] ( 84.78, 39.32) circle (  2.13);

\path[fill=fillColor,fill opacity=0.20] ( 56.38, 98.39) circle (  2.13);

\path[fill=fillColor,fill opacity=0.20] ( 75.39, 77.90) circle (  2.13);

\path[fill=fillColor,fill opacity=0.20] ( 81.73, 70.92) circle (  2.13);

\path[fill=fillColor,fill opacity=0.20] ( 91.78, 48.27) circle (  2.13);

\path[fill=fillColor,fill opacity=0.20] ( 86.53, 51.46) circle (  2.13);

\path[fill=fillColor,fill opacity=0.20] ( 83.47, 64.38) circle (  2.13);

\path[fill=fillColor,fill opacity=0.20] ( 81.94, 64.12) circle (  2.13);

\path[fill=fillColor,fill opacity=0.20] ( 80.41, 58.52) circle (  2.13);

\path[fill=fillColor,fill opacity=0.20] ( 78.23, 59.04) circle (  2.13);

\path[fill=fillColor,fill opacity=0.20] (129.14, 48.79) circle (  2.13);

\path[fill=fillColor,fill opacity=0.20] (114.72, 44.14) circle (  2.13);

\path[fill=fillColor,fill opacity=0.20] (102.70, 53.09) circle (  2.13);

\path[fill=fillColor,fill opacity=0.20] (119.96, 46.72) circle (  2.13);

\path[fill=fillColor,fill opacity=0.20] (100.95, 48.79) circle (  2.13);

\path[fill=fillColor,fill opacity=0.20] ( 93.31, 54.64) circle (  2.13);

\path[fill=fillColor,fill opacity=0.20] (100.73, 53.78) circle (  2.13);

\path[fill=fillColor,fill opacity=0.20] ( 94.18, 55.16) circle (  2.13);

\path[fill=fillColor,fill opacity=0.20] ( 86.31, 57.14) circle (  2.13);

\path[fill=fillColor,fill opacity=0.20] ( 88.28, 45.09) circle (  2.13);

\path[fill=fillColor,fill opacity=0.20] ( 79.32, 44.83) circle (  2.13);

\path[fill=fillColor,fill opacity=0.20] ( 72.33, 82.72) circle (  2.13);

\path[fill=fillColor,fill opacity=0.20] ( 90.90, 66.27) circle (  2.13);

\path[fill=fillColor,fill opacity=0.20] ( 99.64, 54.99) circle (  2.13);

\path[fill=fillColor,fill opacity=0.20] ( 97.46, 49.39) circle (  2.13);

\path[fill=fillColor,fill opacity=0.20] ( 88.06, 48.53) circle (  2.13);

\path[fill=fillColor,fill opacity=0.20] ( 84.78, 49.99) circle (  2.13);

\path[fill=fillColor,fill opacity=0.20] ( 83.04, 57.40) circle (  2.13);

\path[fill=fillColor,fill opacity=0.20] ( 76.48, 69.11) circle (  2.13);

\path[fill=fillColor,fill opacity=0.20] ( 79.98, 62.14) circle (  2.13);

\path[fill=fillColor,fill opacity=0.20] ( 66.21, 95.29) circle (  2.13);

\path[fill=fillColor,fill opacity=0.20] (117.34, 42.33) circle (  2.13);

\path[fill=fillColor,fill opacity=0.20] (103.36, 63.00) circle (  2.13);

\path[fill=fillColor,fill opacity=0.20] (109.69, 61.53) circle (  2.13);

\path[fill=fillColor,fill opacity=0.20] (122.15, 47.33) circle (  2.13);

\path[fill=fillColor,fill opacity=0.20] (106.20, 48.96) circle (  2.13);

\path[fill=fillColor,fill opacity=0.20] ( 96.58, 54.04) circle (  2.13);

\path[fill=fillColor,fill opacity=0.20] (106.63, 55.08) circle (  2.13);

\path[fill=fillColor,fill opacity=0.20] ( 83.25, 60.41) circle (  2.13);

\path[fill=fillColor,fill opacity=0.20] ( 88.06, 56.80) circle (  2.13);

\path[fill=fillColor,fill opacity=0.20] ( 87.62, 45.34) circle (  2.13);

\path[fill=fillColor,fill opacity=0.20] ( 74.30, 61.10) circle (  2.13);

\path[fill=fillColor,fill opacity=0.20] ( 82.60, 63.86) circle (  2.13);

\path[fill=fillColor,fill opacity=0.20] ( 90.68, 63.69) circle (  2.13);

\path[fill=fillColor,fill opacity=0.20] ( 92.43, 62.91) circle (  2.13);

\path[fill=fillColor,fill opacity=0.20] ( 90.90, 55.33) circle (  2.13);

\path[fill=fillColor,fill opacity=0.20] ( 88.28, 45.26) circle (  2.13);

\path[fill=fillColor,fill opacity=0.20] ( 90.68, 49.39) circle (  2.13);

\path[fill=fillColor,fill opacity=0.20] ( 81.29, 71.87) circle (  2.13);

\path[fill=fillColor,fill opacity=0.20] ( 89.81, 73.85) circle (  2.13);

\path[fill=fillColor,fill opacity=0.20] ( 82.60, 64.98) circle (  2.13);

\path[fill=fillColor,fill opacity=0.20] ( 95.93, 56.54) circle (  2.13);

\path[fill=fillColor,fill opacity=0.20] ( 89.81, 55.85) circle (  2.13);

\path[fill=fillColor,fill opacity=0.20] ( 88.28, 68.16) circle (  2.13);

\path[fill=fillColor,fill opacity=0.20] ( 97.89, 60.33) circle (  2.13);

\path[fill=fillColor,fill opacity=0.20] (107.73, 50.51) circle (  2.13);

\path[fill=fillColor,fill opacity=0.20] (100.52, 54.13) circle (  2.13);

\path[fill=fillColor,fill opacity=0.20] ( 98.33, 54.21) circle (  2.13);

\path[fill=fillColor,fill opacity=0.20] (101.17, 49.74) circle (  2.13);

\path[fill=fillColor,fill opacity=0.20] ( 92.87, 54.21) circle (  2.13);

\path[fill=fillColor,fill opacity=0.20] ( 85.88, 53.35) circle (  2.13);

\path[fill=fillColor,fill opacity=0.20] ( 85.66, 49.05) circle (  2.13);

\path[fill=fillColor,fill opacity=0.20] ( 66.43, 75.40) circle (  2.13);

\path[fill=fillColor,fill opacity=0.20] ( 72.11, 75.48) circle (  2.13);

\path[fill=fillColor,fill opacity=0.20] ( 80.63, 70.66) circle (  2.13);

\path[fill=fillColor,fill opacity=0.20] ( 82.38, 65.24) circle (  2.13);

\path[fill=fillColor,fill opacity=0.20] ( 91.99, 57.06) circle (  2.13);

\path[fill=fillColor,fill opacity=0.20] ( 86.75, 53.53) circle (  2.13);

\path[fill=fillColor,fill opacity=0.20] ( 87.62, 53.61) circle (  2.13);

\path[fill=fillColor,fill opacity=0.20] ( 88.50, 62.40) circle (  2.13);

\path[fill=fillColor,fill opacity=0.20] ( 79.32, 67.65) circle (  2.13);

\path[fill=fillColor,fill opacity=0.20] ( 75.83, 60.67) circle (  2.13);

\path[fill=fillColor,fill opacity=0.20] ( 64.90, 74.02) circle (  2.13);

\path[fill=fillColor,fill opacity=0.20] ( 86.31, 70.75) circle (  2.13);

\path[fill=fillColor,fill opacity=0.20] (110.13, 53.61) circle (  2.13);

\path[fill=fillColor,fill opacity=0.20] ( 83.25, 63.77) circle (  2.13);

\path[fill=fillColor,fill opacity=0.20] ( 87.62, 62.31) circle (  2.13);

\path[fill=fillColor,fill opacity=0.20] ( 95.05, 53.70) circle (  2.13);

\path[fill=fillColor,fill opacity=0.20] ( 98.33, 50.08) circle (  2.13);

\path[fill=fillColor,fill opacity=0.20] (101.17, 54.39) circle (  2.13);

\path[fill=fillColor,fill opacity=0.20] ( 91.56, 51.11) circle (  2.13);

\path[fill=fillColor,fill opacity=0.20] (106.63, 43.62) circle (  2.13);

\path[fill=fillColor,fill opacity=0.20] (102.92, 50.68) circle (  2.13);

\path[fill=fillColor,fill opacity=0.20] ( 90.47, 54.99) circle (  2.13);

\path[fill=fillColor,fill opacity=0.20] ( 83.25, 58.52) circle (  2.13);

\path[fill=fillColor,fill opacity=0.20] ( 71.46, 84.10) circle (  2.13);

\path[fill=fillColor,fill opacity=0.20] ( 89.59, 65.32) circle (  2.13);

\path[fill=fillColor,fill opacity=0.20] ( 90.68, 53.96) circle (  2.13);

\path[fill=fillColor,fill opacity=0.20] ( 85.00, 57.92) circle (  2.13);

\path[fill=fillColor,fill opacity=0.20] ( 89.59, 57.92) circle (  2.13);

\path[fill=fillColor,fill opacity=0.20] ( 90.68, 50.34) circle (  2.13);

\path[fill=fillColor,fill opacity=0.20] ( 86.75, 46.38) circle (  2.13);

\path[fill=fillColor,fill opacity=0.20] ( 77.79, 48.44) circle (  2.13);

\path[fill=fillColor,fill opacity=0.20] ( 76.26, 56.37) circle (  2.13);

\path[fill=fillColor,fill opacity=0.20] ( 67.52, 74.80) circle (  2.13);

\path[fill=fillColor,fill opacity=0.20] ( 72.99, 86.08) circle (  2.13);

\path[fill=fillColor,fill opacity=0.20] ( 97.24, 49.48) circle (  2.13);

\path[fill=fillColor,fill opacity=0.20] ( 96.15, 57.23) circle (  2.13);

\path[fill=fillColor,fill opacity=0.20] ( 96.80, 54.21) circle (  2.13);

\path[fill=fillColor,fill opacity=0.20] (105.32, 51.72) circle (  2.13);

\path[fill=fillColor,fill opacity=0.20] (102.26, 53.18) circle (  2.13);

\path[fill=fillColor,fill opacity=0.20] ( 99.64, 52.58) circle (  2.13);

\path[fill=fillColor,fill opacity=0.20] (100.95, 49.31) circle (  2.13);

\path[fill=fillColor,fill opacity=0.20] ( 98.99, 45.52) circle (  2.13);

\path[fill=fillColor,fill opacity=0.20] (103.36, 48.19) circle (  2.13);

\path[fill=fillColor,fill opacity=0.20] ( 95.93, 59.38) circle (  2.13);

\path[fill=fillColor,fill opacity=0.20] ( 94.84, 65.41) circle (  2.13);

\path[fill=fillColor,fill opacity=0.20] ( 78.67, 74.45) circle (  2.13);

\path[fill=fillColor,fill opacity=0.20] ( 88.06, 78.84) circle (  2.13);

\path[fill=fillColor,fill opacity=0.20] ( 99.42, 54.64) circle (  2.13);

\path[fill=fillColor,fill opacity=0.20] ( 84.35, 52.92) circle (  2.13);

\path[fill=fillColor,fill opacity=0.20] ( 89.37, 54.56) circle (  2.13);

\path[fill=fillColor,fill opacity=0.20] ( 89.81, 48.27) circle (  2.13);

\path[fill=fillColor,fill opacity=0.20] ( 85.22, 46.38) circle (  2.13);

\path[fill=fillColor,fill opacity=0.20] ( 86.97, 48.96) circle (  2.13);

\path[fill=fillColor,fill opacity=0.20] ( 78.45, 55.76) circle (  2.13);

\path[fill=fillColor,fill opacity=0.20] ( 75.17, 68.51) circle (  2.13);

\path[fill=fillColor,fill opacity=0.20] ( 60.31, 82.55) circle (  2.13);

\path[fill=fillColor,fill opacity=0.20] ( 75.17, 96.58) circle (  2.13);

\path[fill=fillColor,fill opacity=0.20] ( 97.46, 56.71) circle (  2.13);

\path[fill=fillColor,fill opacity=0.20] ( 94.18, 62.74) circle (  2.13);

\path[fill=fillColor,fill opacity=0.20] ( 97.46, 53.78) circle (  2.13);

\path[fill=fillColor,fill opacity=0.20] (107.29, 45.95) circle (  2.13);

\path[fill=fillColor,fill opacity=0.20] (108.38, 49.99) circle (  2.13);

\path[fill=fillColor,fill opacity=0.20] (102.70, 58.18) circle (  2.13);

\path[fill=fillColor,fill opacity=0.20] ( 98.33, 61.19) circle (  2.13);

\path[fill=fillColor,fill opacity=0.20] (107.29, 54.56) circle (  2.13);

\path[fill=fillColor,fill opacity=0.20] ( 98.33, 50.34) circle (  2.13);

\path[fill=fillColor,fill opacity=0.20] (100.52, 56.11) circle (  2.13);

\path[fill=fillColor,fill opacity=0.20] (100.52, 62.83) circle (  2.13);

\path[fill=fillColor,fill opacity=0.20] ( 86.10, 76.17) circle (  2.13);

\path[fill=fillColor,fill opacity=0.20] ( 57.69,100.80) circle (  2.13);

\path[fill=fillColor,fill opacity=0.20] ( 55.51,106.49) circle (  2.13);

\path[fill=fillColor,fill opacity=0.20] ( 75.61, 72.64) circle (  2.13);

\path[fill=fillColor,fill opacity=0.20] ( 88.28, 57.83) circle (  2.13);

\path[fill=fillColor,fill opacity=0.20] ( 86.53, 54.04) circle (  2.13);

\path[fill=fillColor,fill opacity=0.20] ( 81.73, 57.66) circle (  2.13);

\path[fill=fillColor,fill opacity=0.20] ( 89.37, 58.43) circle (  2.13);

\path[fill=fillColor,fill opacity=0.20] ( 86.97, 55.68) circle (  2.13);

\path[fill=fillColor,fill opacity=0.20] ( 86.10, 59.21) circle (  2.13);

\path[fill=fillColor,fill opacity=0.20] ( 77.36, 70.32) circle (  2.13);

\path[fill=fillColor,fill opacity=0.20] ( 72.77, 77.64) circle (  2.13);

\path[fill=fillColor,fill opacity=0.20] ( 65.77, 84.18) circle (  2.13);

\path[fill=fillColor,fill opacity=0.20] ( 68.40, 99.51) circle (  2.13);

\path[fill=fillColor,fill opacity=0.20] ( 97.68, 54.56) circle (  2.13);

\path[fill=fillColor,fill opacity=0.20] ( 94.18, 66.53) circle (  2.13);

\path[fill=fillColor,fill opacity=0.20] ( 95.05, 62.74) circle (  2.13);

\path[fill=fillColor,fill opacity=0.20] (100.95, 53.78) circle (  2.13);

\path[fill=fillColor,fill opacity=0.20] (102.05, 56.80) circle (  2.13);

\path[fill=fillColor,fill opacity=0.20] ( 99.21, 58.18) circle (  2.13);

\path[fill=fillColor,fill opacity=0.20] (100.08, 57.66) circle (  2.13);

\path[fill=fillColor,fill opacity=0.20] (103.58, 59.21) circle (  2.13);

\path[fill=fillColor,fill opacity=0.20] (100.08, 61.62) circle (  2.13);

\path[fill=fillColor,fill opacity=0.20] (104.23, 66.10) circle (  2.13);

\path[fill=fillColor,fill opacity=0.20] (102.26, 62.14) circle (  2.13);

\path[fill=fillColor,fill opacity=0.20] ( 88.06, 58.26) circle (  2.13);

\path[fill=fillColor,fill opacity=0.20] ( 55.94,106.14) circle (  2.13);

\path[fill=fillColor,fill opacity=0.20] ( 85.00, 81.86) circle (  2.13);

\path[fill=fillColor,fill opacity=0.20] ( 93.74, 61.79) circle (  2.13);

\path[fill=fillColor,fill opacity=0.20] ( 85.22, 61.10) circle (  2.13);

\path[fill=fillColor,fill opacity=0.20] ( 91.34, 55.42) circle (  2.13);

\path[fill=fillColor,fill opacity=0.20] ( 98.11, 47.76) circle (  2.13);

\path[fill=fillColor,fill opacity=0.20] ( 86.75, 54.90) circle (  2.13);

\path[fill=fillColor,fill opacity=0.20] ( 84.78, 60.93) circle (  2.13);

\path[fill=fillColor,fill opacity=0.20] ( 83.25, 63.95) circle (  2.13);

\path[fill=fillColor,fill opacity=0.20] ( 83.47, 66.61) circle (  2.13);

\path[fill=fillColor,fill opacity=0.20] ( 68.83, 72.13) circle (  2.13);

\path[fill=fillColor,fill opacity=0.20] ( 59.22,106.49) circle (  2.13);

\path[fill=fillColor,fill opacity=0.20] ( 92.21, 59.04) circle (  2.13);

\path[fill=fillColor,fill opacity=0.20] ( 93.09, 67.05) circle (  2.13);

\path[fill=fillColor,fill opacity=0.20] (103.14, 62.74) circle (  2.13);

\path[fill=fillColor,fill opacity=0.20] (103.36, 50.34) circle (  2.13);

\path[fill=fillColor,fill opacity=0.20] (105.76, 49.39) circle (  2.13);

\path[fill=fillColor,fill opacity=0.20] (102.48, 55.94) circle (  2.13);

\path[fill=fillColor,fill opacity=0.20] (100.73, 59.47) circle (  2.13);

\path[fill=fillColor,fill opacity=0.20] (103.79, 54.90) circle (  2.13);

\path[fill=fillColor,fill opacity=0.20] (110.35, 53.70) circle (  2.13);

\path[fill=fillColor,fill opacity=0.20] (104.89, 62.74) circle (  2.13);

\path[fill=fillColor,fill opacity=0.20] ( 98.11, 67.05) circle (  2.13);

\path[fill=fillColor,fill opacity=0.20] ( 76.70, 65.06) circle (  2.13);

\path[fill=fillColor,fill opacity=0.20] ( 55.51,111.65) circle (  2.13);

\path[fill=fillColor,fill opacity=0.20] ( 91.12, 77.81) circle (  2.13);

\path[fill=fillColor,fill opacity=0.20] (101.39, 59.47) circle (  2.13);

\path[fill=fillColor,fill opacity=0.20] ( 98.33, 48.19) circle (  2.13);

\path[fill=fillColor,fill opacity=0.20] (102.48, 43.79) circle (  2.13);

\path[fill=fillColor,fill opacity=0.20] ( 91.78, 51.80) circle (  2.13);

\path[fill=fillColor,fill opacity=0.20] ( 83.69, 55.42) circle (  2.13);

\path[fill=fillColor,fill opacity=0.20] ( 83.91, 60.59) circle (  2.13);

\path[fill=fillColor,fill opacity=0.20] ( 76.26, 66.44) circle (  2.13);

\path[fill=fillColor,fill opacity=0.20] ( 72.77, 66.96) circle (  2.13);

\path[fill=fillColor,fill opacity=0.20] ( 66.87, 78.15) circle (  2.13);

\path[fill=fillColor,fill opacity=0.20] ( 62.72,100.03) circle (  2.13);

\path[fill=fillColor,fill opacity=0.20] ( 91.34, 65.84) circle (  2.13);

\path[fill=fillColor,fill opacity=0.20] ( 86.31, 71.27) circle (  2.13);

\path[fill=fillColor,fill opacity=0.20] ( 98.11, 63.77) circle (  2.13);

\path[fill=fillColor,fill opacity=0.20] (113.41, 52.15) circle (  2.13);

\path[fill=fillColor,fill opacity=0.20] (116.69, 45.69) circle (  2.13);

\path[fill=fillColor,fill opacity=0.20] (117.34, 43.54) circle (  2.13);

\path[fill=fillColor,fill opacity=0.20] (109.04, 45.43) circle (  2.13);

\path[fill=fillColor,fill opacity=0.20] (106.63, 52.06) circle (  2.13);

\path[fill=fillColor,fill opacity=0.20] (101.61, 56.97) circle (  2.13);

\path[fill=fillColor,fill opacity=0.20] (100.08, 58.69) circle (  2.13);

\path[fill=fillColor,fill opacity=0.20] (111.88, 60.24) circle (  2.13);

\path[fill=fillColor,fill opacity=0.20] (104.67, 62.22) circle (  2.13);

\path[fill=fillColor,fill opacity=0.20] ( 74.73, 74.28) circle (  2.13);

\path[fill=fillColor,fill opacity=0.20] ( 53.98,107.69) circle (  2.13);

\path[fill=fillColor,fill opacity=0.20] ( 98.77, 75.31) circle (  2.13);

\path[fill=fillColor,fill opacity=0.20] (102.92, 61.53) circle (  2.13);

\path[fill=fillColor,fill opacity=0.20] ( 97.02, 58.52) circle (  2.13);

\path[fill=fillColor,fill opacity=0.20] ( 97.68, 57.14) circle (  2.13);

\path[fill=fillColor,fill opacity=0.20] ( 84.78, 61.28) circle (  2.13);

\path[fill=fillColor,fill opacity=0.20] ( 82.82, 68.08) circle (  2.13);

\path[fill=fillColor,fill opacity=0.20] ( 81.51, 68.42) circle (  2.13);

\path[fill=fillColor,fill opacity=0.20] ( 72.33, 68.85) circle (  2.13);

\path[fill=fillColor,fill opacity=0.20] ( 72.33, 70.92) circle (  2.13);

\path[fill=fillColor,fill opacity=0.20] ( 68.18, 69.37) circle (  2.13);

\path[fill=fillColor,fill opacity=0.20] ( 70.36, 78.67) circle (  2.13);

\path[fill=fillColor,fill opacity=0.20] ( 97.24, 54.73) circle (  2.13);

\path[fill=fillColor,fill opacity=0.20] (102.70, 73.07) circle (  2.13);

\path[fill=fillColor,fill opacity=0.20] ( 95.71, 72.64) circle (  2.13);

\path[fill=fillColor,fill opacity=0.20] (112.10, 54.47) circle (  2.13);

\path[fill=fillColor,fill opacity=0.20] (112.10, 53.53) circle (  2.13);

\path[fill=fillColor,fill opacity=0.20] (116.90, 56.88) circle (  2.13);

\path[fill=fillColor,fill opacity=0.20] (114.72, 51.37) circle (  2.13);

\path[fill=fillColor,fill opacity=0.20] (110.79, 45.43) circle (  2.13);

\path[fill=fillColor,fill opacity=0.20] (109.47, 48.10) circle (  2.13);

\path[fill=fillColor,fill opacity=0.20] (100.95, 56.11) circle (  2.13);

\path[fill=fillColor,fill opacity=0.20] ( 98.33, 56.80) circle (  2.13);

\path[fill=fillColor,fill opacity=0.20] ( 97.68, 57.31) circle (  2.13);

\path[fill=fillColor,fill opacity=0.20] ( 70.14, 69.80) circle (  2.13);

\path[fill=fillColor,fill opacity=0.20] ( 91.34, 86.42) circle (  2.13);

\path[fill=fillColor,fill opacity=0.20] (100.30, 77.21) circle (  2.13);

\path[fill=fillColor,fill opacity=0.20] ( 91.34, 62.05) circle (  2.13);

\path[fill=fillColor,fill opacity=0.20] ( 87.41, 58.61) circle (  2.13);

\path[fill=fillColor,fill opacity=0.20] ( 84.78, 68.60) circle (  2.13);

\path[fill=fillColor,fill opacity=0.20] ( 82.60, 67.13) circle (  2.13);

\path[fill=fillColor,fill opacity=0.20] ( 80.41, 63.34) circle (  2.13);

\path[fill=fillColor,fill opacity=0.20] ( 72.77, 64.55) circle (  2.13);

\path[fill=fillColor,fill opacity=0.20] ( 70.80, 60.41) circle (  2.13);

\path[fill=fillColor,fill opacity=0.20] ( 81.73, 59.12) circle (  2.13);

\path[fill=fillColor,fill opacity=0.20] ( 70.58, 66.10) circle (  2.13);

\path[fill=fillColor,fill opacity=0.20] ( 58.35, 86.42) circle (  2.13);

\path[fill=fillColor,fill opacity=0.20] ( 78.01, 56.97) circle (  2.13);

\path[fill=fillColor,fill opacity=0.20] ( 87.19, 55.85) circle (  2.13);

\path[fill=fillColor,fill opacity=0.20] ( 98.33, 62.22) circle (  2.13);

\path[fill=fillColor,fill opacity=0.20] ( 92.65, 63.17) circle (  2.13);

\path[fill=fillColor,fill opacity=0.20] (102.92, 62.22) circle (  2.13);

\path[fill=fillColor,fill opacity=0.20] (105.98, 59.81) circle (  2.13);

\path[fill=fillColor,fill opacity=0.20] (106.42, 62.31) circle (  2.13);

\path[fill=fillColor,fill opacity=0.20] (110.79, 67.65) circle (  2.13);

\path[fill=fillColor,fill opacity=0.20] (111.00, 62.74) circle (  2.13);

\path[fill=fillColor,fill opacity=0.20] (111.66, 52.06) circle (  2.13);

\path[fill=fillColor,fill opacity=0.20] (106.85, 50.25) circle (  2.13);

\path[fill=fillColor,fill opacity=0.20] (100.08, 52.84) circle (  2.13);

\path[fill=fillColor,fill opacity=0.20] ( 85.88, 52.58) circle (  2.13);

\path[fill=fillColor,fill opacity=0.20] ( 71.24, 66.70) circle (  2.13);

\path[fill=fillColor,fill opacity=0.20] ( 68.62, 95.12) circle (  2.13);

\path[fill=fillColor,fill opacity=0.20] ( 85.44, 72.13) circle (  2.13);

\path[fill=fillColor,fill opacity=0.20] ( 94.40, 54.04) circle (  2.13);

\path[fill=fillColor,fill opacity=0.20] ( 92.65, 57.57) circle (  2.13);

\path[fill=fillColor,fill opacity=0.20] ( 85.88, 64.29) circle (  2.13);

\path[fill=fillColor,fill opacity=0.20] ( 84.13, 61.79) circle (  2.13);

\path[fill=fillColor,fill opacity=0.20] ( 76.70, 63.00) circle (  2.13);

\path[fill=fillColor,fill opacity=0.20] ( 70.36, 71.87) circle (  2.13);

\path[fill=fillColor,fill opacity=0.20] ( 69.27, 74.11) circle (  2.13);

\path[fill=fillColor,fill opacity=0.20] ( 73.86, 63.69) circle (  2.13);

\path[fill=fillColor,fill opacity=0.20] ( 71.89, 55.08) circle (  2.13);

\path[fill=fillColor,fill opacity=0.20] ( 72.33, 59.55) circle (  2.13);

\path[fill=fillColor,fill opacity=0.20] ( 64.90, 76.52) circle (  2.13);

\path[fill=fillColor,fill opacity=0.20] ( 67.52, 74.37) circle (  2.13);

\path[fill=fillColor,fill opacity=0.20] ( 76.04, 69.97) circle (  2.13);

\path[fill=fillColor,fill opacity=0.20] ( 85.66, 66.70) circle (  2.13);

\path[fill=fillColor,fill opacity=0.20] ( 86.75, 65.84) circle (  2.13);

\path[fill=fillColor,fill opacity=0.20] (116.90, 67.48) circle (  2.13);

\path[fill=fillColor,fill opacity=0.20] ( 93.09, 65.75) circle (  2.13);

\path[fill=fillColor,fill opacity=0.20] ( 93.96, 55.59) circle (  2.13);

\path[fill=fillColor,fill opacity=0.20] (103.58, 54.90) circle (  2.13);

\path[fill=fillColor,fill opacity=0.20] (105.10, 63.08) circle (  2.13);

\path[fill=fillColor,fill opacity=0.20] (105.54, 66.36) circle (  2.13);

\path[fill=fillColor,fill opacity=0.20] (108.38, 67.22) circle (  2.13);

\path[fill=fillColor,fill opacity=0.20] (109.47, 70.06) circle (  2.13);

\path[fill=fillColor,fill opacity=0.20] ( 99.86, 66.10) circle (  2.13);

\path[fill=fillColor,fill opacity=0.20] ( 94.62, 58.00) circle (  2.13);

\path[fill=fillColor,fill opacity=0.20] ( 80.85, 63.43) circle (  2.13);

\path[fill=fillColor,fill opacity=0.20] ( 67.96, 95.29) circle (  2.13);

\path[fill=fillColor,fill opacity=0.20] ( 78.67, 72.99) circle (  2.13);

\path[fill=fillColor,fill opacity=0.20] ( 91.34, 59.98) circle (  2.13);

\path[fill=fillColor,fill opacity=0.20] ( 93.74, 62.74) circle (  2.13);

\path[fill=fillColor,fill opacity=0.20] ( 87.84, 62.57) circle (  2.13);

\path[fill=fillColor,fill opacity=0.20] ( 89.15, 59.12) circle (  2.13);

\path[fill=fillColor,fill opacity=0.20] ( 80.41, 68.51) circle (  2.13);

\path[fill=fillColor,fill opacity=0.20] ( 77.57, 73.25) circle (  2.13);

\path[fill=fillColor,fill opacity=0.20] ( 75.61, 65.58) circle (  2.13);

\path[fill=fillColor,fill opacity=0.20] ( 75.17, 57.49) circle (  2.13);

\path[fill=fillColor,fill opacity=0.20] ( 75.39, 56.88) circle (  2.13);

\path[fill=fillColor,fill opacity=0.20] ( 77.57, 65.93) circle (  2.13);

\path[fill=fillColor,fill opacity=0.20] ( 70.80, 73.93) circle (  2.13);

\path[fill=fillColor,fill opacity=0.20] ( 70.58, 68.60) circle (  2.13);

\path[fill=fillColor,fill opacity=0.20] ( 65.99, 75.57) circle (  2.13);

\path[fill=fillColor,fill opacity=0.20] ( 66.65, 79.53) circle (  2.13);

\path[fill=fillColor,fill opacity=0.20] ( 66.21, 79.70) circle (  2.13);

\path[fill=fillColor,fill opacity=0.20] ( 90.68, 80.74) circle (  2.13);

\path[fill=fillColor,fill opacity=0.20] ( 85.66, 69.28) circle (  2.13);

\path[fill=fillColor,fill opacity=0.20] ( 86.10, 67.22) circle (  2.13);

\path[fill=fillColor,fill opacity=0.20] ( 95.71, 78.07) circle (  2.13);

\path[fill=fillColor,fill opacity=0.20] ( 89.59, 74.62) circle (  2.13);

\path[fill=fillColor,fill opacity=0.20] ( 97.68, 63.69) circle (  2.13);

\path[fill=fillColor,fill opacity=0.20] ( 90.68, 63.17) circle (  2.13);

\path[fill=fillColor,fill opacity=0.20] (101.83, 58.86) circle (  2.13);

\path[fill=fillColor,fill opacity=0.20] (107.95, 58.26) circle (  2.13);

\path[fill=fillColor,fill opacity=0.20] (119.31, 64.72) circle (  2.13);

\path[fill=fillColor,fill opacity=0.20] (105.98, 63.34) circle (  2.13);

\path[fill=fillColor,fill opacity=0.20] ( 96.80, 64.12) circle (  2.13);

\path[fill=fillColor,fill opacity=0.20] ( 92.21, 74.11) circle (  2.13);

\path[fill=fillColor,fill opacity=0.20] ( 78.67, 80.39) circle (  2.13);

\path[fill=fillColor,fill opacity=0.20] ( 75.61, 78.33) circle (  2.13);

\path[fill=fillColor,fill opacity=0.20] ( 67.30, 85.73) circle (  2.13);

\path[fill=fillColor,fill opacity=0.20] ( 80.63, 79.53) circle (  2.13);

\path[fill=fillColor,fill opacity=0.20] ( 91.34, 69.11) circle (  2.13);

\path[fill=fillColor,fill opacity=0.20] ( 97.89, 53.87) circle (  2.13);

\path[fill=fillColor,fill opacity=0.20] ( 98.11, 50.86) circle (  2.13);

\path[fill=fillColor,fill opacity=0.20] ( 87.84, 59.21) circle (  2.13);

\path[fill=fillColor,fill opacity=0.20] ( 79.10, 59.81) circle (  2.13);

\path[fill=fillColor,fill opacity=0.20] ( 78.01, 62.31) circle (  2.13);

\path[fill=fillColor,fill opacity=0.20] ( 75.83, 67.82) circle (  2.13);

\path[fill=fillColor,fill opacity=0.20] ( 76.70, 68.68) circle (  2.13);

\path[fill=fillColor,fill opacity=0.20] ( 73.20, 70.83) circle (  2.13);

\path[fill=fillColor,fill opacity=0.20] ( 71.46, 71.09) circle (  2.13);

\path[fill=fillColor,fill opacity=0.20] ( 71.67, 66.87) circle (  2.13);

\path[fill=fillColor,fill opacity=0.20] ( 72.33, 65.50) circle (  2.13);

\path[fill=fillColor,fill opacity=0.20] ( 74.95, 67.05) circle (  2.13);

\path[fill=fillColor,fill opacity=0.20] ( 71.46, 69.63) circle (  2.13);

\path[fill=fillColor,fill opacity=0.20] ( 65.99, 76.69) circle (  2.13);

\path[fill=fillColor,fill opacity=0.20] ( 76.92, 67.65) circle (  2.13);

\path[fill=fillColor,fill opacity=0.20] ( 71.24, 58.09) circle (  2.13);

\path[fill=fillColor,fill opacity=0.20] ( 67.30, 71.44) circle (  2.13);

\path[fill=fillColor,fill opacity=0.20] ( 68.62, 84.53) circle (  2.13);

\path[fill=fillColor,fill opacity=0.20] ( 74.30, 78.93) circle (  2.13);

\path[fill=fillColor,fill opacity=0.20] ( 72.11, 75.57) circle (  2.13);

\path[fill=fillColor,fill opacity=0.20] ( 71.46, 80.57) circle (  2.13);

\path[fill=fillColor,fill opacity=0.20] ( 69.71, 80.48) circle (  2.13);

\path[fill=fillColor,fill opacity=0.20] ( 77.57, 72.04) circle (  2.13);

\path[fill=fillColor,fill opacity=0.20] ( 74.08, 67.82) circle (  2.13);

\path[fill=fillColor,fill opacity=0.20] ( 76.26, 70.58) circle (  2.13);

\path[fill=fillColor,fill opacity=0.20] ( 78.23, 79.88) circle (  2.13);

\path[fill=fillColor,fill opacity=0.20] ( 74.95, 83.49) circle (  2.13);

\path[fill=fillColor,fill opacity=0.20] ( 76.70, 77.21) circle (  2.13);

\path[fill=fillColor,fill opacity=0.20] ( 76.26, 72.73) circle (  2.13);

\path[fill=fillColor,fill opacity=0.20] ( 82.82, 69.37) circle (  2.13);

\path[fill=fillColor,fill opacity=0.20] ( 80.20, 67.65) circle (  2.13);

\path[fill=fillColor,fill opacity=0.20] ( 88.94, 74.71) circle (  2.13);

\path[fill=fillColor,fill opacity=0.20] ( 89.81, 75.23) circle (  2.13);

\path[fill=fillColor,fill opacity=0.20] ( 84.13, 68.25) circle (  2.13);

\path[fill=fillColor,fill opacity=0.20] ( 88.72, 69.03) circle (  2.13);

\path[fill=fillColor,fill opacity=0.20] ( 99.42, 66.61) circle (  2.13);

\path[fill=fillColor,fill opacity=0.20] (101.39, 60.93) circle (  2.13);

\path[fill=fillColor,fill opacity=0.20] (100.30, 58.00) circle (  2.13);

\path[fill=fillColor,fill opacity=0.20] ( 97.68, 57.49) circle (  2.13);

\path[fill=fillColor,fill opacity=0.20] ( 97.02, 64.46) circle (  2.13);

\path[fill=fillColor,fill opacity=0.20] ( 88.72, 75.83) circle (  2.13);

\path[fill=fillColor,fill opacity=0.20] ( 80.20, 72.64) circle (  2.13);

\path[fill=fillColor,fill opacity=0.20] ( 89.15, 75.31) circle (  2.13);

\path[fill=fillColor,fill opacity=0.20] ( 74.73, 81.94) circle (  2.13);

\path[fill=fillColor,fill opacity=0.20] ( 73.20, 77.81) circle (  2.13);

\path[fill=fillColor,fill opacity=0.20] ( 74.30, 86.59) circle (  2.13);

\path[fill=fillColor,fill opacity=0.20] ( 82.82, 71.18) circle (  2.13);

\path[fill=fillColor,fill opacity=0.20] ( 90.90, 58.69) circle (  2.13);

\path[fill=fillColor,fill opacity=0.20] ( 91.78, 58.26) circle (  2.13);

\path[fill=fillColor,fill opacity=0.20] ( 87.19, 61.45) circle (  2.13);

\path[fill=fillColor,fill opacity=0.20] ( 82.82, 57.75) circle (  2.13);

\path[fill=fillColor,fill opacity=0.20] ( 79.76, 60.07) circle (  2.13);

\path[fill=fillColor,fill opacity=0.20] ( 74.95, 63.26) circle (  2.13);

\path[fill=fillColor,fill opacity=0.20] ( 74.51, 63.43) circle (  2.13);

\path[fill=fillColor,fill opacity=0.20] ( 74.73, 67.39) circle (  2.13);

\path[fill=fillColor,fill opacity=0.20] ( 71.02, 70.15) circle (  2.13);

\path[fill=fillColor,fill opacity=0.20] ( 75.61, 68.25) circle (  2.13);

\path[fill=fillColor,fill opacity=0.20] ( 82.82, 70.06) circle (  2.13);

\path[fill=fillColor,fill opacity=0.20] ( 73.20, 71.78) circle (  2.13);

\path[fill=fillColor,fill opacity=0.20] ( 76.48, 69.46) circle (  2.13);

\path[fill=fillColor,fill opacity=0.20] ( 74.08, 63.34) circle (  2.13);

\path[fill=fillColor,fill opacity=0.20] ( 76.26, 55.94) circle (  2.13);

\path[fill=fillColor,fill opacity=0.20] ( 75.83, 59.21) circle (  2.13);

\path[fill=fillColor,fill opacity=0.20] ( 79.32, 64.98) circle (  2.13);

\path[fill=fillColor,fill opacity=0.20] ( 72.77, 61.28) circle (  2.13);

\path[fill=fillColor,fill opacity=0.20] ( 79.32, 61.10) circle (  2.13);

\path[fill=fillColor,fill opacity=0.20] ( 77.79, 66.96) circle (  2.13);

\path[fill=fillColor,fill opacity=0.20] ( 79.76, 67.65) circle (  2.13);

\path[fill=fillColor,fill opacity=0.20] ( 85.22, 67.82) circle (  2.13);

\path[fill=fillColor,fill opacity=0.20] ( 82.60, 69.46) circle (  2.13);

\path[fill=fillColor,fill opacity=0.20] ( 83.69, 69.97) circle (  2.13);

\path[fill=fillColor,fill opacity=0.20] ( 82.38, 74.62) circle (  2.13);

\path[fill=fillColor,fill opacity=0.20] ( 79.54, 78.58) circle (  2.13);

\path[fill=fillColor,fill opacity=0.20] ( 86.97, 74.71) circle (  2.13);

\path[fill=fillColor,fill opacity=0.20] ( 89.15, 68.34) circle (  2.13);

\path[fill=fillColor,fill opacity=0.20] ( 92.21, 66.44) circle (  2.13);

\path[fill=fillColor,fill opacity=0.20] ( 95.71, 68.08) circle (  2.13);

\path[fill=fillColor,fill opacity=0.20] ( 90.68, 58.00) circle (  2.13);

\path[fill=fillColor,fill opacity=0.20] ( 96.15, 45.52) circle (  2.13);

\path[fill=fillColor,fill opacity=0.20] ( 90.68, 54.13) circle (  2.13);

\path[fill=fillColor,fill opacity=0.20] ( 90.68, 61.79) circle (  2.13);

\path[fill=fillColor,fill opacity=0.20] ( 95.27, 58.35) circle (  2.13);

\path[fill=fillColor,fill opacity=0.20] ( 87.19, 64.89) circle (  2.13);

\path[fill=fillColor,fill opacity=0.20] ( 82.16, 69.37) circle (  2.13);

\path[fill=fillColor,fill opacity=0.20] ( 81.51, 72.04) circle (  2.13);

\path[fill=fillColor,fill opacity=0.20] ( 75.61, 86.34) circle (  2.13);

\path[fill=fillColor,fill opacity=0.20] ( 67.09, 90.04) circle (  2.13);

\path[fill=fillColor,fill opacity=0.20] ( 83.25, 85.73) circle (  2.13);

\path[fill=fillColor,fill opacity=0.20] ( 61.19, 99.17) circle (  2.13);

\path[fill=fillColor,fill opacity=0.20] ( 76.92, 87.54) circle (  2.13);

\path[fill=fillColor,fill opacity=0.20] ( 79.32, 76.86) circle (  2.13);

\path[fill=fillColor,fill opacity=0.20] ( 85.00, 71.87) circle (  2.13);

\path[fill=fillColor,fill opacity=0.20] ( 90.68, 64.20) circle (  2.13);

\path[fill=fillColor,fill opacity=0.20] ( 86.75, 54.39) circle (  2.13);

\path[fill=fillColor,fill opacity=0.20] ( 85.44, 52.84) circle (  2.13);

\path[fill=fillColor,fill opacity=0.20] ( 80.41, 59.98) circle (  2.13);

\path[fill=fillColor,fill opacity=0.20] ( 76.48, 67.73) circle (  2.13);

\path[fill=fillColor,fill opacity=0.20] ( 78.45, 68.08) circle (  2.13);

\path[fill=fillColor,fill opacity=0.20] ( 76.04, 69.28) circle (  2.13);

\path[fill=fillColor,fill opacity=0.20] ( 72.55, 73.68) circle (  2.13);

\path[fill=fillColor,fill opacity=0.20] ( 75.17, 72.82) circle (  2.13);

\path[fill=fillColor,fill opacity=0.20] ( 77.36, 68.25) circle (  2.13);

\path[fill=fillColor,fill opacity=0.20] ( 73.86, 68.08) circle (  2.13);

\path[fill=fillColor,fill opacity=0.20] ( 76.92, 68.94) circle (  2.13);

\path[fill=fillColor,fill opacity=0.20] ( 78.23, 67.39) circle (  2.13);

\path[fill=fillColor,fill opacity=0.20] ( 71.67, 65.75) circle (  2.13);

\path[fill=fillColor,fill opacity=0.20] ( 74.51, 64.20) circle (  2.13);

\path[fill=fillColor,fill opacity=0.20] ( 78.01, 64.72) circle (  2.13);

\path[fill=fillColor,fill opacity=0.20] ( 77.79, 65.15) circle (  2.13);

\path[fill=fillColor,fill opacity=0.20] ( 78.01, 66.18) circle (  2.13);

\path[fill=fillColor,fill opacity=0.20] ( 74.73, 71.78) circle (  2.13);

\path[fill=fillColor,fill opacity=0.20] ( 74.73, 74.02) circle (  2.13);

\path[fill=fillColor,fill opacity=0.20] ( 77.36, 71.27) circle (  2.13);

\path[fill=fillColor,fill opacity=0.20] ( 80.85, 68.85) circle (  2.13);

\path[fill=fillColor,fill opacity=0.20] ( 83.25, 66.01) circle (  2.13);

\path[fill=fillColor,fill opacity=0.20] ( 88.94, 64.81) circle (  2.13);

\path[fill=fillColor,fill opacity=0.20] ( 88.06, 61.96) circle (  2.13);

\path[fill=fillColor,fill opacity=0.20] ( 98.77, 59.12) circle (  2.13);

\path[fill=fillColor,fill opacity=0.20] ( 96.36, 64.89) circle (  2.13);

\path[fill=fillColor,fill opacity=0.20] ( 98.33, 70.66) circle (  2.13);

\path[fill=fillColor,fill opacity=0.20] ( 81.07, 67.73) circle (  2.13);

\path[fill=fillColor,fill opacity=0.20] ( 85.00, 68.42) circle (  2.13);

\path[fill=fillColor,fill opacity=0.20] ( 76.70, 76.35) circle (  2.13);

\path[fill=fillColor,fill opacity=0.20] ( 70.58, 84.96) circle (  2.13);

\path[fill=fillColor,fill opacity=0.20] ( 54.63,108.38) circle (  2.13);

\path[fill=fillColor,fill opacity=0.20] ( 75.17, 89.18) circle (  2.13);

\path[fill=fillColor,fill opacity=0.20] ( 81.73, 70.32) circle (  2.13);

\path[fill=fillColor,fill opacity=0.20] ( 84.57, 60.33) circle (  2.13);

\path[fill=fillColor,fill opacity=0.20] ( 90.47, 66.27) circle (  2.13);

\path[fill=fillColor,fill opacity=0.20] ( 89.37, 71.61) circle (  2.13);

\path[fill=fillColor,fill opacity=0.20] ( 87.19, 67.48) circle (  2.13);

\path[fill=fillColor,fill opacity=0.20] ( 89.37, 65.32) circle (  2.13);

\path[fill=fillColor,fill opacity=0.20] ( 88.94, 66.87) circle (  2.13);

\path[fill=fillColor,fill opacity=0.20] ( 81.73, 64.38) circle (  2.13);

\path[fill=fillColor,fill opacity=0.20] ( 79.76, 60.50) circle (  2.13);

\path[fill=fillColor,fill opacity=0.20] ( 82.60, 56.37) circle (  2.13);

\path[fill=fillColor,fill opacity=0.20] ( 82.38, 58.09) circle (  2.13);

\path[fill=fillColor,fill opacity=0.20] ( 77.79, 63.86) circle (  2.13);

\path[fill=fillColor,fill opacity=0.20] ( 79.32, 64.46) circle (  2.13);

\path[fill=fillColor,fill opacity=0.20] ( 78.67, 59.21) circle (  2.13);

\path[fill=fillColor,fill opacity=0.20] ( 78.23, 55.08) circle (  2.13);

\path[fill=fillColor,fill opacity=0.20] ( 79.32, 53.01) circle (  2.13);

\path[fill=fillColor,fill opacity=0.20] ( 83.47, 57.40) circle (  2.13);

\path[fill=fillColor,fill opacity=0.20] ( 89.37, 64.55) circle (  2.13);

\path[fill=fillColor,fill opacity=0.20] ( 83.04, 65.24) circle (  2.13);

\path[fill=fillColor,fill opacity=0.20] ( 87.62, 63.95) circle (  2.13);

\path[fill=fillColor,fill opacity=0.20] ( 88.50, 64.81) circle (  2.13);

\path[fill=fillColor,fill opacity=0.20] ( 92.87, 62.91) circle (  2.13);

\path[fill=fillColor,fill opacity=0.20] ( 89.81, 66.27) circle (  2.13);

\path[fill=fillColor,fill opacity=0.20] ( 83.69, 76.95) circle (  2.13);

\path[fill=fillColor,fill opacity=0.20] ( 71.67, 83.41) circle (  2.13);

\path[fill=fillColor,fill opacity=0.20] ( 67.52, 91.24) circle (  2.13);

\path[fill=fillColor,fill opacity=0.20] ( 52.01,112.51) circle (  2.13);

\path[fill=fillColor,fill opacity=0.20] ( 67.52, 92.02) circle (  2.13);

\path[fill=fillColor,fill opacity=0.20] ( 81.07, 86.42) circle (  2.13);

\path[fill=fillColor,fill opacity=0.20] ( 88.28, 80.74) circle (  2.13);

\path[fill=fillColor,fill opacity=0.20] ( 79.32, 78.41) circle (  2.13);

\path[fill=fillColor,fill opacity=0.20] ( 93.96, 69.97) circle (  2.13);

\path[fill=fillColor,fill opacity=0.20] ( 92.65, 57.49) circle (  2.13);

\path[fill=fillColor,fill opacity=0.20] ( 97.89, 61.62) circle (  2.13);

\path[fill=fillColor,fill opacity=0.20] ( 90.68, 69.97) circle (  2.13);

\path[fill=fillColor,fill opacity=0.20] ( 93.96, 58.43) circle (  2.13);

\path[fill=fillColor,fill opacity=0.20] ( 85.22, 50.17) circle (  2.13);

\path[fill=fillColor,fill opacity=0.20] ( 85.44, 61.79) circle (  2.13);

\path[fill=fillColor,fill opacity=0.20] ( 82.82, 67.99) circle (  2.13);

\path[fill=fillColor,fill opacity=0.20] ( 85.22, 64.29) circle (  2.13);

\path[fill=fillColor,fill opacity=0.20] ( 85.88, 62.05) circle (  2.13);

\path[fill=fillColor,fill opacity=0.20] ( 87.62, 62.14) circle (  2.13);

\path[fill=fillColor,fill opacity=0.20] ( 83.91, 66.79) circle (  2.13);

\path[fill=fillColor,fill opacity=0.20] ( 89.37, 72.64) circle (  2.13);

\path[fill=fillColor,fill opacity=0.20] ( 86.97, 76.17) circle (  2.13);

\path[fill=fillColor,fill opacity=0.20] ( 75.17, 85.65) circle (  2.13);

\path[fill=fillColor,fill opacity=0.20] ( 74.30, 92.54) circle (  2.13);

\path[fill=fillColor,fill opacity=0.20] ( 76.26, 90.81) circle (  2.13);

\path[fill=fillColor,fill opacity=0.20] ( 52.01,110.71) circle (  2.13);

\path[fill=fillColor,fill opacity=0.20] ( 57.69,102.35) circle (  2.13);

\path[fill=fillColor,fill opacity=0.20] ( 66.21, 86.42) circle (  2.13);

\path[fill=fillColor,fill opacity=0.20] ( 80.85, 88.75) circle (  2.13);

\path[fill=fillColor,fill opacity=0.20] ( 79.76, 93.83) circle (  2.13);

\path[fill=fillColor,fill opacity=0.20] ( 77.14, 88.23) circle (  2.13);

\path[fill=fillColor,fill opacity=0.20] ( 72.55, 86.16) circle (  2.13);

\path[fill=fillColor,fill opacity=0.20] ( 69.27, 86.25) circle (  2.13);

\path[fill=fillColor,fill opacity=0.20] ( 71.67, 87.28) circle (  2.13);

\path[fill=fillColor,fill opacity=0.20] ( 71.24, 90.12) circle (  2.13);

\path[fill=fillColor,fill opacity=0.20] ( 77.36, 91.33) circle (  2.13);

\path[fill=fillColor,fill opacity=0.20] ( 79.10,111.65) circle (  2.13);

\path[fill=fillColor,fill opacity=0.20] ( 74.95, 88.06) circle (  2.13);

\path[fill=fillColor,fill opacity=0.20] ( 73.20, 82.46) circle (  2.13);

\path[fill=fillColor,fill opacity=0.20] ( 91.99, 99.77) circle (  2.13);

\path[fill=fillColor,fill opacity=0.20] (124.77,115.10) circle (  2.13);

\path[fill=fillColor,fill opacity=0.20] (111.66, 83.23) circle (  2.13);

\path[fill=fillColor,fill opacity=0.20] ( 97.68, 58.78) circle (  2.13);

\path[fill=fillColor,fill opacity=0.20] ( 96.58, 49.56) circle (  2.13);

\path[fill=fillColor,fill opacity=0.20] ( 89.15, 59.21) circle (  2.13);

\path[fill=fillColor,fill opacity=0.20] ( 94.18, 61.10) circle (  2.13);

\path[fill=fillColor,fill opacity=0.20] ( 88.72, 54.82) circle (  2.13);

\path[fill=fillColor,fill opacity=0.20] ( 86.75, 69.20) circle (  2.13);

\path[fill=fillColor,fill opacity=0.20] (102.48,109.84) circle (  2.13);

\path[fill=fillColor,fill opacity=0.20] (130.89, 52.84) circle (  2.13);

\path[fill=fillColor,fill opacity=0.20] (112.10, 56.63) circle (  2.13);

\path[fill=fillColor,fill opacity=0.20] (109.91, 46.89) circle (  2.13);

\path[fill=fillColor,fill opacity=0.20] (118.21, 44.48) circle (  2.13);

\path[fill=fillColor,fill opacity=0.20] (108.82, 58.95) circle (  2.13);

\path[fill=fillColor,fill opacity=0.20] (108.38, 56.45) circle (  2.13);

\path[fill=fillColor,fill opacity=0.20] (109.26, 46.64) circle (  2.13);

\path[fill=fillColor,fill opacity=0.20] ( 90.25, 58.09) circle (  2.13);

\path[fill=fillColor,fill opacity=0.20] ( 62.28, 88.75) circle (  2.13);

\path[fill=fillColor,fill opacity=0.20] (128.48, 62.74) circle (  2.13);

\path[fill=fillColor,fill opacity=0.20] (106.20, 48.70) circle (  2.13);

\path[fill=fillColor,fill opacity=0.20] (101.83, 66.96) circle (  2.13);

\path[fill=fillColor,fill opacity=0.20] (124.99, 54.56) circle (  2.13);

\path[fill=fillColor,fill opacity=0.20] (122.80, 52.15) circle (  2.13);

\path[fill=fillColor,fill opacity=0.20] (114.28, 59.04) circle (  2.13);

\path[fill=fillColor,fill opacity=0.20] (119.96, 57.66) circle (  2.13);

\path[fill=fillColor,fill opacity=0.20] (130.67, 51.37) circle (  2.13);

\path[fill=fillColor,fill opacity=0.20] (114.94, 55.25) circle (  2.13);

\path[fill=fillColor,fill opacity=0.20] ( 75.39, 73.07) circle (  2.13);

\path[fill=fillColor,fill opacity=0.20] (124.99, 46.89) circle (  2.13);

\path[fill=fillColor,fill opacity=0.20] (106.20, 52.49) circle (  2.13);

\path[fill=fillColor,fill opacity=0.20] (116.90, 60.07) circle (  2.13);

\path[fill=fillColor,fill opacity=0.20] (146.40, 52.41) circle (  2.13);

\path[fill=fillColor,fill opacity=0.20] (136.13, 50.86) circle (  2.13);

\path[fill=fillColor,fill opacity=0.20] (113.63, 44.31) circle (  2.13);

\path[fill=fillColor,fill opacity=0.20] (122.15, 45.09) circle (  2.13);

\path[fill=fillColor,fill opacity=0.20] (139.41, 53.78) circle (  2.13);

\path[fill=fillColor,fill opacity=0.20] (125.43, 51.11) circle (  2.13);

\path[fill=fillColor,fill opacity=0.20] ( 85.00, 59.81) circle (  2.13);

\path[fill=fillColor,fill opacity=0.20] ( 45.89,108.12) circle (  2.13);

\path[fill=fillColor,fill opacity=0.20] ( 45.45,102.96) circle (  2.13);

\path[fill=fillColor,fill opacity=0.20] ( 52.23,106.06) circle (  2.13);

\path[fill=fillColor,fill opacity=0.20] ( 59.22,114.24) circle (  2.13);

\path[fill=fillColor,fill opacity=0.20] (112.53, 51.63) circle (  2.13);

\path[fill=fillColor,fill opacity=0.20] (120.40, 40.44) circle (  2.13);

\path[fill=fillColor,fill opacity=0.20] (123.90, 44.57) circle (  2.13);

\path[fill=fillColor,fill opacity=0.20] (125.21, 52.58) circle (  2.13);

\path[fill=fillColor,fill opacity=0.20] (126.08, 52.92) circle (  2.13);

\path[fill=fillColor,fill opacity=0.20] (125.43, 38.37) circle (  2.13);

\path[fill=fillColor,fill opacity=0.20] (135.04, 48.62) circle (  2.13);

\path[fill=fillColor,fill opacity=0.20] (119.31, 51.29) circle (  2.13);

\path[fill=fillColor,fill opacity=0.20] ( 59.00,113.37) circle (  2.13);

\path[fill=fillColor,fill opacity=0.20] ( 71.02, 76.95) circle (  2.13);

\path[fill=fillColor,fill opacity=0.20] ( 79.98, 76.60) circle (  2.13);

\path[fill=fillColor,fill opacity=0.20] ( 92.21, 83.75) circle (  2.13);

\path[fill=fillColor,fill opacity=0.20] ( 93.52, 71.52) circle (  2.13);

\path[fill=fillColor,fill opacity=0.20] ( 94.18, 68.16) circle (  2.13);

\path[fill=fillColor,fill opacity=0.20] ( 74.73, 82.55) circle (  2.13);

\path[fill=fillColor,fill opacity=0.20] ( 79.10, 79.79) circle (  2.13);

\path[fill=fillColor,fill opacity=0.20] ( 76.26, 80.31) circle (  2.13);

\path[fill=fillColor,fill opacity=0.20] ( 63.37,107.00) circle (  2.13);

\path[fill=fillColor,fill opacity=0.20] ( 93.09, 63.60) circle (  2.13);

\path[fill=fillColor,fill opacity=0.20] (104.89, 50.86) circle (  2.13);

\path[fill=fillColor,fill opacity=0.20] (114.06, 44.91) circle (  2.13);

\path[fill=fillColor,fill opacity=0.20] (118.87, 51.03) circle (  2.13);

\path[fill=fillColor,fill opacity=0.20] (126.95, 52.66) circle (  2.13);

\path[fill=fillColor,fill opacity=0.20] (134.38, 46.89) circle (  2.13);

\path[fill=fillColor,fill opacity=0.20] (138.32, 45.00) circle (  2.13);

\path[fill=fillColor,fill opacity=0.20] (144.22, 44.66) circle (  2.13);

\path[fill=fillColor,fill opacity=0.20] (122.37, 51.54) circle (  2.13);

\path[fill=fillColor,fill opacity=0.20] ( 84.57,102.18) circle (  2.13);

\path[fill=fillColor,fill opacity=0.20] (104.23, 68.94) circle (  2.13);

\path[fill=fillColor,fill opacity=0.20] (111.44, 56.63) circle (  2.13);

\path[fill=fillColor,fill opacity=0.20] ( 97.68, 49.22) circle (  2.13);

\path[fill=fillColor,fill opacity=0.20] ( 84.35, 51.29) circle (  2.13);

\path[fill=fillColor,fill opacity=0.20] ( 80.85, 57.92) circle (  2.13);

\path[fill=fillColor,fill opacity=0.20] ( 78.45, 71.44) circle (  2.13);

\path[fill=fillColor,fill opacity=0.20] ( 72.11, 87.89) circle (  2.13);

\path[fill=fillColor,fill opacity=0.20] ( 73.20, 90.30) circle (  2.13);

\path[fill=fillColor,fill opacity=0.20] ( 87.41, 76.09) circle (  2.13);

\path[fill=fillColor,fill opacity=0.20] ( 92.21, 58.35) circle (  2.13);

\path[fill=fillColor,fill opacity=0.20] ( 97.02, 59.98) circle (  2.13);

\path[fill=fillColor,fill opacity=0.20] (106.42, 48.96) circle (  2.13);

\path[fill=fillColor,fill opacity=0.20] (126.08, 47.07) circle (  2.13);

\path[fill=fillColor,fill opacity=0.20] (135.04, 51.80) circle (  2.13);

\path[fill=fillColor,fill opacity=0.20] (142.91, 45.09) circle (  2.13);

\path[fill=fillColor,fill opacity=0.20] (104.67, 47.07) circle (  2.13);

\path[fill=fillColor,fill opacity=0.20] (112.53, 45.52) circle (  2.13);

\path[fill=fillColor,fill opacity=0.20] (111.44, 57.23) circle (  2.13);

\path[fill=fillColor,fill opacity=0.20] (106.42, 59.90) circle (  2.13);

\path[fill=fillColor,fill opacity=0.20] ( 94.62, 59.55) circle (  2.13);

\path[fill=fillColor,fill opacity=0.20] ( 89.15, 63.34) circle (  2.13);

\path[fill=fillColor,fill opacity=0.20] ( 78.01, 65.58) circle (  2.13);

\path[fill=fillColor,fill opacity=0.20] ( 82.82, 71.18) circle (  2.13);

\path[fill=fillColor,fill opacity=0.20] ( 80.85, 79.45) circle (  2.13);

\path[fill=fillColor,fill opacity=0.20] (109.26, 49.65) circle (  2.13);

\path[fill=fillColor,fill opacity=0.20] (106.42, 40.01) circle (  2.13);

\path[fill=fillColor,fill opacity=0.20] (107.95, 47.67) circle (  2.13);

\path[fill=fillColor,fill opacity=0.20] (114.28, 53.01) circle (  2.13);

\path[fill=fillColor,fill opacity=0.20] (135.69, 44.05) circle (  2.13);

\path[fill=fillColor,fill opacity=0.20] (135.04, 49.31) circle (  2.13);

\path[fill=fillColor,fill opacity=0.20] (119.96, 46.46) circle (  2.13);

\path[fill=fillColor,fill opacity=0.20] ( 88.94, 49.74) circle (  2.13);

\path[fill=fillColor,fill opacity=0.20] ( 85.00, 82.12) circle (  2.13);

\path[fill=fillColor,fill opacity=0.20] (109.47, 40.78) circle (  2.13);

\path[fill=fillColor,fill opacity=0.20] (107.29, 51.03) circle (  2.13);

\path[fill=fillColor,fill opacity=0.20] (105.10, 60.93) circle (  2.13);

\path[fill=fillColor,fill opacity=0.20] (104.23, 53.78) circle (  2.13);

\path[fill=fillColor,fill opacity=0.20] (101.17, 57.92) circle (  2.13);

\path[fill=fillColor,fill opacity=0.20] ( 87.19, 62.57) circle (  2.13);

\path[fill=fillColor,fill opacity=0.20] ( 86.10, 55.51) circle (  2.13);

\path[fill=fillColor,fill opacity=0.20] ( 92.87, 54.47) circle (  2.13);

\path[fill=fillColor,fill opacity=0.20] ( 89.81, 63.60) circle (  2.13);

\path[fill=fillColor,fill opacity=0.20] ( 73.42, 86.08) circle (  2.13);

\path[fill=fillColor,fill opacity=0.20] ( 88.28, 53.35) circle (  2.13);

\path[fill=fillColor,fill opacity=0.20] (105.32, 54.64) circle (  2.13);

\path[fill=fillColor,fill opacity=0.20] (114.28, 42.24) circle (  2.13);

\path[fill=fillColor,fill opacity=0.20] (113.84, 48.10) circle (  2.13);

\path[fill=fillColor,fill opacity=0.20] (126.08, 59.55) circle (  2.13);

\path[fill=fillColor,fill opacity=0.20] (119.31, 54.90) circle (  2.13);

\path[fill=fillColor,fill opacity=0.20] (119.96, 51.46) circle (  2.13);

\path[fill=fillColor,fill opacity=0.20] (137.01, 52.92) circle (  2.13);

\path[fill=fillColor,fill opacity=0.20] (131.32, 52.58) circle (  2.13);

\path[fill=fillColor,fill opacity=0.20] ( 94.40, 50.34) circle (  2.13);

\path[fill=fillColor,fill opacity=0.20] ( 95.49, 51.11) circle (  2.13);

\path[fill=fillColor,fill opacity=0.20] (138.97, 52.06) circle (  2.13);

\path[fill=fillColor,fill opacity=0.20] (104.45, 48.19) circle (  2.13);

\path[fill=fillColor,fill opacity=0.20] (110.35, 40.78) circle (  2.13);

\path[fill=fillColor,fill opacity=0.20] ( 98.77, 55.33) circle (  2.13);

\path[fill=fillColor,fill opacity=0.20] ( 91.34, 59.98) circle (  2.13);

\path[fill=fillColor,fill opacity=0.20] ( 93.52, 54.64) circle (  2.13);

\path[fill=fillColor,fill opacity=0.20] ( 87.62, 57.83) circle (  2.13);

\path[fill=fillColor,fill opacity=0.20] ( 69.27,102.87) circle (  2.13);

\path[fill=fillColor,fill opacity=0.20] ( 89.15, 57.66) circle (  2.13);

\path[fill=fillColor,fill opacity=0.20] ( 95.05, 66.79) circle (  2.13);

\path[fill=fillColor,fill opacity=0.20] (127.83, 62.65) circle (  2.13);

\path[fill=fillColor,fill opacity=0.20] (118.21, 52.66) circle (  2.13);

\path[fill=fillColor,fill opacity=0.20] (118.43, 53.44) circle (  2.13);

\path[fill=fillColor,fill opacity=0.20] (137.44, 51.98) circle (  2.13);

\path[fill=fillColor,fill opacity=0.20] (138.54, 52.58) circle (  2.13);

\path[fill=fillColor,fill opacity=0.20] (119.31, 50.51) circle (  2.13);

\path[fill=fillColor,fill opacity=0.20] ( 72.77,112.51) circle (  2.13);

\path[fill=fillColor,fill opacity=0.20] ( 98.55, 54.73) circle (  2.13);

\path[fill=fillColor,fill opacity=0.20] (121.49, 38.71) circle (  2.13);

\path[fill=fillColor,fill opacity=0.20] (124.77, 40.01) circle (  2.13);

\path[fill=fillColor,fill opacity=0.20] ( 96.58, 51.63) circle (  2.13);

\path[fill=fillColor,fill opacity=0.20] ( 83.25, 57.31) circle (  2.13);

\path[fill=fillColor,fill opacity=0.20] ( 80.41, 58.26) circle (  2.13);

\path[fill=fillColor,fill opacity=0.20] ( 76.70, 66.79) circle (  2.13);

\path[fill=fillColor,fill opacity=0.20] ( 74.73,106.31) circle (  2.13);

\path[fill=fillColor,fill opacity=0.20] ( 81.07, 63.95) circle (  2.13);

\path[fill=fillColor,fill opacity=0.20] ( 89.37, 68.42) circle (  2.13);

\path[fill=fillColor,fill opacity=0.20] (104.67, 65.50) circle (  2.13);

\path[fill=fillColor,fill opacity=0.20] (111.00, 56.63) circle (  2.13);

\path[fill=fillColor,fill opacity=0.20] (108.38, 53.87) circle (  2.13);

\path[fill=fillColor,fill opacity=0.20] (123.24, 47.67) circle (  2.13);

\path[fill=fillColor,fill opacity=0.20] (147.28, 52.06) circle (  2.13);

\path[fill=fillColor,fill opacity=0.20] (138.97, 45.34) circle (  2.13);

\path[fill=fillColor,fill opacity=0.20] ( 93.74, 39.92) circle (  2.13);

\path[fill=fillColor,fill opacity=0.20] ( 82.60, 70.66) circle (  2.13);

\path[fill=fillColor,fill opacity=0.20] (122.80, 50.08) circle (  2.13);

\path[fill=fillColor,fill opacity=0.20] (131.76, 40.87) circle (  2.13);

\path[fill=fillColor,fill opacity=0.20] ( 93.96, 47.33) circle (  2.13);

\path[fill=fillColor,fill opacity=0.20] ( 80.63, 53.27) circle (  2.13);

\path[fill=fillColor,fill opacity=0.20] ( 75.17, 57.66) circle (  2.13);

\path[fill=fillColor,fill opacity=0.20] ( 72.99, 60.07) circle (  2.13);

\path[fill=fillColor,fill opacity=0.20] ( 62.93,107.09) circle (  2.13);

\path[fill=fillColor,fill opacity=0.20] ( 88.50, 65.41) circle (  2.13);

\path[fill=fillColor,fill opacity=0.20] ( 89.15, 74.45) circle (  2.13);

\path[fill=fillColor,fill opacity=0.20] ( 95.05, 71.95) circle (  2.13);

\path[fill=fillColor,fill opacity=0.20] (108.60, 56.80) circle (  2.13);

\path[fill=fillColor,fill opacity=0.20] (100.08, 53.44) circle (  2.13);

\path[fill=fillColor,fill opacity=0.20] (103.58, 54.47) circle (  2.13);

\path[fill=fillColor,fill opacity=0.20] (121.27, 48.19) circle (  2.13);

\path[fill=fillColor,fill opacity=0.20] (123.90, 39.75) circle (  2.13);

\path[fill=fillColor,fill opacity=0.20] (125.64, 53.35) circle (  2.13);

\path[fill=fillColor,fill opacity=0.20] ( 99.42, 54.04) circle (  2.13);

\path[fill=fillColor,fill opacity=0.20] ( 74.30, 86.42) circle (  2.13);

\path[fill=fillColor,fill opacity=0.20] (107.73, 54.30) circle (  2.13);

\path[fill=fillColor,fill opacity=0.20] (135.91, 48.44) circle (  2.13);

\path[fill=fillColor,fill opacity=0.20] (123.02, 52.58) circle (  2.13);

\path[fill=fillColor,fill opacity=0.20] (102.92, 51.03) circle (  2.13);

\path[fill=fillColor,fill opacity=0.20] (100.30, 49.05) circle (  2.13);

\path[fill=fillColor,fill opacity=0.20] ( 88.50, 55.59) circle (  2.13);

\path[fill=fillColor,fill opacity=0.20] ( 89.37, 57.92) circle (  2.13);

\path[fill=fillColor,fill opacity=0.20] ( 80.41, 54.21) circle (  2.13);

\path[fill=fillColor,fill opacity=0.20] ( 61.40, 77.12) circle (  2.13);

\path[fill=fillColor,fill opacity=0.20] ( 63.15,100.03) circle (  2.13);

\path[fill=fillColor,fill opacity=0.20] ( 78.88, 68.68) circle (  2.13);

\path[fill=fillColor,fill opacity=0.20] ( 88.72, 69.03) circle (  2.13);

\path[fill=fillColor,fill opacity=0.20] ( 99.64, 63.95) circle (  2.13);

\path[fill=fillColor,fill opacity=0.20] (112.75, 63.08) circle (  2.13);

\path[fill=fillColor,fill opacity=0.20] (118.65, 57.40) circle (  2.13);

\path[fill=fillColor,fill opacity=0.20] ( 99.42, 51.89) circle (  2.13);

\path[fill=fillColor,fill opacity=0.20] (102.92, 52.32) circle (  2.13);

\path[fill=fillColor,fill opacity=0.20] (120.84, 51.89) circle (  2.13);

\path[fill=fillColor,fill opacity=0.20] (114.06, 52.49) circle (  2.13);

\path[fill=fillColor,fill opacity=0.20] (107.73, 55.59) circle (  2.13);

\path[fill=fillColor,fill opacity=0.20] ( 97.24, 56.37) circle (  2.13);

\path[fill=fillColor,fill opacity=0.20] ( 81.73, 66.61) circle (  2.13);

\path[fill=fillColor,fill opacity=0.20] (108.82, 53.70) circle (  2.13);

\path[fill=fillColor,fill opacity=0.20] (116.90, 52.23) circle (  2.13);

\path[fill=fillColor,fill opacity=0.20] (114.50, 45.86) circle (  2.13);

\path[fill=fillColor,fill opacity=0.20] (109.26, 51.20) circle (  2.13);

\path[fill=fillColor,fill opacity=0.20] (108.82, 59.90) circle (  2.13);

\path[fill=fillColor,fill opacity=0.20] (102.92, 57.06) circle (  2.13);

\path[fill=fillColor,fill opacity=0.20] ( 88.50, 60.50) circle (  2.13);

\path[fill=fillColor,fill opacity=0.20] ( 73.42, 75.40) circle (  2.13);

\path[fill=fillColor,fill opacity=0.20] ( 63.37, 97.27) circle (  2.13);

\path[fill=fillColor,fill opacity=0.20] ( 82.38, 64.89) circle (  2.13);

\path[fill=fillColor,fill opacity=0.20] ( 94.40, 73.33) circle (  2.13);

\path[fill=fillColor,fill opacity=0.20] ( 95.27, 64.12) circle (  2.13);

\path[fill=fillColor,fill opacity=0.20] (102.70, 50.25) circle (  2.13);

\path[fill=fillColor,fill opacity=0.20] (116.47, 54.39) circle (  2.13);

\path[fill=fillColor,fill opacity=0.20] (116.90, 48.88) circle (  2.13);

\path[fill=fillColor,fill opacity=0.20] (109.04, 40.09) circle (  2.13);

\path[fill=fillColor,fill opacity=0.20] (112.97, 42.93) circle (  2.13);

\path[fill=fillColor,fill opacity=0.20] (105.54, 49.91) circle (  2.13);

\path[fill=fillColor,fill opacity=0.20] ( 94.62, 57.66) circle (  2.13);

\path[fill=fillColor,fill opacity=0.20] ( 81.29, 61.19) circle (  2.13);

\path[fill=fillColor,fill opacity=0.20] ( 66.21, 88.57) circle (  2.13);

\path[fill=fillColor,fill opacity=0.20] ( 91.34, 74.80) circle (  2.13);

\path[fill=fillColor,fill opacity=0.20] (113.41, 52.84) circle (  2.13);

\path[fill=fillColor,fill opacity=0.20] (111.22, 38.80) circle (  2.13);

\path[fill=fillColor,fill opacity=0.20] (111.66, 51.54) circle (  2.13);

\path[fill=fillColor,fill opacity=0.20] (108.82, 56.97) circle (  2.13);

\path[fill=fillColor,fill opacity=0.20] (104.01, 49.91) circle (  2.13);

\path[fill=fillColor,fill opacity=0.20] ( 94.84, 58.18) circle (  2.13);

\path[fill=fillColor,fill opacity=0.20] ( 80.85, 70.66) circle (  2.13);

\path[fill=fillColor,fill opacity=0.20] ( 69.05, 79.27) circle (  2.13);

\path[fill=fillColor,fill opacity=0.20] ( 50.48,115.10) circle (  2.13);

\path[fill=fillColor,fill opacity=0.20] ( 70.80, 92.02) circle (  2.13);

\path[fill=fillColor,fill opacity=0.20] ( 78.23, 72.90) circle (  2.13);

\path[fill=fillColor,fill opacity=0.20] ( 87.62, 74.88) circle (  2.13);

\path[fill=fillColor,fill opacity=0.20] ( 99.42, 70.66) circle (  2.13);

\path[fill=fillColor,fill opacity=0.20] (108.38, 60.76) circle (  2.13);

\path[fill=fillColor,fill opacity=0.20] (114.28, 47.41) circle (  2.13);

\path[fill=fillColor,fill opacity=0.20] (110.13, 40.09) circle (  2.13);

\path[fill=fillColor,fill opacity=0.20] (104.45, 46.46) circle (  2.13);

\path[fill=fillColor,fill opacity=0.20] ( 77.57, 51.20) circle (  2.13);

\path[fill=fillColor,fill opacity=0.20] ( 77.14, 61.53) circle (  2.13);

\path[fill=fillColor,fill opacity=0.20] ( 83.47, 72.13) circle (  2.13);

\path[fill=fillColor,fill opacity=0.20] ( 80.85, 93.22) circle (  2.13);

\path[fill=fillColor,fill opacity=0.20] (124.77, 52.23) circle (  2.13);

\path[fill=fillColor,fill opacity=0.20] (112.10, 55.33) circle (  2.13);

\path[fill=fillColor,fill opacity=0.20] (107.51, 58.52) circle (  2.13);

\path[fill=fillColor,fill opacity=0.20] ( 97.02, 48.27) circle (  2.13);

\path[fill=fillColor,fill opacity=0.20] ( 96.36, 54.99) circle (  2.13);

\path[fill=fillColor,fill opacity=0.20] ( 88.72, 68.51) circle (  2.13);

\path[fill=fillColor,fill opacity=0.20] ( 83.47, 68.77) circle (  2.13);

\path[fill=fillColor,fill opacity=0.20] ( 79.98, 74.80) circle (  2.13);

\path[fill=fillColor,fill opacity=0.20] ( 57.91,107.17) circle (  2.13);

\path[fill=fillColor,fill opacity=0.20] ( 77.36,105.45) circle (  2.13);

\path[fill=fillColor,fill opacity=0.20] ( 74.73, 79.70) circle (  2.13);

\path[fill=fillColor,fill opacity=0.20] ( 93.96, 80.65) circle (  2.13);

\path[fill=fillColor,fill opacity=0.20] ( 95.93, 82.98) circle (  2.13);

\path[fill=fillColor,fill opacity=0.20] (103.14, 65.75) circle (  2.13);

\path[fill=fillColor,fill opacity=0.20] (110.79, 50.68) circle (  2.13);

\path[fill=fillColor,fill opacity=0.20] (117.78, 43.11) circle (  2.13);

\path[fill=fillColor,fill opacity=0.20] (102.05, 55.16) circle (  2.13);

\path[fill=fillColor,fill opacity=0.20] (102.26, 45.78) circle (  2.13);

\path[fill=fillColor,fill opacity=0.20] (134.82, 60.67) circle (  2.13);

\path[fill=fillColor,fill opacity=0.20] (103.79, 53.53) circle (  2.13);

\path[fill=fillColor,fill opacity=0.20] ( 96.58, 57.06) circle (  2.13);

\path[fill=fillColor,fill opacity=0.20] ( 92.65, 59.98) circle (  2.13);

\path[fill=fillColor,fill opacity=0.20] ( 95.05, 62.14) circle (  2.13);

\path[fill=fillColor,fill opacity=0.20] ( 84.35, 63.34) circle (  2.13);

\path[fill=fillColor,fill opacity=0.20] ( 75.39, 66.27) circle (  2.13);

\path[fill=fillColor,fill opacity=0.20] ( 63.37, 98.22) circle (  2.13);

\path[fill=fillColor,fill opacity=0.20] ( 64.90,106.40) circle (  2.13);

\path[fill=fillColor,fill opacity=0.20] ( 78.88, 85.73) circle (  2.13);

\path[fill=fillColor,fill opacity=0.20] ( 95.05, 75.31) circle (  2.13);

\path[fill=fillColor,fill opacity=0.20] ( 95.93, 82.72) circle (  2.13);

\path[fill=fillColor,fill opacity=0.20] (106.42, 80.13) circle (  2.13);

\path[fill=fillColor,fill opacity=0.20] (115.37, 62.57) circle (  2.13);

\path[fill=fillColor,fill opacity=0.20] (127.83, 51.03) circle (  2.13);

\path[fill=fillColor,fill opacity=0.20] ( 83.25, 43.28) circle (  2.13);

\path[fill=fillColor,fill opacity=0.20] ( 87.84, 70.23) circle (  2.13);

\path[fill=fillColor,fill opacity=0.20] (102.05, 57.40) circle (  2.13);

\path[fill=fillColor,fill opacity=0.20] (107.51, 53.09) circle (  2.13);

\path[fill=fillColor,fill opacity=0.20] (100.73, 58.43) circle (  2.13);

\path[fill=fillColor,fill opacity=0.20] ( 97.46, 61.71) circle (  2.13);

\path[fill=fillColor,fill opacity=0.20] ( 88.28, 66.96) circle (  2.13);

\path[fill=fillColor,fill opacity=0.20] ( 87.41, 72.90) circle (  2.13);

\path[fill=fillColor,fill opacity=0.20] ( 82.82, 66.70) circle (  2.13);

\path[fill=fillColor,fill opacity=0.20] ( 79.98, 67.99) circle (  2.13);

\path[fill=fillColor,fill opacity=0.20] ( 64.68, 91.59) circle (  2.13);

\path[fill=fillColor,fill opacity=0.20] ( 64.90,100.63) circle (  2.13);

\path[fill=fillColor,fill opacity=0.20] ( 65.12,101.23) circle (  2.13);

\path[fill=fillColor,fill opacity=0.20] ( 67.52, 96.32) circle (  2.13);

\path[fill=fillColor,fill opacity=0.20] ( 80.20, 83.06) circle (  2.13);

\path[fill=fillColor,fill opacity=0.20] ( 89.81, 75.66) circle (  2.13);

\path[fill=fillColor,fill opacity=0.20] ( 95.27, 73.68) circle (  2.13);

\path[fill=fillColor,fill opacity=0.20] (106.63, 73.93) circle (  2.13);

\path[fill=fillColor,fill opacity=0.20] ( 97.24, 66.18) circle (  2.13);

\path[fill=fillColor,fill opacity=0.20] ( 91.34, 57.14) circle (  2.13);

\path[fill=fillColor,fill opacity=0.20] ( 96.36, 54.39) circle (  2.13);

\path[fill=fillColor,fill opacity=0.20] (101.17, 59.98) circle (  2.13);

\path[fill=fillColor,fill opacity=0.20] ( 93.96, 66.27) circle (  2.13);

\path[fill=fillColor,fill opacity=0.20] ( 89.15, 73.16) circle (  2.13);

\path[fill=fillColor,fill opacity=0.20] ( 87.62, 71.70) circle (  2.13);

\path[fill=fillColor,fill opacity=0.20] (124.11, 58.43) circle (  2.13);

\path[fill=fillColor,fill opacity=0.20] ( 83.91, 60.16) circle (  2.13);

\path[fill=fillColor,fill opacity=0.20] ( 78.67, 77.98) circle (  2.13);

\path[fill=fillColor,fill opacity=0.20] ( 63.15, 95.12) circle (  2.13);

\path[fill=fillColor,fill opacity=0.20] ( 55.72,102.87) circle (  2.13);

\path[fill=fillColor,fill opacity=0.20] ( 46.98,114.24) circle (  2.13);

\path[fill=fillColor,fill opacity=0.20] ( 64.46,104.76) circle (  2.13);

\path[fill=fillColor,fill opacity=0.20] ( 78.01, 85.56) circle (  2.13);

\path[fill=fillColor,fill opacity=0.20] ( 78.67, 92.88) circle (  2.13);

\path[fill=fillColor,fill opacity=0.20] ( 85.44, 84.27) circle (  2.13);

\path[fill=fillColor,fill opacity=0.20] ( 95.93, 74.02) circle (  2.13);

\path[fill=fillColor,fill opacity=0.20] (103.58, 79.88) circle (  2.13);

\path[fill=fillColor,fill opacity=0.20] (115.37, 75.92) circle (  2.13);

\path[fill=fillColor,fill opacity=0.20] ( 81.94, 57.49) circle (  2.13);

\path[fill=fillColor,fill opacity=0.20] ( 79.10, 70.06) circle (  2.13);

\path[fill=fillColor,fill opacity=0.20] ( 91.78, 52.32) circle (  2.13);

\path[fill=fillColor,fill opacity=0.20] ( 93.96, 48.10) circle (  2.13);

\path[fill=fillColor,fill opacity=0.20] ( 92.65, 61.19) circle (  2.13);

\path[fill=fillColor,fill opacity=0.20] ( 92.43, 63.60) circle (  2.13);

\path[fill=fillColor,fill opacity=0.20] ( 88.72, 52.41) circle (  2.13);

\path[fill=fillColor,fill opacity=0.20] ( 83.47, 56.80) circle (  2.13);

\path[fill=fillColor,fill opacity=0.20] ( 86.31, 68.60) circle (  2.13);

\path[fill=fillColor,fill opacity=0.20] ( 81.51, 72.04) circle (  2.13);

\path[fill=fillColor,fill opacity=0.20] ( 71.89, 76.26) circle (  2.13);

\path[fill=fillColor,fill opacity=0.20] ( 70.58, 77.90) circle (  2.13);

\path[fill=fillColor,fill opacity=0.20] ( 63.37, 89.26) circle (  2.13);

\path[fill=fillColor,fill opacity=0.20] ( 55.94,114.24) circle (  2.13);

\path[fill=fillColor,fill opacity=0.20] ( 74.51,100.72) circle (  2.13);

\path[fill=fillColor,fill opacity=0.20] ( 84.78, 76.00) circle (  2.13);

\path[fill=fillColor,fill opacity=0.20] ( 97.89, 79.45) circle (  2.13);

\path[fill=fillColor,fill opacity=0.20] ( 99.86, 81.77) circle (  2.13);

\path[fill=fillColor,fill opacity=0.20] (108.60, 73.25) circle (  2.13);

\path[fill=fillColor,fill opacity=0.20] (118.21, 71.09) circle (  2.13);

\path[fill=fillColor,fill opacity=0.20] (102.92, 61.45) circle (  2.13);

\path[fill=fillColor,fill opacity=0.20] ( 97.46, 48.96) circle (  2.13);

\path[fill=fillColor,fill opacity=0.20] ( 98.33, 56.63) circle (  2.13);

\path[fill=fillColor,fill opacity=0.20] ( 90.68, 62.83) circle (  2.13);

\path[fill=fillColor,fill opacity=0.20] ( 87.62, 64.03) circle (  2.13);

\path[fill=fillColor,fill opacity=0.20] ( 87.62, 69.03) circle (  2.13);

\path[fill=fillColor,fill opacity=0.20] ( 78.23, 68.60) circle (  2.13);

\path[fill=fillColor,fill opacity=0.20] ( 75.83, 65.15) circle (  2.13);

\path[fill=fillColor,fill opacity=0.20] ( 78.45, 68.25) circle (  2.13);

\path[fill=fillColor,fill opacity=0.20] ( 79.10, 69.03) circle (  2.13);

\path[fill=fillColor,fill opacity=0.20] ( 70.58, 66.87) circle (  2.13);

\path[fill=fillColor,fill opacity=0.20] ( 73.86, 73.68) circle (  2.13);

\path[fill=fillColor,fill opacity=0.20] ( 70.80, 90.81) circle (  2.13);

\path[fill=fillColor,fill opacity=0.20] ( 63.37,108.81) circle (  2.13);

\path[fill=fillColor,fill opacity=0.20] ( 97.02,114.24) circle (  2.13);

\path[fill=fillColor,fill opacity=0.20] ( 61.62,111.65) circle (  2.13);

\path[fill=fillColor,fill opacity=0.20] ( 62.72, 99.94) circle (  2.13);

\path[fill=fillColor,fill opacity=0.20] ( 65.77,106.92) circle (  2.13);

\path[fill=fillColor,fill opacity=0.20] ( 69.27,108.72) circle (  2.13);

\path[fill=fillColor,fill opacity=0.20] ( 75.17, 86.08) circle (  2.13);

\path[fill=fillColor,fill opacity=0.20] ( 76.70, 59.12) circle (  2.13);

\path[fill=fillColor,fill opacity=0.20] ( 74.95, 62.14) circle (  2.13);

\path[fill=fillColor,fill opacity=0.20] ( 82.38, 68.94) circle (  2.13);

\path[fill=fillColor,fill opacity=0.20] ( 84.78, 64.89) circle (  2.13);

\path[fill=fillColor,fill opacity=0.20] ( 83.47, 72.82) circle (  2.13);

\path[fill=fillColor,fill opacity=0.20] ( 90.68, 79.88) circle (  2.13);

\path[fill=fillColor,fill opacity=0.20] (104.23, 64.38) circle (  2.13);

\path[fill=fillColor,fill opacity=0.20] (108.60, 48.88) circle (  2.13);

\path[fill=fillColor,fill opacity=0.20] ( 89.59, 46.89) circle (  2.13);

\path[fill=fillColor,fill opacity=0.20] ( 90.68, 63.43) circle (  2.13);

\path[fill=fillColor,fill opacity=0.20] ( 94.84, 66.01) circle (  2.13);

\path[fill=fillColor,fill opacity=0.20] ( 98.55, 59.38) circle (  2.13);

\path[fill=fillColor,fill opacity=0.20] ( 93.52, 63.26) circle (  2.13);

\path[fill=fillColor,fill opacity=0.20] ( 78.67, 66.36) circle (  2.13);

\path[fill=fillColor,fill opacity=0.20] ( 81.07, 60.59) circle (  2.13);

\path[fill=fillColor,fill opacity=0.20] ( 77.14, 64.72) circle (  2.13);

\path[fill=fillColor,fill opacity=0.20] ( 76.26, 63.77) circle (  2.13);

\path[fill=fillColor,fill opacity=0.20] ( 71.02, 53.96) circle (  2.13);

\path[fill=fillColor,fill opacity=0.20] ( 81.51, 57.14) circle (  2.13);

\path[fill=fillColor,fill opacity=0.20] ( 90.25, 63.00) circle (  2.13);

\path[fill=fillColor,fill opacity=0.20] ( 85.88, 62.65) circle (  2.13);

\path[fill=fillColor,fill opacity=0.20] ( 83.69, 65.75) circle (  2.13);

\path[fill=fillColor,fill opacity=0.20] ( 85.88, 71.52) circle (  2.13);

\path[fill=fillColor,fill opacity=0.20] ( 77.57, 74.62) circle (  2.13);

\path[fill=fillColor,fill opacity=0.20] ( 76.70, 73.07) circle (  2.13);

\path[fill=fillColor,fill opacity=0.20] ( 70.14, 67.65) circle (  2.13);

\path[fill=fillColor,fill opacity=0.20] ( 70.80, 66.27) circle (  2.13);

\path[fill=fillColor,fill opacity=0.20] ( 77.57, 73.42) circle (  2.13);

\path[fill=fillColor,fill opacity=0.20] ( 74.95, 76.35) circle (  2.13);

\path[fill=fillColor,fill opacity=0.20] ( 75.83, 73.25) circle (  2.13);

\path[fill=fillColor,fill opacity=0.20] ( 76.26, 73.68) circle (  2.13);

\path[fill=fillColor,fill opacity=0.20] ( 81.29, 73.85) circle (  2.13);

\path[fill=fillColor,fill opacity=0.20] ( 75.83, 68.42) circle (  2.13);

\path[fill=fillColor,fill opacity=0.20] ( 72.77, 62.31) circle (  2.13);

\path[fill=fillColor,fill opacity=0.20] ( 79.10, 65.15) circle (  2.13);

\path[fill=fillColor,fill opacity=0.20] ( 80.85, 67.99) circle (  2.13);

\path[fill=fillColor,fill opacity=0.20] ( 75.83, 63.34) circle (  2.13);

\path[fill=fillColor,fill opacity=0.20] ( 75.61, 61.36) circle (  2.13);

\path[fill=fillColor,fill opacity=0.20] ( 77.36, 60.50) circle (  2.13);

\path[fill=fillColor,fill opacity=0.20] ( 88.06, 59.47) circle (  2.13);

\path[fill=fillColor,fill opacity=0.20] ( 91.99, 68.51) circle (  2.13);

\path[fill=fillColor,fill opacity=0.20] ( 92.21, 73.68) circle (  2.13);

\path[fill=fillColor,fill opacity=0.20] ( 92.21, 55.68) circle (  2.13);

\path[fill=fillColor,fill opacity=0.20] ( 94.62, 37.94) circle (  2.13);

\path[fill=fillColor,fill opacity=0.20] ( 85.66, 65.67) circle (  2.13);

\path[fill=fillColor,fill opacity=0.20] ( 93.09, 56.11) circle (  2.13);

\path[fill=fillColor,fill opacity=0.20] (102.48, 58.43) circle (  2.13);

\path[fill=fillColor,fill opacity=0.20] ( 98.77, 60.33) circle (  2.13);

\path[fill=fillColor,fill opacity=0.20] ( 86.75, 63.77) circle (  2.13);

\path[fill=fillColor,fill opacity=0.20] ( 87.62, 66.01) circle (  2.13);

\path[fill=fillColor,fill opacity=0.20] ( 88.28, 57.57) circle (  2.13);

\path[fill=fillColor,fill opacity=0.20] ( 82.16, 53.78) circle (  2.13);

\path[fill=fillColor,fill opacity=0.20] ( 77.14, 63.86) circle (  2.13);

\path[fill=fillColor,fill opacity=0.20] ( 77.57, 66.44) circle (  2.13);

\path[fill=fillColor,fill opacity=0.20] ( 74.30, 64.46) circle (  2.13);

\path[fill=fillColor,fill opacity=0.20] ( 84.13, 63.34) circle (  2.13);

\path[fill=fillColor,fill opacity=0.20] ( 88.94, 61.02) circle (  2.13);

\path[fill=fillColor,fill opacity=0.20] ( 78.88, 70.23) circle (  2.13);

\path[fill=fillColor,fill opacity=0.20] ( 75.17, 79.79) circle (  2.13);

\path[fill=fillColor,fill opacity=0.20] ( 77.36, 70.66) circle (  2.13);

\path[fill=fillColor,fill opacity=0.20] ( 79.54, 61.96) circle (  2.13);

\path[fill=fillColor,fill opacity=0.20] ( 79.10, 66.10) circle (  2.13);

\path[fill=fillColor,fill opacity=0.20] ( 79.54, 63.17) circle (  2.13);

\path[fill=fillColor,fill opacity=0.20] ( 83.25, 56.02) circle (  2.13);

\path[fill=fillColor,fill opacity=0.20] ( 83.69, 60.93) circle (  2.13);

\path[fill=fillColor,fill opacity=0.20] ( 82.60, 65.93) circle (  2.13);

\path[fill=fillColor,fill opacity=0.20] ( 87.19, 64.55) circle (  2.13);

\path[fill=fillColor,fill opacity=0.20] ( 76.04, 68.68) circle (  2.13);

\path[fill=fillColor,fill opacity=0.20] ( 74.51, 75.83) circle (  2.13);

\path[fill=fillColor,fill opacity=0.20] ( 76.92, 76.60) circle (  2.13);

\path[fill=fillColor,fill opacity=0.20] ( 80.41, 74.45) circle (  2.13);

\path[fill=fillColor,fill opacity=0.20] ( 78.67, 73.33) circle (  2.13);

\path[fill=fillColor,fill opacity=0.20] ( 87.19, 63.86) circle (  2.13);

\path[fill=fillColor,fill opacity=0.20] ( 94.18, 56.54) circle (  2.13);

\path[fill=fillColor,fill opacity=0.20] ( 91.99, 62.91) circle (  2.13);

\path[fill=fillColor,fill opacity=0.20] ( 81.51, 59.38) circle (  2.13);

\path[fill=fillColor,fill opacity=0.20] ( 92.65, 63.26) circle (  2.13);

\path[fill=fillColor,fill opacity=0.20] ( 91.56, 61.10) circle (  2.13);

\path[fill=fillColor,fill opacity=0.20] ( 91.99, 54.13) circle (  2.13);

\path[fill=fillColor,fill opacity=0.20] ( 91.56, 52.15) circle (  2.13);

\path[fill=fillColor,fill opacity=0.20] ( 90.90, 57.75) circle (  2.13);

\path[fill=fillColor,fill opacity=0.20] ( 83.25, 64.46) circle (  2.13);

\path[fill=fillColor,fill opacity=0.20] ( 82.60, 66.36) circle (  2.13);

\path[fill=fillColor,fill opacity=0.20] ( 88.94, 63.34) circle (  2.13);

\path[fill=fillColor,fill opacity=0.20] ( 88.28, 57.49) circle (  2.13);

\path[fill=fillColor,fill opacity=0.20] ( 84.57, 63.00) circle (  2.13);

\path[fill=fillColor,fill opacity=0.20] ( 84.35, 73.59) circle (  2.13);

\path[fill=fillColor,fill opacity=0.20] ( 90.03, 71.44) circle (  2.13);

\path[fill=fillColor,fill opacity=0.20] ( 90.03, 67.56) circle (  2.13);

\path[fill=fillColor,fill opacity=0.20] ( 88.72, 66.36) circle (  2.13);

\path[fill=fillColor,fill opacity=0.20] ( 81.73, 60.50) circle (  2.13);

\path[fill=fillColor,fill opacity=0.20] ( 90.47, 55.59) circle (  2.13);

\path[fill=fillColor,fill opacity=0.20] (103.58, 62.57) circle (  2.13);

\path[fill=fillColor,fill opacity=0.20] ( 81.73, 69.89) circle (  2.13);

\path[fill=fillColor,fill opacity=0.20] ( 80.20, 65.93) circle (  2.13);

\path[fill=fillColor,fill opacity=0.20] ( 88.72, 61.88) circle (  2.13);

\path[fill=fillColor,fill opacity=0.20] ( 87.84, 64.81) circle (  2.13);

\path[fill=fillColor,fill opacity=0.20] ( 85.00, 65.32) circle (  2.13);

\path[fill=fillColor,fill opacity=0.20] ( 91.34, 63.26) circle (  2.13);

\path[fill=fillColor,fill opacity=0.20] ( 91.34, 58.52) circle (  2.13);

\path[fill=fillColor,fill opacity=0.20] ( 83.69, 53.53) circle (  2.13);

\path[fill=fillColor,fill opacity=0.20] ( 85.44, 58.52) circle (  2.13);

\path[fill=fillColor,fill opacity=0.20] ( 74.30, 57.75) circle (  2.13);

\path[fill=fillColor,fill opacity=0.20] ( 81.73, 59.38) circle (  2.13);

\path[fill=fillColor,fill opacity=0.20] ( 83.04, 60.07) circle (  2.13);

\path[fill=fillColor,fill opacity=0.20] ( 92.65, 54.73) circle (  2.13);

\path[fill=fillColor,fill opacity=0.20] ( 91.78, 53.18) circle (  2.13);

\path[fill=fillColor,fill opacity=0.20] ( 94.40, 49.82) circle (  2.13);

\path[fill=fillColor,fill opacity=0.20] (105.98, 41.38) circle (  2.13);

\path[fill=fillColor,fill opacity=0.20] ( 95.93, 41.73) circle (  2.13);

\path[fill=fillColor,fill opacity=0.20] ( 94.40, 49.56) circle (  2.13);

\path[fill=fillColor,fill opacity=0.20] ( 86.75, 54.73) circle (  2.13);

\path[fill=fillColor,fill opacity=0.20] ( 84.78, 54.21) circle (  2.13);

\path[fill=fillColor,fill opacity=0.20] ( 78.23, 52.41) circle (  2.13);

\path[fill=fillColor,fill opacity=0.20] ( 74.95, 54.82) circle (  2.13);

\path[fill=fillColor,fill opacity=0.20] ( 83.04, 54.04) circle (  2.13);

\path[fill=fillColor,fill opacity=0.20] ( 78.67, 54.04) circle (  2.13);

\path[fill=fillColor,fill opacity=0.20] ( 79.32, 56.37) circle (  2.13);

\path[fill=fillColor,fill opacity=0.20] ( 86.10, 52.32) circle (  2.13);

\path[fill=fillColor,fill opacity=0.20] ( 79.98, 45.34) circle (  2.13);

\path[fill=fillColor,fill opacity=0.20] ( 78.23, 45.43) circle (  2.13);

\path[fill=fillColor,fill opacity=0.20] ( 51.79,115.10) circle (  2.13);

\path[fill=fillColor,fill opacity=0.20] ( 52.23,105.28) circle (  2.13);

\path[fill=fillColor,fill opacity=0.20] ( 54.19,101.15) circle (  2.13);

\path[fill=fillColor,fill opacity=0.20] ( 86.97,103.39) circle (  2.13);

\path[fill=fillColor,fill opacity=0.20] ( 59.66, 98.82) circle (  2.13);

\path[fill=fillColor,fill opacity=0.20] ( 58.35, 93.31) circle (  2.13);

\path[fill=fillColor,fill opacity=0.20] ( 72.55,102.52) circle (  2.13);

\path[fill=fillColor,fill opacity=0.20] ( 55.29,108.12) circle (  2.13);

\path[fill=fillColor,fill opacity=0.20] ( 55.51,107.17) circle (  2.13);

\path[fill=fillColor,fill opacity=0.20] ( 51.79,115.10) circle (  2.13);

\path[fill=fillColor,fill opacity=0.20] ( 45.67,115.96) circle (  2.13);

\path[fill=fillColor,fill opacity=0.20] ( 54.63,106.31) circle (  2.13);

\path[fill=fillColor,fill opacity=0.20] ( 77.57,100.46) circle (  2.13);

\path[fill=fillColor,fill opacity=0.20] ( 64.46,100.72) circle (  2.13);

\path[fill=fillColor,fill opacity=0.20] ( 71.02, 92.71) circle (  2.13);

\path[fill=fillColor,fill opacity=0.20] ( 73.42, 78.76) circle (  2.13);

\path[fill=fillColor,fill opacity=0.20] ( 68.62, 72.99) circle (  2.13);

\path[fill=fillColor,fill opacity=0.20] ( 73.20, 72.64) circle (  2.13);

\path[fill=fillColor,fill opacity=0.20] ( 74.08, 72.64) circle (  2.13);

\path[fill=fillColor,fill opacity=0.20] (140.72, 81.60) circle (  2.13);

\path[fill=fillColor,fill opacity=0.20] ( 89.37,103.90) circle (  2.13);

\path[fill=fillColor,fill opacity=0.20] ( 75.83, 94.52) circle (  2.13);

\path[fill=fillColor,fill opacity=0.20] ( 84.78, 81.60) circle (  2.13);

\path[fill=fillColor,fill opacity=0.20] ( 95.71, 75.92) circle (  2.13);

\path[fill=fillColor,fill opacity=0.20] (103.36, 69.03) circle (  2.13);

\path[fill=fillColor,fill opacity=0.20] ( 88.06, 62.91) circle (  2.13);

\path[fill=fillColor,fill opacity=0.20] ( 84.57, 55.68) circle (  2.13);

\path[fill=fillColor,fill opacity=0.20] ( 83.04, 54.30) circle (  2.13);

\path[fill=fillColor,fill opacity=0.20] ( 81.73, 68.77) circle (  2.13);

\path[fill=fillColor,fill opacity=0.20] ( 78.01, 73.68) circle (  2.13);

\path[fill=fillColor,fill opacity=0.20] ( 57.25,110.88) circle (  2.13);

\path[fill=fillColor,fill opacity=0.20] ( 79.76, 87.97) circle (  2.13);

\path[fill=fillColor,fill opacity=0.20] ( 93.31, 79.02) circle (  2.13);

\path[fill=fillColor,fill opacity=0.20] (126.52, 65.24) circle (  2.13);

\path[fill=fillColor,fill opacity=0.20] (134.82, 59.98) circle (  2.13);

\path[fill=fillColor,fill opacity=0.20] (103.14, 57.49) circle (  2.13);

\path[fill=fillColor,fill opacity=0.20] ( 93.09, 63.60) circle (  2.13);

\path[fill=fillColor,fill opacity=0.20] ( 85.00, 71.61) circle (  2.13);

\path[fill=fillColor,fill opacity=0.20] ( 86.31, 61.45) circle (  2.13);

\path[fill=fillColor,fill opacity=0.20] (102.05, 53.96) circle (  2.13);

\path[fill=fillColor,fill opacity=0.20] ( 83.91, 69.80) circle (  2.13);

\path[fill=fillColor,fill opacity=0.20] ( 74.73, 47.93) circle (  2.13);

\path[fill=fillColor,fill opacity=0.20] ( 67.09, 94.00) circle (  2.13);

\path[fill=fillColor,fill opacity=0.20] (124.99, 69.80) circle (  2.13);

\path[fill=fillColor,fill opacity=0.20] (112.53, 64.63) circle (  2.13);

\path[fill=fillColor,fill opacity=0.20] ( 93.31, 64.46) circle (  2.13);

\path[fill=fillColor,fill opacity=0.20] ( 91.78, 67.13) circle (  2.13);

\path[fill=fillColor,fill opacity=0.20] ( 93.09, 62.31) circle (  2.13);

\path[fill=fillColor,fill opacity=0.20] ( 94.40, 62.65) circle (  2.13);

\path[fill=fillColor,fill opacity=0.20] ( 83.91, 76.52) circle (  2.13);

\path[fill=fillColor,fill opacity=0.20] ( 81.51, 74.80) circle (  2.13);

\path[fill=fillColor,fill opacity=0.20] ( 85.44, 62.65) circle (  2.13);

\path[fill=fillColor,fill opacity=0.20] ( 79.98, 60.67) circle (  2.13);

\path[fill=fillColor,fill opacity=0.20] ( 91.34, 52.06) circle (  2.13);

\path[fill=fillColor,fill opacity=0.20] (102.92, 51.98) circle (  2.13);

\path[fill=fillColor,fill opacity=0.20] ( 86.53, 87.80) circle (  2.13);

\path[fill=fillColor,fill opacity=0.20] (103.36, 59.47) circle (  2.13);

\path[fill=fillColor,fill opacity=0.20] (119.31, 59.64) circle (  2.13);

\path[fill=fillColor,fill opacity=0.20] ( 99.64, 64.29) circle (  2.13);

\path[fill=fillColor,fill opacity=0.20] ( 93.31, 69.28) circle (  2.13);

\path[fill=fillColor,fill opacity=0.20] ( 84.78, 64.12) circle (  2.13);

\path[fill=fillColor,fill opacity=0.20] ( 95.05, 60.50) circle (  2.13);

\path[fill=fillColor,fill opacity=0.20] ( 78.45, 82.03) circle (  2.13);

\path[fill=fillColor,fill opacity=0.20] ( 79.98, 75.23) circle (  2.13);

\path[fill=fillColor,fill opacity=0.20] ( 91.34, 47.58) circle (  2.13);

\path[fill=fillColor,fill opacity=0.20] (101.39, 40.44) circle (  2.13);

\path[fill=fillColor,fill opacity=0.20] ( 98.33, 50.08) circle (  2.13);

\path[fill=fillColor,fill opacity=0.20] (123.24, 53.09) circle (  2.13);

\path[fill=fillColor,fill opacity=0.20] (123.24, 57.23) circle (  2.13);

\path[fill=fillColor,fill opacity=0.20] (128.27, 53.53) circle (  2.13);

\path[fill=fillColor,fill opacity=0.20] (138.32, 40.26) circle (  2.13);

\path[fill=fillColor,fill opacity=0.20] ( 88.50, 89.87) circle (  2.13);

\path[fill=fillColor,fill opacity=0.20] (107.07, 54.47) circle (  2.13);

\path[fill=fillColor,fill opacity=0.20] ( 95.71, 55.25) circle (  2.13);

\path[fill=fillColor,fill opacity=0.20] (100.52, 65.41) circle (  2.13);

\path[fill=fillColor,fill opacity=0.20] ( 95.27, 58.95) circle (  2.13);

\path[fill=fillColor,fill opacity=0.20] ( 96.80, 55.68) circle (  2.13);

\path[fill=fillColor,fill opacity=0.20] ( 92.43, 58.86) circle (  2.13);

\path[fill=fillColor,fill opacity=0.20] ( 78.23, 82.55) circle (  2.13);

\path[fill=fillColor,fill opacity=0.20] ( 63.81, 87.63) circle (  2.13);

\path[fill=fillColor,fill opacity=0.20] ( 86.97, 47.24) circle (  2.13);

\path[fill=fillColor,fill opacity=0.20] ( 94.62, 54.73) circle (  2.13);

\path[fill=fillColor,fill opacity=0.20] (102.92, 55.42) circle (  2.13);

\path[fill=fillColor,fill opacity=0.20] (120.18, 49.39) circle (  2.13);

\path[fill=fillColor,fill opacity=0.20] (131.54, 56.11) circle (  2.13);

\path[fill=fillColor,fill opacity=0.20] (121.27, 51.89) circle (  2.13);

\path[fill=fillColor,fill opacity=0.20] ( 98.55, 93.91) circle (  2.13);

\path[fill=fillColor,fill opacity=0.20] (131.76, 50.60) circle (  2.13);

\path[fill=fillColor,fill opacity=0.20] ( 85.88, 54.21) circle (  2.13);

\path[fill=fillColor,fill opacity=0.20] ( 87.84, 63.08) circle (  2.13);

\path[fill=fillColor,fill opacity=0.20] ( 92.43, 49.31) circle (  2.13);

\path[fill=fillColor,fill opacity=0.20] ( 91.34, 48.79) circle (  2.13);

\path[fill=fillColor,fill opacity=0.20] ( 84.13, 63.86) circle (  2.13);

\path[fill=fillColor,fill opacity=0.20] (112.10, 68.08) circle (  2.13);

\path[fill=fillColor,fill opacity=0.20] ( 77.14, 67.05) circle (  2.13);

\path[fill=fillColor,fill opacity=0.20] ( 97.46, 52.06) circle (  2.13);

\path[fill=fillColor,fill opacity=0.20] ( 99.86, 51.98) circle (  2.13);

\path[fill=fillColor,fill opacity=0.20] ( 96.36, 46.38) circle (  2.13);

\path[fill=fillColor,fill opacity=0.20] (105.98, 43.11) circle (  2.13);

\path[fill=fillColor,fill opacity=0.20] (123.02, 52.23) circle (  2.13);

\path[fill=fillColor,fill opacity=0.20] (125.43, 56.54) circle (  2.13);

\path[fill=fillColor,fill opacity=0.20] (118.43, 38.80) circle (  2.13);

\path[fill=fillColor,fill opacity=0.20] (139.63, 98.56) circle (  2.13);

\path[fill=fillColor,fill opacity=0.20] (112.75, 53.35) circle (  2.13);

\path[fill=fillColor,fill opacity=0.20] ( 91.99, 56.02) circle (  2.13);

\path[fill=fillColor,fill opacity=0.20] ( 84.35, 60.67) circle (  2.13);

\path[fill=fillColor,fill opacity=0.20] ( 85.44, 46.89) circle (  2.13);

\path[fill=fillColor,fill opacity=0.20] ( 82.16, 46.81) circle (  2.13);

\path[fill=fillColor,fill opacity=0.20] ( 80.85, 61.45) circle (  2.13);

\path[fill=fillColor,fill opacity=0.20] ( 78.01, 66.53) circle (  2.13);

\path[fill=fillColor,fill opacity=0.20] ( 74.73, 56.97) circle (  2.13);

\path[fill=fillColor,fill opacity=0.20] ( 73.64, 55.16) circle (  2.13);

\path[fill=fillColor,fill opacity=0.20] ( 69.93, 77.29) circle (  2.13);

\path[fill=fillColor,fill opacity=0.20] (116.03, 40.35) circle (  2.13);

\path[fill=fillColor,fill opacity=0.20] (102.70, 49.65) circle (  2.13);

\path[fill=fillColor,fill opacity=0.20] (100.73, 41.56) circle (  2.13);

\path[fill=fillColor,fill opacity=0.20] (109.47, 47.33) circle (  2.13);

\path[fill=fillColor,fill opacity=0.20] (118.21, 56.80) circle (  2.13);

\path[fill=fillColor,fill opacity=0.20] (139.63, 45.26) circle (  2.13);

\path[fill=fillColor,fill opacity=0.20] (145.53, 42.33) circle (  2.13);

\path[fill=fillColor,fill opacity=0.20] (102.26, 72.47) circle (  2.13);

\path[fill=fillColor,fill opacity=0.20] ( 97.02, 62.14) circle (  2.13);

\path[fill=fillColor,fill opacity=0.20] ( 91.12, 63.26) circle (  2.13);

\path[fill=fillColor,fill opacity=0.20] ( 86.53, 57.92) circle (  2.13);

\path[fill=fillColor,fill opacity=0.20] ( 92.87, 56.20) circle (  2.13);

\path[fill=fillColor,fill opacity=0.20] ( 84.57, 58.52) circle (  2.13);

\path[fill=fillColor,fill opacity=0.20] ( 85.00, 58.18) circle (  2.13);

\path[fill=fillColor,fill opacity=0.20] ( 79.98, 53.01) circle (  2.13);

\path[fill=fillColor,fill opacity=0.20] ( 88.06, 57.75) circle (  2.13);

\path[fill=fillColor,fill opacity=0.20] ( 69.71, 70.15) circle (  2.13);

\path[fill=fillColor,fill opacity=0.20] ( 55.07, 96.67) circle (  2.13);

\path[fill=fillColor,fill opacity=0.20] ( 75.61, 63.08) circle (  2.13);

\path[fill=fillColor,fill opacity=0.20] (101.83, 52.23) circle (  2.13);

\path[fill=fillColor,fill opacity=0.20] (110.57, 46.46) circle (  2.13);

\path[fill=fillColor,fill opacity=0.20] (111.22, 41.21) circle (  2.13);

\path[fill=fillColor,fill opacity=0.20] (105.32, 46.64) circle (  2.13);

\path[fill=fillColor,fill opacity=0.20] (115.16, 44.05) circle (  2.13);

\path[fill=fillColor,fill opacity=0.20] (125.64, 40.69) circle (  2.13);

\path[fill=fillColor,fill opacity=0.20] (141.38, 43.28) circle (  2.13);

\path[fill=fillColor,fill opacity=0.20] (129.58, 48.53) circle (  2.13);

\path[fill=fillColor,fill opacity=0.20] ( 98.33, 96.75) circle (  2.13);

\path[fill=fillColor,fill opacity=0.20] (102.48, 69.89) circle (  2.13);

\path[fill=fillColor,fill opacity=0.20] ( 96.58, 63.95) circle (  2.13);

\path[fill=fillColor,fill opacity=0.20] ( 92.21, 69.11) circle (  2.13);

\path[fill=fillColor,fill opacity=0.20] ( 79.76, 70.66) circle (  2.13);

\path[fill=fillColor,fill opacity=0.20] ( 81.07, 64.81) circle (  2.13);

\path[fill=fillColor,fill opacity=0.20] ( 86.97, 53.27) circle (  2.13);

\path[fill=fillColor,fill opacity=0.20] ( 83.25, 56.11) circle (  2.13);

\path[fill=fillColor,fill opacity=0.20] ( 77.36, 72.56) circle (  2.13);

\path[fill=fillColor,fill opacity=0.20] ( 71.24, 77.03) circle (  2.13);

\path[fill=fillColor,fill opacity=0.20] ( 61.40, 78.84) circle (  2.13);

\path[fill=fillColor,fill opacity=0.20] ( 53.76, 98.74) circle (  2.13);

\path[fill=fillColor,fill opacity=0.20] ( 97.68, 50.86) circle (  2.13);

\path[fill=fillColor,fill opacity=0.20] (109.04, 44.57) circle (  2.13);

\path[fill=fillColor,fill opacity=0.20] (107.95, 43.11) circle (  2.13);

\path[fill=fillColor,fill opacity=0.20] (102.26, 48.96) circle (  2.13);

\path[fill=fillColor,fill opacity=0.20] (106.20, 44.66) circle (  2.13);

\path[fill=fillColor,fill opacity=0.20] (116.69, 41.64) circle (  2.13);

\path[fill=fillColor,fill opacity=0.20] (110.57, 51.63) circle (  2.13);

\path[fill=fillColor,fill opacity=0.20] (117.78, 45.17) circle (  2.13);

\path[fill=fillColor,fill opacity=0.20] (136.57, 37.94) circle (  2.13);

\path[fill=fillColor,fill opacity=0.20] (125.43, 51.98) circle (  2.13);

\path[fill=fillColor,fill opacity=0.20] (109.04, 53.53) circle (  2.13);

\path[fill=fillColor,fill opacity=0.20] (101.83, 83.67) circle (  2.13);

\path[fill=fillColor,fill opacity=0.20] ( 96.58, 62.48) circle (  2.13);

\path[fill=fillColor,fill opacity=0.20] ( 88.06, 61.88) circle (  2.13);

\path[fill=fillColor,fill opacity=0.20] ( 77.79, 69.20) circle (  2.13);

\path[fill=fillColor,fill opacity=0.20] ( 77.57, 62.14) circle (  2.13);

\path[fill=fillColor,fill opacity=0.20] ( 81.73, 53.61) circle (  2.13);

\path[fill=fillColor,fill opacity=0.20] ( 76.26, 64.81) circle (  2.13);

\path[fill=fillColor,fill opacity=0.20] ( 74.95, 76.78) circle (  2.13);

\path[fill=fillColor,fill opacity=0.20] ( 82.60, 76.95) circle (  2.13);

\path[fill=fillColor,fill opacity=0.20] ( 74.30, 76.86) circle (  2.13);

\path[fill=fillColor,fill opacity=0.20] ( 64.46, 80.91) circle (  2.13);

\path[fill=fillColor,fill opacity=0.20] ( 56.82, 91.59) circle (  2.13);

\path[fill=fillColor,fill opacity=0.20] ( 64.90,115.10) circle (  2.13);

\path[fill=fillColor,fill opacity=0.20] ( 91.12, 68.42) circle (  2.13);

\path[fill=fillColor,fill opacity=0.20] (114.28, 55.25) circle (  2.13);

\path[fill=fillColor,fill opacity=0.20] (104.23, 47.84) circle (  2.13);

\path[fill=fillColor,fill opacity=0.20] (103.58, 47.24) circle (  2.13);

\path[fill=fillColor,fill opacity=0.20] (103.58, 43.11) circle (  2.13);

\path[fill=fillColor,fill opacity=0.20] (127.39, 47.15) circle (  2.13);

\path[fill=fillColor,fill opacity=0.20] (129.80, 60.85) circle (  2.13);

\path[fill=fillColor,fill opacity=0.20] (109.47, 57.40) circle (  2.13);

\path[fill=fillColor,fill opacity=0.20] (116.90, 38.46) circle (  2.13);

\path[fill=fillColor,fill opacity=0.20] (135.26, 38.89) circle (  2.13);

\path[fill=fillColor,fill opacity=0.20] (120.40, 58.00) circle (  2.13);

\path[fill=fillColor,fill opacity=0.20] (116.69, 60.41) circle (  2.13);

\path[fill=fillColor,fill opacity=0.20] (100.08, 46.12) circle (  2.13);

\path[fill=fillColor,fill opacity=0.20] (116.90, 80.13) circle (  2.13);

\path[fill=fillColor,fill opacity=0.20] ( 87.19, 58.78) circle (  2.13);

\path[fill=fillColor,fill opacity=0.20] ( 85.00, 59.73) circle (  2.13);

\path[fill=fillColor,fill opacity=0.20] ( 83.69, 54.47) circle (  2.13);

\path[fill=fillColor,fill opacity=0.20] ( 84.57, 57.14) circle (  2.13);

\path[fill=fillColor,fill opacity=0.20] ( 72.11, 66.70) circle (  2.13);

\path[fill=fillColor,fill opacity=0.20] ( 75.83, 67.91) circle (  2.13);

\path[fill=fillColor,fill opacity=0.20] ( 96.80, 59.12) circle (  2.13);

\path[fill=fillColor,fill opacity=0.20] ( 78.45, 65.75) circle (  2.13);

\path[fill=fillColor,fill opacity=0.20] ( 72.99, 72.47) circle (  2.13);

\path[fill=fillColor,fill opacity=0.20] ( 63.81, 72.04) circle (  2.13);

\path[fill=fillColor,fill opacity=0.20] ( 61.84, 87.80) circle (  2.13);

\path[fill=fillColor,fill opacity=0.20] ( 70.36, 91.85) circle (  2.13);

\path[fill=fillColor,fill opacity=0.20] ( 83.04, 69.54) circle (  2.13);

\path[fill=fillColor,fill opacity=0.20] ( 93.31, 73.25) circle (  2.13);

\path[fill=fillColor,fill opacity=0.20] (104.23, 63.86) circle (  2.13);

\path[fill=fillColor,fill opacity=0.20] (113.41, 46.55) circle (  2.13);

\path[fill=fillColor,fill opacity=0.20] (116.25, 45.17) circle (  2.13);

\path[fill=fillColor,fill opacity=0.20] (115.37, 58.95) circle (  2.13);

\path[fill=fillColor,fill opacity=0.20] (109.47, 65.84) circle (  2.13);

\path[fill=fillColor,fill opacity=0.20] (116.03, 54.21) circle (  2.13);

\path[fill=fillColor,fill opacity=0.20] (118.43, 45.09) circle (  2.13);

\path[fill=fillColor,fill opacity=0.20] (119.74, 50.25) circle (  2.13);

\path[fill=fillColor,fill opacity=0.20] (122.80, 54.39) circle (  2.13);

\path[fill=fillColor,fill opacity=0.20] (116.25, 58.43) circle (  2.13);

\path[fill=fillColor,fill opacity=0.20] (114.50, 66.01) circle (  2.13);

\path[fill=fillColor,fill opacity=0.20] (107.73, 82.72) circle (  2.13);

\path[fill=fillColor,fill opacity=0.20] ( 90.03, 70.75) circle (  2.13);

\path[fill=fillColor,fill opacity=0.20] ( 81.51, 66.87) circle (  2.13);

\path[fill=fillColor,fill opacity=0.20] ( 84.13, 65.84) circle (  2.13);

\path[fill=fillColor,fill opacity=0.20] ( 82.60, 62.05) circle (  2.13);

\path[fill=fillColor,fill opacity=0.20] ( 86.75, 58.35) circle (  2.13);

\path[fill=fillColor,fill opacity=0.20] ( 88.72, 55.25) circle (  2.13);

\path[fill=fillColor,fill opacity=0.20] ( 90.25, 59.12) circle (  2.13);

\path[fill=fillColor,fill opacity=0.20] ( 87.62, 63.34) circle (  2.13);

\path[fill=fillColor,fill opacity=0.20] ( 77.57, 61.71) circle (  2.13);

\path[fill=fillColor,fill opacity=0.20] ( 73.64, 66.10) circle (  2.13);

\path[fill=fillColor,fill opacity=0.20] ( 67.52, 74.54) circle (  2.13);

\path[fill=fillColor,fill opacity=0.20] ( 60.75, 78.58) circle (  2.13);

\path[fill=fillColor,fill opacity=0.20] ( 54.41, 93.83) circle (  2.13);

\path[fill=fillColor,fill opacity=0.20] ( 56.82,107.17) circle (  2.13);

\path[fill=fillColor,fill opacity=0.20] ( 64.90, 96.58) circle (  2.13);

\path[fill=fillColor,fill opacity=0.20] ( 56.16,101.15) circle (  2.13);

\path[fill=fillColor,fill opacity=0.20] ( 72.55, 83.06) circle (  2.13);

\path[fill=fillColor,fill opacity=0.20] (103.58, 63.26) circle (  2.13);

\path[fill=fillColor,fill opacity=0.20] ( 93.09, 59.12) circle (  2.13);

\path[fill=fillColor,fill opacity=0.20] ( 91.78, 68.77) circle (  2.13);

\path[fill=fillColor,fill opacity=0.20] (104.89, 64.12) circle (  2.13);

\path[fill=fillColor,fill opacity=0.20] (108.82, 45.86) circle (  2.13);

\path[fill=fillColor,fill opacity=0.20] (114.72, 44.40) circle (  2.13);

\path[fill=fillColor,fill opacity=0.20] (109.47, 57.31) circle (  2.13);

\path[fill=fillColor,fill opacity=0.20] (116.03, 56.02) circle (  2.13);

\path[fill=fillColor,fill opacity=0.20] (118.65, 45.00) circle (  2.13);

\path[fill=fillColor,fill opacity=0.20] (112.75, 54.82) circle (  2.13);

\path[fill=fillColor,fill opacity=0.20] (115.16, 61.19) circle (  2.13);

\path[fill=fillColor,fill opacity=0.20] (113.63, 42.33) circle (  2.13);

\path[fill=fillColor,fill opacity=0.20] (106.42, 47.76) circle (  2.13);

\path[fill=fillColor,fill opacity=0.20] (115.16, 78.67) circle (  2.13);

\path[fill=fillColor,fill opacity=0.20] ( 84.35, 88.49) circle (  2.13);

\path[fill=fillColor,fill opacity=0.20] ( 91.34, 83.06) circle (  2.13);

\path[fill=fillColor,fill opacity=0.20] ( 97.24, 73.42) circle (  2.13);

\path[fill=fillColor,fill opacity=0.20] ( 85.88, 73.85) circle (  2.13);

\path[fill=fillColor,fill opacity=0.20] ( 85.44, 74.45) circle (  2.13);

\path[fill=fillColor,fill opacity=0.20] ( 78.23, 66.79) circle (  2.13);

\path[fill=fillColor,fill opacity=0.20] ( 83.25, 58.09) circle (  2.13);

\path[fill=fillColor,fill opacity=0.20] ( 78.88, 56.11) circle (  2.13);

\path[fill=fillColor,fill opacity=0.20] ( 72.33, 59.21) circle (  2.13);

\path[fill=fillColor,fill opacity=0.20] ( 69.27, 56.97) circle (  2.13);

\path[fill=fillColor,fill opacity=0.20] ( 71.24, 63.17) circle (  2.13);

\path[fill=fillColor,fill opacity=0.20] ( 68.40, 74.37) circle (  2.13);

\path[fill=fillColor,fill opacity=0.20] ( 75.17, 78.24) circle (  2.13);

\path[fill=fillColor,fill opacity=0.20] ( 60.31, 84.87) circle (  2.13);

\path[fill=fillColor,fill opacity=0.20] ( 55.51, 93.05) circle (  2.13);

\path[fill=fillColor,fill opacity=0.20] ( 83.91, 87.97) circle (  2.13);

\path[fill=fillColor,fill opacity=0.20] ( 66.21, 83.32) circle (  2.13);

\path[fill=fillColor,fill opacity=0.20] ( 74.51, 67.56) circle (  2.13);

\path[fill=fillColor,fill opacity=0.20] ( 74.73, 68.68) circle (  2.13);

\path[fill=fillColor,fill opacity=0.20] ( 74.08, 79.70) circle (  2.13);

\path[fill=fillColor,fill opacity=0.20] ( 78.23, 69.63) circle (  2.13);

\path[fill=fillColor,fill opacity=0.20] ( 83.91, 62.14) circle (  2.13);

\path[fill=fillColor,fill opacity=0.20] ( 95.71, 53.78) circle (  2.13);

\path[fill=fillColor,fill opacity=0.20] ( 92.21, 46.64) circle (  2.13);

\path[fill=fillColor,fill opacity=0.20] ( 99.42, 52.06) circle (  2.13);

\path[fill=fillColor,fill opacity=0.20] ( 91.12, 48.88) circle (  2.13);

\path[fill=fillColor,fill opacity=0.20] (104.23, 47.50) circle (  2.13);

\path[fill=fillColor,fill opacity=0.20] (104.67, 55.85) circle (  2.13);

\path[fill=fillColor,fill opacity=0.20] (109.04, 56.54) circle (  2.13);

\path[fill=fillColor,fill opacity=0.20] (110.57, 44.57) circle (  2.13);

\path[fill=fillColor,fill opacity=0.20] (119.09, 45.52) circle (  2.13);

\path[fill=fillColor,fill opacity=0.20] (115.81, 52.92) circle (  2.13);

\path[fill=fillColor,fill opacity=0.20] (115.37, 39.14) circle (  2.13);

\path[fill=fillColor,fill opacity=0.20] (105.32, 74.80) circle (  2.13);

\path[fill=fillColor,fill opacity=0.20] ( 85.88, 80.65) circle (  2.13);

\path[fill=fillColor,fill opacity=0.20] ( 79.54, 73.85) circle (  2.13);

\path[fill=fillColor,fill opacity=0.20] ( 76.70, 63.26) circle (  2.13);

\path[fill=fillColor,fill opacity=0.20] ( 79.32, 56.11) circle (  2.13);

\path[fill=fillColor,fill opacity=0.20] ( 79.76, 61.36) circle (  2.13);

\path[fill=fillColor,fill opacity=0.20] ( 78.67, 68.08) circle (  2.13);

\path[fill=fillColor,fill opacity=0.20] ( 81.51, 64.46) circle (  2.13);

\path[fill=fillColor,fill opacity=0.20] ( 77.79, 58.43) circle (  2.13);

\path[fill=fillColor,fill opacity=0.20] ( 65.34, 64.55) circle (  2.13);

\path[fill=fillColor,fill opacity=0.20] ( 77.14, 70.58) circle (  2.13);

\path[fill=fillColor,fill opacity=0.20] ( 64.90, 77.64) circle (  2.13);

\path[fill=fillColor,fill opacity=0.20] ( 57.25, 89.87) circle (  2.13);

\path[fill=fillColor,fill opacity=0.20] ( 52.01,103.13) circle (  2.13);

\path[fill=fillColor,fill opacity=0.20] ( 66.43, 79.36) circle (  2.13);

\path[fill=fillColor,fill opacity=0.20] ( 76.04, 74.54) circle (  2.13);

\path[fill=fillColor,fill opacity=0.20] ( 78.67, 56.88) circle (  2.13);

\path[fill=fillColor,fill opacity=0.20] ( 89.59, 52.15) circle (  2.13);

\path[fill=fillColor,fill opacity=0.20] ( 94.84, 71.78) circle (  2.13);

\path[fill=fillColor,fill opacity=0.20] ( 94.40, 77.47) circle (  2.13);

\path[fill=fillColor,fill opacity=0.20] ( 84.57, 65.75) circle (  2.13);

\path[fill=fillColor,fill opacity=0.20] ( 86.10, 63.00) circle (  2.13);

\path[fill=fillColor,fill opacity=0.20] (106.42, 67.73) circle (  2.13);

\path[fill=fillColor,fill opacity=0.20] ( 98.77, 52.75) circle (  2.13);

\path[fill=fillColor,fill opacity=0.20] ( 90.47, 41.47) circle (  2.13);

\path[fill=fillColor,fill opacity=0.20] ( 95.49, 49.91) circle (  2.13);

\path[fill=fillColor,fill opacity=0.20] ( 92.21, 59.47) circle (  2.13);

\path[fill=fillColor,fill opacity=0.20] ( 96.58, 60.85) circle (  2.13);

\path[fill=fillColor,fill opacity=0.20] (103.58, 55.33) circle (  2.13);

\path[fill=fillColor,fill opacity=0.20] (100.95, 55.16) circle (  2.13);

\path[fill=fillColor,fill opacity=0.20] (100.52, 51.46) circle (  2.13);

\path[fill=fillColor,fill opacity=0.20] (129.36, 40.87) circle (  2.13);

\path[fill=fillColor,fill opacity=0.20] (114.50, 43.71) circle (  2.13);

\path[fill=fillColor,fill opacity=0.20] (121.06, 47.24) circle (  2.13);

\path[fill=fillColor,fill opacity=0.20] (128.70, 40.52) circle (  2.13);

\path[fill=fillColor,fill opacity=0.20] (104.89, 74.80) circle (  2.13);

\path[fill=fillColor,fill opacity=0.20] ( 86.10, 91.93) circle (  2.13);

\path[fill=fillColor,fill opacity=0.20] ( 77.79, 84.87) circle (  2.13);

\path[fill=fillColor,fill opacity=0.20] ( 79.10, 74.71) circle (  2.13);

\path[fill=fillColor,fill opacity=0.20] ( 82.16, 72.73) circle (  2.13);

\path[fill=fillColor,fill opacity=0.20] ( 77.36, 70.40) circle (  2.13);

\path[fill=fillColor,fill opacity=0.20] ( 76.70, 63.86) circle (  2.13);

\path[fill=fillColor,fill opacity=0.20] ( 77.79, 60.59) circle (  2.13);

\path[fill=fillColor,fill opacity=0.20] ( 79.32, 66.79) circle (  2.13);

\path[fill=fillColor,fill opacity=0.20] ( 75.61, 63.77) circle (  2.13);

\path[fill=fillColor,fill opacity=0.20] ( 74.51, 67.22) circle (  2.13);

\path[fill=fillColor,fill opacity=0.20] ( 70.80, 75.48) circle (  2.13);

\path[fill=fillColor,fill opacity=0.20] ( 68.62, 75.92) circle (  2.13);

\path[fill=fillColor,fill opacity=0.20] ( 62.50, 74.02) circle (  2.13);

\path[fill=fillColor,fill opacity=0.20] ( 59.66, 86.25) circle (  2.13);

\path[fill=fillColor,fill opacity=0.20] ( 52.66,104.59) circle (  2.13);

\path[fill=fillColor,fill opacity=0.20] ( 69.71, 91.59) circle (  2.13);

\path[fill=fillColor,fill opacity=0.20] ( 73.86, 75.31) circle (  2.13);

\path[fill=fillColor,fill opacity=0.20] ( 76.04, 77.47) circle (  2.13);

\path[fill=fillColor,fill opacity=0.20] ( 85.66, 76.17) circle (  2.13);

\path[fill=fillColor,fill opacity=0.20] ( 77.36, 57.66) circle (  2.13);

\path[fill=fillColor,fill opacity=0.20] ( 85.88, 54.04) circle (  2.13);

\path[fill=fillColor,fill opacity=0.20] ( 87.84, 70.83) circle (  2.13);

\path[fill=fillColor,fill opacity=0.20] (101.17, 73.93) circle (  2.13);

\path[fill=fillColor,fill opacity=0.20] ( 85.66, 66.96) circle (  2.13);

\path[fill=fillColor,fill opacity=0.20] ( 90.90, 57.06) circle (  2.13);

\path[fill=fillColor,fill opacity=0.20] ( 90.68, 50.60) circle (  2.13);

\path[fill=fillColor,fill opacity=0.20] ( 91.99, 53.96) circle (  2.13);

\path[fill=fillColor,fill opacity=0.20] ( 89.81, 56.37) circle (  2.13);

\path[fill=fillColor,fill opacity=0.20] ( 86.97, 58.52) circle (  2.13);

\path[fill=fillColor,fill opacity=0.20] ( 98.99, 67.30) circle (  2.13);

\path[fill=fillColor,fill opacity=0.20] ( 99.86, 64.29) circle (  2.13);

\path[fill=fillColor,fill opacity=0.20] ( 99.64, 51.20) circle (  2.13);

\path[fill=fillColor,fill opacity=0.20] (100.08, 50.77) circle (  2.13);

\path[fill=fillColor,fill opacity=0.20] (101.39, 51.03) circle (  2.13);

\path[fill=fillColor,fill opacity=0.20] (108.16, 47.76) circle (  2.13);

\path[fill=fillColor,fill opacity=0.20] ( 97.02, 51.20) circle (  2.13);

\path[fill=fillColor,fill opacity=0.20] (117.34, 50.60) circle (  2.13);

\path[fill=fillColor,fill opacity=0.20] (136.79, 48.88) circle (  2.13);

\path[fill=fillColor,fill opacity=0.20] (103.36, 83.15) circle (  2.13);

\path[fill=fillColor,fill opacity=0.20] ( 77.57, 75.48) circle (  2.13);

\path[fill=fillColor,fill opacity=0.20] ( 73.42, 79.45) circle (  2.13);

\path[fill=fillColor,fill opacity=0.20] ( 80.41, 74.02) circle (  2.13);

\path[fill=fillColor,fill opacity=0.20] ( 79.98, 71.78) circle (  2.13);

\path[fill=fillColor,fill opacity=0.20] ( 78.88, 70.75) circle (  2.13);

\path[fill=fillColor,fill opacity=0.20] ( 76.04, 71.01) circle (  2.13);

\path[fill=fillColor,fill opacity=0.20] ( 76.92, 69.71) circle (  2.13);

\path[fill=fillColor,fill opacity=0.20] ( 71.89, 69.11) circle (  2.13);

\path[fill=fillColor,fill opacity=0.20] ( 68.62, 68.51) circle (  2.13);

\path[fill=fillColor,fill opacity=0.20] ( 71.46, 65.32) circle (  2.13);

\path[fill=fillColor,fill opacity=0.20] ( 68.18, 64.63) circle (  2.13);

\path[fill=fillColor,fill opacity=0.20] ( 65.56, 68.60) circle (  2.13);

\path[fill=fillColor,fill opacity=0.20] ( 59.44, 87.37) circle (  2.13);

\path[fill=fillColor,fill opacity=0.20] ( 53.54,105.28) circle (  2.13);

\path[fill=fillColor,fill opacity=0.20] ( 58.35, 90.12) circle (  2.13);

\path[fill=fillColor,fill opacity=0.20] ( 64.03, 83.92) circle (  2.13);

\path[fill=fillColor,fill opacity=0.20] ( 72.11, 88.32) circle (  2.13);

\path[fill=fillColor,fill opacity=0.20] ( 64.68, 96.15) circle (  2.13);

\path[fill=fillColor,fill opacity=0.20] ( 70.14, 82.20) circle (  2.13);

\path[fill=fillColor,fill opacity=0.20] ( 69.49, 83.06) circle (  2.13);

\path[fill=fillColor,fill opacity=0.20] ( 70.58, 87.02) circle (  2.13);

\path[fill=fillColor,fill opacity=0.20] ( 71.46, 81.86) circle (  2.13);

\path[fill=fillColor,fill opacity=0.20] ( 76.04, 77.03) circle (  2.13);

\path[fill=fillColor,fill opacity=0.20] ( 80.20, 76.69) circle (  2.13);

\path[fill=fillColor,fill opacity=0.20] ( 79.76, 78.67) circle (  2.13);

\path[fill=fillColor,fill opacity=0.20] ( 81.94, 67.30) circle (  2.13);

\path[fill=fillColor,fill opacity=0.20] ( 86.53, 60.07) circle (  2.13);

\path[fill=fillColor,fill opacity=0.20] ( 83.25, 64.81) circle (  2.13);

\path[fill=fillColor,fill opacity=0.20] ( 84.13, 69.89) circle (  2.13);

\path[fill=fillColor,fill opacity=0.20] ( 91.34, 69.71) circle (  2.13);

\path[fill=fillColor,fill opacity=0.20] ( 98.55, 59.21) circle (  2.13);

\path[fill=fillColor,fill opacity=0.20] ( 93.09, 49.91) circle (  2.13);

\path[fill=fillColor,fill opacity=0.20] ( 84.57, 59.21) circle (  2.13);

\path[fill=fillColor,fill opacity=0.20] ( 87.62, 67.05) circle (  2.13);

\path[fill=fillColor,fill opacity=0.20] ( 95.71, 64.89) circle (  2.13);

\path[fill=fillColor,fill opacity=0.20] (100.95, 60.16) circle (  2.13);

\path[fill=fillColor,fill opacity=0.20] (100.73, 49.65) circle (  2.13);

\path[fill=fillColor,fill opacity=0.20] ( 92.87, 51.54) circle (  2.13);

\path[fill=fillColor,fill opacity=0.20] ( 99.64, 54.99) circle (  2.13);

\path[fill=fillColor,fill opacity=0.20] ( 98.99, 48.01) circle (  2.13);

\path[fill=fillColor,fill opacity=0.20] ( 95.93, 52.41) circle (  2.13);

\path[fill=fillColor,fill opacity=0.20] ( 91.78, 54.47) circle (  2.13);

\path[fill=fillColor,fill opacity=0.20] (115.37, 48.44) circle (  2.13);

\path[fill=fillColor,fill opacity=0.20] (123.24, 57.06) circle (  2.13);

\path[fill=fillColor,fill opacity=0.20] ( 82.60, 85.47) circle (  2.13);

\path[fill=fillColor,fill opacity=0.20] ( 92.65, 84.44) circle (  2.13);

\path[fill=fillColor,fill opacity=0.20] ( 75.17, 74.19) circle (  2.13);

\path[fill=fillColor,fill opacity=0.20] ( 79.10, 64.81) circle (  2.13);

\path[fill=fillColor,fill opacity=0.20] ( 77.36, 71.35) circle (  2.13);

\path[fill=fillColor,fill opacity=0.20] ( 72.99, 73.33) circle (  2.13);

\path[fill=fillColor,fill opacity=0.20] ( 69.71, 67.73) circle (  2.13);

\path[fill=fillColor,fill opacity=0.20] ( 76.48, 70.75) circle (  2.13);

\path[fill=fillColor,fill opacity=0.20] ( 79.32, 69.80) circle (  2.13);

\path[fill=fillColor,fill opacity=0.20] ( 76.26, 66.53) circle (  2.13);

\path[fill=fillColor,fill opacity=0.20] ( 71.24, 72.04) circle (  2.13);

\path[fill=fillColor,fill opacity=0.20] ( 55.29, 96.15) circle (  2.13);

\path[fill=fillColor,fill opacity=0.20] ( 56.82, 82.72) circle (  2.13);

\path[fill=fillColor,fill opacity=0.20] ( 47.20,112.51) circle (  2.13);

\path[fill=fillColor,fill opacity=0.20] ( 69.27,106.31) circle (  2.13);

\path[fill=fillColor,fill opacity=0.20] ( 54.85, 94.34) circle (  2.13);

\path[fill=fillColor,fill opacity=0.20] ( 63.81, 90.55) circle (  2.13);

\path[fill=fillColor,fill opacity=0.20] ( 65.77, 91.50) circle (  2.13);

\path[fill=fillColor,fill opacity=0.20] ( 62.72, 65.32) circle (  2.13);

\path[fill=fillColor,fill opacity=0.20] ( 74.08, 53.44) circle (  2.13);

\path[fill=fillColor,fill opacity=0.20] ( 81.29, 71.78) circle (  2.13);

\path[fill=fillColor,fill opacity=0.20] ( 75.17, 78.15) circle (  2.13);

\path[fill=fillColor,fill opacity=0.20] ( 80.85, 78.07) circle (  2.13);

\path[fill=fillColor,fill opacity=0.20] ( 78.23, 94.17) circle (  2.13);

\path[fill=fillColor,fill opacity=0.20] (109.69, 97.27) circle (  2.13);

\path[fill=fillColor,fill opacity=0.20] ( 72.77, 94.60) circle (  2.13);

\path[fill=fillColor,fill opacity=0.20] ( 82.82, 82.20) circle (  2.13);

\path[fill=fillColor,fill opacity=0.20] ( 83.91, 63.95) circle (  2.13);

\path[fill=fillColor,fill opacity=0.20] ( 93.96, 62.74) circle (  2.13);

\path[fill=fillColor,fill opacity=0.20] ( 92.43, 60.67) circle (  2.13);

\path[fill=fillColor,fill opacity=0.20] ( 93.52, 64.20) circle (  2.13);

\path[fill=fillColor,fill opacity=0.20] ( 97.02, 73.25) circle (  2.13);

\path[fill=fillColor,fill opacity=0.20] ( 95.49, 78.15) circle (  2.13);

\path[fill=fillColor,fill opacity=0.20] ( 89.81, 59.55) circle (  2.13);

\path[fill=fillColor,fill opacity=0.20] ( 91.34, 64.55) circle (  2.13);

\path[fill=fillColor,fill opacity=0.20] ( 94.62, 67.56) circle (  2.13);

\path[fill=fillColor,fill opacity=0.20] ( 93.31, 61.02) circle (  2.13);

\path[fill=fillColor,fill opacity=0.20] ( 95.05, 51.37) circle (  2.13);

\path[fill=fillColor,fill opacity=0.20] ( 86.10, 46.12) circle (  2.13);

\path[fill=fillColor,fill opacity=0.20] ( 93.52, 58.35) circle (  2.13);

\path[fill=fillColor,fill opacity=0.20] ( 93.31, 65.15) circle (  2.13);

\path[fill=fillColor,fill opacity=0.20] ( 90.03, 55.33) circle (  2.13);

\path[fill=fillColor,fill opacity=0.20] ( 85.00, 57.06) circle (  2.13);

\path[fill=fillColor,fill opacity=0.20] ( 92.87, 57.23) circle (  2.13);

\path[fill=fillColor,fill opacity=0.20] ( 93.09, 53.61) circle (  2.13);

\path[fill=fillColor,fill opacity=0.20] ( 99.64, 64.63) circle (  2.13);

\path[fill=fillColor,fill opacity=0.20] ( 79.32, 73.42) circle (  2.13);

\path[fill=fillColor,fill opacity=0.20] ( 77.14, 78.15) circle (  2.13);

\path[fill=fillColor,fill opacity=0.20] ( 81.94, 77.64) circle (  2.13);

\path[fill=fillColor,fill opacity=0.20] ( 77.36, 89.00) circle (  2.13);

\path[fill=fillColor,fill opacity=0.20] ( 71.67, 76.86) circle (  2.13);

\path[fill=fillColor,fill opacity=0.20] ( 76.04, 61.45) circle (  2.13);

\path[fill=fillColor,fill opacity=0.20] ( 74.08, 65.15) circle (  2.13);

\path[fill=fillColor,fill opacity=0.20] ( 73.42, 69.71) circle (  2.13);

\path[fill=fillColor,fill opacity=0.20] ( 70.58, 74.97) circle (  2.13);

\path[fill=fillColor,fill opacity=0.20] ( 69.93, 81.34) circle (  2.13);

\path[fill=fillColor,fill opacity=0.20] ( 75.39, 67.48) circle (  2.13);

\path[fill=fillColor,fill opacity=0.20] ( 69.49, 53.78) circle (  2.13);

\path[fill=fillColor,fill opacity=0.20] ( 68.62, 76.17) circle (  2.13);

\path[fill=fillColor,fill opacity=0.20] ( 68.83, 81.68) circle (  2.13);

\path[fill=fillColor,fill opacity=0.20] ( 63.81, 66.61) circle (  2.13);

\path[fill=fillColor,fill opacity=0.20] ( 64.25, 69.11) circle (  2.13);

\path[fill=fillColor,fill opacity=0.20] ( 64.03, 73.33) circle (  2.13);

\path[fill=fillColor,fill opacity=0.20] ( 71.24, 76.86) circle (  2.13);

\path[fill=fillColor,fill opacity=0.20] ( 64.25, 81.43) circle (  2.13);

\path[fill=fillColor,fill opacity=0.20] ( 68.18, 73.42) circle (  2.13);

\path[fill=fillColor,fill opacity=0.20] ( 75.61, 68.25) circle (  2.13);

\path[fill=fillColor,fill opacity=0.20] (115.81, 71.52) circle (  2.13);

\path[fill=fillColor,fill opacity=0.20] ( 78.01, 67.22) circle (  2.13);

\path[fill=fillColor,fill opacity=0.20] ( 71.67, 59.90) circle (  2.13);

\path[fill=fillColor,fill opacity=0.20] (101.83, 67.39) circle (  2.13);

\path[fill=fillColor,fill opacity=0.20] ( 82.16, 82.89) circle (  2.13);

\path[fill=fillColor,fill opacity=0.20] ( 75.61, 84.27) circle (  2.13);

\path[fill=fillColor,fill opacity=0.20] ( 82.16, 66.53) circle (  2.13);

\path[fill=fillColor,fill opacity=0.20] ( 77.36, 59.81) circle (  2.13);

\path[fill=fillColor,fill opacity=0.20] ( 81.73, 82.72) circle (  2.13);

\path[fill=fillColor,fill opacity=0.20] ( 80.63, 85.39) circle (  2.13);

\path[fill=fillColor,fill opacity=0.20] ( 87.19, 84.44) circle (  2.13);

\path[fill=fillColor,fill opacity=0.20] ( 85.22, 75.40) circle (  2.13);

\path[fill=fillColor,fill opacity=0.20] ( 92.87, 77.47) circle (  2.13);

\path[fill=fillColor,fill opacity=0.20] (104.45, 76.60) circle (  2.13);

\path[fill=fillColor,fill opacity=0.20] ( 94.62, 82.12) circle (  2.13);

\path[fill=fillColor,fill opacity=0.20] ( 84.57, 73.76) circle (  2.13);

\path[fill=fillColor,fill opacity=0.20] ( 89.15, 68.68) circle (  2.13);

\path[fill=fillColor,fill opacity=0.20] ( 93.52, 67.22) circle (  2.13);

\path[fill=fillColor,fill opacity=0.20] ( 90.03, 62.65) circle (  2.13);

\path[fill=fillColor,fill opacity=0.20] ( 92.65, 57.49) circle (  2.13);

\path[fill=fillColor,fill opacity=0.20] ( 94.40, 60.33) circle (  2.13);

\path[fill=fillColor,fill opacity=0.20] ( 92.21, 59.47) circle (  2.13);

\path[fill=fillColor,fill opacity=0.20] ( 87.41, 64.46) circle (  2.13);

\path[fill=fillColor,fill opacity=0.20] ( 89.37, 67.73) circle (  2.13);

\path[fill=fillColor,fill opacity=0.20] ( 85.22, 64.72) circle (  2.13);

\path[fill=fillColor,fill opacity=0.20] ( 78.88, 68.08) circle (  2.13);

\path[fill=fillColor,fill opacity=0.20] ( 90.68, 65.75) circle (  2.13);

\path[fill=fillColor,fill opacity=0.20] ( 94.40, 61.53) circle (  2.13);

\path[fill=fillColor,fill opacity=0.20] ( 97.24, 76.17) circle (  2.13);

\path[fill=fillColor,fill opacity=0.20] ( 80.20, 78.50) circle (  2.13);

\path[fill=fillColor,fill opacity=0.20] ( 77.36, 78.76) circle (  2.13);

\path[fill=fillColor,fill opacity=0.20] ( 71.46, 80.22) circle (  2.13);

\path[fill=fillColor,fill opacity=0.20] ( 81.73, 75.31) circle (  2.13);

\path[fill=fillColor,fill opacity=0.20] ( 77.79, 72.04) circle (  2.13);

\path[fill=fillColor,fill opacity=0.20] ( 80.85, 67.56) circle (  2.13);

\path[fill=fillColor,fill opacity=0.20] ( 74.73, 61.28) circle (  2.13);

\path[fill=fillColor,fill opacity=0.20] ( 75.39, 64.03) circle (  2.13);

\path[fill=fillColor,fill opacity=0.20] ( 76.04, 64.72) circle (  2.13);

\path[fill=fillColor,fill opacity=0.20] ( 84.78, 60.07) circle (  2.13);

\path[fill=fillColor,fill opacity=0.20] ( 78.45, 57.83) circle (  2.13);

\path[fill=fillColor,fill opacity=0.20] ( 81.73, 63.34) circle (  2.13);

\path[fill=fillColor,fill opacity=0.20] ( 76.04, 71.78) circle (  2.13);

\path[fill=fillColor,fill opacity=0.20] ( 74.30, 66.61) circle (  2.13);

\path[fill=fillColor,fill opacity=0.20] ( 77.79, 63.26) circle (  2.13);

\path[fill=fillColor,fill opacity=0.20] ( 79.76, 65.06) circle (  2.13);

\path[fill=fillColor,fill opacity=0.20] ( 81.29, 68.34) circle (  2.13);

\path[fill=fillColor,fill opacity=0.20] ( 83.91, 72.82) circle (  2.13);

\path[fill=fillColor,fill opacity=0.20] (123.02, 71.61) circle (  2.13);

\path[fill=fillColor,fill opacity=0.20] ( 91.78, 63.69) circle (  2.13);

\path[fill=fillColor,fill opacity=0.20] ( 88.06, 69.54) circle (  2.13);

\path[fill=fillColor,fill opacity=0.20] ( 86.10, 84.96) circle (  2.13);

\path[fill=fillColor,fill opacity=0.20] ( 94.18, 86.42) circle (  2.13);

\path[fill=fillColor,fill opacity=0.20] ( 98.99, 72.99) circle (  2.13);

\path[fill=fillColor,fill opacity=0.20] ( 93.96, 62.83) circle (  2.13);

\path[fill=fillColor,fill opacity=0.20] ( 90.03, 70.83) circle (  2.13);

\path[fill=fillColor,fill opacity=0.20] ( 96.80, 76.86) circle (  2.13);

\path[fill=fillColor,fill opacity=0.20] ( 94.18, 66.61) circle (  2.13);

\path[fill=fillColor,fill opacity=0.20] ( 87.84, 62.91) circle (  2.13);

\path[fill=fillColor,fill opacity=0.20] ( 91.99, 61.36) circle (  2.13);

\path[fill=fillColor,fill opacity=0.20] ( 86.75, 61.10) circle (  2.13);

\path[fill=fillColor,fill opacity=0.20] ( 86.75, 72.38) circle (  2.13);

\path[fill=fillColor,fill opacity=0.20] ( 93.52, 69.89) circle (  2.13);

\path[fill=fillColor,fill opacity=0.20] (102.48, 71.35) circle (  2.13);

\path[fill=fillColor,fill opacity=0.20] ( 78.67, 76.09) circle (  2.13);

\path[fill=fillColor,fill opacity=0.20] ( 84.13, 74.19) circle (  2.13);

\path[fill=fillColor,fill opacity=0.20] ( 84.35, 76.52) circle (  2.13);

\path[fill=fillColor,fill opacity=0.20] ( 84.35, 79.10) circle (  2.13);

\path[fill=fillColor,fill opacity=0.20] ( 73.42, 80.57) circle (  2.13);

\path[fill=fillColor,fill opacity=0.20] ( 75.61, 73.85) circle (  2.13);

\path[fill=fillColor,fill opacity=0.20] ( 79.32, 65.41) circle (  2.13);

\path[fill=fillColor,fill opacity=0.20] ( 76.26, 57.23) circle (  2.13);

\path[fill=fillColor,fill opacity=0.20] ( 79.98, 55.51) circle (  2.13);

\path[fill=fillColor,fill opacity=0.20] ( 75.39, 69.89) circle (  2.13);

\path[fill=fillColor,fill opacity=0.20] ( 76.70, 77.81) circle (  2.13);

\path[fill=fillColor,fill opacity=0.20] ( 83.91, 65.41) circle (  2.13);

\path[fill=fillColor,fill opacity=0.20] ( 82.82, 60.16) circle (  2.13);

\path[fill=fillColor,fill opacity=0.20] ( 82.82, 64.98) circle (  2.13);

\path[fill=fillColor,fill opacity=0.20] (100.30, 67.65) circle (  2.13);

\path[fill=fillColor,fill opacity=0.20] ( 81.29, 72.21) circle (  2.13);

\path[fill=fillColor,fill opacity=0.20] ( 84.35, 76.52) circle (  2.13);

\path[fill=fillColor,fill opacity=0.20] ( 85.00, 79.53) circle (  2.13);

\path[fill=fillColor,fill opacity=0.20] ( 92.65, 79.53) circle (  2.13);

\path[fill=fillColor,fill opacity=0.20] (100.52, 77.64) circle (  2.13);

\path[fill=fillColor,fill opacity=0.20] ( 84.57, 84.18) circle (  2.13);

\path[fill=fillColor,fill opacity=0.20] ( 94.18, 94.69) circle (  2.13);

\path[fill=fillColor,fill opacity=0.20] (105.54, 91.85) circle (  2.13);

\path[fill=fillColor,fill opacity=0.20] ( 88.50, 62.40) circle (  2.13);

\path[fill=fillColor,fill opacity=0.20] (108.16, 69.89) circle (  2.13);

\path[fill=fillColor,fill opacity=0.20] ( 82.38, 78.93) circle (  2.13);

\path[fill=fillColor,fill opacity=0.20] ( 89.59, 69.80) circle (  2.13);

\path[fill=fillColor,fill opacity=0.20] ( 80.41, 64.81) circle (  2.13);

\path[fill=fillColor,fill opacity=0.20] ( 86.75, 75.31) circle (  2.13);

\path[fill=fillColor,fill opacity=0.20] ( 82.82, 87.37) circle (  2.13);

\path[fill=fillColor,fill opacity=0.20] ( 97.02, 76.26) circle (  2.13);

\path[fill=fillColor,fill opacity=0.20] ( 87.41, 63.51) circle (  2.13);

\path[fill=fillColor,fill opacity=0.20] ( 90.68, 73.85) circle (  2.13);

\path[fill=fillColor,fill opacity=0.20] ( 94.18, 81.43) circle (  2.13);

\path[fill=fillColor,fill opacity=0.20] ( 89.59, 80.65) circle (  2.13);

\path[fill=fillColor,fill opacity=0.20] ( 88.72, 81.77) circle (  2.13);

\path[fill=fillColor,fill opacity=0.20] ( 89.37, 78.24) circle (  2.13);

\path[fill=fillColor,fill opacity=0.20] ( 97.46, 88.23) circle (  2.13);

\path[fill=fillColor,fill opacity=0.20] ( 95.93, 95.64) circle (  2.13);

\path[fill=fillColor,fill opacity=0.20] ( 88.94, 87.63) circle (  2.13);

\path[fill=fillColor,fill opacity=0.20] (100.08, 95.12) circle (  2.13);

\path[fill=fillColor,fill opacity=0.20] (101.83, 79.96) circle (  2.13);

\path[fill=fillColor,fill opacity=0.20] (100.95, 77.03) circle (  2.13);

\path[fill=fillColor,fill opacity=0.20] ( 90.47, 69.71) circle (  2.13);

\path[fill=fillColor,fill opacity=0.20] ( 93.52, 89.61) circle (  2.13);

\path[fill=fillColor,fill opacity=0.20] (108.60, 59.64) circle (  2.13);

\path[fill=fillColor,fill opacity=0.20] (130.23, 50.77) circle (  2.13);

\path[fill=fillColor,fill opacity=0.20] (102.92, 52.06) circle (  2.13);

\path[fill=fillColor,fill opacity=0.20] ( 93.52, 61.53) circle (  2.13);

\path[fill=fillColor,fill opacity=0.20] ( 89.15, 73.85) circle (  2.13);

\path[fill=fillColor,fill opacity=0.20] ( 79.10,115.10) circle (  2.13);

\path[fill=fillColor,fill opacity=0.20] ( 82.82,112.51) circle (  2.13);

\path[fill=fillColor,fill opacity=0.20] ( 78.67, 99.17) circle (  2.13);

\path[fill=fillColor,fill opacity=0.20] ( 76.70, 96.41) circle (  2.13);

\path[fill=fillColor,fill opacity=0.20] ( 85.44,107.35) circle (  2.13);

\path[fill=fillColor,fill opacity=0.20] ( 92.87,109.67) circle (  2.13);

\path[fill=fillColor,fill opacity=0.20] (100.95, 55.25) circle (  2.13);

\path[fill=fillColor,fill opacity=0.20] (110.13, 58.69) circle (  2.13);

\path[fill=fillColor,fill opacity=0.20] (136.57, 59.30) circle (  2.13);

\path[fill=fillColor,fill opacity=0.20] (110.57, 48.96) circle (  2.13);

\path[fill=fillColor,fill opacity=0.20] ( 97.02, 58.61) circle (  2.13);

\path[fill=fillColor,fill opacity=0.20] (119.53, 74.45) circle (  2.13);

\path[fill=fillColor,fill opacity=0.20] ( 85.88, 74.62) circle (  2.13);

\path[fill=fillColor,fill opacity=0.20] ( 72.77, 81.00) circle (  2.13);

\path[fill=fillColor,fill opacity=0.20] ( 92.21, 91.16) circle (  2.13);

\path[fill=fillColor,fill opacity=0.20] ( 84.57, 76.43) circle (  2.13);

\path[fill=fillColor,fill opacity=0.20] ( 85.66, 74.11) circle (  2.13);

\path[fill=fillColor,fill opacity=0.20] ( 87.19, 74.80) circle (  2.13);

\path[fill=fillColor,fill opacity=0.20] ( 90.25, 69.71) circle (  2.13);

\path[fill=fillColor,fill opacity=0.20] ( 95.27, 64.46) circle (  2.13);

\path[fill=fillColor,fill opacity=0.20] ( 94.84, 72.47) circle (  2.13);

\path[fill=fillColor,fill opacity=0.20] ( 81.29, 87.37) circle (  2.13);

\path[fill=fillColor,fill opacity=0.20] ( 67.52, 98.22) circle (  2.13);

\path[fill=fillColor,fill opacity=0.20] (105.10, 76.69) circle (  2.13);

\path[fill=fillColor,fill opacity=0.20] (107.73, 46.55) circle (  2.13);

\path[fill=fillColor,fill opacity=0.20] (122.58, 70.66) circle (  2.13);

\path[fill=fillColor,fill opacity=0.20] (123.90, 65.67) circle (  2.13);

\path[fill=fillColor,fill opacity=0.20] (136.35, 57.57) circle (  2.13);

\path[fill=fillColor,fill opacity=0.20] (100.08, 66.18) circle (  2.13);

\path[fill=fillColor,fill opacity=0.20] ( 98.33, 69.11) circle (  2.13);

\path[fill=fillColor,fill opacity=0.20] ( 98.55, 63.60) circle (  2.13);

\path[fill=fillColor,fill opacity=0.20] ( 98.11, 59.90) circle (  2.13);

\path[fill=fillColor,fill opacity=0.20] ( 77.36, 65.75) circle (  2.13);

\path[fill=fillColor,fill opacity=0.20] ( 88.50,101.49) circle (  2.13);

\path[fill=fillColor,fill opacity=0.20] ( 92.87, 54.90) circle (  2.13);

\path[fill=fillColor,fill opacity=0.20] ( 96.36, 41.90) circle (  2.13);

\path[fill=fillColor,fill opacity=0.20] ( 96.36, 49.82) circle (  2.13);

\path[fill=fillColor,fill opacity=0.20] ( 87.84, 60.85) circle (  2.13);

\path[fill=fillColor,fill opacity=0.20] ( 90.90, 64.46) circle (  2.13);

\path[fill=fillColor,fill opacity=0.20] ( 91.34, 59.98) circle (  2.13);

\path[fill=fillColor,fill opacity=0.20] (126.52, 59.73) circle (  2.13);

\path[fill=fillColor,fill opacity=0.20] ( 74.51, 65.58) circle (  2.13);

\path[fill=fillColor,fill opacity=0.20] ( 60.97, 71.95) circle (  2.13);

\path[fill=fillColor,fill opacity=0.20] (106.85, 74.37) circle (  2.13);

\path[fill=fillColor,fill opacity=0.20] (113.84, 52.32) circle (  2.13);

\path[fill=fillColor,fill opacity=0.20] (115.37, 73.25) circle (  2.13);

\path[fill=fillColor,fill opacity=0.20] (151.43, 65.84) circle (  2.13);

\path[fill=fillColor,fill opacity=0.20] (123.02, 73.68) circle (  2.13);

\path[fill=fillColor,fill opacity=0.20] (101.17, 58.78) circle (  2.13);

\path[fill=fillColor,fill opacity=0.20] ( 91.56, 48.27) circle (  2.13);

\path[fill=fillColor,fill opacity=0.20] ( 89.37, 65.75) circle (  2.13);

\path[fill=fillColor,fill opacity=0.20] ( 85.44, 72.99) circle (  2.13);

\path[fill=fillColor,fill opacity=0.20] ( 76.48, 70.58) circle (  2.13);

\path[fill=fillColor,fill opacity=0.20] ( 94.84, 95.72) circle (  2.13);

\path[fill=fillColor,fill opacity=0.20] (102.05, 48.96) circle (  2.13);

\path[fill=fillColor,fill opacity=0.20] (106.85, 39.23) circle (  2.13);

\path[fill=fillColor,fill opacity=0.20] (113.41, 45.78) circle (  2.13);

\path[fill=fillColor,fill opacity=0.20] (123.24, 48.88) circle (  2.13);

\path[fill=fillColor,fill opacity=0.20] (114.50, 50.77) circle (  2.13);

\path[fill=fillColor,fill opacity=0.20] (106.63, 50.77) circle (  2.13);

\path[fill=fillColor,fill opacity=0.20] ( 97.46, 49.74) circle (  2.13);

\path[fill=fillColor,fill opacity=0.20] ( 81.51, 60.24) circle (  2.13);

\path[fill=fillColor,fill opacity=0.20] ( 73.64, 72.21) circle (  2.13);

\path[fill=fillColor,fill opacity=0.20] ( 76.48, 66.61) circle (  2.13);

\path[fill=fillColor,fill opacity=0.20] ( 52.01, 74.45) circle (  2.13);

\path[fill=fillColor,fill opacity=0.20] (102.70, 80.82) circle (  2.13);

\path[fill=fillColor,fill opacity=0.20] (112.53, 49.91) circle (  2.13);

\path[fill=fillColor,fill opacity=0.20] (124.55, 66.01) circle (  2.13);

\path[fill=fillColor,fill opacity=0.20] (124.11, 56.71) circle (  2.13);

\path[fill=fillColor,fill opacity=0.20] (145.53, 63.51) circle (  2.13);

\path[fill=fillColor,fill opacity=0.20] (117.78, 49.05) circle (  2.13);

\path[fill=fillColor,fill opacity=0.20] ( 94.18, 42.68) circle (  2.13);

\path[fill=fillColor,fill opacity=0.20] ( 88.50, 64.63) circle (  2.13);

\path[fill=fillColor,fill opacity=0.20] ( 92.21, 69.28) circle (  2.13);

\path[fill=fillColor,fill opacity=0.20] ( 80.20, 66.79) circle (  2.13);

\path[fill=fillColor,fill opacity=0.20] (103.14, 56.97) circle (  2.13);

\path[fill=fillColor,fill opacity=0.20] (125.43, 40.95) circle (  2.13);

\path[fill=fillColor,fill opacity=0.20] (131.11, 46.64) circle (  2.13);

\path[fill=fillColor,fill opacity=0.20] (148.80, 49.91) circle (  2.13);

\path[fill=fillColor,fill opacity=0.20] (109.47, 43.97) circle (  2.13);

\path[fill=fillColor,fill opacity=0.20] ( 95.93, 38.37) circle (  2.13);

\path[fill=fillColor,fill opacity=0.20] ( 83.69, 51.54) circle (  2.13);

\path[fill=fillColor,fill opacity=0.20] ( 66.21, 68.94) circle (  2.13);

\path[fill=fillColor,fill opacity=0.20] ( 83.69, 70.92) circle (  2.13);

\path[fill=fillColor,fill opacity=0.20] ( 96.15, 41.81) circle (  2.13);

\path[fill=fillColor,fill opacity=0.20] (110.35, 57.06) circle (  2.13);

\path[fill=fillColor,fill opacity=0.20] (124.11, 52.92) circle (  2.13);

\path[fill=fillColor,fill opacity=0.20] (114.28, 43.54) circle (  2.13);

\path[fill=fillColor,fill opacity=0.20] (111.22, 47.67) circle (  2.13);

\path[fill=fillColor,fill opacity=0.20] (112.32, 48.19) circle (  2.13);

\path[fill=fillColor,fill opacity=0.20] ( 99.64, 51.37) circle (  2.13);

\path[fill=fillColor,fill opacity=0.20] ( 89.59, 56.45) circle (  2.13);

\path[fill=fillColor,fill opacity=0.20] (114.28, 54.64) circle (  2.13);

\path[fill=fillColor,fill opacity=0.20] ( 87.62, 93.91) circle (  2.13);

\path[fill=fillColor,fill opacity=0.20] ( 99.42, 41.56) circle (  2.13);

\path[fill=fillColor,fill opacity=0.20] (128.27, 38.54) circle (  2.13);

\path[fill=fillColor,fill opacity=0.20] ( 95.49, 55.33) circle (  2.13);

\path[fill=fillColor,fill opacity=0.20] ( 83.91, 56.71) circle (  2.13);

\path[fill=fillColor,fill opacity=0.20] ( 77.79, 51.46) circle (  2.13);

\path[fill=fillColor,fill opacity=0.20] ( 63.15, 57.75) circle (  2.13);

\path[fill=fillColor,fill opacity=0.20] ( 91.78, 65.24) circle (  2.13);

\path[fill=fillColor,fill opacity=0.20] ( 90.03, 52.49) circle (  2.13);

\path[fill=fillColor,fill opacity=0.20] ( 94.62, 60.41) circle (  2.13);

\path[fill=fillColor,fill opacity=0.20] (121.49, 56.88) circle (  2.13);

\path[fill=fillColor,fill opacity=0.20] (106.20, 47.24) circle (  2.13);

\path[fill=fillColor,fill opacity=0.20] (107.29, 49.99) circle (  2.13);

\path[fill=fillColor,fill opacity=0.20] (111.22, 56.11) circle (  2.13);

\path[fill=fillColor,fill opacity=0.20] (110.13, 52.84) circle (  2.13);

\path[fill=fillColor,fill opacity=0.20] ( 95.27, 50.25) circle (  2.13);

\path[fill=fillColor,fill opacity=0.20] ( 91.34, 53.18) circle (  2.13);

\path[fill=fillColor,fill opacity=0.20] ( 98.77, 74.62) circle (  2.13);

\path[fill=fillColor,fill opacity=0.20] (112.97, 40.69) circle (  2.13);

\path[fill=fillColor,fill opacity=0.20] (120.84, 46.38) circle (  2.13);

\path[fill=fillColor,fill opacity=0.20] (132.42, 39.83) circle (  2.13);

\path[fill=fillColor,fill opacity=0.20] (105.76, 45.52) circle (  2.13);

\path[fill=fillColor,fill opacity=0.20] (105.32, 55.16) circle (  2.13);

\path[fill=fillColor,fill opacity=0.20] (101.61, 63.43) circle (  2.13);

\path[fill=fillColor,fill opacity=0.20] ( 78.01, 67.48) circle (  2.13);

\path[fill=fillColor,fill opacity=0.20] ( 68.18, 64.03) circle (  2.13);

\path[fill=fillColor,fill opacity=0.20] ( 76.26, 69.37) circle (  2.13);

\path[fill=fillColor,fill opacity=0.20] ( 90.25, 62.48) circle (  2.13);

\path[fill=fillColor,fill opacity=0.20] ( 99.21, 57.57) circle (  2.13);

\path[fill=fillColor,fill opacity=0.20] ( 92.65, 58.26) circle (  2.13);

\path[fill=fillColor,fill opacity=0.20] ( 88.06, 56.37) circle (  2.13);

\path[fill=fillColor,fill opacity=0.20] ( 94.84, 56.20) circle (  2.13);

\path[fill=fillColor,fill opacity=0.20] ( 99.64, 57.06) circle (  2.13);

\path[fill=fillColor,fill opacity=0.20] (117.56, 52.32) circle (  2.13);

\path[fill=fillColor,fill opacity=0.20] (109.47, 47.84) circle (  2.13);

\path[fill=fillColor,fill opacity=0.20] ( 98.77, 52.06) circle (  2.13);

\path[fill=fillColor,fill opacity=0.20] ( 82.38, 57.92) circle (  2.13);

\path[fill=fillColor,fill opacity=0.20] ( 65.77, 65.58) circle (  2.13);

\path[fill=fillColor,fill opacity=0.20] ( 59.88, 76.60) circle (  2.13);

\path[fill=fillColor,fill opacity=0.20] ( 97.02, 69.63) circle (  2.13);

\path[fill=fillColor,fill opacity=0.20] (118.65, 43.54) circle (  2.13);

\path[fill=fillColor,fill opacity=0.20] (119.96, 54.82) circle (  2.13);

\path[fill=fillColor,fill opacity=0.20] (114.94, 58.09) circle (  2.13);

\path[fill=fillColor,fill opacity=0.20] (100.52, 60.07) circle (  2.13);

\path[fill=fillColor,fill opacity=0.20] ( 87.19, 55.85) circle (  2.13);

\path[fill=fillColor,fill opacity=0.20] ( 86.97, 54.56) circle (  2.13);

\path[fill=fillColor,fill opacity=0.20] ( 80.85, 61.71) circle (  2.13);

\path[fill=fillColor,fill opacity=0.20] ( 82.38, 70.32) circle (  2.13);

\path[fill=fillColor,fill opacity=0.20] ( 88.50, 48.53) circle (  2.13);

\path[fill=fillColor,fill opacity=0.20] ( 89.37, 49.65) circle (  2.13);

\path[fill=fillColor,fill opacity=0.20] (102.26, 53.61) circle (  2.13);

\path[fill=fillColor,fill opacity=0.20] ( 96.58, 61.96) circle (  2.13);

\path[fill=fillColor,fill opacity=0.20] (102.05, 56.71) circle (  2.13);

\path[fill=fillColor,fill opacity=0.20] ( 93.96, 45.60) circle (  2.13);

\path[fill=fillColor,fill opacity=0.20] ( 91.99, 48.10) circle (  2.13);

\path[fill=fillColor,fill opacity=0.20] ( 84.78, 51.98) circle (  2.13);

\path[fill=fillColor,fill opacity=0.20] ( 73.86, 55.08) circle (  2.13);

\path[fill=fillColor,fill opacity=0.20] ( 94.84, 78.93) circle (  2.13);

\path[fill=fillColor,fill opacity=0.20] (104.89, 45.60) circle (  2.13);

\path[fill=fillColor,fill opacity=0.20] (119.96, 45.00) circle (  2.13);

\path[fill=fillColor,fill opacity=0.20] (121.49, 49.99) circle (  2.13);

\path[fill=fillColor,fill opacity=0.20] (107.95, 64.55) circle (  2.13);

\path[fill=fillColor,fill opacity=0.20] ( 96.15, 68.85) circle (  2.13);

\path[fill=fillColor,fill opacity=0.20] ( 83.25, 58.43) circle (  2.13);

\path[fill=fillColor,fill opacity=0.20] ( 82.38, 59.38) circle (  2.13);

\path[fill=fillColor,fill opacity=0.20] ( 79.76, 65.50) circle (  2.13);

\path[fill=fillColor,fill opacity=0.20] ( 62.50, 76.09) circle (  2.13);

\path[fill=fillColor,fill opacity=0.20] ( 77.79,102.27) circle (  2.13);

\path[fill=fillColor,fill opacity=0.20] ( 82.82, 66.53) circle (  2.13);

\path[fill=fillColor,fill opacity=0.20] ( 85.66, 49.65) circle (  2.13);

\path[fill=fillColor,fill opacity=0.20] ( 94.18, 55.85) circle (  2.13);

\path[fill=fillColor,fill opacity=0.20] ( 98.11, 59.38) circle (  2.13);

\path[fill=fillColor,fill opacity=0.20] ( 98.77, 53.70) circle (  2.13);

\path[fill=fillColor,fill opacity=0.20] ( 92.87, 57.57) circle (  2.13);

\path[fill=fillColor,fill opacity=0.20] ( 93.09, 57.57) circle (  2.13);

\path[fill=fillColor,fill opacity=0.20] ( 93.31, 53.78) circle (  2.13);

\path[fill=fillColor,fill opacity=0.20] ( 88.94, 53.44) circle (  2.13);

\path[fill=fillColor,fill opacity=0.20] ( 83.69, 49.74) circle (  2.13);

\path[fill=fillColor,fill opacity=0.20] (120.62, 61.45) circle (  2.13);

\path[fill=fillColor,fill opacity=0.20] (114.50, 51.29) circle (  2.13);

\path[fill=fillColor,fill opacity=0.20] (124.77, 43.19) circle (  2.13);

\path[fill=fillColor,fill opacity=0.20] (101.61, 51.46) circle (  2.13);

\path[fill=fillColor,fill opacity=0.20] ( 92.87, 62.91) circle (  2.13);

\path[fill=fillColor,fill opacity=0.20] ( 83.47, 65.06) circle (  2.13);

\path[fill=fillColor,fill opacity=0.20] ( 77.14, 68.34) circle (  2.13);

\path[fill=fillColor,fill opacity=0.20] ( 78.67, 69.89) circle (  2.13);

\path[fill=fillColor,fill opacity=0.20] ( 72.11, 71.87) circle (  2.13);

\path[fill=fillColor,fill opacity=0.20] ( 84.13, 91.93) circle (  2.13);

\path[fill=fillColor,fill opacity=0.20] ( 83.25, 75.48) circle (  2.13);

\path[fill=fillColor,fill opacity=0.20] ( 92.21, 66.79) circle (  2.13);

\path[fill=fillColor,fill opacity=0.20] (108.16, 53.61) circle (  2.13);

\path[fill=fillColor,fill opacity=0.20] (101.61, 41.47) circle (  2.13);

\path[fill=fillColor,fill opacity=0.20] ( 94.40, 46.98) circle (  2.13);

\path[fill=fillColor,fill opacity=0.20] (101.83, 59.98) circle (  2.13);

\path[fill=fillColor,fill opacity=0.20] ( 95.93, 63.17) circle (  2.13);

\path[fill=fillColor,fill opacity=0.20] ( 95.27, 54.04) circle (  2.13);

\path[fill=fillColor,fill opacity=0.20] ( 88.06, 50.43) circle (  2.13);

\path[fill=fillColor,fill opacity=0.20] ( 68.40, 66.36) circle (  2.13);

\path[fill=fillColor,fill opacity=0.20] ( 99.42, 73.93) circle (  2.13);

\path[fill=fillColor,fill opacity=0.20] (120.84, 60.07) circle (  2.13);

\path[fill=fillColor,fill opacity=0.20] (138.97, 44.83) circle (  2.13);

\path[fill=fillColor,fill opacity=0.20] ( 99.86, 51.29) circle (  2.13);

\path[fill=fillColor,fill opacity=0.20] ( 88.94, 60.93) circle (  2.13);

\path[fill=fillColor,fill opacity=0.20] ( 75.17, 63.17) circle (  2.13);

\path[fill=fillColor,fill opacity=0.20] ( 74.95, 67.99) circle (  2.13);

\path[fill=fillColor,fill opacity=0.20] ( 73.42, 69.80) circle (  2.13);

\path[fill=fillColor,fill opacity=0.20] ( 57.03, 73.42) circle (  2.13);

\path[fill=fillColor,fill opacity=0.20] ( 94.18, 76.17) circle (  2.13);

\path[fill=fillColor,fill opacity=0.20] (105.32, 69.97) circle (  2.13);

\path[fill=fillColor,fill opacity=0.20] (108.60, 62.74) circle (  2.13);

\path[fill=fillColor,fill opacity=0.20] (118.00, 49.22) circle (  2.13);

\path[fill=fillColor,fill opacity=0.20] (101.17, 43.19) circle (  2.13);

\path[fill=fillColor,fill opacity=0.20] (109.26, 49.65) circle (  2.13);

\path[fill=fillColor,fill opacity=0.20] ( 96.80, 55.68) circle (  2.13);

\path[fill=fillColor,fill opacity=0.20] ( 94.18, 55.16) circle (  2.13);

\path[fill=fillColor,fill opacity=0.20] ( 64.25, 73.42) circle (  2.13);

\path[fill=fillColor,fill opacity=0.20] ( 57.47, 78.15) circle (  2.13);

\path[fill=fillColor,fill opacity=0.20] ( 83.69, 88.57) circle (  2.13);

\path[fill=fillColor,fill opacity=0.20] (103.14, 63.60) circle (  2.13);

\path[fill=fillColor,fill opacity=0.20] ( 99.86, 54.82) circle (  2.13);

\path[fill=fillColor,fill opacity=0.20] ( 99.21, 48.19) circle (  2.13);

\path[fill=fillColor,fill opacity=0.20] ( 95.49, 49.82) circle (  2.13);

\path[fill=fillColor,fill opacity=0.20] ( 90.25, 54.82) circle (  2.13);

\path[fill=fillColor,fill opacity=0.20] ( 86.75, 54.47) circle (  2.13);

\path[fill=fillColor,fill opacity=0.20] ( 86.75, 60.24) circle (  2.13);

\path[fill=fillColor,fill opacity=0.20] ( 74.08, 65.24) circle (  2.13);

\path[fill=fillColor,fill opacity=0.20] ( 65.12, 66.44) circle (  2.13);

\path[fill=fillColor,fill opacity=0.20] ( 88.94, 74.80) circle (  2.13);

\path[fill=fillColor,fill opacity=0.20] (102.26, 67.05) circle (  2.13);

\path[fill=fillColor,fill opacity=0.20] (111.44, 56.63) circle (  2.13);

\path[fill=fillColor,fill opacity=0.20] (108.60, 54.30) circle (  2.13);

\path[fill=fillColor,fill opacity=0.20] (115.59, 61.62) circle (  2.13);

\path[fill=fillColor,fill opacity=0.20] (113.41, 63.86) circle (  2.13);

\path[fill=fillColor,fill opacity=0.20] ( 98.77, 56.37) circle (  2.13);

\path[fill=fillColor,fill opacity=0.20] ( 89.15, 46.72) circle (  2.13);

\path[fill=fillColor,fill opacity=0.20] ( 88.28, 46.38) circle (  2.13);

\path[fill=fillColor,fill opacity=0.20] ( 99.86, 64.63) circle (  2.13);

\path[fill=fillColor,fill opacity=0.20] ( 95.93, 58.35) circle (  2.13);

\path[fill=fillColor,fill opacity=0.20] ( 88.28, 50.51) circle (  2.13);

\path[fill=fillColor,fill opacity=0.20] ( 88.94, 53.18) circle (  2.13);

\path[fill=fillColor,fill opacity=0.20] ( 87.62, 53.35) circle (  2.13);

\path[fill=fillColor,fill opacity=0.20] ( 78.45, 50.08) circle (  2.13);

\path[fill=fillColor,fill opacity=0.20] ( 71.02, 65.41) circle (  2.13);

\path[fill=fillColor,fill opacity=0.20] ( 68.62, 81.94) circle (  2.13);

\path[fill=fillColor,fill opacity=0.20] ( 74.30, 90.12) circle (  2.13);

\path[fill=fillColor,fill opacity=0.20] ( 85.22, 71.78) circle (  2.13);

\path[fill=fillColor,fill opacity=0.20] ( 86.31, 80.57) circle (  2.13);

\path[fill=fillColor,fill opacity=0.20] ( 90.47, 69.28) circle (  2.13);

\path[fill=fillColor,fill opacity=0.20] ( 97.46, 63.34) circle (  2.13);

\path[fill=fillColor,fill opacity=0.20] (105.32, 59.90) circle (  2.13);

\path[fill=fillColor,fill opacity=0.20] (107.73, 59.73) circle (  2.13);

\path[fill=fillColor,fill opacity=0.20] (104.67, 70.23) circle (  2.13);

\path[fill=fillColor,fill opacity=0.20] (121.27, 68.85) circle (  2.13);

\path[fill=fillColor,fill opacity=0.20] ( 88.06, 50.68) circle (  2.13);

\path[fill=fillColor,fill opacity=0.20] ( 77.79, 44.23) circle (  2.13);

\path[fill=fillColor,fill opacity=0.20] ( 83.25, 64.72) circle (  2.13);

\path[fill=fillColor,fill opacity=0.20] ( 74.30, 74.97) circle (  2.13);

\path[fill=fillColor,fill opacity=0.20] ( 64.03, 64.38) circle (  2.13);

\path[fill=fillColor,fill opacity=0.20] (106.85, 71.35) circle (  2.13);

\path[fill=fillColor,fill opacity=0.20] ( 97.02, 56.54) circle (  2.13);

\path[fill=fillColor,fill opacity=0.20] ( 81.29, 52.58) circle (  2.13);

\path[fill=fillColor,fill opacity=0.20] ( 79.76, 53.27) circle (  2.13);

\path[fill=fillColor,fill opacity=0.20] ( 80.63, 51.63) circle (  2.13);

\path[fill=fillColor,fill opacity=0.20] ( 76.04, 53.18) circle (  2.13);

\path[fill=fillColor,fill opacity=0.20] ( 85.22, 66.53) circle (  2.13);

\path[fill=fillColor,fill opacity=0.20] ( 70.36, 82.89) circle (  2.13);

\path[fill=fillColor,fill opacity=0.20] ( 99.21, 86.25) circle (  2.13);

\path[fill=fillColor,fill opacity=0.20] ( 81.07, 71.09) circle (  2.13);

\path[fill=fillColor,fill opacity=0.20] ( 85.88, 70.66) circle (  2.13);

\path[fill=fillColor,fill opacity=0.20] ( 90.47, 59.04) circle (  2.13);

\path[fill=fillColor,fill opacity=0.20] ( 91.34, 55.59) circle (  2.13);

\path[fill=fillColor,fill opacity=0.20] ( 92.65, 66.96) circle (  2.13);

\path[fill=fillColor,fill opacity=0.20] ( 91.78, 70.83) circle (  2.13);

\path[fill=fillColor,fill opacity=0.20] ( 98.33, 67.48) circle (  2.13);

\path[fill=fillColor,fill opacity=0.20] ( 92.65, 64.29) circle (  2.13);

\path[fill=fillColor,fill opacity=0.20] ( 83.91, 56.97) circle (  2.13);

\path[fill=fillColor,fill opacity=0.20] ( 83.25, 48.36) circle (  2.13);

\path[fill=fillColor,fill opacity=0.20] ( 74.08, 58.18) circle (  2.13);

\path[fill=fillColor,fill opacity=0.20] ( 53.76, 58.95) circle (  2.13);

\path[fill=fillColor,fill opacity=0.20] ( 90.47, 75.31) circle (  2.13);

\path[fill=fillColor,fill opacity=0.20] (102.92, 61.19) circle (  2.13);

\path[fill=fillColor,fill opacity=0.20] ( 92.21, 67.39) circle (  2.13);

\path[fill=fillColor,fill opacity=0.20] ( 78.23, 59.21) circle (  2.13);

\path[fill=fillColor,fill opacity=0.20] ( 77.14, 54.47) circle (  2.13);

\path[fill=fillColor,fill opacity=0.20] ( 77.36, 64.63) circle (  2.13);

\path[fill=fillColor,fill opacity=0.20] ( 78.01, 70.23) circle (  2.13);

\path[fill=fillColor,fill opacity=0.20] ( 71.89, 64.29) circle (  2.13);

\path[fill=fillColor,fill opacity=0.20] ( 76.04, 73.76) circle (  2.13);

\path[fill=fillColor,fill opacity=0.20] ( 60.75, 89.09) circle (  2.13);

\path[fill=fillColor,fill opacity=0.20] ( 84.13, 71.70) circle (  2.13);

\path[fill=fillColor,fill opacity=0.20] ( 89.37, 73.16) circle (  2.13);

\path[fill=fillColor,fill opacity=0.20] ( 90.68, 71.52) circle (  2.13);

\path[fill=fillColor,fill opacity=0.20] ( 87.41, 53.96) circle (  2.13);

\path[fill=fillColor,fill opacity=0.20] ( 91.12, 44.40) circle (  2.13);

\path[fill=fillColor,fill opacity=0.20] (104.45, 54.30) circle (  2.13);

\path[fill=fillColor,fill opacity=0.20] ( 92.43, 61.88) circle (  2.13);

\path[fill=fillColor,fill opacity=0.20] ( 91.78, 66.10) circle (  2.13);

\path[fill=fillColor,fill opacity=0.20] ( 88.50, 69.71) circle (  2.13);

\path[fill=fillColor,fill opacity=0.20] ( 86.97, 65.15) circle (  2.13);

\path[fill=fillColor,fill opacity=0.20] ( 75.39, 60.24) circle (  2.13);

\path[fill=fillColor,fill opacity=0.20] ( 83.47, 62.48) circle (  2.13);

\path[fill=fillColor,fill opacity=0.20] ( 82.82, 69.71) circle (  2.13);

\path[fill=fillColor,fill opacity=0.20] ( 84.57, 97.62) circle (  2.13);

\path[fill=fillColor,fill opacity=0.20] ( 89.37, 76.52) circle (  2.13);

\path[fill=fillColor,fill opacity=0.20] ( 99.64, 71.01) circle (  2.13);

\path[fill=fillColor,fill opacity=0.20] ( 82.82, 65.75) circle (  2.13);

\path[fill=fillColor,fill opacity=0.20] ( 80.20, 67.13) circle (  2.13);

\path[fill=fillColor,fill opacity=0.20] ( 89.37, 64.63) circle (  2.13);

\path[fill=fillColor,fill opacity=0.20] ( 80.41, 49.31) circle (  2.13);

\path[fill=fillColor,fill opacity=0.20] ( 74.30, 58.00) circle (  2.13);

\path[fill=fillColor,fill opacity=0.20] ( 73.20, 73.25) circle (  2.13);

\path[fill=fillColor,fill opacity=0.20] ( 75.39, 69.37) circle (  2.13);

\path[fill=fillColor,fill opacity=0.20] ( 67.52, 76.69) circle (  2.13);

\path[fill=fillColor,fill opacity=0.20] ( 60.09, 88.66) circle (  2.13);

\path[fill=fillColor,fill opacity=0.20] ( 85.66, 76.09) circle (  2.13);

\path[fill=fillColor,fill opacity=0.20] ( 89.81, 68.77) circle (  2.13);

\path[fill=fillColor,fill opacity=0.20] ( 91.12, 68.68) circle (  2.13);

\path[fill=fillColor,fill opacity=0.20] ( 92.43, 63.86) circle (  2.13);

\path[fill=fillColor,fill opacity=0.20] ( 89.15, 51.46) circle (  2.13);

\path[fill=fillColor,fill opacity=0.20] ( 94.18, 53.35) circle (  2.13);

\path[fill=fillColor,fill opacity=0.20] ( 96.80, 63.08) circle (  2.13);

\path[fill=fillColor,fill opacity=0.20] (102.26, 59.64) circle (  2.13);

\path[fill=fillColor,fill opacity=0.20] ( 86.10, 56.88) circle (  2.13);

\path[fill=fillColor,fill opacity=0.20] ( 91.12, 65.75) circle (  2.13);

\path[fill=fillColor,fill opacity=0.20] ( 85.22, 68.85) circle (  2.13);

\path[fill=fillColor,fill opacity=0.20] ( 81.94, 70.23) circle (  2.13);

\path[fill=fillColor,fill opacity=0.20] ( 81.94, 70.32) circle (  2.13);

\path[fill=fillColor,fill opacity=0.20] ( 75.83, 68.85) circle (  2.13);

\path[fill=fillColor,fill opacity=0.20] ( 79.98, 92.36) circle (  2.13);

\path[fill=fillColor,fill opacity=0.20] ( 91.78, 63.17) circle (  2.13);

\path[fill=fillColor,fill opacity=0.20] (103.79, 65.24) circle (  2.13);

\path[fill=fillColor,fill opacity=0.20] ( 91.34, 80.05) circle (  2.13);

\path[fill=fillColor,fill opacity=0.20] ( 93.09, 59.81) circle (  2.13);

\path[fill=fillColor,fill opacity=0.20] ( 87.19, 67.13) circle (  2.13);

\path[fill=fillColor,fill opacity=0.20] ( 82.16, 63.86) circle (  2.13);

\path[fill=fillColor,fill opacity=0.20] ( 79.10, 67.30) circle (  2.13);

\path[fill=fillColor,fill opacity=0.20] ( 78.23, 68.77) circle (  2.13);

\path[fill=fillColor,fill opacity=0.20] ( 75.39, 67.91) circle (  2.13);

\path[fill=fillColor,fill opacity=0.20] ( 71.02, 68.60) circle (  2.13);

\path[fill=fillColor,fill opacity=0.20] ( 72.11, 75.83) circle (  2.13);

\path[fill=fillColor,fill opacity=0.20] ( 86.97, 73.85) circle (  2.13);

\path[fill=fillColor,fill opacity=0.20] ( 90.47, 63.17) circle (  2.13);

\path[fill=fillColor,fill opacity=0.20] ( 94.18, 57.57) circle (  2.13);

\path[fill=fillColor,fill opacity=0.20] ( 93.74, 59.64) circle (  2.13);

\path[fill=fillColor,fill opacity=0.20] ( 94.62, 64.98) circle (  2.13);

\path[fill=fillColor,fill opacity=0.20] ( 97.24, 64.55) circle (  2.13);

\path[fill=fillColor,fill opacity=0.20] ( 97.68, 62.91) circle (  2.13);

\path[fill=fillColor,fill opacity=0.20] ( 93.31, 60.24) circle (  2.13);

\path[fill=fillColor,fill opacity=0.20] ( 91.56, 55.16) circle (  2.13);

\path[fill=fillColor,fill opacity=0.20] ( 82.38, 59.38) circle (  2.13);

\path[fill=fillColor,fill opacity=0.20] ( 77.79, 68.60) circle (  2.13);

\path[fill=fillColor,fill opacity=0.20] ( 79.10, 64.89) circle (  2.13);

\path[fill=fillColor,fill opacity=0.20] ( 75.17, 66.27) circle (  2.13);

\path[fill=fillColor,fill opacity=0.20] ( 70.36, 75.14) circle (  2.13);

\path[fill=fillColor,fill opacity=0.20] ( 90.47, 73.50) circle (  2.13);

\path[fill=fillColor,fill opacity=0.20] ( 97.46, 66.53) circle (  2.13);

\path[fill=fillColor,fill opacity=0.20] ( 82.38, 68.42) circle (  2.13);

\path[fill=fillColor,fill opacity=0.20] ( 90.03, 67.73) circle (  2.13);

\path[fill=fillColor,fill opacity=0.20] ( 89.59, 58.00) circle (  2.13);

\path[fill=fillColor,fill opacity=0.20] ( 90.03, 51.11) circle (  2.13);

\path[fill=fillColor,fill opacity=0.20] ( 86.31, 66.87) circle (  2.13);

\path[fill=fillColor,fill opacity=0.20] ( 91.99, 75.14) circle (  2.13);

\path[fill=fillColor,fill opacity=0.20] ( 79.76, 77.29) circle (  2.13);

\path[fill=fillColor,fill opacity=0.20] ( 76.92, 71.35) circle (  2.13);

\path[fill=fillColor,fill opacity=0.20] ( 73.64, 71.87) circle (  2.13);

\path[fill=fillColor,fill opacity=0.20] ( 77.57, 81.43) circle (  2.13);

\path[fill=fillColor,fill opacity=0.20] ( 82.60, 68.08) circle (  2.13);

\path[fill=fillColor,fill opacity=0.20] ( 83.69, 62.22) circle (  2.13);

\path[fill=fillColor,fill opacity=0.20] (104.23, 59.98) circle (  2.13);

\path[fill=fillColor,fill opacity=0.20] ( 99.86, 59.90) circle (  2.13);

\path[fill=fillColor,fill opacity=0.20] (100.73, 68.25) circle (  2.13);

\path[fill=fillColor,fill opacity=0.20] (101.61, 77.55) circle (  2.13);

\path[fill=fillColor,fill opacity=0.20] ( 97.24, 74.02) circle (  2.13);

\path[fill=fillColor,fill opacity=0.20] (102.05, 62.83) circle (  2.13);

\path[fill=fillColor,fill opacity=0.20] ( 86.75, 51.63) circle (  2.13);

\path[fill=fillColor,fill opacity=0.20] ( 83.69, 48.10) circle (  2.13);

\path[fill=fillColor,fill opacity=0.20] ( 71.89, 66.36) circle (  2.13);

\path[fill=fillColor,fill opacity=0.20] ( 69.93, 79.36) circle (  2.13);

\path[fill=fillColor,fill opacity=0.20] ( 66.21, 67.22) circle (  2.13);

\path[fill=fillColor,fill opacity=0.20] ( 55.72, 69.63) circle (  2.13);

\path[fill=fillColor,fill opacity=0.20] ( 64.03, 73.68) circle (  2.13);

\path[fill=fillColor,fill opacity=0.20] ( 81.29, 70.83) circle (  2.13);

\path[fill=fillColor,fill opacity=0.20] ( 95.49, 74.02) circle (  2.13);

\path[fill=fillColor,fill opacity=0.20] ( 95.49, 63.77) circle (  2.13);

\path[fill=fillColor,fill opacity=0.20] ( 94.40, 61.45) circle (  2.13);

\path[fill=fillColor,fill opacity=0.20] ( 87.84, 73.85) circle (  2.13);

\path[fill=fillColor,fill opacity=0.20] ( 92.21, 74.71) circle (  2.13);

\path[fill=fillColor,fill opacity=0.20] ( 85.66, 67.91) circle (  2.13);

\path[fill=fillColor,fill opacity=0.20] ( 82.60, 62.14) circle (  2.13);

\path[fill=fillColor,fill opacity=0.20] ( 76.04, 67.82) circle (  2.13);

\path[fill=fillColor,fill opacity=0.20] ( 74.51, 84.78) circle (  2.13);

\path[fill=fillColor,fill opacity=0.20] ( 71.02, 91.50) circle (  2.13);

\path[fill=fillColor,fill opacity=0.20] ( 69.27, 86.16) circle (  2.13);

\path[fill=fillColor,fill opacity=0.20] ( 69.27, 99.08) circle (  2.13);

\path[fill=fillColor,fill opacity=0.20] ( 83.25, 79.02) circle (  2.13);

\path[fill=fillColor,fill opacity=0.20] ( 85.00, 75.83) circle (  2.13);

\path[fill=fillColor,fill opacity=0.20] ( 89.59, 79.27) circle (  2.13);

\path[fill=fillColor,fill opacity=0.20] ( 92.87, 76.52) circle (  2.13);

\path[fill=fillColor,fill opacity=0.20] (100.30, 72.73) circle (  2.13);

\path[fill=fillColor,fill opacity=0.20] ( 97.02, 68.85) circle (  2.13);

\path[fill=fillColor,fill opacity=0.20] ( 97.24, 71.18) circle (  2.13);

\path[fill=fillColor,fill opacity=0.20] ( 90.68, 77.98) circle (  2.13);

\path[fill=fillColor,fill opacity=0.20] ( 83.47, 69.37) circle (  2.13);

\path[fill=fillColor,fill opacity=0.20] ( 69.05, 61.28) circle (  2.13);

\path[fill=fillColor,fill opacity=0.20] ( 68.40, 67.30) circle (  2.13);

\path[fill=fillColor,fill opacity=0.20] ( 65.77, 75.31) circle (  2.13);

\path[fill=fillColor,fill opacity=0.20] ( 67.09, 84.70) circle (  2.13);

\path[fill=fillColor,fill opacity=0.20] ( 53.76, 77.55) circle (  2.13);

\path[fill=fillColor,fill opacity=0.20] ( 73.20, 72.56) circle (  2.13);

\path[fill=fillColor,fill opacity=0.20] (106.20, 63.60) circle (  2.13);

\path[fill=fillColor,fill opacity=0.20] ( 99.42, 74.19) circle (  2.13);

\path[fill=fillColor,fill opacity=0.20] ( 87.62, 77.64) circle (  2.13);

\path[fill=fillColor,fill opacity=0.20] ( 95.49, 62.48) circle (  2.13);

\path[fill=fillColor,fill opacity=0.20] ( 99.64, 45.43) circle (  2.13);

\path[fill=fillColor,fill opacity=0.20] ( 90.47, 47.15) circle (  2.13);

\path[fill=fillColor,fill opacity=0.20] ( 81.73, 58.61) circle (  2.13);

\path[fill=fillColor,fill opacity=0.20] ( 87.84, 67.82) circle (  2.13);

\path[fill=fillColor,fill opacity=0.20] ( 76.48, 73.59) circle (  2.13);

\path[fill=fillColor,fill opacity=0.20] ( 75.39, 80.74) circle (  2.13);

\path[fill=fillColor,fill opacity=0.20] ( 72.77, 85.90) circle (  2.13);

\path[fill=fillColor,fill opacity=0.20] ( 70.80, 87.11) circle (  2.13);

\path[fill=fillColor,fill opacity=0.20] ( 78.23,101.66) circle (  2.13);

\path[fill=fillColor,fill opacity=0.20] ( 88.72, 85.82) circle (  2.13);

\path[fill=fillColor,fill opacity=0.20] (110.35, 75.66) circle (  2.13);

\path[fill=fillColor,fill opacity=0.20] (114.72, 76.86) circle (  2.13);

\path[fill=fillColor,fill opacity=0.20] (111.88, 71.78) circle (  2.13);

\path[fill=fillColor,fill opacity=0.20] ( 89.15, 71.95) circle (  2.13);

\path[fill=fillColor,fill opacity=0.20] ( 81.94, 77.03) circle (  2.13);

\path[fill=fillColor,fill opacity=0.20] ( 78.67, 67.30) circle (  2.13);

\path[fill=fillColor,fill opacity=0.20] ( 68.18, 64.03) circle (  2.13);

\path[fill=fillColor,fill opacity=0.20] ( 68.62, 72.38) circle (  2.13);

\path[fill=fillColor,fill opacity=0.20] ( 61.62, 70.32) circle (  2.13);

\path[fill=fillColor,fill opacity=0.20] ( 59.22, 75.57) circle (  2.13);

\path[fill=fillColor,fill opacity=0.20] ( 53.10, 95.20) circle (  2.13);

\path[fill=fillColor,fill opacity=0.20] ( 55.29,103.30) circle (  2.13);

\path[fill=fillColor,fill opacity=0.20] (142.25, 77.03) circle (  2.13);

\path[fill=fillColor,fill opacity=0.20] (145.09, 64.89) circle (  2.13);

\path[fill=fillColor,fill opacity=0.20] (107.29, 66.01) circle (  2.13);

\path[fill=fillColor,fill opacity=0.20] (100.08, 56.71) circle (  2.13);

\path[fill=fillColor,fill opacity=0.20] (102.05, 51.37) circle (  2.13);

\path[fill=fillColor,fill opacity=0.20] ( 92.87, 54.56) circle (  2.13);

\path[fill=fillColor,fill opacity=0.20] ( 85.00, 53.87) circle (  2.13);

\path[fill=fillColor,fill opacity=0.20] ( 83.25, 52.92) circle (  2.13);

\path[fill=fillColor,fill opacity=0.20] ( 79.54, 61.10) circle (  2.13);

\path[fill=fillColor,fill opacity=0.20] ( 80.20, 76.35) circle (  2.13);

\path[fill=fillColor,fill opacity=0.20] ( 72.11, 84.87) circle (  2.13);

\path[fill=fillColor,fill opacity=0.20] ( 69.49, 74.97) circle (  2.13);

\path[fill=fillColor,fill opacity=0.20] ( 67.74, 71.01) circle (  2.13);

\path[fill=fillColor,fill opacity=0.20] ( 65.99, 83.32) circle (  2.13);

\path[fill=fillColor,fill opacity=0.20] ( 91.56, 92.36) circle (  2.13);

\path[fill=fillColor,fill opacity=0.20] ( 84.13, 95.55) circle (  2.13);

\path[fill=fillColor,fill opacity=0.20] ( 86.75, 94.00) circle (  2.13);

\path[fill=fillColor,fill opacity=0.20] ( 97.02, 84.78) circle (  2.13);

\path[fill=fillColor,fill opacity=0.20] ( 95.71, 76.00) circle (  2.13);

\path[fill=fillColor,fill opacity=0.20] ( 87.62, 66.87) circle (  2.13);

\path[fill=fillColor,fill opacity=0.20] ( 88.50, 50.68) circle (  2.13);

\path[fill=fillColor,fill opacity=0.20] ( 59.66, 50.94) circle (  2.13);

\path[fill=fillColor,fill opacity=0.20] ( 62.50, 77.03) circle (  2.13);

\path[fill=fillColor,fill opacity=0.20] ( 47.42, 93.65) circle (  2.13);

\path[fill=fillColor,fill opacity=0.20] ( 54.41, 92.62) circle (  2.13);

\path[fill=fillColor,fill opacity=0.20] ( 89.37, 44.31) circle (  2.13);

\path[fill=fillColor,fill opacity=0.20] (103.79, 59.90) circle (  2.13);

\path[fill=fillColor,fill opacity=0.20] (101.61, 70.23) circle (  2.13);

\path[fill=fillColor,fill opacity=0.20] ( 93.74, 66.61) circle (  2.13);

\path[fill=fillColor,fill opacity=0.20] ( 83.69, 60.07) circle (  2.13);

\path[fill=fillColor,fill opacity=0.20] ( 85.66, 56.54) circle (  2.13);

\path[fill=fillColor,fill opacity=0.20] ( 86.10, 62.05) circle (  2.13);

\path[fill=fillColor,fill opacity=0.20] ( 76.48, 71.27) circle (  2.13);

\path[fill=fillColor,fill opacity=0.20] ( 76.26, 76.26) circle (  2.13);

\path[fill=fillColor,fill opacity=0.20] ( 70.36, 75.92) circle (  2.13);

\path[fill=fillColor,fill opacity=0.20] ( 70.80, 72.56) circle (  2.13);

\path[fill=fillColor,fill opacity=0.20] ( 79.98, 81.60) circle (  2.13);

\path[fill=fillColor,fill opacity=0.20] ( 82.16, 87.02) circle (  2.13);

\path[fill=fillColor,fill opacity=0.20] ( 86.97, 89.18) circle (  2.13);

\path[fill=fillColor,fill opacity=0.20] ( 83.25, 79.53) circle (  2.13);

\path[fill=fillColor,fill opacity=0.20] ( 83.47, 79.70) circle (  2.13);

\path[fill=fillColor,fill opacity=0.20] ( 74.51, 78.76) circle (  2.13);

\path[fill=fillColor,fill opacity=0.20] ( 71.89, 68.16) circle (  2.13);

\path[fill=fillColor,fill opacity=0.20] ( 67.74, 66.27) circle (  2.13);

\path[fill=fillColor,fill opacity=0.20] ( 58.13, 66.36) circle (  2.13);

\path[fill=fillColor,fill opacity=0.20] ( 51.79, 76.78) circle (  2.13);

\path[fill=fillColor,fill opacity=0.20] ( 66.65,101.58) circle (  2.13);

\path[fill=fillColor,fill opacity=0.20] ( 70.58, 63.86) circle (  2.13);

\path[fill=fillColor,fill opacity=0.20] ( 97.24, 62.83) circle (  2.13);

\path[fill=fillColor,fill opacity=0.20] (100.30, 64.29) circle (  2.13);

\path[fill=fillColor,fill opacity=0.20] ( 91.34, 67.30) circle (  2.13);

\path[fill=fillColor,fill opacity=0.20] ( 91.56, 63.00) circle (  2.13);

\path[fill=fillColor,fill opacity=0.20] ( 87.84, 58.26) circle (  2.13);

\path[fill=fillColor,fill opacity=0.20] ( 79.54, 64.89) circle (  2.13);

\path[fill=fillColor,fill opacity=0.20] ( 84.13, 72.21) circle (  2.13);

\path[fill=fillColor,fill opacity=0.20] ( 83.69, 76.26) circle (  2.13);

\path[fill=fillColor,fill opacity=0.20] ( 78.45, 81.17) circle (  2.13);

\path[fill=fillColor,fill opacity=0.20] ( 77.36, 81.60) circle (  2.13);

\path[fill=fillColor,fill opacity=0.20] ( 67.74, 68.42) circle (  2.13);

\path[fill=fillColor,fill opacity=0.20] ( 68.83, 59.21) circle (  2.13);

\path[fill=fillColor,fill opacity=0.20] ( 69.71, 71.70) circle (  2.13);

\path[fill=fillColor,fill opacity=0.20] ( 70.14, 90.21) circle (  2.13);

\path[fill=fillColor,fill opacity=0.20] ( 75.39, 99.25) circle (  2.13);

\path[fill=fillColor,fill opacity=0.20] ( 81.07, 85.99) circle (  2.13);

\path[fill=fillColor,fill opacity=0.20] ( 77.57, 88.06) circle (  2.13);

\path[fill=fillColor,fill opacity=0.20] ( 87.41, 86.68) circle (  2.13);

\path[fill=fillColor,fill opacity=0.20] ( 89.37, 66.96) circle (  2.13);

\path[fill=fillColor,fill opacity=0.20] ( 94.40, 67.22) circle (  2.13);

\path[fill=fillColor,fill opacity=0.20] ( 93.31, 74.37) circle (  2.13);

\path[fill=fillColor,fill opacity=0.20] ( 82.82, 75.83) circle (  2.13);

\path[fill=fillColor,fill opacity=0.20] ( 64.90, 75.48) circle (  2.13);

\path[fill=fillColor,fill opacity=0.20] ( 63.81, 69.37) circle (  2.13);

\path[fill=fillColor,fill opacity=0.20] ( 70.14, 70.15) circle (  2.13);

\path[fill=fillColor,fill opacity=0.20] ( 94.18, 65.06) circle (  2.13);

\path[fill=fillColor,fill opacity=0.20] ( 91.12, 71.01) circle (  2.13);

\path[fill=fillColor,fill opacity=0.20] ( 89.37, 66.61) circle (  2.13);

\path[fill=fillColor,fill opacity=0.20] ( 91.34, 55.76) circle (  2.13);

\path[fill=fillColor,fill opacity=0.20] ( 90.25, 58.43) circle (  2.13);

\path[fill=fillColor,fill opacity=0.20] ( 94.62, 69.11) circle (  2.13);

\path[fill=fillColor,fill opacity=0.20] ( 95.93, 73.85) circle (  2.13);

\path[fill=fillColor,fill opacity=0.20] ( 88.72, 76.60) circle (  2.13);

\path[fill=fillColor,fill opacity=0.20] ( 81.51, 72.30) circle (  2.13);

\path[fill=fillColor,fill opacity=0.20] ( 75.83, 72.13) circle (  2.13);

\path[fill=fillColor,fill opacity=0.20] ( 74.73, 74.11) circle (  2.13);

\path[fill=fillColor,fill opacity=0.20] ( 74.51, 70.32) circle (  2.13);

\path[fill=fillColor,fill opacity=0.20] ( 81.29, 67.05) circle (  2.13);

\path[fill=fillColor,fill opacity=0.20] ( 70.80, 68.85) circle (  2.13);

\path[fill=fillColor,fill opacity=0.20] ( 71.02, 69.11) circle (  2.13);

\path[fill=fillColor,fill opacity=0.20] ( 71.24, 72.56) circle (  2.13);

\path[fill=fillColor,fill opacity=0.20] ( 72.55, 81.86) circle (  2.13);

\path[fill=fillColor,fill opacity=0.20] ( 75.17, 88.23) circle (  2.13);

\path[fill=fillColor,fill opacity=0.20] ( 78.88, 92.45) circle (  2.13);

\path[fill=fillColor,fill opacity=0.20] ( 72.77, 90.47) circle (  2.13);

\path[fill=fillColor,fill opacity=0.20] ( 72.99, 78.33) circle (  2.13);

\path[fill=fillColor,fill opacity=0.20] ( 77.14, 73.25) circle (  2.13);

\path[fill=fillColor,fill opacity=0.20] ( 79.98, 81.08) circle (  2.13);

\path[fill=fillColor,fill opacity=0.20] ( 77.36, 83.06) circle (  2.13);

\path[fill=fillColor,fill opacity=0.20] ( 79.10, 79.70) circle (  2.13);

\path[fill=fillColor,fill opacity=0.20] ( 80.20, 80.48) circle (  2.13);

\path[fill=fillColor,fill opacity=0.20] ( 78.01, 80.91) circle (  2.13);

\path[fill=fillColor,fill opacity=0.20] ( 78.23, 75.31) circle (  2.13);

\path[fill=fillColor,fill opacity=0.20] ( 83.25, 68.16) circle (  2.13);

\path[fill=fillColor,fill opacity=0.20] ( 92.65, 57.31) circle (  2.13);

\path[fill=fillColor,fill opacity=0.20] ( 90.25, 53.70) circle (  2.13);

\path[fill=fillColor,fill opacity=0.20] ( 93.52, 69.63) circle (  2.13);

\path[fill=fillColor,fill opacity=0.20] ( 91.34, 82.12) circle (  2.13);

\path[fill=fillColor,fill opacity=0.20] ( 93.52, 68.94) circle (  2.13);

\path[fill=fillColor,fill opacity=0.20] ( 90.68, 52.41) circle (  2.13);

\path[fill=fillColor,fill opacity=0.20] ( 82.82, 59.30) circle (  2.13);

\path[fill=fillColor,fill opacity=0.20] ( 61.19, 68.51) circle (  2.13);

\path[fill=fillColor,fill opacity=0.20] ( 46.55, 77.72) circle (  2.13);

\path[fill=fillColor,fill opacity=0.20] ( 67.09, 99.68) circle (  2.13);

\path[fill=fillColor,fill opacity=0.20] ( 61.19, 78.76) circle (  2.13);

\path[fill=fillColor,fill opacity=0.20] ( 72.99, 81.00) circle (  2.13);

\path[fill=fillColor,fill opacity=0.20] ( 71.24, 77.64) circle (  2.13);

\path[fill=fillColor,fill opacity=0.20] ( 82.82, 67.56) circle (  2.13);

\path[fill=fillColor,fill opacity=0.20] ( 98.11, 65.58) circle (  2.13);

\path[fill=fillColor,fill opacity=0.20] (109.69, 67.22) circle (  2.13);

\path[fill=fillColor,fill opacity=0.20] ( 93.74, 62.65) circle (  2.13);

\path[fill=fillColor,fill opacity=0.20] ( 87.62, 65.41) circle (  2.13);

\path[fill=fillColor,fill opacity=0.20] ( 90.25, 79.88) circle (  2.13);

\path[fill=fillColor,fill opacity=0.20] ( 85.66, 86.16) circle (  2.13);

\path[fill=fillColor,fill opacity=0.20] ( 80.85, 78.24) circle (  2.13);

\path[fill=fillColor,fill opacity=0.20] ( 78.67, 73.07) circle (  2.13);

\path[fill=fillColor,fill opacity=0.20] ( 76.26, 66.44) circle (  2.13);

\path[fill=fillColor,fill opacity=0.20] ( 76.48, 79.45) circle (  2.13);

\path[fill=fillColor,fill opacity=0.20] ( 79.10, 75.83) circle (  2.13);

\path[fill=fillColor,fill opacity=0.20] ( 73.42, 65.32) circle (  2.13);

\path[fill=fillColor,fill opacity=0.20] ( 69.93, 62.48) circle (  2.13);

\path[fill=fillColor,fill opacity=0.20] ( 71.89, 72.56) circle (  2.13);

\path[fill=fillColor,fill opacity=0.20] ( 77.79, 81.34) circle (  2.13);

\path[fill=fillColor,fill opacity=0.20] ( 74.73, 74.54) circle (  2.13);

\path[fill=fillColor,fill opacity=0.20] ( 75.17, 68.25) circle (  2.13);

\path[fill=fillColor,fill opacity=0.20] ( 80.41, 75.23) circle (  2.13);

\path[fill=fillColor,fill opacity=0.20] ( 78.45, 72.90) circle (  2.13);

\path[fill=fillColor,fill opacity=0.20] ( 80.63, 60.41) circle (  2.13);

\path[fill=fillColor,fill opacity=0.20] ( 80.41, 60.93) circle (  2.13);

\path[fill=fillColor,fill opacity=0.20] ( 89.81, 70.06) circle (  2.13);

\path[fill=fillColor,fill opacity=0.20] ( 95.71, 68.42) circle (  2.13);

\path[fill=fillColor,fill opacity=0.20] ( 90.90, 62.14) circle (  2.13);

\path[fill=fillColor,fill opacity=0.20] ( 78.01, 68.51) circle (  2.13);

\path[fill=fillColor,fill opacity=0.20] ( 68.40, 76.00) circle (  2.13);

\path[fill=fillColor,fill opacity=0.20] ( 66.43, 66.61) circle (  2.13);

\path[fill=fillColor,fill opacity=0.20] ( 67.96, 63.26) circle (  2.13);

\path[fill=fillColor,fill opacity=0.20] ( 69.71, 78.15) circle (  2.13);

\path[fill=fillColor,fill opacity=0.20] ( 78.88, 87.63) circle (  2.13);

\path[fill=fillColor,fill opacity=0.20] ( 55.94, 95.64) circle (  2.13);

\path[fill=fillColor,fill opacity=0.20] ( 74.51, 73.59) circle (  2.13);

\path[fill=fillColor,fill opacity=0.20] ( 90.47, 62.91) circle (  2.13);

\path[fill=fillColor,fill opacity=0.20] ( 79.76, 52.92) circle (  2.13);

\path[fill=fillColor,fill opacity=0.20] ( 82.38, 59.12) circle (  2.13);

\path[fill=fillColor,fill opacity=0.20] ( 88.28, 77.98) circle (  2.13);

\path[fill=fillColor,fill opacity=0.20] ( 88.72, 79.62) circle (  2.13);

\path[fill=fillColor,fill opacity=0.20] ( 92.87, 71.95) circle (  2.13);

\path[fill=fillColor,fill opacity=0.20] ( 91.56, 75.92) circle (  2.13);

\path[fill=fillColor,fill opacity=0.20] ( 94.40, 72.13) circle (  2.13);

\path[fill=fillColor,fill opacity=0.20] ( 96.80, 72.47) circle (  2.13);

\path[fill=fillColor,fill opacity=0.20] ( 90.90, 83.58) circle (  2.13);

\path[fill=fillColor,fill opacity=0.20] ( 88.06, 74.11) circle (  2.13);

\path[fill=fillColor,fill opacity=0.20] ( 79.32, 60.67) circle (  2.13);

\path[fill=fillColor,fill opacity=0.20] ( 79.32, 68.77) circle (  2.13);

\path[fill=fillColor,fill opacity=0.20] ( 83.69, 77.29) circle (  2.13);

\path[fill=fillColor,fill opacity=0.20] ( 85.44, 73.25) circle (  2.13);

\path[fill=fillColor,fill opacity=0.20] (103.14, 82.80) circle (  2.13);

\path[fill=fillColor,fill opacity=0.20] ( 86.75, 76.17) circle (  2.13);

\path[fill=fillColor,fill opacity=0.20] ( 86.31, 55.68) circle (  2.13);

\path[fill=fillColor,fill opacity=0.20] ( 85.00, 54.99) circle (  2.13);

\path[fill=fillColor,fill opacity=0.20] ( 90.03, 67.30) circle (  2.13);

\path[fill=fillColor,fill opacity=0.20] ( 91.99, 71.95) circle (  2.13);

\path[fill=fillColor,fill opacity=0.20] ( 78.45, 69.80) circle (  2.13);

\path[fill=fillColor,fill opacity=0.20] ( 60.53, 71.52) circle (  2.13);

\path[fill=fillColor,fill opacity=0.20] ( 52.01, 82.29) circle (  2.13);

\path[fill=fillColor,fill opacity=0.20] ( 48.51, 94.09) circle (  2.13);

\path[fill=fillColor,fill opacity=0.20] ( 45.02,105.37) circle (  2.13);

\path[fill=fillColor,fill opacity=0.20] ( 45.24,115.96) circle (  2.13);

\path[fill=fillColor,fill opacity=0.20] ( 55.29, 86.77) circle (  2.13);

\path[fill=fillColor,fill opacity=0.20] ( 59.44, 70.06) circle (  2.13);

\path[fill=fillColor,fill opacity=0.20] ( 61.19, 58.78) circle (  2.13);

\path[fill=fillColor,fill opacity=0.20] ( 69.49, 63.26) circle (  2.13);

\path[fill=fillColor,fill opacity=0.20] ( 74.73, 64.72) circle (  2.13);

\path[fill=fillColor,fill opacity=0.20] ( 71.02, 62.14) circle (  2.13);

\path[fill=fillColor,fill opacity=0.20] ( 80.41, 68.34) circle (  2.13);

\path[fill=fillColor,fill opacity=0.20] ( 91.99, 63.43) circle (  2.13);

\path[fill=fillColor,fill opacity=0.20] ( 84.35, 54.56) circle (  2.13);

\path[fill=fillColor,fill opacity=0.20] ( 83.69, 61.79) circle (  2.13);

\path[fill=fillColor,fill opacity=0.20] (118.87, 64.72) circle (  2.13);

\path[fill=fillColor,fill opacity=0.20] ( 99.21, 54.82) circle (  2.13);

\path[fill=fillColor,fill opacity=0.20] ( 91.34, 58.26) circle (  2.13);

\path[fill=fillColor,fill opacity=0.20] ( 90.25, 73.07) circle (  2.13);

\path[fill=fillColor,fill opacity=0.20] ( 95.93, 78.58) circle (  2.13);

\path[fill=fillColor,fill opacity=0.20] ( 88.50, 72.82) circle (  2.13);

\path[fill=fillColor,fill opacity=0.20] (101.17, 69.37) circle (  2.13);

\path[fill=fillColor,fill opacity=0.20] (105.98, 71.01) circle (  2.13);

\path[fill=fillColor,fill opacity=0.20] ( 96.36, 71.61) circle (  2.13);

\path[fill=fillColor,fill opacity=0.20] ( 85.22, 63.60) circle (  2.13);

\path[fill=fillColor,fill opacity=0.20] ( 80.20, 55.42) circle (  2.13);

\path[fill=fillColor,fill opacity=0.20] ( 75.17, 55.68) circle (  2.13);

\path[fill=fillColor,fill opacity=0.20] ( 73.64, 61.36) circle (  2.13);

\path[fill=fillColor,fill opacity=0.20] ( 69.05, 68.51) circle (  2.13);

\path[fill=fillColor,fill opacity=0.20] ( 62.50, 73.50) circle (  2.13);

\path[fill=fillColor,fill opacity=0.20] ( 52.23, 86.77) circle (  2.13);

\path[fill=fillColor,fill opacity=0.20] ( 50.48, 84.70) circle (  2.13);

\path[fill=fillColor,fill opacity=0.20] ( 50.92, 77.64) circle (  2.13);

\path[fill=fillColor,fill opacity=0.20] ( 57.25, 74.97) circle (  2.13);

\path[fill=fillColor,fill opacity=0.20] ( 54.85, 69.54) circle (  2.13);

\path[fill=fillColor,fill opacity=0.20] ( 49.17, 62.40) circle (  2.13);

\path[fill=fillColor,fill opacity=0.20] ( 51.79, 63.00) circle (  2.13);

\path[fill=fillColor,fill opacity=0.20] (122.15, 59.55) circle (  2.13);

\path[fill=fillColor,fill opacity=0.20] ( 85.00, 44.14) circle (  2.13);

\path[fill=fillColor,fill opacity=0.20] ( 78.67, 49.22) circle (  2.13);

\path[fill=fillColor,fill opacity=0.20] ( 78.45, 66.18) circle (  2.13);

\path[fill=fillColor,fill opacity=0.20] ( 74.95, 62.91) circle (  2.13);

\path[fill=fillColor,fill opacity=0.20] ( 71.67, 56.11) circle (  2.13);

\path[fill=fillColor,fill opacity=0.20] ( 78.01, 63.69) circle (  2.13);

\path[fill=fillColor,fill opacity=0.20] ( 74.30, 62.91) circle (  2.13);

\path[fill=fillColor,fill opacity=0.20] ( 70.14, 47.58) circle (  2.13);

\path[fill=fillColor,fill opacity=0.20] ( 63.15, 46.21) circle (  2.13);

\path[fill=fillColor,fill opacity=0.20] ( 65.12, 65.32) circle (  2.13);

\path[fill=fillColor,fill opacity=0.20] (105.10, 77.98) circle (  2.13);

\path[fill=fillColor,fill opacity=0.20] ( 66.43, 80.31) circle (  2.13);

\path[fill=fillColor,fill opacity=0.20] ( 58.78, 83.41) circle (  2.13);

\path[fill=fillColor,fill opacity=0.20] ( 59.22, 64.38) circle (  2.13);

\path[fill=fillColor,fill opacity=0.20] ( 56.82, 75.40) circle (  2.13);

\path[fill=fillColor,fill opacity=0.20] ( 54.85, 75.31) circle (  2.13);

\path[fill=fillColor,fill opacity=0.20] ( 54.85, 76.00) circle (  2.13);

\path[fill=fillColor,fill opacity=0.20] ( 56.38, 85.39) circle (  2.13);

\path[fill=fillColor,fill opacity=0.20] ( 59.22, 82.20) circle (  2.13);

\path[fill=fillColor,fill opacity=0.20] ( 46.98, 73.85) circle (  2.13);

\path[fill=fillColor,fill opacity=0.20] ( 48.08, 78.58) circle (  2.13);

\path[fill=fillColor,fill opacity=0.20] ( 50.70, 90.73) circle (  2.13);

\path[fill=fillColor,fill opacity=0.20] ( 54.41, 98.65) circle (  2.13);

\path[fill=fillColor,fill opacity=0.20] ( 81.51, 99.25) circle (  2.13);

\path[fill=fillColor,fill opacity=0.20] ( 53.98,101.23) circle (  2.13);

\path[fill=fillColor,fill opacity=0.20] ( 85.66, 91.33) circle (  2.13);

\path[fill=fillColor,fill opacity=0.20] (133.07, 90.81) circle (  2.13);

\path[fill=fillColor,fill opacity=0.20] ( 95.93, 85.73) circle (  2.13);

\path[fill=fillColor,fill opacity=0.20] ( 81.51, 81.08) circle (  2.13);

\path[fill=fillColor,fill opacity=0.20] ( 78.23, 84.01) circle (  2.13);

\path[fill=fillColor,fill opacity=0.20] ( 73.42,108.38) circle (  2.13);

\path[fill=fillColor,fill opacity=0.20] ( 93.52, 91.16) circle (  2.13);

\path[fill=fillColor,fill opacity=0.20] (101.17, 82.72) circle (  2.13);

\path[fill=fillColor,fill opacity=0.20] ( 97.24, 72.04) circle (  2.13);

\path[fill=fillColor,fill opacity=0.20] ( 98.11, 64.29) circle (  2.13);

\path[fill=fillColor,fill opacity=0.20] ( 94.84, 68.34) circle (  2.13);

\path[fill=fillColor,fill opacity=0.20] ( 81.73, 69.89) circle (  2.13);

\path[fill=fillColor,fill opacity=0.20] ( 79.76, 70.32) circle (  2.13);

\path[fill=fillColor,fill opacity=0.20] ( 85.66, 80.05) circle (  2.13);

\path[fill=fillColor,fill opacity=0.20] ( 70.14, 83.32) circle (  2.13);

\path[fill=fillColor,fill opacity=0.20] ( 69.49, 77.72) circle (  2.13);

\path[fill=fillColor,fill opacity=0.20] ( 72.77,102.52) circle (  2.13);

\path[fill=fillColor,fill opacity=0.20] ( 86.31, 77.90) circle (  2.13);

\path[fill=fillColor,fill opacity=0.20] ( 92.87, 69.89) circle (  2.13);

\path[fill=fillColor,fill opacity=0.20] ( 99.64, 67.99) circle (  2.13);

\path[fill=fillColor,fill opacity=0.20] ( 95.93, 59.30) circle (  2.13);

\path[fill=fillColor,fill opacity=0.20] ( 92.21, 55.76) circle (  2.13);

\path[fill=fillColor,fill opacity=0.20] ( 89.15, 56.71) circle (  2.13);

\path[fill=fillColor,fill opacity=0.20] ( 80.20, 60.41) circle (  2.13);

\path[fill=fillColor,fill opacity=0.20] ( 81.29, 62.91) circle (  2.13);

\path[fill=fillColor,fill opacity=0.20] ( 77.14, 62.48) circle (  2.13);

\path[fill=fillColor,fill opacity=0.20] ( 79.32, 62.05) circle (  2.13);

\path[fill=fillColor,fill opacity=0.20] ( 78.23, 73.42) circle (  2.13);

\path[fill=fillColor,fill opacity=0.20] ( 66.43, 93.40) circle (  2.13);

\path[fill=fillColor,fill opacity=0.20] ( 60.31,112.51) circle (  2.13);

\path[fill=fillColor,fill opacity=0.20] ( 84.35, 75.74) circle (  2.13);

\path[fill=fillColor,fill opacity=0.20] ( 95.05, 61.19) circle (  2.13);

\path[fill=fillColor,fill opacity=0.20] ( 96.58, 56.71) circle (  2.13);

\path[fill=fillColor,fill opacity=0.20] ( 96.15, 59.12) circle (  2.13);

\path[fill=fillColor,fill opacity=0.20] ( 91.12, 56.45) circle (  2.13);

\path[fill=fillColor,fill opacity=0.20] ( 87.62, 57.31) circle (  2.13);

\path[fill=fillColor,fill opacity=0.20] ( 85.00, 55.85) circle (  2.13);

\path[fill=fillColor,fill opacity=0.20] ( 82.38, 56.63) circle (  2.13);

\path[fill=fillColor,fill opacity=0.20] ( 77.57, 62.14) circle (  2.13);

\path[fill=fillColor,fill opacity=0.20] ( 73.64, 70.75) circle (  2.13);

\path[fill=fillColor,fill opacity=0.20] ( 74.51, 83.15) circle (  2.13);

\path[fill=fillColor,fill opacity=0.20] ( 62.93, 88.83) circle (  2.13);

\path[fill=fillColor,fill opacity=0.20] ( 62.28, 87.54) circle (  2.13);

\path[fill=fillColor,fill opacity=0.20] ( 71.02, 94.60) circle (  2.13);

\path[fill=fillColor,fill opacity=0.20] ( 95.93, 67.91) circle (  2.13);

\path[fill=fillColor,fill opacity=0.20] ( 94.84, 65.67) circle (  2.13);

\path[fill=fillColor,fill opacity=0.20] ( 90.90, 57.49) circle (  2.13);

\path[fill=fillColor,fill opacity=0.20] ( 87.19, 60.24) circle (  2.13);

\path[fill=fillColor,fill opacity=0.20] ( 85.44, 64.03) circle (  2.13);

\path[fill=fillColor,fill opacity=0.20] ( 81.73, 61.96) circle (  2.13);

\path[fill=fillColor,fill opacity=0.20] ( 76.26, 61.88) circle (  2.13);

\path[fill=fillColor,fill opacity=0.20] ( 78.67, 60.76) circle (  2.13);

\path[fill=fillColor,fill opacity=0.20] ( 76.70, 57.49) circle (  2.13);

\path[fill=fillColor,fill opacity=0.20] ( 64.90, 65.41) circle (  2.13);

\path[fill=fillColor,fill opacity=0.20] ( 60.53, 83.49) circle (  2.13);

\path[fill=fillColor,fill opacity=0.20] ( 61.84, 90.47) circle (  2.13);

\path[fill=fillColor,fill opacity=0.20] ( 74.08, 82.37) circle (  2.13);

\path[fill=fillColor,fill opacity=0.20] ( 96.36, 60.41) circle (  2.13);

\path[fill=fillColor,fill opacity=0.20] ( 92.43, 64.46) circle (  2.13);

\path[fill=fillColor,fill opacity=0.20] ( 95.49, 53.70) circle (  2.13);

\path[fill=fillColor,fill opacity=0.20] ( 83.69, 58.26) circle (  2.13);

\path[fill=fillColor,fill opacity=0.20] ( 76.70, 71.78) circle (  2.13);

\path[fill=fillColor,fill opacity=0.20] ( 79.32, 69.54) circle (  2.13);

\path[fill=fillColor,fill opacity=0.20] ( 76.92, 64.72) circle (  2.13);

\path[fill=fillColor,fill opacity=0.20] ( 73.42, 65.93) circle (  2.13);

\path[fill=fillColor,fill opacity=0.20] ( 67.74, 59.55) circle (  2.13);

\path[fill=fillColor,fill opacity=0.20] ( 53.98, 60.85) circle (  2.13);

\path[fill=fillColor,fill opacity=0.20] ( 68.62, 94.43) circle (  2.13);

\path[fill=fillColor,fill opacity=0.20] ( 80.20, 71.52) circle (  2.13);

\path[fill=fillColor,fill opacity=0.20] ( 91.78, 52.66) circle (  2.13);

\path[fill=fillColor,fill opacity=0.20] ( 91.12, 55.16) circle (  2.13);

\path[fill=fillColor,fill opacity=0.20] (104.01, 51.63) circle (  2.13);

\path[fill=fillColor,fill opacity=0.20] ( 84.13, 57.06) circle (  2.13);

\path[fill=fillColor,fill opacity=0.20] ( 77.14, 68.25) circle (  2.13);

\path[fill=fillColor,fill opacity=0.20] ( 74.08, 72.64) circle (  2.13);

\path[fill=fillColor,fill opacity=0.20] ( 71.89, 65.84) circle (  2.13);

\path[fill=fillColor,fill opacity=0.20] ( 67.96, 68.60) circle (  2.13);

\path[fill=fillColor,fill opacity=0.20] ( 56.16, 77.98) circle (  2.13);

\path[fill=fillColor,fill opacity=0.20] ( 72.77, 62.83) circle (  2.13);

\path[fill=fillColor,fill opacity=0.20] ( 79.10, 61.36) circle (  2.13);

\path[fill=fillColor,fill opacity=0.20] ( 77.57, 64.38) circle (  2.13);

\path[fill=fillColor,fill opacity=0.20] ( 72.33, 72.04) circle (  2.13);

\path[fill=fillColor,fill opacity=0.20] ( 76.04, 82.46) circle (  2.13);

\path[fill=fillColor,fill opacity=0.20] ( 93.31, 63.08) circle (  2.13);

\path[fill=fillColor,fill opacity=0.20] ( 89.59, 61.19) circle (  2.13);

\path[fill=fillColor,fill opacity=0.20] ( 85.44, 57.49) circle (  2.13);

\path[fill=fillColor,fill opacity=0.20] ( 78.45, 59.64) circle (  2.13);

\path[fill=fillColor,fill opacity=0.20] ( 78.45, 62.22) circle (  2.13);

\path[fill=fillColor,fill opacity=0.20] ( 73.86, 67.22) circle (  2.13);

\path[fill=fillColor,fill opacity=0.20] ( 68.62, 70.49) circle (  2.13);

\path[fill=fillColor,fill opacity=0.20] ( 61.62, 76.78) circle (  2.13);

\path[fill=fillColor,fill opacity=0.20] ( 50.04, 92.19) circle (  2.13);

\path[fill=fillColor,fill opacity=0.20] ( 76.92, 71.35) circle (  2.13);

\path[fill=fillColor,fill opacity=0.20] ( 74.30, 63.00) circle (  2.13);

\path[fill=fillColor,fill opacity=0.20] ( 79.98, 76.78) circle (  2.13);

\path[fill=fillColor,fill opacity=0.20] ( 88.06, 75.57) circle (  2.13);

\path[fill=fillColor,fill opacity=0.20] ( 79.76, 73.33) circle (  2.13);

\path[fill=fillColor,fill opacity=0.20] ( 78.67, 69.20) circle (  2.13);

\path[fill=fillColor,fill opacity=0.20] ( 77.14, 58.35) circle (  2.13);

\path[fill=fillColor,fill opacity=0.20] ( 66.87, 93.22) circle (  2.13);

\path[fill=fillColor,fill opacity=0.20] ( 64.68, 92.19) circle (  2.13);

\path[fill=fillColor,fill opacity=0.20] (112.10, 68.51) circle (  2.13);

\path[fill=fillColor,fill opacity=0.20] ( 94.84, 63.77) circle (  2.13);

\path[fill=fillColor,fill opacity=0.20] ( 84.13, 54.73) circle (  2.13);

\path[fill=fillColor,fill opacity=0.20] ( 82.60, 58.26) circle (  2.13);

\path[fill=fillColor,fill opacity=0.20] ( 83.25, 61.19) circle (  2.13);

\path[fill=fillColor,fill opacity=0.20] ( 72.77, 62.65) circle (  2.13);

\path[fill=fillColor,fill opacity=0.20] ( 66.21, 83.58) circle (  2.13);

\path[fill=fillColor,fill opacity=0.20] ( 86.97, 62.91) circle (  2.13);

\path[fill=fillColor,fill opacity=0.20] ( 84.13, 66.10) circle (  2.13);

\path[fill=fillColor,fill opacity=0.20] ( 88.72, 70.23) circle (  2.13);

\path[fill=fillColor,fill opacity=0.20] ( 86.97, 64.98) circle (  2.13);

\path[fill=fillColor,fill opacity=0.20] ( 86.10, 65.67) circle (  2.13);

\path[fill=fillColor,fill opacity=0.20] ( 81.94, 70.58) circle (  2.13);

\path[fill=fillColor,fill opacity=0.20] ( 75.83, 68.34) circle (  2.13);

\path[fill=fillColor,fill opacity=0.20] ( 78.88, 75.05) circle (  2.13);

\path[fill=fillColor,fill opacity=0.20] ( 59.88, 91.85) circle (  2.13);

\path[fill=fillColor,fill opacity=0.20] (110.57, 51.29) circle (  2.13);

\path[fill=fillColor,fill opacity=0.20] (114.28, 54.64) circle (  2.13);

\path[fill=fillColor,fill opacity=0.20] ( 88.06, 48.01) circle (  2.13);

\path[fill=fillColor,fill opacity=0.20] ( 91.99, 55.25) circle (  2.13);

\path[fill=fillColor,fill opacity=0.20] ( 89.15, 64.29) circle (  2.13);

\path[fill=fillColor,fill opacity=0.20] ( 72.33, 59.64) circle (  2.13);

\path[fill=fillColor,fill opacity=0.20] ( 75.83, 68.51) circle (  2.13);

\path[fill=fillColor,fill opacity=0.20] ( 68.40, 80.13) circle (  2.13);

\path[fill=fillColor,fill opacity=0.20] ( 86.53, 74.45) circle (  2.13);

\path[fill=fillColor,fill opacity=0.20] ( 94.84, 56.63) circle (  2.13);

\path[fill=fillColor,fill opacity=0.20] ( 94.84, 58.95) circle (  2.13);

\path[fill=fillColor,fill opacity=0.20] ( 90.90, 55.33) circle (  2.13);

\path[fill=fillColor,fill opacity=0.20] ( 81.51, 53.96) circle (  2.13);

\path[fill=fillColor,fill opacity=0.20] ( 85.44, 50.68) circle (  2.13);

\path[fill=fillColor,fill opacity=0.20] ( 86.53, 57.31) circle (  2.13);

\path[fill=fillColor,fill opacity=0.20] ( 77.36, 66.27) circle (  2.13);

\path[fill=fillColor,fill opacity=0.20] ( 84.78, 68.42) circle (  2.13);

\path[fill=fillColor,fill opacity=0.20] ( 59.66,114.24) circle (  2.13);

\path[fill=fillColor,fill opacity=0.20] ( 97.68, 49.13) circle (  2.13);

\path[fill=fillColor,fill opacity=0.20] ( 98.77, 51.89) circle (  2.13);

\path[fill=fillColor,fill opacity=0.20] ( 84.78, 56.28) circle (  2.13);

\path[fill=fillColor,fill opacity=0.20] ( 85.66, 59.81) circle (  2.13);

\path[fill=fillColor,fill opacity=0.20] ( 82.16, 65.50) circle (  2.13);

\path[fill=fillColor,fill opacity=0.20] ( 74.95, 59.90) circle (  2.13);

\path[fill=fillColor,fill opacity=0.20] ( 74.51, 58.26) circle (  2.13);

\path[fill=fillColor,fill opacity=0.20] ( 68.62, 65.75) circle (  2.13);

\path[fill=fillColor,fill opacity=0.20] ( 94.62, 58.26) circle (  2.13);

\path[fill=fillColor,fill opacity=0.20] (102.26, 49.74) circle (  2.13);

\path[fill=fillColor,fill opacity=0.20] ( 94.62, 62.65) circle (  2.13);

\path[fill=fillColor,fill opacity=0.20] ( 92.21, 57.92) circle (  2.13);

\path[fill=fillColor,fill opacity=0.20] ( 92.43, 52.41) circle (  2.13);

\path[fill=fillColor,fill opacity=0.20] ( 87.41, 44.05) circle (  2.13);

\path[fill=fillColor,fill opacity=0.20] ( 91.12, 49.91) circle (  2.13);

\path[fill=fillColor,fill opacity=0.20] ( 81.94, 60.16) circle (  2.13);

\path[fill=fillColor,fill opacity=0.20] ( 78.67, 59.04) circle (  2.13);

\path[fill=fillColor,fill opacity=0.20] ( 91.56, 82.63) circle (  2.13);

\path[fill=fillColor,fill opacity=0.20] ( 65.56, 86.68) circle (  2.13);

\path[fill=fillColor,fill opacity=0.20] ( 82.60, 60.33) circle (  2.13);

\path[fill=fillColor,fill opacity=0.20] ( 89.81, 66.01) circle (  2.13);

\path[fill=fillColor,fill opacity=0.20] ( 85.88, 66.61) circle (  2.13);

\path[fill=fillColor,fill opacity=0.20] ( 82.60, 61.53) circle (  2.13);

\path[fill=fillColor,fill opacity=0.20] ( 81.29, 59.12) circle (  2.13);

\path[fill=fillColor,fill opacity=0.20] ( 71.89, 58.95) circle (  2.13);

\path[fill=fillColor,fill opacity=0.20] ( 66.65, 64.46) circle (  2.13);

\path[fill=fillColor,fill opacity=0.20] (104.67, 56.88) circle (  2.13);

\path[fill=fillColor,fill opacity=0.20] ( 98.11, 59.21) circle (  2.13);

\path[fill=fillColor,fill opacity=0.20] ( 91.99, 70.58) circle (  2.13);

\path[fill=fillColor,fill opacity=0.20] ( 96.58, 59.21) circle (  2.13);

\path[fill=fillColor,fill opacity=0.20] ( 99.21, 49.74) circle (  2.13);

\path[fill=fillColor,fill opacity=0.20] ( 92.87, 45.60) circle (  2.13);

\path[fill=fillColor,fill opacity=0.20] ( 99.21, 54.13) circle (  2.13);

\path[fill=fillColor,fill opacity=0.20] ( 88.06, 62.40) circle (  2.13);

\path[fill=fillColor,fill opacity=0.20] ( 80.41, 55.16) circle (  2.13);

\path[fill=fillColor,fill opacity=0.20] ( 78.23, 66.70) circle (  2.13);

\path[fill=fillColor,fill opacity=0.20] ( 76.04, 69.89) circle (  2.13);

\path[fill=fillColor,fill opacity=0.20] ( 87.84, 57.57) circle (  2.13);

\path[fill=fillColor,fill opacity=0.20] ( 90.47, 59.55) circle (  2.13);

\path[fill=fillColor,fill opacity=0.20] ( 77.79, 57.66) circle (  2.13);

\path[fill=fillColor,fill opacity=0.20] ( 79.32, 62.83) circle (  2.13);

\path[fill=fillColor,fill opacity=0.20] ( 75.17, 74.80) circle (  2.13);

\path[fill=fillColor,fill opacity=0.20] ( 65.34, 81.00) circle (  2.13);

\path[fill=fillColor,fill opacity=0.20] ( 67.74, 78.67) circle (  2.13);

\path[fill=fillColor,fill opacity=0.20] ( 96.36, 58.00) circle (  2.13);

\path[fill=fillColor,fill opacity=0.20] ( 87.41, 63.26) circle (  2.13);

\path[fill=fillColor,fill opacity=0.20] ( 89.59, 63.17) circle (  2.13);

\path[fill=fillColor,fill opacity=0.20] ( 92.43, 53.53) circle (  2.13);

\path[fill=fillColor,fill opacity=0.20] ( 93.52, 54.73) circle (  2.13);

\path[fill=fillColor,fill opacity=0.20] ( 87.41, 55.94) circle (  2.13);

\path[fill=fillColor,fill opacity=0.20] ( 93.31, 58.09) circle (  2.13);

\path[fill=fillColor,fill opacity=0.20] ( 92.65, 62.31) circle (  2.13);

\path[fill=fillColor,fill opacity=0.20] ( 80.41, 57.57) circle (  2.13);

\path[fill=fillColor,fill opacity=0.20] ( 89.37, 54.13) circle (  2.13);

\path[fill=fillColor,fill opacity=0.20] ( 46.55, 94.77) circle (  2.13);

\path[fill=fillColor,fill opacity=0.20] ( 66.43, 63.95) circle (  2.13);

\path[fill=fillColor,fill opacity=0.20] ( 83.04, 49.82) circle (  2.13);

\path[fill=fillColor,fill opacity=0.20] ( 82.16, 62.57) circle (  2.13);

\path[fill=fillColor,fill opacity=0.20] ( 76.70, 72.82) circle (  2.13);

\path[fill=fillColor,fill opacity=0.20] ( 78.67, 77.90) circle (  2.13);

\path[fill=fillColor,fill opacity=0.20] ( 72.77, 82.29) circle (  2.13);

\path[fill=fillColor,fill opacity=0.20] ( 69.71, 79.79) circle (  2.13);

\path[fill=fillColor,fill opacity=0.20] ( 66.87, 78.67) circle (  2.13);

\path[fill=fillColor,fill opacity=0.20] ( 78.88, 75.57) circle (  2.13);

\path[fill=fillColor,fill opacity=0.20] ( 84.78, 53.96) circle (  2.13);

\path[fill=fillColor,fill opacity=0.20] ( 83.91, 47.84) circle (  2.13);

\path[fill=fillColor,fill opacity=0.20] ( 91.12, 49.82) circle (  2.13);

\path[fill=fillColor,fill opacity=0.20] ( 93.31, 57.57) circle (  2.13);

\path[fill=fillColor,fill opacity=0.20] (100.08, 62.74) circle (  2.13);

\path[fill=fillColor,fill opacity=0.20] ( 98.55, 61.45) circle (  2.13);

\path[fill=fillColor,fill opacity=0.20] ( 95.49, 57.75) circle (  2.13);

\path[fill=fillColor,fill opacity=0.20] ( 91.12, 60.50) circle (  2.13);

\path[fill=fillColor,fill opacity=0.20] ( 84.13, 62.91) circle (  2.13);

\path[fill=fillColor,fill opacity=0.20] ( 82.60, 63.43) circle (  2.13);

\path[fill=fillColor,fill opacity=0.20] ( 74.51, 84.96) circle (  2.13);

\path[fill=fillColor,fill opacity=0.20] ( 52.45, 60.41) circle (  2.13);

\path[fill=fillColor,fill opacity=0.20] ( 71.89, 62.91) circle (  2.13);

\path[fill=fillColor,fill opacity=0.20] ( 80.63, 74.28) circle (  2.13);

\path[fill=fillColor,fill opacity=0.20] ( 86.10, 68.85) circle (  2.13);

\path[fill=fillColor,fill opacity=0.20] ( 78.01, 67.82) circle (  2.13);

\path[fill=fillColor,fill opacity=0.20] ( 72.33, 78.76) circle (  2.13);

\path[fill=fillColor,fill opacity=0.20] ( 68.62, 84.10) circle (  2.13);

\path[fill=fillColor,fill opacity=0.20] ( 70.80, 88.83) circle (  2.13);

\path[fill=fillColor,fill opacity=0.20] ( 76.26, 65.32) circle (  2.13);

\path[fill=fillColor,fill opacity=0.20] ( 81.07, 63.26) circle (  2.13);

\path[fill=fillColor,fill opacity=0.20] ( 90.90, 49.39) circle (  2.13);

\path[fill=fillColor,fill opacity=0.20] ( 95.05, 40.18) circle (  2.13);

\path[fill=fillColor,fill opacity=0.20] ( 94.18, 51.11) circle (  2.13);

\path[fill=fillColor,fill opacity=0.20] ( 97.89, 61.53) circle (  2.13);

\path[fill=fillColor,fill opacity=0.20] (104.45, 60.07) circle (  2.13);

\path[fill=fillColor,fill opacity=0.20] (107.07, 55.25) circle (  2.13);

\path[fill=fillColor,fill opacity=0.20] ( 99.42, 55.59) circle (  2.13);

\path[fill=fillColor,fill opacity=0.20] ( 93.96, 62.57) circle (  2.13);

\path[fill=fillColor,fill opacity=0.20] ( 91.12, 66.10) circle (  2.13);

\path[fill=fillColor,fill opacity=0.20] ( 81.07, 71.52) circle (  2.13);

\path[fill=fillColor,fill opacity=0.20] ( 52.88, 94.09) circle (  2.13);

\path[fill=fillColor,fill opacity=0.20] ( 54.19, 63.77) circle (  2.13);

\path[fill=fillColor,fill opacity=0.20] ( 77.36, 63.34) circle (  2.13);

\path[fill=fillColor,fill opacity=0.20] ( 90.90, 66.18) circle (  2.13);

\path[fill=fillColor,fill opacity=0.20] ( 79.98, 66.87) circle (  2.13);

\path[fill=fillColor,fill opacity=0.20] ( 75.61, 74.45) circle (  2.13);

\path[fill=fillColor,fill opacity=0.20] ( 73.42, 84.96) circle (  2.13);

\path[fill=fillColor,fill opacity=0.20] ( 69.49, 80.13) circle (  2.13);

\path[fill=fillColor,fill opacity=0.20] ( 82.82, 72.64) circle (  2.13);

\path[fill=fillColor,fill opacity=0.20] ( 89.15, 62.65) circle (  2.13);

\path[fill=fillColor,fill opacity=0.20] ( 86.75, 61.36) circle (  2.13);

\path[fill=fillColor,fill opacity=0.20] ( 97.02, 54.39) circle (  2.13);

\path[fill=fillColor,fill opacity=0.20] (101.83, 59.81) circle (  2.13);

\path[fill=fillColor,fill opacity=0.20] (109.91, 63.34) circle (  2.13);

\path[fill=fillColor,fill opacity=0.20] (105.10, 58.00) circle (  2.13);

\path[fill=fillColor,fill opacity=0.20] (104.45, 53.09) circle (  2.13);

\path[fill=fillColor,fill opacity=0.20] (100.52, 48.62) circle (  2.13);

\path[fill=fillColor,fill opacity=0.20] (100.08, 50.77) circle (  2.13);

\path[fill=fillColor,fill opacity=0.20] (101.39, 62.57) circle (  2.13);

\path[fill=fillColor,fill opacity=0.20] ( 94.18, 61.88) circle (  2.13);

\path[fill=fillColor,fill opacity=0.20] ( 83.04, 63.51) circle (  2.13);

\path[fill=fillColor,fill opacity=0.20] ( 53.32, 92.54) circle (  2.13);

\path[fill=fillColor,fill opacity=0.20] ( 80.85, 66.87) circle (  2.13);

\path[fill=fillColor,fill opacity=0.20] ( 76.26, 66.87) circle (  2.13);

\path[fill=fillColor,fill opacity=0.20] ( 85.66, 73.59) circle (  2.13);

\path[fill=fillColor,fill opacity=0.20] ( 79.54, 72.13) circle (  2.13);

\path[fill=fillColor,fill opacity=0.20] ( 76.70, 74.62) circle (  2.13);

\path[fill=fillColor,fill opacity=0.20] ( 74.51, 74.88) circle (  2.13);

\path[fill=fillColor,fill opacity=0.20] ( 72.99, 75.66) circle (  2.13);

\path[fill=fillColor,fill opacity=0.20] ( 80.85, 71.09) circle (  2.13);

\path[fill=fillColor,fill opacity=0.20] ( 88.50, 67.39) circle (  2.13);

\path[fill=fillColor,fill opacity=0.20] ( 90.68, 63.17) circle (  2.13);

\path[fill=fillColor,fill opacity=0.20] ( 92.65, 52.49) circle (  2.13);

\path[fill=fillColor,fill opacity=0.20] ( 97.46, 60.24) circle (  2.13);

\path[fill=fillColor,fill opacity=0.20] ( 96.15, 76.17) circle (  2.13);

\path[fill=fillColor,fill opacity=0.20] (106.85, 71.61) circle (  2.13);

\path[fill=fillColor,fill opacity=0.20] (105.76, 55.94) circle (  2.13);

\path[fill=fillColor,fill opacity=0.20] ( 97.02, 47.15) circle (  2.13);

\path[fill=fillColor,fill opacity=0.20] ( 99.42, 43.28) circle (  2.13);

\path[fill=fillColor,fill opacity=0.20] ( 99.86, 52.32) circle (  2.13);

\path[fill=fillColor,fill opacity=0.20] (102.48, 64.29) circle (  2.13);

\path[fill=fillColor,fill opacity=0.20] ( 89.37, 60.33) circle (  2.13);

\path[fill=fillColor,fill opacity=0.20] ( 74.30, 64.72) circle (  2.13);

\path[fill=fillColor,fill opacity=0.20] ( 48.08, 94.00) circle (  2.13);

\path[fill=fillColor,fill opacity=0.20] ( 62.72, 72.30) circle (  2.13);

\path[fill=fillColor,fill opacity=0.20] ( 71.89, 71.27) circle (  2.13);

\path[fill=fillColor,fill opacity=0.20] ( 87.19, 76.60) circle (  2.13);

\path[fill=fillColor,fill opacity=0.20] ( 81.07, 72.13) circle (  2.13);

\path[fill=fillColor,fill opacity=0.20] ( 78.01, 71.52) circle (  2.13);

\path[fill=fillColor,fill opacity=0.20] ( 82.60, 77.29) circle (  2.13);

\path[fill=fillColor,fill opacity=0.20] ( 73.20, 80.31) circle (  2.13);

\path[fill=fillColor,fill opacity=0.20] ( 64.90, 86.59) circle (  2.13);

\path[fill=fillColor,fill opacity=0.20] ( 73.20, 74.02) circle (  2.13);

\path[fill=fillColor,fill opacity=0.20] ( 86.97, 63.86) circle (  2.13);

\path[fill=fillColor,fill opacity=0.20] ( 97.24, 61.28) circle (  2.13);

\path[fill=fillColor,fill opacity=0.20] ( 91.34, 54.82) circle (  2.13);

\path[fill=fillColor,fill opacity=0.20] (101.17, 52.49) circle (  2.13);

\path[fill=fillColor,fill opacity=0.20] (100.73, 60.41) circle (  2.13);

\path[fill=fillColor,fill opacity=0.20] (102.05, 67.99) circle (  2.13);

\path[fill=fillColor,fill opacity=0.20] (104.45, 66.87) circle (  2.13);

\path[fill=fillColor,fill opacity=0.20] (105.32, 55.94) circle (  2.13);

\path[fill=fillColor,fill opacity=0.20] (100.08, 43.71) circle (  2.13);

\path[fill=fillColor,fill opacity=0.20] (103.79, 49.65) circle (  2.13);

\path[fill=fillColor,fill opacity=0.20] (105.76, 63.95) circle (  2.13);

\path[fill=fillColor,fill opacity=0.20] (102.48, 66.61) circle (  2.13);

\path[fill=fillColor,fill opacity=0.20] ( 86.53, 66.53) circle (  2.13);

\path[fill=fillColor,fill opacity=0.20] ( 69.49, 80.65) circle (  2.13);

\path[fill=fillColor,fill opacity=0.20] (104.23, 76.17) circle (  2.13);

\path[fill=fillColor,fill opacity=0.20] ( 86.75, 69.11) circle (  2.13);

\path[fill=fillColor,fill opacity=0.20] ( 84.13, 67.22) circle (  2.13);

\path[fill=fillColor,fill opacity=0.20] ( 81.29, 75.92) circle (  2.13);

\path[fill=fillColor,fill opacity=0.20] ( 72.11, 74.62) circle (  2.13);

\path[fill=fillColor,fill opacity=0.20] ( 73.20, 73.59) circle (  2.13);

\path[fill=fillColor,fill opacity=0.20] ( 63.37, 95.64) circle (  2.13);

\path[fill=fillColor,fill opacity=0.20] ( 72.77, 76.52) circle (  2.13);

\path[fill=fillColor,fill opacity=0.20] ( 75.83, 75.74) circle (  2.13);

\path[fill=fillColor,fill opacity=0.20] ( 95.71, 65.41) circle (  2.13);

\path[fill=fillColor,fill opacity=0.20] ( 98.77, 59.55) circle (  2.13);

\path[fill=fillColor,fill opacity=0.20] ( 90.68, 60.16) circle (  2.13);

\path[fill=fillColor,fill opacity=0.20] ( 96.80, 56.54) circle (  2.13);

\path[fill=fillColor,fill opacity=0.20] ( 99.86, 54.82) circle (  2.13);

\path[fill=fillColor,fill opacity=0.20] (107.29, 56.11) circle (  2.13);

\path[fill=fillColor,fill opacity=0.20] (101.83, 57.75) circle (  2.13);

\path[fill=fillColor,fill opacity=0.20] ( 98.77, 52.66) circle (  2.13);

\path[fill=fillColor,fill opacity=0.20] (105.32, 49.82) circle (  2.13);

\path[fill=fillColor,fill opacity=0.20] (130.45, 61.19) circle (  2.13);

\path[fill=fillColor,fill opacity=0.20] (100.30, 65.32) circle (  2.13);

\path[fill=fillColor,fill opacity=0.20] ( 88.50, 63.86) circle (  2.13);

\path[fill=fillColor,fill opacity=0.20] ( 68.62, 77.55) circle (  2.13);

\path[fill=fillColor,fill opacity=0.20] ( 68.62, 77.29) circle (  2.13);

\path[fill=fillColor,fill opacity=0.20] ( 72.33, 59.21) circle (  2.13);

\path[fill=fillColor,fill opacity=0.20] ( 78.88, 63.95) circle (  2.13);

\path[fill=fillColor,fill opacity=0.20] ( 76.26, 74.02) circle (  2.13);

\path[fill=fillColor,fill opacity=0.20] ( 74.08, 65.15) circle (  2.13);

\path[fill=fillColor,fill opacity=0.20] ( 73.42, 68.25) circle (  2.13);

\path[fill=fillColor,fill opacity=0.20] ( 69.49, 82.63) circle (  2.13);

\path[fill=fillColor,fill opacity=0.20] ( 69.27, 88.83) circle (  2.13);

\path[fill=fillColor,fill opacity=0.20] ( 77.14, 74.54) circle (  2.13);

\path[fill=fillColor,fill opacity=0.20] ( 80.20, 69.03) circle (  2.13);

\path[fill=fillColor,fill opacity=0.20] ( 87.19, 74.37) circle (  2.13);

\path[fill=fillColor,fill opacity=0.20] ( 96.58, 67.82) circle (  2.13);

\path[fill=fillColor,fill opacity=0.20] (101.17, 62.83) circle (  2.13);

\path[fill=fillColor,fill opacity=0.20] ( 98.99, 64.46) circle (  2.13);

\path[fill=fillColor,fill opacity=0.20] ( 97.24, 53.53) circle (  2.13);

\path[fill=fillColor,fill opacity=0.20] ( 99.86, 47.33) circle (  2.13);

\path[fill=fillColor,fill opacity=0.20] (101.83, 57.40) circle (  2.13);

\path[fill=fillColor,fill opacity=0.20] (138.75, 58.43) circle (  2.13);

\path[fill=fillColor,fill opacity=0.20] (101.17, 51.72) circle (  2.13);

\path[fill=fillColor,fill opacity=0.20] ( 93.96, 54.64) circle (  2.13);

\path[fill=fillColor,fill opacity=0.20] ( 83.69, 58.35) circle (  2.13);

\path[fill=fillColor,fill opacity=0.20] ( 83.47, 56.71) circle (  2.13);

\path[fill=fillColor,fill opacity=0.20] ( 88.28, 68.51) circle (  2.13);

\path[fill=fillColor,fill opacity=0.20] ( 55.51, 81.08) circle (  2.13);

\path[fill=fillColor,fill opacity=0.20] ( 68.83, 69.20) circle (  2.13);

\path[fill=fillColor,fill opacity=0.20] ( 77.36, 64.81) circle (  2.13);

\path[fill=fillColor,fill opacity=0.20] ( 77.79, 61.28) circle (  2.13);

\path[fill=fillColor,fill opacity=0.20] ( 81.94, 59.73) circle (  2.13);

\path[fill=fillColor,fill opacity=0.20] ( 78.88, 61.19) circle (  2.13);

\path[fill=fillColor,fill opacity=0.20] ( 75.61, 67.56) circle (  2.13);

\path[fill=fillColor,fill opacity=0.20] ( 71.02, 76.26) circle (  2.13);

\path[fill=fillColor,fill opacity=0.20] ( 85.22, 65.06) circle (  2.13);

\path[fill=fillColor,fill opacity=0.20] ( 91.78, 57.83) circle (  2.13);

\path[fill=fillColor,fill opacity=0.20] ( 97.68, 67.30) circle (  2.13);

\path[fill=fillColor,fill opacity=0.20] ( 98.11, 67.13) circle (  2.13);

\path[fill=fillColor,fill opacity=0.20] (102.92, 60.33) circle (  2.13);

\path[fill=fillColor,fill opacity=0.20] (104.23, 59.30) circle (  2.13);

\path[fill=fillColor,fill opacity=0.20] (111.66, 48.88) circle (  2.13);

\path[fill=fillColor,fill opacity=0.20] (105.32, 43.79) circle (  2.13);

\path[fill=fillColor,fill opacity=0.20] ( 98.77, 57.57) circle (  2.13);

\path[fill=fillColor,fill opacity=0.20] ( 99.86, 59.04) circle (  2.13);

\path[fill=fillColor,fill opacity=0.20] ( 94.40, 53.78) circle (  2.13);

\path[fill=fillColor,fill opacity=0.20] ( 81.07, 61.36) circle (  2.13);

\path[fill=fillColor,fill opacity=0.20] ( 73.42, 64.20) circle (  2.13);

\path[fill=fillColor,fill opacity=0.20] ( 57.91, 68.60) circle (  2.13);

\path[fill=fillColor,fill opacity=0.20] ( 51.57, 91.85) circle (  2.13);

\path[fill=fillColor,fill opacity=0.20] ( 93.09, 64.20) circle (  2.13);

\path[fill=fillColor,fill opacity=0.20] ( 80.20, 61.45) circle (  2.13);

\path[fill=fillColor,fill opacity=0.20] ( 81.73, 62.22) circle (  2.13);

\path[fill=fillColor,fill opacity=0.20] ( 75.83, 57.57) circle (  2.13);

\path[fill=fillColor,fill opacity=0.20] ( 74.51, 62.57) circle (  2.13);

\path[fill=fillColor,fill opacity=0.20] ( 69.05, 68.34) circle (  2.13);

\path[fill=fillColor,fill opacity=0.20] ( 68.18, 71.87) circle (  2.13);

\path[fill=fillColor,fill opacity=0.20] ( 74.73, 70.58) circle (  2.13);

\path[fill=fillColor,fill opacity=0.20] ( 74.51, 73.93) circle (  2.13);

\path[fill=fillColor,fill opacity=0.20] ( 83.47, 73.59) circle (  2.13);

\path[fill=fillColor,fill opacity=0.20] ( 93.96, 63.08) circle (  2.13);

\path[fill=fillColor,fill opacity=0.20] ( 97.24, 60.50) circle (  2.13);

\path[fill=fillColor,fill opacity=0.20] ( 95.49, 63.60) circle (  2.13);

\path[fill=fillColor,fill opacity=0.20] (100.08, 60.33) circle (  2.13);

\path[fill=fillColor,fill opacity=0.20] ( 98.33, 58.35) circle (  2.13);

\path[fill=fillColor,fill opacity=0.20] ( 99.21, 59.55) circle (  2.13);

\path[fill=fillColor,fill opacity=0.20] (111.22, 55.33) circle (  2.13);

\path[fill=fillColor,fill opacity=0.20] ( 92.87, 52.32) circle (  2.13);

\path[fill=fillColor,fill opacity=0.20] ( 86.10, 55.25) circle (  2.13);

\path[fill=fillColor,fill opacity=0.20] ( 77.14, 59.90) circle (  2.13);

\path[fill=fillColor,fill opacity=0.20] ( 70.36, 80.39) circle (  2.13);

\path[fill=fillColor,fill opacity=0.20] ( 46.98, 85.90) circle (  2.13);

\path[fill=fillColor,fill opacity=0.20] ( 82.82, 65.84) circle (  2.13);

\path[fill=fillColor,fill opacity=0.20] ( 82.60, 66.53) circle (  2.13);

\path[fill=fillColor,fill opacity=0.20] ( 76.70, 72.13) circle (  2.13);

\path[fill=fillColor,fill opacity=0.20] ( 74.73, 71.70) circle (  2.13);

\path[fill=fillColor,fill opacity=0.20] ( 71.89, 76.43) circle (  2.13);

\path[fill=fillColor,fill opacity=0.20] ( 72.11, 76.43) circle (  2.13);

\path[fill=fillColor,fill opacity=0.20] ( 71.67, 76.69) circle (  2.13);

\path[fill=fillColor,fill opacity=0.20] ( 69.49, 76.52) circle (  2.13);

\path[fill=fillColor,fill opacity=0.20] ( 70.14, 78.58) circle (  2.13);

\path[fill=fillColor,fill opacity=0.20] ( 79.32, 70.83) circle (  2.13);

\path[fill=fillColor,fill opacity=0.20] ( 85.22, 63.69) circle (  2.13);

\path[fill=fillColor,fill opacity=0.20] ( 88.94, 67.91) circle (  2.13);

\path[fill=fillColor,fill opacity=0.20] ( 96.80, 65.67) circle (  2.13);

\path[fill=fillColor,fill opacity=0.20] ( 98.55, 56.54) circle (  2.13);

\path[fill=fillColor,fill opacity=0.20] ( 96.80, 57.14) circle (  2.13);

\path[fill=fillColor,fill opacity=0.20] (100.52, 60.59) circle (  2.13);

\path[fill=fillColor,fill opacity=0.20] (102.48, 59.81) circle (  2.13);

\path[fill=fillColor,fill opacity=0.20] (104.67, 62.05) circle (  2.13);

\path[fill=fillColor,fill opacity=0.20] ( 95.49, 60.24) circle (  2.13);

\path[fill=fillColor,fill opacity=0.20] ( 88.94, 59.81) circle (  2.13);

\path[fill=fillColor,fill opacity=0.20] ( 89.59, 63.08) circle (  2.13);

\path[fill=fillColor,fill opacity=0.20] ( 87.62, 61.62) circle (  2.13);

\path[fill=fillColor,fill opacity=0.20] ( 74.95, 63.26) circle (  2.13);

\path[fill=fillColor,fill opacity=0.20] ( 52.01, 84.10) circle (  2.13);

\path[fill=fillColor,fill opacity=0.20] ( 52.01, 84.70) circle (  2.13);

\path[fill=fillColor,fill opacity=0.20] ( 69.05, 68.16) circle (  2.13);

\path[fill=fillColor,fill opacity=0.20] ( 90.25, 75.83) circle (  2.13);

\path[fill=fillColor,fill opacity=0.20] ( 78.23, 78.24) circle (  2.13);

\path[fill=fillColor,fill opacity=0.20] ( 74.51, 80.57) circle (  2.13);

\path[fill=fillColor,fill opacity=0.20] ( 82.16, 76.86) circle (  2.13);

\path[fill=fillColor,fill opacity=0.20] ( 79.98, 71.95) circle (  2.13);

\path[fill=fillColor,fill opacity=0.20] ( 70.14, 69.89) circle (  2.13);

\path[fill=fillColor,fill opacity=0.20] ( 62.50, 89.26) circle (  2.13);

\path[fill=fillColor,fill opacity=0.20] ( 58.56,109.07) circle (  2.13);

\path[fill=fillColor,fill opacity=0.20] ( 71.89, 69.28) circle (  2.13);

\path[fill=fillColor,fill opacity=0.20] ( 77.14, 66.61) circle (  2.13);

\path[fill=fillColor,fill opacity=0.20] ( 78.88, 69.89) circle (  2.13);

\path[fill=fillColor,fill opacity=0.20] ( 89.59, 64.20) circle (  2.13);

\path[fill=fillColor,fill opacity=0.20] ( 98.99, 65.41) circle (  2.13);

\path[fill=fillColor,fill opacity=0.20] (101.61, 71.61) circle (  2.13);

\path[fill=fillColor,fill opacity=0.20] (105.10, 64.29) circle (  2.13);

\path[fill=fillColor,fill opacity=0.20] (115.16, 50.86) circle (  2.13);

\path[fill=fillColor,fill opacity=0.20] ( 93.09, 48.79) circle (  2.13);

\path[fill=fillColor,fill opacity=0.20] ( 91.12, 53.44) circle (  2.13);

\path[fill=fillColor,fill opacity=0.20] ( 83.91, 58.00) circle (  2.13);

\path[fill=fillColor,fill opacity=0.20] ( 83.04, 62.31) circle (  2.13);

\path[fill=fillColor,fill opacity=0.20] ( 78.67, 69.20) circle (  2.13);

\path[fill=fillColor,fill opacity=0.20] ( 70.36, 80.74) circle (  2.13);

\path[fill=fillColor,fill opacity=0.20] ( 61.62, 84.53) circle (  2.13);

\path[fill=fillColor,fill opacity=0.20] ( 59.88, 78.76) circle (  2.13);

\path[fill=fillColor,fill opacity=0.20] ( 45.67, 84.01) circle (  2.13);

\path[fill=fillColor,fill opacity=0.20] ( 75.17, 73.25) circle (  2.13);

\path[fill=fillColor,fill opacity=0.20] ( 75.83, 70.92) circle (  2.13);

\path[fill=fillColor,fill opacity=0.20] ( 77.57, 71.52) circle (  2.13);

\path[fill=fillColor,fill opacity=0.20] ( 77.57, 73.68) circle (  2.13);

\path[fill=fillColor,fill opacity=0.20] ( 76.26, 73.93) circle (  2.13);

\path[fill=fillColor,fill opacity=0.20] ( 80.41, 70.23) circle (  2.13);

\path[fill=fillColor,fill opacity=0.20] ( 80.63, 63.86) circle (  2.13);

\path[fill=fillColor,fill opacity=0.20] ( 70.58, 66.87) circle (  2.13);

\path[fill=fillColor,fill opacity=0.20] ( 66.43, 79.70) circle (  2.13);

\path[fill=fillColor,fill opacity=0.20] ( 68.40, 81.17) circle (  2.13);

\path[fill=fillColor,fill opacity=0.20] ( 66.87, 75.31) circle (  2.13);

\path[fill=fillColor,fill opacity=0.20] ( 71.89, 86.51) circle (  2.13);

\path[fill=fillColor,fill opacity=0.20] ( 72.11, 68.85) circle (  2.13);

\path[fill=fillColor,fill opacity=0.20] ( 74.95, 70.32) circle (  2.13);

\path[fill=fillColor,fill opacity=0.20] ( 78.01, 79.36) circle (  2.13);

\path[fill=fillColor,fill opacity=0.20] ( 87.41, 78.41) circle (  2.13);

\path[fill=fillColor,fill opacity=0.20] ( 97.24, 73.07) circle (  2.13);

\path[fill=fillColor,fill opacity=0.20] ( 93.52, 70.66) circle (  2.13);

\path[fill=fillColor,fill opacity=0.20] ( 91.78, 70.75) circle (  2.13);

\path[fill=fillColor,fill opacity=0.20] ( 92.21, 64.29) circle (  2.13);

\path[fill=fillColor,fill opacity=0.20] ( 92.87, 60.16) circle (  2.13);

\path[fill=fillColor,fill opacity=0.20] ( 87.84, 60.76) circle (  2.13);

\path[fill=fillColor,fill opacity=0.20] ( 78.45, 60.76) circle (  2.13);

\path[fill=fillColor,fill opacity=0.20] ( 68.18, 64.81) circle (  2.13);

\path[fill=fillColor,fill opacity=0.20] ( 65.12, 72.82) circle (  2.13);

\path[fill=fillColor,fill opacity=0.20] ( 59.22, 78.24) circle (  2.13);

\path[fill=fillColor,fill opacity=0.20] ( 59.66, 93.31) circle (  2.13);

\path[fill=fillColor,fill opacity=0.20] ( 74.08, 70.75) circle (  2.13);

\path[fill=fillColor,fill opacity=0.20] ( 76.26, 72.38) circle (  2.13);

\path[fill=fillColor,fill opacity=0.20] ( 70.36, 77.21) circle (  2.13);

\path[fill=fillColor,fill opacity=0.20] ( 79.98, 74.80) circle (  2.13);

\path[fill=fillColor,fill opacity=0.20] ( 83.04, 66.61) circle (  2.13);

\path[fill=fillColor,fill opacity=0.20] ( 79.54, 68.34) circle (  2.13);

\path[fill=fillColor,fill opacity=0.20] ( 75.39, 72.47) circle (  2.13);

\path[fill=fillColor,fill opacity=0.20] ( 75.39, 71.87) circle (  2.13);

\path[fill=fillColor,fill opacity=0.20] ( 77.79, 74.45) circle (  2.13);

\path[fill=fillColor,fill opacity=0.20] ( 74.73, 70.40) circle (  2.13);

\path[fill=fillColor,fill opacity=0.20] ( 66.43, 78.84) circle (  2.13);

\path[fill=fillColor,fill opacity=0.20] ( 77.36, 80.39) circle (  2.13);

\path[fill=fillColor,fill opacity=0.20] ( 79.76, 78.24) circle (  2.13);

\path[fill=fillColor,fill opacity=0.20] ( 88.94, 80.31) circle (  2.13);

\path[fill=fillColor,fill opacity=0.20] ( 98.11, 81.34) circle (  2.13);

\path[fill=fillColor,fill opacity=0.20] ( 89.59, 73.42) circle (  2.13);

\path[fill=fillColor,fill opacity=0.20] ( 89.81, 65.50) circle (  2.13);

\path[fill=fillColor,fill opacity=0.20] ( 91.12, 64.12) circle (  2.13);

\path[fill=fillColor,fill opacity=0.20] ( 75.83, 61.88) circle (  2.13);

\path[fill=fillColor,fill opacity=0.20] ( 67.96, 63.34) circle (  2.13);

\path[fill=fillColor,fill opacity=0.20] (103.36, 75.23) circle (  2.13);

\path[fill=fillColor,fill opacity=0.20] ( 60.31, 88.49) circle (  2.13);

\path[fill=fillColor,fill opacity=0.20] ( 58.13, 93.14) circle (  2.13);

\path[fill=fillColor,fill opacity=0.20] ( 83.04, 80.57) circle (  2.13);

\path[fill=fillColor,fill opacity=0.20] ( 69.71, 73.59) circle (  2.13);

\path[fill=fillColor,fill opacity=0.20] ( 71.02, 71.52) circle (  2.13);

\path[fill=fillColor,fill opacity=0.20] ( 78.01, 71.09) circle (  2.13);

\path[fill=fillColor,fill opacity=0.20] ( 75.39, 75.14) circle (  2.13);

\path[fill=fillColor,fill opacity=0.20] ( 74.95, 75.48) circle (  2.13);

\path[fill=fillColor,fill opacity=0.20] ( 73.20, 65.84) circle (  2.13);

\path[fill=fillColor,fill opacity=0.20] ( 75.61, 70.32) circle (  2.13);

\path[fill=fillColor,fill opacity=0.20] ( 76.92, 80.48) circle (  2.13);

\path[fill=fillColor,fill opacity=0.20] ( 77.36, 76.09) circle (  2.13);

\path[fill=fillColor,fill opacity=0.20] ( 72.77, 76.60) circle (  2.13);

\path[fill=fillColor,fill opacity=0.20] ( 68.62, 84.01) circle (  2.13);

\path[fill=fillColor,fill opacity=0.20] ( 78.88, 77.90) circle (  2.13);

\path[fill=fillColor,fill opacity=0.20] ( 72.77, 80.22) circle (  2.13);

\path[fill=fillColor,fill opacity=0.20] ( 71.89, 82.72) circle (  2.13);

\path[fill=fillColor,fill opacity=0.20] ( 65.56, 83.67) circle (  2.13);

\path[fill=fillColor,fill opacity=0.20] ( 68.40, 85.73) circle (  2.13);

\path[fill=fillColor,fill opacity=0.20] ( 67.30, 79.45) circle (  2.13);

\path[fill=fillColor,fill opacity=0.20] ( 62.06, 78.76) circle (  2.13);

\path[fill=fillColor,fill opacity=0.20] ( 63.59, 77.90) circle (  2.13);

\path[fill=fillColor,fill opacity=0.20] ( 68.83, 76.43) circle (  2.13);

\path[fill=fillColor,fill opacity=0.20] ( 78.67, 73.50) circle (  2.13);

\path[fill=fillColor,fill opacity=0.20] ( 80.85, 79.79) circle (  2.13);

\path[fill=fillColor,fill opacity=0.20] ( 78.45, 71.18) circle (  2.13);

\path[fill=fillColor,fill opacity=0.20] ( 74.51, 71.61) circle (  2.13);

\path[fill=fillColor,fill opacity=0.20] ( 72.77, 64.20) circle (  2.13);

\path[fill=fillColor,fill opacity=0.20] ( 70.14, 53.87) circle (  2.13);

\path[fill=fillColor,fill opacity=0.20] (106.85, 60.16) circle (  2.13);

\path[fill=fillColor,fill opacity=0.20] ( 64.68, 71.52) circle (  2.13);

\path[fill=fillColor,fill opacity=0.20] ( 57.03, 82.55) circle (  2.13);

\path[fill=fillColor,fill opacity=0.20] ( 51.35,101.84) circle (  2.13);

\path[fill=fillColor,fill opacity=0.20] ( 66.65, 84.78) circle (  2.13);

\path[fill=fillColor,fill opacity=0.20] ( 61.84, 70.49) circle (  2.13);

\path[fill=fillColor,fill opacity=0.20] ( 67.52, 73.25) circle (  2.13);

\path[fill=fillColor,fill opacity=0.20] ( 69.49, 72.73) circle (  2.13);

\path[fill=fillColor,fill opacity=0.20] ( 71.46, 64.29) circle (  2.13);

\path[fill=fillColor,fill opacity=0.20] ( 74.08, 70.66) circle (  2.13);

\path[fill=fillColor,fill opacity=0.20] ( 77.57, 82.80) circle (  2.13);

\path[fill=fillColor,fill opacity=0.20] ( 78.01, 81.34) circle (  2.13);

\path[fill=fillColor,fill opacity=0.20] ( 76.70, 72.47) circle (  2.13);

\path[fill=fillColor,fill opacity=0.20] ( 79.32, 76.26) circle (  2.13);

\path[fill=fillColor,fill opacity=0.20] ( 71.46, 79.53) circle (  2.13);

\path[fill=fillColor,fill opacity=0.20] ( 72.99, 70.92) circle (  2.13);

\path[fill=fillColor,fill opacity=0.20] ( 81.07, 65.24) circle (  2.13);

\path[fill=fillColor,fill opacity=0.20] ( 78.01, 68.42) circle (  2.13);

\path[fill=fillColor,fill opacity=0.20] ( 75.61, 69.37) circle (  2.13);

\path[fill=fillColor,fill opacity=0.20] ( 76.92, 81.00) circle (  2.13);

\path[fill=fillColor,fill opacity=0.20] ( 88.28, 87.45) circle (  2.13);

\path[fill=fillColor,fill opacity=0.20] ( 86.75, 72.21) circle (  2.13);

\path[fill=fillColor,fill opacity=0.20] ( 87.19, 64.03) circle (  2.13);

\path[fill=fillColor,fill opacity=0.20] ( 80.41, 75.74) circle (  2.13);

\path[fill=fillColor,fill opacity=0.20] ( 74.08, 78.07) circle (  2.13);

\path[fill=fillColor,fill opacity=0.20] ( 71.89, 76.35) circle (  2.13);

\path[fill=fillColor,fill opacity=0.20] ( 71.24, 79.19) circle (  2.13);

\path[fill=fillColor,fill opacity=0.20] ( 70.14, 76.35) circle (  2.13);

\path[fill=fillColor,fill opacity=0.20] ( 72.55, 75.66) circle (  2.13);

\path[fill=fillColor,fill opacity=0.20] ( 91.56, 79.45) circle (  2.13);

\path[fill=fillColor,fill opacity=0.20] ( 85.00, 73.93) circle (  2.13);

\path[fill=fillColor,fill opacity=0.20] ( 66.43, 59.98) circle (  2.13);

\path[fill=fillColor,fill opacity=0.20] ( 69.05, 60.85) circle (  2.13);

\path[fill=fillColor,fill opacity=0.20] ( 69.05, 71.87) circle (  2.13);

\path[fill=fillColor,fill opacity=0.20] ( 49.17, 84.87) circle (  2.13);

\path[fill=fillColor,fill opacity=0.20] ( 55.72,101.84) circle (  2.13);

\path[fill=fillColor,fill opacity=0.20] ( 54.63, 79.53) circle (  2.13);

\path[fill=fillColor,fill opacity=0.20] ( 67.52, 70.66) circle (  2.13);

\path[fill=fillColor,fill opacity=0.20] ( 65.56, 79.27) circle (  2.13);

\path[fill=fillColor,fill opacity=0.20] ( 69.27, 83.06) circle (  2.13);

\path[fill=fillColor,fill opacity=0.20] ( 68.83, 74.97) circle (  2.13);

\path[fill=fillColor,fill opacity=0.20] ( 67.09, 63.69) circle (  2.13);

\path[fill=fillColor,fill opacity=0.20] ( 73.20, 71.18) circle (  2.13);

\path[fill=fillColor,fill opacity=0.20] ( 83.04, 81.77) circle (  2.13);

\path[fill=fillColor,fill opacity=0.20] ( 75.17, 66.44) circle (  2.13);

\path[fill=fillColor,fill opacity=0.20] ( 79.54, 59.12) circle (  2.13);

\path[fill=fillColor,fill opacity=0.20] ( 87.41, 74.62) circle (  2.13);

\path[fill=fillColor,fill opacity=0.20] ( 74.95, 77.55) circle (  2.13);

\path[fill=fillColor,fill opacity=0.20] ( 70.58, 78.84) circle (  2.13);

\path[fill=fillColor,fill opacity=0.20] ( 76.48, 85.30) circle (  2.13);

\path[fill=fillColor,fill opacity=0.20] ( 76.48, 74.97) circle (  2.13);

\path[fill=fillColor,fill opacity=0.20] ( 74.30, 70.23) circle (  2.13);

\path[fill=fillColor,fill opacity=0.20] ( 72.99, 84.01) circle (  2.13);

\path[fill=fillColor,fill opacity=0.20] ( 75.61, 77.21) circle (  2.13);

\path[fill=fillColor,fill opacity=0.20] ( 78.01, 86.59) circle (  2.13);

\path[fill=fillColor,fill opacity=0.20] ( 78.67, 93.48) circle (  2.13);

\path[fill=fillColor,fill opacity=0.20] ( 76.04, 75.23) circle (  2.13);

\path[fill=fillColor,fill opacity=0.20] ( 78.23, 63.34) circle (  2.13);

\path[fill=fillColor,fill opacity=0.20] ( 80.85, 55.59) circle (  2.13);

\path[fill=fillColor,fill opacity=0.20] ( 66.43, 69.71) circle (  2.13);

\path[fill=fillColor,fill opacity=0.20] ( 55.51, 75.57) circle (  2.13);

\path[fill=fillColor,fill opacity=0.20] ( 62.93, 69.80) circle (  2.13);

\path[fill=fillColor,fill opacity=0.20] ( 59.22, 99.60) circle (  2.13);

\path[fill=fillColor,fill opacity=0.20] ( 60.75, 90.55) circle (  2.13);

\path[fill=fillColor,fill opacity=0.20] ( 62.50, 79.79) circle (  2.13);

\path[fill=fillColor,fill opacity=0.20] ( 64.90, 66.10) circle (  2.13);

\path[fill=fillColor,fill opacity=0.20] ( 62.93, 55.51) circle (  2.13);

\path[fill=fillColor,fill opacity=0.20] ( 67.74, 58.69) circle (  2.13);

\path[fill=fillColor,fill opacity=0.20] ( 70.80, 74.62) circle (  2.13);

\path[fill=fillColor,fill opacity=0.20] ( 73.20, 74.45) circle (  2.13);

\path[fill=fillColor,fill opacity=0.20] ( 71.46, 71.01) circle (  2.13);

\path[fill=fillColor,fill opacity=0.20] ( 75.17, 83.49) circle (  2.13);

\path[fill=fillColor,fill opacity=0.20] ( 75.39, 84.01) circle (  2.13);

\path[fill=fillColor,fill opacity=0.20] ( 80.41, 82.29) circle (  2.13);

\path[fill=fillColor,fill opacity=0.20] ( 77.36, 72.13) circle (  2.13);

\path[fill=fillColor,fill opacity=0.20] ( 85.88, 81.60) circle (  2.13);

\path[fill=fillColor,fill opacity=0.20] ( 84.57, 72.82) circle (  2.13);

\path[fill=fillColor,fill opacity=0.20] ( 88.50, 62.22) circle (  2.13);

\path[fill=fillColor,fill opacity=0.20] ( 81.51, 59.55) circle (  2.13);

\path[fill=fillColor,fill opacity=0.20] ( 88.72, 66.87) circle (  2.13);

\path[fill=fillColor,fill opacity=0.20] ( 74.51, 79.10) circle (  2.13);

\path[fill=fillColor,fill opacity=0.20] ( 58.13, 80.31) circle (  2.13);

\path[fill=fillColor,fill opacity=0.20] ( 53.10, 72.82) circle (  2.13);

\path[fill=fillColor,fill opacity=0.20] ( 53.76, 84.96) circle (  2.13);

\path[fill=fillColor,fill opacity=0.20] ( 59.00, 75.92) circle (  2.13);

\path[fill=fillColor,fill opacity=0.20] ( 61.62, 72.99) circle (  2.13);

\path[fill=fillColor,fill opacity=0.20] ( 69.05, 66.87) circle (  2.13);

\path[fill=fillColor,fill opacity=0.20] ( 67.74, 72.56) circle (  2.13);

\path[fill=fillColor,fill opacity=0.20] ( 82.38, 66.36) circle (  2.13);

\path[fill=fillColor,fill opacity=0.20] ( 80.85, 50.68) circle (  2.13);

\path[fill=fillColor,fill opacity=0.20] ( 76.70, 79.79) circle (  2.13);

\path[fill=fillColor,fill opacity=0.20] ( 81.29, 57.92) circle (  2.13);

\path[fill=fillColor,fill opacity=0.20] ( 71.89, 40.61) circle (  2.13);

\path[fill=fillColor,fill opacity=0.20] ( 68.18, 47.24) circle (  2.13);

\path[fill=fillColor,fill opacity=0.20] ( 70.36, 62.57) circle (  2.13);

\path[fill=fillColor,fill opacity=0.20] ( 64.68, 72.47) circle (  2.13);

\path[fill=fillColor,fill opacity=0.20] ( 79.54, 67.48) circle (  2.13);

\path[fill=fillColor,fill opacity=0.20] ( 67.74, 63.26) circle (  2.13);

\path[fill=fillColor,fill opacity=0.20] ( 60.31, 75.14) circle (  2.13);

\path[fill=fillColor,fill opacity=0.20] ( 55.29, 90.12) circle (  2.13);

\path[fill=fillColor,fill opacity=0.20] ( 53.98,103.56) circle (  2.13);

\path[fill=fillColor,fill opacity=0.20] ( 55.29, 86.85) circle (  2.13);

\path[fill=fillColor,fill opacity=0.20] ( 69.27, 76.95) circle (  2.13);

\path[fill=fillColor,fill opacity=0.20] ( 68.18, 62.91) circle (  2.13);

\path[fill=fillColor,fill opacity=0.20] ( 76.70, 52.41) circle (  2.13);

\path[fill=fillColor,fill opacity=0.20] ( 79.32, 52.75) circle (  2.13);

\path[fill=fillColor,fill opacity=0.20] ( 86.97, 57.06) circle (  2.13);

\path[fill=fillColor,fill opacity=0.20] ( 83.91, 68.68) circle (  2.13);

\path[fill=fillColor,fill opacity=0.20] ( 58.56, 61.02) circle (  2.13);

\path[fill=fillColor,fill opacity=0.20] ( 49.61, 71.27) circle (  2.13);

\path[fill=fillColor,fill opacity=0.20] ( 46.77, 82.72) circle (  2.13);

\path[fill=fillColor,fill opacity=0.20] ( 58.35, 89.52) circle (  2.13);

\path[fill=fillColor,fill opacity=0.20] ( 62.50, 77.38) circle (  2.13);

\path[fill=fillColor,fill opacity=0.20] ( 78.01, 76.86) circle (  2.13);

\path[fill=fillColor,fill opacity=0.20] ( 65.99, 85.65) circle (  2.13);

\path[fill=fillColor,fill opacity=0.20] ( 69.49, 85.90) circle (  2.13);

\path[fill=fillColor,fill opacity=0.20] ( 78.45, 76.86) circle (  2.13);

\path[fill=fillColor,fill opacity=0.20] ( 48.95, 86.59) circle (  2.13);
\end{scope}
\begin{scope}
\path[clip] (159.87, 34.04) rectangle (277.04,119.86);
\definecolor[named]{fillColor}{rgb}{0.90,0.90,0.90}

\path[fill=fillColor] (159.87, 34.04) rectangle (277.03,119.86);
\definecolor[named]{drawColor}{rgb}{0.95,0.95,0.95}

\path[draw=drawColor,line width= 0.3pt,line join=round,line cap=round] (159.87, 34.15) --
	(277.04, 34.15);

\path[draw=drawColor,line width= 0.3pt,line join=round,line cap=round] (159.87, 51.37) --
	(277.04, 51.37);

\path[draw=drawColor,line width= 0.3pt,line join=round,line cap=round] (159.87, 68.60) --
	(277.04, 68.60);

\path[draw=drawColor,line width= 0.3pt,line join=round,line cap=round] (159.87, 85.82) --
	(277.04, 85.82);

\path[draw=drawColor,line width= 0.3pt,line join=round,line cap=round] (159.87,103.04) --
	(277.04,103.04);

\path[draw=drawColor,line width= 0.3pt,line join=round,line cap=round] (169.78, 34.04) --
	(169.78,119.86);

\path[draw=drawColor,line width= 0.3pt,line join=round,line cap=round] (191.63, 34.04) --
	(191.63,119.86);

\path[draw=drawColor,line width= 0.3pt,line join=round,line cap=round] (213.48, 34.04) --
	(213.48,119.86);

\path[draw=drawColor,line width= 0.3pt,line join=round,line cap=round] (235.33, 34.04) --
	(235.33,119.86);

\path[draw=drawColor,line width= 0.3pt,line join=round,line cap=round] (257.18, 34.04) --
	(257.18,119.86);
\definecolor[named]{drawColor}{rgb}{1.00,1.00,1.00}

\path[draw=drawColor,line width= 0.6pt,line join=round,line cap=round] (159.87, 42.76) --
	(277.04, 42.76);

\path[draw=drawColor,line width= 0.6pt,line join=round,line cap=round] (159.87, 59.98) --
	(277.04, 59.98);

\path[draw=drawColor,line width= 0.6pt,line join=round,line cap=round] (159.87, 77.21) --
	(277.04, 77.21);

\path[draw=drawColor,line width= 0.6pt,line join=round,line cap=round] (159.87, 94.43) --
	(277.04, 94.43);

\path[draw=drawColor,line width= 0.6pt,line join=round,line cap=round] (159.87,111.65) --
	(277.04,111.65);

\path[draw=drawColor,line width= 0.6pt,line join=round,line cap=round] (180.71, 34.04) --
	(180.71,119.86);

\path[draw=drawColor,line width= 0.6pt,line join=round,line cap=round] (202.56, 34.04) --
	(202.56,119.86);

\path[draw=drawColor,line width= 0.6pt,line join=round,line cap=round] (224.41, 34.04) --
	(224.41,119.86);

\path[draw=drawColor,line width= 0.6pt,line join=round,line cap=round] (246.26, 34.04) --
	(246.26,119.86);

\path[draw=drawColor,line width= 0.6pt,line join=round,line cap=round] (268.11, 34.04) --
	(268.11,119.86);
\definecolor[named]{fillColor}{rgb}{0.00,0.00,0.00}

\path[fill=fillColor,fill opacity=0.20] (185.95, 85.65) circle (  2.13);

\path[fill=fillColor,fill opacity=0.20] (193.16, 85.73) circle (  2.13);

\path[fill=fillColor,fill opacity=0.20] (191.41, 82.46) circle (  2.13);

\path[fill=fillColor,fill opacity=0.20] (196.88, 75.31) circle (  2.13);

\path[fill=fillColor,fill opacity=0.20] (186.61, 74.28) circle (  2.13);

\path[fill=fillColor,fill opacity=0.20] (177.65, 87.20) circle (  2.13);

\path[fill=fillColor,fill opacity=0.20] (187.48, 85.73) circle (  2.13);

\path[fill=fillColor,fill opacity=0.20] (202.56, 76.09) circle (  2.13);

\path[fill=fillColor,fill opacity=0.20] (213.05, 77.98) circle (  2.13);

\path[fill=fillColor,fill opacity=0.20] (214.14, 77.90) circle (  2.13);

\path[fill=fillColor,fill opacity=0.20] (207.58, 69.63) circle (  2.13);

\path[fill=fillColor,fill opacity=0.20] (209.99, 64.72) circle (  2.13);

\path[fill=fillColor,fill opacity=0.20] (210.42, 72.90) circle (  2.13);

\path[fill=fillColor,fill opacity=0.20] (200.15, 83.84) circle (  2.13);

\path[fill=fillColor,fill opacity=0.20] (186.83, 87.37) circle (  2.13);

\path[fill=fillColor,fill opacity=0.20] (177.21, 87.28) circle (  2.13);

\path[fill=fillColor,fill opacity=0.20] (170.66, 90.38) circle (  2.13);

\path[fill=fillColor,fill opacity=0.20] (194.69, 90.99) circle (  2.13);

\path[fill=fillColor,fill opacity=0.20] (211.30, 82.12) circle (  2.13);

\path[fill=fillColor,fill opacity=0.20] (222.00, 74.11) circle (  2.13);

\path[fill=fillColor,fill opacity=0.20] (219.38, 69.37) circle (  2.13);

\path[fill=fillColor,fill opacity=0.20] (211.52, 63.86) circle (  2.13);

\path[fill=fillColor,fill opacity=0.20] (206.93, 62.91) circle (  2.13);

\path[fill=fillColor,fill opacity=0.20] (202.12, 69.63) circle (  2.13);

\path[fill=fillColor,fill opacity=0.20] (202.56, 77.81) circle (  2.13);

\path[fill=fillColor,fill opacity=0.20] (201.47, 83.58) circle (  2.13);

\path[fill=fillColor,fill opacity=0.20] (194.91, 88.32) circle (  2.13);

\path[fill=fillColor,fill opacity=0.20] (188.14, 88.57) circle (  2.13);

\path[fill=fillColor,fill opacity=0.20] (183.77, 90.81) circle (  2.13);

\path[fill=fillColor,fill opacity=0.20] (189.67, 83.67) circle (  2.13);

\path[fill=fillColor,fill opacity=0.20] (206.27, 74.11) circle (  2.13);

\path[fill=fillColor,fill opacity=0.20] (220.69, 71.44) circle (  2.13);

\path[fill=fillColor,fill opacity=0.20] (221.13, 73.85) circle (  2.13);

\path[fill=fillColor,fill opacity=0.20] (216.54, 69.37) circle (  2.13);

\path[fill=fillColor,fill opacity=0.20] (216.54, 63.60) circle (  2.13);

\path[fill=fillColor,fill opacity=0.20] (213.70, 66.87) circle (  2.13);

\path[fill=fillColor,fill opacity=0.20] (205.62, 74.37) circle (  2.13);

\path[fill=fillColor,fill opacity=0.20] (203.00, 77.29) circle (  2.13);

\path[fill=fillColor,fill opacity=0.20] (214.79, 73.85) circle (  2.13);

\path[fill=fillColor,fill opacity=0.20] (214.79, 79.96) circle (  2.13);

\path[fill=fillColor,fill opacity=0.20] (200.81, 90.81) circle (  2.13);

\path[fill=fillColor,fill opacity=0.20] (182.24, 96.75) circle (  2.13);

\path[fill=fillColor,fill opacity=0.20] (195.57, 74.28) circle (  2.13);

\path[fill=fillColor,fill opacity=0.20] (215.67, 55.33) circle (  2.13);

\path[fill=fillColor,fill opacity=0.20] (222.22, 57.06) circle (  2.13);

\path[fill=fillColor,fill opacity=0.20] (215.67, 71.78) circle (  2.13);

\path[fill=fillColor,fill opacity=0.20] (218.07, 75.05) circle (  2.13);

\path[fill=fillColor,fill opacity=0.20] (219.60, 71.09) circle (  2.13);

\path[fill=fillColor,fill opacity=0.20] (209.33, 68.85) circle (  2.13);

\path[fill=fillColor,fill opacity=0.20] (204.96, 71.70) circle (  2.13);

\path[fill=fillColor,fill opacity=0.20] (199.50, 78.15) circle (  2.13);

\path[fill=fillColor,fill opacity=0.20] (215.01, 75.57) circle (  2.13);

\path[fill=fillColor,fill opacity=0.20] (215.23, 80.48) circle (  2.13);

\path[fill=fillColor,fill opacity=0.20] (208.46, 82.37) circle (  2.13);

\path[fill=fillColor,fill opacity=0.20] (204.96, 78.24) circle (  2.13);

\path[fill=fillColor,fill opacity=0.20] (201.90, 73.85) circle (  2.13);

\path[fill=fillColor,fill opacity=0.20] (195.57, 80.57) circle (  2.13);

\path[fill=fillColor,fill opacity=0.20] (184.64, 97.70) circle (  2.13);

\path[fill=fillColor,fill opacity=0.20] (197.97, 73.42) circle (  2.13);

\path[fill=fillColor,fill opacity=0.20] (215.89, 59.04) circle (  2.13);

\path[fill=fillColor,fill opacity=0.20] (218.73, 59.38) circle (  2.13);

\path[fill=fillColor,fill opacity=0.20] (213.92, 68.51) circle (  2.13);

\path[fill=fillColor,fill opacity=0.20] (211.74, 73.33) circle (  2.13);

\path[fill=fillColor,fill opacity=0.20] (210.86, 71.61) circle (  2.13);

\path[fill=fillColor,fill opacity=0.20] (206.71, 66.44) circle (  2.13);

\path[fill=fillColor,fill opacity=0.20] (203.65, 69.54) circle (  2.13);

\path[fill=fillColor,fill opacity=0.20] (190.32, 83.67) circle (  2.13);

\path[fill=fillColor,fill opacity=0.20] (209.77, 81.68) circle (  2.13);

\path[fill=fillColor,fill opacity=0.20] (221.13, 76.78) circle (  2.13);

\path[fill=fillColor,fill opacity=0.20] (216.54, 82.89) circle (  2.13);

\path[fill=fillColor,fill opacity=0.20] (215.45, 77.55) circle (  2.13);

\path[fill=fillColor,fill opacity=0.20] (219.60, 70.06) circle (  2.13);

\path[fill=fillColor,fill opacity=0.20] (213.05, 67.05) circle (  2.13);

\path[fill=fillColor,fill opacity=0.20] (210.64, 65.06) circle (  2.13);

\path[fill=fillColor,fill opacity=0.20] (208.68, 68.25) circle (  2.13);

\path[fill=fillColor,fill opacity=0.20] (185.52,104.33) circle (  2.13);

\path[fill=fillColor,fill opacity=0.20] (199.50, 79.96) circle (  2.13);

\path[fill=fillColor,fill opacity=0.20] (215.45, 73.33) circle (  2.13);

\path[fill=fillColor,fill opacity=0.20] (215.89, 70.32) circle (  2.13);

\path[fill=fillColor,fill opacity=0.20] (213.48, 68.42) circle (  2.13);

\path[fill=fillColor,fill opacity=0.20] (211.74, 68.42) circle (  2.13);

\path[fill=fillColor,fill opacity=0.20] (206.71, 69.11) circle (  2.13);

\path[fill=fillColor,fill opacity=0.20] (206.93, 67.22) circle (  2.13);

\path[fill=fillColor,fill opacity=0.20] (201.68, 68.60) circle (  2.13);

\path[fill=fillColor,fill opacity=0.20] (214.79, 70.58) circle (  2.13);

\path[fill=fillColor,fill opacity=0.20] (220.26, 71.35) circle (  2.13);

\path[fill=fillColor,fill opacity=0.20] (218.95, 71.09) circle (  2.13);

\path[fill=fillColor,fill opacity=0.20] (222.00, 61.36) circle (  2.13);

\path[fill=fillColor,fill opacity=0.20] (218.95, 58.69) circle (  2.13);

\path[fill=fillColor,fill opacity=0.20] (217.42, 62.48) circle (  2.13);

\path[fill=fillColor,fill opacity=0.20] (218.51, 61.79) circle (  2.13);

\path[fill=fillColor,fill opacity=0.20] (206.93, 60.59) circle (  2.13);

\path[fill=fillColor,fill opacity=0.20] (199.72, 87.02) circle (  2.13);

\path[fill=fillColor,fill opacity=0.20] (213.26, 78.67) circle (  2.13);

\path[fill=fillColor,fill opacity=0.20] (213.70, 74.88) circle (  2.13);

\path[fill=fillColor,fill opacity=0.20] (211.74, 72.04) circle (  2.13);

\path[fill=fillColor,fill opacity=0.20] (215.23, 68.85) circle (  2.13);

\path[fill=fillColor,fill opacity=0.20] (210.64, 66.87) circle (  2.13);

\path[fill=fillColor,fill opacity=0.20] (208.46, 66.44) circle (  2.13);

\path[fill=fillColor,fill opacity=0.20] (201.25, 66.70) circle (  2.13);

\path[fill=fillColor,fill opacity=0.20] (206.05, 79.53) circle (  2.13);

\path[fill=fillColor,fill opacity=0.20] (227.03, 56.37) circle (  2.13);

\path[fill=fillColor,fill opacity=0.20] (219.38, 56.80) circle (  2.13);

\path[fill=fillColor,fill opacity=0.20] (224.19, 50.68) circle (  2.13);

\path[fill=fillColor,fill opacity=0.20] (223.10, 49.31) circle (  2.13);

\path[fill=fillColor,fill opacity=0.20] (213.05, 59.04) circle (  2.13);

\path[fill=fillColor,fill opacity=0.20] (212.61, 58.95) circle (  2.13);

\path[fill=fillColor,fill opacity=0.20] (206.71, 55.59) circle (  2.13);

\path[fill=fillColor,fill opacity=0.20] (203.21, 59.64) circle (  2.13);

\path[fill=fillColor,fill opacity=0.20] (196.88, 62.91) circle (  2.13);

\path[fill=fillColor,fill opacity=0.20] (199.94, 94.26) circle (  2.13);

\path[fill=fillColor,fill opacity=0.20] (209.33, 79.36) circle (  2.13);

\path[fill=fillColor,fill opacity=0.20] (212.39, 73.76) circle (  2.13);

\path[fill=fillColor,fill opacity=0.20] (211.30, 72.73) circle (  2.13);

\path[fill=fillColor,fill opacity=0.20] (215.23, 66.79) circle (  2.13);

\path[fill=fillColor,fill opacity=0.20] (214.14, 57.31) circle (  2.13);

\path[fill=fillColor,fill opacity=0.20] (206.49, 57.83) circle (  2.13);

\path[fill=fillColor,fill opacity=0.20] (201.68, 63.17) circle (  2.13);

\path[fill=fillColor,fill opacity=0.20] (189.89, 69.54) circle (  2.13);

\path[fill=fillColor,fill opacity=0.20] (224.85, 50.77) circle (  2.13);

\path[fill=fillColor,fill opacity=0.20] (232.27, 42.59) circle (  2.13);

\path[fill=fillColor,fill opacity=0.20] (218.51, 49.65) circle (  2.13);

\path[fill=fillColor,fill opacity=0.20] (219.82, 43.28) circle (  2.13);

\path[fill=fillColor,fill opacity=0.20] (215.01, 48.19) circle (  2.13);

\path[fill=fillColor,fill opacity=0.20] (207.37, 58.00) circle (  2.13);

\path[fill=fillColor,fill opacity=0.20] (211.30, 50.43) circle (  2.13);

\path[fill=fillColor,fill opacity=0.20] (204.52, 48.96) circle (  2.13);

\path[fill=fillColor,fill opacity=0.20] (199.94, 61.53) circle (  2.13);

\path[fill=fillColor,fill opacity=0.20] (194.04, 65.75) circle (  2.13);

\path[fill=fillColor,fill opacity=0.20] (195.78,101.23) circle (  2.13);

\path[fill=fillColor,fill opacity=0.20] (210.42, 79.19) circle (  2.13);

\path[fill=fillColor,fill opacity=0.20] (217.85, 70.06) circle (  2.13);

\path[fill=fillColor,fill opacity=0.20] (215.89, 65.32) circle (  2.13);

\path[fill=fillColor,fill opacity=0.20] (216.98, 56.71) circle (  2.13);

\path[fill=fillColor,fill opacity=0.20] (215.23, 49.74) circle (  2.13);

\path[fill=fillColor,fill opacity=0.20] (205.62, 53.87) circle (  2.13);

\path[fill=fillColor,fill opacity=0.20] (199.06, 59.38) circle (  2.13);

\path[fill=fillColor,fill opacity=0.20] (195.57, 59.30) circle (  2.13);

\path[fill=fillColor,fill opacity=0.20] (176.99, 75.57) circle (  2.13);

\path[fill=fillColor,fill opacity=0.20] (185.73, 94.60) circle (  2.13);

\path[fill=fillColor,fill opacity=0.20] (240.58, 38.63) circle (  2.13);

\path[fill=fillColor,fill opacity=0.20] (222.00, 42.50) circle (  2.13);

\path[fill=fillColor,fill opacity=0.20] (214.36, 48.44) circle (  2.13);

\path[fill=fillColor,fill opacity=0.20] (217.20, 46.38) circle (  2.13);

\path[fill=fillColor,fill opacity=0.20] (216.76, 51.20) circle (  2.13);

\path[fill=fillColor,fill opacity=0.20] (215.01, 50.51) circle (  2.13);

\path[fill=fillColor,fill opacity=0.20] (214.58, 42.42) circle (  2.13);

\path[fill=fillColor,fill opacity=0.20] (209.33, 46.72) circle (  2.13);

\path[fill=fillColor,fill opacity=0.20] (201.03, 58.43) circle (  2.13);

\path[fill=fillColor,fill opacity=0.20] (204.09, 62.48) circle (  2.13);

\path[fill=fillColor,fill opacity=0.20] (183.11, 77.55) circle (  2.13);

\path[fill=fillColor,fill opacity=0.20] (210.86, 79.79) circle (  2.13);

\path[fill=fillColor,fill opacity=0.20] (221.35, 66.36) circle (  2.13);

\path[fill=fillColor,fill opacity=0.20] (221.79, 53.61) circle (  2.13);

\path[fill=fillColor,fill opacity=0.20] (216.76, 46.64) circle (  2.13);

\path[fill=fillColor,fill opacity=0.20] (213.26, 52.49) circle (  2.13);

\path[fill=fillColor,fill opacity=0.20] (208.89, 56.28) circle (  2.13);

\path[fill=fillColor,fill opacity=0.20] (200.59, 56.02) circle (  2.13);

\path[fill=fillColor,fill opacity=0.20] (201.03, 56.97) circle (  2.13);

\path[fill=fillColor,fill opacity=0.20] (190.76, 64.63) circle (  2.13);

\path[fill=fillColor,fill opacity=0.20] (204.96, 61.28) circle (  2.13);

\path[fill=fillColor,fill opacity=0.20] (221.79, 46.64) circle (  2.13);

\path[fill=fillColor,fill opacity=0.20] (213.48, 47.24) circle (  2.13);

\path[fill=fillColor,fill opacity=0.20] (214.79, 42.85) circle (  2.13);

\path[fill=fillColor,fill opacity=0.20] (218.07, 46.46) circle (  2.13);

\path[fill=fillColor,fill opacity=0.20] (214.36, 52.66) circle (  2.13);

\path[fill=fillColor,fill opacity=0.20] (211.30, 48.88) circle (  2.13);

\path[fill=fillColor,fill opacity=0.20] (209.77, 46.81) circle (  2.13);

\path[fill=fillColor,fill opacity=0.20] (207.15, 51.89) circle (  2.13);

\path[fill=fillColor,fill opacity=0.20] (208.68, 56.28) circle (  2.13);

\path[fill=fillColor,fill opacity=0.20] (198.19, 62.22) circle (  2.13);

\path[fill=fillColor,fill opacity=0.20] (180.71, 84.44) circle (  2.13);

\path[fill=fillColor,fill opacity=0.20] (198.63, 85.90) circle (  2.13);

\path[fill=fillColor,fill opacity=0.20] (208.89, 68.25) circle (  2.13);

\path[fill=fillColor,fill opacity=0.20] (218.95, 53.96) circle (  2.13);

\path[fill=fillColor,fill opacity=0.20] (214.14, 51.11) circle (  2.13);

\path[fill=fillColor,fill opacity=0.20] (210.64, 56.54) circle (  2.13);

\path[fill=fillColor,fill opacity=0.20] (215.89, 54.21) circle (  2.13);

\path[fill=fillColor,fill opacity=0.20] (209.33, 53.87) circle (  2.13);

\path[fill=fillColor,fill opacity=0.20] (204.09, 60.67) circle (  2.13);

\path[fill=fillColor,fill opacity=0.20] (197.31, 66.18) circle (  2.13);

\path[fill=fillColor,fill opacity=0.20] (186.83, 82.98) circle (  2.13);

\path[fill=fillColor,fill opacity=0.20] (211.52, 48.44) circle (  2.13);

\path[fill=fillColor,fill opacity=0.20] (208.89, 51.72) circle (  2.13);

\path[fill=fillColor,fill opacity=0.20] (209.33, 46.72) circle (  2.13);

\path[fill=fillColor,fill opacity=0.20] (221.35, 41.64) circle (  2.13);

\path[fill=fillColor,fill opacity=0.20] (214.79, 47.93) circle (  2.13);

\path[fill=fillColor,fill opacity=0.20] (218.07, 52.75) circle (  2.13);

\path[fill=fillColor,fill opacity=0.20] (218.73, 54.90) circle (  2.13);

\path[fill=fillColor,fill opacity=0.20] (210.21, 58.35) circle (  2.13);

\path[fill=fillColor,fill opacity=0.20] (206.71, 57.92) circle (  2.13);

\path[fill=fillColor,fill opacity=0.20] (205.84, 58.95) circle (  2.13);

\path[fill=fillColor,fill opacity=0.20] (194.91, 73.76) circle (  2.13);

\path[fill=fillColor,fill opacity=0.20] (198.41, 80.82) circle (  2.13);

\path[fill=fillColor,fill opacity=0.20] (213.05, 62.83) circle (  2.13);

\path[fill=fillColor,fill opacity=0.20] (212.83, 60.59) circle (  2.13);

\path[fill=fillColor,fill opacity=0.20] (209.33, 57.92) circle (  2.13);

\path[fill=fillColor,fill opacity=0.20] (214.58, 53.61) circle (  2.13);

\path[fill=fillColor,fill opacity=0.20] (216.76, 57.06) circle (  2.13);

\path[fill=fillColor,fill opacity=0.20] (206.93, 63.08) circle (  2.13);

\path[fill=fillColor,fill opacity=0.20] (201.25, 66.79) circle (  2.13);

\path[fill=fillColor,fill opacity=0.20] (187.48, 71.44) circle (  2.13);

\path[fill=fillColor,fill opacity=0.20] (208.46, 61.10) circle (  2.13);

\path[fill=fillColor,fill opacity=0.20] (206.49, 52.84) circle (  2.13);

\path[fill=fillColor,fill opacity=0.20] (209.11, 53.27) circle (  2.13);

\path[fill=fillColor,fill opacity=0.20] (213.48, 47.58) circle (  2.13);

\path[fill=fillColor,fill opacity=0.20] (217.42, 47.58) circle (  2.13);

\path[fill=fillColor,fill opacity=0.20] (215.01, 54.30) circle (  2.13);

\path[fill=fillColor,fill opacity=0.20] (207.15, 56.97) circle (  2.13);

\path[fill=fillColor,fill opacity=0.20] (215.45, 60.85) circle (  2.13);

\path[fill=fillColor,fill opacity=0.20] (210.86, 63.86) circle (  2.13);

\path[fill=fillColor,fill opacity=0.20] (208.24, 59.12) circle (  2.13);

\path[fill=fillColor,fill opacity=0.20] (199.94, 62.05) circle (  2.13);

\path[fill=fillColor,fill opacity=0.20] (203.00, 76.86) circle (  2.13);

\path[fill=fillColor,fill opacity=0.20] (209.33, 68.42) circle (  2.13);

\path[fill=fillColor,fill opacity=0.20] (208.02, 64.72) circle (  2.13);

\path[fill=fillColor,fill opacity=0.20] (208.68, 63.34) circle (  2.13);

\path[fill=fillColor,fill opacity=0.20] (211.95, 64.29) circle (  2.13);

\path[fill=fillColor,fill opacity=0.20] (209.11, 61.02) circle (  2.13);

\path[fill=fillColor,fill opacity=0.20] (204.74, 61.96) circle (  2.13);

\path[fill=fillColor,fill opacity=0.20] (197.97, 66.01) circle (  2.13);

\path[fill=fillColor,fill opacity=0.20] (189.01, 66.79) circle (  2.13);

\path[fill=fillColor,fill opacity=0.20] (176.12, 77.90) circle (  2.13);

\path[fill=fillColor,fill opacity=0.20] (210.64, 64.55) circle (  2.13);

\path[fill=fillColor,fill opacity=0.20] (212.83, 57.31) circle (  2.13);

\path[fill=fillColor,fill opacity=0.20] (208.89, 57.92) circle (  2.13);

\path[fill=fillColor,fill opacity=0.20] (213.48, 60.24) circle (  2.13);

\path[fill=fillColor,fill opacity=0.20] (222.22, 53.87) circle (  2.13);

\path[fill=fillColor,fill opacity=0.20] (217.85, 49.48) circle (  2.13);

\path[fill=fillColor,fill opacity=0.20] (219.82, 56.37) circle (  2.13);

\path[fill=fillColor,fill opacity=0.20] (205.40, 61.19) circle (  2.13);

\path[fill=fillColor,fill opacity=0.20] (208.02, 60.50) circle (  2.13);

\path[fill=fillColor,fill opacity=0.20] (205.84, 60.93) circle (  2.13);

\path[fill=fillColor,fill opacity=0.20] (204.74, 61.53) circle (  2.13);

\path[fill=fillColor,fill opacity=0.20] (196.88, 69.54) circle (  2.13);

\path[fill=fillColor,fill opacity=0.20] (200.15, 84.53) circle (  2.13);

\path[fill=fillColor,fill opacity=0.20] (208.24, 77.47) circle (  2.13);

\path[fill=fillColor,fill opacity=0.20] (209.55, 70.32) circle (  2.13);

\path[fill=fillColor,fill opacity=0.20] (210.42, 65.06) circle (  2.13);

\path[fill=fillColor,fill opacity=0.20] (207.58, 58.61) circle (  2.13);

\path[fill=fillColor,fill opacity=0.20] (207.58, 60.16) circle (  2.13);

\path[fill=fillColor,fill opacity=0.20] (202.56, 65.93) circle (  2.13);

\path[fill=fillColor,fill opacity=0.20] (200.59, 64.63) circle (  2.13);

\path[fill=fillColor,fill opacity=0.20] (190.98, 64.03) circle (  2.13);

\path[fill=fillColor,fill opacity=0.20] (208.89, 63.86) circle (  2.13);

\path[fill=fillColor,fill opacity=0.20] (217.63, 61.45) circle (  2.13);

\path[fill=fillColor,fill opacity=0.20] (210.86, 59.98) circle (  2.13);

\path[fill=fillColor,fill opacity=0.20] (216.32, 60.33) circle (  2.13);

\path[fill=fillColor,fill opacity=0.20] (217.63, 63.95) circle (  2.13);

\path[fill=fillColor,fill opacity=0.20] (222.00, 58.78) circle (  2.13);

\path[fill=fillColor,fill opacity=0.20] (215.45, 51.37) circle (  2.13);

\path[fill=fillColor,fill opacity=0.20] (210.42, 54.56) circle (  2.13);

\path[fill=fillColor,fill opacity=0.20] (211.08, 58.35) circle (  2.13);

\path[fill=fillColor,fill opacity=0.20] (210.42, 55.33) circle (  2.13);

\path[fill=fillColor,fill opacity=0.20] (202.78, 58.95) circle (  2.13);

\path[fill=fillColor,fill opacity=0.20] (196.22, 70.58) circle (  2.13);

\path[fill=fillColor,fill opacity=0.20] (201.03, 93.40) circle (  2.13);

\path[fill=fillColor,fill opacity=0.20] (217.42, 72.21) circle (  2.13);

\path[fill=fillColor,fill opacity=0.20] (211.52, 60.93) circle (  2.13);

\path[fill=fillColor,fill opacity=0.20] (204.09, 57.57) circle (  2.13);

\path[fill=fillColor,fill opacity=0.20] (211.52, 58.52) circle (  2.13);

\path[fill=fillColor,fill opacity=0.20] (209.77, 61.71) circle (  2.13);

\path[fill=fillColor,fill opacity=0.20] (206.93, 60.93) circle (  2.13);

\path[fill=fillColor,fill opacity=0.20] (204.96, 60.24) circle (  2.13);

\path[fill=fillColor,fill opacity=0.20] (192.94, 68.68) circle (  2.13);

\path[fill=fillColor,fill opacity=0.20] (214.14, 64.63) circle (  2.13);

\path[fill=fillColor,fill opacity=0.20] (211.74, 62.22) circle (  2.13);

\path[fill=fillColor,fill opacity=0.20] (219.38, 59.04) circle (  2.13);

\path[fill=fillColor,fill opacity=0.20] (221.79, 56.97) circle (  2.13);

\path[fill=fillColor,fill opacity=0.20] (218.07, 56.37) circle (  2.13);

\path[fill=fillColor,fill opacity=0.20] (218.29, 59.90) circle (  2.13);

\path[fill=fillColor,fill opacity=0.20] (215.23, 62.22) circle (  2.13);

\path[fill=fillColor,fill opacity=0.20] (218.51, 59.73) circle (  2.13);

\path[fill=fillColor,fill opacity=0.20] (213.92, 56.11) circle (  2.13);

\path[fill=fillColor,fill opacity=0.20] (213.26, 53.78) circle (  2.13);

\path[fill=fillColor,fill opacity=0.20] (202.12, 61.53) circle (  2.13);

\path[fill=fillColor,fill opacity=0.20] (190.54, 80.82) circle (  2.13);

\path[fill=fillColor,fill opacity=0.20] (215.45, 66.79) circle (  2.13);

\path[fill=fillColor,fill opacity=0.20] (209.99, 59.98) circle (  2.13);

\path[fill=fillColor,fill opacity=0.20] (208.02, 53.35) circle (  2.13);

\path[fill=fillColor,fill opacity=0.20] (206.05, 51.63) circle (  2.13);

\path[fill=fillColor,fill opacity=0.20] (203.65, 55.51) circle (  2.13);

\path[fill=fillColor,fill opacity=0.20] (203.87, 55.68) circle (  2.13);

\path[fill=fillColor,fill opacity=0.20] (207.15, 60.07) circle (  2.13);

\path[fill=fillColor,fill opacity=0.20] (200.15, 72.64) circle (  2.13);

\path[fill=fillColor,fill opacity=0.20] (182.24, 83.92) circle (  2.13);

\path[fill=fillColor,fill opacity=0.20] (202.56, 65.67) circle (  2.13);

\path[fill=fillColor,fill opacity=0.20] (212.17, 68.25) circle (  2.13);

\path[fill=fillColor,fill opacity=0.20] (215.01, 64.29) circle (  2.13);

\path[fill=fillColor,fill opacity=0.20] (215.89, 55.94) circle (  2.13);

\path[fill=fillColor,fill opacity=0.20] (222.22, 53.44) circle (  2.13);

\path[fill=fillColor,fill opacity=0.20] (219.60, 49.82) circle (  2.13);

\path[fill=fillColor,fill opacity=0.20] (220.48, 49.31) circle (  2.13);

\path[fill=fillColor,fill opacity=0.20] (216.98, 61.45) circle (  2.13);

\path[fill=fillColor,fill opacity=0.20] (215.89, 71.35) circle (  2.13);

\path[fill=fillColor,fill opacity=0.20] (212.61, 66.79) circle (  2.13);

\path[fill=fillColor,fill opacity=0.20] (205.62, 58.35) circle (  2.13);

\path[fill=fillColor,fill opacity=0.20] (201.90, 59.64) circle (  2.13);

\path[fill=fillColor,fill opacity=0.20] (202.78, 70.49) circle (  2.13);

\path[fill=fillColor,fill opacity=0.20] (201.25, 82.80) circle (  2.13);

\path[fill=fillColor,fill opacity=0.20] (212.39, 64.20) circle (  2.13);

\path[fill=fillColor,fill opacity=0.20] (213.48, 56.20) circle (  2.13);

\path[fill=fillColor,fill opacity=0.20] (214.36, 56.54) circle (  2.13);

\path[fill=fillColor,fill opacity=0.20] (207.58, 59.47) circle (  2.13);

\path[fill=fillColor,fill opacity=0.20] (203.87, 56.63) circle (  2.13);

\path[fill=fillColor,fill opacity=0.20] (200.59, 53.61) circle (  2.13);

\path[fill=fillColor,fill opacity=0.20] (199.94, 64.20) circle (  2.13);

\path[fill=fillColor,fill opacity=0.20] (193.60, 77.55) circle (  2.13);

\path[fill=fillColor,fill opacity=0.20] (172.84, 85.30) circle (  2.13);

\path[fill=fillColor,fill opacity=0.20] (205.62, 53.53) circle (  2.13);

\path[fill=fillColor,fill opacity=0.20] (206.49, 63.43) circle (  2.13);

\path[fill=fillColor,fill opacity=0.20] (216.98, 66.61) circle (  2.13);

\path[fill=fillColor,fill opacity=0.20] (221.79, 56.71) circle (  2.13);

\path[fill=fillColor,fill opacity=0.20] (225.50, 53.96) circle (  2.13);

\path[fill=fillColor,fill opacity=0.20] (223.32, 55.08) circle (  2.13);

\path[fill=fillColor,fill opacity=0.20] (222.66, 51.46) circle (  2.13);

\path[fill=fillColor,fill opacity=0.20] (219.16, 53.61) circle (  2.13);

\path[fill=fillColor,fill opacity=0.20] (209.77, 64.29) circle (  2.13);

\path[fill=fillColor,fill opacity=0.20] (205.18, 68.34) circle (  2.13);

\path[fill=fillColor,fill opacity=0.20] (202.12, 61.62) circle (  2.13);

\path[fill=fillColor,fill opacity=0.20] (204.74, 60.16) circle (  2.13);

\path[fill=fillColor,fill opacity=0.20] (187.92, 74.71) circle (  2.13);

\path[fill=fillColor,fill opacity=0.20] (206.71, 78.24) circle (  2.13);

\path[fill=fillColor,fill opacity=0.20] (220.91, 72.90) circle (  2.13);

\path[fill=fillColor,fill opacity=0.20] (218.51, 76.60) circle (  2.13);

\path[fill=fillColor,fill opacity=0.20] (218.51, 66.96) circle (  2.13);

\path[fill=fillColor,fill opacity=0.20] (208.24, 60.16) circle (  2.13);

\path[fill=fillColor,fill opacity=0.20] (201.90, 63.51) circle (  2.13);

\path[fill=fillColor,fill opacity=0.20] (197.75, 62.48) circle (  2.13);

\path[fill=fillColor,fill opacity=0.20] (198.19, 64.89) circle (  2.13);

\path[fill=fillColor,fill opacity=0.20] (198.41, 69.37) circle (  2.13);

\path[fill=fillColor,fill opacity=0.20] (193.16, 63.86) circle (  2.13);

\path[fill=fillColor,fill opacity=0.20] (180.05, 64.38) circle (  2.13);

\path[fill=fillColor,fill opacity=0.20] (193.60, 67.99) circle (  2.13);

\path[fill=fillColor,fill opacity=0.20] (206.93, 65.75) circle (  2.13);

\path[fill=fillColor,fill opacity=0.20] (208.46, 69.63) circle (  2.13);

\path[fill=fillColor,fill opacity=0.20] (210.86, 67.13) circle (  2.13);

\path[fill=fillColor,fill opacity=0.20] (217.20, 59.12) circle (  2.13);

\path[fill=fillColor,fill opacity=0.20] (218.07, 54.39) circle (  2.13);

\path[fill=fillColor,fill opacity=0.20] (220.48, 59.21) circle (  2.13);

\path[fill=fillColor,fill opacity=0.20] (221.35, 62.57) circle (  2.13);

\path[fill=fillColor,fill opacity=0.20] (222.00, 57.83) circle (  2.13);

\path[fill=fillColor,fill opacity=0.20] (209.55, 56.37) circle (  2.13);

\path[fill=fillColor,fill opacity=0.20] (205.84, 61.19) circle (  2.13);

\path[fill=fillColor,fill opacity=0.20] (201.90, 60.16) circle (  2.13);

\path[fill=fillColor,fill opacity=0.20] (200.15, 57.57) circle (  2.13);

\path[fill=fillColor,fill opacity=0.20] (185.30, 71.52) circle (  2.13);

\path[fill=fillColor,fill opacity=0.20] (207.37, 92.45) circle (  2.13);

\path[fill=fillColor,fill opacity=0.20] (212.83, 75.92) circle (  2.13);

\path[fill=fillColor,fill opacity=0.20] (211.08, 62.31) circle (  2.13);

\path[fill=fillColor,fill opacity=0.20] (203.65, 67.13) circle (  2.13);

\path[fill=fillColor,fill opacity=0.20] (197.75, 63.26) circle (  2.13);

\path[fill=fillColor,fill opacity=0.20] (199.06, 67.48) circle (  2.13);

\path[fill=fillColor,fill opacity=0.20] (200.81, 69.89) circle (  2.13);

\path[fill=fillColor,fill opacity=0.20] (196.88, 68.25) circle (  2.13);

\path[fill=fillColor,fill opacity=0.20] (190.98, 73.50) circle (  2.13);

\path[fill=fillColor,fill opacity=0.20] (183.99, 77.47) circle (  2.13);

\path[fill=fillColor,fill opacity=0.20] (178.30, 75.83) circle (  2.13);

\path[fill=fillColor,fill opacity=0.20] (193.60, 74.37) circle (  2.13);

\path[fill=fillColor,fill opacity=0.20] (203.87, 70.15) circle (  2.13);

\path[fill=fillColor,fill opacity=0.20] (207.58, 70.58) circle (  2.13);

\path[fill=fillColor,fill opacity=0.20] (206.71, 76.52) circle (  2.13);

\path[fill=fillColor,fill opacity=0.20] (207.37, 77.55) circle (  2.13);

\path[fill=fillColor,fill opacity=0.20] (211.74, 65.75) circle (  2.13);

\path[fill=fillColor,fill opacity=0.20] (213.48, 56.11) circle (  2.13);

\path[fill=fillColor,fill opacity=0.20] (213.05, 57.92) circle (  2.13);

\path[fill=fillColor,fill opacity=0.20] (215.89, 62.31) circle (  2.13);

\path[fill=fillColor,fill opacity=0.20] (219.16, 62.31) circle (  2.13);

\path[fill=fillColor,fill opacity=0.20] (222.22, 59.12) circle (  2.13);

\path[fill=fillColor,fill opacity=0.20] (210.86, 56.20) circle (  2.13);

\path[fill=fillColor,fill opacity=0.20] (206.49, 59.04) circle (  2.13);

\path[fill=fillColor,fill opacity=0.20] (194.04, 67.13) circle (  2.13);

\path[fill=fillColor,fill opacity=0.20] (190.54, 80.05) circle (  2.13);

\path[fill=fillColor,fill opacity=0.20] (208.68, 62.83) circle (  2.13);

\path[fill=fillColor,fill opacity=0.20] (211.74, 62.57) circle (  2.13);

\path[fill=fillColor,fill opacity=0.20] (204.52, 65.06) circle (  2.13);

\path[fill=fillColor,fill opacity=0.20] (198.19, 67.05) circle (  2.13);

\path[fill=fillColor,fill opacity=0.20] (201.03, 70.06) circle (  2.13);

\path[fill=fillColor,fill opacity=0.20] (194.47, 72.64) circle (  2.13);

\path[fill=fillColor,fill opacity=0.20] (191.85, 74.11) circle (  2.13);

\path[fill=fillColor,fill opacity=0.20] (194.47, 74.02) circle (  2.13);

\path[fill=fillColor,fill opacity=0.20] (196.66, 71.35) circle (  2.13);

\path[fill=fillColor,fill opacity=0.20] (189.67, 70.75) circle (  2.13);

\path[fill=fillColor,fill opacity=0.20] (171.97, 79.79) circle (  2.13);

\path[fill=fillColor,fill opacity=0.20] (193.16, 82.46) circle (  2.13);

\path[fill=fillColor,fill opacity=0.20] (224.63, 74.28) circle (  2.13);

\path[fill=fillColor,fill opacity=0.20] (207.37, 74.28) circle (  2.13);

\path[fill=fillColor,fill opacity=0.20] (206.05, 78.93) circle (  2.13);

\path[fill=fillColor,fill opacity=0.20] (205.18, 74.62) circle (  2.13);

\path[fill=fillColor,fill opacity=0.20] (207.15, 68.42) circle (  2.13);

\path[fill=fillColor,fill opacity=0.20] (211.74, 66.61) circle (  2.13);

\path[fill=fillColor,fill opacity=0.20] (213.05, 61.10) circle (  2.13);

\path[fill=fillColor,fill opacity=0.20] (216.98, 54.90) circle (  2.13);

\path[fill=fillColor,fill opacity=0.20] (217.42, 55.68) circle (  2.13);

\path[fill=fillColor,fill opacity=0.20] (211.30, 58.95) circle (  2.13);

\path[fill=fillColor,fill opacity=0.20] (209.55, 63.17) circle (  2.13);

\path[fill=fillColor,fill opacity=0.20] (209.11, 68.34) circle (  2.13);

\path[fill=fillColor,fill opacity=0.20] (204.31, 70.92) circle (  2.13);

\path[fill=fillColor,fill opacity=0.20] (189.89, 77.38) circle (  2.13);

\path[fill=fillColor,fill opacity=0.20] (206.27, 82.46) circle (  2.13);

\path[fill=fillColor,fill opacity=0.20] (217.42, 73.50) circle (  2.13);

\path[fill=fillColor,fill opacity=0.20] (203.65, 72.13) circle (  2.13);

\path[fill=fillColor,fill opacity=0.20] (203.65, 66.79) circle (  2.13);

\path[fill=fillColor,fill opacity=0.20] (199.50, 65.06) circle (  2.13);

\path[fill=fillColor,fill opacity=0.20] (196.44, 67.91) circle (  2.13);

\path[fill=fillColor,fill opacity=0.20] (197.97, 70.32) circle (  2.13);

\path[fill=fillColor,fill opacity=0.20] (194.47, 77.12) circle (  2.13);

\path[fill=fillColor,fill opacity=0.20] (194.04, 81.68) circle (  2.13);

\path[fill=fillColor,fill opacity=0.20] (198.19, 79.70) circle (  2.13);

\path[fill=fillColor,fill opacity=0.20] (190.98, 75.57) circle (  2.13);

\path[fill=fillColor,fill opacity=0.20] (190.98, 79.27) circle (  2.13);

\path[fill=fillColor,fill opacity=0.20] (184.86, 80.57) circle (  2.13);

\path[fill=fillColor,fill opacity=0.20] (180.49, 74.62) circle (  2.13);

\path[fill=fillColor,fill opacity=0.20] (179.83, 72.47) circle (  2.13);

\path[fill=fillColor,fill opacity=0.20] (181.15, 78.76) circle (  2.13);

\path[fill=fillColor,fill opacity=0.20] (176.12, 82.12) circle (  2.13);

\path[fill=fillColor,fill opacity=0.20] (176.34, 80.22) circle (  2.13);

\path[fill=fillColor,fill opacity=0.20] (185.52, 82.63) circle (  2.13);

\path[fill=fillColor,fill opacity=0.20] (179.40, 88.57) circle (  2.13);

\path[fill=fillColor,fill opacity=0.20] (180.05, 88.57) circle (  2.13);

\path[fill=fillColor,fill opacity=0.20] (180.27, 87.45) circle (  2.13);

\path[fill=fillColor,fill opacity=0.20] (180.93, 90.99) circle (  2.13);

\path[fill=fillColor,fill opacity=0.20] (180.05, 90.38) circle (  2.13);

\path[fill=fillColor,fill opacity=0.20] (186.61, 81.86) circle (  2.13);

\path[fill=fillColor,fill opacity=0.20] (196.00, 79.88) circle (  2.13);

\path[fill=fillColor,fill opacity=0.20] (187.70, 85.22) circle (  2.13);

\path[fill=fillColor,fill opacity=0.20] (189.23, 83.06) circle (  2.13);

\path[fill=fillColor,fill opacity=0.20] (191.85, 78.15) circle (  2.13);

\path[fill=fillColor,fill opacity=0.20] (194.47, 78.58) circle (  2.13);

\path[fill=fillColor,fill opacity=0.20] (200.15, 78.15) circle (  2.13);

\path[fill=fillColor,fill opacity=0.20] (198.19, 77.72) circle (  2.13);

\path[fill=fillColor,fill opacity=0.20] (197.53, 78.76) circle (  2.13);

\path[fill=fillColor,fill opacity=0.20] (206.93, 74.02) circle (  2.13);

\path[fill=fillColor,fill opacity=0.20] (208.46, 68.60) circle (  2.13);

\path[fill=fillColor,fill opacity=0.20] (211.74, 66.01) circle (  2.13);

\path[fill=fillColor,fill opacity=0.20] (215.67, 60.67) circle (  2.13);

\path[fill=fillColor,fill opacity=0.20] (215.45, 56.63) circle (  2.13);

\path[fill=fillColor,fill opacity=0.20] (213.70, 55.59) circle (  2.13);

\path[fill=fillColor,fill opacity=0.20] (217.20, 55.25) circle (  2.13);

\path[fill=fillColor,fill opacity=0.20] (206.93, 57.23) circle (  2.13);

\path[fill=fillColor,fill opacity=0.20] (208.46, 59.90) circle (  2.13);

\path[fill=fillColor,fill opacity=0.20] (204.09, 63.69) circle (  2.13);

\path[fill=fillColor,fill opacity=0.20] (196.00, 75.40) circle (  2.13);

\path[fill=fillColor,fill opacity=0.20] (190.76, 89.87) circle (  2.13);

\path[fill=fillColor,fill opacity=0.20] (202.12, 84.96) circle (  2.13);

\path[fill=fillColor,fill opacity=0.20] (212.39, 73.25) circle (  2.13);

\path[fill=fillColor,fill opacity=0.20] (205.18, 65.93) circle (  2.13);

\path[fill=fillColor,fill opacity=0.20] (204.74, 63.95) circle (  2.13);

\path[fill=fillColor,fill opacity=0.20] (203.65, 65.93) circle (  2.13);

\path[fill=fillColor,fill opacity=0.20] (200.59, 72.90) circle (  2.13);

\path[fill=fillColor,fill opacity=0.20] (195.78, 79.36) circle (  2.13);

\path[fill=fillColor,fill opacity=0.20] (199.50, 78.67) circle (  2.13);

\path[fill=fillColor,fill opacity=0.20] (199.50, 77.38) circle (  2.13);

\path[fill=fillColor,fill opacity=0.20] (197.31, 79.62) circle (  2.13);

\path[fill=fillColor,fill opacity=0.20] (196.66, 81.00) circle (  2.13);

\path[fill=fillColor,fill opacity=0.20] (196.66, 73.33) circle (  2.13);

\path[fill=fillColor,fill opacity=0.20] (205.40, 65.06) circle (  2.13);

\path[fill=fillColor,fill opacity=0.20] (198.84, 66.01) circle (  2.13);

\path[fill=fillColor,fill opacity=0.20] (194.26, 67.48) circle (  2.13);

\path[fill=fillColor,fill opacity=0.20] (198.41, 64.72) circle (  2.13);

\path[fill=fillColor,fill opacity=0.20] (200.59, 66.27) circle (  2.13);

\path[fill=fillColor,fill opacity=0.20] (201.25, 73.68) circle (  2.13);

\path[fill=fillColor,fill opacity=0.20] (198.84, 76.52) circle (  2.13);

\path[fill=fillColor,fill opacity=0.20] (200.81, 69.97) circle (  2.13);

\path[fill=fillColor,fill opacity=0.20] (199.72, 62.57) circle (  2.13);

\path[fill=fillColor,fill opacity=0.20] (196.44, 64.98) circle (  2.13);

\path[fill=fillColor,fill opacity=0.20] (197.97, 71.35) circle (  2.13);

\path[fill=fillColor,fill opacity=0.20] (199.28, 70.15) circle (  2.13);

\path[fill=fillColor,fill opacity=0.20] (201.90, 68.42) circle (  2.13);

\path[fill=fillColor,fill opacity=0.20] (200.37, 69.20) circle (  2.13);

\path[fill=fillColor,fill opacity=0.20] (202.78, 65.06) circle (  2.13);

\path[fill=fillColor,fill opacity=0.20] (203.43, 65.32) circle (  2.13);

\path[fill=fillColor,fill opacity=0.20] (206.71, 70.75) circle (  2.13);

\path[fill=fillColor,fill opacity=0.20] (206.93, 67.39) circle (  2.13);

\path[fill=fillColor,fill opacity=0.20] (209.11, 62.83) circle (  2.13);

\path[fill=fillColor,fill opacity=0.20] (208.02, 66.87) circle (  2.13);

\path[fill=fillColor,fill opacity=0.20] (209.99, 67.30) circle (  2.13);

\path[fill=fillColor,fill opacity=0.20] (212.83, 60.85) circle (  2.13);

\path[fill=fillColor,fill opacity=0.20] (214.14, 57.23) circle (  2.13);

\path[fill=fillColor,fill opacity=0.20] (215.67, 61.02) circle (  2.13);

\path[fill=fillColor,fill opacity=0.20] (209.33, 67.73) circle (  2.13);

\path[fill=fillColor,fill opacity=0.20] (204.09, 71.44) circle (  2.13);

\path[fill=fillColor,fill opacity=0.20] (201.47, 74.37) circle (  2.13);

\path[fill=fillColor,fill opacity=0.20] (189.67, 79.53) circle (  2.13);

\path[fill=fillColor,fill opacity=0.20] (187.70, 83.23) circle (  2.13);

\path[fill=fillColor,fill opacity=0.20] (182.46, 84.44) circle (  2.13);

\path[fill=fillColor,fill opacity=0.20] (205.40, 69.63) circle (  2.13);

\path[fill=fillColor,fill opacity=0.20] (212.83, 66.96) circle (  2.13);

\path[fill=fillColor,fill opacity=0.20] (204.31, 67.73) circle (  2.13);

\path[fill=fillColor,fill opacity=0.20] (205.62, 68.85) circle (  2.13);

\path[fill=fillColor,fill opacity=0.20] (204.31, 67.65) circle (  2.13);

\path[fill=fillColor,fill opacity=0.20] (203.43, 63.26) circle (  2.13);

\path[fill=fillColor,fill opacity=0.20] (199.06, 64.98) circle (  2.13);

\path[fill=fillColor,fill opacity=0.20] (198.84, 70.32) circle (  2.13);

\path[fill=fillColor,fill opacity=0.20] (197.31, 72.30) circle (  2.13);

\path[fill=fillColor,fill opacity=0.20] (204.96, 74.37) circle (  2.13);

\path[fill=fillColor,fill opacity=0.20] (199.06, 74.62) circle (  2.13);

\path[fill=fillColor,fill opacity=0.20] (201.25, 70.92) circle (  2.13);

\path[fill=fillColor,fill opacity=0.20] (197.97, 66.10) circle (  2.13);

\path[fill=fillColor,fill opacity=0.20] (201.25, 64.98) circle (  2.13);

\path[fill=fillColor,fill opacity=0.20] (203.00, 70.58) circle (  2.13);

\path[fill=fillColor,fill opacity=0.20] (200.59, 75.05) circle (  2.13);

\path[fill=fillColor,fill opacity=0.20] (199.72, 70.92) circle (  2.13);

\path[fill=fillColor,fill opacity=0.20] (199.50, 66.53) circle (  2.13);

\path[fill=fillColor,fill opacity=0.20] (206.49, 64.20) circle (  2.13);

\path[fill=fillColor,fill opacity=0.20] (199.50, 64.20) circle (  2.13);

\path[fill=fillColor,fill opacity=0.20] (201.03, 68.42) circle (  2.13);

\path[fill=fillColor,fill opacity=0.20] (201.47, 69.71) circle (  2.13);

\path[fill=fillColor,fill opacity=0.20] (203.87, 64.98) circle (  2.13);

\path[fill=fillColor,fill opacity=0.20] (208.02, 59.73) circle (  2.13);

\path[fill=fillColor,fill opacity=0.20] (208.02, 57.57) circle (  2.13);

\path[fill=fillColor,fill opacity=0.20] (208.68, 60.33) circle (  2.13);

\path[fill=fillColor,fill opacity=0.20] (208.89, 65.58) circle (  2.13);

\path[fill=fillColor,fill opacity=0.20] (208.46, 67.13) circle (  2.13);

\path[fill=fillColor,fill opacity=0.20] (205.18, 67.99) circle (  2.13);

\path[fill=fillColor,fill opacity=0.20] (200.81, 73.25) circle (  2.13);

\path[fill=fillColor,fill opacity=0.20] (199.28, 77.90) circle (  2.13);

\path[fill=fillColor,fill opacity=0.20] (199.72, 79.88) circle (  2.13);

\path[fill=fillColor,fill opacity=0.20] (195.78, 85.47) circle (  2.13);

\path[fill=fillColor,fill opacity=0.20] (187.70, 94.86) circle (  2.13);

\path[fill=fillColor,fill opacity=0.20] (179.83,100.20) circle (  2.13);

\path[fill=fillColor,fill opacity=0.20] (178.52,101.49) circle (  2.13);

\path[fill=fillColor,fill opacity=0.20] (198.19, 78.41) circle (  2.13);

\path[fill=fillColor,fill opacity=0.20] (201.47, 74.71) circle (  2.13);

\path[fill=fillColor,fill opacity=0.20] (202.78, 75.31) circle (  2.13);

\path[fill=fillColor,fill opacity=0.20] (206.27, 69.80) circle (  2.13);

\path[fill=fillColor,fill opacity=0.20] (200.37, 57.66) circle (  2.13);

\path[fill=fillColor,fill opacity=0.20] (201.68, 54.47) circle (  2.13);

\path[fill=fillColor,fill opacity=0.20] (201.90, 58.86) circle (  2.13);

\path[fill=fillColor,fill opacity=0.20] (204.96, 63.77) circle (  2.13);

\path[fill=fillColor,fill opacity=0.20] (203.00, 74.37) circle (  2.13);

\path[fill=fillColor,fill opacity=0.20] (201.68, 77.38) circle (  2.13);

\path[fill=fillColor,fill opacity=0.20] (202.34, 69.11) circle (  2.13);

\path[fill=fillColor,fill opacity=0.20] (202.12, 62.40) circle (  2.13);

\path[fill=fillColor,fill opacity=0.20] (202.34, 63.08) circle (  2.13);

\path[fill=fillColor,fill opacity=0.20] (200.81, 67.99) circle (  2.13);

\path[fill=fillColor,fill opacity=0.20] (204.74, 70.15) circle (  2.13);

\path[fill=fillColor,fill opacity=0.20] (198.41, 64.03) circle (  2.13);

\path[fill=fillColor,fill opacity=0.20] (199.72, 61.62) circle (  2.13);

\path[fill=fillColor,fill opacity=0.20] (204.74, 65.84) circle (  2.13);

\path[fill=fillColor,fill opacity=0.20] (202.34, 68.77) circle (  2.13);

\path[fill=fillColor,fill opacity=0.20] (204.74, 70.49) circle (  2.13);

\path[fill=fillColor,fill opacity=0.20] (203.65, 70.49) circle (  2.13);

\path[fill=fillColor,fill opacity=0.20] (204.31, 67.65) circle (  2.13);

\path[fill=fillColor,fill opacity=0.20] (208.46, 66.44) circle (  2.13);

\path[fill=fillColor,fill opacity=0.20] (201.90, 71.52) circle (  2.13);

\path[fill=fillColor,fill opacity=0.20] (198.84, 78.07) circle (  2.13);

\path[fill=fillColor,fill opacity=0.20] (193.60, 85.04) circle (  2.13);

\path[fill=fillColor,fill opacity=0.20] (187.92, 91.42) circle (  2.13);

\path[fill=fillColor,fill opacity=0.20] (194.04, 77.72) circle (  2.13);

\path[fill=fillColor,fill opacity=0.20] (200.37, 72.73) circle (  2.13);

\path[fill=fillColor,fill opacity=0.20] (202.12, 71.09) circle (  2.13);

\path[fill=fillColor,fill opacity=0.20] (209.11, 72.21) circle (  2.13);

\path[fill=fillColor,fill opacity=0.20] (205.18, 77.29) circle (  2.13);

\path[fill=fillColor,fill opacity=0.20] (204.74, 76.09) circle (  2.13);

\path[fill=fillColor,fill opacity=0.20] (204.52, 70.40) circle (  2.13);

\path[fill=fillColor,fill opacity=0.20] (203.43, 67.82) circle (  2.13);

\path[fill=fillColor,fill opacity=0.20] (200.37, 67.48) circle (  2.13);

\path[fill=fillColor,fill opacity=0.20] (200.59, 69.80) circle (  2.13);

\path[fill=fillColor,fill opacity=0.20] (197.10, 72.38) circle (  2.13);

\path[fill=fillColor,fill opacity=0.20] (198.84, 69.46) circle (  2.13);

\path[fill=fillColor,fill opacity=0.20] (199.50, 68.51) circle (  2.13);

\path[fill=fillColor,fill opacity=0.20] (199.94, 74.45) circle (  2.13);

\path[fill=fillColor,fill opacity=0.20] (197.31, 80.82) circle (  2.13);

\path[fill=fillColor,fill opacity=0.20] (198.84, 85.47) circle (  2.13);

\path[fill=fillColor,fill opacity=0.20] (191.20, 92.79) circle (  2.13);

\path[fill=fillColor,fill opacity=0.20] (191.63,100.80) circle (  2.13);

\path[fill=fillColor,fill opacity=0.20] (191.41, 92.10) circle (  2.13);

\path[fill=fillColor,fill opacity=0.20] (192.94, 86.25) circle (  2.13);

\path[fill=fillColor,fill opacity=0.20] (191.20, 89.61) circle (  2.13);

\path[fill=fillColor,fill opacity=0.20] (190.98, 88.75) circle (  2.13);

\path[fill=fillColor,fill opacity=0.20] (197.53, 48.44) circle (  2.13);

\path[fill=fillColor,fill opacity=0.20] (250.19, 76.09) circle (  2.13);

\path[fill=fillColor,fill opacity=0.20] (237.52, 60.50) circle (  2.13);

\path[fill=fillColor,fill opacity=0.20] (240.36, 73.50) circle (  2.13);

\path[fill=fillColor,fill opacity=0.20] (240.14, 65.06) circle (  2.13);

\path[fill=fillColor,fill opacity=0.20] (238.17, 63.00) circle (  2.13);

\path[fill=fillColor,fill opacity=0.20] (239.70, 62.40) circle (  2.13);

\path[fill=fillColor,fill opacity=0.20] (239.92, 51.29) circle (  2.13);

\path[fill=fillColor,fill opacity=0.20] (214.14, 57.06) circle (  2.13);

\path[fill=fillColor,fill opacity=0.20] (235.55, 42.33) circle (  2.13);

\path[fill=fillColor,fill opacity=0.20] (234.24, 45.95) circle (  2.13);

\path[fill=fillColor,fill opacity=0.20] (243.20, 51.03) circle (  2.13);

\path[fill=fillColor,fill opacity=0.20] (247.79, 43.54) circle (  2.13);

\path[fill=fillColor,fill opacity=0.20] (250.85, 47.76) circle (  2.13);

\path[fill=fillColor,fill opacity=0.20] (258.28, 52.15) circle (  2.13);

\path[fill=fillColor,fill opacity=0.20] (260.24, 51.29) circle (  2.13);

\path[fill=fillColor,fill opacity=0.20] (230.96, 60.24) circle (  2.13);

\path[fill=fillColor,fill opacity=0.20] (196.44, 77.21) circle (  2.13);

\path[fill=fillColor,fill opacity=0.20] (266.14, 54.64) circle (  2.13);

\path[fill=fillColor,fill opacity=0.20] (253.03, 50.51) circle (  2.13);

\path[fill=fillColor,fill opacity=0.20] (251.72, 38.80) circle (  2.13);

\path[fill=fillColor,fill opacity=0.20] (256.96, 43.19) circle (  2.13);

\path[fill=fillColor,fill opacity=0.20] (252.81, 57.92) circle (  2.13);

\path[fill=fillColor,fill opacity=0.20] (200.81, 67.82) circle (  2.13);

\path[fill=fillColor,fill opacity=0.20] (242.33, 45.69) circle (  2.13);

\path[fill=fillColor,fill opacity=0.20] (269.86, 46.12) circle (  2.13);

\path[fill=fillColor,fill opacity=0.20] (243.85, 52.06) circle (  2.13);

\path[fill=fillColor,fill opacity=0.20] (217.63, 50.17) circle (  2.13);

\path[fill=fillColor,fill opacity=0.20] (231.40, 40.95) circle (  2.13);

\path[fill=fillColor,fill opacity=0.20] (261.12, 47.58) circle (  2.13);

\path[fill=fillColor,fill opacity=0.20] (253.25, 48.70) circle (  2.13);

\path[fill=fillColor,fill opacity=0.20] (247.57, 49.82) circle (  2.13);

\path[fill=fillColor,fill opacity=0.20] (256.09, 49.05) circle (  2.13);

\path[fill=fillColor,fill opacity=0.20] (264.83, 48.96) circle (  2.13);

\path[fill=fillColor,fill opacity=0.20] (259.15, 48.53) circle (  2.13);

\path[fill=fillColor,fill opacity=0.20] (222.88, 49.31) circle (  2.13);

\path[fill=fillColor,fill opacity=0.20] (194.47, 63.95) circle (  2.13);

\path[fill=fillColor,fill opacity=0.20] (232.06, 90.99) circle (  2.13);

\path[fill=fillColor,fill opacity=0.20] (229.87, 74.80) circle (  2.13);

\path[fill=fillColor,fill opacity=0.20] (217.85, 75.31) circle (  2.13);

\path[fill=fillColor,fill opacity=0.20] (207.80, 68.51) circle (  2.13);

\path[fill=fillColor,fill opacity=0.20] (203.65, 66.96) circle (  2.13);

\path[fill=fillColor,fill opacity=0.20] (203.43, 74.97) circle (  2.13);

\path[fill=fillColor,fill opacity=0.20] (196.44, 82.03) circle (  2.13);

\path[fill=fillColor,fill opacity=0.20] (172.19, 71.01) circle (  2.13);

\path[fill=fillColor,fill opacity=0.20] (216.54, 59.90) circle (  2.13);

\path[fill=fillColor,fill opacity=0.20] (232.06, 52.84) circle (  2.13);

\path[fill=fillColor,fill opacity=0.20] (235.77, 54.39) circle (  2.13);

\path[fill=fillColor,fill opacity=0.20] (240.36, 52.15) circle (  2.13);

\path[fill=fillColor,fill opacity=0.20] (244.07, 49.99) circle (  2.13);

\path[fill=fillColor,fill opacity=0.20] (251.94, 51.98) circle (  2.13);

\path[fill=fillColor,fill opacity=0.20] (256.31, 46.98) circle (  2.13);

\path[fill=fillColor,fill opacity=0.20] (246.48, 42.42) circle (  2.13);

\path[fill=fillColor,fill opacity=0.20] (207.58, 49.48) circle (  2.13);

\path[fill=fillColor,fill opacity=0.20] (175.03, 63.51) circle (  2.13);

\path[fill=fillColor,fill opacity=0.20] (225.06, 83.92) circle (  2.13);

\path[fill=fillColor,fill opacity=0.20] (228.12, 67.05) circle (  2.13);

\path[fill=fillColor,fill opacity=0.20] (228.78, 60.93) circle (  2.13);

\path[fill=fillColor,fill opacity=0.20] (214.58, 64.89) circle (  2.13);

\path[fill=fillColor,fill opacity=0.20] (207.58, 64.81) circle (  2.13);

\path[fill=fillColor,fill opacity=0.20] (215.89, 60.50) circle (  2.13);

\path[fill=fillColor,fill opacity=0.20] (208.89, 68.08) circle (  2.13);

\path[fill=fillColor,fill opacity=0.20] (180.05, 76.95) circle (  2.13);

\path[fill=fillColor,fill opacity=0.20] (217.85, 57.57) circle (  2.13);

\path[fill=fillColor,fill opacity=0.20] (233.80, 58.95) circle (  2.13);

\path[fill=fillColor,fill opacity=0.20] (260.46, 56.20) circle (  2.13);

\path[fill=fillColor,fill opacity=0.20] (252.16, 48.53) circle (  2.13);

\path[fill=fillColor,fill opacity=0.20] (247.79, 46.03) circle (  2.13);

\path[fill=fillColor,fill opacity=0.20] (253.69, 44.74) circle (  2.13);

\path[fill=fillColor,fill opacity=0.20] (255.65, 47.15) circle (  2.13);

\path[fill=fillColor,fill opacity=0.20] (228.12, 49.39) circle (  2.13);

\path[fill=fillColor,fill opacity=0.20] (198.84, 59.98) circle (  2.13);

\path[fill=fillColor,fill opacity=0.20] (220.69, 71.09) circle (  2.13);

\path[fill=fillColor,fill opacity=0.20] (226.59, 57.49) circle (  2.13);

\path[fill=fillColor,fill opacity=0.20] (229.00, 66.87) circle (  2.13);

\path[fill=fillColor,fill opacity=0.20] (222.88, 66.53) circle (  2.13);

\path[fill=fillColor,fill opacity=0.20] (218.29, 67.56) circle (  2.13);

\path[fill=fillColor,fill opacity=0.20] (216.11, 68.16) circle (  2.13);

\path[fill=fillColor,fill opacity=0.20] (213.48, 60.41) circle (  2.13);

\path[fill=fillColor,fill opacity=0.20] (210.86, 63.95) circle (  2.13);

\path[fill=fillColor,fill opacity=0.20] (201.25, 76.86) circle (  2.13);

\path[fill=fillColor,fill opacity=0.20] (217.63, 56.54) circle (  2.13);

\path[fill=fillColor,fill opacity=0.20] (226.16, 40.61) circle (  2.13);

\path[fill=fillColor,fill opacity=0.20] (242.33, 51.20) circle (  2.13);

\path[fill=fillColor,fill opacity=0.20] (258.06, 41.56) circle (  2.13);

\path[fill=fillColor,fill opacity=0.20] (271.39, 45.86) circle (  2.13);

\path[fill=fillColor,fill opacity=0.20] (258.28, 46.21) circle (  2.13);

\path[fill=fillColor,fill opacity=0.20] (251.28, 43.79) circle (  2.13);

\path[fill=fillColor,fill opacity=0.20] (209.77, 52.32) circle (  2.13);

\path[fill=fillColor,fill opacity=0.20] (212.17, 83.15) circle (  2.13);

\path[fill=fillColor,fill opacity=0.20] (221.57, 60.59) circle (  2.13);

\path[fill=fillColor,fill opacity=0.20] (242.33, 55.08) circle (  2.13);

\path[fill=fillColor,fill opacity=0.20] (233.80, 55.08) circle (  2.13);

\path[fill=fillColor,fill opacity=0.20] (222.88, 58.78) circle (  2.13);

\path[fill=fillColor,fill opacity=0.20] (216.98, 67.48) circle (  2.13);

\path[fill=fillColor,fill opacity=0.20] (213.26, 66.79) circle (  2.13);

\path[fill=fillColor,fill opacity=0.20] (211.95, 59.12) circle (  2.13);

\path[fill=fillColor,fill opacity=0.20] (207.15, 61.88) circle (  2.13);

\path[fill=fillColor,fill opacity=0.20] (186.39, 69.37) circle (  2.13);

\path[fill=fillColor,fill opacity=0.20] (235.33, 64.63) circle (  2.13);

\path[fill=fillColor,fill opacity=0.20] (228.12, 60.85) circle (  2.13);

\path[fill=fillColor,fill opacity=0.20] (233.59, 54.82) circle (  2.13);

\path[fill=fillColor,fill opacity=0.20] (242.33, 61.62) circle (  2.13);

\path[fill=fillColor,fill opacity=0.20] (252.16, 52.92) circle (  2.13);

\path[fill=fillColor,fill opacity=0.20] (249.75, 43.71) circle (  2.13);

\path[fill=fillColor,fill opacity=0.20] (260.24, 52.32) circle (  2.13);

\path[fill=fillColor,fill opacity=0.20] (221.79, 44.31) circle (  2.13);

\path[fill=fillColor,fill opacity=0.20] (185.08, 51.20) circle (  2.13);

\path[fill=fillColor,fill opacity=0.20] (204.74, 78.67) circle (  2.13);

\path[fill=fillColor,fill opacity=0.20] (227.90, 60.50) circle (  2.13);

\path[fill=fillColor,fill opacity=0.20] (235.11, 58.78) circle (  2.13);

\path[fill=fillColor,fill opacity=0.20] (224.19, 49.31) circle (  2.13);

\path[fill=fillColor,fill opacity=0.20] (219.38, 46.89) circle (  2.13);

\path[fill=fillColor,fill opacity=0.20] (209.77, 57.92) circle (  2.13);

\path[fill=fillColor,fill opacity=0.20] (204.09, 57.57) circle (  2.13);

\path[fill=fillColor,fill opacity=0.20] (204.09, 53.09) circle (  2.13);

\path[fill=fillColor,fill opacity=0.20] (201.25, 58.35) circle (  2.13);

\path[fill=fillColor,fill opacity=0.20] (168.47, 65.24) circle (  2.13);

\path[fill=fillColor,fill opacity=0.20] (214.36, 73.33) circle (  2.13);

\path[fill=fillColor,fill opacity=0.20] (219.38, 60.50) circle (  2.13);

\path[fill=fillColor,fill opacity=0.20] (223.32, 66.96) circle (  2.13);

\path[fill=fillColor,fill opacity=0.20] (230.09, 70.23) circle (  2.13);

\path[fill=fillColor,fill opacity=0.20] (233.59, 61.28) circle (  2.13);

\path[fill=fillColor,fill opacity=0.20] (240.80, 51.03) circle (  2.13);

\path[fill=fillColor,fill opacity=0.20] (249.10, 47.33) circle (  2.13);

\path[fill=fillColor,fill opacity=0.20] (240.36, 49.48) circle (  2.13);

\path[fill=fillColor,fill opacity=0.20] (242.54, 57.49) circle (  2.13);

\path[fill=fillColor,fill opacity=0.20] (246.48, 59.30) circle (  2.13);

\path[fill=fillColor,fill opacity=0.20] (196.00, 56.11) circle (  2.13);

\path[fill=fillColor,fill opacity=0.20] (197.10, 84.78) circle (  2.13);

\path[fill=fillColor,fill opacity=0.20] (218.51, 60.50) circle (  2.13);

\path[fill=fillColor,fill opacity=0.20] (235.11, 59.47) circle (  2.13);

\path[fill=fillColor,fill opacity=0.20] (227.69, 55.85) circle (  2.13);

\path[fill=fillColor,fill opacity=0.20] (218.95, 47.15) circle (  2.13);

\path[fill=fillColor,fill opacity=0.20] (218.95, 50.51) circle (  2.13);

\path[fill=fillColor,fill opacity=0.20] (206.93, 51.03) circle (  2.13);

\path[fill=fillColor,fill opacity=0.20] (201.68, 46.64) circle (  2.13);

\path[fill=fillColor,fill opacity=0.20] (191.41, 56.97) circle (  2.13);

\path[fill=fillColor,fill opacity=0.20] (217.42, 63.43) circle (  2.13);

\path[fill=fillColor,fill opacity=0.20] (230.09, 61.10) circle (  2.13);

\path[fill=fillColor,fill opacity=0.20] (225.28, 57.06) circle (  2.13);

\path[fill=fillColor,fill opacity=0.20] (223.97, 70.32) circle (  2.13);

\path[fill=fillColor,fill opacity=0.20] (235.11, 68.77) circle (  2.13);

\path[fill=fillColor,fill opacity=0.20] (237.74, 50.17) circle (  2.13);

\path[fill=fillColor,fill opacity=0.20] (235.55, 47.33) circle (  2.13);

\path[fill=fillColor,fill opacity=0.20] (233.37, 52.75) circle (  2.13);

\path[fill=fillColor,fill opacity=0.20] (234.90, 55.33) circle (  2.13);

\path[fill=fillColor,fill opacity=0.20] (197.53, 60.93) circle (  2.13);

\path[fill=fillColor,fill opacity=0.20] (169.78, 70.23) circle (  2.13);

\path[fill=fillColor,fill opacity=0.20] (202.12, 64.72) circle (  2.13);

\path[fill=fillColor,fill opacity=0.20] (237.96, 55.85) circle (  2.13);

\path[fill=fillColor,fill opacity=0.20] (241.89, 56.63) circle (  2.13);

\path[fill=fillColor,fill opacity=0.20] (234.24, 49.13) circle (  2.13);

\path[fill=fillColor,fill opacity=0.20] (232.71, 47.84) circle (  2.13);

\path[fill=fillColor,fill opacity=0.20] (231.40, 55.85) circle (  2.13);

\path[fill=fillColor,fill opacity=0.20] (214.36, 55.16) circle (  2.13);

\path[fill=fillColor,fill opacity=0.20] (203.65, 56.37) circle (  2.13);

\path[fill=fillColor,fill opacity=0.20] (216.76, 67.13) circle (  2.13);

\path[fill=fillColor,fill opacity=0.20] (245.60, 63.86) circle (  2.13);

\path[fill=fillColor,fill opacity=0.20] (245.17, 52.06) circle (  2.13);

\path[fill=fillColor,fill opacity=0.20] (233.15, 55.85) circle (  2.13);

\path[fill=fillColor,fill opacity=0.20] (230.31, 67.65) circle (  2.13);

\path[fill=fillColor,fill opacity=0.20] (234.46, 71.35) circle (  2.13);

\path[fill=fillColor,fill opacity=0.20] (234.90, 61.96) circle (  2.13);

\path[fill=fillColor,fill opacity=0.20] (227.90, 55.76) circle (  2.13);

\path[fill=fillColor,fill opacity=0.20] (216.98, 50.68) circle (  2.13);

\path[fill=fillColor,fill opacity=0.20] (191.85, 47.50) circle (  2.13);

\path[fill=fillColor,fill opacity=0.20] (167.38, 63.00) circle (  2.13);

\path[fill=fillColor,fill opacity=0.20] (212.61, 58.35) circle (  2.13);

\path[fill=fillColor,fill opacity=0.20] (227.03, 54.21) circle (  2.13);

\path[fill=fillColor,fill opacity=0.20] (231.62, 47.93) circle (  2.13);

\path[fill=fillColor,fill opacity=0.20] (244.73, 45.95) circle (  2.13);

\path[fill=fillColor,fill opacity=0.20] (240.58, 59.04) circle (  2.13);

\path[fill=fillColor,fill opacity=0.20] (223.10, 68.16) circle (  2.13);

\path[fill=fillColor,fill opacity=0.20] (216.76, 61.02) circle (  2.13);

\path[fill=fillColor,fill opacity=0.20] (201.47, 79.45) circle (  2.13);

\path[fill=fillColor,fill opacity=0.20] (212.83, 65.06) circle (  2.13);

\path[fill=fillColor,fill opacity=0.20] (245.38, 67.82) circle (  2.13);

\path[fill=fillColor,fill opacity=0.20] (248.01, 62.57) circle (  2.13);

\path[fill=fillColor,fill opacity=0.20] (237.52, 58.61) circle (  2.13);

\path[fill=fillColor,fill opacity=0.20] (231.62, 55.08) circle (  2.13);

\path[fill=fillColor,fill opacity=0.20] (221.57, 61.19) circle (  2.13);

\path[fill=fillColor,fill opacity=0.20] (211.08, 61.79) circle (  2.13);

\path[fill=fillColor,fill opacity=0.20] (204.52, 57.92) circle (  2.13);

\path[fill=fillColor,fill opacity=0.20] (189.23, 54.64) circle (  2.13);

\path[fill=fillColor,fill opacity=0.20] (198.84, 75.57) circle (  2.13);

\path[fill=fillColor,fill opacity=0.20] (213.05, 60.85) circle (  2.13);

\path[fill=fillColor,fill opacity=0.20] (224.19, 50.86) circle (  2.13);

\path[fill=fillColor,fill opacity=0.20] (231.62, 52.06) circle (  2.13);

\path[fill=fillColor,fill opacity=0.20] (226.81, 58.00) circle (  2.13);

\path[fill=fillColor,fill opacity=0.20] (221.57, 63.34) circle (  2.13);

\path[fill=fillColor,fill opacity=0.20] (216.98, 65.15) circle (  2.13);

\path[fill=fillColor,fill opacity=0.20] (206.27, 73.76) circle (  2.13);

\path[fill=fillColor,fill opacity=0.20] (220.04, 61.62) circle (  2.13);

\path[fill=fillColor,fill opacity=0.20] (232.93, 65.93) circle (  2.13);

\path[fill=fillColor,fill opacity=0.20] (265.49, 70.66) circle (  2.13);

\path[fill=fillColor,fill opacity=0.20] (260.46, 66.10) circle (  2.13);

\path[fill=fillColor,fill opacity=0.20] (228.78, 57.75) circle (  2.13);

\path[fill=fillColor,fill opacity=0.20] (203.00, 51.80) circle (  2.13);

\path[fill=fillColor,fill opacity=0.20] (185.73, 51.63) circle (  2.13);

\path[fill=fillColor,fill opacity=0.20] (183.77, 61.79) circle (  2.13);

\path[fill=fillColor,fill opacity=0.20] (185.08, 62.83) circle (  2.13);

\path[fill=fillColor,fill opacity=0.20] (181.80, 59.47) circle (  2.13);

\path[fill=fillColor,fill opacity=0.20] (206.05, 74.80) circle (  2.13);

\path[fill=fillColor,fill opacity=0.20] (220.04, 54.99) circle (  2.13);

\path[fill=fillColor,fill opacity=0.20] (223.10, 55.94) circle (  2.13);

\path[fill=fillColor,fill opacity=0.20] (224.85, 56.11) circle (  2.13);

\path[fill=fillColor,fill opacity=0.20] (223.97, 52.92) circle (  2.13);

\path[fill=fillColor,fill opacity=0.20] (220.04, 58.69) circle (  2.13);

\path[fill=fillColor,fill opacity=0.20] (211.95, 68.85) circle (  2.13);

\path[fill=fillColor,fill opacity=0.20] (210.64, 73.85) circle (  2.13);

\path[fill=fillColor,fill opacity=0.20] (230.96, 81.51) circle (  2.13);

\path[fill=fillColor,fill opacity=0.20] (254.56, 79.10) circle (  2.13);

\path[fill=fillColor,fill opacity=0.20] (249.10, 69.03) circle (  2.13);

\path[fill=fillColor,fill opacity=0.20] (251.50, 59.90) circle (  2.13);

\path[fill=fillColor,fill opacity=0.20] (203.43, 49.56) circle (  2.13);

\path[fill=fillColor,fill opacity=0.20] (201.25, 49.39) circle (  2.13);

\path[fill=fillColor,fill opacity=0.20] (168.04, 64.03) circle (  2.13);

\path[fill=fillColor,fill opacity=0.20] (192.51, 85.22) circle (  2.13);

\path[fill=fillColor,fill opacity=0.20] (210.21, 53.44) circle (  2.13);

\path[fill=fillColor,fill opacity=0.20] (217.42, 55.25) circle (  2.13);

\path[fill=fillColor,fill opacity=0.20] (228.56, 58.86) circle (  2.13);

\path[fill=fillColor,fill opacity=0.20] (223.75, 54.21) circle (  2.13);

\path[fill=fillColor,fill opacity=0.20] (218.29, 52.49) circle (  2.13);

\path[fill=fillColor,fill opacity=0.20] (218.07, 58.43) circle (  2.13);

\path[fill=fillColor,fill opacity=0.20] (215.01, 68.34) circle (  2.13);

\path[fill=fillColor,fill opacity=0.20] (210.64, 70.83) circle (  2.13);

\path[fill=fillColor,fill opacity=0.20] (232.71, 87.63) circle (  2.13);

\path[fill=fillColor,fill opacity=0.20] (239.70, 85.47) circle (  2.13);

\path[fill=fillColor,fill opacity=0.20] (224.63, 71.18) circle (  2.13);

\path[fill=fillColor,fill opacity=0.20] (211.95, 58.09) circle (  2.13);

\path[fill=fillColor,fill opacity=0.20] (177.65, 51.46) circle (  2.13);

\path[fill=fillColor,fill opacity=0.20] (183.55, 59.30) circle (  2.13);

\path[fill=fillColor,fill opacity=0.20] (199.72, 61.62) circle (  2.13);

\path[fill=fillColor,fill opacity=0.20] (217.63, 62.14) circle (  2.13);

\path[fill=fillColor,fill opacity=0.20] (224.85, 56.88) circle (  2.13);

\path[fill=fillColor,fill opacity=0.20] (222.88, 56.54) circle (  2.13);

\path[fill=fillColor,fill opacity=0.20] (213.92, 58.26) circle (  2.13);

\path[fill=fillColor,fill opacity=0.20] (211.52, 63.69) circle (  2.13);

\path[fill=fillColor,fill opacity=0.20] (210.21, 57.66) circle (  2.13);

\path[fill=fillColor,fill opacity=0.20] (208.02, 53.78) circle (  2.13);

\path[fill=fillColor,fill opacity=0.20] (231.62, 86.85) circle (  2.13);

\path[fill=fillColor,fill opacity=0.20] (233.59, 86.51) circle (  2.13);

\path[fill=fillColor,fill opacity=0.20] (204.31, 90.30) circle (  2.13);

\path[fill=fillColor,fill opacity=0.20] (185.95, 72.73) circle (  2.13);

\path[fill=fillColor,fill opacity=0.20] (170.66, 48.19) circle (  2.13);

\path[fill=fillColor,fill opacity=0.20] (170.66, 57.23) circle (  2.13);

\path[fill=fillColor,fill opacity=0.20] (182.89, 75.66) circle (  2.13);

\path[fill=fillColor,fill opacity=0.20] (202.56, 60.24) circle (  2.13);

\path[fill=fillColor,fill opacity=0.20] (224.85, 58.00) circle (  2.13);

\path[fill=fillColor,fill opacity=0.20] (216.11, 58.78) circle (  2.13);

\path[fill=fillColor,fill opacity=0.20] (216.98, 53.18) circle (  2.13);

\path[fill=fillColor,fill opacity=0.20] (224.63, 54.99) circle (  2.13);

\path[fill=fillColor,fill opacity=0.20] (216.98, 60.24) circle (  2.13);

\path[fill=fillColor,fill opacity=0.20] (217.85, 58.35) circle (  2.13);

\path[fill=fillColor,fill opacity=0.20] (218.07, 73.50) circle (  2.13);

\path[fill=fillColor,fill opacity=0.20] (217.63, 71.52) circle (  2.13);

\path[fill=fillColor,fill opacity=0.20] (245.82, 78.76) circle (  2.13);

\path[fill=fillColor,fill opacity=0.20] (229.22, 81.34) circle (  2.13);

\path[fill=fillColor,fill opacity=0.20] (182.89, 60.59) circle (  2.13);

\path[fill=fillColor,fill opacity=0.20] (208.89, 59.21) circle (  2.13);

\path[fill=fillColor,fill opacity=0.20] (174.37, 76.60) circle (  2.13);

\path[fill=fillColor,fill opacity=0.20] (194.26, 62.14) circle (  2.13);

\path[fill=fillColor,fill opacity=0.20] (209.99, 50.86) circle (  2.13);

\path[fill=fillColor,fill opacity=0.20] (224.85, 45.09) circle (  2.13);

\path[fill=fillColor,fill opacity=0.20] (225.50, 58.09) circle (  2.13);

\path[fill=fillColor,fill opacity=0.20] (223.10, 66.96) circle (  2.13);

\path[fill=fillColor,fill opacity=0.20] (218.07, 63.43) circle (  2.13);

\path[fill=fillColor,fill opacity=0.20] (209.33, 59.81) circle (  2.13);

\path[fill=fillColor,fill opacity=0.20] (213.70, 68.85) circle (  2.13);

\path[fill=fillColor,fill opacity=0.20] (209.55, 82.80) circle (  2.13);

\path[fill=fillColor,fill opacity=0.20] (211.52, 77.03) circle (  2.13);

\path[fill=fillColor,fill opacity=0.20] (216.98, 81.51) circle (  2.13);

\path[fill=fillColor,fill opacity=0.20] (223.53, 74.19) circle (  2.13);

\path[fill=fillColor,fill opacity=0.20] (230.53, 66.61) circle (  2.13);

\path[fill=fillColor,fill opacity=0.20] (215.01, 70.23) circle (  2.13);

\path[fill=fillColor,fill opacity=0.20] (211.52, 59.38) circle (  2.13);

\path[fill=fillColor,fill opacity=0.20] (177.65, 83.06) circle (  2.13);

\path[fill=fillColor,fill opacity=0.20] (188.57, 64.81) circle (  2.13);

\path[fill=fillColor,fill opacity=0.20] (208.89, 58.95) circle (  2.13);

\path[fill=fillColor,fill opacity=0.20] (229.43, 61.36) circle (  2.13);

\path[fill=fillColor,fill opacity=0.20] (229.00, 61.79) circle (  2.13);

\path[fill=fillColor,fill opacity=0.20] (222.00, 62.14) circle (  2.13);

\path[fill=fillColor,fill opacity=0.20] (212.39, 58.35) circle (  2.13);

\path[fill=fillColor,fill opacity=0.20] (209.33, 68.60) circle (  2.13);

\path[fill=fillColor,fill opacity=0.20] (207.58, 77.64) circle (  2.13);

\path[fill=fillColor,fill opacity=0.20] (199.94, 70.15) circle (  2.13);

\path[fill=fillColor,fill opacity=0.20] (197.97, 63.17) circle (  2.13);

\path[fill=fillColor,fill opacity=0.20] (198.63, 71.44) circle (  2.13);

\path[fill=fillColor,fill opacity=0.20] (203.00, 88.66) circle (  2.13);

\path[fill=fillColor,fill opacity=0.20] (206.93, 74.37) circle (  2.13);

\path[fill=fillColor,fill opacity=0.20] (208.24, 67.13) circle (  2.13);

\path[fill=fillColor,fill opacity=0.20] (211.74, 64.46) circle (  2.13);

\path[fill=fillColor,fill opacity=0.20] (216.32, 70.92) circle (  2.13);

\path[fill=fillColor,fill opacity=0.20] (229.00, 72.99) circle (  2.13);

\path[fill=fillColor,fill opacity=0.20] (223.53, 65.84) circle (  2.13);

\path[fill=fillColor,fill opacity=0.20] (197.31, 59.30) circle (  2.13);

\path[fill=fillColor,fill opacity=0.20] (166.51, 56.71) circle (  2.13);

\path[fill=fillColor,fill opacity=0.20] (174.15, 85.82) circle (  2.13);

\path[fill=fillColor,fill opacity=0.20] (189.67, 69.37) circle (  2.13);

\path[fill=fillColor,fill opacity=0.20] (213.26, 59.81) circle (  2.13);

\path[fill=fillColor,fill opacity=0.20] (221.35, 58.18) circle (  2.13);

\path[fill=fillColor,fill opacity=0.20] (221.13, 54.04) circle (  2.13);

\path[fill=fillColor,fill opacity=0.20] (217.63, 62.48) circle (  2.13);

\path[fill=fillColor,fill opacity=0.20] (212.17, 76.86) circle (  2.13);

\path[fill=fillColor,fill opacity=0.20] (206.27, 74.37) circle (  2.13);

\path[fill=fillColor,fill opacity=0.20] (205.84, 65.06) circle (  2.13);

\path[fill=fillColor,fill opacity=0.20] (201.25, 62.40) circle (  2.13);

\path[fill=fillColor,fill opacity=0.20] (195.57, 62.31) circle (  2.13);

\path[fill=fillColor,fill opacity=0.20] (210.64, 70.32) circle (  2.13);

\path[fill=fillColor,fill opacity=0.20] (199.72, 81.77) circle (  2.13);

\path[fill=fillColor,fill opacity=0.20] (199.94, 77.29) circle (  2.13);

\path[fill=fillColor,fill opacity=0.20] (201.68, 64.98) circle (  2.13);

\path[fill=fillColor,fill opacity=0.20] (199.06, 54.99) circle (  2.13);

\path[fill=fillColor,fill opacity=0.20] (200.37, 54.39) circle (  2.13);

\path[fill=fillColor,fill opacity=0.20] (203.43, 60.07) circle (  2.13);

\path[fill=fillColor,fill opacity=0.20] (220.69, 60.16) circle (  2.13);

\path[fill=fillColor,fill opacity=0.20] (224.85, 55.33) circle (  2.13);

\path[fill=fillColor,fill opacity=0.20] (221.35, 62.22) circle (  2.13);

\path[fill=fillColor,fill opacity=0.20] (213.70, 63.34) circle (  2.13);

\path[fill=fillColor,fill opacity=0.20] (197.31, 52.66) circle (  2.13);

\path[fill=fillColor,fill opacity=0.20] (171.09, 52.92) circle (  2.13);

\path[fill=fillColor,fill opacity=0.20] (167.38, 68.34) circle (  2.13);

\path[fill=fillColor,fill opacity=0.20] (175.68, 86.34) circle (  2.13);

\path[fill=fillColor,fill opacity=0.20] (202.56, 69.11) circle (  2.13);

\path[fill=fillColor,fill opacity=0.20] (211.52, 56.54) circle (  2.13);

\path[fill=fillColor,fill opacity=0.20] (215.89, 52.58) circle (  2.13);

\path[fill=fillColor,fill opacity=0.20] (221.57, 57.23) circle (  2.13);

\path[fill=fillColor,fill opacity=0.20] (217.42, 63.08) circle (  2.13);

\path[fill=fillColor,fill opacity=0.20] (219.60, 66.36) circle (  2.13);

\path[fill=fillColor,fill opacity=0.20] (217.85, 64.20) circle (  2.13);

\path[fill=fillColor,fill opacity=0.20] (208.68, 61.45) circle (  2.13);

\path[fill=fillColor,fill opacity=0.20] (206.05, 66.96) circle (  2.13);

\path[fill=fillColor,fill opacity=0.20] (205.40, 69.20) circle (  2.13);

\path[fill=fillColor,fill opacity=0.20] (201.47, 62.65) circle (  2.13);

\path[fill=fillColor,fill opacity=0.20] (201.68, 57.92) circle (  2.13);

\path[fill=fillColor,fill opacity=0.20] (201.25, 56.71) circle (  2.13);

\path[fill=fillColor,fill opacity=0.20] (203.00, 56.02) circle (  2.13);

\path[fill=fillColor,fill opacity=0.20] (198.63, 61.02) circle (  2.13);

\path[fill=fillColor,fill opacity=0.20] (202.56, 58.95) circle (  2.13);

\path[fill=fillColor,fill opacity=0.20] (203.65, 53.44) circle (  2.13);

\path[fill=fillColor,fill opacity=0.20] (203.00, 58.09) circle (  2.13);

\path[fill=fillColor,fill opacity=0.20] (196.22, 63.08) circle (  2.13);

\path[fill=fillColor,fill opacity=0.20] (199.50, 67.30) circle (  2.13);

\path[fill=fillColor,fill opacity=0.20] (201.25, 69.71) circle (  2.13);

\path[fill=fillColor,fill opacity=0.20] (198.41, 63.26) circle (  2.13);

\path[fill=fillColor,fill opacity=0.20] (200.81, 58.35) circle (  2.13);

\path[fill=fillColor,fill opacity=0.20] (203.21, 61.19) circle (  2.13);

\path[fill=fillColor,fill opacity=0.20] (203.87, 63.17) circle (  2.13);

\path[fill=fillColor,fill opacity=0.20] (208.68, 65.41) circle (  2.13);

\path[fill=fillColor,fill opacity=0.20] (214.36, 61.71) circle (  2.13);

\path[fill=fillColor,fill opacity=0.20] (216.98, 60.93) circle (  2.13);

\path[fill=fillColor,fill opacity=0.20] (219.38, 66.96) circle (  2.13);

\path[fill=fillColor,fill opacity=0.20] (217.63, 63.17) circle (  2.13);

\path[fill=fillColor,fill opacity=0.20] (210.42, 53.35) circle (  2.13);

\path[fill=fillColor,fill opacity=0.20] (192.73, 52.15) circle (  2.13);

\path[fill=fillColor,fill opacity=0.20] (175.46, 51.37) circle (  2.13);

\path[fill=fillColor,fill opacity=0.20] (215.89, 60.59) circle (  2.13);

\path[fill=fillColor,fill opacity=0.20] (190.76, 75.31) circle (  2.13);

\path[fill=fillColor,fill opacity=0.20] (186.39, 64.46) circle (  2.13);

\path[fill=fillColor,fill opacity=0.20] (196.00, 48.44) circle (  2.13);

\path[fill=fillColor,fill opacity=0.20] (211.52, 49.91) circle (  2.13);

\path[fill=fillColor,fill opacity=0.20] (215.45, 65.93) circle (  2.13);

\path[fill=fillColor,fill opacity=0.20] (217.20, 69.28) circle (  2.13);

\path[fill=fillColor,fill opacity=0.20] (219.38, 61.36) circle (  2.13);

\path[fill=fillColor,fill opacity=0.20] (218.95, 62.74) circle (  2.13);

\path[fill=fillColor,fill opacity=0.20] (215.01, 69.97) circle (  2.13);

\path[fill=fillColor,fill opacity=0.20] (220.04, 72.30) circle (  2.13);

\path[fill=fillColor,fill opacity=0.20] (218.95, 67.91) circle (  2.13);

\path[fill=fillColor,fill opacity=0.20] (215.45, 59.12) circle (  2.13);

\path[fill=fillColor,fill opacity=0.20] (211.08, 55.85) circle (  2.13);

\path[fill=fillColor,fill opacity=0.20] (214.14, 59.64) circle (  2.13);

\path[fill=fillColor,fill opacity=0.20] (213.05, 60.50) circle (  2.13);

\path[fill=fillColor,fill opacity=0.20] (211.95, 58.09) circle (  2.13);

\path[fill=fillColor,fill opacity=0.20] (212.39, 57.66) circle (  2.13);

\path[fill=fillColor,fill opacity=0.20] (208.24, 58.35) circle (  2.13);

\path[fill=fillColor,fill opacity=0.20] (208.68, 60.67) circle (  2.13);

\path[fill=fillColor,fill opacity=0.20] (216.32, 62.22) circle (  2.13);

\path[fill=fillColor,fill opacity=0.20] (213.70, 63.60) circle (  2.13);

\path[fill=fillColor,fill opacity=0.20] (217.42, 63.69) circle (  2.13);

\path[fill=fillColor,fill opacity=0.20] (218.29, 63.51) circle (  2.13);

\path[fill=fillColor,fill opacity=0.20] (221.57, 67.82) circle (  2.13);

\path[fill=fillColor,fill opacity=0.20] (231.40, 69.89) circle (  2.13);

\path[fill=fillColor,fill opacity=0.20] (225.06, 60.33) circle (  2.13);

\path[fill=fillColor,fill opacity=0.20] (212.61, 55.51) circle (  2.13);

\path[fill=fillColor,fill opacity=0.20] (211.74, 60.07) circle (  2.13);

\path[fill=fillColor,fill opacity=0.20] (179.83, 55.25) circle (  2.13);

\path[fill=fillColor,fill opacity=0.20] (169.56, 55.42) circle (  2.13);

\path[fill=fillColor,fill opacity=0.20] (179.40, 71.61) circle (  2.13);

\path[fill=fillColor,fill opacity=0.20] (166.07, 76.00) circle (  2.13);

\path[fill=fillColor,fill opacity=0.20] (175.68, 69.46) circle (  2.13);

\path[fill=fillColor,fill opacity=0.20] (208.46, 75.66) circle (  2.13);

\path[fill=fillColor,fill opacity=0.20] (196.88, 78.76) circle (  2.13);

\path[fill=fillColor,fill opacity=0.20] (192.29, 67.56) circle (  2.13);

\path[fill=fillColor,fill opacity=0.20] (198.84, 56.80) circle (  2.13);

\path[fill=fillColor,fill opacity=0.20] (205.84, 61.45) circle (  2.13);

\path[fill=fillColor,fill opacity=0.20] (224.85, 66.96) circle (  2.13);

\path[fill=fillColor,fill opacity=0.20] (217.85, 61.71) circle (  2.13);

\path[fill=fillColor,fill opacity=0.20] (216.54, 52.92) circle (  2.13);

\path[fill=fillColor,fill opacity=0.20] (215.23, 50.08) circle (  2.13);

\path[fill=fillColor,fill opacity=0.20] (221.57, 52.92) circle (  2.13);

\path[fill=fillColor,fill opacity=0.20] (213.48, 58.43) circle (  2.13);

\path[fill=fillColor,fill opacity=0.20] (214.14, 62.05) circle (  2.13);

\path[fill=fillColor,fill opacity=0.20] (206.27, 61.88) circle (  2.13);

\path[fill=fillColor,fill opacity=0.20] (203.43, 61.19) circle (  2.13);

\path[fill=fillColor,fill opacity=0.20] (192.73, 61.10) circle (  2.13);

\path[fill=fillColor,fill opacity=0.20] (198.41, 56.63) circle (  2.13);

\path[fill=fillColor,fill opacity=0.20] (202.34, 51.11) circle (  2.13);

\path[fill=fillColor,fill opacity=0.20] (197.31, 51.63) circle (  2.13);

\path[fill=fillColor,fill opacity=0.20] (212.83, 55.59) circle (  2.13);

\path[fill=fillColor,fill opacity=0.20] (193.38, 53.61) circle (  2.13);

\path[fill=fillColor,fill opacity=0.20] (192.07, 48.88) circle (  2.13);

\path[fill=fillColor,fill opacity=0.20] (192.94, 48.53) circle (  2.13);

\path[fill=fillColor,fill opacity=0.20] (174.81, 49.65) circle (  2.13);

\path[fill=fillColor,fill opacity=0.20] (168.47, 56.28) circle (  2.13);

\path[fill=fillColor,fill opacity=0.20] (165.19, 76.26) circle (  2.13);

\path[fill=fillColor,fill opacity=0.20] (166.94, 67.91) circle (  2.13);

\path[fill=fillColor,fill opacity=0.20] (172.41, 59.73) circle (  2.13);

\path[fill=fillColor,fill opacity=0.20] (169.35, 53.44) circle (  2.13);

\path[fill=fillColor,fill opacity=0.20] (169.35, 52.32) circle (  2.13);

\path[fill=fillColor,fill opacity=0.20] (177.21, 51.29) circle (  2.13);

\path[fill=fillColor,fill opacity=0.20] (172.41, 51.37) circle (  2.13);

\path[fill=fillColor,fill opacity=0.20] (173.50, 56.97) circle (  2.13);

\path[fill=fillColor,fill opacity=0.20] (169.78, 63.34) circle (  2.13);

\path[fill=fillColor,fill opacity=0.20] (175.25, 68.51) circle (  2.13);

\path[fill=fillColor,fill opacity=0.20] (177.65, 71.44) circle (  2.13);

\path[fill=fillColor,fill opacity=0.20] (170.00, 76.09) circle (  2.13);

\path[fill=fillColor,fill opacity=0.20] (165.19, 79.45) circle (  2.13);

\path[fill=fillColor,fill opacity=0.20] (233.59, 69.37) circle (  2.13);

\path[fill=fillColor,fill opacity=0.20] (183.77, 59.90) circle (  2.13);

\path[fill=fillColor,fill opacity=0.20] (172.62, 68.77) circle (  2.13);

\path[fill=fillColor,fill opacity=0.20] (204.09, 86.25) circle (  2.13);

\path[fill=fillColor,fill opacity=0.20] (215.01, 93.31) circle (  2.13);

\path[fill=fillColor,fill opacity=0.20] (230.74, 87.63) circle (  2.13);

\path[fill=fillColor,fill opacity=0.20] (224.85, 86.94) circle (  2.13);

\path[fill=fillColor,fill opacity=0.20] (215.89, 86.59) circle (  2.13);

\path[fill=fillColor,fill opacity=0.20] (206.05, 80.48) circle (  2.13);

\path[fill=fillColor,fill opacity=0.20] (202.56, 72.82) circle (  2.13);

\path[fill=fillColor,fill opacity=0.20] (201.90, 72.30) circle (  2.13);

\path[fill=fillColor,fill opacity=0.20] (202.78, 78.50) circle (  2.13);

\path[fill=fillColor,fill opacity=0.20] (196.66, 86.08) circle (  2.13);

\path[fill=fillColor,fill opacity=0.20] (191.85, 90.30) circle (  2.13);

\path[fill=fillColor,fill opacity=0.20] (191.63, 93.14) circle (  2.13);

\path[fill=fillColor,fill opacity=0.20] (223.10, 86.16) circle (  2.13);

\path[fill=fillColor,fill opacity=0.20] (212.39, 71.01) circle (  2.13);

\path[fill=fillColor,fill opacity=0.20] (258.71, 74.88) circle (  2.13);

\path[fill=fillColor,fill opacity=0.20] (216.98, 82.29) circle (  2.13);

\path[fill=fillColor,fill opacity=0.20] (211.74, 80.82) circle (  2.13);

\path[fill=fillColor,fill opacity=0.20] (204.52, 74.80) circle (  2.13);

\path[fill=fillColor,fill opacity=0.20] (199.06, 70.58) circle (  2.13);

\path[fill=fillColor,fill opacity=0.20] (198.41, 67.99) circle (  2.13);

\path[fill=fillColor,fill opacity=0.20] (196.22, 69.71) circle (  2.13);

\path[fill=fillColor,fill opacity=0.20] (189.23, 81.77) circle (  2.13);

\path[fill=fillColor,fill opacity=0.20] (180.71, 90.81) circle (  2.13);

\path[fill=fillColor,fill opacity=0.20] (174.37, 91.33) circle (  2.13);

\path[fill=fillColor,fill opacity=0.20] (214.58, 98.74) circle (  2.13);

\path[fill=fillColor,fill opacity=0.20] (218.29, 80.05) circle (  2.13);

\path[fill=fillColor,fill opacity=0.20] (209.33, 66.44) circle (  2.13);

\path[fill=fillColor,fill opacity=0.20] (211.08, 66.70) circle (  2.13);

\path[fill=fillColor,fill opacity=0.20] (211.08, 71.18) circle (  2.13);

\path[fill=fillColor,fill opacity=0.20] (207.58, 69.80) circle (  2.13);

\path[fill=fillColor,fill opacity=0.20] (204.52, 68.42) circle (  2.13);

\path[fill=fillColor,fill opacity=0.20] (198.19, 71.61) circle (  2.13);

\path[fill=fillColor,fill opacity=0.20] (197.75, 72.56) circle (  2.13);

\path[fill=fillColor,fill opacity=0.20] (192.73, 73.68) circle (  2.13);

\path[fill=fillColor,fill opacity=0.20] (180.05, 82.72) circle (  2.13);

\path[fill=fillColor,fill opacity=0.20] (213.70, 78.41) circle (  2.13);

\path[fill=fillColor,fill opacity=0.20] (252.16, 77.64) circle (  2.13);

\path[fill=fillColor,fill opacity=0.20] (214.79, 96.24) circle (  2.13);

\path[fill=fillColor,fill opacity=0.20] (213.26, 79.62) circle (  2.13);

\path[fill=fillColor,fill opacity=0.20] (208.68, 69.89) circle (  2.13);

\path[fill=fillColor,fill opacity=0.20] (205.84, 64.63) circle (  2.13);

\path[fill=fillColor,fill opacity=0.20] (205.18, 63.43) circle (  2.13);

\path[fill=fillColor,fill opacity=0.20] (203.65, 63.17) circle (  2.13);

\path[fill=fillColor,fill opacity=0.20] (201.47, 64.38) circle (  2.13);

\path[fill=fillColor,fill opacity=0.20] (197.53, 71.78) circle (  2.13);

\path[fill=fillColor,fill opacity=0.20] (191.20, 77.81) circle (  2.13);

\path[fill=fillColor,fill opacity=0.20] (183.11, 78.93) circle (  2.13);

\path[fill=fillColor,fill opacity=0.20] (168.25, 84.96) circle (  2.13);

\path[fill=fillColor,fill opacity=0.20] (199.06, 67.05) circle (  2.13);

\path[fill=fillColor,fill opacity=0.20] (203.65, 66.79) circle (  2.13);

\path[fill=fillColor,fill opacity=0.20] (212.39, 63.43) circle (  2.13);

\path[fill=fillColor,fill opacity=0.20] (193.60,106.14) circle (  2.13);

\path[fill=fillColor,fill opacity=0.20] (215.23, 86.77) circle (  2.13);

\path[fill=fillColor,fill opacity=0.20] (215.67, 73.25) circle (  2.13);

\path[fill=fillColor,fill opacity=0.20] (206.71, 69.89) circle (  2.13);

\path[fill=fillColor,fill opacity=0.20] (204.96, 64.98) circle (  2.13);

\path[fill=fillColor,fill opacity=0.20] (202.34, 62.31) circle (  2.13);

\path[fill=fillColor,fill opacity=0.20] (200.15, 65.50) circle (  2.13);

\path[fill=fillColor,fill opacity=0.20] (199.94, 67.56) circle (  2.13);

\path[fill=fillColor,fill opacity=0.20] (197.10, 72.38) circle (  2.13);

\path[fill=fillColor,fill opacity=0.20] (187.26, 79.62) circle (  2.13);

\path[fill=fillColor,fill opacity=0.20] (205.62, 66.10) circle (  2.13);

\path[fill=fillColor,fill opacity=0.20] (205.40, 61.79) circle (  2.13);

\path[fill=fillColor,fill opacity=0.20] (214.36, 63.69) circle (  2.13);

\path[fill=fillColor,fill opacity=0.20] (224.19, 56.63) circle (  2.13);

\path[fill=fillColor,fill opacity=0.20] (225.06, 53.87) circle (  2.13);

\path[fill=fillColor,fill opacity=0.20] (219.60, 55.33) circle (  2.13);

\path[fill=fillColor,fill opacity=0.20] (213.05, 61.02) circle (  2.13);

\path[fill=fillColor,fill opacity=0.20] (207.37, 69.28) circle (  2.13);

\path[fill=fillColor,fill opacity=0.20] (193.82,101.15) circle (  2.13);

\path[fill=fillColor,fill opacity=0.20] (218.07, 77.29) circle (  2.13);

\path[fill=fillColor,fill opacity=0.20] (212.83, 66.27) circle (  2.13);

\path[fill=fillColor,fill opacity=0.20] (206.27, 68.08) circle (  2.13);

\path[fill=fillColor,fill opacity=0.20] (205.18, 63.60) circle (  2.13);

\path[fill=fillColor,fill opacity=0.20] (200.15, 60.67) circle (  2.13);

\path[fill=fillColor,fill opacity=0.20] (197.10, 66.53) circle (  2.13);

\path[fill=fillColor,fill opacity=0.20] (197.97, 71.35) circle (  2.13);

\path[fill=fillColor,fill opacity=0.20] (194.91, 75.48) circle (  2.13);

\path[fill=fillColor,fill opacity=0.20] (208.89, 63.60) circle (  2.13);

\path[fill=fillColor,fill opacity=0.20] (218.29, 66.36) circle (  2.13);

\path[fill=fillColor,fill opacity=0.20] (222.22, 63.43) circle (  2.13);

\path[fill=fillColor,fill opacity=0.20] (219.38, 54.82) circle (  2.13);

\path[fill=fillColor,fill opacity=0.20] (234.46, 48.62) circle (  2.13);

\path[fill=fillColor,fill opacity=0.20] (229.87, 46.98) circle (  2.13);

\path[fill=fillColor,fill opacity=0.20] (219.60, 53.09) circle (  2.13);

\path[fill=fillColor,fill opacity=0.20] (215.89, 60.24) circle (  2.13);

\path[fill=fillColor,fill opacity=0.20] (210.64, 63.51) circle (  2.13);

\path[fill=fillColor,fill opacity=0.20] (227.90, 67.48) circle (  2.13);

\path[fill=fillColor,fill opacity=0.20] (191.41,104.76) circle (  2.13);

\path[fill=fillColor,fill opacity=0.20] (217.85, 83.15) circle (  2.13);

\path[fill=fillColor,fill opacity=0.20] (207.80, 66.10) circle (  2.13);

\path[fill=fillColor,fill opacity=0.20] (202.12, 66.96) circle (  2.13);

\path[fill=fillColor,fill opacity=0.20] (203.00, 67.56) circle (  2.13);

\path[fill=fillColor,fill opacity=0.20] (200.81, 62.74) circle (  2.13);

\path[fill=fillColor,fill opacity=0.20] (198.63, 62.65) circle (  2.13);

\path[fill=fillColor,fill opacity=0.20] (197.75, 69.97) circle (  2.13);

\path[fill=fillColor,fill opacity=0.20] (195.13, 80.39) circle (  2.13);

\path[fill=fillColor,fill opacity=0.20] (206.71, 84.78) circle (  2.13);

\path[fill=fillColor,fill opacity=0.20] (211.30, 49.56) circle (  2.13);

\path[fill=fillColor,fill opacity=0.20] (211.95, 58.35) circle (  2.13);

\path[fill=fillColor,fill opacity=0.20] (219.60, 58.52) circle (  2.13);

\path[fill=fillColor,fill opacity=0.20] (234.68, 48.96) circle (  2.13);

\path[fill=fillColor,fill opacity=0.20] (238.61, 44.40) circle (  2.13);

\path[fill=fillColor,fill opacity=0.20] (237.52, 46.55) circle (  2.13);

\path[fill=fillColor,fill opacity=0.20] (230.09, 51.11) circle (  2.13);

\path[fill=fillColor,fill opacity=0.20] (224.41, 48.88) circle (  2.13);

\path[fill=fillColor,fill opacity=0.20] (213.05, 46.81) circle (  2.13);

\path[fill=fillColor,fill opacity=0.20] (201.68, 60.67) circle (  2.13);

\path[fill=fillColor,fill opacity=0.20] (196.22, 77.64) circle (  2.13);

\path[fill=fillColor,fill opacity=0.20] (212.83, 88.14) circle (  2.13);

\path[fill=fillColor,fill opacity=0.20] (215.23, 64.98) circle (  2.13);

\path[fill=fillColor,fill opacity=0.20] (203.21, 64.98) circle (  2.13);

\path[fill=fillColor,fill opacity=0.20] (203.43, 73.59) circle (  2.13);

\path[fill=fillColor,fill opacity=0.20] (210.42, 67.65) circle (  2.13);

\path[fill=fillColor,fill opacity=0.20] (203.65, 57.49) circle (  2.13);

\path[fill=fillColor,fill opacity=0.20] (196.00, 62.91) circle (  2.13);

\path[fill=fillColor,fill opacity=0.20] (194.04, 77.81) circle (  2.13);

\path[fill=fillColor,fill opacity=0.20] (194.91, 83.75) circle (  2.13);

\path[fill=fillColor,fill opacity=0.20] (188.57, 84.18) circle (  2.13);

\path[fill=fillColor,fill opacity=0.20] (219.16, 64.98) circle (  2.13);

\path[fill=fillColor,fill opacity=0.20] (217.42, 49.39) circle (  2.13);

\path[fill=fillColor,fill opacity=0.20] (221.35, 47.33) circle (  2.13);

\path[fill=fillColor,fill opacity=0.20] (227.03, 39.32) circle (  2.13);

\path[fill=fillColor,fill opacity=0.20] (229.22, 44.31) circle (  2.13);

\path[fill=fillColor,fill opacity=0.20] (234.02, 49.91) circle (  2.13);

\path[fill=fillColor,fill opacity=0.20] (239.27, 49.91) circle (  2.13);

\path[fill=fillColor,fill opacity=0.20] (237.96, 44.83) circle (  2.13);

\path[fill=fillColor,fill opacity=0.20] (226.16, 38.20) circle (  2.13);

\path[fill=fillColor,fill opacity=0.20] (218.07, 51.11) circle (  2.13);

\path[fill=fillColor,fill opacity=0.20] (196.00, 70.23) circle (  2.13);

\path[fill=fillColor,fill opacity=0.20] (196.00, 67.48) circle (  2.13);

\path[fill=fillColor,fill opacity=0.20] (208.46, 81.68) circle (  2.13);

\path[fill=fillColor,fill opacity=0.20] (222.44, 60.16) circle (  2.13);

\path[fill=fillColor,fill opacity=0.20] (211.74, 61.45) circle (  2.13);

\path[fill=fillColor,fill opacity=0.20] (207.15, 69.80) circle (  2.13);

\path[fill=fillColor,fill opacity=0.20] (205.40, 63.08) circle (  2.13);

\path[fill=fillColor,fill opacity=0.20] (198.84, 54.04) circle (  2.13);

\path[fill=fillColor,fill opacity=0.20] (192.94, 59.04) circle (  2.13);

\path[fill=fillColor,fill opacity=0.20] (194.91, 68.68) circle (  2.13);

\path[fill=fillColor,fill opacity=0.20] (196.44, 72.30) circle (  2.13);

\path[fill=fillColor,fill opacity=0.20] (193.82, 72.64) circle (  2.13);

\path[fill=fillColor,fill opacity=0.20] (225.28, 63.08) circle (  2.13);

\path[fill=fillColor,fill opacity=0.20] (228.56, 43.62) circle (  2.13);

\path[fill=fillColor,fill opacity=0.20] (218.07, 51.20) circle (  2.13);

\path[fill=fillColor,fill opacity=0.20] (222.88, 42.42) circle (  2.13);

\path[fill=fillColor,fill opacity=0.20] (225.72, 40.78) circle (  2.13);

\path[fill=fillColor,fill opacity=0.20] (226.16, 48.88) circle (  2.13);

\path[fill=fillColor,fill opacity=0.20] (230.09, 49.31) circle (  2.13);

\path[fill=fillColor,fill opacity=0.20] (235.99, 45.26) circle (  2.13);

\path[fill=fillColor,fill opacity=0.20] (238.39, 44.48) circle (  2.13);

\path[fill=fillColor,fill opacity=0.20] (234.68, 40.18) circle (  2.13);

\path[fill=fillColor,fill opacity=0.20] (224.41, 42.59) circle (  2.13);

\path[fill=fillColor,fill opacity=0.20] (204.74, 56.11) circle (  2.13);

\path[fill=fillColor,fill opacity=0.20] (196.00, 64.55) circle (  2.13);

\path[fill=fillColor,fill opacity=0.20] (196.00, 84.10) circle (  2.13);

\path[fill=fillColor,fill opacity=0.20] (219.16, 62.57) circle (  2.13);

\path[fill=fillColor,fill opacity=0.20] (214.14, 60.33) circle (  2.13);

\path[fill=fillColor,fill opacity=0.20] (205.84, 63.08) circle (  2.13);

\path[fill=fillColor,fill opacity=0.20] (204.31, 56.28) circle (  2.13);

\path[fill=fillColor,fill opacity=0.20] (200.15, 54.13) circle (  2.13);

\path[fill=fillColor,fill opacity=0.20] (199.28, 61.62) circle (  2.13);

\path[fill=fillColor,fill opacity=0.20] (197.53, 66.96) circle (  2.13);

\path[fill=fillColor,fill opacity=0.20] (200.15, 66.70) circle (  2.13);

\path[fill=fillColor,fill opacity=0.20] (198.84, 63.08) circle (  2.13);

\path[fill=fillColor,fill opacity=0.20] (194.04, 66.27) circle (  2.13);

\path[fill=fillColor,fill opacity=0.20] (222.88, 71.44) circle (  2.13);

\path[fill=fillColor,fill opacity=0.20] (231.62, 53.44) circle (  2.13);

\path[fill=fillColor,fill opacity=0.20] (226.16, 55.59) circle (  2.13);

\path[fill=fillColor,fill opacity=0.20] (224.41, 49.48) circle (  2.13);

\path[fill=fillColor,fill opacity=0.20] (220.04, 53.61) circle (  2.13);

\path[fill=fillColor,fill opacity=0.20] (226.16, 51.63) circle (  2.13);

\path[fill=fillColor,fill opacity=0.20] (232.06, 43.45) circle (  2.13);

\path[fill=fillColor,fill opacity=0.20] (234.02, 46.03) circle (  2.13);

\path[fill=fillColor,fill opacity=0.20] (234.02, 47.50) circle (  2.13);

\path[fill=fillColor,fill opacity=0.20] (238.17, 40.09) circle (  2.13);

\path[fill=fillColor,fill opacity=0.20] (215.23, 43.79) circle (  2.13);

\path[fill=fillColor,fill opacity=0.20] (196.00, 57.31) circle (  2.13);

\path[fill=fillColor,fill opacity=0.20] (177.87,101.32) circle (  2.13);

\path[fill=fillColor,fill opacity=0.20] (203.65, 74.45) circle (  2.13);

\path[fill=fillColor,fill opacity=0.20] (209.99, 63.43) circle (  2.13);

\path[fill=fillColor,fill opacity=0.20] (206.71, 63.95) circle (  2.13);

\path[fill=fillColor,fill opacity=0.20] (205.40, 60.33) circle (  2.13);

\path[fill=fillColor,fill opacity=0.20] (207.15, 56.37) circle (  2.13);

\path[fill=fillColor,fill opacity=0.20] (204.52, 61.19) circle (  2.13);

\path[fill=fillColor,fill opacity=0.20] (203.43, 70.06) circle (  2.13);

\path[fill=fillColor,fill opacity=0.20] (199.28, 71.44) circle (  2.13);

\path[fill=fillColor,fill opacity=0.20] (201.90, 60.93) circle (  2.13);

\path[fill=fillColor,fill opacity=0.20] (198.84, 54.82) circle (  2.13);

\path[fill=fillColor,fill opacity=0.20] (192.73, 66.96) circle (  2.13);

\path[fill=fillColor,fill opacity=0.20] (186.83, 88.23) circle (  2.13);

\path[fill=fillColor,fill opacity=0.20] (219.16, 58.43) circle (  2.13);

\path[fill=fillColor,fill opacity=0.20] (227.03, 57.23) circle (  2.13);

\path[fill=fillColor,fill opacity=0.20] (225.50, 54.30) circle (  2.13);

\path[fill=fillColor,fill opacity=0.20] (223.75, 60.76) circle (  2.13);

\path[fill=fillColor,fill opacity=0.20] (232.71, 53.18) circle (  2.13);

\path[fill=fillColor,fill opacity=0.20] (236.21, 41.13) circle (  2.13);

\path[fill=fillColor,fill opacity=0.20] (237.08, 51.72) circle (  2.13);

\path[fill=fillColor,fill opacity=0.20] (235.33, 53.61) circle (  2.13);

\path[fill=fillColor,fill opacity=0.20] (236.21, 39.40) circle (  2.13);

\path[fill=fillColor,fill opacity=0.20] (215.23, 43.36) circle (  2.13);

\path[fill=fillColor,fill opacity=0.20] (199.50, 49.39) circle (  2.13);

\path[fill=fillColor,fill opacity=0.20] (181.58, 76.00) circle (  2.13);

\path[fill=fillColor,fill opacity=0.20] (179.18, 89.69) circle (  2.13);

\path[fill=fillColor,fill opacity=0.20] (203.43, 69.89) circle (  2.13);

\path[fill=fillColor,fill opacity=0.20] (208.89, 63.08) circle (  2.13);

\path[fill=fillColor,fill opacity=0.20] (209.77, 61.88) circle (  2.13);

\path[fill=fillColor,fill opacity=0.20] (204.52, 59.30) circle (  2.13);

\path[fill=fillColor,fill opacity=0.20] (206.05, 58.52) circle (  2.13);

\path[fill=fillColor,fill opacity=0.20] (205.18, 65.06) circle (  2.13);

\path[fill=fillColor,fill opacity=0.20] (203.87, 69.28) circle (  2.13);

\path[fill=fillColor,fill opacity=0.20] (206.71, 62.57) circle (  2.13);

\path[fill=fillColor,fill opacity=0.20] (202.34, 57.49) circle (  2.13);

\path[fill=fillColor,fill opacity=0.20] (194.47, 65.84) circle (  2.13);

\path[fill=fillColor,fill opacity=0.20] (194.69, 76.00) circle (  2.13);

\path[fill=fillColor,fill opacity=0.20] (190.54, 83.92) circle (  2.13);

\path[fill=fillColor,fill opacity=0.20] (204.52, 73.50) circle (  2.13);

\path[fill=fillColor,fill opacity=0.20] (217.20, 59.73) circle (  2.13);

\path[fill=fillColor,fill opacity=0.20] (222.22, 57.92) circle (  2.13);

\path[fill=fillColor,fill opacity=0.20] (228.12, 53.61) circle (  2.13);

\path[fill=fillColor,fill opacity=0.20] (229.00, 55.94) circle (  2.13);

\path[fill=fillColor,fill opacity=0.20] (233.15, 54.04) circle (  2.13);

\path[fill=fillColor,fill opacity=0.20] (231.40, 47.58) circle (  2.13);

\path[fill=fillColor,fill opacity=0.20] (234.02, 53.18) circle (  2.13);

\path[fill=fillColor,fill opacity=0.20] (232.49, 52.41) circle (  2.13);

\path[fill=fillColor,fill opacity=0.20] (230.74, 43.02) circle (  2.13);

\path[fill=fillColor,fill opacity=0.20] (232.27, 46.55) circle (  2.13);

\path[fill=fillColor,fill opacity=0.20] (222.88, 48.10) circle (  2.13);

\path[fill=fillColor,fill opacity=0.20] (199.28, 45.52) circle (  2.13);

\path[fill=fillColor,fill opacity=0.20] (197.31, 77.03) circle (  2.13);

\path[fill=fillColor,fill opacity=0.20] (183.99, 80.82) circle (  2.13);

\path[fill=fillColor,fill opacity=0.20] (203.21, 59.30) circle (  2.13);

\path[fill=fillColor,fill opacity=0.20] (204.74, 54.90) circle (  2.13);

\path[fill=fillColor,fill opacity=0.20] (205.84, 59.21) circle (  2.13);

\path[fill=fillColor,fill opacity=0.20] (208.68, 59.04) circle (  2.13);

\path[fill=fillColor,fill opacity=0.20] (205.18, 58.00) circle (  2.13);

\path[fill=fillColor,fill opacity=0.20] (224.41, 60.50) circle (  2.13);

\path[fill=fillColor,fill opacity=0.20] (203.87, 61.79) circle (  2.13);

\path[fill=fillColor,fill opacity=0.20] (200.37, 64.03) circle (  2.13);

\path[fill=fillColor,fill opacity=0.20] (198.63, 68.60) circle (  2.13);

\path[fill=fillColor,fill opacity=0.20] (198.63, 72.82) circle (  2.13);

\path[fill=fillColor,fill opacity=0.20] (196.44, 78.33) circle (  2.13);

\path[fill=fillColor,fill opacity=0.20] (194.26, 77.98) circle (  2.13);

\path[fill=fillColor,fill opacity=0.20] (191.41, 79.88) circle (  2.13);

\path[fill=fillColor,fill opacity=0.20] (196.88, 84.61) circle (  2.13);

\path[fill=fillColor,fill opacity=0.20] (211.95, 66.96) circle (  2.13);

\path[fill=fillColor,fill opacity=0.20] (217.63, 58.61) circle (  2.13);

\path[fill=fillColor,fill opacity=0.20] (219.82, 58.95) circle (  2.13);

\path[fill=fillColor,fill opacity=0.20] (220.48, 53.53) circle (  2.13);

\path[fill=fillColor,fill opacity=0.20] (226.59, 46.89) circle (  2.13);

\path[fill=fillColor,fill opacity=0.20] (239.92, 46.98) circle (  2.13);

\path[fill=fillColor,fill opacity=0.20] (228.12, 51.20) circle (  2.13);

\path[fill=fillColor,fill opacity=0.20] (231.84, 51.63) circle (  2.13);

\path[fill=fillColor,fill opacity=0.20] (233.59, 44.83) circle (  2.13);

\path[fill=fillColor,fill opacity=0.20] (224.19, 48.19) circle (  2.13);

\path[fill=fillColor,fill opacity=0.20] (232.49, 56.71) circle (  2.13);

\path[fill=fillColor,fill opacity=0.20] (236.64, 49.31) circle (  2.13);

\path[fill=fillColor,fill opacity=0.20] (200.15, 48.19) circle (  2.13);

\path[fill=fillColor,fill opacity=0.20] (218.73, 79.53) circle (  2.13);

\path[fill=fillColor,fill opacity=0.20] (173.72, 74.71) circle (  2.13);

\path[fill=fillColor,fill opacity=0.20] (186.61, 62.74) circle (  2.13);

\path[fill=fillColor,fill opacity=0.20] (199.28, 59.73) circle (  2.13);

\path[fill=fillColor,fill opacity=0.20] (205.62, 57.83) circle (  2.13);

\path[fill=fillColor,fill opacity=0.20] (205.62, 57.14) circle (  2.13);

\path[fill=fillColor,fill opacity=0.20] (203.00, 56.71) circle (  2.13);

\path[fill=fillColor,fill opacity=0.20] (208.89, 57.75) circle (  2.13);

\path[fill=fillColor,fill opacity=0.20] (199.50, 65.75) circle (  2.13);

\path[fill=fillColor,fill opacity=0.20] (198.84, 70.15) circle (  2.13);

\path[fill=fillColor,fill opacity=0.20] (197.75, 69.89) circle (  2.13);

\path[fill=fillColor,fill opacity=0.20] (195.35, 76.52) circle (  2.13);

\path[fill=fillColor,fill opacity=0.20] (195.13, 75.66) circle (  2.13);

\path[fill=fillColor,fill opacity=0.20] (196.00, 70.49) circle (  2.13);

\path[fill=fillColor,fill opacity=0.20] (193.60, 78.58) circle (  2.13);

\path[fill=fillColor,fill opacity=0.20] (189.01, 89.35) circle (  2.13);

\path[fill=fillColor,fill opacity=0.20] (193.16, 79.62) circle (  2.13);

\path[fill=fillColor,fill opacity=0.20] (192.51, 78.84) circle (  2.13);

\path[fill=fillColor,fill opacity=0.20] (197.53, 76.43) circle (  2.13);

\path[fill=fillColor,fill opacity=0.20] (205.84, 76.86) circle (  2.13);

\path[fill=fillColor,fill opacity=0.20] (214.79, 72.30) circle (  2.13);

\path[fill=fillColor,fill opacity=0.20] (217.42, 64.03) circle (  2.13);

\path[fill=fillColor,fill opacity=0.20] (216.11, 59.12) circle (  2.13);

\path[fill=fillColor,fill opacity=0.20] (214.58, 58.09) circle (  2.13);

\path[fill=fillColor,fill opacity=0.20] (217.42, 54.82) circle (  2.13);

\path[fill=fillColor,fill opacity=0.20] (228.78, 47.07) circle (  2.13);

\path[fill=fillColor,fill opacity=0.20] (226.37, 42.33) circle (  2.13);

\path[fill=fillColor,fill opacity=0.20] (232.93, 47.41) circle (  2.13);

\path[fill=fillColor,fill opacity=0.20] (230.96, 50.25) circle (  2.13);

\path[fill=fillColor,fill opacity=0.20] (229.65, 47.24) circle (  2.13);

\path[fill=fillColor,fill opacity=0.20] (223.53, 52.66) circle (  2.13);

\path[fill=fillColor,fill opacity=0.20] (233.15, 57.23) circle (  2.13);

\path[fill=fillColor,fill opacity=0.20] (221.13, 49.82) circle (  2.13);

\path[fill=fillColor,fill opacity=0.20] (199.50, 55.85) circle (  2.13);

\path[fill=fillColor,fill opacity=0.20] (204.31, 81.68) circle (  2.13);

\path[fill=fillColor,fill opacity=0.20] (173.50, 92.71) circle (  2.13);

\path[fill=fillColor,fill opacity=0.20] (177.43, 76.60) circle (  2.13);

\path[fill=fillColor,fill opacity=0.20] (183.55, 63.86) circle (  2.13);

\path[fill=fillColor,fill opacity=0.20] (190.54, 60.24) circle (  2.13);

\path[fill=fillColor,fill opacity=0.20] (204.31, 55.94) circle (  2.13);

\path[fill=fillColor,fill opacity=0.20] (202.56, 54.21) circle (  2.13);

\path[fill=fillColor,fill opacity=0.20] (202.78, 65.24) circle (  2.13);

\path[fill=fillColor,fill opacity=0.20] (200.81, 74.19) circle (  2.13);

\path[fill=fillColor,fill opacity=0.20] (198.84, 70.92) circle (  2.13);

\path[fill=fillColor,fill opacity=0.20] (195.35, 72.73) circle (  2.13);

\path[fill=fillColor,fill opacity=0.20] (197.97, 75.31) circle (  2.13);

\path[fill=fillColor,fill opacity=0.20] (202.12, 74.71) circle (  2.13);

\path[fill=fillColor,fill opacity=0.20] (198.84, 75.92) circle (  2.13);

\path[fill=fillColor,fill opacity=0.20] (196.88, 70.15) circle (  2.13);

\path[fill=fillColor,fill opacity=0.20] (192.94, 64.81) circle (  2.13);

\path[fill=fillColor,fill opacity=0.20] (190.98, 75.57) circle (  2.13);

\path[fill=fillColor,fill opacity=0.20] (187.26, 87.11) circle (  2.13);

\path[fill=fillColor,fill opacity=0.20] (192.73, 83.58) circle (  2.13);

\path[fill=fillColor,fill opacity=0.20] (197.97, 82.20) circle (  2.13);

\path[fill=fillColor,fill opacity=0.20] (213.26, 77.55) circle (  2.13);

\path[fill=fillColor,fill opacity=0.20] (209.33, 77.47) circle (  2.13);

\path[fill=fillColor,fill opacity=0.20] (206.71, 76.78) circle (  2.13);

\path[fill=fillColor,fill opacity=0.20] (204.31, 70.92) circle (  2.13);

\path[fill=fillColor,fill opacity=0.20] (213.70, 72.21) circle (  2.13);

\path[fill=fillColor,fill opacity=0.20] (218.51, 70.49) circle (  2.13);

\path[fill=fillColor,fill opacity=0.20] (214.79, 62.05) circle (  2.13);

\path[fill=fillColor,fill opacity=0.20] (210.64, 56.37) circle (  2.13);

\path[fill=fillColor,fill opacity=0.20] (213.70, 57.06) circle (  2.13);

\path[fill=fillColor,fill opacity=0.20] (216.98, 56.37) circle (  2.13);

\path[fill=fillColor,fill opacity=0.20] (220.69, 55.85) circle (  2.13);

\path[fill=fillColor,fill opacity=0.20] (223.53, 52.58) circle (  2.13);

\path[fill=fillColor,fill opacity=0.20] (226.16, 47.84) circle (  2.13);

\path[fill=fillColor,fill opacity=0.20] (230.31, 49.99) circle (  2.13);

\path[fill=fillColor,fill opacity=0.20] (224.85, 57.23) circle (  2.13);

\path[fill=fillColor,fill opacity=0.20] (223.10, 56.11) circle (  2.13);

\path[fill=fillColor,fill opacity=0.20] (232.27, 50.08) circle (  2.13);

\path[fill=fillColor,fill opacity=0.20] (230.53, 49.22) circle (  2.13);

\path[fill=fillColor,fill opacity=0.20] (194.47, 58.69) circle (  2.13);

\path[fill=fillColor,fill opacity=0.20] (171.53, 79.27) circle (  2.13);

\path[fill=fillColor,fill opacity=0.20] (196.88, 94.00) circle (  2.13);

\path[fill=fillColor,fill opacity=0.20] (174.37, 80.05) circle (  2.13);

\path[fill=fillColor,fill opacity=0.20] (184.42, 69.11) circle (  2.13);

\path[fill=fillColor,fill opacity=0.20] (198.84, 63.26) circle (  2.13);

\path[fill=fillColor,fill opacity=0.20] (205.40, 63.08) circle (  2.13);

\path[fill=fillColor,fill opacity=0.20] (205.40, 69.03) circle (  2.13);

\path[fill=fillColor,fill opacity=0.20] (205.62, 69.20) circle (  2.13);

\path[fill=fillColor,fill opacity=0.20] (204.96, 66.79) circle (  2.13);

\path[fill=fillColor,fill opacity=0.20] (204.96, 67.91) circle (  2.13);

\path[fill=fillColor,fill opacity=0.20] (205.62, 70.06) circle (  2.13);

\path[fill=fillColor,fill opacity=0.20] (201.03, 66.87) circle (  2.13);

\path[fill=fillColor,fill opacity=0.20] (196.44, 60.93) circle (  2.13);

\path[fill=fillColor,fill opacity=0.20] (194.91, 65.67) circle (  2.13);

\path[fill=fillColor,fill opacity=0.20] (194.91, 73.33) circle (  2.13);

\path[fill=fillColor,fill opacity=0.20] (198.84, 70.23) circle (  2.13);

\path[fill=fillColor,fill opacity=0.20] (193.60, 73.42) circle (  2.13);

\path[fill=fillColor,fill opacity=0.20] (194.26, 87.11) circle (  2.13);

\path[fill=fillColor,fill opacity=0.20] (196.44, 85.99) circle (  2.13);

\path[fill=fillColor,fill opacity=0.20] (201.03, 77.03) circle (  2.13);

\path[fill=fillColor,fill opacity=0.20] (209.55, 65.93) circle (  2.13);

\path[fill=fillColor,fill opacity=0.20] (210.86, 64.89) circle (  2.13);

\path[fill=fillColor,fill opacity=0.20] (206.27, 66.87) circle (  2.13);

\path[fill=fillColor,fill opacity=0.20] (205.40, 61.79) circle (  2.13);

\path[fill=fillColor,fill opacity=0.20] (216.98, 62.14) circle (  2.13);

\path[fill=fillColor,fill opacity=0.20] (226.81, 62.40) circle (  2.13);

\path[fill=fillColor,fill opacity=0.20] (216.54, 52.84) circle (  2.13);

\path[fill=fillColor,fill opacity=0.20] (218.51, 49.22) circle (  2.13);

\path[fill=fillColor,fill opacity=0.20] (221.35, 56.37) circle (  2.13);

\path[fill=fillColor,fill opacity=0.20] (217.42, 58.61) circle (  2.13);

\path[fill=fillColor,fill opacity=0.20] (216.98, 60.59) circle (  2.13);

\path[fill=fillColor,fill opacity=0.20] (218.51, 61.45) circle (  2.13);

\path[fill=fillColor,fill opacity=0.20] (221.35, 53.44) circle (  2.13);

\path[fill=fillColor,fill opacity=0.20] (222.44, 52.41) circle (  2.13);

\path[fill=fillColor,fill opacity=0.20] (217.63, 63.34) circle (  2.13);

\path[fill=fillColor,fill opacity=0.20] (220.69, 60.41) circle (  2.13);

\path[fill=fillColor,fill opacity=0.20] (230.74, 47.84) circle (  2.13);

\path[fill=fillColor,fill opacity=0.20] (222.44, 45.00) circle (  2.13);

\path[fill=fillColor,fill opacity=0.20] (190.76, 51.98) circle (  2.13);

\path[fill=fillColor,fill opacity=0.20] (168.04, 77.03) circle (  2.13);

\path[fill=fillColor,fill opacity=0.20] (172.62,102.44) circle (  2.13);

\path[fill=fillColor,fill opacity=0.20] (176.99, 87.28) circle (  2.13);

\path[fill=fillColor,fill opacity=0.20] (189.01, 69.28) circle (  2.13);

\path[fill=fillColor,fill opacity=0.20] (194.69, 64.55) circle (  2.13);

\path[fill=fillColor,fill opacity=0.20] (197.53, 66.61) circle (  2.13);

\path[fill=fillColor,fill opacity=0.20] (204.96, 60.50) circle (  2.13);

\path[fill=fillColor,fill opacity=0.20] (204.31, 62.48) circle (  2.13);

\path[fill=fillColor,fill opacity=0.20] (203.43, 68.94) circle (  2.13);

\path[fill=fillColor,fill opacity=0.20] (211.52, 60.50) circle (  2.13);

\path[fill=fillColor,fill opacity=0.20] (201.68, 57.40) circle (  2.13);

\path[fill=fillColor,fill opacity=0.20] (199.06, 72.90) circle (  2.13);

\path[fill=fillColor,fill opacity=0.20] (197.75, 74.62) circle (  2.13);

\path[fill=fillColor,fill opacity=0.20] (198.19, 56.63) circle (  2.13);

\path[fill=fillColor,fill opacity=0.20] (196.66, 56.97) circle (  2.13);

\path[fill=fillColor,fill opacity=0.20] (192.94, 79.62) circle (  2.13);

\path[fill=fillColor,fill opacity=0.20] (200.15, 94.00) circle (  2.13);

\path[fill=fillColor,fill opacity=0.20] (208.68, 73.42) circle (  2.13);

\path[fill=fillColor,fill opacity=0.20] (205.84, 55.85) circle (  2.13);

\path[fill=fillColor,fill opacity=0.20] (193.82, 62.83) circle (  2.13);

\path[fill=fillColor,fill opacity=0.20] (193.38, 73.50) circle (  2.13);

\path[fill=fillColor,fill opacity=0.20] (201.03, 74.19) circle (  2.13);

\path[fill=fillColor,fill opacity=0.20] (201.25, 66.36) circle (  2.13);

\path[fill=fillColor,fill opacity=0.20] (203.21, 53.09) circle (  2.13);

\path[fill=fillColor,fill opacity=0.20] (204.09, 47.50) circle (  2.13);

\path[fill=fillColor,fill opacity=0.20] (206.71, 54.39) circle (  2.13);

\path[fill=fillColor,fill opacity=0.20] (207.80, 59.47) circle (  2.13);

\path[fill=fillColor,fill opacity=0.20] (214.58, 57.49) circle (  2.13);

\path[fill=fillColor,fill opacity=0.20] (217.20, 52.23) circle (  2.13);

\path[fill=fillColor,fill opacity=0.20] (217.42, 46.29) circle (  2.13);

\path[fill=fillColor,fill opacity=0.20] (231.18, 46.89) circle (  2.13);

\path[fill=fillColor,fill opacity=0.20] (217.42, 54.04) circle (  2.13);

\path[fill=fillColor,fill opacity=0.20] (214.36, 56.88) circle (  2.13);

\path[fill=fillColor,fill opacity=0.20] (212.39, 59.12) circle (  2.13);

\path[fill=fillColor,fill opacity=0.20] (215.01, 60.59) circle (  2.13);

\path[fill=fillColor,fill opacity=0.20] (217.42, 55.59) circle (  2.13);

\path[fill=fillColor,fill opacity=0.20] (217.42, 54.04) circle (  2.13);

\path[fill=fillColor,fill opacity=0.20] (218.51, 62.05) circle (  2.13);

\path[fill=fillColor,fill opacity=0.20] (225.50, 62.65) circle (  2.13);

\path[fill=fillColor,fill opacity=0.20] (221.57, 50.77) circle (  2.13);

\path[fill=fillColor,fill opacity=0.20] (213.05, 42.50) circle (  2.13);

\path[fill=fillColor,fill opacity=0.20] (185.08, 54.73) circle (  2.13);

\path[fill=fillColor,fill opacity=0.20] (175.25, 91.76) circle (  2.13);

\path[fill=fillColor,fill opacity=0.20] (179.62, 87.02) circle (  2.13);

\path[fill=fillColor,fill opacity=0.20] (181.58, 79.45) circle (  2.13);

\path[fill=fillColor,fill opacity=0.20] (193.16, 68.51) circle (  2.13);

\path[fill=fillColor,fill opacity=0.20] (194.26, 71.70) circle (  2.13);

\path[fill=fillColor,fill opacity=0.20] (199.06, 73.59) circle (  2.13);

\path[fill=fillColor,fill opacity=0.20] (204.31, 60.67) circle (  2.13);

\path[fill=fillColor,fill opacity=0.20] (209.33, 59.12) circle (  2.13);

\path[fill=fillColor,fill opacity=0.20] (201.68, 72.73) circle (  2.13);

\path[fill=fillColor,fill opacity=0.20] (200.15, 68.42) circle (  2.13);

\path[fill=fillColor,fill opacity=0.20] (200.37, 53.61) circle (  2.13);

\path[fill=fillColor,fill opacity=0.20] (204.96, 59.04) circle (  2.13);

\path[fill=fillColor,fill opacity=0.20] (193.82, 75.83) circle (  2.13);

\path[fill=fillColor,fill opacity=0.20] (192.29, 79.19) circle (  2.13);

\path[fill=fillColor,fill opacity=0.20] (199.72, 71.44) circle (  2.13);

\path[fill=fillColor,fill opacity=0.20] (199.50, 68.34) circle (  2.13);

\path[fill=fillColor,fill opacity=0.20] (198.19, 76.95) circle (  2.13);

\path[fill=fillColor,fill opacity=0.20] (205.40, 80.48) circle (  2.13);

\path[fill=fillColor,fill opacity=0.20] (204.09, 68.34) circle (  2.13);

\path[fill=fillColor,fill opacity=0.20] (196.00, 73.76) circle (  2.13);

\path[fill=fillColor,fill opacity=0.20] (189.23, 87.54) circle (  2.13);

\path[fill=fillColor,fill opacity=0.20] (195.35, 90.38) circle (  2.13);

\path[fill=fillColor,fill opacity=0.20] (202.12, 82.29) circle (  2.13);

\path[fill=fillColor,fill opacity=0.20] (202.78, 68.34) circle (  2.13);

\path[fill=fillColor,fill opacity=0.20] (206.05, 50.34) circle (  2.13);

\path[fill=fillColor,fill opacity=0.20] (209.99, 42.50) circle (  2.13);

\path[fill=fillColor,fill opacity=0.20] (207.58, 45.78) circle (  2.13);

\path[fill=fillColor,fill opacity=0.20] (207.58, 47.24) circle (  2.13);

\path[fill=fillColor,fill opacity=0.20] (207.15, 48.88) circle (  2.13);

\path[fill=fillColor,fill opacity=0.20] (210.21, 54.47) circle (  2.13);

\path[fill=fillColor,fill opacity=0.20] (217.20, 55.25) circle (  2.13);

\path[fill=fillColor,fill opacity=0.20] (211.30, 51.98) circle (  2.13);

\path[fill=fillColor,fill opacity=0.20] (216.32, 50.60) circle (  2.13);

\path[fill=fillColor,fill opacity=0.20] (218.29, 51.98) circle (  2.13);

\path[fill=fillColor,fill opacity=0.20] (215.45, 52.15) circle (  2.13);

\path[fill=fillColor,fill opacity=0.20] (209.11, 52.58) circle (  2.13);

\path[fill=fillColor,fill opacity=0.20] (211.08, 56.45) circle (  2.13);

\path[fill=fillColor,fill opacity=0.20] (210.64, 58.43) circle (  2.13);

\path[fill=fillColor,fill opacity=0.20] (209.33, 55.59) circle (  2.13);

\path[fill=fillColor,fill opacity=0.20] (210.64, 53.27) circle (  2.13);

\path[fill=fillColor,fill opacity=0.20] (217.42, 56.88) circle (  2.13);

\path[fill=fillColor,fill opacity=0.20] (226.16, 58.09) circle (  2.13);

\path[fill=fillColor,fill opacity=0.20] (212.17, 50.17) circle (  2.13);

\path[fill=fillColor,fill opacity=0.20] (200.37, 53.18) circle (  2.13);

\path[fill=fillColor,fill opacity=0.20] (179.40, 82.37) circle (  2.13);

\path[fill=fillColor,fill opacity=0.20] (172.41, 88.32) circle (  2.13);

\path[fill=fillColor,fill opacity=0.20] (181.80, 87.97) circle (  2.13);

\path[fill=fillColor,fill opacity=0.20] (184.42, 83.06) circle (  2.13);

\path[fill=fillColor,fill opacity=0.20] (191.85, 73.16) circle (  2.13);

\path[fill=fillColor,fill opacity=0.20] (200.37, 69.46) circle (  2.13);

\path[fill=fillColor,fill opacity=0.20] (202.12, 66.96) circle (  2.13);

\path[fill=fillColor,fill opacity=0.20] (199.94, 60.93) circle (  2.13);

\path[fill=fillColor,fill opacity=0.20] (201.25, 57.06) circle (  2.13);

\path[fill=fillColor,fill opacity=0.20] (206.27, 56.71) circle (  2.13);

\path[fill=fillColor,fill opacity=0.20] (201.47, 62.14) circle (  2.13);

\path[fill=fillColor,fill opacity=0.20] (197.75, 68.08) circle (  2.13);

\path[fill=fillColor,fill opacity=0.20] (206.93, 64.81) circle (  2.13);

\path[fill=fillColor,fill opacity=0.20] (194.69, 64.03) circle (  2.13);

\path[fill=fillColor,fill opacity=0.20] (192.94, 70.06) circle (  2.13);

\path[fill=fillColor,fill opacity=0.20] (196.66, 67.99) circle (  2.13);

\path[fill=fillColor,fill opacity=0.20] (200.37, 63.08) circle (  2.13);

\path[fill=fillColor,fill opacity=0.20] (197.10, 70.15) circle (  2.13);

\path[fill=fillColor,fill opacity=0.20] (190.10, 97.44) circle (  2.13);

\path[fill=fillColor,fill opacity=0.20] (190.76, 85.13) circle (  2.13);

\path[fill=fillColor,fill opacity=0.20] (189.01, 76.43) circle (  2.13);

\path[fill=fillColor,fill opacity=0.20] (198.84, 81.08) circle (  2.13);

\path[fill=fillColor,fill opacity=0.20] (194.26, 80.48) circle (  2.13);

\path[fill=fillColor,fill opacity=0.20] (200.81, 67.82) circle (  2.13);

\path[fill=fillColor,fill opacity=0.20] (196.22, 64.20) circle (  2.13);

\path[fill=fillColor,fill opacity=0.20] (204.09, 71.35) circle (  2.13);

\path[fill=fillColor,fill opacity=0.20] (204.52, 71.52) circle (  2.13);

\path[fill=fillColor,fill opacity=0.20] (204.52, 70.49) circle (  2.13);

\path[fill=fillColor,fill opacity=0.20] (198.19, 77.90) circle (  2.13);

\path[fill=fillColor,fill opacity=0.20] (188.57, 80.74) circle (  2.13);

\path[fill=fillColor,fill opacity=0.20] (176.34, 88.92) circle (  2.13);

\path[fill=fillColor,fill opacity=0.20] (166.29,113.37) circle (  2.13);

\path[fill=fillColor,fill opacity=0.20] (200.59, 84.70) circle (  2.13);

\path[fill=fillColor,fill opacity=0.20] (208.68, 78.58) circle (  2.13);

\path[fill=fillColor,fill opacity=0.20] (208.68, 75.31) circle (  2.13);

\path[fill=fillColor,fill opacity=0.20] (210.64, 65.75) circle (  2.13);

\path[fill=fillColor,fill opacity=0.20] (225.50, 56.71) circle (  2.13);

\path[fill=fillColor,fill opacity=0.20] (210.64, 56.97) circle (  2.13);

\path[fill=fillColor,fill opacity=0.20] (215.01, 57.40) circle (  2.13);

\path[fill=fillColor,fill opacity=0.20] (213.70, 56.71) circle (  2.13);

\path[fill=fillColor,fill opacity=0.20] (212.17, 57.23) circle (  2.13);

\path[fill=fillColor,fill opacity=0.20] (212.61, 55.51) circle (  2.13);

\path[fill=fillColor,fill opacity=0.20] (213.05, 53.78) circle (  2.13);

\path[fill=fillColor,fill opacity=0.20] (211.95, 55.68) circle (  2.13);

\path[fill=fillColor,fill opacity=0.20] (211.95, 58.00) circle (  2.13);

\path[fill=fillColor,fill opacity=0.20] (216.32, 58.78) circle (  2.13);

\path[fill=fillColor,fill opacity=0.20] (209.11, 56.63) circle (  2.13);

\path[fill=fillColor,fill opacity=0.20] (216.54, 55.08) circle (  2.13);

\path[fill=fillColor,fill opacity=0.20] (211.52, 55.85) circle (  2.13);

\path[fill=fillColor,fill opacity=0.20] (198.19, 62.22) circle (  2.13);

\path[fill=fillColor,fill opacity=0.20] (182.24, 80.57) circle (  2.13);

\path[fill=fillColor,fill opacity=0.20] (182.24, 92.36) circle (  2.13);

\path[fill=fillColor,fill opacity=0.20] (184.86, 87.11) circle (  2.13);

\path[fill=fillColor,fill opacity=0.20] (183.11, 78.76) circle (  2.13);

\path[fill=fillColor,fill opacity=0.20] (187.04, 74.45) circle (  2.13);

\path[fill=fillColor,fill opacity=0.20] (194.26, 71.18) circle (  2.13);

\path[fill=fillColor,fill opacity=0.20] (206.27, 58.69) circle (  2.13);

\path[fill=fillColor,fill opacity=0.20] (200.15, 50.34) circle (  2.13);

\path[fill=fillColor,fill opacity=0.20] (201.25, 54.39) circle (  2.13);

\path[fill=fillColor,fill opacity=0.20] (200.81, 58.18) circle (  2.13);

\path[fill=fillColor,fill opacity=0.20] (200.15, 65.58) circle (  2.13);

\path[fill=fillColor,fill opacity=0.20] (199.28, 73.42) circle (  2.13);

\path[fill=fillColor,fill opacity=0.20] (199.06, 68.85) circle (  2.13);

\path[fill=fillColor,fill opacity=0.20] (192.29, 58.78) circle (  2.13);

\path[fill=fillColor,fill opacity=0.20] (201.25, 58.95) circle (  2.13);

\path[fill=fillColor,fill opacity=0.20] (197.53, 65.06) circle (  2.13);

\path[fill=fillColor,fill opacity=0.20] (193.82, 72.47) circle (  2.13);

\path[fill=fillColor,fill opacity=0.20] (191.85, 77.64) circle (  2.13);

\path[fill=fillColor,fill opacity=0.20] (191.41, 75.57) circle (  2.13);

\path[fill=fillColor,fill opacity=0.20] (193.38, 70.92) circle (  2.13);

\path[fill=fillColor,fill opacity=0.20] (198.84, 69.89) circle (  2.13);

\path[fill=fillColor,fill opacity=0.20] (196.66, 74.11) circle (  2.13);

\path[fill=fillColor,fill opacity=0.20] (193.82, 81.94) circle (  2.13);

\path[fill=fillColor,fill opacity=0.20] (191.41, 89.09) circle (  2.13);

\path[fill=fillColor,fill opacity=0.20] (192.94, 81.43) circle (  2.13);

\path[fill=fillColor,fill opacity=0.20] (197.10, 73.42) circle (  2.13);

\path[fill=fillColor,fill opacity=0.20] (200.59, 70.23) circle (  2.13);

\path[fill=fillColor,fill opacity=0.20] (201.25, 67.56) circle (  2.13);

\path[fill=fillColor,fill opacity=0.20] (204.31, 65.24) circle (  2.13);

\path[fill=fillColor,fill opacity=0.20] (205.40, 64.63) circle (  2.13);

\path[fill=fillColor,fill opacity=0.20] (203.00, 60.67) circle (  2.13);

\path[fill=fillColor,fill opacity=0.20] (204.09, 60.24) circle (  2.13);

\path[fill=fillColor,fill opacity=0.20] (206.05, 69.97) circle (  2.13);

\path[fill=fillColor,fill opacity=0.20] (205.62, 73.85) circle (  2.13);

\path[fill=fillColor,fill opacity=0.20] (204.74, 65.67) circle (  2.13);

\path[fill=fillColor,fill opacity=0.20] (201.47, 63.34) circle (  2.13);

\path[fill=fillColor,fill opacity=0.20] (200.59, 72.13) circle (  2.13);

\path[fill=fillColor,fill opacity=0.20] (203.00, 78.93) circle (  2.13);

\path[fill=fillColor,fill opacity=0.20] (184.86, 88.49) circle (  2.13);

\path[fill=fillColor,fill opacity=0.20] (175.25,104.16) circle (  2.13);

\path[fill=fillColor,fill opacity=0.20] (190.32,113.37) circle (  2.13);

\path[fill=fillColor,fill opacity=0.20] (196.00,102.52) circle (  2.13);

\path[fill=fillColor,fill opacity=0.20] (207.58, 92.10) circle (  2.13);

\path[fill=fillColor,fill opacity=0.20] (216.32, 78.15) circle (  2.13);

\path[fill=fillColor,fill opacity=0.20] (212.17, 70.23) circle (  2.13);

\path[fill=fillColor,fill opacity=0.20] (207.37, 68.51) circle (  2.13);

\path[fill=fillColor,fill opacity=0.20] (209.11, 67.05) circle (  2.13);

\path[fill=fillColor,fill opacity=0.20] (216.98, 66.10) circle (  2.13);

\path[fill=fillColor,fill opacity=0.20] (214.14, 65.15) circle (  2.13);

\path[fill=fillColor,fill opacity=0.20] (207.58, 64.63) circle (  2.13);

\path[fill=fillColor,fill opacity=0.20] (209.77, 70.15) circle (  2.13);

\path[fill=fillColor,fill opacity=0.20] (209.55, 73.16) circle (  2.13);

\path[fill=fillColor,fill opacity=0.20] (203.87, 70.49) circle (  2.13);

\path[fill=fillColor,fill opacity=0.20] (198.84, 79.27) circle (  2.13);

\path[fill=fillColor,fill opacity=0.20] (178.74,103.39) circle (  2.13);

\path[fill=fillColor,fill opacity=0.20] (181.36, 85.13) circle (  2.13);

\path[fill=fillColor,fill opacity=0.20] (183.33, 65.24) circle (  2.13);

\path[fill=fillColor,fill opacity=0.20] (192.51, 60.24) circle (  2.13);

\path[fill=fillColor,fill opacity=0.20] (202.34, 63.26) circle (  2.13);

\path[fill=fillColor,fill opacity=0.20] (206.71, 66.36) circle (  2.13);

\path[fill=fillColor,fill opacity=0.20] (204.96, 66.79) circle (  2.13);

\path[fill=fillColor,fill opacity=0.20] (196.44, 64.29) circle (  2.13);

\path[fill=fillColor,fill opacity=0.20] (201.25, 61.71) circle (  2.13);

\path[fill=fillColor,fill opacity=0.20] (198.84, 59.98) circle (  2.13);

\path[fill=fillColor,fill opacity=0.20] (200.59, 61.02) circle (  2.13);

\path[fill=fillColor,fill opacity=0.20] (196.66, 67.48) circle (  2.13);

\path[fill=fillColor,fill opacity=0.20] (196.44, 71.52) circle (  2.13);

\path[fill=fillColor,fill opacity=0.20] (198.84, 68.25) circle (  2.13);

\path[fill=fillColor,fill opacity=0.20] (197.97, 65.15) circle (  2.13);

\path[fill=fillColor,fill opacity=0.20] (201.90, 62.05) circle (  2.13);

\path[fill=fillColor,fill opacity=0.20] (199.94, 58.26) circle (  2.13);

\path[fill=fillColor,fill opacity=0.20] (198.41, 59.81) circle (  2.13);

\path[fill=fillColor,fill opacity=0.20] (203.43, 61.62) circle (  2.13);

\path[fill=fillColor,fill opacity=0.20] (200.37, 62.05) circle (  2.13);

\path[fill=fillColor,fill opacity=0.20] (199.94, 66.61) circle (  2.13);

\path[fill=fillColor,fill opacity=0.20] (201.03, 70.83) circle (  2.13);

\path[fill=fillColor,fill opacity=0.20] (217.20, 70.92) circle (  2.13);

\path[fill=fillColor,fill opacity=0.20] (209.33, 65.67) circle (  2.13);

\path[fill=fillColor,fill opacity=0.20] (206.05, 60.07) circle (  2.13);

\path[fill=fillColor,fill opacity=0.20] (207.58, 61.71) circle (  2.13);

\path[fill=fillColor,fill opacity=0.20] (211.08, 65.67) circle (  2.13);

\path[fill=fillColor,fill opacity=0.20] (213.05, 61.10) circle (  2.13);

\path[fill=fillColor,fill opacity=0.20] (209.11, 57.06) circle (  2.13);

\path[fill=fillColor,fill opacity=0.20] (202.56, 66.10) circle (  2.13);

\path[fill=fillColor,fill opacity=0.20] (196.88, 79.10) circle (  2.13);

\path[fill=fillColor,fill opacity=0.20] (187.92, 87.20) circle (  2.13);

\path[fill=fillColor,fill opacity=0.20] (178.52, 94.52) circle (  2.13);

\path[fill=fillColor,fill opacity=0.20] (199.28,103.73) circle (  2.13);

\path[fill=fillColor,fill opacity=0.20] (195.57, 98.13) circle (  2.13);

\path[fill=fillColor,fill opacity=0.20] (198.41, 90.64) circle (  2.13);

\path[fill=fillColor,fill opacity=0.20] (205.40, 87.45) circle (  2.13);

\path[fill=fillColor,fill opacity=0.20] (206.05, 86.08) circle (  2.13);

\path[fill=fillColor,fill opacity=0.20] (200.37, 85.39) circle (  2.13);

\path[fill=fillColor,fill opacity=0.20] (199.94, 92.28) circle (  2.13);

\path[fill=fillColor,fill opacity=0.20] (176.12,100.46) circle (  2.13);

\path[fill=fillColor,fill opacity=0.20] (185.95, 82.98) circle (  2.13);

\path[fill=fillColor,fill opacity=0.20] (193.82, 69.71) circle (  2.13);

\path[fill=fillColor,fill opacity=0.20] (199.28, 64.98) circle (  2.13);

\path[fill=fillColor,fill opacity=0.20] (209.11, 62.14) circle (  2.13);

\path[fill=fillColor,fill opacity=0.20] (206.05, 63.69) circle (  2.13);

\path[fill=fillColor,fill opacity=0.20] (202.34, 67.56) circle (  2.13);

\path[fill=fillColor,fill opacity=0.20] (199.28, 66.01) circle (  2.13);

\path[fill=fillColor,fill opacity=0.20] (201.68, 64.03) circle (  2.13);

\path[fill=fillColor,fill opacity=0.20] (203.43, 65.41) circle (  2.13);

\path[fill=fillColor,fill opacity=0.20] (202.34, 64.81) circle (  2.13);

\path[fill=fillColor,fill opacity=0.20] (205.62, 66.10) circle (  2.13);

\path[fill=fillColor,fill opacity=0.20] (202.12, 69.03) circle (  2.13);

\path[fill=fillColor,fill opacity=0.20] (201.03, 66.87) circle (  2.13);

\path[fill=fillColor,fill opacity=0.20] (200.59, 63.60) circle (  2.13);

\path[fill=fillColor,fill opacity=0.20] (204.74, 62.65) circle (  2.13);

\path[fill=fillColor,fill opacity=0.20] (205.40, 61.79) circle (  2.13);

\path[fill=fillColor,fill opacity=0.20] (203.65, 62.48) circle (  2.13);

\path[fill=fillColor,fill opacity=0.20] (210.42, 66.44) circle (  2.13);

\path[fill=fillColor,fill opacity=0.20] (212.17, 72.30) circle (  2.13);

\path[fill=fillColor,fill opacity=0.20] (211.95, 70.83) circle (  2.13);

\path[fill=fillColor,fill opacity=0.20] (213.48, 61.53) circle (  2.13);

\path[fill=fillColor,fill opacity=0.20] (212.17, 58.69) circle (  2.13);

\path[fill=fillColor,fill opacity=0.20] (211.08, 65.24) circle (  2.13);

\path[fill=fillColor,fill opacity=0.20] (204.52, 68.60) circle (  2.13);

\path[fill=fillColor,fill opacity=0.20] (191.41, 69.63) circle (  2.13);

\path[fill=fillColor,fill opacity=0.20] (181.36, 80.57) circle (  2.13);

\path[fill=fillColor,fill opacity=0.20] (175.68, 98.22) circle (  2.13);

\path[fill=fillColor,fill opacity=0.20] (180.93, 92.71) circle (  2.13);

\path[fill=fillColor,fill opacity=0.20] (203.00, 81.51) circle (  2.13);

\path[fill=fillColor,fill opacity=0.20] (194.69, 71.87) circle (  2.13);

\path[fill=fillColor,fill opacity=0.20] (194.47, 73.33) circle (  2.13);

\path[fill=fillColor,fill opacity=0.20] (196.66, 71.61) circle (  2.13);

\path[fill=fillColor,fill opacity=0.20] (200.59, 62.40) circle (  2.13);

\path[fill=fillColor,fill opacity=0.20] (203.43, 60.67) circle (  2.13);

\path[fill=fillColor,fill opacity=0.20] (205.84, 66.79) circle (  2.13);

\path[fill=fillColor,fill opacity=0.20] (206.71, 68.42) circle (  2.13);

\path[fill=fillColor,fill opacity=0.20] (209.11, 66.96) circle (  2.13);

\path[fill=fillColor,fill opacity=0.20] (208.68, 67.48) circle (  2.13);

\path[fill=fillColor,fill opacity=0.20] (207.80, 66.87) circle (  2.13);

\path[fill=fillColor,fill opacity=0.20] (212.17, 63.08) circle (  2.13);

\path[fill=fillColor,fill opacity=0.20] (213.48, 60.67) circle (  2.13);

\path[fill=fillColor,fill opacity=0.20] (211.30, 60.41) circle (  2.13);

\path[fill=fillColor,fill opacity=0.20] (218.29, 58.52) circle (  2.13);

\path[fill=fillColor,fill opacity=0.20] (219.16, 60.76) circle (  2.13);

\path[fill=fillColor,fill opacity=0.20] (208.89, 65.41) circle (  2.13);

\path[fill=fillColor,fill opacity=0.20] (206.27, 63.60) circle (  2.13);

\path[fill=fillColor,fill opacity=0.20] (202.34, 60.67) circle (  2.13);

\path[fill=fillColor,fill opacity=0.20] (197.75, 66.96) circle (  2.13);

\path[fill=fillColor,fill opacity=0.20] (188.57, 79.10) circle (  2.13);

\path[fill=fillColor,fill opacity=0.20] (184.86, 83.67) circle (  2.13);

\path[fill=fillColor,fill opacity=0.20] (188.79, 79.02) circle (  2.13);

\path[fill=fillColor,fill opacity=0.20] (197.10, 75.23) circle (  2.13);

\path[fill=fillColor,fill opacity=0.20] (219.60, 78.24) circle (  2.13);

\path[fill=fillColor,fill opacity=0.20] (199.06, 73.93) circle (  2.13);

\path[fill=fillColor,fill opacity=0.20] (206.93, 60.16) circle (  2.13);

\path[fill=fillColor,fill opacity=0.20] (212.17, 55.85) circle (  2.13);

\path[fill=fillColor,fill opacity=0.20] (211.95, 62.14) circle (  2.13);

\path[fill=fillColor,fill opacity=0.20] (215.23, 61.45) circle (  2.13);

\path[fill=fillColor,fill opacity=0.20] (213.05, 59.81) circle (  2.13);

\path[fill=fillColor,fill opacity=0.20] (208.89, 65.15) circle (  2.13);

\path[fill=fillColor,fill opacity=0.20] (214.58, 66.96) circle (  2.13);

\path[fill=fillColor,fill opacity=0.20] (199.94, 63.34) circle (  2.13);

\path[fill=fillColor,fill opacity=0.20] (194.47, 65.06) circle (  2.13);

\path[fill=fillColor,fill opacity=0.20] (195.13, 71.61) circle (  2.13);

\path[fill=fillColor,fill opacity=0.20] (183.99, 77.21) circle (  2.13);

\path[fill=fillColor,fill opacity=0.20] (179.62, 82.46) circle (  2.13);

\path[fill=fillColor,fill opacity=0.20] (196.66, 75.40) circle (  2.13);

\path[fill=fillColor,fill opacity=0.20] (200.59, 65.67) circle (  2.13);

\path[fill=fillColor,fill opacity=0.20] (195.57, 67.13) circle (  2.13);

\path[fill=fillColor,fill opacity=0.20] (196.22, 72.90) circle (  2.13);

\path[fill=fillColor,fill opacity=0.20] (196.44, 77.21) circle (  2.13);

\path[fill=fillColor,fill opacity=0.20] (189.89, 82.72) circle (  2.13);

\path[fill=fillColor,fill opacity=0.20] (185.30, 88.75) circle (  2.13);

\path[fill=fillColor,fill opacity=0.20] (182.46, 88.23) circle (  2.13);

\path[fill=fillColor,fill opacity=0.20] (177.65, 83.84) circle (  2.13);

\path[fill=fillColor,fill opacity=0.20] (197.31, 97.44) circle (  2.13);

\path[fill=fillColor,fill opacity=0.20] (200.37,105.28) circle (  2.13);

\path[fill=fillColor,fill opacity=0.20] (185.08,103.47) circle (  2.13);

\path[fill=fillColor,fill opacity=0.20] (204.74, 59.90) circle (  2.13);

\path[fill=fillColor,fill opacity=0.20] (209.99, 56.11) circle (  2.13);

\path[fill=fillColor,fill opacity=0.20] (210.21, 62.91) circle (  2.13);

\path[fill=fillColor,fill opacity=0.20] (209.55, 72.21) circle (  2.13);

\path[fill=fillColor,fill opacity=0.20] (205.62, 71.52) circle (  2.13);

\path[fill=fillColor,fill opacity=0.20] (180.05, 99.60) circle (  2.13);

\path[fill=fillColor,fill opacity=0.20] (177.21, 80.48) circle (  2.13);

\path[fill=fillColor,fill opacity=0.20] (182.24, 71.52) circle (  2.13);

\path[fill=fillColor,fill opacity=0.20] (183.99, 81.17) circle (  2.13);

\path[fill=fillColor,fill opacity=0.20] (183.11, 78.24) circle (  2.13);

\path[fill=fillColor,fill opacity=0.20] (180.49, 70.49) circle (  2.13);

\path[fill=fillColor,fill opacity=0.20] (201.90, 51.89) circle (  2.13);

\path[fill=fillColor,fill opacity=0.20] (215.67, 48.01) circle (  2.13);

\path[fill=fillColor,fill opacity=0.20] (217.20, 57.92) circle (  2.13);

\path[fill=fillColor,fill opacity=0.20] (223.32, 59.81) circle (  2.13);

\path[fill=fillColor,fill opacity=0.20] (225.28, 60.85) circle (  2.13);

\path[fill=fillColor,fill opacity=0.20] (220.04, 66.61) circle (  2.13);

\path[fill=fillColor,fill opacity=0.20] (212.61, 75.31) circle (  2.13);

\path[fill=fillColor,fill opacity=0.20] (202.34, 85.90) circle (  2.13);

\path[fill=fillColor,fill opacity=0.20] (180.49,103.21) circle (  2.13);

\path[fill=fillColor,fill opacity=0.20] (181.15, 88.83) circle (  2.13);

\path[fill=fillColor,fill opacity=0.20] (182.46, 87.63) circle (  2.13);

\path[fill=fillColor,fill opacity=0.20] (190.54, 73.25) circle (  2.13);

\path[fill=fillColor,fill opacity=0.20] (199.06, 61.96) circle (  2.13);

\path[fill=fillColor,fill opacity=0.20] (201.90, 67.39) circle (  2.13);

\path[fill=fillColor,fill opacity=0.20] (198.63, 66.53) circle (  2.13);

\path[fill=fillColor,fill opacity=0.20] (194.47, 64.03) circle (  2.13);

\path[fill=fillColor,fill opacity=0.20] (190.10, 66.36) circle (  2.13);

\path[fill=fillColor,fill opacity=0.20] (185.52, 73.50) circle (  2.13);

\path[fill=fillColor,fill opacity=0.20] (185.73, 72.21) circle (  2.13);

\path[fill=fillColor,fill opacity=0.20] (219.38, 45.60) circle (  2.13);

\path[fill=fillColor,fill opacity=0.20] (220.91, 62.91) circle (  2.13);

\path[fill=fillColor,fill opacity=0.20] (224.19, 58.00) circle (  2.13);

\path[fill=fillColor,fill opacity=0.20] (232.93, 52.58) circle (  2.13);

\path[fill=fillColor,fill opacity=0.20] (234.02, 48.88) circle (  2.13);

\path[fill=fillColor,fill opacity=0.20] (229.00, 53.53) circle (  2.13);

\path[fill=fillColor,fill opacity=0.20] (217.42, 59.90) circle (  2.13);

\path[fill=fillColor,fill opacity=0.20] (206.93, 61.88) circle (  2.13);

\path[fill=fillColor,fill opacity=0.20] (197.75, 78.15) circle (  2.13);

\path[fill=fillColor,fill opacity=0.20] (206.71, 93.57) circle (  2.13);

\path[fill=fillColor,fill opacity=0.20] (190.10, 81.86) circle (  2.13);

\path[fill=fillColor,fill opacity=0.20] (197.53, 74.80) circle (  2.13);

\path[fill=fillColor,fill opacity=0.20] (196.66, 60.33) circle (  2.13);

\path[fill=fillColor,fill opacity=0.20] (201.68, 51.37) circle (  2.13);

\path[fill=fillColor,fill opacity=0.20] (204.96, 61.19) circle (  2.13);

\path[fill=fillColor,fill opacity=0.20] (206.71, 70.66) circle (  2.13);

\path[fill=fillColor,fill opacity=0.20] (205.40, 61.71) circle (  2.13);

\path[fill=fillColor,fill opacity=0.20] (197.97, 58.78) circle (  2.13);

\path[fill=fillColor,fill opacity=0.20] (195.57, 70.15) circle (  2.13);

\path[fill=fillColor,fill opacity=0.20] (198.84, 50.60) circle (  2.13);

\path[fill=fillColor,fill opacity=0.20] (227.25, 57.49) circle (  2.13);

\path[fill=fillColor,fill opacity=0.20] (225.72, 71.27) circle (  2.13);

\path[fill=fillColor,fill opacity=0.20] (233.59, 59.21) circle (  2.13);

\path[fill=fillColor,fill opacity=0.20] (234.68, 45.69) circle (  2.13);

\path[fill=fillColor,fill opacity=0.20] (233.15, 38.28) circle (  2.13);

\path[fill=fillColor,fill opacity=0.20] (232.27, 39.23) circle (  2.13);

\path[fill=fillColor,fill opacity=0.20] (222.44, 38.54) circle (  2.13);

\path[fill=fillColor,fill opacity=0.20] (217.20, 38.37) circle (  2.13);

\path[fill=fillColor,fill opacity=0.20] (206.27, 56.20) circle (  2.13);

\path[fill=fillColor,fill opacity=0.20] (179.18, 93.22) circle (  2.13);

\path[fill=fillColor,fill opacity=0.20] (188.14, 61.10) circle (  2.13);

\path[fill=fillColor,fill opacity=0.20] (192.29, 71.70) circle (  2.13);

\path[fill=fillColor,fill opacity=0.20] (196.44, 78.07) circle (  2.13);

\path[fill=fillColor,fill opacity=0.20] (199.28, 61.19) circle (  2.13);

\path[fill=fillColor,fill opacity=0.20] (204.74, 54.99) circle (  2.13);

\path[fill=fillColor,fill opacity=0.20] (208.02, 58.78) circle (  2.13);

\path[fill=fillColor,fill opacity=0.20] (206.27, 63.51) circle (  2.13);

\path[fill=fillColor,fill opacity=0.20] (205.18, 63.43) circle (  2.13);

\path[fill=fillColor,fill opacity=0.20] (194.69, 56.88) circle (  2.13);

\path[fill=fillColor,fill opacity=0.20] (185.73, 59.81) circle (  2.13);

\path[fill=fillColor,fill opacity=0.20] (177.21, 72.99) circle (  2.13);

\path[fill=fillColor,fill opacity=0.20] (197.97, 65.15) circle (  2.13);

\path[fill=fillColor,fill opacity=0.20] (225.06, 51.80) circle (  2.13);

\path[fill=fillColor,fill opacity=0.20] (225.50, 64.89) circle (  2.13);

\path[fill=fillColor,fill opacity=0.20] (234.24, 63.51) circle (  2.13);

\path[fill=fillColor,fill opacity=0.20] (235.33, 44.74) circle (  2.13);

\path[fill=fillColor,fill opacity=0.20] (233.80, 40.01) circle (  2.13);

\path[fill=fillColor,fill opacity=0.20] (216.11, 42.16) circle (  2.13);

\path[fill=fillColor,fill opacity=0.20] (205.62, 66.87) circle (  2.13);

\path[fill=fillColor,fill opacity=0.20] (175.25, 89.44) circle (  2.13);

\path[fill=fillColor,fill opacity=0.20] (192.94, 69.54) circle (  2.13);

\path[fill=fillColor,fill opacity=0.20] (201.68, 55.16) circle (  2.13);

\path[fill=fillColor,fill opacity=0.20] (199.94, 61.62) circle (  2.13);

\path[fill=fillColor,fill opacity=0.20] (201.25, 65.50) circle (  2.13);

\path[fill=fillColor,fill opacity=0.20] (204.31, 59.98) circle (  2.13);

\path[fill=fillColor,fill opacity=0.20] (207.80, 58.69) circle (  2.13);

\path[fill=fillColor,fill opacity=0.20] (209.33, 54.04) circle (  2.13);

\path[fill=fillColor,fill opacity=0.20] (206.93, 54.82) circle (  2.13);

\path[fill=fillColor,fill opacity=0.20] (204.31, 67.48) circle (  2.13);

\path[fill=fillColor,fill opacity=0.20] (196.22, 64.12) circle (  2.13);

\path[fill=fillColor,fill opacity=0.20] (173.28, 65.24) circle (  2.13);

\path[fill=fillColor,fill opacity=0.20] (197.10, 58.09) circle (  2.13);

\path[fill=fillColor,fill opacity=0.20] (220.91, 43.36) circle (  2.13);

\path[fill=fillColor,fill opacity=0.20] (223.53, 53.87) circle (  2.13);

\path[fill=fillColor,fill opacity=0.20] (234.46, 54.64) circle (  2.13);

\path[fill=fillColor,fill opacity=0.20] (238.17, 43.19) circle (  2.13);

\path[fill=fillColor,fill opacity=0.20] (230.31, 50.17) circle (  2.13);

\path[fill=fillColor,fill opacity=0.20] (229.22, 56.80) circle (  2.13);

\path[fill=fillColor,fill opacity=0.20] (226.37, 47.41) circle (  2.13);

\path[fill=fillColor,fill opacity=0.20] (216.54, 43.11) circle (  2.13);

\path[fill=fillColor,fill opacity=0.20] (208.02, 56.63) circle (  2.13);

\path[fill=fillColor,fill opacity=0.20] (194.26, 88.57) circle (  2.13);

\path[fill=fillColor,fill opacity=0.20] (183.11, 72.21) circle (  2.13);

\path[fill=fillColor,fill opacity=0.20] (198.63, 52.15) circle (  2.13);

\path[fill=fillColor,fill opacity=0.20] (202.56, 53.35) circle (  2.13);

\path[fill=fillColor,fill opacity=0.20] (208.46, 40.95) circle (  2.13);

\path[fill=fillColor,fill opacity=0.20] (210.42, 41.38) circle (  2.13);

\path[fill=fillColor,fill opacity=0.20] (213.05, 53.87) circle (  2.13);

\path[fill=fillColor,fill opacity=0.20] (215.45, 56.88) circle (  2.13);

\path[fill=fillColor,fill opacity=0.20] (212.17, 52.32) circle (  2.13);

\path[fill=fillColor,fill opacity=0.20] (208.89, 54.82) circle (  2.13);

\path[fill=fillColor,fill opacity=0.20] (203.43, 64.63) circle (  2.13);

\path[fill=fillColor,fill opacity=0.20] (189.67, 74.80) circle (  2.13);

\path[fill=fillColor,fill opacity=0.20] (202.34, 40.09) circle (  2.13);

\path[fill=fillColor,fill opacity=0.20] (219.38, 46.12) circle (  2.13);

\path[fill=fillColor,fill opacity=0.20] (220.91, 50.43) circle (  2.13);

\path[fill=fillColor,fill opacity=0.20] (234.68, 48.19) circle (  2.13);

\path[fill=fillColor,fill opacity=0.20] (237.30, 45.17) circle (  2.13);

\path[fill=fillColor,fill opacity=0.20] (231.18, 40.61) circle (  2.13);

\path[fill=fillColor,fill opacity=0.20] (227.90, 53.18) circle (  2.13);

\path[fill=fillColor,fill opacity=0.20] (225.28, 64.12) circle (  2.13);

\path[fill=fillColor,fill opacity=0.20] (222.44, 51.46) circle (  2.13);

\path[fill=fillColor,fill opacity=0.20] (216.32, 42.33) circle (  2.13);

\path[fill=fillColor,fill opacity=0.20] (204.52, 53.53) circle (  2.13);

\path[fill=fillColor,fill opacity=0.20] (191.63, 84.27) circle (  2.13);

\path[fill=fillColor,fill opacity=0.20] (177.43, 91.76) circle (  2.13);

\path[fill=fillColor,fill opacity=0.20] (194.04, 76.78) circle (  2.13);

\path[fill=fillColor,fill opacity=0.20] (204.31, 50.86) circle (  2.13);

\path[fill=fillColor,fill opacity=0.20] (208.46, 43.71) circle (  2.13);

\path[fill=fillColor,fill opacity=0.20] (220.04, 56.63) circle (  2.13);

\path[fill=fillColor,fill opacity=0.20] (217.42, 54.73) circle (  2.13);

\path[fill=fillColor,fill opacity=0.20] (214.14, 46.21) circle (  2.13);

\path[fill=fillColor,fill opacity=0.20] (209.55, 56.71) circle (  2.13);

\path[fill=fillColor,fill opacity=0.20] (196.22, 68.25) circle (  2.13);

\path[fill=fillColor,fill opacity=0.20] (174.81, 90.90) circle (  2.13);

\path[fill=fillColor,fill opacity=0.20] (187.92, 50.08) circle (  2.13);

\path[fill=fillColor,fill opacity=0.20] (213.05, 48.10) circle (  2.13);

\path[fill=fillColor,fill opacity=0.20] (217.85, 56.28) circle (  2.13);

\path[fill=fillColor,fill opacity=0.20] (216.98, 55.42) circle (  2.13);

\path[fill=fillColor,fill opacity=0.20] (222.88, 56.54) circle (  2.13);

\path[fill=fillColor,fill opacity=0.20] (227.25, 47.41) circle (  2.13);

\path[fill=fillColor,fill opacity=0.20] (231.18, 41.13) circle (  2.13);

\path[fill=fillColor,fill opacity=0.20] (232.27, 48.79) circle (  2.13);

\path[fill=fillColor,fill opacity=0.20] (224.63, 52.66) circle (  2.13);

\path[fill=fillColor,fill opacity=0.20] (220.04, 44.83) circle (  2.13);

\path[fill=fillColor,fill opacity=0.20] (216.76, 45.95) circle (  2.13);

\path[fill=fillColor,fill opacity=0.20] (204.74, 57.14) circle (  2.13);

\path[fill=fillColor,fill opacity=0.20] (177.43, 88.06) circle (  2.13);

\path[fill=fillColor,fill opacity=0.20] (196.66, 77.90) circle (  2.13);

\path[fill=fillColor,fill opacity=0.20] (206.49, 58.78) circle (  2.13);

\path[fill=fillColor,fill opacity=0.20] (209.77, 48.10) circle (  2.13);

\path[fill=fillColor,fill opacity=0.20] (215.23, 45.52) circle (  2.13);

\path[fill=fillColor,fill opacity=0.20] (215.45, 53.87) circle (  2.13);

\path[fill=fillColor,fill opacity=0.20] (215.23, 64.89) circle (  2.13);

\path[fill=fillColor,fill opacity=0.20] (215.01, 63.77) circle (  2.13);

\path[fill=fillColor,fill opacity=0.20] (211.95, 50.60) circle (  2.13);

\path[fill=fillColor,fill opacity=0.20] (202.78, 56.54) circle (  2.13);

\path[fill=fillColor,fill opacity=0.20] (184.20, 76.52) circle (  2.13);

\path[fill=fillColor,fill opacity=0.20] (178.96, 72.56) circle (  2.13);

\path[fill=fillColor,fill opacity=0.20] (203.00, 59.12) circle (  2.13);

\path[fill=fillColor,fill opacity=0.20] (213.70, 60.76) circle (  2.13);

\path[fill=fillColor,fill opacity=0.20] (216.32, 56.63) circle (  2.13);

\path[fill=fillColor,fill opacity=0.20] (213.92, 58.43) circle (  2.13);

\path[fill=fillColor,fill opacity=0.20] (218.95, 57.23) circle (  2.13);

\path[fill=fillColor,fill opacity=0.20] (223.53, 48.70) circle (  2.13);

\path[fill=fillColor,fill opacity=0.20] (228.78, 42.33) circle (  2.13);

\path[fill=fillColor,fill opacity=0.20] (227.47, 44.57) circle (  2.13);

\path[fill=fillColor,fill opacity=0.20] (219.60, 46.64) circle (  2.13);

\path[fill=fillColor,fill opacity=0.20] (219.38, 54.21) circle (  2.13);

\path[fill=fillColor,fill opacity=0.20] (211.74, 59.30) circle (  2.13);

\path[fill=fillColor,fill opacity=0.20] (198.84, 66.18) circle (  2.13);

\path[fill=fillColor,fill opacity=0.20] (179.62, 83.41) circle (  2.13);

\path[fill=fillColor,fill opacity=0.20] (194.26, 72.21) circle (  2.13);

\path[fill=fillColor,fill opacity=0.20] (204.52, 64.98) circle (  2.13);

\path[fill=fillColor,fill opacity=0.20] (212.61, 55.33) circle (  2.13);

\path[fill=fillColor,fill opacity=0.20] (217.85, 49.74) circle (  2.13);

\path[fill=fillColor,fill opacity=0.20] (218.73, 56.71) circle (  2.13);

\path[fill=fillColor,fill opacity=0.20] (215.23, 63.00) circle (  2.13);

\path[fill=fillColor,fill opacity=0.20] (211.95, 65.24) circle (  2.13);

\path[fill=fillColor,fill opacity=0.20] (207.15, 63.69) circle (  2.13);

\path[fill=fillColor,fill opacity=0.20] (194.91, 61.45) circle (  2.13);

\path[fill=fillColor,fill opacity=0.20] (191.41, 71.35) circle (  2.13);

\path[fill=fillColor,fill opacity=0.20] (202.12, 69.28) circle (  2.13);

\path[fill=fillColor,fill opacity=0.20] (209.11, 51.98) circle (  2.13);

\path[fill=fillColor,fill opacity=0.20] (213.92, 46.55) circle (  2.13);

\path[fill=fillColor,fill opacity=0.20] (218.07, 53.53) circle (  2.13);

\path[fill=fillColor,fill opacity=0.20] (226.16, 50.08) circle (  2.13);

\path[fill=fillColor,fill opacity=0.20] (220.26, 46.03) circle (  2.13);

\path[fill=fillColor,fill opacity=0.20] (224.41, 48.36) circle (  2.13);

\path[fill=fillColor,fill opacity=0.20] (223.10, 48.62) circle (  2.13);

\path[fill=fillColor,fill opacity=0.20] (217.85, 56.02) circle (  2.13);

\path[fill=fillColor,fill opacity=0.20] (213.26, 69.46) circle (  2.13);

\path[fill=fillColor,fill opacity=0.20] (207.37, 68.68) circle (  2.13);

\path[fill=fillColor,fill opacity=0.20] (193.60, 80.48) circle (  2.13);

\path[fill=fillColor,fill opacity=0.20] (194.26, 71.35) circle (  2.13);

\path[fill=fillColor,fill opacity=0.20] (210.86, 62.31) circle (  2.13);

\path[fill=fillColor,fill opacity=0.20] (215.67, 60.33) circle (  2.13);

\path[fill=fillColor,fill opacity=0.20] (219.16, 57.40) circle (  2.13);

\path[fill=fillColor,fill opacity=0.20] (217.85, 59.38) circle (  2.13);

\path[fill=fillColor,fill opacity=0.20] (212.39, 58.09) circle (  2.13);

\path[fill=fillColor,fill opacity=0.20] (208.46, 56.63) circle (  2.13);

\path[fill=fillColor,fill opacity=0.20] (204.52, 67.22) circle (  2.13);

\path[fill=fillColor,fill opacity=0.20] (191.63, 71.52) circle (  2.13);

\path[fill=fillColor,fill opacity=0.20] (194.47, 72.47) circle (  2.13);

\path[fill=fillColor,fill opacity=0.20] (196.00, 69.80) circle (  2.13);

\path[fill=fillColor,fill opacity=0.20] (203.65, 62.74) circle (  2.13);

\path[fill=fillColor,fill opacity=0.20] (208.46, 45.09) circle (  2.13);

\path[fill=fillColor,fill opacity=0.20] (223.10, 48.53) circle (  2.13);

\path[fill=fillColor,fill opacity=0.20] (226.37, 45.69) circle (  2.13);

\path[fill=fillColor,fill opacity=0.20] (223.10, 43.28) circle (  2.13);

\path[fill=fillColor,fill opacity=0.20] (220.48, 54.99) circle (  2.13);

\path[fill=fillColor,fill opacity=0.20] (218.73, 59.90) circle (  2.13);

\path[fill=fillColor,fill opacity=0.20] (216.54, 60.50) circle (  2.13);

\path[fill=fillColor,fill opacity=0.20] (209.11, 64.12) circle (  2.13);

\path[fill=fillColor,fill opacity=0.20] (198.84, 69.71) circle (  2.13);

\path[fill=fillColor,fill opacity=0.20] (184.86, 97.19) circle (  2.13);

\path[fill=fillColor,fill opacity=0.20] (191.20, 76.60) circle (  2.13);

\path[fill=fillColor,fill opacity=0.20] (208.46, 61.10) circle (  2.13);

\path[fill=fillColor,fill opacity=0.20] (214.58, 64.03) circle (  2.13);

\path[fill=fillColor,fill opacity=0.20] (214.58, 62.22) circle (  2.13);

\path[fill=fillColor,fill opacity=0.20] (209.55, 56.80) circle (  2.13);

\path[fill=fillColor,fill opacity=0.20] (204.31, 56.28) circle (  2.13);

\path[fill=fillColor,fill opacity=0.20] (205.62, 57.57) circle (  2.13);

\path[fill=fillColor,fill opacity=0.20] (203.87, 64.20) circle (  2.13);

\path[fill=fillColor,fill opacity=0.20] (194.26, 75.40) circle (  2.13);

\path[fill=fillColor,fill opacity=0.20] (187.92, 66.36) circle (  2.13);

\path[fill=fillColor,fill opacity=0.20] (199.06, 61.53) circle (  2.13);

\path[fill=fillColor,fill opacity=0.20] (209.11, 62.31) circle (  2.13);

\path[fill=fillColor,fill opacity=0.20] (211.74, 63.51) circle (  2.13);

\path[fill=fillColor,fill opacity=0.20] (213.26, 54.21) circle (  2.13);

\path[fill=fillColor,fill opacity=0.20] (223.53, 49.13) circle (  2.13);

\path[fill=fillColor,fill opacity=0.20] (228.12, 48.70) circle (  2.13);

\path[fill=fillColor,fill opacity=0.20] (225.28, 50.77) circle (  2.13);

\path[fill=fillColor,fill opacity=0.20] (222.22, 58.86) circle (  2.13);

\path[fill=fillColor,fill opacity=0.20] (220.69, 65.67) circle (  2.13);

\path[fill=fillColor,fill opacity=0.20] (222.22, 58.86) circle (  2.13);

\path[fill=fillColor,fill opacity=0.20] (206.49, 55.08) circle (  2.13);

\path[fill=fillColor,fill opacity=0.20] (190.32, 72.13) circle (  2.13);

\path[fill=fillColor,fill opacity=0.20] (184.42, 76.52) circle (  2.13);

\path[fill=fillColor,fill opacity=0.20] (201.03, 65.15) circle (  2.13);

\path[fill=fillColor,fill opacity=0.20] (210.86, 66.27) circle (  2.13);

\path[fill=fillColor,fill opacity=0.20] (210.42, 59.55) circle (  2.13);

\path[fill=fillColor,fill opacity=0.20] (209.11, 48.88) circle (  2.13);

\path[fill=fillColor,fill opacity=0.20] (209.11, 57.14) circle (  2.13);

\path[fill=fillColor,fill opacity=0.20] (203.43, 64.03) circle (  2.13);

\path[fill=fillColor,fill opacity=0.20] (201.90, 59.47) circle (  2.13);

\path[fill=fillColor,fill opacity=0.20] (199.06, 60.93) circle (  2.13);

\path[fill=fillColor,fill opacity=0.20] (183.55, 77.55) circle (  2.13);

\path[fill=fillColor,fill opacity=0.20] (185.08, 68.77) circle (  2.13);

\path[fill=fillColor,fill opacity=0.20] (197.31, 53.70) circle (  2.13);

\path[fill=fillColor,fill opacity=0.20] (210.64, 55.08) circle (  2.13);

\path[fill=fillColor,fill opacity=0.20] (214.58, 64.03) circle (  2.13);

\path[fill=fillColor,fill opacity=0.20] (213.92, 70.66) circle (  2.13);

\path[fill=fillColor,fill opacity=0.20] (213.70, 64.72) circle (  2.13);

\path[fill=fillColor,fill opacity=0.20] (223.97, 53.53) circle (  2.13);

\path[fill=fillColor,fill opacity=0.20] (231.84, 54.56) circle (  2.13);

\path[fill=fillColor,fill opacity=0.20] (231.40, 63.08) circle (  2.13);

\path[fill=fillColor,fill opacity=0.20] (231.84, 64.12) circle (  2.13);

\path[fill=fillColor,fill opacity=0.20] (229.22, 60.93) circle (  2.13);

\path[fill=fillColor,fill opacity=0.20] (220.69, 60.50) circle (  2.13);

\path[fill=fillColor,fill opacity=0.20] (201.68, 64.38) circle (  2.13);

\path[fill=fillColor,fill opacity=0.20] (191.63, 76.78) circle (  2.13);

\path[fill=fillColor,fill opacity=0.20] (204.09, 67.65) circle (  2.13);

\path[fill=fillColor,fill opacity=0.20] (208.89, 55.94) circle (  2.13);

\path[fill=fillColor,fill opacity=0.20] (209.99, 50.86) circle (  2.13);

\path[fill=fillColor,fill opacity=0.20] (207.58, 59.47) circle (  2.13);

\path[fill=fillColor,fill opacity=0.20] (201.90, 60.93) circle (  2.13);

\path[fill=fillColor,fill opacity=0.20] (201.90, 51.54) circle (  2.13);

\path[fill=fillColor,fill opacity=0.20] (204.96, 49.82) circle (  2.13);

\path[fill=fillColor,fill opacity=0.20] (198.41, 57.49) circle (  2.13);

\path[fill=fillColor,fill opacity=0.20] (177.87, 78.07) circle (  2.13);

\path[fill=fillColor,fill opacity=0.20] (185.52, 76.78) circle (  2.13);

\path[fill=fillColor,fill opacity=0.20] (197.10, 60.67) circle (  2.13);

\path[fill=fillColor,fill opacity=0.20] (203.21, 54.47) circle (  2.13);

\path[fill=fillColor,fill opacity=0.20] (209.99, 59.21) circle (  2.13);

\path[fill=fillColor,fill opacity=0.20] (215.45, 57.83) circle (  2.13);

\path[fill=fillColor,fill opacity=0.20] (217.42, 54.64) circle (  2.13);

\path[fill=fillColor,fill opacity=0.20] (218.51, 59.04) circle (  2.13);

\path[fill=fillColor,fill opacity=0.20] (224.63, 56.80) circle (  2.13);

\path[fill=fillColor,fill opacity=0.20] (230.53, 55.68) circle (  2.13);

\path[fill=fillColor,fill opacity=0.20] (227.90, 62.57) circle (  2.13);

\path[fill=fillColor,fill opacity=0.20] (232.49, 61.71) circle (  2.13);

\path[fill=fillColor,fill opacity=0.20] (222.88, 56.71) circle (  2.13);

\path[fill=fillColor,fill opacity=0.20] (203.87, 65.24) circle (  2.13);

\path[fill=fillColor,fill opacity=0.20] (198.63, 74.71) circle (  2.13);

\path[fill=fillColor,fill opacity=0.20] (207.58, 65.06) circle (  2.13);

\path[fill=fillColor,fill opacity=0.20] (208.68, 60.59) circle (  2.13);

\path[fill=fillColor,fill opacity=0.20] (207.37, 56.37) circle (  2.13);

\path[fill=fillColor,fill opacity=0.20] (207.37, 52.92) circle (  2.13);

\path[fill=fillColor,fill opacity=0.20] (203.65, 52.58) circle (  2.13);

\path[fill=fillColor,fill opacity=0.20] (205.40, 54.56) circle (  2.13);

\path[fill=fillColor,fill opacity=0.20] (202.12, 56.97) circle (  2.13);

\path[fill=fillColor,fill opacity=0.20] (190.10, 58.52) circle (  2.13);

\path[fill=fillColor,fill opacity=0.20] (176.56, 72.99) circle (  2.13);

\path[fill=fillColor,fill opacity=0.20] (186.61, 88.83) circle (  2.13);

\path[fill=fillColor,fill opacity=0.20] (193.16, 61.45) circle (  2.13);

\path[fill=fillColor,fill opacity=0.20] (201.90, 60.16) circle (  2.13);

\path[fill=fillColor,fill opacity=0.20] (202.56, 69.37) circle (  2.13);

\path[fill=fillColor,fill opacity=0.20] (208.68, 63.95) circle (  2.13);

\path[fill=fillColor,fill opacity=0.20] (215.89, 50.94) circle (  2.13);

\path[fill=fillColor,fill opacity=0.20] (225.72, 43.11) circle (  2.13);

\path[fill=fillColor,fill opacity=0.20] (223.75, 51.80) circle (  2.13);

\path[fill=fillColor,fill opacity=0.20] (227.90, 57.83) circle (  2.13);

\path[fill=fillColor,fill opacity=0.20] (222.66, 53.09) circle (  2.13);

\path[fill=fillColor,fill opacity=0.20] (219.16, 49.31) circle (  2.13);

\path[fill=fillColor,fill opacity=0.20] (224.41, 53.09) circle (  2.13);

\path[fill=fillColor,fill opacity=0.20] (210.64, 64.29) circle (  2.13);

\path[fill=fillColor,fill opacity=0.20] (195.35, 70.58) circle (  2.13);

\path[fill=fillColor,fill opacity=0.20] (203.21, 75.57) circle (  2.13);

\path[fill=fillColor,fill opacity=0.20] (208.02, 63.95) circle (  2.13);

\path[fill=fillColor,fill opacity=0.20] (208.68, 48.27) circle (  2.13);

\path[fill=fillColor,fill opacity=0.20] (209.11, 45.52) circle (  2.13);

\path[fill=fillColor,fill opacity=0.20] (207.80, 54.99) circle (  2.13);

\path[fill=fillColor,fill opacity=0.20] (206.27, 61.02) circle (  2.13);

\path[fill=fillColor,fill opacity=0.20] (202.34, 59.73) circle (  2.13);

\path[fill=fillColor,fill opacity=0.20] (195.78, 55.42) circle (  2.13);

\path[fill=fillColor,fill opacity=0.20] (191.85, 54.56) circle (  2.13);

\path[fill=fillColor,fill opacity=0.20] (180.49, 65.32) circle (  2.13);

\path[fill=fillColor,fill opacity=0.20] (182.02, 89.26) circle (  2.13);

\path[fill=fillColor,fill opacity=0.20] (194.91, 79.62) circle (  2.13);

\path[fill=fillColor,fill opacity=0.20] (203.00, 67.39) circle (  2.13);

\path[fill=fillColor,fill opacity=0.20] (201.47, 63.60) circle (  2.13);

\path[fill=fillColor,fill opacity=0.20] (200.15, 70.06) circle (  2.13);

\path[fill=fillColor,fill opacity=0.20] (210.21, 64.55) circle (  2.13);

\path[fill=fillColor,fill opacity=0.20] (218.29, 52.66) circle (  2.13);

\path[fill=fillColor,fill opacity=0.20] (225.06, 52.92) circle (  2.13);

\path[fill=fillColor,fill opacity=0.20] (225.28, 60.76) circle (  2.13);

\path[fill=fillColor,fill opacity=0.20] (225.06, 60.67) circle (  2.13);

\path[fill=fillColor,fill opacity=0.20] (270.51, 57.92) circle (  2.13);

\path[fill=fillColor,fill opacity=0.20] (212.61, 50.51) circle (  2.13);

\path[fill=fillColor,fill opacity=0.20] (209.33, 51.72) circle (  2.13);

\path[fill=fillColor,fill opacity=0.20] (200.37, 69.20) circle (  2.13);

\path[fill=fillColor,fill opacity=0.20] (190.32, 80.57) circle (  2.13);

\path[fill=fillColor,fill opacity=0.20] (201.03, 67.48) circle (  2.13);

\path[fill=fillColor,fill opacity=0.20] (209.99, 50.94) circle (  2.13);

\path[fill=fillColor,fill opacity=0.20] (208.02, 47.41) circle (  2.13);

\path[fill=fillColor,fill opacity=0.20] (206.27, 58.18) circle (  2.13);

\path[fill=fillColor,fill opacity=0.20] (206.27, 61.71) circle (  2.13);

\path[fill=fillColor,fill opacity=0.20] (201.47, 57.31) circle (  2.13);

\path[fill=fillColor,fill opacity=0.20] (204.52, 55.68) circle (  2.13);

\path[fill=fillColor,fill opacity=0.20] (199.28, 61.19) circle (  2.13);

\path[fill=fillColor,fill opacity=0.20] (193.38, 63.60) circle (  2.13);

\path[fill=fillColor,fill opacity=0.20] (181.36, 64.20) circle (  2.13);

\path[fill=fillColor,fill opacity=0.20] (181.36, 81.43) circle (  2.13);

\path[fill=fillColor,fill opacity=0.20] (195.57, 60.85) circle (  2.13);

\path[fill=fillColor,fill opacity=0.20] (204.31, 71.95) circle (  2.13);

\path[fill=fillColor,fill opacity=0.20] (208.68, 73.93) circle (  2.13);

\path[fill=fillColor,fill opacity=0.20] (214.14, 59.30) circle (  2.13);

\path[fill=fillColor,fill opacity=0.20] (208.46, 56.37) circle (  2.13);

\path[fill=fillColor,fill opacity=0.20] (213.70, 63.95) circle (  2.13);

\path[fill=fillColor,fill opacity=0.20] (216.76, 61.45) circle (  2.13);

\path[fill=fillColor,fill opacity=0.20] (218.29, 60.41) circle (  2.13);

\path[fill=fillColor,fill opacity=0.20] (219.82, 63.43) circle (  2.13);

\path[fill=fillColor,fill opacity=0.20] (214.79, 65.24) circle (  2.13);

\path[fill=fillColor,fill opacity=0.20] (211.30, 68.25) circle (  2.13);

\path[fill=fillColor,fill opacity=0.20] (208.89, 68.16) circle (  2.13);

\path[fill=fillColor,fill opacity=0.20] (198.63, 61.53) circle (  2.13);

\path[fill=fillColor,fill opacity=0.20] (194.69, 85.99) circle (  2.13);

\path[fill=fillColor,fill opacity=0.20] (208.02, 62.65) circle (  2.13);

\path[fill=fillColor,fill opacity=0.20] (209.55, 53.35) circle (  2.13);

\path[fill=fillColor,fill opacity=0.20] (206.71, 59.38) circle (  2.13);

\path[fill=fillColor,fill opacity=0.20] (200.81, 61.28) circle (  2.13);

\path[fill=fillColor,fill opacity=0.20] (202.78, 55.25) circle (  2.13);

\path[fill=fillColor,fill opacity=0.20] (205.18, 57.57) circle (  2.13);

\path[fill=fillColor,fill opacity=0.20] (201.25, 65.93) circle (  2.13);

\path[fill=fillColor,fill opacity=0.20] (193.60, 69.54) circle (  2.13);

\path[fill=fillColor,fill opacity=0.20] (191.20, 62.74) circle (  2.13);

\path[fill=fillColor,fill opacity=0.20] (188.57, 61.10) circle (  2.13);

\path[fill=fillColor,fill opacity=0.20] (177.65, 86.59) circle (  2.13);

\path[fill=fillColor,fill opacity=0.20] (180.27, 79.79) circle (  2.13);

\path[fill=fillColor,fill opacity=0.20] (191.41, 57.23) circle (  2.13);

\path[fill=fillColor,fill opacity=0.20] (201.90, 45.34) circle (  2.13);

\path[fill=fillColor,fill opacity=0.20] (204.52, 61.88) circle (  2.13);

\path[fill=fillColor,fill opacity=0.20] (205.84, 70.58) circle (  2.13);

\path[fill=fillColor,fill opacity=0.20] (217.63, 60.76) circle (  2.13);

\path[fill=fillColor,fill opacity=0.20] (215.45, 54.99) circle (  2.13);

\path[fill=fillColor,fill opacity=0.20] (216.32, 65.50) circle (  2.13);

\path[fill=fillColor,fill opacity=0.20] (212.17, 66.70) circle (  2.13);

\path[fill=fillColor,fill opacity=0.20] (211.52, 62.91) circle (  2.13);

\path[fill=fillColor,fill opacity=0.20] (220.91, 61.79) circle (  2.13);

\path[fill=fillColor,fill opacity=0.20] (210.86, 58.61) circle (  2.13);

\path[fill=fillColor,fill opacity=0.20] (202.56, 66.96) circle (  2.13);

\path[fill=fillColor,fill opacity=0.20] (200.37, 78.84) circle (  2.13);

\path[fill=fillColor,fill opacity=0.20] (186.17, 81.34) circle (  2.13);

\path[fill=fillColor,fill opacity=0.20] (185.73, 95.29) circle (  2.13);

\path[fill=fillColor,fill opacity=0.20] (198.63, 92.36) circle (  2.13);

\path[fill=fillColor,fill opacity=0.20] (204.09, 70.40) circle (  2.13);

\path[fill=fillColor,fill opacity=0.20] (204.09, 55.25) circle (  2.13);

\path[fill=fillColor,fill opacity=0.20] (205.84, 52.84) circle (  2.13);

\path[fill=fillColor,fill opacity=0.20] (204.96, 53.09) circle (  2.13);

\path[fill=fillColor,fill opacity=0.20] (202.56, 53.27) circle (  2.13);

\path[fill=fillColor,fill opacity=0.20] (196.88, 60.59) circle (  2.13);

\path[fill=fillColor,fill opacity=0.20] (198.41, 69.20) circle (  2.13);

\path[fill=fillColor,fill opacity=0.20] (195.35, 74.62) circle (  2.13);

\path[fill=fillColor,fill opacity=0.20] (197.53, 66.44) circle (  2.13);

\path[fill=fillColor,fill opacity=0.20] (191.85, 53.53) circle (  2.13);

\path[fill=fillColor,fill opacity=0.20] (180.93, 64.03) circle (  2.13);

\path[fill=fillColor,fill opacity=0.20] (183.55, 69.63) circle (  2.13);

\path[fill=fillColor,fill opacity=0.20] (197.75, 59.04) circle (  2.13);

\path[fill=fillColor,fill opacity=0.20] (199.28, 53.61) circle (  2.13);

\path[fill=fillColor,fill opacity=0.20] (195.78, 51.89) circle (  2.13);

\path[fill=fillColor,fill opacity=0.20] (203.87, 58.43) circle (  2.13);

\path[fill=fillColor,fill opacity=0.20] (209.33, 63.77) circle (  2.13);

\path[fill=fillColor,fill opacity=0.20] (212.83, 60.93) circle (  2.13);

\path[fill=fillColor,fill opacity=0.20] (216.76, 59.81) circle (  2.13);

\path[fill=fillColor,fill opacity=0.20] (224.19, 62.05) circle (  2.13);

\path[fill=fillColor,fill opacity=0.20] (207.58, 64.29) circle (  2.13);

\path[fill=fillColor,fill opacity=0.20] (207.80, 72.13) circle (  2.13);

\path[fill=fillColor,fill opacity=0.20] (208.02, 77.29) circle (  2.13);

\path[fill=fillColor,fill opacity=0.20] (201.03, 69.46) circle (  2.13);

\path[fill=fillColor,fill opacity=0.20] (192.29, 74.45) circle (  2.13);

\path[fill=fillColor,fill opacity=0.20] (191.20, 85.99) circle (  2.13);

\path[fill=fillColor,fill opacity=0.20] (199.94, 68.51) circle (  2.13);

\path[fill=fillColor,fill opacity=0.20] (208.89, 58.18) circle (  2.13);

\path[fill=fillColor,fill opacity=0.20] (205.18, 53.09) circle (  2.13);

\path[fill=fillColor,fill opacity=0.20] (211.52, 53.70) circle (  2.13);

\path[fill=fillColor,fill opacity=0.20] (203.00, 64.12) circle (  2.13);

\path[fill=fillColor,fill opacity=0.20] (203.87, 83.06) circle (  2.13);

\path[fill=fillColor,fill opacity=0.20] (204.52, 79.45) circle (  2.13);

\path[fill=fillColor,fill opacity=0.20] (201.90, 50.25) circle (  2.13);

\path[fill=fillColor,fill opacity=0.20] (187.70, 64.38) circle (  2.13);

\path[fill=fillColor,fill opacity=0.20] (182.46, 70.32) circle (  2.13);

\path[fill=fillColor,fill opacity=0.20] (192.94, 64.12) circle (  2.13);

\path[fill=fillColor,fill opacity=0.20] (201.25, 68.68) circle (  2.13);

\path[fill=fillColor,fill opacity=0.20] (203.21, 67.91) circle (  2.13);

\path[fill=fillColor,fill opacity=0.20] (203.21, 62.83) circle (  2.13);

\path[fill=fillColor,fill opacity=0.20] (205.40, 61.02) circle (  2.13);

\path[fill=fillColor,fill opacity=0.20] (204.74, 62.31) circle (  2.13);

\path[fill=fillColor,fill opacity=0.20] (205.40, 61.71) circle (  2.13);

\path[fill=fillColor,fill opacity=0.20] (204.96, 59.90) circle (  2.13);

\path[fill=fillColor,fill opacity=0.20] (198.41, 66.10) circle (  2.13);

\path[fill=fillColor,fill opacity=0.20] (197.53, 78.67) circle (  2.13);

\path[fill=fillColor,fill opacity=0.20] (197.31, 78.15) circle (  2.13);

\path[fill=fillColor,fill opacity=0.20] (211.74, 53.44) circle (  2.13);

\path[fill=fillColor,fill opacity=0.20] (208.24, 44.83) circle (  2.13);

\path[fill=fillColor,fill opacity=0.20] (207.37, 55.85) circle (  2.13);

\path[fill=fillColor,fill opacity=0.20] (201.47, 80.57) circle (  2.13);

\path[fill=fillColor,fill opacity=0.20] (204.31, 78.41) circle (  2.13);

\path[fill=fillColor,fill opacity=0.20] (208.46, 50.86) circle (  2.13);

\path[fill=fillColor,fill opacity=0.20] (209.99, 48.53) circle (  2.13);

\path[fill=fillColor,fill opacity=0.20] (203.00, 69.63) circle (  2.13);

\path[fill=fillColor,fill opacity=0.20] (192.94, 74.19) circle (  2.13);

\path[fill=fillColor,fill opacity=0.20] (182.02, 78.50) circle (  2.13);

\path[fill=fillColor,fill opacity=0.20] (185.08, 76.09) circle (  2.13);

\path[fill=fillColor,fill opacity=0.20] (192.51, 66.44) circle (  2.13);

\path[fill=fillColor,fill opacity=0.20] (199.94, 63.69) circle (  2.13);

\path[fill=fillColor,fill opacity=0.20] (208.46, 63.60) circle (  2.13);

\path[fill=fillColor,fill opacity=0.20] (209.99, 65.84) circle (  2.13);

\path[fill=fillColor,fill opacity=0.20] (207.80, 68.85) circle (  2.13);

\path[fill=fillColor,fill opacity=0.20] (201.47, 73.59) circle (  2.13);

\path[fill=fillColor,fill opacity=0.20] (197.75, 78.24) circle (  2.13);

\path[fill=fillColor,fill opacity=0.20] (194.26, 82.46) circle (  2.13);

\path[fill=fillColor,fill opacity=0.20] (190.10, 84.01) circle (  2.13);

\path[fill=fillColor,fill opacity=0.20] (185.95, 88.06) circle (  2.13);

\path[fill=fillColor,fill opacity=0.20] (197.75, 83.06) circle (  2.13);

\path[fill=fillColor,fill opacity=0.20] (207.37, 60.07) circle (  2.13);

\path[fill=fillColor,fill opacity=0.20] (211.95, 57.83) circle (  2.13);

\path[fill=fillColor,fill opacity=0.20] (208.46, 73.85) circle (  2.13);

\path[fill=fillColor,fill opacity=0.20] (206.93, 67.39) circle (  2.13);

\path[fill=fillColor,fill opacity=0.20] (209.33, 46.38) circle (  2.13);

\path[fill=fillColor,fill opacity=0.20] (208.02, 56.11) circle (  2.13);

\path[fill=fillColor,fill opacity=0.20] (205.18, 75.57) circle (  2.13);

\path[fill=fillColor,fill opacity=0.20] (195.35, 66.18) circle (  2.13);

\path[fill=fillColor,fill opacity=0.20] (197.31, 53.35) circle (  2.13);

\path[fill=fillColor,fill opacity=0.20] (186.83, 93.65) circle (  2.13);

\path[fill=fillColor,fill opacity=0.20] (194.04, 71.87) circle (  2.13);

\path[fill=fillColor,fill opacity=0.20] (200.59, 62.65) circle (  2.13);

\path[fill=fillColor,fill opacity=0.20] (201.03, 58.18) circle (  2.13);

\path[fill=fillColor,fill opacity=0.20] (208.24, 54.56) circle (  2.13);

\path[fill=fillColor,fill opacity=0.20] (204.09, 65.41) circle (  2.13);

\path[fill=fillColor,fill opacity=0.20] (192.73, 80.31) circle (  2.13);

\path[fill=fillColor,fill opacity=0.20] (188.79, 93.74) circle (  2.13);

\path[fill=fillColor,fill opacity=0.20] (185.08, 97.96) circle (  2.13);

\path[fill=fillColor,fill opacity=0.20] (199.72, 81.34) circle (  2.13);

\path[fill=fillColor,fill opacity=0.20] (209.11, 72.73) circle (  2.13);

\path[fill=fillColor,fill opacity=0.20] (208.46, 57.66) circle (  2.13);

\path[fill=fillColor,fill opacity=0.20] (207.58, 45.78) circle (  2.13);

\path[fill=fillColor,fill opacity=0.20] (207.80, 53.35) circle (  2.13);

\path[fill=fillColor,fill opacity=0.20] (208.68, 63.08) circle (  2.13);

\path[fill=fillColor,fill opacity=0.20] (203.00, 63.86) circle (  2.13);

\path[fill=fillColor,fill opacity=0.20] (201.25, 67.22) circle (  2.13);

\path[fill=fillColor,fill opacity=0.20] (202.78, 64.46) circle (  2.13);

\path[fill=fillColor,fill opacity=0.20] (196.88, 45.52) circle (  2.13);

\path[fill=fillColor,fill opacity=0.20] (192.94, 40.26) circle (  2.13);

\path[fill=fillColor,fill opacity=0.20] (187.04, 63.86) circle (  2.13);

\path[fill=fillColor,fill opacity=0.20] (181.36, 85.22) circle (  2.13);

\path[fill=fillColor,fill opacity=0.20] (183.11, 94.17) circle (  2.13);

\path[fill=fillColor,fill opacity=0.20] (191.20, 68.08) circle (  2.13);

\path[fill=fillColor,fill opacity=0.20] (194.26, 51.98) circle (  2.13);

\path[fill=fillColor,fill opacity=0.20] (200.37, 63.43) circle (  2.13);

\path[fill=fillColor,fill opacity=0.20] (196.44, 69.63) circle (  2.13);

\path[fill=fillColor,fill opacity=0.20] (189.45, 76.09) circle (  2.13);

\path[fill=fillColor,fill opacity=0.20] (183.33, 92.36) circle (  2.13);

\path[fill=fillColor,fill opacity=0.20] (175.90,110.53) circle (  2.13);

\path[fill=fillColor,fill opacity=0.20] (179.62,115.96) circle (  2.13);

\path[fill=fillColor,fill opacity=0.20] (192.07, 91.93) circle (  2.13);

\path[fill=fillColor,fill opacity=0.20] (203.21, 64.55) circle (  2.13);

\path[fill=fillColor,fill opacity=0.20] (206.49, 55.16) circle (  2.13);

\path[fill=fillColor,fill opacity=0.20] (209.77, 52.49) circle (  2.13);

\path[fill=fillColor,fill opacity=0.20] (207.15, 49.22) circle (  2.13);

\path[fill=fillColor,fill opacity=0.20] (206.71, 55.33) circle (  2.13);

\path[fill=fillColor,fill opacity=0.20] (203.00, 72.30) circle (  2.13);

\path[fill=fillColor,fill opacity=0.20] (200.37, 78.67) circle (  2.13);

\path[fill=fillColor,fill opacity=0.20] (199.94, 65.06) circle (  2.13);

\path[fill=fillColor,fill opacity=0.20] (198.19, 56.37) circle (  2.13);

\path[fill=fillColor,fill opacity=0.20] (192.94, 64.89) circle (  2.13);

\path[fill=fillColor,fill opacity=0.20] (193.38, 64.98) circle (  2.13);

\path[fill=fillColor,fill opacity=0.20] (190.10, 59.90) circle (  2.13);

\path[fill=fillColor,fill opacity=0.20] (186.83, 70.83) circle (  2.13);

\path[fill=fillColor,fill opacity=0.20] (182.02, 92.45) circle (  2.13);

\path[fill=fillColor,fill opacity=0.20] (186.39, 91.50) circle (  2.13);

\path[fill=fillColor,fill opacity=0.20] (192.07, 81.77) circle (  2.13);

\path[fill=fillColor,fill opacity=0.20] (194.91, 71.01) circle (  2.13);

\path[fill=fillColor,fill opacity=0.20] (194.26, 66.61) circle (  2.13);

\path[fill=fillColor,fill opacity=0.20] (198.63, 65.75) circle (  2.13);

\path[fill=fillColor,fill opacity=0.20] (220.48, 58.35) circle (  2.13);

\path[fill=fillColor,fill opacity=0.20] (182.24, 76.60) circle (  2.13);

\path[fill=fillColor,fill opacity=0.20] (175.68,111.57) circle (  2.13);

\path[fill=fillColor,fill opacity=0.20] (184.86, 88.23) circle (  2.13);

\path[fill=fillColor,fill opacity=0.20] (203.21, 71.01) circle (  2.13);

\path[fill=fillColor,fill opacity=0.20] (209.99, 57.75) circle (  2.13);

\path[fill=fillColor,fill opacity=0.20] (206.05, 48.44) circle (  2.13);

\path[fill=fillColor,fill opacity=0.20] (205.18, 53.78) circle (  2.13);

\path[fill=fillColor,fill opacity=0.20] (206.27, 68.60) circle (  2.13);

\path[fill=fillColor,fill opacity=0.20] (205.40, 77.81) circle (  2.13);

\path[fill=fillColor,fill opacity=0.20] (199.94, 76.95) circle (  2.13);

\path[fill=fillColor,fill opacity=0.20] (199.94, 81.77) circle (  2.13);

\path[fill=fillColor,fill opacity=0.20] (197.97, 80.48) circle (  2.13);

\path[fill=fillColor,fill opacity=0.20] (197.31, 61.79) circle (  2.13);

\path[fill=fillColor,fill opacity=0.20] (196.00, 52.32) circle (  2.13);

\path[fill=fillColor,fill opacity=0.20] (196.22, 64.20) circle (  2.13);

\path[fill=fillColor,fill opacity=0.20] (193.38, 70.23) circle (  2.13);

\path[fill=fillColor,fill opacity=0.20] (190.10, 61.96) circle (  2.13);

\path[fill=fillColor,fill opacity=0.20] (187.70, 62.22) circle (  2.13);

\path[fill=fillColor,fill opacity=0.20] (183.33, 77.47) circle (  2.13);

\path[fill=fillColor,fill opacity=0.20] (178.74, 95.55) circle (  2.13);

\path[fill=fillColor,fill opacity=0.20] (178.09,101.41) circle (  2.13);

\path[fill=fillColor,fill opacity=0.20] (178.74, 99.94) circle (  2.13);

\path[fill=fillColor,fill opacity=0.20] (181.36, 90.64) circle (  2.13);

\path[fill=fillColor,fill opacity=0.20] (180.93, 95.29) circle (  2.13);

\path[fill=fillColor,fill opacity=0.20] (179.18, 97.44) circle (  2.13);

\path[fill=fillColor,fill opacity=0.20] (188.79, 80.48) circle (  2.13);

\path[fill=fillColor,fill opacity=0.20] (189.45, 73.93) circle (  2.13);

\path[fill=fillColor,fill opacity=0.20] (192.73, 68.77) circle (  2.13);

\path[fill=fillColor,fill opacity=0.20] (197.31, 74.54) circle (  2.13);

\path[fill=fillColor,fill opacity=0.20] (199.94, 73.33) circle (  2.13);

\path[fill=fillColor,fill opacity=0.20] (198.63, 60.76) circle (  2.13);

\path[fill=fillColor,fill opacity=0.20] (199.72, 60.33) circle (  2.13);

\path[fill=fillColor,fill opacity=0.20] (198.84, 55.08) circle (  2.13);

\path[fill=fillColor,fill opacity=0.20] (192.07, 55.51) circle (  2.13);

\path[fill=fillColor,fill opacity=0.20] (182.02, 83.75) circle (  2.13);

\path[fill=fillColor,fill opacity=0.20] (195.57, 74.80) circle (  2.13);

\path[fill=fillColor,fill opacity=0.20] (202.12, 68.94) circle (  2.13);

\path[fill=fillColor,fill opacity=0.20] (202.56, 72.73) circle (  2.13);

\path[fill=fillColor,fill opacity=0.20] (201.25, 71.78) circle (  2.13);

\path[fill=fillColor,fill opacity=0.20] (201.68, 67.56) circle (  2.13);

\path[fill=fillColor,fill opacity=0.20] (204.09, 68.08) circle (  2.13);

\path[fill=fillColor,fill opacity=0.20] (208.02, 72.64) circle (  2.13);

\path[fill=fillColor,fill opacity=0.20] (206.71, 67.82) circle (  2.13);

\path[fill=fillColor,fill opacity=0.20] (200.59, 57.66) circle (  2.13);

\path[fill=fillColor,fill opacity=0.20] (199.50, 56.20) circle (  2.13);

\path[fill=fillColor,fill opacity=0.20] (198.19, 65.84) circle (  2.13);

\path[fill=fillColor,fill opacity=0.20] (195.78, 69.03) circle (  2.13);

\path[fill=fillColor,fill opacity=0.20] (195.35, 60.41) circle (  2.13);

\path[fill=fillColor,fill opacity=0.20] (194.91, 57.31) circle (  2.13);

\path[fill=fillColor,fill opacity=0.20] (194.47, 67.13) circle (  2.13);

\path[fill=fillColor,fill opacity=0.20] (195.35, 73.25) circle (  2.13);

\path[fill=fillColor,fill opacity=0.20] (195.78, 71.70) circle (  2.13);

\path[fill=fillColor,fill opacity=0.20] (194.91, 61.02) circle (  2.13);

\path[fill=fillColor,fill opacity=0.20] (196.22, 59.98) circle (  2.13);

\path[fill=fillColor,fill opacity=0.20] (195.13, 69.63) circle (  2.13);

\path[fill=fillColor,fill opacity=0.20] (197.31, 69.11) circle (  2.13);

\path[fill=fillColor,fill opacity=0.20] (195.13, 66.53) circle (  2.13);

\path[fill=fillColor,fill opacity=0.20] (195.35, 68.85) circle (  2.13);

\path[fill=fillColor,fill opacity=0.20] (195.78, 61.79) circle (  2.13);

\path[fill=fillColor,fill opacity=0.20] (195.78, 57.49) circle (  2.13);

\path[fill=fillColor,fill opacity=0.20] (195.35, 66.61) circle (  2.13);

\path[fill=fillColor,fill opacity=0.20] (200.81, 63.43) circle (  2.13);

\path[fill=fillColor,fill opacity=0.20] (196.88, 53.96) circle (  2.13);

\path[fill=fillColor,fill opacity=0.20] (200.59, 55.16) circle (  2.13);

\path[fill=fillColor,fill opacity=0.20] (200.59, 50.34) circle (  2.13);

\path[fill=fillColor,fill opacity=0.20] (214.14, 47.93) circle (  2.13);

\path[fill=fillColor,fill opacity=0.20] (201.68, 58.52) circle (  2.13);

\path[fill=fillColor,fill opacity=0.20] (197.97, 65.58) circle (  2.13);

\path[fill=fillColor,fill opacity=0.20] (194.04, 70.58) circle (  2.13);

\path[fill=fillColor,fill opacity=0.20] (186.17, 80.13) circle (  2.13);

\path[fill=fillColor,fill opacity=0.20] (175.46,102.78) circle (  2.13);

\path[fill=fillColor,fill opacity=0.20] (193.60, 75.05) circle (  2.13);

\path[fill=fillColor,fill opacity=0.20] (198.41, 72.56) circle (  2.13);

\path[fill=fillColor,fill opacity=0.20] (206.49, 65.75) circle (  2.13);

\path[fill=fillColor,fill opacity=0.20] (208.24, 65.75) circle (  2.13);

\path[fill=fillColor,fill opacity=0.20] (208.24, 64.63) circle (  2.13);

\path[fill=fillColor,fill opacity=0.20] (205.62, 62.31) circle (  2.13);

\path[fill=fillColor,fill opacity=0.20] (203.00, 67.13) circle (  2.13);

\path[fill=fillColor,fill opacity=0.20] (198.41, 67.82) circle (  2.13);

\path[fill=fillColor,fill opacity=0.20] (201.47, 60.59) circle (  2.13);

\path[fill=fillColor,fill opacity=0.20] (203.43, 61.19) circle (  2.13);

\path[fill=fillColor,fill opacity=0.20] (199.50, 67.30) circle (  2.13);

\path[fill=fillColor,fill opacity=0.20] (201.25, 69.71) circle (  2.13);

\path[fill=fillColor,fill opacity=0.20] (203.21, 72.73) circle (  2.13);

\path[fill=fillColor,fill opacity=0.20] (198.41, 63.08) circle (  2.13);

\path[fill=fillColor,fill opacity=0.20] (196.22, 52.58) circle (  2.13);

\path[fill=fillColor,fill opacity=0.20] (198.19, 59.04) circle (  2.13);

\path[fill=fillColor,fill opacity=0.20] (199.28, 64.63) circle (  2.13);

\path[fill=fillColor,fill opacity=0.20] (201.68, 58.78) circle (  2.13);

\path[fill=fillColor,fill opacity=0.20] (201.47, 61.62) circle (  2.13);

\path[fill=fillColor,fill opacity=0.20] (199.06, 62.74) circle (  2.13);

\path[fill=fillColor,fill opacity=0.20] (200.37, 56.02) circle (  2.13);

\path[fill=fillColor,fill opacity=0.20] (201.68, 56.97) circle (  2.13);

\path[fill=fillColor,fill opacity=0.20] (203.43, 56.20) circle (  2.13);

\path[fill=fillColor,fill opacity=0.20] (204.31, 50.25) circle (  2.13);

\path[fill=fillColor,fill opacity=0.20] (206.71, 55.16) circle (  2.13);

\path[fill=fillColor,fill opacity=0.20] (204.74, 57.66) circle (  2.13);

\path[fill=fillColor,fill opacity=0.20] (195.57, 62.91) circle (  2.13);

\path[fill=fillColor,fill opacity=0.20] (184.20, 82.98) circle (  2.13);

\path[fill=fillColor,fill opacity=0.20] (190.76, 91.42) circle (  2.13);

\path[fill=fillColor,fill opacity=0.20] (195.13, 93.31) circle (  2.13);

\path[fill=fillColor,fill opacity=0.20] (198.41, 87.37) circle (  2.13);

\path[fill=fillColor,fill opacity=0.20] (199.72, 69.89) circle (  2.13);

\path[fill=fillColor,fill opacity=0.20] (197.53, 74.54) circle (  2.13);

\path[fill=fillColor,fill opacity=0.20] (197.75, 80.74) circle (  2.13);

\path[fill=fillColor,fill opacity=0.20] (201.90, 71.61) circle (  2.13);

\path[fill=fillColor,fill opacity=0.20] (200.59, 61.79) circle (  2.13);

\path[fill=fillColor,fill opacity=0.20] (202.34, 59.98) circle (  2.13);

\path[fill=fillColor,fill opacity=0.20] (206.71, 59.30) circle (  2.13);

\path[fill=fillColor,fill opacity=0.20] (207.15, 59.55) circle (  2.13);

\path[fill=fillColor,fill opacity=0.20] (206.05, 60.59) circle (  2.13);

\path[fill=fillColor,fill opacity=0.20] (204.52, 52.49) circle (  2.13);

\path[fill=fillColor,fill opacity=0.20] (203.65, 46.03) circle (  2.13);

\path[fill=fillColor,fill opacity=0.20] (205.40, 47.07) circle (  2.13);

\path[fill=fillColor,fill opacity=0.20] (207.37, 50.34) circle (  2.13);

\path[fill=fillColor,fill opacity=0.20] (206.93, 53.61) circle (  2.13);

\path[fill=fillColor,fill opacity=0.20] (205.18, 60.85) circle (  2.13);

\path[fill=fillColor,fill opacity=0.20] (203.00, 67.05) circle (  2.13);

\path[fill=fillColor,fill opacity=0.20] (205.18, 63.00) circle (  2.13);

\path[fill=fillColor,fill opacity=0.20] (206.49, 61.19) circle (  2.13);

\path[fill=fillColor,fill opacity=0.20] (205.84, 66.27) circle (  2.13);

\path[fill=fillColor,fill opacity=0.20] (201.90, 72.56) circle (  2.13);

\path[fill=fillColor,fill opacity=0.20] (194.04, 83.67) circle (  2.13);

\path[fill=fillColor,fill opacity=0.20] (184.86, 87.37) circle (  2.13);

\path[fill=fillColor,fill opacity=0.20] (186.17,107.09) circle (  2.13);

\path[fill=fillColor,fill opacity=0.20] (188.79,102.61) circle (  2.13);

\path[fill=fillColor,fill opacity=0.20] (187.26, 88.14) circle (  2.13);

\path[fill=fillColor,fill opacity=0.20] (187.92, 82.72) circle (  2.13);

\path[fill=fillColor,fill opacity=0.20] (196.44, 75.57) circle (  2.13);

\path[fill=fillColor,fill opacity=0.20] (203.87, 59.81) circle (  2.13);

\path[fill=fillColor,fill opacity=0.20] (205.18, 55.33) circle (  2.13);

\path[fill=fillColor,fill opacity=0.20] (204.96, 59.73) circle (  2.13);

\path[fill=fillColor,fill opacity=0.20] (203.43, 58.18) circle (  2.13);

\path[fill=fillColor,fill opacity=0.20] (202.56, 56.88) circle (  2.13);

\path[fill=fillColor,fill opacity=0.20] (200.59, 59.81) circle (  2.13);

\path[fill=fillColor,fill opacity=0.20] (201.90, 63.08) circle (  2.13);

\path[fill=fillColor,fill opacity=0.20] (201.25, 75.83) circle (  2.13);

\path[fill=fillColor,fill opacity=0.20] (198.41, 84.27) circle (  2.13);

\path[fill=fillColor,fill opacity=0.20] (194.91, 82.98) circle (  2.13);

\path[fill=fillColor,fill opacity=0.20] (189.01, 94.43) circle (  2.13);

\path[fill=fillColor,fill opacity=0.20] (182.89,102.52) circle (  2.13);

\path[fill=fillColor,fill opacity=0.20] (186.61, 92.19) circle (  2.13);

\path[fill=fillColor,fill opacity=0.20] (191.41, 83.92) circle (  2.13);

\path[fill=fillColor,fill opacity=0.20] (191.41, 88.57) circle (  2.13);

\path[fill=fillColor,fill opacity=0.20] (196.66, 98.48) circle (  2.13);

\path[fill=fillColor,fill opacity=0.20] (187.70,101.41) circle (  2.13);

\path[fill=fillColor,fill opacity=0.20] (183.77, 96.67) circle (  2.13);

\path[fill=fillColor,fill opacity=0.20] (190.98,101.58) circle (  2.13);

\path[fill=fillColor,fill opacity=0.20] (189.67,115.96) circle (  2.13);

\path[fill=fillColor,fill opacity=0.20] (180.27,115.10) circle (  2.13);

\path[fill=fillColor,fill opacity=0.20] (178.52,112.51) circle (  2.13);

\path[fill=fillColor,fill opacity=0.20] (204.31, 82.03) circle (  2.13);

\path[fill=fillColor,fill opacity=0.20] (203.21, 81.77) circle (  2.13);

\path[fill=fillColor,fill opacity=0.20] (203.43, 80.57) circle (  2.13);

\path[fill=fillColor,fill opacity=0.20] (198.84, 87.37) circle (  2.13);

\path[fill=fillColor,fill opacity=0.20] (202.34, 85.65) circle (  2.13);

\path[fill=fillColor,fill opacity=0.20] (207.37, 82.46) circle (  2.13);

\path[fill=fillColor,fill opacity=0.20] (207.58, 79.10) circle (  2.13);

\path[fill=fillColor,fill opacity=0.20] (206.05, 73.93) circle (  2.13);

\path[fill=fillColor,fill opacity=0.20] (205.18, 69.46) circle (  2.13);

\path[fill=fillColor,fill opacity=0.20] (200.15, 70.49) circle (  2.13);

\path[fill=fillColor,fill opacity=0.20] (192.73, 76.09) circle (  2.13);

\path[fill=fillColor,fill opacity=0.20] (181.58, 79.53) circle (  2.13);

\path[fill=fillColor,fill opacity=0.20] (207.37, 89.09) circle (  2.13);

\path[fill=fillColor,fill opacity=0.20] (206.49, 84.10) circle (  2.13);

\path[fill=fillColor,fill opacity=0.20] (213.92, 81.08) circle (  2.13);

\path[fill=fillColor,fill opacity=0.20] (210.86, 76.69) circle (  2.13);

\path[fill=fillColor,fill opacity=0.20] (210.86, 69.80) circle (  2.13);

\path[fill=fillColor,fill opacity=0.20] (206.71, 65.84) circle (  2.13);

\path[fill=fillColor,fill opacity=0.20] (203.87, 66.61) circle (  2.13);

\path[fill=fillColor,fill opacity=0.20] (201.03, 69.54) circle (  2.13);

\path[fill=fillColor,fill opacity=0.20] (190.10, 75.14) circle (  2.13);

\path[fill=fillColor,fill opacity=0.20] (171.97, 80.13) circle (  2.13);

\path[fill=fillColor,fill opacity=0.20] (200.59,100.20) circle (  2.13);

\path[fill=fillColor,fill opacity=0.20] (207.58, 84.44) circle (  2.13);

\path[fill=fillColor,fill opacity=0.20] (212.39, 71.52) circle (  2.13);

\path[fill=fillColor,fill opacity=0.20] (215.01, 68.68) circle (  2.13);

\path[fill=fillColor,fill opacity=0.20] (211.74, 65.41) circle (  2.13);

\path[fill=fillColor,fill opacity=0.20] (209.11, 61.53) circle (  2.13);

\path[fill=fillColor,fill opacity=0.20] (204.31, 62.05) circle (  2.13);

\path[fill=fillColor,fill opacity=0.20] (200.37, 65.50) circle (  2.13);

\path[fill=fillColor,fill opacity=0.20] (193.82, 69.37) circle (  2.13);

\path[fill=fillColor,fill opacity=0.20] (181.58, 78.33) circle (  2.13);

\path[fill=fillColor,fill opacity=0.20] (166.94, 90.81) circle (  2.13);

\path[fill=fillColor,fill opacity=0.20] (180.05,100.54) circle (  2.13);

\path[fill=fillColor,fill opacity=0.20] (205.18, 91.07) circle (  2.13);

\path[fill=fillColor,fill opacity=0.20] (208.24, 74.97) circle (  2.13);

\path[fill=fillColor,fill opacity=0.20] (208.68, 59.81) circle (  2.13);

\path[fill=fillColor,fill opacity=0.20] (207.37, 53.70) circle (  2.13);

\path[fill=fillColor,fill opacity=0.20] (207.80, 55.16) circle (  2.13);

\path[fill=fillColor,fill opacity=0.20] (204.96, 59.47) circle (  2.13);

\path[fill=fillColor,fill opacity=0.20] (199.06, 63.60) circle (  2.13);

\path[fill=fillColor,fill opacity=0.20] (192.29, 63.34) circle (  2.13);

\path[fill=fillColor,fill opacity=0.20] (183.77, 65.24) circle (  2.13);

\path[fill=fillColor,fill opacity=0.20] (168.69, 77.98) circle (  2.13);

\path[fill=fillColor,fill opacity=0.20] (204.31, 74.80) circle (  2.13);

\path[fill=fillColor,fill opacity=0.20] (205.40, 65.50) circle (  2.13);

\path[fill=fillColor,fill opacity=0.20] (206.71, 57.14) circle (  2.13);

\path[fill=fillColor,fill opacity=0.20] (205.18, 47.24) circle (  2.13);

\path[fill=fillColor,fill opacity=0.20] (205.40, 48.36) circle (  2.13);

\path[fill=fillColor,fill opacity=0.20] (201.03, 57.66) circle (  2.13);

\path[fill=fillColor,fill opacity=0.20] (198.84, 62.22) circle (  2.13);

\path[fill=fillColor,fill opacity=0.20] (187.48, 62.48) circle (  2.13);

\path[fill=fillColor,fill opacity=0.20] (172.41, 67.05) circle (  2.13);

\path[fill=fillColor,fill opacity=0.20] (207.58, 81.25) circle (  2.13);

\path[fill=fillColor,fill opacity=0.20] (208.89, 77.81) circle (  2.13);

\path[fill=fillColor,fill opacity=0.20] (203.87, 66.61) circle (  2.13);

\path[fill=fillColor,fill opacity=0.20] (205.40, 59.64) circle (  2.13);

\path[fill=fillColor,fill opacity=0.20] (207.37, 56.54) circle (  2.13);

\path[fill=fillColor,fill opacity=0.20] (209.77, 45.95) circle (  2.13);

\path[fill=fillColor,fill opacity=0.20] (201.03, 47.41) circle (  2.13);

\path[fill=fillColor,fill opacity=0.20] (200.81, 56.37) circle (  2.13);

\path[fill=fillColor,fill opacity=0.20] (199.06, 60.07) circle (  2.13);

\path[fill=fillColor,fill opacity=0.20] (186.17, 65.50) circle (  2.13);

\path[fill=fillColor,fill opacity=0.20] (170.00, 84.78) circle (  2.13);

\path[fill=fillColor,fill opacity=0.20] (209.55, 74.28) circle (  2.13);

\path[fill=fillColor,fill opacity=0.20] (210.86, 71.52) circle (  2.13);

\path[fill=fillColor,fill opacity=0.20] (210.64, 73.59) circle (  2.13);

\path[fill=fillColor,fill opacity=0.20] (207.37, 77.21) circle (  2.13);

\path[fill=fillColor,fill opacity=0.20] (205.18, 68.85) circle (  2.13);

\path[fill=fillColor,fill opacity=0.20] (209.11, 59.90) circle (  2.13);

\path[fill=fillColor,fill opacity=0.20] (207.37, 55.08) circle (  2.13);

\path[fill=fillColor,fill opacity=0.20] (208.02, 48.62) circle (  2.13);

\path[fill=fillColor,fill opacity=0.20] (204.09, 52.06) circle (  2.13);

\path[fill=fillColor,fill opacity=0.20] (200.15, 59.55) circle (  2.13);

\path[fill=fillColor,fill opacity=0.20] (199.72, 61.10) circle (  2.13);

\path[fill=fillColor,fill opacity=0.20] (188.57, 70.75) circle (  2.13);

\path[fill=fillColor,fill opacity=0.20] (212.17, 68.94) circle (  2.13);

\path[fill=fillColor,fill opacity=0.20] (207.37, 68.16) circle (  2.13);

\path[fill=fillColor,fill opacity=0.20] (206.49, 67.48) circle (  2.13);

\path[fill=fillColor,fill opacity=0.20] (208.89, 66.79) circle (  2.13);

\path[fill=fillColor,fill opacity=0.20] (196.44, 68.77) circle (  2.13);

\path[fill=fillColor,fill opacity=0.20] (211.74, 60.41) circle (  2.13);

\path[fill=fillColor,fill opacity=0.20] (208.02, 56.02) circle (  2.13);

\path[fill=fillColor,fill opacity=0.20] (208.46, 55.33) circle (  2.13);

\path[fill=fillColor,fill opacity=0.20] (202.78, 57.49) circle (  2.13);

\path[fill=fillColor,fill opacity=0.20] (196.00, 59.04) circle (  2.13);

\path[fill=fillColor,fill opacity=0.20] (197.10, 61.02) circle (  2.13);

\path[fill=fillColor,fill opacity=0.20] (192.29, 72.30) circle (  2.13);

\path[fill=fillColor,fill opacity=0.20] (196.88, 71.52) circle (  2.13);

\path[fill=fillColor,fill opacity=0.20] (208.46, 68.16) circle (  2.13);

\path[fill=fillColor,fill opacity=0.20] (206.05, 64.81) circle (  2.13);

\path[fill=fillColor,fill opacity=0.20] (202.34, 66.27) circle (  2.13);

\path[fill=fillColor,fill opacity=0.20] (201.03, 66.44) circle (  2.13);

\path[fill=fillColor,fill opacity=0.20] (209.11, 64.81) circle (  2.13);

\path[fill=fillColor,fill opacity=0.20] (204.31, 71.95) circle (  2.13);

\path[fill=fillColor,fill opacity=0.20] (182.67, 70.49) circle (  2.13);

\path[fill=fillColor,fill opacity=0.20] (206.93, 60.16) circle (  2.13);

\path[fill=fillColor,fill opacity=0.20] (206.71, 58.35) circle (  2.13);

\path[fill=fillColor,fill opacity=0.20] (205.40, 59.64) circle (  2.13);

\path[fill=fillColor,fill opacity=0.20] (203.00, 59.73) circle (  2.13);

\path[fill=fillColor,fill opacity=0.20] (197.31, 52.84) circle (  2.13);

\path[fill=fillColor,fill opacity=0.20] (197.53, 53.44) circle (  2.13);

\path[fill=fillColor,fill opacity=0.20] (197.53, 69.71) circle (  2.13);

\path[fill=fillColor,fill opacity=0.20] (188.14, 78.93) circle (  2.13);

\path[fill=fillColor,fill opacity=0.20] (192.73, 71.09) circle (  2.13);

\path[fill=fillColor,fill opacity=0.20] (195.35, 63.43) circle (  2.13);

\path[fill=fillColor,fill opacity=0.20] (199.28, 62.14) circle (  2.13);

\path[fill=fillColor,fill opacity=0.20] (197.53, 65.06) circle (  2.13);

\path[fill=fillColor,fill opacity=0.20] (204.31, 68.08) circle (  2.13);

\path[fill=fillColor,fill opacity=0.20] (197.10, 68.68) circle (  2.13);

\path[fill=fillColor,fill opacity=0.20] (200.37, 72.47) circle (  2.13);

\path[fill=fillColor,fill opacity=0.20] (186.83, 82.98) circle (  2.13);

\path[fill=fillColor,fill opacity=0.20] (185.73, 87.37) circle (  2.13);

\path[fill=fillColor,fill opacity=0.20] (199.94, 65.15) circle (  2.13);

\path[fill=fillColor,fill opacity=0.20] (208.02, 58.61) circle (  2.13);

\path[fill=fillColor,fill opacity=0.20] (202.78, 58.69) circle (  2.13);

\path[fill=fillColor,fill opacity=0.20] (200.15, 62.05) circle (  2.13);

\path[fill=fillColor,fill opacity=0.20] (200.59, 55.42) circle (  2.13);

\path[fill=fillColor,fill opacity=0.20] (199.28, 51.37) circle (  2.13);

\path[fill=fillColor,fill opacity=0.20] (201.90, 68.16) circle (  2.13);

\path[fill=fillColor,fill opacity=0.20] (191.41, 96.75) circle (  2.13);

\path[fill=fillColor,fill opacity=0.20] (192.29, 85.39) circle (  2.13);

\path[fill=fillColor,fill opacity=0.20] (196.88, 76.09) circle (  2.13);

\path[fill=fillColor,fill opacity=0.20] (201.03, 66.96) circle (  2.13);

\path[fill=fillColor,fill opacity=0.20] (194.26, 60.76) circle (  2.13);

\path[fill=fillColor,fill opacity=0.20] (194.69, 59.81) circle (  2.13);

\path[fill=fillColor,fill opacity=0.20] (194.69, 65.84) circle (  2.13);

\path[fill=fillColor,fill opacity=0.20] (194.91, 66.79) circle (  2.13);

\path[fill=fillColor,fill opacity=0.20] (201.47, 66.61) circle (  2.13);

\path[fill=fillColor,fill opacity=0.20] (186.17, 79.79) circle (  2.13);

\path[fill=fillColor,fill opacity=0.20] (181.36, 72.04) circle (  2.13);

\path[fill=fillColor,fill opacity=0.20] (199.06, 59.64) circle (  2.13);

\path[fill=fillColor,fill opacity=0.20] (204.31, 58.35) circle (  2.13);

\path[fill=fillColor,fill opacity=0.20] (216.76, 63.43) circle (  2.13);

\path[fill=fillColor,fill opacity=0.20] (201.68, 64.63) circle (  2.13);

\path[fill=fillColor,fill opacity=0.20] (199.28, 57.83) circle (  2.13);

\path[fill=fillColor,fill opacity=0.20] (203.21, 65.24) circle (  2.13);

\path[fill=fillColor,fill opacity=0.20] (198.63, 82.89) circle (  2.13);

\path[fill=fillColor,fill opacity=0.20] (200.15, 82.46) circle (  2.13);

\path[fill=fillColor,fill opacity=0.20] (200.37, 74.37) circle (  2.13);

\path[fill=fillColor,fill opacity=0.20] (201.47, 65.32) circle (  2.13);

\path[fill=fillColor,fill opacity=0.20] (197.31, 58.35) circle (  2.13);

\path[fill=fillColor,fill opacity=0.20] (200.59, 56.28) circle (  2.13);

\path[fill=fillColor,fill opacity=0.20] (199.28, 63.00) circle (  2.13);

\path[fill=fillColor,fill opacity=0.20] (194.47, 65.58) circle (  2.13);

\path[fill=fillColor,fill opacity=0.20] (199.72, 65.32) circle (  2.13);

\path[fill=fillColor,fill opacity=0.20] (191.20, 77.98) circle (  2.13);

\path[fill=fillColor,fill opacity=0.20] (167.82, 75.31) circle (  2.13);

\path[fill=fillColor,fill opacity=0.20] (175.46, 64.89) circle (  2.13);

\path[fill=fillColor,fill opacity=0.20] (194.47, 62.40) circle (  2.13);

\path[fill=fillColor,fill opacity=0.20] (214.79, 58.61) circle (  2.13);

\path[fill=fillColor,fill opacity=0.20] (202.34, 62.05) circle (  2.13);

\path[fill=fillColor,fill opacity=0.20] (200.59, 61.88) circle (  2.13);

\path[fill=fillColor,fill opacity=0.20] (204.09, 58.18) circle (  2.13);

\path[fill=fillColor,fill opacity=0.20] (200.59, 66.96) circle (  2.13);

\path[fill=fillColor,fill opacity=0.20] (195.13, 97.79) circle (  2.13);

\path[fill=fillColor,fill opacity=0.20] (196.88, 73.25) circle (  2.13);

\path[fill=fillColor,fill opacity=0.20] (201.03, 63.34) circle (  2.13);

\path[fill=fillColor,fill opacity=0.20] (206.27, 59.47) circle (  2.13);

\path[fill=fillColor,fill opacity=0.20] (198.63, 58.00) circle (  2.13);

\path[fill=fillColor,fill opacity=0.20] (197.10, 63.60) circle (  2.13);

\path[fill=fillColor,fill opacity=0.20] (204.52, 69.71) circle (  2.13);

\path[fill=fillColor,fill opacity=0.20] (198.63, 71.87) circle (  2.13);

\path[fill=fillColor,fill opacity=0.20] (185.73, 79.10) circle (  2.13);

\path[fill=fillColor,fill opacity=0.20] (175.68, 64.20) circle (  2.13);

\path[fill=fillColor,fill opacity=0.20] (193.16, 50.77) circle (  2.13);

\path[fill=fillColor,fill opacity=0.20] (202.56, 53.35) circle (  2.13);

\path[fill=fillColor,fill opacity=0.20] (206.71, 58.95) circle (  2.13);

\path[fill=fillColor,fill opacity=0.20] (201.03, 56.71) circle (  2.13);

\path[fill=fillColor,fill opacity=0.20] (200.37, 61.28) circle (  2.13);

\path[fill=fillColor,fill opacity=0.20] (199.94, 79.53) circle (  2.13);

\path[fill=fillColor,fill opacity=0.20] (197.97, 88.14) circle (  2.13);

\path[fill=fillColor,fill opacity=0.20] (205.18, 74.11) circle (  2.13);

\path[fill=fillColor,fill opacity=0.20] (210.21, 61.53) circle (  2.13);

\path[fill=fillColor,fill opacity=0.20] (209.55, 61.88) circle (  2.13);

\path[fill=fillColor,fill opacity=0.20] (200.59, 60.24) circle (  2.13);

\path[fill=fillColor,fill opacity=0.20] (199.72, 57.31) circle (  2.13);

\path[fill=fillColor,fill opacity=0.20] (204.31, 63.43) circle (  2.13);

\path[fill=fillColor,fill opacity=0.20] (201.90, 69.46) circle (  2.13);

\path[fill=fillColor,fill opacity=0.20] (199.06, 70.92) circle (  2.13);

\path[fill=fillColor,fill opacity=0.20] (191.63, 78.33) circle (  2.13);

\path[fill=fillColor,fill opacity=0.20] (171.97, 90.30) circle (  2.13);

\path[fill=fillColor,fill opacity=0.20] (167.38, 72.99) circle (  2.13);

\path[fill=fillColor,fill opacity=0.20] (177.43, 56.28) circle (  2.13);

\path[fill=fillColor,fill opacity=0.20] (192.73, 54.56) circle (  2.13);

\path[fill=fillColor,fill opacity=0.20] (205.62, 59.04) circle (  2.13);

\path[fill=fillColor,fill opacity=0.20] (203.43, 58.95) circle (  2.13);

\path[fill=fillColor,fill opacity=0.20] (199.94, 60.41) circle (  2.13);

\path[fill=fillColor,fill opacity=0.20] (200.15, 65.32) circle (  2.13);

\path[fill=fillColor,fill opacity=0.20] (196.22, 77.47) circle (  2.13);

\path[fill=fillColor,fill opacity=0.20] (211.95, 95.03) circle (  2.13);

\path[fill=fillColor,fill opacity=0.20] (211.08, 79.62) circle (  2.13);

\path[fill=fillColor,fill opacity=0.20] (210.64, 68.85) circle (  2.13);

\path[fill=fillColor,fill opacity=0.20] (211.74, 65.50) circle (  2.13);

\path[fill=fillColor,fill opacity=0.20] (207.80, 59.64) circle (  2.13);

\path[fill=fillColor,fill opacity=0.20] (207.58, 55.94) circle (  2.13);

\path[fill=fillColor,fill opacity=0.20] (202.12, 59.38) circle (  2.13);

\path[fill=fillColor,fill opacity=0.20] (199.06, 61.19) circle (  2.13);

\path[fill=fillColor,fill opacity=0.20] (199.50, 61.02) circle (  2.13);

\path[fill=fillColor,fill opacity=0.20] (189.23, 74.11) circle (  2.13);

\path[fill=fillColor,fill opacity=0.20] (178.96, 77.21) circle (  2.13);

\path[fill=fillColor,fill opacity=0.20] (183.77, 65.84) circle (  2.13);

\path[fill=fillColor,fill opacity=0.20] (186.83, 60.93) circle (  2.13);

\path[fill=fillColor,fill opacity=0.20] (200.59, 58.18) circle (  2.13);

\path[fill=fillColor,fill opacity=0.20] (204.74, 56.37) circle (  2.13);

\path[fill=fillColor,fill opacity=0.20] (202.56, 59.30) circle (  2.13);

\path[fill=fillColor,fill opacity=0.20] (198.19, 68.85) circle (  2.13);

\path[fill=fillColor,fill opacity=0.20] (194.91, 76.17) circle (  2.13);

\path[fill=fillColor,fill opacity=0.20] (190.54,102.52) circle (  2.13);

\path[fill=fillColor,fill opacity=0.20] (210.21, 91.76) circle (  2.13);

\path[fill=fillColor,fill opacity=0.20] (217.20, 83.84) circle (  2.13);

\path[fill=fillColor,fill opacity=0.20] (213.48, 70.83) circle (  2.13);

\path[fill=fillColor,fill opacity=0.20] (209.99, 65.32) circle (  2.13);

\path[fill=fillColor,fill opacity=0.20] (209.33, 62.74) circle (  2.13);

\path[fill=fillColor,fill opacity=0.20] (213.70, 59.64) circle (  2.13);

\path[fill=fillColor,fill opacity=0.20] (212.61, 58.78) circle (  2.13);

\path[fill=fillColor,fill opacity=0.20] (208.68, 58.95) circle (  2.13);

\path[fill=fillColor,fill opacity=0.20] (203.00, 58.09) circle (  2.13);

\path[fill=fillColor,fill opacity=0.20] (192.51, 58.95) circle (  2.13);

\path[fill=fillColor,fill opacity=0.20] (181.36, 69.80) circle (  2.13);

\path[fill=fillColor,fill opacity=0.20] (170.88, 80.57) circle (  2.13);

\path[fill=fillColor,fill opacity=0.20] (169.56, 68.16) circle (  2.13);

\path[fill=fillColor,fill opacity=0.20] (180.27, 59.98) circle (  2.13);

\path[fill=fillColor,fill opacity=0.20] (198.19, 57.83) circle (  2.13);

\path[fill=fillColor,fill opacity=0.20] (203.21, 61.71) circle (  2.13);

\path[fill=fillColor,fill opacity=0.20] (201.25, 68.94) circle (  2.13);

\path[fill=fillColor,fill opacity=0.20] (201.03, 70.66) circle (  2.13);

\path[fill=fillColor,fill opacity=0.20] (194.04, 75.05) circle (  2.13);

\path[fill=fillColor,fill opacity=0.20] (210.21, 85.39) circle (  2.13);

\path[fill=fillColor,fill opacity=0.20] (220.04, 75.05) circle (  2.13);

\path[fill=fillColor,fill opacity=0.20] (219.38, 66.53) circle (  2.13);

\path[fill=fillColor,fill opacity=0.20] (212.83, 56.37) circle (  2.13);

\path[fill=fillColor,fill opacity=0.20] (211.08, 57.40) circle (  2.13);

\path[fill=fillColor,fill opacity=0.20] (206.93, 62.31) circle (  2.13);

\path[fill=fillColor,fill opacity=0.20] (205.84, 59.81) circle (  2.13);

\path[fill=fillColor,fill opacity=0.20] (206.71, 59.04) circle (  2.13);

\path[fill=fillColor,fill opacity=0.20] (204.31, 63.08) circle (  2.13);

\path[fill=fillColor,fill opacity=0.20] (200.15, 66.96) circle (  2.13);

\path[fill=fillColor,fill opacity=0.20] (188.57, 68.94) circle (  2.13);

\path[fill=fillColor,fill opacity=0.20] (177.87, 73.50) circle (  2.13);

\path[fill=fillColor,fill opacity=0.20] (170.88, 87.80) circle (  2.13);

\path[fill=fillColor,fill opacity=0.20] (175.90, 73.68) circle (  2.13);

\path[fill=fillColor,fill opacity=0.20] (181.15, 64.38) circle (  2.13);

\path[fill=fillColor,fill opacity=0.20] (194.69, 63.86) circle (  2.13);

\path[fill=fillColor,fill opacity=0.20] (203.87, 65.06) circle (  2.13);

\path[fill=fillColor,fill opacity=0.20] (203.87, 69.37) circle (  2.13);

\path[fill=fillColor,fill opacity=0.20] (202.34, 70.15) circle (  2.13);

\path[fill=fillColor,fill opacity=0.20] (198.41, 75.74) circle (  2.13);

\path[fill=fillColor,fill opacity=0.20] (214.14, 83.92) circle (  2.13);

\path[fill=fillColor,fill opacity=0.20] (218.29, 72.21) circle (  2.13);

\path[fill=fillColor,fill opacity=0.20] (227.69, 67.30) circle (  2.13);

\path[fill=fillColor,fill opacity=0.20] (218.07, 62.31) circle (  2.13);

\path[fill=fillColor,fill opacity=0.20] (213.05, 55.51) circle (  2.13);

\path[fill=fillColor,fill opacity=0.20] (215.23, 60.33) circle (  2.13);

\path[fill=fillColor,fill opacity=0.20] (208.68, 66.79) circle (  2.13);

\path[fill=fillColor,fill opacity=0.20] (202.78, 62.05) circle (  2.13);

\path[fill=fillColor,fill opacity=0.20] (202.78, 60.24) circle (  2.13);

\path[fill=fillColor,fill opacity=0.20] (201.90, 65.84) circle (  2.13);

\path[fill=fillColor,fill opacity=0.20] (196.22, 71.35) circle (  2.13);

\path[fill=fillColor,fill opacity=0.20] (177.65, 78.76) circle (  2.13);

\path[fill=fillColor,fill opacity=0.20] (166.07, 93.14) circle (  2.13);

\path[fill=fillColor,fill opacity=0.20] (182.02, 67.82) circle (  2.13);

\path[fill=fillColor,fill opacity=0.20] (201.90, 64.20) circle (  2.13);

\path[fill=fillColor,fill opacity=0.20] (208.24, 66.70) circle (  2.13);

\path[fill=fillColor,fill opacity=0.20] (206.05, 72.21) circle (  2.13);

\path[fill=fillColor,fill opacity=0.20] (201.03, 71.52) circle (  2.13);

\path[fill=fillColor,fill opacity=0.20] (200.15, 78.67) circle (  2.13);

\path[fill=fillColor,fill opacity=0.20] (217.42, 87.80) circle (  2.13);

\path[fill=fillColor,fill opacity=0.20] (222.44, 72.90) circle (  2.13);

\path[fill=fillColor,fill opacity=0.20] (220.69, 69.80) circle (  2.13);

\path[fill=fillColor,fill opacity=0.20] (223.10, 68.16) circle (  2.13);

\path[fill=fillColor,fill opacity=0.20] (215.45, 62.83) circle (  2.13);

\path[fill=fillColor,fill opacity=0.20] (213.92, 60.50) circle (  2.13);

\path[fill=fillColor,fill opacity=0.20] (214.14, 64.03) circle (  2.13);

\path[fill=fillColor,fill opacity=0.20] (212.17, 64.12) circle (  2.13);

\path[fill=fillColor,fill opacity=0.20] (207.37, 62.57) circle (  2.13);

\path[fill=fillColor,fill opacity=0.20] (196.88, 66.96) circle (  2.13);

\path[fill=fillColor,fill opacity=0.20] (190.98, 69.03) circle (  2.13);

\path[fill=fillColor,fill opacity=0.20] (184.20, 70.06) circle (  2.13);

\path[fill=fillColor,fill opacity=0.20] (180.71, 80.31) circle (  2.13);

\path[fill=fillColor,fill opacity=0.20] (165.41, 79.62) circle (  2.13);

\path[fill=fillColor,fill opacity=0.20] (178.74, 67.73) circle (  2.13);

\path[fill=fillColor,fill opacity=0.20] (193.82, 63.08) circle (  2.13);

\path[fill=fillColor,fill opacity=0.20] (198.63, 68.94) circle (  2.13);

\path[fill=fillColor,fill opacity=0.20] (208.46, 70.92) circle (  2.13);

\path[fill=fillColor,fill opacity=0.20] (207.37, 66.44) circle (  2.13);

\path[fill=fillColor,fill opacity=0.20] (204.52, 77.29) circle (  2.13);

\path[fill=fillColor,fill opacity=0.20] (219.82,103.73) circle (  2.13);

\path[fill=fillColor,fill opacity=0.20] (214.36, 80.48) circle (  2.13);

\path[fill=fillColor,fill opacity=0.20] (219.82, 66.27) circle (  2.13);

\path[fill=fillColor,fill opacity=0.20] (223.97, 63.60) circle (  2.13);

\path[fill=fillColor,fill opacity=0.20] (219.82, 64.12) circle (  2.13);

\path[fill=fillColor,fill opacity=0.20] (212.39, 63.00) circle (  2.13);

\path[fill=fillColor,fill opacity=0.20] (212.61, 58.00) circle (  2.13);

\path[fill=fillColor,fill opacity=0.20] (217.20, 60.67) circle (  2.13);

\path[fill=fillColor,fill opacity=0.20] (208.89, 64.89) circle (  2.13);

\path[fill=fillColor,fill opacity=0.20] (203.43, 58.09) circle (  2.13);

\path[fill=fillColor,fill opacity=0.20] (192.94, 58.09) circle (  2.13);

\path[fill=fillColor,fill opacity=0.20] (185.08, 69.89) circle (  2.13);

\path[fill=fillColor,fill opacity=0.20] (175.68, 74.97) circle (  2.13);

\path[fill=fillColor,fill opacity=0.20] (183.99, 74.28) circle (  2.13);

\path[fill=fillColor,fill opacity=0.20] (166.94, 76.69) circle (  2.13);

\path[fill=fillColor,fill opacity=0.20] (175.46, 65.15) circle (  2.13);

\path[fill=fillColor,fill opacity=0.20] (187.48, 65.24) circle (  2.13);

\path[fill=fillColor,fill opacity=0.20] (201.90, 66.10) circle (  2.13);

\path[fill=fillColor,fill opacity=0.20] (206.71, 64.63) circle (  2.13);

\path[fill=fillColor,fill opacity=0.20] (204.09, 64.98) circle (  2.13);

\path[fill=fillColor,fill opacity=0.20] (204.52, 69.20) circle (  2.13);

\path[fill=fillColor,fill opacity=0.20] (198.41, 87.97) circle (  2.13);

\path[fill=fillColor,fill opacity=0.20] (216.76, 99.17) circle (  2.13);

\path[fill=fillColor,fill opacity=0.20] (221.57, 82.37) circle (  2.13);

\path[fill=fillColor,fill opacity=0.20] (225.72, 73.76) circle (  2.13);

\path[fill=fillColor,fill opacity=0.20] (219.60, 58.61) circle (  2.13);

\path[fill=fillColor,fill opacity=0.20] (220.48, 53.96) circle (  2.13);

\path[fill=fillColor,fill opacity=0.20] (227.03, 59.55) circle (  2.13);

\path[fill=fillColor,fill opacity=0.20] (219.38, 57.06) circle (  2.13);

\path[fill=fillColor,fill opacity=0.20] (232.06, 56.11) circle (  2.13);

\path[fill=fillColor,fill opacity=0.20] (219.38, 55.68) circle (  2.13);

\path[fill=fillColor,fill opacity=0.20] (208.46, 59.47) circle (  2.13);

\path[fill=fillColor,fill opacity=0.20] (198.41, 66.10) circle (  2.13);

\path[fill=fillColor,fill opacity=0.20] (192.29, 59.98) circle (  2.13);

\path[fill=fillColor,fill opacity=0.20] (178.52, 57.23) circle (  2.13);

\path[fill=fillColor,fill opacity=0.20] (174.15, 71.44) circle (  2.13);

\path[fill=fillColor,fill opacity=0.20] (185.30, 83.15) circle (  2.13);

\path[fill=fillColor,fill opacity=0.20] (169.56, 66.10) circle (  2.13);

\path[fill=fillColor,fill opacity=0.20] (183.99, 58.69) circle (  2.13);

\path[fill=fillColor,fill opacity=0.20] (197.75, 57.92) circle (  2.13);

\path[fill=fillColor,fill opacity=0.20] (207.15, 60.76) circle (  2.13);

\path[fill=fillColor,fill opacity=0.20] (205.18, 60.85) circle (  2.13);

\path[fill=fillColor,fill opacity=0.20] (203.87, 65.67) circle (  2.13);

\path[fill=fillColor,fill opacity=0.20] (197.97, 81.51) circle (  2.13);

\path[fill=fillColor,fill opacity=0.20] (225.72, 82.12) circle (  2.13);

\path[fill=fillColor,fill opacity=0.20] (228.56, 69.20) circle (  2.13);

\path[fill=fillColor,fill opacity=0.20] (234.02, 65.67) circle (  2.13);

\path[fill=fillColor,fill opacity=0.20] (222.88, 63.60) circle (  2.13);

\path[fill=fillColor,fill opacity=0.20] (223.97, 58.52) circle (  2.13);

\path[fill=fillColor,fill opacity=0.20] (216.76, 58.61) circle (  2.13);

\path[fill=fillColor,fill opacity=0.20] (221.35, 59.64) circle (  2.13);

\path[fill=fillColor,fill opacity=0.20] (219.38, 60.59) circle (  2.13);

\path[fill=fillColor,fill opacity=0.20] (209.77, 62.74) circle (  2.13);

\path[fill=fillColor,fill opacity=0.20] (202.12, 58.86) circle (  2.13);

\path[fill=fillColor,fill opacity=0.20] (187.48, 61.53) circle (  2.13);

\path[fill=fillColor,fill opacity=0.20] (183.11, 69.71) circle (  2.13);

\path[fill=fillColor,fill opacity=0.20] (182.67, 69.97) circle (  2.13);

\path[fill=fillColor,fill opacity=0.20] (179.83, 71.70) circle (  2.13);

\path[fill=fillColor,fill opacity=0.20] (188.79, 85.13) circle (  2.13);

\path[fill=fillColor,fill opacity=0.20] (166.94, 60.41) circle (  2.13);

\path[fill=fillColor,fill opacity=0.20] (185.30, 57.14) circle (  2.13);

\path[fill=fillColor,fill opacity=0.20] (197.31, 63.60) circle (  2.13);

\path[fill=fillColor,fill opacity=0.20] (205.40, 71.70) circle (  2.13);

\path[fill=fillColor,fill opacity=0.20] (206.27, 72.13) circle (  2.13);

\path[fill=fillColor,fill opacity=0.20] (200.37, 69.46) circle (  2.13);

\path[fill=fillColor,fill opacity=0.20] (200.59, 73.25) circle (  2.13);

\path[fill=fillColor,fill opacity=0.20] (198.41, 91.33) circle (  2.13);

\path[fill=fillColor,fill opacity=0.20] (223.53, 87.63) circle (  2.13);

\path[fill=fillColor,fill opacity=0.20] (225.50, 66.61) circle (  2.13);

\path[fill=fillColor,fill opacity=0.20] (230.74, 61.10) circle (  2.13);

\path[fill=fillColor,fill opacity=0.20] (247.35, 62.65) circle (  2.13);

\path[fill=fillColor,fill opacity=0.20] (219.38, 60.67) circle (  2.13);

\path[fill=fillColor,fill opacity=0.20] (219.60, 59.04) circle (  2.13);

\path[fill=fillColor,fill opacity=0.20] (221.13, 60.33) circle (  2.13);

\path[fill=fillColor,fill opacity=0.20] (214.36, 60.85) circle (  2.13);

\path[fill=fillColor,fill opacity=0.20] (206.49, 65.50) circle (  2.13);

\path[fill=fillColor,fill opacity=0.20] (196.66, 69.54) circle (  2.13);

\path[fill=fillColor,fill opacity=0.20] (182.24, 68.85) circle (  2.13);

\path[fill=fillColor,fill opacity=0.20] (182.02, 68.85) circle (  2.13);

\path[fill=fillColor,fill opacity=0.20] (174.81, 76.95) circle (  2.13);

\path[fill=fillColor,fill opacity=0.20] (182.67, 83.58) circle (  2.13);

\path[fill=fillColor,fill opacity=0.20] (188.14, 82.55) circle (  2.13);

\path[fill=fillColor,fill opacity=0.20] (180.27, 70.32) circle (  2.13);

\path[fill=fillColor,fill opacity=0.20] (189.89, 70.83) circle (  2.13);

\path[fill=fillColor,fill opacity=0.20] (198.41, 68.51) circle (  2.13);

\path[fill=fillColor,fill opacity=0.20] (204.96, 67.05) circle (  2.13);

\path[fill=fillColor,fill opacity=0.20] (201.25, 65.15) circle (  2.13);

\path[fill=fillColor,fill opacity=0.20] (200.81, 65.15) circle (  2.13);

\path[fill=fillColor,fill opacity=0.20] (201.03, 76.78) circle (  2.13);

\path[fill=fillColor,fill opacity=0.20] (196.88, 95.12) circle (  2.13);

\path[fill=fillColor,fill opacity=0.20] (222.44, 89.18) circle (  2.13);

\path[fill=fillColor,fill opacity=0.20] (230.09, 75.92) circle (  2.13);

\path[fill=fillColor,fill opacity=0.20] (225.28, 69.54) circle (  2.13);

\path[fill=fillColor,fill opacity=0.20] (220.26, 61.96) circle (  2.13);

\path[fill=fillColor,fill opacity=0.20] (219.82, 58.18) circle (  2.13);

\path[fill=fillColor,fill opacity=0.20] (217.63, 60.50) circle (  2.13);

\path[fill=fillColor,fill opacity=0.20] (208.89, 58.61) circle (  2.13);

\path[fill=fillColor,fill opacity=0.20] (200.37, 53.53) circle (  2.13);

\path[fill=fillColor,fill opacity=0.20] (191.41, 55.76) circle (  2.13);

\path[fill=fillColor,fill opacity=0.20] (190.32, 63.60) circle (  2.13);

\path[fill=fillColor,fill opacity=0.20] (183.99, 71.70) circle (  2.13);

\path[fill=fillColor,fill opacity=0.20] (191.41, 76.26) circle (  2.13);

\path[fill=fillColor,fill opacity=0.20] (184.64, 77.55) circle (  2.13);

\path[fill=fillColor,fill opacity=0.20] (179.62, 83.67) circle (  2.13);

\path[fill=fillColor,fill opacity=0.20] (168.04, 92.97) circle (  2.13);

\path[fill=fillColor,fill opacity=0.20] (183.77, 63.51) circle (  2.13);

\path[fill=fillColor,fill opacity=0.20] (183.55, 56.45) circle (  2.13);

\path[fill=fillColor,fill opacity=0.20] (198.41, 59.30) circle (  2.13);

\path[fill=fillColor,fill opacity=0.20] (201.25, 68.08) circle (  2.13);

\path[fill=fillColor,fill opacity=0.20] (203.87, 71.27) circle (  2.13);

\path[fill=fillColor,fill opacity=0.20] (205.84, 71.44) circle (  2.13);

\path[fill=fillColor,fill opacity=0.20] (200.81, 74.37) circle (  2.13);

\path[fill=fillColor,fill opacity=0.20] (197.10, 79.45) circle (  2.13);

\path[fill=fillColor,fill opacity=0.20] (199.06, 88.83) circle (  2.13);

\path[fill=fillColor,fill opacity=0.20] (233.80,106.83) circle (  2.13);

\path[fill=fillColor,fill opacity=0.20] (217.42, 82.80) circle (  2.13);

\path[fill=fillColor,fill opacity=0.20] (224.41, 68.25) circle (  2.13);

\path[fill=fillColor,fill opacity=0.20] (227.25, 66.53) circle (  2.13);

\path[fill=fillColor,fill opacity=0.20] (212.61, 69.37) circle (  2.13);

\path[fill=fillColor,fill opacity=0.20] (205.84, 67.48) circle (  2.13);

\path[fill=fillColor,fill opacity=0.20] (200.59, 61.36) circle (  2.13);

\path[fill=fillColor,fill opacity=0.20] (208.46, 59.73) circle (  2.13);

\path[fill=fillColor,fill opacity=0.20] (189.23, 58.35) circle (  2.13);

\path[fill=fillColor,fill opacity=0.20] (185.73, 57.23) circle (  2.13);

\path[fill=fillColor,fill opacity=0.20] (182.02, 64.98) circle (  2.13);

\path[fill=fillColor,fill opacity=0.20] (165.19, 78.93) circle (  2.13);

\path[fill=fillColor,fill opacity=0.20] (178.96, 86.08) circle (  2.13);

\path[fill=fillColor,fill opacity=0.20] (190.98, 89.35) circle (  2.13);

\path[fill=fillColor,fill opacity=0.20] (173.72, 92.45) circle (  2.13);

\path[fill=fillColor,fill opacity=0.20] (170.00, 74.54) circle (  2.13);

\path[fill=fillColor,fill opacity=0.20] (175.68, 62.83) circle (  2.13);

\path[fill=fillColor,fill opacity=0.20] (180.05, 58.61) circle (  2.13);

\path[fill=fillColor,fill opacity=0.20] (193.60, 66.61) circle (  2.13);

\path[fill=fillColor,fill opacity=0.20] (200.59, 73.93) circle (  2.13);

\path[fill=fillColor,fill opacity=0.20] (198.63, 65.84) circle (  2.13);

\path[fill=fillColor,fill opacity=0.20] (202.34, 65.58) circle (  2.13);

\path[fill=fillColor,fill opacity=0.20] (198.63, 73.76) circle (  2.13);

\path[fill=fillColor,fill opacity=0.20] (198.84, 77.38) circle (  2.13);

\path[fill=fillColor,fill opacity=0.20] (200.37, 80.22) circle (  2.13);

\path[fill=fillColor,fill opacity=0.20] (200.59, 90.64) circle (  2.13);

\path[fill=fillColor,fill opacity=0.20] (212.39, 99.17) circle (  2.13);

\path[fill=fillColor,fill opacity=0.20] (209.77, 83.67) circle (  2.13);

\path[fill=fillColor,fill opacity=0.20] (209.11, 81.60) circle (  2.13);

\path[fill=fillColor,fill opacity=0.20] (205.84, 77.38) circle (  2.13);

\path[fill=fillColor,fill opacity=0.20] (201.90, 68.94) circle (  2.13);

\path[fill=fillColor,fill opacity=0.20] (198.84, 65.24) circle (  2.13);

\path[fill=fillColor,fill opacity=0.20] (186.17, 68.16) circle (  2.13);

\path[fill=fillColor,fill opacity=0.20] (192.07, 69.20) circle (  2.13);

\path[fill=fillColor,fill opacity=0.20] (183.55, 66.87) circle (  2.13);

\path[fill=fillColor,fill opacity=0.20] (180.27, 68.51) circle (  2.13);

\path[fill=fillColor,fill opacity=0.20] (185.73, 74.28) circle (  2.13);

\path[fill=fillColor,fill opacity=0.20] (186.39, 79.88) circle (  2.13);

\path[fill=fillColor,fill opacity=0.20] (201.03, 87.28) circle (  2.13);

\path[fill=fillColor,fill opacity=0.20] (171.31, 84.35) circle (  2.13);

\path[fill=fillColor,fill opacity=0.20] (176.34, 64.46) circle (  2.13);

\path[fill=fillColor,fill opacity=0.20] (184.86, 61.96) circle (  2.13);

\path[fill=fillColor,fill opacity=0.20] (194.04, 54.13) circle (  2.13);

\path[fill=fillColor,fill opacity=0.20] (197.75, 56.80) circle (  2.13);

\path[fill=fillColor,fill opacity=0.20] (195.78, 69.63) circle (  2.13);

\path[fill=fillColor,fill opacity=0.20] (201.03, 74.37) circle (  2.13);

\path[fill=fillColor,fill opacity=0.20] (203.87, 72.64) circle (  2.13);

\path[fill=fillColor,fill opacity=0.20] (199.06, 74.62) circle (  2.13);

\path[fill=fillColor,fill opacity=0.20] (193.82, 77.21) circle (  2.13);

\path[fill=fillColor,fill opacity=0.20] (196.44, 85.99) circle (  2.13);

\path[fill=fillColor,fill opacity=0.20] (215.89, 93.40) circle (  2.13);

\path[fill=fillColor,fill opacity=0.20] (209.33, 83.84) circle (  2.13);

\path[fill=fillColor,fill opacity=0.20] (202.12, 73.68) circle (  2.13);

\path[fill=fillColor,fill opacity=0.20] (192.51, 67.82) circle (  2.13);

\path[fill=fillColor,fill opacity=0.20] (185.95, 64.72) circle (  2.13);

\path[fill=fillColor,fill opacity=0.20] (185.30, 68.94) circle (  2.13);

\path[fill=fillColor,fill opacity=0.20] (179.18, 74.97) circle (  2.13);

\path[fill=fillColor,fill opacity=0.20] (179.40, 73.07) circle (  2.13);

\path[fill=fillColor,fill opacity=0.20] (180.27, 70.83) circle (  2.13);

\path[fill=fillColor,fill opacity=0.20] (185.08, 74.54) circle (  2.13);

\path[fill=fillColor,fill opacity=0.20] (183.11, 75.83) circle (  2.13);

\path[fill=fillColor,fill opacity=0.20] (190.32, 75.92) circle (  2.13);

\path[fill=fillColor,fill opacity=0.20] (182.24, 84.61) circle (  2.13);

\path[fill=fillColor,fill opacity=0.20] (172.19, 90.04) circle (  2.13);

\path[fill=fillColor,fill opacity=0.20] (174.59, 79.36) circle (  2.13);

\path[fill=fillColor,fill opacity=0.20] (175.46, 69.37) circle (  2.13);

\path[fill=fillColor,fill opacity=0.20] (192.07, 61.96) circle (  2.13);

\path[fill=fillColor,fill opacity=0.20] (182.67, 62.22) circle (  2.13);

\path[fill=fillColor,fill opacity=0.20] (187.70, 66.79) circle (  2.13);

\path[fill=fillColor,fill opacity=0.20] (196.22, 65.24) circle (  2.13);

\path[fill=fillColor,fill opacity=0.20] (200.37, 65.93) circle (  2.13);

\path[fill=fillColor,fill opacity=0.20] (199.28, 72.04) circle (  2.13);

\path[fill=fillColor,fill opacity=0.20] (194.91, 70.58) circle (  2.13);

\path[fill=fillColor,fill opacity=0.20] (193.60, 65.24) circle (  2.13);

\path[fill=fillColor,fill opacity=0.20] (195.35, 70.58) circle (  2.13);

\path[fill=fillColor,fill opacity=0.20] (192.94, 78.24) circle (  2.13);

\path[fill=fillColor,fill opacity=0.20] (190.54, 79.36) circle (  2.13);

\path[fill=fillColor,fill opacity=0.20] (196.88, 82.98) circle (  2.13);

\path[fill=fillColor,fill opacity=0.20] (204.09, 89.00) circle (  2.13);

\path[fill=fillColor,fill opacity=0.20] (190.54, 89.26) circle (  2.13);

\path[fill=fillColor,fill opacity=0.20] (199.06, 86.68) circle (  2.13);

\path[fill=fillColor,fill opacity=0.20] (198.41, 87.28) circle (  2.13);

\path[fill=fillColor,fill opacity=0.20] (199.50, 89.95) circle (  2.13);

\path[fill=fillColor,fill opacity=0.20] (198.41, 92.54) circle (  2.13);

\path[fill=fillColor,fill opacity=0.20] (195.78, 91.85) circle (  2.13);

\path[fill=fillColor,fill opacity=0.20] (206.93, 86.42) circle (  2.13);

\path[fill=fillColor,fill opacity=0.20] (209.33, 83.67) circle (  2.13);

\path[fill=fillColor,fill opacity=0.20] (215.67, 81.43) circle (  2.13);

\path[fill=fillColor,fill opacity=0.20] (212.61, 80.05) circle (  2.13);

\path[fill=fillColor,fill opacity=0.20] (209.55, 73.76) circle (  2.13);

\path[fill=fillColor,fill opacity=0.20] (209.55, 65.24) circle (  2.13);

\path[fill=fillColor,fill opacity=0.20] (201.90, 66.27) circle (  2.13);

\path[fill=fillColor,fill opacity=0.20] (199.50, 71.95) circle (  2.13);

\path[fill=fillColor,fill opacity=0.20] (185.08, 70.32) circle (  2.13);

\path[fill=fillColor,fill opacity=0.20] (182.02, 68.25) circle (  2.13);

\path[fill=fillColor,fill opacity=0.20] (175.03, 72.99) circle (  2.13);

\path[fill=fillColor,fill opacity=0.20] (173.28, 76.60) circle (  2.13);

\path[fill=fillColor,fill opacity=0.20] (182.89, 80.39) circle (  2.13);

\path[fill=fillColor,fill opacity=0.20] (168.91, 84.61) circle (  2.13);

\path[fill=fillColor,fill opacity=0.20] (179.83, 88.66) circle (  2.13);

\path[fill=fillColor,fill opacity=0.20] (188.57, 90.64) circle (  2.13);

\path[fill=fillColor,fill opacity=0.20] (190.10, 89.18) circle (  2.13);

\path[fill=fillColor,fill opacity=0.20] (181.58, 94.95) circle (  2.13);

\path[fill=fillColor,fill opacity=0.20] (177.87, 85.39) circle (  2.13);

\path[fill=fillColor,fill opacity=0.20] (169.13, 78.41) circle (  2.13);

\path[fill=fillColor,fill opacity=0.20] (175.25, 75.48) circle (  2.13);

\path[fill=fillColor,fill opacity=0.20] (185.52, 69.20) circle (  2.13);

\path[fill=fillColor,fill opacity=0.20] (188.79, 65.15) circle (  2.13);

\path[fill=fillColor,fill opacity=0.20] (189.01, 66.79) circle (  2.13);

\path[fill=fillColor,fill opacity=0.20] (188.14, 62.65) circle (  2.13);

\path[fill=fillColor,fill opacity=0.20] (190.54, 58.61) circle (  2.13);

\path[fill=fillColor,fill opacity=0.20] (199.94, 63.69) circle (  2.13);

\path[fill=fillColor,fill opacity=0.20] (196.44, 66.61) circle (  2.13);

\path[fill=fillColor,fill opacity=0.20] (196.66, 66.01) circle (  2.13);

\path[fill=fillColor,fill opacity=0.20] (198.41, 66.70) circle (  2.13);

\path[fill=fillColor,fill opacity=0.20] (197.97, 70.40) circle (  2.13);

\path[fill=fillColor,fill opacity=0.20] (192.07, 72.64) circle (  2.13);

\path[fill=fillColor,fill opacity=0.20] (196.00, 72.13) circle (  2.13);

\path[fill=fillColor,fill opacity=0.20] (201.47, 72.13) circle (  2.13);

\path[fill=fillColor,fill opacity=0.20] (202.34, 72.04) circle (  2.13);

\path[fill=fillColor,fill opacity=0.20] (208.24, 70.75) circle (  2.13);

\path[fill=fillColor,fill opacity=0.20] (202.12, 69.80) circle (  2.13);

\path[fill=fillColor,fill opacity=0.20] (202.34, 67.99) circle (  2.13);

\path[fill=fillColor,fill opacity=0.20] (199.94, 67.39) circle (  2.13);

\path[fill=fillColor,fill opacity=0.20] (206.27, 71.18) circle (  2.13);

\path[fill=fillColor,fill opacity=0.20] (206.27, 74.54) circle (  2.13);

\path[fill=fillColor,fill opacity=0.20] (206.93, 75.31) circle (  2.13);

\path[fill=fillColor,fill opacity=0.20] (205.18, 70.23) circle (  2.13);

\path[fill=fillColor,fill opacity=0.20] (201.68, 62.65) circle (  2.13);

\path[fill=fillColor,fill opacity=0.20] (192.29, 59.21) circle (  2.13);

\path[fill=fillColor,fill opacity=0.20] (182.02, 58.95) circle (  2.13);

\path[fill=fillColor,fill opacity=0.20] (175.68, 65.41) circle (  2.13);

\path[fill=fillColor,fill opacity=0.20] (168.04, 75.31) circle (  2.13);

\path[fill=fillColor,fill opacity=0.20] (175.46, 79.45) circle (  2.13);

\path[fill=fillColor,fill opacity=0.20] (178.74, 84.10) circle (  2.13);

\path[fill=fillColor,fill opacity=0.20] (172.84, 95.98) circle (  2.13);

\path[fill=fillColor,fill opacity=0.20] (179.40,106.57) circle (  2.13);

\path[fill=fillColor,fill opacity=0.20] (182.46, 92.62) circle (  2.13);

\path[fill=fillColor,fill opacity=0.20] (180.05, 91.33) circle (  2.13);

\path[fill=fillColor,fill opacity=0.20] (179.83, 85.13) circle (  2.13);

\path[fill=fillColor,fill opacity=0.20] (172.19, 73.42) circle (  2.13);

\path[fill=fillColor,fill opacity=0.20] (178.09, 63.26) circle (  2.13);

\path[fill=fillColor,fill opacity=0.20] (175.46, 56.71) circle (  2.13);

\path[fill=fillColor,fill opacity=0.20] (182.02, 57.06) circle (  2.13);

\path[fill=fillColor,fill opacity=0.20] (189.01, 62.74) circle (  2.13);

\path[fill=fillColor,fill opacity=0.20] (192.94, 63.51) circle (  2.13);

\path[fill=fillColor,fill opacity=0.20] (194.91, 61.02) circle (  2.13);

\path[fill=fillColor,fill opacity=0.20] (199.94, 61.19) circle (  2.13);

\path[fill=fillColor,fill opacity=0.20] (197.31, 61.45) circle (  2.13);

\path[fill=fillColor,fill opacity=0.20] (200.15, 63.08) circle (  2.13);

\path[fill=fillColor,fill opacity=0.20] (204.74, 67.56) circle (  2.13);

\path[fill=fillColor,fill opacity=0.20] (208.02, 72.13) circle (  2.13);

\path[fill=fillColor,fill opacity=0.20] (203.43, 71.01) circle (  2.13);

\path[fill=fillColor,fill opacity=0.20] (203.65, 65.93) circle (  2.13);

\path[fill=fillColor,fill opacity=0.20] (196.66, 64.29) circle (  2.13);

\path[fill=fillColor,fill opacity=0.20] (188.14, 63.69) circle (  2.13);

\path[fill=fillColor,fill opacity=0.20] (190.54, 63.95) circle (  2.13);

\path[fill=fillColor,fill opacity=0.20] (189.45, 68.42) circle (  2.13);

\path[fill=fillColor,fill opacity=0.20] (183.55, 72.38) circle (  2.13);

\path[fill=fillColor,fill opacity=0.20] (185.30, 72.90) circle (  2.13);

\path[fill=fillColor,fill opacity=0.20] (183.99, 69.54) circle (  2.13);

\path[fill=fillColor,fill opacity=0.20] (178.96, 63.00) circle (  2.13);

\path[fill=fillColor,fill opacity=0.20] (174.59, 62.74) circle (  2.13);

\path[fill=fillColor,fill opacity=0.20] (173.50, 72.13) circle (  2.13);

\path[fill=fillColor,fill opacity=0.20] (183.55, 89.87) circle (  2.13);

\path[fill=fillColor,fill opacity=0.20] (182.46, 84.87) circle (  2.13);

\path[fill=fillColor,fill opacity=0.20] (182.24, 76.78) circle (  2.13);

\path[fill=fillColor,fill opacity=0.20] (174.15, 75.48) circle (  2.13);

\path[fill=fillColor,fill opacity=0.20] (167.82, 74.80) circle (  2.13);

\path[fill=fillColor,fill opacity=0.20] (180.27, 70.06) circle (  2.13);

\path[fill=fillColor,fill opacity=0.20] (179.40, 63.51) circle (  2.13);

\path[fill=fillColor,fill opacity=0.20] (184.20, 63.00) circle (  2.13);

\path[fill=fillColor,fill opacity=0.20] (188.14, 64.89) circle (  2.13);

\path[fill=fillColor,fill opacity=0.20] (193.38, 61.62) circle (  2.13);

\path[fill=fillColor,fill opacity=0.20] (195.35, 55.42) circle (  2.13);

\path[fill=fillColor,fill opacity=0.20] (199.50, 58.61) circle (  2.13);

\path[fill=fillColor,fill opacity=0.20] (196.88, 67.48) circle (  2.13);

\path[fill=fillColor,fill opacity=0.20] (188.57, 67.13) circle (  2.13);

\path[fill=fillColor,fill opacity=0.20] (183.55, 63.34) circle (  2.13);

\path[fill=fillColor,fill opacity=0.20] (178.09, 63.08) circle (  2.13);

\path[fill=fillColor,fill opacity=0.20] (174.15, 65.06) circle (  2.13);

\path[fill=fillColor,fill opacity=0.20] (176.78, 63.86) circle (  2.13);

\path[fill=fillColor,fill opacity=0.20] (171.09, 63.00) circle (  2.13);

\path[fill=fillColor,fill opacity=0.20] (170.66, 62.83) circle (  2.13);

\path[fill=fillColor,fill opacity=0.20] (181.36, 67.39) circle (  2.13);

\path[fill=fillColor,fill opacity=0.20] (185.73, 77.64) circle (  2.13);

\path[fill=fillColor,fill opacity=0.20] (185.08, 80.39) circle (  2.13);

\path[fill=fillColor,fill opacity=0.20] (186.61, 81.86) circle (  2.13);

\path[fill=fillColor,fill opacity=0.20] (179.40, 93.57) circle (  2.13);

\path[fill=fillColor,fill opacity=0.20] (181.80, 99.08) circle (  2.13);

\path[fill=fillColor,fill opacity=0.20] (176.56, 87.37) circle (  2.13);

\path[fill=fillColor,fill opacity=0.20] (177.21, 75.48) circle (  2.13);

\path[fill=fillColor,fill opacity=0.20] (171.97, 75.83) circle (  2.13);

\path[fill=fillColor,fill opacity=0.20] (180.49, 77.21) circle (  2.13);

\path[fill=fillColor,fill opacity=0.20] (193.38, 72.56) circle (  2.13);

\path[fill=fillColor,fill opacity=0.20] (181.36, 65.67) circle (  2.13);

\path[fill=fillColor,fill opacity=0.20] (178.96, 61.53) circle (  2.13);

\path[fill=fillColor,fill opacity=0.20] (180.05, 64.03) circle (  2.13);

\path[fill=fillColor,fill opacity=0.20] (170.66, 67.05) circle (  2.13);

\path[fill=fillColor,fill opacity=0.20] (177.43, 70.32) circle (  2.13);

\path[fill=fillColor,fill opacity=0.20] (169.56, 76.86) circle (  2.13);

\path[fill=fillColor,fill opacity=0.20] (181.36, 83.58) circle (  2.13);

\path[fill=fillColor,fill opacity=0.20] (181.80, 82.29) circle (  2.13);

\path[fill=fillColor,fill opacity=0.20] (182.24, 76.35) circle (  2.13);

\path[fill=fillColor,fill opacity=0.20] (190.76, 76.52) circle (  2.13);

\path[fill=fillColor,fill opacity=0.20] (210.64, 82.03) circle (  2.13);

\path[fill=fillColor,fill opacity=0.20] (192.07, 90.99) circle (  2.13);

\path[fill=fillColor,fill opacity=0.20] (186.83, 88.75) circle (  2.13);

\path[fill=fillColor,fill opacity=0.20] (185.30, 84.35) circle (  2.13);

\path[fill=fillColor,fill opacity=0.20] (187.26, 82.72) circle (  2.13);

\path[fill=fillColor,fill opacity=0.20] (189.67, 77.29) circle (  2.13);

\path[fill=fillColor,fill opacity=0.20] (176.56, 74.19) circle (  2.13);

\path[fill=fillColor,fill opacity=0.20] (178.74, 80.13) circle (  2.13);

\path[fill=fillColor,fill opacity=0.20] (182.89, 88.23) circle (  2.13);
\end{scope}
\begin{scope}
\path[clip] (  0.00,  0.00) rectangle (289.08,144.54);
\definecolor[named]{drawColor}{rgb}{0.50,0.50,0.50}

\node[text=drawColor,anchor=base,inner sep=0pt, outer sep=0pt, scale=  0.96] at ( 60.53, 20.31) {0.02};

\node[text=drawColor,anchor=base,inner sep=0pt, outer sep=0pt, scale=  0.96] at ( 82.38, 20.31) {0.03};

\node[text=drawColor,anchor=base,inner sep=0pt, outer sep=0pt, scale=  0.96] at (104.23, 20.31) {0.04};

\node[text=drawColor,anchor=base,inner sep=0pt, outer sep=0pt, scale=  0.96] at (126.08, 20.31) {0.05};

\node[text=drawColor,anchor=base,inner sep=0pt, outer sep=0pt, scale=  0.96] at (147.93, 20.31) {0.06};
\end{scope}
\begin{scope}
\path[clip] (  0.00,  0.00) rectangle (289.08,144.54);
\definecolor[named]{drawColor}{rgb}{0.50,0.50,0.50}

\path[draw=drawColor,line width= 0.6pt,line join=round,line cap=round] ( 60.53, 29.77) -- ( 60.53, 34.04);

\path[draw=drawColor,line width= 0.6pt,line join=round,line cap=round] ( 82.38, 29.77) -- ( 82.38, 34.04);

\path[draw=drawColor,line width= 0.6pt,line join=round,line cap=round] (104.23, 29.77) -- (104.23, 34.04);

\path[draw=drawColor,line width= 0.6pt,line join=round,line cap=round] (126.08, 29.77) -- (126.08, 34.04);

\path[draw=drawColor,line width= 0.6pt,line join=round,line cap=round] (147.93, 29.77) -- (147.93, 34.04);
\end{scope}
\begin{scope}
\path[clip] (  0.00,  0.00) rectangle (289.08,144.54);
\definecolor[named]{drawColor}{rgb}{0.50,0.50,0.50}

\node[text=drawColor,anchor=base,inner sep=0pt, outer sep=0pt, scale=  0.96] at (180.71, 20.31) {0.02};

\node[text=drawColor,anchor=base,inner sep=0pt, outer sep=0pt, scale=  0.96] at (202.56, 20.31) {0.03};

\node[text=drawColor,anchor=base,inner sep=0pt, outer sep=0pt, scale=  0.96] at (224.41, 20.31) {0.04};

\node[text=drawColor,anchor=base,inner sep=0pt, outer sep=0pt, scale=  0.96] at (246.26, 20.31) {0.05};

\node[text=drawColor,anchor=base,inner sep=0pt, outer sep=0pt, scale=  0.96] at (268.11, 20.31) {0.06};
\end{scope}
\begin{scope}
\path[clip] (  0.00,  0.00) rectangle (289.08,144.54);
\definecolor[named]{drawColor}{rgb}{0.50,0.50,0.50}

\path[draw=drawColor,line width= 0.6pt,line join=round,line cap=round] (180.71, 29.77) -- (180.71, 34.04);

\path[draw=drawColor,line width= 0.6pt,line join=round,line cap=round] (202.56, 29.77) -- (202.56, 34.04);

\path[draw=drawColor,line width= 0.6pt,line join=round,line cap=round] (224.41, 29.77) -- (224.41, 34.04);

\path[draw=drawColor,line width= 0.6pt,line join=round,line cap=round] (246.26, 29.77) -- (246.26, 34.04);

\path[draw=drawColor,line width= 0.6pt,line join=round,line cap=round] (268.11, 29.77) -- (268.11, 34.04);
\end{scope}
\begin{scope}
\path[clip] (  0.00,  0.00) rectangle (289.08,144.54);
\definecolor[named]{drawColor}{rgb}{0.00,0.00,0.00}

\node[text=drawColor,anchor=base,inner sep=0pt, outer sep=0pt, scale=  1.20] at (158.36,  9.03) {$\rho$ $[\mu m^{-2}]$};
\end{scope}
\begin{scope}
\path[clip] (  0.00,  0.00) rectangle (289.08,144.54);
\definecolor[named]{drawColor}{rgb}{0.00,0.00,0.00}

\node[text=drawColor,rotate= 90.00,anchor=base,inner sep=0pt, outer sep=0pt, scale=  1.20] at ( 17.30, 76.95) {RD $[\times 10^{-9}mm^2/s]$};
\end{scope}
\end{tikzpicture}

					\end{adjustbox}
					\end{minipage}
				}		
	\caption{Scatterplots of DTI metrics and $a$ and $\rho$. The $r$ value denotes the correspondenceding correlation coefficient.}
	\label{fig:chap 9 DTI correlations}	
\end{figure}	
Figure~\ref{fig:chap 9 DTI correlations} presents the correlation between the standard DTI metrics and the $a$ and $\rho$ estimates. The correlations we see here agree with the findings of \citep{Barazany:2009} and \citep{Alexander:2010}. While $MD$ is not correlated either of the microstructure indices, the directional diffusivities $AD$ and $RD$ both show moderate correlations with $a$ and $\rho$. Of course it is not surprising to find $RD$ negatively correlated with $a$ and positively correlated with $\rho$ respectively, as it is know both axonal packing density, and axon diameter all influence $RD$ measurements \citep{Beaulieu:2002}. The observed correlation between $AD$ and $a$ (positive) and $\rho$ (negative) is less intuitively explainable, however consistent with previous findings \citep{Barazany:2009,Alexander:2010}. Alexander et al. speculate that the increase of axon size and decrease of packing density are associated with lower fibre coherency, and thus are causing an increasing amount of diffusion impedance along the dominant diffusion direction. The correlation between AD and the $a$ and $\rho$ indices might also be the result of varying CSF contamination, since the regions high axon diameter are mostly found in the thinner midbody region are more affected by partial volume effects than those in genu and splenium. Since we find both $AD$ and $RD$ correlated with the $a$ and $\rho$ indices, their correlation with $FA$ follows by definition.  
\egroup %png export
\FloatBarrier
\subsection*{Axon diameter and axon density indices in the SC}
Figure~\ref{fig:chap 9 SC results} shows the $a$ and $\rho$ maps acquired in one healthy volunteer. We clearly see the bilateral symmetry of the parameter maps as expected from the basic anatomy of the spinal cord. The estimates of $a$ and $\rho$ indices are within a similar range of values measured in the CC. Furthermore, both $a$ and $\rho$ allow good discrimination between motor and sensory WM tracts. The largest $a$ (10.7 $\pm$ 2$\mu m$) and lowest $\rho$ (0.035$\pm$0.017$\mu m^{-2}$) are found in the lateral tract. The dorsal sensory tract shows the lowest $a$ (9.1$\pm$1.3$\mu m$) and highest $\rho$ (0.046$\pm$0.017$\mu m^{-2}$). The contrast between LT and DC is consistent over several slices in our dataset and agrees with our earlier findings in fixed monkey cervical cord. Figure~\ref{fig:chap 9 SC results} also illustrates well the challenges in SC imaging. The posterior halo of low $a$ is the result of motion artifacts during the acquisition which can be cause e.g. by swallowing or breathing. Cardiac motion also makes cardiac gating a neccesity, which in turn reduces the data we can acquire within the 25 minute windows and consequently reduced the SNR in our data. However, the results here show first evidence that our \SFasym{} protocol can be used successfully in the SC application, despite the more challenging imaging environment.
\begin{figure}[ht]
	\centering
	\begin{minipage}{0.39\textwidth}
		\subfloat[]
		{
			\pgfimage[width=0.9\textwidth]{chapter9/figs/SC_slices}
		}
	\end{minipage}
	\begin{minipage}{0.59\textwidth}
			\subfloat[]
			{
				\pgfimage[width=\textwidth]{chapter9/figs/SC_diam}
			}\\
			\subfloat[]
			{
				\pgfimage[width=\textwidth]{chapter9/figs/SC_dens}
			}
	\end{minipage}
	\caption{(a) SC slice alignment  and  (b\&c) maps of $a$ and $\rho$ in one healthy volunteer. Annotations on the first result slice denote the location of the dorsal column (DC) and left and right lateral tracts (LT).}
	\label{fig:chap 9 SC results}
\end{figure}

\section{Discussion}
This work presents a novel imaging and analysis pipeline for measuring axon diameter and density indices in the CC in-vivo, which expands on our \SFasym{} protocol optimisation we introduced in the previous chapter. We combined small FOV imaging and careful optimisation of the MR protocols and post-processing techniques to gain both high spatial resolution while maximising SNR. We show here axon diameter and axon density maps of better quality than previous studies have shown before. For the first time present results of a larger subject cohort of 5 subjects, allowing to infer scan and rescan reproducibilty with more confidence. Our results show that $a$ and $\rho$ show very good reproducibility consistently over all investigated subjects. Furthermore, a first test of our protocol in healthy cervical produced compelling results that are in good agreement with our findings excised monkey cervical cord.

\subsection*{Limitations \& further work}
\paragraph{Interpretation of $a$ and $\rho$}The model we use here is a very simplistic approximation of the complex micro-anatomy of real biological tissue. The actual estimates of axon diameter and densities differ considerably from what is expected from histology or ex-vivo scans \citep{Alexander:2010}. Most of this disparity can be explained by the limited gradient strength available on clinical system.  With limited gradient strength, small axons cause very little signal attenuation and become indistinguishable from each other \citep{Laett:2007,Yeh:2010}. Our simulation experiments in Chapter \ref{XX} show a limit of sensitivity of of 2--4$\mu m$ even in the very idealised situation of perfectly aligned single-radius cylinders. Consequently, the $a$ and $\rho$ indices must not be seen as accurate reflections of the complete axon diameter distribution as they are likely driven only by a small number of large axons in the WM. 

Other more complex models have been also been suggested, adding more tissue compartments, a distribution of axon diameters and/or permeable membranes. However, with the given limits both in scan time and gradient hardware, their practical value for in-vivo clinical applications is questionable. Recent studies e.g. by \citet{Panagiotaki:2012, Ferizi:2012} have studied those more complex models in the CC WM tissue with much more extensive datasets than we used here. Their findings suggest that a simple two compartment model similar to ours explains diffusion in coherent WM tissue reasonably well. For our purpose of clinical adaptation, this model provides the best trade-off between explaining the diffusion in WM while keeping data requirements reasonably low. Furthermore, the shorter protocol we propose is an important step to enable more widespread adaptation of our imaging pipeline, which in turn will lead to a better understanding of its parameters interpretation.


Furthermore, alternative acquisition methods such as oscillating gradients \citep{Does:2003, Colvin:2008} or multiple wave-vector acquisitions \citep{Komlosh:2008,Koch:2008,Avram:2012} promise more sensitivity to smaller axon diameters. Recently the protocol optimisation framework has been extended to support such non-rectangular gradient waveforms\citet{Drobnjak:2010,Siow:2012a}. The method we present here is seasily combined with any other pulse sequences to provide better discrimination of small axon diameters.

\paragraph{T2 estimation: } We chose here to estimate T2 directly from the b=0 weighted images. Using this approach, we are limited to estimate rather high \glspl{TE} due to the nature of the single shot EPI technique used here. Furthermore, due to the sparsity of the \glspl{TE} samples, we can only account for mono-exponential T2-decay. A more comprehensive T2 decay curve could potentially be estimated using the \gls{CPMG} sequence \citep{Pell:2006}. However, this approach would add significant scan time to our protocol. Furthermore, such acquisition does not suffer from the same image distortions as the EPI DWI images and therefore requires further registration, which might confound the results. Nevertheless, a better T2 decay curve estimate might also be used to correct for T2 differences between intra- and extra-axonal compartments. Fitting a single T2 to all compartments could lead to errors in  the $f_{intra}$ estimates and might have an affect on our $\rho$ parameter maps. While we focussed on a single-compartment T2 estimation that to simply fitting, our tissue model can easily incorporate different T2s for individual compartments. 

\paragraph{Validation: } We have validate our result here by comparision with previous reports of tissue parameter estimates by diffusion MRI and independent histopathology. While several systematic reports of the tissue microstructure are available in the CC, validation of our results is more complicated in the SC as it is much less documented. Future work needs is required that provides a direct comparison between our MRI parameters and independent histology for a better interpretation of our estimated parameters in healthy tissue but more importantly also in the presence of pathological tissue alteration. Following up on our work on \SFasym{}, we are now in the process of setting a validation study of our protocol using post-mortem human spinal cord of healthy and MS tissue in collaboration with Dr DeLuca from the Nuffield Department of Neurosciences, University of Oxford.

\paragraph{Clinical application: } The protocols we present here are designed with clinical adaptation in mind. Due ot the short acquisition time, our 25 minute protocol can be easily incorporated into existing studies. A disadvantage of our method is that it does not offer whole brain coverage. However, the alternative \OI{} method is also intrinsically limiting by the tissue model to application in highly coherent WM structures. In fact, if only such structures are to be studies anyway, our method provides much better spatial resolution, SNR in shorter acquisition time. Furthermore, many neurological diseases such as Alzheimers disease, Schizophrenia or MS have severe impact on the CC and diagnosis. In those diseases, diagnosnis and therapy monitoring might benefit from better tissue characterisation. Furthermore, the first results in the SC we showed also promise future application in the cord, e.g. to gain better biomarkers for diagnosis and therapeutic outcome in SCI. Wider application in other parts in the CNS might come from the adaption of a more complex tissue model which incorporates fibre dispersion and fibre crossing \cite{Zhang:2011}.

\section{Conclusion}
We have demonstrated that our microstructure estimates agree with reported post-mortem evaluation of the CC fibre density distribution. Further, we showed good inter- and intra-subject reproducibility. The scan time of the protocol is short enough to be easily incorporated into clinical studies. In future work, we are planning to use this approach in subjects with known altered microstructure of the CC.