%!TEX root = ./report.tex
    
\section{Reproducibity of q-space imaging in the spinal cord} 
In this study we investigate accuracy and sensitivity of tract-specific q-space imaging metrics in healthy controls. As discussed above (see Section~\ref{sec:qspace}), several studies on experimental MRI systems have shown that QSI can provide accurate information about microscopic restriction in excised tissue \citep{Assaf:2000,Bar-Shir:2008,Ong:2008}. Although the conditions for true QSI, such as the short gradient pulse, are impossible to achieve in clinical systems, studies such as \citet{Farrell:2008} have shown the great potential in the assessment of spinal cord white matter pathology. However, most clinical QSI studies only focused on a small number of patients and failed to demonstrate the reliability of QSI. The aim of this study is to report reproducibility of QSI metrics in the cervical spinal cord on a standard 3T clinical MRI scanner. We also assessed QSI measures both in-plane (XY) and parallel to the main spinal cord axis (Z), not presented before. We compare QSI measures derived in gray matter and different ascending and descending white tracts of the cervical spinal cord in healthy subjects and investigate associations between QSI parameters and conventional apparent diffusion coefficient (ADC) measures, both in plane and along the cord. 
\subsection*{Methods} 
\paragraph{Study design \& Data acquisition:} We recruited 9 right-handed male healthy subjects (mean age 35±11yrs) to be scanned on a 3T Tim Trio (Siemens Healthcare, Erlangen). Three subjects were recalled for a second scan on a different day to assess intra-subject reproducibility of QSI derived parameters. We performed cardiac-gated high b-value axial DWI (matrix=96x96, b-spline interpolated to 192x192 in image space, FOV=144x144mm2, slice thickness=5mm, 20 slices, TE=110ms, TR≈4000ms) with 32 b values between 0-3000s/mm2 in b=50s/mm2 steps (gradient duration=45ms, diffusion time=55ms, maximum gradient strength=23mT/m). Three different DWI directions were acquired: two directions perpendicular (XY) and one parallel (Z) to the main spinal cord axis. The two perpendicular diffusion directions were averaged to increase the signal-to-noise ratio. The measurements were linearly regridded to be equidistant in q-space and the DPDF was computed using inverse fast Fourier transformation. To increase the resolution of the DPDF, the signal was extrapolated in q-space to a maximum q=166mm$^{-1}$ by fitting a bi-exponential decay curve to the DWI data as in \citet{Cohen:2002, Farrell:2008}. Maps of the full width at half maximum and zero displacement probability were derived for XY and Z as described in Section~\ref{sec:qspace}. For comparison we also computed the apparent diffusion coefficient (see section \ref{subsec:adc}) from the monoexponential part of the decay curve (b≤1100s/mm2) as in \citet{Farrell:2008} for both XY and Z directions. 
\paragraph{ROI analysis:} We semi-automatically delineate the whole cervical spinal cord area (SCA) between levels C1 and C3 on the b=0 images using the active surface segmentation by \citet{Horsfield:2010} available in Jim6. We perform a morphological erosion (2 iterations) of the obtained segmentation mask to exclude voxels with potential partial-volume average effect from surrounding cerebro-spinal fluid. In addition, four regions of interest were manually placed in specific white matter tracts and one ROI was positioned in the gray matter on all slices between level C1 and C3. The four white matter regions comprised the left and right tracts (l\&r-LT) running in the lateral columns and the anterior (AT) and posterior tracts (PT) similar to \citet{Hesseltine:2006,Freund:2010}. 
\paragraph{Statistical processing:} We report reproducibility as the intra-subject coefficient of variation (COV=SD measurements/mean measurements) for the three scan/rescan subjects and the inter-subject COV among all nine subjects. Further, we compare significant differences in the group mean values of the ADC parameters (ADCxy, ADCz) and QSI metrics (P0xy, P0z, FWHMxy, FWHMz) between tracts by performing the Hotellings-T2 test (confidence interval=99\%). To investigate the relevance of measurements in the Z direction, we compute the same significance test of XY-only QSI parameters (P0xy, FWHMxy). Finally, we investigate the relationship between individual ADC and QSI measurements in XY and Z directions for each tract using the Spearman’s $\rho$ correlation coefficient. 
\subsection*{Results} 

\begin{figure}
  \centering
  \pgfimage[width=12cm]{pictures/chap3/sec3/cov.pdf}
  \captionbelow{Intra- and inter-subject COV for QSI and ADC parameters.}
  \label{fig:experiment3_QSIvariations}
\end{figure}

\paragraph{Reproducibility:} In both intra-subject scan/rescan experiments and among subjects we observe a consistently lower COV in QSI metrics compared to ADC measurements (see Figure~\ref{fig:experiment3_QSIvariations}). In particular, ADCxy shows the largest intra- and inter-subject variation (>25\%) in most tracts. In contrast, tract-specific QSI measurements vary less, and the majority of observed CoVs are between 5-10\%. 
\paragraph{Tract-specific differences:} Figure~\ref{fig:experiment3_tractplots} reports QSI and ADC values among all 9 subjects. We find significant group differences in QSI and ADC parameters between different white matter tracts. In particular, there are significant differences in the ADCxy and ADCz between the PT, AT and lateral tracts (p<0.01) that are not observable with QSI parameters. On the other hand, XY and Z QSI parameters showed significant differences between left and right lateral tracts (p<0.01), as well as differences between AT and l-LT (p<0.05). However, perpendicular QSI metrics alone do not show significant differences in any white matter tract. Both ADC and QSI metrics are significantly different between white matter tracts and gray matter (p<0.001). Figure~\ref{fig:experiment3_singlesubjectDPDF} which shows the detailed displacement profiles in one exemplary subject. It becomes even more apparent that the tract specific differences in the DPDFs are not limited to the XY plane but the profiles are also distinguishable in the Z direction.

\begin{figure}
  \centering
  \pgfimage[width=\textwidth]{pictures/chap3/sec3/qval-plots.pdf}
  \captionbelow{Mean and standard error of tract-specific QSI and ADC in XY and Z direction over all subjects.}
  \label{fig:experiment3_tractplots}
\end{figure} 

\begin{figure}
  \centering
  \pgfimage[width=9cm]{pictures/chap3/sec3/demo-tracts.pdf}
  \captionbelow{Exemplary FWHM maps and DPDFs in five voxels placed in white matter tracts AT, PT, l\&r-LT as well as inside GM for XY and Z directions.}
  \label{fig:experiment3_singlesubjectDPDF}
\end{figure}

%TODO: hotelling and correlation matrix 
\paragraph{QSI and ADC correlation:} We further observe significant correlations between ADC and QSI parameters in both XY and Z direction in all tracts. In XY direction, the strongest associations between FWHMxy and ADCx are found in AT and PT (p<0.001, $\rho$>0.8), although weaker correlations are also found in the r-LT (p<0.05, $\rho$=0.66). In Z direction, we find strong positive correlations only in PT and l-LT between ADCz and FWHMz (p<0.001, $\rho$>0.9) and negative correlation with P0z (p<0.001, $\rho$<-0.8) respectively. Over the whole SCA, we found correlation between all XY and Z measurements: the strongest correlation is found between ADCz and P0z (p<0.01, $\rho$=-0.9) and a weak correlation is found between ADCxy and FWHMxy(p<0.05, $\rho$>0.7) and P0xy(p<0.05, $\rho$<-0.7). 
\subsection*{Discussion \& Conclusion} QSI metrics obtained without sequence development, using standard DWI protocol available on a 3T clinical scanner, show a good reproducibility that is superior to simple ADC analysis. We observe tract-specific correlations between ADC and QSI parameters. However, especially in the lateral tracts, associations are weaker than in the anterior and posterior tracts, suggesting additional information in both XY and Z from QSI analysis in these columns. We further demonstrate that QSI parameters provides complementary metrics that allow discrimination of white matter tracts in healthy controls that cannot be distinguished with ADC alone. Our findings also suggest that the Z direction provides additional information to perpendicular measurements.
