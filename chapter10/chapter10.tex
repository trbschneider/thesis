%!TEX root = ../thesis.tex
\section{Summary}
The overarching aim of this dissertation was to develop imaging markers that can be useful in the clinical assessment of spinal cord pathologies such as traumatic \gls{SCI} and \gls{MS}. Over several studies, we have explored both well established methods such as \gls{ADC} and \gls{DTI} and experimental techniques such as \gls{QSI} and the ActiveImaging framework.
\paragraph{}
In Chapter\ref{chapter3} we have devised a new imaging protocol to visualize and quantify collateral nerves in the cord with \gls{DTI}. While the size of the study was small, we developed a sound methodical framework for \gls{DTI} acquisition and processing in the cord, which proved helpful for any analysis of \gls{SC} data beyond the scope of the study itself. The observation on partial volume averaging lead to further development of a novel partial volume correction method for whole cord averages of whole-cord {\gls{DWI}} metrics. We showed our PVA correction helps to reduce bias in average whole cord \gls{DTI} metrics and improves inter-subject variability. The achievable resolution in {\gls{SC}} DWI is low and PVA is a common problem to all {\gls{SC}} DWI techniques. Our is expected to improve other DWI derived imaging markers, e.g., those investigated in Experiments 3 and 4.


In Experiment 3, we turn towards the more experimental \gls{QSI}. The setup and analysis of in-vivo \gls{QSI} is more challenging, hence \gls{QSI} measures have only been reported in a few case studies of in-vivo human {\gls{SC}} so far. For the first time, we perform \gls{QSI} on a larger group of 9 healthy volunteers and examined inter- and intra-subject variability of \gls{QSI} measures over the whole cord area and specific white matter tracts in cervical cord. We demonstrate that variability of \gls{QSI} metrics is low both in individual subjects and among all controls, while conventional \gls{ADC} measures show higher variability. Further, \gls{QSI} metrics complement \gls{ADC} measures when distinguishing features of different white matter tracts.


In Experiment 4, we construct an imaging protocol that is specifically designed to directly estimate axon diameter and density in structures with known single fibre orientation such as the {\gls{SC}}. We adapt the framework of Alexander\cite{Alexander:2008} to this special case. In simulation we demonstrate improved efficacy of tissue microstructure estimates when applying our protocol compared to the more general protocol used in \cite{Alexander:2010}. We chose to perform the first test of the in-vivo implementation of our method  in the corpus callosum rather. Similar to the {\gls{SC}}, fibres in the corpus callosum follow one main direction. However, the \gls{CC} has the advantage of having a larger white matter area while suffering less from motion and aliasing artifacts than the {\gls{SC}}.
% 
% \section{Future work}
% Our focus in the remaining time of the PhD will be the completion of the experiments detailed above. In detail the required work will be:
% \paragraph*{Position dependency of diffusion metrics in the presence of collateral fibres}
% This study was performed on a 1.5T scanner which was replaced with a new 3T scanner in September 2009. Since we don't have access to the 1.5T system anymore, further work would require time consuming re-optimisation of the protocol and re-acquisition of data. Although the preliminary results justify further investigation, the continuation of this project is beyond the scope of this PhD.
% \paragraph*{Fuzzy partial volume correction of DTI parameters in the spinal cord}
% This work was recently accepted for presentation at ISMRM 2011 and also has been applied in the analysis of clinical \gls{SC} \gls{DTI} data (paper in preparation). We plan to finalise this project by submitting the description and proof of our method to a peer-reviewed journal, e.g., Magnetic Resonance in Medicine (Note) or Journal of Magnetic Resonance Imaging.
% \paragraph*{Reproducibity of QSI metrics}
% This study is of particular importance for upcoming clinical applications such as \cite{Ciccarelli:2011}. We therefore aim to finalise this study and publish the results in a peer-reviewed journal such as MRM as soon as possible. The steps required for completion are:
% \begin{enumerate}
%   \item Finish acquisition of scan and re-scan data for 10-15 healthy subjects
%   \item Analyse data and produce tract-specific reproducibibilty and reliability statistics
%   \item Write up results and submit to journal    
% \end{enumerate}             
% \paragraph*{Axon diameter and density estimation in single-fibre structures}
% The publication of the papers by \cite{Barazany:2009} and \cite{Alexander:2010} have generated great interest for axon diameter and density estimates. We regard the methods described in Section~\ref{sec:chap3:experiment4} as most promising new biomarkers for the investigation of \gls{SCI} recovery. Therefore our primary aim for future work is establish a clinically viable protocol for a standard clinical system. Based on the results of the MICCAI2010 submission, we plan to submit a methological paper to a peer-reviewed journal, possibly Magnetic Resonance in Medicine. Finally our next aim will be to demonstrate feasibility of microstructure estimation in in-vivo human spinal cord. We plan to carry out an initial feasibility study on healthy volunteers. The following steps are required:
% \begin{enumerate}
%   \item Finish optimisation of \gls{SF} acquisition protocol in \gls{CC}
%   \item Acquisition and analysis of \gls{CC} microstructure from 5 healthy volunteers using the \gls{SF} protocol
%   \item Write up and publication of \gls{CC} results
%   \item Develop acquisition protocol for \gls{SC} imaging
%   \item Show feasibility of \gls{SC} axon diameter and density imaging in 5 healthy volunteers
% \end{enumerate}
