%!TEX root = ../thesis.tex
The overarching aim of this dissertation was to develop imaging markers that can be helpful in the clinical assessment of spinal cord pathologies such as traumatic \gls{SCI} and \gls{MS}. In several studies we have explored both established methods such as \gls{ADC} and \gls{DTI} as well as more experimental approaches such as \gls{QSI} and the ActiveImaging framework.
\paragraph{}
In Chapter~\ref{chapter3} we have devised a new imaging protocol to visualize and quantify collateral nerves in the cord with \gls{DTI}. While the size of the study was small, we developed a sound methodical framework for \gls{DTI} acquisition and processing in the cord, which proved helpful for any analysis of \gls{SC} data beyond the scope of the study itself. The observations in this study also lead to the development of a novel partial volume correction method for whole cord averages of whole-cord {\gls{DTI}} metrics. We showed our PVA correction helps to reduce bias in average whole cord \gls{DTI} metrics and improves inter-subject variability. The achievable resolution in {\gls{SC}} DWI is low and PVA is a common problem to all {\gls{SC}} DWI techniques. In addition we have presented an example of a successful application of our \gls{PVA} correction method for the analysis of \gls{DTI} data acquired in chronic \gls{SCI} patients. 
\paragraph{}
In chapters~\ref{chapter5}\&\ref{chapter6} we turned towards \gls{QSI}, which offers the theoretically the most complete discription of the diffusion process in any tissue. However, in practise the setup and analysis of \emph{in-vivo} \gls{QSI} is very challenging, and only few \gls{QSI} studies have been reported \emph{in-vivo} human {\gls{SC}} so far. For the first time we presented here a systematic study of inter- and intra-subject variability of \gls{QSI} measures over the whole cord area and specific white matter tracts in cervical cord. We demonstrate that variability and reproducibility of \gls{QSI} metrics is very good, and, as shown in Chapter~\ref{chapter6}, can be improved even more when combined with modern scanner hardware and a carefully optimised \gls{SC} imaging set-up. We were unable to reproduce the clear distinction between WM tracts in the cord as seen in \emph{ex-vivo} high-field MRI experiments by \citet{Ong:2012}. However \gls{QSI} metrics in different white matter tracts complemented conventional \gls{ADC} estimates when distinguishing features of different white matter tracts. While we were not able to demonstrate a clear advantage of \gls{QSI} in healthy SC over conventional analysis, \gls{QSI} might be more senstitive to WM damage such as Wallerian degeneration, as shown by \citet{Farrell:2010} in a rat axotonomy model. Future work will explore the feasibility \gls{QSI} to such SC pathologies \emph{in-vivo} under realistic clincial condition, in a similar fashion to our study of healthy SC we presented here. The \gls{QSI} protocols and analysis pipeline we presented in chapters~\ref{chapter5} and \ref{chapter6} for two different scanners are currently at used at the two sites to study different spinal cord patient cohorts: (1) patients with brachial plexus avulsion scanned at the Wellcome Centre for Neuroimaging, UCL, using the Siemens Trio 3T system (2) \gls{MS} patients as part of a longitudinal study at Department of Neuroinflammation, UCL Institute of Neurology using the Philips 3T TX Achieva machine.

\paragraph{}
Finally, we presented in chapters~\ref{chapter7}--\ref{chapter9} a new imaging method, that is specifically designed to provide direct estimates of axon diameter and density indices in structures with known single fibre orientation such as the {\gls{SC}}. We thoroughly evaluated our method, going from using computer simulation via \emph{ex-vivo} monkey spinal cord samples to application in live humans, first in the corpus callosum and finally the spinal cord. We demonstrate that‚ our method produces very repeatable maps of axon diameter and axon density indices with very good SNR. However, the key achievement here is that our proposed protocol can be acquired in $\approx$30 minutes, which is crucial for future adoption into clinical studies. Our method extends naturally to different models or imaging sequences. We intend to extend the algorithm contraints of strictly unidirectional fibre directions to incorporate some degree of dispersion for a more realistic representation of healthy and pathologic white matter. We are also planning to use our protocols to study \emph{ex-vivo} healthy and MS human cord to better understand the role of our parameter estimates in the presence of tissue alteration. In the long term, our method must be evaluated in the context of a larger clinical study, e.g., of \gls{SCI} to determine further it's clinical benefit for \gls{SC} disease diagnosis and management purposes. 
\paragraph{}
A common theme that emerged from all the work we presented here is the importance of a holistic approach in optimising the imaging pipeline to the desired \gls{DWI} method and vice versa. We have demonstrated the clear benefits of adapting the acquisition protocol to the specific \gls{DWI} analysis (e.g. in Chapter~\ref{chapter3} and Chapter~\ref{chapter7}). On the other hand we have also shown, e.g. in chapters~\ref{chapter6} and \ref{chapter9} that a careful optimisation of the imaging parameters themself, such as image quality and positioning are equally important for any successful \gls{DWI} study. We believe that our contributions meet the initial goal of this thesis to improve existing acquisition protocols and analysis methods devise new imaging biomarkers for the study of \gls{SC} with diffusion MRI.
% 
% \section{Future work}
% Our focus in the remaining time of the PhD will be the completion of the experiments detailed above. In detail the required work will be:
% \paragraph*{Position dependency of diffusion metrics in the presence of collateral fibres}
% This study was performed on a 1.5T scanner which was replaced with a new 3T scanner in September 2009. Since we don't have access to the 1.5T system anymore, further work would require time consuming re-optimisation of the protocol and re-acquisition of data. Although the preliminary results justify further investigation, the continuation of this project is beyond the scope of this PhD.
% \paragraph*{Fuzzy partial volume correction of DTI parameters in the spinal cord}
% This work was recently accepted for presentation at ISMRM 2011 and also has been applied in the analysis of clinical \gls{SC} \gls{DTI} data (paper in preparation). We plan to finalise this project by submitting the description and proof of our method to a peer-reviewed journal, e.g., Magnetic Resonance in Medicine (Note) or Journal of Magnetic Resonance Imaging.
% \paragraph*{Reproducibity of QSI metrics}
% This study is of particular importance for upcoming clinical applications such as \cite{Ciccarelli:2011}. We therefore aim to finalise this study and publish the results in a peer-reviewed journal such as MRM as soon as possible. The steps required for completion are:
% \begin{enumerate}
%   \item Finish acquisition of scan and re-scan data for 10-15 healthy subjects
%   \item Analyse data and produce tract-specific reproducibibilty and reliability statistics
%   \item Write up results and submit to journal    
% \end{enumerate}             
% \paragraph*{Axon diameter and density estimation in single-fibre structures}
% The publication of the papers by \cite{Barazany:2009} and \cite{Alexander:2010} have generated great interest for axon diameter and density estimates. We regard the methods described in Section~\ref{sec:chap3:experiment4} as most promising new biomarkers for the investigation of \gls{SCI} recovery. Therefore our primary aim for future work is establish a clinically viable protocol for a standard clinical system. Based on the results of the MICCAI2010 submission, we plan to submit a methological paper to a peer-reviewed journal, possibly Magnetic Resonance in Medicine. Finally our next aim will be to demonstrate feasibility of microstructure estimation in \emph{in-vivo} human spinal cord. We plan to carry out an initial feasibility study on healthy volunteers. The following steps are required:
% \begin{enumerate}
%   \item Finish optimisation of \gls{SF} acquisition protocol in \gls{CC}
%   \item Acquisition and analysis of \gls{CC} microstructure from 5 healthy volunteers using the \gls{SF} protocol
%   \item Write up and publication of \gls{CC} results
%   \item Develop acquisition protocol for \gls{SC} imaging
%   \item Show feasibility of \gls{SC} axon diameter and density imaging in 5 healthy volunteers
% \end{enumerate}
