%!TEX root = ./thesis.tex
%%%%%%%%%%%%%%%%%%%%%%%%%%%%%%%%%%%%%%%%%
% Thesis Configuration File
%
% The main lines to change in this file are in the DOCUMENT VARIABLES
% section, the rest of the file is for advanced configuration.
%
%%%%%%%%%%%%%%%%%%%%%%%%%%%%%%%%%%%%%%%%%

%----------------------------------------------------------------------------------------
%	DOCUMENT VARIABLES
%	Fill in the lines below to enter your infcmation into the thesis template
%	Each of the commands can be cited anywhere in the thesis
%----------------------------------------------------------------------------------------

% Remove drafting to get rid of the '[ Date - classicthesis version 4.0 ]' text at the bottom of every page
\PassOptionsToPackage{pdfspacing,eulerchapternumbers,listings,subfig,parts,floatperchapter,manychapters,dottedtoc,subfig}{classicthesis}
% Available options: drafting parts nochapters linedheaders eulerchapternumbers beramono eulermath pdfspacing minionprospacing tocaligned dottedtoc manychapters listings floatperchapter subfig
% Adding 'dottedtoc' will make page numbers in the table of contents flushed right with dots leading to them
\setcounter{tocdepth}{1}

%%% THESIS VARIABLES
\newcommand{\myTitle}{Spinal Cord Diffusion Imaging\xspace}
\newcommand{\mySubtitle}{Imaging Challenges and Prognostic Value \xspace}
\newcommand{\myDegree}{Doctor of Philosophy\xspace}
\newcommand{\myName}{Torben Schneider\xspace}
\newcommand{\myProf}{Claudia Wheeler-Kingshott\xspace}
\newcommand{\myOtherProf}{Daniel Alexander\xspace}
\newcommand{\mySupervisor}{Put name here\xspace}
\newcommand{\myFaculty}{UCL Institute of Neurology\xspace}
\newcommand{\myDepartment}{Department of Neuroinflammation\xspace}
\newcommand{\myUni}{University College London\xspace}
\newcommand{\myLocation}{London, United Kingdom\xspace}
\newcommand{\myTime}{\today\xspace}
\newcommand{\myVersion}{}

%%% COUNTER FIXES
\newcounter{dummy} % Necessary for correct hyperlinks (to index, bib, etc.)

%----------------------------------------------------------------------------------------
%	PACKAGES
%----------------------------------------------------------------------------------------

%%%% PACKAGES FOR CLASSIC THESIS
\usepackage{xspace} % To get the spacing after macros right
\usepackage{mparhack} % To get marginpar right
\usepackage[dvipsnames]{xcolor}
\usepackage[british,UKenglish,USenglish,english,american]{babel}
%\pdfprotrudechars=2
%\pdfadjustspacing=2


%%%% THESIS PACKAGES
%math
\usepackage{amssymb}
\usepackage{amsmath}
\usepackage{xfrac}
\newcommand{\mat}[1]{\ensuremath{\mathbf{#1}}}

%tikz & pgf
\usepackage{tikz}
\usepackage{pgfplots}
\usetikzlibrary[calc]
\usetikzlibrary[decorations.pathreplacing]
\usetikzlibrary[svg.path]
\usepgfplotslibrary{polar}
\usepackage{pgfboxplot} %custom
\usepackage{sequenceplot} %custom
\usetikzlibrary{external}
\tikzexternalize[mode=list and make]
\tikzsetexternalprefix{tikz/}
\tikzset{external/system call={lualatex \tikzexternalcheckshellescape -halt-on-error -interaction=batchmode -jobname "\image" "\texsource"}}
%optional png setup
\tikzset{
  	% Add size information to the .dpth file (png is in density not size)
   /pgf/images/external info,
    png export/.style={
        external/system call=%
        {lualatex \tikzexternalcheckshellescape -halt-on-error -interaction=batchmode -jobname "\image" "\texsource"; convert -density 72 -transparent white "\image.pdf" "\image.png"},
	},
    % Use the png export AND the import
%    use png/.style={png export,png import},
    use png/.style={},
    png import/.code={%
        \tikzset{%
            /pgf/images/include external/.code={%
                % Here you can alter to whatever you want
                % \pgfexternalwidth is only available if /pgf/images/external info
                % is set
            		\includegraphics[width=\pgfexternalwidth,height=\pgfexternalheight]%
					{##1.png}%
            }%
        }%
    }%
}


%figures
\usepackage{graphicx}
\usepackage{float}
\usepackage{placeins}
\usepackage[export]{adjustbox}
\usepackage{subfig}  



%bibliography 
%\usepackage{natbib}
\usepackage{csquotes}
\usepackage[style=numeric-comp,firstinits=true,maxcitenames=1,natbib=true,backend=biber, doi=false,isbn=false,url=false]{biblatex}
%
%
%\newcounter{cbx:totalcite}
%\newbibmacro{textcite:count}{%
%  \iffirstcite
%    {\setcounter{cbx:totalcite}{1}}
%    {\addtocounter{cbx:totalcite}{1}}}
%
%\newbibmacro{textcite:delim}{%
%  \iffirstcite
%    {}
%    {\iflastcite
%       {\ifnumgreater{\value{cbx:totalcite}}{2}
%          {\finalandcomma\space\bibstring{and}\space}
%          {\addspace\bibstring{and}\space}}
%       {\addcomma\space}}}
%
%\def\iffirstcite{%
%  \ifboolexpr{ ( test {\ifnumequal{\value{multicitetotal}}{0}}
%                 and test {\ifnumequal{\value{citecount}}{1}} )
%               or ( test {\ifnumgreater{\value{multicitetotal}}{0}}
%                    and test {\ifnumequal{\value{multicitecount}}{1}}
%                    and test {\ifnumequal{\value{citecount}}{1}} ) }}
%
%\def\iflastcite{%
%  \ifboolexpr{ test {\ifnumequal{\value{citecount}}{\value{citetotal}}}
%               and test {\ifnumequal{\value{multicitecount}}{\value{multicitetotal}}} }}
%
%\DeclareCiteCommand{\textcite}
%  {}% <precode>
%  {\usebibmacro{citeindex}% <loopcode>
%   \usebibmacro{textcite:count}%
%   \usebibmacro{textcite:delim}%
%   \usebibmacro{textcite}}
%  {} % <sepcode>
%  {\usebibmacro{textcite:postnote}}% <postcode>
%
%\DeclareMultiCiteCommand{\textcites}{\textcite}{}

%for possessive cite
\newcommand{\posscite}[1]{\citeauthor{#1}'s \autocite*{#1}} 

% glossaries
\usepackage[nomain,acronym,nonumberlist,section=chapter]{glossaries}
\glossarystyle{listdotted}
\setlength{\glslistdottedwidth}{12em}




%title
%\usepackage{titlesec}
%\usepackage[title, titletoc]{appendix}
\usepackage{appendix}

%misc
\usepackage[super]{nth}

%drafttools
\usepackage{todonotes}
\usepackage{ifdraft}


%----------------------------------------------------------------------------------------
%	FLOATS: TABLES, FIGURES AND CAPTIONS SETUP
%----------------------------------------------------------------------------------------
%tables
\usepackage{multirow}
\usepackage{ctable}
\usepackage{array}
\usepackage{booktabs}
\usepackage{colortbl}
\usepackage{tabularx} % Better tables
\usepackage{etoolbox} % Modify tables


%new columns with fixed width
\newcolumntype{x}[1]{ >{\centering\arraybackslash\hspace{0pt}}p{#1}}
\newcolumntype{y}[1]{ >{\arraybackslash\hspace{0pt}}p{#1}}


\setlength{\extrarowheight}{3pt} % Increase table row height
\newcommand{\tableheadline}[1]{\multicolumn{1}{c}{\spacedlowsmallcaps{#1}}}
\newcommand{\myfloatalign}{\centering} % To be used with each float for alignment

\captionsetup[table]{position=t}
\setkomafont{caption}{\captfont}
\setkomafont{captionlabel}{\usekomafont{caption}}
%\setcapindent{1em}
\captionsetup{format=hang,font+={small,it},labelfont+={bf}}


%change font and size of tables
\makeatletter
\AtBeginEnvironment{table}{%
  \def\@floatboxreset{\reset@font\small\floatfont\@setminipage}%
}
\patchcmd{\@xfloat}{\normalsize}{\selectfont}{}{}
\makeatother

%frame tables
\makeatletter
\usepackage{environ}
\NewEnviron{tableframe}[1][\textwidth]{%
	\tikzset{external/export next=false}
	\begin{tikzpicture}
		\node [tablebox] (box){
			\begin{minipage}{#1}
       		 	\BODY
		    \end{minipage}%  	  	
  		};
  	\end{tikzpicture}
}
\NewEnviron{captionframe}[1][\textwidth]{%
	\tikzset{external/export next=false}
	\begin{tikzpicture}
		\node [captionbox] (box){
			\begin{minipage}{#1}
       		 	\BODY
		    \end{minipage}%  	  	
  		};
  	\end{tikzpicture}
}

%\AtBeginEnvironment{table}{\begin{tableframe}}%
%\AtEndEnvironment{table}{\end{tableframe}}%
\makeatother


%----------------------------------------------------------------------------------------
%	HYPERREFERENCES
%----------------------------------------------------------------------------------------

\PassOptionsToPackage{hyperfootnotes=false}{hyperref}
\usepackage{hyperref}  % backref linktocpage pagebackref
\hypersetup{
% Uncomment the line below to remove all links (to references, figures, tables, etc)
%draft, 
colorlinks=true, linktocpage=true, pdfstartpage=3, pdfstartview=FitV,
% Uncomment the line below if you want to have black links (e.g. for printing black and white)
%colorlinks=false, linktocpage=false, pdfborder={0 0 0}, pdfstartpage=3, pdfstartview=FitV, 
breaklinks=true, pdfpagemode=UseNone, pageanchor=true, pdfpagemode=UseOutlines,
plainpages=false, bookmarksnumbered, bookmarksopen=true, bookmarksopenlevel=1,
hypertexnames=true, pdfhighlight=/O, urlcolor=webbrown, linkcolor=RoyalBlue, citecolor=webgreen,
%------------------------------------------------
% PDF file meta-information
pdftitle={\myTitle},
pdfauthor={\textcopyright\ \myName, \myUni, \myFaculty},
pdfsubject={},
pdfkeywords={},
pdfcreator={xelatex},
pdfproducer={LaTeX}
%------------------------------------------------
}   

%----------------------------------------------------------------------------------------
%	BACKREFERENCES
%----------------------------------------------------------------------------------------

\usepackage{ifthen} % Allows the user of the \ifthenelse command
\newboolean{enable-backrefs} % Variable to enable backrefs in the bibliography
\setboolean{enable-backrefs}{false} % Variable value: true or false

\newcommand{\backrefnotcitedstring}{\relax} % (Not cited.)
\newcommand{\backrefcitedsinglestring}[1]{(Cited on page~#1.)}
\newcommand{\backrefcitedmultistring}[1]{(Cited on pages~#1.)}
\ifthenelse{\boolean{enable-backrefs}} % If backrefs were enabled
{
\PassOptionsToPackage{hyperpageref}{backref}
\usepackage{backref} % to be loaded after hyperref package 
\renewcommand{\backreftwosep}{ and~} % separate 2 pages
\renewcommand{\backreflastsep}{, and~} % separate last of longer list
\renewcommand*{\backref}[1]{}  % disable standard
\renewcommand*{\backrefalt}[4]{% detailed backref
\ifcase #1 
\backrefnotcitedstring
\or
\backrefcitedsinglestring{#2}
\else
\backrefcitedmultistring{#2}
\fi}
}{\relax} 


%----------------------------------------------------------------------------------------
%	CHAPTER AND LAYOUT MODS
%----------------------------------------------------------------------------------------

\newcommand\chaptersub[2]{\chapter
  [#1\hfil\hbox{}\protect\linebreak{\itshape#2}]%
  {#1}%
}


%----------------------------------------------------------------------------------------
%	AUTOREFERENCES SETUP
%	Redefines how references in text are prefaced for different 
%	languages (e.g. "Section 1.2" or "section 1.2")
%----------------------------------------------------------------------------------------

\makeatletter
\@ifpackageloaded{babel}
{
\addto\extrasbritish{
\renewcommand*{\figureautorefname}{Figure}
\renewcommand*{\tableautorefname}{Table}
\renewcommand*{\partautorefname}{Part}
\renewcommand*{\chapterautorefname}{Chapter}
\renewcommand*{\sectionautorefname}{Section}
\renewcommand*{\subsectionautorefname}{Section}
\renewcommand*{\subsubsectionautorefname}{Section}
}

\providecommand{\subfigureautorefname}{\figureautorefname} % Fix to getting autorefs for subfigures right
}{\relax}
\makeatother

%----------------------------------------------------------------------------------------

\usepackage{classicthesis} 


%----------------------------------------------------------------------------------------
%	CHANGING TEXT AREA 
%----------------------------------------------------------------------------------------

\linespread{1.05} % a bit more for Palatino
%\areaset[current]{312pt}{761pt} % 686 (factor 2.2) + 33 head + 42 head \the\footskip
%\setlength{\marginparwidth}{7em}%
%\setlength{\marginparsep}{2em}%

%----------------------------------------------------------------------------------------
%	USING DIFFERENT FONTS
%----------------------------------------------------------------------------------------


%\usepackage{tgpagella}
%\fontfamily{qpl}\selectfont
%\usepackage{libertineotf}
\usepackage{fontspec,microtype}
\PassOptionsToPackage{protrusion=true, expansion=true,final}{microtype}
\newfontfeature{Microtype}{protrusion=default;expansion=default;}
\directlua{fonts.protrusions.setups.default.factor=500}
\directlua{fonts.expansions.setups.default.stretch=10.5}

% \setmainfont[
% 	Path = ./fonts/ ,
% 	Extension = .ttf ,
% 	UprightFont = *regular,
% 	BoldFont = *bold,
% 	ItalicFont = *italic,
% 	BoldItalicFont = *bolditalic,
% ]{serif72beta}


\setmainfont[
	Microtype,
	Ligatures=TeX,
	Path = ./fonts/ ,
	Extension = .otf ,
	UprightFont = *-Regular,
	BoldFont = *-Bold,
	ItalicFont = *-It,
	BoldItalicFont = *-BoldIt,
]{MinionPro}

%\setmainfont[
%	Path = ./fonts/ ,
%	Extension = .otf ,
%	UprightFont = *,
%	BoldFont = *-Bold,
%	ItalicFont = *-Italic,
%	BoldItalicFont = *-BoldItalic,
%]{BemboStd}

% \setmainfont[
% 	Microtype,
% 	Ligatures=TeX,
% 	Path = ./fonts/ ,
% 	Extension = .otf ,
% 	UprightFont = *-regular,
% 	BoldFont = *-bold,
% 	ItalicFont = *-italic,
% 	BoldItalicFont = *-bolditalic,
% 	SmallCapsFont = LinLibertine_C,	
% ]{texgyrepagella}

% \setmainfont[
% 	Microtype,
% 	Ligatures=TeX,
% 	Path = ./fonts/ ,
% 	Extension = .otf ,
% 	UprightFont = *-regular,
% 	BoldFont = *-bold,
% 	ItalicFont = *-italic,
% 	BoldItalicFont = *-bolditalic,
% 	SmallCapsFont = LinLibertine_C,	
% ]{texgyreschola}

%\newfontfamily\chapfont[
%	Path=./fonts/,
%	UprightFont = SourceSansPro-Bold.otf,
%	BoldFont = SourceSansPro-Bold.otf,
%	ItalicFont = SourceSansPro-BoldIt.otf,
%	BoldItalicFont = SourceSansPro-BoldIt.otf,
%]{sourcesans}

%\newfontfamily\subsectfont[
%	Path = ./fonts/ ,
%	Path=./fonts/,
%	UprightFont = SourceSansPro-Regular.otf,
%	BoldFont = SourceSansPro-Semibold.otf,
%	ItalicFont = SourceSansPro-It.otf,
%	BoldItalicFont = SourceSansPro-SemiboldIt.otf,
%]{sourcesans}



% \newfontfamily\chapfont[
% 	Scale=MatchUppercase,
% 	Path = ./fonts/ ,
% 	Extension = .otf ,
% 	UprightFont = *-regular,
% 	BoldFont = *-bold,
% 	ItalicFont = *-italic,
% 	BoldItalicFont = *-bolditalic
%  ]{texgyreadventor}
% 
% \newfontfamily\sectfont[
% 	Scale=MatchUppercase,
% 	Path = ./fonts/ ,
% 	Extension = .otf ,
% 	UprightFont = *-regular,
% 	BoldFont = *-bold,
% 	ItalicFont = *-italic,
% 	BoldItalicFont = *-bolditalic
%  ]{texgyreadventor}
% 
% \newfontfamily\subsectfont[
% 	Scale=MatchUppercase,
% 	Path = ./fonts/ ,
% 	Extension = .otf ,
% 	UprightFont = *-regular,
% 	BoldFont = *-bold,
% 	ItalicFont = *-italic,
% 	BoldItalicFont = *-bolditalic
%  ]{texgyreadventor}
 
\newfontfamily\chapfont[
	Scale=MatchUppercase,
 	Path = ./fonts/ ,
	Extension = .ttf ,
 	UprightFont = *Display.ttf,
 	BoldFont = *_Bd.ttf,
 	ItalicFont = *_It.ttf,
 	BoldItalicFont = *_BdIt.ttf,
 ]{Aller}


\newfontfamily\sectfont[
	Scale=MatchUppercase,
 	Path = ./fonts/ ,
	Extension = .ttf ,
 	UprightFont = *Display.ttf,
 	BoldFont = *_Bd.ttf,
 	ItalicFont = *_It.ttf,
 	BoldItalicFont = *_BdIt.ttf,
 ]{Aller}

\newfontfamily\subsectfont[
	Scale=MatchUppercase,
 	Path = ./fonts/ ,
	Extension = .ttf ,
 	UprightFont = *_Rg.ttf,
 	BoldFont = *_Bd.ttf,
 	ItalicFont = *_It.ttf,
 	BoldItalicFont = *_BdIt.ttf,
 ]{Aller}


\newfontfamily\captfont[
	Microtype,
	Ligatures=TeX,
	Scale=MatchUppercase,
 	Path = ./fonts/ ,
	Extension = .ttf ,
 	UprightFont = *_Lt.ttf,
 	BoldFont = *_Bd.ttf,
 	ItalicFont = *_LtIt.ttf,
 	BoldItalicFont = *_BdIt.ttf,
 ]{Aller}

\newfontfamily\floatfont[
	Ligatures=TeX,
	Scale=MatchLowercase,
	Path = ./fonts/ ,
	Path=./fonts/,
	UprightFont = SourceSansPro-Regular.otf,
	BoldFont = SourceSansPro-Semibold.otf,
	ItalicFont = SourceSansPro-It.otf,
	BoldItalicFont = SourceSansPro-SemiboldIt.otf,
 ]{sourcesans}

\newfontfamily\hdrfont[
	Ligatures=TeX,
	Scale=MatchLowercase,
	Path = ./fonts/ ,
	Path=./fonts/,
	UprightFont = SourceSansPro-Regular.otf,
	BoldFont = SourceSansPro-Semibold.otf,
	ItalicFont = SourceSansPro-It.otf,
	BoldItalicFont = SourceSansPro-SemiboldIt.otf,
 ]{sourcesans} 

\newfontfamily\titfont[
	Ligatures=TeX,
	Scale=MatchLowercase,
	Path = ./fonts/ ,
	Path=./fonts/,
	UprightFont = SourceSansPro-Black.otf,
	ItalicFont = SourceSansPro-It.otf,
 ]{sourcesans} 


%%% all math font business
\usepackage[partial=upright,nabla=upright]{unicode-math}
\setmathfont[Scale=MatchLowercase]{TG Pagella Math}
\setmathfont[range={\mathcal,\mathbfcal},StylisticSet=1]{XITS Math}
%\usepackage[sc]{mathpazo}
%\usepackage{fourier}
%change microtype features

%make lines flow better with gyre
\emergencystretch=1.2em

\DeclareMathSizes{12}{11}{9}{7}
\DeclareMathSizes{11}{10}{8}{7}


%for tables and tikz figures change font to sans to match tables
% \setmathfont[version=mysans,Scale=MatchLowercase,range={\mathit,\mathup},Path = ./fonts/]{SourceSansPro-It.otf}%
% \setmathfont[version=mysans,Scale=MatchLowercase]{XITS Math}
% 
%change tikz font for everything
\tikzstyle{every picture}+=[font=\floatfont]

\RequirePackage{textcase} % for \MakeTextUppercase
 

    \titleformat{\part}[display]
        {\centering\huge\titfont}%
        {\thispagestyle{empty}\partname~\MakeTextUppercase{\thepart}}{1em}%
        {\textbf}[\bigskip\normalfont\normalsize\color{Black}]
        \renewcommand{\thepart}{\roman{part}}%
   
	

    
\titleformat{\chapter}[display]
         {\relax}{\mbox{}\oldmarginpar{\vspace*{-3\baselineskip}\color{halfgray}\chapterNumber\thechapter}}{0pt}%
         {\Large\chapfont\textbf}[\vspace*{.0\baselineskip}\titlerule]%   
 
     % sections
     \titleformat{\section}
         {\relax}{\sectfont\textbf\thesection}{0.5em}{\sectfont\textbf}
	 \titlespacing*{\section}{0pt}{5ex}{3ex}
     % subsections
     \titleformat{\subsection}
         {\relax}{\subsectfont\normalsize\textbf\thesubsection}{1em}		{\subsectfont\normalsize\textbf}
	 \titlespacing*{\subsection}{0pt}{3ex}{1ex}
     % subsections
     \titleformat{\subsubsection}
         {\relax}{\subsectfont\normalsize\itshape\thesubsubsection}{1em}		{\subsectfont\normalsize\itshape}
	 \titlespacing*{\subsubsection}{0pt}{3ex}{1ex}


     % paragraphs
     \titleformat{\paragraph}[runin]
         {\normalfont\normalsize\itshape}{\theparagraph}{0pt}{\normalfont\normalsize\itshape}    % no small caps for paragraphs
    % descriptionlabels
		\renewcommand{\descriptionlabel}[1]{\hspace*{\labelsep}{\itshape{#1}}} 


%%%TOC
\renewcommand{\cftpartfont}{\titfont}%
\renewcommand{\cftpartpagefont}{\floatfont}%

\renewcommand{\cftchappresnum}{\floatfont}%
\renewcommand{\cftchapfont}{\floatfont\textbf}%
\renewcommand{\cftchappagefont}{\normalfont}%

\renewcommand{\cftsecpresnum}{\floatfont}%
\renewcommand{\cftsecfont}{\floatfont}%
	
\makeatletter
%%%%% headers
%% kick the spaced caps into oblivion
%\renewcommand{\spacedlowsmallcaps}[1]{set in ssc ::: #1 :::}%
\clearscrheadings
\setheadsepline{0pt}

\renewcommand{\chaptermark}[1]{\markboth{{#1}}{{#1}}}
\renewcommand{\sectionmark}[1]{\markright{\thesection\enspace{#1}}}
 
\lehead{\mbox{\llap{\small\thepage\kern2em}\textit{\headmark}\hfil}}
\rohead{\mbox{\hfil{\textit{\headmark}}\rlap{\small\kern2em\thepage}}}
%
\renewcommand{\headfont}{\hdrfont\small}  
%
%\DeclareRobustCommand{\fixBothHeadlines}[2]{} % <--- ToDo
%    % hack to get the content headlines right (thanks, Lorenzo!)
%\def\toc@heading{%
%	\chapter*{\contentsname}
%	\@mkboth
%		{\hdrfont\itshape \contentsname}
%		{\hdrfont\itshape \contentsname}
%}


%restore original chapter definition
\renewcommand*{\chapter}{\oldchap}

\let\oldchapa=\chapter
\renewcommand*{\chapter}{%
    \secdef{\Chap}{\ChapS}%
}
\renewcommand*\ChapS[1]{\oldchapa*{#1}}%
\renewcommand*\Chap[2][]{\oldchapa[\texorpdfstring{{#1}}{#1}]{#2}}%

%restore old part definition
\renewcommand*{\part}{\oldpart}
    
\let\oldparta=\part%
\renewcommand*{\part}{%
	     \secdef{\Part}{\PartS}%
}%
\renewcommand\PartS[1]{\oldparta*{#1}}%
\renewcommand\Part[2][]{\oldparta[\texorpdfstring{{#1}}{#1}]{#2}}% spacedallcaps 

\renewcommand{\tocEntry}[1]{
	\texorpdfstring{#1}{#1}%
}

\makeatother

%----------------------------------------------------------------------------------------
%	USEFUL COMMANDS
%----------------------------------------------------------------------------------------
%%%%% USEFUL COMMANDS %%%%%%%%%%

\newcommand{\q}{{\ensuremath{\mathbf{q}}}}

\newcommand{\SF}{SF}
\newcommand{\OI}{OI}
\newcommand{\SD}{\SF{}PULSES}
\newcommand{\DO}{\SF{}DIRS}
\newcommand{\FD}{\SF{}}

\newcommand{\SFshort}{\SF$_{90}$}
\newcommand{\SFlong}{\SF$_{360}$}
\newcommand{\OIlong}{\OI$_{360}$}

\newcommand{\FDmod}{\FD{\ensuremath{_{mod}}}}
\newcommand{\SFasym}{A\SF{}}

\tikzstyle{protocolbox} = [draw=black, thin, rectangle, rounded corners, inner xsep=5pt, outer ysep=0pt] \tikzstyle{protocolheader} = [draw=none, fill=gray!30, rounded corners, inner sep = 5pt] \tikzstyle{protocoltext} = [draw=black, fill=white, rounded corners, inner sep = 5pt]

%vertebrae colorscheme
\definecolor{darkgray}{HTML}{666666}
\definecolor{mediumgray}{HTML}{999999}
\definecolor{lightgray}{HTML}{CCCCCC}
\definecolor{green}{HTML}{ABDB25}


\setlength{\abovecaptionskip}{4pt} % space above main figure caption was 8pt
\setlength{\belowcaptionskip}{4pt} % space above main figure caption was 8pt
\tikzstyle{tablebox} = [draw=none, fill=lightgray!50!white, inner xsep=5pt, inner ysep=5pt, outer sep=0pt]
\tikzstyle{captionbox} = [draw=none, fill=mediumgray, inner xsep=5pt, inner ysep=0pt, outer sep=0pt]


%make all titlecase
\usepackage{stringstrings}
\addlcwords{all of the and a an is before on}
\DeclareRobustCommand*\MakeTitlecase[1]{%
  \caselower[e]{#1}%
  \capitalizetitle{\thestring}%
}

\makeatletter
