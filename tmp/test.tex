
\documentclass[draft]{article}
\usepackage{subfig}
\usepackage{pgf}
\usepackage{pgfplots}
\usepackage{blindtext}
\usepackage{nonfloat}
\usepackage{booktabs}
\usepackage{natbib}


\newcommand{\captionbelow}{\caption}
\newcommand{\captionabove}{\caption}
\newcommand{\gls}{\textbf}
\newcommand{\glspl}{\textbf}

\begin{document}

\chapter{Q-space imaging of the healthy cervical spinal cord}
\label{sec:chap5 QSI in cord}
In this chapter we describe a study that investigates accuracy and sensitivity of spinal cord \gls{QSI} metrics in healthy controls. As discussed above (see Section~\ref{sec:qspace}), various studies on experimental MRI systems have shown that \gls{QSI} can provide accurate information about microscopic restriction in excised tissue \citep{Assaf:2000,Bar-Shir:2008,Ong:2008}. \gls{QSI} requires an extensive sampling of different $q$-values. This restricts the number of diffusion gradient directions that can be sampled when scan time is limited. Therefore, application of \gls{QSI} in the \gls{CNS} mostly focuses on the \gls{SC} since its relatively simple white matter structure doesn't require high angular resolution of gradient directions. \Citep{Ong:2008,Ong:2010,Ong:2011} measured \gls{QSI} parameters in different white matter tracts of excised rat spinal cord and were able to correlate the \gls{QSI} parameters with the axon diameter in different white matter regions.

Although the conditions for true \gls{QSI}, such as the short gradient pulse, are impossible to achieve in clinical systems, studies such as \cite{TODO} in the human brain and \citet{Farrell:2008} in the spinal cord have shown the great potential in the assessment of {\gls{SC}} white matter and white pathologies such as MS. With the emerge of \gls{QSI}

However, most clinical \gls{QSI} studies only focus on a small number of patients and failed to demonstrate the reliability of \gls{QSI}. The aim of this study is to report reproducibility of \gls{QSI} metrics in the cervical {\gls{SC}} on a standard 3T clinical MRI scanner. We also assessed \gls{QSI} measures both in-plane (XY) and parallel to the main {\gls{SC}} axis (Z), not presented before. We compare \gls{QSI} measures derived in gray matter and different ascending and descending white tracts of the cervical {\gls{SC}} in healthy subjects and investigate associations between \gls{QSI} parameters and conventional apparent diffusion coefficient (\gls{ADC}) measures, both in plane and along the cord.

We the following sections will present data from two experiments. The first experiment investigated 9 healthy controls that were scanned at the Wellcome Trust Centre for Neuroimaging Imaging centre as part of a pilot study to investigate the effect of brachial plexus avulsion. Our preliminary findings were submitted for presentation at the Annual Meeting of the International Society of Magnetic Resonance in Medicine and were accepted for oral presentation \citep{Schneider:XXX}. The inital set-up had several shortcomings and we decided to reimplement an improved protocol on our in-house Philips 3T MRI scanner to scan 10 healthy volunteers. The results of this experiment are shown in section~\ref{TODO}.

\section{Experiment 1}
\subsection{Methods}
\paragraph{Study design}
9 right-handed male healthy subjects were recruited (mean age 35�11yrs) to be scanned on a 3T Tim Trio (Siemens Healthcare, Erlangen). Three subjects were recalled for a second scan on a different day to assess intra-subject reproducibility of \gls{QSI} derived parameters.
\paragraph{Data acquisition}
In each subject we perform cardiac-gated high {\gls{bvalue}} axial{\gls{DWI}}(matrix=96x96, b-spline interpolated to 192x192 in image space, FOV=144x144$mm^2$, slice thickness=5mm, 20 slices, TE=110ms, TR$\approx$4000ms). The \gls{QSI} set-up is based on parameters found in the most recent clinical \gls{QSI} study \citep{Farrell:2008}. However, our gradient system only allowed maximum \gls{gstr} of 23mT/m (\citet{Farrell:2008}: 60mT/m). To achieve similar $q$-values is was necessary increase the gradient duration \gls{smalldel} to 51ms. Reproduction of the protocol was further complicated by a limitation in the scanner software, which only permits $b$-value to be specified in multiples of 50 $m/s^2$ and means that $q$-values can not be exactly linearly spaced. We acquire a total of 32 $b$-values between 0-3000s/mm2 in three different{\gls{DWI}}directions: two directions perpendicular (XY) and one parallel (Z) to the main {\gls{SC}} axis. The full protocol is given in Table~\ref{tab:chap5exp1 protocol}.
\paragraph{Data processing}
Similar to \citet{Farrell:2008} the two perpendicular diffusion directions were averaged to increase the signal-to-noise ratio. The measurements are linearly regridded to be equidistant in q-space and the  {\gls{dpdf}} is computed using inverse Fast Fourier Transformation. To increase the resolution of the  {\gls{dpdf}}, the signal was extrapolated in q-space to a maximum q=166mm$^{-1}$ by fitting a bi-exponential decay curve to the{\gls{DWI}}data as suggested in \citet{Cohen:2002, Farrell:2008}. Figure~\ref{fig:chapter5 porcessing pipeline} illustrates the processing pipeline. Maps of the full width at half maximum and zero displacement probability were derived for XY and Z as described in Section~\ref{sec:qspace}. For comparison we also computed the apparent diffusion coefficient (see section \ref{subsec:adc}) from the monoexponential part of the decay curve (b < 1100s/mm2) as in \citet{Farrell:2008} for both XY and Z directions using a constrained non-linear least squared fitting algorithm. Figure~\ref{fig:chapter5 exemplary maps} shows both \gls{ADC} maps and the four \gls{QSI} parameter maps in one randomly chosen subject.

\begin{figure}
\centering
\subfloat[ADC$_{xy}$ $\times 10^{-9}m^2/s$]{
    \pgfimage[width=0.4\textwidth]{test}
}
\subfloat[ADC$_{z}$ $\times 10^{-9}m^2/s$]{
    \pgfimage[width=0.4\textwidth]{test}
}\\
\subfloat[P0$_{xy}$]{
    \pgfimage[width=0.4\textwidth]{test}
}
\subfloat[FWHM$_{xy}$ $\times 10^{-6}m$]{
    \pgfimage[width=0.4\textwidth]{test}
}\\
\subfloat[P0$_{z}$]{
    \pgfimage[width=0.4\textwidth]{test}
}
\subfloat[FWHM$_{z}$ $\times 10^{-6}m$]{
    \pgfimage[width=0.4\textwidth]{test}
}
\caption{ADC maps and QSI parameter maps in one exemplary subject at the level of the C2-C3 disc.}
\label{fig:chapter5 exemplary maps}
\end{figure}

\paragraph{ROI analysis} We semi-automatically delineate the whole cervical {\gls{SC}} area (SCA) between levels C1 and C3 on the b=0 images using the active surface segmentation by \citet{Horsfield:2010} available in Jim6. We perform a morphological erosion (2 iterations) of the obtained segmentation mask to exclude voxels with potential partial-volume average effect from surrounding \gls{CSF}. In addition, four regions of interest were manually placed in specific white matter tracts and one ROI was positioned in the gray matter on all slices between level C1 and C3. The four white matter regions comprised the left and right tracts (l\&r-LT) running in the lateral columns and the anterior (AT) and posterior tracts (PT) similar to \citet{Hesseltine:2006,Freund:2010}.

  \begin{figure}
      \centering
%      \pgfimage{chapter5/figs/exp1_ROIs.pdf}
      \caption{}
      \label{fig:chapter5 exp1 ROIs}
  \end{figure}

  \begin{table}
      \centering
     \caption{QSI Protocol}
     \begin{tabular}{|l|l|l|l|l|}
       \hline
       % after \\: \hline or \cline{col1-col2} \cline{col3-col4} ...
       a & b & c & d & e \\
       \hline
     \end{tabular}
     \label{fig:chapter5 exp1 ROIs}
  \end{table}


\paragraph{Statistical processing} We compare scan/re-scan reproducibility by computing the absolute difference and relative difference in ADC and \gls{QSI} parameters over the defined \glspl{ROI}.


Further, we investigate the correlation between individual \gls{ADC} and \gls{QSI} measurements in XY and Z directions. We pool all voxel-wise measurements over the segmented \gls{SC} area and compute Pearson's correlation coefficient over all voxels. We test for statistical significant of the correlations with a confidence interval of 95\%.


We then compare significant differences in individual metrics using a paired two-tailed t-test and further investigate statisitical significance in the group mean values of the \gls{ADC} parameters and \gls{QSI} metrics between tracts by performing the Hotellings-T$^2$ test (confidence interval=95\%). To investigate the relevance of measurements in the different \gls{DWI} directions, we compute the same significance test of XY-only \gls{QSI} parameters (P0$_{xy}$, FWHM$_{xy}$) and compare with Z-only (P0$_z$, FWHM$_z$) and the combination of both (P0$_{xy}$, FWHM$_{xy}$, P0$_z$, FWHM$_z$).

\subsection{Results}
\paragraph{Reproducibility}
\label{par:chapter5 exp1 reproducibility}
We show absolute and relative differences between scan and rescan of three healthy subjects in ADC$_{xy}$ and ADC$_z$ (see Table~\ref{tab:chap5exp1 scan rescan adc} and \gls{QSI} metrics in XY and Z direction (see Table~\ref{tab:chap5exp1 scan rescan qsi}).We observe a general trend of measurements perpendicular to the long \gls{SC} fibres presenting higher variation between scan and rescan than parallel measurements in \gls{ADC} and both \gls{QSI} metrics in all subjects. In particular ADC$_{xy}$ shows very high intra-subject variation between 20-40\% on average in all white matter \glspl{ROI} and only in GM gives an acceptable reproducibility rate of less than 11\%. In particular ADC$_z$ appears more reproducible in all three subjects with average relative variation between 5-16\%.


The perpendicular \gls{QSI} metrics P0${_{xy}}$ and FWHM$_{xy}$ present good reproducibility rates of 6-12\% and are up to 4 times lower than ADC$_{xy}$ measurements in corresponding \glspl{ROI}. In both P0$_{z}$ and FWHM$_{z}$ we observe relative change between 4-13\% similar to values in ADC$_z$.%
\\[2cm]
\begin{minipage}{\linewidth}
    \centering%
    \tabcaption{Absolute and relative change (in percent) between scan and rescan of perpendicular and parallel diffusivities in 3 healthy volunteers}
    \footnotesize
    \subfloat[ADC$_{xy}$ $\times$ $10^{-9}m^2/s$]
        {
            \begin{tabular}{rrrrrr}
            \addlinespace
            \toprule
            subject & rLT   & lLT   & AT    & PT    & GM \\
            \midrule
            1     & 0.10 (30.4\%) & 0.00 (4.7\%) & 0.07 (27.6\%) & 0.06 (24.1\%) & 0.09 (12.0\%) \\
            2     & 0.06 (16.9\%) & 0.06 (34.4\%) & 0.12 (44.6\%) & 0.03 (11.0\%) & 0.05 (12.0\%) \\
            3     & 0.09 (25.5\%) & 0.12 (51.9\%) & 0.24 (57.2\%) & 0.20 (82.5\%) & 0.04 (8.6\%) \\
                  &       &       &       &       &  \\
            mean  & 0.08 (24.3\%) & 0.06 (30.4\%) & 0.14 (43.1\%) & 0.10 (39.2\%) & 0.06 (10.9\%) \\
            \bottomrule
            \end{tabular}%
        }\\
        \subfloat[ADC$_z$ $\times$ $10^{-9}m^2/s$]
        {
            \begin{tabular}{rrrrrr}
            \addlinespace
            \toprule
            subject & rLT   & lLT   & AT    & PT    & GM \\
            \midrule
            1     & 0.04 (3.3\%) & 0.07 (4.7\%) & 0.18 (12.2\%) & 0.03 (2.1\%) & 0.03 (2.4\%) \\
            2     & 0.13 (9.0\%) & 0.17 (9.8\%) & 0.40 (23.2\%) & 0.03 (1.6\%) & 0.30 (16.9\%) \\
            3     & 0.16 (12.5\%) & 0.10 (6.2\%) & 0.21 (12.9\%) & 0.16 (10.2\%) & 0.28 (16.6\%) \\
                  &       &       &       &       &  \\
            mean  & 0.11 (8.3\%) & 0.12 (6.9\%) & 0.26 (16.1\%) & 0.07 (4.7\%) & 0.20 (12.0\%) \\
            \bottomrule
            \end{tabular}%
        }
    \label{tab:chap5exp1 scan rescan adc}
\end{minipage}%
\\[2cm]
\begin{minipage}{\linewidth}
        \centering
        \tabcaption{Absolute and relative change (in percent) between scan and rescan of perpendicular and parallel QSI metrics in 3 healthy volunteers for all tract-specific ROIs}
        \footnotesize
        \subfloat[P0$_{xy}$]
        {
            \begin{tabular}{rrrrrr}
            \addlinespace
            \toprule
            subject & rLT   & lLT   & AT    & PT    & GM \\
            \midrule
            1     & 0.01 (3.1\%) & 0.02 (6.8\%) & 0.01 (3.4\%) & 0.00 (1.7\%) & 0.00 (1.9\%) \\
            2     & 0.00 (0.4\%) & 0.00 (0.3\%) & 0.01 (4.3\%) & 0.01 (3.3\%) & 0.00 (2.3\%) \\
            3     & 0.01 (6.1\%) & 0.06 (28.2\%) & 0.04 (19.9\%) & 0.06 (26.7\%) & 0.03 (14.8\%) \\
                  &       &       &       &       &  \\
            mean  & 0.01 (3.2\%) & 0.03 (11.8\%) & 0.02 (9.2\%) & 0.02 (10.6\%) & 0.01 (6.3\%) \\
            \bottomrule
            \end{tabular}%
        }\\
        \subfloat[FWHM$_{xy}$ $\times$ $10^{-6}m$]
        {
            \begin{tabular}{rrrrrr}
            \addlinespace
            \toprule
            subject & rLT   & lLT   & AT    & PT    & GM \\
            \midrule
            1     & 0.52 (2.5\%) & 0.67 (4.8\%) & 0.67 (3.5\%) & 0.29 (1.5\%) & 0.62 (2.4\%) \\
            2     & 0.03 (0.1\%) & 0.29 (1.6\%) & 0.76 (4.1\%) & 0.36 (1.9\%) & 0.32 (1.5\%) \\
            3     & 1.10 (5.2\%) & 5.69 (29.6\%) & 4.72 (20.6\%) & 5.10 (27.5\%) & 3.29 (15.5\%) \\
                  &       &       &       &       &  \\
            mean  & 0.55 (2.6\%) & 2.22 (12.0\%) & 2.05 (9.4\%) & 1.92 (10.3\%) & 1.41 (6.5\%) \\
            \bottomrule
            \end{tabular}%
        }\\
        \subfloat[P0$_{z}$]
        {
            \begin{tabular}{rrrrrr}
            \addlinespace
            \toprule
            subject & rLT   & lLT   & AT    & PT    & GM \\
            \midrule
            1     & 0.00 (4.5\%) & 0.00 (0.6\%) & 0.01 (6.9\%) & 0.00 (2.5\%) & 0.01 (5.6\%) \\
            2     & 0.01 (9.2\%) & 0.01 (11.4\%) & 0.01 (11.0\%) & 0.00 (3.4\%) & 0.01 (10.6\%) \\
            3     & 0.01 (6.0\%) & 0.00 (0.1\%) & 0.00 (1.2\%) & 0.01 (7.8\%) & 0.01 (14.9\%) \\
                  &       &       &       &       &  \\
            mean  & 0.01 (6.6\%) & 0.00 (4.1\%) & 0.01 (6.3\%) & 0.00 (4.6\%) & 0.01 (10.4\%) \\
            \bottomrule
            \end{tabular}%
        }\\
        \subfloat[FWHM$_{z}$ $\times$ $10^{-6}m$]
        {
            \begin{tabular}{rrrrrr}
            \addlinespace
            \toprule
            subject & rLT   & lLT   & AT    & PT    & GM \\
            \midrule
            1     & 1.42 (3.9\%) & 1.67 (3.9\%) & 4.20 (10.4\%) & 1.07 (2.7\%) & 3.86 (10.7\%) \\
            2     & 5.72 (15.4\%) & 9.04 (22.2\%) & 4.69 (11.6\%) & 4.47 (10.9\%) & 4.76 (12.5\%) \\
            3     & 1.08 (3.0\%) & 0.13 (0.3\%) & 1.66 (4.1\%) & 4.89 (12.4\%) & 6.17 (16.1\%) \\
                  &       &       &       &       &  \\
            mean  & 2.74 (7.4\%) & 3.61 (8.8\%) & 3.52 (8.7\%) & 3.48 (8.6\%) & 4.93 (13.1\%) \\
            \bottomrule
            \end{tabular}%
        }
   \label{tab:chap5exp1 scan rescan qsi}
\end{minipage}%


\paragraph{Differences between tract-specific ROI measurements}
We compare the average values and standard devation over all 9 subjects between tract-specific \glspl{ROI} for \gls{ADC} values in Figure~\ref{fig:chapter5 exp1 ADC vals} and \gls{QSI} metrics in Figure~\ref{fig:chapter5 exp1 ADC vals}. As a general trend, we observe higher inter-subject variation in XY measurements compared to Z measurements among all 9 subjects which is in line with the intra-subject variation observed in Paragraph~\ref{par:chapter5 exp1 reproducibility}.


Tables~\ref{tab:chap5exp1_adc single ttest} and \ref{tab:chap5exp1_qsi single ttest} present $p$-values for pairwise differences between different tract-\glspl{ROI} for ADC and \gls{QSI} metrics. Most apparent differences can be found between the GM \gls{ROI} and the white matter regions in ADC$_{xy}$ and both P0$_{xy}$/FWHM$_{xy}$ with high statistical significance ($p<0.01$ between WM tracts GM for all X metrics). Z metrics in the GM \gls{ROI} appear more similar to WM \glspl{ROI}. Significant differences are only found between the right LT and GM in ADC$_z$. The \gls{QSI}-Z metrics only shows significant differences between GM and rRT in P0$_z$ (p=0.01) and between GM and AT and PT (FWHM$_z$).


Between WM \glspl{ROI} only the left LT but not the right LT is significantly different from both AT and PT in ADC$_{xy}$ perpendicular to long white matter fibres. Parallel to the long \gls{SC} axis where we only find ADC$_{z}$ in the right LT significantly lower from AT and PT. Left and right LT show significant differences in both ADC$_{xy}$ and ADC$_{z}$ while we find no difference between AT or PT. In \gls{QSI} metrics we find the same tracts as with ADC to be significantly different in XY and Z direction. However, $p$-values are increased in \gls{QSI} compared to corresponding ADC, but remained below $p<0.05$.

\begin{minipage}{\linewidth}
   \centering
   \input{chapter5/figs/exp1_adcvals.tex}
   \figcaption{Mean and standard devation of perpendicular and parallel diffusivities in all ROIs over all 9 volunteers.}
   \label{fig:chapter5 exp1 ADC vals}
\end{minipage}\\
\begin{minipage}{\linewidth}
%  \centering
   \input{chapter5/figs/exp1_qspacevals.tex}
   \figcaption{Mean and standard devation of perpendicular and parallel QSI metrics in all ROIs over all 9 volunteers.}
   \label{fig:chapter5 exp1 QSI vals}
\end{minipage}%

  \begin{table}
    \footnotesize
    \centering
    \caption{Significance of pair-wise differences between SC tracts in diffusion coefficients ADC$_{xy}$ and ADC$_{z}$ (confidence interval: 95\%)}
    \subfloat[ADC$_{xy}$]{
        \begin{tabular}{rrrrr}
        \addlinespace
        \toprule
        & LCST  & AT    & PT    & GM \\
        \midrule
        RCST  & \emph{0.01}  & 0.60  & 0.84  & \emph{<0.01} \\
        LCST  &       & \emph{<0.01}  & \emph{<0.01}  & \emph{<0.01} \\
        AT    &       &       & 0.56  & \emph{<0.01} \\
        PT    &       &       &       & \emph{<0.01} \\
        \bottomrule
        \end{tabular}%
    }\hspace{0.5cm}
    \subfloat[ADC$_z$]
    {
        \begin{tabular}{rrrrr}
        \addlinespace
        \toprule
              & LCST  & AT    & PT    & GM \\
        \midrule
        RCST  & \emph{0.01}  & \emph{<0.01}  & \emph{<0.01}  & \emph{<0.01} \\
        LCST  &       & 0.85  & \emph{<0.01}  & 0.57 \\
        AT    &       &       & 0.44  & 0.30 \\
        PT    &       &       &       & 0.74 \\
        \bottomrule
        \end{tabular}%
    }
\label{tab:chap5exp1_adc single ttest}%
\end{table}%
\end{document}

