%!TEX root = ./thesis.tex
%%%%%%%%%%%%%%%%%%%%%%%%%%%%%%%%%%%%%%%%%
% Thesis Configuration File
%
% The main lines to change in this file are in the DOCUMENT VARIABLES
% section, the rest of the file is for advanced configuration.
%
%%%%%%%%%%%%%%%%%%%%%%%%%%%%%%%%%%%%%%%%%

%----------------------------------------------------------------------------------------
%	DOCUMENT VARIABLES
%	Fill in the lines below to enter your information into the thesis template
%	Each of the commands can be cited anywhere in the thesis
%----------------------------------------------------------------------------------------

% Remove drafting to get rid of the '[ Date - classicthesis version 4.0 ]' text at the bottom of every page
\PassOptionsToPackage{eulerchapternumbers,listings,drafting, subfig,parts,floatperchapter,manychapters,dottedtoc,subfig}{classicthesisxetex}
% Available options: drafting parts nochapters linedheaders eulerchapternumbers beramono eulermath pdfspacing minionprospacing tocaligned dottedtoc manychapters listings floatperchapter subfig
% Adding 'dottedtoc' will make page numbers in the table of contents flushed right with dots leading to them
\setcounter{tocdepth}{1}

%%% THESIS VARIABLES
\newcommand{\myTitle}{Spinal Cord Diffusion Imaging: Imaging Challenges and Prognostic Value\xspace}
\newcommand{\mySubtitle}{qq \xspace}
\newcommand{\myDegree}{PhD\xspace}
\newcommand{\myName}{Torben Schneider\xspace}
\newcommand{\myProf}{Claudia Wheeler-Kingshott\xspace}
\newcommand{\myOtherProf}{Daniel Alexander\xspace}
\newcommand{\mySupervisor}{Put name here\xspace}
\newcommand{\myFaculty}{UCL Institute of Neurology\xspace}
\newcommand{\myDepartment}{Department of Neuroinflammation\xspace}
\newcommand{\myUni}{University College London\xspace}
\newcommand{\myLocation}{London, United Kingdom\xspace}
\newcommand{\myTime}{\today\xspace}
\newcommand{\myVersion}{version 1.0\xspace}

%%% COUNTER FIXES
\newcounter{dummy} % Necessary for correct hyperlinks (to index, bib, etc.)

%----------------------------------------------------------------------------------------
%	PACKAGES
%----------------------------------------------------------------------------------------

%%%% PACKAGES FOR CLASSIC THESIS
\usepackage{xspace} % To get the spacing after macros right
\usepackage{mparhack} % To get marginpar right
\usepackage[dvipsnames]{xcolor}
\usepackage{polyglossia}
\setdefaultlanguage[variant=british]{english}


%%%% THESIS PACKAGES
%math
\usepackage{amssymb}
\usepackage{amsmath}
\usepackage{xfrac}
\newcommand{\mat}[1]{\ensuremath{\mathbf{#1}}}

%tikz & pgf
\usepackage{tikz}
\usepackage{pgfplots}
\usetikzlibrary[calc]
\usetikzlibrary[decorations.pathreplacing]
\usetikzlibrary[svg.path]
\usepgfplotslibrary{polar}
\usepackage{pgfboxplot} %custom
\usepackage{sequenceplot} %custom
\usetikzlibrary{external}
\tikzexternalize[mode=list and make]
\tikzsetexternalprefix{tikz/}
\tikzset{external/system call={xelatex \tikzexternalcheckshellescape -halt-on-error -interaction=batchmode -jobname "\image" "\texsource"}}
%optional png setup
\tikzset{
  	% Add size information to the .dpth file (png is in density not size)
   /pgf/images/external info,
    png export/.style={
        external/system call=%
        {xelatex \tikzexternalcheckshellescape -halt-on-error -interaction=batchmode -jobname "\image" "\texsource"; convert -density 72 -transparent white "\image.pdf" "\image.png"},
	},
    % Use the png export AND the import
    use png/.style={png export,png import},
    png import/.code={%
        \tikzset{%
            /pgf/images/include external/.code={%
                % Here you can alter to whatever you want
                % \pgfexternalwidth is only available if /pgf/images/external info
                % is set
            		\includegraphics[width=\pgfexternalwidth,height=\pgfexternalheight]%
					{##1.png}%
            }%
        }%
    }%
}


%figures
\usepackage{graphicx}
\usepackage{float}
\usepackage{placeins}
\usepackage[export]{adjustbox}
\usepackage{subfig}  



%library & glossaries
\usepackage{natbib}
\usepackage[nomain,acronym,nonumberlist,section=chapter]{glossaries}
\glossarystyle{listdotted}
\setlength{\glslistdottedwidth}{12em}




%title
%\usepackage{titlesec}
%\usepackage[title, titletoc]{appendix}
\usepackage{appendix}

%misc
\usepackage[super]{nth}

%drafttools
\usepackage{todonotes}
\usepackage{ifdraft}


%----------------------------------------------------------------------------------------
%	FLOATS: TABLES, FIGURES AND CAPTIONS SETUP
%----------------------------------------------------------------------------------------
%tables
\usepackage{multirow}
\usepackage{ctable}
\usepackage{array}
\usepackage{booktabs}
\usepackage{colortbl}
\usepackage{tabularx} % Better tables


%new columns with fixed width
\newcolumntype{x}[1]{ >{\centering\arraybackslash\hspace{0pt}}p{#1}}
\newcolumntype{y}[1]{ >{\arraybackslash\hspace{0pt}}p{#1}}


\setlength{\extrarowheight}{3pt} % Increase table row height
\newcommand{\tableheadline}[1]{\multicolumn{1}{c}{\spacedlowsmallcaps{#1}}}
\newcommand{\myfloatalign}{\centering} % To be used with each float for alignment
\usepackage{caption}
\captionsetup{format=hang,font=small,labelfont={bf}}


%----------------------------------------------------------------------------------------
%	HYPERREFERENCES
%----------------------------------------------------------------------------------------

\PassOptionsToPackage{hyperfootnotes=false}{hyperref}
\usepackage{hyperref}  % backref linktocpage pagebackref
\hypersetup{
% Uncomment the line below to remove all links (to references, figures, tables, etc)
%draft, 
colorlinks=true, linktocpage=true, pdfstartpage=3, pdfstartview=FitV,
% Uncomment the line below if you want to have black links (e.g. for printing black and white)
%colorlinks=false, linktocpage=false, pdfborder={0 0 0}, pdfstartpage=3, pdfstartview=FitV, 
breaklinks=true, pdfpagemode=UseNone, pageanchor=true, pdfpagemode=UseOutlines,
plainpages=false, bookmarksnumbered, bookmarksopen=true, bookmarksopenlevel=1,
hypertexnames=true, pdfhighlight=/O, urlcolor=webbrown, linkcolor=RoyalBlue, citecolor=webgreen,
%------------------------------------------------
% PDF file meta-information
pdftitle={\myTitle},
pdfauthor={\textcopyright\ \myName, \myUni, \myFaculty},
pdfsubject={},
pdfkeywords={},
pdfcreator={xelatex},
pdfproducer={LaTeX}
%------------------------------------------------
}   

%----------------------------------------------------------------------------------------
%	BACKREFERENCES
%----------------------------------------------------------------------------------------

\usepackage{ifthen} % Allows the user of the \ifthenelse command
\newboolean{enable-backrefs} % Variable to enable backrefs in the bibliography
\setboolean{enable-backrefs}{false} % Variable value: true or false

\newcommand{\backrefnotcitedstring}{\relax} % (Not cited.)
\newcommand{\backrefcitedsinglestring}[1]{(Cited on page~#1.)}
\newcommand{\backrefcitedmultistring}[1]{(Cited on pages~#1.)}
\ifthenelse{\boolean{enable-backrefs}} % If backrefs were enabled
{
\PassOptionsToPackage{hyperpageref}{backref}
\usepackage{backref} % to be loaded after hyperref package 
\renewcommand{\backreftwosep}{ and~} % separate 2 pages
\renewcommand{\backreflastsep}{, and~} % separate last of longer list
\renewcommand*{\backref}[1]{}  % disable standard
\renewcommand*{\backrefalt}[4]{% detailed backref
\ifcase #1 
\backrefnotcitedstring
\or
\backrefcitedsinglestring{#2}
\else
\backrefcitedmultistring{#2}
\fi}
}{\relax} 

%----------------------------------------------------------------------------------------
%	AUTOREFERENCES SETUP
%	Redefines how references in text are prefaced for different 
%	languages (e.g. "Section 1.2" or "section 1.2")
%----------------------------------------------------------------------------------------

\makeatletter
\@ifpackageloaded{babel}
{
\addto\extrasbritish{
\renewcommand*{\figureautorefname}{Figure}
\renewcommand*{\tableautorefname}{Table}
\renewcommand*{\partautorefname}{Part}
\renewcommand*{\chapterautorefname}{Chapter}
\renewcommand*{\sectionautorefname}{Section}
\renewcommand*{\subsectionautorefname}{Section}
\renewcommand*{\subsubsectionautorefname}{Section}
}

\providecommand{\subfigureautorefname}{\figureautorefname} % Fix to getting autorefs for subfigures right
}{\relax}
\makeatother

%----------------------------------------------------------------------------------------

\usepackage{classicthesisxetex} 
%\usepackage{classicthesis} 


%----------------------------------------------------------------------------------------
%	CHANGING TEXT AREA 
%----------------------------------------------------------------------------------------

\linespread{1.05} % a bit more for Palatino
%\areaset[current]{312pt}{761pt} % 686 (factor 2.2) + 33 head + 42 head \the\footskip
%\setlength{\marginparwidth}{7em}%
%\setlength{\marginparsep}{2em}%

%----------------------------------------------------------------------------------------
%	USING DIFFERENT FONTS
%----------------------------------------------------------------------------------------

\usepackage[math]{anttor}
%\usepackage{cmbright}

\usepackage{xltxtra}
\defaultfontfeatures{Mapping=tex-text}

%\newfontfamily\electra[
%	Path = ./fonts/ ,
%	Extension = .otf ,
%	UprightFont = *-Regular,
%	BoldFont = *-Bold,
%	ItalicFont = *-Cursive,
%	BoldItalicFont = *-BoldCursive,
%]{ElectraLTStd}
%\newcommand*{\textel}[1]{{\electra #1}}

%\setmainfont[
%	Path = ./fonts/ ,
%	Extension = .otf ,
%	UprightFont = *,
%	BoldFont = *-Bold,
%	ItalicFont = *-Italic,
%	BoldItalicFont = *-BoldItalic,
%]{BemboStd}

%\setmainfont[
%	Path = ./fonts/ ,
%	Extension = .ttf ,
%	UprightFont = *-Regular,
%	BoldFont = *-Bold,
%	ItalicFont = *-Italic,
%	BoldItalicFont = *-BoldItalic,
%	SmallCapsFont = GriffosSCapsFont % change font for sc
%]{DroidSerif}

%\setmainfont[
%	Path = ./fonts/ ,
%	Extension = .ttf ,
%	UprightFont = *regular,
%	BoldFont = *bold,
%	ItalicFont = *italic,
%	BoldItalicFont = *bolditalic,
%]{serif72beta}

%\setmainfont[
%	Path = ./fonts/ ,
%	Extension = .ttf ,
%	UprightFont = *R,
%	BoldFont = *B,
%	ItalicFont = *I,
%	BoldItalicFont = *BI,
%]{GenBas}% Gentium Basic Book

%\setmainfont[
%	Path = ./fonts/ ,
%	Extension = .ttf ,
%	UprightFont = *-Regular,
%	BoldFont = *-Bold,
%	ItalicFont = *-Italic,
%	BoldItalicFont = *Cursive-Regular,
%]{Neuton}

%\setmainfont[
%	Path = ./fonts/ ,
%	%Extension = .otf ,
%	UprightFont = *-Regular-OTF.otf,
%	BoldFont = *-Bold-OTF.otf,
%	ItalicFont = *-Italic-OTF.ttf,
%	BoldItalicFont = *-BoldItalic-OTF.otf,
%]{Volkhov}

%
\setmainfont[
	Path = ./fonts/ ,
	Extension = .otf ,
	UprightFont = *-regular,
	BoldFont = *-bold,
	ItalicFont = *-italic,
	BoldItalicFont = *-bolditalic,
]{texgyrepagella}

%
%\setmainfont[
%	Path = ./fonts/ ,
%	Extension = .ttf ,
%	UprightFont = *-Regular,
%	BoldFont = *-Bold,
%	ItalicFont = *-Italic,
%	BoldItalicFont = *-BoldItalic,
%]{Vollkorn}

%\setmainfont[
%	Path = ./fonts/ ,
%	Extension = .ttf ,
%	UprightFont = *_Lt,
%	BoldFont = *_Bd,
%	ItalicFont = *_LtIt,
%	BoldItalicFont = *_BdIt,
%]{Aller}

%\setmainfont[
%	Path = ./fonts/ ,
%	Extension = .otf ,
%	UprightFont = *-Regular,
%	BoldFont = *-Bold,
%	ItalicFont = *-Italic,
%	BoldItalicFont = *-BoldItalic,
%]{Alegreya}

%\setmainfont[
%	Path = ./fonts/FontinSans_Cyrillic_46b/ ,
%	Extension = .otf ,
%	UprightFont = *_R_46b,
%	BoldFont = *_B_46b,
%	ItalicFont = *_I_46b,
%	BoldItalicFont = *_BI_46b,
%]{FontinSans_Cyrillic}

\newfontfamily\sans[
	Path=./fonts/,
	UprightFont = AlegreyaSC-Regular.otf,
	BoldFont = *-Bold.ttf,
	ItalicFont = *-Italic.ttf,
	BoldItalicFont = *-BoldItalic.ttf,
]{Vollkorn}

\newcommand*{\textsans}[1]{{\sans #1}}

\titleformat{\chapter}[display]
        {\relax}{\mbox{}\oldmarginpar{\vspace*{-3\baselineskip}\color{halfgray}\chapterNumber\thechapter}}{0pt}%
        {\Large\sans}[\vspace*{.0\baselineskip}\titlerule]%   


%\setmainfont[
%	Path = ./fonts/ ,
%	Extension = .otf ,
%	UprightFont = *_Light,
%	BoldFont = *_Bold,
%	ItalicFont = *_Italic_Light,
%	BoldItalicFont = *_Italic_Bold,
%]{Midiet_Sans}


%----------------------------------------------------------------------------------------
%	USEFUL COMMANDS
%----------------------------------------------------------------------------------------
%%%%% USEFUL COMMANDS %%%%%%%%%%
\newcommand{\SF}{{\ensuremath{\mathcal{SF}}}}
\newcommand{\OI}{{\ensuremath{\mathcal{OI}}}}
\newcommand{\SD}{{\ensuremath{\mathcal{SF}_{pulses}}}}
\newcommand{\DO}{{\ensuremath{\mathcal{SF}_{dirs}}}}
\newcommand{\FD}{{\SF}}

\newcommand{\SFshort}{\SF$_{90}$}
\newcommand{\SFlong}{\SF$_{360}$}
\newcommand{\OIlong}{\OI$_{360}$}

\newcommand{\FDmod}{{\ensuremath{\FD_{mod}}}}
\newcommand{\SFasym}{{\ensuremath{a\mathcal{SF}}}}

% spaces for figure list and table list
%\usepackage{etoolbox}
%\newcommand{\killdeactivateaddvspace}{\let\deactivateaddvspace\relax}
%\AtEndPreamble{\addtocontents{lof}{\protect\killdeactivateaddvspace}}

%\usepackage{arsclassica} 