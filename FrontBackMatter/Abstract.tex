%!TEX root = ../thesis.tex
% Abstract

\pdfbookmark[1]{Abstract}{Abstract} % Bookmark name visible in a PDF viewer

\begingroup
\let\clearpage\relax
\let\cleardoublepage\relax
\let\cleardoublepage\relax

\chapter*{Abstract} % Abstract name
{\small
The aim of this thesis is to explore the potential of imaging markers derived from diffusion-weighted MRI (DW MRI) in the spinal cord to characterise healthy white matter pathways and provide sensitivity to axonal damage, regeneration and collateral sprouting in spinal cord disease. 


With new innovative treatment strategies emerging for spinal cord pathologies such as spinal cord injury and Multiple Sclerosis, there is a need for new \emph{in-vivo} biomarkers that can be specific to structural and functional changes and their underlying mechanisms on a microscopic scale. DW MRI has the potential to quantifying those microstructural characteristics beyond the scale of  conventional MRI.


In the first part of this dissertation we investigate Diffusion Tensor Imaging (DTI), which is the most established DW MRI analysis technique in clinical practice. In two studies we assess DTI in the context of spinal cord imaging. In the first experiment we show that DTI is sensitive to the presence of collateral fibres, e.g., at inter-vertebral level where peripheral nerves enter the spinal tract. In the second experiment we propose a new method for the correction of partial volume effect on whole cord DTI measurements, that is specifically tailored for the imaging and analysis challenges in the cord. 


The second part of this thesis comprises two studies of q-space imaging (QSI) in healthy controls. In theory, QSI offers a more comprehensive description of the diffusion process, but is challenging to set up on a clinical MRI scanner. We present here two QSI protocols, set up for two different scanners with different gradient hardware, receive coils and software limitations. For the first time we perform a systematic study of QSI that assesses the reproducibility and specificity to different white pathways \emph{in-vivo} in the cervical cord within a group of healthy volunteers. Both studies showed superior reproducibility of QSI over conventional analysis, although the results of using QSI parameters to distinguish individual white matter tracts in the cord were inconclusive.


The third part of this thesis describes a new imaging method that uses a complex multi-compartment model, which relates DW MRI data to microstructural parameters like axon diameter and density. We designed our method to incorporate the known fibre structure of the spinal cord. In a first step we validated our approach in in a post-mortem cervical spinal cord sample of a velveteen monkey. We then demonstrated clinical feasibility and good reproducibility of our protocols for \emph{in-vivo} human studies, using the corpus callosum as a model system for structures with unidirectional fibre architecture. Finally we present first estimation results of axon diameter and density of the cervical spinal cord \emph{in-vivo} in a healthy control that agree with our earlier findings in the \emph{ex-vivo} monkey spinal cord sample.
\vfill
}
\endgroup			

