%!TEX root = ./report.tex
Spinal Cord Injury (SCI) can have devastating effects on the life of people affected by it. Thanks to innovative treatment strategies, SCI patients can now hope in the concrete possibility of new therapies leading to recovery of feeling and motor functions, with a dramatic repercussion on their future quality of life.

Magnetic Resonance Imaging (MRI) routine scans of the spinal cord are often aiding the diagnosis of SCI, but they have a limited prognostic value because of their qualitative nature and because of their lack of specificity in terms of underlying mechanisms such as inflammation, axonal loss and gliosis. There is the need for in vivo imaging biomarkers for human spinal cord examinations, which are sensitive to tissue changes and which are capable of quantifying underlying structural and functional changes.

MRI offers the possibility of translating the application of novel quantitative techniques that have been successful in the brain to the spinal cord. In particular, it has been already shown that parameters derived from diffusion imaging can be specific to the underlying tissue structure and that such parameters are able to discriminate between SCI patients and controls \cite{TODO}. Moreover, from diffusion imaging data it is possible to reconstruct fibre paths and the field is now looking at ways to measure more specific parameters such as the axonal density.
\section{Problem Statement}
Although some optimisation work for spinal cord applications has already taken place, there is the need for a dedicated effort to develop spinal cord diffusion imaging with the aim of optimising the whole process from the acquisition design to the analysis methods, based on the spinal cord tissue properties and the expected underlying mechanisms of tissue damage and recovery.

\section{Project Aims}
\begin{enumerate}
	\item Improve the in vivo imaging characterisation of spinal cord white matter pathways in healthy volunteers
	\item Introduce imaging biomarkers that are able to detect partial preservation of long fibre pathways in spared tissue after spinal cord injury and that can distinguish between axonal damage and functional recovery through modelling of axonal regeneration and collateral sprouting
	\item Evaluate the prognostic value of the proposed quantitative parameters. 
\end{enumerate}
\section{Summary of contributions}
\begin{enumerate}
	\item 	
\end{enumerate}
