%!TEX root = ../thesis.tex

\chapter{Q-space imaging of the healthy cervical spinal cord}
\label{sec:chap5 QSI in cord}
In this chapter we investigate accuracy and sensitivity of spinal cord \gls{QSI} metrics in healthy controls and evaluate its potential for clinical application. As discussed above (see Section~\ref{sec:qspace}), various studies on experimental MRI systems have shown that \gls{QSI} can provide accurate information about microscopic restriction in excised tissue \citep{Assaf:2000,Bar-Shir:2008,Ong:2008}. \gls{QSI} requires an extensive sampling of different $q$-values in along a single axis. This restricts the number of diffusion gradient directions that can be sampled when scan time is limited. While in most CNS, e.g. in the brain, high angular resolution of gradient directions is required to capture the variety of different fibre direction, this is less of a problem in \gls{SC} due its relatively simple white matter structure. Although the conditions for true \gls{QSI}, such as the short gradient pulse, are impossible to achieve in clinical systems. Despite all it's difficulties, previous proof-of-concept studies have shown the great potential in the assessment of {\gls{SC}} white matter and white pathologies such as MS in the human brain \citep{Assaf:XXX, XXX,XXX} and in the spinal cord \citet{Farrell:2008}.

Following up on the encouraging results of the previous studies, we aim here to study the reproducibility of \gls{QSI} metrics in the cervical {\gls{SC}} on a standard 3T clinical MRI scanner. We also assessed \gls{QSI} measures both in-plane (XY) and parallel to the main {\gls{SC}} axis (Z), not presented before. Our particular intest is to explore \gls{QSI} measures in gray matter and different ascending and descending white matter tracts of the cervical {\gls{SC}}. Previous work in in-vitro rat spinal cord by \citep{Ong:2008,Ong:2010,Ong:2011} suggest that \gls{QSI} parameters correlate with the axon diameter in different white matter regions. Here we test whether \gls{QSI} can discriminate between \gls{WM} in the cervical in healthy subjects and compare conventional \gls{ADC} measures, both in plane and along the cord. !Furthermore we also test whether any combination of \gls{QSI} derived FWHM and P0 metrics can better distinguish between WM \glspl{ROI} than the individual metrics alone.!

The remainder of this chapter will present data from the following two experiments. The first experiment investigated 9 healthy controls that were scanned at the Wellcome Trust Centre for Neuroimaging Imaging centre as part of a pilot study to study the effect of brachial plexus avulsion (and was also used to study \gls{NMO}). Our preliminary findings in healthy controls were submitted for presentation at the Annual Meeting of the International Society of Magnetic Resonance in Medicine and were accepted for oral presentation \citep{Schneider:XXX}. Following the encouraging results in this first experiment, we re-implemented an improved protocol on the Philips 3T MRI scanner, that was newly installed in in our lab in 2010. The results of a second \gls{QSI} pilot study using the new protocols comprise the second part of this chapter (section~\ref{TODO}).

\section{Experiment 1}
\subsection{Methods}
\paragraph{Study design}
Twenty right-handed male healthy subjects were recruited (mean age 35�11yrs) to be scanned on a 3T Tim Trio (Siemens Healthcare, Erlangen). Six subjects were recalled for a second scan on a different day to assess intra-subject reproducibility of \gls{QSI} derived parameters. 

\paragraph{Data acquisition}
In each subject we perform cardiac-gated high {\gls{bvalue}} axial{\gls{DWI}}(matrix=96x96, b-spline interpolated to 192x192 in image space, FOV=144x144$mm^2$, slice thickness=5mm, 20 slices, TE=110ms, TR$\approx$4000ms). The \gls{QSI} set-up is based on parameters found in the most recent clinical \gls{QSI} study \citep{Farrell:2008}. However, our gradient system only allowed maximum \gls{gstr} of 23mT/m (\citet{Farrell:2008}: 60mT/m). To achieve similar $q$-values is was necessary increase the gradient duration \gls{smalldel} to 51ms. Reproduction of the protocol was further complicated by a limitation in the scanner software, which only permits $b$-value to be specified in multiples of 50 $m/s^2$ and means that $q$-values can not be exactly linearly spaced. We acquire a total of 32 $b$-values between 0-3000s/mm2 in three different{\gls{DWI}}directions: two directions perpendicular (XY) and one parallel (Z) to the main {\gls{SC}} axis. The full protocol is given in Table~\ref{tab:chap5exp1 protocol}.

After an initial quality check, we found that the prescription of the axial \gls{DWI} slices varying largely between different subjects. Figure~\ref{XX} shows two representive cases for correct and incorrect positioning observed in the dataset. \gls{QSI} is very sensitive to its alignment to the fibre direction as shown e.g. in \citep{Avram:2002} and the variation in slice positioning might overshadow the subtle differences between \gls{WM} we are interested in. We therefore measure the angulation between imaging plane and \gls{SC} longitudinal axis as seen on a T2w sagittal scan at C2/C3. We excluded 11 subjects and their subsequent data with where the angle was less than 80$^\circ$ (ideally we assume the axial imagies perfectly perpendicular, i.e 90$^\circ$)

\paragraph{Data processing}
Similar to \citet{Farrell:2008} the two perpendicular diffusion directions were averaged to increase the signal-to-noise ratio. The measurements are linearly regridded to be equidistant in q-space and the  {\gls{dpdf}} is computed using inverse Fast Fourier Transformation. To increase the resolution of the  {\gls{dpdf}}, the signal was extrapolated in q-space to a maximum q=166mm$^{-1}$ by fitting a bi-exponential decay curve to the{\gls{DWI}}data as suggested in \citet{Cohen:2002, Farrell:2008}. Figure~\ref{fig:chapter5 exp1 processing pipeline} illustrates the processing pipeline. Maps of the full width at half maximum and zero displacement probability were derived for XY and Z as described in Section~\ref{sec:qspace}. For comparison we also computed the apparent diffusion coefficient (see section \ref{subsec:adc}) from the monoexponential part of the decay curve (b < 1100s/mm2) as in \citet{Farrell:2008} for both XY and Z directions using a constrained non-linear least squared fitting algorithm. Figure~\ref{fig:chapter5 exemplary maps} shows both \gls{ADC} maps and the four \gls{QSI} parameter maps in one randomly chosen subject.


\begin{figure}
  \pgfimage[width=\textwidth]{chapter5/figs/QSIprocessing.pdf}
  \caption{Cartoon of the individual steps in our QSI processing pipeline.}
  \label{fig:chapter5 exp1 processing pipeline}
\end{figure}


\begin{figure}
\centering
\subfloat[ADC$_{xy}$ $\times 10^{-9}m^2/s$]{
    \pgfimage[width=0.4\textwidth]{chapter5/figs/exp1_ADCX.png}
}
\subfloat[ADC$_{z}$ $\times 10^{-9}m^2/s$]{
    \pgfimage[width=0.4\textwidth]{chapter5/figs/exp1_ADCZ.png}
}\\
\subfloat[P0$_{xy}$]{
    \pgfimage[width=0.4\textwidth]{chapter5/figs/exp1_P0X.png}
}
\subfloat[FWHM$_{xy}$ $\times 10^{-6}m$]{
    \pgfimage[width=0.4\textwidth]{chapter5/figs/exp1_FWHMX.png}
}\\
\subfloat[P0$_{z}$]{
    \pgfimage[width=0.4\textwidth]{chapter5/figs/exp1_P0Z.png}
}
\subfloat[FWHM$_{z}$ $\times 10^{-6}m$]{
    \pgfimage[width=0.4\textwidth]{chapter5/figs/exp1_FWHMZ.png}
}
\caption{ADC maps and QSI parameter maps in one exemplary subject at the level of the C2--C3 disc.}
\label{fig:chapter5 exemplary maps}
\end{figure}

\paragraph{ROI analysis} We semi-automatically delineate the whole cervical {\gls{SC}} area (SCA) between levels C1 and C3 on the b=0 images using the active surface segmentation by \citet{Horsfield:2010} available in Jim6. We perform a morphological erosion (2 iterations) of the obtained segmentation mask to exclude voxels with potential partial-volume average effect from surrounding \gls{CSF}. In addition, four regions of interest were manually placed in specific white matter tracts and one ROI was positioned in the gray matter on all slices between level C1 and C3. The four white matter regions comprised the left and right tracts (l\&r-LT) running in the lateral columns and the anterior (AT) and posterior tracts (PT) similar to \citet{Hesseltine:2006,Freund:2010} (see Figure~\ref{fig:chapter5 exp1 ROIs}).

  \begin{figure}
      \centering
      \pgfimage[height=5cm]{chapter5/figs/rois.png}
      \caption{Illustration of WM ROIs drawn on b0 image of the cord.}
      \label{fig:chapter5 exp1 ROIs}
  \end{figure}

  \begin{table}
      \centering
     \caption{QSI protocol displaying: Gradient strength (G), q-value (q) and b-value (b) for each of the 32 DWI volumes.}
        \begin{tabular}{rrr}
        \addlinespace
            \multicolumn{3}{l}{}\\
        \toprule
            G $[mT/m]$ & q $[cm^{-1}]$ & b $[s/mm^2]$ \\
            \cmidrule(r){1-1}\cmidrule(lr){2-2}\cmidrule(l){3-3}
            0.0   & 0.0   & 0 \\
            3.0   & 56.2  & 50 \\
            4.2   & 79.4  & 100 \\
            5.1   & 97.3  & 150 \\
            5.9   & 112.3 & 200 \\
            6.6   & 125.6 & 250 \\
            8.4   & 158.9 & 400 \\
            9.8   & 186.3 & 550 \\
            11.5  & 217.5 & 750 \\
            12.9  & 244.8 & 950 \\
            14.5  & 275.2 & 1200 \\
            15.9  & 302.5 & 1450 \\
            17.5  & 332.3 & 1750 \\
            19.2  & 364.0 & 2100 \\
            20.7  & 393.2 & 2450 \\
            22.2  & 420.3 & 2800 \\
            \bottomrule
        \end{tabular}%
        \hspace{0.2cm}%
        \begin{tabular}{rrr}
        \addlinespace
            \multicolumn{3}{l}{\textit{... continued}}\\
        \toprule
            G $[mT/m]$ & q $[cm^{-1}]$ & b $[s/mm^2]$ \\
            \cmidrule(r){1-1}\cmidrule(lr){2-2}\cmidrule(l){3-3}
            0.0   & 0.0   & 0 \\
            3.0   & 56.2  & 50 \\
            4.2   & 79.4  & 100 \\
            5.1   & 97.3  & 150 \\
            5.9   & 112.3 & 200 \\
            7.8   & 148.6 & 350 \\
            9.4   & 177.6 & 500 \\
            10.7  & 202.5 & 650 \\
            12.2  & 231.6 & 850 \\
            13.9  & 263.4 & 1100 \\
            15.4  & 291.9 & 1350 \\
            16.7  & 317.7 & 1600 \\
            18.2  & 346.2 & 1900 \\
            19.9  & 376.8 & 2250 \\
            21.3  & 405.0 & 2600 \\
            22.9  & 435.1 & 3000 \\
            \bottomrule
        \end{tabular}%
     \label{tab:chapter5 exp1 QSI protocol}
\end{table}

\paragraph{Statistical processing} We compare scan/re-scan reproducibility by computing the absolute difference and relative difference in ADC and \gls{QSI} parameters over the defined \glspl{ROI}.
%
%
Further, we investigate the correlation between individual \gls{ADC} and \gls{QSI} measurements in XY and Z directions. We pool all voxel-wise measurements over the segmented \gls{SC} area and compute Pearson's correlation coefficient over all voxels. We test for statistical significant of the correlations with a confidence interval of 95\%.


We then compare significant differences in individual metrics using a paired two-tailed t-test and further investigate statisitical significance in the group mean values of the \gls{ADC} parameters and \gls{QSI} metrics between tracts by performing the Hotellings-T$^2$ test (confidence interval=95\%). To investigate the relevance of measurements in the different \gls{DWI} directions, we compute the same significance test of XY-only \gls{QSI} parameters (P0$_{xy}$, FWHM$_{xy}$) and compare with Z-only (P0$_z$, FWHM$_z$) and the combination of both (P0$_{xy}$, FWHM$_{xy}$, P0$_z$, FWHM$_z$).

\subsection{Results}
\paragraph{Reproducibility}
\label{par:chapter5 exp1 reproducibility}
Tables~\ref{tab:chap5exp1 scan rescan} shows absolute and relative differences between scan and rescan of three healthy subjects in ADC$_{xy}$ and ADC$_z$ (\ref{tab:chap5exp1 scan rescan adc}) and \gls{QSI} metrics in XY and Z direction (\ref{tab:chap5exp1 scan rescan qsi}). We observe a general trend of measurements perpendicular to the long \gls{SC} fibres presenting higher variation between scan and rescan than parallel measurements in \gls{ADC} and both \gls{QSI} metrics in all subjects. In particular ADC$_{xy}$ shows very high intra-subject variation between 20--40\% on average in all white matter \glspl{ROI}, while only GM values show good reproducibility value of less than 11\% variation. In particular ADC$_z$ appears more reproducible in all three subjects with average relative variation between 5--16\%.


The perpendicular \gls{QSI} metrics P0${_{xy}}$ and FWHM$_{xy}$ present good reproducibility values of 6--12\% and are up to 4 times lower than ADC$_{xy}$ measurements in corresponding \glspl{ROI}. In both P0$_{z}$ and FWHM$_{z}$ we observe relative change between 4--13\% similar to values in ADC$_z$.%
\begin{table}
\begin{minipage}{\linewidth}
    \centering%
    \caption{Absolute and relative change (in percent) between scan and rescan of diffusivities and QSI parameters in 3 healthy volunteers}
    \footnotesize
	\subfloat[Perpendicular (ADC$_{xy}$) and parallel diffusivity (ADC$_{z}$)]
	{
    		\begin{minipage}{\linewidth}
	        \begin{tabular}{rrrrrr}
            \addlinespace
			\multicolumn{6}{c}{\textbf{ADC$_{xy}$ $\times$ $10^{-9}m^2/s$}}\\
			\toprule
            subject & rLT   & lLT   & AT    & PT    & GM \\
            \midrule
            1     & 0.10 (30.4\%) & 0.00 (4.7\%) & 0.07 (27.6\%) & 0.06 (24.1\%) & 0.09 (12.0\%) \\
            2     & 0.06 (16.9\%) & 0.06 (34.4\%) & 0.12 (44.6\%) & 0.03 (11.0\%) & 0.05 (12.0\%) \\
            3     & 0.09 (25.5\%) & 0.12 (51.9\%) & 0.24 (57.2\%) & 0.20 (82.5\%) & 0.04 (8.6\%) \\
                  &       &       &       &       &  \\
            mean  & 0.08 (24.3\%) & 0.06 (30.4\%) & 0.14 (43.1\%) & 0.10 (39.2\%) & 0.06 (10.9\%) \\
            \bottomrule
            \end{tabular}%
			\\[0.5ex]
		\begin{tabular}{rrrrrr}
			\addlinespace
			\multicolumn{6}{c}{\textbf{ADC$_{z}$ $\times$ $10^{-9}m^2/s$}}\\
			\toprule
			subject & rLT   & lLT   & AT    & PT    & GM \\
		            \midrule
		            1     & 0.04 (3.3\%) & 0.07 (4.7\%) & 0.18 (12.2\%) & 0.03 (2.1\%) & 0.03 (2.4\%) \\
		            2     & 0.13 (9.0\%) & 0.17 (9.8\%) & 0.40 (23.2\%) & 0.03 (1.6\%) & 0.30 (16.9\%) \\
		            3     & 0.16 (12.5\%) & 0.10 (6.2\%) & 0.21 (12.9\%) & 0.16 (10.2\%) & 0.28 (16.6\%) \\
		                  &       &       &       &       &  \\
		            mean  & 0.11 (8.3\%) & 0.12 (6.9\%) & 0.26 (16.1\%) & 0.07 (4.7\%) & 0.20 (12.0\%) \\
		            \bottomrule
		\end{tabular}%
		\end{minipage}
		\label{tab:chap5exp1 scan rescan adc}
		}\\	
	\subfloat[Perpendicular and parallel QSI parameters]
		    {
	    		\begin{minipage}{\linewidth}
				\begin{tabular}{rrrrrr}
		            \addlinespace
				\multicolumn{6}{c}{\textbf{P0$_{xy}$}}\\
				\toprule
				subject & rLT   & lLT   & AT    & PT    & GM \\
		            \midrule
		            1     & 0.01 (3.1\%) & 0.02 (6.8\%) & 0.01 (3.4\%) & 0.00 (1.7\%) & 0.00 (1.9\%) \\
		            2     & 0.00 (0.4\%) & 0.00 (0.3\%) & 0.01 (4.3\%) & 0.01 (3.3\%) & 0.00 (2.3\%) \\
		            3     & 0.01 (6.1\%) & 0.06 (28.2\%) & 0.04 (19.9\%) & 0.06 (26.7\%) & 0.03 (14.8\%) \\
		                  &       &       &       &       &  \\
		            mean  & 0.01 (3.2\%) & 0.03 (11.8\%) & 0.02 (9.2\%) & 0.02 (10.6\%) & 0.01 (6.3\%) \\
		            \bottomrule
		            \end{tabular}%
		        \\[0.5ex]
		        \begin{tabular}{rrrrrr}
		            \addlinespace
				\multicolumn{6}{c}{\textbf{FWHM$_{xy}$}}\\			
		            \toprule
		            subject & rLT   & lLT   & AT    & PT    & GM \\
		            \midrule
		            1     & 0.52 (2.5\%) & 0.67 (4.8\%) & 0.67 (3.5\%) & 0.29 (1.5\%) & 0.62 (2.4\%) \\
		            2     & 0.03 (0.1\%) & 0.29 (1.6\%) & 0.76 (4.1\%) & 0.36 (1.9\%) & 0.32 (1.5\%) \\
		            3     & 1.10 (5.2\%) & 5.69 (29.6\%) & 4.72 (20.6\%) & 5.10 (27.5\%) & 3.29 (15.5\%) \\
		                  &       &       &       &       &  \\
		            mean  & 0.55 (2.6\%) & 2.22 (12.0\%) & 2.05 (9.4\%) & 1.92 (10.3\%) & 1.41 (6.5\%) \\
		            \bottomrule
		        \end{tabular}%
			\\[0.5ex]
		        \begin{tabular}{rrrrrr}
		            \addlinespace
				\multicolumn{6}{c}{\textbf{P0$_{z}$}}\\
		            \toprule
		            subject & rLT   & lLT   & AT    & PT    & GM \\
		            \midrule
		            1     & 0.00 (4.5\%) & 0.00 (0.6\%) & 0.01 (6.9\%) & 0.00 (2.5\%) & 0.01 (5.6\%) \\
		            2     & 0.01 (9.2\%) & 0.01 (11.4\%) & 0.01 (11.0\%) & 0.00 (3.4\%) & 0.01 (10.6\%) \\
		            3     & 0.01 (6.0\%) & 0.00 (0.1\%) & 0.00 (1.2\%) & 0.01 (7.8\%) & 0.01 (14.9\%) \\
		                  &       &       &       &       &  \\
		            mean  & 0.01 (6.6\%) & 0.00 (4.1\%) & 0.01 (6.3\%) & 0.00 (4.6\%) & 0.01 (10.4\%) \\
		            \bottomrule
		        \end{tabular}%
		        \\[0.5ex]
		        \begin{tabular}{rrrrrr}
		            \addlinespace
				\multicolumn{6}{c}{\textbf{FWHM$_{z}$}}\\		
		            \toprule
		            subject & rLT   & lLT   & AT    & PT    & GM \\
		            \midrule
		            1     & 1.42 (3.9\%) & 1.67 (3.9\%) & 4.20 (10.4\%) & 1.07 (2.7\%) & 3.86 (10.7\%) \\
		            2     & 5.72 (15.4\%) & 9.04 (22.2\%) & 4.69 (11.6\%) & 4.47 (10.9\%) & 4.76 (12.5\%) \\
		            3     & 1.08 (3.0\%) & 0.13 (0.3\%) & 1.66 (4.1\%) & 4.89 (12.4\%) & 6.17 (16.1\%) \\
		                  &       &       &       &       &  \\
		            mean  & 2.74 (7.4\%) & 3.61 (8.8\%) & 3.52 (8.7\%) & 3.48 (8.6\%) & 4.93 (13.1\%) \\
		            \bottomrule
		        \end{tabular}%
				\label{tab:chap5exp1 scan rescan qsi}
	    		\end{minipage}
		    }
\end{minipage}%
\label{tab:chap5exp1 scan rescan}
\end{table}

\paragraph{Differences between tract-specific ROI measurements}
We compare the average values and standard devation over all 9 subjects between tract-specific \glspl{ROI} for \gls{ADC} values in Figure~\ref{fig:chapter5 exp1 ADC vals} and \gls{QSI} metrics in Figure~\ref{fig:chapter5 exp1 QSI vals}. As a general trend, we observe higher inter-subject variation in XY measurements compared to Z measurements among all 9 subjects which is in line with our results of intra-subject variation shown above.


Table~\ref{tab:chap5exp1 single ttest} present $p$-values for pairwise differences between different tract-\glspl{ROI} for ADC and \gls{QSI} metrics. The most noteable differences are found between the GM \gls{ROI} and the white matter regions in ADC$_{xy}$ and both P0$_{xy}$/FWHM$_{xy}$ with high statistical significance ($p<0.01$ between WM tracts GM for all \gls{QSI}$_{xy}$ metrics), while ADC$_z$ and \gls{QSI}$_{z}$ metrics are less different between GM \gls{ROI} and WM \glspl{ROI}. In fact, significant differences are only found between rLT and GM in ADC$_z$. The \gls{QSI}$_z$ metrics only show significant differences between GM and rRT in P0$_z$ (p=0.01) and between GM and AT and PT (FWHM$_z$).


Between WM \glspl{ROI} only the left LT but not the right LT is significantly different from both AT and PT in ADC$_{xy}$ perpendicular to long white matter fibres. Parallel to the long \gls{SC} axis we only find ADC$_{z}$ in the right LT significantly lower from AT and PT. Left and right LT show significant differences in both ADC$_{xy}$ and ADC$_{z}$ while we find no difference between AT or PT. In \gls{QSI} metrics we find the same tracts as with ADC to be significantly different in XY and Z direction. However, $p$-values are increased in \gls{QSI} compared to corresponding ADC, but remain below $p<0.05$.%
\begin{figure}[h!tp]
	\subfloat[]
	{
		\begin{minipage}{\linewidth}
		   \centering
		   \pgfplotsset{cutoff_vs_dti_barchart/.style={ybar,
                                            bar width=20pt,
                                            width=6cm,
                                            height=6cm,
                                            xtick={{1},{2},{3},{4},{5},{6}},
                                            xticklabels={rLT,lLT,AT,PT,GM,SCA}, fill=red},
                                            yticklabel style={/pgf/number format/.cd,
                                                              fixed,
                                                              fixed zerofill,
                                                              precision=2}}
\pgfplotsset{cutoff_vs_dti_barchart plot/.style={fill=olive!40!white,error bars/.cd, y dir=both, y explicit}}

\begin{tikzpicture}
\begin{axis}[cutoff_vs_dti_barchart, title=$ADC_{xy}$ $\times 10^{-9} m^2/s$, ymin=0]
    \addplot+[cutoff_vs_dti_barchart plot] table[y=ADCX, y error=ADCXerr] {chapter5/figs/exp1_qspacevals.dat};
\end{axis}
\end{tikzpicture}
\begin{tikzpicture}
\begin{axis}[cutoff_vs_dti_barchart, title=$ADC_{z}$ $\times 10^{-9} m^2/s$, ymin=0]
    \addplot+[cutoff_vs_dti_barchart plot] table[y=ADCZ, y error=ADCZerr] {chapter5/figs/exp1_qspacevals.dat};
\end{axis}
\end{tikzpicture}

		   \label{fig:chapter5 exp1 ADC vals}
		\end{minipage}%
	}\\
	\subfloat[]
	{
		\begin{minipage}{\linewidth}
	   	% Table generated by Excel2LaTeX from sheet 'Sheet2'

%!TEX root = ../chap4.tex
\pgfplotsset{cutoff_vs_dti_barchart/.style={ybar,
                                            bar width=20pt,
                                            width=6cm,
                                            height=6cm,
                                            xtick={{1},{2},{3},{4},{5},{6}},
                                            xticklabels={rLT,lLT,AT,PT,GM,SCA}},
                                            yticklabel style={/pgf/number format/.cd,
                                                              fixed,
                                                              fixed zerofill,
                                                              precision=2}}
\pgfplotsset{cutoff_vs_dti_barchart plot/.style={error bars/.cd, y dir=both, y explicit}}
\begin{tikzpicture}
\begin{axis}[cutoff_vs_dti_barchart, title=$P0_{xy}$, ymin=0]
    \addplot+[cutoff_vs_dti_barchart plot] table[y=P0X, y error=P0Xerr] {chapter5+6/figs/exp1_qspacevals.dat};
\end{axis}
\end{tikzpicture}
\begin{tikzpicture}
\begin{axis}[cutoff_vs_dti_barchart, title=$P0_{z}$, ymin=0]
    \addplot+[cutoff_vs_dti_barchart plot] table[y=P0Z, y error=P0Zerr] {chapter5+6/figs/exp1_qspacevals.dat};
\end{axis}
\end{tikzpicture}\\
\begin{tikzpicture}
\begin{axis}[cutoff_vs_dti_barchart, title=$FWHM_{xy}$ $\times 10^{-6} m$, ymin=0]
    \addplot+[cutoff_vs_dti_barchart plot] table[y=FWHMX, y error=FWHMXerr] {chapter5+6/figs/exp1_qspacevals.dat};
\end{axis}
\end{tikzpicture}
\begin{tikzpicture}
\begin{axis}[cutoff_vs_dti_barchart, title=$FWHM_{z}$ $\times 10^{-6} m$, ymin=0]
    \addplot+[cutoff_vs_dti_barchart plot] table[y=FWHMZ, y error=FWHMZerr] {chapter5+6/figs/exp1_qspacevals.dat};
\end{axis}
\end{tikzpicture}	   
		\end{minipage}%
	}
   \caption{Mean and standard deviation of perpendicular and parallel ADC and QSI metrics for all ROIs over all 9 volunteers.}
   \label{fig:chapter5 exp1 QSI vals}	
\end{figure}

\begin{table}
	\footnotesize
    \centering
    \caption{Significance of pair-wise differences between SC tracts in diffusivities and QSI parameters (confidence interval: 95\%). Statistically significant differences are marked as follows: \textbf{bold} if $p<0.05$, \textbf{\emph{bold-italic}} if $p<0.01$.}	
    \subfloat[]
       	{
   				\begin{minipage}{\linewidth}%
              	\begin{tabular}{rrrrr}
				\addlinespace
				\multicolumn{5}{c}{\textbf{ADC$_{xy}$ $\times$ $10^{-9}m^2/s$}}\\
				\toprule
                      & lLT  & AT    & PT    & GM \\
                \midrule
                rLT  & \textbf{0.01}  & 0.60  & 0.84  & \textbf{\emph{<0.01}} \\
                lLT  &       & \textbf{\emph{<0.01}}  & \textbf{\emph{<0.01}}  & \textbf{\emph{<0.01}} \\
                AT    &       &       & 0.56  & \textbf{\emph{<0.01}} \\
                PT    &       &       &       & \textbf{\emph{<0.01}} \\
                \bottomrule
                \end{tabular}%
    			\hspace{0.5cm}
				%%
				%%
		        \begin{tabular}{rrrrr}
		        \addlinespace
				\multicolumn{5}{c}{\textbf{ADC$_{z}$ $\times$ $10^{-9}m^2/s$}}\\
				\toprule
		              & lLT  & AT    & PT    & GM \\
		        \midrule
		        rLT  & \textbf{0.01}  & \textbf{\emph{<0.01}}  & \textbf{\emph{<0.01}}  & \textbf{\emph{<0.01}} \\
		        lLT  &       & 0.85  & \textbf{\emph{<0.01}}  & 0.57 \\
		        AT    &       &       & 0.44  & 0.30 \\
		        PT    &       &       &       & 0.74 \\
		        \bottomrule
		        \end{tabular}%
				\end{minipage}%
				\label{tab:chap5exp1_adc single ttest}%
				
  	   }\\
  	   	\subfloat[]
  	  	{			
		\begin{minipage}{\linewidth}
  				    \begin{tabular}{rrrrr}
  				    \addlinespace
					\multicolumn{5}{c}{\textbf{P0$_{xy}$}}\\
					\toprule
  				          & lLT  & AT    & PT    & GM \\
  				    \midrule
  				    rLT  & \textbf{0.04}  & 0.27  & 0.48  & \textbf{\emph{<0.01}} \\
  				    lLT  &       & \textbf{0.05}  & 0.48  & \textbf{\emph{<0.01}} \\
  				    AT    &       &       & 0.97  & \textbf{\emph{<0.01}} \\
  				    PT    &       &       &       & \textbf{\emph{<0.01}} \\
  				    \bottomrule
  				    \end{tabular}%
  	  			    \hspace{0.5cm}
  				    \begin{tabular}{rrrrr}
  				    \addlinespace
					\multicolumn{5}{c}{\textbf{P0$_{z}$}}\\
					\toprule
  				          & lLT  & AT    & PT    & GM \\
  				    \midrule
  				    rLT  & \textbf{0.01}  & \textbf{\emph{<0.01}}  & \textbf{\emph{<0.01}}  & \textbf{0.01} \\
  				    lLT  &       & 0.94  & \textbf{\emph{<0.01}}  & 0.77 \\
  				    AT    &       &       & 0.40  & 0.69 \\
  				    PT    &       &       &       & 0.16 \\
  				    \bottomrule
  				    \end{tabular}%
  					\\[0.5ex]
  	 				\begin{tabular}{rrrrr}
  	 			    \addlinespace
					\multicolumn{5}{c}{\textbf{FWHM$_{xy}$}}\\		
					\toprule
  				        & lLT  & AT    & PT    & GM \\
  				    \midrule
  				    rLT  & \textbf{0.04}  & 0.56  & 0.37  & \textbf{\emph{<0.01}} \\
  				    lLT  &       & \textbf{0.02}  & 0.37  & \textbf{\emph{<0.01}} \\
  				    AT    &       &       & 0.72  & \textbf{0.01} \\
  				    PT    &       &       &       & \textbf{\emph{<0.01}} \\
  				    \bottomrule
  				    \end{tabular}%
  	  			    \hspace{0.5cm}
  				    \begin{tabular}{rrrrr}
  				    \addlinespace
					\multicolumn{5}{c}{\textbf{FWHM$_{z}$}}\\		
					\toprule
  				          & lLT  & AT    & PT    & GM \\
  				    \midrule
  				    rLT  & 0.21  & \textbf{0.02}  & \textbf{0.03}  & 0.99 \\
  				    lLT  &       & 0.20  & \textbf{0.03}  & 0.13 \\
  				    AT    &       &       & 1.00  & \textbf{0.01} \\
  				    PT    &       &       &       & \textbf{0.01} \\
  				    \bottomrule
  				    \end{tabular}%
  				\end{minipage}%
  	  	  	  }
  \label{tab:chap5exp1 single ttest}%
\end{table}%

\paragraph{Multi-variate differences between tract-specific ROI measurements}
The single parameter comparisions above indicate that both \gls{ADC} and \gls{QSI} metrics can discriminate some WM tracts, but offer complementary information in perpendicular and parallel measurements. The multivariate Hotelling's-T$^2$ allows us to test whether a combination of XY and Z metrics is better suited to characterize and discriminate WM measures in different \glspl{ROI}. In the following, we present results for the following combinations of parameters:
\begin{itemize}
    \item Both diffusivity parameters ADC$_{xy}$ and ADC$_z$ (Table~\ref{tab:chap5exp1_adc hotelling})
    \item Perpendicular only \gls{QSI} metrics P0$_{xy}$ and FWHM$_{xy}$ (Table~\ref{tab:chap5exp1_adc hotelling})
    \item Parallel only \gls{QSI} metrics P0$_z$ and FWHM$_z$ (Table~\ref{tab:chap5exp1_qsiz hotelling})
    \item Perpendicular and parallel \gls{QSI} metrics P0$_{xy}$, FWHM$_{xy}$, P0$_z$ and FWHM$_z$ (Table~\ref{tab:chap5exp1_qsiall hotelling})
\end{itemize}


Similar to the single t-test results shown above, GM and WM \glspl{ROI} can clearly be distinguished using either of the combinations of \gls{ADC} and \gls{QSI} parameters. However, GM/WM differences are more pronounced in XY than in Z direction. The combined ADC metrics show significant differences between the both lateral tracts and also l/r LT and the posterior WM \gls{ROI}. AT is only significantly different from the right but not the left LT. 

For combination of \gls{QSI} parameters in Z only as well as the combination of both XY and Z, the only two emerging differences are found between lLT/rLT and rLT/PT, both with $p<0.05$.%

\begin{table}
  \centering
  \footnotesize
  \caption{Hotelling's-T$^2$ significance of pair-wise tract-specific differences for ADC and QSI parameters. (\textbf{bold} marks $p<0.05$, \textbf{\emph{bold-italic}} marks $p<0.01$).}
  \subfloat[Combined ADC$_{xy}$,ADC$_{z}$]
  {
	    \begin{tabular}{rrrrr}
	    \addlinespace
	    \toprule
	          & lLT  & AT    & PT    & GM \\
	    \midrule
	    rLT  & \textbf{\emph{<0.01}}  & \textbf{0.02}  & \textbf{0.01}  & \textbf{\emph{<0.01}} \\
	    lLT  &       & 0.10  & \textbf{0.01}  & \textbf{\emph{<0.01}} \\
	    AT    &       &       & 0.85  & \textbf{\emph{<0.01}} \\
	    PT    &       &       &       & \textbf{\emph{<0.01}} \\
	    \bottomrule
	    \end{tabular}%
	  \label{tab:chap5exp1_adc hotelling}%
  }\hspace{0.5cm}
  \subfloat[Combined perpendicular QSI parameters (P0$_{xy}$,FWHM$_{xy})$]{
      \begin{tabular}{rrrrr}
        \addlinespace
        \toprule
              & lLT  & AT    & PT    & GM \\
        \midrule
        rLT  & 0.13 & 0.79  & 0.71 &  \textbf{0.01} \\
        lLT  &       & 0.26  & 0.12 & \textbf{\emph{<0.01}} \\
        AT    &       &       & 0.76 & \textbf{0.02}  \\
        PT    &       &       &       & \textbf{\emph{<0.01}} \\
        \bottomrule
      \end{tabular}%
      \label{tab:chap5exp1_qsix hotelling}%
  }\\
  \subfloat[Combined parallel QSI parameters (P0$_{z}$,FWHM$_{z}$)]
  {
    \begin{tabular}{rrrrr}
    \addlinespace
    \toprule
          & lLT  & AT    & PT    & GM \\
    \midrule
    rLT  & \textbf{0.03}  & 0.08  & \textbf{0.02}  & \textbf{0.02} \\
    lLT  &       & 0.60  & 0.86  & 0.18 \\
    AT    &       &       & 0.49  & \textbf{0.01} \\
    PT    &       &       &       & \textbf{\emph{<0.01}} \\
    \bottomrule
    \end{tabular}%
    \label{tab:chap5exp1_qsiz hotelling}%
  }\hspace{0.5cm}
  \subfloat[Combined perpendicular and parallel QSI parameters (P0$_{xy}$,FWHM$_{xy}$,P0$_{z}$,FWHM$_{z}$)]
  {
    \begin{tabular}{rrrrr}
    \addlinespace
    \toprule
          & lLT  & AT    & PT    & GM \\
    \midrule
    rLT  & \textbf{0.04}  & 0.25  & \textbf{0.01}  & \textbf{\emph{<0.01}} \\
    lLT  &       & 0.62  & 0.50  & \textbf{\emph{<0.01}} \\
    AT    &       &       & 0.75  & \textbf{\emph{<0.01}} \\
    PT    &       &       &       & \textbf{\emph{<0.01}} \\
    \bottomrule
    \end{tabular}%
    \label{tab:chap5exp1_qsiall hotelling}%
  }
  \label{tab:chap5exp1  hotelling}%
\end{table}%

\paragraph{Voxel-wise correlation of ADC and QSI metrics}
Table~\ref{tab:chapter5 exp1 correlations} shows the Pearson correlation coefficient $\mathit{r}$ and statistical significance of the correlation between ADC and \gls{QSI} parameters over all \gls{SC} voxels in all subjects. We observe significant correlations between \gls{ADC}$_{xy}$ and \gls{ADC}$_z$. Further we find significant correlations between
and \gls{QSI} parameters both within and across XY and Z direction. Interestingly, we find both FWHM$_{xy}$ and P0$_z$ are correlated with each other and also with both ADC parameters, while the other two \gls{QSI} parameters P0$_{xy}$ and FWHM$_z$ did neither correlate with each other nor any other metrics.%

\begin{table}
	\centering
    \caption{Pearson-correlation coefficient and significance between all ADC and QSI metrics. The $p$-values $<0.01$ are marked \textbf\emph{bold-italic}.}
    \begin{tabular}{rrrrrrrr}
	    \addlinespace
	    \toprule
	          &       & ADC$_{xy}$  & ADC$_{z}$  & P0$_{xy}$   & FWHM$_{xy}$   & P0$_{z}$   & FWHM$_{z}$ \\
	    \midrule
	    \multicolumn{1}{c}{\multirow{2}[0]{*}{ADC$_{xy}$}} & $\mathit{r}$   &       & 0.58  & 0.00  & -0.74 & 0.20  & 0.01 \\
	    \multicolumn{1}{c}{} & {p} & {} & {\textbf{\emph{<0.01}}} & {0.91} & {\textbf{\emph{<0.01}}} & {\textbf{\emph{<0.01}}} & {0.56} \\[2.0ex]
	    \multicolumn{1}{c}{\multirow{2}[0]{*}{ADC$_{z}$}} & $\mathit{r}$   & 0.58  &       & 0.00  & -0.29 & 0.71  & 0.00 \\
	    \multicolumn{1}{c}{} & {p} & {\textbf{\emph{<0.01}}} & {} & {0.87} & {\textbf{\emph{<0.01}}} & {\textbf{\emph{<0.01}}} & {0.82} \\[2.0ex]
	    \multicolumn{1}{c}{\multirow{2}[0]{*}{P0$_{xy}$}} & $\mathit{r}$   & 0.00  & 0.00  &       & 0.00  & 0.00  & 0.00 \\
	    \multicolumn{1}{c}{} & {p} & {0.91} & {0.87} & {} & {0.82} & {0.99} & {1.00} \\[2.0ex]
	    \multicolumn{1}{c}{\multirow{2}[0]{*}{FWHM$_{xy}$}} & $\mathit{r}$   & -0.74 & -0.29 & 0.00  &       & -0.18 & -0.01 \\
	    \multicolumn{1}{c}{} & {p} & {\textbf{\emph{<0.01}}} & {\textbf{\emph{<0.01}}} & {0.82} & {} & {\textbf{\emph{<0.01}}} & {0.70} \\[2.0ex]
	    \multicolumn{1}{c}{\multirow{2}[0]{*}{P0$_{z}$}} & $\mathit{r}$   & 0.20  & 0.71  & 0.00  & -0.18 &       & 0.01 \\
	    \multicolumn{1}{c}{} & {p} & {\textbf{\emph{<0.01}}} & {\textbf{\emph{<0.01}}} & {0.99} & \textbf{\emph{<0.01}} & {} & {0.52} \\[2.0ex]
	    \multicolumn{1}{c}{\multirow{2}[0]{*}{FWHM$_{z}$}} & $\mathit{r}$   & 0.01  & 0.00  & 0.00  & -0.01 & 0.01  &  \\
	    \multicolumn{1}{c}{} & {p} & {0.56} & {0.82} & {1.00} & {0.70} & {0.52} & {} \\
    \bottomrule
    \end{tabular}%
  \label{tab:chapter5 exp1 correlations}
\end{table}%

\subsection{Discussion} \gls{QSI} metrics obtained without sequence development, using standard{\gls{DWI}}protocol available on a 3T clinical scanner, show a good reproducibility that is superior to simple \gls{ADC} analysis. We observe tract-specific correlations between \gls{ADC} and \gls{QSI} parameters between several WM tracts. However some of the associations in \gls{QSI} metrics are weaker in XY compared to Z, particularly between lateral and posterior tracts. Together with the findings of weak of correlation between \gls{QSI} and ADC metrics in both XY and Z, our results suggest that the Z direction provides additional information to perpendicular measurements. Our results also suggest that on a clinical scanner \gls{QSI} might not be able to reliably distinguish between individual WM tracts.

\paragraph{} The results of this experiment need to be interpreted with caution due to the limitations in hardware and software in the experimental setup. In particular the low gradient strength used in this study might conceal differences between tracts. Simulations in \Citet{Laett:2008} have shown in simulation that insufficient gradient strength might lead to overestimation of compartment size and suggest gradients of minimum 60 mT to distinguish compartment sizes found in human WM. Further, the linear regridding that was necessar because of scanner software limitation might introduce further error in our measurements. Also, the in-plane resolution of the acquired images was insufficient to delineate individual WM, so the images had to interpolated before analysis. However, all these required pre-processing steps might weaken the confidence in our results. We therefore decided to repeat the experiment on our in-house scanner which possesses a more powerful gradient system. The availability of development tools to modify the pulse sequence on this scanner also allowed us to more options recreate and the adapt the protocol of \citet{Farrell:2008}.

\section{Experiment 2}
The aim of this study is to improve the major factors we identified as potential confounding factors in the Experiment 1. We carefully optimise the acquisition to achieve an increase in spatial resolution and signal-to-noise ratio of the axial \gls{DWI} measurements, and higher diffusion gradient strength as well as linear spacing of $q$-values.

\subsection{Methods}

\paragraph{Study design}
We recruit 10 healthy volunteers (4 male/6 female) to be scanned on a 3T Philips Achieva 3TX (Philips Healthcare, Eindhoven). Five subjects are rescanned at a different time  to assess intra-subject reproducibility of the derived parameters.
\paragraph{Data acquisition}
To ensure consistent positioning of the \gls{DWI} columes among all volunteers, we acquire a structural scan of the whole cervical cord using a sagittal T2 weighted turbo-spin-echo sequence (voxel size=1x1x3 mm$^3$, FOV=256x247$mm^2$, TR=4000ms, TE=63ms, 2 averages). We then position the \gls{DWI} images based on the structural scan so that the centre of the acquisitions volume is aligned with the C2/C3 disc and the slice directions is parallel to the cord at this level.


We use a cardiac gated \gls{DWI} acquisition with the following imaging parameters: voxel size=1x1x5 mm$^3$, FOV=64x64$mm^2$, TR=9RR, TE=129ms). To avoid aliasing artifacts from surrounding tissue we use a ZOOM acquisition with outer-volume suppression as described in \cite{TODO:ZOOM with OVS}. We acquire 32 \gls{DWI} equidistantly spaced $q$-values in two directions perpendicular (XY) and in one parallel (Z) direction with respect to the main {\gls{SC}} axis. To achieve the maximum possible gradient strength on our scanner we exploit the layout of the   gradient amplifiers in our scanner, which each can generate a maximum \gls{gstr} of 62mT/m in the along the major axes of the scanner bore. Assuming axial symmetry of the axons along the long axis of the spinal cord, we modify the scanner software to drive multiple gradient amplifiers in two orthogonal directions perpendicular to major SC fibre direction (see Figure~\ref{fig:chapter5_exp2_overplus_cartoon} for illustration). This allows us to generate a guaranteed maximum \gls{gstr} of $\sqrt{2} * 62mT/m = 87mT/m$ in XY direction.  In Z direction we use a maximum \gls{gstr} of 62 $mT/m$. We use the same $q$-values in this experiment as described in \citet{Farrell:2008}. However the increase in \gls{gstr} allows us to reduce the gradient duration from XXms to 11.4ms in XY direction. The full protocol is given in Table~\ref{tab:chapter 5 exp2 QSI protocol}.


\begin{figure}
  \pgfimage[width=\textwidth]{chapter5/figs/overplus_cartoon.pdf}
  \caption{Cartoon of our implemented gradient strength modification method.}
  \label{fig:chapter5_exp2_overplus_cartoon}
\end{figure}

\begin{table}
      \centering
     \caption{QSI protocol displaying: Gradient strength (G), q-value (q) and b-value for each of the 32 DWI volumes.}
     \subfloat[Protocol for X and Y direction]{
         \begin{tabular}{rrr}
            \addlinespace
                \multicolumn{3}{l}{}\\
            \toprule
                G $[mT/m]$ & q $[cm^{-1}]$ & b $[s/mm^2]$ \\
                \cmidrule(r){1-1}\cmidrule(lr){2-2}\cmidrule(l){3-3}
                0.0   & 0.0   & 0 \\
                5.8   & 66.2  & 22 \\
                11.7  & 132.8 & 90 \\
                17.5  & 198.6 & 200 \\
                23.3  & 264.5 & 355 \\
                29.1  & 330.3 & 554 \\
                35.0  & 397.3 & 802 \\
                40.8  & 463.1 & 1089 \\
                46.6  & 528.9 & 1421 \\
                52.5  & 595.9 & 1803 \\
                58.3  & 661.7 & 2224 \\
                64.1  & 727.5 & 2688 \\
                69.9  & 793.4 & 3197 \\
                75.8  & 860.3 & 3759 \\
                81.6  & 926.2 & 4357 \\
                87.4  & 992.0 & 4998 \\
                \bottomrule
            \end{tabular}%
            \hspace{0.2cm}%
            \begin{tabular}{rrr}
            \addlinespace
                \multicolumn{3}{l}{\textit{... continued}}\\
            \toprule
                G $[mT/m]$ & q $[cm^{-1}]$ & b $[s/mm^2]$ \\
                \cmidrule(r){1-1}\cmidrule(lr){2-2}\cmidrule(l){3-3}
                0.0  & 0.0  & 0 \\
                2.9  & 33.0  & 6 \\
                8.7  & 99.2  & 50 \\
                14.6  & 165.7 & 139 \\
                20.4  & 231.5 & 272 \\
                26.2  & 297.4 & 449 \\
                32.1  & 364.3 & 674 \\
                37.9  & 430.2 & 940 \\
                43.7  & 496.0 & 1250 \\
                49.5  & 561.8 & 1603 \\
                55.4  & 628.8 & 2008 \\
                61.2  & 694.6 & 2451 \\
                67    & 760.5 & 2937 \\
                72.9  & 827.4 & 3477 \\
                78.7  & 893.2 & 4053 \\
                84.5  & 959.1 & 4672 \\
                \bottomrule
            \end{tabular}%
     }\\[1cm]
     \subfloat[Protocol for Z direction]{
         \begin{tabular}{rrr}
            \addlinespace
                \multicolumn{3}{l}{}\\
            \toprule
                G $[mT/m]$ & q $[cm^{-1}]$ & b $[s/mm^2]$ \\
                \cmidrule(r){1-1}\cmidrule(lr){2-2}\cmidrule(l){3-3}
                0.0   & 0.0   & 0 \\
                4.1   & 46.9  & 11 \\
                8.3   & 94.2  & 45 \\
                12.4  & 140.9 & 101 \\
                16.5  & 187.6 & 179 \\
                20.6  & 234.2 & 279 \\
                24.8  & 281.7 & 403 \\
                28.9  & 328.4 & 548 \\
                33.0  & 375.1 & 715 \\
                37.2  & 422.6 & 907 \\
                41.3  & 469.3 & 1119 \\
                45.5  & 516.0 & 1352 \\
                49.6  & 562.7 & 1608 \\
                53.8  & 610.2 & 1891 \\
                57.9  & 656.9 & 2191 \\
                62.0  & 703.5 & 2514 \\
                \bottomrule
            \end{tabular}%
            \hspace{0.2cm}%
            \begin{tabular}{rrr}
            \addlinespace
                \multicolumn{3}{l}{\textit{... continued}}\\
            \toprule
                G $[mT/m]$ & q $[cm^{-1}]$ & b $[s/mm^2]$ \\
                \cmidrule(r){1-1}\cmidrule(lr){2-2}\cmidrule(l){3-3}
                0.0   & 0.0   & 0 \\
                2.1   & 23.4  & 3 \\
                6.2   & 70.4  & 25 \\
                10.4  & 117.5 & 70 \\
                14.5  & 164.2 & 137 \\
                18.6  & 210.9 & 226 \\
                22.8  & 258.4 & 339 \\
                26.9  & 305.1 & 473 \\
                31.0  & 351.8 & 629 \\
                35.1  & 398.5 & 806 \\
                39.3  & 446.0 & 1010 \\
                43.4  & 492.6 & 1233 \\
                47.5  & 539.3 & 1477 \\
                51.7  & 586.8 & 1749 \\
                55.8  & 633.5 & 2038 \\
                59.9  & 680.2 & 2350 \\
                \bottomrule
            \end{tabular}%
     }
     \label{tab:chapter5 exp1 QSI protocol}
\end{table}



\paragraph{Data processing}
We apply the same data processing pipeline as in the first experiment (see Section~\ref{sec:chapter 5 exp1 methods}) with the exception of the linear regridding of acquired q-values, which is not necessary in this data set. Figure~\ref{fig:chapter5 exemplary maps} shows \gls{ADC} maps, \gls{P0}  and \gls{FWHM} in one representative volunteer.

\paragraph{ROI analysis}
As in the previous experiment, we segment whole cervical {\gls{SC}} and place \glspl{ROI} in the lateral columns and the anterior and posterior tracts between level C1/2 and C3 in all subjects.

\paragraph{Statistical processing} We derive the same statistics from this dataset as in the first study. We present the absolute difference and relative difference in ADC and \gls{QSI} parameters over the defined \glspl{ROI} in the scan/re-scan cases. Further we show results of t-tests between between different tracts for individual metrics and the multivariate Hotelling-T$^2$ test for combination of parameters. We also investigate voxel-wise correlations between the six metrics using Pearson correlation coefficient.


\subsection{Results and Discussion}
\paragraph{Reproducibility}
\label{par:chapter5 exp2 reproducibility}
Table~\ref{tab:chap5exp2 scan rescan adc} and Table~\ref{tab:chap5exp2 scan rescan qsi} show the intra-subject variability for ADC and QSI metrics in all five subjects. As in Experiment 1 we observe better reproduciblity of ADC and QSI metrics in all \glspl{ROI} in Z compared to XY direction. Similar to results in the first experiment, we find that QSI metrics are generally more reproducible than ADC values in all \glspl{ROI}. However, the variation in the measured values is considerably lower in this study. The intra-subject variation of ADC values is reduced to less than 26\% in XY and less than 7\% in Z (Experiment 1: 40\% in XY and 16\% in Z). Moreover, we find low variation of QSI metrics within subject in XY (less than 10\%) and Z (less than 5\%). The combination of improved \gls{SNR} by using a modified imaging protocol and more careful positioning contribute to the overall good reproducibility of our measurements.%
\\[2cm]
\begin{minipage}{\linewidth}
    \tabcaption{Absolute and relative change (in percent) between scan and rescan of perpendicular and parallel diffusivities in 4 healthy volunteers}
    \footnotesize
    \centering
        \subfloat[ADC$_{xy}$ $\times$ $10^{-9}m^2/s$]
        {
            \begin{tabular}{rrrrrrr}
                \addlinespace
                \toprule
                subject& rLT   & lLT   & AT    & PT    & GM  \\
                \midrule
                1     & 0.02 (7.6\%) & 0.04 (11.5\%) & 0.01 (1.9\%) & 0.03 (9.5\%) & 0.01 (1.0\%) \\
                2     & 0.03 (10.1\%) & 0.10 (34.9\%) & 0.07 (15.0\%) & 0.21 (47.3\%) & 0.01 (3.0\%) \\
                3     & 0.02 (6.5\%) & 0.09 (24.4\%) & 0.09 (18.6\%) & 0.13 (37.5\%) & 0.08 (14.8\%) \\
                4     & 0.11 (29.8\%) & 0.06 (16.4\%) & 0.17 (32.1\%) & 0.03 (10.4\%) & 0.03 (5.3\%) \\
                      &       &       &       &       &         \\
                mean  & 0.05 (13.5\%) & 0.07 (21.8\%) & 0.08 (16.9\%) & 0.10 (26.2\%) & 0.03 (6.0\%) \\
                \bottomrule
            \end{tabular}%
        }\\
        \subfloat[ADC$_z$ $\times$ $10^{-9}m^2/s$]
        {
            % Table generated by Excel2LaTeX from sheet 'Sheet2'
            \begin{tabular}{rrrrrrr}
                \addlinespace
                \toprule
                subject & rLT   & lLT   & AT    & PT    & GM\\
                \midrule
                1     & 0.22 (10.9\%) & 0.07 (3.4\%) & 0.23 (12.3\%) & 0.24 (11.9\%) & 0.31 (19.9\%)\\
                2     & 0.33 (17.4\%) & 0.12 (5.9\%) & 0.23 (14.0\%) & 0.32 (14.3\%) & 0.34 (17.6\%) \\
                3     & 0.18 (9.3\%) & 0.01 (0.4\%) & 0.05 (2.6\%) & 0.03 (1.3\%) & 0.04 (2.1\%) \\
                4     & 0.19 (9.5\%) & 0.05 (2.7\%) & 0.01 (0.6\%) & 0.12 (5.6\%) & 0.13 (7.4\%) \\
                      &       &       &       &       &        \\
                mean  & 0.23 (11.8\%) & 0.06 (3.1\%) & 0.13 (7.4\%) & 0.18 (8.3\%) & 0.21 (11.8\%) \\
                \bottomrule
            \end{tabular}%

        }
    \label{tab:chap5exp2 scan rescan adc}
\end{minipage}%
\\[1cm]
\begin{minipage}{\linewidth}
        \tabcaption{Absolute and relative change (in percent) between scan and rescan of perpendicular and parallel QSI metrics in 4 healthy volunteers for all tract-specific ROIs}
        \footnotesize
        \centering
        \subfloat[P0$_{xy}$]
        {
          % Table generated by Excel2LaTeX from sheet 'Sheet2'
            \begin{tabular}{rrrrrrr}
            \addlinespace
            \toprule
            subject & rLT   & lLT   & AT    & PT    & GM    \\
            \midrule
            1     & 0.00 (0.0\%) & 0.04 (18.4\%) & 0.02 (8.6\%) & 0.00 (1.8\%) & 0.00 (2.5\%) \\
            2     & 0.00 (0.0\%) & 0.03 (11.1\%) & 0.01 (4.4\%) & 0.03 (15.8\%) & 0.01 (3.2\%)\\
            3     & 0.01 (5.7\%) & 0.00 (0.0\%) & 0.01 (4.7\%) & 0.02 (10.6\%) & 0.02 (10.9\%) \\
            4     & 0.01 (3.8\%) & 0.01 (3.4\%) & 0.00 (2.2\%) & 0.00 (0.0\%) & 0.01 (4.0\%)  \\
                  &       &       &       &       &        \\
            mean  & 0.01 (0.0\%) & 0.02 (0.0\%) & 0.01 (0.0\%) & 0.01 (0.0\%) & 0.01 (0.0\%)  \\
            \bottomrule
            \end{tabular}%

        }\\
        \subfloat[FWHM$_{xy}$ $\times$ $10^{-6}m$]
        {
            % Table generated by Excel2LaTeX from sheet 'Sheet2'
            \begin{tabular}{rrrrrrr}
            \addlinespace
            \toprule
            subject & rLT   & lLT   & AT    & PT    & GM    \\
            \midrule
            1     & 0.00 (0.0\%) & 3.00 (14.6\%) & 1.60 (7.1\%) & 0.40 (1.9\%) & 0.70 (2.5\%) \\
            2     & 0.20 (0.9\%) & 1.90 (9.4\%) & 1.00 (4.0\%) & 3.20 (13.6\%) & 1.00 (4.7\%) \\
            3     & 1.20 (5.9\%) & 0.30 (1.4\%) & 0.50 (2.1\%) & 2.00 (9.3\%) & 1.90 (7.6\%) \\
            4     & 0.40 (1.8\%) & 0.10 (0.4\%) & 0.10 (0.4\%) & 0.40 (1.8\%) & 0.80 (3.1\%)  \\
                  &       &       &       &       &       \\
            mean  & 0.45 (2.2\%) & 1.33 (6.5\%) & 0.80 (3.4\%) & 1.50 (6.7\%) & 1.10 (4.5\%) \\
            \bottomrule
            \end{tabular}%

        }\\
        \subfloat[P0$_{z}$]
        {
          % Table generated by Excel2LaTeX from sheet 'Sheet2'
            \begin{tabular}{rrrrrrr}
            \addlinespace
            \toprule
            subject & rLT   & lLT   & AT    & PT    & GM   \\
            \midrule
            1     & 0.01 (4.8\%) & 0.00 (2.9\%) & 0.01 (4.6\%) & 0.01 (7.5\%) & 0.01 (10.1\%)\\
            2     & 0.01 (11.1\%) & 0.00 (2.0\%) & 0.01 (7.0\%) & 0.01 (5.8\%) & 0.01 (10.9\%) \\
            3     & 0.01 (5.7\%) & 0.00 (1.2\%) & 0.00 (1.9\%) & 0.00 (0.4\%) & 0.00 (1.0\%)  \\
            4     & 0.01 (5.9\%) & 0.00 (0.0\%) & 0.00 (0.9\%) & 0.00 (3.1\%) & 0.00 (3.6\%) \\
                  &       &       &       &       &       \\
            mean  & 0.01 (6.9\%) & 0.00 (1.5\%) & 0.00 (3.6\%) & 0.00 (4.2\%) & 0.01 (6.4\%)  \\
            \bottomrule
            \end{tabular}%

        }\\
        \subfloat[FWHM$_{z}$ $\times$ $10^{-6}m$]
        {
        % Table generated by Excel2LaTeX from sheet 'Sheet2'
            \begin{tabular}{rrrrrrr}
            \addlinespace
            \toprule
            subject & rLT   & lLT   & AT    & PT    & GM  \\
            \midrule
            1     & 1.80 (3.0\%) & 1.00 (1.6\%) & 3.00 (5.3\%) & 5.40 (8.8\%) & 4.40 (8.4\%) \\
            2     & 6.10 (10.4\%) & 0.80 (1.3\%) & 3.50 (6.4\%) & 4.30 (6.5\%) & 8.50 (14.1\%) \\
            3     & 4.20 (7.1\%) & 1.20 (1.8\%) & 1.60 (2.7\%) & 0.40 (0.6\%) & 1.60 (2.6\%) \\
            4     & 3.50 (5.6\%) & 1.00 (1.7\%) & 0.50 (0.9\%) & 0.00 (0.0\%) & 1.60 (2.8\%) \\
                  &       &       &       &       &   & \\
            mean  & 3.90 (6.5\%) & 1.00 (1.6\%) & 2.15 (3.8\%) & 2.53 (4.0\%) & 4.03 (7.0\%) \\
            \bottomrule
            \end{tabular}%

        }
   \label{tab:chap5exp2 scan rescan qsi}
\end{minipage}


\paragraph{Differences between tract-specific ROI measurements}
\label{par:chapter5 exp2 tract specific}
Figure~\ref{fig:chapter5 exp2 ADC vals} and Figure~\ref{fig:chapter5 exp2 ADC vals} show the average and standard deviation of \gls{ADC} and \gls{QSI} values over all 10 subjects in each \gls{ROI}. Table~\ref{fig:chapter5 exp2 ADC vals} and Table~\ref{fig:chapter5 exp2 ADC vals} present the corresponding pairwise p-values of the t-test for statistically significant difference between \glspl{ROI}. As in Experiment 1 we observe lower ADC$_{xy}$ values in WM \glspl{ROI} compared to GM values. Similarly FWHM$_{xy}$ is lower and P0$_{xy}$ higher in WM than in GM. In Z direction ADC values are highest in WM and lower in GM. Furthermore, we find lower inter-subject variation in all Z values compared to their XY counterparts - a trend we already observed in Experiment 1 and in the results of intra-subject variation of this study.

We find FWHM$_{z}$ increased and decreased P0$_{z}$ in WM \glspl{ROI}. ADC and QSI values in both XY and Z are significantly different between the WM regions PT, left and right LT and GM. No statistical difference is observed between left and right WM \gls{ROI} in neither ADC or QSI values. However, left and right LT present significantly different than the PT in ADC$_z$ as well as in QSI values P0$_z$ and FWHM$_z$ but are not distinguishable in either of the XY metrics.

Results in AT are notably different than other WM \glspl{ROI} and shows no significant difference with GM in any metric. This can partly be explained by the largest standard deviation of all investigated \glspl{ROI}. However, average ADC and QSI values are closer to the values observed in GM than any of the other WM measurements. This is reflected by the $p$-values that are close or below the threshold for statistical significance between AT and the other WM \glspl{ROI} but not with the GM region. However, it must be noted that the AT is the most difficult \gls{ROI} to locate due to it's small size in the cervical part of the \gls{SC} and suffers \gls{CSF} contribution from the median fissure in this region. Therefore the resulting measurements in this region might rather be caused by partial volume effects with \gls{CSF} and GM than reflect a difference in underlying microstructure of the WM in the AT.%
\\[2cm]
\begin{minipage}{\linewidth}
      \centering
      \pgfplotsset{cutoff_vs_dti_barchart/.style={ybar,
                                            bar width=20pt,
                                            width=6cm,
                                            height=6cm,
                                            xtick={{1},{2},{3},{4},{5},{6}},
                                            xticklabels={rLT,lLT,AT,PT,GM,SCA}, fill=red},
                                            yticklabel style={/pgf/number format/.cd,
                                                              fixed,
                                                              fixed zerofill,
                                                              precision=2}}
\pgfplotsset{cutoff_vs_dti_barchart plot/.style={fill=olive!40!white,error bars/.cd, y dir=both, y explicit}}

\begin{tikzpicture}
\begin{axis}[cutoff_vs_dti_barchart, title=$ADC_{xy}$ $\times 10^{-9} m^2/s$, ymin=0]
    \addplot+[cutoff_vs_dti_barchart plot] table[y=ADCX, y error=ADCXerr] {chapter5+6/figs/exp2_qspacevals.dat};
\end{axis}
\end{tikzpicture}
\begin{tikzpicture}
\begin{axis}[cutoff_vs_dti_barchart, title=$ADC_{z}$ $\times 10^{-9} m^2/s$, ymin=0]
    \addplot+[cutoff_vs_dti_barchart plot] table[y=ADCZ, y error=ADCZerr] {chapter5+6/figs/exp2_qspacevals.dat};
\end{axis}
\end{tikzpicture}

      \figcaption{Mean and standard deviation of perpendicular and parallel diffusivities in all ROIs over all 10 volunteers.}
      \label{fig:chapter5 exp2 ADC vals}
\end{minipage}%
\\[1cm]
\begin{minipage}{\linewidth}
      \centering
      % Table generated by Excel2LaTeX from sheet 'Sheet2'

%!TEX root = ../chap4.tex
\pgfplotsset{cutoff_vs_dti_barchart/.style={ybar,
                                            bar width=20pt,
                                            width=6cm,
                                            height=6cm,
                                            xtick={{1},{2},{3},{4},{5},{6}},
                                            xticklabels={rLT,lLT,AT,PT,GM,SCA}},
                                            yticklabel style={/pgf/number format/.cd,
                                                              fixed,
                                                              fixed zerofill,
                                                              precision=2}}
\pgfplotsset{cutoff_vs_dti_barchart plot/.style={error bars/.cd, y dir=both, y explicit}}
\begin{tikzpicture}
\begin{axis}[cutoff_vs_dti_barchart, title=$P0_{xy}$, ymin=0]
    \addplot+[cutoff_vs_dti_barchart plot] table[y=P0X, y error=P0Xerr] {chapter5+6/figs/exp2_qspacevals.dat};
\end{axis}
\end{tikzpicture}
\begin{tikzpicture}
\begin{axis}[cutoff_vs_dti_barchart, title=$P0_{z}$, ymin=0]
    \addplot+[cutoff_vs_dti_barchart plot] table[y=P0Z, y error=P0Zerr] {chapter5+6/figs/exp2_qspacevals.dat};
\end{axis}
\end{tikzpicture}\\
\begin{tikzpicture}
\begin{axis}[cutoff_vs_dti_barchart, title=$FWHM_{xy}$ $\times 10^{-6} m$, ymin=0]
    \addplot+[cutoff_vs_dti_barchart plot] table[y=FWHMX, y error=FWHMXerr] {chapter5+6/figs/exp2_qspacevals.dat};
\end{axis}
\end{tikzpicture}
\begin{tikzpicture}
\begin{axis}[cutoff_vs_dti_barchart, title=$FWHM_{z}$ $\times 10^{-6} m$, ymin=0]
    \addplot+[cutoff_vs_dti_barchart plot] table[y=FWHMZ, y error=FWHMZerr] {chapter5+6/figs/exp2_qspacevals.dat};
\end{axis}
\end{tikzpicture}


      \figcaption{Mean and standard deviation of perpendicular and parallel QSI metrics in all ROIs over all 10 volunteers.}
      \label{fig:chapter5 exp2 QSI vals}
\end{minipage}%
\\[1cm]
\begin{minipage}{\linewidth}
    \footnotesize
    \centering
    \tabcaption{Significance of pair-wise differences between SC tracts in diffusion coefficients ADC$_{xy}$ and ADC$_{z}$ (confidence interval: 95\%)}
    \subfloat[ADC$_{xy}$]{
            \begin{tabular}{rrrrr}
            \addlinespace
            \toprule
                  & lLT   & AT    & PT    & GM \\
            \midrule
            rLT   & 0.51  & \emph{<0.01}  & 0.53  & \emph{<0.01} \\
            lLT   &       & \emph{<0.01}  & 0.60  & \emph{<0.01} \\
            AT    &       &       & \emph{0.03}  & 0.73 \\
            PT    &       &       &       & \emph{0.02} \\
            \bottomrule
            \end{tabular}%

    }\hspace{0.5cm}
    \subfloat[ADC$_z$]
    {
        \begin{tabular}{rrrrr}
        \addlinespace
        \toprule
              & lLT   & AT    & PT    & GM \\
        \midrule
        rLT   & 0.83  & 0.06  & \emph{0.03}  & \emph{0.02} \\
        lLT   &       & \emph{0.01}  & 0.06  & \emph{<0.01} \\
        AT    &       &       & \emph{<0.01}  & 0.44 \\
        PT    &       &       &       & \emph{<0.01} \\
        \bottomrule
        \end{tabular}%

    }
\label{tab:chap5exp2_adc single ttest}%
\end{minipage}%
\\[2cm]
\begin{minipage}{\linewidth}
  \centering
  \footnotesize
  \tabcaption{Significance of pair-wise differences between SC tracts in QSI metrics perpendicular (P0$_{xy}$ and FWHM$_{xy}$) and parallel (P0$_{xy}$ and FWHM$_{xy}$) to long SC axis (confidence interval: 95\%)}
  \subfloat[P0$_{xy}$]{
    % Table generated by Excel2LaTeX from sheet 'Sheet1'
        \begin{tabular}{rrrrr}
        \addlinespace
        \toprule
              &       &       &       &  \\
        \midrule
              & lLT   & AT    & PT    & GM \\
        rLT   & 0.96  & \emph{<0.01}  & 0.82  & \emph{<0.01} \\
        lLT   &       & \emph{<0.01}  & 0.83  & \emph{<0.01} \\
        AT    &       &       & \emph{<0.01}  & 0.36 \\
        PT    &       &       &       & \emph{0.01} \\
        \bottomrule
        \end{tabular}%

  }\hspace{0.5cm}
  \subfloat[FWHM$_{xy}$]
  {% Table generated by Excel2LaTeX from sheet 'Sheet1'
        \begin{tabular}{rrrrr}
        \addlinespace
        \toprule
              & lLT   & AT    & PT    & GM \\
        \midrule
        rLT   & 0.50  & \emph{<0.01}  & 0.58  & \emph{<0.01} \\
        lLT   &       & \emph{<0.01}  & 0.91  & \emph{<0.01} \\
        AT    &       &       & \emph{<0.01}  & 0.24 \\
        PT    &       &       &       & \emph{<0.01} \\
        \bottomrule
        \end{tabular}%
  }\\
  \subfloat[P0$_z$]
  {
        \begin{tabular}{rrrrr}
        \addlinespace
        \toprule
              & lLT   & AT    & PT    & GM \\
        \midrule
        rLT   & 0.93  & 0.06  & \emph{0.02}  & 0.05 \\
        lLT   &       & \emph{<0.01}  & 0.05  & \emph{0.01} \\
        AT    &       &       & \emph{<0.01}  & 0.74 \\
        PT    &       &       &       & \emph{<0.01} \\
        \bottomrule
        \end{tabular}%

  }\hspace{0.5cm}
  \subfloat[FWHM$_z$]
  {
    % Table generated by Excel2LaTeX from sheet 'Sheet1'
        \begin{tabular}{rrrrr}
        \addlinespace
        \toprule
              & lLT   & AT    & PT    & GM \\
        \midrule
        rLT   & 0.46  & 0.12  & \emph{<0.01}  & 0.19 \\
        lLT   &       & \emph{0.01}  & \emph{0.03}  & \emph{0.04} \\
        AT    &       &       & \emph{<0.01}  & 0.85 \\
        PT    &       &       &       & \emph{<0.01} \\
        \bottomrule
        \end{tabular}%

  }
  \label{tab:chap5exp2_qsi single ttest}%
\end{minipage}%
\\[2cm]
\paragraph{Multi-variate differences between tract-specific ROI measurements}
\label{par:chapter5 exp2 tract specific}
Table~\ref{tab:chap5exp2_qsi hotelling} and Table~\ref{tab:chap5exp2_qsi hotelling} show the the statistical significance of tract-specific differences using combinations of ADC$_{xy}$, ADC$_{z}$, and P0 and FWHM metrics in XY and Z. As expected from single metric results, the combination ADC$_{xy}$+ADC$_{z}$ most sensitive to differences between WM (except AT) and GM as well as between both LTs and the PT. The combination of both QSI metrics in XY revealed highly significant difference between GM and all WM regions except AT, but not between any pair of WM regions. Differences between GM and WM are weaker in the combined QSI metrics in Z. However, P0$_z$+FWHM$_z$ revealed differences between PT and rLT and PT and AT that are not present in XY values. Combined XY and Z QSI metrics show the weakest differences. We find p<0.05 only between GM and rLT, PT and between PT and AT.%
\\[2cm]
\begin{minipage}{\linewidth}
  \centering
  \tabcaption{Hotelling's-T$^2$ significance of pair-wise tract-specific differences ADC$_{xy}$+ADC$_{z}$ (confidence interval: 95\%)}
    % Table generated by Excel2LaTeX from sheet 'Sheet1'
        \begin{tabular}{rrrrr}
        \addlinespace
        \toprule
              & lLT   & AT    & PT    & GM \\
        \midrule
        rLT   & 0.93  & \emph{0.04}  & \emph{0.03}  & \emph{<0.01} \\
        lLT   &       & 0.09  & 0.22  & \emph{<0.01} \\
        AT    &       &       & \emph{<0.01}  & 0.81 \\
        PT    &       &       &       & \emph{<0.01} \\
        \bottomrule
        \end{tabular}%
  \label{tab:chap5exp2_adc hotelling}%
\end{minipage}%
\\[1cm]
\begin{minipage}{\linewidth}
  \centering
  \footnotesize
  \tabcaption{Hotelling's-T$^2$ significance of pair-wise tract-specific differences for combinations of QSI parameters (confidence interval: 95\%)}
  \subfloat[Combined perpendicular QSI parameters (P0$_{xy}$+FWHM$_{xy})$]{
        \begin{tabular}{rrrrr}
        \addlinespace
        \toprule
              & lLT   & AT    & PT    & GM \\
        \midrule
        rLT   & 0.44  & 0.10  & 0.60  & \emph{<0.01} \\
        lLT   &       & 0.19  & 0.92  & \emph{0.01} \\
        AT    &       &       & 0.23  & 0.43 \\
        PT    &       &       &       & \emph{0.01} \\
        \bottomrule
        \end{tabular}%
      \label{tab:chap5exp2_qsix hotelling}%
  }\hspace{0.5cm}
  \subfloat[Combined parallel QSI parameters (P0$_{z}$+FWHM$_{z}$)]
  {
        \begin{tabular}{rrrrr}
        \addlinespace
        \toprule
              & lLT   & AT    & PT    & GM \\
        \midrule
        rLT   & 0.57  & 0.08  & \emph{0.02}  & \emph{0.04} \\
        lLT   &       & 0.22  & 0.32  & 0.19 \\
        AT    &       &       & \emph{<0.01}  & 0.63 \\
        PT    &       &       &       & \emph{0.01} \\
        \bottomrule
        \end{tabular}%
        \label{tab:chap5exp2_qsiz hotelling}%
  }\\
  \subfloat[Combined perpendicular and parallel QSI parameters (P0$_{xy}$+FWHM$_{xy}$+P0$_{z}$+FWHM$_{z}$)]
  {
        \begin{tabular}{rrrrr}
        \addlinespace
        \toprule
              & lLT   & AT    & PT    & GM \\
        \midrule
        rLT   & 0.71  & 0.22  & 0.17  & \emph{\emph{0.02}} \\
        lLT   &       & 0.45  & 0.69  & 0.08 \\
        AT    &       &       & \emph{0.03}  & 0.66 \\
        PT    &       &       &       &\emph{0.02} \\
        \bottomrule
        \end{tabular}%

    \label{tab:chap5exp2_qsiall hotelling}%
  }
  \label{tab:chap5exp2_qsi hotelling}%
\end{minipage}%
\\[2cm]
\paragraph{Correlation between ADC and QSI}
\label{par:chapter5 exp2 correlation}
Table~\ref{tab:chapter5 exp2 correlations} shows the Pearson coefficient and p-value for voxel-wise correlations between the investigated ADC and QSI metrics. We observe a strong correspondence (p<0.01) between ADC measurements and P0 and FWMH QSI metrics in X as well as FWHM in Z. P0$_z$ is the only parameter that does not correlate with any of the other metrics, suggesting additional information that is neither captured in the ADC$_z$ value nor captured with any of the XY measurements.%
\\[2cm]
\begin{minipage}{\linewidth}
 \tabcaption{Pearson-correlation coefficient and significance between all ADC and QSI metrics. P-values $<0.01$ are displayed as bold.}
 \centering
    \begin{tabular}{rrrrrrrr}
    \addlinespace
    \toprule
              &       & ADC$_{xy}$  & ADC$_{z}$  & P0$_{xy}$   & FWHM$_{xy}$   & P0$_{z}$   & FWHM$_{z}$ \\
    \midrule
    \multicolumn{1}{c}{\multirow{2}[0]{*}{ADC$_{xy}$}} & rho   & 1.00  & 0.43  & -0.15 & -0.25 & -0.01 & 0.15 \\
    \multicolumn{1}{c}{} & \textit{p} & \textit{} & \textbf{\textit{<0.01}} & \textbf{\textit{<0.01}} & \textbf{\textit{<0.01}} & \textit{0.60} & \textbf{\textit{<0.01}} \\
    \multicolumn{1}{c}{\multirow{2}[0]{*}{ADC$_{z}$}} & rho   & 0.43  & 1.00  & -0.46 & -0.30 & 0.00  & 0.21 \\
    \multicolumn{1}{c}{} & \textit{p} & \textbf{\textit{<0.01}} & \textit{} & \textbf{\textit{<0.01}} & \textbf{\textit{<0.01}} & \textit{0.85} & \textbf{\textit{<0.01}} \\
    \multicolumn{1}{c}{\multirow{2}[0]{*}{P0$_{xy}$}} & rho   & -0.15 & -0.46 & 1.00  & -0.05 & 0.01  & 0.16 \\
    \multicolumn{1}{c}{} & \textit{p} & \textbf{\textit{<0.01}} & \textbf{\textit{<0.01}} & \textit{} & \textbf{\textit{<0.01}} & \textit{0.55} & \textbf{\textit{<0.01}} \\
    \multicolumn{1}{c}{\multirow{2}[0]{*}{FWHM$_{xy}$}} & rho   & -0.25 & -0.30 & -0.05 & 1.00  & 0.00  & -0.80 \\
    \multicolumn{1}{c}{} & \textit{p} & \textbf{\textit{<0.01}} & \textbf{\textit{<0.01}} & \textbf{\textit{<0.01}} & \textit{} & \textit{0.92} & \textbf{\textit{<0.01}} \\
    \multicolumn{1}{c}{\multirow{2}[0]{*}{P0$_{z}$}} & rho   & -0.01 & 0.00  & 0.01  & 0.00  & 1.00  & 0.00 \\
    \multicolumn{1}{c}{} & \textit{p} & \textit{0.60} & \textit{0.85} & \textit{0.55} & \textit{0.92} & \textit{} & \textit{0.84} \\
    \multicolumn{1}{c}{\multirow{2}[0]{*}{FWHM$_{z}$}} & rho   & 0.15  & 0.21  & 0.16  & -0.80 & 0.00  & 1.00 \\
    \multicolumn{1}{c}{} & \textit{p} & \textbf{\textit{<0.01}} & \textbf{\textit{<0.01}} & \textbf{\textit{<0.01}} & \textbf{\textit{<0.01}} & \textit{0.84} & \textit{} \\
    \bottomrule
    \end{tabular}%  
  \label{tab:chapter5 exp2 correlations}
\end{minipage}%
\\[2cm]
\subsection{Conclusion}
\label{par:chapter5 exp2 correlation}
We have performed two experiments to investigate reproducibility of QSI metrics and their ability to discriminate individual WM and GM tracts. Further we compared the QSI performance with standard ADC analysis. For the first time, we also report QSI parameters measured parallel to the \gls{SC} long axis. Despite a suboptimal imaging setup of the initial experiment we found better intra- and inter-subject reproducibility in QSI compared to ADC in all investigates \glspl{ROI}. Furthermore, both QSI and ADC did discriminate GM and WM as well as between some WM \glspl{ROI}, although we didn't QSI metrics didn't increase significance of the found differences. Furthermore, we found that measurements in Z helped to distinguish \glspl{ROI} with more accuracy and complemented ADC and QSI values in XY direction. 


The encouraging initial results motivated a second study, in which we tackle some of the major limiting factors of the first Experiments, in particular low gradient strength and low spatial resolution. We introduced an improved imaging protocol and obtained data with higher quality to validate the findings in the first Experiment 1 in another cohort of 10 healthy subjects. We confirmed the general trends found in intra- and inter-subject reproducibility, although overall reproducibility was reduced in all metrics as an effect of the optimised imaging protocol. However, similar to the first experiment, we were able to distinguish WM and GM but only found significant differences between few WM regions. Furthermore, the significant WM regions were different between the two experiments  and hence could be a result of random effect due to the small sample size or ROI displacement rather than true microstructural differences. However, we were able to confirm in this experiment that ADC and QSI metrics in Z provides useful information about the microstructure parallel to the principle fibre direction. 

%\subsection{Future work}
%One of


