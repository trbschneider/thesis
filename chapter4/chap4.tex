%!TEX root = ../thesis.tex

\chapter{Fuzzy partial volume correction of average DTI metrics in the spinal cord}
As seen in the previous Chapter in \gls{SC} \gls{DTI}, a large proportion of voxels are usually affected by {\gls{PVA}} from surrounding {\gls{CSF}} due to the small size of the cord and the limited spatial resolution. While the effects of \gls{PVA} on brain \gls{DTI} have been studied extensively for over a decade, its effects on \gls{SC} \gls{DTI} are by far less well explored.

In the previous chapter we opted for a very conservative approach, excluding all voxels within the boundary of the \gls{SC}, which drastically reduces the number of effective voxels for further analysis. In this chapter we present an alternative approach to \gls{ROI} analysis of \gls{SC}-\gls{DTI} which aims to reduce the \gls{PVA} effect while it retains the information contained in boundary voxels.  
\section{Motivation}
\label{sec:chap4motivation}
In the cord, a large proportion of voxels are usually affected by {\gls{PVA}} from surrounding {\gls{CSF}} due to the small size of the cord and the limited spatial resolution. Water molecules in {\gls{CSF}} are less hindered than in nervous tissue, resulting in increased diffusivity measures and decreased anisotropy in {\gls{PVA}} voxels \citep{Alexander:2001,Pfefferbaum:2003, Vos:2011, Pasternak:2011}. This can lead to biased average measurements over the whole cord volume and could potentially conceal subtle disease effects. \Gls{PVA} corruption can be dealt with by using \gls{CSF}-suppressing pulse sequences such as \gls{FLAIR} \citep{Chou:2005, Papadakis:2002}. However \gls{FLAIR} has several disadvantages such as low \gls{SNR} and high motion sensitivity, and, moreover, is unsuitable for cardiac gating due to its long inversion recovery preparation. As a consequence, \gls{FLAIR}-\gls{DTI} is not a viable alternative in the spinal cord.

\gls{PVA} post-acquisition correction methods have been proposed that fit a combination of \gls{DTI} and \gls{CSF} compartments to the diffusion data \citet{Pierpaoli:2003, Pasternak:2009}. While these methods show promising results on brain DTI data, we found that they are not applicable to \gls{SC} data due to the much lower \gls{SNR}.

Therefore in common practice, {\gls{CSF}} affected voxels are excluded from analysis with a subjective and manual editing of the outlined {\gls{ROI}}. However, objectively deciding which voxels to exclude is difficult and might introduce an observer-specific error to the measurements. Furthermore retaining information while excluding those voxels can be problematic, particularly when the cord area is small and only few unaffected voxels exist, e.g. in patients with {\gls{SC}} atrophy. We introduce a novel partial volume correction method for average \gls{DTI} parameters that avoids the manual exclusion of {\gls{PVA}} affected voxels. Instead, we introduce a contribution weighting factor for each affected voxel that depends on its distance to the interface between {\gls{SC}} voxels and {\gls{CSF}}. We test our approach in healthy volunteers and patients with chronic \gls{SCI} and demonstrate that our method significantly reduces {\gls{PVA}} effects on mean \gls{DTI} indices.
\section{Subjects and data acquisition}
\label{sec:chap4data acquisition}
\subsection{Subjects}
The dataset we use in this study was acquired for a study of chronic \gls{SCI}. The data consists of nine male SCI subjects (mean age=45.7 yrs, SD=10.3, range=29--61; n=9) who fulfilled the following inclusion criteria: (1) Bilateral upper and lower limb impairment; (2) No head or brain lesion associated with the trauma leading to the injury; (3) No seizure, no medical or mental illness; (4) no \gls{MRI} contraindications. Furthermore the dataset includes ten age- and gender-matched right handed healthy subjects (mean age=38.8 yrs, SD=15.5, range=25--65,) without any history of neurological or psychiatric illness.

\subsection{Image acquisition}
\gls{DTI} was acquired on a 1.5T whole body Magnetom Sonata MRI scanner (Siemens Medical Systems, Erlangen, Germany) with a single shot echo planar imaging sequence employing the twice refocused spin-echo method for diffusion encoding \citep{Reese:2003}. Two axial datasets were collected using peripheral gating for reducing artifacts associated with cardiac induced gating motion \citep{Wheeler-Kingshott:2002}. The two datasets were also acquired with alternating phase encoding blip directions to remove susceptibility induced geometric distortions \citep{Andersson:2003}. Each dataset consisted of 68 images with a low b-value of $100 s/mm^{2}$ for the first 7 images and a high b-value of $1000 s/mm^{2}$ for the remaining 61 directions. The diffusion encoding gradient directions were distributed evenly on the surface of the unit sphere \citep{Jansons:2003}. The slice-to-slice repetition time was $180 ms$, the echo time was $90 ms$ and the excitation flip angle was $90�$. The image volumes consisted of 20 axial slices with thickness of 5$mm$ and an in-plane resolution of 1.5 $mm^2$, with no inter-slice gaps, acquisition matrix of $96 \times 96$, field of view of $144 \times 144 mm^2$, and bandwidth 1408 Hz/Pixel. The large {\gls{FOV}} was necessary to avoid wrap-around artifacts from surrounding shoulder and neck tissue.

Interleaved slice sampling was chosen to avoid cross talk between adjacent slices. The acquisition was pulse triggered and took approximately 20 mins. The two datasets with different phase encoded directions were combined into a single dataset with reduced susceptibility induced geometric distortions as described in \citep{Andersson:2003}. Then, all volumes in image space were sinc interpolated to a $192\times192$ image matrix, resulting in an in-plane resolution of $0.75mm^2$.

\subsection{Data analysis}
The diffusion tensor model was fitted to the interpolated data on a voxel-by-voxel basis using the freely available Camino toolkit \citep{Cook:2006}. Before further processing, all images were manually checked for remaining artifacts. The \gls{DTI} dataset of one control subject had to be excluded from further analysis due to distortion artifacts. From the estimated tensor, \gls{FA}, \gls{RD}, \gls{L1} and \gls{MD} maps were calculated for each subject. For each we also compute the mean $b=100s/mm^2$ and perform a semi-automatic spinal cord segmentation on the image using the active surface segmentation implemented in Jim6 \citep{Horsfield:2010}.
%%%%%%%%%%%%%%%%%%%%%%%%%%%%%%%%%%%%%%%%%%%%%%%%%%%%%%%%%%%%%%%
%
% PVA method
%
%%%%%%%%%%%%%%%%%%%%%%%%%%%%%%%%%%%%%%%%%%%%%%%%%%%%%%%%%%%%%%%
\section{Fuzzy partial volume correction method}
We propose a novel method that computes the average over the DTI metrics by using the morphology of the initial spinal cord segmentation. By definition, the boundary voxels of the \gls{SC} are most affected by the \gls{PVA} effect, while voxels in the center remain unaffected. We can exploit the simple outline of the \gls{SC} and compute the distance map of the initial segmentation using binary morphology operators \cite{Serra:1982}. This operation computes the minimal distance $d$ to the border in each voxel. An example of a distance map of the spinal cord segmentation is shown in Figure~\ref{fig:chap4distancemap}. 


A clear relationship between the normalised distance $\hat{d}=d/max(d)$ and \gls{DTI} parameters can be seen in both controls (Figure~\ref{fig:chap4scatterplotsdistance controls}) and patients (Figure~\ref{fig:chap4scatterplotsdistance patients}). These plots suggest a correlation between all DTI parameters and $d$ for voxels close to the boundary. Voxel with high $\hat{d}$ appear uncorrelated and therefore unaffected by \gls{PVA}. We observe decreased \gls{FA} and increased \gls{MD} and \gls{L1} and \gls{RD} for low $\hat{d}$ values compared to higher $\hat{d}$ values. It is important to note that the \gls{PVA} effect is not only observed in the immediate boundary of the segmented \gls{SC} but also affects voxels close to the boundary $\hat{d}<0.3$. A possible explanation is that the observed \gls{PVA} here originates not only from the the presence of multiple tissue types in a single voxel but factors in additional segmentation errors, movement and blurring due to a large point-spread function.

    \begin{figure}
    \centering
        \pgfimage[width=0.45\textwidth]{chapter4/figures/distancemap.png}
        \caption{Isolines of distance map of SC segmentation overlayed on FA map in one slice of one control subject.}
        \label{fig:chap4distancemap}
    \end{figure}
    

    
    \begin{figure}
        \centering
        \subfloat[Healthy controls]
        {
                \begin{minipage}[b]{\textwidth}
                \centering
                \pgfimage[width=0.37\textwidth]{chapter4/figures/v-wFA_c.pdf}
                \pgfimage[width=0.37\textwidth]{chapter4/figures/v-wMD_c.pdf}\\
                \pgfimage[width=0.37\textwidth]{chapter4/figures/v-wL1_c.pdf}
                \pgfimage[width=0.37\textwidth]{chapter4/figures/v-wRD_c.pdf}
                \end{minipage}
        }
        \\
        
        \subfloat[SCI patients]
        {
                \begin{minipage}[b]{\textwidth}
                \centering
                \pgfimage[width=0.37\textwidth]{chapter4/figures/v-wFA_p.pdf}
                \pgfimage[width=0.37\textwidth]{chapter4/figures/v-wMD_p.pdf}\\
                \pgfimage[width=0.37\textwidth]{chapter4/figures/v-wL1_p.pdf}
                \pgfimage[width=0.37\textwidth]{chapter4/figures/v-wRD_p.pdf}
                \end{minipage}
        }
        \caption{Scatterplots of normalised voxel distance $\hat{d}=d/max(d)$ against DTI parameters for controls and patient groups}
        \label{fig:chap4scatterplotsdistance}
    \end{figure}
%%%%%%%%%%%%%%%%%%%%%%%%%%%%%%%%%%%%%%%%%%%%%%%%%%%%%%%%%%%%%%%
%
% Weighting function
%
%%%%%%%%%%%%%%%%%%%%%%%%%%%%%%%%%%%%%%%%%%%%%%%%%%%%%%%%%%%%%%%
\subsection{Weighting function}
We aim to determine a weighting function that reflects the confidence in each voxel whether it belongs to the spinal cord tissue or not. We identified the criteria for a suitable candidate weighting function as follows:
\begin{enumerate}
  \item Voxels closer to boundary are assigned less confidence, i.e., are weighted less than voxels close to centre
  \item After a certain cut-off distance, voxels are assumed to be within spinal cord
  \item Boundary voxels are weighted less in larger spinal cord volumes than in small spinal cord volumes.
\end{enumerate}
As stated above, criteria 1 and 2 are inferred directly from the observations made in both control and patient groups in Figure~\ref{fig:chap4scatterplotsdistance}. Criteria 3 is included because of the relation between larger volume and increased \gls{PVA} as shown e.g. by \citet{Pasternak:2011}. We choose the weighting function $w(voxel)$ that fulfills all criteria above as:
\begin{equation}
	w(voxel) =\left\{
	\begin{array}{lll}
		d/\mbox{max}(d)&\mbox{ if } d\leq c\\
		1&\mbox{ otherwise }
	\end{array}
	\right.,	
\label{eq:chap4fuzzyweightingfunction}
\end{equation}

where $c$ is a given cutoff distance and $d$ is the distance of a voxel to the boundary as defined earlier. Figure \ref{fig:chap4weightingisolines illustration cartoon} gives a graphical representation of the chosen weighting function $w(voxel)$. We choose a linear weighting because the monotonous relationship between distance and \gls{PVA} in voxels with small $d$ as seen in Figure~\ref{fig:chap4scatterplotsdistance} while it is also easy to interpret and implement.

\begin{figure}[tbh]
  \begin{center}
    \pgfimage[width=7cm]{chapter4/figures/weighting-illustration.pdf}
  \end{center}
  \caption{1-d illustration of weighting function defined in Eq.~\ref{eq:chap4fuzzyweightingfunction}.}
  \label{fig:chap4weightingisolines illustration cartoon}
\end{figure}

\subsection{Data analysis}
To determine the appropriate cutoff distance $c$ we then apply the fuzzy weighted averages for the derived \gls{DTI} parameters and compare them with the unweighted average. After the $c$ is chosen, we compute the average differences between unweighted and fuzzy weighted whole \gls{ROI} measurements in controls and SCI patients for all \gls{DTI} parameters and also compute the unweighted and weighted histogram to compare the distribution of \gls{DTI} values for both methods and both groups.


To quantify the difference between our method and the standard average, we test the statistical significance of the difference using a pairwise two-tailed t-test for controls and patients groups independently. We also perform an unpaired two-tailed t-test on the differences between the two groups to investigate the influence of our method on the significance of group-wise changes between healthy subjects and chronic \gls{SCI} patients. All statistical tests assume a confidence interval of 95\%.
%%%%%%%%%%%%%%%%%%%%%%%%%%%%%%%%%%%%%%%%%%%%%%%%%%%%%%%%%%%%%%%
%
% Results
%
%%%%%%%%%%%%%%%%%%%%%%%%%%%%%%%%%%%%%%%%%%%%%%%%%%%%%%%%%%%%%%%
\section{Results}
\label{sec:chap4 results}
\subsection{Determining the cutoff distance}
Our definition of the fuzzy weighting function (see Equation \ref{eq:chap4fuzzyweightingfunction}) requires the cutoff parameter $c$ to be determined. This has to be done on the basis of the study specific protocol as image acquisition setup and postprocessing steps influence the dimensions of \gls{PVA}. The isolines for the different values of $c$ are illustrated in Figure \ref{fig:chap4weightingisolines illustration} in a slice of one control subject. In Figure \ref{fig:chap4cutoff_vs_DTImean} and Table~\ref{tab:chap4cutoff distance values} we present average \gls{DTI} metrics and standard deviations for different $c$ in controls and patients. We observe higher diffusivity values and decreased FA in both groups for $c\leq2$ compared to larger $c$. For $c\geq3$, the average metrics reach a plateau which can be clearly seen in Figure~\ref{fig:chap4cutoff_vs_DTImean}. Furthermore, the estimated average FA and diffusivities at $c\geq3$ in the control group agree with previously reported values in healthy human cervical spinal cord \citep{Wheeler-Kingshott:2002a,Ellingson:2007}. Based on these observations, we choose a cutoff value of $c=3$ for this experiment.

\begin{figure}
\centering
    \subfloat[standard average]
    {
        \pgfimage[width=0.19\textwidth]{chapter4/figures/cutoff_isolines_1.png}
    }
    \subfloat[$c=2$]
    {
        \pgfimage[width=0.19\textwidth]{chapter4/figures/cutoff_isolines_2.png}
    }
    \subfloat[$c=3$]
    {
        \pgfimage[width=0.19\textwidth]{chapter4/figures/cutoff_isolines_3.png}
    }
    \subfloat[$c=3$]
    {
        \pgfimage[width=0.19\textwidth]{chapter4/figures/cutoff_isolines_4.png}
    }
    \subfloat[$c=6$]
    {
        \pgfimage[width=0.19\textwidth]{chapter4/figures/cutoff_isolines_5.png}
    }
    \caption{Illustration of weighting isolines for different cutoff distances $c\in \{0,2,3,4,5\}$ overlayed on FA in one slice of one control.}
    \label{fig:chap4weightingisolines illustration}
\end{figure}

\begin{table}
    \centering
    \caption{Mean and standard deviation of DTI parameters for controls (con) and SCI patients (pat) with respect to chosen cutoff distance $c$. The column of the chosen cutoff distance $c=3$ is marked red. Statistical significant differences between healthy controls and SCI patients are marked with $^**p<0.01$, $^{*}p<0.05$.  }
    \begin{tabular}{rrrrrrrrrr}
    \addlinespace
    \toprule
          &       & \multicolumn{2}{c}{FA} & \multicolumn{2}{c}{MD} & \multicolumn{2}{c}{$\lambda_1$} & \multicolumn{2}{c}{RD} \\
          &       & \multicolumn{1}{c}{con} & \multicolumn{1}{c}{pat} & \multicolumn{1}{c}{con} & \multicolumn{1}{c}{pat} & \multicolumn{1}{c}{con} & \multicolumn{1}{c}{pat} & \multicolumn{1}{c}{con} & \multicolumn{1}{c}{pat} \\
    \midrule
    \multicolumn{1}{c}{\multirow{3}[0]{*}{c=0}} & \multicolumn{1}{c}{mean} & 0.52  & 0.42  & 1.23  & 1.30  & 1.95  & 1.89  & 0.87  & 1.00 \\
    \multicolumn{1}{c}{} & \multicolumn{1}{c}{std} & 0.03  & 0.04  & 0.11  & 0.18  & 0.14  & 0.25  & 0.10  & 0.15 \\
    \multicolumn{1}{c}{} & \multicolumn{1}{c}{\textbf{p}} & \multicolumn{2}{c}{\textbf{<0.01$^{**}$}} & \multicolumn{2}{c}{\textbf{0.33}} & \multicolumn{2}{c}{\textbf{0.57}} & \multicolumn{2}{c}{\textbf{\emph{0.04$^*$}}} \\
    \midrule
    \multicolumn{1}{c}{\multirow{3}[0]{*}{c=2}} & \multicolumn{1}{c}{mean} & 0.60  & 0.50  & 1.04  & 1.09  & 1.83  & 1.72  & 0.65  & 0.77 \\
    \multicolumn{1}{c}{} & \multicolumn{1}{c}{std} & 0.04  & 0.05  & 0.09  & 0.16  & 0.12  & 0.23  & 0.09  & 0.13 \\
    \multicolumn{1}{c}{} & \multicolumn{1}{c}{\textbf{p}} & \multicolumn{2}{c}{\textbf{\emph{<0.01$^{**}$}}} & \multicolumn{2}{c}{\textbf{0.42}} & \multicolumn{2}{c}{\textbf{0.24}} & \multicolumn{2}{c}{\textbf{\emph{0.03$^*$}}} \\
    \midrule
    \multicolumn{1}{c}{\multirow{3}[0]{*}{\textcolor{red}{c=3}}} & \multicolumn{1}{c}{mean} & 0.63  & 0.52  & 0.99  & 1.05  & 1.78  & 1.68  & 0.60  & 0.73 \\
    \multicolumn{1}{c}{} & \multicolumn{1}{c}{std} & 0.04  & 0.05  & 0.08  & 0.15  & 0.11  & 0.22  & 0.08  & 0.12 \\
    \multicolumn{1}{c}{} & \multicolumn{1}{c}{\textbf{p}} & \multicolumn{2}{c}{\textbf{\emph{<0.01$^{**}$}}} & \multicolumn{2}{c}{\textbf{0.34}} & \multicolumn{2}{c}{\textbf{0.22}} & \multicolumn{2}{c}{\textbf{\emph{0.02$^*$}}} \\
    \midrule
    \multicolumn{1}{c}{\multirow{3}[0]{*}{c=4}} & \multicolumn{1}{c}{mean} & 0.62  & 0.51  & 1.00  & 1.06  & 1.78  & 1.69  & 0.61  & 0.75 \\
    \multicolumn{1}{c}{} & \multicolumn{1}{c}{std} & 0.04  & 0.05  & 0.08  & 0.15  & 0.11  & 0.22  & 0.08  & 0.13 \\
    \multicolumn{1}{c}{} & \multicolumn{1}{c}{\textbf{p}} & \multicolumn{2}{c}{\textbf{\emph{<0.01$^{**}$}}} & \multicolumn{2}{c}{\textbf{0.28}} & \multicolumn{2}{c}{\textbf{0.27}} & \multicolumn{2}{c}{\textbf{\emph{0.01$^*$}}} \\
    \midrule
    \multicolumn{1}{c}{\multirow{3}[0]{*}{c=5}} & \multicolumn{1}{c}{mean} & 0.61  & 0.50  & 1.02  & 1.08  & 1.79  & 1.70  & 0.63  & 0.77 \\
    \multicolumn{1}{c}{} & \multicolumn{1}{c}{std} & 0.04  & 0.05  & 0.08  & 0.15  & 0.11  & 0.23  & 0.08  & 0.13 \\
    \multicolumn{1}{c}{} & \multicolumn{1}{c}{\textbf{p}} & \multicolumn{2}{c}{\textbf{\emph{<0.01$^{**}$}}} & \multicolumn{2}{c}{\textbf{0.28}} & \multicolumn{2}{c}{\textbf{0.30}} & \multicolumn{2}{c}{\textbf{\emph{0.01$^*$}}} \\
    \bottomrule
    \end{tabular}%
    \label{tab:chap4cutoff distance values}
\end{table}%

 
\begin{figure}
    \begin{center}
        %!TEX root = ../chap4.tex
\pgfplotsset{cutoff_vs_dti_barchart/.style={ybar, width=0.49\textwidth, height=0.49\textwidth, xtick={{1},{2},{3},{4},{5}}, xticklabels={no,{2},\textbf{3},{4},{5}}}, xlabel={cutoff distance [voxel]}, xmin=0.5, xmax=5.5}
\begin{tikzpicture}
\begin{axis}[cutoff_vs_dti_barchart, title=FA, ymax=0.72]
    \addplot+[error bars/.cd, y dir=both, y explicit] table[y=FA, y error=FAstd] {chapter4/figures/distance_vs_DTImean_c.dat};
    \addplot+[error bars/.cd, y dir=both, y explicit] table[y=FA, y error=FAstd] {chapter4/figures/distance_vs_DTImean_p.dat};
    \addplot+[only marks, mark color=black, black, fill=black, mark=text, text mark={**}] coordinates {(1,0.57) (2,0.68) (3,0.69) (4,0.69) (5,0.68)};    
\end{axis}
\end{tikzpicture}
\begin{tikzpicture}
\begin{axis}[cutoff_vs_dti_barchart, title=MD ($\times 10^9 mm^2/s$)]
    \addplot+[error bars/.cd, y dir=both, y explicit] table[y=MD, y error=MDstd] {chapter4/figures/distance_vs_DTImean_c.dat};
    \addplot+[error bars/.cd, y dir=both, y explicit] table[y=MD, y error=MDstd] {chapter4/figures/distance_vs_DTImean_p.dat};
    \legend{controls, patients}
\end{axis}
\end{tikzpicture}\\
\begin{tikzpicture}
\begin{axis}[cutoff_vs_dti_barchart, title=L1 ($\times 10^9 mm^2/s$)]
    \addplot+[error bars/.cd, y dir=both, y explicit] table[y=L1, y error=L1std] {chapter4/figures/distance_vs_DTImean_c.dat};
    \addplot+[error bars/.cd, y dir=both, y explicit] table[y=L1, y error=L1std] {chapter4/figures/distance_vs_DTImean_p.dat};
\end{axis}
\end{tikzpicture}
\begin{tikzpicture}
\begin{axis}[cutoff_vs_dti_barchart, title=RD ($\times 10^9 mm^2/s$)]
    \addplot+[error bars/.cd, y dir=both, y explicit] table[y=RD, y error=RDstd] {chapter4/figures/distance_vs_DTImean_c.dat};
    \addplot+[error bars/.cd, y dir=both, y explicit] table[y=RD, y error=RDstd] {chapter4/figures/distance_vs_DTImean_p.dat};
    \addplot+[only marks, mark color=black, black, fill=black, mark=text, text mark={*}] coordinates {(1,1.2) (2,0.95) (3,0.94) (4,0.95) (5,0.97)};
\end{axis}
\end{tikzpicture}
    
    \end{center}
    
    
    \begin{flushleft}
          $^*p<0.01$, $^{**}p<0.05$
    \end{flushleft}
    \caption{Weighted mean and standard deviation of DTI parameters over all controls/patients with respect to chosen cutoff distance. The chosen cutoff distance $c=3$ is is marked bold in all figures.}

    \label{fig:chap4cutoff_vs_DTImean}
\end{figure}


\subsection{Comparison of unweighted and fuzzy weighted average DTI metrics}
Table~\ref{tab:experiment2_average change} summarises the relative change in DTI metrics between the unweighted and \gls{PVA}-weighted average \gls{DTI} metrics. In both control and patient groups we observe significant differences between the unweighted and \gls{PVA}-weighted averages. In \gls{RD} we see the largest decreased of approx. 30\% in both patients and controls. \gls{FA} is increased by 22\% and \gls{MD} is reduced by 19\% in both groups. We also find moderate but significant decrease in \gls{L1} by \~10\%. Furthermore we observe a reduction in the standard deviation of the diffusivities between 10\% in \gls{L1} and 25\% in \gls{MD} and \gls{RD}. In contrast, the variability in \gls{FA} increases by more than 30\% on average in both controls and patients compared to the unweighted average. However, the absolute standard deviation was low in both unweighted and fuzzy corrected averages (see Table~\ref{tab:chap4cutoff distance values}).

Table~\ref{tab:chap4cutoff distance values} lists the statistical significance of the group-wise differences between controls and SCI patients for the unweighted whole \gls{SC} average and the fuzzy weighted averages for all \gls{DTI} parameters. \gls{FA} are significantly different between both groups (p<0.01) independently of the analysis method. \gls{L1} and \gls{MD} values show no significant differences (all p>0.05). The \gls{PVA} correction had most effect on the group-wise difference of \gls{RD}: significance was decreased from $p=0.04$ between unweighted averages to $p=0.02$ in corrected average with chosen cutoff parameter $(c=3)$.


Figure~\ref{fig:chap4histograms DTI metrics} presents histograms of all measured \gls{DTI} parameters over all voxels of all control subjects and patients respectively. In each figure we present the normalised unweighted and fuzzy weighted histogram. In \gls{FA} we see clear peaks at 0.7 (controls) and 0.6 (patients) in the fuzzy weighted histogram, while \gls{FA} values in the unweighted histogram are skewed towards low values both patients and controls show high peaks at 0.2. Peak position in the histograms of the diffusivity parameters are similar in unweighted and fuzzy weighted histograms, but the unweighted histograms show broader peaks and the distributions are generally more skewed towards high diffusion coefficients.
    \begin{table}
    \centering
    \caption{Averaged relative change of $\Delta$mean and $\Delta$standard deviation between uncorrected and PVA-corrected DTI measurements over all subjects.}
    \begin{tabular}{l >{\raggedleft\arraybackslash}p{1.5cm} >{\raggedleft\arraybackslash}p{1.5cm} >{\raggedleft\arraybackslash}p{1.5cm} >{\raggedleft\arraybackslash} p{1.5cm}}
        \toprule
        & \multicolumn{2}{c}{controls} & \multicolumn{2}{c}{patients}\\
        & \centering $\Delta$  mean (\%) & \centering\arraybackslash $\Delta$ std (\%) & \centering $\Delta$  mean (\%) & \centering\arraybackslash $\Delta$ std (\%)\\
        \cmidrule(l){2-2}\cmidrule(l){3-3}\cmidrule(l){4-4}\cmidrule(r){5-5}
        FA    & +21.6$^*$  & +30.0 & +21.6  & +37.2 \\
        MD    & -19.2$^*$  & -25.8 & -19.4  & -19.2 \\
        $\lambda_1$    & -8.6$^*$  & -26.6  & -11.3 & -12.4 \\
        RD    & -31.0$^*$  & -15.2 & -27.0  & -19.8 \\
        \bottomrule
        \multicolumn{5}{l}{$^*$Significance p<0.01}\\
    \end{tabular}
    \label{tab:experiment2_average change}

    \end{table}
    \begin{figure}
        \centering
        \pgfplotsset{hist_dti/.style={const plot, opacity=100}}
\pgfplotsset{hist_dti_controls fuz/.style={blue, thick, fill=blue}}
\pgfplotsset{hist_dti_controls unw/.style={blue, thick, fill=blue!10!white, opacity=60}}
\pgfplotsset{hist_dti_patients fuz/.style={red, thick,fill=red, opacity=50}}
\pgfplotsset{hist_dti_patients unw/.style={red, thick,fill=red!10!white, opacity=60}}
\pgfplotsset{hist_dti_axis/.style={ymin=0, xmin=0, legend style={area legend, font=\footnotesize, anchor=north, at={(0.5,-0.2)}, legend columns=-1}, width=0.49\textwidth, height=0.2\textheight}}

\begin{tikzpicture}[scale=0.9]
    \begin{axis}[title=FA controls, hist_dti_axis]
         \addplot[hist_dti, hist_dti_controls fuz] table[x=bins, y=FA] {chapter4/figures/hist_FA_c.txt} \closedcycle;
         \addplot[hist_dti, hist_dti_controls unw] table[x=bins, y=FA] {chapter4/figures/hist_FA_c_unw.txt}\closedcycle;
         %\legend{weighted ($c=3$), unweighted};
    \end{axis}
\end{tikzpicture}
\begin{tikzpicture}[scale=0.9]
    \begin{axis}[title=FA patients, hist_dti_axis]
         \addplot[hist_dti, hist_dti_patients fuz] table[x=bins, y=FA] {chapter4/figures/hist_FA_p.txt} \closedcycle;
         \addplot[hist_dti, hist_dti_patients unw] table[x=bins, y=FA] {chapter4/figures/hist_FA_p_unw.txt}\closedcycle;
         %\legend{weighted ($c=3$), unweighted};
    \end{axis}
\end{tikzpicture}\\
\begin{tikzpicture}[scale=0.9]
    \begin{axis}[title={MD controls $[\mu m^2/ms]$}, hist_dti_axis]
         \addplot[hist_dti, hist_dti_controls fuz] table[x=bins, y=MD] {chapter4/figures/hist_MD_c.txt} \closedcycle;
         \addplot[hist_dti, hist_dti_controls unw] table[x=bins, y=MD] {chapter4/figures/hist_MD_c_unw.txt}\closedcycle;
         %\legend{weighted ($c=3$), unweighted};
    \end{axis}
\end{tikzpicture}
\begin{tikzpicture}[scale=0.9]
    \begin{axis}[title={MD patients $[\mu m^2/ms]$}, hist_dti_axis]
         \addplot[hist_dti, hist_dti_patients fuz] table[x=bins, y=MD] {chapter4/figures/hist_MD_p.txt} \closedcycle;
         \addplot[hist_dti, hist_dti_patients unw] table[x=bins, y=MD] {chapter4/figures/hist_MD_p_unw.txt}\closedcycle;
%         \legend{weighted ($c=3$), unweighted};
    \end{axis}
\end{tikzpicture}\\
\begin{tikzpicture}[scale=0.9]
    \begin{axis}[title={$\lambda_1$ controls $[\mu m^2/ms]$}, hist_dti_axis]
         \addplot[hist_dti, hist_dti_controls fuz] table[x=bins, y=L1] {chapter4/figures/hist_L1_c.txt} \closedcycle;
         \addplot[hist_dti, hist_dti_controls unw] table[x=bins, y=L1] {chapter4/figures/hist_L1_c_unw.txt}\closedcycle;
         %\legend{weighted ($c=3$), unweighted};
    \end{axis}
\end{tikzpicture}
\begin{tikzpicture}[scale=0.9]
    \begin{axis}[title={$\lambda_1$ patients $[\mu m^2/ms]$}, hist_dti_axis]
         \addplot[hist_dti, hist_dti_patients fuz] table[x=bins, y=L1] {chapter4/figures/hist_L1_p.txt} \closedcycle;
         \addplot[hist_dti, hist_dti_patients unw] table[x=bins, y=L1] {chapter4/figures/hist_L1_p_unw.txt}\closedcycle;
%         \legend{weighted ($c=3$), unweighted};
    \end{axis}
\end{tikzpicture}\\
\begin{tikzpicture}[scale=0.9]
    \begin{axis}[title={RD controls $[\mu m^2/ms]$}, hist_dti_axis]
         \addplot[hist_dti, hist_dti_controls fuz] table[x=bins, y=RD] {chapter4/figures/hist_RD_c.txt} \closedcycle;
         \addplot[hist_dti, hist_dti_controls unw] table[x=bins, y=RD] {chapter4/figures/hist_RD_c_unw.txt}\closedcycle;
         \legend{weighted ($c=3$), unweighted};
    \end{axis}
\end{tikzpicture}
\begin{tikzpicture}[scale=0.9]
    \begin{axis}[title={RD patients $[\mu m^2/ms]$}, hist_dti_axis]
         \addplot[hist_dti, hist_dti_patients fuz] table[x=bins, y=RD] {chapter4/figures/hist_RD_p.txt} \closedcycle;
         \addplot[hist_dti, hist_dti_patients unw] table[x=bins, y=RD] {chapter4/figures/hist_RD_p_unw.txt}\closedcycle;
         \legend{weighted ($c=3$), unweighted};
    \end{axis}
\end{tikzpicture}

        \caption{Relative weighted and unweighted histogram of all DTI parameters for pooled SC voxels of controls (blue) and patient(red) groups.}
        \label{fig:chap4histograms DTI metrics}
    \end{figure}
\section{Discussion}
 We demonstrate the effect of \gls{PVA} using \gls{DTI} data of a cohort of 10 controls and 9 chronic \gls{SCI} patients. We show that \gls{DTI} metrics in the vicinity of the \gls{CSF}/\gls{SC} interface are affected and that there is a monotonous relationship between \gls{DTI} metrics in a voxel and its distance to the border of the \gls{SC} segmentation in our dataset. The significantly lower values of \gls{FA} and higher diffusivities \gls{MD}, \gls{L1} and \gls{RD} suggest that these voxels suffer most likely from \gls{CSF} contribution. These differences are expected from simulations as in \citet{Alexander:2001, Pasternak:2011} and have been shown in brain \gls{DTI} parameters that suffer from \gls{CSF} contribution \citep{Pfefferbaum:2002, Vos:2011}.
\paragraph{Choice of cutoff value}
We choose the cutoff parameters $c$ depending on the average parameter over all \gls{DTI} metrics in our control group. We chose the value that achieves a stable plateau in all \gls{DTI} parameters, assuming that this reflects the elimination of \gls{CSF} contribution. Although value of $c=3$ worked best in this study, we expect the optimal choice of $c$ to be highly dependent on study specific parameters such as the choice of the pulse sequence, additional pre-processing steps and also the accuracy of the initial \gls{SC} segmentation.

\paragraph{Effect of \gls{PVA} correction}
The weighted distributions in Figure~\ref{fig:chap4histograms DTI metrics} show that our method corrects the \gls{CSF} bias in the measurements by reducing the artificial tails of the parameter distribution. Quantitatively, we observe a large reduction in \gls{RD}, \gls{MD} and \gls{FA} parameters and smaller changes in \gls{L1}. In agreement with earlier studies of the \gls{PVA} effect in the brain, \gls{RD} is reduced the most, while \gls{AD} is the least affected parameter. However, in contrast to those brain studies, we also observe a large change in \gls{FA}. A possible explanation might be that the CSF contaminated voxels with low FA values have a bigger effect on the FA than the in the brain due to much the smaller volume of the \gls{SC} and therefore higher number of near-border voxels.

\paragraph{Group wise differences}
We investigated whether our method affects group wise statistics between patients and controls in our study. \gls{FA} and \gls{RD} are both significantly diminished in uncorrected and corrected metrics in \gls{SCI} patients compared to controls. However, the \gls{PVA} correction increased the confidence of our results particularly in the \gls{RD} metric, which may help to detect more subtle changes e.g. in Multiple Sclerosis. 

\section{Conclusion}
We propose a novel fuzzy partial volume correction method that removes \gls{CSF} contribution effects in measurements of \gls{DTI} parameters over the whole spinal cord volume. We avoid fully excluding all potentially \gls{CSF} contaminated voxels, and introduce a weighting factor that is dependent on the size of the cord and therefore accounts for the variability in number of white matter voxels. We show that our method produces reliable \gls{DTI} metrics that agree with previously measured values in the cord more reliable measurements. We demonstrate that this method increases significance of group differences between \gls{SCI} patients, i.e. can potentially increase the statistical power of larger clinical studies.

\section{Limitations and future work} 
In this study we attribute changes in \gls{DTI} parameters exclusively to the \gls{PVA} effect. This is clearly an oversimplicfication as several other factors such as field inhomogenities, field strengths may have an impact on the observed effect. The influence of the \gls{PVA} will differ across different scanners/coils/centres. As a consequence, the cutoff parameter we found optimal for our study will reflect the influence of all these parameters to some extend, so care has to be taken when using this method to compare datasets that come from different scanners or are acquired with different protocols.


Further work is needed to validate this method against other \gls{PVA} correction methods. Although validation is challenging due to a lack of ground truth value, in future work we might compare our corrected results with with fully \gls{CSF} suppressed measurements, e.g., by using the FLAIR DTI technique \citep{Chou:2005, Papadakis:2002}, although this technique also suffers from the lack of cardiac-gating and lower SNR.   

It is important to point out that in this experiment we don't correct for partial volume effects between grey and white matter inside the cord. This is mainly because the lack of contrast between gray and white matter on the low-b value images does not allow gray and white matter segmentation.  Furthermore, our approach uses only macroscopic morphological \emph{a-priori} knowledge the spinal cord. The method is likely to fail if there is no clear relation between the morphology, i.e., voxel distance to border and present \gls{PVA} effect. Future extension of this method could combine the morphologic approach with diffusion models that account for \gls{CSF} contribution in the raw \gls{DTI} signal, e.g., as in \citet{Pierpaoli:2001, Pasternak:2008}. By using the combination of macroscopic and microscopic information we can potentially overcome the limitation of the modeling approach in the low \gls{SNR} regime in the spinal cord.



\begin{subappendices}
\section*{Application to clinical study}
Spinal cord injury produces degenerative changes in descending and ascending pathways that manifest in morphometric changes in spinal and cranial white and grey matter in humans \citep{Jurkiewicz:2006,Ellingson:2008a,Wrigley:2009}. However little is known whether in the chronic phase of injury ongoing microanatomical spinal changes (i.e. axonal degeneration and progressive demyelination) contribute to disability and macroanatomic changes such as atrophy.


Axonal degeneration, although may not be visible on conventional \gls{MRI} scans, alters the diffusion parameters in the cord \citep{Pierpaoli:2001}. In particularly \gls{FA} has been reported to be a marker of both axonal count \citep{Gouw:2001} and myelin content \citep{Schmierer:2007} and \gls{L1} and \gls{RD} are thought to reflect the integrity of axons and myelin in the injured spinal cord, respectively \citep{Ou:2009,Song:2002,Song:2005,Zhang:2009}. Importantly, in the acute phase of injury, \gls{L1} correlates with spared axons in a mouse model of \gls{SCI}. Moreover, both AD and RD predict functional recovery \citep{Kim:2006, Freund:2010a}. Thus, \gls{L1} and \gls{RD} can disclose additional information on axonal as well as myelin integrity to standard \gls{FA} and \gls{MD} metrics.


We demonstrated previously that spinal atrophy is related to central microanatomical changes in cranial regions containing the CST and reduced axonal integrity relates to cortical motor reorganization \citep{Freund:2011}. However, three key questions still remain unresolved: (1) Do the directional diffusivities (i.e. \gls{L1}/gls{RD}) disclose additional information on the pathological processes occurring rostral to the site of injury? (2) Do microanatomical changes correlate with macroanatomical changes? (3) Do micronanatomical changes relate to clinical disability?

In this paper we hypothesize that the directional diffusivities reveal insights into the degenerative mechanisms and that \gls{DTI} metrics provide clinically relevant information in the sequel of \gls{SCI}.

\subsection{Clinical assessment}
All participants underwent \gls{MRI} scans, and were clinically assessed bilaterally on the \gls{9HPT} \citep{Goodkin:1988}, as well as on \gls{MVC} and \gls{ARAT} with their dominant hand. The reciprocal of the average of two trials for each hand of the \GLS{9HPT} and the average of two trials of the \gls{MVC} were calculated. Three \gls{SCI} subjects were unable to perform the \GLS{9HPT} with their non-dominant hand and were scored with the maximum time allowed for the \GLS{9HPT} (300 sec) \citep{Hoogervorst:2002}. To assess differences in motor performance between \gls{SCI} subjects and controls, a two-sample t-test was used. A p-value <0.05 was considered significant.

\subsection{Methods}
\paragraph{Structural MRI}
In addition to the \gls{DTI} scans described above, we acquire \gls{T1}  {3D-MDEFT} structural images of the whole brain, brainstem and cervical cord (down to C5) \citep{Deichmann:2004} on a 1.5T whole body Magnetom Sonata MRI scanner (Siemens Medical Systems, Erlangen, Germany). The scan parameters were: isotropic 1 mm3 resolution, field of view: 256x256 mm2, matrix size 256x256, 176 sagittal partitions, TR=12.24 ms, TE=3.56 ms, TI=530 ms, flip angle 23�, fat saturation, bandwidth 106Hz/Pixel. The acquisition time was 14 minutes. \gls{SCA} was measured using a semi-automated segmentation method \citep{Freund:2010,Losseff:1996}.
\paragraph{Tract-specific regions of interests}
In each participant, four \glspl{ROI} were manually drawn on the average low-diffusion weighted image between C1 and C3 similar to those used in (Freund et al., 2010c). The four \glspl{ROI} comprised the left and right \gls{CST} running in the lateral columns and sensory tracts in the anterior and posterior columns \ref{}. The mean value of each diffusion parameter within each \gls{ROI} was calculated.

\paragraph{Statistical analysis}
We used the two-sample t-test to investigate differences in each \gls{DTI} metrics and \gls{SCA} between \gls{SCI} subjects and controls. To identify independent associations between morphometric measures and clinical measures with the diffusion metrics in \gls{SCI} subjects, two multiple linear regression models were set up. For the morphometric measure, the DTI metrics were used as the dependent and SCA as the independent, adjusting for age. For the clinical measures, \GLS{9HPT}, \gls{ARAT}, \gls{MVC}, ASIA upper and lower limb motor score were used, in turn, as dependent variable and the diffusion metrics were used as covariates, together with the corresponding SCA and age. Stata 9.2 (www.stata.com) was used. Results associated with p<0.05 are reported.

\subsection{Results}

\paragraph{Clinical data}
Nine male \gls{SCI} subjects (mean period post \gls{SCI} was 14.8 years (SD 7.2, range 7-30) had lesions of the cervical cord between C5-C8 \ref{tab:chap4 clinical scores SCI}. \gls{SCI} subjects were quadriplegic as reflected by reduced ASIA motor and sensory scores of upper and lower limb. \gls{SCI} subjects were also impaired on the \gls{ARAT} (mean score of 38.7 of max 57) (Table 1). \gls{SCI} subjects were impaired on the dominant hand \GLS{9HPT} score [98.28 sec (SD 84.93) vs. 17.04 sec (SD 1.56), p=0.001], non-dominant hand \GLS{9HPT} score [141.22 sec (SD 121.98) vs. 18.14 sec (SD 1.35], p=0.001) and grip strength [0.13 mv (SD 0.1) vs. 0.49 mV (SD 0.26), p<0.001] when compared to controls (Table~\ref{tab:chap4 clinical scores SCI})
\begin{table}[htbp]
  \centering
  \caption{Individual clinical and behavioural data for the SCI subjects with means}
    \begin{tabular}{llp{0.15\textwidth}p{0.1\textwidth}p{0.1\textwidth}p{0.06\textwidth}p{0.06\textwidth}p{0.06\textwidth}p{0.06\textwidth}ll}
    \addlinespace
    \toprule
     &   &   &  &  & \multicolumn{2}{c}{motor score} & \multicolumn{2}{c}{9HPT} &  &  \\
    Subject & Age & Aetiology of the injury & Time since injury (years) & Level of motor impairment/ASIA & Upper limb & Lower limb & dh & ndh & MVC & ARAT \\
    \midrule
    1     & 43    & fracture & 14    & C6/D  & 25    & 30.5  & 68    & 54.35 & 0.22  & 36 \\
    2     & 29    & fracture & 9     & C6/B  & 19    & 0     & 52.6  & 59.2  & 0.05  & 42 \\
    3     & 44    & fracture & 7     & C7/C  & 14    & 19    & 56.75 & 118.4 & 0.25  & 57 \\
    4     & 35    & fracture & 14    & C5/A  & 18.5  & 0     & 190.5 & 300   & 0.02  & 26 \\
    5     & 61    & fracture & 19    & C6/A  & 34    & 0     & 68.3  & 76.5  & 0.05  & 26.5 \\
    6     & 40    & disc prolapse & 19    & C5/C  & 20.5  & 18.5  & 283   & 300   & 0.01  & 26 \\
    7     & 53    & fracture & 7     & C8/D  & 43.3  & 48    & 38.55 & 42.45 & 0.25  & 53 \\
    8     & 56    & fracture & 15    & C5/D  & 23.5  & 34.5  & 105   & 300   & 0.11  & 25 \\
    9     & 50    & fracture & 30    & C5/D  & 25    & 22    & 21.8  & 20.1  & 0.22  & 57 \\
          &       &       &       &       &       &       &       &       &       &  \\
    Mean  & 45.7  &       & 14.9  &       & 24.75 & 19.16 & 98.3  & 141.2 & 0.13  & 38.72 \\
    SD    & 9.7   &       & 6.8   &       & 8.4   & 16    & 84.94 & 121.98 & 0.1   & 13.13 \\
    \bottomrule
    \end{tabular}%
    \raggedright


    ASIA= American Spinal Injury Association impairment scale; dh=dominant hand; ndh= non-dominant hand; \GLS{9HPT}= Nine Hole Peg Test; \gls{MVC}= maximum voluntary contraction, \gls{ARAT}= Arm Research Arm Test; SD= standard deviation
  \label{tab:chap4 clinical scores SCI}%
\end{table}%

\paragraph{Spinal atrophy}
Cord area was reduced by more than 30\% in \gls{SCI} subjects when compared to controls [52.4 mm2 (SD 7.2) vs. 79.82 mm2 (SD 8.3), p<0.001] (Table 1).

\paragraph{Differences in \gls{DTI} metrics between \gls{SCI} subjects and controls}
Compared to controls, \gls{SCI} subjects showed lower FA in all \glspl{ROI} and higher MD in the anterior column of the spinal cord. Turning to the directional diffusivities, we found a higher \gls{RD} in the left \gls{CST}-\gls{ROI}, anterior columns and cross-section of the cervical cord and lower \gls{L1} in the right \gls{CST}-\gls{ROI} (Table~\ref{tab:chap4 multiple roi DTI}).


\begin{table}[htbp]
  \centering
  \caption{Values of the diffusion metrics obtained in the cervical spinal cord in \glspl{ROI} in the left and right lateral \gls{CST}, anterior and posterior columns and cross sectional cord area, in \gls{SCI} subjects and controls.}
    \begin{tabular}{rrrr}
    \addlinespace
    \toprule
          & 9 SCI subjects & 10 Controls &  \\
    \midrule
    \multirow{2}[0]{*}{FA} & Mean  & Mean  & \multirow{2}[0]{*}{P-value} \\
          & (SD)  & (SD)  &  \\
    L-CST & 0.63 (0.08) & 0.76 (0.05) & <0.0001 \\
    R-CST & 0.67 (0.07) & 0.75 (0.04) & 0.008 \\
    PC    & 0.37 (0.06) & 0.5 (0.09) & 0.004 \\
    AC    & 0.53 (0.1) & 0.73 (0.07) & <0.0001 \\
    SCA   & 0.52 (0.06) & 0.63 (0.04) & <0.0001 \\
    \multicolumn{4}{r}{RD (mm2/sx10-3)} \\
    L-CST & 0.35 (0.06) & 0.25 (0.08) & 0.047 \\
    R-CST & 0.48 (0.18) & 0.35 (0.06) & 0.35 \\
    PC    & 0.91 (0.13) & 0.75 (0.19) & 0.063 \\
    AC    & 0.71 (0.23) & 0.42 (0.09) & 0.002 \\
    SCA   & 0.73 (0.1) & 0.59 (0.08) & 0.009 \\
    \multicolumn{4}{r}{AD (mm2/sx10-3)} \\
    L-CST & 1.45 (0.33) & 1.64 (0.18) & 0.13 \\
    R-CST & 1.43 (0.23) & 1.67 (0.12) & 0.012 \\
    PC    & 1.81 (0.24) & 1.88 (0.24) & 0.41 \\
    AC    & 1.76 (0.31) & 1.79 (0.18) & 0.75 \\
    SCA   & 1.67 (0.23) & 1.77 (0.12) & 0.26 \\
    \multicolumn{4}{r}{MD (mm2/sx10-3)} \\
    L-CST & 0.81 (0.22) & 0.78 (0.08) & 0.76 \\
    R-CST & 0.75 (0.15) & 0.80 (0.063) & 0.32 \\
    PC    & 1.17 (.014) & 1.09 (0.17) & 0.29 \\
    AC    & 1.06 (0.24) & 0.88 (0.08) & 0.039 \\
    SCA   & 1.05 (0.16) & 0.98 (0.08) & 0.28 \\
    \bottomrule
    \end{tabular}%

    Note: L=left, R=right, \gls{CST}=cortico-spinal tract, PC=posterior columns, AC=anterior columns. RD= radial diffusitivity, AD=axial diffusitivity, FA= fractional anisotropy, MD= mean diffusitivity, SCA=cross sectional cord area
  \label{tab:chap4 multiple roi DTI}%
\end{table}%



\paragraph{Associations between FA and spinal atrophy}
In \gls{SCI} subjects, lower \gls{FA} of the cross-section of the cervical cord was negatively associated with lower \gls{SCA} (coef. 0.007, p=0.002, 95\% confidence interval (CI) 0.004, 0.011) independently of age. In other words, greater spinal atrophy is associated with greater changes to the tissue architecture.


\paragraph{Relationship between FA and clinical measures of motor and sensory function}

\gls{FA} of the right \gls{CST}-\gls{ROI} was associated with measures of manual dexterity and tactile sensation independently of SCA and age. In particularly, \gls{FA} of the right \gls{CST}-\gls{ROI} was associated with dominant hand \GLS{9HPT} score (coef. 0.16, p=0.005, CI 0.069, .0245) and non-dominant hand \GLS{9HPT} score (coef. 0.18, p=0.007, CI 0.07, 0.289). \gls{FA} in the posterior columns correlated with pin prick score (coef. -471.5, p=0.011, CI -792.64, -150.37) and light touch score (coef. -402.55, p=0.009, CI -664.01, -141.09) (Fig 2). In short, \gls{FA} in cervical regions rostral to the site of initial trauma predict motor and sensory impairment independently of spinal atrophy. No significant correlation was detected with MD, RD and AD.

\subsection{Discussion}

This study establishes independent associations between clinical impairment and cervical cord atrophy with disintegration of the tissue architecture rostral to the site in subjects of a chronic cervical injury. Besides these associations, we observed trauma induced quantitative changes of the standard diffusion metrics reflected by reduced \gls{FA} and increased MD. Importantly, we complement these findings as we demonstrate for the first time increased RD and reduced AD in SI subjects when compared with controls. Changes in the directional diffusivities (AD/RD) suggest that both axonal degeneration and demyelination in the chronically injured cervical cord are ongoing mechanisms. In view of the clinical correlations we suggest that pathological damage detected by \gls{DTI} may be a significant factor contributing to disability in chronic \gls{SCI} subjects.

\paragraph{Microanatomical changes in the injured chronic cervical cord}
Trauma results in macroanatomical changes that manifest in spinal and cortical atrophy \citep{Freund:2010,Jurkiewicz:2006,Wrigley:2009}. The pathological processes underlying these changes in man are not resolved and may be results of multiple microanatomical changes such as axonal disintegration \citep{Pierpaoli:2001, Buss:2003}, progressive demyelination \citep{Buss:2003, Totoiu:2005, Waxman:1989}, loss of large diameter axons \citep{Blight:1986} or damage to the spinal grey matter \citep{Kakulas:1987}. In agreement with other studies investigating \gls{DTI} metrics in the injured spinal cord \citep{Ellingson:2008, Shanmuganathan:2008} \gls{FA} was decreased in all \glspl{ROI} and \gls{MD} increased in the anterior column. The directional diffusivities we provide evidence that \gls{RD} and \gls{L1} also changed following \gls{SCI}. In particular, \gls{RD} was higher in the left \gls{CST}-\gls{ROI}, anterior columns and cross-section of the cervical cord while \gls{L1} was lower in the right \gls{CST}-\gls{ROI}. \gls{RD} and \gls{L1} have the potential to provide information on the pathological processes in addition to that derived from the standard diffusion metrics in various animal models of spinal involvement, e.g., \citet{Budde:2008,Feng:2009,Kim:2006,Kim:2009,Song:2002, Zhang:2009}.  Thus, these alterations of the axonal architecture are suggestive of both axonal degeneration and progressive demyelination at sites rostral to the trauma of the cervical cord.

\paragraph{Association between spinal atrophy and reduced FA}
Moving beyond the group effects we explored the relationship between spinal atrophy and measures of axonal microstructure rostral to the site of initial trauma. Importantly, we report a clear negative relationship between cord area and \gls{FA}. Thus we show that both \gls{FA} and spinal atrophy, which are correlated, may be indicators of axonal loss. However, spinal atrophy may be more directly linked to axonal loss as \gls{FA} is sensitive to any geometric tissue alterations.

Relationship between spinal changes in microstructure and clinical disability
The interruption of information flow between the brain and spinal cord is largely responsible for the ensuing clinical impairment \citep{Dietz:2006}. In healthy individuals, the \gls{CST} carries important information for the movement control of manual dexterity \citep{Lemon:2008}. Pathways that mediate afferent information flow are located in the posterior columns and they carry information for two-point sensation, pressure, and vibration \citep{TODO}. Previously, we have demonstrated associations between reduced cord area (i.e. tissue loss) and upper limb motor function \citep{Freund:2010a}. Here we find clear associations between reduced \gls{FA} measured in \glspl{ROI} that correspond to the lateral \gls{CST} and the ascending pathways in the posterior columns and impaired motor and sensory function in \gls{SCI} subjects, respectively.
In particularly, reduced \gls{FA} in the \gls{ROI} placed over the right \gls{CST} was independently associated with slower \GLS{9HPT} performance. Asymmetry
Reduced pin prick and light touch score correlated with reduced \gls{FA} in the \gls{ROI} placed over the posterior columns. These clinically relevant relationships suggest that the integrity of the motor and sensory function is related to microanatomical changes and is a significant factor in chronic \gls{SCI} disability. Thus, alterations in the microstructure of the axonal architecture of the injured spinal cord in descending and ascending pathways are clinical eloquent and deserve further assessment in longitudinal studies.

\paragraph{Methodological considerations}
From a technical point of view, we chose to manually draw individual \glspl{ROI} to assure that the \glspl{ROI} are positioned in the columns that contain the correct ascending and descending tracts \citep{Freund:2010}. For the analysis of mean diffusion parameters over the whole cross-section of the cervical cord, we chose an automatic segmentation method that avoids inter- and intra-observer bias. The method uses thresholded (\gls{FA} >0.15) \gls{FA} maps since they provide better contrast between \gls{CSF} and \gls{SCA} than the low diffusion weighted images alone, although this method could be considered biased towards high \gls{FA} values.

\subsection{Conclusion}
In conclusion, we have demonstrated that trauma induced reduction in axonal and myelin integrity in the cervical injured cord rostral to initial site of trauma. Importantly, we have provided evidence of clinical eloquent correlations between spinal pathways affected by axonal disintegration and demyelination and motor and sensory function, respectively. In addition we have shown that microanatomical sequel of \gls{SCI} is associated with spinal atrophy. Thus, progressive disintegration of tissue architecture rostral to spinal injury may contribute to spinal atrophy and changing clinical status in chronic \gls{SCI}. Future longitudinal studies in larger cohorts of \gls{SCI} subjects are necessary to investigate whether \gls{DTI} metrics can serve as sensitive biomarkers but the results of this study involving only a small group provide promising initial results.
\end{subappendices}
